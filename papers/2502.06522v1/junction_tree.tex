
\subsection{Junction Tree Algorithm} \label{sec:junction_tree}

In this section, we prove the following theorem for directed {\hopset}.

\begin{theorem} \label{thm:junction_tree}
    There is a polynomial-time $\widetilde{O} (\beta n^\epsilon \cdot {\opt})$-approximation for directed {\hopset}.
\end{theorem}

To prove the theorem, we give an algorithm similar to a subroutine of the directed pairwise weighted spanner algorithm of~\cite{GKL23}, where ``weighted" refers to edge costs. Just as for hopsets, the directed pairwise weighted spanner problem does not have $n-1$ as a lower bound on the cost of the optimal solution. This allows their techniques to be useful in our setting. 

In the pairwise weighted spanner subroutine of~\cite{GKL23}, they define a variant of the junction tree (the ``distance-preserving junction tree"). Junction trees are rooted trees that satisfy demands, and \textit{good} junction trees satisfy many demands at low cost; that is, they have low ``density." Junction trees have been used in several spanner approximation algorithms (e.g.~\cite{GKL23, CDKL20, GKL24}). In~\cite{GKL23}, they give an algorithm that iteratively buys their version of low density junction trees until all demands are satisfied. Our algorithm will follow the same structure. The main technical work in this section is in showing that low-density \textit{hopbounded} junction trees exist in our setting, and that we can use the subroutine of~\cite{GKL23} to find these hopbounded junction trees, even though their subroutine does not have any hop guarantees.

We first define a hopbounded variant of the junction tree, which we call an $(i,j)$-distance-preserving hopbounded junction tree. We parameterize by $i,j$, where $i+j \leq \beta$, to ensure that both ``sides" of the rooted tree---the in-arborescence and the out-arborescence that make up the tree---are hopbounded by $i$ and $j$, respectively, so that all paths in the tree have at most $\beta$ hops. 

\begin{definition}[$(i,j)$-Distance-Preserving Hopbounded Junction Tree] 
     An \textit{$(i,j)$-distance-preserving hopbounded junction tree}, where $i + j \leq \beta$, is a subgraph of $G_M$ that is a union of an in-arborescence and an out-arborescence, both rooted at some vertex $r \in V$, with the following properties: 1) every leaf of the in-arborescence has a path of at most $i$ hops to $r$, 2) for every leaf $w$ in the out-arborescence, there is a path of at most $j$ hops from $r$ to $w$, and 3) for some node $s$ in the in-arborescence and some node $t$ in the out-arborescence, there is an  $s-t$ path through $r$ with distance at most $Dist(s,t)$. The \textit{density} of an $(i,j)$-distance-preserving hopbounded junction tree $T$ is the ratio of the cost of $T$ to the number demands settled by $T$. 
\end{definition}

Going forward, we will refer to the $(i,j)$-distance-preserving hopbounded junction tree as simply an ``$(i,j)$-junction tree." Our algorithm will find and remove a low-density hopbounded junction tree from $G_M$, add its edges to the current solution, and repeat, until all demand pairs are settled. 
We will give an $O(n^\epsilon)$-approximation for finding these low-density junction trees. The algorithm will return a subgraph with total cost of $\widetilde{O} (\beta^2 n^\epsilon \cdot {{\opt}}^2 )$.

\subsubsection{Existence of Low-Density Junction Trees}
Let $D$ be the set of \textit{unsettled} vertex pairs at some iteration of the algorithm.
We first argue that a hopbounded junction tree with density $O (\beta^2 \cdot {\opt}^2 / \mbox{ } |D| )$ always exists (at any iteration), where ${\opt}$ is the cost of the optimal solution to the problem instance. 

\begin{lemma}
\label{lem:existence}
    For any set of unsettled demands $D$, there exists an $(i,j)$-junction tree with density $O(\beta \cdot {{\opt}}^2 / |D|) )$.
\end{lemma}
\begin{proof}
    Let $H$ be an optimal solution subgraph to the graph $G_M$, and let $S$ be the cheapest subgraph of $G_M$ that settles all demands in $D$.
    We have that $c(S) \leq c(H) = {\opt}$.
    %$|H_D \cap E'| \leq |H \cap E'| = {\opt}$.
    We now look at the set of paths in $S$ that settle the demands in $D$. Each of these $|D|$ paths must use some edge in $S \cap \widetilde{E}$; otherwise, the demand is already settled and cannot be in $D$. Due to averaging, there must be some edge $e \in S \cap \widetilde{E}$ that belongs to at least $|D| / |S \cap \widetilde{E}| \geq |D| / {\opt}$ of these paths. 
    
    Let $D_e \subseteq D$ be the set of demands settled by paths through $e = (u,v)$ in $S$. Let $P_{e}$ be these demand-settling paths (that is, $P_{e}$ is the set of paths that settle the demands in $D_e$ through $e$). 
    We now place the demands in $D_e$ into $O(\beta)$ classes as follows. Let $x,y$ be nonnegative integers such that $x+y = \beta$. We say that a demand $(s,t) \in D_e$ is in class $(x,y)$ if its corresponding path in $P_{e}$ has at most $x$ hops from $s$ to $u$ and at most $y$ hops from $u$ to $t$. Note that every demand in $D_e$ belongs to at least one class. Recall that $e$ belongs to at least $|D| \mbox{ } / \mbox{ } {\opt}$ demand-settling paths in $S$.
    Again due to averaging, there must be some class containing $\Omega\left( |D| \mbox{ } / \mbox{ } (\beta \cdot {\opt}) \right)$ demands. Let $(i,j)$ denote such a class, let $D_e^{(i,j)} \subseteq D_e$ denote the set of demands in this class, and let $P_e^{(i,j)}$ be their corresponding demand-settling paths. 

    Now consider the tree created by adding all paths in $P_{e}^{(i,j)}$, rooted at vertex $u$. This tree is an $(i, j)$-junction tree, as it is the union of an in-arborescence where each leaf has a path of at most $i$ hops to $u$, and an out-arborescence where $u$ has a path with at most $j$ hops to each leaf, where $i+j \leq \beta$. This tree has cost at most ${\opt}$ and settles at least $|D_e^{(i,j)}| = \Omega \left( |D| \mbox{ } / \mbox{ } (\beta \cdot {\opt}) \right)$ demands, and thus has density $O( \beta \cdot {{\opt}}^2 / \mbox{ } |D| )$.
\end{proof}

\subsubsection{Layered Graph Reduction} \label{sec:layered_reduction}
We want to show that we can \textit{find} a junction tree with low enough density at each iteration of the algorithm. To do so, we will use the junction-tree finding subroutine provided in~\cite{GKL23}. Their subroutine, however, finds non-hop-constrained junction trees. We therefore transform our input graph in order to use their subroutine to find $(i,j)$-junction trees. We build the following $\beta$-layered graph out of $G_M$. 

\paragraph{Layered Graph Construction.} To construct the layered graph $G_L = (V_L, E_L)$ with costs $c_L : E_L \rightarrow \{0,1\}$, weights $\ell_L : E_L \rightarrow \{1, 2, \dots, \texttt{poly}(n) \}$, demand set $\mathcal{D}_L$, and distance bounds $Dist_L : \mathcal{D}_L \rightarrow \mathbb{N}_{\geq 0}$, 
we first create $\beta + 1$ copies of each vertex $u \in V$ so that $u$ corresponds to vertices $u_0, u_1, \dots, u_{\beta}$ in $V_L$.
For each edge $(u,v) \in E_M$ we add edges $\{ (u_i, v_{i+1})$ : $i \in [0, \beta-1] \}$ to $E_L$. 
For each edge $e = (u_i, v_{i+1})$ of this type, we set $\ell_{L}(e) = \ell(u,v)$. We also set $c_{L}(e) = 1$ if $(u,v) \in \widetilde{E}$; otherwise we set $c_{L}(e) = 0$.  
For each vertex in $V_L$, we also add edges $\{ (u_i, u_{i+1}) : i \in [0, \beta-1]  \}$ to $E_L$, and set their costs and weights to $0$. 
Finally, we add a demand $(s_0, t_\beta)$ to $\mathcal{D}_L$ for demand each $(s,t) \in \mathcal{D}$.


By design, $(i,j)$-junction trees in $G_M$ correspond to $(i,j)$-junction trees in $G_L$ (and vice versa) with the same density. The proof of this is straightforward, \iflong and is given in the following pair of lemmas. \else and is deferred to Appendix~\ref{app:reduction}. We say that {\jt} is the optimization problem of finding the minimum density $(i,j)$-junction tree in an input graph, over all possible values of $i,j$. \fi

\iflong
\begin{lemma}
\label{cl:input_to_layered}
    For any $(i,j)$-junction tree $T$ in $G_M$, there exists an $(i,j)$-junction tree $T_L$ in $G_L$ with the same density. 
\end{lemma}
\begin{proof}
    Fix an $(i,j)$-junction tree $T$ in $G_M$. We build an $(i,j)$-junction tree $T_L$ in $G_L$ with the same density. Tree $T$ is the union of an in-arborescence, which we denote as $A^{in}$, and an out-arborescence, denoted by $A^{out}$, both of which are rooted at some node $r$. We add node $r_i \in V_L$ to $T_L$ as its root. 
    For each vertex $u \in A^{in}$, let $h_u$ be the number of hops (edges) on the $u-r$ path in $A^{in}$; likewise, for each vertex $w \in A^{out}$, let $h_w$ be the number of hops on the $r-w$ path in $A^{out}$. For each edge $(u,v) \in A^{in}$, we add edge $(u_{i-h_u}, v_{i-h_u+1})$ (and corresponding nodes) to $T_L$. Similarly, for each edge $(u,v) \in A^{out}$, we add edge $(u_{i+h_u}, v_{i+h_u+1})$ (and corresponding nodes) to $T_L$.
   
    Finally, we add ``dummy paths" to $T_L$ to ensure that the corresponding demands in $\mathcal{D}_L$ are settled. These dummy paths are need to handle demand-settling paths in $T$ that have fewer that $\beta$ hops. Let $A^{in}_L$ and $A^{out}_L$ denote the in- and out-arborescences of $T_L$, respectively. For each vertex $v_x \in A^{in}_L$, we add edges $\{ (v_k, v_{k+1}) :  0 \leq k < x  \}$ (and corresponding nodes) to $T_L$ (if they don't already exist in $T_L$). Similarly for each vertex $v_x \in A^{out}_L$, we add edges $\{ (v_k, v_{k+1}) :  x \leq k < \beta  \}$ (and corresponding nodes) to $T_L$ (if they don't already exist in $T_L$). With these dummy paths, we have that if demand $(s,t)$ is settled by a path in $T$ with fewer than $\beta$ hops, then the corresponding demand $(s_0, t_\beta)$ is also settled in $T_L$.  

    Tree $T_L$ has the same cost as $T$: for every edge $(u,v) \in T$ we add an edge $(u_k, v_{k+1})$, for some integer $k$, to $T_L$. Recall that $(u,v)$ and $(u_k, v_{k+1})$ have the same cost, for any $0 \leq k < \beta$. All other edges in $T_L$ (i.e., all edges on dummy paths) are of the form $(u_k, u_{k+1})$, for some $k$, and have cost $0$. 
    
    The number of demands satisfied in both trees is also the same, which we show by mapping each demand $(s,t) \in \mathcal{D}$ settled by $T$ to a unique demand $(s_0, t_\beta) \in \mathcal{D}_L$ settled by $T_L$ (and vice versa). We now show that if $(s,t)$ is settled by $T$, then $(s_0, t_\beta)$ is settled by $T_L$. Let $P = \{s, a, b, \dots, t\}$ be the path in $T$ that settles $(s,t)$. Then, path $P_L = \{s_0, \dots, s_k, a_{k+1}, b_{k+2}, \dots, t_{m}, \dots , t_\beta \}$ is in $T_L$, for some $k, m$. Paths $P$ and $P_L$ have the same length---any edge $(u,v) \in P$ has the same length as its corresponding edge $(u_k, v_{k+1})$ (for some $k$) in $P_L$, and all other edges (i.e., edges from the dummy subpaths $(s_0, \dots, s_k)$ and $(t_m, \dots, t_\beta)$, if they exist) have length $0$. We therefore have that $d_{T_L}(s_0,t_\beta) = d_T(s,t) \leq  Dist(s,t) = Dist_L(s_0,t_\beta)$. Also, path $P_L$ has exactly $\beta$ hops. Thus, demand $(s_0, t_\beta)$ is settled by $T_L$. It is also not difficult to see that each demand $(s_0, t_\beta) \in \mathcal{D}_L$ settled by $T_L$ can be mapped to a unique demand $(s,t) \in \mathcal{D}$ settled by $T$, using similar arguments. Trees $T$ and $T_L$ therefore have the same cost and settle the same number of demands, and so have the same density. 
\end{proof}

\begin{lemma}
\label{cl:layered_to_input}
    For any $(i,j)$-junction tree $T_L$ in $G_L$, there exists an $(i,j)$-junction tree $T$ in $G_M$ with the same density.
\end{lemma}
\begin{proof}
     Given an $(i,j)$-junction tree $T_L$ of $G_L$, we can build an $(i,j)$-junction tree $T$ in $G_M$ with the same density. Note that edges only exist between adjacent layers in $G_L$ (and in $T_L$)---namely, all edges in $T_L$ are of the from $(u_k, v_{k+1})$ for some $k$. For each $k \in [0, \beta]$ and for each edge $(u_k, v_{k+1})$ in $T_L$ such that $u \neq v$, we add edge $(u, v) \in G_M$ (and corresponding nodes) to $T$. 
    
    Trees $T_L$ and $T$ have the same cost: For each edge $(u_k, v_{k+1}) \in T_L$ (for some $k$) with cost $1$, we add edge $(u,v)$ to $T$, which also has cost $1$. For all other edges in $T_L$ (all of which have no cost), we either add the corresponding edge to $T$, which also has no cost, or we add no edge. 
    
    Both trees also settle the same number of demands. Each demand $(s_0, t_\beta) \in \mathcal{D}_L$ settled by $T_L$ can be mapped to a unique demand $(s,t) \in \mathcal{D}$ settled by $T$. Let $P_L = \{s_0, \dots, s_k, a_{k+1}, b_{k+2}, \dots, t_{m}, \dots , t_\beta \}$ be the path that settles $(s_0, t_\beta)$ in $T_L$. Then, the path $P = \{s, a, b, \dots, t\}$ is in $T$. Paths $P$ and $P_L$ have the same length---any edge of the form $(u_k,v_{k+1}) \in P_L$ (for some $k$), where $u \neq v$, has the same length as its corresponding edge $(u, v) \in P$. All other edges in $P_L$ (i.e., edges from the dummy subpaths if they exist) have length $0$. We therefore have that $d_T(s,t) = d_{T_L}(s_0,t_\beta) \leq Dist_{L}(s_0,t_\beta) = Dist(s,t)$. Path $P$ also has at most $\beta$ hops, so $(s,t)$ is settled by $T$. It is also not difficult to see that each demand $(s, t) \in \mathcal{D}$ settled by $T$ can be mapped to a unique demand $(s_0,t_\beta) \in \mathcal{D}_L$ settled by $T_L$, using similar arguments. We've shown that $T_L$ and $T$ have the same cost and settle the same number of demands, and so they have the same density.
\end{proof}

Let $\Delta(T)$ denote the density of junction tree $T$. The above lemmas imply the following:

\begin{corollary}
\label{cor:equivalent}
    Let $T^*$ be the min-density $(i,j)$-junction tree (over all possible $i,j$) in $G_M$, and let $T_L^*$ be the min-density $(i,j)$-junction tree (over all possible $i,j$) in $G_L$. Then, $\Delta(T^*) = \Delta(T_L^*)$.
\end{corollary}
\begin{proof}
    By Lemma~\ref{cl:input_to_layered}, we have that $\Delta(T_L^*) \leq \Delta(T^*)$.  By Lemma~\ref{cl:layered_to_input}, $\Delta(T^*) \leq \Delta(T_L^*)$. Therefore, $\Delta(T^*) = \Delta(T_L^*)$.
\end{proof}

We now use this Corollary to reduce from finding min-density $(i,j)$-junction trees in $G_M$ to finding them in $G_L$. We say that {\jt} is the optimization problem of finding the minimum density $(i,j)$-junction tree in an input graph, over all possible values of $i,j$.

\begin{lemma}
\label{lem:reduction}
     If there is an $\alpha$-approximation algorithm for {\jt} on $G_L$, then there is also an $\alpha$-approximation algorithm for {\jt} on $G_M$. 
\end{lemma}
\begin{proof}
    Suppose we have an $\alpha$-approximation for {\jt} on graph $G_L$. Then, the following is an algorithm for {\jt} on $G_M$. First, run the $\alpha$-approximation algorithm on $G_L$, and let $T_L$ be the tree returned by the algorithm. Using the procedure described in Lemma~\ref{cl:layered_to_input}, we can build (in polynomial time) a valid $(i,j)$-junction tree $T$ of $G_M$ with the same density as $T_L$. The density of $T$ is as follows:
    \begin{align*}
        \Delta(T) = \Delta(T_L) \leq \alpha \cdot \Delta(T_L^*) = \alpha \cdot \Delta(T^*).
    \end{align*}
    The final equality is due to to Corollary~\ref{cor:equivalent}.
\end{proof}

\else
\begin{lemma}
\label{lem:reduction}
     If there is an $\alpha$-approximation algorithm for {\jt} on $G_L$, then there is also an $\alpha$-approximation algorithm for {\jt} on $G_M$. 
\end{lemma}
\fi

\subsubsection{Junction Tree-Finding Subroutine}
We now show that we can find low-density junction trees at each iteration of the algorithm. Although $(i,j)$-junction trees are hopbounded by definition,  we can now use the following length-bounded junction tree-finding subroutine of~\cite{GKL23} to find hopbounded junction trees, thanks to the reduction to the $\beta$-layered graph $G_L$.

\begin{lemma}[Lemma 16 of \cite{GKL23}]
\label{lem:og_JT_alg}
    For any constant $\epsilon > 0$, there is a polynomial-time approximation algorithm for finding the minimum density distance-preserving junction tree. That is, there is a polynomial time algorithm which, given a weighted directed $n$-vertex graph $G = (V,E)$ where each edge $e \in E$ has cost $c(e) \in \mathbb{R}_{\geq 0}$ and integral length $\ell(e) \in \{0,1, \dots, \poly(n)\}$, terminal pairs $\mathcal{D} \subseteq V \times V$, and distance bounds $Dist : \mathcal{D} \rightarrow \mathbb{N}$ for each terminal pair $(s,t) \in \mathcal{D}$, approximates the following problem to within an $O(n^\epsilon)$ factor:
    \begin{itemize}
        \item Find a non-empty set of edges $F \subseteq E$ minimizing the ratio:
    \end{itemize}
    \begin{align*}
        \min_{r \in V} \frac{\sum_{e \in F} c(e)}{|\{(s,t) \in \mathcal{D} : d_{F,r}(s,t) \leq Dist(s,t) \}|}
    \end{align*}
    where $d_{F,r}(s,t)$ is the length of the shortest path using edges in $F$ which connects $s$ to $t$ while going through $r$ (if such a path exists). We call this problem {\ljt}.
\end{lemma}

This gives an $O(n^\epsilon)$-approximation algorithm for finding the min-density $(i,j)$-junction tree on $G_M$. \iflong \else The proof of the following lemma can be found in Appendix~\ref{app:jt_alg}. \fi

\begin{lemma}
\label{lem:hop_JT_approx}
    There is an $O(n^\epsilon)$-approximation for {\jt} on $G_M$.
\end{lemma}
\iflong 
\begin{proof}
    By Lemma~\ref{lem:reduction}, we can prove the lemma by giving an $O(n^\epsilon)$-approximation for {\jt} on $G_L$. The algorithm is as follows: Simply run the algorithm of Lemma~\ref{lem:og_JT_alg} on $G_L$.

    We now show that the tree $T_L$ returned by this algorithm is an $(i,j)$-junction tree of $G_L$, where $i+j = \beta$, and that the density of $T_L$ is at most a factor $O(n^\epsilon)$ of the density of the optimal tree.
    Tree $T_L$ has some root $r_i$.
    Fix a demand $(s_0,t_\beta) \in \mathcal{D}_L$ that has length at most $Dist_L(s,t)$ in $T_L$. Due to the structure of $G_L$, the path from $s_0$ to $t_\be$ in $T_L$ must have $i$ hops from $s_0$ to $r_i$ and $\beta - i$ hops from $r_i$ to $t_\beta$. Thus $T_L$ is an $(i,j)$-junction tree. To see that this is an $O(n^\epsilon)$-approximation, first observe that the optimal density $(i,j)$-junction tree is a feasible solution to {\ljt} on $G_L$. As for the approximation ratio, the algorithm gives a $O(|V_L|^\epsilon) = O(\beta^\epsilon |V|^\epsilon) = O(n^{\epsilon'})$ approximation, where $n = |V|$ and $\epsilon' > 0$ is an arbitrarily small constant.
\end{proof}
\else
\fi

\begin{lemma}
\label{lem:hop_JT_alg}
    There is a polynomial-time algorithm that finds an $(i,j)$-junction tree with density $O(\beta n^\epsilon \cdot {{\opt}}^2 / |D|) )$, where $D$ is the set of unsettled demands in $G$.
\end{lemma}
\begin{proof}
    By Lemma~\ref{lem:existence}, there exists an $(i,j)$-junction tree with density $O(\beta \cdot {{\opt}}^2 / |D|) )$. We can run the $O(n^\epsilon)$-approximation algorithm (Lemma~\ref{lem:hop_JT_approx}) on $G_M$, which outputs an $(i,j)$-junction tree with density $O(\beta n^\epsilon \cdot {{\opt}}^2 / |D|) )$.
\end{proof}

\subsubsection{Proof of Theorem \ref{thm:junction_tree}}

By iteratively buying these low-density $(i,j)$-junction trees, we get an  $O(\beta n^\epsilon \cdot {{\opt}}^2 )$-approximation for {\hopset}.

\begin{proof}
    The algorithm for {\hopset} builds and returns subgraph $H$, which is initialized as empty. Let $D$ be initialized as the set of unsettled demands in the input.
    The algorithm first finds an $(i,j)$-junction tree $T$ with density $O(\beta n^\epsilon \cdot {{\opt}}^2 / |D|) )$, as described in Lemma~\ref{lem:hop_JT_alg}. It then removes $T$ from $G_M$, adds $T$ to $H$, and removes all settled demands from $D$. This process repeats until all demands are settled. 
     Now we show that the algorithm gives an $O(\beta n^\epsilon \cdot {{\opt}}^2 )$-approximation. Suppose the algorithm runs for $\ell$ total iterations. For iteration $k$ of the algorithm, let $T_k$ be the $(i,j)$-junction tree found at that iteration, let $c(T_k)$ be its cost, and let $s_k$ be the number of demands settled by $T_k$. Let $D_k$ be the set of unsettled demands at the start of iteration $k$. The cost of $H$ is the following:
     \begin{align*}
         c(H) = \sum_{k=1}^{\ell} c(T_k) 
         = \sum_{k=1}^{\ell} O\left(   \frac{\beta n^\epsilon \cdot {{\opt}}^2 }{|D_k|}    \cdot s_k \right)  
         &= O \left( \beta n^\epsilon \cdot {{\opt}}^2     \sum_{k=1}^{\ell} \frac{s_k}{|D_k|} \right)  \\
         &= O \left( \beta n^\epsilon \cdot {{\opt}}^2   \cdot \log n \right) \tag{$|D|$\textsuperscript{th} Harmonic number} \\
         &= \widetilde{O}  \left( \beta n^\epsilon \cdot {{\opt}} \right) \cdot {\opt}.
     \end{align*}
    Note that $\sum_{k=1}^{\ell} \frac{s_k}{|D_k|} = O(H_{|D|}) = O(\log n)$, where $H_{|D|}$ is the $|D|$\textsuperscript{th} Harmonic number.
\end{proof}




