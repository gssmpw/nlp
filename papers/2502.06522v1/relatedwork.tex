\section{Related Work}
\label{sec:related_work}

In earlier applications, sparse $(1+\epsilon)$-approximate hopsets for undirected graphs were used to obtain low parallel depth single-source shortest paths algorithms \cite{klein1997, cohen2000, miller2015, elkin2019RNC}. Similarly, fast hopset constructions for undirected graphs were studied in many other settings such as distributed \cite{elkin2019journal, elkin2017, censor2021, elkin2017, dinitz2019, DM24} and dynamic settings \cite{bernstein2011,henzinger2014, gutenberg2020, chechik2018, LN2022}.

Another line of work focuses on fast computation of hopsets for directed graphs and shortcut sets that preserve pairwise reachability, rather than distances, while reducing the diameter. These have gained attention particularly due to their application in parallel reachability and directed shortest path computation \cite{ullman1990high, cao2020efficient, cao2020improved, cao2023exact, fineman2018nearly,jambulapati2019parallel}.

More recently, hopsets and shortcut sets have been studied from an extremal (or existential) point of view.
In particular, for $(1+\epsilon)$-hopsets, there are almost (up to $1/\epsilon$ factors) matching upper bounds \cite{elkin2019journal, huang2019} and lower bounds \cite{abboud2018} in \emph{undirected graphs}. One trade-off that is widely used in $(1+\epsilon)$-approximate single-source shortest paths applications is that for any graph, there exists a $(1+\epsilon)$-hopset of size $n^{1+o(1)}$ with hopbound $n^{o(1)}$.  On the other hand, \cite{abboud2018} showed that there are instances in which this trade-off is tight. For larger stretch, better size/hopbound trade-offs are known.

In directed graphs there are polynomial gaps between existential lower bounds and upper bounds, both for approximate hopsets and shortcut sets. A widely used folklore sampling approach implies an $\widetilde{O}(n)$ size for exact hopset and shortcut sets, with hopbound $O(\sqrt{n})$ \footnote{The upperbounds for directed hopsets and shortcut sets give a smooth trade-off between size and hopbound. For simplicity, we discuss the important linear-size regime.}.
A breakthrough result of \cite{KP22} went beyond this long-standing bound by constructing shortcut sets of size $\widetilde{O}(n)$ with hopbound $\widetilde{O}(n^{1/3})$. 
Later, \cite{BW23} showed that the same upper bound also holds for directed $(1+\epsilon)$-hopsets (up to log factors in weights and aspect ratio). More on these existential bounds can be found in Section \ref{sec:existential}, where we trade them off with our approximation algorithms\iflong \else, and in Appendix~\ref{app:tradeoffs} \fi. Another line of work focused on obtaining subsequently better existential lower bounds for directed graphs \cite{hesse2003directed, huang2021lower, kogan2023towards,  williams2024simpler}. The current best known lower bound for a linear size shortcut set is $\widetilde{\Omega}(n^{1/4})$ \cite{BH23folklore, williams2024simpler}. 

Another recent result \cite{BH23folklore} showed that for \textit{exact hopsets} both in directed and undirected \textit{weighted} graphs, the folklore sampling (see Section \ref{sec:existential}) is existentially optimal, implying that the separation between approximate hopsets for directed and undirected graphs, does not hold for exact hopsets.


On the approximation algorithms side, most relevant to hopsets are the weighted pairwise spanner algorithms \cite{GKL23}, and 2-hop covers \cite{CHHZ2003}. In particular, weighted pairwise spanners (like hopsets and unlike $k$-spanners) do not have $n$ as a trivial lower bound on $\opt$, and hence some of the difficulties encountered when designing algorithms are similar. 


Approximation algorithms for 2-hop covers~\cite{CHHZ2003}, hub labelings~\cite{CHHZ2003,BGGN17}, and 2-spanners \cite{KP94,DK11} in particular are closely related to our hopbound $2$ regime (see \iflong Section~\ref{sec:2hop}\else Appendix~\ref{app:2hop}\fi).  In all of these problems, the requirement of covering using 2-paths leads to algorithms that are essentially solving some variant of Set Cover.  Our situation is similar, so we can use similar techniques, but our Set Cover variant is slightly different from these other problems, most notably because we do not ``pay'' for edges that already exist in the graph. 



%%%%%%%%%%%%%%%%%%%%%%%%%%%%%%%%%%%%%%%%%