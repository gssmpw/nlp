
\subsection{Star Sampling with Randomized LP Rounding Algorithm} \label{sec:star_sampling_rounding}


In this section we prove the following theorem.

\begin{theorem} \label{thm:bbmry_alg}
    There is a randomized algorithm for directed {\hopset} with expected approximation ratio $O( n \ln{n} \, \big/ \sqrt{\opt})$.
\end{theorem}

The pair of algorithms we give closely follow the $\widetilde{O}(n^{2/3})$- and $\widetilde{O}(\sqrt{n})$-approximations for the unweighted $k$-spanner problem, given by~\cite{DK11} and~\cite{BBMRY11}. The $k$-spanner algorithm is a trade-off between two algorithms: an arborescence sampling algorithm for settling a class of edges (or demands) that they call ``thick," and a randomized LP rounding algorithm for settling ``thin" edges. For hopsets we will settle these thick demands by sampling directed stars instead of arborescences, and for thin demands we will use a similar LP rounding approach. Although our hopset algorithms are similar to the $k$-spanner algorithms, we get a different approximation ($\widetilde{O}( n \big/ \sqrt{\opt})$ for hopsets versus $\widetilde{O}(\sqrt{n})$ for spanners). This is because \cite{DK11, BBMRY11} take advantage of the fact that for spanners, ${\opt} \geq n-1$. This is not the case for hopsets so we get a different approximation out of the algorithms, in terms of {\opt}. 

We note that our $O( n \ln{n} \, \big/ \sqrt{\opt})$-approximation is achieved by trading off the two aforementioned algorithms (star-sampling and randomized-rounding). To later achieve the optimal trade-off with other algorithms, one should a priori treat each of these two algorithms as separate, with their own individual approximation ratios. It is however equivalent to trade these two algorithms off first and treat them as one combined algorithm, which we do going forward. This is because these are our only algorithms that will depend on the ``local neighborhood size" parameter.

To define thick and thin demands, we must first define subgraphs $G^{s,t}$ for all demands $(s,t)$, as in \cite{DK11, BBMRY11}:

\begin{definition}
    For a demand $(s,t) \in \mathcal{D}$, let $G^{s,t} = (V^{s,t}, E^{s,t})$ be the subgraph of $G_M$ induced by the vertices on paths in $\mathcal{P}_{s,t}$. We call $|V^{s,t}|$ the \textit{local neighborhood size}.
\end{definition}

\begin{definition}[Thick and Thin Demands] 
    Let $b$ be a parameter in $[1,n]$. If $|V^{s,t}| \geq n/b$ then the corresponding demand $(s,t)$ is thick, otherwise it is thin. We shall always assume that $b = \sqrt{{\opt}}$.
\end{definition}

Let $\mathcal{D}_{thick}$ and $\mathcal{D}_{thin}$ be the set of all thick and thin demands, respectively. We will run two algorithms to build two edge sets, $E'$ and $E''$, such that all thick demands are settled by $E'$ and all thin demands are settled by $E''$. The set $E'$ will have cost $O(bn\ln{n})$ in expectation, while $E''$ will have cost $O((n/b)\ln{n} \cdot {\opt})$ in expectation. The optimal trade-off of these algorithms has $b = \sqrt{\opt}$, so each edge set will have cost $O(n\ln{n} \cdot \sqrt{{\opt} })$ in expectation.

\subsubsection{Star-Sampling Algorithm for Thick Demands} \label{sec:thick}
We describe the random sampling subroutine for constructing the edge set $E'$, which will settle all thick demands (Algorithm~\ref{alg:star_sample}). 

\begin{algorithm}[h]
\DontPrintSemicolon

\textbf{Input:} 
Graph $G_M = (V, E_M)$ \\

Let $E' \gets \emptyset, \; S \gets \emptyset$ \tcp*{Set $S$ is only used for the analysis} 

\ForEach{index $i = 1, 2, \dots, b\ln{n}$}{
    $v \gets$ a uniformly random element from $V$ \\
    $T^{in}_v \gets$ inward star of $G_M$ rooted at $v$ \\
    $T^{out}_v \gets$ outward star of $G_M$ rooted at $v$ \\
    $E' \gets E' \cup T^{in}_v \cup T^{out}_v , \; S \gets S \cup \{v\}$ 
} 
\ForEach{unsettled demand $(s,t) \in \mathcal{D}_{thick}$}{
    $E' \gets E' \cup (s,t)$
} 
\textbf{Return} $E'$ \;

\caption{\label{alg:star_sample} Star-Sampling Algorithm}
\end{algorithm}

This algorithm is nearly identical to that of~\cite{DK11}. The only difference is that, since we operate on the weighted transitive closure of $G$, we build directed in- and out-stars as opposed to the shortest path in- and out-arborescences used for the spanner setting. 

We now show that $E'$ has the desired cost in expectation. While~\cite{DK11} proves this for spanners, it's easy to see that a near identical argument also holds for hopsets in $G_M$. We restate the proof \iflong \else in Appendix~\ref{app:thick_proof} \fi for completeness.

\begin{lemma}[\hspace{1sp}\cite{DK11}] \label{lem:thick}
    Algorithm~\ref{alg:star_sample}, in polynomial time, computes an edge set $E'$ that settles all thick demands and has expected cost $O(bn\ln{n} )$. If $b = \sqrt{{\opt}}$, then the expected size is $O(n\ln{n} \cdot \sqrt{{\opt}})$.
\end{lemma}
\iflong
\begin{proof}  
    After the execution of the first for loop in Algorithm~\ref{alg:star_sample}, $|E'| \leq 2(n-1)b \ln{n}$.

    If some vertex $v$ from a set $V^{s,t}$ appears in the set $S$ of vertices selected by Algorithm~\ref{alg:star_sample}, then $T^{in}_v$ and $T^{out}_v$ contain shortest, $1$-hop paths from $s$ to $v$ and from $v$ to $t$, respectively. Thus, both paths are contained in $E'$. Since $v \in V^{s,t}$, the sum of lengths of these two paths is at most $Dist(s,t)$. Therefore, if $S \cap V^{s,t} \neq \emptyset$, then the demand $(s,t)$ is settled. For a thick demand $(s,t)$, the set $S \cap V^{s,t}$ is empty with probability at most $(1-1/b)^{b\ln{n}} \leq e^{-\ln{n}} = 1/n$. Thus, the expected number of unsettled thick demands added to $E'$ in the final for loop of Algorithm~\ref{alg:star_sample} is at most $|\mathcal{D}|/n \leq n$.

    The final for loop ensures that $E'$, returned by the algorithm, settles all thick demands. Computing in- and out-stars and determining whether a demand is thick can be done in polynomial time.
\end{proof}
\else
\fi


\subsubsection{Randomized LP Rounding Algorithm for Thin Demands} 
We now give the algorithm for finding a set $E''$ to settle thin demands. \cite{BBMRY11} introduces the notion ``anti-spanners," which is crucial for the algorithm and analysis for settling thin demands. In particular, they formulate an anti-spanner covering LP that captures the problem of settling all thin demands. They then solve the LP (with high probability) by constructing a separation oracle that utilizes randomized rounding. We will also use randomized LP rounding, though instead of rounding the solution to an ``anti-hopset" covering LP, we will round based on~\ref{lp:hopset}.  Our LP is stronger than the ``anti-hopset" covering LP, since our LP is for \textit{fractional} cuts against valid paths, while the anti-hopset covering LP is only for integer cuts.

Going forward, we will assume without loss of generality that we know the value of the optimal solution---${\opt}$ is in $\{0, 1, \dots, n^2 \}$, so we can just try each of these values for ${\opt}$ and return the smallest hopset found over all tries. We can therefore replace the objective function of~\ref{lp:hopset} with the following:
\begin{align*}
    \sum_{e \in \widetilde{E}} x_e \leq {\opt} \tag{4}
\end{align*}

We use this modified version of~\ref{lp:hopset} for the randomized rounding algorithm. Given a fractional solution $\bm{\mathrm{x}}^*$ to~\ref{lp:hopset}, our algorithm will return an edge set $E''$ that, with high probability, will cost at most $2{\opt} \cdot 2(n/b)  \ln{n}$ and satisfy all thin demands (see Algorithm~\ref{alg:random_rounding}). We say that the algorithm \textit{fails} if $c(E'') > 2{\opt} \cdot 2(n/b)  \ln{n}$ or if $E''$ doesn't satisfy all thin demands. The algorithm will fail with low probability.

\begin{algorithm}[h]
\DontPrintSemicolon

\textbf{Input:} Graph $G_M = (V, E_M)$, \ref{lp:hopset} fractional solution $\bm{\mathrm{x}}^*$ \\

Let $E'' \gets \emptyset$ \; \;

\tcp{sample edges into $E''$}
\ForEach{edge $e \in E_M $}{
    Let $p_e \gets \min(1, 2(n/b) \ln{n} \cdot x^*_e)$ \;
    Add $e$ to $E''$ with probability $p_e$  } \;

\If{$E''$ settles all thin demands}{
    \textbf{Return} $E''$} 
    
\Else{ 
    \textbf{Return} $E_M \setminus E$  }

\caption{\label{alg:random_rounding} Randomized LP Rounding Algorithm }
\end{algorithm}

To show that with high probability, Algorithm~\ref{alg:random_rounding} does not fail, we start by defining ``anti-hopsets," the analogous of anti-spanners in the hopset setting. For a given demand $(s,t)$, an anti-hopset is a set of edges such that removing them from $G_M$ results in no valid paths from $s$ to $t$ in what remains.

\begin{definition}[Anti-Hopsets]
    An edge set $C \subseteq E$ is an anti-hopset for demand $(s,t) \in \mathcal{D}$ if there is no $\beta$-hopbounded path of length at most $Dist(s,t)$ in $G_M \setminus C$. If no proper subset of an anti-hopset $C$ is an anti-hopset, we say that $C$ is a minimal.
\end{definition}

Thus, a set of edges is a valid hopset if and only if it is a hitting set for the collection of (minimal) anti-hopsets---that is, to be a valid hopset, an edge set must include at least one edge from every anti-hopset.
We now show that the probability is exponentially small that the algorithm fails. The argument is very similar to that given by~\cite{BBMRY11} for spanners; we state it \iflong here \else in Appendix~\ref{app:thin_proof} \fi for completeness.

\begin{lemma}[Theorem 2.2 of \cite{BBMRY11}] \label{lem:thin_fail}
    The probability that the algorithm fails is exponentially small in $n$.
\end{lemma}
\iflong
\begin{proof}
    There are two different events that can cause the algorithm to fail:   
    \begin{enumerate}
        \item The cost of the sampled set $E''$ is too high---that is, $c(E'') > 2{\opt} \cdot 2(n/b)  \ln{n}$. The expected cost of $E''$ is at most $2(n/b) \ln{n} \, \cdot \sum_{e \in E_M \setminus E} x_e \leq {\opt} \cdot 2(n/b)  \ln{n}$. By the Chernoff bound (recall that $b = \sqrt{{\opt}}$), we have that $\Pr[c(E'') > 2{\opt} \cdot 2(n/b)  \ln{n} ] \leq e^{-c \cdot {\opt} \cdot (n/b) \ln{n}} = e^{-c n\ln{n} \cdot \sqrt{{\opt}}} = e^{-\Omega(n \ln(n))}$. Thus, the probability that the algorithm fails because $c(E'') > 2{\opt} \cdot 2(n/b)  \ln{n}$ is exponentially small.
        \item $E''$ does not settle all thin demands. We prove that the probability that $E''$ does not settle all thin demands (that is, that $E''$ does not intersect all minimal anti-hopsets for thin demands) is exponentially small in the following Lemma.
    \end{enumerate}
    
    \begin{lemma}[Lemma 2.3 of~\cite{BBMRY11}] \label{lem:thin_not_settle}
        The probability that there exists a demand $(s,t)$ and a minimal anti-hopset $C$ for it such that $C \subset E^{s,t} \setminus E''$ is at most $|\mathcal{D}_{thin}| \cdot \left( 1 / b n \right)^{n/b}$. In particular, if $b = \sqrt{{\opt}}$, then the probability is at most $|\mathcal{D}_{thin}| \cdot (1 \, / \, n\sqrt{{\opt}})^{n/\sqrt{\opt}}$.
    \end{lemma}
    \begin{proof}
        First, we bound the total number of minimal anti-hopsets for thin demands. 

        \begin{proposition}[Proposition 2.1 of \cite{BBMRY11}] \label{prop:thin1}
            If $(s,t)$ is a thin demand, then there are at most $(n/b)^{n/b}$ minimal anti-hopsets for $(s,t)$.
        \end{proposition}
        \begin{proof}
            Fix a thin demand $(s,t)$ and consider an arbitrary minimal anti-hopset $C$ for $(s,t)$. For the rest of this argument, for any two paths of the same length between the same pair of vertices, we consider one path to be shorter than the other if that path has fewer hops. More specifically, for any vertex pair $x, y \in V$ and any two $x-y$ paths $P, P'$ that have the same length, we say that $P$ is shorter than $P'$ if $P$ has fewer hops (edges).
            
            Let $A_C$ be the outward shortest path tree (arborescence) rooted at $s$ in the graph $(V^{s,t}, E^{s,t} \setminus C)$. Our tie-breaking of same-length paths with different hops ensures that $A_C$ includes shortest paths with the lowest number of hops (recall that every edge $(u,v) \in E_M$ is a shortest path from $u$ to $v$). 
            Denote by $f^{(\beta)}_{A_C}(u)$ the $\beta$-hopbounded distance from $s$ to $u$ in the tree $A_C$. If there is no $\beta$-hopbounded directed path from $s$ to $u$ in $A_C$, let $f^{(\beta)}_{A_C}(u) = \infty$. 
            
            We show that $C = \left\{ (u,v) \in E^{s,t} : f^{(\beta)}_{A_C}(u) + \ell(u,v) <  f^{(\beta)}_{A_C}(v) \right\}$, and thus, $A_C$ uniquely determines $C$ for a given thin demand $(s,t)$. If $(u,v) \in C$, then, since $C$ is a \textit{minimal} anti-hopset, there exists a $\beta$-hopbounded path from $s$ to $t$ of length at most $Dist(s,t)$ in the graph $(G_M \setminus C) \cup \{(u,v)\}$. This path must lie in $(G^{s,t} \setminus C) \cup \{(u,v)\}$ and must contain the edge $(u,v)$. Thus, the $\beta$-hopbounded distance from $s$ to $t$ in the graph $(G^{s,t} \setminus C) \cup \{(u,v)\}$ is at most $Dist(s,t)$ and is strictly less than $f^{(\beta)}_{A_C}(t)$. Hence, $A_C$ is not the shortest path tree in the graph $(G^{s,t} \setminus C) \cup \{(u,v)\}$. Therefore, $f^{(\beta)}_{A_C}(u) + \ell(u,v) <  f^{(\beta)}_{A_C}(v)$. If $(u,v) \in E^{s,t}$ satisfies the condition $f^{(\beta)}_{A_C}(u) + \ell(u,v) <  f^{(\beta)}_{A_C}(v)$, then $(u,v) \notin E^{s,t} \setminus C$ (otherwise, $A_C$ would not be the shortest path tree), hence $(u,v) \in C$.

            We now count the number of outward trees rooted at $s$ in $G^{s,t} \setminus C$. For every vertex $u \in V^{s,t}$, we may choose the parent vertex in at most $|V^{s,t}|$ possible ways (if a vertex is isolated we assume that it is its own parent), thus the total number of trees is at most $|V^{s,t}|^{|V^{s,t}|} \leq (n / b)^{n/b}$
        \end{proof}

        \begin{proposition}[Proposition 2.2 of \cite{BBMRY11}] \label{prop:thin2}
            For a demand $(s,t)$ and a minimal anti-hopset $C$ for $(s,t)$, the probability that $E'' \cap C = \emptyset$ is at most $e^{-2(n/b)\ln{n}}$.
        \end{proposition}
        \begin{proof}
            Suppose there is an anti-hopset edge $e \in C$ such that $x_e^* \geq (2(n/b)\ln{n})^{-1}$. In this case, $e$ is in $E''$ with probability $1$, and we are done. Otherwise, the probability that $e$ is in $E''$ is exactly $2(n/b)\ln{n} \cdot x_e$. In this case, the probability that $E''$ does not include $e$ is 
            \begin{align*}
                \prod_{e \in C} \left( 1 - 2 (n/b) \ln{n} \right) < \exp{\left(-\sum_{x \in C} 2 (n/b) \ln{n} \cdot x^*_e \right)} \leq e^{-2 (n/b) \ln{n}}.
            \end{align*}
            The first inequality holds from the fact that $1-x < \exp{(-x)}$ for $x > 0$. For the  last inequality, observe that every anti-hopset is an integer cut against valid paths. Thus, each anti-hopset $C$ corresponds to an~\ref{lp:hopset} constraint of the form $\sum_{e \in E_M} z_e x_e \geq 1$, where $\bm{\mathrm{z}} \in \mathcal{Z}_{s,t}$ is the indicator vector for cut $C$.
        \end{proof}
        
    To finish the proof of Lemma~\ref{lem:thin_not_settle}, we use Propositions~\ref{prop:thin1} and~\ref{prop:thin2} to take a union bound over all minimal anti-hopsets for all thin demands. Let $\mathcal{S}_{s,t}$ be the set of all minimal anti-hopsets for a thin demand $(s,t)$, and let $\mathcal{S}$ be the collection of all minimal anti-hopsets for all thin demands. The probability that $E''$ does not intersect all minimal anti-hopsets in $\mathcal{S}$ is the following:
    \begin{align*}
        \Pr[E'' \text{ is not a hitting set for } \mathcal{S}] &\leq  \sum_{(s,t) \in \mathcal{D}_{thin}} \sum_{C \in \mathcal{S}_{s,t}} e^{-2 (n/b) \ln{n}} \\
        &\leq  |\mathcal{D}_{thin}| \cdot \left( \frac{n}{b} \right)^{n/b} \cdot e^{-2 (n/b) \ln{n}} \\
        &= |\mathcal{D}_{thin}| \cdot \left( \frac{1}{b n} \right)^{n/b} \\
        &= |\mathcal{D}_{thin}| \cdot \left( \frac{1}{n \sqrt{{\opt}}} \right)^{\frac{n}{\sqrt{{\opt}}}} \tag{$b = \sqrt{{\opt}}$}.
    \end{align*}
    If ${\opt} = \widetilde{O}(n^{2})$, then we can achieve an $\widetilde{O}(1)$-approximation by just returning $E_M \setminus E''$. Thus, we can assume without loss of generality that ${\opt} = \widetilde{o}(n^{2})$.  We therefore have that the probability that $E''$ doesn't cover all minimal anti-hopsets for a thin demand $(s,t)$ is exponentially small.
    \end{proof}

\end{proof}
\else
\fi


\subsubsection{Proof of Theorem~\ref{thm:bbmry_alg}}
\begin{proof}
    All thick demands can be satisfied by running Algorithm~\ref{alg:star_sample} to build $E'$, which has expected cost $O(n \ln{n} \cdot \sqrt{{\opt}})$ (by Lemma~\ref{lem:thick}) and runs in polynomial time. The thin demands can be satisfied by running Algorithm~\ref{alg:random_rounding}, which runs in polynomial time. Algorithm~\ref{alg:random_rounding} fails with exponentially small probability (in which case we return all possible hopset edges, $\widetilde{E}$), and thus the expected cost of $E''$ is at most ${\opt} \cdot  2(n/b) \ln{n} + o(1) = O(n \ln{n} \cdot \sqrt{{\opt}} )$ (Lemma~\ref{lem:thin_fail}). Thus the overall approximation ratio is $O( n \ln{n} / \sqrt{{\opt}})$. 
\end{proof}





