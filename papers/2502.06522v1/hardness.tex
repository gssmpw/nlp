
In this section we prove Theorem~\ref{thm:LB-main}. 
 In particular, we prove hardness of approximation for directed shortcut sets on directed graphs, which immediately implies hardness for directed hopsets for any stretch bound as long as the hopbound is at least $3$.  We use a reduction from Min-Rep (the natural minimization version of \LabelCover) in a very similar way to the hardness proofs for graph spanners from~\cite{Kor01,EP07,DKR16,CD16}.  The main technical difficulty / difference is that these previous reductions for spanners heavily use the fact that only edges from the original graph can be included in the spanner.  This is not true of shortcut sets (or hopsets), and makes it simpler to reason about the space of ``all feasible spanners'' than it is to reason about the space of ``all possible shortcut sets''.  

The Min-Rep problem, first introduced by~\cite{Kor01} and since used to prove hardness of approximation for many network design problems, can be thought of as a minimization version of \LabelCover with the additional property that the alphabets are represented explicitly as vertices in a graph.  A Min-Rep instance consists of an undirected bipartite graph $G = (A, B, E)$, together with partitions of $A_1, A_2, \dots, A_{m}$ of $A$ and $B_1, B_2, \dots, B_m$ of $B$ with the property that $|A_i| = |A_j| = |B_i| = |B_j$ for every $i,j \in [m]$\footnote{In some versions of the problem we do not assume that each partition has the same number of parts or that each $|A_i|=|B_i|$, but those assumptions can both be made without loss of generality so we do so for convenience}.  Each part of one of the partitions is called a \emph{group}.  If there is at least one edge between $A_i$ and $B_j$, then we say that $(i,j)$ is a \emph{superedge}.  Our goal is to choose a set $S \subseteq A \cup B$ of vertices of $V$ if minimum size so that, for every superedge $(i,j)$, there is some vertex $x \in A_i \cap S$ and $y \in B_j \cap S$ with $\{x,y\} \in E$.  In other words, we must choose vertices to \emph{cover} each superedge.  Any such cover is called a REP-cover, and our goal is to find the minimum size REP-cover.  Note that we must choose at least one vertex from every group (assuming that each group is involved in at least one superedge), and hence $OPT \geq 2m$.  

Since Min-Rep is essentially \LabelCover, which is a two-query PCP, the PCP theorem implies hardness of approximation for Min-Rep.  More formally, the following hardness is known~\cite{Kor01}.

\begin{theorem}[\cite{Kor01}] \label{thm:MinRep-hardness}
    Assuming that $NP \not\subseteq DTIME(2^{polylog(n)})$, then for any constant $\epsilon > 0$ there is no polynomial-time algorithm that approximates Min-Rep better than $2^{\log^{1-\epsilon} n}$.  In particular, no polynomial time algorithm can distinguish between when $OPT = 2m$ and when $OPT \geq 2m \cdot 2^{\log^{1-\epsilon} n}$
\end{theorem}


\subsection{Reduction from Min-Rep to shortcut sets}
We can now give our formal reduction from Min-Rep to shortcut sets with hopbound $\beta \geq 3$.  Consider some instance of Min-Rep $G = (A, B, E)$ with partition $A_1, A_2, \dots, A_m$ and $B_1, B_2, \dots, B_m$.  First we create two nodes for every group: let 
\begin{align*}
&A' = \{a_1^1, a_1^2, a_2^1, a_2^2, \dots, a_m^1, a_m^2\} & \text{and} &  &B' = \{b_1^1, b_1^2, b_2^1,b_2^2, \dots, b_m^1, b_m^2\}.
\end{align*}
Then for every edge $e = \{a, b\}$, we create a set of $\beta - 3$ vertices $V_e = \{v_e^1, v_e^2, \dots, v_e^{\beta-3}\}$ (note that these sets are empty if $\beta=3$).  Our final vertex set is 
\begin{align*}
    V' = V \cup A' \cup B' \cup (\cup_{e \in E} V_e).
\end{align*}

We now create the (directed) edges of the graph.  For each $e = \{a,b\} \in E$ we create a path:
\begin{align*}
    P_e &= \{(a, v_e^1)\} \cup \{(v_e^i, v_e^{i+1}) : i \in [\beta-4]\} \cup \{(b, v_e^{\beta-3})\}.
\end{align*}
If $\beta = 3$, we simply set $P_e = \{(a,b)\}$.  We can now specify our final edge set:
\begin{align*}
    E' = &\left(\cup_{e \in E} P_e\right) \\
    &\cup \{(a_i^1, a_i^2) : i \in [m]\} \cup \{(b_i^2, b_i^1) : i \in [m]\} \\
    &\cup \{(a_i^2, x) : i \in [m], x \in A_i\} \cup \{(x, b_i^2) : i \in [m], x \in B_i\}.
\end{align*}

Our final shortcut set instance is $G' = (V', E')$ with hopbound $\beta$.

\subsection{Analysis}
The analysis of this reduction has two parts, completeness and soundness (so-called due to their connections back to probabilistically checkable proofs).  For completeness, we show that if the original Min-Rep instance has $OPT = \gamma$ then there is a shortcut set of size at most $\gamma$.  For soundness, we show that if there is a shortcut set of size $\gamma$, then there is a REP-cover of size at most $O(\gamma)$.  Together with Theorem~\ref{thm:MinRep-hardness}, these will imply our desired hardness bound for shortcut sets.  To get Theorem~\ref{thm:LB-main}, one just needs to observe that in $G'$, if there is a path from some node $u$ to some node $v$ then in fact \emph{every} path from $u$ to $v$ has exactly the same length.  Hence any shortcut set is a hopset with the same hopbound and stretch $1$, and any hopset with stretch $\alpha$ is also shortcut set.

\subsubsection{Completeness}
We begin with completeness, which is the easier direction.  We prove the following lemma.

\begin{lemma} \label{lem:completeness}
    If there is a REP-cover of the Min-Rep instance with size $\gamma$, then there is a shortcut set for $G'$ with hopbound $\beta$ and size $\gamma$.
\end{lemma}
\begin{proof}
Suppose that there is a REP-cover $S \subseteq V$ with $|S|= \gamma$.  Consider the edge set 
\begin{align*}
    S' &= \{(a_i^1, x) : i \in [m], x \in S \cap A_i\} \cup \{(x, b_i^1) : i \in [m], x \in S \cap B_i\}.
\end{align*}
In other words, we add an edge between each ``outer'' node representing a group and the nodes in the REP-cover that are in that group (directed oppositely for the left and right sides of the bipartite graph).  Clearly $|S'| = |S| = \gamma$, so it remains to show that $S'$ is a shortcut set with hopbound $\beta$.  

An easy observation is that in $G'$, the only nodes $x, y$ where $y$ is reachable from $x$ but there are more than $\beta$ hops on every $x \leadsto y$ path are when $x = a_i^1$ and $y = b_j^1$ for some $i,j \in [m]$ where $(i,j)$ is a superedge.  So consider such an $a_i^1, b_j^1$.  Then since $(i,j)$ is a superedge and $S$ is a REP-cover, we know that there is some $a,b \in S$ with $a \in A_i$ and $b \in B_j$ and $\{a,b\} \in E$.  Then by definition $S'$ includes the edges $(a_i^1, a)$ and $(b, b_j^1)$.  Thus there is a path from $a_i^1$ to $b_j^1$ which first uses the edge to $a$ added by $S$, then uses the edges of $P_{\{a,b\}}$ to get to $b$, and then uses the edge to $b_j^1$ added by $S$.  This path has exactly $\beta$ hops by construction.  Hence $S'$ is a shortcut set with hopbound $\beta$.
\end{proof}

\subsubsection{Soundness}
We now argue about soundness.  To do this, we first need the following definition.

\begin{definition} \label{def:canonical}
    A shortcut edge is called \emph{canonical} if it is of the form $(a_i^1, a)$ for some $i \in [m]$ and $a \in A_i$ or of the form $(b, b_i^1)$ for some $i \in [m]$ and $b \in B_i$.  A shortcut set $S'$ is called \emph{canonical} if every edge in $S'$ is canonical. 
\end{definition}

We now show that without loss of generality, we may assume that any shortcut set is canonical.

\begin{lemma} \label{lem:canonical}
    Suppose that $S$ is a shortcut set of $G'$ with hopbound $\beta$ and size $\gamma$.  Then there is a canonical shortcut set $S'$ of $G'$ with hopbound $\beta$ and size at most $2 \gamma$.
\end{lemma}
\begin{proof}
    Consider some $(x,y) \in S$ which is not canonical.  Since it is not canonical but also must exist in the transitive closure of $G'$, there must exist a super edge $(i,j)$ so that $x$ and $y$ are as follows.
    \begin{itemize}
        \item $x$ is either $a_i^1, a_i^2$, is in $A_i$, or is in $P_e$ for some $e = \{a,b\}$ with $a \in A_i$ and $b \in B_j$.
        \item $y$ is either $b_j^1, b_j^2$, is in $B_j$, or is in $P_e$ for some $e = \{a,b\}$ with $a \in A_i$ and $b \in B_j$.
    \end{itemize}
    In particular, this implies that $(x,y)$ is only useful in decreasing the number of hops to at most $\beta$ for the pair $(a_i^1, b_j^1)$.  In other words, if we \emph{removed} $(x,y)$ from $S$, then the only pair of nodes which be reachable but not within $\beta$ hops would be $(a_i^1, b_j^1)$.  So choose some arbitrary edge $\{a,b\} \in E$ with $a \in A_i$ and $b \in B_j$, and let $(a_i^1, a)$ and $(b, b_j^1)$ be the \emph{replacement edges} for $(x,y)$.  Note that these replacement edges are both canonical, and suffice to decrease the number of hops from $a_i^1$ to $b_j^1$ to $\beta$.

    So we let $S'$ consist of all of the canonical edges of $S$ together with the two replacement edges for every noncanonical edge in $S$.  Clearly $|S'| \leq 2|S| =2\gamma$, and $S'$ is a shortcut set with hopbound $\beta$.  
\end{proof}

With this lemma in hand, it is now relatively straighforward to prove soundness.

\begin{lemma} \label{lem:soundness}
    If there is a shortcut set for $G'$ with hopbound $\beta$ and size $\gamma$, then there is a REP-cover of the Min-Rep instance with size at most $2\gamma$.
\end{lemma}
\begin{proof}
    Let $\hat S$ be a shortcut set for $G'$ with hopbound $\beta$ and size $\gamma$.  By Lemma~\ref{lem:canonical}, there is a shortcut set $S'$ for $G'$ with hopbound $\beta$ and size at most $2\gamma$.  Let $S$ be the set of nodes such that they are an endpoint of some edge in $S'$ and are in $A \cup B$.  In other words, for every canonical edge $(a_i^1, a) \in S'$ we add $a$ to $S$, and similarly for every canonical edge $(b, b_i^1) \in S'$ we add $b$ to $S$.  Clearly $|S| \leq |S'| \leq 2\gamma$, so we just need to prove that $S$ is a REP-cover.

    Consider some superedge $(i,j)$.  Then $G'$, we know that $a_i^1$ can reach $b_j^1$ but every such path has length $\beta+2$.  Since $S'$ is a canonical shortcut set, this means that it must add an edge $(a_i^1, a)$ and an edge $(b, b_j^1)$ for some $a \in A_i$ and $b \in B_i$ with $(a,b) \in E$.  Hence $a,b \in S$ by definition.  Thus $S$ is a REP-cover as claimed.  
\end{proof}