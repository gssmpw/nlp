\section{Recommendations}
\label{section:recommendations}

We argue that for public infrastructure and private companies to productively and ethically make use of genetic data, several amendments need to be made to the current genetic data governance system. 
%We structure recommendations that rely on protecting the values outlined in our risk assessment framework. 
Our three recommendations address open privacy concerns, legislative scoping for policy changes, and best practices for bodies handling genetic data.

\subsection{Recommendation 1: Redefining Genetic Data}

\textbf{Issue}: Legal policy surrounding genetic privacy notably excludes deidentified or anonymized data from protection.

\textbf{Recommendation}: Given that we argue that genetic data is unique compared to any other identifying data, we suggest genetic data be defined using the following language: 

``Genetic data refers to any information relating to an individual’s genetic characteristics, including but not limited to DNA or RNA sequences, gene expression profiles, or any data derived from a biological sample—including that of a relative—regardless of format. \textbf{This data is always considered identifiable} by its nature—or personally identifiable information (PII)—as it pertains to unique biological attributes that can potentially be linked to a specific individual, their biological relatives, or identifiable group. For the purposes of this definition, any de-identified, pseudonymized, or anonymized genetic information shall be treated as genetic data, regardless of measures taken to mask individual identities, recognizing that the inherent characteristics of genetic information may enable re-identification through advanced technological or data cross-referencing methods.''

\subsection{Recommendation 2: Extending Protections for Genetic Discrimination}

\textbf{Issue}: GINA is a vital piece of federal legislation that protects against genetic discrimination in employment and health insurance domains. However, other domains in which there exists potential for genetic discrimination  are \textit{not} protected by GINA: other insurance domains (life, long-term care, disability), housing, and education. 

\textbf{Recommendation}: We recommend that additional federal laws should be enacted to extend GINA's coverage beyond employment and health insurance. The most comprehensive state genetic anti-discrimination law is California's CalGINA, which includes housing, mortgage brokerage, education, and more.  However, despite these extensions, CalGINA does not cover life, disability, or long-term care insurance. We suggest that extension bills for GINA should explicitly cover life, disability, long-term care, as well as education and any other opportunities for advancement to be protected against genetic discrimination. It should be written in such a way that prohibits \textit{any} barrier to opportunity, even those not anticipated at the time of writing. Additionally, legislation should be written to explicitly bar genetic risk for \emph{complex traits} known to have significant environmental influences (such as cardiovascular disease) from being considered a preexisting condition.


\subsection{Recommendation 3: A Genetic Data Regulation Framework}

\textbf{Issue}: Current regulations were designed to govern one application of genetic data—or Pillar—at a time. This has led to ``leaky protections'', where the use of genetic data in one Pillar can affect opportunities and decisions in other Pillars (e.g., clinical tests being used in life insurance).

\textbf{Recommendation}: To eliminate \emph{leaky protections}, we suggest a comprehensive and uniform regulatory framework that encompasses all usage domains whether genetic data is collected, analyzed, or used in any type of inference. Below, we build upon specific values inherent in privacy rights to ensure that responsibility is placed on the organizations that hold and analyze data, rather than the individuals from whom it was collected \cite{solove_limitations_2022}. 

\begin{enumerate}
    \item \textbf{Entity approval}: Entities—whether they be universities, health care providers, hospital systems, or corporations—that collect or house any type of genetic data (DNA, RNA, or even genetic test results) must have prior regulatory approval. Approval requires a clear commitment to the basic rights of the individuals whose genetic data the entity will collect or own, such as data privacy, security, the right to deletion, etc. 
    %Broadly requiring approval for any public or private entity to collect or house genetic data will ensure that genetic data is given the same protections across all domains. 
    
    \item \textbf{Test approval}: After \emph{entity approval}, we propose regulations for the inferential tests, specifically those that return results to either an individual or a medical practitioner for any actionable result. At a minimum, entities would be required to publicly release detailed white papers for each test that detail laboratory procedures, quality control steps, inferential models used, presentation of results, and any other necessary details that would allow one to recreate the analysis if given the same data. 
    %An open question here is whether to require further regulatory approval for each test and what that regulatory process would encompass. It is not the goal to stymy genetics research, but to ensure that the science underlying tests is sufficiently rigorous.
    
    \item \textbf{Powers given to the individual}
    \begin{itemize}
        \item Individuals should be able to request \textbf{data removal}. This would include specifically (1) the destruction of the physical sample, (2) deletion of data, (3) removal of any identifiers, and (4) that any removed data no longer influence the results of any downstream models. 
        %This last point is purposefully vague, but could include periodic re-training of models with the requested data deleted.
        
        \item \textbf{Third-party data transfers} should only occur between companies authorized to collect or own genetic data, with recipient companies subject to the same regulations. We recommend providing individuals with the option of blanket \textbf{opt-out} from data transfers and specific \textbf{opt-in} choices for each third party. Individuals must be notified of the transfer and given a reasonable time to opt-out. 
        %They can also request data deletion after the transfer, as the receiving entity is bound by the same regulations, including right to deletion.

        
        % should only occur between companies that are both authorized to collect or own genetic data. Additionally, recipient companies must be subject to the same regulations as the data collector. We recommend that any third-party data transfers for an active company include a \textbf{blanket opt-out} option for individuals to exclude themselves from brokered data, along with specific \textbf{opt-in} choices for each third party involved. Individuals must be preemptively notified of the transfer and given a reasonable time frame to opt out. Because the receiving entity is subject to the same rules and regulations that ensure the right to deletion at any time, individuals can still request their data be deleted after the transfer.









        \item \textbf{Research usage} is common among DTC companies, academia, and hospitals, where data provided per a test can be pooled together as a dataset for research (both internally and via external collaborations). We suggest an \textbf{opt-out} strategy here for individuals, with a specification on whether they consent to their data becoming publicly available in any form (e.g. public datasets, open sourced GWAS weights, trained models, etc).  
        
        \item \textbf{Incidental genetic discoveries} can be possible in research use cases where data for a particular test was used for other tests. In these cases, where knowledge of a particular trait can also be harmful and uninformative to the individual, we suggest an \textbf{opt-in} strategy for an individual to blanket choose if they wish to hear any secondary discoveries, or forward any secondary discoveries to their medical practitioner, to be enacted on at their discretion. 
        
    \end{itemize}

    
    \item \textbf{Bankruptcy and Acquisition}: As detailed by third-party data transfers, the owning of genetic data would also apply to companies who acquire any genetic data as part of assets through acquisitions or bankruptcy. Thus, all entities involved must already have approval to handle genetic data. In the rare event that there are no entities to handle the data, we suggest a similar protocol structure as nuclear waste \cite{doctorow_personal_data_2008}, where all physical sample, data, and models are subsequently destroyed and cannot be recovered.
\end{enumerate}

 % It is important to consider who will enforce such a regulatory framework, which could possibly be a new regulatory body or fall under the purview of pre-existing regulatory bodies such as the FDA, FTC, or HHS. We recommend that the regulation be implemented at the federal level to prevent \textit{leaky protections}. \sverror{this doesn't answer the question though of WHO would enforce it}
%, which can occur with state-based regulations, where individuals may lose or be uncertain about their protections depending on their state of residence.

