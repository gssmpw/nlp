\section{Risk Assessment Framework}
\label{sec:risks}

Our goal in this section is to develop a risk assessment framework for genetic data governance.
%Having established the background of genetic data, the genetic data collectors, and current protections—we next formalize the \textit{risks} that the existing system poses to the individual. 
Similar to the aims of the Blueprint for the AI Bill of Rights \cite{park2023ai}, we motivate our framework with three central questions: (1) What values (i.e., moral principles and civil liberties) should be preserved? (2) What are the vulnerabilities in the current system that can compromise these values? (3) What are the specific harms that can result from the vulnerabilities? 


\subsection{What values should be preserved?}

We seek to identify values that are future-proof and can be encoded in a system of genetic data collection. By examining the Four Pillars, six key values emerge:

\begin{enumerate}
    \item \textbf{Right to action}: The individual, and the individual only, has the choice to submit (and the freedom to not submit) their genome\footnote{We use ``genome'' to refer to a physical DNA samples of an individual and any sequencing/genotyping data.}.
    %they shall not be required to do so by any entity.
    \item \textbf{Ownership of the genome}: The individual owns their genome and therefore controls the usage of their genome, including \textit{who} has access to it and for what reason.
    \item \textbf{Right to privacy}: The individual has a right to privacy with regards to their genome and inferences made from their genome.
    \item \textbf{Right to knowledge}: The individual has a right to know or \textit{not} know about inferences made from their genome. 
    %An important implication of the genetic dragnet is that individuals may not have control over others (e.g., close relatives) inadvertently revealing insights about their genome. This is particularly important for individuals who do \textit{not} participate in genetic testing (or who selectively participate in certain tests, but not others). \frerror{Is this too complicated by bringing up the relational aspect?} \sverror{I think it is, especially in comparison with the previous shorter points}
    \item \textbf{Protecting opportunities for advancement}: Genetic data should not be used to deprive the individual of opportunities in any domain, including (but not limited to) education, access to financial tools, health insurance, housing, social services, and reproductive choices. 
    %Here, any discriminatory behavior that deprives opportunity should be protected against, recognizing that reproductive autonomy is fundamental to an individual's ability to make life choices and advance according to their own goals. \sverror{I'm not sure what this second sentence is adding}
    \item \textbf{Benefits of inclusion}: The Belmont principle states ``those who bear the burdens of research (i.e., those who are exposed to the discomforts, inconveniences, and risks) should receive the benefits in equal measure to the burdens''\cite{belmont_report}. We believe that this applies to individuals and their genetic data.\footnote{Variant Bio is a biotech company that collects genetic data from Indigenous groups from around the world and participates in revenue sharing with the communities they collect data from.}
\end{enumerate}

\subsection{What are the vulnerabilities in the current system that can compromise these values?}

The current genetic data ecosystem has many vulnerabilities that compromise the values listed above. These include:
\begin{enumerate}
    \item \textbf{Unsettled science}: The role of genetics in shaping complex traits is poorly understood. As a consequence, the deterministic nature of genetics may be overstated, particularly for behavioral and cognitive traits. 
    \item \textbf{Rapid evolution of genetic data/methods}: As genetic data collection, sequencing, and methodologies quickly evolve, legal protections fall behind. For example, education discrimination is not included in GINA, which was passed five years before the first large GWAS on educational attainment \cite{rietveld_gwas_2013}.
    %Many individuals can be negatively affected because of lack of scientific clarity (e.g. predicting IQ through genetic tests is often confounded by lack of causal confidence and environmental socioeconomic factors \cite{selzam_s_comparing_2019}). 
    \item \textbf{Guilt by association}: DNA databases used for criminal investigations can impact not only the individuals whose DNA is stored, but also their biological relatives, as genetic information is shared among family members. 
  %  \item \textbf{Third-party data collectors}: \rverror{This one doesn't seem clear cut; and overlaps with many others -- do we lose anything but cutting it out altogether? or how do we make it distinct?} there are companies that allow users to upload their genetic data (that they have downloaded from 23andMe or AncestryDNA) for free and in exchange users receive additional relative matches (e.g., GEDMatch) or disease risk reports (e.g., Promethease). The barrier to entry in order to start one of these companies is low—e.g., Promethease reports polygenic risk scores using publicly available GWAS weights—which may result in lower trust. GEDMatch\footnote{GEDMatch also had an option allowing users to `opt-in' to sharing data with law enforcement, but in 2019 a Florida judge overrode user preferences and granted a detective full access to the database \cite{hill_your_2019}.}, which collected genetic data from 1.2M individuals, was acquired by genetic forensics company Verogen, which was then acquired by the German company Qiagen.
 
    \item \textbf{Geographical legislative patchwork}: Genetic anti-discrimination and privacy legislation widely differs between states. Legal policy in each state for data removal, third-party sharing, and many other aspects of genetic data are unclear. Individuals who move between states or share data with entities in other states can be affected by this dependency on state specific policy. Aside from patchwork policy, many state policies do not sufficiently protect genetic data specifically. 
%    \item \textbf{Security practices}:
%    \frerror{Is there a term to refer to individual versus entity security practices?}
 %   Individual and entity security practices can control whether malicious actors can gain access to their data (e.g. 23andMe hack where individual security passwords were used to access related profiles and their identifying data; see Section \ref{case:dtc}). 
\end{enumerate}


\subsection{What are the harms that arise as a result?}
Lastly, we will outline several consequences that could arise from violations of our key values through the above vulnerabilities. These downstream effects, which we refer to as harms, can affect an individual and their immediate genetic relatives and 
%and their personal data footprint,
can also have a broader effect on communities and the population at large.
%the population level scale \cwerror{Don't like this phrasing}. 

\begin{enumerate}
    \item \textbf{Leakage to the family:} The genetic dragnet means that any conclusion about an individual's genetic trait can be linked back to their relatives, even if no action is taken by their relatives in either (1) submitting data or (2) opting in to receive information. Any derived secondhand knowledge can immediately impact insurance (health, life, etc.) and medical treatment plans, as well as compromise identity.
    %, even though the original determination was not made on genetic relatives. The relational nature of genetic data and its leaky quality thus compromise both the privacy and personal liberties of family members. Not only can the relational nature of genetics affect immediate known relatives, but decisions also impact future children, disease carriers, and preventative care. 

    \item \textbf{Loss of anonymity: } It is straightforward to identify some traits about an individual from their genetic data, and subsequently from their genetic relatives. Common examples include: race, gender, ethnicity, and markers for certain diseases. Leaky genetic data interfaces can thus compromise information about an individual, their genetic relatives, and unborn children that the individual (and implicated relatives) may wish to maintain as private. 
    %Exposure can lead to future vulnerability such as affecting opportunities of advancement, discrimination in availing certain insurance policies or services, and extortion or targeted attacks -- all of which compromising a critical value of privacy and personal liberties. 
    As the refinement of reference datasets continues and data collection grows\footnote{It has been estimated that, in a sample of three million European-descent Americans, 99\% of individuals would have a least one third cousin in the dataset \cite{erlich_identity_2018}.}, anonymity becomes even more elusive.
    
    \item \textbf{Loss of Data Control: } 
    % Private institutions that sell genetic tests own the end-to-end process of how individuals' data is collected, stored, and monetized. Data transfer policies remain a black box as it is unclear which deals an individual's data may be included in, how the data is monetized, and which parties benefit from these trades. Even if users opt-out of participation, ungoverned data flows mean that their genetic data footprint will exist in perpetuity. The lack of transparency in the event of bankruptcy or acquisition has the same implication: In 2024, Tempus AI acquired Ambry Genetics, which sequences 400,000 patients annually, with the stated intent to "leverage this data and augment Tempus' current data offering" \cite{tempus2025}. Consequently, any decision of an individual to submit data for a genetic test implicitly includes loss of complete ownership over their genetic data, and any leaky interfaces can compromise their privacy without the appropriate safeguards.\sverror{I don't understand how the tempus example is relevant}
    Private institutions offering genetic tests control the entire lifecycle of individuals' data from collection and storage to monetization. This control creates significant transparency issues, as it is often unclear how data is monetized, which parties benefit, or what agreements include an individual’s data. Users may opt out of participation, but unregulated data flows ensure that their genetic footprint persists indefinitely. This lack of transparency is particularly concerning in cases of bankruptcy or acquisition. For example, in 2024, Tempus AI acquired Ambry Genetics, a company that sequences genetic data for 400,000 patients annually \cite{tempus2025}. Tempus stated its intent to "leverage this data and augment Tempus' current data offering," raising questions about how patient data is repurposed and monetized in such deals. Ultimately, an individual’s decision to submit genetic data for testing often entails relinquishing full ownership of their genetic information. Without robust safeguards, privacy risks emerge from opaque data practices and leaky interfaces, leaving individuals vulnerable to misuse or unauthorized access to their genetic data. 
    
    \item \textbf{Misinformed Actions: } The interpretation of genetic test results and the subsequent actions taken by individuals are profoundly personal, often influencing lifestyle changes, significant financial decisions, and critical medical choices. For instance, individuals with BRCA1 gene mutations indicating an elevated breast cancer risk may opt for preventive interventions. These decisions are particularly sensitive as the interpretation of genetic data evolves alongside advancements in scientific methods. However, the relational nature of genetic data complicates individual autonomy and access to information. It is possible for someone to receive conclusions about themselves indirectly through the test results of genetic relatives. This dynamic, combined with the widespread availability of unregulated genetic tests and the evolving nature of genetic science, increases the potential for harm if individuals act on information that is incomplete, inaccurate, or probabilistic rather than deterministic.
    % The interpretation of genetic test results and the actions taken by users are deeply personal. \textcolor{blue}{These results can drive significant lifestyle changes, major financial decisions, and critical medical choices, such as pursuing aggressive interventions when BRCA1 gene mutations indicate elevated breast cancer risk} \sverror{someone make this sound better :)} \cwerror{better?}. Such decisions are particularly crucial, given that our understanding and interpretation of genetic test results continue to evolve alongside advancements in scientific inference methods. However, the inherently relational nature of genetic data poses challenges to an individual’s autonomy and access to information, as it is possible for someone to receive secondhand conclusions about themselves based on tests results of their genetic relatives. \textcolor{blue}{This characteristic, coupled with evolving science and mass availability of unregulated genetic tests, creates the risk of significant harm if individuals act on incorrect, incomplete, or non-deterministic information.} 
  


    \item \textbf{Financial impact}: Individuals may face financial repercussions if insurance or legal policies fail to provide adequate protection or if coverage is denied. Beyond medical procedures, financial impacts can also arise on a case-by-case basis, such as in situations where genetic data is held for ransom, legal defense is required, or individuals are forced to purchase direct-to-consumer tests (e.g., in order to purchase a life insurance policy).
\end{enumerate}