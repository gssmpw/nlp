\section{Conclusions and  Open Challenges}
\label{section:challenges}

In this paper we outline the genetic data ecosystem and today's state of public protections. We identify critical gaps in current regulatory federal and state-level frameworks that prevent the productive, ethical, and safe adoption of genetic data for broad societal applications. To address this, we propose a novel risk assessment framework that offers key public protections and preserve's individuals personal liberties, and finally make three concrete policy recommendations. 
% While our policy recommendations account for improved user protections, we identify three remaining open challenges in regulating genetic data use, that lie beyond the scope of our work: First, our recommendations only cover \textbf{voluntary} genetic data collections.  In the case of involuntary genetic data collection, we reference Georgetown’s Center of Privacy and Technology (CPT) report and their accompanying suggestions \cite{glaberson_raiding_2024}. Second, the \textbf{relational} aspect of genetic data considerably implicates any biological relatives of individuals in a dataset. A relational theory of privacy is a complex phenomenon that remains highly discussed in academic literature \cite{viljoen_relational_2024, costello_genetic_2022, zigomitros_survey_2020, reviglio_i_2020}.  As such, we envision a future regulatory framework that would encompass the relational aspect of genetic data, of which we highlight three possible considerations/applications:

% These questions remain largely unanswered and require contributions from legal scholars, bioethicists, moral philosophers, privacy experts, human rights advocates, and public policy researchers. We raise these questions for the purpose of thoroughness in our evaluation of the use of genetic data in today's ecosystem.

While our policy recommendations focus on enhancing user protections, we also identify three key challenges in regulating genetic data usage that fall outside the scope of our work and that would benefit from further research.
%We present these challenges to ensure a comprehensive evaluation of genetic data use in the current ecosystem.

\subsection{The relational theory of privacy.}

\subsubsection{Should genetic data submissions or inclusion into genetic datasets require consent from biological relatives?}  %The framework should address issues regarding competing privacy interests between relatives. For example, w
When an individual does a genetic test they are, indirectly, testing (part of) their relative's genome.  What (if any) consent or privacy measures should be taken to protect their biological relatives who may not consent to such a test? Additionally, what should happen upon death? Do children ``own'' their parents' genome (and what ownership would, say, a full sibling have?) and can they make decisions (e.g., to delete the data)?

\subsubsection{Who has ownership of children’s DNA\@?}
Do parents have a ``fiduciary duty'' to protect their child's genetic data and protect them from the system vulnerabilities that we have outlined? Part of this fiduciary duty would involve whether a genetic test should be taken in the first place, e.g., the difference between a medically-necessary test and a DTC genetic test\footnote{23andMe's policy for minors is as follows: individuals under 18 that give assent and have a parent/guardian's permission can submit their sample \cite{23andme_research_2024}.} for some insights into the child's ancestry and genealogical relatives. 
%Return of results to the child (e.g., by their parents) may come with harm. The child may not understand the statistical nuances of genetic tests for complex traits, resulting in a harmful deterministic view of their future health and social outcomes. 
The relational aspect of genetic data also comes into play here; for example, should consent be required from both parents to sequence their child's genome? See \cite{bala_who_2023} for a thorough discussion of child's genome ownership. 

%By sequencing the child's genome, half of the parents' genomes are also sequenced. Companies or clinical services that sequence children or embryos (e.g. Orchid genetics performs whole genome sequencing on embryos for disorders and predispositions \cite{OrchidHealth2025}) call into question who owns this DNA, especially after the sequenced individuals become adults.
    % move to open challenges; remove biosecure

\subsection{International genetic data transfers.}
 Our paper focuses on American regulatory and legislative bodies and American companies. In the DTC genetic testing space, individuals may be less likely to trust foreign companies: Nucleus Genomics\footnote{https://mynucleus.com} proudly states on their home page that customer genomes are sequenced in the U.S. and the data is never sent/stored overseas\footnote{23andMe also genotypes samples in the U.S. \cite{23andme_what_2024}, but this is not a selling point of theirs.}. Should there be restrictions on genetic data transfer between countries? For example, one hypothetical scenario might involve a political refugee who has medically-necessary genetic testing done within the U.S, but because of leaky data transfer rules, genetic information is transferred to their home country where it might be used to identify family members now at risk for persecution.


\subsection{The deployment of AI models for genetic data.}
\label{challenges:ai}

Our paper has largely shied away from discussions about genetic inference and AI for two reasons: (1) deployment of AI technologies for genetic inference is nascent and, most importantly, (2) risks and vulnerabilities exist even with current simple linear models for genetic inference. AI will, we believe, exacerbate the risks we have outlined as they will likely result in seemingly better trait prediction by modeling nonlinear interactions and taking advantage of gene-environment correlations\cite{yang_advancing_2024}. However, the uncertainties associated with the basic scientific inference (that we outline in Section~\ref{section:genetic_data}) will be compounded because of the lack of interpretability and complexity of AI models. 

%Large language models that incorporate genetic data and other data modalities have already demonstrated improved accuracy, e.g., Google's Med-Gemini-Polygenic \cite{yang_advancing_2024}. As we described in Section 2, even linear models suffer from interpretability, with statistical geneticists unable to disentangle genetic and environmental confounders, meaning there is not a clear understanding as to what is \textit{biological}; this task would be made more difficult if AI technologies were deployed.