\section{Background: Current Protections}
\label{section:protections}


\begin{figure}[H]
    \centering
    \includegraphics[width=0.6\textwidth]{figures/final_four_pillars_v6.png}
    \caption{\textbf{Four Pillars of genetic data collection and their regulatory considerations.} Data collectors within each pillar operate with different collection strategies and intent to store, use, and transfer the data. However, there is only a patchwork of federal and state-level legislation that governs each Pillar. The resulting regulatory gaps lead to \textit{leaky protections} through which genetic data—or knowledge derived from genetic data—can be used in other pillars with little to no oversight. Real-world examples\protect\footnotemark{} of leakage are indicated by red arrows.}
    \label{fig:4_pillars}
\end{figure}
\footnotetext{Orchid Health is a private company that uses publicly available GWAS weights to compute polygenic risk scores (PGS) of embryos. Nucleus Genomics is also a company that similarly uses whole genome sequencing to compute PGS to predict health and behavioral traits.}




In this section, we outline the foundation of today's genetic data collectors and the current regulatory and legislative environments that they are subject to. To better understand the vast landscape over which genetic data is being collected, we build upon the Four Pillars \cite{wan_sociotechnical_2022} of genetic data collectors. We categorize these data collectors by their motivations for collecting genetic data and their relationships with the individuals from whom the data is collected. We define the updated Four Pillars as the following:
\begin{itemize}
    \item \textbf{Forensics, Tracking, Surveillance:} Entities focused on monitoring, identifying, and analyzing genetic data to support public health outcomes, security, and law enforcement objectives.
    \item \textbf{Direct to Consumer (DTC):} Entities developing products to provide consumers with direct access to genetic insights.
    \item \textbf{Clinical:} Entities requiring medical professionals' oversight to conduct tests, often enabling insurance coverage and generating clinically actionable insights.
    \item \textbf{Research:} Entities engaging in research and development (R\&D) and basic research across private, public, and academic sectors. This includes public consortia, biobanks, and collaborations with DTC entities.
\end{itemize}

The ``genetic data ecosystem'' encompasses not only the entities within the Four Pillars but also the broader network of legislatures, regulators, and individuals involved in or affected by genetic data collection and use. By outlining this ecosystem, we aim to shed light on the regulatory gaps that enable genetic data leakage across pillars. We provide real-world examples of this leakage in Figure \ref{fig:4_pillars}.

%after an examination of federal legislation, policy, and regulatory bodies, that attempt to provide baseline protections across the United States. We find that federal data protections for individuals are limited, and that a patchwork of state-level legislation creates regulatory gaps, allowing private companies handling genetic data to discriminate based on jurisdiction and geography. We thus determine that today's state of protections is \emph{leaky}, and finally identify areas in urgent need of reform.

%%%%%%%

\subsection{Federal protections}

\subsubsection{Genetic Information Non-Discrimination Act (GINA)}
GINA was enacted in 2008 and prohibits genetic discrimination by employers or health insurers. Specifically, GINA bars employers from making hiring, firing, or promotion decisions based on genetic information (and information disclosed about relatives) and bans health insurance providers from denying coverage or charging higher premiums based on genetic information including genetic predisposition for disease \cite{gina_2008}. GINA does not apply to life, disability, or long-term care insurance \cite{green_strategic_2020} or any other venues outside of employment and health insurance (e.g., education). The US Equal Employment Opportunity Commission (EEOC) regulates workplace discrimination for employees and thus, is responsible for enforcing GINA in the workplace. Enforcement for health insurers falls under the purview of several federal agencies, including the Department of Health and Human Services (HHS); Department of Labor; Centers for Medicare and Medicaid Services; and the Department of the Treasury \cite{protections_ohrp_federal_2009}. 


\subsubsection{Health Information Portability and Accountability Act (HIPAA)} HIPAA, signed into law in 1996, grants security and privacy protections for patient health information (PHI). 
%and Genetic Information Nondiscrimination Act (GINA). HIPAA was passed by Congress and signed into law in 1996. 
HIPAA applies to covered entities that maintain this information, such as healthcare and health insurance companies, but does not apply to many other entities that might retain health information, including search engines, medical information sites, or dating sites \cite{citron_new_2020}. Importantly, HIPAA does not apply to two of the largest DTC genetic testing companies 23andMe and AncestryDNA \cite{sklar_be_2020}. HIPAA does not cover deidentified data: PHI data with unique identifying information (names, address, phone numbers, dates, etc.) removed do not fall under HIPAA. Research consortia, biobanks, and research collaborations with DTC and clinical entities primarily use genetic data that is considered deidentified, and so HIPAA generally does not apply. However, the relational aspect of genetic data has motivated calls to ``consider genomic data as, in principle, always identifiable'' \cite{bonomi_privacy_2020}. The inability to completely strip data of \emph{all} identifiable features introduces critical vulnerabilities for genome owners and exacerbates the risk of data leakage across the Four Pillars.


\subsubsection{Food and Drug Administration (FDA)}
The FDA ``is responsible for protecting the public health by ensuring the safety, efficacy, and security of human and veterinary drugs, biological products, and medical devices'' \cite{commissioner_what_2023}. In the context of DTC genetic testing companies, the FDA only regulates well-defined high risk medical tests with clinical actions, such as 23andMe's health tests for diseases like Parkinson's, breast cancer, and late-onset Alzheimer's. If the FDA approves a specific test, Centers for Medicare and Medicaid Services then choose to approve the test for insurance coverage \cite{daval_authority_2023}. Most genetic tests are not covered by Medicare, particularly DTC tests. 

\subsubsection{Federal Trade Commission (FTC)}
The FTC enforces federal laws that protect consumers from ``fraud, deception and unfair business practices'' \cite{federal_trade_commission_coop_nodate}. The FTC has warned DTC genetic testing companies that the results they return to customers must be backed by ``reliable science'', and also warns companies against making exaggerated claims about the use of AI in their products \cite{jillson_dna_2024}. Additionally, the FTC remains vigilant of deceptive practices regarding privacy policy changes and ``dark patterns'' designed to coerce consumers into consenting to data sharing\footnote{In 2023, the Commission fined 1Health.io/Vitagene \$75,000, claiming they ``left sensitive genetic and health data unsecured, deceived consumers about their ability to get their data deletes, and changed its privacy policy retroactively without adequately notifying and obtaining consent'' \cite{ftc_1health_2023}.}. The power and broad scope of the FTC—and their stated intentions of cracking down on DTC genetic testing companies \cite{jillson_dna_2024}—make it a major player in the regulatory space, and it will likely grow in importance if the federal government prioritizes genetic data privacy. 

\subsubsection{The Common Rule}
This is a federal policy that protects human subjects in research settings \cite{sciences_federal_2014}. It was first codified in 1981 by the Department of Health, Education, and Welfare (now HHS) but adopted more widely by other federal agencies in 1991. The Common Rule institutionalizes protections for vulnerable participants, protections overseen by Institutional Review Boards (IRBs); IRBs are groups ``formally designated to review and monitor biomedical research involving human subjects'' \cite{research_institutional_2024}. Revisions enacted in 2017 particularly focus on informed consent for participants, but do not cover deidentified data \cite{menikoff_common_2017, protections_ohrp_federal_2009}. IRBs ensure that consistent protection of human subjects are maintained and are necessary for research to begin and proceed if involving human participants. The maintenance of the Common Rule and HIPAA, in the case of medical data, is performed by IRBs.
\subsection{State protections}

\begin{figure}[ht]
    \centering
    \includegraphics[width=\textwidth]{figures/legislation_map.png}
    \caption{\textbf{Beyond HIPAA and GINA, state-level genetic discrimination and privacy legislation varies widely in the United States}. \textbf{Left:} States with genetic anti-discrimination legislation, distinguishing between states with both health insurance and employment protections (red) and those with health insurance protections only (orange). \textbf{Right}: states with general privacy laws that include genetic data (blue) and genetic privacy-specific laws (green).}
    \label{fig:genetic_legislation}
\end{figure}

State laws offer far more protections than their federal counterparts, but a patchwork of these laws means a patchwork in protections, as shown in Figure \ref{fig:genetic_legislation}. Here, we focus on two categories of laws: genetic anti-discrimination laws and data privacy laws.

\subsubsection{Genetic Anti-Discrimination Laws}
Most states have passed some version of a genetic anti-discrimination law (initially prompted by the passage of GINA) although the sector, scope, and strength of the laws differs between states. 
%Genetic anti-discrimination laws were, generally, passed around the time of GINA and typically offer similar protections (i.e., outlawing genetic discrimination in employment or health insurance). 
Some states offer fewer protections than GINA: Georgia, for example, has a statute outlawing genetic discrimination by health insurers, but not employers. Other states offer \textit{more} protections than GINA. California's CalGINA \cite{california_senate_bill_559_2011} outlaws genetic discrimination in several sectors: employment, health insurance, housing, and by any state agency or entity receiving state funding (such as emergency medical services). Florida includes life, long-term care, and disability insurers \cite{florida_house_bill_1189_2020}. Many states (including Massachusetts and New Mexico) that include life insurance in their non-discrimination laws exempt companies if genetic data reliably, and based on ``sound actuarial principles'' \cite{massachusetts_genetic_law_2025}, gives information relating to mortality or morbidity (see Section \ref{case:insurance} for a detailed case study).

\subsubsection{Genetic Data Privacy Laws}

Many states have genetic data privacy laws, either as a standalone law or as part of a broader data privacy law. These laws\footnote{Some of these laws have also led to a change in company services in that state. For example, the passing of Illinois' BIPA (Biometric Information Protection Act) resulted in Amazon and Google removing certain services of Alexa and Google Photos respectively after lawsuits which claimed that the company violated BIPA.} were generally passed after GINA with the rise of DTC genetic testing companies. Standalone genetic data privacy laws tend to be written specifically to cover DTC genetic testing companies and are often extensions of genetic non-discrimination laws passed around the time of GINA.

Both data privacy and genetic privacy laws have similar provisions \cite{nhgri_genome_database_2025}, including a customer right to request deletion of genetic data and samples and the necessity of consent for data transfers. Some states allow consumers to bring lawsuits for violations of the privacy law (a private right of action). In general though, privacy laws do not apply to deidentified genetic data (as per HIPAA). An important contrast to the US federal and state bills on privacy is the EU General Data Protection Regulation (GDPR). GDPR not only regulates all personal data within the EU but also the transfer of data outside of the EU, data minimization, and compliance. Genetic data is labeled as ``special category data'' in GDPR as it has many identifiers associated with it. 

% See Appendix \ref{appendix:dtc_laws} for more discussion on provisions for DTC genetic data use.


% \sverror{and then continue here} \rverror{We moved the previous paragraph to the appendix.}
% Some of these laws have also led to a change in company services in that state. For example, the passing of Illinois' BIPA (Biometric Information Protection Act) resulted in Amazon and Google removing certain services of Alexa and Google Photos respectively after lawsuits which claimed that the company violated BIPA.  
