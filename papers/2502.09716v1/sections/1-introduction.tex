\section{Introduction}

% questions we want to answer
% structure of the paper
% main takeaways


% first para
% historically
% in light of all of this 
% here's our takeway
% here's our contributions

% superexponential

The scale of genetic data generation has increased at a staggering rate since 2001 \cite{bick_genomic_2024, bycroft_uk_2018, loos_15_2020}. A primary driver of this is decreased DNA sequencing costs \cite{wetterstrand_dna_2023}, which have enabled the curation of large genetic datasets for personal and public health, ancestry inference, relative identification and advanced forensics. For example, there are whole genome sequences combined with medical records from approximately 400,000 US residents in the National Institutes of Health (NIH) All of Us Research Study, with plans to increase sampling to include at least 1 million US residents \cite{nih_all_2024}. Direct-to-consumer (DTC) genetic testing companies have much more data: 23andMe has 14 million customer sequences \cite{23andme_reports_2023} and Ancestry has 25 million \cite{ancestry_company_facts_2025}. The increase in genetic data and improvements in technologies has resulted in the following aspirational ``bold prediction'' for 2030 by the NIH: ``a person’s complete genome sequence along with informative annotations can be securely and readily accessible on their smartphone'' \cite{green_strategic_2020}. 
From a scientific perspective, the collection of genetic data at this scale has been critically valuable to understanding health, ancestry, and driving public health initiatives.


The propensity for genetic discrimination grows with poor protections for privacy. Since 2020, the U.S. Department of Homeland Security has accelerated its collection of genetic data, growing the CODIS database by more than 1.5 million people, the majority of whom are people of color \cite{glaberson_raiding_2024}. 23andMe—which boasts more than 14 million customers \cite{23andme_reports_2023}—suffered a data breach after a hacker accessed 14,000 customer profiles\footnote{While genetic data was not accessed, ancestry information and relative names were\cite{23andme_addressing_2023}}, primarily targeting individuals of Chinese and Ashkenazi Jewish descent \cite{carballo_23andme_2024}. 
%Additionally, the financial woes of the company have left customers wondering what will happen to their genetic data in the event of bankruptcy or acquisition \cite{allyn_23andme_2024}. 
These pressing, real-world discrimination and privacy concerns are exacerbated by the sordid history of eugenics and forced sterilization in U.S. states in the 20th century (a model later adopted by Nazi Germany \cite{spiegel_jeremiah_2019}), and more recent practices of
%fact that the state of public protections is weak. \rverror{moved the jarring para here: The unchecked integration of genetic data in society raises serious concerns in light of genetic theory's role in the eugenics movement of the 20th century. The idea of genetic determinism was used to successfully advocate for forced sterilizations in 27 U.S. states  Since the 1960s, this movement has transitioned into modern practices of 
genetic determinism \cite{epstein_is_2003}, racial pseudoscience \cite{lala_genes_2024, duello_race_2021, harmon_geneticists_2018, panofsky_how_2021}, race-motivated violence \cite{carlson_counter_2022}, privacy and consent violations \cite{strand_shedding_2016, sterling_genetic_2011, jillson_dna_2024}, and broad scientific distrust \cite{panacer_ethical_2023, kaye_tension_2012, saulsberry_need_2013}. These are serious concerns that must be addressed today given the number of recent proposals for genetic data applications in public infrastructure, (e.g., to assess intelligence proxies, like IQ \cite{r_new_2018}; insurance premiums \cite{karlsson_linner_genetic_2022}; or educational outcomes \cite{harden_genetic_2020}). The only federal-level legislation currently in effect is the Genetic Information Non-Discrimination Act (GINA) of 2008. GINA prohibits the use of genetic information in employment decisions and discrimination based on genetic information in health insurance coverage but does not 
%GINA is noted for being the United States' only preemptive genetic anti-discrimination act, and for the overwhelming bipartisan support (Senate: 95-0, House: 414-1 \cite{senate_vote_113_2008}) when it passed. 
%However, it does not 
prohibit genetic discrimination for long-term care, disability, or life insurance. 
%More protections for the public are thus critical in today's age of genetic data collection.
%\sverror{these two paras above seem a little jarring together. I'd suggest that you start with the second para and insert the concerns around eugenics inside it. I can make the edit.} \rverror{See above}

\paragraph{Our Work.} In this paper, we make three contributions towards the construction of a robust genetic data governance system. First, we expand upon Wan et al.'s \cite{wan_sociotechnical_2022} Four Pillars of genetic data collectors and make the argument that inconsistent and \emph{leaky} regulation across these pillars introduces opportunity for genetic discrimination and privacy violations. Second, we outline a risk assessment framework for modern-day genetic data governance and outline key values it should preserve. Third, we make three concrete recommendations to: (1) legally redefine genetic data; (2) expand GINA; and (3) create a regulatory framework for genetic data collectors. 
%, our paper seeks to identify the vulnerabilities of existing genetic data governance and make policy recommendations to address them. 


%To ensure that the scientific and social benefits of genetic data use are fully realized, we posit that a comprehensive genetic data governance system should be devised from answers to the following questions. 

%\begin{enumerate}
%    \item What values, in an era of widespread genetic testing, should a genetic data governance system seek to preserve?
%    \item What policy guardrails must we put in place to ensure that genetic inference technology serves rather than constrains human potential?
%    \item As we rush to integrate genetic predictions into critical societal decisions, are we adequately accounting for the statistical limitations and demographic biases inherent in these technologies?
%\end{enumerate}


%We answer these questions through the contributions of our work: 

Our paper is structured as follows: In Section \ref{section:genetic_data}, we provide an overview of genetic data and inference, with a focus on the nuances of genetic trait prediction. In Section \ref{section:protections}, we introduce the extended Four Pillars and describe the existing legislative and regulatory structure governing genetic data use. In Section \ref{sec:risks}, we outline our risk assessment framework and apply it to four case studies in Section \ref{sec:studies}. We give our three policy recommendations in Section \ref{section:recommendations} and conclude, in Section \ref{section:challenges}, with three open challenges that are beyond the scope of this paper. 
