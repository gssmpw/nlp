\section{How is meaning derived from genetic data?}
\label{section:genetic_data}
\frerror{Sohini's review}
In this section, we provide a brief overview of genetic data and the nuances of genetic inference. An individual's genome is over 3 billion DNA bases long and derived equally from both genetic parents. This data contains information about one's ancestry, genetic disease risk and other genetically influenced traits (from anthropometric traits to behavioral traits). This information is sprinkled throughout the genome, and it is the goal of geneticists to extract this information and transform it into useful quantities (e.g., disease risk, ancestry proportions). However, this remains a challenging task: while some traits are determined by a single gene (e.g., Cystic Fibrosis or Huntington's Disease), many are a result of the interplay of multiple small DNA sequence fragments (\emph{variants}) in various combinations, but have no effect in isolation. 


\subsection{Your genetic data is shaped by your ancestors and shared with your relatives.}


A Reddit user aptly claimed, ``Your genetic data is only as secure as your relatives' passwords'' \rverror{reddit claim needs to go to DTC case study}: The inheritance of DNA—whereby each parent passes down a random half of their own genome to their offspring—means that relatives share DNA (and that the amount of DNA shared is higher among closer relatives). Consequently, a decision made about any one genome has implications for all of that individual's genetic relatives. This relational characteristic of genetic data complicates discussions of privacy, as formalized in \cite{costello_genetic_2022}. We use the term \textit{genetic dragnet} to describe the relational aspect of genetic data and the idea that individual decisions about their own data cast a wide dragnet over their genetic relatives. The power of the relational nature of genetic data is what enabled the identification of the Golden State Killer in 2018\footnote{Authorities used DNA left at a crime scene to identify the Golden State Killer 3 decades later. They found distant relatives through DNA matching on a website called GEDMatch and used publicly-available and user-uploaded genealogical records to identify him \cite{kaiser_we_2018}. This was possible despite the fact that the closest relatives they could identify shared a great-great-great-great grandfather with the suspect \cite{zabel_killer_2019}}.

At the same time, an individual's genome is unique to them (even identical twins' genomes have differences \cite{ormond_whole_2024}). While an individual's genome, by itself, is not identifiable, the relational aspect of genetic data can allow for re-identification \cite{shabani_reidentifiability_2019, bonomi_privacy_2020}, with some calling to ``consider genomic data as, in principle, always identifiable'' \cite{bonomi_privacy_2020}. \rverror{Can we move the following line to some other section?} We similarly take the conservative approach of \cite{bonomi_privacy_2020} and argue that genetic data is personally identifiable information (PII) because identifiability \textit{can} occur indirectly through, for example, relative matching. 
\sverror{While we do claim this, it's confusing to bring in a contribution in what is supposed to be a "science of genetics" background.}
% \subsection{Genetic data is used to make inferences about groups of people}
\subsection{Geneticists use statistical models for trait prediction.}

Inference using genetic data requires some knowledge of a variant's effect on a trait. This can sometimes be tested in animals by designing experiments to ask, what is the effect on the trait if we change the variant? However, this approach is infeasible and unethical to pursue humans, and so researchers instead rely on large genetic datasets to ask the question, do individuals who have the variant \textit{tend} to have the disease (or trait of interest) compared to those who do not have the variant? To answer this question, researchers typically conduct genome-wide association studies (GWAS), 
which identify genetic variants that are associated with a trait, and quantify the effect of each variant on the trait. The effect sizes of variants—also called \textit{weights}—are directly interpretable, e.g., in a GWAS for height, an effect size of 0.01 would mean that the variant increases height by an average of 0.01cm.

Trait or disease risk prediction is done using a polygenic score (PGS). 

\cwerror{Which explanation do we like better?}

1. For an individual we know \textit{which} variants are in their genome (from sequencing) and the effect of those variants (from the GWAS). A PGS is simply a sum of the effect sizes of the variants in an individual's genome.

2. A PGS is a weighted sum of an individual's genome, which is represented as an array of 0s, 1s, and 2s; the weights are the effect sizes estimated by the GWAS.

PGS remains the gold standard in trait prediction, as geneticists rely on the interpretibility of genetic features that is offered by linear models to identify direct variant-trait associations (see Section \ref{challenges:ai} for more discussion on frontier models for genetic data)\cite{fritzsche_ethical_2023}.



\textcolor{blue}{I think this paragraph repeats 3.3 -- so either we remove it from here, or combine the 2 and refine in 3.3 (I think the latter makes more sense.} \rverror{In simple terms, to interpret a genome, we need a reference or ground truth.} Reference genetic datasets, analogous to the training dataset in machine learning, are thus key to genetic analyses. These datasets contain genetic data and their ``ground truth'' annotations of trait values, whether it be ancestry, disease status, height, etc. The composition of the reference dataset—\emph{who} is in the dataset and how their data is labeled\footnote{`True'' ancestry has been arbitrarily (and implicitly) set by the field to correspond to: where your ancestors lived 600 years ago \cite{coop_genetic_2023}. Individuals in the same ancestry group are distantly related in the same timescale (i.e., likely many ancestors in common dating back approximately 600 years ago). }—thus affects analysis. For example, a Londoner could be encoded as English, British, Northwestern European, or European.



% \subsection{Genetic data is collected in \textit{cohorts}}
\subsection{Insight from genetic data is a function of the cohort.}

A group of individuals in a particular genetic study is referred to as a ``cohort''. When a cohort is used as a reference, \sverror{if you're bringing up reference sets here, why are you talking about reference sets in the previous subsection?} it is essential to understand and communicate \textit{who} is included, as the composition of the cohort can introduce biases that impact downstream analyses. For example, the average participant in the UK Biobank (which contains genetic and medical data from 500,000 UK residents aged 40-69 at the time of recruitment) tends to be more educated and healthier than the average Briton \cite{fry_comparison_2017}. The group-level ancestry composition of the cohort (i.e., the distant genetic relatedness of the cohort; analogous to the training data distribution) can introduce biases.



In addition to group composition, non-genetic factors such as environmental exposures and social determinants of health can also introduce biases.``Cohort matching'' refers to the similarity between a test sample and the reference cohort, including factors like ancestry, age, geography, and social/environmental influences, and is a key component of genetic inference studies. For example, \cite{mostafavi_variable_2020} find that genetic data prediction accuracy varies within an ancestrally-homogeneous white British cohort. Similarly, a BMI GWAS in a younger cohort, has higher prediction accuracy for younger individuals compared to older ones \rverror{cite}.
