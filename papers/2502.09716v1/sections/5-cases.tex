\section{Case Studies}

\label{sec:studies}

We apply our risk assessment framework to four case studies which highlight regulatory gaps across the Four Pillars. These case studies, ranging from past events to speculative scenarios, all involve privacy loss or potential discrimination. We choose case studies that illustrate genetic data leakage in both, a pillar-to-pillar and one-to-many pillar settings.

% While some align with specific Pillars, others span multiple Pillars due to the leaky protections in which genetic data, initially collected within one Pillar, can be used across others.

\subsection{Genetics and Education}
\label{case:education}

Polygenic scores (PGS) are popular in the social/behavioral sciences for predicting social outcome traits, such as educational attainment (EA; number of schooling years completed by an adult). We will use EA as the example throughout this case study, but note that there is also interest in predicting standardized testing scores, performance in mathematics, and other traits with substantial environmental influences. 


A common metric for assessing PGS accuracy is the percentage of trait variance it explains: higher percentages indicate better predictive performance. A recent 23andMe EA PGS, based on data from over 3 million customers of European descent, explains 12-16\% of the variance \footnote{As a useful comparison, mother's education explains 15\% of the variance in EA \cite{lee_gene_2018}} in educational attainment (EA) \cite{okbay_polygenic_2022}. This effectively means that 50-70\% of individuals with PGS scores in the top 10\% for EA are predicted to graduate college. However, the PGS accuracy significantly decreases when applied to African American customers. This is an example of the commonly observed ``portability problem'' \cite{martin_human_2017}, where a PGS derived from GWAS in one population predicts poorly in another due to confounding\footnote{Confounding in GWAS can be genetic (non-causal variants correlated with causal ones) or environmental (non-causal variants correlated with causal environmental factors), leading to potential statistical artifacts in effect sizes.}.


There have been several calls to use EA PGS to inform education policy. Harden et al. propose the use of math-performance PGS to identify ``leaks'' in the education system: for example, by identifying high math PGS students who perform poorly, they claim educators could pinpoint \textit{why} and \textit{how} students are failing to reach their potential\footnote{An example they use: 31\% of high PGS students in good schools take calculus, compared to 24\% of the same-scoring students in poor-performing schools.} \cite{harden_genetic_2020}. Plomin \& von Stumm take it further: they use the term ``precision education'' (akin to precision medicine) to propose a tailor-made, individualized education that is genetics-informed \cite{r_new_2018}. Statements like this, combined with statements such as ``students with higher polygenic scores for years of education have, on average, higher cognitive ability, better grades and come from families with higher SES [socioeconomic status]'' \cite{smith-woolley_differences_2018} are cause for concern because they invoke a sense of genetic determinism. However, other predictors (parents' educational status, socioeconomic status) explain similar amounts of variance in EA \cite{lee_gene_2018, morris_can_2020} and—unlike DNA—are mutable through social policy changes. 

Several of our values would be violated if children were required to submit their DNA (Right to Action) or educational opportunities were denied to children based on their genetic potential (Opportunities for Advancement). Through the vulnerabilities of unsettled science and the rapid evolution of genetic methods, harms such as leakage to the family can occur and affect not only children but their families and future.

% \sverror{a bigger issue I have with this case is that we are asserting that EA PGS is bad. but are not explaining why. Is it because the predictive power is low? or that the science is dicey? we need to explain that} \cwerror{Added the sentence at the end of the 3rd paragraph—is this better?}


\subsection{DTC Genetic Data Brokerage and Leakage}
\label{case:dtc}

Several FDA-approved 23andMe tests\footnote{23andMe tests with FDA 510(k) clearance are verified as substantially equivalent to existing devices or methods. Only health-related tests, such as those identifying genetic risks, carrier status, or drug responses, qualify for this review.} provide customers with results containing sensitive information, including carrier screening reports, genetic health risk assessments, and BRCA1/BRCA2 variant analyses. In addition, they offer genealogy (to identify relatives) and ancestry services. In an attempt to protect customer privacy, 23andMe offers an opt-in/opt-out policy when it comes to sharing their samples in studies with third-parties.


% FDA approval for certain tests\footnote{23andMe tests with FDA 510(k) clearance are verified as substantially equivalent to existing devices or methods. Only health-related tests, such as those identifying genetic risks, carrier status, or drug responses, qualify for this review.}, including carrier screening reports, genetic health risk reports, and BRCA1/BRCA2 variant tests. In addition, they offer genealogy (to identify relatives) and ancestry services. 23andMe consent forms give the user a choice of opting in or out of sharing their samples in studies with third parties.

In 2023, hackers accessed 14,000+ 23andMe user accounts by using recycled login information (credential stuffing). Due to the relatives linkage feature in genetic data reports, the hackers were able to then access information about an additional 5.5M users \cite{23andme_addressing_2023}; there were also reports that the hacker targeted Ashkenazi Jewish and Chinese ancestry profiles \cite{carballo_23andme_2024}. Among the many values violated, the Right of Privacy and Ownership of the Genome are particularly relevant to this case as 23andMe did not immediately notify compromised users or the public. Several class action lawsuits were filed (e.g., \cite{santana_23andme_2023}), with plaintiffs complaining that 23andMe failed to adequately protect their sensitive information. 23andMe responded by enforcing multifactor authentication (MFA) of user accounts, which was previously optional \cite{whittaker_23andme_2023}, and a serious technological vulnerability. A key gap that allowed for this issue to occur is the lack of consensus around security responsibility, which resulted in the disproportionate and harmful targeting of underrepresented groups. The 23andMe breach highlights the consequences of the relational nature of genetic data, where an individual's information can be compromised through poor public protections, even if they personally take precautions. As one Reddit user quipped ``Your genetic data is only as secure as your relatives' passwords.'' 
% \footnote{\url{https://www.reddit.com/r/23andme/comments/1ayw9y3/what_specific_privacy_concerns_do_you_have_about/}}."

\subsection{Genetic Data Collection of Detained Noncitizens}
\label{case:cpt}

A recent Georgetown University report from the Center of Privacy and Technology (CPT) examines the U.S. federal government's practice of collecting DNA from detained migrants \cite{glaberson_raiding_2024}. This practice, which began with the 2005 DNA Fingerprint Act and expanded significantly in 2020—a mandate that the Department of Homeland Security must collect DNA from \textit{all} detainees, even briefly detained\footnote{States have varying rules about DNA collection from suspects and detainees: 28 states allow DNA to be taken from someone \textit{before} they are convicted of any crime and most states restrict DNA collection to cases involving felony charges or convictions, rather than allowing collection for any arrest or detention \cite{samuels_collecting_dna_2012}.}—has led to a dramatic increase in detainee representation in CODIS (the federal DNA database), rising from 0.21\% in 2019 to 9.21\% in 2023. By 2020, approximately 25,000 noncitizens were added to the database under the ``detainee'' classification.

The CPT report highlights several concerning aspects of this practice, particularly its disproportionate impact on migrants of color and issues of consent. In our Risk Assessment Framework, this immediately violates the Right to Action and Right to Knowledge. Many migrants undergo DNA collection without understanding its implications, sometimes believing it to be a COVID-19 test or submitting under threat of criminal prosecution. The report argues that this practice violates the Fourth Amendment in collecting DNA from detainees without probable cause. Additionally, even though CODIS was designed to be privacy-protective by collecting only 20 markers—a tiny slice of the genome believed to be medically neutral—this limited data can still identify relatives through partial matching and, as an analysis 25 years later would show, are slightly informative of disease risk \cite{banuelos_associations_2022}. The expansive collection of genetic data has far-reaching implications through the guilt-by-association vulnerability. Since CODIS records are difficult to expunge, they can restrict Opportunities for Advancement not only for detainees but also for their relatives and descendants.

\subsection{Underwriting Life Insurance with AI and Genetic Data}
\label{case:insurance}

Life insurance underwriting is the process whereby an insurance company uses personal and health information to assess the risk of insuring an applicant. The relationship between applicants and insurance companies is already fraught: \cite{devnos_genomics_2016} find that patients are more likely to share their genetic info with Google than insurance companies. The future of life insurance underwriting is expected to become increasingly computational and automated via the usage of AI and the collection of more personalized, individual genetic data \cite{filabi_ai-enabled_2021, balasubramanian_insurance_nodate, rothstein_time_2018, koleva-kolarova_financing_2022}. While medical records and demographic factors can be used for mortality analysis, the inclusion of genetic factors can mean that risk prediction can be performed much earlier in an applicant's life without necessarily the same amount of records as an older individual, emphasizing the risk of genetic discrimination \cite{karlsson_linner_genetic_2022} based on the \textit{potential} to have risk factors (e.g., a genetic risk for high blood pressure versus clinically-measured high blood pressure). Life insurance is not covered by GINA and life insurance companies can access medical records (which may include genetic test results) as part of an application. Bills have been introduced in several states that would restrict the use of genetic information in underwriting, but these efforts largely failed: As of 2022, out of thirty-seven proposed bills across all states, three were introduced, eight were signed by the governor, demonstrating the geographical dependency of policy-based protection \cite{vermont_legislature_httpslegislaturevermontgovdocuments2022workgroupshouse20commercegenetic20testingwitness20documentswjill20rickardgenetic20testing20legislation20across20states4-20-2022pdf_nodate}.

Life insurance companies may be interested in using polygenic scores (PGS) to assess an individual's risk for various diseases. Indeed, in early 2024, U.S.-based life insurance company MassMutual and U.K.-based Genomics plc announced a partnership, offering free genetic testing to MassMutual's life insurance customers \cite{massmutual_genomics_2024}. However, the press release stressed that MassMutual would \textit{not} receive individual results and that \textit{current} premiums/policies would be unaffected \cite{massmutual_genomics_2024}. This example particularly highlights the leaky interface between pillars, where the Right of Privacy and Right of Action are called into question. AI insurance underwriting algorithms already suffer from racial biases \cite{lee_ai_2022}. That PGS suffer from the portability problem (itself caused by systematic biases in training datasets) means that these biases could be perpetuated in genetics-informed life insurance underwriting. Additionally, given the relational aspect of genetic data, genetic-based underwriting could affect other biological relatives' applications, harming not only relatives but any others with the same genetic markers that are associated with mortality risks. The potential harms of using genetic data as a part of underwriting in life insurance are one such example of violations of genetic privacy with the rapid evolution and usages of genetic data with unsettled science. 