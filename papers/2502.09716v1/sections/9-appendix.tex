\appendix

\section{Four Pillars}
\begin{table}[!htb]
\centering

\label{tab:pillars}
\begin{tabular}{@{}llll@{}}
\toprule
\multicolumn{1}{c}{\textbf{}} & \multicolumn{1}{c}{\textbf{Pillar}}                                                 & \multicolumn{1}{c}{\textbf{Scope}}                                                                                                                                                                  & \multicolumn{1}{c}{\textbf{Relevant Protections}}                                                                                                                                                                                                                                                                                                                                            \\ \midrule
1                             & \begin{tabular}[c]{@{}l@{}}Tracking, \\ Surveillance, \\ and Forensics\end{tabular} & \begin{tabular}[c]{@{}l@{}}Defined by specific functions \\ the government implements for its own \\ interest and whether individuals \\ submit their data voluntarily or \\ involuntarily.\end{tabular} & \begin{tabular}[c]{@{}l@{}}Genetic data is classified as \\ Personally Identifying Information \\ (PII) data.\end{tabular}  \\ \midrule
2                             & Research                                                                            & \begin{tabular}[c]{@{}l@{}}Defined by institutional regulatory \\ environments and scientific research \\ questions.\end{tabular}                                                                   & \begin{tabular}[c]{@{}l@{}}The impact of research findings depends on the genetic \\ diversity and number of individuals in a study. \\ Investigators, research consortia, national agencies and \\ international collaborations curate population-scale genetic  \\ data that includes many other PII and meta data.\end{tabular}                                                                                                                                           \\ \midrule
3                             & \begin{tabular}[c]{@{}l@{}}Direct to \\ Consumer \\ (DTC)\end{tabular}              & \begin{tabular}[c]{@{}l@{}}Defined by commercial products \\ with the intent to provide \\ consumers with insights about their \\ genome.\end{tabular}                                                                            & \begin{tabular}[c]{@{}l@{}}By removing any third parties, recreational genetic testing \\ democratizes individual access to information about their \\ own genome. In contrast individuals participating in \\ research studies may not receive any genetic results or may \\ not receive "incidental genetic findings" which are unrelated \\ to the central question of the study.\end{tabular} \\ \midrule
4                             & Clinical                                                                            & \begin{tabular}[c]{@{}l@{}}Defined by clinical regulatory \\ environment and interface with \\ medical professionals.\end{tabular}                                                                  & \begin{tabular}[c]{@{}l@{}} FDA, HIPAA\end{tabular}                                                                         \\ \bottomrule
\end{tabular}
\vspace{3 pt}
\caption{\textbf{Four Pillars of genetic data collection and their regulatory considerations.} Data collectors within each pillar operate with different collection strategies and intent to store, use, and transfer the data. However, there is a patchwork of federal and state-level legislation that governs each Pillar. The resulting regulatory gaps lead to \textit{leaky protections} \rverror{for genome owners? the public?}. To introduce a robust genetic data governance system, we explore the associated risks of the leaky protections in our Risk Assessment Framework (Section \ref{sec:risks}), apply it to case studies (Section \ref{sec:studies}), and make policy recommendations (Section \ref{section:recommendations}). \frerror{Make sure that we're all in agreement that this table is accurate for our argument.}}
\end{table}

\section{Genetic Privacy Laws for DTC Companies}
\label{appendix:dtc_laws}
\sverror{I feel like we can skip the next two paragraphs if we need the space}
These provisions are similar between all states with genetic privacy laws encompassing DTC genetic testing companies. However, there are some small differences, e.g., laws may differ in the timeline that a company must abide by for data deletion.

In California, genetic data that is used in a company's quality control pipeline does not need to be deleted. For data transfers to third parties, some states require express consent for each individual transfer, whereas in other states a company would only need express consent once. California does not offer private right of action (only Wyoming, Alaska, and Illinois do), but enforcement falls under several jurisdictions, from city prosecutors to the attorney general.

Common provisions of a data governance system include:

    \begin{itemize}
       \item Right to deletion: the company must comply with consumer requests to delete genetic data (and customer accounts) and destroy the biological sample.
       \item Express consent for data transfers: the company must receive express consent from the consumer before sharing their genetic data with a third party. 
       \item Right of action: \textit{who} enforces the law? In states with private right of action laws, the consumer may bring forth a lawsuit against the companies; however, in most states, only the state attorney general can enforce the law.
       \item De-identified data: privacy laws generally do not apply to genetic data that is de-identified (meaning it is stripped of any identifying metadata). 
    \end{itemize}


\section{Secure Genetic Data for Public Use}
\rverror{A lot of this can be moved to the appendix I think} Some genetic data is already classified as public data, existing in publicly accessible data banks or databases for users to interact with, download, and analyze themselves whether in an academic setting or not. Here, we point to security protocols that now exist with specific genetic databases (such as NIH dbGap), where access must be specifically requested by an user account with email and contact on file. In addition, the responsibility and liability of the security and privacy of the data is given to not only to the individuals and groups that download and use the data, but also to accompanying university IT departments. The study leads  are also able to change data access policies if required, to require any third-parties to delete their data and to inform individuals with informed consent. The example of academic datasets emphasizing integrity with data use and responsibility onto multiple parties for each download is one such example of ensuring data privacy for public datasets, where registration, identification, and interaction with the dataset leadership is fixed. 