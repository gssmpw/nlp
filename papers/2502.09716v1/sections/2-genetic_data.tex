\section{Background: How meaning is derived from genetic data}
\label{section:genetic_data}
In this section, we provide a brief overview of genetic data and genetic inference. An individual's genome is over 3 billion DNA bases\footnote{Units of DNA} long and derived equally from both genetic parents. Genetic variants—units of DNA that differ between individuals—are scattered throughout the genome and contain information about genetically influenced traits (from physical traits to behavioral traits). To predict traits or disease risk, geneticists analyze an individual's unique set of genetic variants. However, this remains a challenging task: while diseases like cystic fibrosis or Huntington's disease are determined by a single genetic variant, most traits are influenced by many variants across the genome as well as non-genetic factors (\emph{complex traits}).

\subsection{Your genetic data is shaped by your ancestors and shared with your relatives.}

% CHAT GPTd/edited FOR CLARITY
The inheritance of DNA—where each parent passes down a random half of their genome to their offspring—results in relatives sharing portions of their DNA, with closer relatives sharing a larger proportion. This shared genetic material means that decisions made about any individual's genome inherently extend to their genetic relatives. This relational nature of genetic data—the \emph{genetic dragnet}—has been noted to significantly complicate discussions about privacy \cite{costello_genetic_2022}. A powerful example of the genetic dragnet came in 2018, when law enforcement leveraged it to identify the Golden State Killer. Authorities used DNA left at a crime scene to identify distant relatives through DNA matching on a public website, GEDMatch. By combining these matches with publicly available genealogical records, they identified the suspect despite the closest identifiable relatives sharing only a great-great-great-great-grandfather \cite{kaiser_we_2018, zabel_killer_2019}. At the same time, an individual's genome is unique to them: even the genomes of identical twins contain differences \cite{ormond_whole_2024}. While a single genome in isolation may not always be immediately identifiable the relational nature of genetic data can enable re-identification. Researchers have argued that genomic data should be treated as "always identifiable in principle" due to this inherent interconnectedness \cite{shabani_reidentifiability_2019, bonomi_privacy_2020}.

\subsection{Geneticists use statistical models for trait prediction.}
Inference using genetic data requires some knowledge of a variant's effect on a trait or disease. This can sometimes be tested in animals by controlled experiments that manipulate the animal's genome. However, this approach is infeasible and unethical to pursue in humans, and so researchers instead rely on large genetic datasets to ask "Do individuals who have the variant \textit{tend} to have the disease (or trait of interest) compared to those who do not have the variant?" To answer this question, researchers typically conduct genome-wide association studies (GWAS), which identify genetic variants that are associated with a trait and quantify the \emph{effect} of each variant on the trait. The learned effect sizes of variants (also called \textit{weights}) are directly interpretable\footnote{For example, in a GWAS for height, an effect size of 0.01 would mean that the variant increases height by an average of 0.01 cm.} and can be used for trait prediction. The polygenic score, considered the gold standard for trait prediction, is calculated as a weighted sum of genetic variants, with each variant scaled by its corresponding GWAS effect size. See Section \ref{section:challenges} for a discussion of frontier models for genetic data \cite{fritzsche_ethical_2023}.


\subsection{Insight from genetic data is a function of the cohort.}
The reference genetic dataset used for a GWAS is analogous to the training dataset in machine learning. Reference datasets contain genetic data and their ``ground truth'' annotations of trait values, whether it be ancestry, disease status, height, etc. Non-genetic factors of a cohort also influence a GWAS. Confounders like environmental exposures and social determinants of
health can introduce bias in a genetic study. For example, the average participant in the UK Biobank \footnote{The UK Biobank is a leading resource which contains genetic and medical data from 500,000 UK residents aged 40-69 at the time of recruitment.} tends to be more educated and healthier than the average Briton \cite{fry_comparison_2017}. The choice of the \emph{cohort}—that is \emph{who} is in the dataset and how their data is labeled\footnote{`True'' ancestry has been arbitrarily (and implicitly) set by the field to correspond to: where your ancestors lived 600 years ago \cite{coop_genetic_2023}. Individuals in the same ancestry group are distantly related in the same timescale (i.e., likely many ancestors in common dating back approximately 600 years ago). }—is thus critical to a GWAS outcome. Prioritizing "cohort matching" in a genetic study ensures that the test sample closely aligns with the reference cohort in key factors such as ancestry, age, geography, and social or environmental influences. This alignment minimizes confounding and enables a more accurate interpretation of genetic effects on traits.\footnote{For example, \cite{mostafavi_variable_2020} find that genetic data prediction accuracy varies within an ancestrally-homogeneous white British cohort, e.g., weights from a BMI GWAS conducted in a younger cohort have higher prediction accuracy via a PGS in younger individuals than older individuals.}
