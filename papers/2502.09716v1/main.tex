%%
%% This is file `sample-manuscript.tex',
%% generated with the docstrip utility.
%%
%% The original source files were:
%%
%% samples.dtx  (with options: `all,proceedings,bibtex,manuscript')
%% 
%% IMPORTANT NOTICE:
%% 
%% For the copyright see the source file.
%% 
%% Any modified versions of this file must be renamed
%% with new filenames distinct from sample-manuscript.tex.
%% 
%% For distribution of the original source see the terms
%% for copying and modification in the file samples.dtx.
%% 
%% This generated file may be distributed as long as the
%% original source files, as listed above, are part of the
%% same distribution. (The sources need not necessarily be
%% in the same archive or directory.)
%%
%%
%% Commands for TeXCount
%TC:macro \cite [option:text,text]
%TC:macro \citep [option:text,text]
%TC:macro \citet [option:text,text]
%TC:envir table 0 1
%TC:envir table* 0 1
%TC:envir tabular [ignore] word
%TC:envir displaymath 0 word
%TC:envir math 0 word
%TC:envir comment 0 0
%%
%% The first command in your LaTeX source must be the \documentclass
%% command.
%%
%% For submission and review of your manuscript please change the
%% command to \documentclass[manuscript, screen, review]{acmart}.
%%
%% When submitting camera ready or to TAPS, please change the command
%% to \documentclass[sigconf]{acmart} or whichever template is required
%% for your publication.
%%
%%
% \documentclass[manuscript,screen,review]{acmart}
%authordraft
\documentclass[manuscript=false,authorversion=true,review=false]{acmart}
\usepackage{footnote}
\usepackage{setspace}
% CHANGE TO DRAFT HERE FOR REVIEW
\usepackage[final,inline,nomargin,index]{fixme}
\fxsetup{theme=color,mode=multiuser,inlineface=\itshape,envface=\itshape}

\FXRegisterAuthor{sv}{asv}{\colorbox{gray!10!white}{\color{black}Suresh}}
\FXRegisterAuthor{rv}{arv}{\colorbox{blue!10!white}{\color{black}Ria}}
\FXRegisterAuthor{fr}{afr}{\colorbox{red!10!white}{\color{black}For Review}}
\FXRegisterAuthor{cw}{acw}{\colorbox{green!10!white}{\color{black}Cole}}
\FXRegisterAuthor{sr}{asr}{\colorbox{purple!10!white}{\color{black}Sohini}}
\FXRegisterAuthor{vr}{avr}{\colorbox{orange!10!white}{\color{black}Vivek}}

\newcommand{\rv}[1]{\rverror{#1}}
%\doublespacing
%%
%% \BibTeX command to typeset BibTeX logo in the docs
\AtBeginDocument{%
  \providecommand\BibTeX{{%
    Bib\TeX}}}

%% Rights management information.  This information is sent to you
%% when you complete the rights form.  These commands have SAMPLE
%% values in them; it is your responsibility as an author to replace
%% the commands and values with those provided to you when you
%% complete the rights form.
%\acmConference[FAccT '25]{FAccT '25: ACM Conference on Fairness,
%Accountability, and Transparency}{June 23-26, 2025}{Athens, Greece}
%\acmBooktitle{FAccT '25: ACM Conference on Fairness,
%Accountability, and Transparency,
%  June 23-26, 2025, Athens, Greece}
  
%\setcopyright{acmlicensed}
%\copyrightyear{2025}
%\acmYear{2025}
%\acmDOI{XXXXXXX.XXXXXXX}
%% These commands are for a PROCEEDINGS abstract or paper.
%\acmConference[Conference acronym 'XX]{Make sure to enter the correct
% conference title from your rights confirmation emai}{June 03--05,
%  2018}{Woodstock, NY}
%%
%%  Uncomment \acmBooktitle if the title of the proceedings is different
%%  from ``Proceedings of ...''!
%%
%%\acmBooktitle{Woodstock '18: ACM Symposium on Neural Gaze Detection,
%%  June 03--05, 2018, Woodstock, NY}
%\acmISBN{978-1-4503-XXXX-X/18/06}


%%
%% Submission ID.
%% Use this when submitting an article to a sponsored event. You'll
%% receive a unique submission ID from the organizers
%% of the event, and this ID should be used as the parameter to this command.
%%\acmSubmissionID{123-A56-BU3}

%%
%% For managing citations, it is recommended to use bibliography
%% files in BibTeX format.
%%
%% You can then either use BibTeX with the ACM-Reference-Format style,
%% or BibLaTeX with the acmnumeric or acmauthoryear sytles, that include
%% support for advanced citation of software artefact from the
%% biblatex-software package, also separately available on CTAN.
%%
%% Look at the sample-*-biblatex.tex files for templates showcasing
%% the biblatex styles.
%%

%%
%% The majority of ACM publications use numbered citations and
%% references.  The command \citestyle{authoryear} switches to the
%% "author year" style.
%%
%% If you are preparing content for an event
%% sponsored by ACM SIGGRAPH, you must use the "author year" style of
%% citations and references.
%% Uncommenting
%% the next command will enable that style.
%%\citestyle{acmauthoryear}


%%
%% end of the preamble, start of the body of the document source

\begin{document}

%%
%% The "title" command has an optional parameter,
%% allowing the author to define a "short title" to be used in page headers.
\title[Genetic Data Governance in Crisis]{Genetic Data Governance in Crisis: Policy Recommendations for 
  Safeguarding Privacy and Preventing Discrimination}

%%
%% The "author" command and its associated commands are used to define
%% the authors and their affiliations.
%% Of note is the shared affiliation of the first two authors, and the
%% "authornote" and "authornotemark" commands
%% used to denote shared contribution to the research.
% \author{Ben Trovato}
% \authornote{Both authors contributed equally to this research.}
% \email{trovato@corporation.com}
% \orcid{1234-5678-9012}
% \author{G.K.M. Tobin}
% \authornotemark[1]
% \email{webmaster@marysville-ohio.com}
% \affiliation{%
%   \institution{Institute for Clarity in Documentation}
%   \city{Dublin}
%   \state{Ohio}
%   \country{USA}
% }

\author{Vivek Ramanan}
\affiliation{%
  \institution{Brown University}
  \city{Providence}
  \country{USA}}

\author{Ria Vinod}
\affiliation{%
  \institution{Brown University}
  \city{Providence}
  \country{USA}
}

\author{Cole Williams}
\affiliation{%
 \institution{Brown University}
 \city{Providence}
 \country{USA}}

\author{Sohini Ramachandran}
\affiliation{%
  \institution{Brown University}
  \city{Providence}
  \country{USA}}

\author{Suresh Venkatasubramanian}
\affiliation{%
  \institution{Brown University}
  \city{Providence}
  \country{USA}}


%%
%% By default, the full list of authors will be used in the page
%% headers. Often, this list is too long, and will overlap
%% other information printed in the page headers. This command allows
%% the author to define a more concise list
%% of authors' names for this purpose.
\renewcommand{\shortauthors}{Ramanan, Vinod, Williams et al.}

%%
%% The abstract is a short summary of the work to be presented in the
%% article.
\begin{abstract}
Genetic data collection has become ubiquitous today. The ability to meaningfully interpret genetic data has motivated its widespread use across forensics, clinical practice, and research, providing crucial insights into human health and ancestry while driving important public health initiatives. Easy access to genetic testing has fueled a rapid expansion of direct-to-consumer offerings, many of which are recreational in nature. However, the growth of genetic datasets and their applications has created significant privacy and discrimination risks, particularly as our understanding of the scientific basis for genetic traits continues to evolve. In this paper, we organize the uses of genetic data along four distinct `pillars': clinical practice, research, forensic and government use, and recreational use. Using our scientific understanding of genetics, genetic inference methods and their associated risks, and existing regulatory mechanisms, we build a risk assessment framework that identifies key values that any governance system must preserve. We then analyze case studies from each of the pillars using this framework to assess how well existing legal and regulatory frameworks preserve desired values. Our investigation reveals critical gaps in existing regulatory frameworks and identifies specific threats to privacy and personal liberties, particularly through genetic discrimination. To address these challenges, we call for and propose comprehensive regulatory reforms including: (1) updating the legal definition of genetic data to protect against modern technological capabilities, (2) expanding the Genetic Information Nondiscrimination Act (GINA) to cover currently unprotected domains, and (3) establishing a unified regulatory framework under a single governing body to oversee all applications of genetic data. 

We conclude with three open questions about genetic data: the challenges posed by its relational nature, including consent for relatives and minors; the complexities of international data transfer; and its potential integration into large language models.


\end{abstract}

%%
%% The code below is generated by the tool at http://dl.acm.org/ccs.cfm.
%% Please copy and paste the code instead of the example below.
%%
\begin{CCSXML}
<ccs2012>
   <concept>
       <concept_id>10010405.10010444.10010093.10010934</concept_id>
       <concept_desc>Applied computing~Computational genomics</concept_desc>
       <concept_significance>500</concept_significance>
       </concept>
   <concept>
       <concept_id>10010405.10010444.10010449</concept_id>
       <concept_desc>Applied computing~Health informatics</concept_desc>
       <concept_significance>300</concept_significance>
       </concept>
   <concept>
       <concept_id>10010405.10010455.10010458</concept_id>
       <concept_desc>Applied computing~Law</concept_desc>
       <concept_significance>100</concept_significance>
       </concept>
   <concept>
       <concept_id>10010405.10010444.10010446</concept_id>
       <concept_desc>Applied computing~Consumer health</concept_desc>
       <concept_significance>300</concept_significance>
       </concept>
 </ccs2012>
\end{CCSXML}

\ccsdesc[500]{Applied computing~Computational genomics}
\ccsdesc[300]{Applied computing~Health informatics}
\ccsdesc[100]{Applied computing~Law}
\ccsdesc[300]{Applied computing~Consumer health}
\ccsdesc[500]{}
% \ccsdesc[300]{Do Not Use This Code~Generate the Correct Terms for Your Paper}
% \ccsdesc{Do Not Use This Code~Generate the Correct Terms for Your Paper}
% \ccsdesc[100]{Do Not Use This Code~Generate the Correct Terms for Your Paper}

%%
%% Keywords. The author(s) should pick words that accurately describe
%% the work being presented. Separate the keywords with commas.
\keywords{Data Governance, Genetic Privacy, Regulatory Frameworks, Genetic Discrimination, Policy, Data Protection, Genetic Inference}

% \received{20 February 2007}
% \received[revised]{12 March 2009}
% \received[accepted]{5 June 2009}

\renewcommand\englishlistfixmename{List of Notes}

% \FXRegisterAuthor{sv}{asv}{\colorbox{gray!10!white}{\color{black}Suresh}}
% \FXRegisterAuthor{rv}{arv}{\colorbox{blue!10!white}{\color{black}Ria}}
% \FXRegisterAuthor{vr}{avr}{\colorbox{red!10!white}{\color{black}Vivek}}
% \FXRegisterAuthor{cw}{acw}{\colorbox{green!10!white}{\color{black}Cole}}
% \FXRegisterAuthor{sr}{asr}{\colorbox{purple!10!white}{\color{black}Sohini}}

%%
%% This command processes the author and affiliation and title
%% information and builds the first part of the formatted document.
\maketitle

\section{Introduction}

% State of the world (robots for creative activites)
The term ``robot,'' originally signifying `forced labor,' has long been associated with labor and work. Robots have demonstrated their utility in various automated productive and social contexts, where the primary goals are improving productivity, safety, and fostering social interactions with humans~\cite{simoes2022designing, weidemann2021role, honig2018understanding}. However, an increasing number of cases feature using of robots in creative settings. Unlike productive contexts, where the focus is on efficiency and task completion~\cite{arents2022smart}, or social contexts, where communication and trust are prioritized~\cite{nam2020trust, saunderson2019robots}, creative environments prioritize artistic innovation and expression~\cite{hsueh2024counts}. This shift fundamentally alters the dynamics of human-robot interaction, redefining the roles and expectations for both humans and robots.

For instance, robots’ social behaviors are leveraged to support the generation and expression of creative ideas~\cite{hu2021exploring, sandoval2022human, alves2020creativity}, and programmable robotic movements and trajectories are employed to inspire artistic activities such as sketching~\cite{lin2020your}. These studies often engage participants from creative fields who possess limited prior experience with robotics, and are typically conducted in short-term, experimental settings. Consequently, the findings from these studies remain constrained since much can be learned from professional practitioners' experiences to inform system design such as digital fabrication~\cite{hirsch2023nothing}. There is a notable gap in research examining the long-term, active, and practical experience of integrating robotic systems into the creative processes. As a result, the deeper insights into how robots facilitate and shape creative processes, beyond simply augmenting human creativity, remain underexplored. In this study, we aim to better understand the impacts of robots on creative processes and outcomes.

As early as Leonardo da Vinci's 16th century ``Automaton,'' artists have explored the creative affordances of robotic systems~\cite{shanken2002cybernetics, pagliarini2009development, jeon2017robotic}. The artistic creation process typically encompasses various stages, including the exploration of materials and techniques, ongoing experimentation and iteration, and the continual refinement of the artists' insights into their creative subjects~\cite{lewis2023art, sturdee2022state}. Therefore, investigating the artistic process involving robots offers an opportunity to gain deeper insights into robots' creative potential. Robotic art, in particular, provides a compelling case for this exploration.

We define robotic art as artworks that utilize robotic or automated machines to create artistic experiences and tangible artifacts. One example is robotic installation art, in which robots are programmed to follow specific rules that embody the artist’s expression (\autoref{fig:teaser} (a)). Another example is responsive art, in which robots react to their environment, with behaviors that change over time or in response to spectators (\autoref{fig:teaser} (b)). Additionally, there are robotic creators, which possess a degree of agency, allowing them to collaborate with human artists and produce works that extend beyond mere replication of human-created art (\autoref{fig:teaser} (c) and (d)). As such, robotic art becomes a rich case for exploring human-machine interactions in creative contexts. Gaining a deeper understanding of how robots facilitate artistic expression can provide insights for designing computing systems to support creative activities~\cite{gomez2021robot}.

% Therefore, we did...
We draw on semi-structured, in-depth interviews with renowned professional robotic artists to investigate the use of robots in artistic practice. Specifically, our goal is to understand how artistic exploration of robotic systems challenges conventional assumptions about the functions of robots, such as their roles in automating repetitive tasks or serving human needs. We also explore the implications of robots in the artistic process and examine how creativity may emerge within robotic art. To address these interrelated inquiries, our study focuses on the practice of robotic art, posing the research question: \textit{How do robotic artists utilize robots in their artistic practice?} We approach this inquiry through the perspectives and experiences of robotic artists, who creatively design, modify, and repurpose robotic systems for artistic expression and exploration.

% The key findings are...
Our findings highlight the social, material, and temporal dimensions of artists' practices that shape their creativity and artistic outcomes. The creation of robotic art is largely a social process, as artists receive both explicit and implicit feedback through the audience's reactions and reception of their work. Simultaneously, the embodiment and malfunctions inherent to robotic systems drive artistic experimentation. The temporal processes of creation and exhibition, beyond just the final product, further enhance the creative value. Our empirical analysis presents how creativity emerges through the interplay of social, material, and temporal interactions among artists, robots, audiences, and the environment.

% The contributions of this work are...
We make two main contributions to HCI in this study. 
First, we elucidate the interactive mechanisms among key actors---human creators, machines, audiences, and environments---within the practice of robotic art, a topic that remains underexplored in HCI. Our findings reveal the significance of sociality (e.g., interactions between artists and audiences), materiality (e.g., the embodiment and malfunctions of robots), and temporality (e.g., the processes of creation and exhibition) in shaping creative values. We propose that these three facets are central to the creative process and facilitate the emergence of creativity in robotic art.
Second, drawing from the findings, we offer implications for \textit{socially informed}, \textit{material-attentive}, and \textit{process-oriented} creation with computing systems. We suggest leveraging these three aspects to enhance creativity and the creative experience. Specifically, we discuss the value of incorporating implicit audience feedback, designing with technical malfunctions, and focusing on the post-creation process to foster alternative creative experiences with machines~\cite{alter2010designing, juarez2022glitch}.




\section{Background: How meaning is derived from genetic data}
\label{section:genetic_data}
In this section, we provide a brief overview of genetic data and genetic inference. An individual's genome is over 3 billion DNA bases\footnote{Units of DNA} long and derived equally from both genetic parents. Genetic variants—units of DNA that differ between individuals—are scattered throughout the genome and contain information about genetically influenced traits (from physical traits to behavioral traits). To predict traits or disease risk, geneticists analyze an individual's unique set of genetic variants. However, this remains a challenging task: while diseases like cystic fibrosis or Huntington's disease are determined by a single genetic variant, most traits are influenced by many variants across the genome as well as non-genetic factors (\emph{complex traits}).

\subsection{Your genetic data is shaped by your ancestors and shared with your relatives.}

% CHAT GPTd/edited FOR CLARITY
The inheritance of DNA—where each parent passes down a random half of their genome to their offspring—results in relatives sharing portions of their DNA, with closer relatives sharing a larger proportion. This shared genetic material means that decisions made about any individual's genome inherently extend to their genetic relatives. This relational nature of genetic data—the \emph{genetic dragnet}—has been noted to significantly complicate discussions about privacy \cite{costello_genetic_2022}. A powerful example of the genetic dragnet came in 2018, when law enforcement leveraged it to identify the Golden State Killer. Authorities used DNA left at a crime scene to identify distant relatives through DNA matching on a public website, GEDMatch. By combining these matches with publicly available genealogical records, they identified the suspect despite the closest identifiable relatives sharing only a great-great-great-great-grandfather \cite{kaiser_we_2018, zabel_killer_2019}. At the same time, an individual's genome is unique to them: even the genomes of identical twins contain differences \cite{ormond_whole_2024}. While a single genome in isolation may not always be immediately identifiable the relational nature of genetic data can enable re-identification. Researchers have argued that genomic data should be treated as "always identifiable in principle" due to this inherent interconnectedness \cite{shabani_reidentifiability_2019, bonomi_privacy_2020}.

\subsection{Geneticists use statistical models for trait prediction.}
Inference using genetic data requires some knowledge of a variant's effect on a trait or disease. This can sometimes be tested in animals by controlled experiments that manipulate the animal's genome. However, this approach is infeasible and unethical to pursue in humans, and so researchers instead rely on large genetic datasets to ask "Do individuals who have the variant \textit{tend} to have the disease (or trait of interest) compared to those who do not have the variant?" To answer this question, researchers typically conduct genome-wide association studies (GWAS), which identify genetic variants that are associated with a trait and quantify the \emph{effect} of each variant on the trait. The learned effect sizes of variants (also called \textit{weights}) are directly interpretable\footnote{For example, in a GWAS for height, an effect size of 0.01 would mean that the variant increases height by an average of 0.01 cm.} and can be used for trait prediction. The polygenic score, considered the gold standard for trait prediction, is calculated as a weighted sum of genetic variants, with each variant scaled by its corresponding GWAS effect size. See Section \ref{section:challenges} for a discussion of frontier models for genetic data \cite{fritzsche_ethical_2023}.


\subsection{Insight from genetic data is a function of the cohort.}
The reference genetic dataset used for a GWAS is analogous to the training dataset in machine learning. Reference datasets contain genetic data and their ``ground truth'' annotations of trait values, whether it be ancestry, disease status, height, etc. Non-genetic factors of a cohort also influence a GWAS. Confounders like environmental exposures and social determinants of
health can introduce bias in a genetic study. For example, the average participant in the UK Biobank \footnote{The UK Biobank is a leading resource which contains genetic and medical data from 500,000 UK residents aged 40-69 at the time of recruitment.} tends to be more educated and healthier than the average Briton \cite{fry_comparison_2017}. The choice of the \emph{cohort}—that is \emph{who} is in the dataset and how their data is labeled\footnote{`True'' ancestry has been arbitrarily (and implicitly) set by the field to correspond to: where your ancestors lived 600 years ago \cite{coop_genetic_2023}. Individuals in the same ancestry group are distantly related in the same timescale (i.e., likely many ancestors in common dating back approximately 600 years ago). }—is thus critical to a GWAS outcome. Prioritizing "cohort matching" in a genetic study ensures that the test sample closely aligns with the reference cohort in key factors such as ancestry, age, geography, and social or environmental influences. This alignment minimizes confounding and enables a more accurate interpretation of genetic effects on traits.\footnote{For example, \cite{mostafavi_variable_2020} find that genetic data prediction accuracy varies within an ancestrally-homogeneous white British cohort, e.g., weights from a BMI GWAS conducted in a younger cohort have higher prediction accuracy via a PGS in younger individuals than older individuals.}


\section{Background: Current Protections}
\label{section:protections}


\begin{figure}[H]
    \centering
    \includegraphics[width=0.6\textwidth]{figures/final_four_pillars_v6.png}
    \caption{\textbf{Four Pillars of genetic data collection and their regulatory considerations.} Data collectors within each pillar operate with different collection strategies and intent to store, use, and transfer the data. However, there is only a patchwork of federal and state-level legislation that governs each Pillar. The resulting regulatory gaps lead to \textit{leaky protections} through which genetic data—or knowledge derived from genetic data—can be used in other pillars with little to no oversight. Real-world examples\protect\footnotemark{} of leakage are indicated by red arrows.}
    \label{fig:4_pillars}
\end{figure}
\footnotetext{Orchid Health is a private company that uses publicly available GWAS weights to compute polygenic risk scores (PGS) of embryos. Nucleus Genomics is also a company that similarly uses whole genome sequencing to compute PGS to predict health and behavioral traits.}




In this section, we outline the foundation of today's genetic data collectors and the current regulatory and legislative environments that they are subject to. To better understand the vast landscape over which genetic data is being collected, we build upon the Four Pillars \cite{wan_sociotechnical_2022} of genetic data collectors. We categorize these data collectors by their motivations for collecting genetic data and their relationships with the individuals from whom the data is collected. We define the updated Four Pillars as the following:
\begin{itemize}
    \item \textbf{Forensics, Tracking, Surveillance:} Entities focused on monitoring, identifying, and analyzing genetic data to support public health outcomes, security, and law enforcement objectives.
    \item \textbf{Direct to Consumer (DTC):} Entities developing products to provide consumers with direct access to genetic insights.
    \item \textbf{Clinical:} Entities requiring medical professionals' oversight to conduct tests, often enabling insurance coverage and generating clinically actionable insights.
    \item \textbf{Research:} Entities engaging in research and development (R\&D) and basic research across private, public, and academic sectors. This includes public consortia, biobanks, and collaborations with DTC entities.
\end{itemize}

The ``genetic data ecosystem'' encompasses not only the entities within the Four Pillars but also the broader network of legislatures, regulators, and individuals involved in or affected by genetic data collection and use. By outlining this ecosystem, we aim to shed light on the regulatory gaps that enable genetic data leakage across pillars. We provide real-world examples of this leakage in Figure \ref{fig:4_pillars}.

%after an examination of federal legislation, policy, and regulatory bodies, that attempt to provide baseline protections across the United States. We find that federal data protections for individuals are limited, and that a patchwork of state-level legislation creates regulatory gaps, allowing private companies handling genetic data to discriminate based on jurisdiction and geography. We thus determine that today's state of protections is \emph{leaky}, and finally identify areas in urgent need of reform.

%%%%%%%

\subsection{Federal protections}

\subsubsection{Genetic Information Non-Discrimination Act (GINA)}
GINA was enacted in 2008 and prohibits genetic discrimination by employers or health insurers. Specifically, GINA bars employers from making hiring, firing, or promotion decisions based on genetic information (and information disclosed about relatives) and bans health insurance providers from denying coverage or charging higher premiums based on genetic information including genetic predisposition for disease \cite{gina_2008}. GINA does not apply to life, disability, or long-term care insurance \cite{green_strategic_2020} or any other venues outside of employment and health insurance (e.g., education). The US Equal Employment Opportunity Commission (EEOC) regulates workplace discrimination for employees and thus, is responsible for enforcing GINA in the workplace. Enforcement for health insurers falls under the purview of several federal agencies, including the Department of Health and Human Services (HHS); Department of Labor; Centers for Medicare and Medicaid Services; and the Department of the Treasury \cite{protections_ohrp_federal_2009}. 


\subsubsection{Health Information Portability and Accountability Act (HIPAA)} HIPAA, signed into law in 1996, grants security and privacy protections for patient health information (PHI). 
%and Genetic Information Nondiscrimination Act (GINA). HIPAA was passed by Congress and signed into law in 1996. 
HIPAA applies to covered entities that maintain this information, such as healthcare and health insurance companies, but does not apply to many other entities that might retain health information, including search engines, medical information sites, or dating sites \cite{citron_new_2020}. Importantly, HIPAA does not apply to two of the largest DTC genetic testing companies 23andMe and AncestryDNA \cite{sklar_be_2020}. HIPAA does not cover deidentified data: PHI data with unique identifying information (names, address, phone numbers, dates, etc.) removed do not fall under HIPAA. Research consortia, biobanks, and research collaborations with DTC and clinical entities primarily use genetic data that is considered deidentified, and so HIPAA generally does not apply. However, the relational aspect of genetic data has motivated calls to ``consider genomic data as, in principle, always identifiable'' \cite{bonomi_privacy_2020}. The inability to completely strip data of \emph{all} identifiable features introduces critical vulnerabilities for genome owners and exacerbates the risk of data leakage across the Four Pillars.


\subsubsection{Food and Drug Administration (FDA)}
The FDA ``is responsible for protecting the public health by ensuring the safety, efficacy, and security of human and veterinary drugs, biological products, and medical devices'' \cite{commissioner_what_2023}. In the context of DTC genetic testing companies, the FDA only regulates well-defined high risk medical tests with clinical actions, such as 23andMe's health tests for diseases like Parkinson's, breast cancer, and late-onset Alzheimer's. If the FDA approves a specific test, Centers for Medicare and Medicaid Services then choose to approve the test for insurance coverage \cite{daval_authority_2023}. Most genetic tests are not covered by Medicare, particularly DTC tests. 

\subsubsection{Federal Trade Commission (FTC)}
The FTC enforces federal laws that protect consumers from ``fraud, deception and unfair business practices'' \cite{federal_trade_commission_coop_nodate}. The FTC has warned DTC genetic testing companies that the results they return to customers must be backed by ``reliable science'', and also warns companies against making exaggerated claims about the use of AI in their products \cite{jillson_dna_2024}. Additionally, the FTC remains vigilant of deceptive practices regarding privacy policy changes and ``dark patterns'' designed to coerce consumers into consenting to data sharing\footnote{In 2023, the Commission fined 1Health.io/Vitagene \$75,000, claiming they ``left sensitive genetic and health data unsecured, deceived consumers about their ability to get their data deletes, and changed its privacy policy retroactively without adequately notifying and obtaining consent'' \cite{ftc_1health_2023}.}. The power and broad scope of the FTC—and their stated intentions of cracking down on DTC genetic testing companies \cite{jillson_dna_2024}—make it a major player in the regulatory space, and it will likely grow in importance if the federal government prioritizes genetic data privacy. 

\subsubsection{The Common Rule}
This is a federal policy that protects human subjects in research settings \cite{sciences_federal_2014}. It was first codified in 1981 by the Department of Health, Education, and Welfare (now HHS) but adopted more widely by other federal agencies in 1991. The Common Rule institutionalizes protections for vulnerable participants, protections overseen by Institutional Review Boards (IRBs); IRBs are groups ``formally designated to review and monitor biomedical research involving human subjects'' \cite{research_institutional_2024}. Revisions enacted in 2017 particularly focus on informed consent for participants, but do not cover deidentified data \cite{menikoff_common_2017, protections_ohrp_federal_2009}. IRBs ensure that consistent protection of human subjects are maintained and are necessary for research to begin and proceed if involving human participants. The maintenance of the Common Rule and HIPAA, in the case of medical data, is performed by IRBs.
\subsection{State protections}

\begin{figure}[ht]
    \centering
    \includegraphics[width=\textwidth]{figures/legislation_map.png}
    \caption{\textbf{Beyond HIPAA and GINA, state-level genetic discrimination and privacy legislation varies widely in the United States}. \textbf{Left:} States with genetic anti-discrimination legislation, distinguishing between states with both health insurance and employment protections (red) and those with health insurance protections only (orange). \textbf{Right}: states with general privacy laws that include genetic data (blue) and genetic privacy-specific laws (green).}
    \label{fig:genetic_legislation}
\end{figure}

State laws offer far more protections than their federal counterparts, but a patchwork of these laws means a patchwork in protections, as shown in Figure \ref{fig:genetic_legislation}. Here, we focus on two categories of laws: genetic anti-discrimination laws and data privacy laws.

\subsubsection{Genetic Anti-Discrimination Laws}
Most states have passed some version of a genetic anti-discrimination law (initially prompted by the passage of GINA) although the sector, scope, and strength of the laws differs between states. 
%Genetic anti-discrimination laws were, generally, passed around the time of GINA and typically offer similar protections (i.e., outlawing genetic discrimination in employment or health insurance). 
Some states offer fewer protections than GINA: Georgia, for example, has a statute outlawing genetic discrimination by health insurers, but not employers. Other states offer \textit{more} protections than GINA. California's CalGINA \cite{california_senate_bill_559_2011} outlaws genetic discrimination in several sectors: employment, health insurance, housing, and by any state agency or entity receiving state funding (such as emergency medical services). Florida includes life, long-term care, and disability insurers \cite{florida_house_bill_1189_2020}. Many states (including Massachusetts and New Mexico) that include life insurance in their non-discrimination laws exempt companies if genetic data reliably, and based on ``sound actuarial principles'' \cite{massachusetts_genetic_law_2025}, gives information relating to mortality or morbidity (see Section \ref{case:insurance} for a detailed case study).

\subsubsection{Genetic Data Privacy Laws}

Many states have genetic data privacy laws, either as a standalone law or as part of a broader data privacy law. These laws\footnote{Some of these laws have also led to a change in company services in that state. For example, the passing of Illinois' BIPA (Biometric Information Protection Act) resulted in Amazon and Google removing certain services of Alexa and Google Photos respectively after lawsuits which claimed that the company violated BIPA.} were generally passed after GINA with the rise of DTC genetic testing companies. Standalone genetic data privacy laws tend to be written specifically to cover DTC genetic testing companies and are often extensions of genetic non-discrimination laws passed around the time of GINA.

Both data privacy and genetic privacy laws have similar provisions \cite{nhgri_genome_database_2025}, including a customer right to request deletion of genetic data and samples and the necessity of consent for data transfers. Some states allow consumers to bring lawsuits for violations of the privacy law (a private right of action). In general though, privacy laws do not apply to deidentified genetic data (as per HIPAA). An important contrast to the US federal and state bills on privacy is the EU General Data Protection Regulation (GDPR). GDPR not only regulates all personal data within the EU but also the transfer of data outside of the EU, data minimization, and compliance. Genetic data is labeled as ``special category data'' in GDPR as it has many identifiers associated with it. 

% See Appendix \ref{appendix:dtc_laws} for more discussion on provisions for DTC genetic data use.


% \sverror{and then continue here} \rverror{We moved the previous paragraph to the appendix.}
% Some of these laws have also led to a change in company services in that state. For example, the passing of Illinois' BIPA (Biometric Information Protection Act) resulted in Amazon and Google removing certain services of Alexa and Google Photos respectively after lawsuits which claimed that the company violated BIPA.  


\section{Risk Assessment Framework}
\label{sec:risks}

Our goal in this section is to develop a risk assessment framework for genetic data governance.
%Having established the background of genetic data, the genetic data collectors, and current protections—we next formalize the \textit{risks} that the existing system poses to the individual. 
Similar to the aims of the Blueprint for the AI Bill of Rights \cite{park2023ai}, we motivate our framework with three central questions: (1) What values (i.e., moral principles and civil liberties) should be preserved? (2) What are the vulnerabilities in the current system that can compromise these values? (3) What are the specific harms that can result from the vulnerabilities? 


\subsection{What values should be preserved?}

We seek to identify values that are future-proof and can be encoded in a system of genetic data collection. By examining the Four Pillars, six key values emerge:

\begin{enumerate}
    \item \textbf{Right to action}: The individual, and the individual only, has the choice to submit (and the freedom to not submit) their genome\footnote{We use ``genome'' to refer to a physical DNA samples of an individual and any sequencing/genotyping data.}.
    %they shall not be required to do so by any entity.
    \item \textbf{Ownership of the genome}: The individual owns their genome and therefore controls the usage of their genome, including \textit{who} has access to it and for what reason.
    \item \textbf{Right to privacy}: The individual has a right to privacy with regards to their genome and inferences made from their genome.
    \item \textbf{Right to knowledge}: The individual has a right to know or \textit{not} know about inferences made from their genome. 
    %An important implication of the genetic dragnet is that individuals may not have control over others (e.g., close relatives) inadvertently revealing insights about their genome. This is particularly important for individuals who do \textit{not} participate in genetic testing (or who selectively participate in certain tests, but not others). \frerror{Is this too complicated by bringing up the relational aspect?} \sverror{I think it is, especially in comparison with the previous shorter points}
    \item \textbf{Protecting opportunities for advancement}: Genetic data should not be used to deprive the individual of opportunities in any domain, including (but not limited to) education, access to financial tools, health insurance, housing, social services, and reproductive choices. 
    %Here, any discriminatory behavior that deprives opportunity should be protected against, recognizing that reproductive autonomy is fundamental to an individual's ability to make life choices and advance according to their own goals. \sverror{I'm not sure what this second sentence is adding}
    \item \textbf{Benefits of inclusion}: The Belmont principle states ``those who bear the burdens of research (i.e., those who are exposed to the discomforts, inconveniences, and risks) should receive the benefits in equal measure to the burdens''\cite{belmont_report}. We believe that this applies to individuals and their genetic data.\footnote{Variant Bio is a biotech company that collects genetic data from Indigenous groups from around the world and participates in revenue sharing with the communities they collect data from.}
\end{enumerate}

\subsection{What are the vulnerabilities in the current system that can compromise these values?}

The current genetic data ecosystem has many vulnerabilities that compromise the values listed above. These include:
\begin{enumerate}
    \item \textbf{Unsettled science}: The role of genetics in shaping complex traits is poorly understood. As a consequence, the deterministic nature of genetics may be overstated, particularly for behavioral and cognitive traits. 
    \item \textbf{Rapid evolution of genetic data/methods}: As genetic data collection, sequencing, and methodologies quickly evolve, legal protections fall behind. For example, education discrimination is not included in GINA, which was passed five years before the first large GWAS on educational attainment \cite{rietveld_gwas_2013}.
    %Many individuals can be negatively affected because of lack of scientific clarity (e.g. predicting IQ through genetic tests is often confounded by lack of causal confidence and environmental socioeconomic factors \cite{selzam_s_comparing_2019}). 
    \item \textbf{Guilt by association}: DNA databases used for criminal investigations can impact not only the individuals whose DNA is stored, but also their biological relatives, as genetic information is shared among family members. 
  %  \item \textbf{Third-party data collectors}: \rverror{This one doesn't seem clear cut; and overlaps with many others -- do we lose anything but cutting it out altogether? or how do we make it distinct?} there are companies that allow users to upload their genetic data (that they have downloaded from 23andMe or AncestryDNA) for free and in exchange users receive additional relative matches (e.g., GEDMatch) or disease risk reports (e.g., Promethease). The barrier to entry in order to start one of these companies is low—e.g., Promethease reports polygenic risk scores using publicly available GWAS weights—which may result in lower trust. GEDMatch\footnote{GEDMatch also had an option allowing users to `opt-in' to sharing data with law enforcement, but in 2019 a Florida judge overrode user preferences and granted a detective full access to the database \cite{hill_your_2019}.}, which collected genetic data from 1.2M individuals, was acquired by genetic forensics company Verogen, which was then acquired by the German company Qiagen.
 
    \item \textbf{Geographical legislative patchwork}: Genetic anti-discrimination and privacy legislation widely differs between states. Legal policy in each state for data removal, third-party sharing, and many other aspects of genetic data are unclear. Individuals who move between states or share data with entities in other states can be affected by this dependency on state specific policy. Aside from patchwork policy, many state policies do not sufficiently protect genetic data specifically. 
%    \item \textbf{Security practices}:
%    \frerror{Is there a term to refer to individual versus entity security practices?}
 %   Individual and entity security practices can control whether malicious actors can gain access to their data (e.g. 23andMe hack where individual security passwords were used to access related profiles and their identifying data; see Section \ref{case:dtc}). 
\end{enumerate}


\subsection{What are the harms that arise as a result?}
Lastly, we will outline several consequences that could arise from violations of our key values through the above vulnerabilities. These downstream effects, which we refer to as harms, can affect an individual and their immediate genetic relatives and 
%and their personal data footprint,
can also have a broader effect on communities and the population at large.
%the population level scale \cwerror{Don't like this phrasing}. 

\begin{enumerate}
    \item \textbf{Leakage to the family:} The genetic dragnet means that any conclusion about an individual's genetic trait can be linked back to their relatives, even if no action is taken by their relatives in either (1) submitting data or (2) opting in to receive information. Any derived secondhand knowledge can immediately impact insurance (health, life, etc.) and medical treatment plans, as well as compromise identity.
    %, even though the original determination was not made on genetic relatives. The relational nature of genetic data and its leaky quality thus compromise both the privacy and personal liberties of family members. Not only can the relational nature of genetics affect immediate known relatives, but decisions also impact future children, disease carriers, and preventative care. 

    \item \textbf{Loss of anonymity: } It is straightforward to identify some traits about an individual from their genetic data, and subsequently from their genetic relatives. Common examples include: race, gender, ethnicity, and markers for certain diseases. Leaky genetic data interfaces can thus compromise information about an individual, their genetic relatives, and unborn children that the individual (and implicated relatives) may wish to maintain as private. 
    %Exposure can lead to future vulnerability such as affecting opportunities of advancement, discrimination in availing certain insurance policies or services, and extortion or targeted attacks -- all of which compromising a critical value of privacy and personal liberties. 
    As the refinement of reference datasets continues and data collection grows\footnote{It has been estimated that, in a sample of three million European-descent Americans, 99\% of individuals would have a least one third cousin in the dataset \cite{erlich_identity_2018}.}, anonymity becomes even more elusive.
    
    \item \textbf{Loss of Data Control: } 
    % Private institutions that sell genetic tests own the end-to-end process of how individuals' data is collected, stored, and monetized. Data transfer policies remain a black box as it is unclear which deals an individual's data may be included in, how the data is monetized, and which parties benefit from these trades. Even if users opt-out of participation, ungoverned data flows mean that their genetic data footprint will exist in perpetuity. The lack of transparency in the event of bankruptcy or acquisition has the same implication: In 2024, Tempus AI acquired Ambry Genetics, which sequences 400,000 patients annually, with the stated intent to "leverage this data and augment Tempus' current data offering" \cite{tempus2025}. Consequently, any decision of an individual to submit data for a genetic test implicitly includes loss of complete ownership over their genetic data, and any leaky interfaces can compromise their privacy without the appropriate safeguards.\sverror{I don't understand how the tempus example is relevant}
    Private institutions offering genetic tests control the entire lifecycle of individuals' data from collection and storage to monetization. This control creates significant transparency issues, as it is often unclear how data is monetized, which parties benefit, or what agreements include an individual’s data. Users may opt out of participation, but unregulated data flows ensure that their genetic footprint persists indefinitely. This lack of transparency is particularly concerning in cases of bankruptcy or acquisition. For example, in 2024, Tempus AI acquired Ambry Genetics, a company that sequences genetic data for 400,000 patients annually \cite{tempus2025}. Tempus stated its intent to "leverage this data and augment Tempus' current data offering," raising questions about how patient data is repurposed and monetized in such deals. Ultimately, an individual’s decision to submit genetic data for testing often entails relinquishing full ownership of their genetic information. Without robust safeguards, privacy risks emerge from opaque data practices and leaky interfaces, leaving individuals vulnerable to misuse or unauthorized access to their genetic data. 
    
    \item \textbf{Misinformed Actions: } The interpretation of genetic test results and the subsequent actions taken by individuals are profoundly personal, often influencing lifestyle changes, significant financial decisions, and critical medical choices. For instance, individuals with BRCA1 gene mutations indicating an elevated breast cancer risk may opt for preventive interventions. These decisions are particularly sensitive as the interpretation of genetic data evolves alongside advancements in scientific methods. However, the relational nature of genetic data complicates individual autonomy and access to information. It is possible for someone to receive conclusions about themselves indirectly through the test results of genetic relatives. This dynamic, combined with the widespread availability of unregulated genetic tests and the evolving nature of genetic science, increases the potential for harm if individuals act on information that is incomplete, inaccurate, or probabilistic rather than deterministic.
    % The interpretation of genetic test results and the actions taken by users are deeply personal. \textcolor{blue}{These results can drive significant lifestyle changes, major financial decisions, and critical medical choices, such as pursuing aggressive interventions when BRCA1 gene mutations indicate elevated breast cancer risk} \sverror{someone make this sound better :)} \cwerror{better?}. Such decisions are particularly crucial, given that our understanding and interpretation of genetic test results continue to evolve alongside advancements in scientific inference methods. However, the inherently relational nature of genetic data poses challenges to an individual’s autonomy and access to information, as it is possible for someone to receive secondhand conclusions about themselves based on tests results of their genetic relatives. \textcolor{blue}{This characteristic, coupled with evolving science and mass availability of unregulated genetic tests, creates the risk of significant harm if individuals act on incorrect, incomplete, or non-deterministic information.} 
  


    \item \textbf{Financial impact}: Individuals may face financial repercussions if insurance or legal policies fail to provide adequate protection or if coverage is denied. Beyond medical procedures, financial impacts can also arise on a case-by-case basis, such as in situations where genetic data is held for ransom, legal defense is required, or individuals are forced to purchase direct-to-consumer tests (e.g., in order to purchase a life insurance policy).
\end{enumerate}

\section{Case Studies}

\label{sec:studies}

We apply our risk assessment framework to four case studies which highlight regulatory gaps across the Four Pillars. These case studies, ranging from past events to speculative scenarios, all involve privacy loss or potential discrimination. We choose case studies that illustrate genetic data leakage in both, a pillar-to-pillar and one-to-many pillar settings.

% While some align with specific Pillars, others span multiple Pillars due to the leaky protections in which genetic data, initially collected within one Pillar, can be used across others.

\subsection{Genetics and Education}
\label{case:education}

Polygenic scores (PGS) are popular in the social/behavioral sciences for predicting social outcome traits, such as educational attainment (EA; number of schooling years completed by an adult). We will use EA as the example throughout this case study, but note that there is also interest in predicting standardized testing scores, performance in mathematics, and other traits with substantial environmental influences. 


A common metric for assessing PGS accuracy is the percentage of trait variance it explains: higher percentages indicate better predictive performance. A recent 23andMe EA PGS, based on data from over 3 million customers of European descent, explains 12-16\% of the variance \footnote{As a useful comparison, mother's education explains 15\% of the variance in EA \cite{lee_gene_2018}} in educational attainment (EA) \cite{okbay_polygenic_2022}. This effectively means that 50-70\% of individuals with PGS scores in the top 10\% for EA are predicted to graduate college. However, the PGS accuracy significantly decreases when applied to African American customers. This is an example of the commonly observed ``portability problem'' \cite{martin_human_2017}, where a PGS derived from GWAS in one population predicts poorly in another due to confounding\footnote{Confounding in GWAS can be genetic (non-causal variants correlated with causal ones) or environmental (non-causal variants correlated with causal environmental factors), leading to potential statistical artifacts in effect sizes.}.


There have been several calls to use EA PGS to inform education policy. Harden et al. propose the use of math-performance PGS to identify ``leaks'' in the education system: for example, by identifying high math PGS students who perform poorly, they claim educators could pinpoint \textit{why} and \textit{how} students are failing to reach their potential\footnote{An example they use: 31\% of high PGS students in good schools take calculus, compared to 24\% of the same-scoring students in poor-performing schools.} \cite{harden_genetic_2020}. Plomin \& von Stumm take it further: they use the term ``precision education'' (akin to precision medicine) to propose a tailor-made, individualized education that is genetics-informed \cite{r_new_2018}. Statements like this, combined with statements such as ``students with higher polygenic scores for years of education have, on average, higher cognitive ability, better grades and come from families with higher SES [socioeconomic status]'' \cite{smith-woolley_differences_2018} are cause for concern because they invoke a sense of genetic determinism. However, other predictors (parents' educational status, socioeconomic status) explain similar amounts of variance in EA \cite{lee_gene_2018, morris_can_2020} and—unlike DNA—are mutable through social policy changes. 

Several of our values would be violated if children were required to submit their DNA (Right to Action) or educational opportunities were denied to children based on their genetic potential (Opportunities for Advancement). Through the vulnerabilities of unsettled science and the rapid evolution of genetic methods, harms such as leakage to the family can occur and affect not only children but their families and future.

% \sverror{a bigger issue I have with this case is that we are asserting that EA PGS is bad. but are not explaining why. Is it because the predictive power is low? or that the science is dicey? we need to explain that} \cwerror{Added the sentence at the end of the 3rd paragraph—is this better?}


\subsection{DTC Genetic Data Brokerage and Leakage}
\label{case:dtc}

Several FDA-approved 23andMe tests\footnote{23andMe tests with FDA 510(k) clearance are verified as substantially equivalent to existing devices or methods. Only health-related tests, such as those identifying genetic risks, carrier status, or drug responses, qualify for this review.} provide customers with results containing sensitive information, including carrier screening reports, genetic health risk assessments, and BRCA1/BRCA2 variant analyses. In addition, they offer genealogy (to identify relatives) and ancestry services. In an attempt to protect customer privacy, 23andMe offers an opt-in/opt-out policy when it comes to sharing their samples in studies with third-parties.


% FDA approval for certain tests\footnote{23andMe tests with FDA 510(k) clearance are verified as substantially equivalent to existing devices or methods. Only health-related tests, such as those identifying genetic risks, carrier status, or drug responses, qualify for this review.}, including carrier screening reports, genetic health risk reports, and BRCA1/BRCA2 variant tests. In addition, they offer genealogy (to identify relatives) and ancestry services. 23andMe consent forms give the user a choice of opting in or out of sharing their samples in studies with third parties.

In 2023, hackers accessed 14,000+ 23andMe user accounts by using recycled login information (credential stuffing). Due to the relatives linkage feature in genetic data reports, the hackers were able to then access information about an additional 5.5M users \cite{23andme_addressing_2023}; there were also reports that the hacker targeted Ashkenazi Jewish and Chinese ancestry profiles \cite{carballo_23andme_2024}. Among the many values violated, the Right of Privacy and Ownership of the Genome are particularly relevant to this case as 23andMe did not immediately notify compromised users or the public. Several class action lawsuits were filed (e.g., \cite{santana_23andme_2023}), with plaintiffs complaining that 23andMe failed to adequately protect their sensitive information. 23andMe responded by enforcing multifactor authentication (MFA) of user accounts, which was previously optional \cite{whittaker_23andme_2023}, and a serious technological vulnerability. A key gap that allowed for this issue to occur is the lack of consensus around security responsibility, which resulted in the disproportionate and harmful targeting of underrepresented groups. The 23andMe breach highlights the consequences of the relational nature of genetic data, where an individual's information can be compromised through poor public protections, even if they personally take precautions. As one Reddit user quipped ``Your genetic data is only as secure as your relatives' passwords.'' 
% \footnote{\url{https://www.reddit.com/r/23andme/comments/1ayw9y3/what_specific_privacy_concerns_do_you_have_about/}}."

\subsection{Genetic Data Collection of Detained Noncitizens}
\label{case:cpt}

A recent Georgetown University report from the Center of Privacy and Technology (CPT) examines the U.S. federal government's practice of collecting DNA from detained migrants \cite{glaberson_raiding_2024}. This practice, which began with the 2005 DNA Fingerprint Act and expanded significantly in 2020—a mandate that the Department of Homeland Security must collect DNA from \textit{all} detainees, even briefly detained\footnote{States have varying rules about DNA collection from suspects and detainees: 28 states allow DNA to be taken from someone \textit{before} they are convicted of any crime and most states restrict DNA collection to cases involving felony charges or convictions, rather than allowing collection for any arrest or detention \cite{samuels_collecting_dna_2012}.}—has led to a dramatic increase in detainee representation in CODIS (the federal DNA database), rising from 0.21\% in 2019 to 9.21\% in 2023. By 2020, approximately 25,000 noncitizens were added to the database under the ``detainee'' classification.

The CPT report highlights several concerning aspects of this practice, particularly its disproportionate impact on migrants of color and issues of consent. In our Risk Assessment Framework, this immediately violates the Right to Action and Right to Knowledge. Many migrants undergo DNA collection without understanding its implications, sometimes believing it to be a COVID-19 test or submitting under threat of criminal prosecution. The report argues that this practice violates the Fourth Amendment in collecting DNA from detainees without probable cause. Additionally, even though CODIS was designed to be privacy-protective by collecting only 20 markers—a tiny slice of the genome believed to be medically neutral—this limited data can still identify relatives through partial matching and, as an analysis 25 years later would show, are slightly informative of disease risk \cite{banuelos_associations_2022}. The expansive collection of genetic data has far-reaching implications through the guilt-by-association vulnerability. Since CODIS records are difficult to expunge, they can restrict Opportunities for Advancement not only for detainees but also for their relatives and descendants.

\subsection{Underwriting Life Insurance with AI and Genetic Data}
\label{case:insurance}

Life insurance underwriting is the process whereby an insurance company uses personal and health information to assess the risk of insuring an applicant. The relationship between applicants and insurance companies is already fraught: \cite{devnos_genomics_2016} find that patients are more likely to share their genetic info with Google than insurance companies. The future of life insurance underwriting is expected to become increasingly computational and automated via the usage of AI and the collection of more personalized, individual genetic data \cite{filabi_ai-enabled_2021, balasubramanian_insurance_nodate, rothstein_time_2018, koleva-kolarova_financing_2022}. While medical records and demographic factors can be used for mortality analysis, the inclusion of genetic factors can mean that risk prediction can be performed much earlier in an applicant's life without necessarily the same amount of records as an older individual, emphasizing the risk of genetic discrimination \cite{karlsson_linner_genetic_2022} based on the \textit{potential} to have risk factors (e.g., a genetic risk for high blood pressure versus clinically-measured high blood pressure). Life insurance is not covered by GINA and life insurance companies can access medical records (which may include genetic test results) as part of an application. Bills have been introduced in several states that would restrict the use of genetic information in underwriting, but these efforts largely failed: As of 2022, out of thirty-seven proposed bills across all states, three were introduced, eight were signed by the governor, demonstrating the geographical dependency of policy-based protection \cite{vermont_legislature_httpslegislaturevermontgovdocuments2022workgroupshouse20commercegenetic20testingwitness20documentswjill20rickardgenetic20testing20legislation20across20states4-20-2022pdf_nodate}.

Life insurance companies may be interested in using polygenic scores (PGS) to assess an individual's risk for various diseases. Indeed, in early 2024, U.S.-based life insurance company MassMutual and U.K.-based Genomics plc announced a partnership, offering free genetic testing to MassMutual's life insurance customers \cite{massmutual_genomics_2024}. However, the press release stressed that MassMutual would \textit{not} receive individual results and that \textit{current} premiums/policies would be unaffected \cite{massmutual_genomics_2024}. This example particularly highlights the leaky interface between pillars, where the Right of Privacy and Right of Action are called into question. AI insurance underwriting algorithms already suffer from racial biases \cite{lee_ai_2022}. That PGS suffer from the portability problem (itself caused by systematic biases in training datasets) means that these biases could be perpetuated in genetics-informed life insurance underwriting. Additionally, given the relational aspect of genetic data, genetic-based underwriting could affect other biological relatives' applications, harming not only relatives but any others with the same genetic markers that are associated with mortality risks. The potential harms of using genetic data as a part of underwriting in life insurance are one such example of violations of genetic privacy with the rapid evolution and usages of genetic data with unsettled science. 

\section{Recommendations}
\label{section:recommendations}

We argue that for public infrastructure and private companies to productively and ethically make use of genetic data, several amendments need to be made to the current genetic data governance system. 
%We structure recommendations that rely on protecting the values outlined in our risk assessment framework. 
Our three recommendations address open privacy concerns, legislative scoping for policy changes, and best practices for bodies handling genetic data.

\subsection{Recommendation 1: Redefining Genetic Data}

\textbf{Issue}: Legal policy surrounding genetic privacy notably excludes deidentified or anonymized data from protection.

\textbf{Recommendation}: Given that we argue that genetic data is unique compared to any other identifying data, we suggest genetic data be defined using the following language: 

``Genetic data refers to any information relating to an individual’s genetic characteristics, including but not limited to DNA or RNA sequences, gene expression profiles, or any data derived from a biological sample—including that of a relative—regardless of format. \textbf{This data is always considered identifiable} by its nature—or personally identifiable information (PII)—as it pertains to unique biological attributes that can potentially be linked to a specific individual, their biological relatives, or identifiable group. For the purposes of this definition, any de-identified, pseudonymized, or anonymized genetic information shall be treated as genetic data, regardless of measures taken to mask individual identities, recognizing that the inherent characteristics of genetic information may enable re-identification through advanced technological or data cross-referencing methods.''

\subsection{Recommendation 2: Extending Protections for Genetic Discrimination}

\textbf{Issue}: GINA is a vital piece of federal legislation that protects against genetic discrimination in employment and health insurance domains. However, other domains in which there exists potential for genetic discrimination  are \textit{not} protected by GINA: other insurance domains (life, long-term care, disability), housing, and education. 

\textbf{Recommendation}: We recommend that additional federal laws should be enacted to extend GINA's coverage beyond employment and health insurance. The most comprehensive state genetic anti-discrimination law is California's CalGINA, which includes housing, mortgage brokerage, education, and more.  However, despite these extensions, CalGINA does not cover life, disability, or long-term care insurance. We suggest that extension bills for GINA should explicitly cover life, disability, long-term care, as well as education and any other opportunities for advancement to be protected against genetic discrimination. It should be written in such a way that prohibits \textit{any} barrier to opportunity, even those not anticipated at the time of writing. Additionally, legislation should be written to explicitly bar genetic risk for \emph{complex traits} known to have significant environmental influences (such as cardiovascular disease) from being considered a preexisting condition.


\subsection{Recommendation 3: A Genetic Data Regulation Framework}

\textbf{Issue}: Current regulations were designed to govern one application of genetic data—or Pillar—at a time. This has led to ``leaky protections'', where the use of genetic data in one Pillar can affect opportunities and decisions in other Pillars (e.g., clinical tests being used in life insurance).

\textbf{Recommendation}: To eliminate \emph{leaky protections}, we suggest a comprehensive and uniform regulatory framework that encompasses all usage domains whether genetic data is collected, analyzed, or used in any type of inference. Below, we build upon specific values inherent in privacy rights to ensure that responsibility is placed on the organizations that hold and analyze data, rather than the individuals from whom it was collected \cite{solove_limitations_2022}. 

\begin{enumerate}
    \item \textbf{Entity approval}: Entities—whether they be universities, health care providers, hospital systems, or corporations—that collect or house any type of genetic data (DNA, RNA, or even genetic test results) must have prior regulatory approval. Approval requires a clear commitment to the basic rights of the individuals whose genetic data the entity will collect or own, such as data privacy, security, the right to deletion, etc. 
    %Broadly requiring approval for any public or private entity to collect or house genetic data will ensure that genetic data is given the same protections across all domains. 
    
    \item \textbf{Test approval}: After \emph{entity approval}, we propose regulations for the inferential tests, specifically those that return results to either an individual or a medical practitioner for any actionable result. At a minimum, entities would be required to publicly release detailed white papers for each test that detail laboratory procedures, quality control steps, inferential models used, presentation of results, and any other necessary details that would allow one to recreate the analysis if given the same data. 
    %An open question here is whether to require further regulatory approval for each test and what that regulatory process would encompass. It is not the goal to stymy genetics research, but to ensure that the science underlying tests is sufficiently rigorous.
    
    \item \textbf{Powers given to the individual}
    \begin{itemize}
        \item Individuals should be able to request \textbf{data removal}. This would include specifically (1) the destruction of the physical sample, (2) deletion of data, (3) removal of any identifiers, and (4) that any removed data no longer influence the results of any downstream models. 
        %This last point is purposefully vague, but could include periodic re-training of models with the requested data deleted.
        
        \item \textbf{Third-party data transfers} should only occur between companies authorized to collect or own genetic data, with recipient companies subject to the same regulations. We recommend providing individuals with the option of blanket \textbf{opt-out} from data transfers and specific \textbf{opt-in} choices for each third party. Individuals must be notified of the transfer and given a reasonable time to opt-out. 
        %They can also request data deletion after the transfer, as the receiving entity is bound by the same regulations, including right to deletion.

        
        % should only occur between companies that are both authorized to collect or own genetic data. Additionally, recipient companies must be subject to the same regulations as the data collector. We recommend that any third-party data transfers for an active company include a \textbf{blanket opt-out} option for individuals to exclude themselves from brokered data, along with specific \textbf{opt-in} choices for each third party involved. Individuals must be preemptively notified of the transfer and given a reasonable time frame to opt out. Because the receiving entity is subject to the same rules and regulations that ensure the right to deletion at any time, individuals can still request their data be deleted after the transfer.









        \item \textbf{Research usage} is common among DTC companies, academia, and hospitals, where data provided per a test can be pooled together as a dataset for research (both internally and via external collaborations). We suggest an \textbf{opt-out} strategy here for individuals, with a specification on whether they consent to their data becoming publicly available in any form (e.g. public datasets, open sourced GWAS weights, trained models, etc).  
        
        \item \textbf{Incidental genetic discoveries} can be possible in research use cases where data for a particular test was used for other tests. In these cases, where knowledge of a particular trait can also be harmful and uninformative to the individual, we suggest an \textbf{opt-in} strategy for an individual to blanket choose if they wish to hear any secondary discoveries, or forward any secondary discoveries to their medical practitioner, to be enacted on at their discretion. 
        
    \end{itemize}

    
    \item \textbf{Bankruptcy and Acquisition}: As detailed by third-party data transfers, the owning of genetic data would also apply to companies who acquire any genetic data as part of assets through acquisitions or bankruptcy. Thus, all entities involved must already have approval to handle genetic data. In the rare event that there are no entities to handle the data, we suggest a similar protocol structure as nuclear waste \cite{doctorow_personal_data_2008}, where all physical sample, data, and models are subsequently destroyed and cannot be recovered.
\end{enumerate}

 % It is important to consider who will enforce such a regulatory framework, which could possibly be a new regulatory body or fall under the purview of pre-existing regulatory bodies such as the FDA, FTC, or HHS. We recommend that the regulation be implemented at the federal level to prevent \textit{leaky protections}. \sverror{this doesn't answer the question though of WHO would enforce it}
%, which can occur with state-based regulations, where individuals may lose or be uncertain about their protections depending on their state of residence.



\section{Conclusions and  Open Challenges}
\label{section:challenges}

In this paper we outline the genetic data ecosystem and today's state of public protections. We identify critical gaps in current regulatory federal and state-level frameworks that prevent the productive, ethical, and safe adoption of genetic data for broad societal applications. To address this, we propose a novel risk assessment framework that offers key public protections and preserve's individuals personal liberties, and finally make three concrete policy recommendations. 
% While our policy recommendations account for improved user protections, we identify three remaining open challenges in regulating genetic data use, that lie beyond the scope of our work: First, our recommendations only cover \textbf{voluntary} genetic data collections.  In the case of involuntary genetic data collection, we reference Georgetown’s Center of Privacy and Technology (CPT) report and their accompanying suggestions \cite{glaberson_raiding_2024}. Second, the \textbf{relational} aspect of genetic data considerably implicates any biological relatives of individuals in a dataset. A relational theory of privacy is a complex phenomenon that remains highly discussed in academic literature \cite{viljoen_relational_2024, costello_genetic_2022, zigomitros_survey_2020, reviglio_i_2020}.  As such, we envision a future regulatory framework that would encompass the relational aspect of genetic data, of which we highlight three possible considerations/applications:

% These questions remain largely unanswered and require contributions from legal scholars, bioethicists, moral philosophers, privacy experts, human rights advocates, and public policy researchers. We raise these questions for the purpose of thoroughness in our evaluation of the use of genetic data in today's ecosystem.

While our policy recommendations focus on enhancing user protections, we also identify three key challenges in regulating genetic data usage that fall outside the scope of our work and that would benefit from further research.
%We present these challenges to ensure a comprehensive evaluation of genetic data use in the current ecosystem.

\subsection{The relational theory of privacy.}

\subsubsection{Should genetic data submissions or inclusion into genetic datasets require consent from biological relatives?}  %The framework should address issues regarding competing privacy interests between relatives. For example, w
When an individual does a genetic test they are, indirectly, testing (part of) their relative's genome.  What (if any) consent or privacy measures should be taken to protect their biological relatives who may not consent to such a test? Additionally, what should happen upon death? Do children ``own'' their parents' genome (and what ownership would, say, a full sibling have?) and can they make decisions (e.g., to delete the data)?

\subsubsection{Who has ownership of children’s DNA\@?}
Do parents have a ``fiduciary duty'' to protect their child's genetic data and protect them from the system vulnerabilities that we have outlined? Part of this fiduciary duty would involve whether a genetic test should be taken in the first place, e.g., the difference between a medically-necessary test and a DTC genetic test\footnote{23andMe's policy for minors is as follows: individuals under 18 that give assent and have a parent/guardian's permission can submit their sample \cite{23andme_research_2024}.} for some insights into the child's ancestry and genealogical relatives. 
%Return of results to the child (e.g., by their parents) may come with harm. The child may not understand the statistical nuances of genetic tests for complex traits, resulting in a harmful deterministic view of their future health and social outcomes. 
The relational aspect of genetic data also comes into play here; for example, should consent be required from both parents to sequence their child's genome? See \cite{bala_who_2023} for a thorough discussion of child's genome ownership. 

%By sequencing the child's genome, half of the parents' genomes are also sequenced. Companies or clinical services that sequence children or embryos (e.g. Orchid genetics performs whole genome sequencing on embryos for disorders and predispositions \cite{OrchidHealth2025}) call into question who owns this DNA, especially after the sequenced individuals become adults.
    % move to open challenges; remove biosecure

\subsection{International genetic data transfers.}
 Our paper focuses on American regulatory and legislative bodies and American companies. In the DTC genetic testing space, individuals may be less likely to trust foreign companies: Nucleus Genomics\footnote{https://mynucleus.com} proudly states on their home page that customer genomes are sequenced in the U.S. and the data is never sent/stored overseas\footnote{23andMe also genotypes samples in the U.S. \cite{23andme_what_2024}, but this is not a selling point of theirs.}. Should there be restrictions on genetic data transfer between countries? For example, one hypothetical scenario might involve a political refugee who has medically-necessary genetic testing done within the U.S, but because of leaky data transfer rules, genetic information is transferred to their home country where it might be used to identify family members now at risk for persecution.


\subsection{The deployment of AI models for genetic data.}
\label{challenges:ai}

Our paper has largely shied away from discussions about genetic inference and AI for two reasons: (1) deployment of AI technologies for genetic inference is nascent and, most importantly, (2) risks and vulnerabilities exist even with current simple linear models for genetic inference. AI will, we believe, exacerbate the risks we have outlined as they will likely result in seemingly better trait prediction by modeling nonlinear interactions and taking advantage of gene-environment correlations\cite{yang_advancing_2024}. However, the uncertainties associated with the basic scientific inference (that we outline in Section~\ref{section:genetic_data}) will be compounded because of the lack of interpretability and complexity of AI models. 

%Large language models that incorporate genetic data and other data modalities have already demonstrated improved accuracy, e.g., Google's Med-Gemini-Polygenic \cite{yang_advancing_2024}. As we described in Section 2, even linear models suffer from interpretability, with statistical geneticists unable to disentangle genetic and environmental confounders, meaning there is not a clear understanding as to what is \textit{biological}; this task would be made more difficult if AI technologies were deployed.

%\section{Conclusion}
\label{sec:Conclusion}
This work evaluates proprietary and open-weight models in agentic frameworks for handling ambiguity in software engineering. In code generation, to effectively integrate new information into the solution, an agent must detect ambiguity and ask targeted questions. Our key findings are:
\begin{itemize}[itemsep=0pt, topsep=0pt]
    \item Given an underspecified input, Claude Sonnet 3.5 and Claude Haiku 3.5 with interaction can achieve 80\% of their performance with a well-specified input. In contrast, open-weight models struggle: Deepseek relies on navigational cues to locate relevant files, while Llama 3.1 70B extracts limited information from the user.
    \item LLMs do not interact unless explicitly prompted, and their ambiguity detection is highly sensitive to prompt variations. Only Claude Sonnet 3.5 achieves a higher accuracy of 84\% in distinguishing between well-specified and underspecified input.

    \item Claude Sonnet 3.5, Haiku 3.5, and Deepseek effectively extract new, detailed user information, whereas Llama 3.1 struggles to ask the right questions.
    
\end{itemize}
Despite these advances, a gap remains between resolve rates for underspecified vs. fully specified issues. Open-weight models need better interaction strategies to improve resolution, while proprietary models, particularly Claude Haiku 3.5, require stronger prompting to engage interactively. This work establishes the current state-of-the-art in handling ambiguity through interaction, breaking the resolution process into multiple steps.




% \newpage 
% \newpage
\appendix
\onecolumn

\part{}
\section*{\centering \LARGE{Appendix}}
\mtcsettitle{parttoc}{Contents}
\parttoc

\clearpage

\section{Related Work}
\label{sec:relatedwork}
% \paragraph{Tool Usage and Toolchain Management} Research in this area focuses on how intelligent agents design and optimize tool networks to effectively execute complex tasks, particularly by dynamically generating, selecting, and combining tools based on task requirements.This includes methods for automated tool generation and optimization, emphasizing systems that can adaptively choose and adjust tool combinations according to different task needs.

% \paragraph{Multi-Agent Systems and Collaboration} Research in Multi-Agent Systems has explored how multiple intelligent agents can collaboratively solve complex tasks in dynamic environments. One significant contribution is the development of decentralized algorithms that allow agents to autonomously form beneficial collaborations and adapt to changing tasks without the need for a central server (DeLAMA) ~\citep{tang2024decentralizedlifelongadaptivemultiagentcollaborative}. Another key area of study focuses on collaboration among heterogeneous agents, where different agents with varied capabilities work together on complex tasks, such as cleaning large spaces, using hierarchical decision models to allocate sub-tasks effectively~\citep{liu2023heterogeneousembodiedmultiagentcollaboration}. Additionally, collaborative learning approaches like Collaborative Q-learning (CollaQ) enhance agent teamwork by decomposing the Q-function and introducing reward attribution techniques to improve performance in multi-agent environments, such as the StarCraft challenge ~\citep{zhang2020multiagentcollaborationrewardattribution}. Finally, research has also examined how multi-agent collaboration can enhance the performance of large language models (LLMs) in tasks like simulations and software development, highlighting the potential of intelligent agent collaboration to improve task outcomes~\citep{talebirad2023multiagentcollaborationharnessingpower}.

\paragraph{Code Generation and Task Solving with LLMs} Large Language Models (LLMs) have demonstrated remarkable potential in generating code to solve complex tasks. Prior studies highlight their effectiveness in mathematical computation ~\citep{zhou2023solving, wang2023mathcoder, gou2023tora}, tabular reasoning ~\citep{chen2022program, lyu2023faithful, lu2024chameleon}, and visual understanding ~\citep{suris2023vipergpt, choudhury2023zero, gupta2023visual}. Frameworks such as AutoGen ~\citep{wu2023autogen} and CodeActAgent~\citep{wang2024executable} extend this capability to agent-based tasks by interpreting executable code as actions. These models dynamically invoke basic tools based on environmental feedback, significantly expanding their utility. Despite their successes, these approaches often treat program generation processes independently, failing to model shared task features and limiting the reusability of functional modules across tasks.

\paragraph{Reusable Tool Creation and Abstraction} To address the limitations of single-use program generation, recent efforts have focused on creating reusable tools. CREATOR ~\citep{qian2023creator} separates the processes of planning (tool creation) and execution, while LATM ~\citep{cai2023large} and CRAFT ~\citep{yuan2023craft} pre-build tools using training and validation sets for task solving. However, these methods often generate a large number of tools, presenting challenges for their efficient reuse. Furthermore, while abstraction-based approaches like REGAL ~\citep{stengel2024regal} focus on extracting reusable tools from primitive programs, they primarily construct simple tools with limited functional complexity. Similarly, Trove ~\citep{wang2024trove} adopts a training-free approach by dynamically composing high-level tools during testing, but its reliance on self-consistency can lead to hallucinated knowledge, reducing accuracy in complex tasks.

\paragraph{Tool Selection for Complex Task Solving} Currently, research on tool selection and retrieval methods primarily focuses on selecting appropriate tools through retrieval mechanisms and LLM-based approaches. ToolRerank ~\citep{zheng2024toolrerank} uses adaptive truncation and hierarchy-aware reranking to improve retrieval results, while Re-Invoke ~\citep{chen2024reinvoketool} introduces an unsupervised framework with synthetic queries and multi-view ranking, enhancing both single-tool and multi-tool retrieval. COLT ~\citep{Qu_2024COLT} combines semantic matching with graph-based collaborative learning to capture relationships among tools, outperforming larger models in some cases. AvaTaR~\citep{wu2024avataroptimizingllmagents} automates the optimization of LLM prompts for better tool utilization, and DRAFT~\citep{qu2024DAFT} refines tool documentation through iterative feedback and exploration, helping LLMs better understand external tools. Despite progress, existing methods generally overlook cost-effectiveness and scalability in tool selection, and often struggle to efficiently adapt to new tools and task requirements in dynamic environments, leading to performance and efficiency bottlenecks. In contrast, our approach dynamically prioritizes tools by combining their relevance and structural importance, ensuring computational efficiency and scalability, thus enabling more effective solutions for complex tasks.
\section{Experimental Details}
\label{app:apexp}
\subsection{Open-ended Task}
\label{subsec:open}
\paragraph{Benchmark} We employed the benchmark proposed by Voyager~\citep{wang2023voyager}, using Minecraft as the experimental platform. Minecraft provides a sandbox environment where players gather resources and craft tools to achieve various goals. The simulation is built on MineDojo~\citep{fan2022minedojo} and uses Mineflayer~\citep{PrismarineJS2013} JavaScript APIs for motor control. 

\paragraph{Baselines}
We conducted a comprehensive comparison with four baselines. Except for Voyager, these methods were originally designed for NLP tasks without embodiment. Therefore, we had to reinterpret and adapt them for execution within the MineDojo environment, ensuring compatibility with the specific requirements of our experimental setup.
\begin{itemize}
    \item \textbf{ReAct:} ReAct~\citep{yao2022react} uses chain-of-thought prompting [46] by generating both reasoning traces and action
plans with LLMs. We provide it with our environment feedback and the agent states as observations.
    \item \textbf{Reflexion:} Reflexion~\citep{shinn2023reflexion} is built on top of ReAct~\citep{yao2022react}with self-reflection to infer more intuitive future actions.
    \item \textbf{AutoGPT:} AutoGPT~\citep{richardssignificant} is a popular software tool that automates NLP tasks by decomposing a high-level
goal into multiple subgoals and executing them in a ReAct-style loop. We re-implement AutoGPT by using GPT-4O to do task decomposition and provide it with the agent states, environment feedback,
and execution errors as observations for subgoal execution
We provide it with execution errors and our self-verification module.
    \item \textbf{Voyager:} Voyager~\citep{wang2023voyager} is a system that integrates an automated curriculum, a scalable skill library, and an iterative prompting framework based on environmental feedback to explore, store, and accumulate skill library within the Minecraft environment.
\end{itemize}


\paragraph{Metric}
The evaluation metric is based on the number of iterations required to progress through tool upgrades, from wooden to stone, iron, and finally diamond tools. Each execution of code is considered one iteration.

\paragraph{Model}
We leverage GPT-4o for text completion, along with the text-embedding-ada-002 API for text embedding. We set all temperatures to
0 except for the automatic curriculum, which uses temperature = 0.1 to encourage task diversity. 

\paragraph{Setting}
We set the maximum number of iterations to 160. For both \ours\ and Voyager, all agents are controlled by GPT-4o, with the number of tools retrieved per iteration set to 5. To ensure a fairer comparison, we removed the Tool Requirement Stage and bug-free checks in \ours\ , and allowed a maximum of 3 self-checks per iteration.

\paragraph{Item Types and Levels}
In the Minecraft task, there are different types and levels of items. Diamond tools are the highest level, and rare items such as golden apples also exist. High-level tools require some lower-level items to craft. Table \ref{tab:toollist} lists the key items in the Minecraft task.
\begingroup
\begin{table}[H]
\caption{List of item types and levels in the Minecraft task.}
\label{tab:toollist}
\vskip -0.1in
\setlength{\tabcolsep}{10pt} % 调整列间距
\begin{center}
\begin{small}
\begin{sc}
\begin{tabular}{l|c|c}
\toprule
\textnormal{\textbf{Category}} & \textnormal{\textbf{level}} & \textnormal{\textbf{Items}} \\
\midrule         
\midrule
\multirow{4}{*}{\multirow{3}{*}{\normalfont Tools}} 
              & \normalfont Wooden Tools & \normalfont Wooden\_Shovel,Wooden\_Pickaxe,Wooden\_Axe,Wooden\_Hoe,Wooden\_Sword \\
              \cmidrule{2-3}
              & \normalfont Stone Tools &\normalfont stone\_pickaxe, stone\_shovel,Stone\_Axe,Stone\_Hoe,Stone\_Sword   \\
              \cmidrule{2-3}
              & \normalfont Iron Tools &\normalfont iron\_pickaxe, iron\_axe, iron\_sword, iron\_shovel, iron\_hoe    \\
              \cmidrule{2-3}
              & \normalfont Diamond Tools &\normalfont diamond\_pickaxe, diamond\_sword, diamond\_axe, diamond\_shovel    \\
             
\midrule
\multirow{2}{*}{\multirow{1}{*}{\normalfont  Armor}} 
              & \normalfont Iron Armor &\normalfont iron\_chestplate, iron\_helmet, iron\_leggings  \\
              \cmidrule{2-3}
              & \normalfont Diamond Armor &\normalfont diamond\_chestplate, diamond\_helmet, diamond\_leggings, diamond\_boots     \\

\midrule
\multirow{3}{*}{\multirow{2}{*}{\normalfont  Food}} 
              & \normalfont Raw Food &\normalfont chicken, mutton, porkchop, rabbit, raw\_rabbit, spider\_eye, bone  \\
              \cmidrule{2-3}
              & \normalfont Cooked Food &\normalfont cooked\_beef, cooked\_chicken, cooked\_mutton, cooked\_porkchop, cooked rabbit  \\
              \cmidrule{2-3}
              & \normalfont Advanced Food &\normalfont golden apple    \\

\bottomrule
\end{tabular}
\end{sc}
\end{small}
\end{center}
\vskip -0.1in
\end{table}
\endgroup


\subsection{Agent Task}
\label{subsec:agent}
\paragraph{Benchmark}
We conducted experiments on two types of agent tasks, demonstrating {\ours}'s capabilities in both game-related and data science tasks.
\begin{itemize}
     \item \textbf{TextCraft:} We evaluate {\ours} on the TextCraft dataset~\citep{futuyma1988evolution}, which challenges agents to craft Minecraft items in a text-only environment~\citep{cote2019textworld}. Each task instance provides a goal and a sequence of crafting commands, which include distractors. We use depth-2 splits for testing and reserve a subset of depth-1 recipes for development, resulting in a 99/77 train/test split.
    \item \textbf{InfiAgent-DABench:} We also test {\ours} on the InfiAgent-DABench benchmark~\citep{hu2024infiagent}, which evaluates LLM-based agents on end-to-end data analysis tasks. This benchmark consists of 257 questions across 52 CSV files, with each question corresponding to a unique CSV file. Agents are required to generate code to analyze data and produce the specified output format. We randomly selected 20 CSV files and their associated question-answer pairs as training data, resulting in a train/test split of 98/159 instances.
\end{itemize}

\paragraph{Baselines}
We compare \ours\ with three methods described below.
\begin{itemize}
     \item \textbf{ReAct:} In this setting, we employ the executor to interact iteratively with the environment, adopting the think-act-observe prompting style from ReAct~\citep{yao2022react}.
     \item \textbf{Plan-Execution:} In contrast, the Plan-and-Execute approach~\citep{shridhar2023art, yang2023intercode} generates a plan upfront and assigns each sub-task to the executor. To ensure each step is executable without further decomposition, we provide new prompts with more detailed planning instructions.
    \item \textbf{Reflexion:} In the Reflection setting~\citep{shinn2023reflexion}, the agent engages in self-reflection after each step, drawing on environmental feedback and exploration history. 
\end{itemize}

\paragraph{Metric} 
The most practically important aspect of the solutions is correctness. For Textcraft, we verify whether the agent’s inventory contains the goal item. For DABench, we check if the agent’s final answer matches the ground truth.

\paragraph{Model}
During training, we use GPT-4o to construct the tool library with a temperature setting of 0. In the testing phase, we conduct a comprehensive comparison of various open-source and closed-source models. The open-source models include \textit{Qwen2.5-7B-Instruct, Qwen-Coder-7B-Instruct, Qwen2.5-14B-Instruct, Deepseeker-Coder-6.7B-Instruct, and Deepseeker-Coder-33B-Instruct}, while the closed-source models primarily include \textit{gpt-3.5-turbo-1106} and \textit{Claude-3-haiku}. During testing, the temperature is set to 0.3, and each experiment is repeated 3 times, with the average result reported.

\paragraph{Setting} 
For ReAct, Reflexion, and \ours\ , the maximum number of steps is set to 20. For Plan-Execution, the maximum number of steps for each sub-task is set to 8. In \ours\ , the number of tools retrieved during testing is limited to 3.



\subsection{Single-turn Code Task}
\label{subsec:code}
\paragraph{Benchmark}
To further explore {\ours}'s potential, we evaluated it on single-turn code generation tasks spanning mathematical reasoning, date comprehension, and tabular reasoning:
 \begin{itemize}
     \item \textbf{MATH:} We used a subset of the MATH dataset~\citep{hendrycks2021measuring}, focusing on 405 level-4 and level-5 algebra problems (MATH contains 5 levels of difficulty) that require textual understanding and advanced reasoning. We randomly selected 200 examples from the test set of the MATH dataset to construct the tool network, resulting in a train/test split of 200/405.
     \item \textbf{Date:} We use the date understanding task from BigBenchHard~\citep{srivastava2022beyond}, which consists of short word problems requiring date understanding. We follow the data splits provided by REGAL\citep{stengel2024regal}, resulting in a train/test split of 66/180.
     \item \textbf{TabMWP:} We further extend our general experiments on MATH by testing on TabMWP~\citep{grand2023learning}, a tabular reasoning dataset consisting of math word problems about tabular data. Based on the CRAFT~\citep{yuan2023craft} splits, we selected 470 problems from levels 7 and 8 (TabMWP contains 8 levels) from the 1,000 test examples. Additionally, we randomly selected 200 examples from the TabMWP training set, resulting in a train/test split of 200/470.
\end{itemize}

\paragraph{Baselines}
For these tasks, we use Programs of Thoughts (PoT)~\citep{chen2022program} and other existing tool-making methods as baselines for comparison.

\begin{itemize}
    \item \textbf{PoT:} The LLM utilizes a program to reason through the problem step by step~\citep{chen2022program}.
   \item \textbf{LATM:} LATM~\citep{cai2023large} samples 3 examples from the training set and create a tool for the task, which is further verified by 3 samples from the validation set. The created tool is then applied to all test cases.
    \item \textbf{CREATOR:} CREATOR~\citep{qian2023creator} disentangle planning (tool making) from execution, enabling Large Language Models (LLMs) to autonomously create a specific tool for each test case during inference.
     \item \textbf{CRAFT:} CRAFT~\citep{yuan2023craft} constructs task-specific toolsets by generating a tool for each training example. During testing, it utilizes a tool retrieval module and a reasoning process akin to CREATOR, generating a function first and then producing the corresponding invocation code. 
      % \item \textbf{Trove:} Trove~\citep{wang2024trove} introduces a training-free method based on self-consistency, where LMs interact with the toolbox through three modes—IMPORT, SKIP, and CREATE. Each mode is executed K times, and from the 3K responses, the function from the most consistent and optimal response is added to the toolbox.
      \item \textbf{REGAL:} During training, REGAL~\citep{stengel2024regal} refines primitive programs by extracting functions. In the testing phase, it retrieves both tools and refactored programs—comprising original and refactored versions—to generate a program that effectively solves the task. 
\end{itemize}
\paragraph{Metric}
We use correctness as the evaluation metric, measuring whether the execution outcome of the solution program exactly matches the ground-truth answer(s).
\paragraph{Model}
The models for the single-turn code generation task are the same as those used for the Agent Task, as presented in Section \ref{subsec:agent}.
\paragraph{Setting}
To ensure a fair comparison, we make slight adjustments to each method. For all methods, we allow up to 3 times for format checking and correction, as small models may not always follow the required output format. For PoT, we use 6 fixed examples of basic tool usage as few-shot. CREATOR employs the rectifying process, while for CRAFT, we use the same training set as our method and construct the tool library with GPT-4o, retrieving 3 tools during testing. For Regal, we use PoT along with GPT-4o to obtain ground-truth code, select the correct program, and have GPT-4o reconstruct it. To maintain fairness in tool generation quality, we standardize the few-shot examples of basic tools and retrieve 3 tools, along with 3 usage examples from the current tool library, avoiding errors from pruned tools. For our method, we train with GPT-4o, retrieving 3 tools and their corresponding usage examples during testing, while fixing the basic tool few-shot examples to 3, ensuring consistency with PoT’s total few-shot count.
\section{More Results}
\label{app:apresults}
\subsection{Open-ended Task}
\label{subsec:open-results}
\paragraph{More complex tools} 
Our hierarchical graph architecture offers significant advantages in handling complex tasks and large-scale systems. As shown in Figure \ref{fig:toolnet1}, Trial 1 starts with five nodes occupying three layers, and evolves into a five-layer network, with an increasing number of inter-tool calls. As shown in Figure \ref{fig:toolnet2}, Trial 2 starts with four nodes occupying four layers, and evolves into a five-layer network with more inter-tool calls. As shown in Figure \ref{fig:toolnet3}, Trial 3 starts with four nodes occupying three layers, and evolves into a six-layer network structure, with a growing number of inter-tool calls. Our tool graph becomes progressively more complex, flexibly expanding and optimizing its components. These results demonstrate that our method can generate tools that call each other, and combine them into more complex tools. This not only enhances scalability but also facilitates the creation of more sophisticated tools, enabling the solution of increasingly complex problems.


\paragraph{More types of inventory} Our method is able to generate more inventory types than Voyager. As shown in Table \ref{tab:Number}, we can see that {\ours} produces more inventory types in all three trials compared to Voyager.

The inventory collected by {\ours} in each trial is

\begin{itemize}
    \item \textbf{Trial 1:}  \textit{oak\_log, birch\_log, oak\_planks, birch\_planks, crafting\_table, stick, wooden\_pickaxe, dirt, cobblestone, coal, stone\_pickaxe, raw\_copper, furnace, copper\_ingot, andesite, raw\_iron, granite, iron\_ingot, iron\_pickaxe, shield, diorite, raw\_gold, lapis\_lazuli, redstone, diamond, diamond\_pickaxe, bucket, gold\_ingot, iron\_chestplate, arrow, iron\_sword, iron\_helmet, diamond\_sword, diamond\_helmet, lightning\_rod, chest, iron\_axe, iron\_leggings, sandstone, dandelion, spider\_eye, string, iron\_shovel, copper\_block, iron\_door, iron\_hoe, kelp, bow, dried\_kelp, torch, cooked\_beef, gray\_wool, cobbled\_deepslate, tuff, diamond\_leggings, bone, diamond\_chestplate, chicken, white\_banner, cooked\_chicken, egg, feather, oak\_sapling, apple, acacia\_log, golden\_apple, diamond\_axe}

    \item \textbf{Trial 2:}  \textit{oak\_sapling, oak\_log, stick, oak\_planks, crafting\_table, wooden\_pickaxe, dirt, cobblestone, stone\_pickaxe, diorite, raw\_iron, coal, lapis\_lazuli, gravel, furnace, iron\_ingot, raw\_copper, sandstone, granite, iron\_pickaxe, andesite, raw\_gold, gold\_ingot, diamond, diamond\_pickaxe, redstone, cobbled\_deepslate, bucket, iron\_sword, arrow, bow, bone, birch\_log, chest, amethyst\_block, calcite, smooth\_basalt, iron\_chestplate, diamond\_sword, diamond\_helmet, iron\_leggings, diamond\_boots, water\_bucket, string, orange\_tulip, mutton, white\_wool, porkchop, dandelion, cooked\_porkchop, cooked\_mutton}

    \item \textbf{Trial 3:}  \textit{jungle\_log, stick, oak\_sapling, jungle\_planks, crafting\_table, dirt, wooden\_pickaxe, cobblestone, stone\_pickaxe, raw\_iron, raw\_copper, furnace, iron\_ingot, iron\_pickaxe, coal, diorite, lapis\_lazuli, andesite, moss\_block, clay\_ball, redstone, raw\_gold, cobbled\_deepslate, granite, diamond, diamond\_pickaxe, copper\_ingot, gunpowder, bucket, gravel, gold\_ingot, oak\_log, iron\_sword, iron\_chestplate, chest, diamond\_sword, spruce\_sapling, rotten\_flesh, bone, rose\_bush, water\_bucket, string, oak\_planks, grass\_block, diamond\_helmet, iron\_leggings, emerald, snowball, rabbit\_hide, rabbit, spruce\_log, cooked\_rabbit, diamond\_boots}
\end{itemize}


The inventory collected by Voyager in each trial is
\begin{itemize}
    \item \textbf{Trial 1:}  \textit{oak\_log, birch\_log, oak\_sapling, birch\_sapling, oak\_planks, stick, crafting\_table, wooden\_pickaxe, dirt, cobblestone, stone\_pickaxe, raw\_copper, white\_tulip, coal, furnace, copper\_ingot, granite, raw\_iron, iron\_ingot, lightning\_rod, iron\_pickaxe, pink\_tulip, orange\_tulip, sandstone, shears, shield, diorite, cobbled\_deepslate, iron\_block, chest, tuff, lapis\_lazuli, redstone, diamond, raw\_gold, gold\_ingot, diamond\_pickaxe, diamond\_helmet, diamond\_sword, sand, andesite, arrow, bone, iron\_chestplate, beef, leather, oak\_leaves, porkchop, cooked\_beef, leather\_leggings}

    \item \textbf{Trial 2:}  \textit{dirt, oak\_log, oak\_planks, crafting\_table, stick, oak\_sapling, wooden\_pickaxe, cobblestone, coal, stone\_pickaxe, raw\_iron, granite, lapis\_lazuli, raw\_copper, furnace, iron\_ingot, copper\_ingot, iron\_helmet, iron\_pickaxe, diorite, andesite, salmon, ink\_sac, iron\_chestplate, lightning\_rod, cooked\_salmon, stone, stonecutter, rotten\_flesh, gravel, flint, chest, iron\_leggings, copper\_block, cobbled\_deepslate, tuff, diamond, diamond\_pickaxe, raw\_gold, gold\_ingot, redstone, diamond\_sword, egg, diamond\_boots, diamond\_axe}

    \item \textbf{Trial 3:}  \textit{jungle\_log, jungle\_planks, oak\_sapling, oak\_log, crafting\_table, stick, wooden\_pickaxe, dirt, cobblestone, coal, stone\_pickaxe, raw\_copper, furnace, copper\_ingot, magma\_block, lightning\_rod, stone\_axe, jungle\_boat, kelp, sand, sandstone, glass, raw\_iron, granite, lapis\_lazuli, diorite, iron\_ingot, bucket, iron\_pickaxe, chest, andesite, redstone, dried\_kelp, iron\_chestplate, wooden\_sword, shield, iron\_sword}
\end{itemize}

\vskip -0.2in
\begin{table}[H]
\caption{Number of different inventory types produced by each trial}
\label{tab:Number}
% \vskip 0.1in
\setlength{\tabcolsep}{12pt} % 调整列间距
\renewcommand{\arraystretch}{1.0} % 调整行间距
\begin{center}
% \resizebox{\textwidth}{!}{ % 自动调整表格宽度以适应页面
\begin{small}
\begin{sc}
\begin{tabular}{lccc} % 确保列数与标题一致
\toprule
\textnormal{\textbf{Method}} & \textnormal{\textbf{Trial 1}} & \textnormal{\textbf{Trial 2}} & \textnormal{\textbf{Trial 3}}  \\
\midrule
\normalfont Voyager     & 50  & 45  & 37    \\
\normalfont AETG(Ours)  & 67  & 51  & 53    \\
\bottomrule
\end{tabular}
\end{sc}
\end{small}
% }
\end{center}
\vskip -0.1in
\end{table}


\paragraph{Longer exploration path} To better demonstrate the exploration capabilities of the agent, we compared the exploration trajectories and their lengths. As shown in Figure \ref{fig:linermap}, our agent exhibits longer and more persistent exploration capabilities than Voyager. In Table \ref{tab:length}, the trajectory lengths of our agent are consistently much greater than those of Voyager. {\ours}is able to traverse across multiple terrains, with an average distance 2.66 times longer than Voyager. Additionally, {\ours} can explore across different continental plates, while Voyager remains confined to a single plate, highlighting the exceptional exploration capability of {\ours}.

% \vskip -0.2in
\begin{table}[H]
\caption{Exploration trajectory length in each trial, where \textit{Performance Gain} = $\textit{ours}/\textit{voyager}$.}
\label{tab:length}
% \vskip 0.1in
\setlength{\tabcolsep}{12pt} % 调整列间距
% \renewcommand{\arraystretch}{1.0} % 调整行间距
\begin{center}
% \resizebox{\textwidth}{!}{ % 自动调整表格宽度以适应页面
\begin{small}
\begin{sc}
\begin{tabular}{lcccc} % 确保列数与标题一致
\toprule
\textnormal{\textbf{Method}} & \textnormal{\textbf{Trial 1}} & \textnormal{\textbf{Trial 2}} & \textnormal{\textbf{Trial 3}} & \textnormal{\textbf{\textit{Avg}}}\\
\midrule
\normalfont Voyager     & 1925.74  & 4102.99  & 902.13  & 2310.29   \\
\normalfont {\ours}(Ours)  & 5665.75  & 8908.57  & 3895.06 & 6156.46  \\
\midrule
\normalfont \textit{Performance Gain} & 2.94  & 2.17   & 4.32    & 2.66 \\
\bottomrule
\end{tabular}
\end{sc}
\end{small}
% }
\end{center}
\vskip -0.1in
\end{table}


\vskip -0.2in
\begin{figure}[H]
\vskip 0.2in
\begin{center}
\centerline{\includegraphics[width=1\linewidth]{trial-map.png}}
% \vskip -0.2in
\caption{Map coverage: Three bird’s eye views of Minecraft maps. The trajectories are plotted based on the position coordinates where each agent interacts.}
\label{fig:trialmap}
\end{center}
\vskip -0.3in
\end{figure}


\vskip -0.2in
\begin{figure}[H]
\vskip 0.2in
\begin{center}
\centerline{\includegraphics[width=1\linewidth]{liner-map.png}}
% \vskip -0.2in
\caption{Movement trajectory Map: Three bird’s eye views of Minecraft maps. The trajectories are plotted based on the position coordinates where each agent interacts.}
\label{fig:linermap}
\end{center}
\vskip -0.3in
\end{figure}



\paragraph{Efficient Zero-Shot Generalization to Unseen Tasks} Based on the results presented in Table \ref{tab:newtechtree} and Figure \ref{fig:diamon and compass}, we can clearly observe the significant advantages of {\ours} in the open-ended task. Table \ref{tab:newtechtree} shows the number of iterations required for different methods to complete various tasks (Gold Sword, Compass, Diamond Hoe, Lava Bucket), where fewer iterations indicate higher efficiency. Compared to Voyager and {\ours} (w/o toolnet), {\ours} consistently requires significantly fewer iterations across all tasks, demonstrating substantial improvements in efficiency. Notably, in the Gold Sword task, {\ours} (ours) completes the task in just 14.00±1.73 iterations, whereas Voyager requires 46.33±14.57 iterations, showcasing its superior performance.

Figure \ref{fig:diamon and compass} further visualizes the intermediate progress of different methods on the "Craft a Compass" and "Craft a Diamond Hoe" tasks. It is evident that {\ours} learns and masters the necessary skills for crafting items more quickly. As the number of prompting iterations increases, {\ours} reaches the task objectives significantly earlier than the other methods. Additionally, while {\ours}(w/o Tool Graph) performs better than Voyager, it still lags behind {\ours}, indicating that the ToolNet component plays a crucial role in enhancing the model's capability.

Overall, these experimental results demonstrate that {\ours} not only learns new skills and crafting techniques more efficiently but also that its key module, Tool Graph, is essential for overall performance improvement. This further validates the effectiveness of our approach in self-driven exploration and task generalization.


\begingroup
\begin{table}[H]
\caption{The mastery of the tech tree in the Open-ended Task. The number indicates the number of iterations. The fewer the iterations, the more efficient the method. "N/A" indicates that the number of iterations for obtaining the current type of tool is not available.}
\label{tab:newtechtree}
\vskip 0.1in
\setlength{\tabcolsep}{12pt} % 调整列间距
% \renewcommand{\arraystretch}{1.0} % 调整行间距
\begin{center}
% \resizebox{\textwidth}{!}{ % 自动调整表格宽度以适应页面
\begin{small}
\begin{sc}
\begin{tabular}{lccccc} % 确保列数与标题一致
\toprule
\textnormal{\textbf{Method}} & \textnormal{\textbf{Trial}} & \textnormal{\textbf{Gold Sword}} & \textnormal{\textbf{Compass}} & \textnormal{\textbf{Diamond Pickaxe}} & \textnormal{\textbf{Lava Bucket}} \\
\midrule
\multirow{4}{*}{\multirow{2}{*}{\normalfont Voyager}} 
              & \normalfont Trial 1 & 48 & 16 &  24 & N/A         \\
              & \normalfont Trial 2 & 31 & 17 &  25 & 39         \\
              & \normalfont Trial 3 & 60 & 20 & 18  & N/A         \\
              \cmidrule{2-6}
              & \textit{Average} & 46.33$\pm$14.57 & 17.67$\pm$2.08 & 22.33$\pm$3.79 & 39.00$\pm$0.00 \\
\midrule
\multirow{4}{*}{\multirow{2}{*}{\normalfont {\ours}\textit{\small(w/o toolnet)}}} 
               & \normalfont Trial 1 & 26 & 27 & 23  & N/A         \\
              & \normalfont Trial 2 & 18 & 22 & 18  & N/A        \\
              & \normalfont Trial 3 & 56 & 15 & 30  & N/A          \\
              \cmidrule{2-6}
              & \textit{Average} & 33.33$\pm$20.03 & 21.33$\pm$6.03 & 23.67$\pm$6.03 & N/A$\pm$N/A \\
\midrule
\multirow{4}{*}{\multirow{2}{*}{\normalfont {\ours}\textit{\small(ours)}}} 
              & \normalfont Trial 1 & 13 & 28 & 16  & 19       \\
              & \normalfont Trial 2 & 13 & 10 & 14  & 27       \\
              & \normalfont Trial 3 & 16 & 13  & 13  & 18      \\
              \cmidrule{2-6}
              & \textit{Average} & \textbf{14.00$\pm$1.73} & \textbf{17.00$\pm$9.64} & \textbf{14.33$\pm$1.53} & \textbf{21.33$\pm$4.93} \\
             

\bottomrule
\end{tabular}
\end{sc}
\end{small}
% }
\end{center}
\vskip -0.1in
\end{table}
\endgroup



\begin{figure}[H]
\vskip 0.2in
\begin{center}
\centerline{\includegraphics[width=1\linewidth]{compass_and_diamond.png}}
% \vskip -0.2in
\caption{Zero-shot generalization to unseen tasks. Here, we visualize the intermediate progress of each method on the tasks "Craft a Compass" and "Craft a Diamond Hoe."}
\label{fig:diamon and compass}
\end{center}
\vskip -0.3in
\end{figure}



\subsection{Agent Task}
\label{subsec:agent-results}

Figures \ref{fig:toolnet-dabench} and \ref{fig:toolnet-textcraft} present the tool network evolution diagrams of DA-Bench and TextCraft, which visually reflect the call relationships between different tool functions. In these diagrams, each node represents a specific tool function, edges indicate the call dependencies between tools, and the shading of the nodes reflects the frequency of tool calls—darker colors indicate higher call frequency. From Figure \ref{fig:toolnet-dabench}, it can be observed that in DA-Bench, the tool network expands progressively as the task advances, forming multiple core nodes with higher call frequencies. This suggests that certain key tools are frequently called during the task execution, playing a central role. Additionally, the tool call relationships exhibit a hierarchical and well-organized structure, reflecting DA-Bench's efficiency in tool dependency management.

In contrast, Figure \ref{fig:toolnet-textcraft} illustrates the tool network evolution of TextCraft, which also shows a similar expansion trend overall. However, compared to DA-Bench, the tool call frequency in TextCraft is more evenly distributed across multiple nodes, meaning that the system calls a wider variety of tools during task execution, rather than relying on a few core tools. This distribution pattern may suggest that TextCraft adopts a more diverse tool usage strategy in task execution.

A comparative analysis of the two figures reveals that, although both DA-Bench and TextCraft exhibit certain hierarchical and expansive characteristics in their tool call patterns, DA-Bench relies more heavily on a few core tools, whereas TextCraft displays a more dispersed tool call pattern. This contrast not only highlights the differences in tool usage between the two, but also emphasizes the importance and effectiveness of ToolNet.





\subsection{Single-turn Code Task}
\label{subsec:code-results}

As shown in the Figure\ref{fig:toolnet-math} \ref{fig:toolnet-tabmwp}, this illustrates the evolution of the tool graph for the Math and TabMWP tasks. It is evident that the tool graph gradually becomes more complex, creating multiple layers of tools, making the tool graph more intricate. Since the Date task can be solved with fewer tools, there is no evolution of the tool graph. However, the generated tools can still effectively solve the task, while there exists a multi-level calling relationship.


\section{More Ablations}
\label{app:apablation}
\subsection{Open-ended Task}
\label{subsec:open-ablation}

As shown in Figure \ref{fig:ablation}, AETG significantly outperforms methods that lack certain functional modules in discovering new Minecraft items and skills. It can be observed that the performance is worst when "w/o retrieval" is used, indicating that the absence of retrieval has the greatest impact on overall functionality and plays a crucial role, thereby validating the effectiveness of our retrieval method. The performance with "w/o duplication" is slightly better, indicating its importance is weaker than that of "w/o retrieval." The performance of "w/o check" and "w/o pruning" is better, but still far behind AETG, which further demonstrates the importance and effectiveness of each functional component.

\vskip -0.1in
\begin{figure}[H]
% \vskip 0.2in
\begin{center}
\centerline{\includegraphics[width=0.6\linewidth]{toolnumber-ablation.png}}
% \vskip -0.2in
\caption{Ablation study of the iterative prompting mechanism. AETN surpasses all other options, highlighting the essential significance of each functional module in the iterative prompting mechanism.}
\label{fig:ablation}
\end{center}
\vskip -0.3in
\end{figure}


\subsection{Closed-Ended Task}
\label{subsec:closed-ended}
For the Closed-Ended Task, we select Textcraft from the Agent Task and Date from the Single-turn Code Task to evaluate the effectiveness of several components in our method. The results are shown in the Table \ref{tab:closed-toolnumber}.

\begingroup
\begin{table}[H]
\caption{The number of tools in Close-Ended Task.}
\label{tab:closed-toolnumber}
\vskip -0.1in
\setlength{\tabcolsep}{10pt} % 调整列间距
\begin{center}
\begin{small}
\begin{sc}
\begin{tabular}{l|cc}
\toprule
\textnormal{\textbf{Method}} & \textnormal{\textbf{TextCraft}}  & \textnormal{\textbf{Date}} \\
\midrule         

\normalfont W/o Self-Check & 42 & 9 \\
\midrule  
\normalfont W/o Merging & 49 & 11\\
\midrule  
\normalfont W/o pruning & 46 & 9 \\
\midrule  
\normalfont GATE & 44 & 4 \\


\bottomrule
\end{tabular}
\end{sc}
\end{small}
\end{center}
\vskip -0.1in
\end{table}
\endgroup

\section{Tool Making}
\label{app:toolgarph}
\subsection{Basic Tools}
\label{subsec:basic-tools}
As shown in the Table \ref{tab:basictool} , the basic tools generated by each method are displayed.

\begingroup
\begin{table}[H]
\caption{Basic tools in various methods.}
\label{tab:basictool}
\vskip -0.1in
\setlength{\tabcolsep}{10pt} % 调整列间距
\begin{center}
\begin{small}
\begin{sc}
\begin{tabular}{l|p{12cm}}
\toprule
\textnormal{\textbf{Tasks}} & \textnormal{\textbf{Basic Tools}}  \\
\midrule         

\normalfont Other Tasks & \normalfont ToolRequest, NotebookBlock, Terminate, CreateTool, EditTool, Python, Feedback, SendAPI, Feedback, Retrieval \\
\midrule  
\normalfont Minecraft & \normalfont smeltItem, killMob, waitForMobRemoved, givePlacedItemBack, useChest, exploreUntil, craftItem, mineBlock, shoot, placeItem, craftHelper, smeltItem, mineflayer, killMob, useChest, exploreUntil, craftItem, mineBlock, placeItem \\

\bottomrule
\end{tabular}
\end{sc}
\end{small}
\end{center}
\vskip -0.1in
\end{table}
\endgroup


\subsection{Tool construction Lists}
\label{subsec:tool construction}

\paragraph{CREATOR:}
\begin{itemize}[noitemsep, topsep=0pt]
    \item \textbf{MATH:}  \textit{sum of areas, find largest won matches, find K, total distance after bounces, find common ratio sum, count lattice points with distance squared, find c for radius, find circle equation and constants, polynomial degree product, calculate cells, find fiftieth term, find non domain values, inverse function product, find m and n, sum of fractions from roots, find roots of quadratic, main, find coefficients, compute expression, prime factors, find x y, find second largest angle, find y coordinate, find constants, evaluate expression, find b for one solution, find c, find minimum value, find possible s, solve expression, find cone height, solve abc, find minimum expression, \dots, time to hit ground, sum of reciprocals of roots, solve x floor x product, sum of possible x, find constant a, sum of squares of solutions, find cost per extra hour, is triangular number, find smallest b greater than 2011, solve exponential equation, solve club suit equation, find degree of h, f, find vertical asymptotes, domain width, maximize revenue, future value, total savings, find min interest rate, equation, find integers, sum of x coordinates squared, find integer values of a, smallest c for real domain, smallest integer c, find m, required investment, simplify expression, g, distance between midpoints, compute x and power, greatest possible a, find continued fraction value, find a b, solve mnp, compute sum, sum of integers in range,
    }

    \item \textbf{Date:}  \textit{get us thanksgiving date, get date one week from first monday of 2019, calculate anniversary date, calculate yesterday from last day of january, calculate one week ago from first monday, get first monday of 2019, calculate yesterday, calculate yesterday from rescheduled meeting, calculate date a month ago from rescheduled meeting, calculate yesterday from first monday of 2019, get date 10 days before us thanksgiving, calculate one week ago from egg runout, calculate one week ago from end of first quarter, calculate date 24 hours later, calculate date a month ago, calculate date 24 hours after anniversary, calculate one week from today from rescheduled meeting, \dots, get tomorrow from us thanksgiving, calculate yesterday from day before yesterday, calculate yesterday from anniversary, calculate date 10 days ago, calculate one year ago from egg run out date, calculate tomorrow from yesterday, calculate one week from last day of january, calculate one week from anniversary, calculate yesterday from eggs run out, calculate tomorrow from today, calculate tomorrow from day before yesterday, calculate one week ago from today, calculate one week ago, calculate date one month ago from anniversary, calculate one year ago from given date, calculate one week from given date}

    \item \textbf{TabMWP:}  \textit{calculate total cost, smallest points, price difference, cost of river rafts, calculate median, calculate range, calculate total spent, rate of change, cost difference, cost for rides, rate of change vacation days, total participants, calculate mean glasses, find mode of states visited, rate of change straight A students, calculate median basketball hoops, count bins with toys in range, people with at least 3 trips, count teams with fewer than 80 swimmers, calculate median clubs, count exact pushups, children with less than 2 necklaces, people played exactly 3 times, count people with fewer than 80 pullups, range of states visited, find spent amount, \dots, calculate median miles, people with fewer than 3 seashells, calculate median glasses, cost to buy cockatiels, largest broken lights, calculate spent, calculate ice cream cost, range of soccer fields, patrons with at least 2 books, count bushes with 20 roses, total people played golf, range of articles, count shipments with exactly 60 broken plates, total cost for lip balms, rate of change scholarships, count teams with fewer than 50 members, count tests with 34 problems, find mode of soccer fields, rate of change hockey games, find lowest score, count pizzas with exactly 48 pepperoni, count people with at least 30 points, cost of wooden benches, rate of change students, patients with fewer than 2 trips, find mode, total cost for hazelnuts, calculate mean fan letters, readers with at least 4 hats, count classrooms with 41 desks}
\end{itemize}

\paragraph{CRAFT:}
\begin{itemize}[noitemsep, topsep=0pt]
    \item  \textbf{MATH:}  \textit{find pack size, count distinct solutions, calculate points, find tank capacity, solve exponential log equation, total energy equilateral triangle, inverse square law force, find max value, total logs in stack, sum of multiples of 13, calculate exponential growth, gravitational force, find x for piecewise composition, positive difference, specific piecewise func, day exceeds 200 cents, find lattice points, count integer parameters for integer solutions, count zeros in square of power of ten minus one, energy stored, sum of squares of roots, sum odd integers, find d minus e squared, compute complex series sum, total energy configuration, sum of areas, \dots, max item price, solve two variable system, inverse variation power, total distance hopped, is prime, total distance, find constant term of polynomial, total distance moved, find perpendicular slope, calculate inverse proportionality, find value of A, count integer a, find min items for higher score, apply r n times, find min x, day exceeds threshold, calculate area in square yards, solve log equation, total items produced, find variable for distance condition, solve time at speeds, find largest solution, find weight of object, calculate proportional value, calculate material cost, solve for variable, total elements in arithmetic sequence, transformed domain, find day for algae coverage, calculate energy stored, least value of y, solve bowling ball weight, find min froods}

    \item \textbf{Date:} \textit{get today date, calculate one week ago, calculate n days from future date, calculate n days from date in format, calculate date days ago, calculate n months from date, calculate one week from today, calculate date after event, find palindrome day, calculate date a month ago, calculate date after days and months, calculate relative date, calculate n days from reference, calculate one year ago from today, calculate n hours from date, calculate date n days from, get date today, calculate date 10 days ago from deadline, calculate n weeks from date, \dots, calculate n units from date, calculate n years from date, calculate n weeks from first weekday of year, calculate today from tomorrow, find special day, calculate date 10 days ago from future, calculate n days after event, calculate date from days passed, calculate one week from christmas eve, calculate one year ago, calculate date 24 hours later, calculate n weeks from anniversary, calculate tomorrow from uk format date, calculate n days from date, is palindrome, calculate one week from first monday of year, calculate one week ago from anniversary}

    \item \textbf{TabMWP:} \textit{get frequency, calculate volleyballs in lockers, calculate total cost from package prices, calculate total items from group counts, calculate mode, calculate donation difference for person, count bags with 20 to 40 broken cookies, calculate total items from groups and items per group, count commutes of 50 minutes, get received amount, calculate total items for groups, find probability, calculate vacation cost, calculate rate of change, find received amount for transaction, calculate vote difference between two items for group, count customers, find minimum value in stem leaf, calculate metric wrenches, find smallest number, count books with 30 to 50 characters, \dots, count people with 67 pullups, calculate difference in donations for person, calculate total cost from unit price and weight, calculate total items from ratio, calculate total cost from unit weight prices and weight, calculate donation difference between causes, calculate difference, calculate net income, calculate grasshoppers on twigs, count total members in group, calculate expenses on date, find lightest child, calculate difference in amounts, count votes for item from groups, calculate probability from count table, get table cell value, calculate jeans in hampers, count instances with specific value in stem leaf, calculate donation difference for person and causes, calculate total from frequency and additional count, calculate range, calculate total reviews}
\end{itemize}


\paragraph{REGAL:}
\begin{itemize}[noitemsep, topsep=0pt]
    \item \textbf{MATH:}  \textit{solve for largest side, apply function sequence, solve rational equation, calculate expression sum, max sum of products, find b for perpendicular bisector, vertex of quadratic, calculate work days, calculate c for zero coefficient, simplify and rationalize sympy, find a for binomial square, compound interest, calculate inverse variation, expand expression, calculate average speed, calculate rs, sum sequence, solve for p, max consecutive integers, find x intercept, day exceeding threshold, find smallest sum, solve for ac pair, constant function, sum of distances, evaluate expression, sum finite geometric series, factor expression, find common difference, total coins pirates, calculate geometric first term, calculate closest whole number, calculate x minus y squared, solve letter values, find circle center v2, evaluate expression with sqrt, calculate sum of equations, \dots, calculate x3 plus y3, find negative intervals, calculate floor and abs, solve quadratic and find min, calculate y, solve for a, check equations, rationalize and simplify, calculate xyz, calculate distance, solve for x in simplified equation, calculate expression, calculate exponent, sum arithmetic series, complete square form, calculate x2 plus y2
    }

    \item \textbf{Date:}  \textit{subtract weeks from date, add weeks to date, format date, add days to date, subtract months from date, subtract days from date, subtract years from date, calculate date, calculate days between weekdays}

    \item \textbf{TabMWP:}  \textit{count range, find mode, total participants, count bushes with fewer roses, find max frequency, total items, count in range, calculate total items, count below threshold, count teams with minimum size, calculate total, calculate range, calculate fraction, sum frequencies below threshold, sum frequencies, calculate difference, calculate median, total outcomes, count specific height, count numbers in range, difference between groups, access frequency, calculate proportionality constant, count values below threshold, find median, calculate probability, calculate mode, get frequency, convert stem leaf to numbers, find minimum, get total items, count scores above, rate of change, calculate mean}
\end{itemize}



\subsection{The tool graph evolution diagrams of {\ours} for various tasks.}
\label{subsec:tool-graph}
Below are the tool graph evolution diagrams for various tasks. The Date task does not have a tool network evolution diagram, as date reasoning does not heavily rely on tool diversity.


\begin{figure}[H]
\vskip 0.2in
\begin{center}
\centerline{\includegraphics[width=1\linewidth]{toolnet-trial1.png}}
% \vskip -0.2in
\caption{
The tool graph evolution diagram for Minecraft Trial 1. In this diagram, each node represents a tool function, and the edges represent the invocation relationships between tools. The darker the color, the more frequently the tool is invoked. The network consists of a total of 6 layers, with layers 2 to 6 shown here from top to bottom.}
\label{fig:toolnet1}
\end{center}
\vskip -0.3in
\end{figure}

\vskip -0.2in
\begin{figure}[H]
\vskip 0.2in
\begin{center}
\centerline{\includegraphics[width=1\linewidth]{toolnet-trial2.png}}
% \vskip -0.2in
\caption{The tool graph evolution diagram for Minecraft Trial 2. In this diagram, each node represents a tool function, and the edges represent the invocation relationships between tools. The darker the color, the more frequently the tool is invoked. The network consists of a total of 6 layers, with layers 2 to 6 shown here from top to bottom.}
\label{fig:toolnet2}
\end{center}
\vskip -0.3in
\end{figure}

\vskip -0.2in
\begin{figure}[H]
\vskip 0.2in
\begin{center}
\centerline{\includegraphics[width=1\linewidth]{toolnet-trial3.png}}
% \vskip -0.2in
\caption{The tool graph evolution diagram for Minecraft Trial 3. In this diagram, each node represents a tool function, and the edges represent the invocation relationships between tools. The darker the color, the more frequently the tool is invoked. The network consists of a total of 6 layers, with layers 2 to 7 shown here from top to bottom.}
\label{fig:toolnet3}
\end{center}
\vskip -0.3in
\end{figure}


\begin{figure}[H]
\vskip 0.2in
\begin{center}
\centerline{\includegraphics[width=1\linewidth]{toolnet-dabench.png}}
% \vskip -0.2in
\caption{The tool graph evolution diagram of DA-Bench. In this diagram, each node represents a tool function, and the edges represent the invocation relationships between tools. The darker the color, the more frequently the tool is invoked.}
\label{fig:toolnet-dabench}
\end{center}
\vskip -0.3in
\end{figure}

\begin{figure}[H]
\vskip 0.2in
\begin{center}
\centerline{\includegraphics[width=1\linewidth]{toolnet-textcraft.png}}
% \vskip -0.2in
\caption{The tool graph evolution diagram of TextCraft. In this diagram, each node represents a tool function, and the edges represent the invocation relationships between tools. The darker the color, the more frequently the tool is invoked.}
\label{fig:toolnet-textcraft}
\end{center}
\vskip -0.3in
\end{figure}


\begin{figure}[H]
\vskip 0.2in
\begin{center}
\centerline{\includegraphics[width=1\linewidth]{toolnet-math.png}}
% \vskip -0.2in
\caption{The tool graph evolution diagram of MATH. In this diagram, each node represents a tool function, and the edges represent the invocation relationships between tools. The darker the color, the more frequently the tool is invoked.}
\label{fig:toolnet-math}
\end{center}
\vskip -0.3in
\end{figure}

\begin{figure}[H]
\vskip 0.2in
\begin{center}
\centerline{\includegraphics[width=1\linewidth]{toolnet-tabmwp.png}}
% \vskip -0.2in
\caption{The tool graph evolution diagram of TabMWP. In this diagram, each node represents a tool function, and the edges represent the invocation relationships between tools. The darker the color, the more frequently the tool is invoked.}
\label{fig:toolnet-tabmwp}
\end{center}
\vskip -0.3in
\end{figure}
\section{Prompt Template}
\label{app:prompt}
In this section, we provide the prompt templates of different types used throughout our experiment. These prompts were carefully crafted to ensure that the model's output aligns with the specific objectives of each task.

\subsection{Construction Stage}
In open-ended task online training, we made slight modifications to their prompts based on Voyager~\citep{wang2023voyager}. For close-ended tasks, the prompts used during the construction process are as follows:
\begin{tcolorbox}[title=Task Solver's Prompt, breakable, width=\textwidth,top=0mm]
\begin{Verbatim}[breaklines, fontsize=\footnotesize]
# Instruction #
You are the Task Solver in a collaborative team, specializing in reasoning and Python programming. Your role is to analyze tasks, collaborate with the Tool Manager, and solve problems step by step.
Directly solving tasks without tool analysis is not allowed. Request necessary tools before proceeding when needed, based on the task analysis.

# WORKFLOW #
You can decide which step to take based on the environment and current situation, adapting dynamically as the task progresses.
Stage 1. Tool Requests:
    Requesting tool is mandatory. Request generalized and reusable tools to solve the task. Focus on abstract functionality rather than task-specific details to enhance flexibility and adaptability.
Stage 2. Code and Interact: 
    Write notebook blocks incrementally, executing and interacting with the environment step by step. Avoid bundling all steps into a single block; instead, adjust dynamically based on feedback after each interaction.
Stage 3: Validate and Conclude: 
    When confident in the solution, review your work, validate the results, and conclude the task.

# Custom Library #
===api===

# NOTICE #
1. You must fully understand the action space and its parameters before using it.
2. If code execution fails, you should analyze the error and try to resolve it. If you find that the error is caused by the API, please promptly report the error information to the Tool Manager.
3. Regardless of how simple the issue may seem, you should always aim to summarize and refine the tool requirements.


# Tool Request Guidelines #
1. Keep It Simple: Design tools with single and simple functionality to ensure they are easy to implement, understand, and use. Avoid unnecessary complexity.
2. Define Purpose: Clearly outline the tool’s role within broader workflows. Focus on creating reusable tools that solve abstract problems rather than task-specific ones.
3. Specify Input and Output: Define the required input and expected output formats, prioritizing generic structures (e.g., dictionaries or lists) to enhance flexibility and adaptability.
4. Generalize Functionality: Ensure the tool is not tied to a specific task. Abstract its functionality to make it applicable to similar problems in other contexts.


# ACTION SPACE #
You should Only take One action below in one RESPONSE:
## NotebookBlock Action
* Signature: 
NotebookBlock():
```python
executable python script
```
* Description: The NotebookBlock action allows you to create and execute a Jupyter Notebook cell. The action will add a code block to the notebook with the content wrapped inside the paired ``` symbols. If the block already exists, it can be overwritten based on the specified conditions (e.g., execution errors). Once added or replaced, the block will be executed immediately.
* Restrictions: Only one notebook block can be managed or executed per action.
* Example
- Example1: 
NotebookBlock():
```python
# Calculate the area of a circle with a radius of 5
radius = 5
area = 3.1416 * radius ** 2
print(area)
```

## Tool_request Action
* Signature:
{
    "action_name": "tool_request",
    "argument": {
         "request": [
             ...
         ]
    }
}
* Description: The Tool Request Action allows you to send tool requirements to the Tool Manager and request it to create appropriate tools. You need to provide the action in a JSON format, where the argument field contains a request parameter that accepts a list. Each element in the list is a string describing the desired tool.
* Note:
* Examples:
- Example 1:
{
    "action_name": "tool_request",
    "argument": {
        "request": [
            "I need a tool that calculates the average value of a specified column in a dataset. The input should include the column name."
        ]
    }
}
- Example 2:
{
    "action_name": "tool_request",
    "argument": {
        "request": [
            "I need a tool that filters rows in a dataset based on a specified condition. The input should include the column name and the condition to filter by."
        ]
    }
}


## Terminate Action
* Signature: Terminate(result=the result of the task)
* Description: The Terminate action ends the process and provides the task result. The `result` argument contains the outcome or status of task completion.
* Examples:
  - Example1: Terminate(result="A")
  - Example2: Terminate(result="1.23")

# RESPONSE FORMAT #
For each task input, your response should contain:
1. One RESPONSE should only contain One Stage, One Thought and One Action.
2. An current phase of task completion, outlining the steps from planning to review, ensuring progress and adherence to the workflow.  (prefix "Stage: ").
3. An analysis of the task and the current environment, including reasoning to determine the next action based on your role as a SolvingAgent. (prefix "Thought: ").
4. An action from the **ACTION SPACE** (prefix "Action: "). Specify the action and its parameters for this step.

# RESPONSE EXAMPLE #
Observation: ...(the output of last actions, as provided by the environment and the code output, you don't need to generate it)

Stage:...(One Stage from `WORKFLOW`)
Thought: ...
Action: ...(Use an action from the ACTION SPACE no more than once per response.)

# TASK #
===task===
\end{Verbatim}
\end{tcolorbox}

\begin{tcolorbox}[title=Tool Manager's Prompt, breakable, width=\textwidth,top=0mm]
\begin{Verbatim}[breaklines, fontsize=\footnotesize]
# Instruction #
You are a Tool Manager in a collaborative team, specializing in assembling existing APIs to construct hierarchical and reusable abstract tools based on predefined criteria.
You will be provided with a custom library, similar to Python’s built-in modules, containing various functions related to date reasoning. For each task, you will receive:
1. Tool request: The specific goal or functionality the new tool must achieve.
2. Existing tools: A list of available functions from the custom library that you can utilize.
Your task is to analyze the given request and create a reusable tool by effectively leveraging the relevant functions from the existing tools or utilizing basic tools to achieve the desired functionality. 
If an existing tool from the provided library already fully satisfies the requirements, simply return that tool instead of duplicating functionality. Ensure all responses align with reusability and efficiency principles.

# Custom Library #
===api===

# Creation Criteria #
- **Reusability**: The function could be resued for more complex function.
- **Innovation**: Tools should offer innovation, not merely wrap or replicate existing APIs. Simply re-calling an API without significant enhancements does not qualify as innovation.
- **Completeness**: The function should handle potential edge cases to ensure completeness.
- **Leveraging Existing Functions**: The function should effectively utilize existing functions to enhance efficiency and avoid redundancy.
- **Functionality**: Ensure the tool runs successfully and is bug-free, guaranteeing full functionality.

# ACTION SPACE #
You should Only take One action below in one RESPONSE:
## Create tool Action
* Description: The Create Tool action allows you to develop a new tool and temporarily store it in a private repository accessible only to you. Each invocation creates a single tool at a time. You can repeatedly use this action to build smaller components, which can later be assembled into the final tool.
* Signature: 
Create_tool(tool_name=The name of the tool you want to create):
```python
The source code of tool
```
* Example:
Create_tool(tool_name=“calculate_column_statistics”):
```python
def calculate_column_statistics(dataset: pd.DataFrame, column_name: str) -> Dict[str, float]:
    """
    Calculates basic statistics (mean, median, standard deviation) for a specified column in a dataset.
    Parameters:
    - dataset: A pandas DataFrame containing the data.
    - column_name: The name of the column to calculate statistics for.
    Returns:
    - A dictionary containing the mean, median, and standard deviation of the column.
    """
    if column_name not in dataset.columns:
        raise ValueError(f"Column '{column_name}' not found in the dataset.")
    
    column_data = dataset[column_name]
    stats = {
        "mean": column_data.mean(),
        "median": column_data.median(),
        "std_dev": column_data.std()
    }
    return stats
```
## Edit tool Action
* Description: The Edit Tool action allows you to modify an existing tool and temporarily store it in a private repository that only you can access. You must provide the name of the tool to be updated along with the complete, revised code. Please note that only one tool can be edited at a time.
* Signature: 
Edit_tool(tool_name=The name of the tool you want to create):
```python
The edited source code of tool
```
* Examples:
Edit_tool(tool_name="filter_rows_by_condition"):
```python
def filter_rows_by_condition(dataset: pd.DataFrame, column_name: str, condition: str) -> pd.DataFrame:
    """
    Filters rows in a dataset based on a specified condition for a given column.
    Parameters:
    - dataset: A pandas DataFrame containing the data.
    - column_name: The name of the column to apply the condition to.
    - condition: A string representing the condition, e.g., 'value > 10'.
    Returns:
    - A filtered DataFrame containing only the rows that satisfy the condition.
    """
    if column_name not in dataset.columns:
        raise ValueError(f"Column '{column_name}' not found in the dataset.")
    
    try:
        filtered_dataset = dataset.query(f"{column_name} {condition}")
    except Exception as e:
        raise ValueError(f"Invalid condition: {condition}. Error: {e}")
    
    return filtered_dataset
```

# RESPONSE FORMAT #
For each task input, your response should contain:
1. Each response should contain only one "Thought," and one "Action."
2. Determine how to construct your tool to meet tool request and function creation criteria. Check if any functions in the Existing Tool can be invoked to assist in the tool’s development and ensure alignment with the criteria.(prefix "Thought: ").
3. An action dict from the **ACTION SPACE** (prefix "Action: "). Specify the action and its parameters for this step. 

# RESPONSE EXAMPLE  #
1. If you determine that the tool request cannot be solved using existing tools, choose this mode to provide a clear and complete code solution.

Thought: ...
Action: ...

2. If you determine that the tool request is already satisfied by existing tools, choose this mode to directly reference and return the relevant tool without creating additional solutions.
Thought: ...
Tool: {  
    "tool_name": "Name of Existing tools"
}

# NOTICE #
1. You can directly call and use the tool in the custom library in your code or tool without importing it.
2. You can only create or edit one tool per response, so take it one step at a time.

# TASK #
===task===
\end{Verbatim}
\end{tcolorbox}


\begin{tcolorbox}[title=Prompt of Self-Check Step 1, breakable, width=\textwidth,top=0mm]
\begin{Verbatim}[breaklines, fontsize=\footnotesize]
# Instruction #
You are evaluating whether the tools provided by the Tool Manager meet the required standards. 
You follow a defined workflow, take actions from the ACTION SPACE, and apply the evaluation criteria. 

# Evaluation Criteria #
- **Reusability**: The function should be designed for reuse in more complex scenarios. For instance, in the case of the `craft_wooden_sword()` tool, it would be more versatile if it could accept a quantity as an input parameter.
- **Innovation**: Tools should offer innovation, not merely wrap or replicate existing APIs. Simply re-calling an API without significant enhancements does not qualify as innovation. If an existing tool from the provided library already fully satisfies the requirements, simply return that tool instead of duplicating functionality. Ensure all responses align with reusability and efficiency principles.
- **Completeness**: The function should handle potential edge cases to ensure completeness.
- **Leveraging Existing Functions**: Check if any function in "Existing Function" is helpful for completing the task. If such functions exist but are not invoked in the provided code, relevant feedback should be given.

## Tool Abstraction ##
Tool abstraction is essential for enabling tools to adapt to diverse tasks. Key principles include:
- Design generic functions to handle queries of the same type, based on shared reasoning steps, avoiding specific object names or terms.
- Name functions and write docstrings to reflect the core reasoning pattern and data organization, without referencing specific objects.
- Use general variable names and pass all column names as arguments to enhance adaptability.

# ACTION SPACE #
You should Only take One action below in one RESPONSE:
# Feedback Action
* Signature: {
    "action_name": "Feedback",
    "argument": {
        "feedback": ...
        "passed": true/false
    }
}
* Description: The Feedback Action is represented as a JSON string that provides feedback to the Tool Manager or SolvingAgent. The feedback field contains comments or suggestions, while pass indicates whether the tool meets the requirements (true for approval, false for rejection). Feedback should be concise, constructive, and relevant. If pass is true, the feedback can be left empty; otherwise, it must be provided.
* Example:
- Example1:
{
    "action_name": "Feedback",
    "argument": {
        "feedback": "",
        "passed": true
    }
}
- Example2:
{
    "action_name": "Feedback",
    "argument": {
        "feedback": "The tool correctly solves the equation for small numbers, but fails when the coefficients are very large. Consider optimizing the algorithm for handling larger values and improving computational efficiency.",
        "passed": false
    }
}

# RESPONSE FORMAT #
For each task input, your response should contain:
1. One RESPONSE should ONLY contain One Thought and One Action.
2. An comprehensive analysis of the tool code based on the evaluation criteria.(prefix "Thought: ").
3. An action from the **ACTION SPACE** (prefix "Action: "). 

# EXAMPLE RESPONSE #
Observation: ...(output from the last action, provided by the environment and task input, no need for you to generate it)

Thought: 1. Reusability: ...
2. Innovation: ...
3. Completeness: ...
4. Leveraging Existing Functions: ...

Action: ...(Use an action from the ACTION SPACE once per response.)

# Custom Library #
===api===

# TASK #
===task===
\end{Verbatim}
\end{tcolorbox}

\begin{tcolorbox}[title=Prompt of Self-Check Step 2, breakable, width=\textwidth,top=0mm]
\begin{Verbatim}[breaklines, fontsize=\footnotesize]
# Instruction #
You are verifying whether the tools provided by the Tool Manager execute without runtime errors.
You will use a custom library, similar to the built-in library, which provides everything necessary for the tasks. Your task is only to execute the provided tool code and check for runtime errors, not to evaluate the tool’s functionality or correctness.

# Stage and Workflow #
1. **Ensure Bug-Free Tool Operation**:
	- Execute the tool to ensure it runs without any runtime bugs.
	- You don’t need to verify the function’s functionality; simply call it to check for any runtime errors.
	- If the tool is a retrieved API, skip this step and proceed.
2. **Send Feedback**:
	- After executing the code, provide feedback based on the output, indicating whether the operation was successful or not.

# Notice #
1. If any issues with the tool are found, promptly provide clear and critical feedback to the Tool Manager for resolution. 
2. You should not create or edit functions (tools) with the same name as the Existing Functions in the code.
3. You can directly call the APIs from the custom library without needing to import or declare any external libraries.
4. You don’t need to verify the function’s functionality or set up its standard output; simply call it to check for any errors.

# ACTION SPACE #
You should Only take One action below in one RESPONSE:
## Python Action
* Signature: 
Python(file_path=python_file):
```python
executable_python_code
```
* Description: The Python action will create a python file in the field `file_path` with the content wrapped by paired ``` symbols. If the file already exists, it will be overwritten. After creating the file, the python file will be executed. Remember You can only create one python file.
* Examples:
- Example1
Python(file_path="solution.py"):
```python
# Calculate the area of a circle with a radius of 5
radius = 5
area = 3.1416 * radius ** 2
print(f"The area of the circle is {area} square units.")
```
- Example2
Python(file_path="solution.py"):
```python
# Calculate the perimeter of a rectangle with length 8 and width 3
length = 8
width = 3
perimeter = 2 * (length + width)
print(f"The perimeter of the rectangle is {perimeter} units.")
```

# Feedback Action
* Signature: {
    "action_name": "Feedback",
    "argument": {
        "feedback": ...
        "passed": true/false
    }
}
* Description: The Feedback Action is used to provide feedback to the Tool Manager. The feedback field contains detailed comments or suggestions. If the tool encounters an error, you should set passed to false and provide a detailed feedback. If the tool runs without errors, you can set passed to true and leave feedback as an empty string.
* Examples:
- Example 1:
{
    "action_name": "Feedback",
    "argument": {
        "feedback": ""
        "passed": true
    }
}
- Example 2:
{
    "action_name": "Feedback",
    "argument": {
        "feedback": "The tool encountered an error while executing. The variable 'height' is missing in the function call. Please ensure that all required parameters are provided.",
        "passed": false
    }
}

# RESPONSE FORMAT #
For each task input, your response should contain:
1. One RESPONSE should ONLY contain One Thought and One Action.
2. An analysis of the task and current environment, reasoning through the next evaluation step based on your role as CheckingAgent.(prefix "Thought: ").
3. An action from the **ACTION SPACE** (prefix "Action: "). Specify the action and its parameters for this step.

# EXAMPLE RESPONSE #
Observation: ...(output from the last action, provided by the environment and task input, no need for you to generate it)

Thought: ...
Action: ...(Use an action from the ACTION SPACE once per response.)

# Custom Library #
You can use pandas, sklearn, or other Python libraries as part of the custom library.

* Note: You can directly call these tools without importing or redefining them in your code.

Let's think step by step.
# TASK #
===task===
\end{Verbatim}
\end{tcolorbox}

\subsection{Test Stage}
\label{appsub:test_prompt}
During the test stage, the prompts used for different datasets are as follows:
\begin{tcolorbox}[title=Prompt on DABench, breakable, width=\textwidth,top=0mm]
\begin{Verbatim}[breaklines, fontsize=\footnotesize]
# Instruction #
You are a helpful assistant, skilled in data science tasks.
You will be provided with a task description and related files. 
You should complete tasks by writing notebook code to interact with the environment containing the task files.
Additionally, you must strictly adhere to the task constraints. 
Once the task is completed, you need to format the answer as specified in the answer format and invoke the Terminate action to conclude.
You should use actions from the ACTION SPACE, follow the Response Format, and complete the task within 20 steps.

You may also leverage the following helper functions if needed.
===api===


===example===


# Response Format #
Your each response should contain:
1. One RESPONSE should only contain ONLY One Thought and ONLY One Action.
2. Only an analysis of the task and the current environment, including reasoning to determine the next action. (prefix "Thought: ").
3. Only an action from the **ACTION SPACE** (prefix "Action: "). Specify the action and its parameters for this step.

Observation: ...(Provided by the environment, no need for you to generate it.))

Thought: ...
Action: ...

# ACTION SPACE #
## NotebookBlock Action
* Signature: 
NotebookBlock():
```python
executable python script
```
* Description: The NotebookBlock action allows you to create and execute a Jupyter Notebook cell. The action will add a code block to the notebook with the content wrapped inside the paired ``` symbols. If the block already exists, it can be overwritten based on the specified conditions (e.g., execution errors). Once added or replaced, the block will be executed immediately.
* Restrictions: Each response must contain only one notebook block.
* Note: In a single block, you may call multiple tools or single.
* Example:
Action: NotebookBlock():
```python
# Calculate the area of a circle with a radius of 5
radius = 5
area = 3.1416 * radius ** 2
print(area)
```

# Terminate Action
* Signature: Terminate(result="the result of the task")
* Description: The Terminate action marks the completion of a task and presents the final result. It is a formatting guideline, not an executable Python function. The result parameter must contain a clear, specific answer that strictly complies with the task’s output format, with all required values explicitly provided.
Tips:
    - Ensure the result parameter provides a definite and concrete final answer.
    - Do not include unresolved Python expressions, placeholders, or variables (e.g., @value[{x + y}] or @result[{variable_name}] or "@result[{variable_name}]".format(variable_name)).
    - The output must adhere precisely to the task’s formatting specifications, ensuring clarity and consistency.
* Examples:
- Example 1: 
Answer Format: @shapiro_wilk_statistic[test_statistic] @shapiro_wilk_p_value[p_value]
Action: Terminate(result="@shapiro_wilk_statistic[0.56] @shapiro_wilk_p_value[0.0002]")
- Example 2: 
Answer Format: @total_votes_outliers_num[outlier_num]
where "outlier_num" is an integer representing the number of values considered outliers in the 'total_votes' column.
Action: Terminate(result="@total_votes_outliers[10]")
- Example3:
Action: Terminate(result="@normality_test_result[Not Normal] @p_value[0.000]")

## Response Example
Here are four examples of responses.
## Example1
Thought: The dataset has been loaded successfully and it contains the "Close Price" column. Now, we need to calculate the mean of the "Close Price" column using pandas.
Action: NotebookBlock():
```python
# Calculate the mean of the "Close Price" column
mean_close_price = data["Close Price"].mean()
# Round the result to two decimal places
mean_close_price_rounded = round(mean_close_price, 2)
print(mean_close_price_rounded)
```
## Example2
Thought: We need to filter the dataset to only include rows where the “Volume” is greater than 100,000. This will help focus on high-volume trades.
Action: NotebookBlock():
```python
# Filter rows where "Volume" is greater than 100,000
filtered_data = data[data["Volume"] > 100000]
# Display the filtered dataset
print(filtered_data)
```
## Example3
Thought: To analyze the correlation between “Open Price” and “Close Price,” we will calculate the Pearson correlation coefficient using pandas.
Action: NotebookBlock():
```python
# Calculate the correlation between "Open Price" and "Close Price"
correlation = data["Open Price"].corr(data["Close Price"])
# Print the correlation result
print(correlation)
```
## Example4
Thought: To check for missing values in the dataset, we need to check for null values in each column using pandas.
Action: NotebookBlock():
```python
# Check for missing values in each column
missing_values = data.isnull().sum()
# Display the result
print(missing_values)
```

# Begin #
Let's Begin.
## Task 
===task===
\end{Verbatim}
\end{tcolorbox}


\begin{tcolorbox}[title=Prompt on TextCraft, breakable, width=\textwidth,top=0mm]
\begin{Verbatim}[breaklines, fontsize=\footnotesize]
# Instruction #
You are provided with a set of useful crafting recipes to create items in Minecraft.
Crafting commands follow the format: "craft [target object] using [input ingredients]".
You can either "fetch" an object (ingredient) from the inventory or the environment or "craft" the target object using the provided crafting commands.
You are allowed to use only the crafting commands provided; do not invent or use your own crafting commands.
If a crafting command specifies a generic ingredient, such as "planks", you can substitute it with a specific type of that ingredient (e.g., “dark oak planks”).
To complete the crafting tasks, you will write notebook code utilizing tools from the "Custom Library". You should carefully read and understand the tool’s docstrings and code to fully grasp their functionality and usage.
The tools should be invoked by outputting a block of Python code. Additionally, you may incorporate Python constructs such as for-loops, if-statements, and other logic where necessary.
Please always use actions from the ACTION SPACE and follow the Response Format.


# ACTION SPACE #
## NotebookBlock Action
* Signature: 
NotebookBlock():
```python
executable python script
```
* Description: The NotebookBlock action creates and executes a Jupyter Notebook cell. It adds a code block wrapped in ``` symbols, overwriting existing blocks if specified (e.g., after execution errors). The block is executed immediately after being added or replaced.
* Note: In a single block, you may call multiple tools.

## Terminate Action
* Signature: Terminate(result=the result of the task)
* Description: The Terminate action ends the process and provides the task result. The `result` argument contains the outcome or status of task completion. Only the CheckingAgent has the authority to decide whether a task is finished.
* Examples:
  - Example1: Action: Terminate(result="3")
  - Example2: Action: Terminate(result="Successfully craft 2 oak planks")
  - Example3: Action: Terminate(result="Successfully craft 1 milk")


# Response Format #
For each task input, your response should contain:
1. One RESPONSE should only contain ONLY One Thought and ONLY One Action.
2. An analysis of the task and the current environment, including reasoning to determine the next action. (prefix "Thought: ").
3. An action from the **ACTION SPACE** (prefix "Action: "). Specify the action and its parameters for this step.

## Response Example
### Example1
Thought: To craft a crimson fence, I need to use the crafting command: “craft 3 crimson fence using 2 stick, 4 crimson planks.” This means I need two ingredients: 2 sticks and 4 crimson planks. To obtain 4 crimson planks, I can use the command “craft 4 crimson planks using 1 crimson stems,” so I need to fetch 1 crimson stem first. For the sticks, I can use the command “craft 4 stick using 2 planks.” Since the crafting of sticks requires planks, I can directly use the crimson planks I craft earlier for this purpose.
Steps: 
1. Fetch 1 crimson stems.
2. Craft 4 crimson planks using 1 crimson stems.
3. Craft 4 sticks using 2 crimson planks.
4. Craft 3 crimson fences using 2 sticks and 4 crimson planks.
Action: NotebookBlock()
```python
get_object('1 crimson stems')
```
### Example2
Thought: The inventory is currently empty, which means I need to retrieve the necessary spruce logs from the environment. Since we need spruce planks to craft the spruce trapdoor, and each spruce trapdoor requires 6 spruce planks, I should start by fetching an adequate number of spruce logs. Since each spruce log can craft 4 spruce planks, we need at least 2 spruce logs to ensure we can craft 8 spruce planks (more than enough for 2 spruce trapdoors if needed in the future).
Action: NotebookBlock()
```python
# Step 2: Fetch 2 spruce logs from the environment
get_object('2 spruce logs')
```
### Example3
Thought: We have successfully retrieved 2 spruce logs from the environment. 
Action: Terminate(result="Successfully crafted 8 spruce planks")


# Custom Library #
### Tool `check_inventory`
Source Code:
```python
def check_inventory() -> str:
    """
    Retrieves the current inventory state from the environment.
    The function sends an 'inventory' command to the environment
    and processes the observation to return a string representation
    of the inventory, listing items and their quantities.
    Returns:
        str: A string describing the inventory in the format:
             "Inventory: [item_name] (quantity)"
    """
    obs, _ = step('inventory')
    return obs
```
Usage Example:
```python
check_inventory() 
# If the environment has no items, Output: Inventory: You are not carrying anything.
# If the environment contains 2 oak planks, Output: Inventory: [oak planks] (2)
```
### Tool `get_object`
Source Code:
```python
def get_object(target: str) -> None:
    """
    Retrieves an item from the environment.

    The function prints the response message from the environment, 
    indicating whether the retrieval was successful or not.

    Args:
        target (str): The name of the item to be retrieved.

    Returns:
        None
    """
    obs, _ = step("get " + target)
    print(obs)
```
Usage Example:
Craft Command:
craft 2 yellow dye using 1 sunflower
craft 8 yellow carpet using 8 white carpet, 1 yellow dye
```python
get_object("1 sunflower") # Ouput: Got 1 sunflower
get_object("2 sunflower") # Ouput: Got 2 sunflower
# Note: You cannot retrieve yellow dye directly from the environment; it must first be crafted using sunflowers.
get_object("1 yellow dye") # Output: Could not find yellow dye
```
### Tool `craft_object`
Source Code:
```python
def craft_object(target: str, ingredients: List[str]) -> None:
    """
    Crafts a specified item using the given ingredients.

    This function's `target` and `ingredients` parameters correspond to the craft command: 
    "Craft 'target' using [ingredients]".
    
    **Note:** The `ingredients` must exactly match the command format. For example, if the command requires 
    '1 oak logs', providing '1 oak log' instead will not be recognized.

    Prints the environment's response to indicate whether the crafting operation was successful or not.

    Args:
        target (str): The item to craft along with its quantity (e.g., '4 oak planks').
        ingredients (List[str]): A list of required ingredients with their respective quantities 
                                (e.g., ['1 oak logs']).

    Returns:
        None

    """
    obs, _ = step("craft " + target + " using " + ", ".join(ingredients))
    print(obs)
```
Usage Example:
- Example1
Query: Craft 1 black dye using 1 wither rose
Craft Command:
craft 1 black dye using 1 wither rose
craft 1 black dye using 1 ink sac

```python
get_object('1 wither rose')
craft_object('1 black dye', ['1 wither rose'])
# Ouput:
# Got 1 wither rose
# Crafted 1 minecraft:black dye
```
- Example2
Craft Commands: 4 oak planks using 1 oak logs
```python
get_object("1 oak logs") # Got 1 oak logs
check_inventory() # Inventory: ["oak logs"] (1)
craft_object('4 oak planks', ['1 oak logs'])
# Crafted 4 minecraft:oak planks
```

===api===


===example===

# Example #
Here is a demo of how to fetch and craft objects.
Crafting commands:
craft 3 dark oak sign using 6 dark oak planks, 1 stick
craft 4 dark oak planks using 1 dark oak logs
craft 1 stick using 1 planks
craft 4 stick using 2 bamboo
craft 4 oak planks using 1 oak logs
craft 1 dark oak fence using 2 stick, 4 dark oak planks
craft 1 warped stairs using 6 warped planks
craft 3 oak sign using 6 oak planks, 1 stick

craft 2 oak planks

Thought: To craft 2 oak planks, I need to first check if I can fetch them directly from the environment or my inventory. If they are not available, I will craft them using the appropriate crafting recipe. Based on the crafting commands, I know that I can craft 4 oak planks using 1 oak logs. My plan is as follows:
1. Check if 2 oak planks are already in my inventory or can be fetched from the environment.
2. If not, verify if I have the required ingredient (1 oak logs).
3. Gather the missing ingredient, if necessary.
4. Craft 4 oak planks and ensure 2 are available for the task.
Action: NotebookBlock():
```python
check_inventory()
```

Observation: Inventory: [stick] (1) 

Thought: I could not find 2 oak planks or the required ingredient (1 oak logs) in my inventory. I need to fetch 1 oak logs first.
Action: NotebookBlock():
```
get_object("1 oak logs")
```

Observation: Got 1 oak logs

Thought: I now have the required ingredient to craft oak planks. I will use the crafting command: craft 4 oak planks using 1 oak logs.
Action: NotebookBlock():
```
craft_object("4 oak planks", ["1 oak logs"])
check_inventory()
```

Observation: Crafted 4 minecraft:oak_planks
Inventory: [stick] (1) [oak_planks] (4)

Thought: I now have 2 oak planks in my inventory. Task Completed!
Action: Terminate(result=“Successfully craft 2 oak planks”)


# Begin #
Let's Begin. Please follow the Instruction and focus on solve the task.
# Current items in the inventory #
These are the items currently in your inventory. You cannot use items that are not in your inventory to craft new items.
===inventory===

# Task #
===task===
\end{Verbatim}
\end{tcolorbox}

\begin{tcolorbox}[title=Prompt on MATH, breakable, width=\textwidth,top=0mm]
\begin{Verbatim}[breaklines, fontsize=\footnotesize]
Your task is to solve math competition problems by writing Python programs.

You may also leverage the following helper functions, but must avoid fabricating and calling undefined function names.
```python
===api===
```

Examples: 

Examples: 
Query: Point $P$ lies on the line $x= -3$ and is 10 units from the point $(5,2)$. Find the product of all possible $y$-coordinates that satisfy the given conditions.
Program: 
```python
from sympy import symbols, Eq, solve
# Define symbolic variable for y-coordinate of point P
y = symbols('y')
# Step 1: Given conditions
x1 = -3  # Point P lies on the vertical line x = -3
x2, y2 = 5, 2  # Coordinates of the given point (5, 2)
d = 10  # Distance between point P and (5,2)
# Step 2: Apply the distance formula
# Distance formula: sqrt((x2 - x1)^2 + (y - y2)^2) = d
# Squaring both sides to eliminate the square root:
# (x2 - x1)^2 + (y - y2)^2 = d^2
distance_equation = Eq((x2 - x1)**2 + (y - y2)**2, d**2)
# Step 3: Solve for possible values of y
y_solutions = solve(distance_equation, y)
# Step 4: Compute the product of all possible y-values
product = y_solutions[0] * y_solutions[1]
# Step 5: Output the final result
print("Final Answer:", product)
```

Query: If $3p+4q=8$ and $4p+3q=13$, what is $q$ equal to?
Program:
```python
from sympy import symbols, Eq, solve
# Define symbolic variables for the unknowns p and q
p, q = symbols('p q')
# Step 1: Define the given system of equations
eq1 = Eq(3 * p + 4 * q, 8)  # Equation 1: 3p + 4q = 8
eq2 = Eq(4 * p + 3 * q, 13)  # Equation 2: 4p + 3q = 13
# Step 2: Solve the system of equations for p and q
solution = solve((eq1, eq2), (p, q))
# Step 3: Extract and output the value of q
print("Final Answer:", solution[q])
```

Query: Simplify $\frac{3^4+3^2}{3^3-3}$. Express your answer as a common fraction.
Program:
```python
from sympy import symbols, simplify
# Define the variable
x = symbols('x')
# Define the expression
numerator = 3**4 + 3**2
denominator = 3**3 - 3
fraction = numerator / denominator
# Simplify the fraction
simplified_fraction = simplify(fraction)
# Print the result
print("Final Answer:", simplified_fraction)
```

===example===

## Begin !
Please generate ONLY the code wrapped in ```python...``` to solve the query below.

Query: ===task===
Program:
\end{Verbatim}
\end{tcolorbox}



\begin{tcolorbox}[title=Prompt on Date, breakable, width=\textwidth,top=0mm]
\begin{Verbatim}[breaklines, fontsize=\footnotesize]
Your task is to solve simple word problems by creating Python programs.

You may also leverage the following helper functions, but must avoid fabricating and calling undefined function names, such as `calculate_date_by_years`.
```python
===api===
```

Examples:

Query: In the US, Thanksgiving is on the fourth Thursday of November. Today is the US Thanksgiving of 2001. What is the date one week from today in MM/DD/YYYY?
Program:
```python
# import relevant packages
from datetime import date, time, datetime
from dateutil.relativedelta import relativedelta
import calendar
# 1. What is the date of the first Thursday of November? (independent, support: [])
date_1st_thu = date(2001,11,1)
while date_1st_thu.weekday() != calendar.THURSDAY:
    date_1st_thu += relativedelta(days=1)
# 2. How many days are there in a week? (independent, support: ["External knowledge: There are 7 days in a week."])
n_days_of_a_week = 7
# 3. What is the date today? (depends on 1 and 2, support: ["Today is the US Thanksgiving of 2001", "Thanksgiving is on the fourth Thursday of November"])
days_from_1st_to_4th_thu = (4-1) * n_days_of_a_week
date_today = date_1st_thu + relativedelta(days=days_from_1st_to_4th_thu)
# 4. What is the date one week from today? (depends on 3, support: [])
date_1week_from_today = date_today + relativedelta(weeks=1)
# 5. Final Answer: What is the date one week from today in MM/DD/YYYY? (depends on 4, support: [])
answer = date_1week_from_today.strftime("%m/%d/%Y")
# print the answer
print(answer)
```

Query: Yesterday was 12/31/1929. Today could not be 12/32/1929 because December has only 31 days. What is the date tomorrow in MM/DD/YYYY?
Program:
```python
# import relevant packages
from datetime import date, time, datetime
from dateutil.relativedelta import relativedelta
# 1. What is the date yesterday? (independent, support: ["Yesterday was 12/31/1929"])
date_yesterday = date(1929,12,31)
# 2. What is the date today? (depends on 1, support: ["Today could not be 12/32/1929 because December has only 31 days"])
date_today = date_yesterday + relativedelta(days=1)
# 3. What is the date tomorrow? (depends on 2, support: [])
date_tomorrow = date_today + relativedelta(days=1)
# 4. Final Answer: What is the date tomorrow in MM/DD/YYYY? (depends on 3, support: [])
answer = date_tomorrow.strftime("%m/%d/%Y")
# print the answer
print(answer)
```

Query: The day before yesterday was 11/23/1933. What is the date one week from today in MM/DD/YYYY?
Program:
```python
# import relevant packages
from datetime import date, time, datetime
from dateutil.relativedelta import relativedelta
# 1. What is the date the day before yesterday? (independent, support: ["The day before yesterday was 11/23/1933"])
date_day_before_yesterday = date(1933,11,23)
# 2. What is the date today? (depends on 1, support: [])
date_today = date_day_before_yesterday + relativedelta(days=2)
# 3. What is the date one week from today? (depends on 2, support: [])
date_1week_from_today = date_today + relativedelta(weeks=1)
# 4. Final Answer: What is the date one week from today in MM/DD/YYYY? (depends on 3, support: [])
answer = date_1week_from_today.strftime("%m/%d/%Y")
# print the answer
print(answer)
```

===example===

## Begin !
Please generate ONLY the code wrapped in ```python...``` to solve the query below.

Query: ===task===
Program:
\end{Verbatim}
\end{tcolorbox}



\begin{tcolorbox}[title=Prompt on TabMWP, breakable, width=\textwidth,top=0mm]
\begin{Verbatim}[breaklines, fontsize=\footnotesize]
Your task is to solve table-reasoning problems by writing Python programs.
You are given a table. The first row is the name for each column. Each column is seperated by "|" and each row is seperated by "\n".
Pay attention to the format of the table, and what the question asks.

You may also leverage the following helper functions, but must avoid fabricating and calling undefined function names.
```python
===api===
```


Examples: 
### Table
Name: None
Unit: $
Content:
Date | Description | Received | Expenses | Available Funds
 | Balance: end of July | | | $260.85
8/15 | tote bag | | $6.50 | $254.35
8/16 | farmers market | | $23.40 | $230.95
8/22 | paycheck | $58.65 | | $289.60
### Question
This is Akira's complete financial record for August. How much money did Akira receive on August 22?
### Solution code
```python
records = {
    "7/31": {"Description": "Balance: end of July", "Received": "", "Expenses": "", "Available Funds": 260.85},
    "8/15": {"Description": "tote bag", "Received": "", "Expenses": 6.5, "Available Funds": ""},
    "8/16": {"Description": "farmers market", "Received": "", "Expenses": 23.4, "Available Funds": ""},
    "8/22": {"Description": "paycheck", "Received": 58.65, "Expenses": "", "Available Funds": ""}
}
# Access the amount received on August 22
received_aug_22 = records["8/22"]["Received"]
print("Final Answer: ", received_aug_22)
```

### Table
Name: Orange candies per bag
Unit: bags
Content:
Stem | Leaf 
2 | 2, 3, 9
3 | 
4 | 
5 | 0, 6, 7, 9
6 | 0
7 | 1, 3, 9
8 | 5
### Question
A candy dispenser put various numbers of orange candies into bags. How many bags had at least 32 orange candies?
### Solution code
```python
data = {
    2: [2, 3, 9],
    3: [],
    4: [],
    5: [0, 6, 7, 9],
    6: [0],
    7: [1, 3, 9],
    8: [5]
}
# Initialize the count to zero
count = 0
# Iterate over the keys in the dictionary
for key in data:
    # Combine tenth digit and unit digit
    if key * 10 + data[key] >= 32:
        # Increment the count
        count += 1

# Output the result
print("Final Answer: ", count)
```

### Table
Name: Monthly Savings  
Unit: $  
Content:  
Date  | Description       | Received | Expenses | Available Funds |
       | Balance: end of May |   |   | $500.00 |
6/10  | groceries        |   | $45.75 | $454.25 |
6/15  | gas refill       |   | $30.20 | $424.05 |
6/25  | salary           | $1200.00 |   | $1624.05 |
### Question
How much money did Akira receive on June 25?
### Solution code
```python
import pandas as pd
records = {
    "5/31": {"Description": "Balance: end of May", "Received": "", "Expenses": "", "Available Funds": 500.00},
    "6/10": {"Description": "groceries", "Received": "", "Expenses": 45.75, "Available Funds": ""},
    "6/15": {"Description": "gas refill", "Received": "", "Expenses": 30.2, "Available Funds": ""},
    "6/25": {"Description": "salary", "Received": 1200.00, "Expenses": "", "Available Funds": ""}
}
# Access the amount received on June 25
received_june_25 = records["6/25"]["Received"]
print("Final Answer: ", received_june_25)
```

===example===

## Begin!
Please solve the task below and enclose your code within a single code block using ```python```  format.

===task===
### Solution code
\end{Verbatim}
\end{tcolorbox}









\section{Examples}
\label{app:example}
\subsection{Generated Tools}

\textbf{The tools generated for the Open-ended Tasks are as follows:}
\begin{tcolorbox}[title=CraftDiamondHelmet, width=\textwidth,top=0mm,  breakable]
\begin{Verbatim}[breaklines=true, breakanywhere=true, fontsize=\footnotesize]
async function craftDiamondHelmet(bot) {
  const mcData = require('minecraft-data')(bot.version);
  const Vec3 = require('vec3').Vec3;

  // Check inventory for the number of diamonds
  const diamondCount = bot.inventory.count(mcData.itemsByName["diamond"].id);
  const requiredDiamonds = 5;

  // If not enough diamonds, mine diamond ores
  if (diamondCount < requiredDiamonds) {
    const diamondsToMine = requiredDiamonds - diamondCount;
    bot.chat(`Need ${diamondsToMine} more diamond(s). Mining...`);
    await mineBlock(bot, "diamond_ore", diamondsToMine);
  }

  // Ensure a crafting table is placed nearby
  let craftingTable = bot.findBlock({
    matching: mcData.blocksByName.crafting_table.id,
    maxDistance: 32
  });
  if (!craftingTable) {
    bot.chat("Placing crafting table...");
    await placeItem(bot, "crafting_table", bot.entity.position.offset(1, 0, 0));
    craftingTable = bot.findBlock({
      matching: mcData.blocksByName.crafting_table.id,
      maxDistance: 32
    });
  }

  // Craft the diamond helmet
  bot.chat("Crafting diamond helmet...");
  await craftItem(bot, "diamond_helmet", 1);
  bot.chat("Diamond helmet crafted successfully.");
}
\end{Verbatim}
\end{tcolorbox}


\begin{tcolorbox}[title=CraftItemWithMaterials, width=\textwidth,top=0mm,  breakable]
\begin{Verbatim}[breaklines=true, breakanywhere=true, fontsize=\footnotesize]
async function craftItemWithMaterials(bot, itemName, requiredMaterials) {
  const mcData = require('minecraft-data')(bot.version);
  const Vec3 = require('vec3').Vec3;

  // Check inventory for required materials
  for (const material of requiredMaterials) {
    let itemCount = bot.inventory.count(mcData.itemsByName[material.name].id);
    if (itemCount < material.count) {
      const requiredCount = material.count - itemCount;
      bot.chat(`Need ${requiredCount} more ${material.name}(s).`);
      if (material.name === "diamond") {
        let diamondOre = await bot.findBlock({
          matching: mcData.blocksByName["diamond_ore"].id,
          maxDistance: 32
        });
        if (!diamondOre) {
          bot.chat("No diamond ore found nearby. Exploring...");
          diamondOre = await exploreUntil(bot, new Vec3(1, 0, 1), 60, () => {
            return bot.findBlock({
              matching: mcData.blocksByName["diamond_ore"].id,
              maxDistance: 32
            });
          });
        }
        if (diamondOre) {
          await mineBlock(bot, "diamond_ore", requiredCount);
        } else {
          bot.chat("Failed to find diamond ore after exploring.");
          return;
        }
      } else if (material.name === "stick") {
        const woodenPlanksCount = bot.inventory.count(mcData.itemsByName["oak_planks"].id) + bot.inventory.count(mcData.itemsByName["birch_planks"].id);
        if (woodenPlanksCount < 2) {
          const requiredLogs = Math.ceil((2 - woodenPlanksCount) / 4);
          bot.chat(`Need more wooden planks. Gathering ${requiredLogs} logs...`);
          await obtainWoodLogs(bot, requiredLogs);
          await craftItem(bot, "oak_planks", requiredLogs);
        }
        bot.chat("Crafting sticks...");
        await craftItem(bot, "stick", 1);
      }
    }
  }

  // Ensure a crafting table is placed nearby
  let craftingTable = bot.findBlock({
    matching: mcData.blocksByName.crafting_table.id,
    maxDistance: 32
  });
  if (!craftingTable) {
    bot.chat("Placing crafting table...");
    await placeItem(bot, "crafting_table", bot.entity.position.offset(1, 0, 0));
    craftingTable = bot.findBlock({
      matching: mcData.blocksByName.crafting_table.id,
      maxDistance: 32
    });
  }

  // Craft the item
  bot.chat(`Crafting ${itemName}...`);
  await craftItem(bot, itemName, 1);
  bot.chat(`${itemName} crafted successfully.`);
}

async function craftDiamondAxe(bot) {
  const requiredMaterials = [{
    name: "diamond",
    count: 3
  }, {
    name: "stick",
    count: 2
  }];
  await craftItemWithMaterials(bot, "diamond_axe", requiredMaterials);
}
\end{Verbatim}
\end{tcolorbox}


\textbf{The tools generated for the Agent Tasks are as follows:}
Here, we can clearly see the call relationships between functions, thus forming more complex tools.
\begin{tcolorbox}[title=Tools for DA-Bench, width=\textwidth,top=0mm,  breakable]
\begin{Verbatim}[breaklines=true, breakanywhere=true, fontsize=\footnotesize]
def filter_rows_by_non_null(df: pd.DataFrame, column_name: str) -> pd.DataFrame:
    """
    Filters rows in a dataset based on non-null values in a specified column.
    
    Parameters:
    - df (pd.DataFrame): The input DataFrame.
    - column_name (str): The name of the column to filter by non-null values.
    
    Returns:
    - pd.DataFrame: A DataFrame with rows containing non-null values in the specified column.
    
    Raises:
    - ValueError: If the specified column is not found in the DataFrame.
    """
    # Check if the column exists in the DataFrame
    if column_name not in df.columns:
        raise ValueError(f"Column '{column_name}' not found in the DataFrame.")
    
    # Filter rows based on non-null values in the specified column
    filtered_df = df.dropna(subset=[column_name])
    
    return filtered_df

def convert_column_to_numeric(df: pd.DataFrame, column_name: str) -> pd.DataFrame:
    """
    Converts a specified column in a DataFrame to numeric values, handling non-numeric values appropriately.
    
    Parameters:
    - df (pd.DataFrame): The input DataFrame.
    - column_name (str): The name of the column to convert to numeric values.
    
    Returns:
    - pd.DataFrame: The DataFrame with the specified column converted to numeric values.
    
    Raises:
    - ValueError: If the specified column is not found in the DataFrame.
    """
    # Check if the column exists in the DataFrame
    if column_name not in df.columns:
        raise ValueError(f"Column '{column_name}' not found in the DataFrame.")
    
    # Convert the specified column to numeric values, setting non-numeric values to NaN
    df[column_name] = pd.to_numeric(df[column_name], errors='coerce')
    
    # Filter out rows with non-numeric values in the specified column using the existing tool
    df = filter_rows_by_non_null(df, column_name)
    
    return df

def create_sum_feature(df: pd.DataFrame, new_column_name: str, columns_to_sum: list) -> pd.DataFrame:
    """
    Creates a new feature by summing specified columns in a DataFrame.
    
    Parameters:
    - df (pd.DataFrame): The input DataFrame.
    - new_column_name (str): The name of the new column to be created.
    - columns_to_sum (list): A list of column names to sum.
    
    Returns:
    - pd.DataFrame: The DataFrame with the new feature added.
    
    Raises:
    - ValueError: If any of the specified columns are not found in the DataFrame.
    """
    # Check if all specified columns exist in the DataFrame
    for column in columns_to_sum:
        if column not in df.columns:
            raise ValueError(f"Column '{column}' not found in the DataFrame.")
    
    # Convert specified columns to numeric values
    for column in columns_to_sum:
        df = convert_column_to_numeric(df, column)
    
    # Create the new feature by summing the specified columns
    df[new_column_name] = df[columns_to_sum].sum(axis=1)
    
    return df
\end{Verbatim}
\end{tcolorbox}


\begin{tcolorbox}[title=Tools for TextCraft, width=\textwidth,top=0mm, breakable]
\begin{Verbatim}[breaklines=true, breakanywhere=true, fontsize=\footnotesize]
def gather_materials_for_dye(required_materials: dict) -> bool:
    """
    Gathers the required materials for crafting any dye.
    
    Parameters:
    - required_materials (dict): A dictionary where keys are material names and values are the required quantities.
    
    The tool checks the inventory for these materials and gathers them if they are missing.
    
    Returns:
    - bool: True if all materials were successfully gathered, False otherwise.
    """
    # Gather the required materials
    if not gather_materials(required_materials):
        return False
    
    # Check if we have white dye, if not gather bone meal or lily of the valley to craft it
    inventory = check_inventory()
    if "white dye" in required_materials and "white dye" not in inventory:
        if not gather_materials({"bone meal": 1}) and not gather_materials({"lily of the valley": 1}):
            return False
        # Craft white dye using bone meal or lily of the valley
        if "bone meal" in inventory:
            craft_object("1 white dye", ["1 bone meal"])
        elif "lily of the valley" in inventory:
            craft_object("1 white dye", ["1 lily of the valley"])
    
    # Recheck the inventory to ensure all materials are gathered
    missing_items = check_missing_items([f"{qty} {item}" for item, qty in required_materials.items()])
    if missing_items:
        print(f"Missing items: {missing_items}")
        return False
    
    # Successfully gathered all materials
    return True

def craft_orange_dye(quantity: int) -> bool:
    """
    Crafts the specified quantity of orange dye.
    
    Parameters:
    - quantity (int): The number of orange dye to craft.
    
    Returns:
    - bool: True if the orange dye was successfully crafted, False otherwise.
    """
    # Define the required materials for crafting orange dye
    required_materials = {"orange tulip": quantity, "red dye": quantity, "yellow dye": quantity}
    
    # Gather the required materials using the existing gather_materials_for_dye function
    if not gather_materials_for_dye(required_materials):
        return False
    
    # Check the inventory for available materials
    inventory = check_inventory()
    
    # Craft orange dye using orange tulip if available
    if "orange tulip" in inventory:
        craft_object(f"{quantity} orange dye", [f"{quantity} orange tulip"])
        print(f"Crafted {quantity} orange dye using {quantity} orange tulip")
        return True
    
    # Craft orange dye using red dye and yellow dye if available
    if "red dye" in inventory and "yellow dye" in inventory:
        craft_object(f"{quantity} orange dye", [f"{quantity} red dye", f"{quantity} yellow dye"])
        print(f"Crafted {quantity} orange dye using {quantity} red dye and {quantity} yellow dye")
        return True
    
    # If neither method was successful, return False
    print("Failed to craft orange dye.")
    return False
\end{Verbatim}
\end{tcolorbox}


\textbf{The tools generated for the Single-turn Code Task are as follows:}
\begin{tcolorbox}[title=Tools for MATH, width=\textwidth,top=0mm, breakable]
\begin{Verbatim}[breaklines=true, breakanywhere=true, fontsize=\footnotesize]
def find_integer_satisfying_condition(condition):
    """
    Find the smallest positive integer that satisfies the given condition.

    Parameters:
        condition (function): A lambda function representing the condition to be checked.

    Returns:
        int: The smallest positive integer that satisfies the condition.
    """
    x = 1
    while True:
        if condition(x):
            return x
        x += 1

def calculate_min_correct_answers(total_problems, passing_percentage):
    """
    Calculate the minimum number of correct answers required to pass a test based on the total number of problems and the passing percentage.

    Parameters:
        total_problems (int): The total number of problems on the test.
        passing_percentage (float): The passing percentage required to pass the test.

    Returns:
        int: The minimum number of correct answers required to pass the test.
    """
    if total_problems <= 0:
        return "Total number of problems must be greater than zero."
    if not (0 <= passing_percentage <= 100):
        return "Passing percentage must be between 0 and 100."

    required_correct_answers = (passing_percentage / 100) * total_problems

    # Use find_integer_satisfying_condition to find the minimum integer satisfying the condition
    min_correct_answers = find_integer_satisfying_condition(lambda x: x >= required_correct_answers)
    
    return min_correct_answers
\end{Verbatim}
\end{tcolorbox}

\begin{tcolorbox}[title=Tools for Date, width=\textwidth,top=0mm, breakable]
\begin{Verbatim}[breaklines=true, breakanywhere=true, fontsize=\footnotesize]
def calculate_date_by_days(start_date_str: str, days_to_add: int, date_format="%m/%d/%Y") -> str:
    """
    Calculates the date a specified number of days before or after a given date.

    Parameters:
    - start_date_str (str): The starting date as a string in the format MM/DD/YYYY.
    - days_to_add (int): The number of days to add (positive) or subtract (negative) from the start date.
    - date_format (str): The format of the input and output date string. Default is 'MM/DD/YYYY'.

    Returns:
    - str: The resulting date in the format MM/DD/YYYY.
    
    Raises:
    - ValueError: If the input date string does not match the specified format.
    - OverflowError: If the resulting date is out of the valid range for dates.
    """
    from datetime import datetime, timedelta

    try:
        # Parse the input date string into a date object using the provided format
        start_date = datetime.strptime(start_date_str, date_format).date()

        # Calculate the new date by adding the specified number of days
        new_date = start_date + timedelta(days=days_to_add)

        # Format the new date back into the desired string format
        result_date_str = new_date.strftime(date_format)

        return result_date_str
    except ValueError as e:
        raise ValueError("Incorrect date format. Please ensure the date string matches the provided format.") from e
    except OverflowError as e:
        raise OverflowError("The resulting date is out of the valid range for dates.") from e

def calculate_date_by_days_uk_format(start_date_str: str, days_to_add: int) -> str:
    """
    Calculates the date a specified number of days before or after a given date in UK format (DD/MM/YYYY).

    Parameters:
    - start_date_str (str): The starting date as a string in the format DD/MM/YYYY.
    - days_to_add (int): The number of days to add (positive) or subtract (negative) from the start date.

    Returns:
    - str: The resulting date in the format MM/DD/YYYY.
    
    Raises:
    - ValueError: If the input date string does not match the specified format.
    """
    from datetime import datetime

    try:
        # Convert the input date from DD/MM/YYYY to MM/DD/YYYY
        start_date = datetime.strptime(start_date_str, "%d/%m/%Y")
        
        # Use the existing tool to calculate the new date
        result_date_str = calculate_date_by_days(start_date.strftime("%m/%d/%Y"), days_to_add, "%m/%d/%Y")
        
        return result_date_str
    except ValueError as e:
        raise ValueError("Incorrect date format. Please ensure the date string matches the provided format.") from e
\end{Verbatim}
\end{tcolorbox}


\begin{tcolorbox}[title=Tools for TabMWP, width=\textwidth,top=0mm, breakable]
\begin{Verbatim}[breaklines=true, breakanywhere=true, fontsize=\footnotesize]
import pandas as pd

def stem_and_leaf_to_dataframe(stem_leaf_dict: dict) -> pd.DataFrame:
    """
    Converts a stem-and-leaf plot into a DataFrame.

    Parameters:
    - stem_leaf_dict (dict): A dictionary where keys are the stems and values are lists of leaves.

    Returns:
    - pd.DataFrame: A DataFrame with a single column containing the combined values of stems and leaves.
    """
    # Initialize an empty list to store the combined values
    combined_values = []

    # Iterate through the dictionary to combine stems and leaves
    for stem, leaves in stem_leaf_dict.items():
        for leaf in leaves:
            combined_value = int(f"{stem}{leaf}")
            combined_values.append(combined_value)

    # Create a DataFrame from the combined values
    df = pd.DataFrame(combined_values, columns=["Values"])
    
    return df

import pandas as pd

def count_value_occurrences(stem_leaf_dict: dict, value) -> int:
    """
    Counts the occurrences of a specific value in a DataFrame column created from a stem-and-leaf plot.

    Parameters:
    - stem_leaf_dict (dict): A dictionary where keys are the stems and values are lists of leaves.
    - value: The value to count in the DataFrame.

    Returns:
    - int: The count of the specified value in the DataFrame.
    """
    # Convert the stem-and-leaf plot to a DataFrame using the existing tool
    df = stem_and_leaf_to_dataframe(stem_leaf_dict)
    
    # Count the occurrences of the specified value in the DataFrame
    count = df["Values"].value_counts().get(value, 0)
    
    return count
\end{Verbatim}
\end{tcolorbox}

\newpage 

\bibliographystyle{ACM-Reference-Format}
\bibliography{genetic_privacy,genetic_privacy-news}



\end{document}
\endinput
%%
%% End of file `sample-manuscript.tex'.
