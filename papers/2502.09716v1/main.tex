%%
%% This is file `sample-manuscript.tex',
%% generated with the docstrip utility.
%%
%% The original source files were:
%%
%% samples.dtx  (with options: `all,proceedings,bibtex,manuscript')
%% 
%% IMPORTANT NOTICE:
%% 
%% For the copyright see the source file.
%% 
%% Any modified versions of this file must be renamed
%% with new filenames distinct from sample-manuscript.tex.
%% 
%% For distribution of the original source see the terms
%% for copying and modification in the file samples.dtx.
%% 
%% This generated file may be distributed as long as the
%% original source files, as listed above, are part of the
%% same distribution. (The sources need not necessarily be
%% in the same archive or directory.)
%%
%%
%% Commands for TeXCount
%TC:macro \cite [option:text,text]
%TC:macro \citep [option:text,text]
%TC:macro \citet [option:text,text]
%TC:envir table 0 1
%TC:envir table* 0 1
%TC:envir tabular [ignore] word
%TC:envir displaymath 0 word
%TC:envir math 0 word
%TC:envir comment 0 0
%%
%% The first command in your LaTeX source must be the \documentclass
%% command.
%%
%% For submission and review of your manuscript please change the
%% command to \documentclass[manuscript, screen, review]{acmart}.
%%
%% When submitting camera ready or to TAPS, please change the command
%% to \documentclass[sigconf]{acmart} or whichever template is required
%% for your publication.
%%
%%
% \documentclass[manuscript,screen,review]{acmart}
%authordraft
\documentclass[manuscript=false,authorversion=true,review=false]{acmart}
\usepackage{footnote}
\usepackage{setspace}
% CHANGE TO DRAFT HERE FOR REVIEW
\usepackage[final,inline,nomargin,index]{fixme}
\fxsetup{theme=color,mode=multiuser,inlineface=\itshape,envface=\itshape}

\FXRegisterAuthor{sv}{asv}{\colorbox{gray!10!white}{\color{black}Suresh}}
\FXRegisterAuthor{rv}{arv}{\colorbox{blue!10!white}{\color{black}Ria}}
\FXRegisterAuthor{fr}{afr}{\colorbox{red!10!white}{\color{black}For Review}}
\FXRegisterAuthor{cw}{acw}{\colorbox{green!10!white}{\color{black}Cole}}
\FXRegisterAuthor{sr}{asr}{\colorbox{purple!10!white}{\color{black}Sohini}}
\FXRegisterAuthor{vr}{avr}{\colorbox{orange!10!white}{\color{black}Vivek}}

\newcommand{\rv}[1]{\rverror{#1}}
%\doublespacing
%%
%% \BibTeX command to typeset BibTeX logo in the docs
\AtBeginDocument{%
  \providecommand\BibTeX{{%
    Bib\TeX}}}

%% Rights management information.  This information is sent to you
%% when you complete the rights form.  These commands have SAMPLE
%% values in them; it is your responsibility as an author to replace
%% the commands and values with those provided to you when you
%% complete the rights form.
%\acmConference[FAccT '25]{FAccT '25: ACM Conference on Fairness,
%Accountability, and Transparency}{June 23-26, 2025}{Athens, Greece}
%\acmBooktitle{FAccT '25: ACM Conference on Fairness,
%Accountability, and Transparency,
%  June 23-26, 2025, Athens, Greece}
  
%\setcopyright{acmlicensed}
%\copyrightyear{2025}
%\acmYear{2025}
%\acmDOI{XXXXXXX.XXXXXXX}
%% These commands are for a PROCEEDINGS abstract or paper.
%\acmConference[Conference acronym 'XX]{Make sure to enter the correct
% conference title from your rights confirmation emai}{June 03--05,
%  2018}{Woodstock, NY}
%%
%%  Uncomment \acmBooktitle if the title of the proceedings is different
%%  from ``Proceedings of ...''!
%%
%%\acmBooktitle{Woodstock '18: ACM Symposium on Neural Gaze Detection,
%%  June 03--05, 2018, Woodstock, NY}
%\acmISBN{978-1-4503-XXXX-X/18/06}


%%
%% Submission ID.
%% Use this when submitting an article to a sponsored event. You'll
%% receive a unique submission ID from the organizers
%% of the event, and this ID should be used as the parameter to this command.
%%\acmSubmissionID{123-A56-BU3}

%%
%% For managing citations, it is recommended to use bibliography
%% files in BibTeX format.
%%
%% You can then either use BibTeX with the ACM-Reference-Format style,
%% or BibLaTeX with the acmnumeric or acmauthoryear sytles, that include
%% support for advanced citation of software artefact from the
%% biblatex-software package, also separately available on CTAN.
%%
%% Look at the sample-*-biblatex.tex files for templates showcasing
%% the biblatex styles.
%%

%%
%% The majority of ACM publications use numbered citations and
%% references.  The command \citestyle{authoryear} switches to the
%% "author year" style.
%%
%% If you are preparing content for an event
%% sponsored by ACM SIGGRAPH, you must use the "author year" style of
%% citations and references.
%% Uncommenting
%% the next command will enable that style.
%%\citestyle{acmauthoryear}


%%
%% end of the preamble, start of the body of the document source

\begin{document}

%%
%% The "title" command has an optional parameter,
%% allowing the author to define a "short title" to be used in page headers.
\title[Genetic Data Governance in Crisis]{Genetic Data Governance in Crisis: Policy Recommendations for 
  Safeguarding Privacy and Preventing Discrimination}

%%
%% The "author" command and its associated commands are used to define
%% the authors and their affiliations.
%% Of note is the shared affiliation of the first two authors, and the
%% "authornote" and "authornotemark" commands
%% used to denote shared contribution to the research.
% \author{Ben Trovato}
% \authornote{Both authors contributed equally to this research.}
% \email{trovato@corporation.com}
% \orcid{1234-5678-9012}
% \author{G.K.M. Tobin}
% \authornotemark[1]
% \email{webmaster@marysville-ohio.com}
% \affiliation{%
%   \institution{Institute for Clarity in Documentation}
%   \city{Dublin}
%   \state{Ohio}
%   \country{USA}
% }

\author{Vivek Ramanan}
\affiliation{%
  \institution{Brown University}
  \city{Providence}
  \country{USA}}

\author{Ria Vinod}
\affiliation{%
  \institution{Brown University}
  \city{Providence}
  \country{USA}
}

\author{Cole Williams}
\affiliation{%
 \institution{Brown University}
 \city{Providence}
 \country{USA}}

\author{Sohini Ramachandran}
\affiliation{%
  \institution{Brown University}
  \city{Providence}
  \country{USA}}

\author{Suresh Venkatasubramanian}
\affiliation{%
  \institution{Brown University}
  \city{Providence}
  \country{USA}}


%%
%% By default, the full list of authors will be used in the page
%% headers. Often, this list is too long, and will overlap
%% other information printed in the page headers. This command allows
%% the author to define a more concise list
%% of authors' names for this purpose.
\renewcommand{\shortauthors}{Ramanan, Vinod, Williams et al.}

%%
%% The abstract is a short summary of the work to be presented in the
%% article.
\begin{abstract}
Genetic data collection has become ubiquitous today. The ability to meaningfully interpret genetic data has motivated its widespread use across forensics, clinical practice, and research, providing crucial insights into human health and ancestry while driving important public health initiatives. Easy access to genetic testing has fueled a rapid expansion of direct-to-consumer offerings, many of which are recreational in nature. However, the growth of genetic datasets and their applications has created significant privacy and discrimination risks, particularly as our understanding of the scientific basis for genetic traits continues to evolve. In this paper, we organize the uses of genetic data along four distinct `pillars': clinical practice, research, forensic and government use, and recreational use. Using our scientific understanding of genetics, genetic inference methods and their associated risks, and existing regulatory mechanisms, we build a risk assessment framework that identifies key values that any governance system must preserve. We then analyze case studies from each of the pillars using this framework to assess how well existing legal and regulatory frameworks preserve desired values. Our investigation reveals critical gaps in existing regulatory frameworks and identifies specific threats to privacy and personal liberties, particularly through genetic discrimination. To address these challenges, we call for and propose comprehensive regulatory reforms including: (1) updating the legal definition of genetic data to protect against modern technological capabilities, (2) expanding the Genetic Information Nondiscrimination Act (GINA) to cover currently unprotected domains, and (3) establishing a unified regulatory framework under a single governing body to oversee all applications of genetic data. 

We conclude with three open questions about genetic data: the challenges posed by its relational nature, including consent for relatives and minors; the complexities of international data transfer; and its potential integration into large language models.


\end{abstract}

%%
%% The code below is generated by the tool at http://dl.acm.org/ccs.cfm.
%% Please copy and paste the code instead of the example below.
%%
\begin{CCSXML}
<ccs2012>
   <concept>
       <concept_id>10010405.10010444.10010093.10010934</concept_id>
       <concept_desc>Applied computing~Computational genomics</concept_desc>
       <concept_significance>500</concept_significance>
       </concept>
   <concept>
       <concept_id>10010405.10010444.10010449</concept_id>
       <concept_desc>Applied computing~Health informatics</concept_desc>
       <concept_significance>300</concept_significance>
       </concept>
   <concept>
       <concept_id>10010405.10010455.10010458</concept_id>
       <concept_desc>Applied computing~Law</concept_desc>
       <concept_significance>100</concept_significance>
       </concept>
   <concept>
       <concept_id>10010405.10010444.10010446</concept_id>
       <concept_desc>Applied computing~Consumer health</concept_desc>
       <concept_significance>300</concept_significance>
       </concept>
 </ccs2012>
\end{CCSXML}

\ccsdesc[500]{Applied computing~Computational genomics}
\ccsdesc[300]{Applied computing~Health informatics}
\ccsdesc[100]{Applied computing~Law}
\ccsdesc[300]{Applied computing~Consumer health}
\ccsdesc[500]{}
% \ccsdesc[300]{Do Not Use This Code~Generate the Correct Terms for Your Paper}
% \ccsdesc{Do Not Use This Code~Generate the Correct Terms for Your Paper}
% \ccsdesc[100]{Do Not Use This Code~Generate the Correct Terms for Your Paper}

%%
%% Keywords. The author(s) should pick words that accurately describe
%% the work being presented. Separate the keywords with commas.
\keywords{Data Governance, Genetic Privacy, Regulatory Frameworks, Genetic Discrimination, Policy, Data Protection, Genetic Inference}

% \received{20 February 2007}
% \received[revised]{12 March 2009}
% \received[accepted]{5 June 2009}

\renewcommand\englishlistfixmename{List of Notes}

% \FXRegisterAuthor{sv}{asv}{\colorbox{gray!10!white}{\color{black}Suresh}}
% \FXRegisterAuthor{rv}{arv}{\colorbox{blue!10!white}{\color{black}Ria}}
% \FXRegisterAuthor{vr}{avr}{\colorbox{red!10!white}{\color{black}Vivek}}
% \FXRegisterAuthor{cw}{acw}{\colorbox{green!10!white}{\color{black}Cole}}
% \FXRegisterAuthor{sr}{asr}{\colorbox{purple!10!white}{\color{black}Sohini}}

%%
%% This command processes the author and affiliation and title
%% information and builds the first part of the formatted document.
\maketitle

\section{Introduction}

Despite the remarkable capabilities of large language models (LLMs)~\cite{DBLP:conf/emnlp/QinZ0CYY23,DBLP:journals/corr/abs-2307-09288}, they often inevitably exhibit hallucinations due to incorrect or outdated knowledge embedded in their parameters~\cite{DBLP:journals/corr/abs-2309-01219, DBLP:journals/corr/abs-2302-12813, DBLP:journals/csur/JiLFYSXIBMF23}.
Given the significant time and expense required to retrain LLMs, there has been growing interest in \emph{model editing} (a.k.a., \emph{knowledge editing})~\cite{DBLP:conf/iclr/SinitsinPPPB20, DBLP:journals/corr/abs-2012-00363, DBLP:conf/acl/DaiDHSCW22, DBLP:conf/icml/MitchellLBMF22, DBLP:conf/nips/MengBAB22, DBLP:conf/iclr/MengSABB23, DBLP:conf/emnlp/YaoWT0LDC023, DBLP:conf/emnlp/ZhongWMPC23, DBLP:conf/icml/MaL0G24, DBLP:journals/corr/abs-2401-04700}, 
which aims to update the knowledge of LLMs cost-effectively.
Some existing methods of model editing achieve this by modifying model parameters, which can be generally divided into two categories~\cite{DBLP:journals/corr/abs-2308-07269, DBLP:conf/emnlp/YaoWT0LDC023}.
Specifically, one type is based on \emph{Meta-Learning}~\cite{DBLP:conf/emnlp/CaoAT21, DBLP:conf/acl/DaiDHSCW22}, while the other is based on \emph{Locate-then-Edit}~\cite{DBLP:conf/acl/DaiDHSCW22, DBLP:conf/nips/MengBAB22, DBLP:conf/iclr/MengSABB23}. This paper primarily focuses on the latter.

\begin{figure}[t]
  \centering
  \includegraphics[width=0.48\textwidth]{figures/demonstration.pdf}
  \vspace{-4mm}
  \caption{(a) Comparison of regular model editing and EAC. EAC compresses the editing information into the dimensions where the editing anchors are located. Here, we utilize the gradients generated during training and the magnitude of the updated knowledge vector to identify anchors. (b) Comparison of general downstream task performance before editing, after regular editing, and after constrained editing by EAC.}
  \vspace{-3mm}
  \label{demo}
\end{figure}

\emph{Sequential} model editing~\cite{DBLP:conf/emnlp/YaoWT0LDC023} can expedite the continual learning of LLMs where a series of consecutive edits are conducted.
This is very important in real-world scenarios because new knowledge continually appears, requiring the model to retain previous knowledge while conducting new edits. 
Some studies have experimentally revealed that in sequential editing, existing methods lead to a decrease in the general abilities of the model across downstream tasks~\cite{DBLP:journals/corr/abs-2401-04700, DBLP:conf/acl/GuptaRA24, DBLP:conf/acl/Yang0MLYC24, DBLP:conf/acl/HuC00024}. 
Besides, \citet{ma2024perturbation} have performed a theoretical analysis to elucidate the bottleneck of the general abilities during sequential editing.
However, previous work has not introduced an effective method that maintains editing performance while preserving general abilities in sequential editing.
This impacts model scalability and presents major challenges for continuous learning in LLMs.

In this paper, a statistical analysis is first conducted to help understand how the model is affected during sequential editing using two popular editing methods, including ROME~\cite{DBLP:conf/nips/MengBAB22} and MEMIT~\cite{DBLP:conf/iclr/MengSABB23}.
Matrix norms, particularly the L1 norm, have been shown to be effective indicators of matrix properties such as sparsity, stability, and conditioning, as evidenced by several theoretical works~\cite{kahan2013tutorial}. In our analysis of matrix norms, we observe significant deviations in the parameter matrix after sequential editing.
Besides, the semantic differences between the facts before and after editing are also visualized, and we find that the differences become larger as the deviation of the parameter matrix after editing increases.
Therefore, we assume that each edit during sequential editing not only updates the editing fact as expected but also unintentionally introduces non-trivial noise that can cause the edited model to deviate from its original semantics space.
Furthermore, the accumulation of non-trivial noise can amplify the negative impact on the general abilities of LLMs.

Inspired by these findings, a framework termed \textbf{E}diting \textbf{A}nchor \textbf{C}ompression (EAC) is proposed to constrain the deviation of the parameter matrix during sequential editing by reducing the norm of the update matrix at each step. 
As shown in Figure~\ref{demo}, EAC first selects a subset of dimension with a high product of gradient and magnitude values, namely editing anchors, that are considered crucial for encoding the new relation through a weighted gradient saliency map.
Retraining is then performed on the dimensions where these important editing anchors are located, effectively compressing the editing information.
By compressing information only in certain dimensions and leaving other dimensions unmodified, the deviation of the parameter matrix after editing is constrained. 
To further regulate changes in the L1 norm of the edited matrix to constrain the deviation, we incorporate a scored elastic net ~\cite{zou2005regularization} into the retraining process, optimizing the previously selected editing anchors.

To validate the effectiveness of the proposed EAC, experiments of applying EAC to \textbf{two popular editing methods} including ROME and MEMIT are conducted.
In addition, \textbf{three LLMs of varying sizes} including GPT2-XL~\cite{radford2019language}, LLaMA-3 (8B)~\cite{llama3} and LLaMA-2 (13B)~\cite{DBLP:journals/corr/abs-2307-09288} and \textbf{four representative tasks} including 
natural language inference~\cite{DBLP:conf/mlcw/DaganGM05}, 
summarization~\cite{gliwa-etal-2019-samsum},
open-domain question-answering~\cite{DBLP:journals/tacl/KwiatkowskiPRCP19},  
and sentiment analysis~\cite{DBLP:conf/emnlp/SocherPWCMNP13} are selected to extensively demonstrate the impact of model editing on the general abilities of LLMs. 
Experimental results demonstrate that in sequential editing, EAC can effectively preserve over 70\% of the general abilities of the model across downstream tasks and better retain the edited knowledge.

In summary, our contributions to this paper are three-fold:
(1) This paper statistically elucidates how deviations in the parameter matrix after editing are responsible for the decreased general abilities of the model across downstream tasks after sequential editing.
(2) A framework termed EAC is proposed, which ultimately aims to constrain the deviation of the parameter matrix after editing by compressing the editing information into editing anchors. 
(3) It is discovered that on models like GPT2-XL and LLaMA-3 (8B), EAC significantly preserves over 70\% of the general abilities across downstream tasks and retains the edited knowledge better.

\section{Background: How meaning is derived from genetic data}
\label{section:genetic_data}
In this section, we provide a brief overview of genetic data and genetic inference. An individual's genome is over 3 billion DNA bases\footnote{Units of DNA} long and derived equally from both genetic parents. Genetic variants—units of DNA that differ between individuals—are scattered throughout the genome and contain information about genetically influenced traits (from physical traits to behavioral traits). To predict traits or disease risk, geneticists analyze an individual's unique set of genetic variants. However, this remains a challenging task: while diseases like cystic fibrosis or Huntington's disease are determined by a single genetic variant, most traits are influenced by many variants across the genome as well as non-genetic factors (\emph{complex traits}).

\subsection{Your genetic data is shaped by your ancestors and shared with your relatives.}

% CHAT GPTd/edited FOR CLARITY
The inheritance of DNA—where each parent passes down a random half of their genome to their offspring—results in relatives sharing portions of their DNA, with closer relatives sharing a larger proportion. This shared genetic material means that decisions made about any individual's genome inherently extend to their genetic relatives. This relational nature of genetic data—the \emph{genetic dragnet}—has been noted to significantly complicate discussions about privacy \cite{costello_genetic_2022}. A powerful example of the genetic dragnet came in 2018, when law enforcement leveraged it to identify the Golden State Killer. Authorities used DNA left at a crime scene to identify distant relatives through DNA matching on a public website, GEDMatch. By combining these matches with publicly available genealogical records, they identified the suspect despite the closest identifiable relatives sharing only a great-great-great-great-grandfather \cite{kaiser_we_2018, zabel_killer_2019}. At the same time, an individual's genome is unique to them: even the genomes of identical twins contain differences \cite{ormond_whole_2024}. While a single genome in isolation may not always be immediately identifiable the relational nature of genetic data can enable re-identification. Researchers have argued that genomic data should be treated as "always identifiable in principle" due to this inherent interconnectedness \cite{shabani_reidentifiability_2019, bonomi_privacy_2020}.

\subsection{Geneticists use statistical models for trait prediction.}
Inference using genetic data requires some knowledge of a variant's effect on a trait or disease. This can sometimes be tested in animals by controlled experiments that manipulate the animal's genome. However, this approach is infeasible and unethical to pursue in humans, and so researchers instead rely on large genetic datasets to ask "Do individuals who have the variant \textit{tend} to have the disease (or trait of interest) compared to those who do not have the variant?" To answer this question, researchers typically conduct genome-wide association studies (GWAS), which identify genetic variants that are associated with a trait and quantify the \emph{effect} of each variant on the trait. The learned effect sizes of variants (also called \textit{weights}) are directly interpretable\footnote{For example, in a GWAS for height, an effect size of 0.01 would mean that the variant increases height by an average of 0.01 cm.} and can be used for trait prediction. The polygenic score, considered the gold standard for trait prediction, is calculated as a weighted sum of genetic variants, with each variant scaled by its corresponding GWAS effect size. See Section \ref{section:challenges} for a discussion of frontier models for genetic data \cite{fritzsche_ethical_2023}.


\subsection{Insight from genetic data is a function of the cohort.}
The reference genetic dataset used for a GWAS is analogous to the training dataset in machine learning. Reference datasets contain genetic data and their ``ground truth'' annotations of trait values, whether it be ancestry, disease status, height, etc. Non-genetic factors of a cohort also influence a GWAS. Confounders like environmental exposures and social determinants of
health can introduce bias in a genetic study. For example, the average participant in the UK Biobank \footnote{The UK Biobank is a leading resource which contains genetic and medical data from 500,000 UK residents aged 40-69 at the time of recruitment.} tends to be more educated and healthier than the average Briton \cite{fry_comparison_2017}. The choice of the \emph{cohort}—that is \emph{who} is in the dataset and how their data is labeled\footnote{`True'' ancestry has been arbitrarily (and implicitly) set by the field to correspond to: where your ancestors lived 600 years ago \cite{coop_genetic_2023}. Individuals in the same ancestry group are distantly related in the same timescale (i.e., likely many ancestors in common dating back approximately 600 years ago). }—is thus critical to a GWAS outcome. Prioritizing "cohort matching" in a genetic study ensures that the test sample closely aligns with the reference cohort in key factors such as ancestry, age, geography, and social or environmental influences. This alignment minimizes confounding and enables a more accurate interpretation of genetic effects on traits.\footnote{For example, \cite{mostafavi_variable_2020} find that genetic data prediction accuracy varies within an ancestrally-homogeneous white British cohort, e.g., weights from a BMI GWAS conducted in a younger cohort have higher prediction accuracy via a PGS in younger individuals than older individuals.}


\section{Background: Current Protections}
\label{section:protections}


\begin{figure}[H]
    \centering
    \includegraphics[width=0.6\textwidth]{figures/final_four_pillars_v6.png}
    \caption{\textbf{Four Pillars of genetic data collection and their regulatory considerations.} Data collectors within each pillar operate with different collection strategies and intent to store, use, and transfer the data. However, there is only a patchwork of federal and state-level legislation that governs each Pillar. The resulting regulatory gaps lead to \textit{leaky protections} through which genetic data—or knowledge derived from genetic data—can be used in other pillars with little to no oversight. Real-world examples\protect\footnotemark{} of leakage are indicated by red arrows.}
    \label{fig:4_pillars}
\end{figure}
\footnotetext{Orchid Health is a private company that uses publicly available GWAS weights to compute polygenic risk scores (PGS) of embryos. Nucleus Genomics is also a company that similarly uses whole genome sequencing to compute PGS to predict health and behavioral traits.}




In this section, we outline the foundation of today's genetic data collectors and the current regulatory and legislative environments that they are subject to. To better understand the vast landscape over which genetic data is being collected, we build upon the Four Pillars \cite{wan_sociotechnical_2022} of genetic data collectors. We categorize these data collectors by their motivations for collecting genetic data and their relationships with the individuals from whom the data is collected. We define the updated Four Pillars as the following:
\begin{itemize}
    \item \textbf{Forensics, Tracking, Surveillance:} Entities focused on monitoring, identifying, and analyzing genetic data to support public health outcomes, security, and law enforcement objectives.
    \item \textbf{Direct to Consumer (DTC):} Entities developing products to provide consumers with direct access to genetic insights.
    \item \textbf{Clinical:} Entities requiring medical professionals' oversight to conduct tests, often enabling insurance coverage and generating clinically actionable insights.
    \item \textbf{Research:} Entities engaging in research and development (R\&D) and basic research across private, public, and academic sectors. This includes public consortia, biobanks, and collaborations with DTC entities.
\end{itemize}

The ``genetic data ecosystem'' encompasses not only the entities within the Four Pillars but also the broader network of legislatures, regulators, and individuals involved in or affected by genetic data collection and use. By outlining this ecosystem, we aim to shed light on the regulatory gaps that enable genetic data leakage across pillars. We provide real-world examples of this leakage in Figure \ref{fig:4_pillars}.

%after an examination of federal legislation, policy, and regulatory bodies, that attempt to provide baseline protections across the United States. We find that federal data protections for individuals are limited, and that a patchwork of state-level legislation creates regulatory gaps, allowing private companies handling genetic data to discriminate based on jurisdiction and geography. We thus determine that today's state of protections is \emph{leaky}, and finally identify areas in urgent need of reform.

%%%%%%%

\subsection{Federal protections}

\subsubsection{Genetic Information Non-Discrimination Act (GINA)}
GINA was enacted in 2008 and prohibits genetic discrimination by employers or health insurers. Specifically, GINA bars employers from making hiring, firing, or promotion decisions based on genetic information (and information disclosed about relatives) and bans health insurance providers from denying coverage or charging higher premiums based on genetic information including genetic predisposition for disease \cite{gina_2008}. GINA does not apply to life, disability, or long-term care insurance \cite{green_strategic_2020} or any other venues outside of employment and health insurance (e.g., education). The US Equal Employment Opportunity Commission (EEOC) regulates workplace discrimination for employees and thus, is responsible for enforcing GINA in the workplace. Enforcement for health insurers falls under the purview of several federal agencies, including the Department of Health and Human Services (HHS); Department of Labor; Centers for Medicare and Medicaid Services; and the Department of the Treasury \cite{protections_ohrp_federal_2009}. 


\subsubsection{Health Information Portability and Accountability Act (HIPAA)} HIPAA, signed into law in 1996, grants security and privacy protections for patient health information (PHI). 
%and Genetic Information Nondiscrimination Act (GINA). HIPAA was passed by Congress and signed into law in 1996. 
HIPAA applies to covered entities that maintain this information, such as healthcare and health insurance companies, but does not apply to many other entities that might retain health information, including search engines, medical information sites, or dating sites \cite{citron_new_2020}. Importantly, HIPAA does not apply to two of the largest DTC genetic testing companies 23andMe and AncestryDNA \cite{sklar_be_2020}. HIPAA does not cover deidentified data: PHI data with unique identifying information (names, address, phone numbers, dates, etc.) removed do not fall under HIPAA. Research consortia, biobanks, and research collaborations with DTC and clinical entities primarily use genetic data that is considered deidentified, and so HIPAA generally does not apply. However, the relational aspect of genetic data has motivated calls to ``consider genomic data as, in principle, always identifiable'' \cite{bonomi_privacy_2020}. The inability to completely strip data of \emph{all} identifiable features introduces critical vulnerabilities for genome owners and exacerbates the risk of data leakage across the Four Pillars.


\subsubsection{Food and Drug Administration (FDA)}
The FDA ``is responsible for protecting the public health by ensuring the safety, efficacy, and security of human and veterinary drugs, biological products, and medical devices'' \cite{commissioner_what_2023}. In the context of DTC genetic testing companies, the FDA only regulates well-defined high risk medical tests with clinical actions, such as 23andMe's health tests for diseases like Parkinson's, breast cancer, and late-onset Alzheimer's. If the FDA approves a specific test, Centers for Medicare and Medicaid Services then choose to approve the test for insurance coverage \cite{daval_authority_2023}. Most genetic tests are not covered by Medicare, particularly DTC tests. 

\subsubsection{Federal Trade Commission (FTC)}
The FTC enforces federal laws that protect consumers from ``fraud, deception and unfair business practices'' \cite{federal_trade_commission_coop_nodate}. The FTC has warned DTC genetic testing companies that the results they return to customers must be backed by ``reliable science'', and also warns companies against making exaggerated claims about the use of AI in their products \cite{jillson_dna_2024}. Additionally, the FTC remains vigilant of deceptive practices regarding privacy policy changes and ``dark patterns'' designed to coerce consumers into consenting to data sharing\footnote{In 2023, the Commission fined 1Health.io/Vitagene \$75,000, claiming they ``left sensitive genetic and health data unsecured, deceived consumers about their ability to get their data deletes, and changed its privacy policy retroactively without adequately notifying and obtaining consent'' \cite{ftc_1health_2023}.}. The power and broad scope of the FTC—and their stated intentions of cracking down on DTC genetic testing companies \cite{jillson_dna_2024}—make it a major player in the regulatory space, and it will likely grow in importance if the federal government prioritizes genetic data privacy. 

\subsubsection{The Common Rule}
This is a federal policy that protects human subjects in research settings \cite{sciences_federal_2014}. It was first codified in 1981 by the Department of Health, Education, and Welfare (now HHS) but adopted more widely by other federal agencies in 1991. The Common Rule institutionalizes protections for vulnerable participants, protections overseen by Institutional Review Boards (IRBs); IRBs are groups ``formally designated to review and monitor biomedical research involving human subjects'' \cite{research_institutional_2024}. Revisions enacted in 2017 particularly focus on informed consent for participants, but do not cover deidentified data \cite{menikoff_common_2017, protections_ohrp_federal_2009}. IRBs ensure that consistent protection of human subjects are maintained and are necessary for research to begin and proceed if involving human participants. The maintenance of the Common Rule and HIPAA, in the case of medical data, is performed by IRBs.
\subsection{State protections}

\begin{figure}[ht]
    \centering
    \includegraphics[width=\textwidth]{figures/legislation_map.png}
    \caption{\textbf{Beyond HIPAA and GINA, state-level genetic discrimination and privacy legislation varies widely in the United States}. \textbf{Left:} States with genetic anti-discrimination legislation, distinguishing between states with both health insurance and employment protections (red) and those with health insurance protections only (orange). \textbf{Right}: states with general privacy laws that include genetic data (blue) and genetic privacy-specific laws (green).}
    \label{fig:genetic_legislation}
\end{figure}

State laws offer far more protections than their federal counterparts, but a patchwork of these laws means a patchwork in protections, as shown in Figure \ref{fig:genetic_legislation}. Here, we focus on two categories of laws: genetic anti-discrimination laws and data privacy laws.

\subsubsection{Genetic Anti-Discrimination Laws}
Most states have passed some version of a genetic anti-discrimination law (initially prompted by the passage of GINA) although the sector, scope, and strength of the laws differs between states. 
%Genetic anti-discrimination laws were, generally, passed around the time of GINA and typically offer similar protections (i.e., outlawing genetic discrimination in employment or health insurance). 
Some states offer fewer protections than GINA: Georgia, for example, has a statute outlawing genetic discrimination by health insurers, but not employers. Other states offer \textit{more} protections than GINA. California's CalGINA \cite{california_senate_bill_559_2011} outlaws genetic discrimination in several sectors: employment, health insurance, housing, and by any state agency or entity receiving state funding (such as emergency medical services). Florida includes life, long-term care, and disability insurers \cite{florida_house_bill_1189_2020}. Many states (including Massachusetts and New Mexico) that include life insurance in their non-discrimination laws exempt companies if genetic data reliably, and based on ``sound actuarial principles'' \cite{massachusetts_genetic_law_2025}, gives information relating to mortality or morbidity (see Section \ref{case:insurance} for a detailed case study).

\subsubsection{Genetic Data Privacy Laws}

Many states have genetic data privacy laws, either as a standalone law or as part of a broader data privacy law. These laws\footnote{Some of these laws have also led to a change in company services in that state. For example, the passing of Illinois' BIPA (Biometric Information Protection Act) resulted in Amazon and Google removing certain services of Alexa and Google Photos respectively after lawsuits which claimed that the company violated BIPA.} were generally passed after GINA with the rise of DTC genetic testing companies. Standalone genetic data privacy laws tend to be written specifically to cover DTC genetic testing companies and are often extensions of genetic non-discrimination laws passed around the time of GINA.

Both data privacy and genetic privacy laws have similar provisions \cite{nhgri_genome_database_2025}, including a customer right to request deletion of genetic data and samples and the necessity of consent for data transfers. Some states allow consumers to bring lawsuits for violations of the privacy law (a private right of action). In general though, privacy laws do not apply to deidentified genetic data (as per HIPAA). An important contrast to the US federal and state bills on privacy is the EU General Data Protection Regulation (GDPR). GDPR not only regulates all personal data within the EU but also the transfer of data outside of the EU, data minimization, and compliance. Genetic data is labeled as ``special category data'' in GDPR as it has many identifiers associated with it. 

% See Appendix \ref{appendix:dtc_laws} for more discussion on provisions for DTC genetic data use.


% \sverror{and then continue here} \rverror{We moved the previous paragraph to the appendix.}
% Some of these laws have also led to a change in company services in that state. For example, the passing of Illinois' BIPA (Biometric Information Protection Act) resulted in Amazon and Google removing certain services of Alexa and Google Photos respectively after lawsuits which claimed that the company violated BIPA.  


\section{Risk Assessment Framework}
\label{sec:risks}

Our goal in this section is to develop a risk assessment framework for genetic data governance.
%Having established the background of genetic data, the genetic data collectors, and current protections—we next formalize the \textit{risks} that the existing system poses to the individual. 
Similar to the aims of the Blueprint for the AI Bill of Rights \cite{park2023ai}, we motivate our framework with three central questions: (1) What values (i.e., moral principles and civil liberties) should be preserved? (2) What are the vulnerabilities in the current system that can compromise these values? (3) What are the specific harms that can result from the vulnerabilities? 


\subsection{What values should be preserved?}

We seek to identify values that are future-proof and can be encoded in a system of genetic data collection. By examining the Four Pillars, six key values emerge:

\begin{enumerate}
    \item \textbf{Right to action}: The individual, and the individual only, has the choice to submit (and the freedom to not submit) their genome\footnote{We use ``genome'' to refer to a physical DNA samples of an individual and any sequencing/genotyping data.}.
    %they shall not be required to do so by any entity.
    \item \textbf{Ownership of the genome}: The individual owns their genome and therefore controls the usage of their genome, including \textit{who} has access to it and for what reason.
    \item \textbf{Right to privacy}: The individual has a right to privacy with regards to their genome and inferences made from their genome.
    \item \textbf{Right to knowledge}: The individual has a right to know or \textit{not} know about inferences made from their genome. 
    %An important implication of the genetic dragnet is that individuals may not have control over others (e.g., close relatives) inadvertently revealing insights about their genome. This is particularly important for individuals who do \textit{not} participate in genetic testing (or who selectively participate in certain tests, but not others). \frerror{Is this too complicated by bringing up the relational aspect?} \sverror{I think it is, especially in comparison with the previous shorter points}
    \item \textbf{Protecting opportunities for advancement}: Genetic data should not be used to deprive the individual of opportunities in any domain, including (but not limited to) education, access to financial tools, health insurance, housing, social services, and reproductive choices. 
    %Here, any discriminatory behavior that deprives opportunity should be protected against, recognizing that reproductive autonomy is fundamental to an individual's ability to make life choices and advance according to their own goals. \sverror{I'm not sure what this second sentence is adding}
    \item \textbf{Benefits of inclusion}: The Belmont principle states ``those who bear the burdens of research (i.e., those who are exposed to the discomforts, inconveniences, and risks) should receive the benefits in equal measure to the burdens''\cite{belmont_report}. We believe that this applies to individuals and their genetic data.\footnote{Variant Bio is a biotech company that collects genetic data from Indigenous groups from around the world and participates in revenue sharing with the communities they collect data from.}
\end{enumerate}

\subsection{What are the vulnerabilities in the current system that can compromise these values?}

The current genetic data ecosystem has many vulnerabilities that compromise the values listed above. These include:
\begin{enumerate}
    \item \textbf{Unsettled science}: The role of genetics in shaping complex traits is poorly understood. As a consequence, the deterministic nature of genetics may be overstated, particularly for behavioral and cognitive traits. 
    \item \textbf{Rapid evolution of genetic data/methods}: As genetic data collection, sequencing, and methodologies quickly evolve, legal protections fall behind. For example, education discrimination is not included in GINA, which was passed five years before the first large GWAS on educational attainment \cite{rietveld_gwas_2013}.
    %Many individuals can be negatively affected because of lack of scientific clarity (e.g. predicting IQ through genetic tests is often confounded by lack of causal confidence and environmental socioeconomic factors \cite{selzam_s_comparing_2019}). 
    \item \textbf{Guilt by association}: DNA databases used for criminal investigations can impact not only the individuals whose DNA is stored, but also their biological relatives, as genetic information is shared among family members. 
  %  \item \textbf{Third-party data collectors}: \rverror{This one doesn't seem clear cut; and overlaps with many others -- do we lose anything but cutting it out altogether? or how do we make it distinct?} there are companies that allow users to upload their genetic data (that they have downloaded from 23andMe or AncestryDNA) for free and in exchange users receive additional relative matches (e.g., GEDMatch) or disease risk reports (e.g., Promethease). The barrier to entry in order to start one of these companies is low—e.g., Promethease reports polygenic risk scores using publicly available GWAS weights—which may result in lower trust. GEDMatch\footnote{GEDMatch also had an option allowing users to `opt-in' to sharing data with law enforcement, but in 2019 a Florida judge overrode user preferences and granted a detective full access to the database \cite{hill_your_2019}.}, which collected genetic data from 1.2M individuals, was acquired by genetic forensics company Verogen, which was then acquired by the German company Qiagen.
 
    \item \textbf{Geographical legislative patchwork}: Genetic anti-discrimination and privacy legislation widely differs between states. Legal policy in each state for data removal, third-party sharing, and many other aspects of genetic data are unclear. Individuals who move between states or share data with entities in other states can be affected by this dependency on state specific policy. Aside from patchwork policy, many state policies do not sufficiently protect genetic data specifically. 
%    \item \textbf{Security practices}:
%    \frerror{Is there a term to refer to individual versus entity security practices?}
 %   Individual and entity security practices can control whether malicious actors can gain access to their data (e.g. 23andMe hack where individual security passwords were used to access related profiles and their identifying data; see Section \ref{case:dtc}). 
\end{enumerate}


\subsection{What are the harms that arise as a result?}
Lastly, we will outline several consequences that could arise from violations of our key values through the above vulnerabilities. These downstream effects, which we refer to as harms, can affect an individual and their immediate genetic relatives and 
%and their personal data footprint,
can also have a broader effect on communities and the population at large.
%the population level scale \cwerror{Don't like this phrasing}. 

\begin{enumerate}
    \item \textbf{Leakage to the family:} The genetic dragnet means that any conclusion about an individual's genetic trait can be linked back to their relatives, even if no action is taken by their relatives in either (1) submitting data or (2) opting in to receive information. Any derived secondhand knowledge can immediately impact insurance (health, life, etc.) and medical treatment plans, as well as compromise identity.
    %, even though the original determination was not made on genetic relatives. The relational nature of genetic data and its leaky quality thus compromise both the privacy and personal liberties of family members. Not only can the relational nature of genetics affect immediate known relatives, but decisions also impact future children, disease carriers, and preventative care. 

    \item \textbf{Loss of anonymity: } It is straightforward to identify some traits about an individual from their genetic data, and subsequently from their genetic relatives. Common examples include: race, gender, ethnicity, and markers for certain diseases. Leaky genetic data interfaces can thus compromise information about an individual, their genetic relatives, and unborn children that the individual (and implicated relatives) may wish to maintain as private. 
    %Exposure can lead to future vulnerability such as affecting opportunities of advancement, discrimination in availing certain insurance policies or services, and extortion or targeted attacks -- all of which compromising a critical value of privacy and personal liberties. 
    As the refinement of reference datasets continues and data collection grows\footnote{It has been estimated that, in a sample of three million European-descent Americans, 99\% of individuals would have a least one third cousin in the dataset \cite{erlich_identity_2018}.}, anonymity becomes even more elusive.
    
    \item \textbf{Loss of Data Control: } 
    % Private institutions that sell genetic tests own the end-to-end process of how individuals' data is collected, stored, and monetized. Data transfer policies remain a black box as it is unclear which deals an individual's data may be included in, how the data is monetized, and which parties benefit from these trades. Even if users opt-out of participation, ungoverned data flows mean that their genetic data footprint will exist in perpetuity. The lack of transparency in the event of bankruptcy or acquisition has the same implication: In 2024, Tempus AI acquired Ambry Genetics, which sequences 400,000 patients annually, with the stated intent to "leverage this data and augment Tempus' current data offering" \cite{tempus2025}. Consequently, any decision of an individual to submit data for a genetic test implicitly includes loss of complete ownership over their genetic data, and any leaky interfaces can compromise their privacy without the appropriate safeguards.\sverror{I don't understand how the tempus example is relevant}
    Private institutions offering genetic tests control the entire lifecycle of individuals' data from collection and storage to monetization. This control creates significant transparency issues, as it is often unclear how data is monetized, which parties benefit, or what agreements include an individual’s data. Users may opt out of participation, but unregulated data flows ensure that their genetic footprint persists indefinitely. This lack of transparency is particularly concerning in cases of bankruptcy or acquisition. For example, in 2024, Tempus AI acquired Ambry Genetics, a company that sequences genetic data for 400,000 patients annually \cite{tempus2025}. Tempus stated its intent to "leverage this data and augment Tempus' current data offering," raising questions about how patient data is repurposed and monetized in such deals. Ultimately, an individual’s decision to submit genetic data for testing often entails relinquishing full ownership of their genetic information. Without robust safeguards, privacy risks emerge from opaque data practices and leaky interfaces, leaving individuals vulnerable to misuse or unauthorized access to their genetic data. 
    
    \item \textbf{Misinformed Actions: } The interpretation of genetic test results and the subsequent actions taken by individuals are profoundly personal, often influencing lifestyle changes, significant financial decisions, and critical medical choices. For instance, individuals with BRCA1 gene mutations indicating an elevated breast cancer risk may opt for preventive interventions. These decisions are particularly sensitive as the interpretation of genetic data evolves alongside advancements in scientific methods. However, the relational nature of genetic data complicates individual autonomy and access to information. It is possible for someone to receive conclusions about themselves indirectly through the test results of genetic relatives. This dynamic, combined with the widespread availability of unregulated genetic tests and the evolving nature of genetic science, increases the potential for harm if individuals act on information that is incomplete, inaccurate, or probabilistic rather than deterministic.
    % The interpretation of genetic test results and the actions taken by users are deeply personal. \textcolor{blue}{These results can drive significant lifestyle changes, major financial decisions, and critical medical choices, such as pursuing aggressive interventions when BRCA1 gene mutations indicate elevated breast cancer risk} \sverror{someone make this sound better :)} \cwerror{better?}. Such decisions are particularly crucial, given that our understanding and interpretation of genetic test results continue to evolve alongside advancements in scientific inference methods. However, the inherently relational nature of genetic data poses challenges to an individual’s autonomy and access to information, as it is possible for someone to receive secondhand conclusions about themselves based on tests results of their genetic relatives. \textcolor{blue}{This characteristic, coupled with evolving science and mass availability of unregulated genetic tests, creates the risk of significant harm if individuals act on incorrect, incomplete, or non-deterministic information.} 
  


    \item \textbf{Financial impact}: Individuals may face financial repercussions if insurance or legal policies fail to provide adequate protection or if coverage is denied. Beyond medical procedures, financial impacts can also arise on a case-by-case basis, such as in situations where genetic data is held for ransom, legal defense is required, or individuals are forced to purchase direct-to-consumer tests (e.g., in order to purchase a life insurance policy).
\end{enumerate}

\section{Case Studies}

\label{sec:studies}

We apply our risk assessment framework to four case studies which highlight regulatory gaps across the Four Pillars. These case studies, ranging from past events to speculative scenarios, all involve privacy loss or potential discrimination. We choose case studies that illustrate genetic data leakage in both, a pillar-to-pillar and one-to-many pillar settings.

% While some align with specific Pillars, others span multiple Pillars due to the leaky protections in which genetic data, initially collected within one Pillar, can be used across others.

\subsection{Genetics and Education}
\label{case:education}

Polygenic scores (PGS) are popular in the social/behavioral sciences for predicting social outcome traits, such as educational attainment (EA; number of schooling years completed by an adult). We will use EA as the example throughout this case study, but note that there is also interest in predicting standardized testing scores, performance in mathematics, and other traits with substantial environmental influences. 


A common metric for assessing PGS accuracy is the percentage of trait variance it explains: higher percentages indicate better predictive performance. A recent 23andMe EA PGS, based on data from over 3 million customers of European descent, explains 12-16\% of the variance \footnote{As a useful comparison, mother's education explains 15\% of the variance in EA \cite{lee_gene_2018}} in educational attainment (EA) \cite{okbay_polygenic_2022}. This effectively means that 50-70\% of individuals with PGS scores in the top 10\% for EA are predicted to graduate college. However, the PGS accuracy significantly decreases when applied to African American customers. This is an example of the commonly observed ``portability problem'' \cite{martin_human_2017}, where a PGS derived from GWAS in one population predicts poorly in another due to confounding\footnote{Confounding in GWAS can be genetic (non-causal variants correlated with causal ones) or environmental (non-causal variants correlated with causal environmental factors), leading to potential statistical artifacts in effect sizes.}.


There have been several calls to use EA PGS to inform education policy. Harden et al. propose the use of math-performance PGS to identify ``leaks'' in the education system: for example, by identifying high math PGS students who perform poorly, they claim educators could pinpoint \textit{why} and \textit{how} students are failing to reach their potential\footnote{An example they use: 31\% of high PGS students in good schools take calculus, compared to 24\% of the same-scoring students in poor-performing schools.} \cite{harden_genetic_2020}. Plomin \& von Stumm take it further: they use the term ``precision education'' (akin to precision medicine) to propose a tailor-made, individualized education that is genetics-informed \cite{r_new_2018}. Statements like this, combined with statements such as ``students with higher polygenic scores for years of education have, on average, higher cognitive ability, better grades and come from families with higher SES [socioeconomic status]'' \cite{smith-woolley_differences_2018} are cause for concern because they invoke a sense of genetic determinism. However, other predictors (parents' educational status, socioeconomic status) explain similar amounts of variance in EA \cite{lee_gene_2018, morris_can_2020} and—unlike DNA—are mutable through social policy changes. 

Several of our values would be violated if children were required to submit their DNA (Right to Action) or educational opportunities were denied to children based on their genetic potential (Opportunities for Advancement). Through the vulnerabilities of unsettled science and the rapid evolution of genetic methods, harms such as leakage to the family can occur and affect not only children but their families and future.

% \sverror{a bigger issue I have with this case is that we are asserting that EA PGS is bad. but are not explaining why. Is it because the predictive power is low? or that the science is dicey? we need to explain that} \cwerror{Added the sentence at the end of the 3rd paragraph—is this better?}


\subsection{DTC Genetic Data Brokerage and Leakage}
\label{case:dtc}

Several FDA-approved 23andMe tests\footnote{23andMe tests with FDA 510(k) clearance are verified as substantially equivalent to existing devices or methods. Only health-related tests, such as those identifying genetic risks, carrier status, or drug responses, qualify for this review.} provide customers with results containing sensitive information, including carrier screening reports, genetic health risk assessments, and BRCA1/BRCA2 variant analyses. In addition, they offer genealogy (to identify relatives) and ancestry services. In an attempt to protect customer privacy, 23andMe offers an opt-in/opt-out policy when it comes to sharing their samples in studies with third-parties.


% FDA approval for certain tests\footnote{23andMe tests with FDA 510(k) clearance are verified as substantially equivalent to existing devices or methods. Only health-related tests, such as those identifying genetic risks, carrier status, or drug responses, qualify for this review.}, including carrier screening reports, genetic health risk reports, and BRCA1/BRCA2 variant tests. In addition, they offer genealogy (to identify relatives) and ancestry services. 23andMe consent forms give the user a choice of opting in or out of sharing their samples in studies with third parties.

In 2023, hackers accessed 14,000+ 23andMe user accounts by using recycled login information (credential stuffing). Due to the relatives linkage feature in genetic data reports, the hackers were able to then access information about an additional 5.5M users \cite{23andme_addressing_2023}; there were also reports that the hacker targeted Ashkenazi Jewish and Chinese ancestry profiles \cite{carballo_23andme_2024}. Among the many values violated, the Right of Privacy and Ownership of the Genome are particularly relevant to this case as 23andMe did not immediately notify compromised users or the public. Several class action lawsuits were filed (e.g., \cite{santana_23andme_2023}), with plaintiffs complaining that 23andMe failed to adequately protect their sensitive information. 23andMe responded by enforcing multifactor authentication (MFA) of user accounts, which was previously optional \cite{whittaker_23andme_2023}, and a serious technological vulnerability. A key gap that allowed for this issue to occur is the lack of consensus around security responsibility, which resulted in the disproportionate and harmful targeting of underrepresented groups. The 23andMe breach highlights the consequences of the relational nature of genetic data, where an individual's information can be compromised through poor public protections, even if they personally take precautions. As one Reddit user quipped ``Your genetic data is only as secure as your relatives' passwords.'' 
% \footnote{\url{https://www.reddit.com/r/23andme/comments/1ayw9y3/what_specific_privacy_concerns_do_you_have_about/}}."

\subsection{Genetic Data Collection of Detained Noncitizens}
\label{case:cpt}

A recent Georgetown University report from the Center of Privacy and Technology (CPT) examines the U.S. federal government's practice of collecting DNA from detained migrants \cite{glaberson_raiding_2024}. This practice, which began with the 2005 DNA Fingerprint Act and expanded significantly in 2020—a mandate that the Department of Homeland Security must collect DNA from \textit{all} detainees, even briefly detained\footnote{States have varying rules about DNA collection from suspects and detainees: 28 states allow DNA to be taken from someone \textit{before} they are convicted of any crime and most states restrict DNA collection to cases involving felony charges or convictions, rather than allowing collection for any arrest or detention \cite{samuels_collecting_dna_2012}.}—has led to a dramatic increase in detainee representation in CODIS (the federal DNA database), rising from 0.21\% in 2019 to 9.21\% in 2023. By 2020, approximately 25,000 noncitizens were added to the database under the ``detainee'' classification.

The CPT report highlights several concerning aspects of this practice, particularly its disproportionate impact on migrants of color and issues of consent. In our Risk Assessment Framework, this immediately violates the Right to Action and Right to Knowledge. Many migrants undergo DNA collection without understanding its implications, sometimes believing it to be a COVID-19 test or submitting under threat of criminal prosecution. The report argues that this practice violates the Fourth Amendment in collecting DNA from detainees without probable cause. Additionally, even though CODIS was designed to be privacy-protective by collecting only 20 markers—a tiny slice of the genome believed to be medically neutral—this limited data can still identify relatives through partial matching and, as an analysis 25 years later would show, are slightly informative of disease risk \cite{banuelos_associations_2022}. The expansive collection of genetic data has far-reaching implications through the guilt-by-association vulnerability. Since CODIS records are difficult to expunge, they can restrict Opportunities for Advancement not only for detainees but also for their relatives and descendants.

\subsection{Underwriting Life Insurance with AI and Genetic Data}
\label{case:insurance}

Life insurance underwriting is the process whereby an insurance company uses personal and health information to assess the risk of insuring an applicant. The relationship between applicants and insurance companies is already fraught: \cite{devnos_genomics_2016} find that patients are more likely to share their genetic info with Google than insurance companies. The future of life insurance underwriting is expected to become increasingly computational and automated via the usage of AI and the collection of more personalized, individual genetic data \cite{filabi_ai-enabled_2021, balasubramanian_insurance_nodate, rothstein_time_2018, koleva-kolarova_financing_2022}. While medical records and demographic factors can be used for mortality analysis, the inclusion of genetic factors can mean that risk prediction can be performed much earlier in an applicant's life without necessarily the same amount of records as an older individual, emphasizing the risk of genetic discrimination \cite{karlsson_linner_genetic_2022} based on the \textit{potential} to have risk factors (e.g., a genetic risk for high blood pressure versus clinically-measured high blood pressure). Life insurance is not covered by GINA and life insurance companies can access medical records (which may include genetic test results) as part of an application. Bills have been introduced in several states that would restrict the use of genetic information in underwriting, but these efforts largely failed: As of 2022, out of thirty-seven proposed bills across all states, three were introduced, eight were signed by the governor, demonstrating the geographical dependency of policy-based protection \cite{vermont_legislature_httpslegislaturevermontgovdocuments2022workgroupshouse20commercegenetic20testingwitness20documentswjill20rickardgenetic20testing20legislation20across20states4-20-2022pdf_nodate}.

Life insurance companies may be interested in using polygenic scores (PGS) to assess an individual's risk for various diseases. Indeed, in early 2024, U.S.-based life insurance company MassMutual and U.K.-based Genomics plc announced a partnership, offering free genetic testing to MassMutual's life insurance customers \cite{massmutual_genomics_2024}. However, the press release stressed that MassMutual would \textit{not} receive individual results and that \textit{current} premiums/policies would be unaffected \cite{massmutual_genomics_2024}. This example particularly highlights the leaky interface between pillars, where the Right of Privacy and Right of Action are called into question. AI insurance underwriting algorithms already suffer from racial biases \cite{lee_ai_2022}. That PGS suffer from the portability problem (itself caused by systematic biases in training datasets) means that these biases could be perpetuated in genetics-informed life insurance underwriting. Additionally, given the relational aspect of genetic data, genetic-based underwriting could affect other biological relatives' applications, harming not only relatives but any others with the same genetic markers that are associated with mortality risks. The potential harms of using genetic data as a part of underwriting in life insurance are one such example of violations of genetic privacy with the rapid evolution and usages of genetic data with unsettled science. 

\section{Recommendations}
\label{section:recommendations}

We argue that for public infrastructure and private companies to productively and ethically make use of genetic data, several amendments need to be made to the current genetic data governance system. 
%We structure recommendations that rely on protecting the values outlined in our risk assessment framework. 
Our three recommendations address open privacy concerns, legislative scoping for policy changes, and best practices for bodies handling genetic data.

\subsection{Recommendation 1: Redefining Genetic Data}

\textbf{Issue}: Legal policy surrounding genetic privacy notably excludes deidentified or anonymized data from protection.

\textbf{Recommendation}: Given that we argue that genetic data is unique compared to any other identifying data, we suggest genetic data be defined using the following language: 

``Genetic data refers to any information relating to an individual’s genetic characteristics, including but not limited to DNA or RNA sequences, gene expression profiles, or any data derived from a biological sample—including that of a relative—regardless of format. \textbf{This data is always considered identifiable} by its nature—or personally identifiable information (PII)—as it pertains to unique biological attributes that can potentially be linked to a specific individual, their biological relatives, or identifiable group. For the purposes of this definition, any de-identified, pseudonymized, or anonymized genetic information shall be treated as genetic data, regardless of measures taken to mask individual identities, recognizing that the inherent characteristics of genetic information may enable re-identification through advanced technological or data cross-referencing methods.''

\subsection{Recommendation 2: Extending Protections for Genetic Discrimination}

\textbf{Issue}: GINA is a vital piece of federal legislation that protects against genetic discrimination in employment and health insurance domains. However, other domains in which there exists potential for genetic discrimination  are \textit{not} protected by GINA: other insurance domains (life, long-term care, disability), housing, and education. 

\textbf{Recommendation}: We recommend that additional federal laws should be enacted to extend GINA's coverage beyond employment and health insurance. The most comprehensive state genetic anti-discrimination law is California's CalGINA, which includes housing, mortgage brokerage, education, and more.  However, despite these extensions, CalGINA does not cover life, disability, or long-term care insurance. We suggest that extension bills for GINA should explicitly cover life, disability, long-term care, as well as education and any other opportunities for advancement to be protected against genetic discrimination. It should be written in such a way that prohibits \textit{any} barrier to opportunity, even those not anticipated at the time of writing. Additionally, legislation should be written to explicitly bar genetic risk for \emph{complex traits} known to have significant environmental influences (such as cardiovascular disease) from being considered a preexisting condition.


\subsection{Recommendation 3: A Genetic Data Regulation Framework}

\textbf{Issue}: Current regulations were designed to govern one application of genetic data—or Pillar—at a time. This has led to ``leaky protections'', where the use of genetic data in one Pillar can affect opportunities and decisions in other Pillars (e.g., clinical tests being used in life insurance).

\textbf{Recommendation}: To eliminate \emph{leaky protections}, we suggest a comprehensive and uniform regulatory framework that encompasses all usage domains whether genetic data is collected, analyzed, or used in any type of inference. Below, we build upon specific values inherent in privacy rights to ensure that responsibility is placed on the organizations that hold and analyze data, rather than the individuals from whom it was collected \cite{solove_limitations_2022}. 

\begin{enumerate}
    \item \textbf{Entity approval}: Entities—whether they be universities, health care providers, hospital systems, or corporations—that collect or house any type of genetic data (DNA, RNA, or even genetic test results) must have prior regulatory approval. Approval requires a clear commitment to the basic rights of the individuals whose genetic data the entity will collect or own, such as data privacy, security, the right to deletion, etc. 
    %Broadly requiring approval for any public or private entity to collect or house genetic data will ensure that genetic data is given the same protections across all domains. 
    
    \item \textbf{Test approval}: After \emph{entity approval}, we propose regulations for the inferential tests, specifically those that return results to either an individual or a medical practitioner for any actionable result. At a minimum, entities would be required to publicly release detailed white papers for each test that detail laboratory procedures, quality control steps, inferential models used, presentation of results, and any other necessary details that would allow one to recreate the analysis if given the same data. 
    %An open question here is whether to require further regulatory approval for each test and what that regulatory process would encompass. It is not the goal to stymy genetics research, but to ensure that the science underlying tests is sufficiently rigorous.
    
    \item \textbf{Powers given to the individual}
    \begin{itemize}
        \item Individuals should be able to request \textbf{data removal}. This would include specifically (1) the destruction of the physical sample, (2) deletion of data, (3) removal of any identifiers, and (4) that any removed data no longer influence the results of any downstream models. 
        %This last point is purposefully vague, but could include periodic re-training of models with the requested data deleted.
        
        \item \textbf{Third-party data transfers} should only occur between companies authorized to collect or own genetic data, with recipient companies subject to the same regulations. We recommend providing individuals with the option of blanket \textbf{opt-out} from data transfers and specific \textbf{opt-in} choices for each third party. Individuals must be notified of the transfer and given a reasonable time to opt-out. 
        %They can also request data deletion after the transfer, as the receiving entity is bound by the same regulations, including right to deletion.

        
        % should only occur between companies that are both authorized to collect or own genetic data. Additionally, recipient companies must be subject to the same regulations as the data collector. We recommend that any third-party data transfers for an active company include a \textbf{blanket opt-out} option for individuals to exclude themselves from brokered data, along with specific \textbf{opt-in} choices for each third party involved. Individuals must be preemptively notified of the transfer and given a reasonable time frame to opt out. Because the receiving entity is subject to the same rules and regulations that ensure the right to deletion at any time, individuals can still request their data be deleted after the transfer.









        \item \textbf{Research usage} is common among DTC companies, academia, and hospitals, where data provided per a test can be pooled together as a dataset for research (both internally and via external collaborations). We suggest an \textbf{opt-out} strategy here for individuals, with a specification on whether they consent to their data becoming publicly available in any form (e.g. public datasets, open sourced GWAS weights, trained models, etc).  
        
        \item \textbf{Incidental genetic discoveries} can be possible in research use cases where data for a particular test was used for other tests. In these cases, where knowledge of a particular trait can also be harmful and uninformative to the individual, we suggest an \textbf{opt-in} strategy for an individual to blanket choose if they wish to hear any secondary discoveries, or forward any secondary discoveries to their medical practitioner, to be enacted on at their discretion. 
        
    \end{itemize}

    
    \item \textbf{Bankruptcy and Acquisition}: As detailed by third-party data transfers, the owning of genetic data would also apply to companies who acquire any genetic data as part of assets through acquisitions or bankruptcy. Thus, all entities involved must already have approval to handle genetic data. In the rare event that there are no entities to handle the data, we suggest a similar protocol structure as nuclear waste \cite{doctorow_personal_data_2008}, where all physical sample, data, and models are subsequently destroyed and cannot be recovered.
\end{enumerate}

 % It is important to consider who will enforce such a regulatory framework, which could possibly be a new regulatory body or fall under the purview of pre-existing regulatory bodies such as the FDA, FTC, or HHS. We recommend that the regulation be implemented at the federal level to prevent \textit{leaky protections}. \sverror{this doesn't answer the question though of WHO would enforce it}
%, which can occur with state-based regulations, where individuals may lose or be uncertain about their protections depending on their state of residence.



\section{Conclusions and  Open Challenges}
\label{section:challenges}

In this paper we outline the genetic data ecosystem and today's state of public protections. We identify critical gaps in current regulatory federal and state-level frameworks that prevent the productive, ethical, and safe adoption of genetic data for broad societal applications. To address this, we propose a novel risk assessment framework that offers key public protections and preserve's individuals personal liberties, and finally make three concrete policy recommendations. 
% While our policy recommendations account for improved user protections, we identify three remaining open challenges in regulating genetic data use, that lie beyond the scope of our work: First, our recommendations only cover \textbf{voluntary} genetic data collections.  In the case of involuntary genetic data collection, we reference Georgetown’s Center of Privacy and Technology (CPT) report and their accompanying suggestions \cite{glaberson_raiding_2024}. Second, the \textbf{relational} aspect of genetic data considerably implicates any biological relatives of individuals in a dataset. A relational theory of privacy is a complex phenomenon that remains highly discussed in academic literature \cite{viljoen_relational_2024, costello_genetic_2022, zigomitros_survey_2020, reviglio_i_2020}.  As such, we envision a future regulatory framework that would encompass the relational aspect of genetic data, of which we highlight three possible considerations/applications:

% These questions remain largely unanswered and require contributions from legal scholars, bioethicists, moral philosophers, privacy experts, human rights advocates, and public policy researchers. We raise these questions for the purpose of thoroughness in our evaluation of the use of genetic data in today's ecosystem.

While our policy recommendations focus on enhancing user protections, we also identify three key challenges in regulating genetic data usage that fall outside the scope of our work and that would benefit from further research.
%We present these challenges to ensure a comprehensive evaluation of genetic data use in the current ecosystem.

\subsection{The relational theory of privacy.}

\subsubsection{Should genetic data submissions or inclusion into genetic datasets require consent from biological relatives?}  %The framework should address issues regarding competing privacy interests between relatives. For example, w
When an individual does a genetic test they are, indirectly, testing (part of) their relative's genome.  What (if any) consent or privacy measures should be taken to protect their biological relatives who may not consent to such a test? Additionally, what should happen upon death? Do children ``own'' their parents' genome (and what ownership would, say, a full sibling have?) and can they make decisions (e.g., to delete the data)?

\subsubsection{Who has ownership of children’s DNA\@?}
Do parents have a ``fiduciary duty'' to protect their child's genetic data and protect them from the system vulnerabilities that we have outlined? Part of this fiduciary duty would involve whether a genetic test should be taken in the first place, e.g., the difference between a medically-necessary test and a DTC genetic test\footnote{23andMe's policy for minors is as follows: individuals under 18 that give assent and have a parent/guardian's permission can submit their sample \cite{23andme_research_2024}.} for some insights into the child's ancestry and genealogical relatives. 
%Return of results to the child (e.g., by their parents) may come with harm. The child may not understand the statistical nuances of genetic tests for complex traits, resulting in a harmful deterministic view of their future health and social outcomes. 
The relational aspect of genetic data also comes into play here; for example, should consent be required from both parents to sequence their child's genome? See \cite{bala_who_2023} for a thorough discussion of child's genome ownership. 

%By sequencing the child's genome, half of the parents' genomes are also sequenced. Companies or clinical services that sequence children or embryos (e.g. Orchid genetics performs whole genome sequencing on embryos for disorders and predispositions \cite{OrchidHealth2025}) call into question who owns this DNA, especially after the sequenced individuals become adults.
    % move to open challenges; remove biosecure

\subsection{International genetic data transfers.}
 Our paper focuses on American regulatory and legislative bodies and American companies. In the DTC genetic testing space, individuals may be less likely to trust foreign companies: Nucleus Genomics\footnote{https://mynucleus.com} proudly states on their home page that customer genomes are sequenced in the U.S. and the data is never sent/stored overseas\footnote{23andMe also genotypes samples in the U.S. \cite{23andme_what_2024}, but this is not a selling point of theirs.}. Should there be restrictions on genetic data transfer between countries? For example, one hypothetical scenario might involve a political refugee who has medically-necessary genetic testing done within the U.S, but because of leaky data transfer rules, genetic information is transferred to their home country where it might be used to identify family members now at risk for persecution.


\subsection{The deployment of AI models for genetic data.}
\label{challenges:ai}

Our paper has largely shied away from discussions about genetic inference and AI for two reasons: (1) deployment of AI technologies for genetic inference is nascent and, most importantly, (2) risks and vulnerabilities exist even with current simple linear models for genetic inference. AI will, we believe, exacerbate the risks we have outlined as they will likely result in seemingly better trait prediction by modeling nonlinear interactions and taking advantage of gene-environment correlations\cite{yang_advancing_2024}. However, the uncertainties associated with the basic scientific inference (that we outline in Section~\ref{section:genetic_data}) will be compounded because of the lack of interpretability and complexity of AI models. 

%Large language models that incorporate genetic data and other data modalities have already demonstrated improved accuracy, e.g., Google's Med-Gemini-Polygenic \cite{yang_advancing_2024}. As we described in Section 2, even linear models suffer from interpretability, with statistical geneticists unable to disentangle genetic and environmental confounders, meaning there is not a clear understanding as to what is \textit{biological}; this task would be made more difficult if AI technologies were deployed.

%\section{Conclusion}
We reveal a tradeoff in robust watermarks: Improved redundancy of watermark information enhances robustness, but increased redundancy raises the risk of watermark leakage. We propose DAPAO attack, a framework that requires only one image for watermark extraction, effectively achieving both watermark removal and spoofing attacks against cutting-edge robust watermarking methods. Our attack reaches an average success rate of 87\% in detection evasion (about 60\% higher than existing evasion attacks) and an average success rate of 85\% in forgery (approximately 51\% higher than current forgery studies). 

% \newpage 
% \newpage
\appendix
\section{Appendix}

\subsection{Conversational agent prompts for generating stable diffusion prompts in art-making phase}

\textbf{Role:} You will be able to capture the essence of the sessions and drawings in the recordings based on the art therapy session recordings I have given you and summarize them into a short sentence that will be used to guide the PROMPT for the Stable Diffusion model.

\vspace{0.5em} % 添加一些垂直间距

\textbf{Example input:}

\begin{itemize}[leftmargin=*]
    \item \textbf{USER:} [user-drawn] I drew the ocean. [canvas content] There is nothing on the canvas right now.
    \item \textbf{ASSISTANT:} What kind of ocean is this?
    \item \textbf{USER:} [user-drawn] I drew grass. [canvas content] Now there is an ocean on the canvas.
    \item \textbf{ASSISTANT:} What kind of grass is this?
    \item \textbf{USER:} [user-drawn] I drew the sky. [canvas content] Now there is grass and ocean on the canvas.
    \item \textbf{ASSISTANT:} What kind of sky is this?
    \item \textbf{USER:} [user-drawn] I drew mountains. [canvas content] Now there is sky, grass, and ocean on the canvas.
    \item \textbf{ASSISTANT:} What kind of mountain is this?
    \item \textbf{USER:} [user-drawn] I drew clouds. [canvas content] Now there is sky, mountain, grass, and ocean on the canvas.
    \item \textbf{ASSISTANT:} What kind of cloud is this?
    \item \textbf{USER:} [user dialogue] Colorful clouds, emerald green mountains and grass, choppy ocean
\end{itemize}

\vspace{0.5em} % 添加一些垂直间距

\textbf{Task:}

\begin{enumerate}[label=\textbf{Step \arabic*:}]
    \item \textbf{[Step 0]} Read the given transcript of the art therapy session, focusing on the content of \texttt{user: [user drawing]} and \texttt{user: [user dialog]}: Go to \textbf{[Step 1]}.
    \item \textbf{[Step 1]} Based on the input, find the last entry of user's input with \texttt{[canvas content]}, find the keywords of the screen elements that the canvas now contains (in the example input above, it is: sky, grass, sea), separate the keywords of each element with a comma, and add them to the generated result. Examples: [keyword1], [keyword2], [keyword3], \dots, [keyword n].
    \item \textbf{[Step 2]} Find whether there are more specific descriptions of the keywords of the painting elements in \texttt{[Step 1]} in \texttt{[User Dialog]} according to the input. If not, this step ends into \textbf{[Step 3]}; if there are, combine these descriptions and the keywords corresponding to the descriptions into a new descriptive phrase, and replace the previous keywords with the new phrases. Examples: [description of keyword 1] [keyword 1], [keyword 2 description of keyword 2], [description of keyword 3], \dots. Based on the above example input, the output is: rough sea, lush grass, blue sky.
    \item \textbf{[Step 3]} Based on the input, find out if there is a description of the painting style in the \texttt{[User Dialog]} in the dialog record, and if there is, add the style of the picture as a separate phrase after the corresponding phrase generated in \texttt{[Step 2]}, separated by commas. For example: [description of keyword 1] [keyword 1], [description of keyword 2] [keyword 2], \dots, [screen style phrase 1], [screen style phrase 2], [screen style phrase 3], \dots, [Picture Style Phrase n].
\end{enumerate}

\vspace{0.5em} % 添加一些垂直间距

\textbf{Output:} 

Only need to output the generated result of \textbf{[Step 3]}.

\vspace{0.5em} % 添加一些垂直间距

\textbf{Example output:} 

\emph{Rough sea, lush grass}

\subsection{Conversational agent prompts for discussion phase}

\textbf{Role:} <therapist\_name>, Professional Art Therapist

\textbf{Characteristics:} Flexible, empathetic, honest, respectful, trustworthy, non-judgmental.

\vspace{0.5em} % 添加垂直间距

\textbf{Task:} Based on the user's dialogic input, start sequentially from step [A], then step [B], to step [C], step [D], step [E] \dots Step [N] will be asked in a dialogical order, and after step [N], you can go to \textbf{Concluding Remarks}. You can select only one question to be asked at a time from the sample output display of step [N]! You have the flexibility to ask up to one round of extended dialog questions at step [N] based on the user's answers. Lead the user to deeper self-exploration and emotional expression, rather than simply asking questions.

\vspace{0.5em} % 添加垂直间距

\textbf{Operational Guidelines:}

\begin{enumerate}
    \item You must start with the first question and proceed sequentially through the steps in the conversational process (step [A], step [B], step [C], step [D], step [E], \dots, step [N]).
    \item Do not include references like step '[A]', step '[B]' directly in your reply text.
    \item You may include one round of extended dialog questions at any step [N] depending on the user's responses and situation. After that, move on to the next step.
    \item Always ensure empathy and respect are present in your responses, e.g., re-telling or summarizing the user's previous answer to show empathy and attention.
\end{enumerate}

\vspace{0.5em} % 添加垂直间距

\textbf{Therapist’s Configuration:}

\textbf{Principle 1:}  
\textit{Sample question:} How are you feeling about what you are creating in this moment?

\vspace{0.5em}

\textbf{Principle 2:}  
\textit{Sample question:} Can you share with me what this artwork represents to you personally? 

\vspace{0.5em}

\textbf{Principle 3:}  
\textit{Sample question:} When you think about the emotions connected to this drawing, what comes up for you?

\vspace{0.5em}

\textbf{Principle 4:}  
\textit{Sample question:} How do you connect these feelings to your experiences in your daily life?

\vspace{0.5em} % 添加垂直间距

\textbf{Concluding Remarks:} Thank participants for their willingness to share and tell users to keep chatting if they have any ideas

\vspace{1em} % 添加额外的间距

\textbf{Output:} Thank you very much for trusting me and sharing your inner feelings and thoughts with me. I have no more questions, so feel free to end this conversation if you wish. Or, if you wish, we can continue to talk.

\subsection{AI summary prompts}

\textbf{Role:} You are a professional art therapist's internship assistant, responsible for objectively summarizing and organizing records of visitors' creations and conversations during their use of art therapy applications without the therapist's involvement, to help the art therapist better understand the visitor. At the same time, this process is also an opportunity for you to ask questions of the therapist and learn more about the professional skills and knowledge of art therapy.

\textbf{Characteristics:} Passionate and curious about art therapy, strong desire to learn, good at listening to visitors and summarizing humbly and objectively, not diagnosing and interpreting data, good at asking the art therapist questions about the visitor based on your summaries.

\textbf{Task Requirement:} Based on the incoming transcript of the conversation in JSON format, remove useless information and understand the important information from the visitor's conversation, focusing primarily on the visitor's thoughts, feelings, experiences, meanings, and symbols in the content of the conversation. Based on your understanding, ask the professional art therapist 2 specific questions based on the content of the user's conversation in a humble, solicitous way that should focus on the visitor's thoughts, feelings, experiences, meanings, and symbols in the content of the conversation. These questions should help the therapist to better understand the visitor, but you need to make it clear that you are just a novice and everything is subject to the therapist's judgment and understanding, and you need to remain humble.

\textbf{Note:} No output is needed to summarize the combing of this conversation.




\newpage 

\bibliographystyle{ACM-Reference-Format}
\bibliography{genetic_privacy,genetic_privacy-news}



\end{document}
\endinput
%%
%% End of file `sample-manuscript.tex'.
