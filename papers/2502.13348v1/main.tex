\documentclass[journal]{IEEEtran}
\usepackage[utf8]{inputenc}
\usepackage[T1]{fontenc}
\usepackage{graphicx}
\usepackage{tabularx,booktabs}
\usepackage{amsmath,amssymb,amsthm,mathtools}
\usepackage{cite}
\usepackage{xcolor}
\usepackage{adjustbox}
\usepackage{gensymb}
\usepackage{bbm}
\usepackage{subcaption} % Use this instead of subfig
\usepackage{listings}
\usepackage{stfloats} % Better figure placement
\usepackage{wasysym} % For "\varhexagon" macro
\usepackage[margin=0.5in]{geometry}
% TikZ & PGFPlots
\usepackage{pgfplots} % If using TikZ
\pgfplotsset{compat=newest}
\usetikzlibrary{patterns}
\usepackage{tikz}

% Custom math commands
\newtheorem{lemma}{\textbf{Lemma}}
\newtheorem{corollary}{\textbf{Corollary}}
\newtheorem{theorem}{\textbf{Theorem}}
\newtheorem{approximation}{\textbf{Approximation}}
\newtheorem{remark}{\textbf{Remark}}
\DeclareMathOperator\erf{erf}
\DeclarePairedDelimiter\ceil{\lceil}{\rceil}
\DeclarePairedDelimiter\floor{\lfloor}{\rfloor}

% Small fraction command
\newcommand{\slfrac}[2]{\left.#1\middle/#2\right.}  


\title{System-level Analysis of Dual-Mode Networked Sensing: ISAC Integration \& Coordination Gains}

\author{Yasser Nabil, Hesham ElSawy, Senior Member, IEEE, and Hossam S. Hassanein, Fellow, IEEE
\thanks{Y.\ Nabil is with the Electrical and Computer Engineering Department, Queen’s University, Kingston, Ontario, Canada. E-mail: \texttt{yasser.nabil@queensu.ca}.\\
H.\ ElSawy and H. S.  Hassanein are with the School of Computing, Queen's University, Kingston, Ontario, Canada. E-mails: \texttt{hesham.elsawy@queensu.ca} and \texttt{ hossam@cs.queensu.ca}.}} 

\begin{document}

\maketitle
\begin{abstract}
This paper characterizes integration and coordination gains in dense millimeter-wave ISAC networks through a dual-mode framework that combines monostatic and multistatic sensing. A comprehensive system-level analysis is conducted, accounting for base station (BS) density, power allocation, antenna misalignment, radar cross-section (RCS) fluctuations, clutter, bistatic geometry, channel fading, and self-interference cancellation (SIC) efficiency.
Using stochastic geometry, coverage probabilities and ergodic rates for sensing and communication are derived, revealing tradeoffs among BS density, beamwidth, and power allocation. It is shown that the communication performance sustained reliable operation despite the overlaid sensing functionality. In contrast, the results reveal the foundational role of spatial sensing diversity, driven by the dual-mode operation, to compensate for the weak sensing reflections and vulnerability to imperfect SIC along with interference and clutter. To this end, we identify a system transition from monostatic to multistatic-dominant sensing operation as a function of the SIC efficiency. In the latter case, using six multistatic BSs instead of a single bistatic receiver improved sensing coverage probability by over 100\%, highlighting the coordination gain.  Moreover, comparisons with pure communication networks confirm substantial integration gain. Specifically, dual-mode networked sensing with four cooperative BSs can double throughput, while multistatic sensing alone improves throughput by over 50\%.
\end{abstract}












\begin{IEEEkeywords}
Integrated sensing and communication (ISAC), networked sensing, multistatic, millimeter wave, coverage and ergodic rate analysis, stochastic geometry.
\end{IEEEkeywords}

\IEEEpeerreviewmaketitle


\section{Introduction}


The sixth-generation (6G) envisions a unified framework for wireless connectivity and precise sensing, using a single waveform for both sensing and communication (S\&C) \cite{liu2022integrated,zhang2021enabling,cui2024integrated,lu2024integrated,wei2023integrated,wen2024survey}. This approach, known as Integrated Sensing and Communications (ISAC) \cite{liu2022integrated,zhang2021enabling}, leverages shared channel properties, signal processing, and hardware, enabling base stations (BSs) and mobile users (MUs) to function as sensors. 
ISAC transforms traditional communication infrastructures into large-scale radio sensing systems, enabling reliable communication and supporting applications such as vehicular networks, environmental monitoring, indoor services, industrial automation, and drone-based surveillance \cite{liu2022integrated,zhang2021enabling,cui2024integrated,lu2024integrated,wei2023integrated,wen2024survey}. It also facilitates emerging higher-level applications such as the metaverse, remote surgeries, autonomous driving, and digital twins \cite{liu2022integrated,zhang2021enabling,cui2024integrated,lu2024integrated,wei2023integrated,wen2024survey}.
To this end, ISAC offers two fundamental advantages: integration gain and coordination gain \cite{liu2022integrated,lu2024integrated}. Integration gain stems from the efficient use of shared wireless resources, reducing hardware redundancy and enhancing spectral efficiency. Coordination gain, in turn, arises from the synergy between S\&C. Specifically, sensing-assisted communication, where sensing can improve tasks such as reducing beam training overhead \cite{10433485}, and communication-assisted sensing (or networked sensing), which leverages cellular connectivity for large-scale coordinated sensing with unparalleled capabilities \cite{liu2022integrated,zhang2021enabling,cui2024integrated}.





Millimeter-wave (mmWave) frequencies are well-suited for ISAC, offering wide bandwidth and narrow beamwidth that enhance communication capacity, radar range, and angular resolutions \cite{liu2022integrated,cui2024integrated}, enabling more accurate object localization \cite{liu2022integrated,zhang2021enabling}. Monostatic ISAC, with co-located transmitter (Tx) and receiver (Rx), allows compact integration and lower computational overhead but may suffer from strong self-interference (SI) as it relies on full-duplex (FD) operation entailing imperfect self-interference cancellation (SIC) \cite{wei2023integrated}. In the absence of FD, multistatic ISAC systems can utilize cellular infrastructure, where signals are transmitted by one Tx BS and received by other Rx BSs. These systems leverage spatial diversity and wide angular observations that can enhance sensing accuracy and reliability, even in the presence of environmental blockages \cite{liu2022integrated,zhang2021enabling,cui2024integrated}.
However, the reliability of multistatic sensing without FD remains uncertain. Furthermore, the potential impact of networked sensing on the communication function in large cellular systems needs further investigation. 





Recent studies have investigated networked sensing in ISAC systems \cite{li2023towards,yang2024coordinated,dehkordi2024multistatic,behdad2024multi,babu2024precoding,wei2023symbol}. For instance, \cite{li2023towards} optimizes joint beamforming to enhance sensing signal-to-noise ratio (SNR) while maintaining communication signal-to-interference-plus-noise ratio  (SINR) in cellular multistatic ISAC systems, while \cite{yang2024coordinated} introduces a coordinated beamforming framework accounting for synchronization errors. In \cite{dehkordi2024multistatic}, beamfocusing codewords are employed to address near-field and far-field conditions in urban multistatic mmWave ISAC systems. The authors in \cite{behdad2024multi} leverage a cell-free massive MIMO network to maximize sensing performance via power allocation under communication constraints, and \cite{babu2024precoding} proposes a precoder design framework optimizing sensing and communication under coordinated beamforming and multipoint transmission. Additionally, \cite{wei2023symbol} presents a symbol-level cooperative sensing framework enhancing localization and velocity estimation by centralized fusion of multiple monostatic returns.
Despite these advancements,  these studies focus on multistatic or monostatic sensing from multiple BSs, without exploring dual operation. Furthermore, most overlook interference and clutter analysis, except \cite{behdad2024multi} and \cite{babu2024precoding}, which consider interference from a few BSs rather than a large-scale network.




Achieving integration gains in ISAC systems involves overlapping signals in time and frequency, introducing severe interference and clutter effects \cite{liu2022integrated,zhang2021enabling,wei2023integrated}, which intensify in dense deployments, necessitating large-scale system-level analysis \cite{olson2023coverage}. Moreover, recent ISAC research has shifted from traditional radar metrics to SINR-based evaluations, emphasizing accurate interference modeling to capture system dynamics \cite{olson2023coverage,meng2024network}. Effective ISAC system design requires realistic modeling that incorporates interference, clutter, radar cross-section (RCS) fluctuations, and spatial topology \cite{liu2022integrated,zhang2021enabling,cui2024integrated,lu2024integrated}. 
To this end, stochastic geometry provides powerful tools for analyzing ISAC network dynamics, enabling large-scale system-level planning \cite{meng2024network,olson2023coverage,xu2024performance,sun2024performance,salem2024rethinking,ali2025integrated,10769538}. This is essential for applications requiring extensive surveillance coverage and for integrating ISAC with existing cellular networks \cite{liu2022integrated,zhang2021enabling,cui2024integrated,lu2024integrated}. 

For instance, \cite{olson2023coverage} develops an information-theoretic framework for evaluating coverage probabilities and ergodic rates in mmWave ISAC networks. The authors in \cite{meng2024network} propose a cooperative ISAC scheme using interference nulling via coordinated beamforming to balance S\&C performance. Moreover, \cite{xu2024performance} explores power and spectrum allocation under constraints like small-distance sensing resolution and high data rates, while \cite{sun2024performance} analyzes ISAC performance in urban environments with blockage effects. Additionally, \cite{salem2024rethinking} optimizes spectral and energy efficiency in dense networks through power allocation. While these studies offer valuable system-level insights, they focus on monostatic sensing, overlooking the inherently more complex multistatic scenario.
The recent work in \cite{ali2025integrated} studies a large sub-6 GHz ISAC network with joint monostatic and bistatic detection, utilizing additional bistatic radars that listen only within the same cell. However, this approach limits spatial diversity and it also neglects clutter effects.
Moreover, \cite{10769538} provides a system-level analysis of a cooperative ISAC network operating at sub-6 GHz, effectively deriving the Cramér-Rao Lower Bound (CRLB) of localization accuracy, however, it does not incorporate sensing interference or clutter effects.

Though promising, research on networked sensing is still evolving, particularly in assessing performance in large-scale deployments and investigating how cellular architectures can improve spectral efficiency and sensing capabilities while maintaining communication functionality \cite{liu2022integrated,zhang2021enabling,cui2024integrated,lu2024integrated}. 
To address the identified gaps in the literature, this paper investigates the networked sensing problem in large-scale mmWave ISAC networks, with a focus on downlink communication and location estimation. The main contributions of this work are as follows:
\begin{itemize}
\item The work identifies ISAC integration and coordination gains in large-scale mmWave ISAC networks through a well-founded mathematical model - a first in the literature. 
    \item A novel dual-mode networked sensing approach is explored, integrating monostatic and multistatic sensing using a unified ISAC signal, demonstrating meaningful gains even without FD capabilities.
\item A realistic system-level analysis is conducted, considering interference and clutter from all BSs and environmental scatterers, revealing critical tradeoffs between S\&C across key parameters such as power, beamwidth, and BS density.
\end{itemize}



This paper demonstrates the resilience of the communication function in the presence of sensing, which can accelerate the adoption of ISAC in mmWave cellular networks. Moreover, the findings indicate that FD capabilities are not a prerequisite for achieving tangible ISAC benefits; instead, multistatic sensing alone can provide reasonable gains. Additionally, the results quantify the added value of incorporating more cooperative BSs on sensing performance, enabling service providers to make informed decisions on the optimal number of cooperative BSs to balance ISAC gains against backhaul requirements.






 The structure of the paper is as follows: Section~\ref{sys_modd} outlines the system model. Section~\ref{ana_synn} details the analysis of the dual-mode networked sensing. Section~\ref{ana_commm} focuses on the communication analysis. Section~\ref{num_ress} discusses the numerical results and simulations. Lastly, Section~\ref{con_pp} provides the conclusion.






\section{System Model}\label{sys_modd}

\subsection{ISAC Network and Signal Models}


Consider a dense large-scale mmWave network utilizing a unified ISAC waveform. The BSs are spatially distributed as a Poisson Point Process (PPP), $\boldsymbol{\Phi}_{\mathrm{BS}}$, with intensity $\lambda_{\mathrm{BS}}$. Each BS employs multiple spatial beams such that each beam serves a MU while estimating the location of a random target. Operating as a cooperative networked sensing system of $N$ BSs, the serving BS acts as a monostatic radar, receiving echoes from the target, while  $(N-1)$ neighboring BSs receive reflections from the same target in a multistatic configuration.\footnote{The BSs are interconnected and synchronized via high-speed 5G fronthaul links (typically optical fiber), enabling reliable downlink multistatic sensing and centralized fusion of sensing information \cite{liu2022integrated,zhang2021enabling,cui2024integrated}.} The system is illustrated in Fig.~\ref{sys_mod}.
  




\begin{figure}[t]
\centering
  \includegraphics[width=0.385\textwidth]{sys_mod_5.png}
    \caption{An illustration of the system model with four cooperative BSs, showing a single beam for clarity. The monostatic distance is \( R_1 \), while each \( (R_1, R_n) \) pair forms a bistatic setup, where \( R_n \) is the distance between the target and its \( n^\text{th} \)-nearest BS (\( n \geq 2 \)).
} 
    \label{sys_mod}
\end{figure}






We adopt a low-complexity unified waveform design of a time slot of duration \( T_t \) \cite{xiao2022waveform}. The ISAC signal includes a sensing pulse of width \( T_s \), with the remaining time, \( T_t - T_s \), allocated for joint communication and sensing (i.e., reception of sensing echoes) as shown in Fig.~\ref{sgn_mod}. The sensing pulse duration, \( T_s \), is the reciprocal of the unified signal bandwidth \( W_b \), while \( T_t \) is chosen to satisfy the maximum unambiguous range, ensuring that the round-trip time from the farthest target does not exceed \( T_t \).
Given a total energy budget \( E_t \) per time slot, a bias factor \( 0 \leq \alpha \leq 1 \) allocates \(\alpha\) of the energy for the sensing pulse and \(( 1-\alpha )\) for communication. Thus, the sensing pulse power is \( P_s = \frac{\alpha E_t}{T_s} \), and the communication power is \( P_c = \frac{(1-\alpha) E_t}{T_t - T_s} \).
This waveform is well-suited for multistatic operations. In monostatic sensing, communication data is transmitted while sensing echoes are received over the same beam, necessitating FD operation with SIC. Given the inherent imperfections of SIC \cite{wei2023integrated, xiao2022waveform}, we define \( \zeta \) as the fraction of power remaining after SIC.




 
\begin{remark}
This design is a unified signal approach, not a time-sharing scheme, enabling simultaneous sensing and communication within the same time and frequency. It preserves communication rates through an extremely short sensing pulse relative to the time slot while ensuring strong sensing performance with a pulse structure featuring favorable correlation properties. 
\end {remark}



\subsection{Beamforming Models}

 
\begin{figure}[t]
\centering
  \includegraphics[width=0.43\textwidth]{sig_nn.png}
    \caption{An illustration for the unified ISAC signal.}
    \label{sgn_mod}
\end{figure}


Consider a multi-beam, multi-frequency antenna system where each beam transmits on a distinct frequency but can receive across multiple frequencies. This setup is feasible with modern beamforming techniques and wideband radio frequency (RF) chains, as demonstrated in \cite{hong2017multibeam,8529214}. To this end, each BS generates \( M \) beams covering the \( 2\pi \) space, enabling simultaneous service and sensing of \( M \) MUs and \( M \) targets per cell.\footnote{While multiple targets can be sensed per beam if separated by the range resolution, and multiple MUs can be served via user-grouping techniques, this work focuses on a single user and target per beam to reduce analytical complexity. These methods, though enhancing throughput, are left for future work, as the core findings remain valid and extensible to more complex scenarios.}
Each beam operates on a unique frequency block of \( W_b \) Hz to transmit the unified ISAC signal, effectively eliminating intra-cell inter-beam interference. Moreover, universal frequency reuse is applied, ensuring full bandwidth utilization across all BSs.



For multistatic operation, signals reflected from targets in neighboring cells randomly arrive at any beam. If a beam receives signals transmitted on any of the $(M-1)$ frequencies other than its own transmission frequency, these signals are useful for multistatic sensing; otherwise, they are treated as interference.
To this end, each BS beam is modeled using the cosine model, which provides precise approximation of the main lobe gains as demonstrated in \cite{yu2017coverage} given by 
\begin{equation}\label{ant_be_ga}
G(\theta) =
\left\{
	\begin{array}{ll}
		G_m \mathrm{cos}^2 (\frac{d\theta}{2})  &  |\theta| \leq \frac{\pi}{d}\\
		0             &   \text{otherwise}
	\end{array}
\right.,
\end{equation}
where $G_m$ represents the maximum gain, $d$ determines the spread of the beam, and $\theta$ denotes the beam angle relative to the boresight angle. To ensure complete $2\pi$ coverage, the 3-dB beamwidth of each beam is given by $\theta_B = \frac{\pi}{d} = \frac{2\pi}{M}$, where the number of beams is $M = 2d$. 
Moreover, for simplicity in the analysis, MUs are assumed to use omnidirectional antennas. 


  

\subsection{ Channel and Propagation Models}



Considering the dense mmWave setup, we adopt the widely used line of sight (LoS) ball model \cite{yu2017coverage}. That is, we assume that the MU and the target maintain a LoS link with the serving BS responsible for monostatic operation.  However, the multistatic link between the target and other BSs, as well as co-channel interference signals from other BSs, can be either LoS or non-line of sight (NLoS). According to \cite{akdeniz2014millimeter,rebato2019stochastic}, the probability that a link of distance \( r \) is LoS is given by:

\begin{equation}\label{los_prop}
\boldsymbol{p}_{\boldsymbol{LOS}}(r) = e^{-\gamma r},
\end{equation}
where \(\gamma\) is a parameter determined by the density and geometry of the blockage environment.

To ensure tractability and consistency with widely accepted studies \cite{olson2023coverage,behdad2024multi,yang2024coordinated,salem2024rethinking,dehkordi2024multistatic,10769538}, sensing tasks are considered feasible only when a direct LoS link exists between the target and the BS. This assumption arises because NLoS channels experience excessive delay and require precise knowledge of the reflection geometry. This challenge is intensified in mmWave channels, where significant attenuation causes multipath components to convey much less reliable information.
On the other hand, the communication link and co-channel interference from other BSs are assumed to undergo quasi-static Nakagami-$m$ fading, with fading gains modeled as i.i.d. Gamma random variables (RVs). The shape parameters $m_L$ and $m_N$ correspond to LoS and NLoS conditions, respectively. Furthermore, sensing and communication functions occur within a time slot $T_t$, during which channel and sensing statistics remain constant but vary across time slots.




\begin{remark}
The system's cooperative design introduces spatial diversity, ensuring robust sensing. Even if some cooperative BSs have NLoS links to the target, other BSs with LoS links ensure seamless sensing operation.
\end{remark}

 The RCS, which quantifies a target's ability to scatter electromagnetic energy back to the radar, significantly influences the strength of the return signal, directly affecting sensing performance \cite{barton2004radar,willis2005bistatic}. The monostatic RCS fluctuations of the target, $\sigma_{tm}$, is modeled using the Swerling I model \cite{swerling1960probability}, which follows an exponential probability density function (PDF), given by:
\begin{equation}\label{sw1_rcs}
f(\sigma_{tm})=\frac{1}{\sigma_{\text{av}_t}}\exp{\left(\frac{-\sigma_{tm}}{\sigma_{\text{av}_t}}\right)}, \quad \sigma_{tm} \geq 0,           
\end{equation}
where $\sigma_{\text{av}_t}$ denotes the average RCS of target.
Monostatic setups typically capture stronger reflections, whereas bistatic reflections are weaker and more sensitive to angular variations. Consequently, modeling the bistatic RCS is more challenging, as it depends on the target's capacity to reflect energy from the Tx to the Rx, with the reflection strength heavily influenced by the target's geometry.
To account for the impact of bistatic geometry on the target's RCS, we approximate the bistatic RCS, $\sigma_{tb}$, following the approach in \cite{kell1965derivation} as
\begin{equation}\label{rcs_bi_app}
\sigma_{tb} \approx \sigma_{tm} \cos \left( \frac{\beta}{2} \right),
\end{equation}
where \( \beta \) denotes the bistatic angle. This approximation is reasonable, as the high frequency of mmWave ensures that the target's dimensions are significantly larger than the wavelength, making specular reflections dominant while minimizing diffraction and scattering effects \cite{willis2005bistatic}.





From a sensing perspective, clutter emerges as a new form of interference in ISAC networks, caused by reflections from scatterers other than the targets \cite{liu2022integrated,zhang2021enabling,wei2023integrated}. As in \cite{chen2012integrated}, we model clutter scatterers ($cl$) as a PPP $\bold\Phi_{cl}$  with intensity $\lambda_{Cl}$. 
If the radar resolution cell is defined as the smallest distinguishable unit of space where radar can separate two targets \cite{barton2004radar,willis2005bistatic},  sensing is affected only by clutter from scatterers within the same resolution cell as the target and with comparable RCS. This clutter arrives with nearly the same delay as the target reflection, making it difficult to filter out \cite{shnidman2005radar}.
Conversely, clutter with significantly different RCS is excluded, as it can be mitigated using standard signal processing techniques \cite{rahman2019framework}.


To this end, the monostatic RCS of clutter, \( \sigma_{cm} \), is modeled using the generalized Weibull distribution, which is widely adopted for its mathematical flexibility in representing diverse environmental conditions \cite{sekine1990weibull} with PDF given by:
\begin{equation}\label{clu_wei}
f(\sigma_{cm}) = \frac{k}{\sigma_{\text{av}_{cl}}} \left( \frac{\sigma_{cm}}{\sigma_{\text{av}_{cl}}} \right)^{k-1} \exp \left( - \left( \frac{\sigma_{cm}}{\sigma_{\text{av}_{cl}}} \right)^k \right), \quad \sigma_{cm} \geq 0,     
\end{equation}  
where \( k \) denotes the shape parameter, and \( \sigma_{\text{avg}_c} \) represents the average RCS of clutter.
In particular, the environment considered here assumes clutter with an RCS comparable to that of the target, which corresponds to an exponential distribution when \( k = 1 \) \cite{sekine1990weibull}. 
Furthermore, since the clutter RCS is assumed to be comparable to that of the target, its bistatic counterpart, $\sigma_{cb}$, can be approximated as in (\ref{rcs_bi_app}) such that $\sigma_{cb} \approx \sigma_{cm} \cos \left( \frac{\beta}{2} \right)$.































\section{Dual-mode Networked Sensing Analysis}\label{ana_synn}

In this section, we carry out the dual-mode networked sensing analysis. We adopt parameter estimation-based formulation for sensing, which is based on mutual information and SINR statistics \cite{olson2023coverage,meng2024network}.
The mutual-information-based approach has recently gained widespread adoption in ISAC systems, as it extends the concepts of coverage probability and ergodic rate from communication to sensing, enabling a unified performance evaluation framework for ISAC systems. 
Specifically, the sensing coverage probability is defined as the probability that the rate of information extracted about a typical target's parameters of interest exceeds some threshold \cite{olson2023coverage}.  Meanwhile, the sensing ergodic rate represents the spatial average of this information extraction rate \cite{olson2023coverage}, which establishes a theoretical lower bound on estimation error \cite{liu2023deterministic}.  Mathematically, the sensing coverage probability is expressed as \( \mathbb{P} \left( \text{SINR} > \phi_s \right) \), where $\phi_s$ is a predefined threshold \cite{olson2023coverage}. Here, SINR  is the ratio of the received sensing signal power to the combined interference, clutter, and noise, where a higher SINR enhances the accuracy of target parameter estimation \cite{tang2018spectrally}. 
Moreover,
as derived in \cite{olson2023coverage,meng2024network}, the sensing information rate is expressed as:
\begin{equation}
\mathcal{R}_s = \mathbb{E}[\log(1 + \text{SINR})].   
\end{equation}












\subsection{Types of Interference and Distance Distributions}



\begin{figure*}[h]
\centering
  \includegraphics[width=0.61\textwidth]{full_intr_new2.png}
    \caption{An illustration depicting interference sources in monostatic operation (left) and bistatic operation (right),  where \( L_n \) represents the bistatic distance. Each interference type is shown with a single representative signal for clarity.}
    \label{ill_ntr}
\end{figure*}



The different types of interference for sensing, covering both monostatic and bistatic scenarios, are depicted in Fig. \ref{ill_ntr} and are summarized below:

 
\begin{enumerate}
    \item \textbf{Direct or co-channel interference from other BSs:} represented by red signals in Fig. \ref{ill_ntr}, these signals do not experience the double path-loss affecting the intended sensing signal. They are analyzed under both LoS and NLoS conditions.


    \item \textbf{Intra-clutter interference:} depicted as dashed green signals in Fig. \ref{ill_ntr}, this arises when the intended signal reflects off scatterers within the target's resolution cell. It is analyzed due to its comparable RCS to the target and a similar delay profile, making it difficult to filter out.


    \item \textbf{Inter-clutter interference:} represented by blue signals in Fig. \ref{ill_ntr}, this interference arises from signals originating from other BSs, reflecting off the target or scatterers within the same resolution cell, and reaching the serving BS. These signals can share similar delay or doppler characteristics with the target's echoes, making them difficult to filter out.
 
\end{enumerate}


The monostatic sensing of a given target is conducted by its nearest BS.  Based on the properties of PPP, the PDF of \( R_1 \) is:
\begin{equation}\label{ner_dis}
f(R_1) = 2 \pi \lambda_{\mathrm{BS}} R_1 e^{-\pi \lambda_{\mathrm{BS}} R_1^2}, \quad R_1 \geq 0.
\end{equation}
As in \cite{8057736}, the conditional PDF of the distance \( R_n \) to the \( n^{\text{th}} \) nearest BS, given that the distance to the nearest BS is \( R_1 \), is:
\footnotesize
\begin{equation}\label{cond_dstt}
f(R_n \mid R_1) = \frac{2 (\lambda_{\mathrm{BS}} \pi)^{n-1}}{(n-2)!} (R_n^2 - R_1^2)^{n-2} R_n e^{-\lambda_{\mathrm{BS}} \pi (R_n^2 - R_1^2)}, 0 \leq R_1 \leq R_n
\end{equation}
\normalsize
Finally, owing to the structure of the PPP, the angle \( \beta \) between \( R_1 \) and \( R_n \) is uniformly distributed, with the PDF given by:
\begin{equation}\label{bet_distrb}
f(\beta) = \frac{1}{\pi}, \quad 0 \leq \beta \leq \pi.
\end{equation}




\subsection{Monostatic Sensing Analysis}



\begin{figure*}
\hrule
\begin{equation}\label{scnr_mono}
\text{SINR}_M=\frac{\frac{P_s  G_{m}^2 \sigma_{tm}  \lambda^2  }{(4\pi)^3 R_{1}^{2 \eta_L}}}{\sum\limits_{cl\in \bold\Phi_{cl}\cap A_{rm}}\frac{P_s G_{m}^2 \sigma_{cm} \lambda^2 }{(4\pi)^3 R_{1}^{2 \eta_L}}+\sum\limits_{\substack{\text{BS}_{n }\in \bold\Phi_{L_{IC}}\\ n \neq 1}} \frac{P_s  G_{m}^2 \sigma_{tb}  \lambda^2  }{(4\pi)^3 R_1^{\eta_L}  R_n^{\eta_L}}+\sum\limits_{\substack{\text{BS}_{n}\in \bold\Phi_{L_{IC}}\\ n \neq 1}}\sum\limits_{cl\in \bold\Phi_{cl}\cap A_{rm}}\frac{P_s  G_{m}^2 \sigma_{cb}  \lambda^2  }{(4\pi)^3 R_1^{\eta_L}  R_n^{\eta_L}}+K_B T W_b+I_{L_s}+I_{N_s}+ P_c \zeta  }.
\end{equation}
\hrule
\end{figure*}
A generic frequency resource can be utilized by one beam per BS to transmit a unified ISAC signal. 
To evaluate the SINR, we consider a reference BS at the origin without loss of generality. Direct interference arises when beams from other BSs align with the reference BS’s beam on the same frequency block, and inter-clutter interference arises from BSs if their beams are directed toward the target. Following the common practice in stochastic geometry \cite{elsawy2016modeling}, the next approximation captures their impact.
\begin{approximation}
To ensure mathematical tractability, direct interfering BSs are approximated by a PPP \(\boldsymbol{\Phi}_{s}\) with intensity \(\lambda_{I_s} =   \frac{\lambda_{Bs}}{M^2}\), where \(\frac{1}{M^2}\) is the probability of beam alignment between an interfering BS and the typical BS.  Moreover, let \(\boldsymbol{\Phi}_{L_{IC}}\) denote another PPP of BSs generating inter-clutter signals. These BSs contribute to inter-clutter signals if their beams are directed toward the target with LoS paths and use the same frequency block under analysis. The intensity of this PPP is \( \frac{\boldsymbol{p}_{\text{LOS}}(r) \lambda_{BS}}{M}\), where \(\frac{1}{M}\) is the probability of a BS beam being directed toward the target.
\label{approx_dir}
\end{approximation}
The validity of Approximation \ref{approx_dir} will be verified through system-level simulations. To this end, the set of interfering BSs, \(\boldsymbol{\Phi}_{s}\), is further divided into two independent PPPs: LOS BSs, \(\boldsymbol{\Phi}_{L_s}\), with intensity \(\boldsymbol{p}_{\text{LOS}}(r) \lambda_{I_s}\), and NLOS BSs, \(\boldsymbol{\Phi}_{N_s}\), with intensity \((1 - \boldsymbol{p}_{\text{LOS}}(r)) \lambda_{I_s}\). 
To estimate location, we consider the monostatic range resolution cell area, defined as \( A_{rm} = \frac{c \, \theta_B \, R_1}{2 \, W_b} \) \cite{barton2004radar}, where \( c \) is the speed of light. 
This is the smallest region where the radar cannot distinguish closely spaced targets, perceiving them as a single target.
Given the typically narrow resolution cell \cite{barton2004radar} and that only clutter within the same resolution cell as the target is considered, it is reasonable to approximate the range to both as \( R_1 \) for tractability.
Using the radar range equation \cite{barton2004radar}, the monostatic SINR at the typical BS for a target at \( R_1 \) is given in (\ref{scnr_mono}) on the next page. The numerator represents the desired echo power, while the denominator represents, intra-clutter, inter-clutter reflected from the target, inter-clutter reflected from other scatterers, thermal noise, LoS direct interference, NLoS direct interference, and residual SI respectively.
For simplicity, a constant gain \( G_m \) is assumed within the 3-dB beamwidth, while a gain of zero is assumed outside this range.
In (\ref{scnr_mono}), \(\eta_L\) is the LoS path-loss exponent, and \(\lambda\) is the signal wavelength.  Moreover, \(K_B\) is the Boltzmann constant, and \(T\) is the system temperature.
\(I_{L_s} = \sum\limits_{\substack{\text{BS}_i \in \boldsymbol{\Phi}_{L_s}}} P_c h_{L,i} G_m^2 C_L r_i^{-\eta_L}\) and \(I_{N_s} = \sum\limits_{\substack{\text{BS}_i \in \boldsymbol{\Phi}_{N_s}}} P_c h_{N,i} G_m^2 C_N r_i^{-\eta_N}\), where \(h_{L,i}\) and \(h_{N,i}\) are the channel gains of the \(i^{\text{th}}\) LoS and NLoS interfering BS, \(C_L\) and \(C_N\) are the path-loss intercepts for LoS and NLoS,  \(r_i\) is the distance between \(\text{BS}_i\) and the origin, and $\eta_N$ is the  NLOS path-loss exponent.
\begin{remark}
Although (\ref{scnr_mono}) represents the monostatic SINR, the bistatic RCS appears in the inter-clutter interference terms. This accounts for a bistatic interference setup, where signals from interfering BSs reflect off the target or scatterers in the same resolution cell and return to the serving BS. This interference strength depends on the geometry, captured by the bistatic RCS.
\end{remark}
Before evaluating the monostatic sensing coverage probability, we first derive the Laplace transform (LT) of the aggregate interference for different interference types. For direct interference, we define an interference protection region centered at the reference BS (at the origin) with a radius equal to the distance to the nearest neighboring BS, whose PDF is given in (\ref{ner_dis}). The LT of the aggregate direct interference is given in the following lemma.
\begin{lemma} \label{lemma_LT_DMR}
The LT of the aggregate direct interference seen by the typical BS located at the origin is given in (\ref{dir_int}) at the beginning of the next page,
\begin{figure*}[h]
\begin{equation}\label{dir_int}
\begin{aligned}
&\mathcal{L}_{\text{direct interference}}(s)= \int_{0}^\infty \left( \exp \left( - 2\pi \lambda_{BS} \frac{1}{M^2} \int_{R_d}^\infty \bold p_{\bold{LOS}}\left(r\right)  \left( 1 - \left( 1 + \frac{s P_c G_m^2 C_L r^{-\eta_L}}{m_L} \right)^{-m_L} \right) r dr \right) \right) \\
&\times \exp \left( - 2\pi \lambda_{BS} \frac{1}{M^2} \int_{R_d}^\infty \left( 1 - \bold p_{\bold{LOS}}\left(r\right)  \right) \left( 1 - \left( 1 + \frac{s P_c G_m^2 C_N r^{-\eta_N}}{m_N} \right)^{-m_N} \right) r dr \right) 
\times 2 \pi \lambda_{BS} R_d e^{-\pi \lambda_{BS} R_d^2} dR_d
\end{aligned}
\end{equation}
\hrule
\end{figure*}
where $\bold p_{\bold{LOS}}\left(r\right) $ is given by (\ref{los_prop}).
\begin{IEEEproof}
See Appendix A.
\end{IEEEproof}
\end{lemma}


The LT of intra-clutter interference, arising from scatterers within the target’s resolution cell and with comparable RCS, is presented in the following lemma.
\begin{lemma} \label{intra_clu}
The LT of monostatic intra-clutter interference is
\begin{equation}\label{intra_clu_int}
\begin{aligned}
\mathbb{E}_{cl, \sigma_{cm}} & \left[ \exp\left( - \frac{\phi_s \sum_{cl \in \bold\Phi_{cl} \cap A_{rm}} \sigma_{cm}}{\sigma_{\text{av}_t}} \right) \right] = \\
&\exp \left( - \lambda_{cl} \frac{c \theta_B R_1}{2 W_b} \cdot \frac{\phi_s  \sigma_{\text{av}_{cl}}}{\sigma_{\text{av}_t} + \phi_s  \sigma_{\text{av}_{cl}}} \right)
\end{aligned}
\end{equation}
\begin{IEEEproof}
See Appendix B.
\end{IEEEproof}
\end{lemma}

For inter-clutter signals, an interference protection region is established around the target, with a radius corresponding to the distance to the second nearest BS, given that the serving BS is located at \( R_1 \). The PDF of this distance is provided in (\ref{cond_dstt}) for \( n=2 \).  Accordingly, the LT of inter-clutter interference, including reflections from both the target and other scatterers within the same resolution cell, is presented in the following lemma.
\begin{lemma} \label{inter_clu}
The LT of monostatic inter-clutter interference is given in (\ref{inter_clu_int}) on the next page.
\begin{figure*}[h]
\begin{equation}\label{inter_clu_int}
\begin{aligned}
\mathcal{L}_{\text{inter-clutter}}(s) = & \int_{R_1}^{\infty} \int_{0}^{\pi} 
\exp \left( - 2 \pi \lambda_{BS} \frac{1}{M} \int_{R_{IC}}^{\infty} \bold p_{\bold{LOS}}\left(r\right)  \left( 1 - \frac{1}{1 + s \cos\left( \frac{\beta}{2} \right) r^{-\eta_L} \sigma_{\text{av}_t}} \right) r \, dr \right) \\
& \times \exp \left( - \lambda_{BS} \cdot \frac{1}{M} \int_{R_{IC}}^{\infty}  \bold p_{\bold{LOS}}\left(r\right)  \left( 1 - \exp \left( - \lambda_{cl} \cdot \frac{c \theta_B R_1}{2 W_b} \cdot \frac{s \cos \left( \frac{\beta}{2} \right) r^{-\eta_L} \sigma_{\text{av}_{cl}}}{1 + s \cos \left( \frac{\beta}{2} \right) r^{-\eta_L} \sigma_{\text{av}_{cl}}} \right) \right) 2 \pi r \, dr \right) \\
& \times 2 \lambda_{BS}  R_{IC} e^{-\lambda_{BS} \pi (R_{IC}^2 - R_1^2)}  \,  d\beta  \, dR_{IC}
\end{aligned}
\end{equation}
\hrule
\end{figure*}
\begin{IEEEproof}
See Appendix C.
\end{IEEEproof}
\end{lemma}
Building on the previous lemmas, the monostatic sensing coverage probability is presented in the following theorem.
\begin{theorem} \label{mono_det}
The monostatic sensing coverage probability for a target at a distance \(R_1\) with SINR threshold \(\phi_s\) is provided in (\ref{coV_mon}) on next page,
\begin{figure*}[h]
\begin{equation}\label{coV_mon}
\begin{aligned}
 \mathcal{P}_M (R_1) &=  \underbrace{\mathbb{E}_{cl, \sigma_{cm}}\left[ \exp\left( - \frac{\phi_s  \sum_{cl \in \bold\Phi_{cl} \cap A_{rm}} \sigma_{cm}}{\sigma_{\text{av}_t}} \right) \right]}_{\text{Intra-clutter effect}} 
\times \underbrace{\mathcal{L}_{I_{IC1}}\left( \frac{\phi_s  R_1^{\eta_L}}{\sigma_{\text{av}_t}} \right)
\times \mathcal{L}_{I_{IC2}}\left( \frac{\phi_s  R_1^{\eta_L}}{\sigma_{\text{av}_t}} \right)}_{\text{Inter-clutter effect}} 
\times \underbrace{\exp\left( - \frac{\phi_s  (4\pi)^3 R_1^{2 \eta_L} K_B T W_b}{P_s G_m^2 \lambda^2 \sigma_{\text{av}_t}} \right)}_{\text{Noise effect}} \\
&  \times \underbrace{\mathcal{L}_{I_{L_s}}\left( \frac{\phi_s  (4\pi)^3 R_1^{2 \eta_L}}{P_s G_m^2 \lambda^2 \sigma_{\text{av}_t}} \right)
\times \mathcal{L}_{I_{N_s}}\left( \frac{\phi_s  (4\pi)^3 R_1^{2 \eta_L}}{P_s G_m^2 \lambda^2 \sigma_{\text{av}_t}} \right)}_{\text{Direct interference effect}} 
\times \underbrace{\exp\left( - \frac{\phi_s  (4\pi)^3 R_1^{2 \eta_L} P_c \zeta}{P_s G_m^2 \lambda^2 \sigma_{\text{av}_t}} \right)}_{\text{Self-interference effect}}
\end{aligned}
\end{equation}
\hrule
\end{figure*}
where \(\bold{p}_{\text{LOS}}\left(R_1\right)\) is from (\ref{los_prop}), the direct interference effect from (\ref{dir_int}), the intra-clutter effect from (\ref{intra_clu_int}), and the inter-clutter effect from (\ref{inter_clu_int}).
\begin{IEEEproof}
See Appendix D.
\end{IEEEproof}
\end{theorem}
%Finally, the average monostatic sensing coverage probability is obtained by integrating (\ref{coV_mon}) over the PDF of $R_1$ provided in (\ref{ner_dis}).
\subsection{Bistatic Sensing Analysis}
Bistatic sensing involves two BSs: the source BS at a distance \( R_1 \) from the target and the \( n^\text{th} \) cooperative BS at a distance \( R_n \) from the target.
The Rx BS can process the bistatic target return through any of its beams, except for the beam that uses the same frequency resource for transmission. In this case, the power of the bistatic reflection is too weak to be processed compared to the monostatic echo. Nevertheless, this reflection contributes to inter-clutter interference in the monostatic scenario.
 To this end, the SINR for bistatic sensing at the \( n^\text{th} \)-nearest BS from the target is given in (\ref{scnr_bi}) on the next page.
\begin{figure*}
\begin{equation}\label{scnr_bi}
\text{SINR}_{B_n}=\frac{\mathbbm{1}_{\{c_1\}}\frac{P_s  G_{m}^2 \sigma_{tb}  \lambda^2  }{(4\pi)^3 R_1^{\eta_L}  R_n^{\eta_L}}}{\sum\limits_{cl\in \bold\Phi_{cl}\cap A_{rb}}\frac{P_s G_{m}^2 \sigma_{cb} \lambda^2 }{(4\pi)^3 R_1^{\eta_L}  R_n^{\eta_L}}+\sum\limits_{\substack{\text{BS}_{v}\in \bold\Phi_{L_{IC}}\\ v \neq 1, v \neq n}}\frac{P_s  G_{m}^2 \sigma_{tb}  \lambda^2  }{(4\pi)^3 R_v^{\eta_L}  R_n^{\eta_L}}+\sum\limits_{\substack{\text{BS}_{v}\in \bold\Phi_{L_{IC}}\\ v \neq 1, v \neq n}}\sum\limits_{cl\in \bold\Phi_{cl}\cap A_{rb}}\frac{P_s  G_{m}^2 \sigma_{cb}  \lambda^2  }{(4\pi)^3 R_v^{\eta_L}  R_n^{\eta_L}}+K_B T W_b+I_{L_s}+I_{N_s} }
\end{equation}
\hrule
\end{figure*}
The numerator represents the desired power received from the target, while the denominator terms correspond to intra-clutter, inter-clutter reflection from the target, inter-clutter reflections from other scatterers, thermal noise, LoS direct interference, and NLoS direct interference. Note that, (\ref{scnr_bi}) is free from SI. 
The bistatic range-limited resolution cell area is \( A_{rb} \), and the indicator function \( \mathbbm{1}_{\{\cdot\}} \) equals one if \( \{\cdot\} \) is true and zero otherwise. The condition \( c_1 \) specifies that the target is LoS with the Rx BS, and the bistatic return is received on a different beam than the one using the same frequency, occurring with probability \( \frac{M-1}{M} \). Moreover, \( R_v \) represents the first link distance for inter-clutter interference between an interfering BS and the target (see Fig.~\ref{ill_ntr}).


In the bistatic setup, the range-limited resolution cell is larger and exhibits a more intricate structure than the monostatic one \cite{willis2005bistatic}. However, for small bistatic distances \( L \) relative to \(( R_1 + R_n )\), typical of dense networks, it is approximated as \cite{willis2005bistatic,ram2022estimation}:
\begin{equation}\label{bis_res_are}
  A_{rb} \approx \frac{c R_n \theta_B}{2 W_b \cos^2\left(\frac{\beta}{2}\right)}.  
\end{equation}
For \(\beta = 0\), this simplifies to the monostatic case, ensuring consistency between both setups.

Following the same approach used in the monostatic analysis, the LT of various interference sources in bistatic sensing is detailed in the following lemmas.
\begin{lemma} \label{intra_clu_bi}
The LT of bistatic intra-clutter interference is:
\begin{equation}\label{intra_clu_int_bi}
\begin{aligned}
\mathbb{E}_{cl, \sigma_{cm}} &\left[ \exp \left( - \frac{\phi_s \sum_{cl \in \bold\Phi_{cl} \cap A_{rb}} \sigma_{cm}}{\sigma_{\text{av}_t}} \right) \right]=\\
& \exp \left( - \lambda_{cl} \frac{c  R_n  \theta_B}{2 W_b\cos^2\left(\frac{\beta}{2}\right)} \cdot \frac{\phi_s \sigma_{\text{av}_{cl}}}{\sigma_{\text{av}_t}+ \phi_s \sigma_{\text{av}_{cl}}} \right) 
\end{aligned}
\end{equation}
\begin{IEEEproof}
Similar to the proof of Lemma \ref{intra_clu}, but using the area of bistatic range-limited resolution cell given by (\ref{bis_res_are}).
\end{IEEEproof}
\end{lemma}



\begin{lemma} \label{inter_clu_b}
The LT of bistatic inter-clutter interference is given in (\ref{inter_clu_int_bi}) at the bottom of the next page.
\begin{figure*}[b]
\hrule
\begin{equation}\label{inter_clu_int_bi}
\begin{aligned}
\mathcal{L}_{\text{inter-clutter}}(s) = & \int_{R_1}^{\infty} \int_{0}^{\pi} 
\exp \left( - 2 \pi \lambda_{BS} \frac{1}{M} \int_{R_{IC}}^{\infty} \bold p_{\bold{LOS}}\left(r\right)  \left( 1 - \frac{1}{1 + s \cos\left( \frac{\beta_I}{2} \right) r^{-\eta_L} \sigma_{\text{av}_t}} \right) r \, dr \right) \\
& \times \exp \left( - \lambda_{BS} \cdot \frac{1}{M} \int_{R_{IC}}^{\infty}  \bold p_{\bold{LOS}}\left(r\right)  \left( 1 - \exp \left( - \lambda_{cl} \cdot \frac{c  R_n  \theta_B}{2 W_b\cos^2\left(\frac{\beta}{2}\right)} \cdot \frac{s \cos \left( \frac{\beta_I}{2} \right) r^{-\eta_L} \sigma_{\text{av}_{cl}}}{1 + s \cos \left( \frac{\beta_I}{2} \right) r^{-\eta_L} \sigma_{\text{av}_{cl}}} \right) \right) 2 \pi r \, dr \right) \\
& \times 2 \lambda_{BS}  R_{IC} e^{-\lambda_{BS} \pi (R_{IC}^2 - R_1^2)}  \,  d\beta_I  \, dR_{IC}
\end{aligned}
\end{equation}
\end{figure*}
\begin{IEEEproof}
Similar to the proof of Lemma \ref{inter_clu}, but using the area of bistatic range-limited resolution cell given by (\ref{bis_res_are}).
\end{IEEEproof}
\end{lemma}
Building on the previous lemmas, the bistatic sensing
coverage probability is presented in the following theorem.
\begin{theorem} \label{bi_det}
The bistatic sensing coverage probability at the $n^\text{th}$ nearest BS to the target, where $n \geq 2$, given that the target is at a distance $R_1$ from the source BS and the SINR threshold is $\phi_s$, is expressed in (\ref{bi_cov_pro}) at the bottom of next page,
\begin{figure*}
\hrule
\small
\begin{equation}\label{bi_cov_pro}
\begin{aligned}
\mathcal{P}_{B_n} (R_1) = & \int_{R_1}^{\infty} \int_{0}^{\pi}  \frac{ \bold p_{\bold{LOS}}\left(R_n\right)(M-1)}{M} \times \underbrace{\mathbb{E}_{cl, \sigma_{cm}} \left[ \exp \left( - \frac{\phi_s \sum_{cl \in \bold\Phi_{cl} \cap A_{rb}} \sigma_{cm}}{\sigma_{\text{av}_t}} \right) \right]}_{\text{Intra-clutter effect}} \\
& \times \underbrace{\mathcal{L}_{I_{IC1_b}} \left( \frac{\phi_s R_1^{\eta_L}}{\sigma_{\text{av}_t} \cos \left( \frac{\beta}{2} \right)} \right) \mathcal{L}_{I_{IC2_b}} \left( \frac{\phi_s R_1^{\eta_L}}{\sigma_{\text{av}_t} \cos \left( \frac{\beta}{2} \right)} \right)}_{\text{Inter-clutter effect}} \times \underbrace{\exp \left( - \frac{\phi_s (4\pi)^3 R_1^{\eta_L} R_n^{\eta_L} K_B T W_b}{P_s G_m^2 \lambda^2 \cos \left( \frac{\beta}{2} \right) \sigma_{\text{av}_t}} \right)}_{\text{Noise effect}} \\
& \times \underbrace{\mathcal{L}_{I_{L_r}} \left( \frac{\phi_s (4\pi)^3 R_1^{\eta_L} R_n^{\eta_L}}{P_s G_m^2 \lambda^2 \cos \left( \frac{\beta}{2} \right) \sigma_{\text{av}_t}} \right) \mathcal{L}_{I_{N_r}} \left( \frac{\phi_s (4\pi)^3 R_1^{\eta_L} R_n^{\eta_L}}{P_s G_m^2 \lambda^2 \cos \left( \frac{\beta}{2} \right) \sigma_{\text{av}_t}} \right)}_{\text{Direct interference effect}} \frac{2 (\lambda_{BS} \pi)^{n-1}}{(n-2)!} (R_n^2 - R_1^2)^{n-2} R_n e^{-\lambda_{BS} \pi (R_n^2 - R_1^2)} \cdot \frac{1}{\pi} \, d\beta \, dR_n.
\end{aligned}
\end{equation}
\normalsize
\end{figure*}
where $\bold p_{\bold{LOS}}\left(R_n\right)$  is given by (\ref{los_prop}), the direct interference effect is given by (\ref{dir_int}), the intra-clutter effect is given by (\ref{intra_clu_int_bi}), and the inter-clutter effect is given by (\ref{inter_clu_int_bi}).
\begin{IEEEproof}
See Appendix E.
\end{IEEEproof}
\end{theorem}








\subsection{The Cooperative Networked Sensing }
Building on the monostatic and bistatic sensing coverage probabilities, we analyze the dual-mode cooperative networked sensing, where a target is sensed by its \( N \) nearest BSs using monostatic and multistatic operations. While coherent signal combining enhances sensing performance \cite{yang2011phase}, it demands stringent synchronization and precise phase alignment between transmitted and received signals, making it computationally intensive and costly especially in large-scale multistatic systems \cite{yang2011phase,sadeghi2021target}. By contrast, the inherent timing differences caused by varying signal travel distances motivate the use of non-coherent processing \cite{sadeghi2021target}, allowing each BS to handle its received signals independently.
To simplify the analysis, we adopt a selection combining strategy, which selects the BS with the highest SINR. This approach ensures reliable performance while reducing computational complexity, backhaul overhead, and signal processing demands.
To this end, the networked sensing coverage probability is the probability that the SINR at any of the $N$ cooperative  BSs exceeds a threshold \( \phi_s \), mathematically expressed as:
\begin{equation}\label{joint_fus_form}
  \mathcal{P}_{\text{net}}(R_1) = 1 - \left[(1 - \mathcal{P}_M(R_1)) \prod_{n=2}^{N} (1 - \mathcal{P}_{B_n}(R_1))\right].  
\end{equation}
To evaluate the average cooperative sensing coverage probability, we compute the expectation of (\ref{joint_fus_form}) with respect to \(R_1\), whose PDF is given in (\ref{ner_dis}). 
\begin{remark}
(\ref{joint_fus_form}) is derived as the complement of the probability that none of the cooperative BSs' SINRs exceed the threshold \( \phi_s \). This assumes independence among the \( N \) BSs, justified by the uncertainties in BS locations, LoS conditions, interference, fading gains, RCS fluctuations, and antenna orientations. Nevertheless, the validity of this independence assumption will be further assessed through comprehensive system-level simulations.
\end{remark}
 


\subsection{Sensing information Rate}

The sensing rate is quantified using a metric based on the mutual information between sensing returns and the target parameter (i.e., range), capturing the rate at which information about this parameter is acquired. The networked sensing information rate, reflecting the achievable throughput from \( N \) cooperative BSs, is derived in the following theorem.
\begin{theorem} \label{ses_rate}
The average dual-mode networked sensing information rate obtained through cooperation among \( N \) BSs is:
\begin{equation}
\begin{aligned}
\mathcal{R}_{s_{\text{avg}}}&=M\int_0^\infty\bigg(\int_0^\infty 1 - \bigg[\big(1 - \mathcal{P}_M(R_1,t)\big)\\
& \quad \prod_{n=2}^{N} \big(1 - \mathcal{P}_{B_n}(R_1,t)\big)\bigg] \, dt\bigg)
 2 \pi \lambda_{BS} R_1 e^{-\pi \lambda_{BS} R_1^2} \; dR_1
\end{aligned}
\end{equation}
where $\mathcal{P}_M (R_1,t)$ and $\mathcal{P}_{B_n} (R_1,t)$ can be obtained by replacing $\phi_s$ by $( e^t - 1)$ in Theorem \ref{mono_det} and Theorem \ref{bi_det} respectively. 
\begin{IEEEproof}
Utilizing the integral representation of the expectation of a non-negative RV, the expected value of \(\log(1 + \text{SINR}_{\text{net}}(R_1)))\) can be expressed as 
$ \int_0^\infty \mathbb{P}( \text{SINR}_{\text{net}}(R_1) >( e^t - 1)) \, dt$.
Hence, the theorem is proved by utilizing the networked sensing coverage probability expression in (\ref{joint_fus_form}), substituting \(\phi_s\) with \((e^t - 1)\) in Theorems \ref{mono_det} and \ref{bi_det}, and then taking the expectation over \( R_1 \). The scaling factor \( M \) is included because there are \( M \) independent beams, where a target is sensed over each beam simultaneously.
\end{IEEEproof}
\end{theorem}








\section{Communication Analysis}\label{ana_commm}

This section focuses on analyzing the communication performance of the proposed ISAC system.

% Note that the possible reflection from  targets at MU are part of the infinite multipath interference signals
%  received at the MU which is already accounted for through NLoS interference. Hence, it can be neglected for traceability and it has a negligible effect on communication performance as confirmed in \cite{ali2024successive}.

\subsection{SINR Formulation}

We start by calculating the SINR at a typical MU, which is assumed to be located at the origin and served by the nearest BS.
For mathematical tractability, interfering BSs are divided into two independent PPPs: 
\(\boldsymbol{\Phi}_{L_c}\) for LOS BSs with intensity \(\boldsymbol{p}_{\text{LOS}}(r) \times \lambda_{BS}\), 
and \(\boldsymbol{\Phi}_{N_c}\) for NLOS BSs with intensity \((1 - \boldsymbol{p}_{\text{LOS}}(r)) \times \lambda_{BS}\). Moreover,
an interference protection region is defined with a radius equal to the distance between the MU and its serving BS, 
as no other BS can be closer to the MU than the serving BS.
 Hence, the SINR at the typical MU is given as follows:
\begin{equation}\label{sinr_comm}
 \mathrm{SINR_C}=\frac{P_c h_{L,o} G(\theta_m) C_L R_o^{-\eta_L}}{I_{L_c}+I_{N_c}+K_B T W_b}, 
\end{equation}
where \( h_{L,o} \) represents the fading gain for the intended channel, and \( R_o \) denotes the distance of the intended link. 
The LOS and NLOS interference are expressed as 
\( I_{L_c} = \sum\limits_{\substack{\text{BS}_i \in \boldsymbol{\Phi}_{L_c}}} P_c h_{L,i} G(\theta_i) C_L r_i^{-\eta_L} \) 
and 
\( I_{N_c} = \sum\limits_{\substack{\text{BS}_i \in \boldsymbol{\Phi}_{N_c}}} P_c h_{N,i} G(\theta_i) C_N r_i^{-\eta_N} \), 
respectively. Here, \( \theta_i \) is the angle between the line connecting the interfering \( \text{BS}_i \) and its intended MU, 
and the line connecting \( \text{BS}_i \) to the reference MU located at the origin.

\begin{remark}
The possible reflections from targets within the ISAC network at the MU are inherently part of the multipath signals received and are already accounted for in the fading channel gains. 
However, the impact of introducing sensing on communication in the unified ISAC signal approach appears in key system parameters such as power, beamwidth, and BS density, where the requirements for sensing and communication may conflict (i.e., enhancing one can compromise the other).
\end{remark}



\subsection{The Impact of Misalignment Error}


Reducing the beamwidth of BS beams improves sensing and communication. Narrower beams enhance antenna gain, boosting communication coverage and rate. For sensing, they improve angular resolution, enable sensing of more targets, and reduce clutter and interference by shrinking the resolution cell area. However, excessively narrow beams make the system prone to misalignment errors, drastically reducing coverage and rate. Thus, it is crucial to study misalignment effects in ISAC networks and determine an optimal beamwidth that balances improved sensing with reliable communication performance.
In a realistic scenario, a misalignment error angle \(\theta_m\) exists between the BS and the MU, caused by various sources of uncertainty. This angle can be modeled as a zero-mean truncated Gaussian RV \cite{bahadori2019device}, with 
$\theta_m \sim N_t(0, a^2, -\theta_M, \theta_M),$
where \(a^2\) denotes the variance and \(\theta_M\) is the maximum error angle, satisfying \(|\theta_M| < \pi\). To proceed with the analysis, the PDF of the antenna gain \(G(\theta_m)\) must be derived instead of the PDF of \(\theta_m\).

\begin{lemma} \label{lemma_gain_mis}
The PDF of the antenna gain with a truncated Gaussian misalignment error  $\theta_m \sim N_t(0,a^2,- \theta_M ,\theta_M )$  is 
\begin{equation}\label{pdf_gain_ant}
 f\left(G_c\right)= \frac{\sqrt{\frac{2}{d^2 a^2}}\exp{\left(-\frac{2}{d^2 a^2}\;\arccos^2\left(\sqrt{\frac{G_c}{G_m}}\right)\right)}}{\erf{\left(\frac{\theta_M}{\sqrt{2 a^2}}\right)} \sqrt{\pi G_c \left(G_m - G_c\right) }}
 \end{equation}
where 
  \[
  \begin{aligned}[b]
 &G_c \in \left[G_m \; \cos^2 \left(\frac{d\theta_M}{2}\right),G_m\right] \;\;\; \text{for}\;\;\; |\theta_M|< \frac{\pi}{d}\\
&G_c \in \left[0,G_m\right]  \;\;\; \text{for}\;\;\; \frac{\pi}{d} \leq |\theta_M| < \pi 
\end{aligned}
\]
\begin{IEEEproof}
The proof follows starting from the definition of the cumulative distribution function (CDF) of the gain, performing a transformation of a RV based on the relationship between \(\theta_m\) and \(G(\theta_m)\) given in (\ref{ant_be_ga}), substituting the PDF of \(\theta_m\) (a truncated Gaussian), and then taking the derivative.
\end{IEEEproof}
\end{lemma}



 

\subsection{Communication Coverage Probability}


We begin by calculating the LT of the interference.
\begin{lemma} \label{lemma_LT_comm}
The LT of the aggregate LOS and NLOS interference seen by the typical MU located at the origin is given by (\ref{LOS_int_com}) and (\ref{NLOS_int_com}) respectively.
\begin{equation}\label{LOS_int_com}
\begin{aligned}
\mathcal{L}_{I_{L_c}} (s)&=\exp\left(- \lambda_{BS} \int_{-\frac{\pi}{d}}^{\frac{\pi}{d}}\int_{R_o}^{\infty}\bold p_{\bold{LOS}}\left(r\right) \right.\\
&  \left. \left(1-\left( 1+\frac{s P_c G (\theta_i)C_L r^{-\eta_L}}{m_L}\right)^{-m_L}    \right) r\;dr \;d\theta_i \right)
\end{aligned}
\end{equation}
\begin{equation}\label{NLOS_int_com}
\begin{aligned}
\mathcal{L}_{I_{N_c}} (s)&=\exp\left(- \lambda_{BS} \int_{-\frac{\pi}{d}}^{\frac{\pi}{d}}\int_{R_o}^{\infty}\left(1-\bold p_{\bold{LOS}}\left(r\right)\right) \right.\\
& \left. \left(1-\left( 1+\frac{s P_c G (\theta_i)C_N r^{-\eta_N}}{m_N}\right)^{-m_N}    \right) r\;dr \;d\theta_i \right)
\end{aligned}
\end{equation}
where $\bold p_{\bold{LOS}}\left(r\right)$  is given in (\ref{los_prop}).
\begin{IEEEproof}
See Appendix G.
\end{IEEEproof}
\end{lemma}
Utilizing the results of Lemma \ref{lemma_gain_mis} and Lemma \ref{lemma_LT_comm}, the coverage probability is derived in the following theorem.


\begin{theorem} \label{com_cov_prob}
The average communication coverage probability with a predefined SINR threshold $\phi_c$ is given as follows:
\begin{equation}
\begin{aligned}[b]
\mathcal{P}_{\text{com}}&= \int_{0}^{\infty} 2 \pi \lambda_{BS} R_o e^{-\pi \lambda_{BS} R_o^2} \sum_{n=1}^{m_L} \left(-1\right)^{n+1} {m_L \choose n}\\
&\int \exp\left(-\;\frac{k_L\;n \;\phi_c \; R_o^{\eta_L}\; K_B T W_b}{P_c G_c C_L}\right) \mathcal{L}_{I_{L_c}} \left(\frac{k_L\;n \;\phi_c \; R_o^{\eta_L}}{P_c G_c C_L}\right)\\
&\mathcal{L}_{I_{N_c}} \left(\frac{k_L\;n \;\phi_c \; R_o^{\eta_L}}{P_c G_c C_L}\right)f\left(G_c\right) dG_c\; dR_o.
\end{aligned}
\end{equation}
where \(k_L = m_L(m_L!)^{-\frac{1}{m_L}}\), while \(\mathcal{L}_{I_{L_c}}\) and \(\mathcal{L}_{I_{N_c}}\) are defined in (\ref{LOS_int_com}) and (\ref{NLOS_int_com}), respectively. Additionally, \(f(G_c)\) and its integration limits are provided in Lemma \ref{lemma_gain_mis}.
\begin{IEEEproof}
See Appendix H.
\end{IEEEproof}
\end{theorem}



\subsection{Communication Rate}

Similar to calculating the sensing rate, the average communication rate per cell is calculated in the following theorem.
\begin{theorem} \label{comm_rate}
The average communication  rate is given as follows:
\small
\begin{equation}
\mathcal{R}_{c_{\text{avg}}}=\frac{M(T_t - T_s)}{T_t} \int_0^\infty \mathcal{P}_{\text{com}}(t)\; dt 
\end{equation}
\normalsize
where $\mathcal{P}_{\text{com}}(t)$ is given by replacing $\phi_c$ by $( e^t - 1)$ in Theorem \ref{com_cov_prob}, where the term \(\frac{T_t - T_s}{T_t}\) is to account for the actual communication duration, and the multiplication by $M$ since a BS is serving $M$ users over $M$ beams simultaneously.
\begin{IEEEproof}
Similar to the proof of Theorem \ref{ses_rate} 
\end{IEEEproof}
\end{theorem}







\section{Numerical and Simulation Results}\label{num_ress} 

In this section, we first validate the derived formulas using Monte Carlo simulations and examine the influence of various parameters on system performance. We then present system-level design insights, highlighting the coordination and integration gains in large-scale ISAC networks.
 In each simulation round, two PPP realizations are generated for BSs and clutter scatterers, while MUs and targets are randomly placed within the Voronoi cell of the typical BS. The actual distances between the target and the $N$ nearest cooperative BSs are computed, along with the corresponding bistatic angles. The RCSs of targets and clutter, as well as channel fading gains, are sampled from their respective PDFs. Various types of interference are computed from all BSs based on the actual geometry. The communication SINR is evaluated at the MU, while the sensing SINRs are computed at the $N$ cooperative BSs, with the highest among them selected as the sensing SINR for this realization. Coverage probabilities and rates are obtained by averaging over $2 \times 10^5$ simulation runs. The simulation area is 20 km$^2$,
and the operating frequency lies within the 28 GHz band. Moreover, \( T_t \) is calculated such that the maximum unambiguous range is \( 3R_{\text{eff}} \) \cite{olson2023coverage}, where \( R_{\text{eff}} = \sqrt{\frac{1}{\pi \lambda_{\text{BS}}}} \) represents the effective radius of the Voronoi cell.
Unless stated otherwise, all numerical values are taken from Table~\ref{tab:system_parameters}.


\begin{table} 
  \begin{center}
     \caption{ Numerical Parameters}
    \begin{tabular}{l@{}cl}
      \hline
      \textbf{Parameter Description} & \textbf{Symbol} & \textbf{Value} \\
      \hline
Number of antenna beams         & $M$                     & $12$                            \\ 
3-dB beamwidth and   spread parameter            & $\theta_B$,  $d$           & $30^\circ $, 6 \\ 
 Antenna beam maximum gain               & $G_m$                   & $10 \, \text{dBi}$              \\ 
Path loss intercepts                  & $C_L$, $C_N$                      & $-61.4 \, \text{dB}$, $-72 \, \text{dB}$     \cite{yu2017coverage}        \\ 
Path loss exponents       & $\eta_L$   , $\eta_N$                & $2$ , $4$          \cite{yu2017coverage}                     \\ 
Nakagami fading parameters   & $m_L$, $m_N$                   & $3$, $2$     \cite{yu2017coverage}                          \\ 
LoS blockage parameter        & $\gamma$                & $0.0149$      \cite{rebato2019stochastic}                  \\ 
Bandwidth           & $W_b$                      & $208$ MHz    \cite{ram2022estimation}                   \\ 
Sensing pulse duration           & $T_s=\frac{1}{W_b}  $                    & $4.8$ ns                         \\ 
Clutter density                 & $\lambda_{\text{cl}}$   & $0.01\; \text{m}^{-2}$           \cite{ram2022estimation}               \\ 
Number of collaborative BSs       & $N$    &  4\\
Average RCS of target and clutter           & $\sigma_{\text{avt}}$, $\sigma_{\text{av}_{cl}}$   & $1 \; \text{m}^2$     \cite{xiao2022waveform,ram2022estimation}                         \\ 
Transmission powers & $P_s$, $P_c$                   & $0.9 \, \text{W}$, $0.1 \, \text{W}$    \cite{xiao2022waveform}           \\ 
Temperature                     & $T$                     & $300 \, \text{K}$               \\ 
Time slot duration            & $T_t$               &        $0.7134\,\mu\text{s}$    \\ 
Fraction of power remaining after SIC  & $\zeta$                 & $10^{-12}$                      \\ 
 Base station density            & $\lambda_{\text{BS}}$   & $250 \, \text{BS/km}^2$       \\ 
SINR thresholds            &   $\phi_s$,  $\phi_c$                       &     0 dB                           \\
Variance and maximum  error angle &  $a^2$, $\theta_M$  & 1, 0.2 $\pi$    \cite{bahadori2019device} \\ 
 \end{tabular}
\label{tab:system_parameters}
  \end{center}
\end{table}


\begin{figure*}
\centering
\subfloat[\label{eff_sources_mono}]{%
  % This file was created by matlab2tikz.
%
%The latest updates can be retrieved from
%  http://www.mathworks.com/matlabcentral/fileexchange/22022-matlab2tikz-matlab2tikz
%where you can also make suggestions and rate matlab2tikz.
%
\definecolor{mycolor1}{rgb}{0.00000,0.44700,0.74100}%
\definecolor{mycolor2}{rgb}{0.85000,0.32500,0.09800}%
\definecolor{mycolor3}{rgb}{0.92900,0.69400,0.12500}%
\definecolor{mycolor4}{rgb}{0.49400,0.18400,0.55600}%
\definecolor{mycolor5}{rgb}{0.46600,0.67400,0.18800}%
\definecolor{mycolor6}{rgb}{0.30100,0.74500,0.93300}%
\definecolor{mycolor7}{rgb}{0.63500,0.07800,0.18400}%
%
\begin{tikzpicture}[scale=0.33, transform shape,font=\Large]

\begin{axis}[%
width=4.521in,
height=3.566in,
at={(0.758in,0.481in)},
scale only axis,
xmin=-10,
xmax=10,
xlabel style={font=\large\color{white!15!black}},
xlabel={SINR Threshold (dB)},
ymin=0.1,
ymax=1,
ylabel style={font=\large\color{white!15!black}},
ylabel={Sensing Coverage Probability},
axis background/.style={fill=white},
xmajorgrids,
ymajorgrids,
legend style={at={(0.003,0.451)}, anchor=south west, legend cell align=left, align=left, draw=white!15!black,font=\large}
]
% \addplot [line width=0.6mm, color=mycolor1]
%   table[row sep=crcr]{%
% -10	0.9974\\
% -9	0.9967\\
% -8	0.9958\\
% -7	0.9947\\
% -6	0.9934\\
% -5	0.9916\\
% -4	0.9895\\
% -3	0.9868\\
% -2	0.9834\\
% -1	0.9792\\
% 0	0.9738\\
% 1	0.9672\\
% 2	0.9588\\
% 3	0.9485\\
% 4	0.9356\\
% 5	0.9196\\
% 6	0.8998\\
% 7	0.8755\\
% 8	0.8459\\
% 9	0.8101\\
% 10	0.7671\\
% };
% \addlegendentry{Noise effect only}

\addplot [line width=0.6mm, color=mycolor1]
  table[row sep=crcr]{%
-10	0.9685\\
-9	0.9659\\
-8	0.9631\\
-7	0.9604\\
-6	0.9575\\
-5	0.9544\\
-4	0.9511\\
-3	0.9474\\
-2	0.9433\\
-1	0.9385\\
0	0.9328\\
1	0.926\\
2	0.9176\\
3	0.9074\\
4	0.8949\\
5	0.8794\\
6	0.8604\\
7	0.8371\\
8	0.8087\\
9	0.7743\\
10	0.7331\\
};
\addlegendentry{Direct interference + noise}

\addplot [line width=0.6mm, color=mycolor2]
  table[row sep=crcr]{%
-10	0.9619\\
-9	0.9577\\
-8	0.9533\\
-7	0.9484\\
-6	0.9431\\
-5	0.9373\\
-4	0.9309\\
-3	0.9239\\
-2	0.9162\\
-1	0.9077\\
0	0.8983\\
1	0.8879\\
2	0.8762\\
3	0.863\\
4	0.8479\\
5	0.8305\\
6	0.8101\\
7	0.7861\\
8	0.7577\\
9	0.7241\\
10	0.6845\\
};
\addlegendentry{Direct + intra-clutter + noise}

\addplot [line width=0.6mm, color=mycolor3]
  table[row sep=crcr]{%
-10	0.9601\\
-9	0.9555\\
-8	0.9505\\
-7	0.9449\\
-6	0.9387\\
-5	0.9319\\
-4	0.9242\\
-3	0.9157\\
-2	0.9061\\
-1	0.8953\\
0	0.8832\\
1	0.8696\\
2	0.8542\\
3	0.8367\\
4	0.8167\\
5	0.7939\\
6	0.7676\\
7	0.7374\\
8	0.7027\\
9	0.663\\
10	0.6178\\
};
\addlegendentry{Direct + intra-clutter + inter-clutter + noise}




\addplot [color=white, draw=none, mark=x, thick, mark size=4pt, mark options={ black}]
  table[row sep=crcr]{%
-10	0.9613\\
-9	0.9569\\
-8	0.9521\\
-7	0.9461\\
-6	0.9399\\
-5	0.9323\\
-4	0.924\\
-3	0.915\\
-2	0.9052\\
-1	0.8942\\
0	0.8811\\
1	0.8664\\
2	0.8504\\
3	0.832\\
4	0.8114\\
5	0.7879\\
6	0.7613\\
7	0.7308\\
8	0.6968\\
9	0.6566\\
10	0.611\\
};
\addlegendentry{Simulations}






\addplot [line width=0.6mm, color=mycolor1]
  table[row sep=crcr]{%
-10	0.4889\\
-9	0.4849\\
-8	0.4803\\
-7	0.4749\\
-6	0.4687\\
-5	0.4614\\
-4	0.453\\
-3	0.4432\\
-2	0.4318\\
-1	0.4187\\
0	0.4037\\
1	0.3866\\
2	0.3673\\
3	0.3457\\
4	0.322\\
5	0.2961\\
6	0.2683\\
7	0.2391\\
8	0.2091\\
9	0.1788\\
10	0.1492\\
};


\addplot [line width=0.6mm, color=mycolor2]
  table[row sep=crcr]{%
-10	0.4364\\
-9	0.4277\\
-8	0.4183\\
-7	0.4081\\
-6	0.3972\\
-5	0.3856\\
-4	0.3732\\
-3	0.3601\\
-2	0.3464\\
-1	0.332\\
0	0.317\\
1	0.3013\\
2	0.2848\\
3	0.2675\\
4	0.2492\\
5	0.23\\
6	0.2096\\
7	0.1883\\
8	0.1663\\
9	0.1438\\
10	0.1214\\
};


\addplot [line width=0.6mm, color=mycolor3]
  table[row sep=crcr]{%
-10	0.433\\
-9	0.424\\
-8	0.4142\\
-7	0.4036\\
-6	0.3922\\
-5	0.38\\
-4	0.3669\\
-3	0.3531\\
-2	0.3386\\
-1	0.3233\\
0	0.3072\\
1	0.2905\\
2	0.2729\\
3	0.2545\\
4	0.2351\\
5	0.2149\\
6	0.1939\\
7	0.1721\\
8	0.15\\
9	0.1279\\
10	0.1064\\
};



\addplot [color=white, draw=none, mark=x, thick, mark size=4pt, mark options={ black}]
  table[row sep=crcr]{%
-10	0.433\\
-9	0.424\\
-8	0.4142\\
-7	0.4036\\
-6	0.3922\\
-5	0.38\\
-4	0.3669\\
-3	0.3531\\
-2	0.3386\\
-1	0.3233\\
0	0.3072\\
1	0.2905\\
2	0.2729\\
3	0.2545\\
4	0.2351\\
5	0.2149\\
6	0.1939\\
7	0.1721\\
8	0.15\\
9	0.1279\\
10	0.1064\\
};


\definecolor{darkgreen}{rgb}{0.00000,0.39200,0.00000} % Define dark green

\draw [stealth-,thick, darkgreen] (axis cs:-1.8,0.92) -- (axis cs:-1.5,0.83) node[below]{Monostatic};
\draw [thick, darkgreen] (-2,0.92) ellipse (0.5cm and 0.5cm); % Ellipse position unchanged

\draw [stealth-,thick, darkgreen] (axis cs:-4.5,0.42) -- (axis cs:-3.8,0.22) node[below]{Bistatic}; % Slightly lowered arrow
\draw [thick, darkgreen] (axis cs:-4.5,0.42) ellipse (0.7cm and 0.7cm); % Slightly lowered ellipse






\end{axis}



\begin{axis}[%
width=5.833in,
height=4.375in,
at={(0in,0in)},
scale only axis,
xmin=0,
xmax=1,
ymin=0,
ymax=1,
axis line style={draw=none},
ticks=none,
axis x line*=bottom,
axis y line*=left
]
\end{axis}
\end{tikzpicture}%
}%
\subfloat[ \label{dist_effect}]{%
   % This file was created by matlab2tikz.
%
%The latest updates can be retrieved from
%  http://www.mathworks.com/matlabcentral/fileexchange/22022-matlab2tikz-matlab2tikz
%where you can also make suggestions and rate matlab2tikz.
%



% Define custom colors
\definecolor{mycolor1}{rgb}{0.00000,0.44700,0.74100}%
\definecolor{mycolor2}{rgb}{0.85000,0.32500,0.09800}%
\definecolor{mycolor3}{rgb}{0.92900,0.69400,0.12500}%
\definecolor{mycolor4}{rgb}{0.49400,0.18400,0.55600}%




\begin{tikzpicture}[scale=0.33, transform shape,font=\Large]

% Main plot
\begin{axis}[%
width=4.521in,
height=3.566in,
at={(0.758in,0.481in)},
scale only axis,
xmin=-10,
xmax=5,
xlabel style={font=\large\color{white!15!black}},
xlabel={SINR Threshold (dB)},
ylabel style={font=\large\color{white!15!black}},
ylabel={Sensing Coverage Probability},
ymin=0.3,
ymax=1,
axis background/.style={fill=white},
xmajorgrids,
ymajorgrids,
legend style={at={(0.05,0.1)}, anchor=south west, legend cell align=left, align=left, draw=white!15!black}
]

% R1 = 5 plots
\addplot [line width=0.6mm, color=mycolor1] table[row sep=crcr]{%
-10 0.99706\\ -9 0.99638\\ -8 0.99564\\ -7 0.99448\\ -6 0.99308\\ -5 0.9915\\ -4 0.98988\\ -3 0.98778\\ -2 0.98594\\ -1 0.9834\\ 0 0.98098\\ 1 0.97814\\ 2 0.97518\\ 3 0.97222\\ 4 0.96892\\ 5 0.965\\ 6 0.96098\\ 7 0.95676\\ 8 0.9523\\ 9 0.94752\\ 10 0.94226\\};
\addlegendentry{Monostatic}

\addplot [line width=0.6mm, color=mycolor2] table[row sep=crcr]{%
-10 0.9986\\ -9 0.99838\\ -8 0.99798\\ -7 0.99732\\ -6 0.99654\\ -5 0.99574\\ -4 0.99472\\ -3 0.99336\\ -2 0.99224\\ -1 0.99082\\ 0 0.98936\\ 1 0.98762\\ 2 0.9857\\ 3 0.98372\\ 4 0.98134\\ 5 0.97872\\ 6 0.97598\\ 7 0.97296\\ 8 0.97008\\ 9 0.96656\\ 10 0.96242\\};
\addlegendentry{Dual-mode Networked Sensing 2}

\addplot [line width=0.6mm, color=mycolor3] table[row sep=crcr]{%
-10 0.99912\\ -9 0.99896\\ -8 0.99876\\ -7 0.99834\\ -6 0.99774\\ -5 0.99714\\ -4 0.9963\\ -3 0.99538\\ -2 0.99456\\ -1 0.99346\\ 0 0.99228\\ 1 0.99088\\ 2 0.9895\\ 3 0.98782\\ 4 0.9857\\ 5 0.98362\\ 6 0.98144\\ 7 0.97926\\ 8 0.97694\\ 9 0.97408\\ 10 0.97042\\};
\addlegendentry{Dual-mode Networked Sensing 3}

\addplot [line width=0.6mm, color=mycolor4] table[row sep=crcr]{%
-10 0.99932\\ -9 0.99918\\ -8 0.99904\\ -7 0.9987\\ -6 0.99826\\ -5 0.99784\\ -4 0.99726\\ -3 0.99652\\ -2 0.99578\\ -1 0.99488\\ 0 0.99386\\ 1 0.99264\\ 2 0.99134\\ 3 0.98992\\ 4 0.98818\\ 5 0.98636\\ 6 0.98456\\ 7 0.98276\\ 8 0.98068\\ 9 0.97796\\ 10 0.97456\\};
\addlegendentry{Dual-mode Networked Sensing 4}


\addplot [color=white, draw=none, mark=x, thick, mark size=4pt, mark options={ black}]
  table[row sep=crcr]{%
-10	0.919558676028084\\
-9	0.908757147156184\\
-8	0.896074691296055\\
-7	0.879919759277834\\
-6	0.86029455966336\\
-5	0.838031409422827\\
-4	0.809751223898491\\
-3	0.776151545363908\\
-2	0.737014925373134\\
-1	0.695079033701163\\
0	0.644942928039702\\
1	0.590033750248164\\
2	0.52757665677547\\
3	0.461635842152112\\
4	0.389866929521932\\
5	0.317849758787043\\
};
\addlegendentry{Simulations}


\addplot [color=white, draw=none, mark=x, thick, mark size=4pt, mark options={ black}]
  table[row sep=crcr]{%
-10	0.949468405215647\\
-9	0.94103721536764\\
-8	0.931492822005823\\
-7	0.918655967903711\\
-6	0.902354473499649\\
-5	0.882404721416425\\
-4	0.858187631131981\\
-3	0.82827517447657\\
-2	0.791283582089552\\
-1	0.751406700467243\\
0	0.702133995037221\\
1	0.646615048640063\\
2	0.581048466864491\\
3	0.510374839283948\\
4	0.43181862986693\\
5	0.352033080634045\\
6	0.27323971324757\\
7	0.199097242665097\\
8	0.135008801095247\\
9	0.0851357928550569\\
10	0.0465179437439379\\
};



\addplot [color=white, draw=none, mark=x, thick, mark size=4pt, mark options={ black}]
  table[row sep=crcr]{%
-10	0.9641925777332\\
-9	0.957648711004113\\
-8	0.94984439313322\\
-7	0.938455366098295\\
-6	0.924055705841098\\
-5	0.906011803541062\\
-4	0.883784593865521\\
-3	0.856131605184447\\
-2	0.821034825870647\\
-1	0.781986280942439\\
0	0.732248138957816\\
1	0.67564026206075\\
2	0.608565776458951\\
3	0.534150924735437\\
4	0.45200591424347\\
5	0.367726690952053\\
6	0.284533045271531\\
7	0.206221175547051\\
8	0.139409348718952\\
9	0.0872773289204711\\
10	0.0475848690591659\\
};


\addplot [color=white, draw=none, mark=x, thick, mark size=4pt, mark options={ black}]
  table[row sep=crcr]{%
-10	0.973119358074223\\
-9	0.967559434246163\\
-8	0.960546129906636\\
-7	0.950351053159478\\
-6	0.936900110209398\\
-5	0.91999599879964\\
-4	0.89897092616645\\
-3	0.872502492522433\\
-2	0.838009950248756\\
-1	0.799463167312854\\
0	0.74941935483871\\
1	0.691522731784793\\
2	0.622373887240356\\
3	0.545781821778261\\
4	0.460877279448004\\
5	0.374401890321945\\
6	0.288637925955023\\
7	0.208595819841036\\
8	0.140641502053589\\
9	0.0879003212304098\\
10	0.0478176527643065\\
};


% R1 = 35 plots
\addplot [line width=0.6mm, color=mycolor1] table[row sep=crcr]{%
-10 0.9168\\ -9 0.90594\\ -8 0.89258\\ -7 0.87728\\ -6 0.85866\\ -5 0.83778\\ -4 0.81048\\ -3 0.77848\\ -2 0.7407\\ -1 0.69918\\ 0 0.64978\\ 1 0.5944\\ 2 0.53338\\ 3 0.46676\\ 4 0.39552\\ 5 0.32284\\ 6 0.25172\\ 7 0.18484\\ 8 0.12672\\ 9 0.08074\\ 10 0.04488\\};


\addplot [line width=0.6mm, color=mycolor2] table[row sep=crcr]{%
-10 0.94662\\ -9 0.93812\\ -8 0.92786\\ -7 0.9159\\ -6 0.90064\\ -5 0.88214\\ -4 0.85896\\ -3 0.83076\\ -2 0.79524\\ -1 0.75584\\ 0 0.7074\\ 1 0.6514\\ 2 0.58744\\ 3 0.51604\\ 4 0.43808\\ 5 0.35756\\ 6 0.27824\\ 7 0.2029\\ 8 0.13806\\ 9 0.08746\\ 10 0.04796\\};


\addplot [line width=0.6mm, color=mycolor3] table[row sep=crcr]{%
-10 0.9613\\ -9 0.95468\\ -8 0.94614\\ -7 0.93564\\ -6 0.9223\\ -5 0.90574\\ -4 0.88458\\ -3 0.8587\\ -2 0.82514\\ -1 0.7866\\ 0 0.73774\\ 1 0.68064\\ 2 0.61526\\ 3 0.54008\\ 4 0.45856\\ 5 0.3735\\ 6 0.28974\\ 7 0.21016\\ 8 0.14256\\ 9 0.08966\\ 10 0.04906\\};


\addplot [line width=0.6mm, color=mycolor4] table[row sep=crcr]{%
-10 0.9702\\ -9 0.96456\\ -8 0.9568\\ -7 0.9475\\ -6 0.93512\\ -5 0.91972\\ -4 0.89978\\ -3 0.87512\\ -2 0.8422\\ -1 0.80418\\ 0 0.75504\\ 1 0.69664\\ 2 0.62922\\ 3 0.55184\\ 4 0.46756\\ 5 0.38028\\ 6 0.29392\\ 7 0.21258\\ 8 0.14382\\ 9 0.0903\\ 10 0.0493\\};











\definecolor{darkgreen}{rgb}{0.00000,0.39200,0.00000} % Define dark green

\draw [stealth-,thick, darkgreen] (axis cs:-1.8,0.99) -- (axis cs:-1.5,0.92) node[below]{$R_1=5$ m};
\draw [thick, darkgreen] (-2,0.99) ellipse (0.5cm and 0.2cm); % Ellipse position unchanged

\draw [stealth-,thick, darkgreen] (axis cs:-4.5,0.87) -- (axis cs:-3.8,0.7) node[below]{$R_1=35$ m}; % Arrow adjusted
\draw [thick, darkgreen] (-4.5,0.87) ellipse (0.7cm and 0.7cm); % Ellipse adjusted








% Highlight zoom region
\draw [dashed, thick] (axis cs:-1,0.97) rectangle (axis cs:1,1);

% Arrow from the middle of the bottom side of the zoom region to above the inset
\draw[->, thick] (axis cs:0,0.97) -- (3.7in,4.55in);

\end{axis}

% Zoom-in inset
\begin{axis}[%
width=1.2in,
height=0.7in,
at={(4in,3in)}, % Position inset plot
scale only axis,
xmin=-1,
xmax=1,
ymin=0.98,
ymax=1,
ytick={0.98, 0.99, 1}, % Specify y-axis tick values
axis background/.style={fill=white},
xmajorgrids,
ymajorgrids
]



% Zoomed-in plots (only R1 = 5)
\addplot [line width=0.6mm, color=mycolor1] table[row sep=crcr]{-1 0.9834\\ 0 0.98098\\ 1 0.97814\\};
\addplot [line width=0.6mm, color=mycolor2] table[row sep=crcr]{-1 0.99082\\ 0 0.98936\\ 1 0.98762\\};
\addplot [line width=0.6mm, color=mycolor3] table[row sep=crcr]{-1 0.99346\\ 0 0.99228\\ 1 0.99088\\};
\addplot [line width=0.6mm, color=mycolor4] table[row sep=crcr]{-1 0.99488\\ 0 0.99386\\ 1 0.99264\\};



\end{axis}

\end{tikzpicture}



}%
   \subfloat[  \label{beam_tradeoff}]{%
   % This file was created by matlab2tikz.
%
%The latest updates can be retrieved from
%  http://www.mathworks.com/matlabcentral/fileexchange/22022-matlab2tikz-matlab2tikz
%where you can also make suggestions and rate matlab2tikz.
%
\definecolor{mycolor1}{rgb}{0.00000,0.44700,0.74100}%
\definecolor{mycolor2}{rgb}{0.85000,0.32500,0.09800}%
\definecolor{mycolor3}{rgb}{0.92900,0.69400,0.12500}%
%
\begin{tikzpicture}[scale=0.33, transform shape,font=\Large]

\begin{axis}[%
width=4.521in,
height=3.477in,
at={(0.758in,0.57in)},
scale only axis,
xmin=1,
xmax=180,
xlabel style={font=\large\color{white!15!black}},
xlabel={3-dB Beamwidth (degree)},
ymin=0,
ymax=1,
ylabel style={font=\large\color{white!15!black}},
ylabel={Average Coverage Probability},
axis background/.style={fill=white},
xmajorgrids,
ymajorgrids,
legend style={at={(0.116,0.037)}, anchor=south west, legend cell align=left, align=left, draw=white!15!black}
]

\addplot [line width=0.6mm, color=mycolor1]
  table[row sep=crcr]{%
1.0000 0.02880 \\
1.0358 0.02974 \\
1.0716 0.03081 \\
1.1074 0.03179 \\
1.1432 0.03280 \\
1.1790 0.03386 \\
1.2148 0.03486 \\
1.2507 0.03592 \\
1.2865 0.03692 \\
1.3223 0.03796 \\
1.3581 0.03905 \\
1.3939 0.04006 \\
1.4297 0.04103 \\
1.4655 0.04205 \\
1.5013 0.04304 \\
1.5371 0.04407 \\
1.5729 0.04512 \\
1.6087 0.04614 \\
1.6445 0.04717 \\
1.6803 0.04814 \\
1.7161 0.04915 \\
1.7520 0.05023 \\
1.7878 0.05126 \\
1.8236 0.05224 \\
1.8594 0.05330 \\
1.8952 0.05431 \\
1.9310 0.05526 \\
1.9668 0.05628 \\
2.0026 0.05737 \\
2.0384 0.05839 \\
2.0742 0.05934 \\
2.1100 0.06039 \\
2.1458 0.06138 \\
2.1816 0.06239 \\
2.2174 0.06346 \\
2.2533 0.06449 \\
2.2891 0.06549 \\
2.3249 0.06645 \\
2.3607 0.06747 \\
2.3965 0.06850 \\
2.4323 0.06950 \\
2.4681 0.07057 \\
2.5039 0.07151 \\
2.5397 0.07253 \\
2.5755 0.07352 \\
2.6113 0.07458 \\
2.6471 0.07560 \\
2.6829 0.07660 \\
2.7187 0.07757 \\
2.7546 0.07858 \\
2.7904 0.07959 \\
2.8262 0.08059 \\
2.8620 0.08164 \\
2.8978 0.08265 \\
2.9336 0.08363 \\
2.9694 0.08463 \\
3.0052 0.08565 \\
3.0410 0.08663 \\
3.0768 0.08763 \\
3.1126 0.08867 \\
3.1484 0.08973 \\
3.1842 0.09070 \\
3.2200 0.09166 \\
3.2559 0.09270 \\
3.2917 0.09375 \\
3.3275 0.09474 \\
3.3633 0.09570 \\
3.3991 0.09668 \\
3.4349 0.09772 \\
3.4707 0.09876 \\
3.5065 0.09977 \\
3.5423 0.10075 \\
3.5781 0.10170 \\
3.6139 0.10269 \\
3.6497 0.10372 \\
3.6855 0.10474 \\
3.7213 0.10571 \\
3.7572 0.10670 \\
3.7930 0.10773 \\
3.8288 0.10877 \\
3.8646 0.10977 \\
3.9004 0.11075 \\
3.9362 0.11174 \\
3.9720 0.11273 \\
4.0078 0.11371 \\
4.0436 0.11469 \\
4.0794 0.11568 \\
4.1152 0.11668 \\
4.1510 0.11770 \\
4.1868 0.11872 \\
4.2226 0.11973 \\
4.2585 0.12074 \\
4.2943 0.12174 \\
4.3301 0.12274 \\
4.3659 0.12373 \\
4.4017 0.12471 \\
4.4375 0.12569 \\
4.4733 0.12667 \\
4.5091 0.12765 \\
4.5449 0.12864 \\
4.5807 0.12963 \\
4.6165 0.13063 \\
4.6523 0.13164 \\
4.6881 0.13264 \\
4.7239 0.13364 \\
4.7598 0.13463 \\
4.7956 0.13561 \\
4.8314 0.13659 \\
4.8672 0.13756 \\
4.9030 0.13854 \\
4.9388 0.13951 \\
4.9746 0.14050 \\
5.0104 0.14149 \\
5.0462 0.14250 \\
5.0820 0.14350 \\
5.1178 0.14451 \\
5.1536 0.14550 \\
5.1894 0.14647 \\
5.2252 0.14743 \\
5.2611 0.14840 \\
5.2969 0.14938 \\
5.3327 0.15037 \\
5.3685 0.15138 \\
5.4043 0.15239 \\
5.4401 0.15340 \\
5.4759 0.15439 \\
5.5117 0.15536 \\
5.5475 0.15632 \\
5.5833 0.15728 \\
5.6191 0.15824 \\
5.6549 0.15922 \\
5.6907 0.16020 \\
5.7265 0.16119 \\
5.7624 0.16218 \\
5.7982 0.16317 \\
5.8340 0.16416 \\
5.8698 0.16514 \\
5.9056 0.16612 \\
5.9414 0.16709 \\
5.9772 0.16807 \\
6.0130 0.16906 \\
6.0488 0.17004 \\
6.0846 0.17103 \\
6.1204 0.17202 \\
6.1562 0.17301 \\
6.1920 0.17399 \\
6.2278 0.17497 \\
6.2637 0.17594 \\
6.2995 0.17691 \\
6.3353 0.17788 \\
6.3711 0.17884 \\
6.4069 0.17981 \\
6.4427 0.18078 \\
6.4785 0.18176 \\
6.5143 0.18274 \\
6.5501 0.18372 \\
6.5859 0.18470 \\
6.6217 0.18568 \\
6.6575 0.18665 \\
6.6933 0.18762 \\
6.7291 0.18859 \\
6.7650 0.18955 \\
6.8008 0.19051 \\
6.8366 0.19148 \\
6.8724 0.19244 \\
6.9082 0.19340 \\
6.9440 0.19436 \\
6.9798 0.19533 \\
7.0156 0.19630 \\
7.0514 0.19727 \\
7.0872 0.19824 \\
7.1230 0.19921 \\
7.1588 0.20018 \\
7.1946 0.20115 \\
7.2304 0.20212 \\
7.2663 0.20309 \\
7.3021 0.20406 \\
7.3379 0.20503 \\
7.3737 0.20599 \\
7.4095 0.20696 \\
7.4453 0.20792 \\
7.4811 0.20889 \\
7.5169 0.20986 \\
7.5527 0.21082 \\
7.5885 0.21179 \\
7.6243 0.21276 \\
7.6601 0.21373 \\
7.6959 0.21469 \\
7.7317 0.21566 \\
7.7676 0.21663 \\
7.8034 0.21759 \\
7.8392 0.21855 \\
7.8750 0.21951 \\
7.9108 0.22047 \\
7.9466 0.22143 \\
7.9824 0.22238 \\
8.0182 0.22334 \\
8.0540 0.22429 \\
8.0898 0.22525 \\
8.1256 0.22620 \\
8.1614 0.22716 \\
8.1972 0.22811 \\
8.2330 0.22907 \\
8.2689 0.23002 \\
8.3047 0.23098 \\
8.3405 0.23193 \\
8.3763 0.23289 \\
8.4121 0.23385 \\
8.4479 0.23480 \\
8.4837 0.23576 \\
8.5195 0.23671 \\
8.5553 0.23767 \\
8.5911 0.23863 \\
8.6269 0.23958 \\
8.6627 0.24053 \\
8.6985 0.24149 \\
8.7343 0.24244 \\
8.7702 0.24340 \\
8.8060 0.24435 \\
8.8418 0.24530 \\
8.8776 0.24625 \\
8.9134 0.24720 \\
8.9492 0.24815 \\
8.9850 0.24910 \\
9.0208 0.25005 \\
9.0566 0.25100 \\
9.0924 0.25195 \\
9.1282 0.25289 \\
9.1640 0.25384 \\
9.1998 0.25479 \\
9.2356 0.25573 \\
9.2715 0.25668 \\
9.3073 0.25762 \\
9.3431 0.25856 \\
9.3789 0.25951 \\
9.4147 0.26045 \\
9.4505 0.26139 \\
9.4863 0.26233 \\
9.5221 0.26327 \\
9.5579 0.26421 \\
9.5937 0.26515 \\
9.6295 0.26609 \\
9.6653 0.26703 \\
9.7011 0.26797 \\
9.7369 0.26891 \\
9.7728 0.26985 \\
9.8086 0.27078 \\
9.8444 0.27172 \\
9.8802 0.27266 \\
9.9160 0.27360 \\
9.9518 0.27454 \\
9.9876 0.27548 \\
10.0234 0.27641 \\
10.0592 0.27735 \\
10.0950 0.27829 \\
10.1308 0.27923 \\
10.1666 0.28017 \\
10.2024 0.28110 \\
10.2382 0.28204 \\
10.2741 0.28298 \\
10.3099 0.28392 \\
10.3457 0.28486 \\
10.3815 0.28579 \\
10.4173 0.28673 \\
10.4531 0.28767 \\
10.4889 0.28860 \\
10.5247 0.28954 \\
10.5605 0.29048 \\
10.5963 0.29141 \\
10.6321 0.29235 \\
10.6679 0.29328 \\
10.7037 0.29421 \\
10.7395 0.29515 \\
10.7754 0.29608 \\
10.8112 0.29701 \\
10.8470 0.29795 \\
10.8828 0.29888 \\
10.9186 0.29981 \\
10.9544 0.30074 \\
10.9902 0.30167 \\
11.0260 0.30260 \\
11.0618 0.30353 \\
11.0976 0.30445 \\
11.1334 0.30538 \\
11.1692 0.30631 \\
11.2050 0.30724 \\
11.2408 0.30816 \\
11.2767 0.30909 \\
11.3125 0.31001 \\
11.3483 0.31094 \\
11.3841 0.31186 \\
11.4199 0.31279 \\
11.4557 0.31371 \\
11.4915 0.31463 \\
11.5273 0.31556 \\
11.5631 0.31648 \\
11.5989 0.31740 \\
11.6347 0.31832 \\
11.6705 0.31924 \\
11.7063 0.32016 \\
11.7421 0.32108 \\
11.7780 0.32200 \\
11.8138 0.32292 \\
11.8496 0.32384 \\
11.8854 0.32476 \\
11.9212 0.32568 \\
11.9570 0.32660 \\
11.9928 0.32752 \\
12.0286 0.32843 \\
12.0644 0.32935 \\
12.1002 0.33027 \\
12.1360 0.33119 \\
12.1718 0.33210 \\
12.2076 0.33302 \\
12.2434 0.33394 \\
12.2793 0.33485 \\
12.3151 0.33577 \\
12.3509 0.33669 \\
12.3867 0.33760 \\
12.4225 0.33852 \\
12.4583 0.33943 \\
12.4941 0.34035 \\
12.5299 0.34126 \\
12.5657 0.34217 \\
12.6015 0.34309 \\
12.6373 0.34400 \\
12.6731 0.34491 \\
12.7089 0.34583 \\
12.7447 0.34674 \\
12.7806 0.34765 \\
12.8164 0.34856 \\
12.8522 0.34947 \\
12.8880 0.35038 \\
12.9238 0.35129 \\
12.9596 0.35220 \\
12.9954 0.35311 \\
13.0312 0.35402 \\
13.0670 0.35493 \\
13.1028 0.35584 \\
13.1386 0.35675 \\
13.1744 0.35765 \\
13.2102 0.35856 \\
13.2460 0.35947 \\
13.2819 0.36037 \\
13.3177 0.36128 \\
13.3535 0.36218 \\
13.3893 0.36309 \\
13.4251 0.36399 \\
13.4609 0.36490 \\
13.4967 0.36580 \\
13.5325 0.36670 \\
13.5683 0.36761 \\
13.6041 0.36851 \\
13.6399 0.36941 \\
13.6757 0.37031 \\
13.7115 0.37121 \\
13.7473 0.37212 \\
13.7832 0.37302 \\
13.8190 0.37392 \\
13.8548 0.37482 \\
13.8906 0.37572 \\
13.9264 0.37662 \\
13.9622 0.37751 \\
13.9980 0.37841 \\
14.0338 0.37931 \\
14.0696 0.38021 \\
14.1054 0.38111 \\
14.1412 0.38200 \\
14.1770 0.38290 \\
14.2128 0.38380 \\
14.2486 0.38469 \\
14.2845 0.38559 \\
14.3203 0.38648 \\
14.3561 0.38738 \\
14.3919 0.38827 \\
14.4277 0.38917 \\
14.4635 0.39006 \\
14.4993 0.39095 \\
14.5351 0.39185 \\
14.5709 0.39274 \\
14.6067 0.39363 \\
14.6425 0.39452 \\
14.6783 0.39541 \\
14.7141 0.39630 \\
14.7499 0.39720 \\
14.7858 0.39808 \\
14.8216 0.39897 \\
14.8574 0.39986 \\
14.8932 0.40075 \\
14.9290 0.40164 \\
14.9648 0.40253 \\
15.0006 0.40341 \\
15.0364 0.40430 \\
15.0722 0.40519 \\
15.1080 0.40607 \\
15.1438 0.40696 \\
15.1796 0.40784 \\
15.2154 0.40873 \\
15.2513 0.40961 \\
15.2871 0.41050 \\
15.3229 0.41138 \\
15.3587 0.41226 \\
15.3945 0.41315 \\
15.4303 0.41403 \\
15.4661 0.41491 \\
15.5019 0.41579 \\
15.5377 0.41667 \\
15.5735 0.41755 \\
15.6093 0.41843 \\
15.6451 0.41931 \\
15.6809 0.42019 \\
15.7167 0.42107 \\
15.7526 0.42195 \\
15.7884 0.42283 \\
15.8242 0.42371 \\
15.8600 0.42458 \\
15.8958 0.42546 \\
15.9316 0.42634 \\
15.9674 0.42722 \\
16.0032 0.42809 \\
16.0390 0.42897 \\
16.0748 0.42984 \\
16.1106 0.43072 \\
16.1464 0.43160 \\
16.1822 0.43247 \\
16.2180 0.43335 \\
16.2539 0.43422 \\
16.2897 0.43509 \\
16.3255 0.43597 \\
16.3613 0.43684 \\
16.3971 0.43772 \\
16.4329 0.43859 \\
16.4687 0.43946 \\
16.5045 0.44034 \\
16.5403 0.44121 \\
16.5761 0.44208 \\
16.6119 0.44295 \\
16.6477 0.44382 \\
16.6835 0.44470 \\
16.7193 0.44557 \\
16.7552 0.44644 \\
16.7910 0.44731 \\
16.8268 0.44818 \\
16.8626 0.44905 \\
16.8984 0.44992 \\
16.9342 0.45079 \\
16.9700 0.45166 \\
17.0058 0.45253 \\
17.0416 0.45340 \\
17.0774 0.45426 \\
17.1132 0.45513 \\
17.1490 0.45600 \\
17.1848 0.45687 \\
17.2206 0.45773 \\
17.2565 0.45860 \\
17.2923 0.45946 \\
17.3281 0.46033 \\
17.3639 0.46120 \\
17.3997 0.46206 \\
17.4355 0.46293 \\
17.4713 0.46379 \\
17.5071 0.46465 \\
17.5429 0.46552 \\
17.5787 0.46638 \\
17.6145 0.46724 \\
17.6503 0.46810 \\
17.6861 0.46897 \\
17.7219 0.46983 \\
17.7578 0.47069 \\
17.7936 0.47155 \\
17.8294 0.47241 \\
17.8652 0.47327 \\
17.9010 0.47413 \\
17.9368 0.47499 \\
17.9726 0.47584 \\
18.0084 0.47670 \\
18.0442 0.47756 \\
18.0800 0.47842 \\
18.1158 0.47927 \\
18.1516 0.48013 \\
18.1874 0.48098 \\
18.2232 0.48184 \\
18.2591 0.48269 \\
18.2949 0.48355 \\
18.3307 0.48440 \\
18.3665 0.48525 \\
18.4023 0.48610 \\
18.4381 0.48696 \\
18.4739 0.48781 \\
18.5097 0.48866 \\
18.5455 0.48951 \\
18.5813 0.49036 \\
18.6171 0.49121 \\
18.6529 0.49206 \\
18.6887 0.49291 \\
18.7245 0.49376 \\
18.7604 0.49461 \\
18.7962 0.49545 \\
18.8320 0.49630 \\
18.8678 0.49715 \\
18.9036 0.49799 \\
18.9394 0.49884 \\
18.9752 0.49968 \\
19.0110 0.50053 \\
19.0468 0.50137 \\
19.0826 0.50222 \\
19.1184 0.50306 \\
19.1542 0.50390 \\
19.1900 0.50475 \\
19.2258 0.50559 \\
19.2617 0.50643 \\
19.2975 0.50727 \\
19.3333 0.50811 \\
19.3691 0.50896 \\
19.4049 0.50980 \\
19.4407 0.51064 \\
19.4765 0.51147 \\
19.5123 0.51231 \\
19.5481 0.51315 \\
19.5839 0.51399 \\
19.6197 0.51483 \\
19.6555 0.51567 \\
19.6913 0.51650 \\
19.7271 0.51734 \\
19.7630 0.51818 \\
19.7988 0.51901 \\
19.8346 0.51985 \\
19.8704 0.52068 \\
19.9062 0.52152 \\
19.9420 0.52235 \\
19.9778 0.52318 \\
20.0136 0.52402 \\
20.0494 0.52485 \\
20.0852 0.52568 \\
20.1210 0.52651 \\
20.1568 0.52735 \\
20.1926 0.52818 \\
20.2284 0.52901 \\
20.2643 0.52984 \\
20.3001 0.53067 \\
20.3359 0.53150 \\
20.3717 0.53233 \\
20.4075 0.53316 \\
20.4433 0.53399 \\
20.4791 0.53481 \\
20.5149 0.53564 \\
20.5507 0.53647 \\
20.5865 0.53730 \\
20.6223 0.53812 \\
20.6581 0.53895 \\
20.6939 0.53977 \\
20.7297 0.54060 \\
20.7656 0.54143 \\
20.8014 0.54225 \\
20.8372 0.54308 \\
20.8730 0.54390 \\
20.9088 0.54472 \\
20.9446 0.54555 \\
20.9804 0.54637 \\
21.0162 0.54719 \\
21.0520 0.54802 \\
21.0878 0.54884 \\
21.1236 0.54966 \\
21.1594 0.55048 \\
21.1952 0.55130 \\
21.2310 0.55212 \\
21.2669 0.55294 \\
21.3027 0.55376 \\
21.3385 0.55458 \\
21.3743 0.55540 \\
21.4101 0.55622 \\
21.4459 0.55704 \\
21.4817 0.55786 \\
21.5175 0.55868 \\
21.5533 0.55949 \\
21.5891 0.56031 \\
21.6249 0.56113 \\
21.6607 0.56194 \\
21.6965 0.56276 \\
21.7323 0.56358 \\
21.7682 0.56439 \\
21.8040 0.56521 \\
21.8398 0.56602 \\
21.8756 0.56684 \\
21.9114 0.56765 \\
21.9472 0.56847 \\
21.9830 0.56928 \\
22.0188 0.57009 \\
22.0546 0.57091 \\
22.0904 0.57172 \\
22.1262 0.57253 \\
22.1620 0.57335 \\
22.1978 0.57416 \\
22.2336 0.57497 \\
22.2695 0.57578 \\
22.3053 0.57659 \\
22.3411 0.57740 \\
22.3769 0.57821 \\
22.4127 0.57902 \\
22.4485 0.57984 \\
22.4843 0.58064 \\
22.5201 0.58145 \\
22.5559 0.58226 \\
22.5917 0.58307 \\
22.6275 0.58388 \\
22.6633 0.58469 \\
22.6991 0.58550 \\
22.7349 0.58631 \\
22.7708 0.58711 \\
22.8066 0.58792 \\
22.8424 0.58873 \\
22.8782 0.58953 \\
22.9140 0.59034 \\
22.9498 0.59115 \\
22.9856 0.59195 \\
23.0214 0.59276 \\
23.0572 0.59356 \\
23.0930 0.59437 \\
23.1288 0.59517 \\
23.1646 0.59597 \\
23.2004 0.59678 \\
23.2362 0.59758 \\
23.2721 0.59838 \\
23.3079 0.59919 \\
23.3437 0.59999 \\
23.3795 0.60079 \\
23.4153 0.60159 \\
23.4511 0.60239 \\
23.4869 0.60319 \\
23.5227 0.60399 \\
23.5585 0.60479 \\
23.5943 0.60559 \\
23.6301 0.60639 \\
23.6659 0.60719 \\
23.7017 0.60799 \\
23.7375 0.60878 \\
23.7734 0.60958 \\
23.8092 0.61038 \\
23.8450 0.61117 \\
23.8808 0.61197 \\
23.9166 0.61276 \\
23.9524 0.61356 \\
23.9882 0.61435 \\
24.0240 0.61515 \\
24.0598 0.61594 \\
24.0956 0.61673 \\
24.1314 0.61752 \\
24.1672 0.61832 \\
24.2030 0.61911 \\
24.2388 0.61990 \\
24.2747 0.62069 \\
24.3105 0.62148 \\
24.3463 0.62227 \\
24.3821 0.62306 \\
24.4179 0.62385 \\
24.4537 0.62463 \\
24.4895 0.62542 \\
24.5253 0.62621 \\
24.5611 0.62699 \\
24.5969 0.62778 \\
24.6327 0.62857 \\
24.6685 0.62935 \\
24.7043 0.63013 \\
24.7401 0.63092 \\
24.7760 0.63170 \\
24.8118 0.63248 \\
24.8476 0.63326 \\
24.8834 0.63405 \\
24.9192 0.63483 \\
24.9550 0.63561 \\
24.9908 0.63639 \\
25.0266 0.63716 \\
25.0624 0.63794 \\
25.0982 0.63872 \\
25.1340 0.63950 \\
25.1698 0.64027 \\
25.2056 0.64105 \\
25.2414 0.64182 \\
25.2773 0.64260 \\
25.3131 0.64337 \\
25.3489 0.64415 \\
25.3847 0.64492 \\
25.4205 0.64569 \\
25.4563 0.64646 \\
25.4921 0.64723 \\
25.5279 0.64800 \\
25.5637 0.64877 \\
25.5995 0.64954 \\
25.6353 0.65031 \\
25.6711 0.65108 \\
25.7069 0.65184 \\
25.7427 0.65261 \\
25.7786 0.65337 \\
25.8144 0.65414 \\
25.8502 0.65490 \\
25.8860 0.65567 \\
25.9218 0.65643 \\
25.9576 0.65719 \\
25.9934 0.65795 \\
26.0292 0.65871 \\
26.0650 0.65947 \\
26.1008 0.66023 \\
26.1366 0.66099 \\
26.1724 0.66175 \\
26.2082 0.66251 \\
26.2440 0.66327 \\
26.2799 0.66402 \\
26.3157 0.66478 \\
26.3515 0.66553 \\
26.3873 0.66629 \\
26.4231 0.66704 \\
26.4589 0.66780 \\
26.4947 0.66855 \\
26.5305 0.66931 \\
26.5663 0.67006 \\
26.6021 0.67081 \\
26.6379 0.67156 \\
26.6737 0.67231 \\
26.7095 0.67307 \\
26.7453 0.67382 \\
26.7812 0.67457 \\
26.8170 0.67532 \\
26.8528 0.67607 \\
26.8886 0.67682 \\
26.9244 0.67756 \\
26.9602 0.67831 \\
26.9960 0.67906 \\
27.0318 0.67981 \\
27.0676 0.68056 \\
27.1034 0.68130 \\
27.1392 0.68205 \\
27.1750 0.68280 \\
27.2108 0.68354 \\
27.2466 0.68429 \\
27.2825 0.68504 \\
27.3183 0.68578 \\
27.3541 0.68653 \\
27.3899 0.68727 \\
27.4257 0.68802 \\
27.4615 0.68876 \\
27.4973 0.68951 \\
27.5331 0.69025 \\
27.5689 0.69100 \\
27.6047 0.69174 \\
27.6405 0.69248 \\
27.6763 0.69323 \\
27.7121 0.69397 \\
27.7479 0.69471 \\
27.7838 0.69546 \\
27.8196 0.69620 \\
27.8554 0.69695 \\
27.8912 0.69769 \\
27.9270 0.69843 \\
27.9628 0.69917 \\
27.9986 0.69992 \\
28.0344 0.70066 \\
28.0702 0.70140 \\
28.1060 0.70215 \\
28.1418 0.70289 \\
28.1776 0.70363 \\
28.2134 0.70438 \\
28.2492 0.70512 \\
28.2851 0.70586 \\
28.3209 0.70661 \\
28.3567 0.70735 \\
28.3925 0.70809 \\
28.4283 0.70884 \\
28.4641 0.70958 \\
28.4999 0.71032 \\
28.5357 0.71107 \\
28.5715 0.71181 \\
28.6073 0.71255 \\
28.6431 0.71330 \\
28.6789 0.71404 \\
28.7147 0.71479 \\
28.7506 0.71553 \\
28.7864 0.71628 \\
28.8222 0.71702 \\
28.8580 0.71777 \\
28.8938 0.71851 \\
28.9296 0.71926 \\
28.9654 0.72000 \\
29.0012 0.72075 \\
29.0370 0.72150 \\
29.0728 0.72224 \\
29.1086 0.72299 \\
29.1444 0.72374 \\
29.1802 0.72448 \\
29.2160 0.72523 \\
29.2519 0.72598 \\
29.2877 0.72673 \\
29.3235 0.72748 \\
29.3593 0.72823 \\
29.3951 0.72898 \\
29.4309 0.72972 \\
29.4667 0.73048 \\
29.5025 0.73123 \\
29.5383 0.73198 \\
29.5741 0.73273 \\
29.6099 0.73348 \\
29.6457 0.73423 \\
29.6815 0.73498 \\
29.7173 0.73574 \\
29.7532 0.73649 \\
29.7890 0.73725 \\
29.8248 0.73800 \\
29.8606 0.73876 \\
29.8964 0.73951 \\
29.9322 0.74027 \\
29.9680 0.74102 \\
30.0038 0.74178 \\
30.0396 0.74254 \\
30.0754 0.74330 \\
30.1112 0.74405 \\
30.1470 0.74481 \\
30.1828 0.74557 \\
30.2186 0.74633 \\
30.2545 0.74709 \\
30.2903 0.74785 \\
30.3261 0.74862 \\
30.3619 0.74938 \\
30.3977 0.75014 \\
30.4335 0.75090 \\
30.4693 0.75166 \\
30.5051 0.75243 \\
30.5409 0.75319 \\
30.5767 0.75395 \\
30.6125 0.75472 \\
30.6483 0.75548 \\
30.6841 0.75624 \\
30.7199 0.75701 \\
30.7558 0.75777 \\
30.7916 0.75853 \\
30.8274 0.75930 \\
30.8632 0.76006 \\
30.8990 0.76082 \\
30.9348 0.76159 \\
30.9706 0.76235 \\
31.0064 0.76311 \\
31.0422 0.76388 \\
31.0780 0.76464 \\
31.1138 0.76540 \\
31.1496 0.76617 \\
31.1854 0.76693 \\
31.2212 0.76769 \\
31.2571 0.76846 \\
31.2929 0.76922 \\
31.3287 0.76998 \\
31.3645 0.77074 \\
31.4003 0.77150 \\
31.4361 0.77226 \\
31.4719 0.77302 \\
31.5077 0.77378 \\
31.5435 0.77454 \\
31.5793 0.77530 \\
31.6151 0.77606 \\
31.6509 0.77682 \\
31.6867 0.77758 \\
31.7225 0.77833 \\
31.7584 0.77909 \\
31.7942 0.77985 \\
31.8300 0.78060 \\
31.8658 0.78136 \\
31.9016 0.78211 \\
31.9374 0.78286 \\
31.9732 0.78362 \\
32.0090 0.78437 \\
32.0448 0.78512 \\
32.0806 0.78587 \\
32.1164 0.78662 \\
32.1522 0.78737 \\
32.1880 0.78812 \\
32.2238 0.78886 \\
32.2597 0.78961 \\
32.2955 0.79035 \\
32.3313 0.79110 \\
32.3671 0.79184 \\
32.4029 0.79258 \\
32.4387 0.79333 \\
32.4745 0.79407 \\
32.5103 0.79481 \\
32.5461 0.79554 \\
32.5819 0.79628 \\
32.6177 0.79702 \\
32.6535 0.79775 \\
32.6893 0.79849 \\
32.7251 0.79922 \\
32.7610 0.79995 \\
32.7968 0.80068 \\
32.8326 0.80141 \\
32.8684 0.80214 \\
32.9042 0.80286 \\
32.9400 0.80359 \\
32.9758 0.80431 \\
33.0116 0.80504 \\
33.0474 0.80576 \\
33.0832 0.80648 \\
33.1190 0.80719 \\
33.1548 0.80791 \\
33.1906 0.80863 \\
33.2264 0.80934 \\
33.2623 0.81005 \\
33.2981 0.81076 \\
33.3339 0.81147 \\
33.3697 0.81218 \\
33.4055 0.81288 \\
33.4413 0.81359 \\
33.4771 0.81429 \\
33.5129 0.81499 \\
33.5487 0.81569 \\
33.5845 0.81639 \\
33.6203 0.81708 \\
33.6561 0.81778 \\
33.6919 0.81847 \\
33.7277 0.81916 \\
33.7636 0.81985 \\
33.7994 0.82054 \\
33.8352 0.82122 \\
33.8710 0.82190 \\
33.9068 0.82258 \\
33.9426 0.82326 \\
33.9784 0.82394 \\
34.0142 0.82461 \\
34.0500 0.82529 \\
34.0858 0.82596 \\
34.1216 0.82663 \\
34.1574 0.82729 \\
34.1932 0.82796 \\
34.2290 0.82862 \\
34.2649 0.82928 \\
34.3007 0.82994 \\
34.3365 0.83059 \\
34.3723 0.83125 \\
34.4081 0.83190 \\
34.4439 0.83255 \\
34.4797 0.83319 \\
34.5155 0.83384 \\
34.5513 0.83448 \\
34.5871 0.83512 \\
34.6229 0.83576 \\
34.6587 0.83639 \\
34.6945 0.83702 \\
34.7303 0.83765 \\
34.7662 0.83828 \\
34.8020 0.83890 \\
34.8378 0.83953 \\
34.8736 0.84015 \\
34.9094 0.84076 \\
34.9452 0.84138 \\
34.9810 0.84199 \\
35.0168 0.84260 \\
35.0526 0.84320 \\
35.0884 0.84381 \\
35.1242 0.84441 \\
35.1600 0.84501 \\
35.1958 0.84560 \\
35.2316 0.84620 \\
35.2675 0.84679 \\
35.3033 0.84737 \\
35.3391 0.84796 \\
35.3749 0.84854 \\
35.4107 0.84912 \\
35.4465 0.84969 \\
35.4823 0.85027 \\
35.5181 0.85084 \\
35.5539 0.85140 \\
35.5897 0.85197 \\
35.6255 0.85253 \\
35.6613 0.85308 \\
35.6971 0.85364 \\
35.7329 0.85419 \\
35.7688 0.85474 \\
35.8046 0.85528 \\
35.8404 0.85582 \\
35.8762 0.85636 \\
35.9120 0.85690 \\
35.9478 0.85743 \\
35.9836 0.85796 \\
36.0194 0.85848 \\
36.0552 0.85901 \\
36.0910 0.85953 \\
36.1268 0.86004 \\
36.1626 0.86055 \\
36.1984 0.86106 \\
36.2342 0.86157 \\
36.2701 0.86207 \\
36.3059 0.86257 \\
36.3417 0.86307 \\
36.3775 0.86356 \\
36.4133 0.86405 \\
36.4491 0.86454 \\
36.4849 0.86502 \\
36.5207 0.86550 \\
36.5565 0.86598 \\
36.5923 0.86645 \\
36.6281 0.86692 \\
36.6639 0.86739 \\
36.6997 0.86786 \\
36.7355 0.86832 \\
36.7714 0.86878 \\
36.8072 0.86923 \\
36.8430 0.86969 \\
36.8788 0.87014 \\
36.9146 0.87058 \\
36.9504 0.87103 \\
36.9862 0.87147 \\
37.0220 0.87190 \\
37.0578 0.87234 \\
37.0936 0.87277 \\
37.1294 0.87320 \\
37.1652 0.87362 \\
37.2010 0.87405 \\
37.2368 0.87447 \\
37.2727 0.87489 \\
37.3085 0.87530 \\
37.3443 0.87571 \\
37.3801 0.87612 \\
37.4159 0.87653 \\
37.4517 0.87693 \\
37.4875 0.87733 \\
37.5233 0.87773 \\
37.5591 0.87812 \\
37.5949 0.87851 \\
37.6307 0.87890 \\
37.6665 0.87929 \\
37.7023 0.87967 \\
37.7381 0.88005 \\
37.7740 0.88043 \\
37.8098 0.88080 \\
37.8456 0.88118 \\
37.8814 0.88155 \\
37.9172 0.88191 \\
37.9530 0.88228 \\
37.9888 0.88264 \\
38.0246 0.88300 \\
38.0604 0.88336 \\
38.0962 0.88371 \\
38.1320 0.88406 \\
38.1678 0.88441 \\
38.2036 0.88476 \\
38.2394 0.88510 \\
38.2753 0.88544 \\
38.3111 0.88578 \\
38.3469 0.88611 \\
38.3827 0.88645 \\
38.4185 0.88678 \\
38.4543 0.88711 \\
38.4901 0.88743 \\
38.5259 0.88776 \\
38.5617 0.88808 \\
38.5975 0.88840 \\
38.6333 0.88871 \\
38.6691 0.88903 \\
38.7049 0.88934 \\
38.7407 0.88965 \\
38.7766 0.88995 \\
38.8124 0.89026 \\
38.8482 0.89056 \\
38.8840 0.89086 \\
38.9198 0.89116 \\
38.9556 0.89145 \\
38.9914 0.89174 \\
39.0272 0.89203 \\
39.0630 0.89232 \\
39.0988 0.89261 \\
39.1346 0.89289 \\
39.1704 0.89317 \\
39.2062 0.89345 \\
39.2420 0.89373 \\
39.2779 0.89400 \\
39.3137 0.89427 \\
39.3495 0.89454 \\
39.3853 0.89481 \\
39.4211 0.89507 \\
39.4569 0.89534 \\
39.4927 0.89560 \\
39.5285 0.89586 \\
39.5643 0.89612 \\
39.6001 0.89637 \\
39.6359 0.89662 \\
39.6717 0.89687 \\
39.7075 0.89712 \\
39.7433 0.89737 \\
39.7792 0.89761 \\
39.8150 0.89786 \\
39.8508 0.89810 \\
39.8866 0.89833 \\
39.9224 0.89857 \\
39.9582 0.89881 \\
39.9940 0.89904 \\
40.0298 0.89927 \\
40.0656 0.89950 \\
40.1014 0.89972 \\
40.1372 0.89995 \\
40.1730 0.90017 \\
40.2088 0.90039 \\
40.2446 0.90061 \\
40.2805 0.90083 \\
40.3163 0.90104 \\
40.3521 0.90126 \\
40.3879 0.90147 \\
40.4237 0.90168 \\
40.4595 0.90188 \\
40.4953 0.90209 \\
40.5311 0.90229 \\
40.5669 0.90250 \\
40.6027 0.90270 \\
40.6385 0.90290 \\
40.6743 0.90309 \\
40.7101 0.90329 \\
40.7459 0.90348 \\
40.7818 0.90367 \\
40.8176 0.90386 \\
40.8534 0.90405 \\
40.8892 0.90424 \\
40.9250 0.90442 \\
40.9608 0.90461 \\
40.9966 0.90479 \\
41.0324 0.90497 \\
41.0682 0.90515 \\
41.1040 0.90532 \\
41.1398 0.90550 \\
41.1756 0.90567 \\
41.2114 0.90585 \\
41.2472 0.90602 \\
41.2831 0.90619 \\
41.3189 0.90635 \\
41.3547 0.90652 \\
41.3905 0.90668 \\
41.4263 0.90685 \\
41.4621 0.90701 \\
41.4979 0.90717 \\
41.5337 0.90733 \\
41.5695 0.90748 \\
41.6053 0.90764 \\
41.6411 0.90779 \\
41.6769 0.90795 \\
41.7127 0.90810 \\
41.7485 0.90825 \\
41.7844 0.90840 \\
41.8202 0.90854 \\
41.8560 0.90869 \\
41.8918 0.90883 \\
41.9276 0.90898 \\
41.9634 0.90912 \\
41.9992 0.90926 \\
42.0350 0.90940 \\
42.0708 0.90954 \\
42.1066 0.90967 \\
42.1424 0.90981 \\
42.1782 0.90994 \\
42.2140 0.91008 \\
42.2498 0.91021 \\
42.2857 0.91034 \\
42.3215 0.91047 \\
42.3573 0.91060 \\
42.3931 0.91072 \\
42.4289 0.91085 \\
42.4647 0.91097 \\
42.5005 0.91110 \\
42.5363 0.91122 \\
42.5721 0.91134 \\
42.6079 0.91146 \\
42.6437 0.91158 \\
42.6795 0.91170 \\
42.7153 0.91181 \\
42.7512 0.91193 \\
42.7870 0.91204 \\
42.8228 0.91216 \\
42.8586 0.91227 \\
42.8944 0.91238 \\
42.9302 0.91249 \\
42.9660 0.91260 \\
43.0018 0.91271 \\
43.0376 0.91282 \\
43.0734 0.91293 \\
43.1092 0.91303 \\
43.1450 0.91314 \\
43.1808 0.91324 \\
43.2166 0.91334 \\
43.2525 0.91345 \\
43.2883 0.91355 \\
43.3241 0.91365 \\
43.3599 0.91375 \\
43.3957 0.91385 \\
43.4315 0.91394 \\
43.4673 0.91404 \\
43.5031 0.91414 \\
43.5389 0.91423 \\
43.5747 0.91433 \\
43.6105 0.91442 \\
43.6463 0.91452 \\
43.6821 0.91461 \\
43.7179 0.91470 \\
43.7538 0.91479 \\
43.7896 0.91488 \\
43.8254 0.91497 \\
43.8612 0.91506 \\
43.8970 0.91515 \\
43.9328 0.91524 \\
43.9686 0.91532 \\
44.0044 0.91541 \\
44.0402 0.91549 \\
44.0760 0.91558 \\
44.1118 0.91566 \\
44.1476 0.91575 \\
44.1834 0.91583 \\
44.2192 0.91591 \\
44.2551 0.91600 \\
44.2909 0.91608 \\
44.3267 0.91616 \\
44.3625 0.91624 \\
44.3983 0.91632 \\
44.4341 0.91640 \\
44.4699 0.91648 \\
44.5057 0.91656 \\
44.5415 0.91663 \\
44.5773 0.91671 \\
44.6131 0.91679 \\
44.6489 0.91687 \\
44.6847 0.91694 \\
44.7205 0.91702 \\
44.7564 0.91709 \\
44.7922 0.91717 \\
44.8280 0.91724 \\
44.8638 0.91732 \\
44.8996 0.91739 \\
44.9354 0.91747 \\
44.9712 0.91754 \\
45.0070 0.91761 \\
45.0428 0.91769 \\
45.0786 0.91776 \\
45.1144 0.91783 \\
45.1502 0.91790 \\
45.1860 0.91798 \\
45.2218 0.91805 \\
45.2577 0.91812 \\
45.2935 0.91819 \\
45.3293 0.91826 \\
45.3651 0.91833 \\
45.4009 0.91840 \\
45.4367 0.91847 \\
45.4725 0.91854 \\
45.5083 0.91861 \\
45.5441 0.91868 \\
45.5799 0.91875 \\
45.6157 0.91882 \\
45.6515 0.91888 \\
45.6873 0.91895 \\
45.7231 0.91902 \\
45.7590 0.91909 \\
45.7948 0.91915 \\
45.8306 0.91922 \\
45.8664 0.91929 \\
45.9022 0.91935 \\
45.9380 0.91942 \\
45.9738 0.91948 \\
46.0096 0.91955 \\
46.0454 0.91961 \\
46.0812 0.91968 \\
46.1170 0.91974 \\
46.1528 0.91981 \\
46.1886 0.91987 \\
46.2244 0.91993 \\
46.2603 0.92000 \\
46.2961 0.92006 \\
46.3319 0.92012 \\
46.3677 0.92019 \\
46.4035 0.92025 \\
46.4393 0.92031 \\
46.4751 0.92037 \\
46.5109 0.92043 \\
46.5467 0.92050 \\
46.5825 0.92056 \\
46.6183 0.92062 \\
46.6541 0.92068 \\
46.6899 0.92074 \\
46.7257 0.92080 \\
46.7616 0.92086 \\
46.7974 0.92092 \\
46.8332 0.92097 \\
46.8690 0.92103 \\
46.9048 0.92109 \\
46.9406 0.92115 \\
46.9764 0.92121 \\
47.0122 0.92127 \\
47.0480 0.92132 \\
47.0838 0.92138 \\
47.1196 0.92144 \\
47.1554 0.92149 \\
47.1912 0.92155 \\
47.2270 0.92161 \\
47.2629 0.92166 \\
47.2987 0.92172 \\
47.3345 0.92177 \\
47.3703 0.92183 \\
47.4061 0.92188 \\
47.4419 0.92194 \\
47.4777 0.92199 \\
47.5135 0.92205 \\
47.5493 0.92210 \\
47.5851 0.92215 \\
47.6209 0.92221 \\
47.6567 0.92226 \\
47.6925 0.92231 \\
47.7283 0.92237 \\
47.7642 0.92242 \\
47.8000 0.92247 \\
47.8358 0.92252 \\
47.8716 0.92257 \\
47.9074 0.92263 \\
47.9432 0.92268 \\
47.9790 0.92273 \\
48.0148 0.92278 \\
48.0506 0.92283 \\
48.0864 0.92288 \\
48.1222 0.92293 \\
48.1580 0.92298 \\
48.1938 0.92303 \\
48.2296 0.92308 \\
48.2655 0.92313 \\
48.3013 0.92317 \\
48.3371 0.92322 \\
48.3729 0.92327 \\
48.4087 0.92332 \\
48.4445 0.92337 \\
48.4803 0.92341 \\
48.5161 0.92346 \\
48.5519 0.92351 \\
48.5877 0.92356 \\
48.6235 0.92360 \\
48.6593 0.92365 \\
48.6951 0.92369 \\
48.7309 0.92374 \\
48.7668 0.92379 \\
48.8026 0.92383 \\
48.8384 0.92388 \\
48.8742 0.92392 \\
48.9100 0.92397 \\
48.9458 0.92401 \\
48.9816 0.92405 \\
49.0174 0.92410 \\
49.0532 0.92414 \\
49.0890 0.92419 \\
49.1248 0.92423 \\
49.1606 0.92427 \\
49.1964 0.92431 \\
49.2322 0.92436 \\
49.2681 0.92440 \\
49.3039 0.92444 \\
49.3397 0.92448 \\
49.3755 0.92452 \\
49.4113 0.92457 \\
49.4471 0.92461 \\
49.4829 0.92465 \\
49.5187 0.92469 \\
49.5545 0.92473 \\
49.5903 0.92477 \\
49.6261 0.92481 \\
49.6619 0.92485 \\
49.6977 0.92489 \\
49.7335 0.92493 \\
49.7694 0.92497 \\
49.8052 0.92501 \\
49.8410 0.92505 \\
49.8768 0.92509 \\
49.9126 0.92512 \\
49.9484 0.92516 \\
49.9842 0.92520 \\
50.0200 0.92524 \\
50.0558 0.92527 \\
50.0916 0.92531 \\
50.1274 0.92535 \\
50.1632 0.92539 \\
50.1990 0.92542 \\
50.2348 0.92546 \\
50.2707 0.92550 \\
50.3065 0.92553 \\
50.3423 0.92557 \\
50.3781 0.92560 \\
50.4139 0.92564 \\
50.4497 0.92567 \\
50.4855 0.92571 \\
50.5213 0.92574 \\
50.5571 0.92578 \\
50.5929 0.92581 \\
50.6287 0.92585 \\
50.6645 0.92588 \\
50.7003 0.92591 \\
50.7361 0.92595 \\
50.7720 0.92598 \\
50.8078 0.92601 \\
50.8436 0.92605 \\
50.8794 0.92608 \\
50.9152 0.92611 \\
50.9510 0.92615 \\
50.9868 0.92618 \\
51.0226 0.92621 \\
51.0584 0.92624 \\
51.0942 0.92627 \\
51.1300 0.92630 \\
51.1658 0.92634 \\
51.2016 0.92637 \\
51.2374 0.92640 \\
51.2733 0.92643 \\
51.3091 0.92646 \\
51.3449 0.92649 \\
51.3807 0.92652 \\
51.4165 0.92655 \\
51.4523 0.92658 \\
51.4881 0.92661 \\
51.5239 0.92664 \\
51.5597 0.92667 \\
51.5955 0.92669 \\
51.6313 0.92672 \\
51.6671 0.92675 \\
51.7029 0.92678 \\
51.7387 0.92681 \\
51.7746 0.92684 \\
51.8104 0.92686 \\
51.8462 0.92689 \\
51.8820 0.92692 \\
51.9178 0.92695 \\
51.9536 0.92697 \\
51.9894 0.92700 \\
52.0252 0.92703 \\
52.0610 0.92705 \\
52.0968 0.92708 \\
52.1326 0.92711 \\
52.1684 0.92713 \\
52.2042 0.92716 \\
52.2400 0.92718 \\
52.2759 0.92721 \\
52.3117 0.92723 \\
52.3475 0.92726 \\
52.3833 0.92728 \\
52.4191 0.92731 \\
52.4549 0.92733 \\
52.4907 0.92736 \\
52.5265 0.92738 \\
52.5623 0.92741 \\
52.5981 0.92743 \\
52.6339 0.92745 \\
52.6697 0.92748 \\
52.7055 0.92750 \\
52.7413 0.92752 \\
52.7772 0.92755 \\
52.8130 0.92757 \\
52.8488 0.92759 \\
52.8846 0.92761 \\
52.9204 0.92764 \\
52.9562 0.92766 \\
52.9920 0.92768 \\
53.0278 0.92770 \\
53.0636 0.92772 \\
53.0994 0.92775 \\
53.1352 0.92777 \\
53.1710 0.92779 \\
53.2068 0.92781 \\
53.2426 0.92783 \\
53.2785 0.92785 \\
53.3143 0.92787 \\
53.3501 0.92789 \\
53.3859 0.92791 \\
53.4217 0.92793 \\
53.4575 0.92795 \\
53.4933 0.92797 \\
53.5291 0.92799 \\
53.5649 0.92801 \\
53.6007 0.92803 \\
53.6365 0.92805 \\
53.6723 0.92807 \\
53.7081 0.92809 \\
53.7439 0.92810 \\
53.7798 0.92812 \\
53.8156 0.92814 \\
53.8514 0.92816 \\
53.8872 0.92818 \\
53.9230 0.92820 \\
53.9588 0.92821 \\
53.9946 0.92823 \\
54.0304 0.92825 \\
54.0662 0.92827 \\
54.1020 0.92828 \\
54.1378 0.92830 \\
54.1736 0.92832 \\
54.2094 0.92833 \\
54.2452 0.92835 \\
54.2811 0.92837 \\
54.3169 0.92838 \\
54.3527 0.92840 \\
54.3885 0.92841 \\
54.4243 0.92843 \\
54.4601 0.92845 \\
54.4959 0.92846 \\
54.5317 0.92848 \\
54.5675 0.92849 \\
54.6033 0.92851 \\
54.6391 0.92852 \\
54.6749 0.92854 \\
54.7107 0.92855 \\
54.7465 0.92857 \\
54.7824 0.92858 \\
54.8182 0.92859 \\
54.8540 0.92861 \\
54.8898 0.92862 \\
54.9256 0.92864 \\
54.9614 0.92865 \\
54.9972 0.92866 \\
55.0330 0.92868 \\
55.0688 0.92869 \\
55.1046 0.92870 \\
55.1404 0.92872 \\
55.1762 0.92873 \\
55.2120 0.92874 \\
55.2478 0.92875 \\
55.2837 0.92877 \\
55.3195 0.92878 \\
55.3553 0.92879 \\
55.3911 0.92880 \\
55.4269 0.92882 \\
55.4627 0.92883 \\
55.4985 0.92884 \\
55.5343 0.92885 \\
55.5701 0.92886 \\
55.6059 0.92887 \\
55.6417 0.92888 \\
55.6775 0.92890 \\
55.7133 0.92891 \\
55.7491 0.92892 \\
55.7850 0.92893 \\
55.8208 0.92894 \\
55.8566 0.92895 \\
55.8924 0.92896 \\
55.9282 0.92897 \\
55.9640 0.92898 \\
55.9998 0.92899 \\
56.0356 0.92900 \\
56.0714 0.92901 \\
56.1072 0.92902 \\
56.1430 0.92903 \\
56.1788 0.92904 \\
56.2146 0.92905 \\
56.2505 0.92906 \\
56.2863 0.92907 \\
56.3221 0.92908 \\
56.3579 0.92908 \\
56.3937 0.92909 \\
56.4295 0.92910 \\
56.4653 0.92911 \\
56.5011 0.92912 \\
56.5369 0.92913 \\
56.5727 0.92913 \\
56.6085 0.92914 \\
56.6443 0.92915 \\
56.6801 0.92916 \\
56.7159 0.92917 \\
56.7518 0.92917 \\
56.7876 0.92918 \\
56.8234 0.92919 \\
56.8592 0.92920 \\
56.8950 0.92920 \\
56.9308 0.92921 \\
56.9666 0.92922 \\
57.0024 0.92922 \\
57.0382 0.92923 \\
57.0740 0.92924 \\
57.1098 0.92925 \\
57.1456 0.92925 \\
57.1814 0.92926 \\
57.2172 0.92926 \\
57.2531 0.92927 \\
57.2889 0.92928 \\
57.3247 0.92928 \\
57.3605 0.92929 \\
57.3963 0.92929 \\
57.4321 0.92930 \\
57.4679 0.92931 \\
57.5037 0.92931 \\
57.5395 0.92932 \\
57.5753 0.92932 \\
57.6111 0.92933 \\
57.6469 0.92933 \\
57.6827 0.92934 \\
57.7185 0.92934 \\
57.7544 0.92935 \\
57.7902 0.92935 \\
57.8260 0.92936 \\
57.8618 0.92936 \\
57.8976 0.92937 \\
57.9334 0.92937 \\
57.9692 0.92938 \\
58.0050 0.92938 \\
58.0408 0.92938 \\
58.0766 0.92939 \\
58.1124 0.92939 \\
58.1482 0.92940 \\
58.1840 0.92940 \\
58.2198 0.92940 \\
58.2557 0.92941 \\
58.2915 0.92941 \\
58.3273 0.92941 \\
58.3631 0.92942 \\
58.3989 0.92942 \\
58.4347 0.92942 \\
58.4705 0.92943 \\
58.5063 0.92943 \\
58.5421 0.92943 \\
58.5779 0.92944 \\
58.6137 0.92944 \\
58.6495 0.92944 \\
58.6853 0.92945 \\
58.7211 0.92945 \\
58.7570 0.92945 \\
58.7928 0.92945 \\
58.8286 0.92946 \\
58.8644 0.92946 \\
58.9002 0.92946 \\
58.9360 0.92946 \\
58.9718 0.92947 \\
59.0076 0.92947 \\
59.0434 0.92947 \\
59.0792 0.92947 \\
59.1150 0.92947 \\
59.1508 0.92948 \\
59.1866 0.92948 \\
59.2224 0.92948 \\
59.2583 0.92948 \\
59.2941 0.92948 \\
59.3299 0.92948 \\
59.3657 0.92948 \\
59.4015 0.92949 \\
59.4373 0.92949 \\
59.4731 0.92949 \\
59.5089 0.92949 \\
59.5447 0.92949 \\
59.5805 0.92949 \\
59.6163 0.92949 \\
59.6521 0.92949 \\
59.6879 0.92949 \\
59.7237 0.92950 \\
59.7596 0.92950 \\
59.7954 0.92950 \\
59.8312 0.92950 \\
59.8670 0.92950 \\
59.9028 0.92950 \\
59.9386 0.92950 \\
59.9744 0.92950 \\
60.0102 0.92950 \\
60.0460 0.92950 \\
60.0818 0.92950 \\
60.1176 0.92950 \\
60.1534 0.92950 \\
60.1892 0.92950 \\
60.2250 0.92950 \\
60.2609 0.92950 \\
60.2967 0.92950 \\
60.3325 0.92950 \\
60.3683 0.92950 \\
60.4041 0.92950 \\
60.4399 0.92950 \\
60.4757 0.92950 \\
60.5115 0.92950 \\
60.5473 0.92950 \\
60.5831 0.92950 \\
60.6189 0.92950 \\
60.6547 0.92950 \\
60.6905 0.92950 \\
60.7263 0.92950 \\
60.7622 0.92950 \\
60.7980 0.92949 \\
60.8338 0.92949 \\
60.8696 0.92949 \\
60.9054 0.92949 \\
60.9412 0.92949 \\
60.9770 0.92949 \\
61.0128 0.92949 \\
61.0486 0.92949 \\
61.0844 0.92949 \\
61.1202 0.92948 \\
61.1560 0.92948 \\
61.1918 0.92948 \\
61.2276 0.92948 \\
61.2635 0.92948 \\
61.2993 0.92948 \\
61.3351 0.92948 \\
61.3709 0.92947 \\
61.4067 0.92947 \\
61.4425 0.92947 \\
61.4783 0.92947 \\
61.5141 0.92947 \\
61.5499 0.92947 \\
61.5857 0.92946 \\
61.6215 0.92946 \\
61.6573 0.92946 \\
61.6931 0.92946 \\
61.7289 0.92945 \\
61.7648 0.92945 \\
61.8006 0.92945 \\
61.8364 0.92945 \\
61.8722 0.92945 \\
61.9080 0.92944 \\
61.9438 0.92944 \\
61.9796 0.92944 \\
62.0154 0.92944 \\
62.0512 0.92943 \\
62.0870 0.92943 \\
62.1228 0.92943 \\
62.1586 0.92942 \\
62.1944 0.92942 \\
62.2302 0.92942 \\
62.2661 0.92942 \\
62.3019 0.92941 \\
62.3377 0.92941 \\
62.3735 0.92941 \\
62.4093 0.92940 \\
62.4451 0.92940 \\
62.4809 0.92940 \\
62.5167 0.92939 \\
62.5525 0.92939 \\
62.5883 0.92939 \\
62.6241 0.92938 \\
62.6599 0.92938 \\
62.6957 0.92938 \\
62.7315 0.92937 \\
62.7674 0.92937 \\
62.8032 0.92936 \\
62.8390 0.92936 \\
62.8748 0.92936 \\
62.9106 0.92935 \\
62.9464 0.92935 \\
62.9822 0.92935 \\
63.0180 0.92934 \\
63.0538 0.92934 \\
63.0896 0.92933 \\
63.1254 0.92933 \\
63.1612 0.92932 \\
63.1970 0.92932 \\
63.2328 0.92932 \\
63.2687 0.92931 \\
63.3045 0.92931 \\
63.3403 0.92930 \\
63.3761 0.92930 \\
63.4119 0.92929 \\
63.4477 0.92929 \\
63.4835 0.92928 \\
63.5193 0.92928 \\
63.5551 0.92927 \\
63.5909 0.92927 \\
63.6267 0.92926 \\
63.6625 0.92926 \\
63.6983 0.92925 \\
63.7341 0.92925 \\
63.7700 0.92924 \\
63.8058 0.92924 \\
63.8416 0.92923 \\
63.8774 0.92923 \\
63.9132 0.92922 \\
63.9490 0.92922 \\
63.9848 0.92921 \\
64.0206 0.92921 \\
64.0564 0.92920 \\
64.0922 0.92920 \\
64.1280 0.92919 \\
64.1638 0.92918 \\
64.1996 0.92918 \\
64.2354 0.92917 \\
64.2713 0.92917 \\
64.3071 0.92916 \\
64.3429 0.92916 \\
64.3787 0.92915 \\
64.4145 0.92914 \\
64.4503 0.92914 \\
64.4861 0.92913 \\
64.5219 0.92912 \\
64.5577 0.92912 \\
64.5935 0.92911 \\
64.6293 0.92911 \\
64.6651 0.92910 \\
64.7009 0.92909 \\
64.7367 0.92909 \\
64.7726 0.92908 \\
64.8084 0.92907 \\
64.8442 0.92907 \\
64.8800 0.92906 \\
64.9158 0.92905 \\
64.9516 0.92905 \\
64.9874 0.92904 \\
65.0232 0.92903 \\
65.0590 0.92903 \\
65.0948 0.92902 \\
65.1306 0.92901 \\
65.1664 0.92900 \\
65.2022 0.92900 \\
65.2380 0.92899 \\
65.2739 0.92898 \\
65.3097 0.92898 \\
65.3455 0.92897 \\
65.3813 0.92896 \\
65.4171 0.92895 \\
65.4529 0.92895 \\
65.4887 0.92894 \\
65.5245 0.92893 \\
65.5603 0.92892 \\
65.5961 0.92892 \\
65.6319 0.92891 \\
65.6677 0.92890 \\
65.7035 0.92889 \\
65.7393 0.92888 \\
65.7752 0.92888 \\
65.8110 0.92887 \\
65.8468 0.92886 \\
65.8826 0.92885 \\
65.9184 0.92884 \\
65.9542 0.92884 \\
65.9900 0.92883 \\
66.0258 0.92882 \\
66.0616 0.92881 \\
66.0974 0.92880 \\
66.1332 0.92879 \\
66.1690 0.92879 \\
66.2048 0.92878 \\
66.2406 0.92877 \\
66.2765 0.92876 \\
66.3123 0.92875 \\
66.3481 0.92874 \\
66.3839 0.92873 \\
66.4197 0.92873 \\
66.4555 0.92872 \\
66.4913 0.92871 \\
66.5271 0.92870 \\
66.5629 0.92869 \\
66.5987 0.92868 \\
66.6345 0.92867 \\
66.6703 0.92866 \\
66.7061 0.92865 \\
66.7419 0.92864 \\
66.7778 0.92863 \\
66.8136 0.92863 \\
66.8494 0.92862 \\
66.8852 0.92861 \\
66.9210 0.92860 \\
66.9568 0.92859 \\
66.9926 0.92858 \\
67.0284 0.92857 \\
67.0642 0.92856 \\
67.1000 0.92855 \\
67.1358 0.92854 \\
67.1716 0.92853 \\
67.2074 0.92852 \\
67.2432 0.92851 \\
67.2791 0.92850 \\
67.3149 0.92849 \\
67.3507 0.92848 \\
67.3865 0.92847 \\
67.4223 0.92846 \\
67.4581 0.92845 \\
67.4939 0.92844 \\
67.5297 0.92843 \\
67.5655 0.92842 \\
67.6013 0.92841 \\
67.6371 0.92840 \\
67.6729 0.92839 \\
67.7087 0.92838 \\
67.7445 0.92837 \\
67.7804 0.92836 \\
67.8162 0.92834 \\
67.8520 0.92833 \\
67.8878 0.92832 \\
67.9236 0.92831 \\
67.9594 0.92830 \\
67.9952 0.92829 \\
68.0310 0.92828 \\
68.0668 0.92827 \\
68.1026 0.92826 \\
68.1384 0.92825 \\
68.1742 0.92824 \\
68.2100 0.92822 \\
68.2458 0.92821 \\
68.2817 0.92820 \\
68.3175 0.92819 \\
68.3533 0.92818 \\
68.3891 0.92817 \\
68.4249 0.92816 \\
68.4607 0.92814 \\
68.4965 0.92813 \\
68.5323 0.92812 \\
68.5681 0.92811 \\
68.6039 0.92810 \\
68.6397 0.92809 \\
68.6755 0.92807 \\
68.7113 0.92806 \\
68.7471 0.92805 \\
68.7830 0.92804 \\
68.8188 0.92803 \\
68.8546 0.92801 \\
68.8904 0.92800 \\
68.9262 0.92799 \\
68.9620 0.92798 \\
68.9978 0.92797 \\
69.0336 0.92795 \\
69.0694 0.92794 \\
69.1052 0.92793 \\
69.1410 0.92792 \\
69.1768 0.92790 \\
69.2126 0.92789 \\
69.2484 0.92788 \\
69.2843 0.92787 \\
69.3201 0.92785 \\
69.3559 0.92784 \\
69.3917 0.92783 \\
69.4275 0.92782 \\
69.4633 0.92780 \\
69.4991 0.92779 \\
69.5349 0.92778 \\
69.5707 0.92776 \\
69.6065 0.92775 \\
69.6423 0.92774 \\
69.6781 0.92773 \\
69.7139 0.92771 \\
69.7497 0.92770 \\
69.7856 0.92769 \\
69.8214 0.92767 \\
69.8572 0.92766 \\
69.8930 0.92765 \\
69.9288 0.92763 \\
69.9646 0.92762 \\
70.0004 0.92761 \\
70.0362 0.92759 \\
70.0720 0.92758 \\
70.1078 0.92756 \\
70.1436 0.92755 \\
70.1794 0.92754 \\
70.2152 0.92752 \\
70.2511 0.92751 \\
70.2869 0.92750 \\
70.3227 0.92748 \\
70.3585 0.92747 \\
70.3943 0.92745 \\
70.4301 0.92744 \\
70.4659 0.92743 \\
70.5017 0.92741 \\
70.5375 0.92740 \\
70.5733 0.92738 \\
70.6091 0.92737 \\
70.6449 0.92736 \\
70.6807 0.92734 \\
70.7165 0.92733 \\
70.7524 0.92731 \\
70.7882 0.92730 \\
70.8240 0.92728 \\
70.8598 0.92727 \\
70.8956 0.92725 \\
70.9314 0.92724 \\
70.9672 0.92722 \\
71.0030 0.92721 \\
71.0388 0.92719 \\
71.0746 0.92718 \\
71.1104 0.92717 \\
71.1462 0.92715 \\
71.1820 0.92714 \\
71.2178 0.92712 \\
71.2537 0.92711 \\
71.2895 0.92709 \\
71.3253 0.92707 \\
71.3611 0.92706 \\
71.3969 0.92704 \\
71.4327 0.92703 \\
71.4685 0.92701 \\
71.5043 0.92700 \\
71.5401 0.92698 \\
71.5759 0.92697 \\
71.6117 0.92695 \\
71.6475 0.92694 \\
71.6833 0.92692 \\
71.7191 0.92691 \\
71.7550 0.92689 \\
71.7908 0.92687 \\
71.8266 0.92686 \\
71.8624 0.92684 \\
71.8982 0.92683 \\
71.9340 0.92681 \\
71.9698 0.92679 \\
72.0056 0.92678 \\
72.0414 0.92676 \\
72.0772 0.92675 \\
72.1130 0.92673 \\
72.1488 0.92671 \\
72.1846 0.92670 \\
72.2204 0.92668 \\
72.2563 0.92667 \\
72.2921 0.92665 \\
72.3279 0.92663 \\
72.3637 0.92662 \\
72.3995 0.92660 \\
72.4353 0.92658 \\
72.4711 0.92657 \\
72.5069 0.92655 \\
72.5427 0.92653 \\
72.5785 0.92652 \\
72.6143 0.92650 \\
72.6501 0.92648 \\
72.6859 0.92647 \\
72.7217 0.92645 \\
72.7576 0.92643 \\
72.7934 0.92642 \\
72.8292 0.92640 \\
72.8650 0.92638 \\
72.9008 0.92637 \\
72.9366 0.92635 \\
72.9724 0.92633 \\
73.0082 0.92631 \\
73.0440 0.92630 \\
73.0798 0.92628 \\
73.1156 0.92626 \\
73.1514 0.92624 \\
73.1872 0.92623 \\
73.2230 0.92621 \\
73.2589 0.92619 \\
73.2947 0.92617 \\
73.3305 0.92616 \\
73.3663 0.92614 \\
73.4021 0.92612 \\
73.4379 0.92610 \\
73.4737 0.92609 \\
73.5095 0.92607 \\
73.5453 0.92605 \\
73.5811 0.92603 \\
73.6169 0.92602 \\
73.6527 0.92600 \\
73.6885 0.92598 \\
73.7243 0.92596 \\
73.7602 0.92594 \\
73.7960 0.92592 \\
73.8318 0.92591 \\
73.8676 0.92589 \\
73.9034 0.92587 \\
73.9392 0.92585 \\
73.9750 0.92583 \\
74.0108 0.92582 \\
74.0466 0.92580 \\
74.0824 0.92578 \\
74.1182 0.92576 \\
74.1540 0.92574 \\
74.1898 0.92572 \\
74.2256 0.92570 \\
74.2615 0.92569 \\
74.2973 0.92567 \\
74.3331 0.92565 \\
74.3689 0.92563 \\
74.4047 0.92561 \\
74.4405 0.92559 \\
74.4763 0.92557 \\
74.5121 0.92555 \\
74.5479 0.92553 \\
74.5837 0.92552 \\
74.6195 0.92550 \\
74.6553 0.92548 \\
74.6911 0.92546 \\
74.7269 0.92544 \\
74.7628 0.92542 \\
74.7986 0.92540 \\
74.8344 0.92538 \\
74.8702 0.92536 \\
74.9060 0.92534 \\
74.9418 0.92532 \\
74.9776 0.92530 \\
75.0134 0.92528 \\
75.0492 0.92526 \\
75.0850 0.92524 \\
75.1208 0.92523 \\
75.1566 0.92521 \\
75.1924 0.92519 \\
75.2282 0.92517 \\
75.2641 0.92515 \\
75.2999 0.92513 \\
75.3357 0.92511 \\
75.3715 0.92509 \\
75.4073 0.92507 \\
75.4431 0.92505 \\
75.4789 0.92503 \\
75.5147 0.92501 \\
75.5505 0.92499 \\
75.5863 0.92497 \\
75.6221 0.92495 \\
75.6579 0.92493 \\
75.6937 0.92491 \\
75.7295 0.92488 \\
75.7654 0.92486 \\
75.8012 0.92484 \\
75.8370 0.92482 \\
75.8728 0.92480 \\
75.9086 0.92478 \\
75.9444 0.92476 \\
75.9802 0.92474 \\
76.0160 0.92472 \\
76.0518 0.92470 \\
76.0876 0.92468 \\
76.1234 0.92466 \\
76.1592 0.92464 \\
76.1950 0.92462 \\
76.2308 0.92460 \\
76.2667 0.92457 \\
76.3025 0.92455 \\
76.3383 0.92453 \\
76.3741 0.92451 \\
76.4099 0.92449 \\
76.4457 0.92447 \\
76.4815 0.92445 \\
76.5173 0.92443 \\
76.5531 0.92441 \\
76.5889 0.92438 \\
76.6247 0.92436 \\
76.6605 0.92434 \\
76.6963 0.92432 \\
76.7321 0.92430 \\
76.7680 0.92428 \\
76.8038 0.92426 \\
76.8396 0.92423 \\
76.8754 0.92421 \\
76.9112 0.92419 \\
76.9470 0.92417 \\
76.9828 0.92415 \\
77.0186 0.92413 \\
77.0544 0.92410 \\
77.0902 0.92408 \\
77.1260 0.92406 \\
77.1618 0.92404 \\
77.1976 0.92402 \\
77.2334 0.92399 \\
77.2693 0.92397 \\
77.3051 0.92395 \\
77.3409 0.92393 \\
77.3767 0.92391 \\
77.4125 0.92388 \\
77.4483 0.92386 \\
77.4841 0.92384 \\
77.5199 0.92382 \\
77.5557 0.92379 \\
77.5915 0.92377 \\
77.6273 0.92375 \\
77.6631 0.92373 \\
77.6989 0.92370 \\
77.7347 0.92368 \\
77.7706 0.92366 \\
77.8064 0.92364 \\
77.8422 0.92361 \\
77.8780 0.92359 \\
77.9138 0.92357 \\
77.9496 0.92355 \\
77.9854 0.92352 \\
78.0212 0.92350 \\
78.0570 0.92348 \\
78.0928 0.92345 \\
78.1286 0.92343 \\
78.1644 0.92341 \\
78.2002 0.92339 \\
78.2360 0.92336 \\
78.2719 0.92334 \\
78.3077 0.92332 \\
78.3435 0.92329 \\
78.3793 0.92327 \\
78.4151 0.92325 \\
78.4509 0.92322 \\
78.4867 0.92320 \\
78.5225 0.92318 \\
78.5583 0.92315 \\
78.5941 0.92313 \\
78.6299 0.92311 \\
78.6657 0.92308 \\
78.7015 0.92306 \\
78.7373 0.92304 \\
78.7732 0.92301 \\
78.8090 0.92299 \\
78.8448 0.92296 \\
78.8806 0.92294 \\
78.9164 0.92292 \\
78.9522 0.92289 \\
78.9880 0.92287 \\
79.0238 0.92285 \\
79.0596 0.92282 \\
79.0954 0.92280 \\
79.1312 0.92277 \\
79.1670 0.92275 \\
79.2028 0.92272 \\
79.2386 0.92270 \\
79.2745 0.92268 \\
79.3103 0.92265 \\
79.3461 0.92263 \\
79.3819 0.92260 \\
79.4177 0.92258 \\
79.4535 0.92256 \\
79.4893 0.92253 \\
79.5251 0.92251 \\
79.5609 0.92248 \\
79.5967 0.92246 \\
79.6325 0.92243 \\
79.6683 0.92241 \\
79.7041 0.92238 \\
79.7399 0.92236 \\
79.7758 0.92233 \\
79.8116 0.92231 \\
79.8474 0.92228 \\
79.8832 0.92226 \\
79.9190 0.92224 \\
79.9548 0.92221 \\
79.9906 0.92219 \\
80.0264 0.92216 \\
80.0622 0.92214 \\
80.0980 0.92211 \\
80.1338 0.92209 \\
80.1696 0.92206 \\
80.2054 0.92204 \\
80.2412 0.92201 \\
80.2771 0.92198 \\
80.3129 0.92196 \\
80.3487 0.92193 \\
80.3845 0.92191 \\
80.4203 0.92188 \\
80.4561 0.92186 \\
80.4919 0.92183 \\
80.5277 0.92181 \\
80.5635 0.92178 \\
80.5993 0.92176 \\
80.6351 0.92173 \\
80.6709 0.92170 \\
80.7067 0.92168 \\
80.7425 0.92165 \\
80.7784 0.92163 \\
80.8142 0.92160 \\
80.8500 0.92158 \\
80.8858 0.92155 \\
80.9216 0.92152 \\
80.9574 0.92150 \\
80.9932 0.92147 \\
81.0290 0.92145 \\
81.0648 0.92142 \\
81.1006 0.92139 \\
81.1364 0.92137 \\
81.1722 0.92134 \\
81.2080 0.92132 \\
81.2438 0.92129 \\
81.2797 0.92126 \\
81.3155 0.92124 \\
81.3513 0.92121 \\
81.3871 0.92119 \\
81.4229 0.92116 \\
81.4587 0.92113 \\
81.4945 0.92111 \\
81.5303 0.92108 \\
81.5661 0.92105 \\
81.6019 0.92103 \\
81.6377 0.92100 \\
81.6735 0.92097 \\
81.7093 0.92095 \\
81.7451 0.92092 \\
81.7810 0.92089 \\
81.8168 0.92087 \\
81.8526 0.92084 \\
81.8884 0.92081 \\
81.9242 0.92079 \\
81.9600 0.92076 \\
81.9958 0.92073 \\
82.0316 0.92071 \\
82.0674 0.92068 \\
82.1032 0.92065 \\
82.1390 0.92062 \\
82.1748 0.92060 \\
82.2106 0.92057 \\
82.2464 0.92054 \\
82.2823 0.92052 \\
82.3181 0.92049 \\
82.3539 0.92046 \\
82.3897 0.92043 \\
82.4255 0.92041 \\
82.4613 0.92038 \\
82.4971 0.92035 \\
82.5329 0.92032 \\
82.5687 0.92030 \\
82.6045 0.92027 \\
82.6403 0.92024 \\
82.6761 0.92021 \\
82.7119 0.92019 \\
82.7477 0.92016 \\
82.7836 0.92013 \\
82.8194 0.92010 \\
82.8552 0.92008 \\
82.8910 0.92005 \\
82.9268 0.92002 \\
82.9626 0.91999 \\
82.9984 0.91996 \\
83.0342 0.91994 \\
83.0700 0.91991 \\
83.1058 0.91988 \\
83.1416 0.91985 \\
83.1774 0.91982 \\
83.2132 0.91980 \\
83.2490 0.91977 \\
83.2849 0.91974 \\
83.3207 0.91971 \\
83.3565 0.91968 \\
83.3923 0.91966 \\
83.4281 0.91963 \\
83.4639 0.91960 \\
83.4997 0.91957 \\
83.5355 0.91954 \\
83.5713 0.91951 \\
83.6071 0.91949 \\
83.6429 0.91946 \\
83.6787 0.91943 \\
83.7145 0.91940 \\
83.7504 0.91937 \\
83.7862 0.91934 \\
83.8220 0.91931 \\
83.8578 0.91929 \\
83.8936 0.91926 \\
83.9294 0.91923 \\
83.9652 0.91920 \\
84.0010 0.91917 \\
84.0368 0.91914 \\
84.0726 0.91911 \\
84.1084 0.91908 \\
84.1442 0.91905 \\
84.1800 0.91903 \\
84.2158 0.91900 \\
84.2517 0.91897 \\
84.2875 0.91894 \\
84.3233 0.91891 \\
84.3591 0.91888 \\
84.3949 0.91885 \\
84.4307 0.91882 \\
84.4665 0.91879 \\
84.5023 0.91876 \\
84.5381 0.91873 \\
84.5739 0.91870 \\
84.6097 0.91868 \\
84.6455 0.91865 \\
84.6813 0.91862 \\
84.7171 0.91859 \\
84.7530 0.91856 \\
84.7888 0.91853 \\
84.8246 0.91850 \\
84.8604 0.91847 \\
84.8962 0.91844 \\
84.9320 0.91841 \\
84.9678 0.91838 \\
85.0036 0.91835 \\
85.0394 0.91832 \\
85.0752 0.91829 \\
85.1110 0.91826 \\
85.1468 0.91823 \\
85.1826 0.91820 \\
85.2184 0.91817 \\
85.2543 0.91814 \\
85.2901 0.91811 \\
85.3259 0.91808 \\
85.3617 0.91805 \\
85.3975 0.91802 \\
85.4333 0.91799 \\
85.4691 0.91796 \\
85.5049 0.91793 \\
85.5407 0.91790 \\
85.5765 0.91787 \\
85.6123 0.91784 \\
85.6481 0.91781 \\
85.6839 0.91778 \\
85.7197 0.91775 \\
85.7556 0.91772 \\
85.7914 0.91769 \\
85.8272 0.91766 \\
85.8630 0.91763 \\
85.8988 0.91760 \\
85.9346 0.91757 \\
85.9704 0.91754 \\
86.0062 0.91750 \\
86.0420 0.91747 \\
86.0778 0.91744 \\
86.1136 0.91741 \\
86.1494 0.91738 \\
86.1852 0.91735 \\
86.2210 0.91732 \\
86.2569 0.91729 \\
86.2927 0.91726 \\
86.3285 0.91723 \\
86.3643 0.91720 \\
86.4001 0.91717 \\
86.4359 0.91713 \\
86.4717 0.91710 \\
86.5075 0.91707 \\
86.5433 0.91704 \\
86.5791 0.91701 \\
86.6149 0.91698 \\
86.6507 0.91695 \\
86.6865 0.91692 \\
86.7223 0.91689 \\
86.7582 0.91685 \\
86.7940 0.91682 \\
86.8298 0.91679 \\
86.8656 0.91676 \\
86.9014 0.91673 \\
86.9372 0.91670 \\
86.9730 0.91667 \\
87.0088 0.91663 \\
87.0446 0.91660 \\
87.0804 0.91657 \\
87.1162 0.91654 \\
87.1520 0.91651 \\
87.1878 0.91648 \\
87.2236 0.91645 \\
87.2595 0.91641 \\
87.2953 0.91638 \\
87.3311 0.91635 \\
87.3669 0.91632 \\
87.4027 0.91629 \\
87.4385 0.91625 \\
87.4743 0.91622 \\
87.5101 0.91619 \\
87.5459 0.91616 \\
87.5817 0.91613 \\
87.6175 0.91609 \\
87.6533 0.91606 \\
87.6891 0.91603 \\
87.7249 0.91600 \\
87.7608 0.91597 \\
87.7966 0.91593 \\
87.8324 0.91590 \\
87.8682 0.91587 \\
87.9040 0.91584 \\
87.9398 0.91581 \\
87.9756 0.91577 \\
88.0114 0.91574 \\
88.0472 0.91571 \\
88.0830 0.91568 \\
88.1188 0.91564 \\
88.1546 0.91561 \\
88.1904 0.91558 \\
88.2262 0.91555 \\
88.2621 0.91551 \\
88.2979 0.91548 \\
88.3337 0.91545 \\
88.3695 0.91542 \\
88.4053 0.91538 \\
88.4411 0.91535 \\
88.4769 0.91532 \\
88.5127 0.91529 \\
88.5485 0.91525 \\
88.5843 0.91522 \\
88.6201 0.91519 \\
88.6559 0.91515 \\
88.6917 0.91512 \\
88.7275 0.91509 \\
88.7634 0.91506 \\
88.7992 0.91502 \\
88.8350 0.91499 \\
88.8708 0.91496 \\
88.9066 0.91492 \\
88.9424 0.91489 \\
88.9782 0.91486 \\
89.0140 0.91482 \\
89.0498 0.91479 \\
89.0856 0.91476 \\
89.1214 0.91472 \\
89.1572 0.91469 \\
89.1930 0.91466 \\
89.2288 0.91462 \\
89.2647 0.91459 \\
89.3005 0.91456 \\
89.3363 0.91452 \\
89.3721 0.91449 \\
89.4079 0.91446 \\
89.4437 0.91442 \\
89.4795 0.91439 \\
89.5153 0.91436 \\
89.5511 0.91432 \\
89.5869 0.91429 \\
89.6227 0.91426 \\
89.6585 0.91422 \\
89.6943 0.91419 \\
89.7301 0.91415 \\
89.7660 0.91412 \\
89.8018 0.91409 \\
89.8376 0.91405 \\
89.8734 0.91402 \\
89.9092 0.91399 \\
89.9450 0.91395 \\
89.9808 0.91392 \\
90.0000 0.91390 \\
90.1804 0.91372 \\
90.3607 0.91354 \\
90.5411 0.91336 \\
90.7214 0.91318 \\
90.9018 0.91299 \\
91.0822 0.91281 \\
91.2625 0.91263 \\
91.4429 0.91245 \\
91.6232 0.91227 \\
91.8036 0.91209 \\
91.9840 0.91191 \\
92.1643 0.91173 \\
92.3447 0.91154 \\
92.5251 0.91136 \\
92.7054 0.91118 \\
92.8858 0.91100 \\
93.0661 0.91082 \\
93.2465 0.91064 \\
93.4269 0.91046 \\
93.6072 0.91028 \\
93.7876 0.91010 \\
93.9679 0.90991 \\
94.1483 0.90973 \\
94.3287 0.90955 \\
94.5090 0.90937 \\
94.6894 0.90919 \\
94.8697 0.90901 \\
95.0501 0.90883 \\
95.2305 0.90865 \\
95.4108 0.90847 \\
95.5912 0.90828 \\
95.7715 0.90810 \\
95.9519 0.90792 \\
96.1323 0.90774 \\
96.3126 0.90756 \\
96.4930 0.90738 \\
96.6733 0.90720 \\
96.8537 0.90702 \\
97.0341 0.90683 \\
97.2144 0.90665 \\
97.3948 0.90647 \\
97.5752 0.90629 \\
97.7555 0.90611 \\
97.9359 0.90593 \\
98.1162 0.90575 \\
98.2966 0.90557 \\
98.4770 0.90539 \\
98.6573 0.90520 \\
98.8377 0.90502 \\
99.0180 0.90484 \\
99.1984 0.90466 \\
99.3788 0.90448 \\
99.5591 0.90430 \\
99.7395 0.90412 \\
99.9198 0.90394 \\
100.1002 0.90375 \\
100.2806 0.90357 \\
100.4609 0.90339 \\
100.6413 0.90321 \\
100.8216 0.90303 \\
101.0020 0.90285 \\
101.1824 0.90267 \\
101.3627 0.90249 \\
101.5431 0.90231 \\
101.7234 0.90212 \\
101.9038 0.90194 \\
102.0842 0.90176 \\
102.2645 0.90158 \\
102.4449 0.90140 \\
102.6253 0.90122 \\
102.8056 0.90104 \\
102.9860 0.90086 \\
103.1663 0.90068 \\
103.3467 0.90049 \\
103.5271 0.90031 \\
103.7074 0.90013 \\
103.8878 0.89995 \\
104.0681 0.89977 \\
104.2485 0.89959 \\
104.4289 0.89941 \\
104.6092 0.89923 \\
104.7896 0.89904 \\
104.9699 0.89886 \\
105.1503 0.89868 \\
105.3307 0.89850 \\
105.5110 0.89832 \\
105.6914 0.89814 \\
105.8717 0.89796 \\
106.0521 0.89778 \\
106.2325 0.89760 \\
106.4128 0.89741 \\
106.5932 0.89723 \\
106.7735 0.89705 \\
106.9539 0.89687 \\
107.1343 0.89669 \\
107.3146 0.89651 \\
107.4950 0.89633 \\
107.6754 0.89615 \\
107.8557 0.89596 \\
108.0361 0.89578 \\
108.2164 0.89560 \\
108.3968 0.89542 \\
108.5772 0.89524 \\
108.7575 0.89506 \\
108.9379 0.89488 \\
109.1182 0.89470 \\
109.2986 0.89452 \\
109.4790 0.89433 \\
109.6593 0.89415 \\
109.8397 0.89397 \\
110.0200 0.89379 \\
110.2004 0.89361 \\
110.3808 0.89343 \\
110.5611 0.89325 \\
110.7415 0.89307 \\
110.9218 0.89289 \\
111.1022 0.89270 \\
111.2826 0.89252 \\
111.4629 0.89234 \\
111.6433 0.89216 \\
111.8236 0.89198 \\
112.0040 0.89180 \\
112.1844 0.89162 \\
112.3647 0.89144 \\
112.5451 0.89125 \\
112.7255 0.89107 \\
112.9058 0.89089 \\
113.0862 0.89071 \\
113.2665 0.89053 \\
113.4469 0.89035 \\
113.6273 0.89017 \\
113.8076 0.88999 \\
113.9880 0.88981 \\
114.1683 0.88962 \\
114.3487 0.88944 \\
114.5291 0.88926 \\
114.7094 0.88908 \\
114.8898 0.88890 \\
115.0701 0.88872 \\
115.2505 0.88854 \\
115.4309 0.88836 \\
115.6112 0.88817 \\
115.7916 0.88799 \\
115.9719 0.88781 \\
116.1523 0.88763 \\
116.3327 0.88745 \\
116.5130 0.88727 \\
116.6934 0.88709 \\
116.8737 0.88691 \\
117.0541 0.88673 \\
117.2345 0.88654 \\
117.4148 0.88636 \\
117.5952 0.88618 \\
117.7756 0.88600 \\
117.9559 0.88582 \\
118.1363 0.88564 \\
118.3166 0.88546 \\
118.4970 0.88528 \\
118.6774 0.88510 \\
118.8577 0.88491 \\
119.0381 0.88473 \\
119.2184 0.88455 \\
119.3988 0.88437 \\
119.5792 0.88419 \\
119.7595 0.88401 \\
119.9399 0.88383 \\
120.1202 0.88365 \\
120.3006 0.88346 \\
120.4810 0.88328 \\
120.6613 0.88310 \\
120.8417 0.88292 \\
121.0220 0.88274 \\
121.2024 0.88256 \\
121.3828 0.88238 \\
121.5631 0.88220 \\
121.7435 0.88202 \\
121.9238 0.88183 \\
122.1042 0.88165 \\
122.2846 0.88147 \\
122.4649 0.88129 \\
122.6453 0.88111 \\
122.8257 0.88093 \\
123.0060 0.88075 \\
123.1864 0.88057 \\
123.3667 0.88038 \\
123.5471 0.88020 \\
123.7275 0.88002 \\
123.9078 0.87984 \\
124.0882 0.87966 \\
124.2685 0.87948 \\
124.4489 0.87930 \\
124.6293 0.87912 \\
124.8096 0.87894 \\
124.9900 0.87875 \\
125.1703 0.87857 \\
125.3507 0.87839 \\
125.5311 0.87821 \\
125.7114 0.87803 \\
125.8918 0.87785 \\
126.0721 0.87767 \\
126.2525 0.87749 \\
126.4329 0.87731 \\
126.6132 0.87712 \\
126.7936 0.87694 \\
126.9739 0.87676 \\
127.1543 0.87658 \\
127.3347 0.87640 \\
127.5150 0.87622 \\
127.6954 0.87604 \\
127.8758 0.87586 \\
128.0561 0.87567 \\
128.2365 0.87549 \\
128.4168 0.87531 \\
128.5972 0.87513 \\
128.7776 0.87495 \\
128.9579 0.87477 \\
129.1383 0.87459 \\
129.3186 0.87441 \\
129.4990 0.87423 \\
129.6794 0.87404 \\
129.8597 0.87386 \\
130.0401 0.87368 \\
130.2204 0.87350 \\
130.4008 0.87332 \\
130.5812 0.87314 \\
130.7615 0.87296 \\
130.9419 0.87278 \\
131.1222 0.87259 \\
131.3026 0.87241 \\
131.4830 0.87223 \\
131.6633 0.87205 \\
131.8437 0.87187 \\
132.0240 0.87169 \\
132.2044 0.87151 \\
132.3848 0.87133 \\
132.5651 0.87115 \\
132.7455 0.87096 \\
132.9259 0.87078 \\
133.1062 0.87060 \\
133.2866 0.87042 \\
133.4669 0.87024 \\
133.6473 0.87006 \\
133.8277 0.86988 \\
134.0080 0.86970 \\
134.1884 0.86952 \\
134.3687 0.86933 \\
134.5491 0.86915 \\
134.7295 0.86897 \\
134.9098 0.86879 \\
135.0902 0.86861 \\
135.2705 0.86843 \\
135.4509 0.86825 \\
135.6313 0.86807 \\
135.8116 0.86788 \\
135.9920 0.86770 \\
136.1723 0.86752 \\
136.3527 0.86734 \\
136.5331 0.86716 \\
136.7134 0.86698 \\
136.8938 0.86680 \\
137.0741 0.86662 \\
137.2545 0.86644 \\
137.4349 0.86625 \\
137.6152 0.86607 \\
137.7956 0.86589 \\
137.9760 0.86571 \\
138.1563 0.86553 \\
138.3367 0.86535 \\
138.5170 0.86517 \\
138.6974 0.86499 \\
138.8778 0.86481 \\
139.0581 0.86462 \\
139.2385 0.86444 \\
139.4188 0.86426 \\
139.5992 0.86408 \\
139.7796 0.86390 \\
139.9599 0.86372 \\
140.1403 0.86354 \\
140.3206 0.86336 \\
140.5010 0.86317 \\
140.6814 0.86299 \\
140.8617 0.86281 \\
141.0421 0.86263 \\
141.2224 0.86245 \\
141.4028 0.86227 \\
141.5832 0.86209 \\
141.7635 0.86191 \\
141.9439 0.86173 \\
142.1242 0.86154 \\
142.3046 0.86136 \\
142.4850 0.86118 \\
142.6653 0.86100 \\
142.8457 0.86082 \\
143.0261 0.86064 \\
143.2064 0.86046 \\
143.3868 0.86028 \\
143.5671 0.86009 \\
143.7475 0.85991 \\
143.9279 0.85973 \\
144.1082 0.85955 \\
144.2886 0.85937 \\
144.4689 0.85919 \\
144.6493 0.85901 \\
144.8297 0.85883 \\
145.0100 0.85865 \\
145.1904 0.85846 \\
145.3707 0.85828 \\
145.5511 0.85810 \\
145.7315 0.85792 \\
145.9118 0.85774 \\
146.0922 0.85756 \\
146.2725 0.85738 \\
146.4529 0.85720 \\
146.6333 0.85702 \\
146.8136 0.85683 \\
146.9940 0.85665 \\
147.1743 0.85647 \\
147.3547 0.85629 \\
147.5351 0.85611 \\
147.7154 0.85593 \\
147.8958 0.85575 \\
148.0762 0.85557 \\
148.2565 0.85538 \\
148.4369 0.85520 \\
148.6172 0.85502 \\
148.7976 0.85484 \\
148.9780 0.85466 \\
149.1583 0.85448 \\
149.3387 0.85430 \\
149.5190 0.85412 \\
149.6994 0.85394 \\
149.8798 0.85375 \\
150.0601 0.85357 \\
150.2405 0.85339 \\
150.4208 0.85321 \\
150.6012 0.85303 \\
150.7816 0.85285 \\
150.9619 0.85267 \\
151.1423 0.85249 \\
151.3226 0.85230 \\
151.5030 0.85212 \\
151.6834 0.85194 \\
151.8637 0.85176 \\
152.0441 0.85158 \\
152.2244 0.85140 \\
152.4048 0.85122 \\
152.5852 0.85104 \\
152.7655 0.85086 \\
152.9459 0.85067 \\
153.1263 0.85049 \\
153.3066 0.85031 \\
153.4870 0.85013 \\
153.6673 0.84995 \\
153.8477 0.84977 \\
154.0281 0.84959 \\
154.2084 0.84941 \\
154.3888 0.84923 \\
154.5691 0.84904 \\
154.7495 0.84886 \\
154.9299 0.84868 \\
155.1102 0.84850 \\
155.2906 0.84832 \\
155.4709 0.84814 \\
155.6513 0.84796 \\
155.8317 0.84778 \\
156.0120 0.84759 \\
156.1924 0.84741 \\
156.3727 0.84723 \\
156.5531 0.84705 \\
156.7335 0.84687 \\
156.9138 0.84669 \\
157.0942 0.84651 \\
157.2745 0.84633 \\
157.4549 0.84615 \\
157.6353 0.84596 \\
157.8156 0.84578 \\
157.9960 0.84560 \\
158.1764 0.84542 \\
158.3567 0.84524 \\
158.5371 0.84506 \\
158.7174 0.84488 \\
158.8978 0.84470 \\
159.0782 0.84451 \\
159.2585 0.84433 \\
159.4389 0.84415 \\
159.6192 0.84397 \\
159.7996 0.84379 \\
159.9800 0.84361 \\
160.1603 0.84343 \\
160.3407 0.84325 \\
160.5210 0.84307 \\
160.7014 0.84288 \\
160.8818 0.84270 \\
161.0621 0.84252 \\
161.2425 0.84234 \\
161.4228 0.84216 \\
161.6032 0.84198 \\
161.7836 0.84180 \\
161.9639 0.84162 \\
162.1443 0.84144 \\
162.3246 0.84125 \\
162.5050 0.84107 \\
162.6854 0.84089 \\
162.8657 0.84071 \\
163.0461 0.84053 \\
163.2265 0.84035 \\
163.4068 0.84017 \\
163.5872 0.83999 \\
163.7675 0.83980 \\
163.9479 0.83962 \\
164.1283 0.83944 \\
164.3086 0.83926 \\
164.4890 0.83908 \\
164.6693 0.83890 \\
164.8497 0.83872 \\
165.0301 0.83854 \\
165.2104 0.83836 \\
165.3908 0.83817 \\
165.5711 0.83799 \\
165.7515 0.83781 \\
165.9319 0.83763 \\
166.1122 0.83745 \\
166.2926 0.83727 \\
166.4729 0.83709 \\
166.6533 0.83691 \\
166.8337 0.83672 \\
167.0140 0.83654 \\
167.1944 0.83636 \\
167.3747 0.83618 \\
167.5551 0.83600 \\
167.7355 0.83582 \\
167.9158 0.83564 \\
168.0962 0.83546 \\
168.2766 0.83528 \\
168.4569 0.83509 \\
168.6373 0.83491 \\
168.8176 0.83473 \\
168.9980 0.83455 \\
169.1784 0.83437 \\
169.3587 0.83419 \\
169.5391 0.83401 \\
169.7194 0.83383 \\
169.8998 0.83365 \\
170.0802 0.83346 \\
170.2605 0.83328 \\
170.4409 0.83310 \\
170.6212 0.83292 \\
170.8016 0.83274 \\
170.9820 0.83256 \\
171.1623 0.83238 \\
171.3427 0.83220 \\
171.5230 0.83201 \\
171.7034 0.83183 \\
171.8838 0.83165 \\
172.0641 0.83147 \\
172.2445 0.83129 \\
172.4248 0.83111 \\
172.6052 0.83093 \\
172.7856 0.83075 \\
172.9659 0.83057 \\
173.1463 0.83038 \\
173.3267 0.83020 \\
173.5070 0.83002 \\
173.6874 0.82984 \\
173.8677 0.82966 \\
174.0481 0.82948 \\
174.2285 0.82930 \\
174.4088 0.82912 \\
174.5892 0.82893 \\
174.7695 0.82875 \\
174.9499 0.82857 \\
175.1303 0.82839 \\
175.3106 0.82821 \\
175.4910 0.82803 \\
175.6713 0.82785 \\
175.8517 0.82767 \\
176.0321 0.82749 \\
176.2124 0.82730 \\
176.3928 0.82712 \\
176.5731 0.82694 \\
176.7535 0.82676 \\
176.9339 0.82658 \\
177.1142 0.82640 \\
177.2946 0.82622 \\
177.4749 0.82604 \\
177.6553 0.82586 \\
177.8357 0.82567 \\
178.0160 0.82549 \\
178.1964 0.82531 \\
178.3768 0.82513 \\
178.5571 0.82495 \\
178.7375 0.82477 \\
178.9178 0.82459 \\
179.0982 0.82441 \\
179.2786 0.82422 \\
179.4589 0.82404 \\
179.6393 0.82386 \\
179.8196 0.82368 \\
180.0000 0.82350 \\
};
\addlegendentry{Misaligned Communication}






\addplot [line width=0.6mm, color=mycolor2]
  table[row sep=crcr]{%
1.250000	0.998900\\
1.428929	0.998798\\
1.607858	0.998697\\
1.786787	0.998678\\
1.965716	0.998559\\
2.144645	0.998495\\
2.323574	0.998347\\
2.502503	0.998295\\
2.681431	0.998206\\
2.860360	0.998095\\
3.039289	0.998008\\
3.218218	0.997895\\
3.397147	0.997799\\
3.576076	0.997713\\
3.755005	0.997596\\
3.933934	0.997482\\
4.112863	0.997383\\
4.291792	0.997298\\
4.470721	0.997115\\
4.649650	0.997083\\
4.828579	0.996917\\
5.007508	0.996898\\
5.186436	0.996767\\
5.365365	0.996656\\
5.544294	0.996546\\
5.723223	0.996446\\
5.902152	0.996348\\
6.081081	0.996267\\
6.260010	0.996156\\
6.438939	0.995992\\
6.617868	0.995913\\
6.796797	0.995859\\
6.975726	0.995767\\
7.154655	0.995634\\
7.333584	0.995506\\
7.512513	0.995392\\
7.691441	0.995282\\
7.870370	0.995173\\
8.049299	0.995070\\
8.228228	0.994977\\
8.407157	0.994891\\
8.586086	0.994791\\
8.765015	0.994663\\
8.943944	0.994535\\
9.122873	0.994437\\
9.301802	0.994366\\
9.480731	0.994297\\
9.659660	0.994209\\
9.838589	0.994105\\
10.017518	0.993988\\
10.196446	0.993866\\
10.375375	0.993742\\
10.554304	0.993622\\
10.733233	0.993510\\
10.912162	0.993403\\
11.091091	0.993297\\
11.270020	0.993187\\
11.448949	0.993070\\
11.627878	0.992949\\
11.806807	0.992827\\
11.985736	0.992709\\
12.164665	0.992598\\
12.343594	0.992492\\
12.522523	0.992390\\
12.701451	0.992289\\
12.880380	0.992187\\
13.059309	0.992081\\
13.238238	0.991973\\
13.417167	0.991863\\
13.596096	0.991753\\
13.775025	0.991643\\
13.953954	0.991535\\
14.132883	0.991427\\
14.311812	0.991320\\
14.490741	0.991213\\
14.669670	0.991105\\
14.848599	0.990995\\
15.027528	0.990882\\
15.206456	0.990767\\
15.385385	0.990649\\
15.564314	0.990530\\
15.743243	0.990410\\
15.922172	0.990291\\
16.101101	0.990172\\
16.280030	0.990054\\
16.458959	0.989939\\
16.637888	0.989826\\
16.816817	0.989716\\
16.995746	0.989606\\
17.174675	0.989498\\
17.353604	0.989391\\
17.532533	0.989283\\
17.711461	0.989176\\
17.890390	0.989067\\
18.069319	0.988957\\
18.248248	0.988846\\
18.427177	0.988733\\
18.606106	0.988619\\
18.785035	0.988504\\
18.963964	0.988388\\
19.142893	0.988271\\
19.321822	0.988153\\
19.500751	0.988034\\
19.679680	0.987915\\
19.858609	0.987795\\
20.037538	0.987675\\
20.216466	0.987554\\
20.395395	0.987433\\
20.574324	0.987312\\
20.753253	0.987191\\
20.932182	0.987070\\
21.111111	0.986948\\
21.290040	0.986826\\
21.468969	0.986704\\
21.647898	0.986582\\
21.826827	0.986460\\
22.005756	0.986338\\
22.184685	0.986216\\
22.363614	0.986093\\
22.542543	0.985971\\
22.721471	0.985849\\
22.900400	0.985726\\
23.079329	0.985604\\
23.258258	0.985482\\
23.437187	0.985359\\
23.616116	0.985237\\
23.795045	0.985115\\
23.973974	0.984992\\
24.152903	0.984870\\
24.331832	0.984747\\
24.510761	0.984625\\
24.689690	0.984502\\
24.868619	0.984380\\
25.047548	0.984257\\
25.226476	0.984135\\
25.405405	0.984012\\
25.584334	0.983889\\
25.763263	0.983766\\
25.942192	0.983643\\
26.121121	0.983521\\
26.300050	0.983397\\
26.478979	0.983274\\
26.657908	0.983151\\
26.836837	0.983027\\
27.015766	0.982904\\
27.194695	0.982780\\
27.373624	0.982656\\
27.552553	0.982531\\
27.731481	0.982407\\
27.910410	0.982282\\
28.089339	0.982157\\
28.268268	0.982032\\
28.447197	0.981906\\
28.626126	0.981780\\
28.805055	0.981654\\
28.983984	0.981527\\
29.162913	0.981400\\
29.341842	0.981273\\
29.520771	0.981145\\
29.699700	0.981016\\
29.878629	0.980888\\
30.057558	0.980758\\
30.236486	0.980629\\
30.415415	0.980499\\
30.594344	0.980368\\
30.773273	0.980237\\
30.952202	0.980106\\
31.131131	0.979974\\
31.310060	0.979842\\
31.488989	0.979709\\
31.667918	0.979576\\
31.846847	0.979443\\
32.025776	0.979310\\
32.204705	0.979176\\
32.383634	0.979042\\
32.562563	0.978908\\
32.741491	0.978773\\
32.920420	0.978638\\
33.099349	0.978503\\
33.278278	0.978368\\
33.457207	0.978233\\
33.636136	0.978097\\
33.815065	0.977962\\
33.993994	0.977826\\
34.172923	0.977690\\
34.351852	0.977554\\
34.530781	0.977418\\
34.709710	0.977282\\
34.888639	0.977146\\
35.067568	0.977010\\
35.246496	0.976873\\
35.425425	0.976737\\
35.604354	0.976601\\
35.783283	0.976465\\
35.962212	0.976329\\
36.141141	0.976193\\
36.320070	0.976057\\
36.498999	0.975921\\
36.677928	0.975785\\
36.856857	0.975649\\
37.035786	0.975513\\
37.214715	0.975377\\
37.393644	0.975242\\
37.572573	0.975106\\
37.751502	0.974970\\
37.930430	0.974835\\
38.109359	0.974699\\
38.288288	0.974563\\
38.467217	0.974427\\
38.646146	0.974292\\
38.825075	0.974156\\
39.004004	0.974020\\
39.182933	0.973884\\
39.361862	0.973748\\
39.540791	0.973612\\
39.719720	0.973476\\
39.898649	0.973340\\
40.077578	0.973204\\
40.256507	0.973067\\
40.435435	0.972931\\
40.614364	0.972795\\
40.793293	0.972658\\
40.972222	0.972521\\
41.151151	0.972385\\
41.330080	0.972248\\
41.509009	0.972111\\
41.687938	0.971973\\
41.866867	0.971836\\
42.045796	0.971698\\
42.224725	0.971561\\
42.403654	0.971423\\
42.582583	0.971285\\
42.761512	0.971147\\
42.940440	0.971008\\
43.119369	0.970870\\
43.298298	0.970731\\
43.477227	0.970592\\
43.656156	0.970453\\
43.835085	0.970313\\
44.014014	0.970174\\
44.192943	0.970034\\
44.371872	0.969894\\
44.550801	0.969753\\
44.729730	0.969613\\
44.908659	0.969472\\
45.087588	0.969331\\
45.266517	0.969189\\
45.445445	0.969048\\
45.624374	0.968906\\
45.803303	0.968764\\
45.982232	0.968621\\
46.161161	0.968478\\
46.340090	0.968335\\
46.519019	0.968192\\
46.697948	0.968049\\
46.876877	0.967905\\
47.055806	0.967761\\
47.234735	0.967617\\
47.413664	0.967472\\
47.592593	0.967327\\
47.771522	0.967182\\
47.950450	0.967037\\
48.129379	0.966892\\
48.308308	0.966746\\
48.487237	0.966600\\
48.666166	0.966454\\
48.845095	0.966308\\
49.024024	0.966161\\
49.202953	0.966014\\
49.381882	0.965867\\
49.560811	0.965720\\
49.739740	0.965572\\
49.918669	0.965424\\
50.097598	0.965276\\
50.276527	0.965128\\
50.455455	0.964980\\
50.634384	0.964831\\
50.813313	0.964682\\
50.992242	0.964533\\
51.171171	0.964384\\
51.350100	0.964234\\
51.529029	0.964085\\
51.707958	0.963935\\
51.886887	0.963785\\
52.065816	0.963635\\
52.244745	0.963484\\
52.423674	0.963333\\
52.602603	0.963183\\
52.781532	0.963031\\
52.960460	0.962880\\
53.139389	0.962729\\
53.318318	0.962577\\
53.497247	0.962425\\
53.676176	0.962273\\
53.855105	0.962121\\
54.034034	0.961969\\
54.212963	0.961816\\
54.391892	0.961664\\
54.570821	0.961511\\
54.749750	0.961358\\
54.928679	0.961205\\
55.107608	0.961051\\
55.286537	0.960898\\
55.465465	0.960744\\
55.644394	0.960590\\
55.823323	0.960436\\
56.002252	0.960282\\
56.181181	0.960128\\
56.360110	0.959973\\
56.539039	0.959818\\
56.717968	0.959664\\
56.896897	0.959509\\
57.075826	0.959354\\
57.254755	0.959198\\
57.433684	0.959043\\
57.612613	0.958888\\
57.791542	0.958732\\
57.970470	0.958576\\
58.149399	0.958420\\
58.328328	0.958264\\
58.507257	0.958108\\
58.686186	0.957952\\
58.865115	0.957795\\
59.044044	0.957639\\
59.222973	0.957482\\
59.401902	0.957325\\
59.580831	0.957168\\
59.759760	0.957011\\
59.938689	0.956854\\
60.117618	0.956697\\
60.296547	0.956539\\
60.475475	0.956382\\
60.654404	0.956224\\
60.833333	0.956066\\
61.012262	0.955908\\
61.191191	0.955750\\
61.370120	0.955592\\
61.549049	0.955434\\
61.727978	0.955276\\
61.906907	0.955117\\
62.085836	0.954959\\
62.264765	0.954800\\
62.443694	0.954641\\
62.622623	0.954483\\
62.801552	0.954324\\
62.980480	0.954165\\
63.159409	0.954005\\
63.338338	0.953846\\
63.517267	0.953687\\
63.696196	0.953527\\
63.875125	0.953367\\
64.054054	0.953208\\
64.232983	0.953048\\
64.411912	0.952888\\
64.590841	0.952728\\
64.769770	0.952567\\
64.948699	0.952407\\
65.127628	0.952247\\
65.306557	0.952086\\
65.485485	0.951925\\
65.664414	0.951765\\
65.843343	0.951604\\
66.022272	0.951443\\
66.201201	0.951282\\
66.380130	0.951120\\
66.559059	0.950959\\
66.737988	0.950797\\
66.916917	0.950636\\
67.095846	0.950474\\
67.274775	0.950312\\
67.453704	0.950151\\
67.632633	0.949989\\
67.811562	0.949826\\
67.990490	0.949664\\
68.169419	0.949502\\
68.348348	0.949340\\
68.527277	0.949177\\
68.706206	0.949014\\
68.885135	0.948852\\
69.064064	0.948689\\
69.242993	0.948526\\
69.421922	0.948363\\
69.600851	0.948199\\
69.779780	0.948036\\
69.958709	0.947873\\
70.137638	0.947709\\
70.316567	0.947545\\
70.495495	0.947382\\
70.674424	0.947218\\
70.853353	0.947054\\
71.032282	0.946890\\
71.211211	0.946726\\
71.390140	0.946561\\
71.569069	0.946397\\
71.747998	0.946232\\
71.926927	0.946068\\
72.105856	0.945903\\
72.284785	0.945738\\
72.463714	0.945573\\
72.642643	0.945408\\
72.821572	0.945243\\
73.000501	0.945078\\
73.179429	0.944912\\
73.358358	0.944747\\
73.537287	0.944581\\
73.716216	0.944416\\
73.895145	0.944250\\
74.074074	0.944084\\
74.253003	0.943918\\
74.431932	0.943752\\
74.610861	0.943586\\
74.789790	0.943419\\
74.968719	0.943253\\
75.147648	0.943086\\
75.326577	0.942920\\
75.505506	0.942753\\
75.684434	0.942586\\
75.863363	0.942419\\
76.042292	0.942252\\
76.221221	0.942085\\
76.400150	0.941918\\
76.579079	0.941750\\
76.758008	0.941583\\
76.936937	0.941415\\
77.115866	0.941247\\
77.294795	0.941079\\
77.473724	0.940912\\
77.652653	0.940744\\
77.831582	0.940575\\
78.010511	0.940407\\
78.189439	0.940239\\
78.368368	0.940070\\
78.547297	0.939902\\
78.726226	0.939733\\
78.905155	0.939564\\
79.084084	0.939396\\
79.263013	0.939227\\
79.441942	0.939058\\
79.620871	0.938888\\
79.799800	0.938719\\
79.978729	0.938550\\
80.157658	0.938380\\
80.336587	0.938211\\
80.515516	0.938041\\
80.694444	0.937871\\
80.873373	0.937701\\
81.052302	0.937531\\
81.231231	0.937361\\
81.410160	0.937191\\
81.589089	0.937021\\
81.768018	0.936850\\
81.946947	0.936680\\
82.125876	0.936509\\
82.304805	0.936338\\
82.483734	0.936167\\
82.662663	0.935997\\
82.841592	0.935825\\
83.020521	0.935654\\
83.199449	0.935483\\
83.378378	0.935312\\
83.557307	0.935140\\
83.736236	0.934969\\
83.915165	0.934797\\
84.094094	0.934625\\
84.273023	0.934454\\
84.451952	0.934282\\
84.630881	0.934110\\
84.809810	0.933937\\
84.988739	0.933765\\
85.167668	0.933593\\
85.346597	0.933420\\
85.525526	0.933248\\
85.704454	0.933075\\
85.883383	0.932902\\
86.062312	0.932729\\
86.241241	0.932557\\
86.420170	0.932383\\
86.599099	0.932210\\
86.778028	0.932037\\
86.956957	0.931864\\
87.135886	0.931690\\
87.314815	0.931517\\
87.493744	0.931343\\
87.672673	0.931169\\
87.851602	0.930995\\
88.030531	0.930821\\
88.209459	0.930647\\
88.388388	0.930473\\
88.567317	0.930299\\
88.746246	0.930125\\
88.925175	0.929950\\
89.104104	0.929776\\
89.283033	0.929601\\
89.461962	0.929426\\
89.640891	0.929251\\
89.819820	0.929076\\
89.998749	0.928901\\
90.177678	0.928726\\
90.356607	0.928551\\
90.535536	0.928375\\
90.714464	0.928200\\
90.893393	0.928024\\
91.072322	0.927849\\
91.251251	0.927673\\
91.430180	0.927497\\
91.609109	0.927321\\
91.788038	0.927145\\
91.966967	0.926969\\
92.145896	0.926793\\
92.324825	0.926617\\
92.503754	0.926440\\
92.682683	0.926264\\
92.861612	0.926087\\
93.040541	0.925910\\
93.219469	0.925733\\
93.398398	0.925557\\
93.577327	0.925380\\
93.756256	0.925202\\
93.935185	0.925025\\
94.114114	0.924848\\
94.293043	0.924671\\
94.471972	0.924493\\
94.650901	0.924315\\
94.829830	0.924138\\
95.008759	0.923960\\
95.187688	0.923782\\
95.366617	0.923604\\
95.545546	0.923426\\
95.724474	0.923248\\
95.903403	0.923070\\
96.082332	0.922891\\
96.261261	0.922713\\
96.440190	0.922534\\
96.619119	0.922356\\
96.798048	0.922177\\
96.976977	0.921998\\
97.155906	0.921819\\
97.334835	0.921640\\
97.513764	0.921461\\
97.692693	0.921282\\
97.871622	0.921102\\
98.050551	0.920923\\
98.229479	0.920744\\
98.408408	0.920564\\
98.587337	0.920384\\
98.766266	0.920205\\
98.945195	0.920025\\
99.124124	0.919845\\
99.303053	0.919665\\
99.481982	0.919484\\
99.660911	0.919304\\
99.839840	0.919124\\
100.018769	0.918943\\
100.197698	0.918763\\
100.376627	0.918582\\
100.555556	0.918402\\
100.734484	0.918221\\
100.913413	0.918040\\
101.092342	0.917859\\
101.271271	0.917678\\
101.450200	0.917497\\
101.629129	0.917315\\
101.808058	0.917134\\
101.986987	0.916953\\
102.165916	0.916771\\
102.344845	0.916589\\
102.523774	0.916408\\
102.702703	0.916226\\
102.881632	0.916044\\
103.060561	0.915862\\
103.239489	0.915680\\
103.418418	0.915498\\
103.597347	0.915315\\
103.776276	0.915133\\
103.955205	0.914950\\
104.134134	0.914768\\
104.313063	0.914585\\
104.491992	0.914402\\
104.670921	0.914220\\
104.849850	0.914037\\
105.028779	0.913854\\
105.207708	0.913671\\
105.386637	0.913487\\
105.565566	0.913304\\
105.744494	0.913121\\
105.923423	0.912937\\
106.102352	0.912754\\
106.281281	0.912570\\
106.460210	0.912386\\
106.639139	0.912202\\
106.818068	0.912018\\
106.996997	0.911834\\
107.175926	0.911650\\
107.354855	0.911466\\
107.533784	0.911282\\
107.712713	0.911097\\
107.891642	0.910913\\
108.070571	0.910728\\
108.249499	0.910544\\
108.428428	0.910359\\
108.607357	0.910174\\
108.786286	0.909989\\
108.965215	0.909804\\
109.144144	0.909619\\
109.323073	0.909434\\
109.502002	0.909249\\
109.680931	0.909064\\
109.859860	0.908878\\
110.038789	0.908693\\
110.217718	0.908507\\
110.396647	0.908321\\
110.575576	0.908135\\
110.754505	0.907950\\
110.933433	0.907764\\
111.112362	0.907578\\
111.291291	0.907391\\
111.470220	0.907205\\
111.649149	0.907019\\
111.828078	0.906832\\
112.007007	0.906646\\
112.185936	0.906459\\
112.364865	0.906273\\
112.543794	0.906086\\
112.722723	0.905899\\
112.901652	0.905712\\
113.080581	0.905525\\
113.259510	0.905338\\
113.438438	0.905151\\
113.617367	0.904964\\
113.796296	0.904776\\
113.975225	0.904589\\
114.154154	0.904401\\
114.333083	0.904214\\
114.512012	0.904026\\
114.690941	0.903838\\
114.869870	0.903650\\
115.048799	0.903462\\
115.227728	0.903274\\
115.406657	0.903086\\
115.585586	0.902898\\
115.764515	0.902709\\
115.943443	0.902521\\
116.122372	0.902333\\
116.301301	0.902144\\
116.480230	0.901955\\
116.659159	0.901767\\
116.838088	0.901578\\
117.017017	0.901389\\
117.195946	0.901200\\
117.374875	0.901011\\
117.553804	0.900822\\
117.732733	0.900632\\
117.911662	0.900443\\
118.090591	0.900254\\
118.269520	0.900064\\
118.448448	0.899875\\
118.627377	0.899685\\
118.806306	0.899495\\
118.985235	0.899305\\
119.164164	0.899115\\
119.343093	0.898925\\
119.522022	0.898735\\
119.700951	0.898545\\
119.879880	0.898355\\
120.058809	0.898165\\
120.237738	0.897974\\
120.416667	0.897784\\
120.595596	0.897593\\
120.774525	0.897402\\
120.953453	0.897212\\
121.132382	0.897021\\
121.311311	0.896830\\
121.490240	0.896639\\
121.669169	0.896448\\
121.848098	0.896257\\
122.027027	0.896066\\
122.205956	0.895874\\
122.384885	0.895683\\
122.563814	0.895491\\
122.742743	0.895300\\
122.921672	0.895108\\
123.100601	0.894916\\
123.279530	0.894725\\
123.458458	0.894533\\
123.637387	0.894341\\
123.816316	0.894149\\
123.995245	0.893956\\
124.174174	0.893764\\
124.353103	0.893572\\
124.532032	0.893380\\
124.710961	0.893187\\
124.889890	0.892995\\
125.068819	0.892802\\
125.247748	0.892609\\
125.426677	0.892416\\
125.605606	0.892224\\
125.784535	0.892031\\
125.963463	0.891838\\
126.142392	0.891644\\
126.321321	0.891451\\
126.500250	0.891258\\
126.679179	0.891065\\
126.858108	0.890871\\
127.037037	0.890678\\
127.215966	0.890484\\
127.394895	0.890290\\
127.573824	0.890097\\
127.752753	0.889903\\
127.931682	0.889709\\
128.110611	0.889515\\
128.289540	0.889321\\
128.468468	0.889127\\
128.647397	0.888933\\
128.826326	0.888738\\
129.005255	0.888544\\
129.184184	0.888349\\
129.363113	0.888155\\
129.542042	0.887960\\
129.720971	0.887766\\
129.899900	0.887571\\
130.078829	0.887376\\
130.257758	0.887181\\
130.436687	0.886986\\
130.615616	0.886791\\
130.794545	0.886596\\
130.973473	0.886401\\
131.152402	0.886205\\
131.331331	0.886010\\
131.510260	0.885814\\
131.689189	0.885619\\
131.868118	0.885423\\
132.047047	0.885228\\
132.225976	0.885032\\
132.404905	0.884836\\
132.583834	0.884640\\
132.762763	0.884444\\
132.941692	0.884248\\
133.120621	0.884052\\
133.299550	0.883855\\
133.478478	0.883659\\
133.657407	0.883463\\
133.836336	0.883266\\
134.015265	0.883070\\
134.194194	0.882873\\
134.373123	0.882676\\
134.552052	0.882480\\
134.730981	0.882283\\
134.909910	0.882086\\
135.088839	0.881889\\
135.267768	0.881692\\
135.446697	0.881495\\
135.625626	0.881297\\
135.804555	0.881100\\
135.983483	0.880903\\
136.162412	0.880705\\
136.341341	0.880508\\
136.520270	0.880310\\
136.699199	0.880113\\
136.878128	0.879915\\
137.057057	0.879717\\
137.235986	0.879519\\
137.414915	0.879321\\
137.593844	0.879123\\
137.772773	0.878925\\
137.951702	0.878727\\
138.130631	0.878528\\
138.309560	0.878330\\
138.488488	0.878132\\
138.667417	0.877933\\
138.846346	0.877735\\
139.025275	0.877536\\
139.204204	0.877337\\
139.383133	0.877139\\
139.562062	0.876940\\
139.740991	0.876741\\
139.919920	0.876542\\
140.098849	0.876343\\
140.277778	0.876144\\
140.456707	0.875944\\
140.635636	0.875745\\
140.814565	0.875546\\
140.993493	0.875346\\
141.172422	0.875147\\
141.351351	0.874947\\
141.530280	0.874747\\
141.709209	0.874548\\
141.888138	0.874348\\
142.067067	0.874148\\
142.245996	0.873948\\
142.424925	0.873748\\
142.603854	0.873548\\
142.782783	0.873348\\
142.961712	0.873148\\
143.140641	0.872947\\
143.319570	0.872747\\
143.498498	0.872547\\
143.677427	0.872346\\
143.856356	0.872145\\
144.035285	0.871945\\
144.214214	0.871744\\
144.393143	0.871543\\
144.572072	0.871342\\
144.751001	0.871141\\
144.929930	0.870940\\
145.108859	0.870739\\
145.287788	0.870538\\
145.466717	0.870337\\
145.645646	0.870136\\
145.824575	0.869934\\
146.003504	0.869733\\
146.182432	0.869531\\
146.361361	0.869330\\
146.540290	0.869128\\
146.719219	0.868926\\
146.898148	0.868725\\
147.077077	0.868523\\
147.256006	0.868321\\
147.434935	0.868119\\
147.613864	0.867917\\
147.792793	0.867715\\
147.971722	0.867513\\
148.150651	0.867310\\
148.329580	0.867108\\
148.508509	0.866906\\
148.687437	0.866703\\
148.866366	0.866501\\
149.045295	0.866298\\
149.224224	0.866095\\
149.403153	0.865893\\
149.582082	0.865690\\
149.761011	0.865487\\
149.939940	0.865284\\
150.118869	0.865081\\
150.297798	0.864878\\
150.476727	0.864675\\
150.655656	0.864471\\
150.834585	0.864268\\
151.013514	0.864065\\
151.192442	0.863861\\
151.371371	0.863658\\
151.550300	0.863454\\
151.729229	0.863251\\
151.908158	0.863047\\
152.087087	0.862843\\
152.266016	0.862640\\
152.444945	0.862436\\
152.623874	0.862232\\
152.802803	0.862028\\
152.981732	0.861824\\
153.160661	0.861619\\
153.339590	0.861415\\
153.518519	0.861211\\
153.697447	0.861007\\
153.876376	0.860802\\
154.055305	0.860598\\
154.234234	0.860393\\
154.413163	0.860189\\
154.592092	0.859984\\
154.771021	0.859779\\
154.949950	0.859574\\
155.128879	0.859369\\
155.307808	0.859165\\
155.486737	0.858960\\
155.665666	0.858754\\
155.844595	0.858549\\
156.023524	0.858344\\
156.202452	0.858139\\
156.381381	0.857934\\
156.560310	0.857728\\
156.739239	0.857523\\
156.918168	0.857317\\
157.097097	0.857112\\
157.276026	0.856906\\
157.454955	0.856700\\
157.633884	0.856494\\
157.812813	0.856289\\
157.991742	0.856083\\
158.170671	0.855877\\
158.349600	0.855671\\
158.528529	0.855465\\
158.707457	0.855258\\
158.886386	0.855052\\
159.065315	0.854846\\
159.244244	0.854640\\
159.423173	0.854433\\
159.602102	0.854227\\
159.781031	0.854020\\
159.959960	0.853814\\
160.138889	0.853607\\
160.317818	0.853400\\
160.496747	0.853193\\
160.675676	0.852987\\
160.854605	0.852780\\
161.033534	0.852573\\
161.212462	0.852366\\
161.391391	0.852159\\
161.570320	0.851951\\
161.749249	0.851744\\
161.928178	0.851537\\
162.107107	0.851330\\
162.286036	0.851122\\
162.464965	0.850915\\
162.643894	0.850707\\
162.822823	0.850500\\
163.001752	0.850292\\
163.180681	0.850084\\
163.359610	0.849876\\
163.538539	0.849669\\
163.717467	0.849461\\
163.896396	0.849253\\
164.075325	0.849045\\
164.254254	0.848837\\
164.433183	0.848629\\
164.612112	0.848420\\
164.791041	0.848212\\
164.969970	0.848004\\
165.148899	0.847795\\
165.327828	0.847587\\
165.506757	0.847378\\
165.685686	0.847170\\
165.864615	0.846961\\
166.043544	0.846753\\
166.222472	0.846544\\
166.401401	0.846335\\
166.580330	0.846126\\
166.759259	0.845917\\
166.938188	0.845708\\
167.117117	0.845499\\
167.296046	0.845290\\
167.474975	0.845081\\
167.653904	0.844872\\
167.832833	0.844663\\
168.011762	0.844453\\
168.190691	0.844244\\
168.369620	0.844034\\
168.548549	0.843825\\
168.727477	0.843615\\
168.906406	0.843406\\
169.085335	0.843196\\
169.264264	0.842986\\
169.443193	0.842777\\
169.622122	0.842567\\
169.801051	0.842357\\
169.979980	0.842147\\
170.158909	0.841937\\
170.337838	0.841727\\
170.516767	0.841517\\
170.695696	0.841306\\
170.874625	0.841096\\
171.053554	0.840886\\
171.232482	0.840676\\
171.411411	0.840465\\
171.590340	0.840255\\
171.769269	0.840044\\
171.948198	0.839834\\
172.127127	0.839623\\
172.306056	0.839412\\
172.484985	0.839201\\
172.663914	0.838991\\
172.842843	0.838780\\
173.021772	0.838569\\
173.200701	0.838358\\
173.379630	0.838147\\
173.558559	0.837936\\
173.737487	0.837725\\
173.916416	0.837513\\
174.095345	0.837302\\
174.274274	0.837091\\
174.453203	0.836879\\
174.632132	0.836668\\
174.811061	0.836456\\
174.989990	0.836245\\
175.168919	0.836033\\
175.347848	0.835822\\
175.526777	0.835610\\
175.705706	0.835398\\
175.884635	0.835186\\
176.063564	0.834975\\
176.242492	0.834763\\
176.421421	0.834551\\
176.600350	0.834339\\
176.779279	0.834127\\
176.958208	0.833914\\
177.137137	0.833702\\
177.316066	0.833490\\
177.494995	0.833278\\
177.673924	0.833065\\
177.852853	0.832853\\
178.031782	0.832641\\
178.210711	0.832428\\
178.389640	0.832215\\
178.568569	0.832003\\
178.747497	0.831790\\
178.926426	0.831577\\
179.105355	0.831365\\
179.284284	0.831152\\
179.463213	0.830939\\
179.642142	0.830726\\
179.821071	0.830513\\
180.000000	0.830300\\
};
\addlegendentry{Aligned Communication}








\addplot [line width=0.6mm, color=mycolor3]
  table[row sep=crcr]{%
1.000000	0.778900\\
1.179179	0.777849\\
1.358358	0.777512\\
1.537538	0.777181\\
1.716717	0.777164\\
1.895896	0.776914\\
2.075075	0.776534\\
2.254254	0.776218\\
2.433433	0.776253\\
2.612613	0.776469\\
2.791792	0.776521\\
2.970971	0.776061\\
3.150150	0.774933\\
3.329329	0.775096\\
3.508509	0.775438\\
3.687688	0.773634\\
3.866867	0.773001\\
4.046046	0.773604\\
4.225225	0.773558\\
4.404404	0.772958\\
4.583584	0.772648\\
4.762763	0.772616\\
4.941942	0.772682\\
5.121121	0.772665\\
5.300300	0.772384\\
5.479479	0.771976\\
5.658659	0.771653\\
5.837838	0.770522\\
6.017017	0.769913\\
6.196196	0.769708\\
6.375375	0.769561\\
6.554555	0.769399\\
6.733734	0.769232\\
6.912913	0.769070\\
7.092092	0.768924\\
7.271271	0.768803\\
7.450450	0.768716\\
7.629630	0.768671\\
7.808809	0.768639\\
7.987988	0.768571\\
8.167167	0.768418\\
8.346346	0.768154\\
8.525526	0.767864\\
8.704705	0.767656\\
8.883884	0.767524\\
9.063063	0.767402\\
9.242242	0.767223\\
9.421421	0.766921\\
9.600601	0.766437\\
9.779780	0.765797\\
9.958959	0.765076\\
10.138138	0.764350\\
10.317317	0.763693\\
10.496496	0.763182\\
10.675676	0.762886\\
10.854855	0.762788\\
11.034034	0.762774\\
11.213213	0.762725\\
11.392392	0.762541\\
11.571572	0.762241\\
11.750751	0.761900\\
11.929930	0.761594\\
12.109109	0.761393\\
12.288288	0.761298\\
12.467467	0.761257\\
12.646647	0.761217\\
12.825826	0.761125\\
13.005005	0.760935\\
13.184184	0.760648\\
13.363363	0.760286\\
13.542543	0.759870\\
13.721722	0.759420\\
13.900901	0.758959\\
14.080080	0.758497\\
14.259259	0.758026\\
14.438438	0.757538\\
14.617618	0.757024\\
14.796797	0.756476\\
14.975976	0.755883\\
15.155155	0.755241\\
15.334334	0.754562\\
15.513514	0.753865\\
15.692693	0.753169\\
15.871872	0.752494\\
16.051051	0.751858\\
16.230230	0.751280\\
16.409409	0.750781\\
16.588589	0.750367\\
16.767768	0.750030\\
16.946947	0.749754\\
17.126126	0.749528\\
17.305305	0.749338\\
17.484484	0.749171\\
17.663664	0.749014\\
17.842843	0.748854\\
18.022022	0.748677\\
18.201201	0.748474\\
18.380380	0.748244\\
18.559560	0.747988\\
18.738739	0.747707\\
18.917918	0.747402\\
19.097097	0.747073\\
19.276276	0.746723\\
19.455455	0.746351\\
19.634635	0.745959\\
19.813814	0.745547\\
19.992993	0.745117\\
20.172172	0.744670\\
20.351351	0.744205\\
20.530531	0.743726\\
20.709710	0.743231\\
20.888889	0.742723\\
21.068068	0.742202\\
21.247247	0.741670\\
21.426426	0.741127\\
21.605606	0.740574\\
21.784785	0.740012\\
21.963964	0.739442\\
22.143143	0.738865\\
22.322322	0.738283\\
22.501502	0.737695\\
22.680681	0.737104\\
22.859860	0.736509\\
23.039039	0.735913\\
23.218218	0.735316\\
23.397397	0.734719\\
23.576577	0.734123\\
23.755756	0.733530\\
23.934935	0.732940\\
24.114114	0.732355\\
24.293293	0.731776\\
24.472472	0.731203\\
24.651652	0.730637\\
24.830831	0.730081\\
25.010010	0.729534\\
25.189189	0.728998\\
25.368368	0.728474\\
25.547548	0.727963\\
25.726727	0.727466\\
25.905906	0.726984\\
26.085085	0.726515\\
26.264264	0.726059\\
26.443443	0.725615\\
26.622623	0.725181\\
26.801802	0.724756\\
26.980981	0.724339\\
27.160160	0.723928\\
27.339339	0.723524\\
27.518519	0.723123\\
27.697698	0.722726\\
27.876877	0.722332\\
28.056056	0.721938\\
28.235235	0.721543\\
28.414414	0.721148\\
28.593594	0.720749\\
28.772773	0.720347\\
28.951952	0.719940\\
29.131131	0.719527\\
29.310310	0.719107\\
29.489489	0.718678\\
29.668669	0.718240\\
29.847848	0.717791\\
30.027027	0.717330\\
30.206206	0.716856\\
30.385385	0.716370\\
30.564565	0.715872\\
30.743744	0.715362\\
30.922923	0.714842\\
31.102102	0.714311\\
31.281281	0.713771\\
31.460460	0.713221\\
31.639640	0.712663\\
31.818819	0.712096\\
31.997998	0.711521\\
32.177177	0.710939\\
32.356356	0.710350\\
32.535536	0.709754\\
32.714715	0.709153\\
32.893894	0.708546\\
33.073073	0.707935\\
33.252252	0.707319\\
33.431431	0.706700\\
33.610611	0.706077\\
33.789790	0.705452\\
33.968969	0.704824\\
34.148148	0.704194\\
34.327327	0.703563\\
34.506507	0.702931\\
34.685686	0.702299\\
34.864865	0.701667\\
35.044044	0.701035\\
35.223223	0.700405\\
35.402402	0.699777\\
35.581582	0.699151\\
35.760761	0.698527\\
35.939940	0.697907\\
36.119119	0.697291\\
36.298298	0.696678\\
36.477477	0.696069\\
36.656657	0.695463\\
36.835836	0.694861\\
37.015015	0.694262\\
37.194194	0.693665\\
37.373373	0.693072\\
37.552553	0.692482\\
37.731732	0.691894\\
37.910911	0.691309\\
38.090090	0.690726\\
38.269269	0.690145\\
38.448448	0.689566\\
38.627628	0.688989\\
38.806807	0.688414\\
38.985986	0.687840\\
39.165165	0.687268\\
39.344344	0.686697\\
39.523524	0.686128\\
39.702703	0.685559\\
39.881882	0.684991\\
40.061061	0.684424\\
40.240240	0.683857\\
40.419419	0.683291\\
40.598599	0.682725\\
40.777778	0.682159\\
40.956957	0.681594\\
41.136136	0.681028\\
41.315315	0.680461\\
41.494494	0.679894\\
41.673674	0.679327\\
41.852853	0.678759\\
42.032032	0.678190\\
42.211211	0.677620\\
42.390390	0.677048\\
42.569570	0.676476\\
42.748749	0.675902\\
42.927928	0.675326\\
43.107107	0.674748\\
43.286286	0.674169\\
43.465465	0.673587\\
43.644645	0.673003\\
43.823824	0.672417\\
44.003003	0.671828\\
44.182182	0.671237\\
44.361361	0.670643\\
44.540541	0.670046\\
44.719720	0.669446\\
44.898899	0.668842\\
45.078078	0.668235\\
45.257257	0.667625\\
45.436436	0.667011\\
45.615616	0.666394\\
45.794795	0.665773\\
45.973974	0.665149\\
46.153153	0.664522\\
46.332332	0.663892\\
46.511512	0.663258\\
46.690691	0.662621\\
46.869870	0.661981\\
47.049049	0.661338\\
47.228228	0.660692\\
47.407407	0.660043\\
47.586587	0.659391\\
47.765766	0.658736\\
47.944945	0.658077\\
48.124124	0.657416\\
48.303303	0.656752\\
48.482482	0.656086\\
48.661662	0.655416\\
48.840841	0.654744\\
49.020020	0.654069\\
49.199199	0.653391\\
49.378378	0.652711\\
49.557558	0.652028\\
49.736737	0.651342\\
49.915916	0.650654\\
50.095095	0.649963\\
50.274274	0.649270\\
50.453453	0.648574\\
50.632633	0.647876\\
50.811812	0.647176\\
50.990991	0.646473\\
51.170170	0.645768\\
51.349349	0.645061\\
51.528529	0.644351\\
51.707708	0.643639\\
51.886887	0.642925\\
52.066066	0.642209\\
52.245245	0.641491\\
52.424424	0.640771\\
52.603604	0.640048\\
52.782783	0.639324\\
52.961962	0.638598\\
53.141141	0.637870\\
53.320320	0.637140\\
53.499499	0.636408\\
53.678679	0.635674\\
53.857858	0.634938\\
54.037037	0.634201\\
54.216216	0.633462\\
54.395395	0.632721\\
54.574575	0.631979\\
54.753754	0.631235\\
54.932933	0.630489\\
55.112112	0.629742\\
55.291291	0.628994\\
55.470470	0.628244\\
55.649650	0.627492\\
55.828829	0.626739\\
56.008008	0.625985\\
56.187187	0.625230\\
56.366366	0.624473\\
56.545546	0.623714\\
56.724725	0.622955\\
56.903904	0.622195\\
57.083083	0.621433\\
57.262262	0.620670\\
57.441441	0.619906\\
57.620621	0.619141\\
57.799800	0.618375\\
57.978979	0.617608\\
58.158158	0.616840\\
58.337337	0.616071\\
58.516517	0.615302\\
58.695696	0.614531\\
58.874875	0.613760\\
59.054054	0.612988\\
59.233233	0.612215\\
59.412412	0.611441\\
59.591592	0.610667\\
59.770771	0.609892\\
59.949950	0.609117\\
60.129129	0.608341\\
60.308308	0.607564\\
60.487487	0.606787\\
60.666667	0.606009\\
60.845846	0.605231\\
61.025025	0.604452\\
61.204204	0.603673\\
61.383383	0.602893\\
61.562563	0.602113\\
61.741742	0.601332\\
61.920921	0.600551\\
62.100100	0.599769\\
62.279279	0.598987\\
62.458458	0.598204\\
62.637638	0.597420\\
62.816817	0.596636\\
62.995996	0.595852\\
63.175175	0.595067\\
63.354354	0.594282\\
63.533534	0.593496\\
63.712713	0.592710\\
63.891892	0.591923\\
64.071071	0.591136\\
64.250250	0.590348\\
64.429429	0.589560\\
64.608609	0.588771\\
64.787788	0.587982\\
64.966967	0.587192\\
65.146146	0.586402\\
65.325325	0.585611\\
65.504505	0.584820\\
65.683684	0.584029\\
65.862863	0.583237\\
66.042042	0.582445\\
66.221221	0.581652\\
66.400400	0.580859\\
66.579580	0.580065\\
66.758759	0.579271\\
66.937938	0.578477\\
67.117117	0.577682\\
67.296296	0.576887\\
67.475475	0.576091\\
67.654655	0.575295\\
67.833834	0.574499\\
68.013013	0.573702\\
68.192192	0.572905\\
68.371371	0.572107\\
68.550551	0.571309\\
68.729730	0.570510\\
68.908909	0.569712\\
69.088088	0.568912\\
69.267267	0.568113\\
69.446446	0.567313\\
69.625626	0.566513\\
69.804805	0.565712\\
69.983984	0.564911\\
70.163163	0.564110\\
70.342342	0.563308\\
70.521522	0.562506\\
70.700701	0.561703\\
70.879880	0.560901\\
71.059059	0.560098\\
71.238238	0.559294\\
71.417417	0.558490\\
71.596597	0.557686\\
71.775776	0.556882\\
71.954955	0.556077\\
72.134134	0.555272\\
72.313313	0.554467\\
72.492492	0.553661\\
72.671672	0.552855\\
72.850851	0.552049\\
73.030030	0.551242\\
73.209209	0.550435\\
73.388388	0.549628\\
73.567568	0.548820\\
73.746747	0.548013\\
73.925926	0.547205\\
74.105105	0.546396\\
74.284284	0.545588\\
74.463463	0.544779\\
74.642643	0.543970\\
74.821822	0.543160\\
75.001001	0.542351\\
75.180180	0.541541\\
75.359359	0.540730\\
75.538539	0.539920\\
75.717718	0.539109\\
75.896897	0.538298\\
76.076076	0.537487\\
76.255255	0.536676\\
76.434434	0.535864\\
76.613614	0.535052\\
76.792793	0.534240\\
76.971972	0.533428\\
77.151151	0.532615\\
77.330330	0.531803\\
77.509510	0.530990\\
77.688689	0.530176\\
77.867868	0.529363\\
78.047047	0.528549\\
78.226226	0.527736\\
78.405405	0.526922\\
78.584585	0.526108\\
78.763764	0.525293\\
78.942943	0.524479\\
79.122122	0.523664\\
79.301301	0.522849\\
79.480480	0.522034\\
79.659660	0.521219\\
79.838839	0.520403\\
80.018018	0.519588\\
80.197197	0.518772\\
80.376376	0.517956\\
80.555556	0.517140\\
80.734735	0.516324\\
80.913914	0.515507\\
81.093093	0.514691\\
81.272272	0.513874\\
81.451451	0.513058\\
81.630631	0.512241\\
81.809810	0.511424\\
81.988989	0.510606\\
82.168168	0.509789\\
82.347347	0.508972\\
82.526527	0.508154\\
82.705706	0.507337\\
82.884885	0.506519\\
83.064064	0.505701\\
83.243243	0.504883\\
83.422422	0.504065\\
83.601602	0.503247\\
83.780781	0.502429\\
83.959960	0.501611\\
84.139139	0.500792\\
84.318318	0.499974\\
84.497497	0.499155\\
84.676677	0.498337\\
84.855856	0.497518\\
85.035035	0.496699\\
85.214214	0.495880\\
85.393393	0.495061\\
85.572573	0.494243\\
85.751752	0.493424\\
85.930931	0.492605\\
86.110110	0.491786\\
86.289289	0.490966\\
86.468468	0.490147\\
86.647648	0.489328\\
86.826827	0.488509\\
87.006006	0.487690\\
87.185185	0.486870\\
87.364364	0.486051\\
87.543544	0.485232\\
87.722723	0.484413\\
87.901902	0.483593\\
88.081081	0.482774\\
88.260260	0.481955\\
88.439439	0.481135\\
88.618619	0.480316\\
88.797798	0.479497\\
88.976977	0.478677\\
89.156156	0.477858\\
89.335335	0.477039\\
89.514515	0.476220\\
89.693694	0.475400\\
89.872873	0.474581\\
90.000000	0.474000\\
90.180361	0.473333\\
90.360721	0.472665\\
90.541082	0.471998\\
90.721443	0.471331\\
90.901804	0.470663\\
91.082164	0.469996\\
91.262525	0.469329\\
91.442886	0.468661\\
91.623246	0.467994\\
91.803607	0.467327\\
91.983968	0.466659\\
92.164329	0.465992\\
92.344689	0.465325\\
92.525050	0.464657\\
92.705411	0.463990\\
92.885772	0.463323\\
93.066132	0.462655\\
93.246493	0.461988\\
93.426854	0.461321\\
93.607214	0.460653\\
93.787575	0.459986\\
93.967936	0.459319\\
94.148297	0.458651\\
94.328657	0.457984\\
94.509018	0.457317\\
94.689379	0.456649\\
94.869739	0.455982\\
95.050100	0.455315\\
95.230461	0.454647\\
95.410822	0.453980\\
95.591182	0.453313\\
95.771543	0.452645\\
95.951904	0.451978\\
96.132265	0.451311\\
96.312625	0.450643\\
96.492986	0.449976\\
96.673347	0.449309\\
96.853707	0.448641\\
97.034068	0.447974\\
97.214429	0.447307\\
97.394790	0.446639\\
97.575150	0.445972\\
97.755511	0.445305\\
97.935872	0.444637\\
98.116232	0.443970\\
98.296593	0.443303\\
98.476954	0.442635\\
98.657315	0.441968\\
98.837675	0.441301\\
99.018036	0.440633\\
99.198397	0.439966\\
99.378758	0.439299\\
99.559118	0.438631\\
99.739479	0.437964\\
99.919840	0.437297\\
100.100200	0.436629\\
100.280561	0.435962\\
100.460922	0.435295\\
100.641283	0.434627\\
100.821643	0.433960\\
101.002004	0.433293\\
101.182365	0.432625\\
101.362725	0.431958\\
101.543086	0.431291\\
101.723447	0.430623\\
101.903808	0.429956\\
102.084168	0.429289\\
102.264529	0.428621\\
102.444890	0.427954\\
102.625251	0.427287\\
102.805611	0.426619\\
102.985972	0.425952\\
103.166333	0.425285\\
103.346693	0.424617\\
103.527054	0.423950\\
103.707415	0.423283\\
103.887776	0.422615\\
104.068136	0.421948\\
104.248497	0.421281\\
104.428858	0.420613\\
104.609218	0.419946\\
104.789579	0.419279\\
104.969940	0.418611\\
105.150301	0.417944\\
105.330661	0.417277\\
105.511022	0.416609\\
105.691383	0.415942\\
105.871743	0.415275\\
106.052104	0.414607\\
106.232465	0.413940\\
106.412826	0.413273\\
106.593186	0.412605\\
106.773547	0.411938\\
106.953908	0.411271\\
107.134269	0.410603\\
107.314629	0.409936\\
107.494990	0.409269\\
107.675351	0.408601\\
107.855711	0.407934\\
108.036072	0.407267\\
108.216433	0.406599\\
108.396794	0.405932\\
108.577154	0.405265\\
108.757515	0.404597\\
108.937876	0.403930\\
109.118236	0.403263\\
109.298597	0.402595\\
109.478958	0.401928\\
109.659319	0.401261\\
109.839679	0.400593\\
110.020040	0.399926\\
110.200401	0.399259\\
110.380762	0.398591\\
110.561122	0.397924\\
110.741483	0.397257\\
110.921844	0.396589\\
111.102204	0.395922\\
111.282565	0.395255\\
111.462926	0.394587\\
111.643287	0.393920\\
111.823647	0.393253\\
112.004008	0.392585\\
112.184369	0.391918\\
112.364729	0.391251\\
112.545090	0.390583\\
112.725451	0.389916\\
112.905812	0.389248\\
113.086172	0.388581\\
113.266533	0.387914\\
113.446894	0.387246\\
113.627255	0.386579\\
113.807615	0.385912\\
113.987976	0.385244\\
114.168337	0.384577\\
114.348697	0.383910\\
114.529058	0.383242\\
114.709419	0.382575\\
114.889780	0.381908\\
115.070140	0.381240\\
115.250501	0.380573\\
115.430862	0.379906\\
115.611222	0.379238\\
115.791583	0.378571\\
115.971944	0.377904\\
116.152305	0.377236\\
116.332665	0.376569\\
116.513026	0.375902\\
116.693387	0.375234\\
116.873747	0.374567\\
117.054108	0.373900\\
117.234469	0.373232\\
117.414830	0.372565\\
117.595190	0.371898\\
117.775551	0.371230\\
117.955912	0.370563\\
118.136273	0.369896\\
118.316633	0.369228\\
118.496994	0.368561\\
118.677355	0.367894\\
118.857715	0.367226\\
119.038076	0.366559\\
119.218437	0.365892\\
119.398798	0.365224\\
119.579158	0.364557\\
119.759519	0.363890\\
119.939880	0.363222\\
120.120240	0.362555\\
120.300601	0.361888\\
120.480962	0.361220\\
120.661323	0.360553\\
120.841683	0.359886\\
121.022044	0.359218\\
121.202405	0.358551\\
121.382766	0.357884\\
121.563126	0.357216\\
121.743487	0.356549\\
121.923848	0.355882\\
122.104208	0.355214\\
122.284569	0.354547\\
122.464930	0.353880\\
122.645291	0.353212\\
122.825651	0.352545\\
123.006012	0.351878\\
123.186373	0.351210\\
123.366733	0.350543\\
123.547094	0.349876\\
123.727455	0.349208\\
123.907816	0.348541\\
124.088176	0.347874\\
124.268537	0.347206\\
124.448898	0.346539\\
124.629259	0.345872\\
124.809619	0.345204\\
124.989980	0.344537\\
125.170341	0.343870\\
125.350701	0.343202\\
125.531062	0.342535\\
125.711423	0.341868\\
125.891784	0.341200\\
126.072144	0.340533\\
126.252505	0.339866\\
126.432866	0.339198\\
126.613226	0.338531\\
126.793587	0.337864\\
126.973948	0.337196\\
127.154309	0.336529\\
127.334669	0.335862\\
127.515030	0.335194\\
127.695391	0.334527\\
127.875752	0.333860\\
128.056112	0.333192\\
128.236473	0.332525\\
128.416834	0.331858\\
128.597194	0.331190\\
128.777555	0.330523\\
128.957916	0.329856\\
129.138277	0.329188\\
129.318637	0.328521\\
129.498998	0.327854\\
129.679359	0.327186\\
129.859719	0.326519\\
130.040080	0.325852\\
130.220441	0.325184\\
130.400802	0.324517\\
130.581162	0.323850\\
130.761523	0.323182\\
130.941884	0.322515\\
131.122244	0.321848\\
131.302605	0.321180\\
131.482966	0.320513\\
131.663327	0.319846\\
131.843687	0.319178\\
132.024048	0.318511\\
132.204409	0.317844\\
132.384770	0.317176\\
132.565130	0.316509\\
132.745491	0.315842\\
132.925852	0.315174\\
133.106212	0.314507\\
133.286573	0.313840\\
133.466934	0.313172\\
133.647295	0.312505\\
133.827655	0.311838\\
134.008016	0.311170\\
134.188377	0.310503\\
134.368737	0.309836\\
134.549098	0.309168\\
134.729459	0.308501\\
134.909820	0.307834\\
135.090180	0.307166\\
135.270541	0.306499\\
135.450902	0.305832\\
135.631263	0.305164\\
135.811623	0.304497\\
135.991984	0.303830\\
136.172345	0.303162\\
136.352705	0.302495\\
136.533066	0.301828\\
136.713427	0.301160\\
136.893788	0.300493\\
137.074148	0.299826\\
137.254509	0.299158\\
137.434870	0.298491\\
137.615230	0.297824\\
137.795591	0.297156\\
137.975952	0.296489\\
138.156313	0.295822\\
138.336673	0.295154\\
138.517034	0.294487\\
138.697395	0.293820\\
138.877756	0.293152\\
139.058116	0.292485\\
139.238477	0.291818\\
139.418838	0.291150\\
139.599198	0.290483\\
139.779559	0.289816\\
139.959920	0.289148\\
140.140281	0.288481\\
140.320641	0.287814\\
140.501002	0.287146\\
140.681363	0.286479\\
140.861723	0.285812\\
141.042084	0.285144\\
141.222445	0.284477\\
141.402806	0.283810\\
141.583166	0.283142\\
141.763527	0.282475\\
141.943888	0.281808\\
142.124248	0.281140\\
142.304609	0.280473\\
142.484970	0.279806\\
142.665331	0.279138\\
142.845691	0.278471\\
143.026052	0.277804\\
143.206413	0.277136\\
143.386774	0.276469\\
143.567134	0.275802\\
143.747495	0.275134\\
143.927856	0.274467\\
144.108216	0.273800\\
144.288577	0.273132\\
144.468938	0.272465\\
144.649299	0.271798\\
144.829659	0.271130\\
145.010020	0.270463\\
145.190381	0.269796\\
145.370741	0.269128\\
145.551102	0.268461\\
145.731463	0.267794\\
145.911824	0.267126\\
146.092184	0.266459\\
146.272545	0.265792\\
146.452906	0.265124\\
146.633267	0.264457\\
146.813627	0.263790\\
146.993988	0.263122\\
147.174349	0.262455\\
147.354709	0.261788\\
147.535070	0.261120\\
147.715431	0.260453\\
147.895792	0.259786\\
148.076152	0.259118\\
148.256513	0.258451\\
148.436874	0.257784\\
148.617234	0.257116\\
148.797595	0.256449\\
148.977956	0.255782\\
149.158317	0.255114\\
149.338677	0.254447\\
149.519038	0.253780\\
149.699399	0.253112\\
149.879760	0.252445\\
150.060120	0.251778\\
150.240481	0.251110\\
150.420842	0.250443\\
150.601202	0.249776\\
150.781563	0.249108\\
150.961924	0.248441\\
151.142285	0.247774\\
151.322645	0.247106\\
151.503006	0.246439\\
151.683367	0.245772\\
151.863727	0.245104\\
152.044088	0.244437\\
152.224449	0.243770\\
152.404810	0.243102\\
152.585170	0.242435\\
152.765531	0.241768\\
152.945892	0.241100\\
153.126253	0.240433\\
153.306613	0.239766\\
153.486974	0.239098\\
153.667335	0.238431\\
153.847695	0.237764\\
154.028056	0.237096\\
154.208417	0.236429\\
154.388778	0.235762\\
154.569138	0.235094\\
154.749499	0.234427\\
154.929860	0.233760\\
155.110220	0.233092\\
155.290581	0.232425\\
155.470942	0.231758\\
155.651303	0.231090\\
155.831663	0.230423\\
156.012024	0.229756\\
156.192385	0.229088\\
156.372745	0.228421\\
156.553106	0.227754\\
156.733467	0.227086\\
156.913828	0.226419\\
157.094188	0.225752\\
157.274549	0.225084\\
157.454910	0.224417\\
157.635271	0.223749\\
157.815631	0.223082\\
157.995992	0.222415\\
158.176353	0.221747\\
158.356713	0.221080\\
158.537074	0.220413\\
158.717435	0.219745\\
158.897796	0.219078\\
159.078156	0.218411\\
159.258517	0.217743\\
159.438878	0.217076\\
159.619238	0.216409\\
159.799599	0.215741\\
159.979960	0.215074\\
160.160321	0.214407\\
160.340681	0.213739\\
160.521042	0.213072\\
160.701403	0.212405\\
160.881764	0.211737\\
161.062124	0.211070\\
161.242485	0.210403\\
161.422846	0.209735\\
161.603206	0.209068\\
161.783567	0.208401\\
161.963928	0.207733\\
162.144289	0.207066\\
162.324649	0.206399\\
162.505010	0.205731\\
162.685371	0.205064\\
162.865731	0.204397\\
163.046092	0.203729\\
163.226453	0.203062\\
163.406814	0.202395\\
163.587174	0.201727\\
163.767535	0.201060\\
163.947896	0.200393\\
164.128257	0.199725\\
164.308617	0.199058\\
164.488978	0.198391\\
164.669339	0.197723\\
164.849699	0.197056\\
165.030060	0.196389\\
165.210421	0.195721\\
165.390782	0.195054\\
165.571142	0.194387\\
165.751503	0.193719\\
165.931864	0.193052\\
166.112224	0.192385\\
166.292585	0.191717\\
166.472946	0.191050\\
166.653307	0.190383\\
166.833667	0.189715\\
167.014028	0.189048\\
167.194389	0.188381\\
167.374749	0.187713\\
167.555110	0.187046\\
167.735471	0.186379\\
167.915832	0.185711\\
168.096192	0.185044\\
168.276553	0.184377\\
168.456914	0.183709\\
168.637275	0.183042\\
168.817635	0.182375\\
168.997996	0.181707\\
169.178357	0.181040\\
169.358717	0.180373\\
169.539078	0.179705\\
169.719439	0.179038\\
169.899800	0.178371\\
170.080160	0.177703\\
170.260521	0.177036\\
170.440882	0.176369\\
170.621242	0.175701\\
170.801603	0.175034\\
170.981964	0.174367\\
171.162325	0.173699\\
171.342685	0.173032\\
171.523046	0.172365\\
171.703407	0.171697\\
171.883768	0.171030\\
172.064128	0.170363\\
172.244489	0.169695\\
172.424850	0.169028\\
172.605210	0.168361\\
172.785571	0.167693\\
172.965932	0.167026\\
173.146293	0.166359\\
173.326653	0.165691\\
173.507014	0.165024\\
173.687375	0.164357\\
173.867735	0.163689\\
174.048096	0.163022\\
174.228457	0.162355\\
174.408818	0.161687\\
174.589178	0.161020\\
174.769539	0.160353\\
174.949900	0.159685\\
175.130261	0.159018\\
175.310621	0.158351\\
175.490982	0.157683\\
175.671343	0.157016\\
175.851703	0.156349\\
176.032064	0.155681\\
176.212425	0.155014\\
176.392786	0.154347\\
176.573146	0.153679\\
176.753507	0.153012\\
176.933868	0.152345\\
177.114228	0.151677\\
177.294589	0.151010\\
177.474950	0.150343\\
177.655311	0.149675\\
177.835671	0.149008\\
178.016032	0.148341\\
178.196393	0.147673\\
178.376754	0.147006\\
178.557114	0.146339\\
178.737475	0.145671\\
178.917836	0.145004\\
179.098196	0.144337\\
179.278557	0.143669\\
179.458918	0.143002\\
179.639279	0.142335\\
179.819639	0.141667\\
180.000000	0.141000\\
  };
\addlegendentry{Dual-mode Networked Sensing}



















\addplot [
    color=white, 
    draw=none, 
    mark=x, 
    thick, 
    mark size=4pt, 
    mark options={black},
]
  table[row sep=crcr]{%
1.0000	0.02871\\
4.5807	0.12924\\
8.1614	0.22648\\
11.7421	0.32012\\
15.3229	0.41015\\
18.9036	0.49650\\
22.4843	0.57890\\
26.0650	0.65749\\
33.2264	0.80691\\
40.3879	0.89877\\
47.5493	0.91933\\
54.7107	0.92576\\
65.4529	0.92616\\
72.6143	0.92372\\
79.7758	0.91956\\
92.5251	0.90863\\
110.5611	0.89057\\
128.5972	0.87250\\
146.6333	0.85445\\
164.6693	0.83638\\
};
\addlegendentry{Simulations}





\addplot [
    color=white, 
    draw=none, 
    mark=x, 
    thick, 
    mark size=4pt, 
    mark options={black},
]
  table[row sep=crcr]{%
1.250000	0.995903\\
13.775025	0.988668\\
26.300050	0.980447\\
38.825075	0.971234\\
51.350100	0.961341\\
63.875125	0.950507\\
76.400150	0.939092\\
88.925175	0.927160\\
101.450200	0.914745\\
113.975225	0.901875\\
126.500250	0.888584\\
139.025275	0.874903\\
151.550300	0.860864\\
164.075325	0.846498\\
176.600350	0.831836\\
};







\addplot [
    color=white, 
    draw=none, 
    mark=x, 
    thick, 
    mark size=4pt, 
    mark options={black},
]
  table[row sep=crcr]{%
1.000000	0.773136\\
13.542543	0.754247\\
26.085085	0.721139\\
38.627628	0.683890\\
51.170170	0.640989\\
63.712713	0.588324\\
76.255255	0.532705\\
88.797798	0.475949\\
101.362725	0.428762\\
113.987976	0.382393\\
126.613226	0.336026\\
139.238477	0.289659\\
151.863727	0.243290\\
164.488978	0.196923\\
177.114228	0.150555\\
  };










\end{axis}

\begin{axis}[%
width=5.833in,
height=4.375in,
at={(0in,0in)},
scale only axis,
xmin=0,
xmax=1,
ymin=0,
ymax=1,
axis line style={draw=none},
ticks=none,
axis x line*=bottom,
axis y line*=left
]
\end{axis}
\end{tikzpicture}%
   }
   \hspace{-0.66cm}
     \subfloat[  \label{bs_avg_tot}]{%
   % This file was created by matlab2tikz.
%
%The latest updates can be retrieved from
%  http://www.mathworks.com/matlabcentral/fileexchange/22022-matlab2tikz-matlab2tikz
%where you can also make suggestions and rate matlab2tikz.
%
\definecolor{mycolor1}{rgb}{0.00000,0.44700,0.74100}%
\definecolor{mycolor2}{rgb}{0.85000,0.32500,0.09800}%
%
\begin{tikzpicture}[scale=0.33, transform shape,font=\Large]

\begin{axis}[%
width=4.521in,
height=3.477in,
at={(0.758in,0.57in)},
scale only axis,
xmin=1,
xmax=500,
xlabel style={font=\large\color{white!15!black}},
xlabel={$\text{BSs Density/ km}^\text{2}$},
ymin=2,
ymax=41,
ylabel style={font=\large\color{white!15!black}},
ylabel={Average Rate  (nats/s/Hz)},
axis background/.style={fill=white},
xmajorgrids,
ymajorgrids,
legend style={at={(0.316,0.057)}, anchor=south west, legend cell align=left, align=left, draw=white!15!black}
]


\addplot [line width=0.6mm, color=mycolor1]
  table[row sep=crcr]{%
10	2.37932777777778\\
30.2	5.57799587628866\\
50.4	8.6849\\
70.6	11.5778113989637\\
90.8	13.8997435344828\\
111	15.8910490636704\\
131.2	17.7826355704698\\
151.4	19.4976349693252\\
171.6	20.9657779036827\\
191.8	22.3941276595745\\
212	23.7028492462312\\
232.2	24.8676038277512\\
252.4	26.0530315668203\\
272.7	27.0710825221239\\
292.9	27.9856311965812\\
313.1	28.7089896907216\\
333.3	29.6136453629032\\
353.5	30.4211102362205\\
373.7	30.9881888246628\\
393.9	31.7447516981132\\
414.1	32.3633411111111\\
434.3	32.8437114337568\\
454.5	33.5133944543828\\
474.7	34.1048585840708\\
494.9	34.3071941074523\\
515.102040816327	34.5158857142857\\
535.30612244898	34.9026285714286\\
555.510204081633	35.2832571428571\\
575.714285714286	35.6657714285714\\
595.918367346939	35.9598857142857\\
616.122448979592	36.2878857142857\\
636.326530612245	36.6254857142857\\
656.530612244898	36.8828571428571\\
676.734693877551	37.1316\\
696.938775510204	37.3717142857143\\
717.142857142857	37.5252\\
737.34693877551	37.7229142857143\\
757.551020408163	37.9309714285714\\
777.755102040816	38.1334857142857\\
797.959183673469	38.3063428571429\\
818.163265306122	38.4848\\
838.367346938776	38.6385142857143\\
858.571428571429	38.8000571428571\\
878.775510204082	38.9079428571429\\
898.979591836735	39.0440571428571\\
919.183673469388	39.0713714285714\\
939.387755102041	39.1737714285714\\
959.591836734694	39.2801142857143\\
979.795918367347	39.4028571428572\\
1000	39.542\\
};
\addlegendentry{Dual-mode Networked Sensing}

\addplot [line width=0.6mm, color=mycolor2]
  table[row sep=crcr]{%
10.000000	27.437364\\
30.204082	35.755709\\
50.408163	38.520553\\
70.612245	39.761418\\
90.816327	40.453270\\
111.020408	40.511550\\
131.224490	40.399799\\
151.428571	39.920911\\
171.632653	39.405524\\
191.836735	38.864594\\
212.040816	38.309726\\
232.244898	37.658984\\
252.448980	37.029868\\
272.653061	36.347645\\
292.857143	35.751329\\
313.061224	35.172898\\
333.265306	34.570598\\
353.469388	33.955722\\
373.673469	33.428646\\
393.877551	32.843270\\
414.081633	32.316868\\
434.285714	31.789932\\
454.489796	31.269669\\
474.693878	30.784219\\
494.897959	30.327161\\
515.102041	29.865955\\
535.306122	29.465751\\
555.510204	29.089850\\
575.714286	28.724412\\
595.918367	28.333438\\
616.122449	27.949873\\
636.326531	27.597796\\
656.530612	27.309886\\
676.734694	27.047476\\
696.938776	26.705471\\
717.142857	26.371944\\
737.346939	26.079903\\
757.551020	25.792530\\
777.755102	25.519985\\
797.959184	25.257380\\
818.163265	24.981133\\
838.367347	24.730662\\
858.571429	24.535717\\
878.775510	24.378532\\
898.979592	24.156601\\
919.183673	23.930964\\
939.387755	23.772724\\
959.591837	23.598513\\
979.795918	23.428949\\
1000.000000	23.263942\\
};
\addlegendentry{Communication}























\addplot [
    color=white, 
    draw=none, 
    mark=x, 
    thick, 
    mark size=4pt, 
    mark options={black},
]
  table[row sep=crcr]{%
10	2.25405714285715\\
30.2040816326531	5.63605714285714\\
50.4081632653061	8.68491428571429\\
70.6122448979592	11.4006285714286\\
90.8163265306122	13.7223428571429\\
111.020408163265	15.7728571428571\\
131.224489795918	17.6053714285714\\
151.428571428571	19.2613142857143\\
171.632653061224	20.7890857142857\\
191.836734693878	22.1584\\
212.040816326531	23.4669714285714\\
232.244897959184	24.6904\\
252.448979591837	25.7563428571429\\
272.65306122449	26.7749142857143\\
292.857142857143	27.6898285714286\\
313.061224489796	28.5325142857143\\
333.265306122449	29.3181142857143\\
353.469387755102	30.066\\
373.673469387755	30.6924571428571\\
393.877551020408	31.3894285714286\\
414.081632653061	32.0077142857143\\
434.285714285714	32.6070285714286\\
454.489795918367	33.1574857142857\\
474.69387755102	33.6286857142857\\
494.897959183673	34.0709714285714\\
515.102040816327	34.5158857142857\\
535.30612244898	34.9026285714286\\
555.510204081633	35.2832571428571\\
575.714285714286	35.6657714285714\\
595.918367346939	35.9598857142857\\
616.122448979592	36.2878857142857\\
636.326530612245	36.6254857142857\\
656.530612244898	36.8828571428571\\
676.734693877551	37.1316\\
696.938775510204	37.3717142857143\\
717.142857142857	37.5252\\
737.34693877551	37.7229142857143\\
757.551020408163	37.9309714285714\\
777.755102040816	38.1334857142857\\
797.959183673469	38.3063428571429\\
818.163265306122	38.4848\\
838.367346938776	38.6385142857143\\
858.571428571429	38.8000571428571\\
878.775510204082	38.9079428571429\\
898.979591836735	39.0440571428571\\
919.183673469388	39.0713714285714\\
939.387755102041	39.1737714285714\\
959.591836734694	39.2801142857143\\
979.795918367347	39.4028571428572\\
1000	39.542\\
};
\addlegendentry{Simulations}

\addplot [
    color=white, 
    draw=none, 
    mark=x, 
    thick, 
    mark size=4pt, 
    mark options={black},
]
  table[row sep=crcr]{%
10 27.3098\\
30.4166666666667 35.4485\\
50.8333333333333 38.4284\\
71.25 39.6332\\
91.6666666666667 40.2268\\
112.083333333333 40.2893\\
132.5 40.0650\\
152.916666666667 39.8224\\
173.333333333333 39.1890\\
193.75 38.6642\\
214.166666666667 37.8805\\
234.583333333333 37.2321\\
255 36.7310\\
275.416666666667 36.0245\\
295.833333333333 35.4335\\
316.25 34.8197\\
336.666666666667 34.3251\\
357.083333333333 33.5933\\
377.5 32.9870\\
397.916666666667 32.4234\\
418.333333333333 31.8776\\
438.75 31.4198\\
459.166666666667 30.9335\\
479.583333333333 30.3918\\
500 30.0211\\
};


\end{axis}

\begin{axis}[%
width=5.833in,
height=4.375in,
at={(0in,0in)},
scale only axis,
xmin=0,
xmax=1,
ymin=0,
ymax=1,
axis line style={draw=none},
ticks=none,
axis x line*=bottom,
axis y line*=left
]
\end{axis}
\end{tikzpicture}%
}%
\caption{ (a) The effect of different interference types on the monostatic and the bistatic ($n=2$) operations. (b) The sensing coverage probability for different target locations. (c) The average coverage probability versus beamwidth spread. (d) The average sensing and communication rates versus BS density. }
\end{figure*}





Fig.~\ref{eff_sources_mono} depicts the effect of different interference types on monostatic sensing (assuming perfect SIC) and bistatic sensing ($n=2$) for a target positioned 20~m from the serving BS. The close alignment between analytical results and simulations confirms the validity of Theorem 1 and Theorem 2.
 The figure demonstrates that focusing on direct (co-channel) interference, as in communication systems, while ignoring clutter-related interference results in overly optimistic sensing coverage estimates. While all interference types contribute comparably in monostatic sensing, their impact on bistatic sensing varies.
The larger bistatic resolution cell introduces more scatterers, making intra-clutter interference dominant. Meanwhile, inter-clutter interference is lower than in the monostatic case due to independent LoS/NLoS conditions for both inter-clutter links (i.e., $R_n$ and $R_v$) in bistatic scenarios, unlike monostatic sensing, where one inter-clutter link (i.e., $R_1$) is always LoS.
 Overall, monostatic sensing achieves higher coverage probability under effective SIC due to proximity to the target, better LoS conditions, higher RCS values with concentrated reflections, and smaller resolution cells that reduce intra-clutter interference.




Fig.~\ref{dist_effect} illustrates the sensing coverage probability for two scenarios: one with a target near the cell center and the other at a farther location, comparing monostatic sensing with networked sensing involving 2, 3, and 4 BSs. The close match between analytical results and simulations validates (\ref{joint_fus_form}) for networked sensing fusion, confirming the independence assumption's accuracy. 
The figure shows that dual-mode networked sensing fusion significantly enhances performance by leveraging spatial diversity. This improvement stems from the fact that, while the average coverage probability of individual bistatic links is lower than that of monostatic sensing, some instantaneous bistatic SINRs can exceed monostatic ones due to inherent randomness in the network.
The figure shows that monostatic sensing is sufficient for targets near the cell center. However, as the target becomes closer to the cell edge, the benefits of networked sensing become more evident, providing significant performance enhancement. While involving additional BSs consistently improves coverage probability, the marginal gains diminish as $n$ increases, making further complexity less practical. Hence, cooperation with a few nearest BSs thus strikes an efficient balance between performance and system complexity.







Fig. \ref{beam_tradeoff} shows the average coverage probability for sensing and communication versus 3-dB beamwidth, validating Theorem \ref{com_cov_prob}. For sensing, narrower beams improve coverage probability by increasing antenna gain, reducing direct interference, minimizing clutter, and enhancing angular resolution. In communication, narrower beams theoretically improve coverage due to higher gain and reduced interference. However, in practice, coverage peaks at an optimal beamwidth and declines sharply beyond it, as narrower beams provide higher gain but are more sensitive to misalignment.  Hence, careful beamwidth selection is essential to balance sensing and communication performance in unified ISAC systems.



Fig.~\ref{bs_avg_tot} illustrates the relationship between average sensing and communication rates and BS density, validating Theorems 3 and 5. The plot reveals an optimal BS density for maximizing communication rates. At low densities, the system is noise-limited, where increased density shortens link distances, improving coverage and rates. Beyond this point, the system becomes interference-limited, as higher density transitions more interference links from NLoS to LoS, reducing coverage and rates.
In contrast, sensing rates consistently increase with BS density, growing rapidly at first and slowing at higher densities. This behavior is driven by: 1) increased probability of LoS links in multistatic sensing due to shorter distances;
2) the intended sensing signal experiences path loss with an exponent twice that of the direct interference signal, resulting in a greater marginal improvement in intended signal strength as BS density increases; and 3) the reduction in the resolution cell area, which minimizes the impact of clutter. The slower growth at higher densities is due to more interference signals transitioning to LoS, amplifying their impact, but this effect remains outweighed by the three  aforementioned factors.






\begin{figure*}
\centering
\subfloat[ \label{imper_sic}]{%
  % This file was created by matlab2tikz.
%
%The latest updates can be retrieved from
%  http://www.mathworks.com/matlabcentral/fileexchange/22022-matlab2tikz-matlab2tikz
%where you can also make suggestions and rate matlab2tikz.
%
\definecolor{mycolor1}{rgb}{0.00000,0.44700,0.74100}%
\definecolor{mycolor2}{rgb}{0.85000,0.32500,0.09800}%
\definecolor{mycolor3}{rgb}{0.92900,0.69400,0.12500}%
\definecolor{mycolor4}{rgb}{0.49400,0.18400,0.55600}%
%
\begin{tikzpicture}[scale=0.42, transform shape,font=\Large]

\begin{axis}[%
width=4.521in,
height=3.477in,
at={(0.758in,0.57in)},
scale only axis,
xmode=log,
xmin=1e-15,
xmax=1e-06,
xminorticks=true,
xlabel style={font=\large\color{white!15!black}},
xlabel={Fraction of Power Remaining After Imperfect SIC $(\zeta)$},
ymin=0,
ymax=1,
ylabel style={font=\large\color{white!15!black}},
ylabel={Sensing Coverage Probability},
axis background/.style={fill=white},
xmajorgrids,
xminorgrids,
ymajorgrids,
legend style={at={(0,0.075)}, anchor=south west, legend cell align=left, align=left, draw=white!15!black, font=\large}
]
\addplot [color=mycolor1, line width=1.5pt, mark=o, mark options={solid, mycolor1}]
  table[row sep=crcr]{%
1e-06	0\\
6.55128556859551e-07	0\\
4.29193426012878e-07	0\\
2.81176869797423e-07	0\\
1.84206996932672e-07	0\\
1.20679264063933e-07	0\\
7.9060432109077e-08	0\\
5.17947467923121e-08	0\\
3.39322177189533e-08	0\\
2.2229964825262e-08	0\\
1.45634847750124e-08	0\\
9.54095476349996e-09	2.85714285714286e-05\\
6.25055192527398e-09	0.0005\\
4.09491506238042e-09	0.0041\\
2.68269579527973e-09	0.0165285714285714\\
1.75751062485479e-09	0.0441142857142857\\
1.15139539932645e-09	0.0906428571428571\\
7.54312006335461e-10	0.156357142857143\\
4.94171336132384e-10	0.238085714285714\\
3.23745754281764e-10	0.331842857142857\\
2.12095088792019e-10	0.432914285714286\\
1.38949549437314e-10	0.531185714285714\\
9.10298177991523e-11	0.619714285714286\\
5.96362331659466e-11	0.692214285714286\\
3.90693993705462e-11	0.747785714285714\\
2.55954792269953e-11	0.788857142857143\\
1.676832936811e-11	0.817971428571428\\
1.09854114198756e-11	0.8378\\
7.19685673001153e-12	0.852028571428571\\
4.71486636345739e-12	0.860771428571429\\
3.08884359647747e-12	0.866871428571429\\
2.02358964772516e-12	0.870985714285714\\
1.32571136559011e-12	0.873642857142857\\
8.68511373751352e-13	0.875957142857143\\
5.68986602901828e-13	0.876785714285714\\
3.72759372031494e-13	0.877485714285714\\
2.44205309454866e-13	0.878028571428571\\
1.59985871960606e-13	0.878828571428571\\
1.04811313415468e-13	0.8789\\
6.866488450043e-14	0.878928571428571\\
4.49843266896945e-14	0.878614285714286\\
2.94705170255181e-14	0.878928571428571\\
1.93069772888325e-14	0.8789\\
1.2648552168553e-14	0.879328571428571\\
8.28642772854686e-15	0.879\\
5.42867543932386e-15	0.879128571428571\\
3.55648030622314e-15	0.879228571428571\\
2.32995181051537e-15	0.879533333333333\\
1.52641796717524e-15	0.87954\\
1e-15	0.879625\\
};
\addlegendentry{Monostatic}

\addplot [color=mycolor2, dashed, line width=1.5pt, mark=square, mark options={solid, mycolor2}]
  table[row sep=crcr]{%
1e-06	0.303625\\
6.55128556859551e-07	0.30446\\
4.29193426012878e-07	0.3043\\
2.81176869797423e-07	0.304071428571429\\
1.84206996932672e-07	0.304571428571429\\
1.20679264063933e-07	0.304671428571429\\
7.9060432109077e-08	0.304614285714286\\
5.17947467923121e-08	0.304971428571429\\
3.39322177189533e-08	0.304814285714286\\
2.2229964825262e-08	0.305171428571429\\
1.45634847750124e-08	0.305471428571429\\
9.54095476349996e-09	0.305057142857143\\
6.25055192527398e-09	0.305314285714286\\
4.09491506238042e-09	0.308042857142857\\
2.68269579527973e-09	0.316385714285714\\
1.75751062485479e-09	0.335728571428571\\
1.15139539932645e-09	0.367714285714286\\
7.54312006335461e-10	0.4132\\
4.94171336132384e-10	0.469857142857143\\
3.23745754281764e-10	0.535471428571429\\
2.12095088792019e-10	0.605971428571428\\
1.38949549437314e-10	0.674714285714286\\
9.10298177991523e-11	0.736085714285714\\
5.96362331659466e-11	0.787028571428571\\
3.90693993705462e-11	0.826185714285714\\
2.55954792269953e-11	0.855042857142857\\
1.676832936811e-11	0.875042857142857\\
1.09854114198756e-11	0.888842857142857\\
7.19685673001153e-12	0.898814285714286\\
4.71486636345739e-12	0.904885714285714\\
3.08884359647747e-12	0.909028571428571\\
2.02358964772516e-12	0.9119\\
1.32571136559011e-12	0.913757142857143\\
8.68511373751352e-13	0.9156\\
5.68986602901828e-13	0.915914285714286\\
3.72759372031494e-13	0.9165\\
2.44205309454866e-13	0.917\\
1.59985871960606e-13	0.917585714285714\\
1.04811313415468e-13	0.917671428571429\\
6.866488450043e-14	0.917614285714286\\
4.49843266896945e-14	0.9175\\
2.94705170255181e-14	0.917757142857143\\
1.93069772888325e-14	0.917857142857143\\
1.2648552168553e-14	0.917985714285714\\
8.28642772854686e-15	0.917828571428571\\
5.42867543932386e-15	0.917842857142857\\
3.55648030622314e-15	0.918042857142857\\
2.32995181051537e-15	0.9182\\
1.52641796717524e-15	0.91842\\
1e-15	0.91825\\
};
\addlegendentry{Dual-mode Networked Sensing 2}

\addplot [color=mycolor3, dashdotted, line width=1.5pt, mark=asterisk, mark options={solid, mycolor3}]
  table[row sep=crcr]{%
1e-06	0.43665\\
6.55128556859551e-07	0.4375\\
4.29193426012878e-07	0.43735\\
2.81176869797423e-07	0.437285714285714\\
1.84206996932672e-07	0.4383\\
1.20679264063933e-07	0.438942857142857\\
7.9060432109077e-08	0.438857142857143\\
5.17947467923121e-08	0.439642857142857\\
3.39322177189533e-08	0.439614285714286\\
2.2229964825262e-08	0.440071428571429\\
1.45634847750124e-08	0.439971428571429\\
9.54095476349996e-09	0.439128571428571\\
6.25055192527398e-09	0.438585714285714\\
4.09491506238042e-09	0.440814285714286\\
2.68269579527973e-09	0.4478\\
1.75751062485479e-09	0.463285714285714\\
1.15139539932645e-09	0.4895\\
7.54312006335461e-10	0.526814285714286\\
4.94171336132384e-10	0.572871428571428\\
3.23745754281764e-10	0.626014285714286\\
2.12095088792019e-10	0.683028571428571\\
1.38949549437314e-10	0.7383\\
9.10298177991523e-11	0.788\\
5.96362331659466e-11	0.828957142857143\\
3.90693993705462e-11	0.860528571428572\\
2.55954792269953e-11	0.883942857142857\\
1.676832936811e-11	0.900342857142857\\
1.09854114198756e-11	0.911471428571429\\
7.19685673001153e-12	0.9193\\
4.71486636345739e-12	0.924257142857143\\
3.08884359647747e-12	0.927371428571429\\
2.02358964772516e-12	0.929585714285714\\
1.32571136559011e-12	0.931014285714286\\
8.68511373751352e-13	0.932485714285714\\
5.68986602901828e-13	0.932685714285714\\
3.72759372031494e-13	0.9333\\
2.44205309454866e-13	0.933542857142857\\
1.59985871960606e-13	0.934085714285714\\
1.04811313415468e-13	0.934057142857143\\
6.866488450043e-14	0.934128571428571\\
4.49843266896945e-14	0.934057142857143\\
2.94705170255181e-14	0.934471428571429\\
1.93069772888325e-14	0.934628571428571\\
1.2648552168553e-14	0.934942857142857\\
8.28642772854686e-15	0.934957142857143\\
5.42867543932386e-15	0.935114285714286\\
3.55648030622314e-15	0.935014285714286\\
2.32995181051537e-15	0.935183333333333\\
1.52641796717524e-15	0.93518\\
1e-15	0.934975\\
};
\addlegendentry{Dual-mode Networked Sensing 3}

\addplot [color=mycolor4, dotted, line width=1.5pt, mark=diamond, mark options={solid, mycolor4}]
  table[row sep=crcr]{%
1e-06	0.509675\\
6.55128556859551e-07	0.51072\\
4.29193426012878e-07	0.5108\\
2.81176869797423e-07	0.510871428571429\\
1.84206996932672e-07	0.511942857142857\\
1.20679264063933e-07	0.512214285714286\\
7.9060432109077e-08	0.512728571428571\\
5.17947467923121e-08	0.513642857142857\\
3.39322177189533e-08	0.513414285714286\\
2.2229964825262e-08	0.514028571428571\\
1.45634847750124e-08	0.513885714285714\\
9.54095476349996e-09	0.512757142857143\\
6.25055192527398e-09	0.512557142857143\\
4.09491506238042e-09	0.514014285714286\\
2.68269579527973e-09	0.520314285714286\\
1.75751062485479e-09	0.534142857142857\\
1.15139539932645e-09	0.5566\\
7.54312006335461e-10	0.5891\\
4.94171336132384e-10	0.629371428571429\\
3.23745754281764e-10	0.6756\\
2.12095088792019e-10	0.725542857142857\\
1.38949549437314e-10	0.773528571428571\\
9.10298177991523e-11	0.816471428571429\\
5.96362331659466e-11	0.852242857142857\\
3.90693993705462e-11	0.879657142857143\\
2.55954792269953e-11	0.900014285714286\\
1.676832936811e-11	0.914171428571429\\
1.09854114198756e-11	0.923757142857143\\
7.19685673001153e-12	0.930457142857143\\
4.71486636345739e-12	0.934785714285714\\
3.08884359647747e-12	0.937271428571429\\
2.02358964772516e-12	0.939185714285714\\
1.32571136559011e-12	0.940371428571429\\
8.68511373751352e-13	0.941642857142857\\
5.68986602901828e-13	0.941814285714286\\
3.72759372031494e-13	0.942442857142857\\
2.44205309454866e-13	0.942628571428571\\
1.59985871960606e-13	0.943185714285714\\
1.04811313415468e-13	0.943085714285714\\
6.866488450043e-14	0.943242857142857\\
4.49843266896945e-14	0.943314285714286\\
2.94705170255181e-14	0.943642857142857\\
1.93069772888325e-14	0.943714285714286\\
1.2648552168553e-14	0.944028571428571\\
8.28642772854686e-15	0.944071428571429\\
5.42867543932386e-15	0.944271428571429\\
3.55648030622314e-15	0.944128571428571\\
2.32995181051537e-15	0.9442\\
1.52641796717524e-15	0.94418\\
1e-15	0.944\\
};
\addlegendentry{Dual-mode Networked Sensing 4}

  % Region A (light pink, left of dashed line)
    \addplot[fill=red!10, draw=none, fill opacity=0.4] 
        coordinates {(1e-15, 0) (2.12095088792019e-10, 0) (2.12095088792019e-10, 1) (1e-15, 1)} -- cycle;

    % Region B (light blue, right of dashed line)
    \addplot[fill=blue!10, draw=none, fill opacity=0.4] 
        coordinates {(2.12095088792019e-10, 0) (1e-6, 0) (1e-6, 1) (2.12095088792019e-10, 1)} -- cycle;

    % Vertical dashed line at x = 2.12095088792019e-10
    \addplot[thick, dashed, black] 
        coordinates {(2.12095088792019e-10, 0) (2.12095088792019e-10, 1)};

    % Labels for regions
   \node[anchor=east, text=magenta] at (axis cs:1e-11, 0.5) {Monostatic Dominant}; % Left region label
    \node[anchor=west, text=magenta] at (axis cs:3.162e-10, 0.8) {Multistatic Dominant};  % Right region label

\end{axis}



\end{tikzpicture}%
}%
\subfloat[ \label{no_fd}]{% 
 % This file was created by matlab2tikz.
%
%The latest updates can be retrieved from
%  http://www.mathworks.com/matlabcentral/fileexchange/22022-matlab2tikz-matlab2tikz
%where you can also make suggestions and rate matlab2tikz.
%
\definecolor{mycolor1}{rgb}{0.00000,0.44700,0.74100}%
\definecolor{mycolor2}{rgb}{0.85000,0.32500,0.09800}%
\definecolor{mycolor3}{rgb}{0.92900,0.69400,0.12500}%
\definecolor{mycolor4}{rgb}{0.49400,0.18400,0.55600}%
\definecolor{mycolor5}{rgb}{0.46600,0.67400,0.18800}%
\definecolor{mycolor6}{rgb}{0.30100,0.74500,0.93300}%
%
\begin{tikzpicture}[scale=0.42, transform shape,font=\Large]

\begin{axis}[%
width=4.521in,
height=3.559in,
at={(0.758in,0.488in)},
scale only axis,
xmin=0,
xmax=1000,
xlabel style={font=\large\color{white!15!black}},
xlabel={$\text{BSs Density/ Km}^\text{2}$},
ymin=0,
ymax=0.9,
ylabel style={font=\large\color{white!15!black}},
ylabel={Average Sensing Coverage Probability},
axis background/.style={fill=white},
xmajorgrids,
ymajorgrids,
legend style={at={(0.68,0.01)}, anchor=south west, legend cell align=left, align=left, draw=white!15!black}
]
 \addplot [
            color=mycolor1, 
            line width=1.5pt,
            mark=o, 
            mark options={solid, mycolor1},
            mark repeat=3 % Places a marker every 2 points
        ] 
  table[row sep=crcr]{%
10	-0.000299999999999995\\
30.2040816326531	0.0154571428571429\\
50.4081632653061	0.0324\\
70.6122448979592	0.0505285714285714\\
90.8163265306122	0.0712619047619048\\
111.020408163265	0.0915809523809524\\
131.224489795918	0.110209523809524\\
151.428571428571	0.128390476190476\\
171.632653061224	0.145995238095238\\
191.836734693878	0.162857142857143\\
212.040816326531	0.17892380952381\\
232.244897959184	0.193633333333333\\
252.448979591837	0.206452380952381\\
272.65306122449	0.219114285714286\\
292.857142857143	0.230471428571429\\
313.061224489796	0.241314285714286\\
333.265306122449	0.251128571428571\\
353.469387755102	0.259509523809524\\
373.673469387755	0.267638095238095\\
393.877551020408	0.275647619047619\\
414.081632653061	0.282780952380952\\
434.285714285714	0.289452380952381\\
454.489795918367	0.296757142857143\\
474.69387755102	0.303357142857143\\
494.897959183673	0.310042857142857\\
515.102040816327	0.314804761904762\\
535.30612244898	0.318042857142857\\
555.510204081633	0.321604761904762\\
575.714285714286	0.32537619047619\\
595.918367346939	0.329214285714286\\
616.122448979592	0.334066666666667\\
636.326530612245	0.337595238095238\\
656.530612244898	0.340780952380952\\
676.734693877551	0.344647619047619\\
696.938775510204	0.347204761904762\\
717.142857142857	0.3498\\
737.34693877551	0.352166666666667\\
757.551020408163	0.35437619047619\\
777.755102040816	0.356952380952381\\
797.959183673469	0.359057142857143\\
818.163265306122	0.361328571428571\\
838.367346938776	0.363442857142857\\
858.571428571429	0.363757142857143\\
878.775510204082	0.365066666666667\\
898.979591836735	0.36717619047619\\
919.183673469388	0.368476190476191\\
939.387755102041	0.370814285714286\\
959.591836734694	0.372407142857143\\
979.795918367347	0.3737\\
1000	0.374692857142857\\
};
\addlegendentry{Bistatic}

 \addplot [
            color=mycolor2, 
            line width=1.5pt,
            mark=x, 
            mark options={solid, mycolor2},
            mark repeat=3 % Places a marker every 2 points
        ]
  table[row sep=crcr]{%
10	-0.000673809523809528\\
30.2040816326531	0.0193428571428571\\
50.4081632653061	0.04195\\
70.6122448979592	0.0671476190476191\\
90.8163265306122	0.0969333333333333\\
111.020408163265	0.127261904761905\\
131.224489795918	0.1559\\
151.428571428571	0.183604761904762\\
171.632653061224	0.210085714285714\\
191.836734693878	0.235452380952381\\
212.040816326531	0.25977619047619\\
232.244897959184	0.282580952380952\\
252.448979591837	0.302933333333333\\
272.65306122449	0.322666666666667\\
292.857142857143	0.340233333333333\\
313.061224489796	0.356833333333333\\
333.265306122449	0.372238095238095\\
353.469387755102	0.386038095238095\\
373.673469387755	0.399057142857143\\
393.877551020408	0.412052380952381\\
414.081632653061	0.422904761904762\\
434.285714285714	0.433438095238095\\
454.489795918367	0.443695238095238\\
474.69387755102	0.453719047619048\\
494.897959183673	0.4629\\
515.102040816327	0.470495238095238\\
535.30612244898	0.476547619047619\\
555.510204081633	0.482314285714286\\
575.714285714286	0.488528571428571\\
595.918367346939	0.495480952380952\\
616.122448979592	0.502585714285714\\
636.326530612245	0.507890476190476\\
656.530612244898	0.512795238095238\\
676.734693877551	0.516552380952381\\
696.938775510204	0.520428571428571\\
717.142857142857	0.524404761904762\\
737.34693877551	0.528485714285714\\
757.551020408163	0.532633333333333\\
777.755102040816	0.537133333333333\\
797.959183673469	0.540890476190476\\
818.163265306122	0.544152380952381\\
838.367346938776	0.546871428571429\\
858.571428571429	0.548166666666667\\
878.775510204082	0.550371428571429\\
898.979591836735	0.552780952380952\\
919.183673469388	0.55532380952381\\
939.387755102041	0.557757142857143\\
959.591836734694	0.559671428571429\\
979.795918367347	0.561292857142857\\
1000	0.562621428571429\\
};
\addlegendentry{Multistatic 2}

 \addplot [
            color=mycolor3, 
            line width=1.5pt,
            mark=square, 
            mark options={solid, mycolor3},
            mark repeat=3 % Places a marker every 2 points
        ]
  table[row sep=crcr]{%
10	-0.000914285714285715\\
30.2040816326531	0.0207\\
50.4081632653061	0.0458142857142857\\
70.6122448979592	0.0744285714285714\\
90.8163265306122	0.108809523809524\\
111.020408163265	0.144304761904762\\
131.224489795918	0.178647619047619\\
151.428571428571	0.211880952380952\\
171.632653061224	0.244042857142857\\
191.836734693878	0.274895238095238\\
212.040816326531	0.304371428571429\\
232.244897959184	0.332047619047619\\
252.448979591837	0.357009523809524\\
272.65306122449	0.380704761904762\\
292.857142857143	0.402919047619048\\
313.061224489796	0.423604761904762\\
333.265306122449	0.442752380952381\\
353.469387755102	0.460042857142857\\
373.673469387755	0.476214285714286\\
393.877551020408	0.492042857142857\\
414.081632653061	0.505390476190476\\
434.285714285714	0.517561904761905\\
454.489795918367	0.529728571428571\\
474.69387755102	0.541795238095238\\
494.897959183673	0.553195238095238\\
515.102040816327	0.563266666666667\\
535.30612244898	0.571371428571429\\
555.510204081633	0.578433333333333\\
575.714285714286	0.585895238095238\\
595.918367346939	0.593766666666667\\
616.122448979592	0.601066666666667\\
636.326530612245	0.607690476190476\\
656.530612244898	0.613861904761905\\
676.734693877551	0.619033333333333\\
696.938775510204	0.624161904761905\\
717.142857142857	0.629571428571429\\
737.34693877551	0.633719047619048\\
757.551020408163	0.63877619047619\\
777.755102040816	0.643257142857143\\
797.959183673469	0.647414285714286\\
818.163265306122	0.65097619047619\\
838.367346938776	0.654271428571429\\
858.571428571429	0.655761904761905\\
878.775510204082	0.658414285714286\\
898.979591836735	0.661214285714286\\
919.183673469388	0.664809523809524\\
939.387755102041	0.667938095238095\\
959.591836734694	0.670235714285714\\
979.795918367347	0.672014285714286\\
1000	0.67327380952381\\
};
\addlegendentry{Multistatic 3}

 \addplot [
            color=mycolor4, 
            line width=1.5pt,
            mark=diamond, 
            mark options={solid, mycolor4},
            mark repeat=3 % Places a marker every 2 points
        ]
  table[row sep=crcr]{%
10	-0.000954761904761911\\
30.2040816326531	0.0211714285714286\\
50.4081632653061	0.0474642857142857\\
70.6122448979592	0.0779238095238095\\
90.8163265306122	0.11492380952381\\
111.020408163265	0.153609523809524\\
131.224489795918	0.19147619047619\\
151.428571428571	0.228433333333333\\
171.632653061224	0.264342857142857\\
191.836734693878	0.299071428571429\\
212.040816326531	0.331971428571429\\
232.244897959184	0.362585714285714\\
252.448979591837	0.390790476190476\\
272.65306122449	0.417433333333333\\
292.857142857143	0.442219047619048\\
313.061224489796	0.465709523809524\\
333.265306122449	0.487290476190476\\
353.469387755102	0.506647619047619\\
373.673469387755	0.525357142857143\\
393.877551020408	0.543442857142857\\
414.081632653061	0.558166666666667\\
434.285714285714	0.571819047619048\\
454.489795918367	0.585590476190476\\
474.69387755102	0.598552380952381\\
494.897959183673	0.61097619047619\\
515.102040816327	0.622247619047619\\
535.30612244898	0.631133333333333\\
555.510204081633	0.6396\\
575.714285714286	0.648342857142857\\
595.918367346939	0.657\\
616.122448979592	0.66507619047619\\
636.326530612245	0.672557142857143\\
656.530612244898	0.679628571428571\\
676.734693877551	0.685428571428571\\
696.938775510204	0.691009523809524\\
717.142857142857	0.697166666666667\\
737.34693877551	0.701833333333333\\
757.551020408163	0.707357142857143\\
777.755102040816	0.712566666666667\\
797.959183673469	0.7163\\
818.163265306122	0.720004761904762\\
838.367346938776	0.723890476190476\\
858.571428571429	0.726080952380952\\
878.775510204082	0.729242857142857\\
898.979591836735	0.73262380952381\\
919.183673469388	0.736028571428571\\
939.387755102041	0.73882380952381\\
959.591836734694	0.741278571428571\\
979.795918367347	0.743392857142857\\
1000	0.745166666666667\\
};
\addlegendentry{Multistatic 4}

 \addplot [
            color=mycolor5, 
            line width=1.5pt,
            mark=pentagon, 
            mark options={solid, mycolor5},
            mark repeat=3 % Places a marker every 2 points
        ]
  table[row sep=crcr]{%
10	-0.000976190476190471\\
30.2040816326531	0.0213714285714286\\
50.4081632653061	0.0483428571428571\\
70.6122448979592	0.0799380952380953\\
90.8163265306122	0.118628571428571\\
111.020408163265	0.159309523809524\\
131.224489795918	0.199461904761905\\
151.428571428571	0.238790476190476\\
171.632653061224	0.277080952380952\\
191.836734693878	0.313957142857143\\
212.040816326531	0.349385714285714\\
232.244897959184	0.382404761904762\\
252.448979591837	0.413480952380952\\
272.65306122449	0.442571428571429\\
292.857142857143	0.469361904761905\\
313.061224489796	0.494233333333333\\
333.265306122449	0.5172\\
353.469387755102	0.537909523809524\\
373.673469387755	0.558414285714286\\
393.877551020408	0.57817619047619\\
414.081632653061	0.594638095238095\\
434.285714285714	0.6093\\
454.489795918367	0.624090476190476\\
474.69387755102	0.6381\\
494.897959183673	0.651357142857143\\
515.102040816327	0.663528571428571\\
535.30612244898	0.673080952380952\\
555.510204081633	0.681671428571429\\
575.714285714286	0.691119047619048\\
595.918367346939	0.70067619047619\\
616.122448979592	0.709433333333333\\
636.326530612245	0.717580952380952\\
656.530612244898	0.725004761904762\\
676.734693877551	0.730928571428571\\
696.938775510204	0.737138095238095\\
717.142857142857	0.743314285714286\\
737.34693877551	0.74827619047619\\
757.551020408163	0.75397619047619\\
777.755102040816	0.759295238095238\\
797.959183673469	0.763047619047619\\
818.163265306122	0.767504761904762\\
838.367346938776	0.771428571428571\\
858.571428571429	0.774357142857143\\
878.775510204082	0.777666666666667\\
898.979591836735	0.781066666666667\\
919.183673469388	0.7842\\
939.387755102041	0.787085714285714\\
959.591836734694	0.7896\\
979.795918367347	0.791778571428572\\
1000	0.793621428571429\\
};
\addlegendentry{Multistatic 5}

 \addplot [
            color=mycolor6, 
            line width=1.5pt,
            mark=triangle, 
            mark options={solid, mycolor6},
            mark repeat=3 % Places a marker every 2 points
        ]
  table[row sep=crcr]{%
10	-0.00102857142857145\\
30.2040816326531	0.0215071428571428\\
50.4081632653061	0.0489357142857143\\
70.6122448979592	0.0812571428571429\\
90.8163265306122	0.120957142857143\\
111.020408163265	0.16297619047619\\
131.224489795918	0.204628571428571\\
151.428571428571	0.245633333333333\\
171.632653061224	0.285585714285714\\
191.836734693878	0.324342857142857\\
212.040816326531	0.361480952380952\\
232.244897959184	0.396485714285714\\
252.448979591837	0.429376190476191\\
272.65306122449	0.460138095238095\\
292.857142857143	0.488442857142857\\
313.061224489796	0.514947619047619\\
333.265306122449	0.539061904761905\\
353.469387755102	0.561071428571429\\
373.673469387755	0.582504761904762\\
393.877551020408	0.60297619047619\\
414.081632653061	0.620342857142857\\
434.285714285714	0.636109523809524\\
454.489795918367	0.6517\\
474.69387755102	0.666680952380952\\
494.897959183673	0.680571428571429\\
515.102040816327	0.693285714285714\\
535.30612244898	0.7035\\
555.510204081633	0.712857142857143\\
575.714285714286	0.722671428571429\\
595.918367346939	0.732552380952381\\
616.122448979592	0.74167619047619\\
636.326530612245	0.750066666666667\\
656.530612244898	0.757614285714286\\
676.734693877551	0.764085714285714\\
696.938775510204	0.770480952380952\\
717.142857142857	0.776847619047619\\
737.34693877551	0.78227619047619\\
757.551020408163	0.787971428571429\\
777.755102040816	0.793342857142857\\
797.959183673469	0.797166666666667\\
818.163265306122	0.801404761904762\\
838.367346938776	0.805338095238095\\
858.571428571429	0.8088\\
878.775510204082	0.812261904761905\\
898.979591836735	0.815861904761905\\
919.183673469388	0.819090476190476\\
939.387755102041	0.821638095238095\\
959.591836734694	0.824278571428572\\
979.795918367347	0.826764285714286\\
1000	0.829095238095239\\
};
\addlegendentry{Multistatic 6}



 % Define points for the line
    \coordinate (A) at (717.142857142857, 0.3498);
    \coordinate (B) at (717.142857142857, 0.776847619047619);
    \coordinate (M) at (717.042857142857, 0.8203238095238095); % Midpoint for the line

    % Draw the line
    \draw[<->, line width=0.8mm, magenta] (A) -- (B);


    % Add text at the middle of the line
    \node[above, text=magenta] at (M) {\textbf{The coordination Gain}};



\end{axis}

\begin{axis}[%
width=5.833in,
height=4.375in,
at={(0in,0in)},
scale only axis,
xmin=0,
xmax=1,
ymin=0,
ymax=1,
axis line style={draw=none},
ticks=none,
axis x line*=bottom,
axis y line*=left
]
\end{axis}
\end{tikzpicture}%
}%
\subfloat[  \label{energy_eff}]{% 
 % This file was created by matlab2tikz.
%
%The latest updates can be retrieved from
%  http://www.mathworks.com/matlabcentral/fileexchange/22022-matlab2tikz-matlab2tikz
%where you can also make suggestions and rate matlab2tikz.
%
\definecolor{mycolor1}{rgb}{0.00000,0.44700,0.74100}%
\definecolor{mycolor2}{rgb}{0.85000,0.32500,0.09800}%
\definecolor{mycolor3}{rgb}{0.92900,0.69400,0.12500}%
\definecolor{mycolor4}{rgb}{0.49400,0.18400,0.55600}%
\definecolor{mycolor5}{rgb}{0.46600,0.67400,0.18800}%
\definecolor{mycolor6}{rgb}{0.30100,0.74500,0.93300}%
%
\begin{tikzpicture}[scale=0.42, transform shape,font=\Large]

\begin{axis}[%
width=4.521in,
height=3.566in,
at={(0.758in,0.481in)},
scale only axis,
xmin=0,
xmax=1,
xlabel style={font=\large\color{white!15!black}},
xlabel={Fraction of energy given to sensing pulse $(\alpha)$},
ymin=0,
ymax=100,
ylabel style={font=\large\color{white!15!black}},
ylabel={Average Rate (nats/s/Hz)},
axis background/.style={fill=white},
xmajorgrids,
ymajorgrids,
legend style={at={(0.174,0)}, anchor=south west, legend cell align=left, align=left, draw=white!15!black,font=\large}
]
 \addplot [
            color=mycolor1, 
            line width=1.5pt,
            mark=o, 
            mark options={solid, mycolor1},
            mark repeat=4 % Places a marker every 2 points
        ] 
  table[row sep=crcr]{%
0	0\\
0.00401606425702811	10.1031392677923\\
0.00803212851405622	16.1027942791471\\
0.0120481927710843	20.3565070536414\\
0.0160642570281124	21.3695924080075\\
0.0200803212851406	22.7495347748971\\
0.0240963855421687	23.9906200482614\\
0.0281124497991968	25.1090767937635\\
0.0321285140562249	26.0792478776615\\
0.036144578313253	27.0035332635756\\
0.0401606425702811	27.819418688001\\
0.0441767068273092	28.4930755915275\\
0.0481927710843374	29.1231325105438\\
0.0522088353413655	29.679246598687\\
0.0562248995983936	30.2341035444197\\
0.0602409638554217	30.8326747603362\\
0.0642570281124498	31.4350174034843\\
0.0682730923694779	31.9413600807384\\
0.072289156626506	32.3678170720878\\
0.0763052208835341	32.7206740895852\\
0.0803212851405622	33.1088453802507\\
0.0843373493975904	33.4839309612794\\
0.0883534136546185	33.8658736827292\\
0.0923694779116466	34.196159279674\\
0.0963855421686747	34.5503305824187\\
0.100401606425703	34.8990733156635\\
0.104417670682731	35.2470160491924\\
0.108433734939759	35.5687873634479\\
0.112449799196787	35.8433586944722\\
0.116465863453815	36.0951300335966\\
0.120481927710843	36.3727299349734\\
0.124497991967871	36.677872683708\\
0.1285140562249	36.9449011602691\\
0.132530120481928	37.1996439269093\\
0.136546184738956	37.4237581330024\\
0.140562248995984	37.7287294532264\\
0.144578313253012	37.9622722273983\\
0.14859437751004	38.2481578401172\\
0.152610441767068	38.4577006228156\\
0.156626506024096	38.5971862875462\\
0.160642570281124	38.7513005185082\\
0.164658634538153	38.9714718688591\\
0.168674698795181	39.2212432086941\\
0.172690763052209	39.4978716818446\\
0.176706827309237	39.7351287404113\\
0.180722891566265	39.9354143835413\\
0.184738955823293	40.0670143367877\\
0.188755020080321	40.247014272839\\
0.192771084337349	40.4842141885688\\
0.196787148594378	40.6447855600939\\
0.200803212851406	40.864556910587\\
0.204819277108434	41.071356837117\\
0.208835341365462	41.2046139326318\\
0.21285140562249	41.3612995912517\\
0.216867469879518	41.4418709911984\\
0.220883534136546	41.543070955245\\
0.224899598393574	41.7864422973537\\
0.228915662650602	41.9578708078788\\
0.232931726907631	42.1585564508667\\
0.236947791164659	42.3027278282183\\
0.240963855421687	42.4168420733911\\
0.244979919678715	42.5361563167165\\
0.248995983935743	42.6762134098154\\
0.253012048192771	42.7719276615253\\
0.257028112449799	43.0317561406444\\
0.261044176706827	43.2162132179693\\
0.265060240963855	43.4219845734361\\
0.269076305220884	43.5854130868033\\
0.273092369477912	43.6810701956763\\
0.27710843373494	43.7783273039808\\
0.281124497991968	43.9531843847164\\
0.285140562248996	43.9978129402897\\
0.289156626506024	44.0975271905784\\
0.293172690763052	44.2619842750088\\
0.29718875502008	44.4524413502021\\
0.301204819277108	44.589241301601\\
0.305220883534137	44.7135841145685\\
0.309236947791165	44.7398698195157\\
0.313253012048193	44.7807840906943\\
0.317269076305221	44.8825554831094\\
0.321285140562249	45.0360982857029\\
0.325301204819277	45.1467839606653\\
0.329317269076305	45.3224981839534\\
0.333333333333333	45.4558124223051\\
0.337349397590361	45.6407837851615\\
0.34136546184739	45.7199266141873\\
0.345381526104418	45.8558122801968\\
0.349397590361446	45.9050122627174\\
0.353413654618474	46.0040407989641\\
0.357429718875502	46.0983264797528\\
0.36144578313253	46.2700978472989\\
0.365461847389558	46.291354982604\\
0.369477911646586	46.4300406476188\\
0.373493975903614	46.5413548937863\\
0.377510040160643	46.6512405690329\\
0.381526104417671	46.7608405300952\\
0.385542168674699	46.8474119279104\\
0.389558232931727	46.9566690319516\\
0.393574297188755	47.0780404174034\\
0.397590361445783	47.2463832147389\\
0.401606425702811	47.3434117516961\\
0.405622489959839	47.4369545756059\\
0.409638554216867	47.4983831252107\\
0.413654618473896	47.5970116615994\\
0.417670682730924	47.7520401779509\\
0.421686746987952	47.8384401472555\\
0.42570281124498	47.9546686773915\\
0.429718875502008	48.011525800049\\
0.433734939759036	48.0931257710589\\
0.437751004016064	48.1030114818325\\
0.441767068273092	48.2207828685632\\
0.44578313253012	48.2729542785996\\
0.449799196787149	48.4110113724091\\
0.453815261044177	48.4894684873927\\
0.457831325301205	48.6060398745497\\
0.461847389558233	48.6813541335071\\
0.465863453815261	48.7875255243589\\
0.469879518072289	48.8332969366691\\
0.473895582329317	48.9453540397156\\
0.477911646586345	49.0623254267305\\
0.481927710843373	49.1210682630037\\
0.485943775100402	49.1951825223874\\
0.48995983935743	49.2696396387921\\
0.493975903614458	49.2887824891341\\
0.497991967871486	49.3425538986021\\
0.502008032128514	49.4344395802435\\
0.506024096385542	49.4913538457378\\
0.51004016064257	49.6472966474787\\
0.514056224899598	49.7533537526569\\
0.518072289156627	49.8402108646562\\
0.522088353413655	49.868096569035\\
0.526104417670683	49.9343251169345\\
0.530120481927711	50.0495822188441\\
0.534136546184739	50.156839323596\\
0.538152610441767	50.2213535863902\\
0.542168674698795	50.3084964125738\\
0.546184738955823	50.3555821101313\\
0.550200803212851	50.4663820707673\\
0.55421686746988	50.6017534512452\\
0.558232931726908	50.646953435187\\
0.562248995983936	50.7272391209495\\
0.566265060240964	50.7643819648966\\
0.570281124497992	50.7684391063124\\
0.57429718875502	50.8352962254172\\
0.578313253012048	50.9534676120057\\
0.582329317269076	51.120496124094\\
0.586345381526104	51.2501532208877\\
0.590361445783133	51.3397531890555\\
0.594377510040161	51.4058103084444\\
0.598393574297189	51.4848388517965\\
0.602409638554217	51.5963245264746\\
0.606425702811245	51.6704959286951\\
0.610441767068273	51.6522673637426\\
0.614457831325301	51.6837530668423\\
0.618473895582329	51.718153054621\\
0.622489959839357	51.8124958782466\\
0.626506024096386	51.9259815522142\\
0.630522088353414	52.0334100854766\\
0.634538152610442	52.0778100697026\\
0.63855421686747	52.1353529064021\\
0.642570281124498	52.2208385903172\\
0.646586345381526	52.2748385711326\\
0.650602409638554	52.304667131964\\
0.654618473895582	52.3871813883634\\
0.65863453815261	52.4123242365737\\
0.662650602409639	52.4607242193786\\
0.666666666666667	52.584381318304\\
0.670682730923695	52.6676955744192\\
0.674698795180723	52.7239812687082\\
0.678714859437751	52.8438097975652\\
0.682730923694779	52.9166097717015\\
0.686746987951807	53.0176383072376\\
0.690763052208835	53.1242668407841\\
0.694779116465863	53.1862096759205\\
0.698795180722892	53.2833524985514\\
0.70281124497992	53.3339239091562\\
0.706827309236948	53.4127810239977\\
0.710843373493976	53.4395238716397\\
0.714859437751004	53.4716952887815\\
0.718875502008032	53.5078667045023\\
0.72289156626506	53.5528952599335\\
0.726907630522088	53.5879809617543\\
0.730923694779116	53.7340951955585\\
0.734939759036145	53.824495163442\\
0.738955823293173	53.95878083002\\
0.742971887550201	54.0280950911089\\
0.746987951807229	54.0204379509721\\
0.751004016064257	54.0583236517981\\
0.755020080321285	54.1659807564079\\
0.759036144578313	54.2018664579444\\
0.763052208835341	54.2511807261388\\
0.767068273092369	54.2833521432807\\
0.771084337349398	54.3789521093168\\
0.775100401606426	54.4594092235899\\
0.779116465863454	54.554266332747\\
0.783132530120482	54.623237736815\\
0.78714859437751	54.7145519900879\\
0.791164658634538	54.7830662514611\\
0.795180722891566	54.9031233516655\\
0.799196787148594	54.9048376367707\\
0.803212851405622	54.9746090405544\\
0.807228915662651	55.0431233019276\\
0.811244979919679	55.1219804167691\\
0.815261044176707	55.1747803980108\\
0.819277108433735	55.2794089322679\\
0.823293172690763	55.3508946211568\\
0.827309236947791	55.4402088751404\\
0.831325301204819	55.4882088580874\\
0.835341365461847	55.5336945562134\\
0.839357429718876	55.5398088397554\\
0.843373493975904	55.6048945309181\\
0.847389558232932	55.6678087942808\\
0.85140562248996	55.6606659396756\\
0.855421686746988	55.7451801953644\\
0.859437751004016	55.8605515829477\\
0.863453815261044	55.9587801194786\\
0.867469879518072	55.9911801079678\\
0.8714859437751	56.0123229575992\\
0.875502008032129	56.0135229571729\\
0.879518072289157	56.099008641088\\
0.883534136546185	56.2042086037136\\
0.887550200803213	56.3686656881439\\
0.891566265060241	56.4222085262646\\
0.895582329317269	56.4770085067957\\
0.899598393574297	56.5520941944057\\
0.903614457831325	56.6279227388946\\
0.907630522088353	56.6449513042734\\
0.911646586345382	56.6863798609836\\
0.91566265060241	56.785865539925\\
0.919678714859438	56.8260940970615\\
0.923694779116466	56.889065503261\\
0.927710843373494	56.9712369026394\\
0.931726907630522	57.017922600339\\
0.93574297188755	57.0336940233073\\
0.939759036144578	57.0951225729121\\
0.943775100401606	57.1724368311589\\
0.947791164658635	57.2399225214689\\
0.951807228915663	57.3447796270734\\
0.955823293172691	57.4782081510987\\
0.959839357429719	57.4888938615881\\
0.963855421686747	57.5536938385666\\
0.967871485943775	57.6679223694131\\
0.971887550200803	57.697179501876\\
0.975903614457831	57.7678080482123\\
0.979919678714859	57.8829508644483\\
0.983935742971888	57.9323794183163\\
0.987951807228916	58.0216936722999\\
0.991967871485944	58.2899792912715\\
0.995983935742972	58.6549505901793\\
1	59.1166075690233\\
};
\addlegendentry{ISAC dual-mode sensing}

 \addplot [
            color=mycolor2, 
            line width=1.5pt,
            mark=asterisk, 
            mark options={solid, mycolor2},
            mark repeat=4 % Places a marker every 2 points
        ] 
  table[row sep=crcr]{%
0	45.7472858545455\\
0.00401606425702811	45.7346769388\\
0.00803212851405622	45.719974808791\\
0.0120481927710843	45.7031776587771\\
0.0160642570281124	45.6842856029284\\
0.0200803212851406	45.6632986412449\\
0.0240963855421687	45.6446258984629\\
0.0281124497991968	45.6119212840477\\
0.0321285140562249	45.5593719299385\\
0.036144578313253	45.4802793730094\\
0.0401606425702811	45.439582313184\\
0.0441767068273092	45.3766093456413\\
0.0481927710843374	45.3658617728\\
0.0522088353413655	45.416363336\\
0.0562248995983936	45.3712733762452\\
0.0602409638554217	45.4399635465274\\
0.0642570281124498	45.436630048\\
0.0682730923694779	45.4230864792743\\
0.072289156626506	45.336344047632\\
0.0763052208835341	45.2935730477855\\
0.0803212851405622	45.283672816\\
0.0843373493975904	45.2288774582033\\
0.0883534136546185	45.2398804133034\\
0.0923694779116466	45.1996289469487\\
0.0963855421686747	45.2383644587304\\
0.100401606425703	45.1959799487877\\
0.104417670682731	45.2116090477877\\
0.108433734939759	45.163379008\\
0.112449799196787	45.1916982842133\\
0.116465863453815	45.2209410429788\\
0.120481927710843	45.191249352\\
0.124497991967871	45.1915488585372\\
0.1285140562249	45.1789497041708\\
0.132530120481928	45.1782119925492\\
0.136546184738956	45.1386963664492\\
0.140562248995984	45.1174553264492\\
0.144578313253012	45.0286351205461\\
0.14859437751004	45.0086148934936\\
0.152610441767068	45.0015409212291\\
0.156626506024096	45.0219897601653\\
0.160642570281124	45.0470831297949\\
0.164658634538153	45.0334116810539\\
0.168674698795181	45.0193148934936\\
0.172690763052209	44.9803046414936\\
0.176706827309237	44.9266426414936\\
0.180722891566265	44.8386436414936\\
0.184738955823293	44.7903267214936\\
0.188755020080321	44.8634576414936\\
0.192771084337349	44.8993070414936\\
0.196787148594378	44.9428720414936\\
0.200803212851406	44.9201109302222\\
0.204819277108434	44.9064560788794\\
0.208835341365462	44.8641917468161\\
0.21285140562249	44.8258310656869\\
0.216867469879518	44.7550796504556\\
0.220883534136546	44.6873849502124\\
0.224899598393574	44.7028477888282\\
0.228915662650602	44.6739889782935\\
0.232931726907631	44.6808096786764\\
0.236947791164659	44.6808022350605\\
0.240963855421687	44.6608070077611\\
0.244979919678715	44.6837813455148\\
0.248995983935743	44.6388964646475\\
0.253012048192771	44.6021149880536\\
0.257028112449799	44.5859606369693\\
0.261044176706827	44.5502969342325\\
0.265060240963855	44.5358727937454\\
0.269076305220884	44.4961862086153\\
0.273092369477912	44.4833893974943\\
0.27710843373494	44.4442136891271\\
0.281124497991968	44.4467685457356\\
0.285140562248996	44.3969265931185\\
0.289156626506024	44.3858723632436\\
0.293172690763052	44.3685184960848\\
0.29718875502008	44.3302895457697\\
0.301204819277108	44.272633184\\
0.305220883534137	44.2604613898915\\
0.309236947791165	44.2722590034382\\
0.313253012048193	44.2577824898534\\
0.317269076305221	44.2824332273251\\
0.321285140562249	44.2749297325459\\
0.325301204819277	44.2632221263601\\
0.329317269076305	44.2396927666403\\
0.333333333333333	44.208749632\\
0.337349397590361	44.1561686179327\\
0.34136546184739	44.1312993031211\\
0.345381526104418	44.0993411652373\\
0.349397590361446	44.0787884352721\\
0.353413654618474	44.0430475565467\\
0.357429718875502	44.0239447029787\\
0.36144578313253	44.0542187025457\\
0.365461847389558	44.064347895\\
0.369477911646586	44.0749449607071\\
0.373493975903614	44.0318750288\\
0.377510040160643	43.9629010568182\\
0.381526104417671	43.9282854919193\\
0.385542168674699	43.8551298927475\\
0.389558232931727	43.797203856\\
0.393574297188755	43.782328672\\
0.397590361445783	43.8260363671071\\
0.401606425702811	43.8002728480001\\
0.405622489959839	43.7849213236363\\
0.409638554216867	43.7378011967273\\
0.413654618473896	43.7245662152727\\
0.417670682730924	43.7180342429091\\
0.421686746987952	43.6837645047273\\
0.42570281124498	43.6939965570909\\
0.429718875502008	43.6887086549091\\
0.433734939759036	43.6767244443636\\
0.437751004016064	43.5892768912727\\
0.441767068273092	43.4509470127273\\
0.44578313253012	43.3683560629091\\
0.449799196787149	43.3212467054545\\
0.453815261044177	43.3127601210909\\
0.457831325301205	43.3304907640909\\
0.461847389558233	43.3098339992727\\
0.465863453815261	43.3033524072727\\
0.469879518072289	43.2914245403636\\
0.473895582329317	43.2543239600001\\
0.477911646586345	43.1866034803636\\
0.481927710843373	43.1833852349091\\
0.485943775100402	43.1207105836363\\
0.48995983935743	43.104980136\\
0.493975903614458	43.1058267727273\\
0.497991967871486	43.045390272\\
0.502008032128514	42.9899989600001\\
0.506024096385542	42.9543650014545\\
0.51004016064257	42.9354780509091\\
0.514056224899598	42.9387828909091\\
0.518072289156627	42.98201184\\
0.522088353413655	42.9643289163636\\
0.526104417670683	42.9539905963636\\
0.530120481927711	42.8962363672727\\
0.534136546184739	42.7857827490909\\
0.538152610441767	42.7333384363636\\
0.542168674698795	42.705392896\\
0.546184738955823	42.693165344\\
0.550200803212851	42.712486056\\
0.55421686746988	42.7129570472727\\
0.558232931726908	42.7096370509091\\
0.562248995983936	42.691067344\\
0.566265060240964	42.6509223680001\\
0.570281124497992	42.5729988480001\\
0.57429718875502	42.5803144032727\\
0.578313253012048	42.5287858989091\\
0.582329317269076	42.5106116756363\\
0.586345381526104	42.452761232\\
0.590361445783133	42.3675647614545\\
0.594377510040161	42.2868346690909\\
0.598393574297189	42.2282348232727\\
0.602409638554217	42.1896222989091\\
0.606425702811245	42.1472575338182\\
0.610441767068273	42.1118658556364\\
0.614457831325301	42.0760304167273\\
0.618473895582329	42.0743802530909\\
0.622489959839357	42.0698190996364\\
0.626506024096386	42.0622629796364\\
0.630522088353414	42.0199521141818\\
0.634538152610442	41.9764794600909\\
0.63855421686747	41.9444326792727\\
0.642570281124498	41.8824174603636\\
0.646586345381526	41.8046655089091\\
0.650602409638554	41.7308059301818\\
0.654618473895582	41.6976190170909\\
0.65863453815261	41.6509991669091\\
0.662650602409639	41.6181809607273\\
0.666666666666667	41.5625652021818\\
0.670682730923695	41.5224423523636\\
0.674698795180723	41.5439474290909\\
0.678714859437751	41.4670360890909\\
0.682730923694779	41.4362428578182\\
0.686746987951807	41.3674249701818\\
0.690763052208835	41.356336488\\
0.694779116465863	41.3274480432727\\
0.698795180722892	41.31399548\\
0.70281124497992	41.2549643621818\\
0.706827309236948	41.2164509469091\\
0.710843373493976	41.1679396032727\\
0.714859437751004	41.0338310792727\\
0.718875502008032	40.9477583607273\\
0.72289156626506	40.854048768\\
0.726907630522088	40.8123622530909\\
0.730923694779116	40.7600760549091\\
0.734939759036145	40.6873048363636\\
0.738955823293173	40.631995208\\
0.742971887550201	40.623432232\\
0.746987951807229	40.5497436436364\\
0.751004016064257	40.4578856461818\\
0.755020080321285	40.4079900530909\\
0.759036144578313	40.3585572789091\\
0.763052208835341	40.3393200289091\\
0.767068273092369	40.3019323149091\\
0.771084337349398	40.2064905912727\\
0.775100401606426	40.15881236\\
0.779116465863454	40.0817013701818\\
0.783132530120482	39.951112768\\
0.78714859437751	39.8629313323636\\
0.791164658634538	39.7747856069091\\
0.795180722891566	39.7509683250909\\
0.799196787148594	39.7686170407273\\
0.803212851405622	39.6992055963636\\
0.807228915662651	39.6125433883636\\
0.811244979919679	39.5322424319091\\
0.815261044176707	39.4394978094546\\
0.819277108433735	39.3285768407273\\
0.823293172690763	39.1875643563636\\
0.827309236947791	39.0469334290909\\
0.831325301204819	38.9500470090909\\
0.835341365461847	38.8758324734546\\
0.839357429718876	38.7649079272727\\
0.843373493975904	38.6158390531818\\
0.847389558232932	38.5357704683636\\
0.85140562248996	38.4948436116364\\
0.855421686746988	38.4467496589091\\
0.859437751004016	38.326909872\\
0.863453815261044	38.1878647829091\\
0.867469879518072	38.0802470483636\\
0.8714859437751	37.9799672032727\\
0.875502008032129	37.8614045643636\\
0.879518072289157	37.6755654290909\\
0.883534136546185	37.5309387963636\\
0.887550200803213	37.4163047589091\\
0.891566265060241	37.2579801636364\\
0.895582329317269	37.0896479323636\\
0.899598393574297	36.898440784\\
0.903614457831325	36.747198248\\
0.907630522088353	36.6329492069091\\
0.911646586345382	36.478975144\\
0.91566265060241	36.3020731749091\\
0.919678714859438	36.1116254407273\\
0.923694779116466	35.9529591912727\\
0.927710843373494	35.6937499789091\\
0.931726907630522	35.4780070029091\\
0.93574297188755	35.2473540763636\\
0.939759036144578	34.9783752632727\\
0.943775100401606	34.6565144923636\\
0.947791164658635	34.2984350429091\\
0.951807228915663	33.9607345090909\\
0.955823293172691	33.60836988\\
0.959839357429719	33.2423519636364\\
0.963855421686747	32.8626807643636\\
0.967871485943775	32.3499932\\
0.971887550200803	31.6738544\\
0.975903614457831	31.0393656\\
0.979919678714859	30.2651444\\
0.983935742971888	29.1952488\\
0.987951807228916	27.6109998\\
0.991967871485944	25.87357444\\
0.995983935742972	22.36506608\\
1	0\\
};
\addlegendentry{ISAC communication}

 \addplot [
            color=mycolor3, 
            line width=1.5pt,
            mark=diamond, 
            mark options={solid, mycolor3},
            mark repeat=4 % Places a marker every 2 points
        ] 
  table[row sep=crcr]{%
0	45.7472858545455\\
0.00401606425702811	55.8378162065923\\
0.00803212851405622	61.8227690879381\\
0.0120481927710843	66.0596847124185\\
0.0160642570281124	67.0538780109359\\
0.0200803212851406	68.412833416142\\
0.0240963855421687	69.6352459467243\\
0.0281124497991968	70.7209980778112\\
0.0321285140562249	71.6386198076\\
0.036144578313253	72.483812636585\\
0.0401606425702811	73.259001001185\\
0.0441767068273092	73.8696849371688\\
0.0481927710843374	74.4889942833438\\
0.0522088353413655	75.095609934687\\
0.0562248995983936	75.6053769206649\\
0.0602409638554217	76.2726383068636\\
0.0642570281124498	76.8716474514843\\
0.0682730923694779	77.3644465600127\\
0.072289156626506	77.7041611197198\\
0.0763052208835341	78.0142471373707\\
0.0803212851405622	78.3925181962507\\
0.0843373493975904	78.7128084194827\\
0.0883534136546185	79.1057540960326\\
0.0923694779116466	79.3957882266227\\
0.0963855421686747	79.7886950411491\\
0.100401606425703	80.0950532644512\\
0.104417670682731	80.4586250969801\\
0.108433734939759	80.7321663714479\\
0.112449799196787	81.0350569786855\\
0.116465863453815	81.3160710765754\\
0.120481927710843	81.5639792869734\\
0.124497991967871	81.8694215422452\\
0.1285140562249	82.1238508644399\\
0.132530120481928	82.3778559194585\\
0.136546184738956	82.5624544994516\\
0.140562248995984	82.8461847796756\\
0.144578313253012	82.9909073479444\\
0.14859437751004	83.2567727336108\\
0.152610441767068	83.4592415440447\\
0.156626506024096	83.6191760477115\\
0.160642570281124	83.7983836483031\\
0.164658634538153	84.004883549913\\
0.168674698795181	84.2405581021877\\
0.172690763052209	84.4781763233382\\
0.176706827309237	84.6617713819049\\
0.180722891566265	84.7740580250349\\
0.184738955823293	84.8573410582813\\
0.188755020080321	85.1104719143326\\
0.192771084337349	85.3835212300624\\
0.196787148594378	85.5876576015875\\
0.200803212851406	85.7846678408092\\
0.204819277108434	85.9778129159964\\
0.208835341365462	86.0688056794479\\
0.21285140562249	86.1871306569386\\
0.216867469879518	86.196950641654\\
0.220883534136546	86.2304559054574\\
0.224899598393574	86.4892900861819\\
0.228915662650602	86.6318597861723\\
0.232931726907631	86.8393661295431\\
0.236947791164659	86.9835300632788\\
0.240963855421687	87.0776490811522\\
0.244979919678715	87.2199376622313\\
0.248995983935743	87.3151098744629\\
0.253012048192771	87.3740426495789\\
0.257028112449799	87.6177167776137\\
0.261044176706827	87.7665101522018\\
0.265060240963855	87.9578573671815\\
0.269076305220884	88.0815992954186\\
0.273092369477912	88.1644595931706\\
0.27710843373494	88.2225409931079\\
0.281124497991968	88.399952930452\\
0.285140562248996	88.3947395334082\\
0.289156626506024	88.483399553822\\
0.293172690763052	88.6305027710936\\
0.29718875502008	88.7827308959718\\
0.301204819277108	88.861874485601\\
0.305220883534137	88.97404550446\\
0.309236947791165	89.0121288229539\\
0.313253012048193	89.0385665805477\\
0.317269076305221	89.1649887104345\\
0.321285140562249	89.3110280182488\\
0.325301204819277	89.4100060870254\\
0.329317269076305	89.5621909505937\\
0.333333333333333	89.6645620543051\\
0.337349397590361	89.7969524030942\\
0.34136546184739	89.8512259173084\\
0.345381526104418	89.9551534454341\\
0.349397590361446	89.9838006979895\\
0.353413654618474	90.0470883555108\\
0.357429718875502	90.1222711827315\\
0.36144578313253	90.3243165498446\\
0.365461847389558	90.355702877604\\
0.369477911646586	90.5049856083259\\
0.373493975903614	90.5732299225863\\
0.377510040160643	90.6141416258511\\
0.381526104417671	90.6891260220145\\
0.385542168674699	90.7025418206579\\
0.389558232931727	90.7538728879516\\
0.393574297188755	90.8603690894034\\
0.397590361445783	91.072419581846\\
0.401606425702811	91.1436845996962\\
0.405622489959839	91.2218758992422\\
0.409638554216867	91.236184321938\\
0.413654618473896	91.3215778768721\\
0.417670682730924	91.47007442086\\
0.421686746987952	91.5222046519828\\
0.42570281124498	91.6486652344824\\
0.429718875502008	91.7002344549581\\
0.433734939759036	91.7698502154225\\
0.437751004016064	91.6922883731052\\
0.441767068273092	91.6717298812905\\
0.44578313253012	91.6413103415087\\
0.449799196787149	91.7322580778636\\
0.453815261044177	91.8022286084836\\
0.457831325301205	91.9365306386406\\
0.461847389558233	91.9911881327798\\
0.465863453815261	92.0908779316316\\
0.469879518072289	92.1247214770327\\
0.473895582329317	92.1996779997157\\
0.477911646586345	92.2489289070941\\
0.481927710843373	92.3044534979128\\
0.485943775100402	92.3158931060237\\
0.48995983935743	92.3746197747921\\
0.493975903614458	92.3946092618614\\
0.497991967871486	92.3879441706021\\
0.502008032128514	92.4244385402436\\
0.506024096385542	92.4457188471923\\
0.51004016064257	92.5827746983878\\
0.514056224899598	92.692136643566\\
0.518072289156627	92.8222227046562\\
0.522088353413655	92.8324254853986\\
0.526104417670683	92.8883157132981\\
0.530120481927711	92.9458185861168\\
0.534136546184739	92.9426220726869\\
0.538152610441767	92.9546920227538\\
0.542168674698795	93.0138893085738\\
0.546184738955823	93.0487474541313\\
0.550200803212851	93.1788681267673\\
0.55421686746988	93.3147104985179\\
0.558232931726908	93.3565904860961\\
0.562248995983936	93.4183064649495\\
0.566265060240964	93.4153043328967\\
0.570281124497992	93.3414379543125\\
0.57429718875502	93.4156106286899\\
0.578313253012048	93.4822535109148\\
0.582329317269076	93.6311077997303\\
0.586345381526104	93.7029144528877\\
0.590361445783133	93.70731795051\\
0.594377510040161	93.6926449775353\\
0.598393574297189	93.7130736750692\\
0.602409638554217	93.7859468253837\\
0.606425702811245	93.8177534625133\\
0.610441767068273	93.764133219379\\
0.614457831325301	93.7597834835696\\
0.618473895582329	93.7925333077119\\
0.622489959839357	93.882314977883\\
0.626506024096386	93.9882445318506\\
0.630522088353414	94.0533621996584\\
0.634538152610442	94.0542895297935\\
0.63855421686747	94.0797855856748\\
0.642570281124498	94.1032560506808\\
0.646586345381526	94.0795040800417\\
0.650602409638554	94.0354730621458\\
0.654618473895582	94.0848004054543\\
0.65863453815261	94.0633234034828\\
0.662650602409639	94.0789051801059\\
0.666666666666667	94.1469465204858\\
0.670682730923695	94.1901379267828\\
0.674698795180723	94.2679286977991\\
0.678714859437751	94.3108458866561\\
0.682730923694779	94.3528526295197\\
0.686746987951807	94.3850632774194\\
0.690763052208835	94.4806033287841\\
0.694779116465863	94.5136577191932\\
0.698795180722892	94.5973479785514\\
0.70281124497992	94.588888271338\\
0.706827309236948	94.6292319709068\\
0.710843373493976	94.6074634749124\\
0.714859437751004	94.5055263680542\\
0.718875502008032	94.4556250652296\\
0.72289156626506	94.4069440279335\\
0.726907630522088	94.4003432148452\\
0.730923694779116	94.4941712504676\\
0.734939759036145	94.5117999998056\\
0.738955823293173	94.59077603802\\
0.742971887550201	94.6515273231089\\
0.746987951807229	94.5701815946085\\
0.751004016064257	94.5162092979799\\
0.755020080321285	94.5739708094988\\
0.759036144578313	94.5604237368535\\
0.763052208835341	94.5905007550479\\
0.767068273092369	94.5852844581898\\
0.771084337349398	94.5854427005895\\
0.775100401606426	94.6182215835899\\
0.779116465863454	94.6359677029288\\
0.783132530120482	94.574350504815\\
0.78714859437751	94.5774833224515\\
0.791164658634538	94.5578518583702\\
0.795180722891566	94.6540916767564\\
0.799196787148594	94.673454677498\\
0.803212851405622	94.673814636918\\
0.807228915662651	94.6556666902912\\
0.811244979919679	94.6542228486782\\
0.815261044176707	94.6142782074654\\
0.819277108433735	94.6079857729952\\
0.823293172690763	94.5384589775204\\
0.827309236947791	94.4871423042313\\
0.831325301204819	94.4382558671783\\
0.835341365461847	94.409527029668\\
0.839357429718876	94.3047167670281\\
0.843373493975904	94.2207335840999\\
0.847389558232932	94.2035792626444\\
0.85140562248996	94.155509551312\\
0.855421686746988	94.1919298542735\\
0.859437751004016	94.1874614549477\\
0.863453815261044	94.1466449023877\\
0.867469879518072	94.0714271563314\\
0.8714859437751	93.9922901608719\\
0.875502008032129	93.8749275215365\\
0.879518072289157	93.7745740701789\\
0.883534136546185	93.7351474000772\\
0.887550200803213	93.784970447053\\
0.891566265060241	93.680188689901\\
0.895582329317269	93.5666564391593\\
0.899598393574297	93.4505349784057\\
0.903614457831325	93.3751209868946\\
0.907630522088353	93.2779005111825\\
0.911646586345382	93.1653550049836\\
0.91566265060241	93.0879387148341\\
0.919678714859438	92.9377195377888\\
0.923694779116466	92.8420246945337\\
0.927710843373494	92.6649868815485\\
0.931726907630522	92.4959296032481\\
0.93574297188755	92.2810480996709\\
0.939759036144578	92.0734978361848\\
0.943775100401606	91.8289513235225\\
0.947791164658635	91.538357564378\\
0.951807228915663	91.3055141361643\\
0.955823293172691	91.0865780310987\\
0.959839357429719	90.7312458252245\\
0.963855421686747	90.4163746029302\\
0.967871485943775	90.0179155694131\\
0.971887550200803	89.371033901876\\
0.975903614457831	88.8071736482123\\
0.979919678714859	88.1480952644483\\
0.983935742971888	87.1276282183163\\
0.987951807228916	85.6326934722999\\
0.991967871485944	84.1635537312715\\
0.995983935742972	81.0200166701793\\
1	59.1166075690233\\
};
\addlegendentry{ISAC dual-mode total}


 \addplot [
            color=mycolor4, 
            line width=1.5pt,
            mark=square, 
            mark options={solid, mycolor4},
            mark repeat=8 % Places a marker every 2 points
        ] 
  table[row sep=crcr]{%
0	45.8188\\
0.00401606425702811	45.8188\\
0.00803212851405622	45.8188\\
0.0120481927710843	45.8188\\
0.0160642570281124	45.8188\\
0.0200803212851406	45.8188\\
0.0240963855421687	45.8188\\
0.0281124497991968	45.8188\\
0.0321285140562249	45.8188\\
0.036144578313253	45.8188\\
0.0401606425702811	45.8188\\
0.0441767068273092	45.8188\\
0.0481927710843374	45.8188\\
0.0522088353413655	45.8188\\
0.0562248995983936	45.8188\\
0.0602409638554217	45.8188\\
0.0642570281124498	45.8188\\
0.0682730923694779	45.8188\\
0.072289156626506	45.8188\\
0.0763052208835341	45.8188\\
0.0803212851405622	45.8188\\
0.0843373493975904	45.8188\\
0.0883534136546185	45.8188\\
0.0923694779116466	45.8188\\
0.0963855421686747	45.8188\\
0.100401606425703	45.8188\\
0.104417670682731	45.8188\\
0.108433734939759	45.8188\\
0.112449799196787	45.8188\\
0.116465863453815	45.8188\\
0.120481927710843	45.8188\\
0.124497991967871	45.8188\\
0.1285140562249	45.8188\\
0.132530120481928	45.8188\\
0.136546184738956	45.8188\\
0.140562248995984	45.8188\\
0.144578313253012	45.8188\\
0.14859437751004	45.8188\\
0.152610441767068	45.8188\\
0.156626506024096	45.8188\\
0.160642570281124	45.8188\\
0.164658634538153	45.8188\\
0.168674698795181	45.8188\\
0.172690763052209	45.8188\\
0.176706827309237	45.8188\\
0.180722891566265	45.8188\\
0.184738955823293	45.8188\\
0.188755020080321	45.8188\\
0.192771084337349	45.8188\\
0.196787148594378	45.8188\\
0.200803212851406	45.8188\\
0.204819277108434	45.8188\\
0.208835341365462	45.8188\\
0.21285140562249	45.8188\\
0.216867469879518	45.8188\\
0.220883534136546	45.8188\\
0.224899598393574	45.8188\\
0.228915662650602	45.8188\\
0.232931726907631	45.8188\\
0.236947791164659	45.8188\\
0.240963855421687	45.8188\\
0.244979919678715	45.8188\\
0.248995983935743	45.8188\\
0.253012048192771	45.8188\\
0.257028112449799	45.8188\\
0.261044176706827	45.8188\\
0.265060240963855	45.8188\\
0.269076305220884	45.8188\\
0.273092369477912	45.8188\\
0.27710843373494	45.8188\\
0.281124497991968	45.8188\\
0.285140562248996	45.8188\\
0.289156626506024	45.8188\\
0.293172690763052	45.8188\\
0.29718875502008	45.8188\\
0.301204819277108	45.8188\\
0.305220883534137	45.8188\\
0.309236947791165	45.8188\\
0.313253012048193	45.8188\\
0.317269076305221	45.8188\\
0.321285140562249	45.8188\\
0.325301204819277	45.8188\\
0.329317269076305	45.8188\\
0.333333333333333	45.8188\\
0.337349397590361	45.8188\\
0.34136546184739	45.8188\\
0.345381526104418	45.8188\\
0.349397590361446	45.8188\\
0.353413654618474	45.8188\\
0.357429718875502	45.8188\\
0.36144578313253	45.8188\\
0.365461847389558	45.8188\\
0.369477911646586	45.8188\\
0.373493975903614	45.8188\\
0.377510040160643	45.8188\\
0.381526104417671	45.8188\\
0.385542168674699	45.8188\\
0.389558232931727	45.8188\\
0.393574297188755	45.8188\\
0.397590361445783	45.8188\\
0.401606425702811	45.8188\\
0.405622489959839	45.8188\\
0.409638554216867	45.8188\\
0.413654618473896	45.8188\\
0.417670682730924	45.8188\\
0.421686746987952	45.8188\\
0.42570281124498	45.8188\\
0.429718875502008	45.8188\\
0.433734939759036	45.8188\\
0.437751004016064	45.8188\\
0.441767068273092	45.8188\\
0.44578313253012	45.8188\\
0.449799196787149	45.8188\\
0.453815261044177	45.8188\\
0.457831325301205	45.8188\\
0.461847389558233	45.8188\\
0.465863453815261	45.8188\\
0.469879518072289	45.8188\\
0.473895582329317	45.8188\\
0.477911646586345	45.8188\\
0.481927710843373	45.8188\\
0.485943775100402	45.8188\\
0.48995983935743	45.8188\\
0.493975903614458	45.8188\\
0.497991967871486	45.8188\\
0.502008032128514	45.8188\\
0.506024096385542	45.8188\\
0.51004016064257	45.8188\\
0.514056224899598	45.8188\\
0.518072289156627	45.8188\\
0.522088353413655	45.8188\\
0.526104417670683	45.8188\\
0.530120481927711	45.8188\\
0.534136546184739	45.8188\\
0.538152610441767	45.8188\\
0.542168674698795	45.8188\\
0.546184738955823	45.8188\\
0.550200803212851	45.8188\\
0.55421686746988	45.8188\\
0.558232931726908	45.8188\\
0.562248995983936	45.8188\\
0.566265060240964	45.8188\\
0.570281124497992	45.8188\\
0.57429718875502	45.8188\\
0.578313253012048	45.8188\\
0.582329317269076	45.8188\\
0.586345381526104	45.8188\\
0.590361445783133	45.8188\\
0.594377510040161	45.8188\\
0.598393574297189	45.8188\\
0.602409638554217	45.8188\\
0.606425702811245	45.8188\\
0.610441767068273	45.8188\\
0.614457831325301	45.8188\\
0.618473895582329	45.8188\\
0.622489959839357	45.8188\\
0.626506024096386	45.8188\\
0.630522088353414	45.8188\\
0.634538152610442	45.8188\\
0.63855421686747	45.8188\\
0.642570281124498	45.8188\\
0.646586345381526	45.8188\\
0.650602409638554	45.8188\\
0.654618473895582	45.8188\\
0.65863453815261	45.8188\\
0.662650602409639	45.8188\\
0.666666666666667	45.8188\\
0.670682730923695	45.8188\\
0.674698795180723	45.8188\\
0.678714859437751	45.8188\\
0.682730923694779	45.8188\\
0.686746987951807	45.8188\\
0.690763052208835	45.8188\\
0.694779116465863	45.8188\\
0.698795180722892	45.8188\\
0.70281124497992	45.8188\\
0.706827309236948	45.8188\\
0.710843373493976	45.8188\\
0.714859437751004	45.8188\\
0.718875502008032	45.8188\\
0.72289156626506	45.8188\\
0.726907630522088	45.8188\\
0.730923694779116	45.8188\\
0.734939759036145	45.8188\\
0.738955823293173	45.8188\\
0.742971887550201	45.8188\\
0.746987951807229	45.8188\\
0.751004016064257	45.8188\\
0.755020080321285	45.8188\\
0.759036144578313	45.8188\\
0.763052208835341	45.8188\\
0.767068273092369	45.8188\\
0.771084337349398	45.8188\\
0.775100401606426	45.8188\\
0.779116465863454	45.8188\\
0.783132530120482	45.8188\\
0.78714859437751	45.8188\\
0.791164658634538	45.8188\\
0.795180722891566	45.8188\\
0.799196787148594	45.8188\\
0.803212851405622	45.8188\\
0.807228915662651	45.8188\\
0.811244979919679	45.8188\\
0.815261044176707	45.8188\\
0.819277108433735	45.8188\\
0.823293172690763	45.8188\\
0.827309236947791	45.8188\\
0.831325301204819	45.8188\\
0.835341365461847	45.8188\\
0.839357429718876	45.8188\\
0.843373493975904	45.8188\\
0.847389558232932	45.8188\\
0.85140562248996	45.8188\\
0.855421686746988	45.8188\\
0.859437751004016	45.8188\\
0.863453815261044	45.8188\\
0.867469879518072	45.8188\\
0.8714859437751	45.8188\\
0.875502008032129	45.8188\\
0.879518072289157	45.8188\\
0.883534136546185	45.8188\\
0.887550200803213	45.8188\\
0.891566265060241	45.8188\\
0.895582329317269	45.8188\\
0.899598393574297	45.8188\\
0.903614457831325	45.8188\\
0.907630522088353	45.8188\\
0.911646586345382	45.8188\\
0.91566265060241	45.8188\\
0.919678714859438	45.8188\\
0.923694779116466	45.8188\\
0.927710843373494	45.8188\\
0.931726907630522	45.8188\\
0.93574297188755	45.8188\\
0.939759036144578	45.8188\\
0.943775100401606	45.8188\\
0.947791164658635	45.8188\\
0.951807228915663	45.8188\\
0.955823293172691	45.8188\\
0.959839357429719	45.8188\\
0.963855421686747	45.8188\\
0.967871485943775	45.8188\\
0.971887550200803	45.8188\\
0.975903614457831	45.8188\\
0.979919678714859	45.8188\\
0.983935742971888	45.8188\\
0.987951807228916	45.8188\\
0.991967871485944	45.8188\\
0.995983935742972	45.8188\\
1	45.8188\\
};
\addlegendentry{Communication-only network}

 \addplot [
            color=mycolor5, 
            line width=1.5pt,
            mark=x, 
            mark options={solid, mycolor5},
            mark repeat=8 % Places a marker every 2 points
        ]
  table[row sep=crcr]{%
0	36.8349\\
0.00401606425702811	36.8349\\
0.00803212851405622	36.8349\\
0.0120481927710843	36.8349\\
0.0160642570281124	36.8349\\
0.0200803212851406	36.8349\\
0.0240963855421687	36.8349\\
0.0281124497991968	36.8349\\
0.0321285140562249	36.8349\\
0.036144578313253	36.8349\\
0.0401606425702811	36.8349\\
0.0441767068273092	36.8349\\
0.0481927710843374	36.8349\\
0.0522088353413655	36.8349\\
0.0562248995983936	36.8349\\
0.0602409638554217	36.8349\\
0.0642570281124498	36.8349\\
0.0682730923694779	36.8349\\
0.072289156626506	36.8349\\
0.0763052208835341	36.8349\\
0.0803212851405622	36.8349\\
0.0843373493975904	36.8349\\
0.0883534136546185	36.8349\\
0.0923694779116466	36.8349\\
0.0963855421686747	36.8349\\
0.100401606425703	36.8349\\
0.104417670682731	36.8349\\
0.108433734939759	36.8349\\
0.112449799196787	36.8349\\
0.116465863453815	36.8349\\
0.120481927710843	36.8349\\
0.124497991967871	36.8349\\
0.1285140562249	36.8349\\
0.132530120481928	36.8349\\
0.136546184738956	36.8349\\
0.140562248995984	36.8349\\
0.144578313253012	36.8349\\
0.14859437751004	36.8349\\
0.152610441767068	36.8349\\
0.156626506024096	36.8349\\
0.160642570281124	36.8349\\
0.164658634538153	36.8349\\
0.168674698795181	36.8349\\
0.172690763052209	36.8349\\
0.176706827309237	36.8349\\
0.180722891566265	36.8349\\
0.184738955823293	36.8349\\
0.188755020080321	36.8349\\
0.192771084337349	36.8349\\
0.196787148594378	36.8349\\
0.200803212851406	36.8349\\
0.204819277108434	36.8349\\
0.208835341365462	36.8349\\
0.21285140562249	36.8349\\
0.216867469879518	36.8349\\
0.220883534136546	36.8349\\
0.224899598393574	36.8349\\
0.228915662650602	36.8349\\
0.232931726907631	36.8349\\
0.236947791164659	36.8349\\
0.240963855421687	36.8349\\
0.244979919678715	36.8349\\
0.248995983935743	36.8349\\
0.253012048192771	36.8349\\
0.257028112449799	36.8349\\
0.261044176706827	36.8349\\
0.265060240963855	36.8349\\
0.269076305220884	36.8349\\
0.273092369477912	36.8349\\
0.27710843373494	36.8349\\
0.281124497991968	36.8349\\
0.285140562248996	36.8349\\
0.289156626506024	36.8349\\
0.293172690763052	36.8349\\
0.29718875502008	36.8349\\
0.301204819277108	36.8349\\
0.305220883534137	36.8349\\
0.309236947791165	36.8349\\
0.313253012048193	36.8349\\
0.317269076305221	36.8349\\
0.321285140562249	36.8349\\
0.325301204819277	36.8349\\
0.329317269076305	36.8349\\
0.333333333333333	36.8349\\
0.337349397590361	36.8349\\
0.34136546184739	36.8349\\
0.345381526104418	36.8349\\
0.349397590361446	36.8349\\
0.353413654618474	36.8349\\
0.357429718875502	36.8349\\
0.36144578313253	36.8349\\
0.365461847389558	36.8349\\
0.369477911646586	36.8349\\
0.373493975903614	36.8349\\
0.377510040160643	36.8349\\
0.381526104417671	36.8349\\
0.385542168674699	36.8349\\
0.389558232931727	36.8349\\
0.393574297188755	36.8349\\
0.397590361445783	36.8349\\
0.401606425702811	36.8349\\
0.405622489959839	36.8349\\
0.409638554216867	36.8349\\
0.413654618473896	36.8349\\
0.417670682730924	36.8349\\
0.421686746987952	36.8349\\
0.42570281124498	36.8349\\
0.429718875502008	36.8349\\
0.433734939759036	36.8349\\
0.437751004016064	36.8349\\
0.441767068273092	36.8349\\
0.44578313253012	36.8349\\
0.449799196787149	36.8349\\
0.453815261044177	36.8349\\
0.457831325301205	36.8349\\
0.461847389558233	36.8349\\
0.465863453815261	36.8349\\
0.469879518072289	36.8349\\
0.473895582329317	36.8349\\
0.477911646586345	36.8349\\
0.481927710843373	36.8349\\
0.485943775100402	36.8349\\
0.48995983935743	36.8349\\
0.493975903614458	36.8349\\
0.497991967871486	36.8349\\
0.502008032128514	36.8349\\
0.506024096385542	36.8349\\
0.51004016064257	36.8349\\
0.514056224899598	36.8349\\
0.518072289156627	36.8349\\
0.522088353413655	36.8349\\
0.526104417670683	36.8349\\
0.530120481927711	36.8349\\
0.534136546184739	36.8349\\
0.538152610441767	36.8349\\
0.542168674698795	36.8349\\
0.546184738955823	36.8349\\
0.550200803212851	36.8349\\
0.55421686746988	36.8349\\
0.558232931726908	36.8349\\
0.562248995983936	36.8349\\
0.566265060240964	36.8349\\
0.570281124497992	36.8349\\
0.57429718875502	36.8349\\
0.578313253012048	36.8349\\
0.582329317269076	36.8349\\
0.586345381526104	36.8349\\
0.590361445783133	36.8349\\
0.594377510040161	36.8349\\
0.598393574297189	36.8349\\
0.602409638554217	36.8349\\
0.606425702811245	36.8349\\
0.610441767068273	36.8349\\
0.614457831325301	36.8349\\
0.618473895582329	36.8349\\
0.622489959839357	36.8349\\
0.626506024096386	36.8349\\
0.630522088353414	36.8349\\
0.634538152610442	36.8349\\
0.63855421686747	36.8349\\
0.642570281124498	36.8349\\
0.646586345381526	36.8349\\
0.650602409638554	36.8349\\
0.654618473895582	36.8349\\
0.65863453815261	36.8349\\
0.662650602409639	36.8349\\
0.666666666666667	36.8349\\
0.670682730923695	36.8349\\
0.674698795180723	36.8349\\
0.678714859437751	36.8349\\
0.682730923694779	36.8349\\
0.686746987951807	36.8349\\
0.690763052208835	36.8349\\
0.694779116465863	36.8349\\
0.698795180722892	36.8349\\
0.70281124497992	36.8349\\
0.706827309236948	36.8349\\
0.710843373493976	36.8349\\
0.714859437751004	36.8349\\
0.718875502008032	36.8349\\
0.72289156626506	36.8349\\
0.726907630522088	36.8349\\
0.730923694779116	36.8349\\
0.734939759036145	36.8349\\
0.738955823293173	36.8349\\
0.742971887550201	36.8349\\
0.746987951807229	36.8349\\
0.751004016064257	36.8349\\
0.755020080321285	36.8349\\
0.759036144578313	36.8349\\
0.763052208835341	36.8349\\
0.767068273092369	36.8349\\
0.771084337349398	36.8349\\
0.775100401606426	36.8349\\
0.779116465863454	36.8349\\
0.783132530120482	36.8349\\
0.78714859437751	36.8349\\
0.791164658634538	36.8349\\
0.795180722891566	36.8349\\
0.799196787148594	36.8349\\
0.803212851405622	36.8349\\
0.807228915662651	36.8349\\
0.811244979919679	36.8349\\
0.815261044176707	36.8349\\
0.819277108433735	36.8349\\
0.823293172690763	36.8349\\
0.827309236947791	36.8349\\
0.831325301204819	36.8349\\
0.835341365461847	36.8349\\
0.839357429718876	36.8349\\
0.843373493975904	36.8349\\
0.847389558232932	36.8349\\
0.85140562248996	36.8349\\
0.855421686746988	36.8349\\
0.859437751004016	36.8349\\
0.863453815261044	36.8349\\
0.867469879518072	36.8349\\
0.8714859437751	36.8349\\
0.875502008032129	36.8349\\
0.879518072289157	36.8349\\
0.883534136546185	36.8349\\
0.887550200803213	36.8349\\
0.891566265060241	36.8349\\
0.895582329317269	36.8349\\
0.899598393574297	36.8349\\
0.903614457831325	36.8349\\
0.907630522088353	36.8349\\
0.911646586345382	36.8349\\
0.91566265060241	36.8349\\
0.919678714859438	36.8349\\
0.923694779116466	36.8349\\
0.927710843373494	36.8349\\
0.931726907630522	36.8349\\
0.93574297188755	36.8349\\
0.939759036144578	36.8349\\
0.943775100401606	36.8349\\
0.947791164658635	36.8349\\
0.951807228915663	36.8349\\
0.955823293172691	36.8349\\
0.959839357429719	36.8349\\
0.963855421686747	36.8349\\
0.967871485943775	36.8349\\
0.971887550200803	36.8349\\
0.975903614457831	36.8349\\
0.979919678714859	36.8349\\
0.983935742971888	36.8349\\
0.987951807228916	36.8349\\
0.991967871485944	36.8349\\
0.995983935742972	36.8349\\
1	36.8349\\
};
\addlegendentry{Time sharing system total}

 \addplot [
            color=mycolor6, 
            line width=1.5pt,
            mark=+, 
            mark options={solid, mycolor6},
            mark repeat=8 % Places a marker every 2 points
        ]
  table[row sep=crcr]{%
0	46.7295550299928\\
0.00401606425702811	50.7486429233465\\
0.00803212851405622	53.9306102817496\\
0.0120481927710843	56.275457105202\\
0.0160642570281124	57.1427548222752\\
0.0200803212851406	58.0604748615407\\
0.0240963855421687	58.8938813127992\\
0.0281124497991968	59.5893776497928\\
0.0321285140562249	60.1891394627341\\
0.036144578313253	60.6717868646158\\
0.0401606425702811	61.1336286480997\\
0.0441767068273092	61.5470468054492\\
0.0481927710843374	61.9843381750386\\
0.0522088353413655	62.4261766053058\\
0.0562248995983936	62.7502050046755\\
0.0602409638554217	63.1745337899726\\
0.0642570281124498	63.5186078472072\\
0.0682730923694779	63.8795179245666\\
0.072289156626506	64.1585659455244\\
0.0763052208835341	64.4143323205783\\
0.0803212851405622	64.6408272990625\\
0.0843373493975904	64.8370685936356\\
0.0883534136546185	65.0877071957937\\
0.0923694779116466	65.2952908512846\\
0.0963855421686747	65.6505810761494\\
0.100401606425703	65.8273870483051\\
0.104417670682731	66.0558297580768\\
0.108433734939759	66.2180187593335\\
0.112449799196787	66.4279423363441\\
0.116465863453815	66.6042621275811\\
0.120481927710843	66.7612345486021\\
0.124497991967871	66.9479368235075\\
0.1285140562249	67.1278495688458\\
0.132530120481928	67.3028763689917\\
0.136546184738956	67.4673871071528\\
0.140562248995984	67.5733154377829\\
0.144578313253012	67.6438401176246\\
0.14859437751004	67.8084227020345\\
0.152610441767068	67.9246980318329\\
0.156626506024096	68.0610941822235\\
0.160642570281124	68.2228834366645\\
0.164658634538153	68.3449935140238\\
0.168674698795181	68.5041229968606\\
0.172690763052209	68.6424548896719\\
0.176706827309237	68.7434425839529\\
0.180722891566265	68.8063562160115\\
0.184738955823293	68.8433268190363\\
0.188755020080321	69.023791653966\\
0.192771084337349	69.1673292464686\\
0.196787148594378	69.3214521503755\\
0.200803212851406	69.3895872887668\\
0.204819277108434	69.4468032278027\\
0.208835341365462	69.5088232107999\\
0.21285140562249	69.5744706534551\\
0.216867469879518	69.6432184460004\\
0.220883534136546	69.716285529774\\
0.224899598393574	69.7937966227121\\
0.228915662650602	69.8754818948145\\
0.232931726907631	69.972401747763\\
0.236947791164659	70.0755471167048\\
0.240963855421687	70.1736673974152\\
0.244979919678715	70.2650892208545\\
0.248995983935743	70.3506308148991\\
0.253012048192771	70.4241237209805\\
0.257028112449799	70.5033681424213\\
0.261044176706827	70.5753078724906\\
0.265060240963855	70.6376473453881\\
0.269076305220884	70.6897877766909\\
0.273092369477912	70.7408569429071\\
0.27710843373494	70.79025086963\\
0.281124497991968	70.8342951388735\\
0.285140562248996	70.8842263906883\\
0.289156626506024	70.9357762650303\\
0.293172690763052	70.992911102409\\
0.29718875502008	71.0586651911854\\
0.301204819277108	71.1252760138038\\
0.305220883534137	71.1899534008868\\
0.309236947791165	71.2614528982113\\
0.313253012048193	71.3350169515299\\
0.317269076305221	71.3978536145004\\
0.321285140562249	71.4539720486341\\
0.325301204819277	71.5065311253431\\
0.329317269076305	71.5546842725846\\
0.333333333333333	71.5981205317375\\
0.337349397590361	71.6461938609062\\
0.34136546184739	71.6918822528517\\
0.345381526104418	71.7426616099331\\
0.349397590361446	71.804750326076\\
0.353413654618474	71.8678376353133\\
0.357429718875502	71.9319875024671\\
0.36144578313253	71.9916347016246\\
0.365461847389558	72.0373138438642\\
0.369477911646586	72.0584364891829\\
0.373493975903614	72.0670284327735\\
0.377510040160643	72.0736585844825\\
0.381526104417671	72.0916376554934\\
0.385542168674699	72.1195857332865\\
0.389558232931727	72.1554382113591\\
0.393574297188755	72.1951774127732\\
0.397590361445783	72.2286375421011\\
0.401606425702811	72.272462011589\\
0.405622489959839	72.3227251743622\\
0.409638554216867	72.3854586256027\\
0.413654618473896	72.4586348896077\\
0.417670682730924	72.5207060053509\\
0.421686746987952	72.5601513514974\\
0.42570281124498	72.5794962769808\\
0.429718875502008	72.5814746347305\\
0.433734939759036	72.5819371394969\\
0.437751004016064	72.5863212315153\\
0.441767068273092	72.5889858408646\\
0.44578313253012	72.5882867468777\\
0.449799196787149	72.5818429511448\\
0.453815261044177	72.5601406496612\\
0.457831325301205	72.5497664495897\\
0.461847389558233	72.5614230430042\\
0.465863453815261	72.5921584360572\\
0.469879518072289	72.6407876401258\\
0.473895582329317	72.6882751208596\\
0.477911646586345	72.7205960621422\\
0.481927710843373	72.7346485244032\\
0.485943775100402	72.7409143970269\\
0.48995983935743	72.7484864733595\\
0.493975903614458	72.7680518420958\\
0.497991967871486	72.796533007698\\
0.502008032128514	72.8288349135038\\
0.506024096385542	72.8566475194363\\
0.51004016064257	72.8813329181048\\
0.514056224899598	72.9143287162566\\
0.518072289156627	72.9510505467242\\
0.522088353413655	72.9846986787828\\
0.526104417670683	73.0147249316399\\
0.530120481927711	73.0191844191127\\
0.534136546184739	73.0154764583505\\
0.538152610441767	73.0165497168675\\
0.542168674698795	73.029643453654\\
0.546184738955823	73.0360596366297\\
0.550200803212851	73.0524319048469\\
0.55421686746988	73.0604766841523\\
0.558232931726908	73.0743307932571\\
0.562248995983936	73.0998018556057\\
0.566265060240964	73.1335734572802\\
0.570281124497992	73.1578656157492\\
0.57429718875502	73.178110197495\\
0.578313253012048	73.1874249950964\\
0.582329317269076	73.1762761000477\\
0.586345381526104	73.1526653885851\\
0.590361445783133	73.1345470208369\\
0.594377510040161	73.1171003861047\\
0.598393574297189	73.0898479184697\\
0.602409638554217	73.056203052314\\
0.606425702811245	73.0255118734532\\
0.610441767068273	73.0008186233602\\
0.614457831325301	73.0033430835064\\
0.618473895582329	73.0229978488353\\
0.622489959839357	73.0411822585242\\
0.626506024096386	73.0527055928004\\
0.630522088353414	73.0658361521465\\
0.634538152610442	73.0707120036377\\
0.63855421686747	73.0749378149124\\
0.642570281124498	73.077028720596\\
0.646586345381526	73.0682051248871\\
0.650602409638554	73.0575307045175\\
0.654618473895582	73.0380699197914\\
0.65863453815261	73.0233040523408\\
0.662650602409639	73.0110929995271\\
0.666666666666667	73.0084584414126\\
0.670682730923695	73.0089984323545\\
0.674698795180723	73.0114986088688\\
0.678714859437751	73.0064820834224\\
0.682730923694779	73.0010395978528\\
0.686746987951807	73.0046097575307\\
0.690763052208835	73.0113910826564\\
0.694779116465863	73.0219907097688\\
0.698795180722892	73.0192437531585\\
0.70281124497992	73.0026376658448\\
0.706827309236948	72.9797345722051\\
0.710843373493976	72.9433584559309\\
0.714859437751004	72.9044845299089\\
0.718875502008032	72.8724268936066\\
0.72289156626506	72.8383611285719\\
0.726907630522088	72.7982356890941\\
0.730923694779116	72.7524803633991\\
0.734939759036145	72.6991494706008\\
0.738955823293173	72.6479863316891\\
0.742971887550201	72.619476754116\\
0.746987951807229	72.5967465021836\\
0.751004016064257	72.5823133835181\\
0.755020080321285	72.5698111098271\\
0.759036144578313	72.5601164750693\\
0.763052208835341	72.5445753449845\\
0.767068273092369	72.5208257389267\\
0.771084337349398	72.4897322254436\\
0.775100401606426	72.4610469407759\\
0.779116465863454	72.4225853341751\\
0.783132530120482	72.3772328662753\\
0.78714859437751	72.3265197691873\\
0.791164658634538	72.276854331973\\
0.795180722891566	72.237643354302\\
0.799196787148594	72.2183504630468\\
0.803212851405622	72.1934036576401\\
0.807228915662651	72.1629979523196\\
0.811244979919679	72.1269900164182\\
0.815261044176707	72.07494737875\\
0.819277108433735	71.9904639276485\\
0.823293172690763	71.8865824174317\\
0.827309236947791	71.7750510337541\\
0.831325301204819	71.6735855849895\\
0.835341365461847	71.5909149290752\\
0.839357429718876	71.5230876774552\\
0.843373493975904	71.4608124025267\\
0.847389558232932	71.4093956865704\\
0.85140562248996	71.3668059414753\\
0.855421686746988	71.293389689368\\
0.859437751004016	71.2163180386106\\
0.863453815261044	71.1369660240586\\
0.867469879518072	71.0541311765542\\
0.8714859437751	70.9669083105733\\
0.875502008032129	70.8767187399544\\
0.879518072289157	70.7820987938239\\
0.883534136546185	70.6239817587482\\
0.887550200803213	70.532510951495\\
0.891566265060241	70.4325448929573\\
0.895582329317269	70.3250056222638\\
0.899598393574297	70.0975294248571\\
0.903614457831325	69.9444087538782\\
0.907630522088353	69.854558182469\\
0.911646586345382	69.6444220907749\\
0.91566265060241	69.5008614099208\\
0.919678714859438	69.3193868511377\\
0.923694779116466	69.1982023722583\\
0.927710843373494	68.9671475529265\\
0.931726907630522	68.89819721716\\
0.93574297188755	68.6706713550164\\
0.939759036144578	68.4396488142556\\
0.943775100401606	68.0974443768165\\
0.947791164658635	67.821534726187\\
0.951807228915663	67.5115789277027\\
0.955823293172691	67.2214180131123\\
0.959839357429719	66.8532234109872\\
0.963855421686747	66.471280835613\\
0.967871485943775	66.0149420727747\\
0.971887550200803	65.443284707476\\
0.975903614457831	64.8028324399145\\
0.979919678714859	64.0142408280158\\
0.983935742971888	62.9015970597764\\
0.987951807228916	61.3233317698199\\
0.991967871485944	59.7609173701149\\
0.995983935742972	56.5251207902601\\
1	34.7824\\
};
\addlegendentry{ISAC without FD (multistatic only) total  }
   \definecolor{darkgreen}{RGB}{0, 100, 0} % A specific dark green shade
% Add a red marker for the maximum point



% Define the points
    \coordinate (A) at (0.70281124497992, 46.8188);
    \coordinate (B) at (0.70281124497992, 94.1423035860089);

 \draw[<->, line width=0.9mm, magenta] (A) -- (B);



   
    \coordinate (M) at (0.608, 73.98055179300445);

    \node[above, text=magenta] at (M) {\textbf{The Integration \hspace{1.5pt}  Gain}};


 
    \coordinate (A2) at (0.566265060240964, 47.0188);
    \coordinate (B2) at (0.566265060240964, 73.0102883444392);

  \draw[<->, line width=0.9mm, magenta] (A2) -- (B2);





\end{axis}

\begin{axis}[%
width=5.833in,
height=4.375in,
at={(0in,0in)},
scale only axis,
xmin=0,
xmax=1,
ymin=0,
ymax=1,
axis line style={draw=none},
ticks=none,
axis x line*=bottom,
axis y line*=left
]
\end{axis}
\end{tikzpicture}%
}%
\caption{  (a)  The effect of imperfect SIC  for a target at a range = 20 m. (b) The networked-sensing performance without the monostatic sensing in case  FD is not available. (c)  The effect of energy allocation between sensing and communication across the ISAC signal.}
\end{figure*}







Fig. \ref{imper_sic} illustrates the impact of imperfect SIC on sensing coverage probability. Efficient SIC implies negligible residual power, which enables high coverage primarily through monostatic sensing. However, as $\zeta$ exceeds a certain threshold, monostatic performance sharply declines to zero, shifting the system to multistatic-dominant sensing. The figure highlights the distinct challenges of FD for sensing compared to communication. While a residual power fraction as small as $10^{-9}$ is acceptable for FD communication \cite{ali2016modeling}, it can severely impair monostatic sensing due to the round-trip path loss unique to sensing. Nonetheless, networked sensing can offer reasonable performance through multistatic contributions, although this demands increased collaboration among more BSs, as detailed in the next figure. 





Fig. \ref{no_fd} examines average sensing coverage probability without FD, where monostatic sensing is infeasible, relying instead on multistatic sensing. The figure shows that increasing the number of cooperating BSs improves performance by exploiting spatial diversity, as bistatic RCS is captured from multiple angles. The performance also improves with higher network density, driven by the same factors discussed in Fig.~\ref{bs_avg_tot}.  For instance, when six BSs cooperate in receiving sensing echoes instead of a single BS Rx, the sensing coverage probability increases by \textbf{108\%} at a BS density of 250 BSs/km$^\text{2}$ and by \textbf{120\%} at 500 BSs/km$^\text{2}$. This improvement is only possible through the cellular infrastructure’s networking capabilities and synchronized operation, enabling efficient coordination that standalone sensing systems cannot achieve. This seamless cooperation among BSs highlights the significant coordination gain in ISAC networks.

















Fig. \ref{energy_eff} analyzes the impact of power allocation on average information rates, assuming \(E_t = 0.2378\,\mu\text{Joule}\) (equivalent to 1 W average power per time slot). The results demonstrate the integration gain of the proposed ISAC system, where the total rate significantly exceeds the individual rates of sensing and communication. This gain is maximized by allocating more energy to the sensing pulse, which helps mitigate SI, compensate for round-trip path loss, and reduce direct interference.
In dense networks, increasing power has a limited effect on communication performance, as it amplifies both the signal and interference, whereas sensing benefits more directly. Additionally, Fig. \ref{energy_eff} compares the proposed system to a time-sharing approach where sensing and communication share equal time sequentially. Although this eliminates SI, the overall performance is very low, as each function is limited to half the time.


To assess the impact of incorporating sensing functionality on communication, the proposed system is contrasted to a communication-only network, where all resources are allocated to communication. The results show that ISAC maintains comparable communication performance while delivering substantial sensing gains. Notably, the total throughput increases by \textbf{106\%} compared to the communication-only network, highlighting the integration gain of the dual-mode ISAC system.
Moreover, Fig. \ref{energy_eff} plots the total ISAC rate without FD, relying on multistatic sensing from the nearest six BSs. Remarkably, the system still achieves a meaningful integration gain, with total throughput increasing by \textbf{60\%} compared to the communication-only network. This demonstrates the proposed ISAC framework’s efficiency even without FD technology. Notably, the optimal power to maximize gain in this scenario is lower than in the FD-enabled case, as there is no monostatic operation or residual SI power.






 


\section{Conclusion}\label{con_pp}




The paper studies the integration and coordination gains in large-scale mmWave ISAC networks using a dual-mode sensing framework that combines monostatic and multistatic approaches. System-level analysis revealed that overlooking sensing specific interference sources leads to overestimated performance, with interference impacting monostatic and bistatic sensing differently.
Analysis of the unified signal approach shows that BS density, beamwidth, and power allocation involve intricate design tradeoffs. Higher BS density improves sensing but communication is optimized at a certain BS density before declining, narrower beams enhance sensing but increase communication misalignment risks, and allocating more energy to sensing pulse than to the rest of the signal enhances total throughput. Additionally, the results confirm the resilience of communication performance despite the integrated sensing functionality, demonstrating ISAC's feasibility in dense networks.

The findings show that dual-mode cooperative sensing enhances sensing performance by leveraging spatial sensing diversity. Specifically, in scenarios with highly imperfect SIC or without FD, the system transitions to multistatic-dominant operation, maintaining reliable sensing. This performance is further enhanced through network densification and the addition of more cooperative BSs, highlighting the coordination gains in ISAC systems. Comparisons with communication-only and time-sharing systems demonstrate the superior integration gains of the proposed approach. Notably, even in multistatic mode alone, the system achieves a reasonable integration gain, demonstrating that FD is not a strict requirement for ISAC systems. Furthermore, the results highlight the impact of additional cooperative BSs on sensing performance, providing valuable insights for service providers to balance ISAC benefits with backhaul constraints.










\appendices
\section{Proof of Lemma 1 }
Using the definition of LT:
\small
\begin{equation}
\begin{aligned}[b]
\mathcal{L}_{I_{L_s}} (s)&=\mathbb{E}_{\bold\Phi_{L_s},h_{L,i}}\left[\exp\left(-sI_{L_s}\right)\right]\\
&=\mathbb{E}_{\bold\Phi_{L_s},h_{L,i}}\left[\exp\left(-s \sum\limits_{\substack{\text{BS}_i\in \bold\Phi_{L_s}}}   P_c h_{L,i}G_m^2 C_L r_i^{-\eta_L} \right)\right]\\
&=\mathbb{E}_{\bold\Phi_{L_s}}\left[\prod\limits_{\substack{\text{BS}_i\in \bold\Phi_{L_s}}}\mathbb{E}_{h_{L,i}}\left[\exp\left(- s P_c h_{L,i}G_m^2 C_L r_i^{-\eta_L}\right)\right]\right].\\
\end{aligned}
\end{equation}
\normalsize
From the Gamma distribution's moment-generating function:
\begin{equation}
\mathcal{L}_{I_{L_s}} (s)=\mathbb{E}_{\bold\Phi_{L_s}}\left[\prod\limits_{\substack{\text{BS}_i\in \bold\Phi_{L_s}}}\left(1+\frac{ s P_c G_m^2 C_L r_i^{-\eta_L}}{m_L}\right)^{-m_L}\right].
\end{equation}
Consider the closet interfering BS at a distance $R_d$, Using polar coordinates and the definition of probability generating functional (PGFL) in PPP:
\begin{equation}
\begin{aligned}
\mathcal{L}_{I_{L_s}} (s)&=\exp\left(-  \frac{2 \pi \lambda_{Bs}}{M^2}    \int_{ R_d}^{\infty}\bold p_{\bold{LOS}}\left(r\right) \right.\\
& \quad \quad \left. \left(1-\left( 1+\frac{s P_c G_m^2 C_L r^{-\eta_L}}{m_L}\right)^{-m_L}    \right) r\;dr  \right).
\end{aligned}
\end{equation}
For NLOS interference, the previous steps are repeated, replacing $\bold p_{\bold{LOS}}\left(r\right)$ by $\left(1-\bold p_{\bold{LOS}}\left(r\right)\right)$, $m_L$ by $m_N$, $C_L$ by $C_N$  and $\eta_L$ by $\eta_N$.
By combining both formulas and taking the expectations over $R_d$ with PDF defined in (\ref{ner_dis}), the lemma is proved.



\section{Proof of Lemma 2}
Since the intra-clutter contribution involves a random number of scatterers and random  RCS within the resolution cell \( A_{rm} \). Therefore, we take the expectation over both the random clutter scatterers and their RCS, then
using the PGFL of PPP:
\begin{equation}
\begin{aligned}
&\mathbb{E}_{cl, \sigma_{cm}} \left[ \exp\left( - \frac{\phi_s \sum_{cl \in \bold\Phi_{cl} \cap A_{rm}} \sigma_{cm}}{\sigma_{\text{av}_t}} \right) \right]\\
&=\mathbb{E}_{cl}\left[ \prod_{cl \in \bold\Phi_{cl} \cap A_{rm}} \mathbb{E}_{\sigma_{cm}}\left[ \exp\left( - \frac{\phi_s  \sigma_{cm}}{\sigma_{\text{av}_t}} \right)\right] \right]\\
&= \exp\left( - \lambda_{cl} \int_{A_{rm}} \mathbb{E}_{\sigma_{cm}}\left[ 1 - \exp\left( - \frac{\phi_s  \sigma_{cm}}{\sigma_{\text{av}_t}} \right) \right] dA_{rm} \right).
\end{aligned}
\end{equation}
By substituting the monostatic range resolution Cell Area:
\begin{equation}\label{fr_intra}
\begin{aligned}
&\mathbb{E}_{cl, \sigma_{cm}} \left[ \exp\left( - \frac{\phi_s \sum_{cl \in \bold\Phi_{cl} \cap A_{rm}} \sigma_{cm}}{\sigma_{\text{av}_t}} \right) \right] \\
&= \exp \left( - \lambda_{cl} \frac{c \theta_B R_1}{2 W_b} \mathbb{E}_{\sigma_{cm}} \left[ 1 - \exp\left( - \frac{\phi_s  \sigma_{cm}}{\sigma_{\text{av}_t}} \right) \right] \right).
\end{aligned}
\end{equation}
Since \( \sigma_{cm} \) follows a Weibull distribution given by (\ref{clu_wei}) with shape parameter \( k= 1 \) which is exponential, then the expectation simplifies to:
\begin{equation}\label{ff_intra}
\mathbb{E}_{\sigma_{cm}} \left[ 1 - \exp\left( - \frac{\phi_s  \sigma_{cm}}{\sigma_{\text{av}_t}} \right) \right] = \frac{\phi_s  \sigma_{\text{av}_{cl}}}{\sigma_{\text{av}_t} + \phi_r  \sigma_{\text{av}_{cl}}}.
\end{equation}
Thus, by substituting (\ref{ff_intra}) in (\ref{fr_intra}), the lemma is proved.


\section{Proof of Lemma 3}


\small
\begin{equation}
\begin{aligned}[b]
\!\!\!\!\mathcal{L}_{I_{IC1}} (s)&=\mathbb{E}_{\bold\Phi_{L_{IC}},\sigma_{tm}}\left[\exp\left(-sI_{IC1}\right)\right]\\
\!\!\!\!&=\mathbb{E}_{\bold\Phi_{L_{IC}},\sigma_{tm}}\left[\exp\left(-s \sum\limits_{\substack{\text{BS}_{n }\in \bold\Phi_{L_{IC}}\\ n \neq 1}} \cos{\left(\frac{\beta}{2}\right)} R_n^{-\eta_L} \sigma_{tm} \right)\right]\\
\!\!\!\!&=\mathbb{E}_{\bold\Phi_{L_{IC}}}\left[\prod\limits_{\substack{\text{BS}_{n }\in \bold\Phi_{L_{IC}}\\ n \neq 1}}\mathbb{E}_{\sigma_{tm}}\left[\exp\left(- s  \cos{\left(\frac{\beta}{2}\right)} R_n^{-\eta_L} \sigma_{tm}\right)\right]\right].\\
\end{aligned}
\end{equation}
\normalsize
Since \( \sigma_{tm} \) is an exponential PDF with mean \( \sigma_{\text{av}_t} \), then:
\begin{equation}
\mathcal{L}_{I_{IC1}}(s) = \mathbb{E}_{\bold\Phi_{LIC}} \left[ \prod_{\text{BS}_n \in \bold\Phi_{L_{IC}}, n \neq 1} \frac{1}{1 + s \cos\left( \frac{\beta}{2} \right) R_n^{-\eta_L} \sigma_{\text{av}_t}} \right].
\end{equation}
Next, we apply the PGFL of PPP:
\begin{equation}
\begin{aligned}
\mathcal{L}_{I_{IC1}}(s) &= \exp\bigg( - 2 \pi \lambda_{BS} \times \frac{1}{M} \int_{R_{IC}}^{\infty} \bold p_{\bold{LOS}}\left(r\right)\\
&\quad \times \bigg( 1 - \frac{1}{1 + s \cos\left( \frac{\beta}{2} \right) r^{-\eta_L} \sigma_{\text{av}_t}} \bigg) r \, dr \bigg).
\end{aligned}
\end{equation}
Similarly, $\mathcal{L}_{I_{IC2}} (s)$ can be expressed as:
\small
\begin{equation}
\begin{aligned}[b]\label{int_clu_inmd}
\mathcal{L}_{I_{IC2}} (s) 
&= \mathbb{E}_{\boldsymbol{\Phi}_{L_{IC}}} 
\Bigg[ \prod\limits_{\substack{\text{BS}_{n} \in \boldsymbol{\Phi}_{L_{IC}} \\ n \neq 1}} 
\mathbb{E}_{cl, \sigma_{cm}} 
\Bigg[ \exp\bigg( 
-s \cos{\left(\frac{\beta}{2}\right)} R_n^{-\eta_L} \\
&\quad \cdot 
\sum\limits_{cl \in \boldsymbol{\Phi}_{cl} \cap A_{rm}} 
\sigma_{cm} 
\bigg) \Bigg] \Bigg].
\end{aligned}
\end{equation}
\normalsize
As in the proof of Lemma 2, we apply the PGFL for the clutter scatterers and multiply by the area of the resolution cell:
\small
\begin{equation}
\begin{aligned}[b]
\!\!\!&\mathbb{E}_{cl, \sigma_{cm}} \left[ \exp \left( - s \cos \left( \frac{\beta}{2} \right) R_n^{-\eta_L} \sum_{cl \in \bold\Phi_{cl} \cap A_{rm}} \sigma_{cm} \right) \right]\\
\!\!\!&= \exp \left( - \lambda_{cl} \cdot \frac{c \theta_B R_1}{2 W_b} \cdot \mathbb{E}_{\sigma_{cm}} \left[ 1 - \exp \left( - s \cos \left( \frac{\beta}{2} \right) R_n^{-\eta_L} \sigma_{cm} \right) \right] \right).
\end{aligned}
\end{equation}
\normalsize
By using the  the Weibull  distribution when $k = 1$ and compute the expectation over $ \sigma_{cm} $:
\small
\begin{equation}
\!\!\!\!\mathbb{E}_{\sigma_{cm}} \left[ 1 - \exp \left( - s \cos \left( \frac{\beta}{2} \right) R_n^{-\eta_L} \sigma_{cm} \right) \right] = \frac{s \cos \left( \frac{\beta}{2} \right) R_n^{-\eta_L} \sigma_{\text{av}_{cl}}}{1 + s \cos \left( \frac{\beta}{2} \right) R_n^{-\eta_L} \sigma_{\text{av}_{cl}}}.
\end{equation}
\normalsize
We substitute back into (\ref{int_clu_inmd}), then apply the PGFL for the BSs:
\small
\begin{equation}
\begin{aligned}
&\mathcal{L}_{I_{IC2}}(s) = \exp \bigg( - \int_{R_{IC}}^\infty \lambda_{BS} \cdot \frac{1}{M} \cdot \boldsymbol{p}_{\text{LOS}}\big(r\big) \\
&\cdot \bigg( 1 - \exp \bigg( - \lambda_{cl} \cdot \frac{c \theta_B R_1}{2 W_b} \cdot \frac{s \cos \bigg( \frac{\beta}{2} \bigg) r^{-\eta_L} \sigma_{\text{av}_{cl}}}{1 + s \cos \bigg( \frac{\beta}{2} \bigg) r^{-\eta_L} \sigma_{\text{av}_{cl}}} \bigg) \bigg) 2 \pi r \, dr \bigg).
\end{aligned}
\end{equation}
\normalsize
Since $R_{IC}$ and $\beta$ are two RVs with PDFs given by (\ref{cond_dstt}) when $n=2$ and (\ref{bet_distrb}) respectively,
then, the lemma is proved by combining both formulas and taking the expectations.





\section{Proof of Theorem 1}


By simplifying (\ref{scnr_mono}) and substituting $\sigma_{tb}=\sigma_{tm} \cos{\left(\frac{\beta}{2}\right)}$ and $\sigma_{cb}=\sigma_{cm}\cos{\left(\frac{\beta}{2}\right)}$, and by letting 
$L_{IC1}= \sum\limits_{\substack{\text{BS}_{n }\in \bold\Phi_{L_{IC}}\\ n \neq 1}} \cos{\left(\frac{\beta}{2}\right)} R_n^{-\eta_L} \sigma_{tm}$  and  $L_{IC2}= \sum\limits_{\substack{\text{BS}_{n }\in \bold\Phi_{L_{IC}}\\ n \neq 1}} \cos{\left(\frac{\beta}{2}\right)} R_n^{-\eta_L} \sum\limits_{cl\in\bold\Phi_{cl}\cap A_{rm}} \sigma_{cm}$, then 
the monostatic sensing coverage probability can be expressed as:
\small
\begin{equation}
\begin{aligned}[b]
\mathcal{P}_M &= \mathbb{P} \left( \text{SINR}_M > \phi_s \right) \\
&= \mathbb{P} \bigg[\sigma_{tm} > \sum\limits_{cl \in \boldsymbol{\Phi}_{cl} \cap A_{rm}} \phi_s \, \sigma_{cm} + \phi_s \, R_1^{\eta_L} L_{IC1}\quad + \\
& \phi_s \, R_1^{\eta_L} L_{IC2} 
+ \frac{\phi_s \, (4\pi)^3 R_{1}^{2 \eta_L}}{P_s G_{m}^2 \lambda^2} 
\bigg( K_B T W_b + I_{L_r} + I_{N_r} + P_c \zeta \bigg) \bigg].
\end{aligned}
\end{equation}
\normalsize
 Using the Swerling I model, with PDF given by (\ref{sw1_rcs}), Then:
 \small
\begin{equation}
\begin{aligned}[b]
&\mathcal{P}_M = \exp\bigg( - \frac{1}{\sigma_{\text{av}_t}} \bigg[\sum_{cl \in \boldsymbol{\Phi}_{cl} \cap A_{rm}} \phi_s \, \sigma_{cm} 
+ \phi_s \, R_1^{\eta_L} L_{IC1} \\
& + \phi_s \, R_1^{\eta_L} L_{IC2} 
+ \frac{\phi_s \, (4\pi)^3 R_1^{2 \eta_L}}{P_s G_m^2 \lambda^2} 
\bigg(K_B T W_b + I_{L_r} + I_{N_r} + P_c \zeta \bigg) \bigg] \bigg).
\end{aligned}
\end{equation}
\normalsize
Finally, the expression can be decomposed into terms representing the LT of different interference sources.



\section{Proof of Theorem 2}

Considering the target in LoS conditions with the  Rx, then by simplifying (\ref{scnr_bi}) and substituting $\sigma_{tb}=\sigma_{tm} \cos{\left(\frac{\beta}{2}\right)}$ and $\sigma_{cb}=\sigma_{cm}\cos{\left(\frac{\beta}{2}\right)}$, and by letting 
$L_{IC1_b}= \sum\limits_{\substack{\text{BS}_{v}\in \bold\Phi_{L_{IC}}\\ v \neq 1, v \neq n}} \cos{\left(\frac{\beta_I}{2}\right)} R_v^{-\eta_L} \sigma_{tm}$  and  $L_{IC2_b}= \sum\limits_{\substack{\text{BS}_{v}\in \bold\Phi_{L_{IC}}\\ v \neq 1, v \neq n}} \cos{\left(\frac{\beta_I}{2}\right)} R_v^{-\eta_L} \sum\limits_{cl\in\bold\Phi_{cl}\cap A_{rb}} \sigma_{cm}$, then
the bistatic sensing coverage probability can be expressed as:
\small
\begin{equation}\label{rad_cov1_b}
\begin{aligned}[b]
&\mathbb{P} \left(\text{SINR}_{B_n} > \phi_s \right) 
= \mathbb{P} \bigg[\sigma_{tm} > \sum\limits_{cl \in \boldsymbol{\Phi}_{cl} \cap A_{rb}} \phi_s \, \sigma_{cm} 
+ \frac{\phi_s \, R_1^{\eta_L} L_{IC1_b}}{\cos{\left(\frac{\beta}{2}\right)}} \\
&+ \frac{\phi_s \, R_1^{\eta_L} L_{IC2_b}}{\cos{\left(\frac{\beta}{2}\right)}} + \frac{\phi_s \, (4\pi)^3 R_1^{\eta_L} R_n^{\eta_L}}{P_s G_{m}^2 \lambda^2 \cos{\left(\frac{\beta}{2}\right)}} 
\big(K_B T W_b + I_{L_r} + I_{N_r} \big) \bigg].
\end{aligned}
\end{equation}
\normalsize
Again, by adopting the Swerling I model, then:
\small
\begin{equation}
\begin{aligned}[b]
\!\!\!&\mathbb{P} \left(\text{SINR}_{B_n} > \phi_s \right) 
= \exp\bigg( - \frac{1}{\sigma_{\text{av}_t}} \bigg[ \sum\limits_{cl \in \boldsymbol{\Phi}_{cl} \cap A_{rb}} \phi_s \, \sigma_{cm} 
+ \frac{\phi_s \, R_1^{\eta_L} L_{IC1_b}}{\cos{\left(\frac{\beta}{2}\right)}} \\
\!\!\!&\quad + \frac{\phi_s \, R_1^{\eta_L} L_{IC2_b}}{\cos{\left(\frac{\beta}{2}\right)}} 
+ \frac{\phi_s \, (4\pi)^3 R_1^{\eta_L} R_n^{\eta_L}}{P_s G_{m}^2 \lambda^2 \cos{\left(\frac{\beta}{2}\right)}} 
\big(K_B T W_b + I_{L_r} + I_{N_r} \big) \bigg] \bigg).
\end{aligned}
\end{equation}
\normalsize
By incorporating the probability that the target is LoS with the RX and that the bistatic return occurs on a beam other than the beam using the same frequency, the expression can be decomposed into terms representing the LT of different interference sources.
Finally, since $R_n$ and $\beta$ are RVs with PDFs specified in (\ref{cond_dstt}) and (\ref{bet_distrb}), the theorem is proved by taking their expectations.











\section{Proof of Lemma 7}



The proof proceeds similarly to that of Lemma 1, utilizing the definition of LT and the moment-generating function of the Gamma distribution, we can reach:
\small
\begin{equation}
\mathcal{L}_{I_{L_c}} (s)=\mathbb{E}_{\bold\Phi_{L_c}}\left[\prod\limits_{\substack{\text{BS}_i\in \phi_{L_c}}}\left(1+\frac{  s P_c G (\theta_i)C_L r_i^{-\eta_L}}{m_L}\right)^{-m_L}\right].
\end{equation}
\normalsize
Combining BSs' location with their orientations and from the definition of PGFL in PPP, then:
\begin{equation}
\begin{aligned}
\mathcal{L}_{I_{L_c}} (s)&=\exp\left(- \lambda_{BS} \int_{-\frac{\pi}{d}}^{\frac{\pi}{d}}\int_{R_o}^{\infty}\bold p_{\bold{LOS}}\left(r\right) \right.\\ 
&  \left.\left(1-\left( 1+\frac{s P_c G(\theta_i)C_L r^{-\eta_L}}{m_L}\right)^{-m_L}    \right) r\;dr \;d\theta_i \right).
 \end{aligned}
\end{equation}
For NLOS interference, the same steps are followed using the NLOS parameters.



\section{Proof of Theorem 4}
\small
\begin{equation}
\begin{aligned}[b]
&\mathcal{P}_{\text{com}}(R_o)=\mathbb{P} \left(\rm{SINR_C}>\phi_c\right) =\mathbb{P}\left(\frac{P_c h_{L,o}G (\theta_m) C_L R_o^{-\eta_L}}{I_{L_c}+I_{N_c}+K_B T W_b}>\phi_c \right)\\
&=\mathbb{P}\left(h_{L,o}>\phi_c  R_o^{\eta_L}\left(P_c G (\theta_m) C_L\right)^{-1}
\times(I_{L_c}+I_{N_c}+K_B T W_c)\right).
\end{aligned}
\end{equation}
\normalsize
Utilizing Alzer’s inequality \cite{alzer1997some}:
\begin{equation}
\begin{aligned}[b]
\mathcal{P}_{\text{com}}(R_o) \approx & \sum_{n=1}^{m_L} \left(-1\right)^{n+1} {m_L \choose n}\\
&\mathbb{E}\left[\exp\left(-\;\frac{k_L\;n \;\phi_c \; R_o^{\eta_L}\; (I_{L_c}+I_{N_c}+K_B T W_b)}{P_c G (\theta_m) C_L}\right)\right],
\end{aligned}
\end{equation}
where $k_L=m_L(m_L!)^{-\;\frac{1}{m_L}}$. Moreover,
 from the definition of LT, and since \(\boldsymbol{\Phi}_{L_c}\) and \(\boldsymbol{\Phi}_{N_c}\) are independent, then:
\small
\begin{equation}
\begin{aligned}[b]
\mathcal{P}_{\text{com}}(R_o) &=\sum_{n=1}^{m_L} \left(-1\right)^{n+1} {m_L \choose n} \exp\left(-\;\frac{k_L\;n \;\phi_c \; R_o^{\eta_L}\; K_B T W_b}{P_c G (\theta_m) C_L}\right)\\
& \mathcal{L}_{I_{L_c}} \left(\frac{k_L\;n \;\phi_c \; R_o^{\eta_L}}{P_c G (\theta_m) C_L}\right) \mathcal{L}_{I_{N_c}} \left(\frac{k_L\;n \;\phi_c \; R_o^{\eta_L}}{P_c G (\theta_m) C_L}\right).
\end{aligned}
\end{equation}
\normalsize
since $G (\theta_m)$ is a random parameter, let $G (\theta_m)= G_c$ with PDF $f\left(G_c\right)$ then:
\small
\begin{equation}
\begin{aligned}[b]
\mathcal{P}_{\text{com}}(R_o)&=\sum_{n=1}^{m_L} \left(-1\right)^{n+1} {m_L \choose n} \int \exp\left(-\;\frac{k_L\;n \;\phi_c \; R_o^{\eta_L}\; K_B T W_b}{P_c G_c C_L}\right)\\
&\mathcal{L}_{I_{L_c}} \left(\frac{k_L\;n \;\phi_c \; R_o^{\eta_L}}{P_c G_c C_L}\right) \mathcal{L}_{I_{N_c}} \left(\frac{k_L\;n \;\phi_c \; R_o^{\eta_L}}{P_c G_c C_L}\right)f\left(G_c\right) dG_c.
\end{aligned}
\end{equation}
\normalsize   
Finally, to find the average coverage probability, we take the exception over $R_o$ whose PDF is given by (\ref{ner_dis}).




\bibliographystyle{ieeetr}
\bibliography{bibliography.bib}



\end{document}


