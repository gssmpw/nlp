% \clearpage
% \setcounter{page}{1}
% \maketitlesupplementary

\appendix

\section*{Appendix Overview}
This technical appendix consists of the following sections:
\begin{itemize}
% \item
% In Section~\ref{method}, we provide more method design.
\item
In Section~\ref{ablation}, we provide more ablation experiments to verify the effectiveness of FlowScene.
\item
In Section~\ref{quantitative}, we provide quantitative results from more experiments.
\item
In Section~\ref{visualizations}, we present more visual qualitative results of the SemanticKITTI val set.
\item
In Section~\ref{sec:limit},  we analyze the shortcomings of our method and directions for future work.
\end{itemize} 

\section{More Ablation Studies}
\label{ablation}
\subsection{Ablation Study for Optical Flow Networks}
Table~\ref{tab:flow} presents the performance of different optical flow networks. We compare several state-of-the-art methods, including PWC-Net~\cite{sun2018pwc}, RAFT~\cite{teed2020raft}, and FlowFormer~\cite{huang2022flowformer}, along with our setting, GMFlow~\cite{xu2022gmflow}, which is highlighted in the last row.
Our setting achieves the highest IoU of 45.01\% and mIoU of 18.13\%, outperforming all other methods in both metrics. These results suggest that GMFlow effectively captures motion cues and integrates them into the semantic scene completion task, providing superior performance over the other optical flow networks tested, with significantly fewer parameters.

\subsection{Ablation Study for Backbone Networks}
Table~\ref{tab:backbone} examines the impact of different backbone networks on the performance of FlowScene. The study compares EfficientNetB7~\cite{tan2019efficientnet}, ResNet50~\cite{he2016resnet}, and RepVit-M2.3~\cite{wang2023repvit}(our setting). Our method, using RepVit-M2.3, achieves the highest IoU of 45.01\% and mIoU of 18.13\%, surpassing both EfficientNetB7 (44.31\% IoU, 17.63\% mIoU) and ResNet50 (44.12\% IoU, 16.98\% mIoU). RepVit-M2.3, though achieving the best performance, maintains a relatively low parameter count of 22.4M. In comparison, EfficientNetB7 has a much higher parameter count of 63.8M, while ResNet50 is more parameter-efficient at 25.6M. RepVit-M2.3 offers a good balance between performance and parameter count, making it an ideal choice for our backbone network.
%--------------------------------
\begin{table}[t]
  \centering\small
\setlength{\tabcolsep}{4pt}{
  \begin{tabular}{l|ccc}
    \toprule
    {\textbf{Method}}& {\textbf{IoU}(\%)}&{\textbf{mIoU}(\%)}&\textbf{Params(M)} \\
    \midrule
    PWC-Net+~\cite{sun2018pwc}& 43.31&17.13&8.8\\
    RAFT~\cite{teed2020raft}&44.12&17.56&5.3 \\
    FlowFormer~\cite{huang2022flowformer}&44.33&17.74&18.2 \\
    \rowcolor{gray!20}{GMFlow}~\cite{xu2022gmflow}&{\textbf{45.01}}&\textbf{18.13}&\textbf{4.7} \\
    \bottomrule
  \end{tabular}
  }
      \caption{Ablation study for optical flow networks.}
  \label{tab:flow}
  % \vspace{-5mm}
\end{table}
%-------------------------------------------------------------
%--------------------------------
\begin{table}[t]
  \centering\small
\setlength{\tabcolsep}{4pt}{
  \begin{tabular}{l|ccc}
    \toprule
    {\textbf{Method}}& {\textbf{IoU}(\%)}&{\textbf{mIoU}(\%)}&\textbf{Params(M)} \\
    \midrule
    EfficientNetB7~\cite{tan2019efficientnet}& 44.31&17.63&63.8\\
    ResNet50~\cite{he2016resnet}&44.12&16.98&25.6 \\
    % RepVit-M1.5~\cite{wang2023repvit}&43.25&17.21&14.0 \\
    \rowcolor{gray!20}{RepVit-M2.3}~\cite{wang2023repvit}&{\textbf{45.01}}&\textbf{18.13}&\textbf{22.4} \\
    \bottomrule
  \end{tabular}
  }
      \caption{Ablation study for backbone networks.}
  \label{tab:backbone}
  % \vspace{-5mm}
\end{table}
%-------------------------------------------------------------
\section{More Quantitative Results}
\label{quantitative}

To provide a more thorough comparison, we provide additional quantitative results of semantic scene completion on the SemanticKITTI validation set in Table~\ref{tab:val}. The results further demonstrate the effectiveness of our approach in enhancing 3D scene perception performance.
Compared with the previous state-of-the-art methods, FlowScene is superior to other HTCL~\cite{li2024htcl} in semantic scene understanding, with a 1.00\% increase in mIoU. In addition, compared with Symphonize~\cite{jiang2024symphonize}, huge improvements are made in both occupancy and semantics. IoU and mIoU enhancement are of great significance for practical applications. It proves that we are not simply reducing a certain metric to achieve semantic scene completion.
%-------------------------------------------------------------------------

\begin{table*}[ht]
  \setlength{\tabcolsep}{2pt}
  \renewcommand\arraystretch{1.2}
  
  \resizebox{\textwidth}{!}{
  \begin{tabular}{l|l|c|r|rrrrrrrrrrrrrrrrrrr|r}
    \toprule
    \textbf{Methods} &\textbf{Published}&\textbf{Inputs}&\textbf{IoU}   
    & \rotatebox{90}{\vcenteredbox{\colorbox[RGB]{255,0,255}{\textcolor[RGB]{255,0,255}{\rule{1px}{1px}}}} \textbf{road} (15.30$\%$)} 
    & \rotatebox{90}{\vcenteredbox{\colorbox[RGB]{75,0,75}{\textcolor[RGB]{75,0,75}{\rule{1px}{1px}}}} \textbf{sidewalk} (11.13$\%$)} 
    & \rotatebox{90}{\vcenteredbox{\colorbox[RGB]{255,150,255}{\textcolor[RGB]{255,150,255}{\rule{1px}{1px}}}} \textbf{parking} (1.12$\%$)} 
    & \rotatebox{90}{\vcenteredbox{\colorbox[RGB]{175,0,75}{\textcolor[RGB]{175,0,75}{\rule{1px}{1px}}}} \textbf{other-grnd} (0.56$\%$)} 
    &  \rotatebox{90}{\vcenteredbox{\colorbox[RGB]{255,200,0}{\textcolor[RGB]{255,200,0}{\rule{1px}{1px}}}} \textbf{building} (14.10$\%$)} 
    &  \rotatebox{90}{\vcenteredbox{\colorbox[RGB]{100,150,245}{\textcolor[RGB]{100,150,245}{\rule{1px}{1px}}}} \textbf{car} (3.92$\%$)} 
    & \rotatebox{90}{\vcenteredbox{\colorbox[RGB]{80,30,180}{\textcolor[RGB]{80,30,180}{\rule{1px}{1px}}}} \textbf{truck} (0.16$\%$)} 
    & \rotatebox{90}{\vcenteredbox{\colorbox[RGB]{100,230,245}{\textcolor[RGB]{100,230,245}{\rule{1px}{1px}}}} \textbf{bicycle} (0.03$\%$)}  
    & \rotatebox{90}{\vcenteredbox{\colorbox[RGB]{30,60,150}{\textcolor[RGB]{30,60,150}{\rule{1px}{1px}}}} \textbf{motocycle} (0.03$\%$)} 
    & \rotatebox{90}{\vcenteredbox{\colorbox[RGB]{0,0,255}{\textcolor[RGB]{0,0,255}{\rule{1px}{1px}}}} \textbf{other-vehicle} (0.20$\%$)} 
    & \rotatebox{90}{\vcenteredbox{\colorbox[RGB]{0,175,0}{\textcolor[RGB]{0,175,0}{\rule{1px}{1px}}}} \textbf{vegetation} (39.3$\%$)} 
    & \rotatebox{90}{\vcenteredbox{\colorbox[RGB]{135,60,0}{\textcolor[RGB]{135,60,0}{\rule{1px}{1px}}}}  \textbf{trunk} (0.51$\%$)} 
    & \rotatebox{90}{\vcenteredbox{\colorbox[RGB]{150,240,80}{\textcolor[RGB]{150,240,80}{\rule{1px}{1px}}}} \textbf{terrain} (9.17$\%$)} 
    & \rotatebox{90}{\vcenteredbox{\colorbox[RGB]{255,30,30}{\textcolor[RGB]{255,30,30}{\rule{1px}{1px}}}} \textbf{person} (0.07$\%$)} 
    & \rotatebox{90}{\vcenteredbox{\colorbox[RGB]{255,40,200}{\textcolor[RGB]{255,40,200}{\rule{1px}{1px}}}} \textbf{bicylist} (0.07$\%$)} 
    & \rotatebox{90}{\vcenteredbox{\colorbox[RGB]{150,30,90}{\textcolor[RGB]{150,30,90}{\rule{1px}{1px}}}}  \textbf{motorcyclist} (0.05$\%$)} 
    &  \rotatebox{90}{\vcenteredbox{\colorbox[RGB]{255,120,50}{\textcolor[RGB]{255,120,50}{\rule{1px}{1px}}}} \textbf{fence} (3.90$\%$)} 
    & \rotatebox{90}{\vcenteredbox{\colorbox[RGB]{255,240,150}{\textcolor[RGB]{255,240,150}{\rule{1px}{1px}}}} \textbf{pole} (0.29$\%$)} 
    & \rotatebox{90}{\vcenteredbox{\colorbox[RGB]{255,0,0}{\textcolor[RGB]{255,0,0}{\rule{1px}{1px}}}} \textbf{traf.-sign} (0.08$\%$)}& \textbf{mIoU}  \\
    
    \midrule
    % LMSCNet$^\dagger$\cite{roldao2020lmscnet}&3DV'2020  & 28.61&40.68&18.22&4.38&0.00&10.31&18.33&0.00&0.00&0.00&0.00&13.66&0.02&20.54&0.00&0.00&0.00&1.21&0.00&0.00&6.70   \\
    % AICNet$^\dagger$\cite{li2020aicnet}&CVPR'2020   &29.59&43.55&20.55&11.97&0.07&12.94&14.71&4.53&0.00&0.00&0.00&15.37&2.90&28.71&0.00&0.00&0.00&2.52&0.06&0.00&8.31 \\
    % JS3C-Net$^\dagger$ \cite{yan2021sparse}&AAAI'2021   &38.98&50.49&23.74&11.94&0.07&15.03&24.65&4.41&0.00&0.00&6.15&18.11&4.33&26.86&0.67&0.27&0.00&3.94&3.77&1.45&10.31\\
    MonoScene \cite{cao2022monoscene}&CVPR'2022 &S& 36.86&56.52&26.72&14.27&0.46&14.09&23.26&6.98&0.61&0.45&1.48&17.89&2.81&29.64&1.86&1.20&0.00&5.84&4.14&2.25&11.08 \\
    
    TPVFormer \cite{huang2023tri}& CVPR'2023&S& 35.61&56.50& 25.87& 20.60& 0.85& 13.88& 23.81& 8.08& 0.36& 0.05 &4.35 &16.92& 2.26& 30.38& 0.51& 0.89& 0.00& 5.94& 3.14& 1.52&11.36 \\
    
    OccFormer\cite{zhang2023occformer}& ICCV'2023 &S& 36.50  & 58.85& 26.88& 19.61& 0.31& 14.40& 25.09& 25.53& 0.81& 1.19& 8.52& 19.63& 3.93& 32.62& 2.78& 2.82& 0.00& 5.61& 4.26& 2.86 & 13.46\\
    % VoxFormer-S\cite{li2023voxformer}& CVPR'2023&44.02&54.76&26.35&15.50&0.70&17.65&25.79&5.63&0.59&0.51&3.77&24.39&5.08&29.96&1.78&3.32&0.00&7.64&7.11&4.18&12.35\\
    Symphonize~\cite{jiang2024symphonize}&CVPR'2024&S&41.92&56.37&27.58&15.28&0.95&21.64&28.68&20.44&2.54&2.82&13.89&25.72&6.60&30.87&3.52&2.24&0.00&8.40&9.57&5.76&14.89\\
        
    VoxFormer-T\cite{li2023voxformer}& CVPR'2023&T& 44.15 &53.57&26.52&19.69&0.42&19.54&26.54&7.26&1.28&0.56&7.81&26.10&6.10&33.06&1.93&1.97&0.00&7.31&9.15&4.94 & 13.35\\
    H2GFormer~\cite{wang2024h2gformer} & AAAI'2024&T&44.69&57.00&29.37&21.74&0.34&20.51&28.21&6.80&0.95&0.91&9.32&27.44&7.80&36.26&1.15&0.10&0.00&7.98&9.88&5.81&14.29 \\
    HASSC~\cite{wang2024HASSC}&CVPR'2024&T&44.58&55.30&29.60&25.90&11.30&23.10&23.00&2.90&1.90&1.50&4.90&24.80&9.80&26.50&1.40&3.00&0.00&14.30&7.00&7.10&14.74\\
    HTCL~\cite{li2024htcl}&ECCV'2024&T&45.51&63.70&32.48&23.27&0.14&24.13&34.30&20.72&3.99&2.80&11.99&26.96&8.79&37.73&2.56&2.70&0.00&11.22&11.49&6.95&17.13\\
    \midrule
    \rowcolor{gray!20}\textbf{Ours}&  &T& {\underline{45.01}} &{\textbf{63.72}}&{\underline{32.10}}& 22.20&\underline{1.31}& {\textbf{25.63}}& {\underline{33.33}}&\textbf{33.47}&2.36& {\textbf{5.09}}&\textbf{16.99}& {\underline{26.35}}& {8.68}& {\underline{36.73}}&\textbf{3.79}& {{1.92}}& {\textbf{0.00}}& {\underline{12.05}}& {\textbf{11.65}}& {\underline{7.05}}&  {\textbf{18.13}}\\
    \bottomrule
  \end{tabular}}
  \caption{Quantitative results on the SemanticKITTI validation set. The best results are in {\textbf{Bold}}.}
  \label{tab:val}
\end{table*}
%-------------------------------------------------------------------------
\section{More Visualizations Results}
\label{visualizations}
We show visualization examples on the Semantickitti validation set, as shown in Figure~\ref{fig:sup1}. From left to right are the input image, the corresponding optical flow and occlusion mask, the front view SSC, and the top view SSC. Due to the motion information brought by the optical flow, the location information of the scene objects is more accurate and the layout is more reasonable.
We report the performance of more visual comparison results on the SemanticKITTI validation set in Figure~\ref{fig:sup2}. We compare with VoxFormer~\cite{li2023voxformer} and BRGScene~\cite{li2023stereoscene}. In general, our method performs more fine-grained segmentation of the scene and maintains clear segmentation boundaries. For example, in the segmentation completion result of cars, we predict clear separation of each car. In contrast, other methods show continuous semantic errors for occluded cars. In addition, our flow can effectively deal with the problem of mutual occlusion between different objects. Finally, we provide a video in the appendix to show the performance more intuitively.
% -------------------------------------------------
\begin{figure*}[t]
\centering
  \includegraphics[width=\textwidth]{supp2.pdf}
  \caption{Qualitative results on the SemanticKITTI validation set.}
  \label{fig:sup1}
\end{figure*}
% -------------------------------------------------

\section{Discussions}
\label{sec:limit}
Flowscene shows strong performance on the benchmark with an improved number of parameters. This is beneficial for deploying real-world autonomous driving applications. But the inference time of the model needs to be improved. Semantic scene completion in multi-camera settings is also worth attention, which is our future work. Meanwhile, the legal challenges of autonomous driving as well as privacy and data security risks are still topics of debate. Finally, the robustness of semantic scene completion is also an issue worth exploring.

% \clearpage
% -------------------------------------------------
\begin{figure*}[t]
\centering
  \includegraphics[width=\textwidth]{supp1.pdf}
  \caption{Qualitative results on the SemanticKITTI validation set.}
  \label{fig:sup2}
\end{figure*}
% -------------------------------------------------
% \clearpage