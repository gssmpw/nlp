\begin{table}[!tb]
\centering
\begin{tabular}{llllll}
\toprule
Method & VLM & Finetuning  & Rank             & Rank          & Rank \\
       &     & Data        & Exc. Real RLB.  & Real RLB.  &  All \\
\midrule
RT-Traj.  & 0-shot GPT-4o & - & 3.40 & 3.63 & 3.47 \\
RT-Traj.  & CaP GPT-4o & - & 3.57 & 3.36 & 3.41 \\
\method \ & VILA & Our Exc. Sim RLB. & 1.78 & 2.39 & 2.13\\
\method \ & VILA & Our & \textbf{1.59} & \textbf{1.28} & \textbf{1.40}\\

\bottomrule
\end{tabular}
\caption{\footnotesize{Ranking-based human evaluation of different VLMs, averaged across various real-world evaluation tasks. Results indicate that \method\ including simulation data is most effective since it captures both spatial and semantic information across diverse tasks from RLBench. This significantly outperforms zero-shot VLM-based trajectory generation, as described in \citet{gu2023rttrajectory}}}
\label{tab:experiments:vlm}
\end{table}
% \begin{table}
% \centering
% \begin{tabular}{l|ccc|c}
% \hline
% & \textbf{VQA Dataset} & \textbf{Real Robot Data} & \textbf{Sim Robot Data} & \textbf{Result}\\
% \hline
% RT-Trajectory (0-shot GPT4-o) &  &  &  & \\
% RT-Trajectory (Code-as-policies GPT4-o) &  &  &  & \\
% VILA & \cmark & & & \\
% \method~ & \cmark & \cmark &   & \\
% \yi{\method~ Freeze Vision} & \cmark & \cmark & \cmark & \\
% \method~ & \cmark & \cmark & \cmark & \\
% \hline
% \end{tabular}
% \label{tab:vlm}
% \caption{Compare Different VLMs }
% \end{table}