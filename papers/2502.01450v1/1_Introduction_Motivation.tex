\vspace{-2ex}
\section{Introduction}
\label{sec:into_motiv}

Understanding human behaviors within social networks is critical across various domains in social sciences. 
In recent years, the rapid growth of Large Language Models (LLMs) has shown great potential for making LLMs act like humans and simulate social networks~\cite{chen2024scalablemultirobotcollaborationlarge}. 
LLMs demonstrate the ability to adapt to different backgrounds and personalities through in-context learning, effectively simulating human beings~\cite{chuang2024simulatingopiniondynamicsnetworks}. 
The traditional studies~\cite{hamidian2019rumordetectionclassificationtwitter,kaligotla2015agent} of social networks predominantly emphasize mathematical equations, statistical analyses, and simplistic agent models. 
However, these approaches often constrain their ability to accurately simulate the diverse personalities and complex dynamics inherent in real social networks, potentially leading to significant impacts on both the processes and outcomes of such studies. 
With the growing capabilities of LLMs~\cite{radford2018improving}, utilizing them as agents to facilitate communication within social networks presents a promising approach to studying human behavior under various conditions.

In this study, we introduce a novel framework utilizing LLM-based agents to examine the spread of rumors and misinformation within social networks.
Our approach enhances the traditional simulations of rumor dynamics by incorporating LLMs as dynamic agents, offering a more realistic exploration of information dissemination.
To accurately simulate users in a social network, we employ an LLM to drive each agent with various personas and their preferences for accepting and spreading rumors, as defined by prompts.
Each agent is associated with a post history that includes all posts from itself and its neighbors, along with a record of its beliefs about each rumor, based on the LLM's output.
%We assess the impact of diverse agent interactions and scalable network designs on rumor propagation, employing hundreds of uniquely prompted agents to achieve comprehensive and authentic results.
Additionally, our dual-simulation framework accounts for both network properties and individual agent characteristics, providing a holistic view of how these factors jointly influence rumor dynamics.
This research not only demonstrates the utility of LLMs in understanding rumor spreading but also provides significant insights into the behavioral simulation capabilities of LLM-based agent societies.