\section{Related Work}
\label{sec:related_work}

\subsection{Modeling rumor spreading} 
In social network analysis, the spread of rumors is an important problem that has garnered extensive research and exploration. 
This is a network science topic where people tend to utilize statistical modeling and probabilistic analysis to formulate the network and define the spread of rumors. 
Common approaches include building statistical models with constraints~\cite{zehmakan2023rumorsspreadfastsocial} and defining multiple parameters that could affect the network~\cite{chen2020rumor}. 
However, these methods may not accurately reflect the individuals and the randomness in real-world societies.
There are works that use traditional agent-based modeling (ABM) to simulate the spread of rumors in a bottom-up approach, including using NetLogo~\cite{wilensky1999netlogo} agents as nodes in a social network~\cite{kaligotla2015agent} and defining mathematical models for agents~\cite{zehmakan2023rumorsspreadfastsocial}. 
However, these agents are still highly dependent on the definition of their mathematical properties.

\subsection{LLM-based Agents} 
In recent years, we have seen the flourishing of LLM~\cite{openai2024chatgpt} and its emergent abilities that perform well in various tasks. 
Recently, many studies have demonstrated the ability of LLMs to drive agents in ABM to simulate general human behavior~\cite{park2023Generate, chuang2024simulatingopiniondynamicsnetworks}.
LLM-based agents also demonstrate strong language comprehension and perform well in tasks guided by natural language instructions~\cite{chen2024scalablemultirobotcollaborationlarge}.
However, these studies primarily focus on utilizing LLMs as individual agents or basic agent communications, overlooking the potential for evaluating LLMs within a network graph to examine rumor propagation in complex social networks.

% This work plans to explore two important questions. 
% The first question this work addresses is how to construct LLM-based agents and their networks. 
% There are multiple design choices to consider, including how to record personas and memories, and whether to include a central LLM for syntactic checking. 
% Determining an optimal design can significantly enhance the exploration of rumor spread. 
% The second question explores how different structures influence the spread and detection of rumors, which will be analyzed through comprehensive studies of network topology characteristics and LLM behaviors.

% This work builds various networks that consist of LLM-based agents as nodes and directional communications as links. 
% Simulating a social network, the communications can be either dyadic one-to-one messages or one-to-many posts. 
% To effectively analyze various social networks, these networks can be modeled after either pre-existing real-world networks (e.g., X, Facebook) or synthetic networks with specific characteristics (e.g., small-world, Scale-Free).

% This work runs simulations in these networks using data from studies that include social network datasets with various rumors and annotations for rumor detection~\cite{hamidian2019rumordetectionclassificationtwitter}. 
% These rumors will be randomly assigned to a node in the network. The spread of the rumor will be recorded at each time step, along with the status of rumor detection (e.g., rumor is denied, rumor is questioned).