\section{Experiments}
\label{sec:eval}

\noindent\textbf{Experiment Setup.} We implemented functions to generate four network types: 
(a) an Erdős-Rényi random network, 
(b) a Scale-Free network, 
(c) a Small World network, and 
(d) a real-world network using Facebook's structure.
%see Figure~\ref{fig:main_figure_2}. 
Each structure has unique properties as shown in Table~\ref{tab:net_analysis}. 
Next, we assigned characteristics and rumors to the agents and randomly mapped them to the nodes of the network. 
Each experiment consisted of 500 iterations, during which an agent was selected in each iteration to make a post, as outlined in the previous section. 
The LLM employed for the agents was the latest version of ChatGPT-4o-mini.
% \S~\ref{sec:approach}.


We conducted a total of three experiments, each designed to examine distinct facets of the problem under investigation: 
(1) the effect of network structure on the dynamics of rumor propagation, 
(2) the impact of initial conditions and spread schemes on the spread of rumors, and 
(3) the role of agent characteristics in shaping the patterns of rumor spread. 
For efficiency, the second and third experiments are evaluated on the Scale-Free network.

%\begin{figure}[t]
%    \centering
%    \includegraphics[width=0.8\linewidth]{Figures/M2/network_structure.png}
%    \caption{Generated networks with agent information.}
%    \label{fig:main_figure_2}
%\end{figure}

% \begin{figure}[ht]
%     \centering
%     \begin{subfigure}[b]{0.16\textwidth}
%         \centering
%         \includegraphics[width=\textwidth]{Figures/demo_network_a.png}
%         \caption{Random Network.}
%         \label{fig:graph_a}
%     \end{subfigure}
%     \hspace{1cm} % Adjust the spacing as needed
%     \begin{subfigure}[b]{0.16\textwidth}
%         \centering
%         \includegraphics[width=\textwidth]{Figures/demo_network_b.png}
%         \caption{Facebook Network~\cite{chen2024scalablemultirobotcollaborationlarge}.}
%         \label{fig:graph_b}
%     \end{subfigure}
%     \caption{Generated networks with agent information.}
%     \label{fig:main_figure_2}
% \end{figure}

\begin{table}[t]
\footnotesize
\centering
\caption{Network Properties Comparison}
\vspace{-0.3cm}
\label{tab:net_analysis}
\begin{tabular}{|l|c|c|c|c|}
\hline
                    & \begin{tabular}[c]{@{}c@{}} \textbf{Erdős}\\\textbf{Rényi}   \end{tabular}
                    & \begin{tabular}[c]{@{}c@{}} \textbf{Scale}\\\textbf{Free}    \end{tabular}
                    & \begin{tabular}[c]{@{}c@{}} \textbf{Small}\\\textbf{World}   \end{tabular}
                    & \begin{tabular}[c]{@{}c@{}} \textbf{Facebook}\\\textbf{\#686} \end{tabular}
                    \\ \hline
\textbf{\# Nodes}   & 100           & 100           & 100           & 168          \\ \hline
\textbf{\# Edges}   & 396           & 390           & 200           & 1656         \\ \hline
\textbf{Avg Degree} & 7.92          & 7.80          & 4.00          & 19.71        \\ \hline
\textbf{Avg Path Len} & 2.42          & 2.37          & 3.88          & 2.43        \\ \hline
\textbf{Diameter}   & 4             & 4             & 7             & 6            \\ \hline
\textbf{Avg CC}     & 0.08          & 0.16          & 0.21          & 0.53         \\ \hline
\end{tabular}
%\vspace{-2ex}
\end{table}



\begin{figure}[t]
    \centering
    \includegraphics[width=0.75\linewidth]{Figures/4_Evaluation/eval1_1.png}
    \vspace{-2ex}
    \caption{Maximum percentage of affected nodes across all rumor-network combinations
    % for all the combinations of rumors and network types. 
    The small world network demonstrates the greatest susceptibility to rumor spread.
    % in general.
    }
    \label{fig:eval_1a}
    \vspace{-2ex}
\end{figure}

\begin{figure}[t]
    \centering
    \includegraphics[width=0.9\linewidth]{Figures/4_Evaluation/eval1_2.png}
    \vspace{-2ex}
    \caption{Propagation of Rumor \#2. The Small-World network shows the greatest susceptibility to rumor spread.}
    \label{fig:eval_1b}
    \vspace{-2ex}
\end{figure}

We tested, in total, the spread of 4 distinct rumors: 
\begin{itemize}
    \item Nicolae Ceaușescu is not dead!
    \item A living dinosaur is found in Yellowstone National Park.
    \item Large Language Models are manned by real people acting as agents.
    \item Drinking 3 ales a day can heal cancer!
\end{itemize}

%By default, we randomly assigned the agent characteristics and mapped the agents to the network nodes. 


\subsection{Effect of Network Structure} 
Figure~\ref{fig:eval_1a} depicts the maximum percentage of the network influenced by each rumor. 
A notable observation is the differential spread of rumors across the network: rumors that are less likely to be disproved tend to propagate more effectively, whereas intelligent agents predominantly reject those easily identifiable as misinformation. 
This behavior can be attributed to the agents' ability to leverage knowledge from the pretrained GPT model, allowing them to recognize and dismiss misinformation~\cite{liu2024largelanguagemodelsdetect}.
We argue that the nature of a rumor plays a critical role in its propagation. 
Rumors about history or engineering are often dismissed by most agents, likely due to extensive coverage during the model’s training. 
Conversely, rumors related to healthcare and nature exhibit a higher likelihood of spreading, potentially due to the agents' limited domain-specific knowledge, which renders them more vulnerable to misinformation.
However, the proprietary and non-transparent nature of ChatGPT's development and training processes prevents concrete verification of this analysis.
Future research focusing on the impact of specific knowledge domains or topics on rumor propagation would provide valuable insights.

Moreover, the structure of the network plays a pivotal role in influencing rumor propagation, with the connectivity and clustering characteristics of nodes being particularly impactful. 
For instance, the Small-World network, characterized by relatively sparse connectivity and moderate clustering, exhibits the highest susceptibility to rumor spread, with up to 50\% of nodes being affected. 
In contrast, as network connectivity increases and clustering decreases—due to greater randomization, as observed in Erdős-Rényi and Scale-Free networks—rumors propagate less effectively.
In the case of the real-world Facebook network, which is characterized by high density and strong clustering, rumors are even less likely to spread widely. 
This behavior can be attributed to the increased connectivity in dense networks, which exposes agents to a diverse array of information sources, thereby reducing the likelihood of rumor propagation. 
Additionally, clustering plays a critical role; nodes within the same cluster often share similar beliefs, creating a form of collective resistance to rumor.

Finally, we analyze the temporal dynamics of rumor propagation. Figure~\ref{fig:eval_1b} illustrates the spread of Rumor \#2 across all networks over time. 
The results reveal an almost linear relationship between rumor spread and time. Notably, an intriguing phenomenon emerges: as iterations progress, some nodes that initially accepted the rumor later reject it.
This behavior can be attributed to interactions with other agents, as each iteration exposes them to new posts and perspectives. Furthermore, the agents' decision-making processes—shaped by their "intelligence," derived from the pretrained GPT model—evolve with the influx of new information, prompting them to revise their stance.
These findings underscore the dynamic nature of rumor propagation and the complex interplay of agent interactions within the network.

%-------------------------------------------------------------------------

\begin{figure}[t]
    \centering
    \includegraphics[width=0.6\linewidth]{Figures/4_Evaluation/eval2.png}
    \vspace{-2ex}
    \caption{All rumors are spread when they originate from agents with more friends, and these agents are more active. Meanwhile, one particular rumor (rumor \#2) is widely spread using more random simulation strategies.}
    \label{fig:eval_2}
    \vspace{-2ex}
\end{figure}

\subsection{Effect of Initialization and Spreading Scheme} 

An additional critical factor influencing rumor propagation is the choice of the \textit{Initialization Strategy}, which determines the initial agents receiving the rumor, and the \textit{Activation Strategy}, which specifies the selection of agents in each iteration.
In this experiment, conducted on a Scale-Free network, the \textit{Initialization Strategy} and \textit{Activation Strategy} were chosen either randomly or based on node degree, as described in Section 3.
% ~\ref{sec:approach}. 

Figure~\ref{fig:eval_2} presents the matrix for the maximum percentage of nodes affected by each rumor under all combinations of \textit{Initialization Strategy} and \textit{Activation Strategy}.
%The results indicate that, in all scenarios, rumors spread more effectively when posting nodes are selected randomly. 
The widespread dissemination of all rumors is observed when both strategies target nodes with the highest degree. 
In this scenario, these popular agents continually spread rumors. The \textit{Activation Strategy} significantly enhances the propagation of all rumors, thereby facilitating their spread throughout the network. 
This outcome can be attributed to the fact that a highly connected \textit{Initialization Strategy} accelerates the rumor’s reach across a larger portion of the network. 
Meanwhile, if the posts are not initially presented to the popular nodes, not all rumors can spread. 
Rumor \#2, which is readily accepted by agents, is efficiently propagated to nearly all agents, whereas other rumors are ignored. 
Furthermore, when the popular nodes are no more active than others in all random strategies, all rumors spread with limited impact.

%-------------------------------------------------------------------------

% \begin{figure}[t]
%     \centering
%     \includegraphics[width=0.9\linewidth]{Figures/4_Evaluation/eval3.png}
%     \vspace{-2ex}
%     \caption{As the personality of the agents shifts from being more likely to accept rumors to being less likely to accept them, we notice a decline in rumor spreading.}
%     \vspace{-2ex}
%     \label{fig:eval_3}
% \end{figure}

\subsection{Effect of Agent's Personas}
The final experiment aimed to examine the impact of agent personas—specifically, the agents' predisposition to accept rumors—on rumor propagation.
As in the previous experiment, this study was conducted exclusively on the Scale-Free network structure. 
We examined the maximum percentage of nodes affected under three distinct agent personality configurations: (a) all agents are highly likely to accept a rumor, (b) each agent's likelihood of accepting a rumor is assigned randomly, and (c) all agents are highly unlikely to accept a rumor.
% Figure~\ref{fig:eval_3} presents the maximum percentage of nodes affected under three distinct agent personality configurations: (a) all agents are highly likely to accept a rumor, (b) each agent's likelihood of accepting a rumor is assigned randomly, and (c) all agents are highly unlikely to accept a rumor.
As expected, the agents' personality configurations significantly influence the spread of rumors. 
The results (see Appendix) reveal a clear decline in rumor propagation as the agents' likelihood of accepting rumors transitions from highly receptive to highly resistant. 
This underscores the critical role of agent characteristics in shaping the dynamics of misinformation.
%This highlights the critical role of agent characteristics in shaping the overall dynamics of misinformation spread within the network. 


%\noindent \textbf{Limitation.} 
%Although the results generally align with expectations and clearly demonstrate the effectiveness of LLM-based agents, some unintuitive findings have emerged. 
%For example, allowing more posts from high-degree nodes appears to suppress the spread of rumors. This may be because nodes with more connections exhibit a personality trait that discourages rumor propagation.
%To accurately understand the behavior and impact of LLM-based agents, it is essential to conduct further experiments aimed at decoupling the characteristics of the LLM from the properties of the network. 
%This includes a more detailed analysis of rumor propagation and mitigation effectiveness.

% in the random network shown in Fig~\ref{fig:graph_a} over 25 simulation iterations, initializing each agent with 2 random messages containing the 4 rumors. According to the results, rumors spread at different speeds, with rumor (1) being shared and accepted by 9 out of 10 agents, as shown in Fig. 3. The other three rumors spread to only 1, 1, and no agents in 20 iterations.

% In another experiment with 20 agents connected by 70 edges in a Scale-Free network, the same four rumors were accepted by only 3, 4, 0, and 1 agents in 25 iterations. These results illustrate that network characteristics and agent personalities can affect the spread of rumors.
% %There are three tasks in this work. First, we need to design various LLM-based agents that can perform the simulation. Second, we need to generate or synthesize various networks for the simulation. Third, we need to simulate the spread of rumors within the networks. We will test all possible combinations of (agent, network) and record the entire process. Finally, we will analyze the results and draw conclusions. The timeline is shown in Table~\ref{tab:milestones}.
% \begin{figure}[ht]

%     \centering
%     \includegraphics[width=0.4\textwidth]{Figures/fig3.png}
%     \caption{Spread of rumor (1) at iterations 0, 2, 4, 8, 16, 24. Yellow nodes indicate agents who believe the rumor, and highlighted nodes and edges show agents act in the iteration.}
%     \label{fig:main_figure_3}
% \end{figure}
% The results show that network generation and agent construction are functioning correctly, allowing the LLM to act as agents and enabling observation of rumor spread. However, the efficiency of prompts and factors influencing rumor spread require further evaluation.