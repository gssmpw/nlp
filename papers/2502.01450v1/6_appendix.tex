\section{Synthetic Networks}
In Table~\ref{tab:net_analysis}, we present three synthetic networks with various properties. Below is the visualization of these networks, where the name on each node denotes the agent associated with that node.

\begin{figure}[h]
    \centering
    \includegraphics[width=0.7\linewidth]{Figures/6_Appendix/Social_Graph_random.png}
    % \vspace{-2ex}
    \caption{Visualization of Erdős-Rényi random network.}
    % \vspace{-3ex}
    \label{fig:app_1_1}
\end{figure}

\begin{figure}[h]
    \centering
    \includegraphics[width=0.7\linewidth]{Figures/6_Appendix/Social_Graph_sc.png}
    % \vspace{-2ex}
    \caption{Visualization of Scale-Free network.}
    % \vspace{-3ex}
    \label{fig:app_1_2}
\end{figure}

\begin{figure}[h]
    \centering
    \includegraphics[width=0.7\linewidth]{Figures/6_Appendix/Social_Graph_sw.png}
    % \vspace{-2ex}
    \caption{Visualization of Small World network.}
    % \vspace{-3ex}
    \label{fig:app_1_3}
\end{figure}

\section{Agent Personas}

In all experiments, each agent's personas are randomly generated by ChatGPT-4, following this structure:

\begin{mdframed}
\begin{quote}
    \tt \small
    \textbf{id:} 3\\
    \textbf{agent\_name:} Leo\\
    \textbf{agent\_age:} 35\\
    \textbf{agent\_job:} Software Developer\\
    \textbf{agent\_traits:} Analytical, Persistent\\
    \textbf{agent\_rumors\_acc:} 3\\
    \textbf{agent\_rumors\_spread:} 3
\end{quote}
\end{mdframed}

\begin{mdframed}
\begin{quote}
    \tt \small
    \textbf{id:} 124\\
    \textbf{agent\_name:} Olivia\\
    \textbf{agent\_age:} 29\\
    \textbf{agent\_job:} Data Scientist\\
    \textbf{agent\_traits:} Curious, Logical\\
    \textbf{agent\_rumors\_acc:} 4\\
    \textbf{agent\_rumors\_spread:} 1
\end{quote}
\end{mdframed}

\section{Prompt Template}

In each iteration, for the selected agents, we use the ChatGPT API to submit prompts to the ChatGPT-4o-mini model. Given an agent persona, a list of rumors, and post history, the exact prompts we used are as follows:

\begin{mdframed}
\begin{quote}
    \tt \small
    \textbf{role:} system\\
    \textbf{content:} You are a helpful assistant.\\
    \textbf{role:} user\\
    \textbf{content:} Hi, \{\textit{agent\_name}\}, you are a \{\textit{agent\_age}\}-year-old \{\textit{agent\_job}\} known for being \{\textit{agent\_traits}\}. Please follow the instructions below.
  You are active on a social network, receiving and sending posts. 
  You \{\textit{likely\_to\_accept\_rumors
  [agent\_rumors\_acc]}\}, and you \{\textit{likely\_to\_forward\_rumors
  [agent\_rumors\_spread]}\}.
  
  Read through the post history, especially the new posts. It can be something you've read in other posts but you need to rephase it your personality.
  You can criticize the posts if you don't agree with them, you can also repeat them or express in your own way.
  Your posts can be seen by all your friends. Here are your friends: \{\textit{friend\_list}\}
  You are about to send a new post [POST] based on your personal preferences. 


  After posting, you will review a list of rumors and decide [CHECK] whether to believe or reject each one. Be honest: if your post mentions a rumor,
  your response must be consistent with what you posted.

  [Action Output Instruction]
  Start with 'POST', then on a new line, specify the content of your new post.
  Then, on a new line, output 'CHECK', followed by True or False for each rumor.
  
  Example\#1: 
  
  POST
  
  I just read that Donald Trump will be president of Greece! OMG! That's interesting.
  
  CHECK
  
  False COVID-19 now named as COVID-114514.
  
  True Donald Trump will be president of Greece.

  Example\#2: 
  
  POST
  
  What a nice day! I enjoy my job as a teacher.
  
  CHECK
  
  False COVID-19 now named as COVID-114514.
  
  False Donald Trump will be president of Greece.

  Before you reviewing the posts, you used to believe:

  You used to believe \{\textit{rumor\_list[str(i)]}\} is True

  The previous post history is: \{\textit{post\_history}\}
  
  Think step-by-step about the task. Be careful not to let the rumor list affect your judgment on post history.
  
  You CANNOT post the information from the rumor list but NOT in your post history.
  
  The rumor list is: \{\textit{rumor\_list}\} Check whether you believe them based on what you read and send.
  
  Try not to exactly repeat what others have said.
  
  Propose exactly one action (POST and CHECK) for yourself in the current round.

  Your response:
\end{quote}
\end{mdframed}

The dictionaries \textit{likely\_to\_accept\_rumors} and \textit{likely\_to\_forward\_rumors} are defined below:

\begin{mdframed}
\textbf{likely\_to\_accept\_rumors:}
\begin{itemize}
    \item[1:] won't easily accept any rumors or new information unless they are confirmed or well-examined
    \item[2:] may suspect rumors but will accept them once they appear frequently in posts or generally make sense
    \item[3:] will accept any new information unless there is significant controversy or criticism
    \item[4:] will easily accept any rumors, even if there are doubts or criticisms
\end{itemize}
\end{mdframed}

\begin{mdframed}
\textbf{likely\_to\_forward\_rumors:}
\begin{itemize}
    \item[1:] prefer not to spread much of the new information seen in others' posts
    \item[2:] may forward posts seen with comments and feelings, or may just share personal experiences
    \item[3:] are willing to share and comment on rumors, posts, and new things seen in posts
\end{itemize}
\end{mdframed}

\section{Supplementary Evaluation Results}

The final experiment of the evaluation section investigated the influence of agent predisposition on rumor propagation within a Scale-Free network.
Three distinct personality configurations—high, random, and low acceptance—were analyzed.
Figure~\ref{fig:eval_3} shows the maximum percentage of nodes affected under those three distinct agent personality configurations.
Consistent with expectations, we observed a clear decline in spread as receptivity decreased, highlighting the impact of agent traits on misinformation dynamics.

\begin{figure}[ht]
    \centering
    \includegraphics[width=0.9\linewidth]{Figures/4_Evaluation/eval3.png}
    \vspace{-2ex}
    \caption{As the personality of the agents shifts from being more likely to accept rumors to being less likely to accept them, we notice a decline in rumor spreading.}
    \vspace{-2ex}
    \label{fig:eval_3}
\end{figure}

