\section{Methods}
\label{sec:approach}

To demonstrate the capabilities of LLMs in simulating the spread of rumors and their mitigation within social networks, we aim to: 
(a) construct various social networks, 
(b) design and implement multiple LLM-based agents as part of an ABM framework operating within the networks, and 
(c) evaluate both the propagation of rumors and the effectiveness of potential mitigation strategies.

%\vspace{-3.5ex}
\subsection{Network Construction}
%\vspace{-2.5ex}
%\subsection{Network Construction} 
Network analysis represents individuals and their relationships as nodes and edges, respectively.
In the context of rumor propagation, the social network models a social media environment where individuals interact with friends and share personal sentiments and rumors. 
To characterize this network, we propose that nodes represent users within the social network, while edges signify the friendship relationships between pairs of users (nodes). 
As friends can view each other’s messages, this network is defined as an undirected network.
%To protect privacy, synthetic data will be used for node information. 
The structure of a network significantly influences behavior in simulations~\cite{Alam2011}. 
To investigate this, we construct various networks for simulation and analysis, employing two approaches:

\noindent\textbf{Synthetic Networks.} 
We algorithmically generate networks with specific characteristics, including Erdős-Rényi networks~\cite{erdos1959random}, Scale-Free networks~\cite{barabasi2003scale}, and Small-World networks~\cite{watts1998collective}. 
These networks enable us to examine the relationship between rumor-spreading and network properties.

\noindent\textbf{Real-World Networks.} 
To simulate more realistic scenarios, we utilize real-world social network data collected from Facebook to generate various networks~\cite{leskovec2012learning}.

\noindent Our objective is to evaluate the spread of rumors across all network types and to explore how network characteristics influence rumor propagation.

\begin{figure}[ht]
    \centering
    \begin{subfigure}[t]{0.21\textwidth}
        \centering
        \includegraphics[width=\textwidth]{Figures/3_Approach/llm1.png}
        \caption{Network with LLM-based agents; each agent maintains its own post history and rumor beliefs.}
        \label{fig:llm_a}
    \end{subfigure}
    \hspace{0.04\textwidth}
    \begin{subfigure}[t]{0.21\textwidth}
        \centering
        \includegraphics[width=\textwidth]{Figures/3_Approach/llm2.png}
        \caption{At each iteration, the agent generates a post, appends it to its own and its neighbor’s history, and updates its rumor belief.
        % At each iteration, the agent generates the post and appends it to both its own and its neighbor's history, and the agent's rumor belief is updated.
        }
        \label{fig:llm_b}
    \end{subfigure}
    \vspace{-2ex}
    \caption{Design of LLM-based multi-agent network.}
    \label{fig:main_figure}
    \vspace{-2ex}
\end{figure}



\begin{algorithm}
\footnotesize
\caption{Simulating Rumor Spread with LLM Agents
% Simulation of Spread of Rumors with LLM Agents
}
\begin{algorithmic}[1]
\State \textbf{Input:} $G$ social network, $N$ agent personas $\{p_{i}\}_{i=1}^{N}$, $L$ list of rumors $\{r_{j}\}_{i=j}^{L}$, number of time steps $T$
\State \textbf{Output:} $B$, the belief in rumors, where $\langle b_{ij} \rangle \in [0,1]$ represents the belief of agent $a_i$ in rumor $r_j$.
\For{$i = 1$ \textbf{to} $N$}\Comment{Agent Initialization}
    \State Assign node $n_i$ of $G$ to agent $a_i$
    \State Initialize agent $a_i$ with persona $p_i$
    \State $friend\_list_i = \{\}$ 
    \For{each edge $e_{ix}$ in $G$ connecting $n_i$ to $n_x$}
        \State Add $a_x$ to $friend\_list_i$
    \EndFor
\EndFor
\For{$j = 1$ \textbf{to} $L$}\Comment{Rumor Initialization}
    \State Select an agent $a_i$ based on Initialization Strategy
    \State Append $r_{j}$ to $a_i$ post history.
\EndFor
\For{$t = 1$ \textbf{to} $T$}\Comment{Simulation}
    \State Select an agent $a_i$ based on Activation Strategy
    \State Agent $a_i$ reads post history and makes a new post
    \For{$a_x$ in $\{a_i, friend\_list_i\}$}
        \State Add the new post to post history of $a_x$
    \EndFor
    \For{$j = 1$ \textbf{to} $L$}
        \State Agent $a_i$ updates opinion of $r_{j}$ to $\langle b_{ij} \rangle$ in $B$
    \EndFor
\EndFor
\end{algorithmic}
\label{alg:alg1}
\end{algorithm}






%\vspace{-3ex}
%\subsection{LLM-based Agent}
% \vspace{-2cm}
\subsection{LLM-based Agent} 
In these networks, each node $n_i$ represents an agent $a_i$ driven by an LLM, as illustrated in Figure~\ref{fig:llm_a} and Algorithm~\ref{alg:alg1}. 
We use a consistent basic prompt structure and the ChatGPT-4o-mini~\cite{openai2024chatgpt} for each agent $a_i$, but customize each prompt with unique information specific to each agent $a_i$. 
The prompt includes the following components: 
(a) Task description and examples that instruct the LLM to generate responses in the correct syntax; 
(b) Agent personas $p_{i}$, including name, age, job, personality traits, and the agent’s willingness to accept and spread rumors, tailored to each agent; 
(c) Rumors previously believed by the agent; 
(d) Posts visible to the agent; (e) A complete list of all rumors. 

Component (a) directs the LLM to generate a new post based on information from (b), (c), and (d), and to evaluate the agent’s belief in each rumor $\langle b_{ij} \rangle$ in the overall rumor matrix $B$ according to (e). 
This setup simulates a user reading posts, creating a new post, and updating opinions on various rumors. Initially, each agent is assigned some random messages, which include the rumors $r_j$ to be tested. 
The agents that have the rumors in their initial post histories are determined by the \textit{Initialization Strategy}. This could either be fully random or a degree-based selection, where the rumors always start with the agents that have the highest degrees (most number of friends).

As Figure~\ref{fig:llm_b} illustrates, in each iteration, an agent $a_i$ is selected in accordance with the \textit{Activation Strategy}, where the probability of an agent's selection may be either fully random or proportional to its degree. This models the realistic social dynamic wherein individuals with a larger number of connections tend to disseminate more posts. The agent $a_i$ feeds the prompt to the LLM and generates a response that contains a new post and its updated belief in each rumor, denoted as $\langle b_{ij} \rangle$. It then appends the new post to its own history as well as to the histories of connected nodes in $friend\_list_i$, while simultaneously updating its opinions on all rumors in $B$.

%For instance, during iteration 3, agent \#2 (Mary) reviews all posts (d), including a rumor about a living dinosaur shared by her friend, agent \#1 (Hanbo). 
%If Mary decides to believe and propagate the rumor, the output generated by the LLM will include a new post affirming the rumor's validity, accompanied by her agreement with its truthfulness (e). 
%This new post is then appended to the history of all her friends, and Mary is recorded as a believer in the rumor.
% From the results, we should be able to observe and analyze the relationships between the spread of rumors and the characteristics of the network and agents. 
% In each network, we can use the history of each time step to analyze the important nodes and links that affect the spread of rumors and identify their characteristics. 
% We can then develop and evaluate different strategies to efficiently stop the rumors. 
% Additionally, we can evaluate and determine which LLM-based agent designs are reasonable compared to real-world behaviors.

%\vspace{-3ex}
%\subsection{Mitigation}
%\vspace{-3ex}
%\noindent \textbf{Mitigation.} 
%Mitigation strategies can be implemented by modifying the network, such as restricting connectivity, or by introducing specialized LLM-based agents, such as fact-checking agents, into the network. 
%The implementation of these mitigation strategies will be demonstrated and discussed in Section M3.


%\vspace{-3ex}
%\subsection{Scalability}
%\vspace{-3ex}
\subsection{Scalability} 
Recent work~\cite{li2024agentsneed} demonstrates that LLM-based agents can be effectively scaled to enhance overall performance. 
This aligns with our observations when scaling our design to a network comprising over 100 nodes and 1000 edges.
%—comparable in size to the Facebook dataset~\cite{leskovec2012learning}. 
%In simulations involving 500 posts and multiple agent interaction rounds, the spread of rumors was successfully simulated and meticulously recorded across the network. 
The complexity of Algorithm~\ref{alg:alg1} is $\mathcal{O}(T(N+B))$, where T is the number of iterations, N is the number of agents, and B is the number of rumors.
In each step, an inference request is sent to the LLM, and the response is appended as input for subsequent requests. 
Thus, further scaling the networks could significantly increase the length of inputs and outputs for each LLM request, leading to a surge in computational costs and potentially reducing accuracy in managing long contexts. 
Future work could include implementing an efficient compression method for post history, parallelizing the agents, and enhancing LLM serving optimizations~\cite{zheng2023efficiently} to improve scalability.
%to address these scaling challenges effectively.