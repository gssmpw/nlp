

\section{Percentile TTFT Constraint Approximation}\label{sxn:extra_theory}

The hardness result of Theorem \ref{thm:decision_nphard} motivates us to consider approximation guarantees for computing a schedule satisfying constraints on the TTFT. One important relaxation of the problem is to instead consider scheduling algorithms that satisfy a constraint on a fixed percentile of the per-query TTFTs. This problem is of interest since latency constrained applications typically seek to bound the P95 or P99 latency of a response in practice. In this section, we show that such a relaxation is tractable when the interaction distance of queries in a queue (controlled by the latency constraint $T$ and maximum prompt length $m$) is bounded.

Theorem \ref{thm:percentile_approx} proves there is an algorithm that runs in polynomial time with respect to the query stream length that either returns a schedule satisfying constraint $T$ on the $(1-p)$-th percentile TTFT or certifies that no feasible schedules exists such that all queries satisfy the TTFT constraint $T$. Note that there is a necessary gap where the the positive case pertain the the percentile TTFT problem and the negative case certifies no schedule satisfies a constraint on the maximum TTFT. This gap is fundamental, since otherwise we could solve the problem for the maximum TTFT in by adding enough unsatisfiable queries and solving the percentile TTFT decision problem.


\begin{theorem}\label{thm:percentile_approx}
    There exists an algorithm that takes a length $n$ query stream $\Qcal$ (see Definition \ref{def:query_stream}) and a TTFT constraint $T > 0$ as input and either:
    \begin{enumerate}
        \item Returns a certificate that no schedule exists for $\Qcal$ that satisfies TTFT constraint $T$, or
        %
        \item Returns a schedule for $\Qcal$ such that the $(1-p)$-th percentile TTFT is at most $T$,
    \end{enumerate}
    under the computational model of Definition \ref{def:llm_computation} with $\cattn=0$. Furthermore, if every query is at most length $m$ and no query is an exact prefix of another, then this algorithm runs in $\Ocal(n \cdot \exp(1/p \log 1/p))$ time when $m$ and $T$ are bounded by a constant.
\end{theorem}

\begin{proof}

    At a high level we will prove that, if there exists a schedule for $\Qcal$ satisfying the TTFT constraint $T$ for all queries, then there exists a subset $\Qcal' \subset \Qcal$ of size $(1-p)n$ that also satisfies the constraint $T$ for all queries. By the contrapositive of the statement, if $\Qcal'$ does not exist, then no such schedule exists for $\Qcal$. We show that $\Qcal'$ can be constructed by decomposing $\Qcal$ into $\Theta(n)$ subproblems, each of which can be solved in time independent from $n$, thereby providing an algorithm that is tractable with respect to $n$.

    Without loss of generality, we assume that $p \cdot n$ is an integer and that $n$ is divisible by $n_0 = \frac{2T}{p}$ for reasons we will explain later. First, we partition the query stream $\mathcal{Q}$ into $d = \frac{n}{n_0} $ disjoint blocks $\mathcal{Q}_1, \dotsc, \mathcal{Q}_d$ that are contiguous with respect to arrival time.
    
    Let this partition satisfy the constraint that for any $i < j$, every query in $\mathcal{Q}_i$ has an equal or earlier arrival time than every query in $\mathcal{Q}_j$. For each block $\mathcal{Q}_k$, we remove $2T$ queries with the latest arrival times to form the reduced block $\mathcal{Q}'_k$. Define $\Qcal' = \bigcup_{k=1}^d \Qcal_k'$.

    First, note that if there exists a feasible schedule for $\Qcal'$ under a uniform TTFT constraint $T$, then there exists a schedule for $\Qcal$ where the $(1-p)$-th percentile TTFT is at most $T$. This schedule can be constructed by following the schedule for all queries in $\Qcal'$ and then processing the remaining queries afterwards. By the contrapositive statement, if there does not exist such a schedule for $\Qcal'$, then there does not exist such a schedule for $\Qcal$.

    We will next show that it is possible to efficiently compute such a schedule for $\Qcal'$ or certify that none exists due to the decomposable nature of the problem. Since $\Qcal_k'$ was constructed by removing the last $2T$ queries of $\Qcal_k$, the latest query arrival time in $\Qcal_k'$ must be at least $T$ units of time before the arrival time of any query in $\Qcal_{k+1}'$ if a schedule for $\Qcal$ satisfying the uniform TTFT constraint exists. This is because no query is an exact prefix of another, and so it must take at least one unit of time to process a query. Therefore, if all queries in $\Qcal_k \setminus \Qcal_k'$ satisfy the TTFT constraint, then their arrival times must span at least a $T$ length interval of time.

    The partition $\{\Qcal_k\}_{k \in [d]}$ was constructed to partition the arrival times of $\Qcal$ into $d$ contiguous intervals. Then, the previous argument implies that the queries in $\Qcal_k'$ must finished being processed before the earliest arrival time in $\Qcal_{k+1}'$, and so the processing time of queries in $\Qcal'$ (the earliest arrival time to the latest completion time) under a processing order of $\Qcal$ satisfying constraint $T$ can be partitioned into $d$ contiguous time windows, each corresponding to a $\Qcal_k'$ block. However, this does not completely decompose the problem, as the processing order of $\Qcal_k'$ may still affect the the next block $\Qcal_{k+1}'$ through prefix reuse.

    We may handle this dependency by keeping track of the feasible last queries for each block that are possible under schedules that satisfy the TTFT constraint, as these dictate the potential cache states when computing the next block. More concretely, if there is a feasible schedule for $\Qcal$, then the following procedure must return a feasible schedule for $\Qcal'$:
    \begin{enumerate}
        \item Let the possible cache initialization of $\Qcal_k'$ be the feasible end queries of $\Qcal_{k-1}'$ or the empty string if $k=1$. Then, in order of $k=1,...,d$, compute all possible pairs of a cache initialization and last query processed in $\Qcal_k$ where a feasible schedule satisfying the constraint $T$ exists.
        %
        \item Consider the queries as vertices in a graph along with a vertex representing the empty string initialization and the set of pairs computed in the last step as directed edges in this graph. Compute a path from the empty string vertex to a vertex representing a query in $\Qcal_d'$.
        %
        \item There exists a feasible a schedule for $\Qcal'$ where the latest processed queries in each $\Qcal_k'$ is provided by the path computed in the last step. Hence, we may compute schedule for each $\Qcal_k'$ with the fixed last query with the constraint that the cache initialization and last query processed is dictated by the returned path.
    \end{enumerate}
    Note that in step one above, if the query in $\Qcal_{k-1}'$ that is processed latest under the feasible schedule for $\Qcal$ is recorded as a possible initialization of $\Qcal_k'$, then the procedure will correctly identify the last query in $\Qcal_k'$ processed under the feasible schedule for $\Qcal$ as a possible initialization for $\Qcal_{k+1}'$. Since the only feasible initialization of the cache at $\Qcal_1'$ is the empty string, by induction we conclude that step one correctly identifies tuples of feasible cache initialization and end queries for block. From this, there must exist a path returned by step two, since the sequence of tuples that occur correspond to the feasible schedule for $\Qcal$ must be a directed path from the empty string vertex to the last query processed under the schedule.

    Next, we show that the above procedure runs in $\Ocal(n \cdot \exp(1/p \cdot \log 1/p))$ time for each of the above steps:
    \begin{enumerate}
        \item To compute the set of tuples for $\Qcal_k'$, we must consider at most $n_0$ possible cache initialization and $n_0!$ orderings of queries in $\Qcal_k'$. We may verify if a fixed combination satisfies the constraint in $\Ocal(n_0)$ time. Therefore, computing set of tuples for a fixed $k$ has the following time complexity:
        \begin{gather*}
            \Ocal(n_0^2 \cdot n_0!)
            = \Ocal(n_0^{n_0 + 2})
            = \exp(n_0 \cdot \log n_0)
            = \exp(1/p \cdot \log 1/p).
        \end{gather*}
        Performing this procedure for each $k$ then takes $\Ocal(n \cdot \exp(1/p \cdot \log 1/p))$ time.
        %
        \item For calculating the path, each directed edge is either between a query in $\Qcal_k'$ to a query in $\Qcal_{k+1}'$ or from the empty string vertex to a query in $\Qcal_1'$. Hence, the directed graph is acyclic, and the degree of every vertex is at most $n_0$. Therefore, finding a path from the empty set vertex to a vertex corresponding to a query in $\Qcal_d'$ can be accomplished by BFS in $\Ocal(n \cdot \poly(n_0))$ time.
        %
        \item Finally, given a feasible sequence of last processed queries for each $\Qcal_k'$, we may compute the schedule for the other queries in $\Qcal_k'$ while considering a fixed cache initialization and last processed query. This can be done by evaluating all possible schedules in $\Ocal(n_0!) = \exp(1/p \cdot \log 1/p)$ time. Doing this for all $d$ blocks then takes $\Ocal(n \cdot \exp(1/p \cdot \log 1/p))$ time.
    \end{enumerate}
    
    Note that computing the partition $\{\Qcal_k\}_{k \in [d]}$ depends solely on the arrival times and so it can be done in $\Ocal(n)$ time. Finally, to complete the proof, note that we do not need to remove the $2T$ last queries in $\Qcal_d$ to construct $\Qcal_d'$, as no other blocks will become dependent on it. Hence, we may adjust the argument by a constant factor, and the assumption that $n$ is exactly divisible by $\frac{2T}{p}$ is not needed. 
\end{proof}
