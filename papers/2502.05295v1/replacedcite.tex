\section{Related Work}
\label{sec:lit-review}

We summarize the most relevant prior work here, with a more detailed discussion in \cref{app:extended-lit}.

\textbf{Classical Spatiotemporal Causal Inference.} Earlier spatiotemporal causal inference methods (e.g., spatial econometrics ____, difference-in-differences ____, and synthetic controls ____) rely on strong assumptions (e.g., parallel trends) and often fail to address interference or time-varying confounders. More recent classical approaches, on the other hand, typically estimate average effects at the regional level or rely on structural and modeling assumptions that may not hold in real-world spatiotemporal contexts ____.

\textbf{Machine Learning for Spatiotemporal Modeling.} Machine learning models (e.g., convolutional and recurrent networks-based methods ____, or graph-based approaches ____) capture spatiotemporal patterns for prediction but lack formal causal adjustments.

\textbf{Time Series Causal Inference.} Time-series causal inference often uses recurrent or transformer-based methods ____ but assumes independent time series, ignoring potential interference effects. Although iterative G-computation ____ or marginal structural models ____ can handle time-varying confounders, most ML extensions ____ exclude interference or cross-series confounding.

\textbf{Neural-Based Spatiotemporal Causal Inference.} In the context of neural spatiotemporal models, ____ integrate spatial representations for causal inference, accounting for spatial confounding and leveraging temporal data to train a UNet model. However, they do not address feedback effects from lagged or time-varying confounders. Most similar to our work, ____ present a climate-focused model that shares certain architectural similarities but emphasizes prediction rather than adjusting for time-varying confounders, leaving causal identification concerns largely unaddressed. 

%\textbf{Our contribution.} The GST-UNet bridges these gaps by combining a U-Net-based architecture for complex spatiotemporal grids with iterative G-computation to handle time-varying confounders. Unlike prior methods, GST-UNet accommodates interference, nonlinear spatiotemporal dynamics, and valid causal identification, enabling fine-grained, location-specific causal effects estimation.