% This must be in the first 5 lines to tell arXiv to use pdfLaTeX, which is strongly recommended.
\pdfoutput=1
% In particular, the hyperref package requires pdfLaTeX in order to break URLs across lines.

\documentclass[11pt]{article}

% Change "review" to "final" to generate the final (sometimes called camera-ready) version.
% Change to "preprint" to generate a non-anonymous version with page numbers.
% \usepackage[review]{acl}
\usepackage[preprint]{acl}
% \usepackage[preprint]{acl}
\usepackage{multirow}
\usepackage{float}
\usepackage{tabularx}
% Standard package includes
\usepackage{times}
\usepackage{latexsym}

% For proper rendering and hyphenation of words containing Latin characters (including in bib files)
\usepackage[T1]{fontenc}

% For Vietnamese characters
% \usepackage[T5]{fontenc}
% See https://www.latex-project.org/help/documentation/encguide.pdf for other character sets

% This assumes your files are encoded as UTF8
\usepackage[utf8]{inputenc}
\usepackage{booktabs} % 用于更好的表格线条

% This is not strictly necessary, and may be commented out,
% but it will improve the layout of the manuscript,
% and will typically save some space.
\usepackage{microtype}
\usepackage{amsthm}
% This is also not strictly necessary, and may be commented out.
% However, it will improve the aesthetics of text in
% the typewriter font.
\usepackage{inconsolata}

\usepackage{enumitem}
%\usepackage[hang, flushmargin]{footmisc}

%Including images in your LaTeX document requires adding
%additional package(s)
\usepackage{graphicx}

\usepackage{xcolor}      % 用于颜色设置
\usepackage{hyperref}    % 用于超链接

% 竟然默认没有这两个包,看来ACL一般不需要
\usepackage{amssymb}
\usepackage{amsmath}

\usepackage{subcaption}
\usepackage{adjustbox}
\usepackage{hyperref}
% 设置超链接颜色为深蓝色
\hypersetup{
    colorlinks=true,        % 使用彩色超链接
    linkcolor=darkblue,     % 文内链接颜色
    urlcolor=darkblue       % 网页链接颜色
}

% 定义深蓝色
\definecolor{darkblue}{HTML}{000099}

\newtheorem{assumption}{Assumption}
\newtheorem{proposition}{Proposition}
\newtheorem{corollary}{Corollary}
\newtheorem{lemma}{Lemma}
\newtheorem{definition}{Definition}

\newenvironment{restatedproposition}[1]{%
    \renewcommand{\theproposition}{#1}%
    \proposition%
}{%
    \endproposition%
}

\newenvironment{restatedcorollary}[1]{%
    \renewcommand{\thecorollary}{#1}%
    \corollary%
}{%
    \endcorollary%
}

% If the title and author information does not fit in the area allocated, uncomment the following
%
%\setlength\titlebox{<dim>}
%
% and set <dim> to something 5cm or larger.

%\title{Revealing and reducing the effect of random noise in pre-training data for language models}
%\title{Do we really have to filter out random noise in pre-training data? An emperical and theoretical study on language models}
%\title{How does random noise in pre-training data affect language models?}
\title{Do we really have to filter out random noise in pre-training data for language models?}

\author{Jinghan Ru, Yuxin Xie, Xianwei Zhuang, Yuguo Yin, Yuexian Zou\thanks{Corresponding author.} \\
        School of Electronic and Computer Engineering, Peking University\\
        \{jinghanru, yuxinxie, xwzhuang, ygyin\}@stu.pku.edu.cn \\
        zouyx@pku.edu.cn \\
        }

\begin{document}
\maketitle
\begin{abstract}
  Web-scale pre-training datasets are the cornerstone of LLMs' success. However, text data curated from the internet inevitably contains random noise caused by decoding errors or unregulated web content. In contrast to previous works that focus on low quality or synthetic data, our study \textbf{provides the first systematic investigation into such random noise through a cohesive ``What-Why-How'' framework.} Surprisingly, we observed that the resulting increase in next-token prediction (NTP) loss was significantly lower than the proportion of random noise. We provide a theoretical justification for this phenomenon, which also elucidates the success of multilingual models. On the other hand, experiments show that the model's performance in downstream tasks is not based solely on the NTP loss, which means that random noise may result in degraded downstream performance. To address the potential adverse effects, we introduce a novel plug-and-play Local Gradient Matching loss, which explicitly enhances the denoising capability of the downstream task head by aligning the gradient of normal and perturbed features without requiring knowledge of the model's parameters. Additional experiments on 8 language and 14 vision benchmarks further validate its effectiveness.
  %Web-scale pre-training datasets are the cornerstone of large language models' success. However, text data curated from the internet inevitably contains various types of noise, whose impact on language models needs to be understood. While existing research primarily focuses on low-quality or synthetic data, the \textit{random noise} introduced by unregulated websites or crawler decoding errors has been largely overlooked. This paper \textbf{investigates the influence of such random noise and proposes strategies to mitigate its impact on downstream tasks}. Surprisingly, we observed that the performance degradation rate was significantly lower than the proportion of noise. We provide a theoretical justification for this phenomenon, which also elucidates the success of multilingual models and can be applied to other modalities. To address the adverse effects of noise, we introduce a novel plug-and-play Local Gradient Matching loss, which explicitly enhances the denoising capability of the downstream task head by aligning the gradient of normal and perturbed features to improve local smoothness without requiring knowledge of the model's parameters. Extensive experiments on 8 language and 14 vision benchmarks validate the effectiveness of our proposed method. \footnote{Code, data, and model checkpoint weights are available at \url{https://anonymous.4open.science/r/lmn-acl-E9D3}}
\end{abstract}

\section{Introduction}

\begin{figure}[t]
  \centering
  \includegraphics[width=\columnwidth]{fig/htdj4.pdf}
  \vspace{-7mm}
  \caption{Overview of the study and methodology. (a) The common scenario in which a GPT model, pre-trained on filtered data $P^c$, demonstrates robust performance. (b) When the pre-training dataset is contaminated with random noise $P^n$, the resultant language model may exhibit unpredictable behavior. (c) Our approach focuses on the effective fine-tuning of black-box noisy models for downstream tasks $P^d$.}
  \label{fig:1}
  \vspace{-5mm}
\end{figure}

Large language models (LLMs), particularly the GPT series \citep{gpt2, gpt3, gpt4}, have fundamentally transformed the research landscape in natural language processing. The remarkable performance of these autoregressive models is largely attributed to pre-training on extensive datasets, which are gathered by crawling text from the whole internet. Given the sheer volume of these datasets, they inevitably encompass a wide variety of noise \citep{pretrainguide, wimbd}. 
Consequently, it is imperative to understand its impact, as the quality of pre-training data plays a decisive role in the effectiveness of LLMs \citep{llama}. 
%Mainstream belief \citep{llama, qwen, doremi} underscores the importance of high-quality data as the bedrock for the successful training of large models. On the other hand, \citet{refinedweb} suggests that performent models can be achieved without any quality filtering of the dataset. This discrepancy highlights the pressing need for a comprehensive understanding of the impact of dataset noise on language models, which is crucial for the development of LLMs \citep{dssurvey}.Significant research has been conducted in this area. 
\citet{allenphysics, xie2023data} highlight that low-quality data can significantly diminish a model's knowledge capacity and performance. \citet{naturemc, colmmc} demonstrate that recursively training LLMs with synthetic data can lead to model collapse. 

However, \textbf{little attention has been paid to the impact of random noise within datasets}. Due to anti-crawling mechanisms \citep{crawldetect}, decoding errors\footnote{\scriptsize \url{https://stackoverflow.com/questions/62499600/gibberish-text-output-because-of-encoding-in-web-scraping}}, and tremendous amounts of unmaintained websites\footnote{\scriptsize \url{https://community.cloudflare.com/t/website-showing-garbage-text}}, the raw data obtained through web crawling inevitably contains a substantial amount of random noise \citep{webnoise, wavlm, webnoise1}. Although theoretically it may not be challenging to remove such noise, practical limitations in computational resources often result in incomplete data cleaning%\footnote{\scriptsize \url{https://github.com/jiangyy/gpt-tokens}} 
\citep{dssurvey, dolma}. For example, it is observed that the Chinese corpus used to train the GPT-4o tokenizer contains a considerable amount of nonsensical data. \footnote{\scriptsize \url{https://github.com/jiangyy/gpt-tokens}}
Therefore, it is of great importance to gain a thorough understanding of the effects of such random noise.%, which will contribute to a deeper insight into the robustness of LLMs.

We conduct extensive experiments based on the OpenWebText dataset \citep{openwebtext} used to pre-train language models with the same architecture and parameter size as GPT-2. Specifically, to simulate random noise shown in Figure~\ref{fig:1}, we randomly generate, with proportions of 1\%, 5\% and 20\%, a sequence of integers within the range of 0 to 50256, according to the vocabulary size of GPT-2's tokenizer, to simulate the tokenization outcome of nonsensical text found on the internet. %which lacks coherence and follows an irregular distribution pattern. Then we concatenate these noise to the end of the clean data. The next-token prediction (NTP) pre-training process then continues as usual. %The purpose is to mimic the unpredictable nature of web-based noise after it has been tokenized.
%The trained models were evaluated on the clean validation set to assess the impact of random noise on the autoregressive loss. 
Interestingly, we observe that the presence of random noise does not lead to a catastrophic failure in model training; instead, \textbf{its effect on autoregressive loss is disproportionately small}, e.g.,~the increase in loss is only about 1\% even with 20\% of the dataset being noisy. %Unlike low-quality or synthetic data that can have a permanently detrimental effect on model performance, the impact of random noise can be mitigated by increasing the training duration. 
We provide a theoretical analysis to explain these phenomena, which also sheds light on the success of multilingual models (where one language may appear as ``noise'' to another%\citep{chomsky, dssurvey}
) and large speech models \citep{wavlm}, indicating the broader implications of studying the effects of random noise.

%Beyond the impact of noise on pre-training loss, it is also crucial to understand its effects in downstream tasks. 
On the other hand, further experiments reveal that a model that exhibits a lower NTP loss experiences a 1.5\% decrease in accuracy on downstream tasks. This indicates that performance on downstream tasks is not solely rely upon the NTP loss. Given the common practice of fine-tuning a pre-trained foundation model rather than undergoing a full pre-training process from scratch, we opt to follow the work of \citet{nml} by exploring how to efficiently fine-tune language models only using extracted features for downstream tasks when the pre-training data and model weights are not accessible, which reflects real-world application scenarios for LLMs. To mitigate the potential adverse effects of noise, we propose a novel plug-and-play Local Gradient Matching (LGM) loss. This method involves artificially adding noise to the output features and minimizing the gradient difference between the noisy and original features. We also provide a theoretical analysis of the LGM loss. Interestingly, when applying the LGM loss to fine-tune clean models such as Llama-3 or ViT-L on 8 language and 14 vision datasets, we observe an unexpected improvement in accuracy, which effectively demonstrates the versatility and broad applicability of the LGM loss beyond its original intent of addressing noise-related issues. 

% \begin{itemize} 
%   \item We investigate the underexplored problem of random noise in pre-training datasets for language models.
%   \item We pre-train multiple GPT-2 models and the empirical results demonstrate that the influence of random noise is relatively insignificant. Then we provide a theoretical analysis, extending our findings to other domains and thus highlighting the significance of this research direction.
%   \item We propose a novel loss for fine-tuning downstream task head, supported by comprehensive experimental and theoretical analyses that confirm its efficacy.
% \end{itemize}

The remaining part is arranged as follows. In Section \ref{sec:2}, we summarize related works. In Section \ref{sec:3} and \ref{sec:4}, we follow a logical structure based on \textbf{``What-Why-How''}:
\vspace{-2mm}
\begin{itemize} 
   \item What: What is the effect of random noise? In Section \ref{sec:3}, we demonstrate through experiments how random noise impacts NTP loss.
   \vspace{-8mm}
   \item Why: Why does it have this effect? The section further explores the underlying reasons by providing a rigorous theoretical analysis.
   \vspace{-3mm}
   \item How: How do we mitigate the potentially harmful effect on downstream tasks? In Section \ref{sec:4}, We introduce the LGM loss and provide a theoretical explanation.
   \vspace{-3mm}
\end{itemize}
Section \ref{sec:4} further provides evidence of the novelty and effectiveness of the LGM loss by conducting extensive experiments on 22 downstream tasks.

In summary, our contributions are as follows: (1) We investigate the underexplored problem of random noise in pre-training datasets for language models. (2) We pre-train multiple GPT-2 models and the empirical results show that the influence of random noise on NTP loss is insignificant. Then we provide a theoretical analysis, extending our findings to other domains and thus highlighting the significance of this research direction. (3) We propose a novel blackbox fine-tuning LGM loss for downstream tasks, supported by comprehensive experimental and theoretical analysis that confirm its efficacy. \textbf{Code, data, and model checkpoint weights are available at} \href{https://anonymous.4open.science/r/lmn-acl-E9D3}{this repo}.
\nopagebreak
\section{Related Works}
\label{sec:2}

\begin{figure*}[ht]
  \centering
  \begin{subfigure}{0.3\linewidth}
    \includegraphics[height=3.9cm, width=\linewidth]{Sec3-Exp/fig-2-1.pdf}
    \caption*{(a)}
  \end{subfigure} \hfill
  \begin{subfigure}{0.3\linewidth}
    \includegraphics[height=3.9cm, width=\linewidth]{Sec3-Exp/fig-2-2.pdf}
    \caption*{(b)}
  \end{subfigure} \hfill
  \begin{subfigure}{0.35\linewidth}
    \includegraphics[height=3.9cm, width=\linewidth]{Sec3-Exp/fig-2-3.pdf}
    \caption*{(c)}
  \end{subfigure}
  \vspace*{-2.5mm}
  \caption{Next-token prediction loss on the clean OpenWebText validation set for GPT-2 models pre-trained on synthetic OpenWebText datasets with varying levels of random noise. (a) Trend of NTP loss as training proceeds. (b) Difference in NTP loss between the noisy and clean models after the same number of training iterations. (c) Difference in loss values after undergoing the same number of training iterations on clean OpenWebText data.}
  \label{fig:2}
  \vspace*{-4.5mm}
\end{figure*}

\textbf{Pre-training Data Analysis for Language Model Training.} \citet{wimbd} analyzed open-source datasets like The Pile and C4, uncovering significant amounts of low-quality content in these datasets. \citet{allenphysics, colmmc} highlighted the negative impact of such data on training. Despite these remarkable contributions, there remains a lack of understanding regarding the specific effects of random noise on language model performance. This paper aims to address this gap.

\textbf{Noisy Model Learning.} Our work draws significant inspiration from Noisy Model Learning (NML) proposed by \citet{nml}. In NML, the authors introduce noise into large datasets like ImageNet by randomly altering labels, then pre-train neural networks on these noisy datasets. The study reveals that moderate label noise enhances in-distribution (ID) sample classification, while out-of-distribution (OOD) performance deteriorates with increasing noise. This paper extends the concept of NML, presenting theoretical insights and methodologies that are applicable across multiple modalities and various problems.

Due to space limitations, the detailed related works are provided in Appendix~\ref{app:relatedwork}.


\begin{figure*}[ht]
  \centering
  \begin{subfigure}{0.31\linewidth}
    \includegraphics[height=4cm, width=\linewidth]{Sec3-Exp/fig-2-4.pdf}
    \caption*{(a)}
  \end{subfigure} \hfill
  \begin{subfigure}{0.31\linewidth}
    \includegraphics[height=4cm, width=\linewidth]{Sec3-Exp/fig-2-5.pdf}
    \caption*{(b)}
  \end{subfigure} \hfill
  \begin{subfigure}{0.33\linewidth}
    \includegraphics[height=4cm, width=\linewidth]{Sec3-Exp/fig-2-7.pdf}
    \caption*{(c)}
  \end{subfigure}
  \vspace*{-2.5mm}
  \caption{Validation experiments.  (a) Loss trends on the random noise in the \textit{training set} of the model trained on the dataset with 5\% random noise. (b) Comparison of the loss between 5\% random noise and Gaussian noise. (c) The loss difference on the clean OpenWebText validation set compared to the baseline for models trained on datasets with 5\% random noise and 5\% Gaussian noise, respectively.}
  \label{fig:3}
  \vspace*{-4mm}
\end{figure*}

\section{Revealing the Effect of Random Noise in Language Model Pre-training}
\label{sec:3}

In this section, we first pre-train multiple GPT-2 models on synthetic noisy OpenWebText corpus to investigate the impact of random noise in the pre-training data. We then provide a theoretical analysis of the results and validate our theory through experiments. Finally, we demonstrate that the insights gained from our investigation have broader applicability beyond the immediate scope of our study. In summary, Section~\ref{sec:3} focuses on understanding the impact of random noise by presenting detailed experiments and theoretical analyses. We delve into how random noise affects NTP loss and introduce insights into why these effects occur. It sets the foundation for our investigation.

The frequently used notation and their descriptions are shown in Appendix~\ref{sec:appa}.

\subsection{Experimental Design}
\textbf{Preliminary.} Let $L$ denote the maximum context length of the language model and let $\mathcal{W}$ represent the model's vocabulary with size $V = |\mathcal{W}|$. We define $\mathcal{X}$ as the set of all discrete sentences that the model can represent, where $\mathcal{X} = \cup_{i=1}^{L}\{0, 1, \dots, V -1\}^{i} = \cup_{i=1}^{L}\mathcal{W}^{i}$ and $\{0, 1, \dots, V -1\}^{i}$ represents prefixes of length $i$. For any discrete set $A$, let $\Delta_A$ denote the set of all probability distributions defined on $A$. Given that next-token prediction (NTP) is actually a classification task given the prefix, we define joint probability distributions $P^c, P^n, P^m \in \Delta_{\mathcal{X} \times \mathcal{W}}$ where $P^c$ represents the distribution of clean data, $P^n$ represents the distribution of noise data, and $P^m$ represents the distribution of the mixed pre-training dataset which contains both clean and noise data. Since the noisy dataset can be viewed as the concatenation of clean data and random noise, it can be formalized by the Huber contamination model \citep{oodfz} as follows:
\vspace{-1mm}
\begin{equation}
  \label{eq:huber}
  P^m = \alpha P^n + (1 - \alpha) P^c
  \vspace{-1.5mm}
\end{equation}
\noindent where we use $\alpha$ to represent the noise proportion. An explanation of Equation~(\ref{eq:huber}) can be found in Appendix~\ref{app:huber}. For any joint probability distribution $P \in \Delta_{\mathcal{X} \times \mathcal{W}}$, let $P_X \in \Delta_{\mathcal{X}}$ and $P_{\cdot \vert X} \in \Delta_{\mathcal{W}}$ represent the marginal and conditional distribution of $P$ over $\mathcal{X}$ and $\mathcal{W}$. 

We use $\mathcal{H}$ to denote the hypothesis space (e.g., all possible parameter configurations given the transformer architecture ). Define $h: \mathcal{X} \rightarrow \mathbb{R}^V \in \mathcal{H}$ as the language model and $\boldsymbol{p}^{h}_{\cdot \vert x}(w)$ as the $w$-th component of the probability distribution induced by $h(x)$. The next-token prediction loss can be expressed as follows:
\begin{equation}
  \label{eq:ntp}
  \mathcal{L}_{ntp}(P, h) = \mathbb{E}_{x \sim P_X} \mathbb{E}_{w \sim P_{\cdot | x}} \big[-\log(\boldsymbol{p}^h_{\cdot | x}(w))\big].
\end{equation}
\vspace{-5.5mm}

\textbf{Data setup.} We utilize the OpenWebText dataset \citep{openwebtext} which comprises 8 billion tokens as an alternative to the original WebText dataset used to train GPT-2 124M models \citep{gpt2}. Concretely, to mimic the unpredictable nature of garbage noise after it has been tokenized, we first generate a sequence of integers where each integer follows a uniform distribution over $[0, 50256)$ (recall that 50256 is the vocabulary size of GPT-2's tokenizer). Furthermore, we notice that beyond uniform distribution, certain tokens appeare with significantly higher frequency in real-world random noise. To reflect this phenomenon, approximately 100 tokens in $[0, 50256)$ are randomly selected and their frequencies increase accordingly. The overall distribution of the data can be referred to in Figure~\ref{fig:dataset}. The generated noise is then added to the clean dataset such that $\alpha$ is 1\%, 5\%, and 20\% respectively. 
Each synthetic noisy dataset is used to pre-train a GPT-2 model. We argue that given the vast diversity of websites on the internet, it is inevitable that datasets used for training large models will inadvertently include such garbled text as shown in Figure~\ref{fig:1}. This kind of noise is not just theoretical; it is a practical issue encountered during web crawling and data collection processes.

We set the context length $L$ to be 1024 and the batch size to be 640. All models are trained for 300,000 iterations. To evaluate the performance, the resulting model checkpoints are tested on the clean OpenWebText validation set, measuring the NTP loss for comparison. Further details regarding datasets and experimental parameters can be found in Appendix~\ref{sec:mer}.

\subsection{Results}
% 三张图:(1)fig2-4 (2) fig2-4+gaussian(前两个+0.1)(3)和rand 5%的对比
In Figure~\ref{fig:2}, we illustrate the evolution of the NTP loss throughout the training process. Although random noise has a negative effect on the model's performance as expected, experimental results yield two intriguing insights: 

(1) In contrast to the low-quality or synthetic data, \textbf{the presence of random noise does not lead to training collapse}, even when the noise level reaches 20\%. While increasing training time on low quality or synthetic data typically degrades model performance \citep{allenphysics,naturemc}, extending the training duration continues to drive down the model's loss in the case of random noise. 

(2) \textbf{The impact of random noise on the loss is disproportionately small}. For instance, 5\% of random noise only results in a 0.2\% increase in the NTP loss. This discrepancy becomes even smaller if the noisy models are calibrated to match the number of training iterations with the baselines trained on clean datasets.

These positive experimental outcomes further corroborate the robustness of language models and provide insights into why pre-training on large-scale datasets that inevitably contain significant amounts of noise can still yield high-performing models. These somewhat unexpected findings naturally prompt us to explore the underlying reasons.


\begin{figure*}[ht]
  \centering
  \begin{subfigure}{0.23\linewidth}
    \includegraphics[height=3.cm, width=\linewidth]{Sec3-Exp/ntp/fig-2-8.pdf}
    \caption*{(a)}
  \end{subfigure} \hfill
  \begin{subfigure}{0.24\linewidth}
    \includegraphics[height=3.cm, width=\linewidth]{Sec3-Exp/ntp/fig-2-9.pdf}
    \caption*{(b)}
  \end{subfigure} \hfill
  \begin{subfigure}{0.23\linewidth}
    \includegraphics[height=3.cm, width=\linewidth]{Sec3-Exp/ntp/fig-2-10.pdf}
    \caption*{(c)}
  \end{subfigure} \hfill
  \begin{subfigure}{0.24\linewidth}
    \includegraphics[height=3.cm, width=\linewidth]{Sec3-Exp/ntp/fig-2-11.pdf}
    \caption*{(d)}
  \end{subfigure}
  \vspace*{-2.5mm}
  \caption{Loss and its difference across different types and levels of noise within the ArXiv and Wikipedia corpora.}
  \label{fig:4}
  \vspace*{-4mm}
\end{figure*}

\subsection{Theoretical Analysis}
In the analysis below, we focus on the impact of random noise on NTP loss, as pre-training loss is crucial for the performance on downstream tasks \citep{mathlm, whyhelp, sameloss, zcy2}. Specifically, we are interested in the difference of NTP Loss between a model $h^*$ trained on a noise-free dataset and a model $h$ trained with a noisy dataset. We begin by noting that sampling from the clean distribution should not yield random gibberish and vice versa. Mathematically, this implies that for any prefix $r$ sampled from $P^n_X$, the probability under the clean distribution $P^c_X(r)$ is zero. Thus, we make the following assumption:

\begin{assumption}
\label{assumption1}
$P^c$ and $P^n$ have disjoint support sets, i.e., ${\text{supp}}(P^c) \cap {\text{supp}}(P^n) = \emptyset$.
\end{assumption}

\noindent The subsequent proposition demonstrates that the error $\epsilon$ introduced to the loss due to random noise is less than the proportion $\alpha$ of random noise in the dataset.

\begin{proposition}
\label{pro1}
Under Assumption~\ref{assumption1}, let $h^*$ be a model trained on $P^c$, with $\mathcal{L}_{ntp}(P^c,h^*)=-\log p_c$ and $\mathcal{L}_{ntp}(P^n,h^*)=-\log p_n$. When the model $h$ is trained on a mixed distribution $P^m$ which includes noise, it attempts to fit $P^n$, leading to an increase in the loss on the clean distribution $P^c$, such that $\mathcal{L}_{ntp}(P^c,h)=-\log(p_c-\epsilon)$ and $\mathcal{L}_{ntp}(P^n,h)=-\log(p_n+\epsilon/k)$ for some $\epsilon > 0$ ($k$ can be shown to be $\Omega(e^{\mathcal{L}_{ntp}(P^n,h)})$). Let $\eta = \alpha p_c - (1-\alpha)kp_n$.
We arrive at the following conclusions:

(1) If $\alpha \leq \frac{kp_n}{p_c + kp_n}$, then for any $0 < \epsilon <p_c$, we have $\mathcal{L}_{ntp}(P^m, h) \ge \mathcal{L}_{ntp}(P^m, h^*)$. This means that when $\alpha$ is sufficiently small, the global minimum on $P^m$ will not be affected by noise.

(2) If $\alpha > \frac{kp_n}{p_c + kp_n}$, then for $\epsilon < \eta$, it holds that $\mathcal{L}_{ntp}(P^m,h) < \mathcal{L}_{ntp}(P^m,h^*)$. This suggests that if $\alpha$ is large enough, the impact on the optimal hypothesis is at least as much as $\alpha p_c - (1-\alpha)kp_n$.

(3) When $\alpha < \frac{1}{3}$ and $k>\frac{\alpha(1-3\alpha)p_c}{(1-\alpha)(2-3\alpha)p_n}$, for $\epsilon \ge 3\eta$ we get $\mathcal{L}_{ntp}(P^m,h^*) < \mathcal{L}_{ntp}(P^m,h)$. Similarly, it can be shown that $\epsilon$ does not exceed $2\eta$ when $\alpha > \max\left(\frac{kp_n}{p_c + kp_n}, \frac{1}{2}\right)$ and $k > \frac{(2\alpha-1)p_c}{2(1-\alpha)p_n}$. This indicates that when $k$ is sufficiently large, the effect of noise is at most $\mathcal{O}(\alpha p_c - (1-\alpha)kp_n)$.

\end{proposition}

The proof can be found in Appendix~\ref{proof_pro1}. 
%Proposition~\ref{pro1} establishes the theoretical performance gap between the optimal model under distribution $P^m$ and the optimal model under distribution $P^c$. In essence, it explores how much a model trained on $P^m$ deviates from the optimal model on $P^c$. The findings indicate that this deviation is possible only when $\alpha$ is large enough, and it's less severe than the proportion of noise, as $\epsilon = \alpha p_c$ results in an increase of $\alpha$ percent of $\mathcal{L}_{ntp}(P^c, h)$. In fact, $\epsilon / k$ always approaches zero because the distribution of random noise can only be learned through memorization of training samples, and since $\alpha \ll 1 - \alpha$ and $k \gg 1$, it follows that $\epsilon \ll \alpha p_c$. This insight elucidates the experimental outcomes. On the other hand, if we want to degrade the performance of $h$ relative to $h$, it is necessary that $\alpha p_c - (1 - \alpha)kp_n > 0$, implying a certain equivalence between $\epsilon < \alpha p_c - (1 - \alpha)kp_n$ and $\mathcal{L}_{ntp}(P^m,h) < \mathcal{L}_{ntp}(P^m,h)$. Furthermore, in order to replace $p_n$ with $\tilde{p}_n = p_n + \frac{\epsilon}{k}$ from Proposition~\ref{pro1}, we have the following corollary:
% \begin{corollary}
%   \label{cor1}
%   Building upon Proposition~\ref{pro1}, if $\alpha > \frac{k p_n}{p_c + k p_n}$, then for $\epsilon < \mathcal{O}(\alpha p_c - (1 - \alpha)k \tilde{p}_n)$, it follows that $\mathcal{L}_{ntp}(P^m,h) < \mathcal{L}_{ntp}(P^m,h)$.
% \end{corollary}
Proposition~\ref{pro1} primarily investigates the performance gap between models trained on $P^m$ and those on $P^c$. It is proved that when $\alpha$ is small enough, the presence of noise has no impact on the optimal model on $P^m$. Even as $\alpha$ approaches $\frac{1}{3}$ or even $\frac{1}{2}$, as long as $k$ is large enough (the analysis regarding $k$ and other parameters is detailed in Appendix~\ref{app:explain}), the loss induced by noise, $\epsilon$, does not exceed $\mathcal{O}(\alpha p_c - (1-\alpha)kp_n)$. Given that $k$ is much greater than 1, this implies $\epsilon$ is much smaller than $\alpha p_c$. This explains the observed experimental results.

With these theoretical results in hand, we then conduct multiple experiments to substantiate their validity. First, we plot the trend of NTP loss on random noise within the training set throughout the learning process, as shown in Figure~\ref{fig:3}(a). It is evident that the loss on random noise decreases at a very slow rate, indicating that the model struggles to efficiently learn the distribution of random noise. %This observation contrasts with previous findings that neural networks can easily fit random labels \citep{rethink}, which needs further investigation. 

Next, we add 5\% Gaussian-distributed noise to the training dataset and compare the results with models trained on 5\% random noise. Specifically, we replace the uniform distribution mentioned above with a Gaussian distribution characterized by a mean of 25128 and standard deviation of 500. As depicted in Figure~\ref{fig:3}(b), the loss on Gaussian noise is lower than that on the random noise. According to Proposition~\ref{pro1}, since the Gaussian distribution corresponds to a high $p_n$, we can \textbf{predict} that a model trained on Gaussian noise will exhibit a lower loss on $P^c$. Figure~\ref{fig:3}(c) confirms our prediction, thus further validating the proportions.

\begin{table*}[ht]
  \centering
  \resizebox{\textwidth}{!}{
    \begin{tabular}{c|cc|cc|cc|cc|cc}
      \toprule
                                    & \multicolumn{2}{c|}{SST-2}                            & \multicolumn{2}{c|}{SST-fine}                         & \multicolumn{2}{c|}{20newsgroup}                             & \multicolumn{2}{c|}{CR}                               & \multicolumn{2}{c}{Avg}         \\
                                    & Linear                    & MLP                       & Linear                    & MLP                       & Linear                    & MLP                       & Linear                    & MLP                       & Linear         & MLP            \\ \hline
      OpenAI's GPT-2$^*$            & 87.4                      & /                         & 49.2                      & /                         & 63.7                      & /                         & 86.8                      & /                         & 71.75          & /              \\ \hline
      0\%                           & 86.71 $\pm$ 0.85          & 87.36 $\pm$ 0.33          & 49.19 $\pm$ 0.32          & 49.18 $\pm$ 0.02          & 63.12 $\pm$ 0.37          & 62.70 $\pm$ 0.86           & 85.65 $\pm$ 0.88          & 84.86 $\pm$ 0.36          & 71.16          & 71.02          \\
      0\% + $\mathcal{L}_{gm}$      & \textbf{87.42 $\pm$ 0.73} & \textbf{87.86 $\pm$ 0.04} & \textbf{49.72 $\pm$ 0.27} & \textbf{49.81 $\pm$ 0.97} & \textbf{63.69 $\pm$ 0.59} & \textbf{62.95 $\pm$ 0.13} & \textbf{86.58 $\pm$ 0.22} & \textbf{86.45 $\pm$ 0.73} & \textbf{71.85} & \textbf{71.76} \\ \hline
      1\%                           & 87.25 $\pm$ 0.79          & \textbf{87.53 $\pm$ 0.27} & 49.32 $\pm$ 0.72          & 49.45 $\pm$ 0.56          & 63.71 $\pm$ 0.02          & 64.65 $\pm$ 0.06          & 84.86 $\pm$ 0.98          & 84.59 $\pm$ 0.59          & 71.28          & 71.55          \\
      1\% + $\mathcal{L}_{gm}$      & \textbf{87.64 $\pm$ 0.91} & 87.25 $\pm$ 0.44          & \textbf{49.59 $\pm$ 0.73} & \textbf{50.01 $\pm$ 0.05} & \textbf{63.92 $\pm$ 0.65} & \textbf{64.72 $\pm$ 0.76} & \textbf{85.12 $\pm$ 0.07} & \textbf{85.25 $\pm$ 0.29} & \textbf{71.56} & \textbf{71.80} \\ \hline
      5\%                           & 86.92 $\pm$ 0.98          & 87.23 $\pm$ 0.41           & 49.04 $\pm$ 0.11          & \textbf{50.09 $\pm$ 0.53} & 63.27 $\pm$ 0.79          & 62.09 $\pm$ 0.28          & 85.30 $\pm$ 0.63          & \textbf{84.32 $\pm$ 0.78} & 71.13          & \textbf{70.93} \\
      5\% + $\mathcal{L}_{gm}$      & \textbf{87.19 $\pm$ 1.02} & \textbf{87.61 $\pm$ 0.51} & \textbf{49.82 $\pm$ 0.17} & 48.95 $\pm$ 0.89          & \textbf{63.78 $\pm$ 0.93} & \textbf{62.37 $\pm$ 0.56} & \textbf{85.57 $\pm$ 0.43} & 84.19 $\pm$ 0.69          & \textbf{71.59} & 70.78         \\ \hline
      20\%                          & 86.60 $\pm$ 1.28           & 86.60 $\pm$ 0.81           & 49.45 $\pm$ 0.78          & 49.63 $\pm$ 0.01          & 63.47 $\pm$ 0.64          & 64.16 $\pm$ 0.92          & \textbf{85.32 $\pm$ 0.60} & \textbf{85.45 $\pm$ 0.86} & 71.26          & 71.26          \\
      20\% + $\mathcal{L}_{gm}$     & \textbf{87.2 $\pm$ 0.99}  & \textbf{86.87 $\pm$ 0.78} & \textbf{49.68 $\pm$ 0.55} & \textbf{50.40 $\pm$ 0.46}  & \textbf{63.58 $\pm$ 0.08} & \textbf{64.21 $\pm$ 0.78} & 85.25 $\pm$ 0.90          & 85.52 $\pm$ 0.24          & \textbf{71.42} & \textbf{71.75} \\ \hline
      Gaussian                      & 85.22 $\pm$ 0.24          & 86.82 $\pm$ 0.72          & 46.15 $\pm$ 0.51          & 49.59 $\pm$ 0.76          & 63.72 $\pm$ 0.35          & \textbf{64.40 $\pm$ 0.76}  & 84.06 $\pm$ 0.74          & \textbf{83.53 $\pm$ 0.70} & 69.78          & 71.08          \\
      Gaussian + $\mathcal{L}_{gm}$ & \textbf{85.94 $\pm$ 0.55} & \textbf{87.25 $\pm$ 0.36} & \textbf{48.23 $\pm$ 0.69} & \textbf{50.29 $\pm$ 0.70} & \textbf{64.06 $\pm$ 0.73} & 64.29 $\pm$ 0.94          & \textbf{84.46 $\pm$ 0.33} & 83.29 $\pm$ 0.47          & \textbf{70.67}          & \textbf{71.45}          \\
    \bottomrule
    \end{tabular}}
  \vspace{-2mm}
  \caption{Accuracy on 4 text classification benchmark. 0\% represents a model trained on $P^c$, 1\% and so on denote the proportion of random noise, and Gaussian refers to Gaussian noise. $^*$ cited from \citet{mathlm}.}
  \label{tab:1}
  \vspace{-2mm}
\end{table*}

\begin{table*}[ht]
  \centering
  \resizebox{\textwidth}{!}{
    \begin{tabular}{c|cc|cc|cc|cc|cc}
      \toprule
                                                 & \multicolumn{2}{c|}{BBC}                              & \multicolumn{2}{c|}{Balanced COPA}                    & \multicolumn{2}{c|}{MRPC}                             & \multicolumn{2}{c|}{WiC}                              & \multicolumn{2}{c}{Avg}         \\
                                                 & Linear                    & MLP                       & Linear                    & MLP                       & Linear                    & MLP                       & Linear                    & MLP                       & Linear         & MLP            \\ \hline
      Llama-3-8B                                 & 96.90 $\pm$ 0.40          & 97.50 $\pm$ 0.20          & 69.00 $\pm$ 0.20          & \textbf{65.60 $\pm$ 0.50} & 72.00 $\pm$ 0.81          & 67.53 $\pm$ 0.93          & 64.14 $\pm$ 0.56          & 59.07 $\pm$ 0.34          & 75.51          & 72.42          \\
      Llama-3-8B + $\mathcal{L}_{gm}$            & \textbf{98.00 $\pm$ 0.50} & \textbf{98.20 $\pm$ 0.40} & \textbf{70.80 $\pm$ 1.70} & 64.80 $\pm$ 0.20          & \textbf{74.89 $\pm$ 0.40} & \textbf{74.14 $\pm$ 1.49} & \textbf{64.71 $\pm$ 0.94} & \textbf{64.21 $\pm$ 0.83} & \textbf{77.10} & \textbf{75.33} \\ \hline
      Llama-3-8B-Instruct                        & 96.80 $\pm$ 0.70          & 96.90 $\pm$ 0.30          & 87.80 $\pm$ 0.70          & 88.80 $\pm$ 0.60          & 72.57 $\pm$ 0.26          & 71.42 $\pm$ 0.13          & 65.92 $\pm$ 0.53          & 61.85 $\pm$ 0.59          & 80.77          & 79.74          \\
      Llama-3-8B-Instruct + $\mathcal{L}_{gm}$   & \textbf{97.70 $\pm$ 0.20} & \textbf{97.80 $\pm$ 0.40} & \textbf{88.40 $\pm$ 0.90} & \textbf{89.60 $\pm$ 0.50} & \textbf{77.79 $\pm$ 0.58} & \textbf{76.81 $\pm$ 0.20} & \textbf{68.64 $\pm$ 0.26} & \textbf{67.71 $\pm$ 0.51} & \textbf{83.13} & \textbf{82.98} \\ \hline
      Llama-3.2-3B-Instruct                      & 97.30 $\pm$ 0.60          & 97.20 $\pm$ 0.80          & 80.40 $\pm$ 0.90          & \textbf{79.60 $\pm$ 0.20} & 77.79 $\pm$ 0.52          & 72.57 $\pm$ 0.31          & 64.07 $\pm$ 0.82          & 57.50 $\pm$ 0.35          & 79.89          & 76.71          \\
      Llama-3.2-3B-Instruct + $\mathcal{L}_{gm}$ & \textbf{97.60 $\pm$ 0.10} & \textbf{97.80 $\pm$ 0.30} & \textbf{81.60 $\pm$ 1.00} & 79.40 $\pm$ 0.10          & \textbf{78.43 $\pm$ 0.78} & \textbf{76.57 $\pm$ 1.12} & \textbf{64.35 $\pm$ 0.62} & \textbf{62.64 $\pm$ 0.07} & \textbf{80.49} & \textbf{79.10} \\ \hline
      Qwen2.5-1.5B-Instruct                      & 97.00 $\pm$ 0.30          & 96.60 $\pm$ 0.70          & 80.80 $\pm$ 0.70          & 82.20 $\pm$ 0.50           & 74.49 $\pm$ 0.71          & 73.39 $\pm$ 0.90          & 65.92 $\pm$ 0.45          & 61.64 $\pm$ 0.20          & 79.55          & 78.45          \\
      Qwen2.5-1.5B-Instruct + $\mathcal{L}_{gm}$ & \textbf{97.40 $\pm$ 0.10} & \textbf{97.20 $\pm$ 0.80} & \textbf{84.00 $\pm$ 0.90} & \textbf{83.40 $\pm$ 0.30}  & \textbf{79.65 $\pm$ 0.62} & \textbf{78.37 $\pm$ 0.84} & \textbf{67.71 $\pm$ 0.49} & \textbf{66.92 $\pm$ 0.55} & \textbf{82.19} & \textbf{81.47} \\ \hline
      Qwen2.5-7B-Instruct                        & 96.30 $\pm$ 0.30          & 96.70 $\pm$ 0.50          & 94.60 $\pm$ 0.90          & 95.80 $\pm$ 0.40          & 83.71 $\pm$ 0.92          & 76.81 $\pm$ 0.51          & 68.92 $\pm$ 0.41          & 64.92 $\pm$ 0.18          & 85.88          & 83.55          \\
      Qwen2.5-7B-Instruct + $\mathcal{L}_{gm}$   & \textbf{97.10 $\pm$ 0.80} & \textbf{97.40 $\pm$ 0.20} & \textbf{95.60 $\pm$ 0.50} & \textbf{96.00 $\pm$ 0.80} & \textbf{84.98 $\pm$ 0.12} & \textbf{83.13 $\pm$ 0.49} & \textbf{72.28 $\pm$ 0.98} & \textbf{70.14 $\pm$ 0.94} & \textbf{87.49} & \textbf{86.66} \\ \bottomrule
      \end{tabular}}
    \vspace{-2mm}
    \caption{Accuracy of LLMs on 4 natural language understanding benchmark.}
    \label{tab:2}
    \vspace{-4mm}
\end{table*}

\subsection{Experiments on Other Text Corpus}
To further investigate the impact of random noise on model generalization, we evaluate the next-token prediction loss of the trained models on data crawled from arXiv and Wikipedia. The results are illustrated in Figure~\ref{fig:4}. Surprisingly, models trained with added noise outperformed those trained on $P^c$. This counterintuitive finding aligns with previous work in visual domains \citep{purenoise}, suggesting that incorporating random noise into training sets might enhance model robustness. Additionally, we observe that the performance of models subjected to Gaussian noise varies across different datasets. These observations warrant further investigation.

\subsection{Broader Impact of the Results}
In addition to providing explanations regarding the impact of random noise on pre-training language models, we aim to extend our proposed theory to other areas, therefore demonstrating the practical value of our research findings.

One immediate direction is the training of multilingual models \citep{mlbert1, mlbert2, yb}. Clearly, tokens corresponding to different languages are distinct, and their distributions naturally satisfy Assumption~\ref{assumption1}. For example, in an English-French bilingual model, let $P^c$ represent English and $P^n$ represent French. Supposing the pre-training corpus consists of an equal distribution of English and French, and given that the two distributions are similar, we can assume that $p_c \approx p_n$, leading to $\epsilon \approx 0$. This provides a theoretical foundation for the success of multilingual models. See Appendix~\ref{app:multilin} for more details.

Beyond language modality, random white noise has received increased attention in the speech domain \citep{wavlm, giga, ydc1, uniaudio}. Since our theory applies to any cross-entropy-like loss, it can also explain why speech models pre-trained on very noisy large-scale datasets, such as Gigaspeech \citep{giga}, which contain significant background noise and prolonged silence at the beginning and end of a few samples, still perform remarkably well. 

\section{Reducing the Noise with Local Gradient Matching}
\label{sec:4}

In Section~\ref{sec:3}, we know that the influence of noise on NTP loss is rather small. However, Figure~\ref{fig:3}(c) and Table~\ref{tab:1} show that the Gaussian noise-trained model with lower NTP loss suffers a 1.5\% decrease in accuracy in downstream tasks. Although this might be mitigated during the pre-training phase, considering that most practitioners fine-tune pre-trained models rather than training them from scratch, we propose a novel black-box fine-tuning method termed Local Gradient Matching loss to tame the influence. Extensive experiments across 8 natural language understanding and 14 image classification benchmark datasets further demonstrate that the proposed method consistently enhances performance across different backbones and modalities. We also provide a theoretical analysis.

\begin{table*}[ht]
  \centering
  \resizebox{\textwidth}{!}{
  \begin{tabular}{c|cc|cc|cc|cc|cc}
  \toprule
  Model                  & \multicolumn{2}{c|}{EfficientNet-B3} & \multicolumn{2}{c|}{ResNetv2-152x2} & \multicolumn{2}{c|}{Swin-L}       & \multicolumn{2}{c|}{ConvNext-L} & \multicolumn{2}{c}{ViT-L}       \\
  Pre-training Data      & \multicolumn{2}{c|}{JFT-300M}        & \multicolumn{2}{c|}{ImageNet-21K}   & \multicolumn{2}{c|}{ImageNet-21K} & \multicolumn{2}{c|}{Laion-2B}   & \multicolumn{2}{c}{Laion-2B}    \\
  Fine-tuning Method     & Linear            & MLP              & Linear           & MLP              & Linear          & MLP             & Linear         & MLP            & Linear         & MLP            \\ \hline
  w/o $\mathcal{L}_{gm}$ & 73.27             & \textbf{76.62}            & 78.14            & 79.60            & 81.43           & 84.19           & 82.89          & 85.71          & 86.86          & 89.12          \\ \hline
  w/ $\mathcal{L}_{gm}$  & \textbf{74.02}    & 75.90   & \textbf{79.49}   & \textbf{79.94}   & \textbf{82.70}  & \textbf{84.42}  & \textbf{84.07} & \textbf{86.27} & \textbf{88.03} & \textbf{89.31} \\ \bottomrule
  \end{tabular}}
  \vspace{-2mm}
  \caption{Average accuracy of 5 vision backbone models on \textbf{14} commonly-used vision datasets.}
  \label{tab3}
  \vspace{-6mm}
\end{table*}

\subsection{Method}
In the preceding analysis, we demonstrate that the population-level loss function is only marginally affected by random noise. However, during the SGD training process, its presence introduces certain noise into the gradients. %Although stochastic gradient noise is crucial for the generalization of deep networks \citep{gradnorm2, gradnoise0}
Prior studies \citep{gradnoise1, gradnoise2} have shown that artificially added gradient noise can hurt the model's generalization. Therefore, inspired by Sharpness-Aware Minimization (SAM) \citep{sharpness0, sharpness, gradnorm1, sharpness2} and noise-robust fine-tuning methods \citep{noisebert1, noisebert2, smart}, we propose explicitly enhancing the denoising capabilities of the downstream task head by aligning local gradients.

\begin{figure}[t]
  \centering
  \includegraphics[width=\linewidth]{Sec4-Exp/fig/htff1.pdf}
  \vspace*{-6mm}
  \caption{Overview of the proposed Local Gradient Mathcing scheme.}
  \label{fig:5}
  \vspace*{-3mm}
\end{figure}

\begin{figure*}[ht]
  \centering
  \begin{subfigure}[t]{0.25\linewidth}
    \includegraphics[height=4cm, width=4cm]{Sec4-Exp/visual/47.pdf}
    \caption*{(a) No regularization.}
  \end{subfigure} \hspace{5mm}
  \begin{subfigure}[t]{0.25\linewidth}
    \includegraphics[height=4cm, width=4cm]{Sec4-Exp/visual/15.pdf}
    \caption*{(b) With $L^2$ regularization.}
  \end{subfigure} \hspace{5mm}
  \begin{subfigure}[t]{0.25\linewidth}
    \includegraphics[height=4cm, width=4cm]{Sec4-Exp/visual/68.pdf}
    \caption*{(c) With $\mathcal{L}_{gm}$.}
  \end{subfigure} %hspace{5mm}
  \begin{subfigure}[t]{0.02\linewidth}
    \includegraphics[height=4cm, keepaspectratio]{Sec4-Exp/visual/bar.pdf} % 添加新的子图
  \end{subfigure}

  \vspace*{-2.5mm}
  \caption{Visualization of input sensitivity for models trained with (a) no (b) $L^2$ (c) $\mathcal{L}_{gm}$ regularization. We randomly select a sample and introduce perturbations on a two-dimensional hyperplane, where different colors represent different labels, and green indicates the correct label.}
  \label{fig6}
  \vspace*{-4mm}
\end{figure*}

Specifically, let $C$ denote the number of classes in the downstream task, and let $g_{\theta}: \mathbb{R}^d \rightarrow \mathbb{R}^C$ represent the linear or MLP classification head parameterized by $\theta$. Let $t^*$ be the feature extracted by $h^*$, $t$ be the feature extracted by $h$, and $y$ be the corresponding label. Let $\ell(\hat{y}, y )$ be the loss function(typically cross-entropy), and $\mathcal{L}_{ce}(\mathcal{D}, g_{\theta}) = \mathbb{E}_{(t, y) \sim \mathcal{D}} \ell(g_{\theta}(t), y)$ be the population-level loss where $\mathcal{D}$ represents the joint distribution of downstream features and labels. Due to the additional randomness introduced by $h$ as a result of noise, $t$ can be viewed as $t^*$ perturbed by minor disturbances. If both $t^*$ and $t$ were known, their distribution could be aligned to achieve denoising. However, in practical applications, it is challenging to obtain $t^*$. To construct contrastive sample pairs without $t^*$, we add Gaussian noise to $t$ to obtain $\hat{t}$:
  \vspace{-2mm}
\begin{equation}
  \label{eq:perturb}
  \hat{t} = t + \gamma \cdot \delta
  \vspace{-2mm}
\end{equation}
where $\delta \sim \mathcal{N}(\mathbf{0}, \mathbf{I}_n)$ denotes the standard normal distribution noise. Our objective is to minimize the discrepancy between the distributions of $g_{\theta}(t)$ and $g_{\theta}(\hat{t})$. Instead of the conventional regularization term $||g_{\theta}(t) - g_{\theta}(\hat{t})||_2$, we propose to align the gradient difference:
  \vspace{-2mm}
\begin{equation}
  \label{eq:lgm}
  \begin{aligned}
    \mathcal{L}_{gm}(\theta) &= || \mathbb{E}_{(t, y) \sim \mathcal{D}} \nabla_{\theta}  \ell(g_{\theta}(t), y ) \\
    &\quad - \mathbb{E}_{(\hat{t}, y) \sim \hat{\mathcal{D}}} \nabla_{\theta}  \ell(g_{\theta}(\hat{t}), y ) ||_2
  \end{aligned}
    \vspace{-2mm}
\end{equation}
Intuitively, if the gradients with respect to $t$ and $\hat{t}$ can be perfectly aligned, then the classification head is insensitive to small perturbations in the input, suggesting that it possesses some denoising capability. Consequently, it should be able to mitigate the noise in $t$, bringing it closer to $t^*$.

\begin{table}[t]
  \centering
  \resizebox{0.8\linewidth}{!}{
      \begin{tabular}{c|c|c|c|c}
      \toprule
      & RTE            & MRPC           & CoLA           & STS-B          \\ \hline
      L$^2$-SP$^*$                  & 70.58          & \textbf{87.74} & 60.54          & 89.38          \\
      L$^2$-SP + $\mathcal{L}_{gm}$ & \textbf{71.25} & 87.62          & \textbf{61.79} & \textbf{89.62} \\ \hline
      SMART$^*$                     & 72.23          & 87.86          & 63.16          & 90.11          \\
      SMART + $\mathcal{L}_{gm}$    & \textbf{72.94} & \textbf{88.61} & \textbf{63.28} & \textbf{90.42} \\ \hline
      LNSR$^*$                      & 73.31          & 88.50          & 63.35          & 90.23          \\
      LNSR + $\mathcal{L}_{gm}$    & \textbf{73.95} & \textbf{89.42} & \textbf{63.82} & \textbf{90.47} \\ \bottomrule
      \end{tabular}}
    \vspace{-2mm}
    \caption{Evaluation of our method combined with SOTA fine-tuning techniques utilizing BERT-Large as the backbone model across 4 datasets. $^*$ cited from \citet{noisebert2}}
    \label{tab4}
    \vspace{-4mm}
\end{table}


\subsection{Theoretical Analysis}
\label{sec:4.2}
To theoretically support the proposed method, we investigate the properties of Equation~(\ref{eq:lgm}) and find that it can be upper bounded by the smoothness, input flatness, and loss function value at $\theta$. Concretely, since we set $\gamma$ in Equation~(\ref{eq:perturb}) to be small, the perturbation can be considered to distribute within an open ball $B(0, \rho)$. Consequently, we have the following result:
\begin{proposition}
  \label{pro2}
  Suppose $\ell(g_{\theta}(t), y)$ is $\beta$-smooth with $\rho$-input flatness $R_{\rho}(\theta)$ ({\rm{c.f.}} Appendix~\ref{app:4.2}), for any $\theta \in \Theta$:
    \vspace{-2mm}
  \begin{equation}
  \mathcal{L}_{gm}(\theta) \le 2 \beta + 2 \mathcal{L}_{ce}(\mathcal{D}, g_{\theta}) + R_{\rho}(\theta).
  \end{equation}
    \vspace{-2mm}
\end{proposition}
\vspace{-5mm}
Proposition~\ref{pro2} demonstrates that $\mathcal{L}_{gm}$ is closely associated with the smoothness of the loss function in both the parameter space and the input space. As a flat minima is widely acknowledged to benefit the generalization of neural networks \citep{flat1, flat2}, it explains the effectiveness of $\mathcal{L}_{gm}$. The final loss function is:
  \vspace{-2mm}
\begin{equation}
  \mathcal{L} = \mathcal{L}_{ce} + \lambda \mathcal{L}_{gm}
  \label{eq:final}
    \vspace{-2mm}
\end{equation}

\subsection{Experiments}
We first conduct extensive experiments using trained GPT-2 models. Then, to further validate the novelty and effectiveness of the LGM loss, we conduct additional experiments using Llama-3 \citep{llama3} and vision models. These experiments are intended to showcase the generalizability of our approach beyond the specific context of GPT-2, demonstrating its applicability across different types of models and tasks. While these experiments enrich our study, \textbf{they are not the core focus but rather supplementary evidence} supporting the broader applicability of our proposed solution. %Detailed information about the models, datasets, hyperparameters, and other experimental settings can be found in Appendix~\ref{app:exp4-detail}.
Details can be found in Appendix~\ref{app:exp4-detail}.

We validate the performance of $\mathcal{L}_{gm}$ on models pre-trained with noisy data using four commonly used classification datasets: SST-2, SST-fine, 20newsgroup, and CR. The training hyperparameters follow those of \citet{mathlm}, where $\gamma=0.01$ and $\lambda=0.15$ apply to all four experiments. In line with the approach described by \citet{nml}, we freeze the model parameters and only fine-tune a linear or MLP classifier head. As shown in Table~\ref{tab:1}, our model achieves competitive results without reaching the number of training iterations of GPT-2, and $\mathcal{L}_{gm}$ consistently boosts performance.

Results in Table~\ref{tab:2} indicate that our method provides a 3\% improvement across multiple NLU datasets with LLM backbone.
In addition, we select five commonly used backbone models in the visual domain and conduct experiments on fourteen datasets. The results are shown in Table~\ref{tab3}. It can be seen that our method is equally applicable to visual tasks, achieving a performance improvement of more than 1\% under the linear probe setting.

\begin{table}[t]
  \centering
  \resizebox{0.9\linewidth}{!}{
      \begin{tabular}{c|cc}
      \toprule
      \multirow{2}{*}{Method}                                                                                 & \multicolumn{2}{c}{SST-2}       \\
                                                                                                              & Linear         & MLP            \\ \hline
      0\%                                                                                                     & 86.71          & 87.36          \\ \hline
      0\% + $||\nabla_{\theta}  \ell(g_{\theta}(t), y )||_2$                                                  & 87.04          & 87.24          \\ \hline
      0\% + $\cos (\nabla_{\theta}  \ell(g_{\theta}(t), y ), \nabla_{\theta}  \ell(g_{\theta}(\hat{t}), y ))$ & 86.89          & 87.52          \\ \hline
      0\% + $\mathcal{L}_{gm}$                                                                                & \textbf{87.42} & \textbf{87.86} \\ \bottomrule
      \end{tabular}
    }
    \vspace{-2mm}
    \caption{Ablation Study. To investigate the effects of reducing $\mathcal{L}_{gm}$, experiments are conducted to examine the impact of separately reducing the norm versus increasing the cosine similarity.}
    \label{tab5}
    \vspace{-4mm}
\end{table}

Furthermore, we visualize the sensitivity of different regularization terms to input perturbations, as illustrated in Figure~\ref{fig6}. 
Compared with other regularization methods, our loss function can increase the size of the region for correct decisions, thereby enhancing the model's robustness to input perturbations. 
We also carry out ablation studies and parameter sensitivity analyses, with results presented in Table~\ref{tab5} and Table~\ref{tab6}, which all demonstrate the effectiveness and robustness of LGM.
% We also conduct ablation studies, the results of which are shown in Table~\ref{tab5}, demonstrating that the effectiveness of our loss function stems from its ability to simultaneously reduce the gradient norm and align the gradient distributions. 
% Finally, we perform parameter sensitivity analyses, as presented in Table~\ref{tab6}, indicating that our method is not sensitive to the two related parameters, thereby proving its robustness.

\begin{table}[t]
  \centering
  \resizebox{0.5\linewidth}{!}{
    \begin{tabular}{cc|cc}
      \toprule
      \multirow{2}{*}{$\gamma$} & \multirow{2}{*}{$\lambda$} & \multicolumn{2}{c}{DTD} \\
                                &                            & Linear      & MLP       \\ \hline
      0.001                     & 0.001                      & 76.31       & 78.54     \\ \hline
      0.05                      & 0.05                       & 76.54       & 79.51     \\ \hline
      0.1                       & 0.1                        & 76.43       & 79.12     \\ \bottomrule
      \end{tabular}
    }
    \vspace{-2mm}
    \caption{Hyperparameter sensitivity experiments on DTD with ConvNext as the backbone.}
    \label{tab6}
    \vspace{-4mm}
\end{table}


\section{Conclusion}
\label{sec:5}
In this paper, we investigate the random noise present in language model pre-training datasets, which is inevitable in real-world scenarios but receives little attention. We pre-train multiple GPT-2 models under varying noise levels and find that random noise has a minor impact on the pre-training loss. We then provide a theoretical explanation for this phenomenon and discover that our theory can elucidate the success of multilingual models. Interestingly, we observe that slight noise can sometimes enhance a model's generalization ability. Then, building on the noisy model learning setup, we propose a novel local gradient matching loss. Extensive experiments across multiple datasets in both language and vision tasks, as well as with various backbone models, validate the effectiveness of our proposed method. We hope this work inspires more researchers to focus on data-centric AI. %and pre-training dataset noise for LLMs.


\section*{Limitations}
In this section, we discuss the limitations of this paper.

Firstly, due to limitations in computational resources and costs, we pre-train only the GPT-2 124M and 774M model(see Appendix~\ref{app:gpt2774}) on the OpenWebText dataset. Compared to today's large language models, both the scale of OpenWebText and that of GPT-2 are relatively small. Additionally, the types of noise considered are limited to uniform and Gaussian distributions. However, based on Proposition~\ref{pro1}, we argue that training GPT-2 on the Synthetic OpenWebText dataset is sufficient to uncover the essence of the issue, as Proposition~\ref{pro1} makes no assumptions about data distribution or model architecture.

Secondly, on the theoretical front, we consider neural networks as black boxes and focus on analyzing the properties of global minima. Due to limited mathematical skills, we do not delve into the dynamical aspects to specifically examine how random noise within datasets influences model gradients, nor do we explore the differences between global and local minima obtained through stochastic gradient descent. However, experimental results indicate that neural networks trained with stochastic gradient descent do not suffer from significant disturbances.

%Finally, following the setup outlined by \citet{nml}, our experiments concentrate on classification tasks for black-box models. Due to limitations in page numbers and computational resources, we do not explore generative tasks like text generation. This area is reserved for future research directions.

% Bibliography entries for the entire Anthology, followed by custom entries
\bibliography{main}
% Custom bibliography entries only
% \bibliography{custom}

% \clearpage
% \onecolumngrid
\newpage
\onecolumn

\appendix

\section{Notations}
\label{sec:appa}

The commonly used notations and their descriptions are as follows.
\begin{table*}[h]
  \centering
  \begin{tabular}{|c|c|}
    \hline
    Notation                                                    & Description                                                                   \\ \hline
    $L$                                                         & context length                                                                \\ \hline
    $d$                                                         & embedding dimension                                                           \\ \hline
    $\mathcal{W}$                                               & vocabulary of words                                                           \\ \hline
    $V=|\mathcal{W}|$                                           & vocabulary size                                                               \\ \hline
    $\mathcal{X} = \cup_{i=1}^{L}\mathcal{W}^{i}$ & model input space                                                                     \\ \hline
    $\mathcal{H}$                                               & model space                                                                   \\ \hline
    $h:\mathcal{X} \rightarrow \mathbb{R}^V \in \mathcal{H}$    & language model                                                                \\ \hline
    %$f^h:\mathcal{X} \rightarrow \mathbb{R}^d$                  & feature extractor corresponding to language model $h$                         \\ \hline
    $\Delta_A$                                                  & distribution defined on a discrete set $A$                                    \\ \hline
    $P^c \in \Delta_{\mathcal{X} \times \mathcal{W}}$           & distribution of clean data                                                    \\ \hline
    $P^n \in \Delta_{\mathcal{X} \times \mathcal{W}}$           & distribution of pure noise data                                               \\ \hline
    $P^m \in \Delta_{\mathcal{X} \times \mathcal{W}}$           & distribution of mixed noisy data                                              \\ \hline
    $\alpha$                                                    & proportion of noise in training data                                          \\ \hline
    $P_X$                                                       & marginal distribution of the joint distribution $P$                           \\ \hline
    $P_{\cdot \vert X}$                                         & conditional distribution of the joint distribution $P$                        \\ \hline
    ${\boldsymbol{p}}^{h}_{\cdot \vert x}(w)$                   & the $w$-th dimension of the probability distribution corresponding to $h(x)$  \\ \hline
    ${\text{supp}}(P^c)$                                        & support of distribution $P^c$                                                 \\ \hline
    $\mathcal{L}_{ntp}(P, h)$                                   & next-token prediction loss of model $h$ on the distribution $P$               \\ \hline
    $g_{\theta}: \mathbb{R}^d \rightarrow \mathbb{R}^C$         & downstream classification head              \\ \hline
    $\theta \in \Theta$                                         & parameters of $g$               \\ \hline
    $t \in \mathcal{T}$                                         & feature of downstream task data extracted by backbone model               \\ \hline
    $y \in \mathcal{Y}$                                         & label of downstream task data               \\ \hline
    $C=|\mathcal{Y}|$                                           & number of classes of the downstream task               \\ \hline
    $\ell(\hat{y}, y)$                                             & downstream task loss function, typically cross-entropy             \\ \hline
    $\mathcal{D}$                                               & joint distribution of downstream feature and label             \\ \hline
    $\mathcal{L}_{ce}(\mathcal{D}, g_{\theta})$                 & population-level loss with downstream data distribution $\mathcal{D}$ and head $g_{\theta}$               \\ \hline
    
  \end{tabular}
  \caption{\label{app-notation}
  Nomenclature.
  }
\end{table*}

\section{Proofs}
\subsection{Explanation of Equation~(\ref{eq:huber})}
\label{app:huber}
Let $\mathcal{M}$ be a measurable space, and let $P_1$ and $P_2$ be probability measures defined on this space. We assume that $N_1$ samples are drawn from $P_1$ and $N_2$ samples from $P_2$. Define $\mu = \frac{N_1}{N_1 + N_2}$, so that $1 - \mu = \frac{N_2}{N_1 + N_2}$.

We aim to show that this collection of $ N_1 + N_2 $ samples can be regarded as drawn from a mixed distribution $$ P_3 = \mu P_1 + (1 - \mu) P_2 $$

First, define a new probability measure $ P_3 $ as $P_3(A) = \alpha P_1(A) + (1 - \alpha) P_2(A)$ for any measurable set $A \subseteq \mathcal{M}$. Here, $P_3$ is a convex combination of $P_1$ and $P_2$, and thus $P_3$ is also a valid probability measure \citep{gtm}.

For any measurable set $ A \subseteq \mathcal{M}$, we examine the probability that a single sample point falls in $A$ by law of total probability:
\begin{enumerate}[label=\textbullet]
  \item A sample from $P_1$ is selected with probability $\mu$, and within this case, the probability of landing in $ A $ is $ P_1(A) $.
  \item A sample from $P_2$ is selected with probability $1 - \mu$, and the probability of it falling in $A$ is $ P_2(A) $.
\end{enumerate}

Thus, the probability of any given sample point falling in $ A $ is
$$
\mu P_1(A) + (1 - \mu) P_2(A) = P_3(A)
$$

% \textbf{Step 3: Equivalence of Sample Distribution.}

Since $N_1$ samples are drawn from $P_1$ and $N_2$ samples from $P_2$, these samples collectively follow the distribution $ P_3 $ as each individual sample's probability of being in any measurable set $ A $ is consistent with $ P_3(A) $. Therefore, drawing $N_1 + N_2$ samples in this manner is equivalent to drawing $N_1 + N_2$ samples from $ P_3 $.

\subsection{Proof of Proposition~\ref{pro1}}
\label{proof_pro1}

Before procedding to the proof, we first establish a useful lemma.
\begin{lemma}
\label{lemma1}
If Assumption \ref{assumption1} holds, then for any $h \in \mathcal{H}$, we have 
$$ \mathcal{L}_{ntp}(P^m, h) = \alpha \mathcal{L}_{ntp}(P^n, h)+(1-\alpha)\mathcal{L}_{ntp}(P^c, h) $$
\end{lemma}

\begin{proof}
  Let $x_i, \, i=1, 2, \ldots, |\mathcal{X}|$ denote all prefixes, and $w_j, \, j=1, 2, \ldots, V$ denote all tokens. For all $x \in \mathcal{X}$, by Equation~(\ref{eq:huber}), we have: 
  
  \begin{align}
  P^m_X(x) &= \sum_{j=1}^{V} P^m(x, w_j) = \sum_{j=1}^{V} \alpha P^n(x, w_j) + (1-\alpha) P^c(x, w_j) \notag \\
  &= \alpha \sum_{j=1}^{V} P^n(x, w_j) + (1-\alpha) \sum_{j=1}^{V} P^c(x, w_j) = \alpha P^n_X(x) + (1-\alpha) P^c_X(x) \label{eq:marginal}
  \end{align}
  This indicates that the marginal distribution possesses additivity. Consequently,
  \begin{align}
  \mathcal{L}_{ntp}(P^m, h) &= \mathbb{E}_{x \sim P^m_X} \mathbb{E}_{w \sim P^m_{\cdot | x}}-\log(\boldsymbol{p}^h_{\cdot | x}(w)) 
   = \sum_{i=1}^{|\mathcal{X}|} P^m_X(x_i) \cdot \mathbb{E}_{w \sim P^m_{\cdot | x_i}} -\log(\boldsymbol{p}^h_{\cdot | x_i}(w)) \notag \\
   &= \sum_{i=1}^{|\mathcal{X}|} [(1-\alpha) P^c_X(x_i) + \alpha P^n_X(x_i)]\cdot \mathbb{E}_{w \sim P^m_{\cdot | x_i}} -\log(\boldsymbol{p}^h_{\cdot | x_i}(w)) \tag*{(Equation~(\ref{eq:marginal}))} \\
   &= (1-\alpha)  \sum_{i=1}^{|\mathcal{X}|} P^c_X(x_i)\mathbb{E}_{w \sim P^m_{\cdot | x_i}} -\log(\boldsymbol{p}^h_{\cdot | x_i}(w)) + \alpha  \sum_{i=1}^{|\mathcal{X}|} P^n_X(x_i)\mathbb{E}_{w \sim P^m_{\cdot | x_i}} -\log(\boldsymbol{p}^h_{\cdot | x_i}(w)) \label{eq1}
   \end{align}
   The conditional distributions do not generally exhibit a linear relationship:
  $$P^m_{\cdot | x}(w|x) = \frac{P^m(x,w)}{P^m_X(x)}=\frac{(1-\alpha)P^c(x,w)+\alpha P^n(x,w)}{(1-\alpha)P^c_X(x)+\alpha P^n_X(x)} \ne  P^c_{\cdot | x}(w|x) \ne P^n_{\cdot | x}(w|x)$$
  However, if ${\text{supp}}(P^c) \cap {\text{supp}}(P^n)=\emptyset$, it immediately follows that:
  $$
    P^m_{\cdot | x}(w|x)=\frac{(1-\alpha)P^c(x,w)+\alpha P^n(x,w)}{(1-\alpha)P^c_X(x)+\alpha P^n_X(x)} = \begin{cases} 
      P^c_{\cdot | x}(w|x) & \text{if } (x,w) \in {\text{supp}}(P^c), \\
      P^n_{\cdot | x}(w|x) & \text{if } (x,w) \in {\text{supp}}(P^n).
    \end{cases} 
  $$
  Consequently,
  \begin{gather}
    \sum_{i=1}^{|\mathcal{X}|} P^c_X(x_i)\mathbb{E}_{w \sim P^m_{\cdot | x_i}} -\log(\boldsymbol{p}^h_{\cdot | x_i}(w)) 
    = \sum_{i=1}^{|\mathcal{X}|} P^c_X(x_i)\mathbb{E}_{w \sim P^c_{\cdot | {x_i}}} -\log(\boldsymbol{p}^h_{\cdot | x_i}(w)) 
    = \mathcal{L}_{ntp}(P^c, h) \label{eq2}
  \end{gather}
  Similarly,
  \begin{align}
  \sum_{i=1}^{|\mathcal{X}|} P^n_X(x_i)\mathbb{E}_{w \sim P^m_{\cdot | x_i}} -\log(\boldsymbol{p}^h_{\cdot | x_i}(w)) = \mathcal{L}_{ntp}(P^n, h) \label{eq3}
  \end{align}
  By substituting Equation~(\ref{eq2}) and Equation~(\ref{eq3}) into Equation~(\ref{eq1}), the proof is completed.
\end{proof}

Now we can prove Proposition~\ref{pro1}.
\begin{restatedproposition}{\ref{pro1}}
  Under Assumption~\ref{assumption1}, let $h^*$ be a model trained on $P^c$, with $\mathcal{L}_{ntp}(P^c,h^*)=-\log p_c$ and $\mathcal{L}_{ntp}(P^n,h^*)=-\log p_n$. When the model $h$ is trained on a mixed distribution $P^m$ which includes noise, it attempts to fit $P^n$, leading to an increase in the loss on the clean distribution $P^c$, such that $\mathcal{L}_{ntp}(P^c,h)=-\log(p_c-\epsilon)$ and $\mathcal{L}_{ntp}(P^n,h)=-\log(p_n+\epsilon/k)$ for some $\epsilon > 0$ ($k$ can be shown to be $\Omega(e^{\mathcal{L}_{ntp}(P^n,h)})$). Let $\eta = \alpha p_c - (1-\alpha)kp_n$.
We arrive at the following conclusions:

(1) If $\alpha \leq \frac{kp_n}{p_c + kp_n}$, then for any $0 < \epsilon <p_c$, we have $\mathcal{L}_{ntp}(P^m, h) \ge \mathcal{L}_{ntp}(P^m, h^*)$. This means that when $\alpha$ is sufficiently small, the global minimum on $P^m$ will not be affected by noise.

(2) If $\alpha > \frac{kp_n}{p_c + kp_n}$, then for $\epsilon < \eta$, it holds that $\mathcal{L}_{ntp}(P^m,h) < \mathcal{L}_{ntp}(P^m,h^*)$. This suggests that if $\alpha$ is large enough, the impact on the optimal hypothesis is at least as much as $\alpha p_c - (1-\alpha)kp_n$.

(3) When $\alpha < \frac{1}{3}$ and $k>\frac{\alpha(1-3\alpha)p_c}{(1-\alpha)(2-3\alpha)p_n}$, for $\epsilon \ge 3\eta$ we get $\mathcal{L}_{ntp}(P^m,h^*) < \mathcal{L}_{ntp}(P^m,h)$. Similarly, it can be shown that $\epsilon$ does not exceed $2\eta$ when $\alpha > \max(\frac{kp_n}{p_c + kp_n}, \frac{1}{2})$ and $k > \frac{(2\alpha-1)p_c}{2(1-\alpha)p_n}$. This indicates that when $k$ is sufficiently large, the effect of noise is at most $\mathcal{O}(\alpha p_c - (1-\alpha)kp_n)$.
\end{restatedproposition}

\begin{proof}
  We first establish that $k$ is $\Omega(e^{\mathcal{L}_{ntp}(P^n,h)})$, thereby ensuring that $\eta \ll \alpha p_c$. Note that
  \begin{align}
    \epsilon &= \frac{1}{e^{\mathcal{L}_{ntp}(P^c, h)}} - \frac{1}{e^{\mathcal{L}_{ntp}(P^c, h)}} = \frac{e^{\mathcal{L}_{ntp}(P^c, h)-\mathcal{L}_{ntp}(P^c, h)}-1}{e^{\mathcal{L}_{ntp}(P^c, h)}} \\
    \frac{\epsilon}{k} &= \frac{1}{e^{\mathcal{L}_{ntp}(P^n, h)}} - \frac{1}{e^{\mathcal{L}_{ntp}(P^n, h)}} = \frac{e^{\mathcal{L}_{ntp}(P^n, h)-\mathcal{L}_{ntp}(P^n, h)}-1}{e^{\mathcal{L}_{ntp}(P^n, h)}}
  \end{align}
  Therefore
  \begin{align}
   k = \frac{\epsilon}{\frac{\epsilon}{k}} &= e^{\mathcal{L}_{ntp}(P^n, h)-\mathcal{L}_{ntp}(P^c, h)} \cdot \frac{e^{\mathcal{L}_{ntp}(P^c, h)-\mathcal{L}_{ntp}(P^c, h)}-1}{e^{\mathcal{L}_{ntp}(P^n, h)-\mathcal{L}_{ntp}(P^n, h)}-1} \notag \\
   &> e^{\mathcal{L}_{ntp}(P^n, h)-\mathcal{L}_{ntp}(P^c, h)} \cdot \frac{\mathcal{L}_{ntp}(P^c, h)-\mathcal{L}_{ntp}(P^c, h)}{e^{\mathcal{L}_{ntp}(P^n, h)-\mathcal{L}_{ntp}(P^n, h)}} \notag \\
    &= %e^{\mathcal{L}_{ntp}(P^n, h)-\mathcal{L}_{ntp}(P^c, h)} \cdot (\mathcal{L}_{ntp}(P^c, h)-\mathcal{L}_{ntp}(P^c, h)) = 
    e^{\mathcal{L}_{ntp}(P^n, h)} \cdot \frac{\mathcal{L}_{ntp}(P^c, h)-\mathcal{L}_{ntp}(P^c, h)}{e^{\mathcal{L}_{ntp}(P^c, h)}} \label{eq4}
  \end{align}
  where $\frac{\mathcal{L}_{ntp}(P^c, h)-\mathcal{L}_{ntp}(P^c, h)}{e^{\mathcal{L}_{ntp}(P^c, h)}}$ only depends on $P^c$, $h$ and $h$. It is worth noting that when \( P^n \) is random noise, \( e^{\mathcal{L}_{ntp}(P^n, h)-\mathcal{L}_{ntp}(P^n, h)}-1 \) is close to $0$, which leads to \( k \) exceeding the lower bound established in Equation~(\ref{eq4}). Then:
  
  (1) If $\alpha \leq \frac{kp_n}{p_c + kp_n}$, we have:
  \begin{align}
    \mathcal{L}_{ntp}(P^m, h^*)-\mathcal{L}_{ntp}(P^m, h) &= (1-\alpha)(\mathcal{L}_{ntp}(P^c, h^*)-\mathcal{L}_{ntp}(P^c, h)) + \alpha (\mathcal{L}_{ntp}(P^n, h^*)-\mathcal{L}_{ntp}(P^n, h)) \notag \\
    &= (1-\alpha) \log \frac{p_c-\epsilon}{p_c} + \alpha \log \frac{p_n + \frac{\epsilon}{k}}{p_n} \label{eq:pro1_2} \\
    &\le (1-\alpha) \cdot \frac{-\epsilon}{p_c} + \alpha \cdot \frac{\frac{\epsilon}{k}}{p_n} \tag{$\log (1 + t) \le t$} \\
    &= \epsilon[\frac{(\alpha - 1)}{p_c} + \frac{\alpha}{kp_n}] = \epsilon \frac{\alpha p_c - (1-\alpha)kp_n}{k p_c p_n}
  \end{align}
  As $\alpha \leq \frac{kp_n}{p_c + kp_n} \iff \alpha p_c - (1-\alpha)kp_n \le 0$, for $\epsilon > 0$ we have $\mathcal{L}_{ntp}(P^m, h^*) \le \mathcal{L}_{ntp}(P^m, h)$.

  (2)when $\alpha > \frac{k p_n}{p_c + k p_n}$ and $\epsilon < \alpha p_c - (1 - \alpha)kp_n$, we have 
  \begin{align}
    \mathcal{L}_{ntp}(P^m, h^*)-\mathcal{L}_{ntp}(P^m, h) &= (1-\alpha) \log \frac{p_c-\epsilon}{p_c} + \alpha \log \frac{p_n + \frac{\epsilon}{k}}{p_n} \tag{Equation~(\ref{eq:pro1_2})}\\
    &\ge (1-\alpha) \frac{-\epsilon}{p_c-\epsilon} + \alpha \frac{\frac{\epsilon}{k}}{p_n+\frac{\epsilon}{k}} \tag{$\log t \ge 1 - \frac{1}{t}$} \\
    &= \epsilon (\frac{\alpha-1}{p_c-\epsilon} + \frac{\alpha}{kp_n+\epsilon}) \notag \\
    &= \frac{\epsilon}{(p_c-\epsilon)(kp_n+\epsilon)}[\alpha(p_c-\epsilon) - (1-\alpha)(kp_n+\epsilon)] \notag \\
    &= \frac{\epsilon}{(p_c-\epsilon)(kp_n+\epsilon)}[\alpha p_c - (1-\alpha)kp_n - \epsilon] \label{eq:lastpro1}
  \end{align}
  As $\epsilon < \alpha p_c - (1 - \alpha)kp_n < \alpha p_c < p_c$, by Equation~(\ref{eq:lastpro1}) we have $\mathcal{L}_{ntp}(P^m, h)-\mathcal{L}_{ntp}(P^m, h) > 0$. 

  (3) Let 
  \begin{equation}
  f(\epsilon) = (1-\alpha) \log \frac{p_c-\epsilon}{p_c} + \alpha \log \frac{p_n + \frac{\epsilon}{k}}{p_n} \overset{p_n'=kp_n}{=} (1-\alpha) \log \frac{p_c-\epsilon}{p_c} + \alpha \log \frac{p_n' + \epsilon}{p_n'}
  \end{equation}

  Take the derivative of $f(\epsilon)$:
  \begin{align}
    f'(\epsilon) &= (1-\alpha) \frac{-\frac{1}{p_c}}{1-\frac{\epsilon}{p_c}} + \alpha \frac{\frac{1}{p_n'}}{1+\frac{\epsilon}{p_n'}} = (1-\alpha) \frac{1}{\epsilon - p_c} + \alpha \frac{1}{p_n'+\epsilon} = \frac{[\alpha p_c-(1-\alpha)p_n']-\epsilon}{(p_c-\epsilon)(p_n'+\epsilon)}
  \end{align}

  Without loss of generality, assume $\eta > 0$, then $f(\epsilon)$ is monotonically increasing on $[0, \eta)$ and monotonically decreasing on  $(\eta, p_c)$. Therefore, to prove that $\mathcal{L}_{ntp}(P^m,h^*) < \mathcal{L}_{ntp}(P^m,h)$ for $\epsilon \geq 3\eta$, we only need to show $f(3\eta) < 0$ when $k>\frac{\alpha(1-3\alpha)p_c}{(1-\alpha)(2-3\alpha)p_n}$. Notice that 
  \begin{align}
    f(3\eta) &= (1-\alpha) \log(1-\frac{3\alpha p_c-3(1-\alpha)p_n'}{p_c}) + \alpha \log(1+\frac{3\alpha p_c-3(1-\alpha)p_n'}{p_n'}) \notag \\
    &= (1-\alpha) \log(1-3\alpha+\frac{3(1-\alpha)}{\frac{p_c}{p_n'}}) + \alpha \log(3\alpha-2+3\alpha \frac{p_c}{p_n'})
  \end{align}

  Let 
  \begin{equation}
    g_{3}(t)= (1-\alpha) \log(1-3\alpha+\frac{3(1-\alpha)}{t}) + \alpha \log(3\alpha-2+3\alpha t)
  \end{equation}
  Take the derivative:
  \begin{align}
    g'_3(t) &= (1-\alpha) \frac{1}{1-3\alpha+\frac{3(1-\alpha)}{t}} \frac{3(\alpha-1)}{t^2} + \alpha \frac{3\alpha}{3\alpha-2+3\alpha t} \\
    &= \frac{-3(1-\alpha)^2}{(1-3\alpha)t^2+(1-\alpha)t} + \frac{3\alpha^2}{3\alpha-2+3\alpha t} \\
    &= \frac{-3(1-\alpha)^2(3\alpha-2+3\alpha t) + 3\alpha^2 [(1-3\alpha)t^2+(1-\alpha)t]}{[(1-3\alpha)t^2+(1-\alpha)t](3\alpha-2+3\alpha t)} \\
    &= \frac{[\alpha t + (\alpha - 1)] [3 \alpha (1-3\alpha) t + 3(1-\alpha)(3\alpha - 2)]}{[(1-3\alpha)t^2+(1-\alpha)t](3\alpha-2+3\alpha t)}
  \end{align}
  First, consider the denominator. Since $\alpha < \frac{1}{3}$, it is clear that $(1-3\alpha)t^2 + (1-\alpha)t > 0$. Given that $t = \frac{p_c}{p_n'} > \frac{1-\alpha}{\alpha}$ (because $\eta > 0$), it follows that $3\alpha - 2 + 3\alpha t > 1 > 0$. Therefore, the denominator is always positive.
  Next, we consider the numerator. Since $\eta > 0$, it follows that $\alpha t + (\alpha - 1) > 0$. Therefore, when $t = \frac{p_c}{p_n'}=\frac{p_c}{k p_n} < \frac{(1-\alpha)(2-3\alpha)}{\alpha (1-3\alpha)}$, we have $g'_3(t) < 0$. This means that $g_3(t)$ is monotonically decreasing on $(\frac{1-\alpha}{\alpha}, \frac{(1-\alpha)(2-3\alpha)}{\alpha (1-3\alpha)}]$. Consequently, $f(3\eta) = g_3(t) \le g_3\left(\frac{1-\alpha}{\alpha}\right) = 0$.

  Following the same line of reasoning, when $\alpha > \frac{1}{2}$, we have
  \begin{align}
    f(2\eta) &= (1-\alpha) \log(1-\frac{2\alpha p_c-2(1-\alpha)p_n'}{p_c}) + \alpha \log(1+\frac{2\alpha p_c-2(1-\alpha)p_n'}{p_n'}) \notag \\
    &= (1-\alpha) \log(1-2\alpha+\frac{2(1-\alpha)}{\frac{p_c}{p_n'}}) + \alpha \log(2\alpha-1+2\alpha \frac{p_c}{p_n'})
  \end{align}

  Let
  \begin{equation}
    g_{2}(t)= (1-\alpha) \log(1-2\alpha+\frac{2(1-\alpha)}{t}) + \alpha \log(2\alpha-1+2\alpha t)
  \end{equation}
  Take the derivative:
  \begin{align}
    g'_2(t) &= (1-\alpha) \frac{1}{1-2\alpha+\frac{2(1-\alpha)}{t}} \frac{2(\alpha-1)}{t^2} + \alpha \frac{2\alpha}{2\alpha-1+2\alpha t} \\
    &= \frac{-2(1-\alpha)^2}{(1-2\alpha)t^2+2(1-\alpha)t} + \frac{2\alpha^2}{2\alpha-1+2\alpha t} \\
    &= \frac{2(1-2\alpha)(\alpha t + 1-\alpha)^2}{[(1-2\alpha)t^2+2(1-\alpha)t](2\alpha-1+2\alpha t)}
  \end{align}
  Therefore, when $\frac{1-\alpha}{\alpha} < t < \frac{2(1-\alpha)}{2\alpha-1}$, we have $g'_2(t) < 0$, which implies that $f(2\eta) < 0$.

\end{proof}

% \subsection{Proof of Corollary~\ref{cor1}}
% Next up, we give a proof to Corollary~\ref{cor1}.

% \begin{restatedcorollary}{\ref{cor1}}
%   Building upon Proposition~\ref{pro1}, if $\alpha > \frac{k p_n}{p_c + k p_n}$, then for $\epsilon < \mathcal{O}(\alpha p_c - (1 - \alpha)k \tilde{p}_n)$, it follows that $\mathcal{L}_{ntp}(P^m,h) < \mathcal{L}_{ntp}(P^m,h)$.
% \end{restatedcorollary}

% \begin{proof}
%   Replacing $k p_n$ in Equation~(\ref{eq:lastpro1}) with $k \tilde{p}_n - \epsilon$, we have:
%   \begin{align}
%     \mathcal{L}_{ntp}(P^m, h)-\mathcal{L}_{ntp}(P^m, h) &> \frac{\epsilon}{(p_c-\epsilon)(kp_n+\epsilon)}[\alpha p_c - (1-\alpha)kp_n - \epsilon] \tag{Equation~(\ref{eq:lastpro1})}\\
%     &= \frac{\epsilon}{(p_c-\epsilon)(kp_n+\epsilon)}[\alpha p_c - (1-\alpha)(k \tilde{p}_n-\epsilon)-\epsilon] \notag \\
%     &= \frac{\epsilon}{(p_c-\epsilon)(kp_n+\epsilon)}[\alpha p_c - (1-\alpha)k \tilde{p}_n - \alpha \epsilon]
%   \end{align}
%   Consequently, for $\epsilon < \frac{1}{\alpha}(\alpha p_c - (1-\alpha)k \tilde{p}_n)$, it holds that $\mathcal{L}_{ntp}(P^m, h) - \mathcal{L}_{ntp}(P^m, h) > 0$, thereby implying that $\epsilon$ is $\mathcal{O}(\alpha p_c - (1 - \alpha)k \tilde{p}_n)$.
% \end{proof}

\subsection{Justification of Proposition~\ref{pro1}}
\label{app:explain}

\begin{figure*}
  \centering
  \begin{subfigure}{0.37\linewidth}
    \includegraphics[height=5.2cm, width=\linewidth]{Sec3-Exp/appendix_data/app_k.pdf}
    \caption*{(a)}
  \end{subfigure}
  \begin{subfigure}{0.55\linewidth}
    \includegraphics[height=5.3cm, width=\linewidth]{Sec3-Exp/appendix_data/app_loss_diff1.pdf}
    \caption*{(b)}
  \end{subfigure}

  \caption{Visualization of $k$ and $\mathcal{L}_{ntp}(P^m, h^*) - \mathcal{L}_{ntp}(P^m, h)$. (a) The trend of $k$ as it changes with training, plotted using the model trained on $P^c$ as $h^*$. (b) Visualization of $\mathcal{L}_{ntp}(P^m, h)$ when the parameter settings are consistent with the experiment.}
  \label{fig:app:k}

\end{figure*}

We plot the trend of $k$ in Figure~\ref{fig:app:k}(a). We compare checkpoints trained for the same iterations on both $P^c$ and $P^m$, where $p_c$ is calculated based on the loss of the model trained on $P^c$, and $p_n$ is determined by the loss of a model trained for 10,000 iterations on $P^m$ when evaluated on $P^n$. It can be observed that the value of $k$ corresponding to random noise is significantly greater than one, which supports the rationality of the assumption made in Proposition~\ref{pro1}.

On the other hand, to extend the proposed theory beyond uniformly distributed random noise (for instance, in multilingual models or Gaussian noise), it is necessary to ensure that $k$ does not become too small in these scenarios. This means that $\mathcal{L}_{ntp}(P^n,h^*)=-\log p_n$ should not be close to $\log V$. One trivial way to increase $p_n$ is to decrease V, the size of vocabulary. Apart from this, we provide two lines of reasoning to justify why $p_n$ can be made large:

(1) Numerous studies on compressing large language models, such as pruning \citep{llmjz1, llmjz2, llmjz3}, quantization \citep{llmlh1, llmlh2, llmlh3}, and distillation \citep{llmzl1, zl2}, have demonstrated that there exists a significant amount of redundancy within the parameters of large models. Therefore, we could first train a model on $P^c$ and then compress it, fine-tuning the surplus parameters on $P^n$. This approach would allow us to improve $p_n$ without altering $p_c$.

(2) A small proportion of data corresponding to $P^n$ can be introduced into $P^c$, making sure that $\alpha$ is extremely small. According to domain adaptation theory \citep{ben-david}, this would only slightly increase $\mathcal{L}_{ntp}(P^c)$. However, existing results \citep{mlbt1, mlbert1, mlbert2} indicate that pre-trained models like BERT or GPT on English text can exhibit strong multilingual capabilities with just a very limited amount of data. Consequently, compared to a model trained solely on $P^c$, the resulting model has a minor difference in $p_c$ but a relatively higher $p_n$.

Both thought experiments above demonstrate that there exist a lot of models within the parameter space $\mathcal{H}$ can perform well on $P^c$ while yielding non-trivial outcomes on $P^n$. Thus, we can ensure that models trained on mixed data distributions will have a sufficiently large $k$.

Additionally, in Figure~\ref{fig:app:k}(b), we illustrate how $\mathcal{L}_{ntp}(P^m)$ varies with changes in $\epsilon$, under settings identical to those used during pre-training. The results depicted in the figure are consistent with our theoretical derivations.


\subsection{Omitted Details in Section~\ref{sec:4.2}}
\label{app:4.2}

\begin{definition}[$\beta$-smooth \citep{unlabeled}]
  \label{def:smooth}
  A loss function $\ell(g_{\theta}(t), y)$ is $\beta$-smooth, if for any $(t, y) \in \mathcal{T} \times \mathcal{Y}$ and any $\theta, \theta' \in \Theta$,
  \begin{equation}
    ||\nabla_{\theta} \ell(g_{\theta}(t), y ) - \nabla_{\theta'} \ell(g_{\theta'}(t), y)||_2 \le \beta ||\theta - \theta'||_2
  \end{equation}
\end{definition}

\begin{definition}[$\rho$-input flatness]
  \label{def:flat}
  The $\rho$-input flatness $R_{\rho}(\theta)$ of loss function $ \ell(g_{\theta}(t), y )$ is defined as:
  \begin{equation}
    R_{\rho}(\theta) = \mathbb{E}_{(t, y) \sim \mathcal{D}} \sup_{\delta' \in B(0, \rho)}  \ell(g_{\theta}(t + \delta'), y ) -  \ell(g_{\theta}(t), y )
  \end{equation}
  where $B(0, \rho) = \{\delta' : ||\delta'||_2 < \rho\}$ is a open ball.
\end{definition}

\begin{lemma}
  \label{lemma2}
  If the loss function $\ell(g_{\theta}(t), y)$ is $\beta$-smooth, then 
  \begin{equation}
    ||\nabla_{\theta} \ell(g_{\theta}(t), y)||^2_2 \le 4 \beta \ell(g_{\theta}(t), y)
  \end{equation}
\end{lemma}

\begin{proof}
  See Lemma 3.1 in \citet{smoothness}.
\end{proof}

\begin{restatedproposition}{\ref{pro2}}
  Suppose $\ell(g_{\theta}(t), y)$ is $\beta$-smooth with $\rho$-input flatness $R_{\rho}(\theta)$, for any $\theta \in \Theta$:
  \begin{equation}
  \mathcal{L}_{gm}(\theta) \le 2 \beta + 2 \mathcal{L}_{ce}(\mathcal{D}, g_{\theta}) + R_{\rho}(\theta)
  \end{equation}
\end{restatedproposition}

\begin{proof}
  \begin{align}
    \mathcal{L}_{gm}(\theta) &= || \mathbb{E}_{(t, y) \sim {\mathcal{D}}} \nabla_{\theta}  \ell(g_{\theta}(t), y ) - \mathbb{E}_{(\hat{t}, y) \sim \hat{\mathcal{D}}} \nabla_{\theta}  \ell(g_{\theta}(\hat{t}), y ) ||_2 \\
    &\le || \mathbb{E}_{(t, y) \sim {\mathcal{D}}} \nabla_{\theta}  \ell(g_{\theta}(t), y )||_2 + ||\mathbb{E}_{(\hat{t}, y) \sim \hat{\mathcal{D}}} \nabla_{\theta}  \ell(g_{\theta}(\hat{t}), y ) ||_2 \tag{Triangle Inequality} \\
    &\le \mathbb{E}_{(t, y) \sim {\mathcal{D}}} ||\nabla_{\theta}  \ell(g_{\theta}(t), y )||_2 + \mathbb{E}_{(\hat{t}, y) \sim \hat{\mathcal{D}}} ||\nabla_{\theta}  \ell(g_{\theta}(\hat{t}), y ) ||_2 \tag{Jensen's Inequality} \\
    &\le \mathbb{E}_{(t, y) \sim {\mathcal{D}}} 2 \sqrt{\beta  \ell(g_{\theta}(t), y )} + \mathbb{E}_{(\hat{t}, y) \sim \hat{\mathcal{D}}} 2 \sqrt{\beta  \ell(g_{\theta}(\hat{t}), y )} \tag{Lemma~\ref{lemma2}} \\
    &\le \mathbb{E}_{(t, y) \sim {\mathcal{D}}} (\beta+ \ell(g_{\theta}(t), y )) + \mathbb{E}_{(\hat{t}, y) \sim {\hat{\mathcal{D}}}} (\beta+ \ell(g_{\theta}(\hat{t}), y )) \tag{AM-GM Inequality} \\
    &= 2\beta + 2 \mathbb{E}_{(t, y) \sim {\mathcal{D}}}  \ell(g_{\theta}(t), y ) + (\mathbb{E}_{(\hat{t}, y) \sim {\hat{\mathcal{D}}}}  \ell(g_{\theta}(\hat{t}), y ) - \mathbb{E}_{(t, y) \sim {\mathcal{D}}}  \ell(g_{\theta}(t), y )) \\
    &\le 2\beta + 2 \mathcal{L}_{ce}(\mathcal{D}, g_{\theta}) + R_{\rho}(\theta)
  \end{align}
  where the last inequality holds because $\hat{t} - t \in B(0, \rho)$.
\end{proof}

\section{Detailed Related Works}
\label{app:relatedwork}
\textbf{Data selection for language model training.} LLMs \citep{ydc2, zxw1, zxw2, xyx1, xyx2, wzy1, cmr1} have fundamentally change the landsape of current AI research \citep{xyx3, cmr2, zxw3, zxw4, ydc3, ydc4}. Large text corpora form the backbone of language models, with data quality being fundamental to their success and safety \citep{yyg1, yyg2}. \citet{wimbd} conducted a systematic analysis of open-source text datasets such as The Pile \citep{pile} (used to train Pythia), C4 \citep{c4} (used to train T5) and RedPajama (used to train LLaMA), revealing that they contain a significant amount of duplicate, toxic, synthetic, and low-quality content. Therefore, it is of great importance to thoroughly understand the impact of low-quality data within these pre-training datasets on the model's performance, reliability, and safety. \citet{allenphysics, allen33} systematically investigated the effect of low-quality data and found that such data can significantly reduce the model's knowledge capacity, sometimes by up to 20 times. Another research direction primarily focuses on the synthetic data of large language models, specifically examining the impacts of using data generated by LLMs for recursive training. The study by \citet{naturemc} was the first to explore this issue and introduced the concept of "model collapse", indicating that recursive training can lead to the loss of information in tail tokens, ultimately resulting in the model producing nonsensical content. \citet{colmmc} mainly provided a theoretical explanation for why model collapse occurs, supporting their arguments with experimental evidence. Consequently, the importance of data selection cannot be overstated. Given that data selection is an NP-hard problem in terms of combinatorial optimization \citep{xmy}, numerous heuristic algorithms have been proposed to expedite the process. \citet{pretrainguide} provided a comprehensive study on pre-training data selection and optimal ratios, offering practical recommendations. \citet{xzk1} proposed dataset pruning, an approach that assesses the impact of omitting training samples on model generalization and creates a minimal training subset with a controlled generalization gap. \citet{gptinfluence} evaluated the impact of individual training samples on the dynamics of GPT model training. \citet{quality_naacl} introduced the Instruction-Following Difficulty metric to assess the quality of instruction-tuning data. \citet{xie2023data} employed importance resampling for data selection. \citet{doremi, beyond} advocated for optimizing data composition and diversity. Despite these notable studies on data selection, they generally acknowledge that dataset noise degenerates model performance but lack a detailed understanding of how and to what extent, particularly in the case of random noise which is inevitable in large-scale datasets. Although \citet{gibberish} investigated the influence of gibberish input, the random noise within the pre-training dataset is still underexplored. This paper aims to bridge the gap.

\textbf{Learning from Noisy Distributions.} The majority of machine learning algorithms assume that training and test samples are independently and identically distributed (i.i.d.), a condition that is often not met in real-world scenarios. For instance, LLMs are pre-trained on datasets with all kinds of noise while their performance is evaluated by the user whose distribution is usually clean and meets real-world scenarios, which violates the i.i.d. assumption. Domain adaptation \citep{yzq, mlc1, zcy1} addresses this issue when the distribution of the training data differs from that of the test data. Although domain adaptation methods attempt to reduce the statistical distribution discrepancy \citep{zkd1, kkw1} or employ adversarial training \citep{jjl1, iosda} to minimize the gap between source and target domains, they typically require access to unlabeled test data under a semi-supervised learning setup, which is impractical for LLM training. Another reason domain adaptation cannot be directly applied here is that domain adaptation theory \citep{ben-david} focuses on the performance of a model trained on one distribution when it is applied to another different but related distribution. This kind of bounds can be easily derived by Lemma~\ref{lemma1}. However, what we aim to investigate here is the extent of performance loss when comparing a model trained on one distribution (noisy dataset) to a model trained on another distribution (clean dataset).

Apart from domain adaptation, there has been extensive research directly investigating noisy training sets. Noisy label learning \citet{ls1, ls2} have explored the impact of incorrect labels on model performance. Regarding input feature noise, \citet{smoothgrad} added perturbations to individual image inputs to enhance model interpretability, and \citet{purenoise} added white noise image into the training dataset to tackle the class imbalance problem. However, most of these efforts have concentrated on image classification and do not consider the pre-training paradigm.

\textbf{Fine-tuning Pre-trained Models.} The approach of initially pre-training model weights on large-scale datasets and subsequently fine-tuning them with downstream data has become the de facto standard in the fields of computer vision \citep{yyq, vargpt, cmr3, zxw5}, natural language processing \citep{whyhelp, ptft3, zxw6, cmr4} and audio processing \citep{uniaudio, zxw7, zxw8}. For instance, \citet{noisebert1, noisebert2} proposed enhancing the performance of models by increasing their resistance to minor perturbations in intermediate layers. Meanwhile, \citet{smart} improved model robustness by adding regularization terms. Besides full-parameter fine-tuning, numerous parameter-efficient fine-tuning algorithms have been extensively studied. \citet{adapter} introduced adapters into the original model architecture, optimizing only these parameters during fine-tuning. \citet{coop, cocoop} efficiently fine-tuned CLIP models \citep{clip} using learnable soft prompts. \citet{lora} optimized models through learning low-rank residual weights. These methods achieved performance close to that of full-parameter fine-tuning while maintaining the generalization ability of the original models. However, they all require access to the model's weights and loading them into GPU memory, which can be challenging for today's large models, especially when state-of-the-art models' parameters are not publicly available. Therefore, in this paper, we follow the NML setup and explore efficient ways to fine-tune the downstream task head under a black-box scenario.

\textbf{Implicit Regularization and Sharpness-aware Minimization.} Achieving good generalization in neural networks optimized using gradient descent algorithms has long been a research focus in deep learning theory. \citet{gradnorm2} explored the properties of stochastic gradient descent (SGD), finding that SGD implicitly constrains the gradient norm. Based on this observation, Sharpness-aware minimization (SAM) \citep{sharpness, sharpness0, sharpness2, xzk2} improves generalization by incorporating the gradient norm as a regularization term. Our method can be seen as drawing inspiration from SAM but differs in that our optimization objective is the model's resilience to input noise rather than seeking flat minima in the parameter space.





\section{Experiments in Section~\ref{sec:3}}
\label{sec:mer}
%训练参数,细节;验证参数,细节

\subsection{Pre-training Dataset}

\begin{figure*}
  \centering
  \begin{subfigure}{0.32\linewidth}
    \includegraphics[height=3.2cm, width=\linewidth]{Sec3-Exp/appendix_data/data_ow.pdf}
    \caption*{(a)}
  \end{subfigure} \hfill
  \begin{subfigure}{0.32\linewidth}
    \includegraphics[height=3.2cm, width=\linewidth]{Sec3-Exp/appendix_data/data_rand.pdf}
    \caption*{(b)}
  \end{subfigure} \hfill
  \begin{subfigure}{0.32\linewidth}
    \includegraphics[height=3.2cm, width=\linewidth]{Sec3-Exp/appendix_data/data_gaussian.pdf}
    \caption*{(c)}
  \end{subfigure}
  %\vspace*{-2.5mm}
  \caption{The prior distribution of tokens in the data from (a) OpenWebText, (b) random noise, and (c) Gaussian noise.}
  \label{fig:dataset}
\end{figure*}

\textbf{OpenWebText Dataset.} The OpenWebText dataset \citep{openwebtext} is a large-scale corpus of English text data, developed to serve as an open-access alternative to proprietary dataset WebText that is utilized by OpenAI for training their GPT-2 models. This dataset originates from the analysis of outbound links clicked on Reddit, undergoing multiple stages of filtering to exclude non-English content, duplicate entries, copyrighted materials, and texts lacking in quality. These links generally direct to web pages available to the public, often shared or debated on Reddit, thereby covering a broad spectrum of subjects that mirror online popular interests and discussions. The dataset includes roughly 18 million documents, amounting to about 20GB of compressed plain text data in uint16 format. Since measures have been implemented to ensure the dataset's reliability by filtering out unsuitable content, we consider it a clean and noise-free dataset. Figure~\ref{fig:dataset}(a) illustrates the distribution of internal tokens.

\textbf{Random Noise.} To simulate the distribution of random gibberish that crawlers might retrieve from the Internet due to various reasons, we manually searched and collected a few websites containing such gibberish and also opened normally functioning websites using different decoding methods to observe the distribution of tokens. We found that, while the distribution of tokens appeared disorganized, their prior probabilities were not evenly distributed. Instead, several tokens had notably high probabilities, which is similar to that observed in the clean data as shown in Figure~\ref{fig:dataset}(a). Thus, on the basis of uniformly distributed random noise, we increased the frequency of certain tokens and then randomized them again. The resulting distribution is illustrated in Figure~\ref{fig:dataset}(b). It can be seen that while maintaining an overall uniform distribution, the frequency of tokens with IDs ranging from 0 to 1000 is higher which closely mirrors real-world scenarios.

\textbf{Gaussian Noise.} Given the diversity and unpredictability of real-world data distributions, we also artificially generated random noise with a prior probability that follows a Gaussian distribution, as shown in Figure~\ref{fig:dataset}(c). The rationale behind choosing the Gaussian distribution is that the noise in many real-world systems can be approximated by it. Additionally, we set the standard deviation $\sigma=500$ to simulate scenarios where random noise exhibits sharp peaks.

\subsection{Training Details of GPT-2}
Our work is based on the source code of nanoGPT\footnote{\url{https://github.com/karpathy/nanoGPT}}. Specifically, we utilized the GPT-2 tokenizer with vocabulary size $V=50256$ to process the OpenWebText dataset, and then appended randomly generated noise to the end of the training set before commencing training. The model's context length is set to \(L = 1024\), with an embedding dimension $d=768$. The GPT-2 model consists of 12 self-attention layers, totaling approximately 124 million parameters. For optimization, we employed AdamW \citep{adamw, adam1} with a learning rate of 6e-4, weight decay of 0.1, and \(\beta\) values of 0.9 and 0.95 for \(\beta_1\) and \(\beta_2\), respectively. A cosine annealing scheduler was used to gradually adjust the learning rate down to 6e-5. We configured the batch size to 16, with a gradient accumulation step of 40, allowing each iteration to process 655,360 tokens (16 * 40 * 1024). Training proceeded for a total of 300,000 iterations.

\subsection{Synthetic Results about Multilingual Models}
\label{app:multilin}
\begin{figure*}
  \centering
  \begin{subfigure}{0.47\linewidth}
    \includegraphics[height=4.5cm, width=\linewidth]{Sec3-Exp/appendix_data/app_loss_diff2.pdf}
    \caption*{(a)}
  \end{subfigure}
  \begin{subfigure}{0.47\linewidth}
    \includegraphics[height=4.5cm, width=\linewidth]{Sec3-Exp/appendix_data/app_loss_0.55.pdf}
    \caption*{(b)}
  \end{subfigure}

  \caption{Visualization of (a) $\mathcal{L}_{ntp}(P^m, h^*) - \mathcal{L}_{ntp}(P^m, h)$ and (b) $\mathcal{L}_{ntp}(P^c, h)$ with $\alpha=0.55,k=1, \mathcal{L}_{ntp}(P^c, h^*)=2.9, \mathcal{L}_{ntp}(P^n, h^*)=2.8$.}
  \label{fig:app:ml}
\end{figure*}

To illustrate our theory's explanatory power concerning multilingual models, we have plotted the scenario where $h^*$ is influenced by $P^n$ under the conditions $\alpha=0.55$, $k=1$, $\mathcal{L}_{ntp}(P^c, h^*)=2.9$, and $\mathcal{L}_{ntp}(P^n, h^*)=2.8$, as shown in Figure~\ref{fig:app:ml}. This setup simulates a model trained on a roughly 1:1 multilingual corpus, where the capacity of one language is affected by the data from another language. As can be observed from the figure, the impact on $p_c$ does not exceed $2\eta = 2(\alpha p_c - (1-\alpha)p_n)$, which translates to an increase of no more than 0.1 in $\mathcal{L}_{ntp}(P^n, h)$. This finding strongly supports the success of multilingual models from a theoretical perspective.

\subsection{Hardware}
We conducted the pre-training process on a server equipped with 8 NVIDIA GeForce RTX 4090 GPUs. It takes approximately 70 hours to train one model using eight 4090 GPUs, so pre-training five GPT-2 models in total requires 2,800 GPU hours.

\subsection{Additional Results on GPT-2 774M and BERT-base}
\label{app:gpt2774}
To further validate the impact of random noise, we scaled up the parameters of GPT-2 from 124M to 774M and conducted pre-training on a dataset containing 5\% random noise. The results are presented in Table~\ref{tab:gpt2_774M}.
\begin{table}[H]
  \centering
  \begin{tabular}{c|c|c|c}
  \toprule
  Noise Percentage & 10,000 iters & 50,000 iters & 100,000 iters \\ 
0\%              & 3.04         & 2.89         & 2.82          \\ 
5\%              & 3.06         & 2.92         & 2.83             \\ \bottomrule
  \end{tabular}
  \caption{NTP loss during GPT-2 774M training}
  \label{tab:gpt2_774M}
\end{table}
These findings sufficiently demonstrate that our conclusions can be extended to models with larger scales, as our theoretical framework does not hinge upon model performance metrics.

Additionally, experiments in Section~\ref{sec:4} involve BERT and vision models to assess the generalizability of the proposed local gradient matching function. This investigation is somewhat tangential to our primary focus—the impact of random noise on training language models using autoregressive approaches. Nonetheless, we pre-trained the BERT-base model under 5\% noise conditions. The outcomes are illustrated below.
\begin{table}[h]
  \centering
  \begin{tabular}{c|c|c|c}
  \toprule
  Noise Percentage & 1M iters & 5m iters & 10m iters \\ 
0\%              & 3.21     & 2.94     & 2.76      \\ 
5\%              & 3.23     & 2.97     & 2.79 \\  \bottomrule
  \end{tabular}
  \caption{NTP loss during BERT-base training.}
  \label{tab:bert}
\end{table}
These results align well with our theoretical predictions.

For vision models, pre-training outcomes can be found in \citet{nml}. Specifically, the authors simulated label noise by randomly replacing ImageNet sample labels and tested on 14 downstream tasks. The slight decrease in accuracy further corroborates our hypothesis across different types of models and datasets, underscoring the robustness of our approach against various levels of noise.

\section{Experiments in Section~\ref{sec:4}}
\label{app:exp4-detail}
\subsection{Detailed Setup for Downstream Natural Language Understanding Experiments}
\subsubsection{Datasets}
We utilize 8 commonly-used text classification benchmark: SST-2, SST-fine, 20newsgroup, CR, BBC, Balanced COPA, MRPC, WiC. The detailed information can be found in Table~\ref{tab:language-data}.

\begin{table}[h]
  \centering
  \begin{tabular}{c|c|c|c}
  \toprule
  Dataset                     & Classes & Train Size & Test Size \\ \hline
  SST-2 \citep{sst2}          & 2       & 6.92k      & 1.82k     \\
  SST-fine \citep{sstfine}    & 5       & 8.54k      & 2.21k     \\
  20newsgroup \citep{20ng}    & 20      & 11.3k      & 7.53k     \\
  CR \citep{cr}               & 2       & 3.39k      & 376       \\
  BBC \citep{bbc}             & 5       & 1.23k      & 1k        \\
  Balanced COPA \citep{bcopa} & 2       & 1k         & 500       \\
  MRPC \citep{mrpc}           & 2       & 3.67k      & 1.73k     \\
  WiC \citep{wic}             & 2       & 5.43k      & 1.4k      \\ \bottomrule
  \end{tabular}
  \caption{Details of the 8 natural language understanding dataset.}
  \label{tab:language-data}
\end{table}

\subsubsection{Prompts}
Since classification tasks can be processed as seq2seq tasks by adding prompts \citep{ilya, mathlm}, we design a unique prompt for each dataset and task. This approach transforms the inputs into a format that large language models can process. The specific designs are shown in Table \ref{tab:prompt}.

\newcolumntype{Y}{>{\centering\arraybackslash}X}
\begin{table}[h]
  \centering
  \begin{tabularx}{0.85\textwidth}{c|Y}
    \toprule
    Dataset       & Prompts                                                                                                                                                                    \\ \hline
    SST-2         & \{text\} this movie is                                                                                                                                                     \\
    SST-fine      & \{text\} this movie is                                                                                                                                                     \\
    20newsgroup   & \{text\} This article is about                                                                                                                                             \\
    CR            & \{text\} the sentiment is                                                                                                                                                  \\
    BBC           & Please classify the topic of the following news: \{text\} Answer:                                                                                                            \\
    Balanced COPA & Given the premise: \{premise\} Find the most plausible alternative for the \{question\}. Option 1: \{choice1\} Option 2: \{choice2\} Which option is more plausible? \\
    MRPC          & Sentence 1: \{text1\} Sentence 2: \{text2\} Is this a paraphrase?                                                                                                       \\
    WiC           & Task: Determine if the word \{phrase1\} has the same meaning in the two sentences below. Sentence 1: \{sentence1\} Sentence 2: \{sentence2\} Your answer:                  \\ \bottomrule
    \end{tabularx}
    \caption{Details of the prompts applied to each dataset.}
  \label{tab:prompt}
\end{table}


\subsubsection{Hyperparameters}
For all experiments in Section~\ref{sec:4}, we utilize a two-layer MLP with hidden dimension equals to feature dimension and ReLU activation function.

For all experiments shown in Table~\ref{tab:1}, we set $\gamma$ in Equation~(\ref{eq:perturb}) to be 0.01 and $\lambda$ in Equation~(\ref{eq:final}) to be 0.15. Following the setup as described by \citet{mathlm}, for each dataset, we conduct a grid search on the validation set to identify the optimal learning rate and batch size. We train for a total of ten epochs with the learning rate ranging within \{3e-4, 6e-4\} and batch size options including \{8, 12, 16, 32\}. For samples without a designated validation set, we randomly select 10\% of the training set samples to form a validation set for the purpose of selecting the best parameters.

For the experiments listed in Table \ref{tab:2}, we set the batch size to 8 and the learning rate to 6e-4 for all linear probe tasks. For all MLP probe tasks, the learning rate is set to 1e-4. Regarding $\gamma$ and $\lambda$, we conduct a grid search on the validation set to find the optimal values.

\subsection{Detailed Setup for Downstream Vision Experiments}
\subsubsection{Datasets}
We select 14 image classification datasets, which serve as a common benchmark for evaluating model performance in the vision community \citep{coop, nml}. Specific information about these 14 datasets is provided in Table \ref{tab:vision-data}.
\begin{table}[h]
  \centering
  \begin{tabular}{c|ccc}
  \toprule
  Dataset                    & Classes & Train Size & Test Size \\ \hline
  StanfordCars \citep{car}   & 196     & 8144       & 8041      \\
  Caltech101 \citep{c101}    & 102     & 3060       & 6084      \\
  CIFAR-10 \citep{c10}       & 10      & 50000      & 10000     \\
  CIFAR-100 \citep{c10}     & 100     & 50000      & 10000     \\
  DTD \citep{dtd}            & 47      & 1880       & 1880      \\
  EuroSAT \citep{sat}        & 10      & 21600      & 5400      \\
  FGVCAircraft \citep{fgvc}  & 102     & 6667       & 3333      \\
  Flowers102 \citep{f102}    & 102     & 2040       & 6149      \\
  Food101 \citep{f101}       & 101     & 75750      & 25250     \\
  OxfordPet \citep{pet}      & 37      & 3680       & 3669      \\
  PatchCamelyon \citep{pcam} & 2       & 262144     & 32768     \\
  RESISC45 \citep{r45}       & 45      & 25200      & 6300      \\
  Rendered SST2 \citep{sst2}   & 2    & 6920       & 1821      \\
  SVHN \citep{svhn}          & 10      & 73257      & 26032     \\ \bottomrule
  \end{tabular}
  \caption{Details of the 14 vision dataset.}
  \label{tab:vision-data}
\end{table}

\subsubsection{Models}
We use five pre-trained general-purpose visual backbone models that cover a variety of architectures, datasets, and training methods. Detailed information is provided in Table \ref{tab:vision-model}.
\begin{table}[h]
  \centering
  \begin{tabular}{c|ccc}
  \toprule
  Model                       & Pre-training Dataset                                                                           & Size   \\ \hline
EfficientNet-B3 \citep{efn} & \begin{tabular}[c]{@{}c@{}}ImageNet-1K \citep{in1k} \\ and JFT-300M \citep{j300m}\end{tabular} & 12.3M  \\
ResNetv2-152x2 \citep{resnet} & ImageNet-21K \citep{in21k}                                                                     & 321.7M \\
Swin-L \citep{swinl}        & ImageNet-21K                                                                                   & 196.7M \\
ConvNext-L \citep{conv}     & \begin{tabular}[c]{@{}c@{}}Laion-2B \citep{laion} \\ and ImageNet-1K\end{tabular}              & 200.1M \\
ViT-L \citep{vit}           & Laion-2B                                                                                       & 428M   \\ \hline
  \end{tabular}
  \caption{Details of the 5 vision models.}
  \label{tab:vision-model}
\end{table}

\subsubsection{Hyperparameters}
In our study, similar to the approach detailed in \citet{nml}, we primarily contrast our proposed method with MLP and LP tuning. For the optimization process, we employ AdamW for fine-tuning the modules over 30 epochs, utilizing a cosine learning rate scheduler. Specifically, for LP, we configure the learning rate at 0.1 without applying any weight decay. In contrast, both the MLP tuning and our method use a more conservative learning rate of 0.001 alongside a weight decay of 1e-4.

\subsubsection{Detailed Experimental Results}
In Table~\ref{tab3}, due to space limitations, we only present the average results, while detailed results are shown in Table~\ref{tab:final}.

\begin{table}[H]
  \centering
  \resizebox{\textwidth}{!}{
    \begin{tabular}{c|ll|ll|ll}
      \toprule
      \multirow{2}{*}{Models}            & \multicolumn{2}{c|}{StanfordCars}                      & \multicolumn{2}{c|}{Caltech101}                        & \multicolumn{2}{c}{CIFAR-10}                           \\
                                         & \multicolumn{1}{c}{Linear} & \multicolumn{1}{c|}{MLP}  & \multicolumn{1}{c}{Linear} & \multicolumn{1}{c|}{MLP}  & \multicolumn{1}{c}{Linear} & \multicolumn{1}{c}{MLP}   \\ \hline
      ViT-L                              & 93.38 $\pm$ 0.76           & 94.41 $\pm$ 1.05          & 92.07 $\pm$ 1.19           & 95.20 $\pm$ 1.12           & 97.99 $\pm$ 0.95           & 98.35 $\pm$ 0.95          \\
      ViT-L + $\mathcal{L}_{gm}$         & \textbf{93.71 $\pm$ 1.37}  & \textbf{94.56 $\pm$ 1.50} & \textbf{95.01 $\pm$ 0.97}  & \textbf{95.29 $\pm$ 1.27} & \textbf{98.07 $\pm$ 0.74}  & \textbf{98.48 $\pm$ 0.60} \\ \hline
      ConvNext-L                         & 86.01 $\pm$ 1.48           & 88.68 $\pm$ 0.89          & 91.02 $\pm$ 0.79           & 94.47 $\pm$ 0.53          & 97.49 $\pm$ 1.36           & 98.09 $\pm$ 0.85          \\
      ConvNext-L+$\mathcal{L}_{gm}$      & \textbf{86.78 $\pm$ 1.32}  & \textbf{89.06 $\pm$ 1.19} & \textbf{94.11 $\pm$ 1.19}  & \textbf{94.93 $\pm$ 0.88} & \textbf{97.59 $\pm$ 0.52}  & \textbf{98.15 $\pm$ 0.71} \\ \hline
      EfficientNet-B3                    & 56.20 $\pm$ 0.54            & \textbf{58.57 $\pm$ 1.11} & 89.43 $\pm$ 0.78           & 91.22 $\pm$ 1.23          & 94.04 $\pm$ 1.19           & 95.73 $\pm$ 1.07          \\
      EfficientNet-B3+$\mathcal{L}_{gm}$ & \textbf{57.02 $\pm$ 1.28}  & 58.15 $\pm$ 1.12          & \textbf{90.25 $\pm$ 1.43}  & \textbf{91.55 $\pm$ 0.90} & \textbf{94.11 $\pm$ 0.88}  & \textbf{95.96 $\pm$ 0.86} \\ \hline
      ResNetv2-152x2                     & 56.95 $\pm$ 1.21           & \textbf{59.18 $\pm$ 1.34} & 91.40 $\pm$ 1.47            & 92.48 $\pm$ 1.30          & 96.28 $\pm$ 1.16           & 96.91 $\pm$ 0.89          \\
      ResNetv2-152x2+$\mathcal{L}_{gm}$  & \textbf{58.78 $\pm$ 1.16}  & 58.67 $\pm$ 1.27          & \textbf{93.83 $\pm$ 0.91}  & \textbf{93.95 $\pm$ 0.94} & \textbf{96.31 $\pm$ 0.85}  & \textbf{97.03 $\pm$ 0.53} \\ \hline
      Swin-L                             & 68.17 $\pm$ 0.98           & \textbf{74.11 $\pm$ 0.60} & 92.58 $\pm$ 0.95           & 94.09 $\pm$ 1.04          & 98.26 $\pm$ 0.89           & 98.61 $\pm$ 0.78          \\
      Swin-L+$\mathcal{L}_{gm}$          & \textbf{69.31 $\pm$ 1.07}  & 73.71 $\pm$ 0.94          & \textbf{93.65 $\pm$ 1.42}  & \textbf{94.62 $\pm$ 0.64} & \textbf{98.41 $\pm$ 0.91}  & \textbf{98.72 $\pm$ 1.28} \\ \bottomrule
    \end{tabular}}
  %\vspace{-2mm}
  %\caption{Average accuracy of 5 vision backbone models on \textbf{14} commonly-used vision datasets.}
  %\label{tab3}
  \vspace{-3mm}
\end{table}
\begin{table}[H]
  \centering
  \resizebox{\textwidth}{!}{
    \begin{tabular}{ll|ll|ll|ll}
      \hline
      \multicolumn{2}{c|}{CIFAR-100}                         & \multicolumn{2}{c|}{EuroSAT}                           & \multicolumn{2}{c|}{FGVCAircraft}                      & \multicolumn{2}{c}{OxfordPet}                          \\
      \multicolumn{1}{c}{Linear} & \multicolumn{1}{c|}{MLP}  & \multicolumn{1}{c}{Linear} & \multicolumn{1}{c|}{MLP}  & \multicolumn{1}{c}{Linear} & \multicolumn{1}{c|}{MLP}  & \multicolumn{1}{c}{Linear} & \multicolumn{1}{c}{MLP}   \\ \hline
      \textbf{88.07 $\pm$ 0.58}  & 89.49 $\pm$ 0.52          & 97.53 $\pm$ 1.13           & 97.75 $\pm$ 0.61          & 65.76 $\pm$ 0.73           & 68.43 $\pm$ 0.78          & 91.65 $\pm$ 1.18           & 93.97 $\pm$ 1.50          \\
      88.06 $\pm$ 0.93           & \textbf{89.58 $\pm$ 0.93} & \textbf{97.83 $\pm$ 0.73}  & \textbf{98.03 $\pm$ 0.60} & \textbf{66.63 $\pm$ 1.08}  & \textbf{68.67 $\pm$ 1.07} & \textbf{93.18 $\pm$ 0.81}  & \textbf{94.17 $\pm$ 1.10} \\ \hline
      \textbf{86.76 $\pm$ 0.91}  & 87.79 $\pm$ 1.27          & 95.57 $\pm$ 1.22           & 96.31 $\pm$ 1.11          & 57.18 $\pm$ 1.12           & 62.25 $\pm$ 0.66          & 94.98 $\pm$ 0.54           & 95.80 $\pm$ 0.75           \\
      86.46 $\pm$ 1.04           & \textbf{87.88 $\pm$ 1.24} & \textbf{96.05 $\pm$ 1.46}  & \textbf{96.74 $\pm$ 1.27} & \textbf{58.35 $\pm$ 0.68}  & \textbf{63.61 $\pm$ 0.83} & \textbf{95.39 $\pm$ 1.21}  & \textbf{95.99 $\pm$ 0.92} \\ \hline
      \textbf{77.34 $\pm$ 0.86}  & 80.28 $\pm$ 1.02          & 94.81 $\pm$ 1.35           & 95.90 $\pm$ 0.88           & 44.73 $\pm$ 1.31           & 46.23 $\pm$ 0.92          & 93.84 $\pm$ 1.14           & 94.79 $\pm$ 1.14          \\
      77.16 $\pm$ 1.11           & \textbf{80.47 $\pm$ 1.43} & \textbf{95.20 $\pm$ 0.52}   & \textbf{96.07 $\pm$ 1.18} & \textbf{45.33 $\pm$ 0.61}  & \textbf{47.07 $\pm$ 0.57} & \textbf{94.63 $\pm$ 1.03}  & \textbf{94.98 $\pm$ 1.14} \\ \hline
      \textbf{84.30 $\pm$ 1.18}   & \textbf{84.68 $\pm$ 1.33} & 97.12 $\pm$ 1.46           & 97.46 $\pm$ 1.28          & 42.03 $\pm$ 0.72           & 48.39 $\pm$ 0.85          & 91.93 $\pm$ 0.68           & 92.99 $\pm$ 1.40          \\
      84.28 $\pm$ 1.22           & 84.29 $\pm$ 1.38          & \textbf{97.35 $\pm$ 1.12}  & \textbf{97.59 $\pm$ 0.72} & \textbf{45.69 $\pm$ 0.80}  & \textbf{48.84 $\pm$ 0.61} & \textbf{92.61 $\pm$ 0.87}  & \textbf{93.45 $\pm$ 1.32} \\ \hline
      89.68 $\pm$ 1.33           & 90.74 $\pm$ 0.98          & \textbf{97.11 $\pm$ 0.72}  & 97.59 $\pm$ 0.63          & 54.96 $\pm$ 1.24           & \textbf{61.17 $\pm$ 1.35} & 92.17 $\pm$ 0.64           & 94.38 $\pm$ 1.21          \\
      \textbf{89.79 $\pm$ 0.52}  & \textbf{91.18 $\pm$ 1.21} & 97.09 $\pm$ 1.11           & \textbf{97.71 $\pm$ 1.08} & \textbf{56.10 $\pm$ 0.67}   & 60.99 $\pm$ 0.73          & \textbf{93.86 $\pm$ 0.85}  & \textbf{94.57 $\pm$ 1.11} \\ \hline
      \end{tabular}}
  %\vspace{-2mm}
  %\caption{Average accuracy of 5 vision backbone models on \textbf{14} commonly-used vision datasets.}
  %\label{tab3}
  \vspace{-3mm}
\end{table}
\begin{table}[H]
  \centering
  \resizebox{\textwidth}{!}{
    \begin{tabular}{ll|ll|ll|ll}
      \hline
      \multicolumn{2}{c|}{Food101}                           & \multicolumn{2}{c|}{Flowers102}                        & \multicolumn{2}{c|}{DTD}                               & \multicolumn{2}{c}{SVHN}                               \\
      \multicolumn{1}{c}{Linear} & \multicolumn{1}{c|}{MLP}  & \multicolumn{1}{c}{Linear} & \multicolumn{1}{c|}{MLP}  & \multicolumn{1}{c}{Linear} & \multicolumn{1}{c|}{MLP}  & \multicolumn{1}{c}{Linear} & \multicolumn{1}{c}{MLP}   \\ \hline
      90.51 $\pm$ 1.31           & 91.04 $\pm$ 1.35          & 94.04 $\pm$ 1.13           & 97.83 $\pm$ 0.74          & 80.53 $\pm$ 1.06           & 83.29 $\pm$ 0.86          & 78.82 $\pm$ 1.25           & 84.74 $\pm$ 1.08          \\
      \textbf{90.62 $\pm$ 1.37}  & \textbf{91.23 $\pm$ 0.52} & \textbf{96.67 $\pm$ 1.41}  & \textbf{98.06 $\pm$ 1.28} & \textbf{82.76 $\pm$ 0.71}  & \textbf{83.77 $\pm$ 0.98} & \textbf{79.80 $\pm$ 0.88}  & \textbf{84.59 $\pm$ 1.38} \\ \hline
      \textbf{89.09 $\pm$ 1.06}  & \textbf{90.21 $\pm$ 0.87} & 94.71 $\pm$ 1.31           & 98.78 $\pm$ 1.17          & 76.01 $\pm$ 1.03           & 78.67 $\pm$ 1.15          & 66.16 $\pm$ 0.87           & 72.76 $\pm$ 0.78          \\
      88.62 $\pm$ 0.72           & 90.10 $\pm$ 1.39          & \textbf{97.12 $\pm$ 0.95}  & \textbf{98.99 $\pm$ 0.99} & \textbf{77.92 $\pm$ 0.72}  & \textbf{80.05 $\pm$ 1.44} & \textbf{68.43 $\pm$ 1.37}  & \textbf{73.18 $\pm$ 0.67} \\ \hline
      \textbf{76.95 $\pm$ 0.82}  & \textbf{81.78 $\pm$ 0.89} & 84.19 $\pm$ 0.61           & 88.97 $\pm$ 1.32          & 69.09 $\pm$ 1.27           & 73.08 $\pm$ 1.30          & 54.26 $\pm$ 0.87           & 61.38 $\pm$ 0.82          \\
      76.35 $\pm$ 0.74           & 81.33 $\pm$ 1.45          & \textbf{86.19 $\pm$ 1.08}  & \textbf{89.42 $\pm$ 0.62} & \textbf{71.27 $\pm$ 1.24}  & \textbf{73.82 $\pm$ 1.40} & \textbf{56.74 $\pm$ 0.73}  & \textbf{63.56 $\pm$ 0.82} \\ \hline
      \textbf{84.83 $\pm$ 1.36}  & 84.15 $\pm$ 1.27          & 96.76 $\pm$ 0.88           & 98.27 $\pm$ 0.53          & 72.23 $\pm$ 0.94           & 76.11 $\pm$ 1.25          & 60.75 $\pm$ 0.89           & 64.87 $\pm$ 1.45          \\
      84.41 $\pm$ 0.88           & \textbf{84.64 $\pm$ 1.04} & \textbf{98.08 $\pm$ 1.23}  & \textbf{98.84 $\pm$ 0.82} & \textbf{74.73 $\pm$ 1.38}  & \textbf{77.12 $\pm$ 1.39} & \textbf{62.04 $\pm$ 1.01}  & \textbf{65.06 $\pm$ 0.62} \\ \hline
      90.23 $\pm$ 0.64           & 92.23 $\pm$ 0.55          & 97.28 $\pm$ 0.92           & 99.51 $\pm$ 1.17          & 75.85 $\pm$ 0.88           & 80.74 $\pm$ 1.45          & 62.77 $\pm$ 1.42           & \textbf{69.53 $\pm$ 1.22} \\
      \textbf{90.26 $\pm$ 1.14}  & \textbf{92.32 $\pm$ 1.00} & \textbf{99.12 $\pm$ 1.05}  & \textbf{99.60 $\pm$ 0.51} & \textbf{77.44 $\pm$ 1.00}  & \textbf{80.91 $\pm$ 0.83} & \textbf{64.83 $\pm$ 0.98}  & 68.97 $\pm$ 1.12          \\ \hline
      \end{tabular}}
  %\vspace{-2mm}
  %\caption{Average accuracy of 5 vision backbone models on \textbf{14} commonly-used vision datasets.}
  %\label{tab3}
  \vspace{-3mm}
\end{table}
\begin{table}[H]
  \centering
  \resizebox{\textwidth}{!}{
    \begin{tabular}{ll|ll|ll|ll}
      \hline
      \multicolumn{2}{c|}{resisc45}                          & \multicolumn{2}{c|}{rsst2}                             & \multicolumn{2}{c|}{pcam}                              & \multicolumn{2}{c}{Avg}                                \\
      \multicolumn{1}{c}{Linear} & \multicolumn{1}{c|}{MLP}  & \multicolumn{1}{c}{Linear} & \multicolumn{1}{c|}{MLP}  & \multicolumn{1}{c}{Linear} & \multicolumn{1}{c|}{MLP}  & \multicolumn{1}{c}{Linear} & \multicolumn{1}{c}{MLP}   \\ \hline
      95.44 $\pm$ 1.41           & 95.79 $\pm$ 0.57          & 67.65 $\pm$ 1.12           & 73.58 $\pm$ 1.31          & \textbf{82.65 $\pm$ 0.57}  & \textbf{83.92 $\pm$ 0.56} & 86.86            & 89.12          \\
      \textbf{95.73 $\pm$ 1.19}  & \textbf{95.93 $\pm$ 1.28} & \textbf{71.82 $\pm$ 0.55}  & \textbf{74.24 $\pm$ 0.76} & 82.55 $\pm$ 1.49           & 83.78 $\pm$ 0.97          & \textbf{88.03}  & \textbf{89.31} \\ \hline
      92.65 $\pm$ 1.25           & 93.09 $\pm$ 0.89          & 60.73 $\pm$ 1.30           & 66.0 $\pm$ 0.55           & 72.21 $\pm$ 0.72           & 77.08 $\pm$ 0.53          & 82.89            & 85.71          \\
      \textbf{92.93 $\pm$ 0.54}  & \textbf{93.31 $\pm$ 1.31} & \textbf{64.14 $\pm$ 0.99}  & \textbf{67.49 $\pm$ 0.63} & \textbf{73.1 $\pm$ 1.35}   & \textbf{78.31 $\pm$ 1.10} & \textbf{84.07}  & \textbf{86.27} \\ \hline
      87.19 $\pm$ 1.05           & 89.01 $\pm$ 0.70          & \textbf{50.46 $\pm$ 1.02}  & 50.74 $\pm$ 1.05          & 53.32 $\pm$ 0.59           & \textbf{51.10 $\pm$ 1.34} & 73.27           & 75.62          \\
      \textbf{87.33 $\pm$ 0.60}  & \textbf{89.17 $\pm$ 1.34} & 50.19 $\pm$ 0.74           & \textbf{51.07 $\pm$ 0.92} & \textbf{54.52 $\pm$ 1.27}  & 50.10 $\pm$ 0.92          & \textbf{74.02}  & \textbf{75.90} \\ \hline
      90.96 $\pm$ 0.81           & 91.19 $\pm$ 0.57          & 50.90 $\pm$ 0.73           & 49.91 $\pm$ 1.08          & 77.62 $\pm$ 0.85           & 77.86 $\pm$ 1.16          & 78.14           & \textbf{79.60} \\
      \textbf{91.17 $\pm$ 0.64}  & \textbf{91.36 $\pm$ 0.73} & \textbf{54.25 $\pm$ 0.69}  & \textbf{49.92 $\pm$ 1.03} & \textbf{79.44 $\pm$ 0.52}  & \textbf{78.48 $\pm$ 0.53} & \textbf{79.49}  & 79.45          \\ \hline
      92.79 $\pm$ 1.25           & 94.09 $\pm$ 0.99          & 50.96 $\pm$ 1.41           & 53.59 $\pm$ 1.20          & 77.32 $\pm$ 0.68           & 78.38 $\pm$ 1.04          & 81.43           & 84.19          \\
      \textbf{93.41 $\pm$ 1.32}  & \textbf{94.42 $\pm$ 0.96} & \textbf{53.48 $\pm$ 0.89}  & \textbf{54.96 $\pm$ 0.90} & \textbf{81.18 $\pm$ 1.34}  & \textbf{79.29 $\pm$ 1.31} & \textbf{82.70}  & \textbf{84.42} \\ \hline
      \end{tabular}}
  %\vspace{-2mm}
  \caption{Detailed accuracy of 5 vision backbone models on \textbf{14} commonly-used vision datasets.}
  \label{tab:final}
  %\vspace{-4mm}
\end{table}

\subsection{Runtime Analysis}
Since all our models are black-box models, we first process all samples into vector-form features and then probe them. All models described in this paper can run on a single NVIDIA RTX 4090 GPU. Extracting all these features requires a total of 10 GPU hours. Subsequently, training these Linear or MLP Probes requires approximately 200 GPU hours in total.

\end{document}
