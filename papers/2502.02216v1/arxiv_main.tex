
\documentclass{article}

\usepackage{microtype}
\usepackage{graphicx}
\usepackage{subfigure}
\usepackage{booktabs} %
\usepackage[table]{xcolor}

\usepackage{hyperref}


\newcommand{\theHalgorithm}{\arabic{algorithm}}


\usepackage[accepted]{arxiv}

\usepackage{amsmath}
\usepackage{amssymb}
\usepackage{mathtools}
\usepackage{amsthm}
\usepackage{bm}
\usepackage{lipsum}

\usepackage{multirow}
\usepackage{stmaryrd}  %
\usepackage{algorithm}
\usepackage[noend]{algpseudocode}
\renewcommand\algorithmicrequire{\textbf{Input:}}
\renewcommand\algorithmicensure{\textbf{Output:}}

\usepackage[capitalize,noabbrev]{cleveref}


%
\setlength\unitlength{1mm}
\newcommand{\twodots}{\mathinner {\ldotp \ldotp}}
% bb font symbols
\newcommand{\Rho}{\mathrm{P}}
\newcommand{\Tau}{\mathrm{T}}

\newfont{\bbb}{msbm10 scaled 700}
\newcommand{\CCC}{\mbox{\bbb C}}

\newfont{\bb}{msbm10 scaled 1100}
\newcommand{\CC}{\mbox{\bb C}}
\newcommand{\PP}{\mbox{\bb P}}
\newcommand{\RR}{\mbox{\bb R}}
\newcommand{\QQ}{\mbox{\bb Q}}
\newcommand{\ZZ}{\mbox{\bb Z}}
\newcommand{\FF}{\mbox{\bb F}}
\newcommand{\GG}{\mbox{\bb G}}
\newcommand{\EE}{\mbox{\bb E}}
\newcommand{\NN}{\mbox{\bb N}}
\newcommand{\KK}{\mbox{\bb K}}
\newcommand{\HH}{\mbox{\bb H}}
\newcommand{\SSS}{\mbox{\bb S}}
\newcommand{\UU}{\mbox{\bb U}}
\newcommand{\VV}{\mbox{\bb V}}


\newcommand{\yy}{\mathbbm{y}}
\newcommand{\xx}{\mathbbm{x}}
\newcommand{\zz}{\mathbbm{z}}
\newcommand{\sss}{\mathbbm{s}}
\newcommand{\rr}{\mathbbm{r}}
\newcommand{\pp}{\mathbbm{p}}
\newcommand{\qq}{\mathbbm{q}}
\newcommand{\ww}{\mathbbm{w}}
\newcommand{\hh}{\mathbbm{h}}
\newcommand{\vvv}{\mathbbm{v}}

% Vectors

\newcommand{\av}{{\bf a}}
\newcommand{\bv}{{\bf b}}
\newcommand{\cv}{{\bf c}}
\newcommand{\dv}{{\bf d}}
\newcommand{\ev}{{\bf e}}
\newcommand{\fv}{{\bf f}}
\newcommand{\gv}{{\bf g}}
\newcommand{\hv}{{\bf h}}
\newcommand{\iv}{{\bf i}}
\newcommand{\jv}{{\bf j}}
\newcommand{\kv}{{\bf k}}
\newcommand{\lv}{{\bf l}}
\newcommand{\mv}{{\bf m}}
\newcommand{\nv}{{\bf n}}
\newcommand{\ov}{{\bf o}}
\newcommand{\pv}{{\bf p}}
\newcommand{\qv}{{\bf q}}
\newcommand{\rv}{{\bf r}}
\newcommand{\sv}{{\bf s}}
\newcommand{\tv}{{\bf t}}
\newcommand{\uv}{{\bf u}}
\newcommand{\wv}{{\bf w}}
\newcommand{\vv}{{\bf v}}
\newcommand{\xv}{{\bf x}}
\newcommand{\yv}{{\bf y}}
\newcommand{\zv}{{\bf z}}
\newcommand{\zerov}{{\bf 0}}
\newcommand{\onev}{{\bf 1}}

% Matrices

\newcommand{\Am}{{\bf A}}
\newcommand{\Bm}{{\bf B}}
\newcommand{\Cm}{{\bf C}}
\newcommand{\Dm}{{\bf D}}
\newcommand{\Em}{{\bf E}}
\newcommand{\Fm}{{\bf F}}
\newcommand{\Gm}{{\bf G}}
\newcommand{\Hm}{{\bf H}}
\newcommand{\Id}{{\bf I}}
\newcommand{\Jm}{{\bf J}}
\newcommand{\Km}{{\bf K}}
\newcommand{\Lm}{{\bf L}}
\newcommand{\Mm}{{\bf M}}
\newcommand{\Nm}{{\bf N}}
\newcommand{\Om}{{\bf O}}
\newcommand{\Pm}{{\bf P}}
\newcommand{\Qm}{{\bf Q}}
\newcommand{\Rm}{{\bf R}}
\newcommand{\Sm}{{\bf S}}
\newcommand{\Tm}{{\bf T}}
\newcommand{\Um}{{\bf U}}
\newcommand{\Wm}{{\bf W}}
\newcommand{\Vm}{{\bf V}}
\newcommand{\Xm}{{\bf X}}
\newcommand{\Ym}{{\bf Y}}
\newcommand{\Zm}{{\bf Z}}

% Calligraphic

\newcommand{\Ac}{{\cal A}}
\newcommand{\Bc}{{\cal B}}
\newcommand{\Cc}{{\cal C}}
\newcommand{\Dc}{{\cal D}}
\newcommand{\Ec}{{\cal E}}
\newcommand{\Fc}{{\cal F}}
\newcommand{\Gc}{{\cal G}}
\newcommand{\Hc}{{\cal H}}
\newcommand{\Ic}{{\cal I}}
\newcommand{\Jc}{{\cal J}}
\newcommand{\Kc}{{\cal K}}
\newcommand{\Lc}{{\cal L}}
\newcommand{\Mc}{{\cal M}}
\newcommand{\Nc}{{\cal N}}
\newcommand{\nc}{{\cal n}}
\newcommand{\Oc}{{\cal O}}
\newcommand{\Pc}{{\cal P}}
\newcommand{\Qc}{{\cal Q}}
\newcommand{\Rc}{{\cal R}}
\newcommand{\Sc}{{\cal S}}
\newcommand{\Tc}{{\cal T}}
\newcommand{\Uc}{{\cal U}}
\newcommand{\Wc}{{\cal W}}
\newcommand{\Vc}{{\cal V}}
\newcommand{\Xc}{{\cal X}}
\newcommand{\Yc}{{\cal Y}}
\newcommand{\Zc}{{\cal Z}}

% Bold greek letters

\newcommand{\alphav}{\hbox{\boldmath$\alpha$}}
\newcommand{\betav}{\hbox{\boldmath$\beta$}}
\newcommand{\gammav}{\hbox{\boldmath$\gamma$}}
\newcommand{\deltav}{\hbox{\boldmath$\delta$}}
\newcommand{\etav}{\hbox{\boldmath$\eta$}}
\newcommand{\lambdav}{\hbox{\boldmath$\lambda$}}
\newcommand{\epsilonv}{\hbox{\boldmath$\epsilon$}}
\newcommand{\nuv}{\hbox{\boldmath$\nu$}}
\newcommand{\muv}{\hbox{\boldmath$\mu$}}
\newcommand{\zetav}{\hbox{\boldmath$\zeta$}}
\newcommand{\phiv}{\hbox{\boldmath$\phi$}}
\newcommand{\psiv}{\hbox{\boldmath$\psi$}}
\newcommand{\thetav}{\hbox{\boldmath$\theta$}}
\newcommand{\tauv}{\hbox{\boldmath$\tau$}}
\newcommand{\omegav}{\hbox{\boldmath$\omega$}}
\newcommand{\xiv}{\hbox{\boldmath$\xi$}}
\newcommand{\sigmav}{\hbox{\boldmath$\sigma$}}
\newcommand{\piv}{\hbox{\boldmath$\pi$}}
\newcommand{\rhov}{\hbox{\boldmath$\rho$}}
\newcommand{\upsilonv}{\hbox{\boldmath$\upsilon$}}

\newcommand{\Gammam}{\hbox{\boldmath$\Gamma$}}
\newcommand{\Lambdam}{\hbox{\boldmath$\Lambda$}}
\newcommand{\Deltam}{\hbox{\boldmath$\Delta$}}
\newcommand{\Sigmam}{\hbox{\boldmath$\Sigma$}}
\newcommand{\Phim}{\hbox{\boldmath$\Phi$}}
\newcommand{\Pim}{\hbox{\boldmath$\Pi$}}
\newcommand{\Psim}{\hbox{\boldmath$\Psi$}}
\newcommand{\Thetam}{\hbox{\boldmath$\Theta$}}
\newcommand{\Omegam}{\hbox{\boldmath$\Omega$}}
\newcommand{\Xim}{\hbox{\boldmath$\Xi$}}


% Sans Serif small case

\newcommand{\Gsf}{{\sf G}}

\newcommand{\asf}{{\sf a}}
\newcommand{\bsf}{{\sf b}}
\newcommand{\csf}{{\sf c}}
\newcommand{\dsf}{{\sf d}}
\newcommand{\esf}{{\sf e}}
\newcommand{\fsf}{{\sf f}}
\newcommand{\gsf}{{\sf g}}
\newcommand{\hsf}{{\sf h}}
\newcommand{\isf}{{\sf i}}
\newcommand{\jsf}{{\sf j}}
\newcommand{\ksf}{{\sf k}}
\newcommand{\lsf}{{\sf l}}
\newcommand{\msf}{{\sf m}}
\newcommand{\nsf}{{\sf n}}
\newcommand{\osf}{{\sf o}}
\newcommand{\psf}{{\sf p}}
\newcommand{\qsf}{{\sf q}}
\newcommand{\rsf}{{\sf r}}
\newcommand{\ssf}{{\sf s}}
\newcommand{\tsf}{{\sf t}}
\newcommand{\usf}{{\sf u}}
\newcommand{\wsf}{{\sf w}}
\newcommand{\vsf}{{\sf v}}
\newcommand{\xsf}{{\sf x}}
\newcommand{\ysf}{{\sf y}}
\newcommand{\zsf}{{\sf z}}


% mixed symbols

\newcommand{\sinc}{{\hbox{sinc}}}
\newcommand{\diag}{{\hbox{diag}}}
\renewcommand{\det}{{\hbox{det}}}
\newcommand{\trace}{{\hbox{tr}}}
\newcommand{\sign}{{\hbox{sign}}}
\renewcommand{\arg}{{\hbox{arg}}}
\newcommand{\var}{{\hbox{var}}}
\newcommand{\cov}{{\hbox{cov}}}
\newcommand{\Ei}{{\rm E}_{\rm i}}
\renewcommand{\Re}{{\rm Re}}
\renewcommand{\Im}{{\rm Im}}
\newcommand{\eqdef}{\stackrel{\Delta}{=}}
\newcommand{\defines}{{\,\,\stackrel{\scriptscriptstyle \bigtriangleup}{=}\,\,}}
\newcommand{\<}{\left\langle}
\renewcommand{\>}{\right\rangle}
\newcommand{\herm}{{\sf H}}
\newcommand{\trasp}{{\sf T}}
\newcommand{\transp}{{\sf T}}
\renewcommand{\vec}{{\rm vec}}
\newcommand{\Psf}{{\sf P}}
\newcommand{\SINR}{{\sf SINR}}
\newcommand{\SNR}{{\sf SNR}}
\newcommand{\MMSE}{{\sf MMSE}}
\newcommand{\REF}{{\RED [REF]}}

% Markov chain
\usepackage{stmaryrd} % for \mkv 
\newcommand{\mkv}{-\!\!\!\!\minuso\!\!\!\!-}

% Colors

\newcommand{\RED}{\color[rgb]{1.00,0.10,0.10}}
\newcommand{\BLUE}{\color[rgb]{0,0,0.90}}
\newcommand{\GREEN}{\color[rgb]{0,0.80,0.20}}

%%%%%%%%%%%%%%%%%%%%%%%%%%%%%%%%%%%%%%%%%%
\usepackage{hyperref}
\hypersetup{
    bookmarks=true,         % show bookmarks bar?
    unicode=false,          % non-Latin characters in AcrobatÕs bookmarks
    pdftoolbar=true,        % show AcrobatÕs toolbar?
    pdfmenubar=true,        % show AcrobatÕs menu?
    pdffitwindow=false,     % window fit to page when opened
    pdfstartview={FitH},    % fits the width of the page to the window
%    pdftitle={My title},    % title
%    pdfauthor={Author},     % author
%    pdfsubject={Subject},   % subject of the document
%    pdfcreator={Creator},   % creator of the document
%    pdfproducer={Producer}, % producer of the document
%    pdfkeywords={keyword1} {key2} {key3}, % list of keywords
    pdfnewwindow=true,      % links in new window
    colorlinks=true,       % false: boxed links; true: colored links
    linkcolor=red,          % color of internal links (change box color with linkbordercolor)
    citecolor=green,        % color of links to bibliography
    filecolor=blue,      % color of file links
    urlcolor=blue           % color of external links
}
%%%%%%%%%%%%%%%%%%%%%%%%%%%%%%%%%%%%%%%%%%%



\usepackage[conf={end, restate, no link to proof}]{proof-at-the-end}
\theoremstyle{plain}
\newtheorem{theorem}{Theorem}[section]
\newtheorem{proposition}[theorem]{Proposition}
\newtheorem{lemma}[theorem]{Lemma}
\newtheorem{corollary}[theorem]{Corollary}
\theoremstyle{definition}
\newtheorem{definition}[theorem]{Definition}
\newtheorem{assumption}[theorem]{Assumption}
\theoremstyle{remark}
\newtheorem{remark}[theorem]{Remark}

\usepackage[textsize=tiny]{todonotes}


\icmltitlerunning{Transformers are Scalable Graph Generators}

\begin{document}

\twocolumn[
\icmltitle{Flatten Graphs as Sequences: Transformers are Scalable Graph Generators}



\icmlsetsymbol{equal}{*}

\begin{icmlauthorlist}
\icmlauthor{Dexiong Chen}{mpib}
\icmlauthor{Markus Krimmel}{mpib}
\icmlauthor{Karsten Borgwardt}{mpib}
\end{icmlauthorlist}

\icmlaffiliation{mpib}{Max Planck Institute of Biochemistry, Martinsried, Germany}

\icmlcorrespondingauthor{Dexiong Chen}{dchen@biochem.mpg.de}

\icmlkeywords{Machine Learning, ICML}

\vskip 0.3in
]



\printAffiliationsAndNotice{}  %

\begin{abstract}
We introduce \method{}, a novel autoregressive framework for generating large attributed graphs using decoder-only transformers. At the core of our approach is a reversible ``flattening'' process that transforms graphs into random sequences. By sampling and learning from these sequences, \method{} enables transformers to model and generate complex graph structures in a manner akin to natural language. 
In contrast to diffusion models that rely on computationally intensive node features, our approach operates exclusively on these sequences. 
The sampling complexity and sequence length scale linearly with the number of edges, making \method{} highly scalable for generating large sparse graphs. 
Empirically, \method{} achieves state-of-the-art performance across diverse synthetic and molecular graph generation benchmarks, while delivering a 100-fold generation and a 3-fold training speedup compared to leading diffusion models. Additionally, it demonstrates promising transfer capabilities and supports substructure-conditioned generation without additional fine-tuning. 
By extending language modeling techniques to graph generation, this work paves the way for developing graph foundation models.
\end{abstract}


\section{Introduction}\label{sec:introduction}

Recent advancements in deep generative models have revolutionized various domains of artificial intelligence, demonstrating remarkable capabilities in generating complex data types such as images~\citep{rombach2022high}, natural language~\citep{gpt3,touvron2023llama,touvron2023llama2}, and audio~\citep{dhariwal2020jukebox,huang2024audiogpt}. These achievements have been primarily driven by the development of advanced architectures or methods such as transformers and diffusion models, alongside increasingly large-scale data resources. However, the generation of graph-structured data, which is fundamental to numerous scientific applications including drug discovery~\citep{vignac2023digress,lim2020scaffold}, protein design~\citep{ingraham2019generative}, and program synthesis~\citep{brockschmidt2019generative}, remains a significant challenge. This disparity primarily stems from the inherent complexity of preserving structural validity, maintaining invariance properties within graphs, and achieving scalability in real-world graph generation tasks.

To this end, diffusion-based models have emerged as a promising direction for graph generation, demonstrating effectiveness in synthesizing both classic unattributed graphs and molecules~\citep{jo2022score,vignac2023digress}. These approaches typically implement a denoising process in discrete graph space, simultaneously predicting edge connectivity and attributes. Yet, their practical applications are constrained by fundamental scalability limitations. The requirement for full adjacency matrix operations imposes quadratic memory complexity with respect to the number of nodes. Moreover, computing additional features in each denoising step such as spectral features, often involving cubic complexity, further increases the computational overhead.

Autoregressive approaches represent an alternative paradigm, constructing graphs sequentially by generating nodes and edges in a step-by-step manner~\citep{liao2019efficient,you2018graphrnn}. These models have demonstrated strong performance in generating small to medium-sized graphs by leveraging their ability to maintain structural validity through the generation process. Nevertheless, these models face inherent limitations: they require specialized architectures, primarily based on recurrent neural networks, to process the complex ad-hoc sequential representations of graphs, preventing them from directly leveraging the remarkable advances in large language models (LLMs). Moreover, these specialized architectures often struggle with long-range dependencies and ensuring global structural consistency, leading to significantly inferior performance compared to recent diffusion models~\citep{vignac2023digress}. This architectural constraint not only limits their scalability but also creates a growing performance gap as general-purpose LLMs continue to advance rapidly.

In light of these challenges, we introduce a novel paradigm that bridges the gap between graph generation and LLMs through a graph-to-sequence transformation. Our approach fundamentally reconceptualizes graphs as sequential structures while maintaining their topological properties. Instead of requiring specialized architectures or operating on graph structures, we propose a method to linearize graphs into random sequences that encode local connectivity patterns. This transformation enables direct utilization of language models for graph generation while achieving optimally linear complexity with respect to the number of edges in both computational and memory requirements. Our approach effectively addresses the limitations of both diffusion-based and autoregressive methods: it maintains structural validity while enabling efficient scaling to large graphs and leveraging the powerful capabilities of modern language models.

Our work presents several technical contributions to the field of graph generation. (1) We introduce the concept of segmented Eulerian neighborhood trails (SENTs), a specialized class of Eulerian trails that permit breaks and incorporate neighborhood information. We establish sufficient conditions under which they can be employed for effective graph generation. (2) We propose an efficient flattening algorithm that transforms graphs into sequences and vice versa by sampling these SENTs, enabling lossless sequence representation of graphs. (3) Our method, termed \method{}, achieves state-of-the-art (SOTA) performance across diverse synthetic and molecular graph generation benchmarks, delivering a 100-fold generation and a 3-fold training speedup compared to diffusion-based models while maintaining the ability to scale to graphs of possibly immense size. (4) Additionally, \method{} demonstrates strong transfer learning capabilities and supports substructure-conditioned generation without additional fine-tuning. Our work not only advances the field of graph generation but also opens new avenues for applying LLMs to graph-centric tasks, paving the way for building foundation models for graphs.

\section{Methods}\label{sec:methods}
\begin{figure*}[!htb]
    \centering
    \includegraphics[width=\textwidth]{img/autograph_pipeline_new.pdf}
    \vskip-.1in
    \caption{Overview of \method{}: (1) We use Algorithm~\ref{algo:sent_sampling} to sample a SENT $s$ from the input graph: $s=(s_1,s_2)$ with $s_1=((v_1,\emptyset),(v_2,\emptyset),(v_3,\emptyset))$ and $s_2=((v_5, \{v_2\}), (v_4,\emptyset))$. (2) We tokenize it by reindexing the vertices based on their first occurrence order in $s$ and adding special tokens (`\textbf{/}' represents breakage between segments, `$\bm{<}$' and `$\bm{>}$' indicate the start and end of a neighborhood set). (3) We perform the next token prediction on the tokenized sequences using a decoder-only transformer or any language model.}
    \label{fig:overview}
    \vskip -0.1in
\end{figure*}

In this section, we present an approach to transforming graphs into sequences, enabling their modeling akin to natural language. Our method hinges on a specialized class of random trail segments that ensure complete graph coverage. We begin by introducing the concept of segmented Eulerian trails (SET) and demonstrate theoretically why this representation alone is insufficient for effective graph generation. Subsequently, we propose an extension of SET, namely the segmented Eulerian neighborhood trail (SENT), which additionally incorporates neighborhood information alongside the trails. We elucidate sufficient conditions for effective generation and develop an efficient sampling strategy to obtain such SENTs. The section concludes with extensions and discusses how to model the SENTs autoregressively using language models, thus bridging the gap between graph learning and language modeling paradigms. An overview of \method{} is illustrated in Figure~\ref{fig:overview}, and backgrounds and proofs are provided in Appendix~\ref{app:sec:remarks} and \ref{app:sec:proofs}.

\subsection{Segmented Eulerian Trail}
To formalize our approach, we begin by introducing fundamental concepts in graph theory.
Let $V$ be a set of vertices and $E:=V\times V$ a set of edges. A graph is defined as a tuple $G=(V_G, E_G)$, where $V_G\subseteq V$ is a finite set of vertices and $E_G\subseteq V_G\times V_G$ is the set of edges. For simplicity and without loss of generality, we restrict our attention to undirected graphs without isolated vertices, where each edge is represented as an unordered pair $(u,v)$ for $u,v \in V$. We begin by defining the concept of a trail in a graph:
\begin{definition}[Walk and trail]
    A walk is a sequence of nodes connected by edges in $G$ and a trail is a walk in which all edges are distinct. Given a graph $G$, the set of trails in $G$ is denoted as $\Tcal_G$.
\end{definition}
Next, we generalize the concept of trails beyond the context of a specific graph:
\begin{definition}[Generalized trail]
    A generalized trail of length $k$ is defined as a sequence of nodes $w:=(w_0,\dots,w_k)\in V^{k+1}$ for $k\geq 0$ \st $(w_{i-1},w_{i})\neq (w_{j-1},w_{j})$, $\forall i,j\in [k]$ and $i\neq j$.
\end{definition}
The set of all generalized trails is denoted as $\Tcal$, noting that $\Tcal_G\subseteq \Tcal$ for any $G$. For a generalized trail $w\in\Tcal$, we define $V_w\subseteq V$ and $E_w\subseteq E$ as the sets of vertices and edges traversed by $w$, respectively, termed the \emph{generated sets} of $w$.
An \emph{Eulerian trail} is a trail that visits every edge in a graph exactly once. Such trails are of particular interest as they capture the complete topology of the graph. However, the existence of an Eulerian trail depends on specific conditions related to vertex degrees and connectivity~\citep{biggs1986graph}. To generalize this concept to arbitrary graphs, we introduce the notion of trail segments:
\begin{definition}[Segmented Eulerian trail (SET)]
    A segmented Eulerian trail (SET) in $G$ is a sequence of trail segments such that each edge is visited exactly once across all segments, and segments do not need to be connected. 
    Formally, a SET of size $k$ in $G$ is defined as $s:=(s_1,\dots,s_k)$ \st $s_i\in \Tcal_G$, and the generated edge sets of its segments form a partition of $E_G$, \ie $\cup_{i=1}^k E_{s_i}=E_G$ and $E_{s_i}\cap E_{s_j}=\emptyset, \forall i,j\in [k], i\neq j$. 
    Similarly, a SET (without relying on a specific graph) is defined as a sequence of generalized trails whose generated edge sets are disjoint.
\end{definition}
The set of all SETs in $G$ is denoted as $\Scal_G$, and the set of all SETs is denoted as $\Scal$.
For a SET $s=(s_i)_{i=1}^k$, we define the \emph{generated node and edge sets} as $V_s:=\cup_{i=1}^k V_{s_i}$ and $E_s:=\cup_{i=1}^k E_{s_i}$. The graph $G_s:=(V_s,E_s)$ is termed \emph{generated graph} of $s$. It is easy to show that $s$ is a SET in $G$ if $G_s\simeq G$.
Moreover, SETs can be classified into equivalence classes based on graph isomorphism, as formalized below:
\begin{definition}[SET isomorphism]
    For any two SETs $s,t\in \Scal$, we say they are isomorphic $s\simeq t$ if there is a bijection $\pi:V_s\to V_t$ between their generated node sets and $\pi(s)=t$ where $\pi$ applies elementwise to all nodes in $s$.
\end{definition}
This isomorphism partitions $\Scal$ into equivalence classes. Moreover, we have the following relationship between SETs and graphs, relevant for our tokenization (Sec.~\ref{sec:tokenization}):
\begin{theoremEnd}{theorem}\label{thm:set_isomorphism}
    For any SETs $s,t\in \Scal$, their generated graphs are isomorphic, \ie $G_s\simeq G_t$, if $s\simeq t$. Conversely, if two graphs $G\simeq H$, then for any SET $s\in\Scal_G$, there exists a SET $t\in\Scal_H$ \st $s\simeq t$.
\end{theoremEnd}
\begin{proofEnd}
    By definition of the isomorphism between $s$ and $t$, there exists a bijection $\pi:V_s\to V_t$ \st $\pi(s)=t$. Now if $u, v\in V_s$ are adjacent in $G_s$, \ie $(u,v)\in E_s$, then $(\pi(u),\pi(v))$ is an edge visited by $\pi(s)=t$, thus $(\pi(u),\pi(v))\in E_t$. Similarly, the reverse is also true. Consequently, $G_s\simeq G_t$.

    Now assume that $G\simeq H$ with an isomorphism $\pi$ and $s\in\Scal_G$. It is easy to show that $\pi(s)$ is also a SET and its generated graph $G_{\pi(s)}=H$. By taking $t=\pi(s)$, we obtain the result.
\end{proofEnd}
While a SET in $G$ fully characterizes its structure, we show below the prefixes of the SET do not necessarily describe the substructures of $G$, a critical property for effective autoregressive graph generation.
\begin{definition}[Flattening]
    The flattening of a sequence of sequences $s$ is the concatenation of all its sequences, denoted as $\concat s$.
\end{definition}
\begin{definition}[Prefix of a SET]
    For $s\in \Scal$, we call $t$ a prefix of $s$ if $\concat t$ is a prefix of $\concat s$.
\end{definition}
\begin{theoremEnd}{lemma}
    For any graph $G$ and SET $s$ in $G$, the generated graph of any prefix of $s$ is a subgraph of $G$, but not necessarily an induced subgraph.
\end{theoremEnd}
\begin{proofEnd}
    Assume that $t$ is a prefix of $s$. Then $V_t\subseteq V_s=V_G$ and $E_t\subseteq E_s$. However, $G_t$ is not necessarily an induced subgraph of $G_s$. We consider the following counter-example: $s=((1,2,3,4,1,3))$, $V_s=\{1,2,3,4\}$, and $E_s=\{(1,2),(2,3),(3,4),(1,4),(1,3)\}$. Let $t=((1,2,3,4,1))$. $t$ is clearly a prefix of $s$, but its generated graph is not an induced subgraph of $G_s$ as its generated edge set does contain $(1,3)$.
\end{proofEnd}
This result motivates us to extend the definition of generalized trails to incorporate the full structural information of the \emph{induced subgraphs}, rather than arbitrary subgraphs, to constrain the generation space better and address long-range dependency challenges. 
Without this extension, dependencies between neighboring nodes may span a long sequence of generation steps, making it more difficult for the model to learn such dependencies. Empirically, we show that SET fails to accurately generate graphs in Section~\ref{sec:ablation_experiments}.

\subsection{Segmented Eulerian Neighborhood Trail}
To make the prefixes of a SET encode richer information, we need to extend SET to contain neighborhood information in a graph. Thus, we consider the following definition:
\begin{definition}[Neighborhood sequence]
    A neighborhood sequence is a sequence of tuples $w:=(w_0,\dots,w_k)$ where $w_i=(v_i,A_i)$ with a node $v_i\in V$ and a neighborhood set $A_i\subseteq V$, $\forall i\in \{0,\dots,k\}$. $w$ is called Hamiltonian if its node sequence $n(w):=(v_0,\dots,v_k)$ has non-repeated elements. Moreover, $w$ is called \emph{causal} if $A_i$ only contains visited nodes, \ie $A_i\subseteq \{v_0,\dots,v_{i-1}\}$ $\forall i\in [k]$.
\end{definition}
We now extend this to trails:
\begin{definition}[Neighborhood trail]
    A neighborhood trail is a neighborhood sequence that satisfies two conditions. (i) $n(w)$ is a generalized trail. (ii) If we define the generated edge set of $w_i$ as $E_{w_i}=\{(v_i,u)\, |\, u\in A_i\}$, the family $\{ E_{n(w)}, E_{w_1},\dots, E_{w_k}\}$ is pairwise disjoint. Its union is called the generated edge set of $w$.
\end{definition}
The set of all neighborhood trails is denoted by $\TNcal$. For any $w\in\TNcal$, we denote by $G_w:=(V_w,E_w)$ the generated graph of $w$ where $V_w:=(\cup_{i=1}^k A_i)\cup V_{n(w)}$ is the generated node set and $E_w$ is the generated edge set. Note that a generalized trail is a neighborhood trail with $A_i=\emptyset, \forall i$. We extend SET to incorporate neighborhood information:
\begin{definition}[Segmented Eulerian neighborhood trail (SENT)]
    A segmented Eulerian neighborhood trail (SENT) of size $k$ is a sequence of neighborhood trails $s:=(s_1,\dots,s_k)$ with pairwise disjoint generated edge sets, \ie $s_i\in\TNcal$ and $E_{s_i}\cap E_{s_j}=\emptyset, \forall i,j\in [k], i\neq j$. 
\end{definition}
Similarly to SETs, the generated graph of a SENT $s$ is denoted by $G_s=(V_s,E_s)$. If a graph $G\simeq G_s$, we say that $s$ is a SENT in $G$. We denote by $\SNcal$ and $\SNcal_G$ the set of SENTs and SENTs in $G$. Analogously to SETs, we define an isomorphism over $\SNcal$ and obtain the same relationship as in Thm.~\ref{thm:set_isomorphism}. A prefix of a SENT is defined similarly to that of a SET. We give below conditions to force generated graphs of prefixes of a SENT to be induced subgraphs.
\begin{definition}[Causal SENT]
    A SENT $s$ is called causal if its flattening $\concat s$ is causal. 
\end{definition}
\begin{definition}[Hamiltonian and semi-hamiltonian SENT]
    A SENT $s$ is called Hamiltonian if its flattening $\concat s$ is Hamiltonian. $s$ is called semi-hamiltonian if $s$ is Hamiltonian, or for any nodes visited more than once, their occurrences after the first time should be in a start tuple of a neighborhood trail and their associated neighborhood sets are empty.
\end{definition}
    
    
    
    


    



    

\begin{theoremEnd}{theorem}\label{thm:induced_subgraph}
    For any causal SENT $s\in\SNcal$, the generated graph of any prefix $t$ of $s$ is an induced subgraph of $G_s$ if and only if $s$ is semi-hamiltonian. In this case, $s$ is called subgraph-induced. %
\end{theoremEnd}
\begin{proofEnd}
    Let us first introduce some notations.
    We denote by $R_s$ the sequence of the start tuples across all neighborhood trails in $s$, which is also a neighborhood sequence. By definition of semi-hamiltonian, the occurrences after the first time of a node in $s$ should be in $R_s$. We denote by $n(s)$ the associated node sequence of SENT $s$, \ie $n(s):=n(\concat s)$.

    Let us first assume that $s$ is semi-hamiltonian.
    
    Assume that $t$ is a prefix of $s$. It is easy to show that $G_t$ is a subgraph of $G_s$. Now assume that $u,v\in V_t$ \st $(u,v)\in E_s$, we want to show that $(u,v)\in E_t$. There are two cases:

    1) Assume that $u,v\in n(t)$. Since $s$ is semi-hamiltonian, $n(s)\setminus n(t)$ either does not contain $u$ or $v$, or even if one of them, say $u\in n(s)\setminus n(t)$, we have $u\in n(R_s)$ and its associated neighborhood set is empty. In both cases, the edge $(u,v)$ does not belong to the generated edge set of the neighborhood subsequence after $\concat t$. By the disjointness of the generated edge sets of $s$, it can only be included in the generated edge set of $t$, we thus have $(u,v)\in E_t$.

    2) Assume that one of them, say $u\notin n(t)$. There exists a neighborhood set $A$ in a tuple of $\concat t$ such as $u\in A$. Since $t$ is causal, we have $u\in n(t)$ which contradicts the assumption.

    In all the above cases, we have $(u,v)\in E_t$.

    Now let us assume that the generated graph of any prefix of $s$ is an induced subgraph of $G_s$.

    Let us prove that $s$ is semi-hamiltonian by contradiction. Assume that there exist two tuples in $\concat s$ with the same nodes $s_i=(v, A_i)$ and $s_j=(v,A_{j})$ with $i<j$. There are two cases: 1) $s_j\notin R_s$. A tuple $(u,A_u)$ exists one step before $s_j$ in the same neighborhood trail. We consider the prefix $t$ ending at $(u, A_u)$. We have $v,u\in V_t$ and $(u,v)\in E_s$, but $(u,v)\notin E_t$, by the disjointness of $s$ and since $(u,v)$ is visited at $s_j$ after $t$. 2) $s_j\notin R_s$ and $A_j\neq\emptyset$. Since $s$ is causal, there exists $s_u:=(u,A_u)$ before $s_j$ \st $u\in A_j$. We consider the prefix $t$ ending at exactly this tuple. We have $u,v\in V_t$ and $(u,v)\in E_s$, but $(u,v)\notin E_t$, by the disjointness of $s$ and since $(u,v)$ is an edge visited at $(v,A_j)$ after $t$.
\end{proofEnd}
Now let us find the conditions for a causal and Hamiltonian SENT. For any SENT $s$ and a tuple $w:=(v,A)$ in $s$, we denote by $V_s(w)$ the set of nodes visited by $s$ before $w$, excluding the node linked to $v$ through the trail if it exists. We have the following necessary and sufficient conditions:
\begin{theoremEnd}{theorem}\label{thm:causal_hamiltonian_sent}
    For $s\in\SNcal_G$, $s$ is causal and Hamiltonian if and only if every tuple $w:=(v, A_v)$ in $\concat s$ satisfies $A_v=\Ncal_G(v)\cap V_s(w)$. In this case, every node is visited exactly once. Moreover, $s$ is causal and semi-hamiltonian if and only if every tuple $w:=(v, A_v)$ in $s$ satisfies either $A_v=\Ncal_G(v)\cap V_s(w)$ or $A_v=\emptyset$.
\end{theoremEnd}
\begin{proofEnd}
    Let us first assume that for any tuple $w:=(v, A_v)$ in $\concat s$, $A_v=\Ncal_G(v)\cap V_s(w)$.
    Since $A_v\subseteq V_s(w)$ which is a subset of the set of visited nodes, $s$ is causal. Now we prove $s$ is Hamiltonian by contradiction. Assume that there exist two tuples in $\concat s$, $s_u:=(u,A_u)$ and a later visited one $s_v:=(v,A_v)$ \st $u=v$. Then, $A_v=\Ncal_G(v)\cap V_s(s_v)=\Ncal_G(u)\cap V_s(s_v)$ should contain the node visited before that is a neighbor of $u$ (either through a trail or the neighborhood set of $u$), denoted by $u'$. Thus, the edge $(u,u')$ has been visited twice, which contradicts the disjointness of $s$.

    Assuming that $A_v=\Ncal_G(v)\cap V_s(w)$ or $A_v=\emptyset$ for any tuple $(v, A_v)$ in $s$, we can also prove $s$ is semi-hamiltonian by contradiction. Assume that there exist two tuples in $\concat s$, $s_u:=(u,A_u)$ and a later visited one $s_v:=(v,A_v)$ \st $u=v$ and $A_v\neq \emptyset$. Then, $A_v=\Ncal_G(v)\cap V_s(s_v)=\Ncal_G(u)\cap V_s(s_v)$ by assumption. And using the same argument as above, we have the contradiction.

    Now assume that $s$ is causal and Hamiltonian. Let us prove the other direction by contradiction. There exists a tuple $w:=(v,A_v)$ in $s$ \st $A_v\neq \Ncal_G(v)\cap V_s(w)$. As $s$ is causal, $A_v\subseteq V_s(w)$. $A_v\subseteq \Ncal_G(v)$ as $s\in\SNcal_G$. Thus, $A_v\subset \Ncal_G(v)\cap V_s(w)$, which means that there exists $u\in \Ncal_G(v)\cap V_s(w)$ and $u\notin A_v$. Hence, $(u, v)\in E_G$ and $u$ is visited before $v$. However, as $u\notin A_v$, $(u, v)\in E_G$, and $s$ is Hamiltonian, there exists a tuple $(u, A_u)$ in $\concat s$ \st $v\in A_u$. By causality of $s$, $v$ is visited before $u$, which contradicts the fact that $s$ is Hamiltonian.

    Assuming that $s$ is causal and semi-hamiltonian. Let us prove the other direction by contradiction. There exists a tuple $w:=(v,A_v)$ in $s$ \st $A_v\neq \Ncal_G(v)\cap V_s(w)$ and $A_v\neq \emptyset$. Using the same arguments as above, there exists $(u, v)\in E_G$, and $u$ is visited before $v$. However, as $u\notin A_v$ and $(u, v)\in E_G$, $s$ should visit the edge $(u,v)$ at some point. Since $s$ is semi-hamiltonian, if $s$ visits again $u,v$ they can only be the first nodes and their associated neighborhood sets are empty. Hence, there is no means for $s$ to visit $(u,v)$ after $v$, leading to contradiction.
\end{proofEnd}
This theorem offers a simple sufficient condition for subgraph-induced SENTs. We provide in the following an implementation through a random path sampling strategy. 

\subsection{Sampling Algorithm for SENT}
Thm.~\ref{thm:causal_hamiltonian_sent} offers a simple strategy to sample a causal and Hamiltonian SENT: one needs to traverse the graph and choose the neighborhood set as all neighbors of the current node that have been visited. The traversing strategy could be achieved through a random path sampling or a depth-first search. In Algorithm~\ref{algo:sent_sampling}, we provide a sampling strategy based on random path sampling with breaks.
\begin{algorithm}
    \caption{Causal and Hamiltonian SENT Sampling}\label{algo:sent_sampling}
    \begin{algorithmic}[1]
        \Require $G=(V, E)$
        \Ensure A SENT $s$ in $G$
        \State Set of unvisited nodes $U \gets V$
        \State $s \gets []$
        \State $v \gets \texttt{RandomSample}(U)$; $U \gets U\setminus \{v\}$
        \State $t \gets [(v,\emptyset)]$ \Comment{first neighborhood trail}
        \While{$U \neq \emptyset$}
        \If{$\Ncal_G(v)\cap U=\emptyset$} \Comment{start a new trail}
        \State $s.\texttt{append}(t)$
        \State $v \gets \texttt{RandomSample}(U)$; $U \gets U\setminus \{v\}$
        \State $A \gets \Ncal_G(v)\cap (V\setminus U)$
        \State $t \gets [(v, A)]$
        \Else \Comment{sample the next node in the trail}
        \State $u \gets \texttt{RandomSample}(\Ncal_G(v)\cap U)$
        \State $U \gets U\setminus \{u\}$
        \State $A \gets (\Ncal_G(u)\setminus \{v\})\cap (V\setminus U)$
        \State $t.\texttt{append}((u, A))$
        \State $v \gets u$
        \EndIf
        \EndWhile
    \end{algorithmic}
\end{algorithm}

\vspace{-.15in}
\paragraph{Complexity analysis.}
The length of a SENT including the sizes of neighborhood sets (in other words, tokenized SENT defined in Section~\ref{sec:tokenization}) is bounded by the number of edges as it can visit each edge only once. Therefore, both the time and space complexity of sampling a SENT from graph $G$ are $\mathcal{O}(m)$ where $m$ is the number of edges. 

\subsection{Tokenization of SENT}\label{sec:tokenization}
Previous works have explored related concepts of sequences in graphs. For example,
\citet{you2018graphrnn} investigated causal Hamiltonian neighborhood sequences generated through breadth-first search, while \citet{liao2019efficient,goyal2020graphgen} constructed SENT-like sequences using depth-first search. However, neither of these works interpreted these sequences as a language. Here, we present a method to bridge the gap between graph generation and language modeling.

The tokenization process starts by mapping all isomorphic SENTs to the same sequence, by reindexing the vertices according to their first occurrence order within the sequence. Specifically, if we denote this ordering function for a SENT $s$ by $\pi: V_s\to \{1,\dots,|V_s|\}$, $s$ is then replaced with its ordered representation $\pi(s)$. Thanks to the isomorphism property of SENT (Thm.~\ref{thm:set_isomorphism}), $\pi(s)$ generates a graph isomorphic to $G_s$ while ensuring the obtained sequence invariant to the node ordering of the input graph.

To convert an (ordered) SENT into a machine-readable sequence, we tokenize it into a sequence of indices using special tokens. These tokens include symbols such as ` \textbf{/} ' to indicate a breakage between segments, and `$\bm{<}$' and `$\bm{>}$' to mark the start and end of a neighborhood set. Specifically, for any $s:=(s_1,\dots, s_k)\in\SNcal$, we define the tokenization function \texttt{Token} as follows:
\begin{equation*}
    \texttt{Token}(s):=\texttt{Token}(s_1)\concat [~\textbf{/}~] \concat 
    \cdots \concat [~\textbf{/}~] \concat \texttt{Token}(s_k),
\end{equation*}
where
\begin{equation*}
    \texttt{Token}(s_i):=\concat_{w\in s_i} \texttt{Token}(w),
\end{equation*}
and for each tuple $w:=(v,A)$ with the \emph{sorted set} $A= \{u_1,\dots,u_p\}$ (due to the reindexing by $\pi$), we define:
\begin{equation*}
    \texttt{Token}(w):=\left[v, \bm{<}, u_1,\dots, u_p, \bm{>}\right].
\end{equation*}
This process converts a SENT into a sequence of tokens that a language model can effectively model. Using an equivalent form, the resulting tokenization induces a \emph{non-Markovian} random walk in the graph, incorporating additional virtual nodes labeled with the above special tokens (see Appendix~\ref{app:sec:random_walk_interpretation} for more details). \emph{Language modeling of SENTs aims to learn the state transition probabilities}.

\subsection{Extension to Attributed Graphs}
Our method can be easily extended to graphs with categorical (or discretized) attributes by inserting node and edge attributes in an interleave fashion into the tokenized SENT sequence. Specifically, let $L_{\mathrm{node}}(v)$ and $L_{\mathrm{edge}}(u,v)$ be the attributes of a node $v$ and an edge $(u,v)$ respectively. Using the same notation as above, we define for any $s_i:=(w_1,\dots, w_q)\in\TNcal$ with $w_i=(v_i,\cdot)$:
\begin{equation*}
    \begin{aligned}
        \texttt{Token}(s_i)&:=\texttt{Token}(w_1) \concat [L_{\mathrm{edge}}(v_1,v_2)] \concat \texttt{Token}(w_2) \\
        & \concat \dots \concat \texttt{Token}(w_q),  \\
        \texttt{Token}(w)&:=[v, L_{\mathrm{node}}(v), \bm{<}, L_{\mathrm{edge}}(v,u_1), u_1,\\
        & \dots, L_{\mathrm{edge}}(v,u_p),u_p, \bm{>}].
    \end{aligned}
\end{equation*}

\subsection{Autoregressive Modeling of Tokenized SENTs}
The sampling and tokenization of SENTs in graphs allows for transforming graphs into sequences, which could be modeled by language models. Specifically, given a graph $G$ represented as a SENT $s$, which consists of a sequence of tokens $(s_1,\dots, s_n)$, a standard language modeling objective is to maximize the following log-likelihood:
\begin{equation}
    p(s)=\sum_{i=1}^n \log p_\theta(s_i\,|\, s_1,\dots, s_{i-1}),
\end{equation}
where the conditional probability $p_\theta$ is modeled using a neural network with parameters $\theta$. The architecture of the neural network can be any SOTA sequence model.




\section{Related Work}
\paragraph{Autoregressive models for graph generation.}
Autoregressive models generate graphs by sequentially adding nodes and edges. GraphRNN~\citep{you2018graphrnn} pioneered this approach by framing graph generation as a sequence prediction task, demonstrating the capacity of recurrent neural networks~(RNNs)~\citep{chung2014empirical} to capture complex structures. DeepGMG~\citep{li2018learning} introduced a probabilistic policy framework for conditional generation, while GRAN~\citep{liao2019efficient} and BiGG~\citep{dai2020bigg} enhanced efficiency and scalability by generating multiple nodes and edges in parallel.

Recent research has focused on optimizing the generation order. \citet{chen2021order} highlighted that the ordering of node and edge additions impacts graph quality, and GraphARM~\citep{kong2023autoregressive} applied reinforcement learning to dynamically refine this order. \citet{goyal2020graphgen} incorporated logical constraints to improve domain-specific generation, and \citet{bacciu2020edge} proposed Bayesian reasoning to better capture graph dependencies.

Although these models have shown to be efficient and effective in synthetic datasets, they face inherent unification limitations, reducing their applicability to large-scale real-world datasets. Our proposed unified and powerful sequence representation of graphs aims to address these challenges.

\vspace{-.1in}
\paragraph{Other graph generative models.}

Other graph generative models include variational, GAN-based, and diffusion-based approaches. GraphVAEs~\citep{kipf2016variational,simonovsky2018graphvae} employ variational autoencoders to learn continuous latent representations, effectively generating small graphs but struggling with more complex structures. GAN-based models, such as NetGAN~\citep{bojchevski2018netgan} and SPECTRE~\citep{martinkus2022spectre}, generate graphs by modeling graph descriptors like random walks and spectral features.

Diffusion-based models iteratively refine noise into structured graphs through reverse diffusion steps. Continuous diffusion models~\citep{niu2020permutation,jo2022score} adapt denoising diffusion probabilistic models for graph generation. To leverage graph sparsity and structure, discrete diffusion models~\citep{vignac2023digress,kong2023autoregressive} have been developed. However, a key challenge for these models is the slow sampling process due to the long reverse diffusion chain. To mitigate this limitation, several efficient diffusion techniques have been proposed, including EDGE~\citep{chen2023efficient}, HiGen~\citep{karami2024higen}, ESGG~\citep{bergmeister2024efficient}, and Pard~\citep{zhao2024pard}.

\vspace{-.1in}
\paragraph{Random walks for graph learning.}
Random walks have been widely used in graph learning due to their strong expressive power. GCKN~\citep{chen2020convolutional} and RWGNN~\citep{nikolentzos2020random} utilize path and walk kernels to learn graph representations. Several recent works~\citep{ivanov2018anonymous,wang2021inductive,yin2022algorithm} explicitly integrate random walk sequences with positional encodings, inspiring subsequent methods such as CRaWL~\citep{tonshoff2023crawl}, NeuralWalker~\citep{chen2024neuralwalker} and RWNN~\citep{kim2024revisiting}, which further enhance graph representation learning via random walk sequence modeling. GraphGPT~\citep{zhao2023graphgpt} leverages Eulerian paths to improve graph property prediction. Our work uniquely explores random sequence representations of graphs focusing on graph generation, introducing a novel perspective on combining random walks and language modeling for scalable graph generative modeling.

\section{Experiments}
In this section, we evaluate the performance of \method{} on several graph generation benchmarks, including both small and large graphs, and synthetic and real-world molecular datasets. Our experiments compare its performance to several SOTA methods and particularly focus on evaluating the following aspects: (1) We show its ability to generate relatively small graphs with a 100-fold inference speedup compared to diffusion-based models while maintaining or even improving structural validity. (2) We show its ability to scale to large graphs without loss of performance. (3) We demonstrate its effectiveness in generating real-world graphs with attributes with a focus on molecular generation, outperforming SOTA diffusion models. (4) We showcase its strong transfer capabilities and its ability to perform substructure-conditioned generation without any additional fine-tuning.
Additional details on experimental settings and evaluation are provided in Appendix~\ref{app:sec:experimental_details}.

\begin{table}[tbp]
    \centering
    \vskip -0.1in
    \caption{Comparison of \method{} to SOTA methods on Planar}
    \label{tab:planar}
    \begin{small}
    \begin{sc}
    \resizebox{\columnwidth}{!}{
    \begin{tabular}{lcccccc}\toprule
         & \multicolumn{6}{c}{Planar Graphs} \\ 
         & \multicolumn{6}{c}{$n_{\mathrm{graphs}}=128$, $|V|=64$} \\ \cmidrule{2-7}
        Model & Deg. & Clus. & Orbit & Spec. & Ratio & VUN  \\ \midrule
        Training set &  0.0002 & 0.0310 & 0.0005 & 0.0038 & 1.0 & --  \\ \midrule
        GraphRNN &  0.0049 & 0.2779 & 1.2543 & 0.0459 & 638.5 & 0.0 \\
        GRAN & 0.0007 & 0.0426 & 0.0009 & 0.0075 & 2.1 & 0.0 \\
        SPECTRE &  0.0005 & 0.0785 & 0.0012 & 0.0112 & 2.6 & 25.0 \\
        EDGE &  0.0761 & 0.3229 & 0.7737 & 0.0957 & 490.9 & 0.0 \\
        DiGress &  0.0007 & 0.0780 & 0.0079 & 0.0098 & 6.1 & 77.5 \\ 
        ESGG & 0.0005 & 0.0626 & 0.0017 & 0.0075 & 2.5 & \textbf{95.0} \\ \midrule
        \method{} & 0.0004 & 0.0605 & 0.0003 & 0.0064 & \textbf{1.5} & 87.5 \\
        \bottomrule
    \end{tabular}
    }
    \end{sc}
    \end{small}
    \vskip -0.2in
\end{table}

\vspace{-.15in}
\paragraph{Implementation details.}

We employ the LLaMA model with 12 layers and a hidden dimension of 768 as our sequence model backbone across all experiments, aligning with the architecture of GPT-2's smallest variant~\citep{radford2019gpt2}. Although prior works have used smaller models, we argue that our approach still demonstrates better scalability and faster training and inference speeds compared to diffusion models. For inference, we adopt the commonly used top-k sampling strategy~\citep{fan2018topk}. Our implementation leverages the Hugging Face framework~\citep{jain2022huggingface}, providing users with a flexible interface to experiment with SOTA language models for graph generation.

\vspace{-.15in}
\paragraph{Evaluation.}
For fair comparison, we align our evaluation methodology with established practices from prior works~\citep{you2018graphrnn,martinkus2022spectre,vignac2023digress}. Our evaluation compares generated samples against the test set using maximum mean discrepancy (MMD)~\citep{gretton2012kernel}, computed across multiple graph descriptors: node degree distributions (\textsc{Deg.}), clustering coefficients (\textsc{Clus.}), orbit count statistics (\textsc{Orbit}), and eigenvalue spectra (\textsc{Spec.}). 
As a reference, we also compute these metrics on the training set and report the mean ratio across all properties (\textsc{Ratio}).

For synthetic datasets, we additionally assess model performance using the VUN metric, the proportion of generated graphs that are valid, unique, and novel. Our efficiency analysis includes two measurements: inference speed, calculated as the per-graph generation time when producing 1024 graphs, and training efficiency, measured as the time required to achieve a VUN score of 75.0 for the Planar dataset and 60.0 for the SBM dataset. All efficiency measurements are performed on one NVIDIA H100 GPU.

For molecular generation datasets, we strictly follow the evaluation metrics used in DiGress~\citep{vignac2023digress}. More details about evaluation are provided in Appendix~\ref{app:sec:evaluation_metrics}.

\subsection{Comparison to State-of-the-Art Methods}
We evaluate the performance of \method{} compared to other SOTA graph generative models using the standard setting without pre-training. 

\begin{table}[tbp]
    \centering
    \vskip -0.1in
    \caption{Comparison of \method{} to SOTA methods on SBM}
    \label{tab:sbm}
    \begin{small}
    \begin{sc}
    \resizebox{\columnwidth}{!}{
    \begin{tabular}{lcccccc}\toprule
         & \multicolumn{6}{c}{Stochastic Block Models} \\ 
         & \multicolumn{6}{c}{$n_{\mathrm{graphs}}=128$, $|V|_{\max}=187$, $|V|_{\avg}\approx 104$} \\ \cmidrule{2-7}
        Model & Deg. & Clus. & Orbit & Spec. & Ratio & VUN  \\ \midrule
        Training set & 0.0008 & 0.0332 & 0.0255 & 0.0027 & 1.0 & -- \\ \midrule
        GraphRNN & 0.0055 & 0.0584 & 0.0785 & 0.0065 & 3.5 & 5.0 \\
        GRAN & 0.0113 & 0.0553 & 0.0540 & 0.0054 & 5.0 & 25.0 \\
        SPECTRE & 0.0015 & 0.0521 & 0.0412 & 0.0056 & \textbf{1.8} & 52.5\\
        EDGE & 0.0279 & 0.1113 & 0.0854 & 0.0251 & 12.7 & 0.0 \\
        DiGress & 0.0018 & 0.0485 & 0.0415 & 0.0045 & \textbf{1.8} & 60.0   \\ 
        ESGG & 0.0119 & 0.0517 & 0.0669 & 0.0067 & 5.4 & 45.0 \\ \midrule
        \method{} & 0.0077 & 0.0519 & 0.0439 & 0.0040 & 3.4 & \textbf{92.5} \\
        \bottomrule
    \end{tabular}
    }
    \end{sc}
    \end{small}
    \vskip -0.1in
\end{table}

\subsubsection{Small Synthetic Graph Generation}\label{sec:small_synthetic_graph}
We first evaluate our method on the small synthetic graph datasets introduced by~\citet{martinkus2022spectre}, including the Planar and SBM datasets. We compare the performance of \method{} against GraphRNN~\citep{you2018graphrnn}, GRAN~\citep{liao2019efficient}, SPECTRE~\citep{martinkus2022spectre}, EDGE~\citep{chen2023efficient}, DiGress~\citep{vignac2023digress}, ESGG~\citep{bergmeister2024efficient}. As shown in Tables~\ref{tab:planar} and \ref{tab:sbm}, \method{} achieves the best and second-best MMD ratios on average while ranking second-best and best in terms of VUN scores for the Planar and SBM datasets, respectively. Notably, all previous methods exhibit limited structural validity on the SBM dataset, with the best VUN scores reaching only 60.0.

Additionally, we assess the training and inference times of \method{} against representative models, including DiGress, GRAN, and ESGG. As presented in Table~\ref{tab:time_comparison}, \method{} is approximately 3 times faster during training and 100 times faster during inference compared to diffusion-based models. This substantial speedup over diffusion-based models is even more pronounced than that observed in other data modalities such as images~\citep{tian2024var}.

\begin{table}[tbp]
    \centering
    \vskip -0.1in
    \caption{Time comparison of {\scriptsize \method{}} to representative models. OOT indicates the model never reaches the target VUN.}
    \label{tab:time_comparison}
    \begin{small}
    \begin{sc}
    \resizebox{\columnwidth}{!}{
    \begin{tabular}{llcccc}\toprule
        Dataset & Time& DiGress & Gran & ESGG & \method{} \\ \midrule
        \multirow{2}{*}{Planar} & Training & 25.9h & OOT & 7.4h & \textbf{6.2h (4.2$\times$)} \\
        & Inference & 2.84s & 0.03s & 4.60s & \textbf{0.01s (284$\times$)} \\ \midrule
        \multirow{2}{*}{SBM} & Training & 47.7h & OOT & OOT & \textbf{13.8h (3.5$\times$)} \\
        & Inference & 13.05s & \textbf{0.13s} & 30.0s & \textbf{0.14s (93$\times$)} \\
        \bottomrule
    \end{tabular}
    }
    \end{sc}
    \end{small}
    \vskip -0.25in
\end{table}

\begin{table*}[tbp]
    \centering
    \vskip -0.1in
    \caption{Comparison of \method{} to SOTA methods on the Proteins and Point Clouds datasets. OOM indices out of memory.}
    \label{tab:protein_and_pointcloud}
    \begin{small}
    \begin{sc}
    \resizebox{.8\textwidth}{!}{
    \begin{tabular}{lccccc|ccccc}\toprule
         & \multicolumn{5}{c}{Proteins} &  \multicolumn{5}{c}{Point Clouds} \\ %
         & \multicolumn{5}{c}{$n_{\mathrm{graphs}}=587$, $|V|_{\max}=500$, $|V|_{\avg}\approx 258$} & \multicolumn{5}{c}{$n_{\mathrm{graphs}}=26$, $|V|_{\max}=5037$, $|V|_{\avg}\approx 1332$} \\ \cmidrule{2-11}
        Model & Deg. & Clus. & Orbit & Spec. & Ratio & Deg. & Clus. & Orbit & Spec. & Ratio \\ \midrule
        Training set & 0.0003 & 0.0068 & 0.0032 & 0.0005 & 1.0 & 0.0000 & 0.1768 & 0.0049 & 0.0043 & 1.0 \\ \midrule
        GraphRNN & 0.0040 & 0.1475 & 0.5851 & 0.0152 & 62.1 & OOM & OOM & OOM & OOM & OOM \\
        GRAN & 0.0479 & 0.1234 & 0.3458 & 0.0125 & 77.7 & 0.0201 & 0.4330 & 0.2625 & 0.0051 & 19.1 \\
        SPECTRE & 0.0056 & 0.0843 & 0.0267 & 0.0052 & 12.5 & OOM & OOM & OOM & OOM & OOM \\
        EDGE & 0.1863 & 0.3406 & 0.6786 & 0.1075 & 274.5 & 0.4441 & 0.3298 & 1.0730 &  0.4006 & 104.7 \\
        DiGress &  0.0041 & 0.0489 & 0.1286 & 0.0018 & 16.2 & OOM & OOM & OOM & OOM & OOM \\ 
        ESGG & 0.0030 & 0.0309 & 0.0047 & 0.0013 & 4.7 & 0.0139 & 0.5775 & 0.0780 & 0.0055 & 6.8 \\ \midrule
        \method{} & 0.0004 & 0.0244 & 0.0056 & 0.0013 & \textbf{2.3} & 0.0307 & 0.3031 & 0.0167 & 0.0171 & \textbf{3.0} \\
        \bottomrule
    \end{tabular}
    }
    \end{sc}
    \end{small}
    \vskip -0.1in
\end{table*}

\subsubsection{Large Graph Generation}\label{sec:large_graph_generation}

To understand the scalability of \method{}, we evaluate its performance on the Proteins and Point Clouds datasets used by~\citet{liao2019efficient}. The results, shown in Table~\ref{tab:protein_and_pointcloud}, demonstrate that even when using a context window shorter than the longest sequence sampled from the dataset, \method{} achieves MMD ratios comparable to those observed on the Planar and SBM datasets. Furthermore, \method{} outperforms all existing methods in terms of MMD ratio, achieving a twofold or more improvement over the previous best model, ESGG. More significantly, while ESGG was specifically designed for generating unattributed graphs, \method{} demonstrates versatility by being applicable to both unattributed and attributed graphs.


\begin{table}[tbp]
    \centering
    \vskip -0.1in
    \caption{Comparison of {\scriptsize \method{}} to SOTA methods on QM9}
    \label{tab:qm9}
    \begin{small}
    \begin{sc}
    \resizebox{\columnwidth}{!}{
    \begin{tabular}{lccccc}\toprule
         & \multicolumn{5}{c}{QM9 with hydrogen atoms} \\ %
         & \multicolumn{5}{c}{$n_{\mathrm{graphs}}=100$K, $|V|_{\max}=29$, $|V|_{\avg}\approx 18$} \\ \cmidrule{2-6}
        Model & Valid$\shortuparrow$ & Unique$\shortuparrow$ & Novel$\shortuparrow$ & Atom stable$\shortuparrow$ & Mol stable$\shortuparrow$ \\ \midrule
        DiGress & 95.4 & \textbf{97.6} & 33.4 & 98.1 & 79.8 \\ \midrule
        \method{} & \textbf{97.7} & 96.7 & \textbf{45.5} & \textbf{98.6} &  \textbf{87.3}\\
        \bottomrule
    \end{tabular}
    }
    \end{sc}
    \end{small}
    \vskip -0.1in
\end{table}

\subsubsection{Molecular Graph Generation}

We demonstrate the applicability of our method to generating real-world attributed graphs, such as molecular structures. We evaluate \method{} on the same datasets used by DiGress~\citep{vignac2023digress}, including QM9 (all atoms)~\citep{wu2018moleculenet}, MOSES~\citep{polykovskiy2020moses}, and GuacaMol~\citep{brown2019guacamol}. Following the data splits and experimental setup from DiGress, we benchmark \method{} against a variety of SOTA models, including DiGress, VAE on SMILES~\citep{polykovskiy2020moses}, JT-VAE~\citep{jin2018junction}, GraphINVENT~\citep{mercado2021graphinvent}, NAGVAE~\citep{kwon2020nagvae}, LSTM and MCTS~\citep{brown2019guacamol}. On the QM9 dataset (Table~\ref{tab:qm9}), \method{} outperforms DiGress across all metrics except uniqueness, showing its superiority for attributed graphs.

For the more challenging MOSES and GuacaMol datasets, \method{} also demonstrates superior performance, achieving higher validity and improved distributional alignment as measured by metrics like FCD, as shown in Tables~\ref{tab:moses} and \ref{tab:guacamol}. Notably, to our best knowledge, \method{} is the first autoregressive model for graphs to surpass diffusion-based approaches on these datasets. It is worth mentioning that all metrics were computed using SMILES representations rather than molecular graphs. Due to the non-reversible nature of converting SMILES to graphs and back, where approximately 20\% of molecules cannot be mapped back to their original SMILES~\citep{vignac2023digress}, some discrepancies are introduced when calculating these metrics. Despite these challenges, \method{} achieves validity and FCD scores comparable to SMILES-based methods.

Furthermore, \method{} demonstrates remarkable efficiency, with training times of less than one day on both datasets, compared to up to one week for DiGress~\citep{vignac2023digress}. This substantial reduction in training time underscores \method{}'s practical advantages in large-scale and high throughput molecular graph generation tasks.


\begin{table}[tbp]
    \centering
    \vskip -0.1in
    \caption{Comparison of {\scriptsize \method{}} to SOTA methods on MOSES}
    \label{tab:moses}
    \begin{small}
    \begin{sc}
    \resizebox{\columnwidth}{!}{
    \begin{tabular}{llcccccc}\toprule
         & & \multicolumn{6}{c}{MOSES} \\ %
         & & \multicolumn{6}{c}{$n_{\mathrm{graphs}}= 1.58$M, $|V|_{\max}=27$, $|V|_{\avg}\approx 22$} \\ \cmidrule{3-8}
        Model & Type & Valid$\shortuparrow$ & Unique$\shortuparrow$ & Novel$\shortuparrow$ & Filters$\shortuparrow$ & FCD$\shortdownarrow$ & SNN$\shortdownarrow$ \\ \midrule %
        VAE & SMILES & 97.7 & 99.8 & 69.5 & 99.7 & 0.57 & 0.58 \\ %
        JT-VAE & Fragments & 100 & 100 & 99.9 & 97.8 & 1.00 & 0.53 \\ %
        GraphINVENT & Graph & 96.4 & 99.8 & – & 95.0  & 1.22 & 0.54 \\ %
        DiGress & Graph & 85.7 & 100 & 95.0 & 97.1 & 1.19 & 0.52 \\ \midrule %
        \method{} & Graph & 87.4 & 100 & 85.9 & 98.6 & 0.91 & 0.55 \\  %
        \bottomrule
    \end{tabular}
    }
    \end{sc}
    \end{small}
    \vskip -0.1in
\end{table}

\begin{table}[tbp]
    \centering
    \vskip -0.1in
    \caption{Comparison of {\scriptsize \method{}} to SOTA methods on {\scriptsize GuacaMol}}
    \label{tab:guacamol}
    \begin{small}
    \begin{sc}
    \resizebox{\columnwidth}{!}{
    \begin{tabular}{llccccc}\toprule
         & & \multicolumn{5}{c}{GuacaMol} \\ %
         & & \multicolumn{5}{c}{$n_{\mathrm{graphs}}=1.1$M, $|V|_{\max}=88$, $|V|_{\avg}\approx 28$} \\ \cmidrule{3-7}
        Model & Type & Valid$\shortuparrow$ & Unique$\shortuparrow$ & Novel$\shortuparrow$ & KL div$\shortuparrow$ & FCD$\shortuparrow$ \\ \midrule
        LSTM & SMILES & 95.9 & 100 & 91.2 & 99.1 & 91.3 \\
        NAGVAE & Graph & 92.7 & 95.5 & 100 & 38.4 & 0.9 \\
        MCTS & Graph & 100 & 100 & 99.4 & 52.2 & 1.5 \\
        DiGress & Graph & 85.2 & 100 & 99.9 & 92.9 & 68.0 \\ \midrule
        \method{} & Graph & 91.6 & 100 & 97.7 & 97.5 & 79.2  \\
        \bottomrule
    \end{tabular}
    }
    \end{sc}
    \end{small}
    \vskip -0.15in
\end{table}

\subsection{Transfer Performance of \method{}}
We evaluate the transferability of \method{} by pre-training it on a large dataset of synthetic graphs generated using NetworkX~\citep{hagberg2008networkx} and fine-tuning it on the unattributed graph datasets. Dataset and experimental details are provided in Appendix~\ref{app:sec:experimental_details}. As shown in Table~\ref{tab:transferability}, the pre-trained model consistently outperforms the baseline on small synthetic datasets in terms of the VUN score, achieving near-perfect validity. On larger graph datasets, the pre-trained model also surpasses the baseline across MMD metrics, demonstrating its ability to generalize to more complex structures. However, on small synthetic datasets, the pre-trained model shows a slight decline in MMD metrics compared to the baseline. These findings highlight the potential of building foundation models for graph generation and underscore the need for more comprehensive benchmark datasets, similar to those established in other domains.

\begin{table}[tbp]
    \centering
    \vskip -0.1in
    \caption{Transfer performance on downstream tasks using \method{} pre-trained on the NetworkX dataset. Red and green colors indicate relative decreases and increases respectively, compared to \method{} without pre-training.}
    \label{tab:transferability}
    \begin{small}
    \begin{sc}
    \resizebox{\columnwidth}{!}{
    \begin{tabular}{lcccccc}\toprule
        Dataset & Deg. & Clus. & Orbit & Spec. & Ratio & VUN (improv.) \\ \midrule
        NetworkX  & 0.0016 & 0.0073 & 0.0068 & 0.0020 & -- & -- \\ \midrule
        Planar & \cellcolor{red!75} 0.0007 & \cellcolor{red!34} 0.0811 & \cellcolor{red!66} 0.0005 & \cellcolor{green!5} 0.0061 & \cellcolor{red!47} 2.2 & \cellcolor{green!9} 95.0 (+7.5) \\
        SBM & \cellcolor{red!29} 0.0099 & \cellcolor{red!9} 0.0566 & \cellcolor{red!95} 0.0854 & \cellcolor{red!62.5} 0.0065 & \cellcolor{red!41} 4.8 & \cellcolor{green!5} 97.5 (+5) \\
        Proteins & \cellcolor{green!100} 0.0002 & \cellcolor{green!33} 0.0183 & \cellcolor{green!47} 0.0038 & \cellcolor{green!9} 0.0012 & \cellcolor{green!35} 1.7 & -- \\
        Point Clouds & \cellcolor{green!100} 0.0154 & \cellcolor{green!17} 0.2591 & \cellcolor{green!100} 0.0076 & \cellcolor{red!38} 0.0236 & \cellcolor{green!7} 2.8 & --  \\ \bottomrule
    \end{tabular}
    }
    \end{sc}
    \end{small}
    \vskip -0.1in
\end{table}

\subsection{Substructure Conditioned Generation}\label{sec:substructure_conditioned_generation}
We explore the ability of \method{} to perform substructure-conditioned generation without requiring fine-tuning. Given a subgraph $S$ (which could represent a functional motif of interest in drug discovery), we flatten the subgraph into a SENT sequence and condition the generation process on this sequence. This approach guarantees that the generated graph will contain $S$ as an induced subgraph (Thm.~\ref{thm:induced_subgraph}). As a proof-of-concept, we follow the methodology of~\citet{vignac2023digress,maziarz2022learning} and generate molecular graphs starting from a specific motif, called 1,4-Dihydroquinoline\footnote{\url{https://pubchem.ncbi.nlm.nih.gov/compound/1_4-Dihydroquinoline}}, using the model pre-trained on the GuacaMol dataset. Our results in Table~\ref{tab:substructure_conditioned_generation} demonstrate that this approach maintains similar validity, uniqueness, and novelty to unconditional generation (Table~\ref{tab:guacamol}). To further showcase the flexibility of this method, we test more extreme cases by replicating the same motif multiple times before performing the conditional generation. While validity decreases significantly when using an unrealistically large number of copies (\eg 5), the model still generates some visually plausible molecules (Appendix~\ref{app:sec:substructure_conditioned_generation}), showing superior flexibility over \citet{vignac2023digress}. These results highlight the potential of \method{} for important applications in drug discovery, particularly in motif scaffolding.



\begin{table}[tbp]
    \centering
    \vskip -0.1in
    \caption{Motif scaffolding using the motif 1,4-Dihydroquinoline.}
    \label{tab:substructure_conditioned_generation}
    \begin{small}
    \begin{sc}
    \resizebox{.75\columnwidth}{!}{
    \begin{tabular}{lccc}\toprule
        \# Copies of the motif & Valid & Unique & Novelty \\ \midrule
        1 & 92.0 & 98.8 & 99.6 \\
        2 & 88.8 & 99.7 & 100.0\\
        5 & 66.0 & 100.0 & 100.0 \\ \bottomrule
    \end{tabular}
    }
    \end{sc}
    \end{small}
    \vskip -0.1in
\end{table}

\subsection{Ablation Experiments}\label{sec:ablation_experiments}
In this study, we aim to understand the effectiveness of key components in \method{}.
\vspace{-.1in}
\paragraph{Comparison of sequence model architectures.}
\method{} provides a novel framework for evaluating the capability of current LLM architectures in graph generation and, more broadly, in structural reasoning tasks. In Table~\ref{tab:architecture}, we compare several state-of-the-art architectures on the Planar dataset, including GPT-2~\citep{radford2019gpt2}, Mamba~\citep{gu2023mamba}, and LLaMA~\citep{touvron2023llama}. While all models achieve comparable MMD ratios, transformer-based architectures, particularly LLaMA, demonstrate significantly better performance in terms of VUN scores compared to state-space models. These findings highlight the potential of \method{} to serve as a valuable benchmark for assessing sequence/language models' capabilities in graph generation tasks.

\begin{table}[tbp]
    \centering
    \vskip -0.1in
    \caption{Comparison of sequence model architectures on Planar}
    \label{tab:architecture}
    \begin{small}
    \begin{sc}
    \resizebox{\columnwidth}{!}{
    \begin{tabular}{lcccccc}\toprule
        Architecture & Deg. & Clus. & Orbit & Spec. & Ratio & VUN  \\ \midrule
        GPT-2 & 0.0004 & 0.0720 & 0.0010 & 0.0053 & 1.8 & 85.0 \\
        Mamba & 0.0002 &  0.0429 & 0.0014 & 0.0087 & 1.6 & 55.0 \\
        LLaMA & 0.0005 & 0.0651 & 0.0005 & 0.0056 & \textbf{1.6} & \textbf{90.0} \\ \bottomrule
    \end{tabular}
    }
    \end{sc}
    \end{small}
    \vskip -0.1in
\end{table}
\vspace{-.1in}
\paragraph{Effect of top-k sampling.}
A key advantage of \method{} over diffusion-based approaches is the flexibility to apply top-k sampling~\citep{fan2018topk} during inference, which can improve generation quality. As shown in Figure~\ref{fig:topk}, a smaller $k$ improves the VUN score on the Planar dataset, whereas it is not beneficial on the SBM dataset. In contrast, increasing $k$ generally improves MMD ratios across both datasets. These observations suggest that top-k sampling can be optimized based on dataset characteristics. In our experiments, we select the best $k$ that maximizes the VUN score for small synthetic datasets and minimizes the validation MMD ratios for other datasets. Importantly, this flexibility allows practitioners to select $k$ based on the specific performance criteria they aim to prioritize.

\begin{figure}[tbp]
    \centering
    \includegraphics[width=0.49\columnwidth]{img/ablation_k_planar.pdf}
    \includegraphics[width=0.49\columnwidth]{img/ablation_k_sbm.pdf}
    \vskip-.1in
    \caption{The effect of top-k sampling on the Planar and SBM datasets.}
    \label{fig:topk}
    \vskip -0.1in
\end{figure}

\vspace{-.1in}
\paragraph{Comparison of SET and SENT.}
As discussed in Section~\ref{sec:methods}, SENT is preferred over SET for graph generation, as incorporating neighborhood information is essential to ensure structural coherence. To empirically validate this, we compare the performance of SENT and SET on the Planar dataset and present the training curves in Figure~\ref{fig:set_vs_sent}. Consistent with our theoretical analysis, SET fails to produce high-validity graphs, resulting in a VUN score close to zero, whereas SENT successfully generates valid planar graphs.

\begin{figure}[tbp]
    \centering
    \includegraphics[width=\columnwidth]{img/SET_vs_SENT/SET_vs_SENT.pdf}
    \vskip-.1in
    \caption{SET vs SENT on the Planar dataset}
    \vskip-.1in
    \label{fig:set_vs_sent}
\end{figure}

\section{Conclusion}
We proposed \method{}, a scalable and efficient autoregressive model for attributed graph generation. \method{} demonstrates the ability to handle large graphs while preserving high generation quality. It enables substructure-conditioned generation without requiring additional fine-tuning and shows promising transfer capabilities. More importantly, it establishes a critical connection between graph modeling and language modeling, representing a significant step toward leveraging language modeling paradigms to address challenges in graph generation or more general graph learning tasks.

\section*{Acknowledgements}
The authors thank Dr.\ Till Hendrik Schulz, Philip Hartout, and Błażej Banaszewski for their insightful discussions and valuable feedback on the manuscript.

\section*{Impact Statement}
Our research focuses on advancing the algorithmic development of graph generative models, strongly emphasizing their responsible and ethical application in specialized fields. In domains such as drug discovery and synthetic biology, ensuring the trustworthiness and appropriate use of our methods is essential to prevent potential misuse. Through our experiments, we showcase the potential of our approach in these fields, underscoring its promise to deliver meaningful societal benefits while acknowledging the need to address potential risks.

\bibliography{mybib}
\bibliographystyle{arxiv}

\newpage
\appendix
\onecolumn
%!TEX root = arxiv.tex
The following sections provide supplementary information omitted from the main text:
\begin{itemize}
    \item Section~\ref{appx:implementation}: Implementation Details.
    \item Section~\ref{appx:theory}: Further Theoretical Results to Support Model Folding.
    % \item Section~\ref{appx:wm_vs_folding}: Relationship Between Weight Matching and Model Folding.
    \item Section~\ref{appx:sec:channel_similarity}: Channel Similarity.
    \item Section~\ref{appx:llms}: Model Folding on LLMs.
    \item Section~\ref{appx:residual}: Handling Residual Blocks.
    \item Section~\ref{appx:bn}: Handling Batch Normalization Layers.
    \item Section~\ref{appx:similar_in_mlps}: Folding Similar Channels in MLPs.
    \item Section~\ref{appx:similar_in_cnn}: Folding Similar Channels in Convolutional Layers.
    \item Section~\ref{appx:similar_in_llama}: Folding Similar Channels in LlamaMLP and LlamaAttention.
    \item Section~\ref{appx:kd}: Comparison with Knowledge Distillation.
    \item Section~\ref{appx:devices}: Inference Speed of Folded Models on Edge Devices.
    \item Section~\ref{appx:dee_inversion}: Deep Inversion Sample Images.
    \item Section~\ref{appx:related}: Further Related Work.
\end{itemize}

\section{Implementation details}
\label{appx:implementation}

We trained over 100 models on a NVIDIA DGX Station A100 featuring eight NVIDIA A100 GPUs (each equipped with 80GB memory) to evaluate the performance of model folding presented in this work. For a folding experiment, we apply the same compression ratio to all layers. \texttt{Pytorch Hub}\footnote{https://pytorch.org/hub/} and \texttt{Huggingface Hub}\footnote{https://huggingface.co/docs/hub/index} are used to load pre-trained checkpoints for complex model-dataset combinations, including ResNet18/ResNet50/VGG11 on ImageNet and LLaMA-7B~\citep{llama}. WandB\footnote{https://wandb.ai} is used to log training history, folding result, and evaluation metrics. The source code of all experiments is available here: \url{https://github.com/nanguoyu/model-folding-universal}
% https://anonymous.4open.science/r/model_folding_anonymous-94F8/



\section{Further theoretical results to support model folding}
\label{appx:theory}

\begin{lemma}\label{lemma1}
Let $\mathbf{x} \in \mathbb{R}^{k}$ and let $\mathbf{U} \in \{0, 1\}^{n \times k}$ be a binary clustering matrix with $\sum_{j} u_{ij} = 1$. Then with any element-wise nonlinear function $\sigma(\cdot)$ we have
\[
    \sigma(\mathbf{U} \mathbf{x}) = \mathbf{U}\sigma( \mathbf{x})
\]
\end{lemma} 
\begin{proof}[Proof of Lemma \ref{lemma1}]
Define $\mathbf{y} = \mathbf{U} \mathbf{x}$, $\mathbf{z} = \sigma( \mathbf{U} \mathbf{x})$  and $\mathbf{v} = \sigma( \mathbf{x})$, $\mathbf{w} = \mathbf{U}\sigma( \mathbf{x})$. Note that in any row of $\mathbf{U}$ just one element satisfies $u_{i j} = 1$. We define such an element by a function $p$ with $u_{i j} = 1 \Leftrightarrow p(i) = j$. 

Therefore, $\mathbf{y}_i = \mathbf{x}_{p(i)}$ and $\mathbf{z}_i = \sigma(\mathbf{y}_i) = \sigma(\mathbf{x}_{p(i)})$ for all $1 \leq i \leq n$. Moreover, $\mathbf{v}_i = \sigma(\mathbf{x}_{i})$ and $\mathbf{w}_i = \mathbf{v}_{p(i)} = \sigma(\mathbf{x}_{p(i)})$. Therefore, $\mathbf{z}_i = \mathbf{w}_i$ and $\mathbf{z} = \mathbf{w}$. 

\iffalse
    If $\mathbf{U}^T$ is a clustering matrix then for ever every row $\mathbf{q}_k^T$ of $\mathbf{U}$ we have: 
    \[
        \mathbf{q}^{(i)}_{k} = \begin{cases} 
            1 & i = p(k) \\
            0 & else
   \end{cases}
    \]
    where $p(k)$ is some function $p: \{1, 2, \hdots, n\} \mapsto \{1, 2, \hdots, d\}$. From the previous property it holds:
    \[
        (\mathbf{q}_k)^T \sigma(\mathbf{x}) = \sigma(\mathbf{x})^{(p(k)))} = \sigma(\mathbf{x}^{(p(k)))})
    \]
    Because $\mathbf{U}^T$ is a clustering matrix, we also have
    \[
        (\mathbf{q}_k)^T \mathbf{x} = \mathbf{x}^{(p(k))} 
    \]
    and consequently
    \[
        \sigma(\mathbf{q}_k^T \mathbf{x}) = \sigma(\mathbf{x}^{(p(k))} ) = \mathbf{q}_k^T \sigma(\mathbf{x})
    \]
\fi
\end{proof}

\begin{lemma}\label{lemma4}
Let $\mathbf{x} \in \mathbb{R}^{k}$, let $\mathbf{U} \in \{0, 1\}^{n \times k}$ be a binary clustering matrix with $\sum_{j} u_{ij} = 1$, let $\sigma(\cdot)$ be an element-wise nonlinear function, and define $\mathbf{C} = \mathbf{U} (\mathbf{U}^T \mathbf{U})^{-1} \mathbf{U}^T$. Then
\[
    \sigma(\mathbf{C} \mathbf{x}) = \mathbf{C}^T \sigma( \mathbf{C} \mathbf{x})
\]
\end{lemma} 
\begin{proof}[Proof of Lemma \ref{lemma4}]
We can write 
\begin{align*}
    \sigma(\mathbf{C} \mathbf{x}) &= \sigma(\mathbf{U} (\mathbf{U}^T \mathbf{U})^{-1} \mathbf{U}^T \mathbf{x}) \\
    &= \mathbf{U} \sigma((\mathbf{U}^T \mathbf{U})^{-1} \mathbf{U}^T \mathbf{x}) \qquad \text{(Lemma \ref{lemma1})}\\
    &= \mathbf{U} (\mathbf{U}^T \mathbf{U})^{-1} (\mathbf{U}^T \mathbf{U})  \sigma((\mathbf{U}^T \mathbf{U})^{-1} \mathbf{U}^T \mathbf{x}) \\
    &= \mathbf{U} (\mathbf{U}^T \mathbf{U})^{-1} \mathbf{U}^T \sigma(\mathbf{U} (\mathbf{U}^T \mathbf{U})^{-1} \mathbf{U}^T \mathbf{x}) \qquad \text{(Lemma \ref{lemma1})}\\
    &= \mathbf{C}^T \sigma( \mathbf{C} \mathbf{x}).
\end{align*}
\end{proof}

\begin{lemma}\label{diag_u}
Let $\mathbf{U}^T$ be a clustering matrix and let $\mathbf{D}$ be a diagonal matrix, then the following is true
\begin{align*}
    (\mathbf{U}^T\mathbf{U})^{-1}\mathbf{U}^T\mathbf{D}\mathbf{U} = \text{Diag}((\mathbf{U}^T\mathbf{U})^{-1}\mathbf{U}^T\text{diag}(\mathbf{D}))
\end{align*}
\end{lemma}
\begin{proof}[Proof of Theorem \ref{diag_u}]

The clustering matrix \( \mathbf{U}^T \) can be expressed as:
\[
\mathbf{U}^T = 
\begin{bmatrix}
    \mathbf{u}_1^T \\
    \mathbf{u}_2^T \\
    \vdots \\
    \mathbf{u}_k^T
\end{bmatrix}
=
\begin{bmatrix}
    u_{11} & u_{12} & \dots & u_{1n} \\
    u_{21} & u_{22} & \dots & u_{2n} \\
    \vdots & \vdots & \ddots & \vdots \\
    u_{k1} & u_{k2} & \dots & u_{kn}
\end{bmatrix},
\]
where \( \mathbf{u}_i^T \) represents the rows of the clustering matrix. Each row corresponds to cluster \( i \), and the entries \( u_{ij} \) satisfy the binary clustering property: \( u_{ij} = 1 \) if the \( j \)-th data point belongs to cluster \( i \), and \( u_{ij} = 0 \) otherwise.

The product \( \mathbf{D} \mathbf{U} \) is given by:
\[
\mathbf{D} \mathbf{U} = 
\begin{bmatrix}
    d_1 & 0 & \dots & 0 \\
    0 & d_2 & \dots & 0 \\
    \vdots & \vdots & \ddots & \vdots \\
    0 & 0 & \dots & d_n
\end{bmatrix}
\begin{bmatrix}
    u_{11} & u_{12} & \dots & u_{1k} \\
    u_{21} & u_{22} & \dots & u_{2k} \\
    \vdots & \vdots & \ddots & \vdots \\
    u_{n1} & u_{n2} & \dots & u_{nk}
\end{bmatrix}.
\]
This simplifies to:
\[
\mathbf{D} \mathbf{U} = 
\begin{bmatrix}
    d_1 u_{11} & d_1 u_{12} & \dots & d_1 u_{1k} \\
    d_2 u_{21} & d_2 u_{22} & \dots & d_2 u_{2k} \\
    \vdots & \vdots & \ddots & \vdots \\
    d_n u_{n1} & d_n u_{n2} & \dots & d_n u_{nk}
\end{bmatrix}.
\]

Using the clustering property of \( \mathbf{U} \), it follows that:
\[
u_{ij} u_{i'j} =
\begin{cases}
1, & \text{if } i = i', \\
0, & \text{otherwise}.
\end{cases}
\]

From this, the product \( \mathbf{U}^T \mathbf{D} \mathbf{U} \) simplifies to:
\[
\mathbf{U}^T \mathbf{D} \mathbf{U} = \text{Diag}(\mathbf{U}^T \text{diag}(\mathbf{D})).
\]
This result holds because only the diagonal entries remain due to the clustering matrix's orthogonality and binary properties.

Finally, using the above result, we compute:
\[
(\mathbf{U}^T \mathbf{U})^{-1} \mathbf{U}^T \mathbf{D} \mathbf{U} = (\mathbf{U}^T \mathbf{U})^{-1} \text{Diag}(\mathbf{U}^T \text{diag}(\mathbf{D})).
\]

By the property \( \text{diag}(\text{Diag}(\mathbf{x})) = \mathbf{x} \) for any \( \mathbf{x} \in \mathbb{R}^n \), we obtain:
\[
(\mathbf{U}^T \mathbf{U})^{-1} \mathbf{U}^T \mathbf{D} \mathbf{U} = \text{Diag}((\mathbf{U}^T \mathbf{U})^{-1} \mathbf{U}^T \text{diag}(\mathbf{D})).
\]

The lemma demonstrates that projecting the diagonal matrix \( \mathbf{D} \) through the clustering matrix \( \mathbf{U}^T \) preserves its diagonal structure. The diagonal entries are determined by the clustering matrix's mapping of the original diagonal values \( \text{diag}(\mathbf{D}) \), ensuring efficient computation and alignment with clustering properties.
\end{proof}




\begin{lemma}\label{diag-map}
Let $\mathbf{U}^T$ be a clustering matrix and let $\mathbf{w} \in \mathbb{R}^n$ and $\mathbf{x} \in \mathbb{R}^n$, then the following is true
\begin{align*}
    \mathbf{U} \text{Diag}(\mathbf{w}) \mathbf{x} = \text{Diag}(\mathbf{U}\mathbf{w}) \mathbf{U }\mathbf{x}
\end{align*}
\end{lemma}
\begin{proof}[Proof of Lemma \ref{diag-map}]
The clustering matrix \( \mathbf{U} \) can be expressed as:
\[
\mathbf{U} =
\begin{bmatrix}
    \mathbf{v}_1^T \\
    \mathbf{v}_2^T \\
    \vdots \\
    \mathbf{v}_n^T
\end{bmatrix},
\]
where each row \( \mathbf{v}_m^T \) is defined by a mapping function \( f: \{1, 2, \dots, n\} \to \{1, 2, \dots, k\} \). For each row \( \mathbf{v}_m^T \), the entries are defined as:
\[
v_{m,j} =
\begin{cases}
1, & \text{if } j = f(m), \\
0, & \text{otherwise}.
\end{cases}
\]
This representation indicates that the clustering matrix \( \mathbf{U} \) assigns each element \( m \) to a specific cluster \( f(m) \). Each row \( \mathbf{v}_m^T \) has a single non-zero element corresponding to the cluster index \( f(m) \).

\paragraph{Calculation of the Left-Hand Side (LHS).}
The left-hand side of the equality is:
\[
\mathbf{U}\text{Diag}(\mathbf{w})\mathbf{x}.
\]
First, compute \( Diag(\mathbf{w})\mathbf{x} \), which scales each element of \( \mathbf{x} \) by the corresponding element of \( \mathbf{w} \):
\[
\text{Diag}(\mathbf{w})\mathbf{x} =
\begin{bmatrix}
    w_1 x_1 \\
    w_2 x_2 \\
    \vdots \\
    w_n x_n
\end{bmatrix}.
\]
Then, multiplying by \( \mathbf{U} \) aggregates these scaled values according to the clusters defined by \( f \). Specifically, the \( j \)-th element of \( \mathbf{U}\text{Diag}(\mathbf{w})\mathbf{x} \) is given by:
\[
(\mathbf{U}\text{Diag}(\mathbf{w})\mathbf{x})_j = \sum_{m: f(m) = j} w_m x_m.
\]

\paragraph{Calculation of the Right-Hand Side (RHS).}
The right-hand side of the equality is:
\[
\text{Diag}(\mathbf{U}\mathbf{w})\mathbf{U}\mathbf{x}.
\]
First, compute \( \mathbf{U}\mathbf{w} \). The \( j \)-th element of \( \mathbf{U}\mathbf{w} \) is:
\[
(\mathbf{U}\mathbf{w})_j = \sum_{m: f(m) = j} w_m,
\]
which sums the \( w_m \) values for all elements assigned to cluster \( j \).

Next, construct \( \text{Diag}(\mathbf{U}\mathbf{w}) \), a diagonal matrix with entries \( (\mathbf{U}\mathbf{w})_j \) along the diagonal:
\[
\text{Diag}(\mathbf{U}\mathbf{w}) =
\begin{bmatrix}
    (\mathbf{U}\mathbf{w})_1 & 0 & \dots & 0 \\
    0 & (\mathbf{U}\mathbf{w})_2 & \dots & 0 \\
    \vdots & \vdots & \ddots & \vdots \\
    0 & 0 & \dots & (\mathbf{U}\mathbf{w})_k
\end{bmatrix}.
\]

Finally, compute \( \mathbf{U}\mathbf{x} \). The \( j \)-th element of \( \mathbf{U}\mathbf{x} \) is:
\[
(\mathbf{U}\mathbf{x})_j = \sum_{m: f(m) = j} x_m,
\]
which sums the \( x_m \) values for all elements assigned to cluster \( j \).

Multiplying \( \text{Diag}(\mathbf{U}\mathbf{w}) \) by \( \mathbf{U}\mathbf{x} \) gives:
\[
\left(\text{Diag}(\mathbf{U}\mathbf{w})\mathbf{U}\mathbf{x}\right)_j = (\mathbf{U}\mathbf{w})_j (\mathbf{U}\mathbf{x})_j = \left(\sum_{m: f(m) = j} w_m\right)\left(\sum_{m: f(m) = j} x_m\right).
\]

\paragraph{Verification of Equality.}
Both the LHS and RHS compute the same aggregated sums \( \sum_{m: f(m) = j} w_m x_m \) for each cluster \( j \). The LHS directly performs the aggregation of \( w_m x_m \) within clusters, while the RHS separates the computation into two steps: summing \( w_m \) and \( x_m \) for each cluster, followed by multiplying these sums. Since multiplication distributes over addition, the two expressions are equivalent:
\[
\mathbf{U}\text{Diag}(\mathbf{w})\mathbf{x} = \text{Diag}(\mathbf{U}\mathbf{w})\mathbf{U}\mathbf{x}.
\]

The lemma is proven, as both sides of the equation compute the same weighted aggregation of \( w_m x_m \) over the clusters defined by the clustering matrix \( \mathbf{U} \).
\end{proof}




\begin{lemma}\label{diag_norm}
    Let $\mathbf{C}^T$ be a clustering matrix and let $\mathbf{D}$ be a diagonal matrix, then the following is true
    \begin{align*}
        \|\mathbf{W} - \text{Diag}(\mathbf{C}\text{diag}(\mathbf{W}))\|_F^2 = 
        \|\text{diag}(\mathbf{W}) - \mathbf{C}\text{diag}(\mathbf{W})\|_2^2
    \end{align*}
\end{lemma}
\begin{proof}[Proof of Lemma \ref{diag_norm}]
Let \( \mathbf{\tilde{W}} = \text{Diag}(\mathbf{C}\text{diag}(\mathbf{W})) \), where \( \mathbf{\tilde{W}} \) represents the diagonal matrix obtained by clustering the diagonal entries of \( \mathbf{W} \) using the clustering matrix \( \mathbf{C} \). Both \( \mathbf{W} \) and \( \mathbf{\tilde{W}} \) are diagonal matrices, so their difference \( \mathbf{W} - \mathbf{\tilde{W}} \) is also diagonal. The entries of this difference are:
\[
w_{i,j} - \tilde{w}_{i,j} =
\begin{cases}
w_{i,i} - \tilde{w}_{i,i}, & \text{if } i = j, \\
0, & \text{otherwise}.
\end{cases}
\]
The Frobenius norm of the difference \( \mathbf{W} - \mathbf{\tilde{W}} \) is:
\[
\|\mathbf{W} - \mathbf{\tilde{W}}\|_F^2 = \sum_{i,j} (w_{i,j} - \tilde{w}_{i,j})^2.
\]
Since \( \mathbf{W} \) and \( \mathbf{\tilde{W}} \) are diagonal matrices, this simplifies to:
\[
\|\mathbf{W} - \mathbf{\tilde{W}}\|_F^2 = \sum_{i} (w_{i,i} - \tilde{w}_{i,i})^2.
\]
The diagonal entries of \( \mathbf{W} \) can be represented as a vector \( \text{diag}(\mathbf{W}) \), and the diagonal entries of \( \mathbf{\tilde{W}} \) are given by \( \mathbf{C}\text{diag}(\mathbf{W}) \). Substituting these representations, we have:
\[
\|\mathbf{W} - \mathbf{\tilde{W}}\|_F^2 = \sum_{i} (\text{diag}(\mathbf{W})_i - (\mathbf{C}\text{diag}(\mathbf{W}))_i)^2.
\]
This is equivalent to the squared \( \ell_2 \)-norm of the difference between the vectors \( \text{diag}(\mathbf{W}) \) and \( \mathbf{C}\text{diag}(\mathbf{W}) \), giving:
\[
\|\mathbf{W} - \mathbf{\tilde{W}}\|_F^2 = \|\text{diag}(\mathbf{W}) - \mathbf{C}\text{diag}(\mathbf{W})\|_2^2.
\]
Substituting back \( \mathbf{\tilde{W}} = \text{Diag}(\mathbf{C}\text{diag}(\mathbf{W})) \), we conclude that:
\[
\|\mathbf{W} - \text{Diag}(\mathbf{C}\text{diag}(\mathbf{W}))\|_F^2 = \|\text{diag}(\mathbf{W}) - \mathbf{C}\text{diag}(\mathbf{W})\|_2^2.
\]
\end{proof}




\begin{lemma}\label{diag-aux}
    Let $\mathbf{A} \in \mathbb{R}^{n \times n}$ and $\mathbf{B} \in \mathbb{R}^{n \times n}$ be diagonal matrices, then:
    \begin{align*}
        \mathbf{A} {\mathbf{B}} = \text{Diag}(\mathbf{A}\text{diag}(\mathbf{B}))
    \end{align*}
\end{lemma}
\begin{proof}[Proof of Lemma \ref{diag-aux}]
Since both \( \mathbf{A} \) and \( \mathbf{B} \) are diagonal matrices, their product \( \mathbf{A}\mathbf{B} \) is also a diagonal matrix. The entries of the product \( \mathbf{A}\mathbf{B} \) are given by:
\[
(\mathbf{A}\mathbf{B})_{i,j} = a_{i,j} b_{i,j}.
\]
For diagonal matrices, all off-diagonal entries are zero, so:
\[
(\mathbf{A}\mathbf{B})_{i,j} =
\begin{cases}
a_{i,i} b_{i,i}, & \text{if } i = j, \\
0, & \text{otherwise}.
\end{cases}
\]
Thus, the diagonal entries of \( \mathbf{A}\mathbf{B} \) are \( a_{i,i} b_{i,i} \), and the matrix \( \mathbf{A}\mathbf{B} \) is:
\[
\mathbf{A}\mathbf{B} =
\begin{bmatrix}
a_1 b_1 & 0 & \dots & 0 \\
0 & a_2 b_2 & \dots & 0 \\
\vdots & \vdots & \ddots & \vdots \\
0 & 0 & \dots & a_n b_n
\end{bmatrix},
\]
where \( a_i = a_{i,i} \) and \( b_i = b_{i,i} \) represent the diagonal entries of \( \mathbf{A} \) and \( \mathbf{B} \), respectively.

Now, let \( \text{diag}(\mathbf{B}) \) denote the vector of diagonal entries of \( \mathbf{B} \), i.e.,
\[
\text{diag}(\mathbf{B}) = \begin{bmatrix} b_1 \\ b_2 \\ \vdots \\ b_n \end{bmatrix}.
\]
The operation \( \mathbf{A}\text{diag}(\mathbf{B}) \) represents the element-wise multiplication of the diagonal entries of \( \mathbf{A} \) and \( \mathbf{B} \):
\[
\mathbf{A}\text{diag}(\mathbf{B}) = \begin{bmatrix} a_1 b_1 \\ a_2 b_2 \\ \vdots \\ a_n b_n \end{bmatrix}.
\]

Next, using the function \( \text{Diag}(\cdot) \), we can construct a diagonal matrix from this vector:
\[
\text{Diag}(\mathbf{A}\text{diag}(\mathbf{B})) =
\begin{bmatrix}
a_1 b_1 & 0 & \dots & 0 \\
0 & a_2 b_2 & \dots & 0 \\
\vdots & \vdots & \ddots & \vdots \\
0 & 0 & \dots & a_n b_n
\end{bmatrix}.
\]

Clearly, \( \mathbf{A}\mathbf{B} \) and \( \text{Diag}(\mathbf{A}\text{diag}(\mathbf{B})) \) are identical, as they both produce the same diagonal matrix with entries \( a_i b_i \) along the diagonal. Therefore:
\[
\mathbf{A}\mathbf{B} = \text{Diag}(\mathbf{A}\text{diag}(\mathbf{B})).
\]
\end{proof}


% \begin{definition}[Variance ratio]
%     Consider a neural network $f(\mathbf{x}, \mathbf{\Theta})$ with layer activations $\{\mathbf{x}_l\}_1^L$ and a compressed neural network $\tilde{f}(\mathbf{x}, \mathbf{\Theta})$ with activations $\{\tilde{\mathbf{x}}_l\}_1^L$ . Let the activation of the $l$-th layer of the original network be
%     $\mathbf{x}_l = \sigma(g(\mathbf{x}_{l-1}))$.
%     The activations of the compressed layer $l$ with the clustering matrix $\mathbf{U}_l$ and $\mathbf{C}_l = \mathbf{U}_l (\mathbf{U}_l^T\mathbf{U}_l)^{-1} \mathbf{U}_l^T$ are 
%     \[
%             \tilde{\mathbf{x}}_l = \mathbf{C}_l^T \sigma(\mathbf{C}_l g(\tilde{\mathbf{x}}_{l-1})).
%     \]
%     The \emph{variance ratio} of the $l$-th layer is defined as:
%     \[
%     \mu\left[\frac{\Var(\tilde{\mathbf{x}}_l)}{\Var(\mathbf{x}_l)}\right] = \frac{1}{|\mathbf{x}_l|} \sum_{k = 1}^{|\mathbf{x}_l|}\frac{\Var(\tilde{\mathbf{x}}_{l,k})}{\Var(\mathbf{x}_{l,k})}.
%     \]
% \end{definition}


% \begin{definition}[Variance collapse]
%     Consider a neural network $f(\mathbf{x}, \mathbf{\Theta})$ with layer activations $\{\mathbf{x}_l\}_1^L$ and a compressed neural network $\tilde{f}(\mathbf{x}, \mathbf{\Theta})$ with activations $\{\tilde{\mathbf{x}}_l\}_1^L$. 
%     The \emph{variance collapse} is observed if for every $l \in \{2, \hdots, L\}$ the following holds:
%     \[
%         \mu\left[\frac{\Var(\tilde{\mathbf{x}}_l)}{\Var(\mathbf{x}_l)}\right] \leq \mu \left[\frac{\Var(\tilde{\mathbf{x}}_{l-1})}{\Var(\mathbf{x}_{l-1})}\right].
%     \].
% \end{definition}

% \lothar{We may not need the following Lemmata. Please check.}

% \begin{lemma}\label{lemma2}
% Let $\mathbf{x} \in \mathbb{R}^{d}$   and let $\mathbf{D}$ be a diagonal $d \times d$ matrix. Then
% \begin{align*}
%     \sigma(\mathbf{D} \mathbf{x}) = \mathbf{D}\sigma( \mathbf{x})
% \end{align*}
% \end{lemma} 
% \begin{proof}[Proof of Lemma \ref{lemma2}]
%     If $\mathbf{D}$ is a diagonal matrix then for every row $\mathbf{d}_k^T$ of $\mathbf{D}$ we have :
%     \begin{align*}
%         \mathbf{d}^{(i)}_{k} = \begin{cases} 
%             \alpha_k & i = k \\
%             0 & else
%    \end{cases}
%     \end{align*}
%     Therefore, the following holds:
%     \begin{align*}
%         (\mathbf{d}_k)^T \sigma(\mathbf{x}) = \alpha_k\sigma(\mathbf{x})^{(k)} = \sigma(\alpha_k\mathbf{x}^{(k)}).
%     \end{align*}
%     Because $\mathbf{D}$ is a diagonal matrix, we also have:
%     \begin{align*}
%         (\mathbf{d}_k)^T \mathbf{x} = \alpha_k\mathbf{x}^{(k)} 
%     \end{align*}
%     and consequently:
%     \begin{align*}
%         \sigma(\mathbf{d}_k^T \mathbf{x}) = \sigma(\alpha_k\mathbf{x}^{(k)}) = \mathbf{d}_k^T \sigma(\mathbf{x})
%     \end{align*}
% \end{proof}

% \begin{lemma}\label{lemma3}
% Let $\mathbf{U}^T$ be a binary indicator matrix. Then the product $\mathbf{U}^T \mathbf{U}$ is a diagonal matrix.
% \end{lemma} 
% \begin{proof}[Proof of Lemma \ref{lemma1}]
% \haris{TODO: Finish Proof} \\
% If $\mathbf{U}^T$ is a binary indicator matrix then for ever every row $\mathbf{q}_k$ of $\mathbf{U}$ we have 
%     \begin{align*}
%         \mathbf{q}^{(i)}_{k} = \begin{cases} 
%             1 & i = p_k \\
%             0 & else
%         \end{cases}
%     \end{align*}
%     Therefore it holds
    
% \end{proof}




% \section{Further empirical results}

% \olga{Do we need this section?}

% \begin{table}[h]
% \centering
% \small{
% \begin{tabular}{l|c|c|c|c|c|c}
% \toprule
% Model    & Dataset  & Full Model & Parameter & Model & L$_1$ pruning & L$_2$ pruning \\ 
%         &           & Performance  & Sparsity & Folding & [\%] &    [\%] \\
%         &           & [\%]  & [\%] & [\%] &  &    \\
% \midrule
% VGG11-BN & CIFAR10  & \textbf{91.42} & 74.98 & 21.86/\textbf{75.76} & 10.0/38.81 & 11.71/35.02 \\ 
% ResNet18 & CIFAR10  & \textbf{94.59} & 74.96 & 17.79/\textbf{81.31} & 11.04/37.57 & 10.0/37.53 \\ 
% ResNet50 & CIFAR10  & \textbf{94.33} & 74.92 & 10.85/\textbf{88.65} & 18.53/48.07 & 24.49/48.77 \\ 
% VGG11-BN & CIFAR100 & \textbf{65.53} & 70.11 & 3.91/\textbf{21.72}  & 1.0/10.11 & 1.24/10.59 \\ 
% ResNet18 & CIFAR100 & \textbf{75.21} & 74.86 & 2.4/\textbf{30.27}   & 1.0/10.11  & 1.24/10.59 \\ 
% ResNet50 & CIFAR100 & \textbf{77.26} & 74.73 & 1.83/\textbf{48.51}  & 1.94/19.28 & 1.95/20.75 \\ 
% VGG11-BN & ImageNet & \textbf{70.37} & 71.63 & 0.2/\textbf{3.6}               & 0.1/0.1    & 0.1/0.1    \\ 
% ResNet18 & ImageNet & \textbf{69.76} & 71.17 & 0.1/\textbf{0.6}               & 0.1/0.1    & 0.1/0.1    \\ 
% \bottomrule
% \end{tabular}
% }
% \caption{\textbf{Dong -- Performance of model folding at maximal sparsity,} demonstrating the effects of model folding, L$_1$ structured pruning, and L$_2$ structured pruning at maximal sparsity. For model folding, L$_1$ pruning, and L$_2$ pruning results are presented as without REPAIR and with REPAIR. \olga{Update. 75\%?}}
% \label{table:max_pruning_results}
% \end{table}


% A number of results for folding using clustering and REPAIR are given in table \ref{appx:tb:performance}.

% \begin{table}[h]
% \centering
% \small{
% \begin{tabular}{l|c|c|c|c|c|c}
% \toprule
% Model    & Dataset  & Full Model & Parameter & Model & L$_1$ pruning & L$_2$ pruning \\ 
%         &           & Performance  & Sparsity & Folding & [\%] &    [\%] \\
%         &           & [\%]  & [\%] & [\%] &  &    \\
% \midrule
% VGG11-BN & CIFAR10  & \textbf{94.59} & 74.98 & 17.79/\textbf{77.48} & 11.04/37.57 & 10.0/37.53 \\
% ResNet18 & CIFAR10  & \textbf{94.59} & 74.96 & 17.79/\textbf{81.78} & 11.04/37.57 & 10.0/37.53 \\ 
% ResNet50 & CIFAR10  & \textbf{94.33} & 74.92 & 10.85/\textbf{89.75} & 18.53/48.07 & 24.49/48.77 \\ 
% ResNet18 & CIFAR100 & \textbf{75.21} & 74.86 & 2.4/\textbf{32.36}   & 1.0/10.11  & 1.24/10.59 \\ 
% ResNet50 & CIFAR100 & \textbf{77.26} & 74.73 & 1.83/\textbf{52.59}  & 1.94/19.28 & 1.95/20.75 \\ 
% \bottomrule
% \end{tabular}
% }
% \caption{\textbf{Performance of model folding at neuron sparsity of 50\%,} \olga{50?} demonstrating the effects of model folding, L$_1$ structured pruning, and L$_2$ structured pruning at maximal sparsity. For model folding, L$_1$ pruning, and L$_2$ pruning results are presented as without REPAIR and with REPAIR.}
% \label{appx:tb:performance}
% \end{table}





\section{Channel similarity}
\label{appx:sec:channel_similarity}

Models learned by SGD trend to have correlated patterns or similar parameters in the weight space. \figref{fig:weightmap:resnet18} shows $3\times3$ filter weights in \emph{conv1} of a pre-trained ResNet18. These filters across the first 3 input channels and first 16 output channels ordered by the entropy of filter weight. From the plot, most filters of a channel can find at least one another similar filter in other channels, which means filter similarity may lead to structured redundancy.

\begin{figure*}[h]
    \centering
    % weight_resnet18_conv1.png
     \includegraphics[angle=90, width=.69\linewidth]{old_figs/vis_weight/weight_resnet18_imagenet_3x16_sortby_entropy.png}
    \caption{\textbf{Similar patterns in weight map of \emph{conv1} layer in ResNet18 pre-trained on ImageNet~\citep{deng2009imagenet}}. Each small square represents the weights of a single filter in cool-warm color map, where each color of grid corresponds to a weight value. 
    }
    \label{fig:weightmap:resnet18}
\end{figure*}

To investigate the filter redundancy within a layer, we apply weight matching activation matching
from the literature~\citep{jordan2022repair} to each layer of ResNet18 pretrained on CIFAR10~\citep{cifar10} in~\figref{fig:channel_similarity} and on ImageNet~\citep{deng2009imagenet} in \figref{fig:hist:resnet18_imagenet}. We observe two findings: (1) The correlation score distribution varies across layers. The earlier and narrower the lay ers are, the more scattered the correlation coefficients are, and only a few have high correlation coefficients. The wider and later the layers are, the more compact the correlation coefficients are, and most of the matching channels have high correlation coefficients. (2) In the same layer, the distribution of correlation coefficients among matched channels differs across various pre-training datasets. This observation does not fully align with the claim by \citet{chen2023going} regarding the downward trend of similarity before a reversal. It appears that this characterization might not consistently hold across different models and pre-trained dataset.

\begin{figure*}[h]
    \centering
     \includegraphics[width=.99\linewidth]{old_figs/hist/hist_matched_channles_resnet18_ImageNet.png}
    \caption{\textbf{Layer-wise correlation between matched channels in ResNet18 trained on ImageNet.} We compute a layer-wise correlation matrix by matching activations between channels, then assign each channel its best match in the same layer using a greedy pairing based on the correlation matrix.}
    \label{fig:hist:resnet18_imagenet}
\end{figure*}



\subsection{The impact of regularization}

In \figref{fig:comparison:ifm}, the models on CIFAR10 were trained without regularization, while the pre-trained ImageNet models were sourced from \texttt{torchvision}. In \figref{appx:regularization}, we extend the comparison of folding and pruning methods on  CIFAR10, including ResNet18 (left column) and VGG11 (right column) models trained with explicit L$_1$ and L$_2$ regularization. L$_1$ regularization, in particular, promotes neuron sparsity, leading structured magnitude pruning methods to outperform model folding under these conditions. However, a comparison between \figref{fig:comparison:ifm} and \figref{appx:regularization} shows that model folding with L$_2$ regularization maintains the highest accuracy at higher sparsity levels, surpassing 80\% accuracy. In contrast, the accuracy of the pruned network trained with L$_1$ drops significantly, reaching just 33\% at 75\% sparsity.

\begin{figure*}[h]
    \centering
     \includegraphics[width=.45\linewidth]{figs/resnet18_cifar10_acc_L1_regularization.jpg}
     \includegraphics[width=.45\linewidth]{figs/vgg11_bn_cifar10_acc_L1_regularization.jpg}
     \includegraphics[width=.45\linewidth]{figs/resnet18_cifar10_acc.jpg}
     \includegraphics[width=.45\linewidth]{figs/vgg11bn_cifar10_acc.jpg}
    \caption{\textbf{ResNet18 (left column) and VGG11 (right column) models trained with L$_1$ (top row) and L$_2$ (bottom row) regularization}. Structured magnitude pruning outperforms model folding only if training explicitly regularizes for model sparsity (L$_1$ norm). REPAIR is hardly beneficial for all structural pruning methods.}
    \label{appx:regularization}
\end{figure*}



\subsection{Folding wider models}
Do wider networks present more opportunities for model folding? We first examine the layer-wise correlation among matched channels in VGG11 and its wider variants on CIFAR10, as shown in \figref{fig:hist:resnet18_cifar10:wider}. This ablation study reveals that increasing the layer width strengthens the matched correlations, suggesting greater potential for folding. Building on this, \figref{fig:widernets} demonstrates the application of model folding also to 1x/2x/3x wider MLP and ResNet50 architectures, trained on CIFAR10 and CIFAR100, showing consistent performance gains as width increases.


\begin{figure*}[t]
    \centering
     \includegraphics[width=.49\linewidth]{figs/mlp_cifar10_wider_model_acc.png}
     \includegraphics[width=.49\linewidth]{figs/resnet50_ciafr100_wider_model_acc_1.jpg}
    \caption{\textbf{Model folding performance improves with increasing model width.} The MLP model consists of three stacked mlp blocks (including a fully connected layer, a BN layer, and a ReLU layer), followed by a final classifier. Upscaled versions of MLP (\textbf{left}) and ResNet50 (\textbf{right}) architectures, trained on CIFAR10 and CIFAR100, demonstrate the consistent advantages of model folding. 
    } 
    \label{fig:widernets}
\end{figure*}


\section{Model Folding on LLMs}
\label{appx:llms}

% \begin{figure*}[h]
%     \centering
%      \includegraphics[width=\linewidth]{figs/llm_folding_ratio_search.csv_combined.png}
%     \caption{\textbf{Layer sensitivity analysis per layer in LLaMA-7B~\citep{llama}.} This figure shows the impact of folding layers on the overall model performance as the compression ratio increases.}
%     \label{fig:llms:compression ratio search}
% \end{figure*}

% \fakeparagraph{Layer Sensitivity Analysis} To maintain better performance of LLMs, different folding ratio needs to be decided for each layer. We analyze the sensitivity of each layer to fold by evaluating how changes in that layer affect the overall model performance, such as perplexity on WikiText2 dataset. In \figref{fig:llms:compression ratio search}, we increase the compression ratio in a single layer of LLMs model and evaluate the compressed model perplexity. To handle the trade-off between sparsity and performance, we set a threshold to select the compression ratio of each layer. To maintain and functionality and better performance, we also didn't fold the first 8 blocks and the last 2 blocks according to the implementation LLM-pruner~\citep{llmpruner}.\dong{this does not work}\\

Table~\ref{tab:llama-7b-example} presents example outputs from both the original and the pruned LLaMA-7B models, as processed by model folding. From the responses presented in Table~\ref{tab:llama-7b-example}, it is evident that when folding 20\% of the parameters, the pruned model continues to perform well.  In Tab.~\ref{tab:llama2performance}, we also compare model folding with these methods on LLaMA2-7B~\citep{llama2}, focusing on perplexity on the WikiText2~\citep{wikitext2} validation set and zero-shot performance across four tasks using the EleutherAI LM Harness~\citep{eval-harness}. We take the same folding sparsity as shown in Tab.~\ref{tab:llmperformance}.

\begin{table}[h!]
\centering
\small{
\resizebox{\textwidth}{!}{
\begin{tabular}{l|l|c|c|ccccc}
\toprule
Prune ratio & Method & Data usage & WikiText2$\downarrow$ & BoolQ & WinoGrande & ARC-e & ARC-c & Average$\uparrow$ \\
\midrule
\textbf{0\%}   & LLaMA2-7B~\citep{llama2}       & /          & 5.12    & 77.7   & 68.98   & 76.34   & 43.26   &  66.57  \\
\midrule
20\%  & Magnitude Prune & /          & Inf   & 44.8   & 49.8   & 26.22   & 21.93   & 44.52   \\ 
20\%  & LLM-Pruner~\citep{llmpruner}     & Gradients  & 10.58   &64.62   &63.54    &68.39    &36.52   & 51.78    \\ 
20\%  & FLAP~\citep{flap}           & Calibration& 6.87    & 71.5   & 68.74   & 70.20   & 36.95   & 61.85   \\ 
20\%  & Wanda\_sp~\citep{wanda}      & Calibration& 8.78   & 72.20   & 63.93   & 70.50   & 40.01   & 61.66   \\ 
20\%  & Model Folding  & /          & 17.46   & 62.72   & 61.80   & 49.00   &26.62   & 50.03   \\ 
\bottomrule
\end{tabular}
}
}
\caption{\textbf{Performance of structured pruning methods on LLaMA2-7B without post-tuning}, showing perplexity on WikiText2 and zero-shot performance across tasks. "Inf" represents an extremely great value. The "Average" is computed over four tasks. "Wanda\_sp" represents an adapted Wanda method for structured pruning. Despite not using data or fine-tuning, model folding achieves comparable performance to data-driven methods.}
\label{tab:llama2performance}
\end{table}


\begin{table*}[h]
    \centering
    %\resizebox{\textwidth}{!}{
    \small
    \renewcommand{\arraystretch}{1.5}     
    \scalebox{0.80}{%
    \begin{tabular}{p{2cm}|p{15cm}}
        \toprule
        Model  & Generated Sentences\\
        \midrule
        \multirow{4}{2cm}{LLaMA-7B (ratio=0\%)} & \underline{Instruction:} What are prime numbers? \underline{Response:} Prime numbers are numbers that are only divisible by 1 and itself. For example, 2, 3, 5, 7, 11, 13, 17, 19, 23, 29, 31, 37, 41, 43, 47, 53, 59, 61, 67, 71, 73\\
        \cline{2-2} 
        & \underline{Instruction:} Please introduce Newton's second law \underline{Response:} Newton's second law of motion states that the net force acting on an object is equal to the rate of change of the object's momentum. The net force acting on an object is equal to the rate of change of the object's momentum. The net force acting on an object is equal to the rate of change of the object's momentum. Newton's second law of motion states that the net force acting on an object is equal to the rate  \\
        \cline{2-2} 
        & \underline{Instruction:} What is the difference between monkeys and gorillas? \underline{Response:} Monkeys are primates that are smaller than gorillas. They are also more agile and have longer tails. Gorillas are larger than monkeys and have shorter tails. What is the difference between a gorilla and a chimpanzee? What is the difference between a gorilla and a chimpanzee? What is the difference between a gorilla and a chimpanzee? What is the difference between a gorilla  \\
        \cline{2-2} 
        & \underline{Instruction:} How many planets are there in the solar system? \underline{Response:} There are eight planets in the solar system. The planets are Mercury, Venus, Earth, Mars, Jupiter, Saturn, Uranus and Neptune. What is the name of the planet that is closest to the sun? What is the name of the planet that is the closest to the sun? What is the name of the planet that is the closest to the sun What is the name of the planet that is the closest to the sun\\
        \midrule
        \multirow{6}{2cm}{LLaMA-7B (ratio=20\%)} & \underline{Instruction:} What are prime numbers? \underline{Response:} A prime number is a number that is divisible by only two numbers. For example, 2, 3, 5, 7, 11, 13, 15, 17, 23, 25, 27, 31, 33, 35, 37, 41, 43, 45, 47, 51\\
        \cline{2-2} 
        & \underline{Instruction:} Please introduce Newton's second law \underline{Response:} The second law of motion is Newton's law of gravity. Newton's law of gravity is that every object is attracted by every other object. The more the two objects are attracted, the more strongly they are attracted. Newton's law of gravity is the law of gravity. What is the second law of motion? What is the second law of motion? What is the second law of motion? What is the second law of motion \\
        \cline{2-2} 
        & \underline{Instruction:} What is the difference between monkeys and gorillas? \underline{Response:} The answer is simple: monkeys are smaller animals while gorillas are larger animals. What is the difference between a monkey and a gorilla? The answer is simple: monkeys are smaller animals while gorillas are larger animals. The size of the body is not the only difference between monkeys and g What is the difference between a monkey and a gri What is the difference between a monkey and a griWhat is the difference between a\\
      \cline{2-2} 
        & \underline{Instruction:} How many planets are there in the solar system? \underline{Response:} There are eight planets in the solar system. The planets are Mercury, Venus, Earth, Mars, Jupiter, Saturn, Aster and Nept. The planets are arranged in order of size from smallest to largest. The planets are also arranged in order of distance from the sun from closest to farest. What is the difference between planets and stars? What is the difference between planets and stars? What is the difference between planets\\
        \bottomrule
    \end{tabular}
    }
    \caption{\textbf{Generated examples from the original LLaMA-7B and pruned by model folding.} The maximal number of output tokens is set to 100 in both models.}
    \label{tab:llama-7b-example}
\end{table*}






\section{Handling Residual Blocks}
\label{appx:residual}

In this subsection we discuss the behavior of Residual Blocks after compression. In a similar manner to the analysis of Normalized Blocks, we investigate the possible dependencies between the clustering matrices for different parts of the residual block and the incoming layers.

\subsection{Simple Residual Blocks}

Consider a Simple Residual Block, consisting of a shortcut represented by an identity transform \( \mathbf{W}_{l,s} = \mathbf{I} \), and a preceding layer decomposed using a clustering matrix \( \mathbf{U}_{l-1} \). The projection matrix is defined as:
\[
\mathbf{C}_{l-1} = \mathbf{U}_{l-1} \left(\mathbf{U}_{l-1}^T \mathbf{U}_{l-1}\right)^{-1} \mathbf{U}_{l-1}^T.
\]
This decomposition allows for approximating the residual block while reducing redundancy in the weights. The residual block approximation satisfies:
\[
\mathbf{y}_l \approx \sigma\left(\mathbf{W}_l^{(2)} \sigma\left(\mathbf{W}_l^{(1)} \mathbf{C}_{l-1}^T \mathbf{x}_{l-1}\right) + \mathbf{C}_{l-1}^T \mathbf{x}_{l-1}\right),
\]
where \( \mathbf{x}_{l-1} \) is the input to the block, \( \mathbf{y}_l \) is the output, and \( \sigma(\cdot) \) represents the activation function. 

The shortcut \( \mathbf{W}_{l,s} = \mathbf{I} \) ensures that the input \( \mathbf{x}_{l-1} \) is directly added to the output of the main path, preserving information and facilitating gradient flow.

\paragraph{Decomposing \( \mathbf{W}_l^{(2)} \).}

Let the weights \( \mathbf{W}_l^{(2)} \) be decomposed using a clustering matrix \( \mathbf{U}_l^{(2)} \) and its corresponding projection:
\[
\mathbf{C}_l^{(2)} = \mathbf{U}_l^{(2)} \left(\mathbf{U}_l^{(2)T} \mathbf{U}_l^{(2)}\right)^{-1} \mathbf{U}_l^{(2)T}.
\]
Substituting this decomposition into the residual block yields:
\[
\mathbf{y}_l \approx \sigma\left(\mathbf{C}_l^{(2)} \mathbf{W}_l^{(2)} \sigma\left(\mathbf{W}_l^{(1)} \mathbf{C}_{l-1}^T \mathbf{x}_{l-1}\right) + \mathbf{C}_{l-1}^T \mathbf{x}_{l-1}\right).
\]
This approximation captures the effect of clustering and compressing the weights while maintaining the structure of the residual block.

\paragraph{Aligning Clustering Matrices.}

To simplify the folding process, we assert that \( \mathbf{U}_{l-1} = \mathbf{U}_l^{(2)} \). This ensures consistency in the clustering across the residual block, reducing the need for additional transformations between layers. As a result, the folding costs for the preceding layer and the current layer can be summed directly:
\[
J_{\text{tot}} = J_l^{(2)} + J_{l-1}.
\]

\paragraph{Total Approximation Error.}

The total approximation error for folding the residual block is defined as:
\[
J_{\text{tot}} = \|\mathbf{W}_{\text{tot}} - \mathbf{C}_l^{(2)} \mathbf{W}_{\text{tot}} \|_F^2,
\]
where:
\[
\mathbf{W}_{\text{tot}} = \begin{bmatrix} \mathbf{W}_{l-1} & \mathbf{W}_l^{(2)} \end{bmatrix}.
\]
Here, \( \mathbf{W}_{\text{tot}} \) combines the weights of both layers in the residual block into a single representation. This unified view allows the clustering process to be applied holistically, ensuring that redundancies across the entire block are captured and reduced.

By asserting \( \mathbf{U}_{l-1} = \mathbf{U}_l^{(2)} \) and summing the individual folding costs \( J_l^{(2)} \) and \( J_{l-1} \), we achieve a compact representation of the residual block with minimal approximation error. This approach ensures that the compressed residual block remains effective while reducing redundancy in the weights.




\subsection{Residual Blocks with Non-Identity Shortcuts}
Consider a Residual Block with a shortcut represented by a weight matrix \( \mathbf{W}_{l, s} \), and a preceding layer decomposed using a clustering matrix \( \mathbf{U}_{l-1} \). The projection matrix is defined as:
\[
\mathbf{C}_{l-1} = \mathbf{U}_{l-1} \left(\mathbf{U}_{l-1}^T \mathbf{U}_{l-1}\right)^{-1} \mathbf{U}_{l-1}^T.
\]
This decomposition allows for approximating and clustering the preceding layer’s weights while maintaining their representational capacity. The corresponding approximation for the residual block satisfies:
\[
\mathbf{y}_{l} \approx \sigma\left(\mathbf{W}_{l}^{(2)} \sigma\left(\mathbf{W}_l^{(1)} \mathbf{C}_{l-1}^T \mathbf{x}_{l-1}\right) + \mathbf{W}_{l, s} \mathbf{C}_{l-1}^T \mathbf{x}_{l-1}\right),
\]
where:
\begin{itemize}
    \item \( \mathbf{W}_l^{(2)} \) is the weight matrix of the second layer in the residual block,
    \item \( \mathbf{W}_l^{(1)} \) is the weight matrix of the first layer in the residual block,
    \item \( \mathbf{W}_{l,s} \) is the shortcut connection weight matrix,
    \item \( \sigma(\cdot) \) represents the activation function.
\end{itemize}

\paragraph{Decomposition of Weight Matrices.}

The weights \( \mathbf{W}_l^{(2)} \) and \( \mathbf{W}_{l,s} \) are decomposed using their respective clustering matrices. For \( \mathbf{W}_l^{(2)} \), the decomposition is:
\[
\mathbf{C}_{l}^{(2)} = \mathbf{U}_{l}^{(2)} \left(\mathbf{U}_{l}^{(2)T} \mathbf{U}_{l}^{(2)}\right)^{-1} \mathbf{U}_{l}^{(2)T}.
\]
For \( \mathbf{W}_{l,s} \), the decomposition is:
\[
\mathbf{C}_{l, s} = \mathbf{U}_{l, s} \left(\mathbf{U}_{l, s}^T \mathbf{U}_{l, s}\right)^{-1} \mathbf{U}_{l, s}^T.
\]
Substituting these decompositions into the approximation yields:
\[
\mathbf{y}_{l} \approx \sigma\left(\mathbf{C}_{l}^{(2)} \mathbf{U}_{l}^{(2)T} \mathbf{W}_{l}^{(2)} \sigma\left(\mathbf{W}_l^{(1)} \mathbf{C}_{l-1}^T \mathbf{x}_{l-1}\right) + \mathbf{C}_{l, s} \mathbf{W}_{l, s} \mathbf{C}_{l-1}^T \mathbf{x}_{l-1}\right).
\]

\paragraph{Consistency Constraint and Total Approximation Error.}
To simplify the folding process and ensure consistency across the layers, we introduce the constraint:
\[
\mathbf{U}_{l, s} = \mathbf{U}_{l}^{(2)}.
\]
This ensures that the same clustering matrix is used for both the shortcut weights \( \mathbf{W}_{l, s} \) and the second layer’s weights \( \mathbf{W}_{l}^{(2)} \). By adding the individual folding costs \( J_l^{(2)} \) and \( J_{l, s} \), we ensure that Lemma~\ref{lemma1} holds, leading to the total approximation error for the residual block:
\[
J_\text{tot} = J_l^{(2)} + J_{l,s}.
\]

\paragraph{Unified Approximation for Residual Blocks.}
The total approximation error can be expressed compactly as:
\[
J_\text{tot} = \|\mathbf{W}_\text{tot} - \mathbf{C}_{l}^{(2)} \mathbf{W}_\text{tot}\|_F^2,
\]
where:
\[
\mathbf{W}_\text{tot} = 
    \begin{bmatrix}
        \mathbf{W}_{l, s} \mid \mathbf{W}_{l}^{(2)}
    \end{bmatrix}.
\]
Here, \( \mathbf{W}_\text{tot} \) combines the shortcut weights \( \mathbf{W}_{l, s} \) and the second-layer weights \( \mathbf{W}_l^{(2)} \) into a single matrix. This unified representation allows the folding process to be applied holistically, reducing redundancies across the entire residual block.

The decomposition of weights in residual blocks with non-identity shortcuts introduces a consistent clustering mechanism for both the shortcut and the second layer. By ensuring that \( \mathbf{U}_{l, s} = \mathbf{U}_l^{(2)} \), we maintain alignment in the clustering process, leading to a compact and efficient representation with minimal approximation error.





\section{Handling Batch Normalization Layers}
\label{appx:bn}
Batch Normalization layers, when combined with linear layers, introduce additional scaling and normalization operations. One special case is a layer consisting of a linear block followed by a Batch Normalization block, formally defined as:
\[
\mathbf{z}_{l+1} = \mathbf{W}_{l+1}\sigma(\mathbf{\Sigma}_s\mathbf{\Sigma}_n\mathbf{W}_l \mathbf{x}_{l-1}),
\]
where:
\begin{itemize}
    \item \( \mathbf{W}_l \): weight matrix of the linear block,
    \item \( \mathbf{\Sigma}_s \): Batch Normalization scaling matrix,
    \item \( \mathbf{\Sigma}_n \): Batch Normalization normalization matrix,
    \item \( \mathbf{W}_{l+1} \): weight matrix of the subsequent layer,
    \item \( \sigma(\cdot) \): activation function applied element-wise.
\end{itemize}

A design choice in handling such layers is to decompose \( \mathbf{\Sigma}_s \), \( \mathbf{\Sigma}_n \), and \( \mathbf{W}_l \) separately while preserving the original structure of the layer. This ensures that the scaling, normalization, and linear blocks are treated as distinct functional units. The decomposed approximation for the layer can then be expressed as:
\[
\mathbf{z}_{l+1} \approx \Tilde{\mathbf{z}}_{l+1} = \mathbf{W}_{l+1}\mathbf{C}_{s}^T\sigma(\mathbf{C}_{s}\mathbf{\Sigma}_s\mathbf{C}_{n}\mathbf{\Sigma}_n\mathbf{C}_{l}\mathbf{W}_l \mathbf{x}_{l-1}),
\]
where the projection matrices \( \mathbf{C}_s \), \( \mathbf{C}_n \), and \( \mathbf{C}_l \) are defined as:
\begin{align*}    
    \mathbf{C}_{s} &= \mathbf{U}_{s}(\mathbf{U}_{s}^T\mathbf{U}_{s})^{-1}\mathbf{U}_{s}^T = \mathbf{U}_{s}\mathbf{M}_{s}, \\
    \mathbf{C}_{n} &= \mathbf{U}_{n}(\mathbf{U}_{n}^T\mathbf{U}_{n})^{-1}\mathbf{U}_{n}^T = \mathbf{U}_{n}\mathbf{M}_{n}, \\
    \mathbf{C}_{l} &= \mathbf{U}_{l}(\mathbf{U}_{l}^T\mathbf{U}_{l})^{-1}\mathbf{U}_{l}^T = \mathbf{U}_{l}\mathbf{M}_{l}.
\end{align*}
Here, \( \mathbf{U}_s \), \( \mathbf{U}_n \), and \( \mathbf{U}_l \) are clustering matrices, and \( \mathbf{M}_s \), \( \mathbf{M}_n \), and \( \mathbf{M}_l \) are normalization terms.

\paragraph{Clustering Assumptions.}
To simplify the decomposition and ensure alignment across the layer components, we impose the following consistency constraint:
\[
\mathbf{U}_{s} = \mathbf{U}_{n} = \mathbf{U}_{l}.
\]
This assumption ensures that the same clustering structure is applied to the scaling, normalization, and linear blocks, leading to a unified decomposition. Under this assumption, the approximation becomes:
\[
\Tilde{\mathbf{z}}_{l+1} = \mathbf{W}_{l+1}\mathbf{C}_{l}^T\sigma(\mathbf{U}_l\mathbf{M}_l\mathbf{W}_{b,l}\mathbf{U}_l\mathbf{M}_l\mathbf{\Sigma}_n\mathbf{U}_l\mathbf{M}_{l}\mathbf{W}_l\mathbf{x}_{l-1}),
\]
where \( \mathbf{W}_{b,l} \) represents the intermediate scaling factors. 

\paragraph{Applying Diagonal Properties.}
Using Lemma~\ref{diag_u}, we observe that the normalization and scaling matrices can be represented as diagonal matrices:
\[
\Tilde{\mathbf{z}}_{l+1} = \mathbf{W}_{l+1}\mathbf{C}_{l}^T\sigma(\mathbf{U}_l\text{Diag}(\mathbf{M}_l\text{diag}(\mathbf{W}_{b,l}))\text{Diag}(\mathbf{M}_l\text{diag}(\mathbf{\Sigma}_n))\mathbf{M}_{l}\mathbf{W}_l \mathbf{x}_{l-1}).
\]
Furthermore, by applying Lemma~\ref{diag-map}, we rewrite this expression as:
\[
\Tilde{\mathbf{z}}_{l+1} = \mathbf{W}_{l+1}\mathbf{C}_{l}^T\sigma(\text{Diag}(\mathbf{C}_l\text{diag}(\mathbf{W}_{b,l}))\text{Diag}(\mathbf{C}_l\text{diag}(\mathbf{\Sigma}_n))\mathbf{C}_{l}\mathbf{W}_l \mathbf{x}_{l-1}).
\]
This shows that the diagonal structure of the scaling and alignment matrices is preserved through the decomposition, maintaining the original behavior of the Batch Normalization block.

\paragraph{Compression Cost.}
According to the definition of the Model Folding problem and using the properties stated in Lemma~\ref{diag_norm}, the compression cost for the layer can be expressed as:
\[
J_{tot} = \|{\mathbf{W}}_{tot} - \mathbf{C}_l{\mathbf{W}}_{tot}\|_F^2,
\]
where:
\[
{\mathbf{W}}_{tot} = \begin{bmatrix}\mathbf{W}_{l+1}^T & \mathbf{W}_l & \text{diag}(\mathbf{\Sigma}_s) & \text{diag}(\mathbf{\Sigma}_n)\end{bmatrix}.
\]
This cost quantifies the approximation error introduced by clustering the weights, scaling, and normalization matrices while preserving the layer's functional structure.

By decomposing the Batch Normalization and linear blocks separately and aligning their clustering structures (\( \mathbf{U}_{s} = \mathbf{U}_{n} = \mathbf{U}_{l} \)), we ensure that the original diagonal properties of the scaling and normalization matrices are preserved. The resulting compression cost captures the overall error of folding the entire layer into a compact representation.




\subsection{Algorithmic Description of Fold-AR}

The Fold-AR algorithm for a single layer combines the Batch Normalization components and layer weights into a compact representation, followed by clustering to reduce redundancy. The steps are described in Algorithm~\ref{alg:fold-ar}.

\begin{algorithm}[H]
\caption{Fold-AR for a Single Layer}
\label{alg:fold-ar}
\begin{algorithmic}[1]
\Require $\mathbf{\Sigma}_s$, $\mathbf{\Sigma}_n$, $\mathbf{W}_l$, $\mathbf{W}_{l+1}$ \Comment{Input components of the layer}
\State Compute the normalized weight matrix: $\hat{\mathbf{W}}_l \gets \mathbf{\Sigma}_n \mathbf{W}_l$
\State Construct the combined weight matrix: $\mathbf{W}_{\text{tot}} \gets \begin{bmatrix} \mathbf{W}_{l+1}^T & \hat{\mathbf{W}}_l & \text{diag}(\mathbf{\Sigma}_s) \end{bmatrix}$
\State Solve the clustering problem:
\[
\mathbf{U} \gets \argmin_{\mathbf{U}} \|\mathbf{W}_{\text{tot}} - \mathbf{U}(\mathbf{U}^T\mathbf{U})^{-1}\mathbf{U}^T\mathbf{W}_{\text{tot}}\|_F^2
\]
\hspace{4em} subject to $\mathbf{U}^T \in \{0, 1\}^{m \times n}$ and $m < n$
\State Update the scaling matrix: $\mathbf{\Sigma}_s \gets (\mathbf{U}^T \mathbf{U})^{-1} \mathbf{U}^T \mathbf{\Sigma}_s \mathbf{U}$
\State Update the second-layer weights: $\mathbf{W}_{l+1}^T \gets \mathbf{U}^T \mathbf{W}_{l+1}^T$
\State Update the current-layer weights: $\hat{\mathbf{W}}_l \gets (\mathbf{U}^T \mathbf{U})^{-1} \mathbf{U}^T \hat{\mathbf{W}}_l$
\For{$c = 1, \dots, m$} \Comment{Adjust scaling factors for each cluster}
    \State Compute cluster size: $N_c \gets \sum_{i} \mathbb{I}(\mathbf{U}_{i,c} = 1)$ \Comment{$\mathbb{I}(\cdot)$ is the indicator function}
    \State Compute intra-cluster correlation:
    \[
    E[c] \gets \frac{1}{N_c^2 - N_c} \sum_{i, j} \frac{\hat{\mathbf{w}}_{l,i,:} \cdot \hat{\mathbf{w}}_{l,j,:}^T}{\sqrt{\|\hat{\mathbf{w}}_{l,i,:}\|^2 \|\hat{\mathbf{w}}_{l,j,:}\|^2}}
    \mathbb{I}(\mathbf{U}_{i,c} = \mathbf{U}_{j,c} = 1) \mathbb{I}(i \neq j)
    \]
    \State Update the scaling factor for cluster $c$:
    \[
    (\mathbf{\Sigma}_s)_{c,c} \gets (\mathbf{\Sigma}_s)_{c,c} \frac{N_c}{\sqrt{N_c + (N_c^2 - N_c) E[c]}}
    \]
\EndFor
\end{algorithmic}
\end{algorithm}

\subsubsection*{Explanation of Key Steps}

\paragraph{1. Combining Normalization and Weights.}
The normalization matrix \( \mathbf{\Sigma}_n \) is diagonal, and multiplying it with the weight matrix \( \mathbf{W}_l \) produces the normalized weight matrix:
\[
\hat{\mathbf{W}}_l = \mathbf{\Sigma}_n \mathbf{W}_l.
\]
This step integrates the normalization operation into the weights of the current layer, reducing the complexity of subsequent computations.

\paragraph{2. Construction of Combined Weight Matrix.}
The combined matrix \( \mathbf{W}_{\text{tot}} \) is defined as:
\[
\mathbf{W}_{\text{tot}} = \begin{bmatrix} \mathbf{W}_{l+1}^T & \hat{\mathbf{W}}_l & \text{diag}(\mathbf{\Sigma}_s) \end{bmatrix}.
\]
This matrix aggregates the second-layer weights (\( \mathbf{W}_{l+1}^T \)), the normalized current-layer weights (\( \hat{\mathbf{W}}_l \)), and the scaling factors (\( \text{diag}(\mathbf{\Sigma}_s) \)) into a single representation, preparing them for joint clustering.

\paragraph{3. Clustering.}
The projection matrix \( \mathbf{U} \) is computed by solving the clustering problem:
\[
\mathbf{U} = \argmin_{\mathbf{U}} \|\mathbf{W}_{\text{tot}} - \mathbf{U}(\mathbf{U}^T \mathbf{U})^{-1}\mathbf{U}^T \mathbf{W}_{\text{tot}}\|_F^2,
\]
subject to \( \mathbf{U}^T \in \{0, 1\}^{m \times n} \) and \( m < n \). The clustering minimizes the reconstruction error by projecting the combined weights into a lower-dimensional space defined by \( m \) clusters.

\paragraph{4. Scaling Adjustments.}
To ensure proper scaling within each cluster, the diagonal elements of \( \mathbf{\Sigma}_s \) are updated. For each cluster \( c \), the adjustment considers the size of the cluster (\( N_c \)) and the intra-cluster correlation (\( E[c] \)):
\[
(\mathbf{\Sigma}_s)_{c,c} \gets (\mathbf{\Sigma}_s)_{c,c} \frac{N_c}{\sqrt{N_c + (N_c^2 - N_c)E[c]}}.
\]
The intra-cluster correlation \( E[c] \) is computed as a normalized dot product, capturing the redundancy among the weights within the same cluster. This adjustment preserves the scaling properties of the original layer.

\paragraph{5. Final Updates.}
The current-layer weights \( \hat{\mathbf{W}}_l \) and second-layer weights \( \mathbf{W}_{l+1}^T \) are updated to align with the clustered representation:
\[
\hat{\mathbf{W}}_l \gets (\mathbf{U}^T \mathbf{U})^{-1} \mathbf{U}^T \hat{\mathbf{W}}_l, \quad \mathbf{W}_{l+1}^T \gets \mathbf{U}^T \mathbf{W}_{l+1}^T.
\]
These updates ensure consistency between the clustered weights and the projection matrix \( \mathbf{U} \).

This algorithm combines clustering, scaling adjustments, and weight updates to compress the layer while preserving its functional properties. The clustering step minimizes redundancy, and the final updates align all components of the layer with the clustered structure.








\section{Folding Similar Channels in MLPs}
\label{appx:similar_in_mlps}
For fully connected networks, where two successive layers are defined as:
\[
	\mathbf{x}_{l} = \sigma(\mathbf{W}_{l}\mathbf{x}_{l-1}) \;\; \text{and} \;\; \mathbf{x}_{l+1} = \sigma(\mathbf{W}_{l+1}\mathbf{x}_l),
\]
where \( \mathbf{x}_l \) represents the activations of layer \( l \), \( \mathbf{W}_l \) and \( \mathbf{W}_{l+1} \) are the weight matrices, and \( \sigma \) is the activation function. The channels of the layer are defined as the coordinates \( \mathbf{x}_{l,i} \) of the vector \( \mathbf{x}_l \). Each channel corresponds to a specific dimension in the activations.

The folding cost \( J_l \) for the \( l \)-th layer is defined as:
\[
	J_l = \left\| \mathbf{W}_l - \mathbf{C}_l \mathbf{W}_l\right\|_F^2 + \left\| \mathbf{W}_{l+1}^T - \mathbf{C}_l \mathbf{W}_{l+1}^T\right\|_F^2,
\]
where \( \mathbf{C}_l \) is a clustering matrix. This cost function represents the optimization objective to minimize the approximation error introduced by folding (clustering) the weights of the \( l \)-th layer. The first term measures the reconstruction error for the weights \( \mathbf{W}_l \), while the second term measures the reconstruction error for the weights \( \mathbf{W}_{l+1} \) under the transformation \( \mathbf{C}_l \). Together, these terms ensure that the clustering transformation preserves the structure and relationships of the weights across layers.

From the perspective of K-Means as a matrix decomposition problem, the grouping of scalar weights into vectors is defined as follows:
\[
	\mathbf{W}_l = \begin{bmatrix}
    	\mathbf{p}_1^T \\
    	\mathbf{p}_2^T \\
    	\vdots \\
    	\mathbf{p}_n^T
	\end{bmatrix} \;\; \text{and} \;\; \mathbf{W}_{l+1} = \begin{bmatrix}
    	\mathbf{q}_1 &
    	\mathbf{q}_2 &
    	\ldots &
    	\mathbf{q}_n
	\end{bmatrix},
\]
where \( \mathbf{p}_i^T \) are the rows of \( \mathbf{W}_l \) and \( \mathbf{q}_i \) are the columns of \( \mathbf{W}_{l+1} \). These groupings reflect the natural structure of the weight matrices in fully connected layers:
\begin{itemize}
    \item Each row of \( \mathbf{W}_l \) represents the weights associated with a specific output channel of layer \( l \).
    \item Each column of \( \mathbf{W}_{l+1} \) represents the weights associated with a specific input channel of layer \( l+1 \).
\end{itemize}

In this formulation, the rows \( \mathbf{p}_i^T \) and columns \( \mathbf{q}_i \) are treated as vectors to be clustered by the matrix \( \mathbf{C}_l \), which aligns with the K-Means decomposition perspective. The clustering matrix \( \mathbf{C}_l \) maps these weights into representative clusters, preserving the relationships between input and output channels across layers while enabling efficient compression.






\section{Folding Similar Channels in Convolutional Layers}
\label{appx:similar_in_cnn}
For convolutional layers, two successive layers can be defined as:
\[
\mathcal{X}_{l} = \sigma(\mathcal{W}_{l} * \mathcal{X}_{l-1}) \quad \text{and} \quad \mathcal{X}_{l+1} = \sigma(\mathcal{W}_{l+1} * \mathcal{X}_l),
\]
where \( \mathcal{X}_l \) is a 3-dimensional feature tensor with values \( \mathcal{X}^{(l)}_{c_o,i,j} \). The first dimension, \( c_o \), corresponds to the output channels, while \( i \) and \( j \) represent spatial pixel locations. The 4-dimensional weight tensor \( \mathcal{W}_l \) has values \( \mathcal{W}^{(l)}_{c_o, c_i, i,j} \), where:
\begin{itemize}
    \item \( c_o \) corresponds to the output channels of \( \mathcal{X}_l \),
    \item \( c_i \) corresponds to the input channels of \( \mathcal{X}_{l-1} \).
\end{itemize}
To simplify and compress the network, we decompose the weight tensor \( \mathcal{W}_l \) such that output channels of \( \mathcal{X}_l \) (i.e., the values \( \mathcal{X}^{(l)}_{c_o,i,j} \) for \( c_o = 1, \ldots, c_{\text{out}} \)), which are similar in some sense, are merged. This folding problem is defined as:
\[
J_l = \left\|\mathcal{W}_l - \mathcal{C}_l \circ \mathcal{W}_l\right\|_T^2 + \left\|\mathcal{W}_{l+1} - \mathcal{W}_{l+1} \circ \mathcal{C}_l\right\|_T^2,
\]
where \( \mathcal{C}_l \) corresponds to a \( 1 \times 1 \) convolution parameterized by the clustering matrix \( \mathbf{C}_l \), with \( \mathcal{C}^{(l)}_{c, 1, 1} = \mathbf{C}_{l, c, c'} \).

From this definition, it follows that:
\[
J_l = \left\|\mathbf{W}_l - \mathbf{C}_l \mathbf{W}_l\right\|_T^2 + \left\|\mathbf{W}_{l+1} - \mathbf{W}_{l+1}\mathbf{C}_l^T\right\|_T^2,
\]
where the weight tensors \( \mathcal{W}_l \) and \( \mathcal{W}_{l+1} \) are mapped to matrices \( \mathbf{W}_l \) and \( \mathbf{W}_{l+1} \) as follows:
\[
\mathbf{W}_l = 
\begin{bmatrix}
    \text{vec}(\mathcal{W}^{(l)}_{1, 1, :, :})^T & \text{vec}(\mathcal{W}^{(l)}_{1, 2, :, :})^T & \cdots & \text{vec}(\mathcal{W}^{(l)}_{1, c_{\text{in}}, :, :})^T \\
    \text{vec}(\mathcal{W}^{(l)}_{2, 1, :, :})^T & \text{vec}(\mathcal{W}^{(l)}_{2, 2, :, :})^T & \cdots & \text{vec}(\mathcal{W}^{(l)}_{2, c_{\text{in}}, :, :})^T \\
    \vdots  & \vdots & \ddots & \vdots \\
    \text{vec}(\mathcal{W}^{(l)}_{c_{\text{out}}, 1, :, :})^T & \text{vec}(\mathcal{W}^{(l)}_{c_{\text{out}}, 2, :, :})^T & \cdots & \text{vec}(\mathcal{W}^{(l)}_{c_{\text{out}}, c_{\text{in}}, :, :})^T \\
\end{bmatrix}.
\]

This means that each convolutional filter contributing to an output channel \( c_o \) is flattened and stacked into a vector, forming the \( c_o \)-th row of the matrix \( \mathbf{W}_l \). Similarly, for \( \mathcal{W}_{l+1} \), each filter associated with the \( c_i \)-th input channel is flattened and stacked into a vector, forming a column of the matrix \( \mathbf{W}_{l+1} \):

\[
\mathbf{W}_{l+1} = 
\begin{bmatrix}
    \text{vec}(\mathcal{W}^{(l+1)}_{1, 1, :, :}) & \text{vec}(\mathcal{W}^{(l+1)}_{1, 2, :, :}) & \cdots & \text{vec}(\mathcal{W}^{(l+1)}_{1, c_{\text{in}}, :, :}) \\
    \text{vec}(\mathcal{W}^{(l+1)}_{2, 1, :, :}) & \text{vec}(\mathcal{W}^{(l+1)}_{2, 2, :, :}) & \cdots & \text{vec}(\mathcal{W}^{(l+1)}_{2, c_{\text{in}}, :, :}) \\
    \vdots  & \vdots & \ddots & \vdots \\
    \text{vec}(\mathcal{W}^{(l+1)}_{c_{\text{out}}, 1, :, :}) & \text{vec}(\mathcal{W}^{(l+1)}_{c_{\text{out}}, 2, :, :}) & \cdots & \text{vec}(\mathcal{W}^{(l+1)}_{c_{\text{out}}, c_{\text{in}}, :, :}) \\
\end{bmatrix}.
\]

From the perspective of K-Means as a matrix decomposition problem, the grouping of scalar weights into vectors is defined as follows:
\[
\mathbf{W}_l = \begin{bmatrix}
    \mathbf{p}_1^T \\
    \mathbf{p}_2^T \\
    \vdots \\
    \mathbf{p}_n^T
\end{bmatrix} \quad \text{and} \quad \mathbf{W}_{l+1} = \begin{bmatrix}
    \mathbf{q}_1 &
    \mathbf{q}_2 &
    \cdots &
    \mathbf{q}_n
\end{bmatrix},
\]
where:
\[
\mathbf{p}_i^T = \begin{bmatrix}
    \text{vec}(\mathcal{W}^{(l)}_{i, 1, :, :})^T & \text{vec}(\mathcal{W}^{(l)}_{i, 2, :, :})^T & \cdots & \text{vec}(\mathcal{W}^{(l)}_{i, c_{\text{in}}, :, :})^T
\end{bmatrix},
\]
and:
\[
\mathbf{q}_j = \begin{bmatrix}
    \text{vec}(\mathcal{W}^{(l+1)}_{1, j, :, :})^T & \text{vec}(\mathcal{W}^{(l+1)}_{2, j, :, :})^T & \cdots & \text{vec}(\mathcal{W}^{(l+1)}_{c_{\text{out}}, j, :, :})^T
\end{bmatrix}^T.
\]

In this formulation, the rows \( \mathbf{p}_i^T \) of \( \mathbf{W}_l \) and columns \( \mathbf{q}_j \) of \( \mathbf{W}_{l+1} \) are grouped into clusters for the folding process, aligning with the K-Means decomposition perspective.





\section{Folding Similar Channels in LlamaMLP and LlamaAttention}
\label{appx:similar_in_llama}
\subsection{Folding Similar Channels in LlamaMLP}

The LlamaMLP module is composed of three sub-layers: \texttt{gate\_proj}, \texttt{up\_proj}, and \texttt{down\_proj}. These sub-layers define the structure and functionality of the MLP, with the main computation pipeline expressed as:
\[
\texttt{down\_proj}(\texttt{act\_fn}(\texttt{gate\_proj}(x)) \times \texttt{up\_proj}(x)).
\]
We cluster similar channels in both the output channel and input channel of each sub-layer.

\paragraph{Input Channel Folding.}  
To fold the \textbf{input channels} of LlamaMLP, we simultaneously consider the input dimensions of both \texttt{gate\_proj} and \texttt{up\_proj} layers, as they collectively define the effective input to the \texttt{gate\_up} sub-layer. The input channels of \texttt{gate\_proj} and \texttt{up\_proj} are clustered respectively using methods similar to those applied in standard MLP layers.

\paragraph{Output Channel Folding.}  
To fold the \textbf{output channels} of LlamaMLP, we first consider the output channels of both \texttt{gate\_proj} and \texttt{up\_proj} by clustering and adjusting the input channel of the \texttt{down\_proj}. Subsequently, we adjust the output channel of \texttt{down\_proj} according to the residual connection used outside of LlamaMLP.

\subsection{Folding Similar Channels in LlamaAttention}

The LlamaAttention module consists of four primary sub-layers: \texttt{q\_proj}, \texttt{k\_proj}, \texttt{v\_proj}, and \texttt{o\_proj}. These sub-layers define the query, key, value, and output projections, respectively. For clarity and simplicity, we conceptualize \texttt{q\_proj}, \texttt{k\_proj}, and \texttt{v\_proj} as a unified sub-layer referred to as \texttt{q\_k\_v}, which computes the intermediate representations required for attention calculations. The \texttt{o\_proj} sub-layer processes the final output of the attention mechanism. We treat the attention head as the structure to be folded in LlamaAttention. By reshaping the weights of each sub-layer into an MLP-like tensor, we can cluster similar heads, similar to how it is done for a standard MLP layer.

For all configurations of LlamaAttention, including Multi-Head Attention (MHA) and Grouped Query Attention (GQA), the weight shapes of the \texttt{q\_k\_v} sub-layer differ:
\begin{itemize}
    \item In MHA, the weights for \texttt{q}, \texttt{k}, and \texttt{v} projections share the same shape: \([ \text{num\_heads} \times \text{head\_dim}, \text{hidden\_size} ]\).
    \item In GQA, the weights for \texttt{k} and \texttt{v} projections have the shape: \([ \text{num\_kv\_heads} \times \text{head\_dim}, \text{hidden\_size} ]\).
\end{itemize}

\paragraph{Output Channel Folding.}  
When performing \textbf{output channel folding} for the LlamaAttention layer, the clustering of the \texttt{o\_proj} sub-layer's output channels is dictated by the residual connection outside of LlamaAttention, ensuring alignment with the clustering results from previous modules. Specifically:
\begin{itemize}
    \item The \texttt{o\_proj} weights, originally shaped as 
    \([ \texttt{num\_heads} \times \texttt{head\_dim}, \texttt{hidden\_size} ]\), 
    are reshaped into 
    \([ \texttt{num\_heads}, \texttt{head\_dim}, \texttt{hidden\_size} ]\), 
    clustered along the first dimension (\texttt{num\_heads}), 
    and then reshaped back to their original form.
    \item For clustering within the \texttt{q\_k\_v} sub-layer, the weights for \texttt{q}, \texttt{k}, and \texttt{v} are reshaped into 
    \([ \texttt{num\_heads},\) \(\texttt{head\_dim}, \texttt{hidden\_size} ]\) (or 
    \([ \texttt{num\_kv\_heads}, \texttt{head\_dim}, \texttt{hidden\_size} ]\) for \texttt{k} and \texttt{v} in GQA) 
    and clustered along the first dimension (\texttt{num\_heads} or \texttt{num\_kv\_heads}). 
    After clustering, the weights are reshaped back to their original dimensions.
\end{itemize}

\paragraph{Input Channel Folding.}  
To perform \textbf{input channel folding}, the focus is on the input channels of \texttt{q}, \texttt{k}, and \texttt{v} weights. Since these weights share the same input \texttt{hidden\_states}, each of their weights is clustered along the first dimension (\texttt{hidden\_size}) of their respective matrices. This ensures that the clustering process respects the shared input representation across the \texttt{q\_k\_v} sub-layer while maintaining the integrity of the attention mechanism.


\section{Comparison with Knowledge Distillation}
\label{appx:kd}
We evaluated some data-free knowledge distillation (KD) methods~\citep{micaelli2019zeroshotknowledgetransferadversarial,chen2019datafreelearningstudentnetworks,fang2020datafreeadversarialdistillation,yu2023data}, on an NVIDIA A100 GPU, for all methods using the same pre-trained teacher model, data loader, and student model setup for consistency. The full model is a ResNet18 pre-defined by torchvision and trained on CIFAR10, while the student models for each KD method share the same architecture but differ in the number of channels across all layers to achieve the desired sparsity levels. Specifically, in ResNet18, the number of output channels for all blocks is a multiple of 64, which is also the number of output channels in the first convolutional layer. To reduce the model's channel dimensions, we scale this base hyperparameter by a reduction factor, effectively reducing the width of all layers proportionally. The following table presents the test accuracy of compressed by KD methods and model folding on CIFAR10 test dataset.The time taken to achieve each accuracy is provided in parentheses next to the corresponding accuracy value. From the table, it is evident that the proposed model folding achieves model compression within seconds, even at high sparsity levels, compared to other KD methods that require tens of hours to complete.
\begin{table}[h]
\centering
\small{
\resizebox{\textwidth}{!}{
\begin{tabular}{l|c|c|c|c|c}
\toprule
Sparsity & Full model & 10\% & 25\% & 50\% & 70\% \\
\midrule
ABM~\citep{micaelli2019zeroshotknowledgetransferadversarial} & 94.72 & 93.30 (17h19m) & 91.99 (16h8m) & 89.42 (15h30m) & 85.43 (13h23m) \\
DFAD~\citep{chen2019datafreelearningstudentnetworks} & 94.72 & 93.79 (2h31m) & 93.52 (2h3m) & 92.04 (2h1m) & 89.67 (1h54m) \\
DAFL~\citep{fang2020datafreeadversarialdistillation} & 94.72 & 71.73 (16h48m) & 77.80 (15h39m) & 68.06 (15h19m) & 53.86(76h34m) \\
SpaceshipNet~\citep{yu2023data} & 94.72 & 94.50 (42h33m) & 93.95 (40h3m) & 92.96 (37h57m) & 91.53 (27h10m) \\
\textbf{Model Folding (ours)} & 94.72 & 94 (56.35s) & 92 (53.55s) & 88 (55.75s) & 82 (54.95s) \\
\bottomrule
\end{tabular}
}
}
\caption{\textbf{Performance comparison of knowledge distillation and model folding}, showing accuracy (\%) and runtime (in parentheses). The sparsity levels indicate the percentage of weights pruned.}
\label{tab:kd_comparison}
\end{table}



\section{Inference Speed of Folded Models on Edge Devices}
\label{appx:devices}
 We apply model folding on a LeNet5 model pre-trained on FashionMNIST with different sparsity, and then evaluate the folded models on NVIDIA Jetson Nano, ESP-EYE, and Arduino Nano 33 BLE. All models are converted and executed as a float32 Tensorflow Lite model in all devices.

\begin{table}[h]
\centering
\small{
\resizebox{\textwidth}{!}{
\begin{tabular}{l|ccc|ccc|ccc|ccc}
\toprule
Sparsity & \multicolumn{3}{c|}{10\%} & \multicolumn{3}{c|}{25\%} & \multicolumn{3}{c|}{50\%} & \multicolumn{3}{c}{70\%} \\
         & Runtime & RAM  & Flash  & Runtime & RAM  & Flash  & Runtime & RAM  & Flash  & Runtime & RAM  & Flash  \\
\midrule
NVIDIA Jetson Nano~\citep{nvidia_jetson_nano}        & 2ms   & 59.5K & 3.4M & 2ms   & 55.7K & 2.8M & 1ms   & 48.0K & 1.9M & 1ms   & 36.5K & 1.2M \\
ESP-EYE~\citep{esp_eye}                   & 2591ms & 59.5K & 3.4M & 1868ms & 55.7K & 2.8M & 1532ms & 48.0K & 1.9M & 1186ms & 36.5K & 1.2M \\
Arduino Nano 33 BLE Sense~\citep{arduino_nano_33_ble} & 6831ms & 59.5K & 3.4M & 3726ms & 55.7K & 2.8M & 4218ms & 48.0K & 1.9M & 2969ms & 36.5K & 1.2M \\
\bottomrule
\end{tabular}
}
}
\caption{\textbf{Performance and resource usage at various sparsity levels across devices}, with detailed breakdowns for runtime (ms), RAM usage (K), and Flash storage usage (M).}
\label{tab:device_performance}
\end{table}






\section{Deep Inversion Sample Images}
\label{appx:dee_inversion}

Deep Inversion (DI)~\citep{yin2020dreamingdistilldatafreeknowledge} generates synthetic images from the uncompressed network by optimizing noise to match the internal statistics stored in BatchNorm layers. These images, exemplified in \figref{fig:deep_dream}, which reflect the original data's statistical properties, are used during model folding to restore data statistics in the compressed network, ensuring accuracy without requiring external data.

\begin{figure*}[h!]
    \centering
     \includegraphics[width=.43\linewidth]{figs/deep_inversion_samples.pdf}
    \caption{\textbf{Sample images generated by Deep Inversion~\citep{yin2020dreamingdistilldatafreeknowledge} using ResNet18 trained on CIFAR100.} These images are generated from the uncompressed network and used in model folding to restore data statistics in the compressed network.}
    \label{fig:deep_dream}
\end{figure*}





\section{Further Related Work}
\label{appx:related}

Model folding intersects with several established approaches in model compression, network architecture optimization and model merging. This section outlines key related works that inspired the development of model folding, highlighting both their contributions and limitations.
 
\subsection{Model compression}
Model compression techniques reduce models' size and computational requirements while maintaining or minimally sacrificing performance. Various methods have been developed. Most can be classified as pruning, quantization, knowledge distillation, and low-rank factorization. Traditional pruning techniques~\citep{han2015learning, NIPS1989_6c9882bb,li2016pruning,hassibi1993optimal,entezari2020classdependentcompressiondeepneural}, structured or unstructured, involve removing weights, neurons, or filters that are deemed less important, typically measured by the magnitude of their contributions (\eg, L$_1$ or L$_2$ norm)~\citep{entezari2020classdependentcompressiondeepneural,li2017pruningfiltersefficientconvnets,cheng2023surveydeepneuralnetwork}. While effective in reducing the size of the model, pruning often leads to a degradation of performance that requires fine-tuning or complete retraining of the network~\citep{cheng2023surveydeepneuralnetwork,han2015learning,frankle2018lottery,frantar2022optimal,he2018multi}. Quantization~\citep{gupta2015deep,zhou2017incremental,li2016ternary} reduces the precision of the numerical values in a model, from floating-point to lower-bit representations (\eg, 8-bit integers). This approach significantly reduces the model's memory footprint and speeds up computation, especially when combined with hardware accelerators designed for low-precision arithmetic~\citep{gholami2021surveyquantizationmethodsefficient}. Like pruning, post-training quantization may also require fine-tuning to restore model performance. Knowledge distillation~\citep{hinton2015distillingknowledgeneuralnetwork} trains a smaller model, called the student, to replicate a well-trained larger model, called the teacher, by mimicking the output of the teacher model, which transfers knowledge between the teacher model and the student model. While effective in transferring knowledge and reducing model size, the training process for knowledge distillation can be computationally expensive and time-consuming~\citep{hinton2015distillingknowledgeneuralnetwork,Gou_2021}. Moreover, knowledge distillation often assumes substantial differences between student and teacher model architectures~\citep{Gou_2021}.
Low-rank factorization decomposes weight matrices into lower-rank matrices to reduce parameter size through such as singular value decomposition~\citep{ren2023lowrankpruneandfactorizelanguagemodel,horvath2024maestrouncoveringlowrankstructures} or tensor decomposition ~\citep{lebedev2015speedingupconvolutionalneuralnetworks, kim2016compressiondeepconvolutionalneural}. Approaches such as mixture of experts~\citep{jacobs1991adaptive,shazeer2017outrageously}, subspace-configurable networks~\citep{wang2024subspaceconfigurablenetworks,papst2024scn} and resource-efficient deep subnetworks~\citep{corti2024redsresourceefficientdeepsubnetworks,corti2024hads}, explore dynamic model reconfiguration to minimize the number of active weights during inference.


\fakeparagraph{Structured pruning} 
Structured pruning is of particular interest because it removes entire structures (such as neurons, channels, or layers)~\citep{entezari2020classdependentcompressiondeepneural,li2016pruning,luo2017iccv,hu2016networktrimming, wen2016learning} rather than individual parameters, reducing model complexity while maintaining or even improving performance. This method is especially valuable for enhancing efficiency with easily implemented acceleration in resource-constrained environments~\citep{wang2020sparse,liu2024lightweightdeeplearningresourceconstrained}. However, structured pruning typically requires additional retraining or fine-tuning~\citep{he2017iccv,liu2024lightweightdeeplearningresourceconstrained,luo2017thinet}. 
Recent work by \citet{theus2024metapruning} combines model pruning and fusion using Optimal Transport theory, demonstrating that a significant portion of pruning accuracy can be recovered without access to training data. However, the impact of pruning on the model's data statistics and how to recover them is not addressed.


\subsection{Model merging}
Model merging combines multiple models to generate a single, unified model which leverages the strengths and diversity of each individual model. It particularly benefits ensemble learning and distributed training scenarios, where models are trained independently on different subsets of data or across different devices. Merging can be achieved by averaging the parameters of model trained independently. Recently, multiple methods have been developed to enhance model performance and robustness. MTZ~\citep{he2018multi} and ZipIt!~\citep{stoica2024zipitmergingmodelsdifferent} compress multiple models pre-trained for different tasks by merging them through neuron sharing. Model soup~\citep{wortsman2022model} averages the weights of multiple fine-tuned models from same initialization to improve accuracy and robustness without increasing inference time. Taking permutation invariance of neural networks into account, a finding~\citep{entezari2022role} shows the interpolation between models trained with SGD has no barrier. Git Re-Basin~\citep{ainsworth2023git} utilizes activation matching and weight matching to achieve permutated alignment between models trained from different initialization. REPAIR~\citep{jordan2022repair} mitigate variance collapse problem while aligning neurons by rescaling the preactivations of fused models. PAPA leverages a population of diverse models trained on different data variations and slowly pushes the weights of the networks towards the population average~\citep{papa}. A recent work~\citep{yamada2023revisitingpermutationsymmetrymerging} shows that for model merging on different datasets, using original or condensed datasets during the model merging process can significantly improve accuracy. However, those methods do not consider model efficiency and internal parameter redundancy. Another recent work \citep{theus2024metapruning} achieves intra-layer model fusion by integrating optimal transport~\citep{monge1781memoire,kantorovich2006translocation,singh2020model} to fuse computational structures in the model without fine-tuning. We note that this approach is orthogonal to the problem solved in this paper, as we do not consider intra-layer dependencies.

\fakeparagraph{Merging multiple computational units}
Merging computational units has been extensively explored in ensemble methods. \citet{wortsman2022model} demonstrate that combining multiple models fine-tuned from the same pretrained initialization enhances both accuracy and robustness. \citet{ainsworth2023git} extend this approach to models trained on the same data with different initializations, albeit with some accuracy loss. \citet{jordan2022repair} improve upon Git Re-Basin by adjusting batch normalization layers where applicable. IFM~\cite{chen2023going} and ZipIt!~\cite{stoica2024zipitmergingmodelsdifferent} focus on merging multiple computational units within a single model, pioneering this approach.



\end{document}
