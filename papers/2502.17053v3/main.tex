%%%%%%%%%%%%%%%%%%%%%%% file template.tex %%%%%%%%%%%%%%%%%%%%%%%%%
%
% This is a general template file for the LaTeX package SVJour3
% for Springer journals.          Springer Heidelberg 2010/09/16
%
% Copy it to a new file with a new name and use it as the basis
% for your article. Delete % signs as needed.
%
% This template includes a few options for different layouts and
% content for various journals. Please consult a previous issue of
% your journal as needed.
%
\RequirePackage{fix-cm}
%
%\documentclass{svjour3}                     % onecolumn (standard format)
%\documentclass[smallcondensed]{svjour3}     % onecolumn (ditto)
%\documentclass[smallextended]{svjour3}       % onecolumn (second format)
\documentclass[twocolumn]{svjour3}          % twocolumn

%
\smartqed  % flush right qed marks, e.g. at end of proof
%
\usepackage{graphicx}
\usepackage{amsmath,amssymb} % define this before the line numbering.
\usepackage[utf8x]{inputenc}
\usepackage{pgfplots}
\pgfplotsset{compat=newest}
\usepackage{tikz}
\usetikzlibrary{backgrounds}
\usepackage{color}
\usepackage{xcolor}
\usepackage{booktabs,siunitx} % better tables
\usepackage{makecell}
% \usepackage{enumitem}
\usepackage[caption=false]{subfig}
\usepackage[noadjust]{cite}
\usepackage[colorlinks=true,allcolors=green]{hyperref}
\usepackage{rotating}

%copied from spoken language paper
\usepackage{times}
\usepackage{pifont}% http://ctan.org/pkg/pifont
\usepackage{arydshln}
\usepackage{adjustbox}
\usepackage{array,multirow}
\newcommand{\cmark}{\ding{51}}%
\newcommand{\xmark}{\ding{55}}%

\usepackage{epsfig}
% \usepackage{subfigure}
\usepackage{paralist}
\usepackage[english]{babel}

\usepackage{authblk}
\usepackage{diagbox}

%% Bibliography stuff.
\usepackage{natbib}
\setcitestyle{authoryear,open={(},close={)}}
% New commands.

\renewcommand{\labelitemi}{\textbullet}

\begin{document}

\title{PointSea: Point Cloud Completion via Self-structure Augmentation}

\titlerunning{PointSea: Point Cloud Completion via Self-structure Augmentation}

\author{Zhe Zhu   \and
        Honghua Chen \and
        Xing He \and
        Mingqiang Wei
}
%\authorrunning{Short form of author list} % if too long for running head

\institute{              
                Zhe Zhu \at zhuzhe0619@nuaa.edu.cn \\ \\
                Honghua Chen$^{\dag}$ \at chenhonghuacn@gmail.com \\ \\
                Xing He \at hexing@nuaa.edu.cn \\ \\
                Mingqiang Wei$^{\dag}$ \at mqwei@nuaa.edu.cn \\ \\
              Nanjing University of Aeronautics and Astronautics \\
              Nanjing, China \\ \\
              $^{\dag}$ Corresponding authors
%              
%             \emph{Present address:} of F. Author  %  if needed
}
\date{Received: date / Accepted: date}
% The correct dates will be entered by the editor

\maketitle


\begin{abstract}
\sloppy{  
Point cloud completion is a fundamental yet not well-solved problem in 3D vision.
Current approaches often rely on 3D coordinate information and/or additional data (e.g., images and scanning viewpoints) to fill in missing parts.  
Unlike these methods, we explore self-structure augmentation and propose \textbf{PointSea} for global-to-local point cloud completion. 
In the global stage, consider how we inspect a defective region of a physical object, we may observe it from various perspectives for a better understanding.
Inspired by this, PointSea augments data representation by leveraging self-projected depth images from multiple views. To reconstruct a compact global shape from the cross-modal input, we incorporate a feature fusion module to fuse features at both intra-view and inter-view levels.
%
In the local stage, to reveal highly detailed structures, we introduce a point generator called the self-structure dual-generator. This generator integrates both learned shape priors and geometric self-similarities for shape refinement. Unlike existing efforts that apply a unified strategy for all points, our dual-path design adapts refinement strategies conditioned on the structural type of each point, addressing the specific incompleteness of each point. 
Comprehensive experiments on widely-used benchmarks demonstrate that PointSea effectively understands global shapes and generates local details from incomplete input, showing clear improvements over existing methods. 
Our code is available at \emph{\textcolor{magenta}{https://github.com/czvvd/SVDFormer\_PointSea}}.
}
\keywords{PointSea \and Point cloud completion \and Self-structure augmentation \and Cross-modal fusion}
\end{abstract}

%% main text
\section{Introduction}


\begin{figure}[t]
\centering
\includegraphics[width=0.6\columnwidth]{figures/evaluation_desiderata_V5.pdf}
\vspace{-0.5cm}
\caption{\systemName is a platform for conducting realistic evaluations of code LLMs, collecting human preferences of coding models with real users, real tasks, and in realistic environments, aimed at addressing the limitations of existing evaluations.
}
\label{fig:motivation}
\end{figure}

\begin{figure*}[t]
\centering
\includegraphics[width=\textwidth]{figures/system_design_v2.png}
\caption{We introduce \systemName, a VSCode extension to collect human preferences of code directly in a developer's IDE. \systemName enables developers to use code completions from various models. The system comprises a) the interface in the user's IDE which presents paired completions to users (left), b) a sampling strategy that picks model pairs to reduce latency (right, top), and c) a prompting scheme that allows diverse LLMs to perform code completions with high fidelity.
Users can select between the top completion (green box) using \texttt{tab} or the bottom completion (blue box) using \texttt{shift+tab}.}
\label{fig:overview}
\end{figure*}

As model capabilities improve, large language models (LLMs) are increasingly integrated into user environments and workflows.
For example, software developers code with AI in integrated developer environments (IDEs)~\citep{peng2023impact}, doctors rely on notes generated through ambient listening~\citep{oberst2024science}, and lawyers consider case evidence identified by electronic discovery systems~\citep{yang2024beyond}.
Increasing deployment of models in productivity tools demands evaluation that more closely reflects real-world circumstances~\citep{hutchinson2022evaluation, saxon2024benchmarks, kapoor2024ai}.
While newer benchmarks and live platforms incorporate human feedback to capture real-world usage, they almost exclusively focus on evaluating LLMs in chat conversations~\citep{zheng2023judging,dubois2023alpacafarm,chiang2024chatbot, kirk2024the}.
Model evaluation must move beyond chat-based interactions and into specialized user environments.



 

In this work, we focus on evaluating LLM-based coding assistants. 
Despite the popularity of these tools---millions of developers use Github Copilot~\citep{Copilot}---existing
evaluations of the coding capabilities of new models exhibit multiple limitations (Figure~\ref{fig:motivation}, bottom).
Traditional ML benchmarks evaluate LLM capabilities by measuring how well a model can complete static, interview-style coding tasks~\citep{chen2021evaluating,austin2021program,jain2024livecodebench, white2024livebench} and lack \emph{real users}. 
User studies recruit real users to evaluate the effectiveness of LLMs as coding assistants, but are often limited to simple programming tasks as opposed to \emph{real tasks}~\citep{vaithilingam2022expectation,ross2023programmer, mozannar2024realhumaneval}.
Recent efforts to collect human feedback such as Chatbot Arena~\citep{chiang2024chatbot} are still removed from a \emph{realistic environment}, resulting in users and data that deviate from typical software development processes.
We introduce \systemName to address these limitations (Figure~\ref{fig:motivation}, top), and we describe our three main contributions below.


\textbf{We deploy \systemName in-the-wild to collect human preferences on code.} 
\systemName is a Visual Studio Code extension, collecting preferences directly in a developer's IDE within their actual workflow (Figure~\ref{fig:overview}).
\systemName provides developers with code completions, akin to the type of support provided by Github Copilot~\citep{Copilot}. 
Over the past 3 months, \systemName has served over~\completions suggestions from 10 state-of-the-art LLMs, 
gathering \sampleCount~votes from \userCount~users.
To collect user preferences,
\systemName presents a novel interface that shows users paired code completions from two different LLMs, which are determined based on a sampling strategy that aims to 
mitigate latency while preserving coverage across model comparisons.
Additionally, we devise a prompting scheme that allows a diverse set of models to perform code completions with high fidelity.
See Section~\ref{sec:system} and Section~\ref{sec:deployment} for details about system design and deployment respectively.



\textbf{We construct a leaderboard of user preferences and find notable differences from existing static benchmarks and human preference leaderboards.}
In general, we observe that smaller models seem to overperform in static benchmarks compared to our leaderboard, while performance among larger models is mixed (Section~\ref{sec:leaderboard_calculation}).
We attribute these differences to the fact that \systemName is exposed to users and tasks that differ drastically from code evaluations in the past. 
Our data spans 103 programming languages and 24 natural languages as well as a variety of real-world applications and code structures, while static benchmarks tend to focus on a specific programming and natural language and task (e.g. coding competition problems).
Additionally, while all of \systemName interactions contain code contexts and the majority involve infilling tasks, a much smaller fraction of Chatbot Arena's coding tasks contain code context, with infilling tasks appearing even more rarely. 
We analyze our data in depth in Section~\ref{subsec:comparison}.



\textbf{We derive new insights into user preferences of code by analyzing \systemName's diverse and distinct data distribution.}
We compare user preferences across different stratifications of input data (e.g., common versus rare languages) and observe which affect observed preferences most (Section~\ref{sec:analysis}).
For example, while user preferences stay relatively consistent across various programming languages, they differ drastically between different task categories (e.g. frontend/backend versus algorithm design).
We also observe variations in user preference due to different features related to code structure 
(e.g., context length and completion patterns).
We open-source \systemName and release a curated subset of code contexts.
Altogether, our results highlight the necessity of model evaluation in realistic and domain-specific settings.





\label{secIntroduction}

\section{RELATED WORK}
\label{sec:relatedwork}
In this section, we describe the previous works related to our proposal, which are divided into two parts. In Section~\ref{sec:relatedwork_exoplanet}, we present a review of approaches based on machine learning techniques for the detection of planetary transit signals. Section~\ref{sec:relatedwork_attention} provides an account of the approaches based on attention mechanisms applied in Astronomy.\par

\subsection{Exoplanet detection}
\label{sec:relatedwork_exoplanet}
Machine learning methods have achieved great performance for the automatic selection of exoplanet transit signals. One of the earliest applications of machine learning is a model named Autovetter \citep{MCcauliff}, which is a random forest (RF) model based on characteristics derived from Kepler pipeline statistics to classify exoplanet and false positive signals. Then, other studies emerged that also used supervised learning. \cite{mislis2016sidra} also used a RF, but unlike the work by \citet{MCcauliff}, they used simulated light curves and a box least square \citep[BLS;][]{kovacs2002box}-based periodogram to search for transiting exoplanets. \citet{thompson2015machine} proposed a k-nearest neighbors model for Kepler data to determine if a given signal has similarity to known transits. Unsupervised learning techniques were also applied, such as self-organizing maps (SOM), proposed \citet{armstrong2016transit}; which implements an architecture to segment similar light curves. In the same way, \citet{armstrong2018automatic} developed a combination of supervised and unsupervised learning, including RF and SOM models. In general, these approaches require a previous phase of feature engineering for each light curve. \par

%DL is a modern data-driven technology that automatically extracts characteristics, and that has been successful in classification problems from a variety of application domains. The architecture relies on several layers of NNs of simple interconnected units and uses layers to build increasingly complex and useful features by means of linear and non-linear transformation. This family of models is capable of generating increasingly high-level representations \citep{lecun2015deep}.

The application of DL for exoplanetary signal detection has evolved rapidly in recent years and has become very popular in planetary science.  \citet{pearson2018} and \citet{zucker2018shallow} developed CNN-based algorithms that learn from synthetic data to search for exoplanets. Perhaps one of the most successful applications of the DL models in transit detection was that of \citet{Shallue_2018}; who, in collaboration with Google, proposed a CNN named AstroNet that recognizes exoplanet signals in real data from Kepler. AstroNet uses the training set of labelled TCEs from the Autovetter planet candidate catalog of Q1–Q17 data release 24 (DR24) of the Kepler mission \citep{catanzarite2015autovetter}. AstroNet analyses the data in two views: a ``global view'', and ``local view'' \citep{Shallue_2018}. \par


% The global view shows the characteristics of the light curve over an orbital period, and a local view shows the moment at occurring the transit in detail

%different = space-based

Based on AstroNet, researchers have modified the original AstroNet model to rank candidates from different surveys, specifically for Kepler and TESS missions. \citet{ansdell2018scientific} developed a CNN trained on Kepler data, and included for the first time the information on the centroids, showing that the model improves performance considerably. Then, \citet{osborn2020rapid} and \citet{yu2019identifying} also included the centroids information, but in addition, \citet{osborn2020rapid} included information of the stellar and transit parameters. Finally, \citet{rao2021nigraha} proposed a pipeline that includes a new ``half-phase'' view of the transit signal. This half-phase view represents a transit view with a different time and phase. The purpose of this view is to recover any possible secondary eclipse (the object hiding behind the disk of the primary star).


%last pipeline applies a procedure after the prediction of the model to obtain new candidates, this process is carried out through a series of steps that include the evaluation with Discovery and Validation of Exoplanets (DAVE) \citet{kostov2019discovery} that was adapted for the TESS telescope.\par
%



\subsection{Attention mechanisms in astronomy}
\label{sec:relatedwork_attention}
Despite the remarkable success of attention mechanisms in sequential data, few papers have exploited their advantages in astronomy. In particular, there are no models based on attention mechanisms for detecting planets. Below we present a summary of the main applications of this modeling approach to astronomy, based on two points of view; performance and interpretability of the model.\par
%Attention mechanisms have not yet been explored in all sub-areas of astronomy. However, recent works show a successful application of the mechanism.
%performance

The application of attention mechanisms has shown improvements in the performance of some regression and classification tasks compared to previous approaches. One of the first implementations of the attention mechanism was to find gravitational lenses proposed by \citet{thuruthipilly2021finding}. They designed 21 self-attention-based encoder models, where each model was trained separately with 18,000 simulated images, demonstrating that the model based on the Transformer has a better performance and uses fewer trainable parameters compared to CNN. A novel application was proposed by \citet{lin2021galaxy} for the morphological classification of galaxies, who used an architecture derived from the Transformer, named Vision Transformer (VIT) \citep{dosovitskiy2020image}. \citet{lin2021galaxy} demonstrated competitive results compared to CNNs. Another application with successful results was proposed by \citet{zerveas2021transformer}; which first proposed a transformer-based framework for learning unsupervised representations of multivariate time series. Their methodology takes advantage of unlabeled data to train an encoder and extract dense vector representations of time series. Subsequently, they evaluate the model for regression and classification tasks, demonstrating better performance than other state-of-the-art supervised methods, even with data sets with limited samples.

%interpretation
Regarding the interpretability of the model, a recent contribution that analyses the attention maps was presented by \citet{bowles20212}, which explored the use of group-equivariant self-attention for radio astronomy classification. Compared to other approaches, this model analysed the attention maps of the predictions and showed that the mechanism extracts the brightest spots and jets of the radio source more clearly. This indicates that attention maps for prediction interpretation could help experts see patterns that the human eye often misses. \par

In the field of variable stars, \citet{allam2021paying} employed the mechanism for classifying multivariate time series in variable stars. And additionally, \citet{allam2021paying} showed that the activation weights are accommodated according to the variation in brightness of the star, achieving a more interpretable model. And finally, related to the TESS telescope, \citet{morvan2022don} proposed a model that removes the noise from the light curves through the distribution of attention weights. \citet{morvan2022don} showed that the use of the attention mechanism is excellent for removing noise and outliers in time series datasets compared with other approaches. In addition, the use of attention maps allowed them to show the representations learned from the model. \par

Recent attention mechanism approaches in astronomy demonstrate comparable results with earlier approaches, such as CNNs. At the same time, they offer interpretability of their results, which allows a post-prediction analysis. \par


\label{secRelatedWork}
%
\section{Method}\label{sec:method}
\begin{figure}
    \centering
    \includegraphics[width=0.85\textwidth]{imgs/heatmap_acc.pdf}
    \caption{\textbf{Visualization of the proposed periodic Bayesian flow with mean parameter $\mu$ and accumulated accuracy parameter $c$ which corresponds to the entropy/uncertainty}. For $x = 0.3, \beta(1) = 1000$ and $\alpha_i$ defined in \cref{appd:bfn_cir}, this figure plots three colored stochastic parameter trajectories for receiver mean parameter $m$ and accumulated accuracy parameter $c$, superimposed on a log-scale heatmap of the Bayesian flow distribution $p_F(m|x,\senderacc)$ and $p_F(c|x,\senderacc)$. Note the \emph{non-monotonicity} and \emph{non-additive} property of $c$ which could inform the network the entropy of the mean parameter $m$ as a condition and the \emph{periodicity} of $m$. %\jj{Shrink the figures to save space}\hanlin{Do we need to make this figure one-column?}
    }
    \label{fig:vmbf_vis}
    \vskip -0.1in
\end{figure}
% \begin{wrapfigure}{r}{0.5\textwidth}
%     \centering
%     \includegraphics[width=0.49\textwidth]{imgs/heatmap_acc.pdf}
%     \caption{\textbf{Visualization of hyper-torus Bayesian flow based on von Mises Distribution}. For $x = 0.3, \beta(1) = 1000$ and $\alpha_i$ defined in \cref{appd:bfn_cir}, this figure plots three colored stochastic parameter trajectories for receiver mean parameter $m$ and accumulated accuracy parameter $c$, superimposed on a log-scale heatmap of the Bayesian flow distribution $p_F(m|x,\senderacc)$ and $p_F(c|x,\senderacc)$. Note the \emph{non-monotonicity} and \emph{non-additive} property of $c$. \jj{Shrink the figures to save space}}
%     \label{fig:vmbf_vis}
%     \vspace{-30pt}
% \end{wrapfigure}


In this section, we explain the detailed design of CrysBFN tackling theoretical and practical challenges. First, we describe how to derive our new formulation of Bayesian Flow Networks over hyper-torus $\mathbb{T}^{D}$ from scratch. Next, we illustrate the two key differences between \modelname and the original form of BFN: $1)$ a meticulously designed novel base distribution with different Bayesian update rules; and $2)$ different properties over the accuracy scheduling resulted from the periodicity and the new Bayesian update rules. Then, we present in detail the overall framework of \modelname over each manifold of the crystal space (\textit{i.e.} fractional coordinates, lattice vectors, atom types) respecting \textit{periodic E(3) invariance}. 

% In this section, we first demonstrate how to build Bayesian flow on hyper-torus $\mathbb{T}^{D}$ by overcoming theoretical and practical problems to provide a low-noise parameter-space approach to fractional atom coordinate generation. Next, we present how \modelname models each manifold of crystal space respecting \textit{periodic E(3) invariance}. 

\subsection{Periodic Bayesian Flow on Hyper-torus \texorpdfstring{$\mathbb{T}^{D}$}{}} 
For generative modeling of fractional coordinates in crystal, we first construct a periodic Bayesian flow on \texorpdfstring{$\mathbb{T}^{D}$}{} by designing every component of the totally new Bayesian update process which we demonstrate to be distinct from the original Bayesian flow (please see \cref{fig:non_add}). 
 %:) 
 
 The fractional atom coordinate system \citep{jiao2023crystal} inherently distributes over a hyper-torus support $\mathbb{T}^{3\times N}$. Hence, the normal distribution support on $\R$ used in the original \citep{bfn} is not suitable for this scenario. 
% The key problem of generative modeling for crystal is the periodicity of Cartesian atom coordinates $\vX$ requiring:
% \begin{equation}\label{eq:periodcity}
% p(\vA,\vL,\vX)=p(\vA,\vL,\vX+\vec{LK}),\text{where}~\vec{K}=\vec{k}\vec{1}_{1\times N},\forall\vec{k}\in\mathbb{Z}^{3\times1}
% \end{equation}
% However, there does not exist such a distribution supporting on $\R$ to model such property because the integration of such distribution over $\R$ will not be finite and equal to 1. Therefore, the normal distribution used in \citet{bfn} can not meet this condition.

To tackle this problem, the circular distribution~\citep{mardia2009directional} over the finite interval $[-\pi,\pi)$ is a natural choice as the base distribution for deriving the BFN on $\mathbb{T}^D$. 
% one natural choice is to 
% we would like to consider the circular distribution over the finite interval as the base 
% we find that circular distributions \citep{mardia2009directional} defined on a finite interval with lengths of $2\pi$ can be used as the instantiation of input distribution for the BFN on $\mathbb{T}^D$.
Specifically, circular distributions enjoy desirable periodic properties: $1)$ the integration over any interval length of $2\pi$ equals 1; $2)$ the probability distribution function is periodic with period $2\pi$.  Sharing the same intrinsic with fractional coordinates, such periodic property of circular distribution makes it suitable for the instantiation of BFN's input distribution, in parameterizing the belief towards ground truth $\x$ on $\mathbb{T}^D$. 
% \yuxuan{this is very complicated from my perspective.} \hanlin{But this property is exactly beautiful and perfectly fit into the BFN.}

\textbf{von Mises Distribution and its Bayesian Update} We choose von Mises distribution \citep{mardia2009directional} from various circular distributions as the form of input distribution, based on the appealing conjugacy property required in the derivation of the BFN framework.
% to leverage the Bayesian conjugacy property of von Mises distribution which is required by the BFN framework. 
That is, the posterior of a von Mises distribution parameterized likelihood is still in the family of von Mises distributions. The probability density function of von Mises distribution with mean direction parameter $m$ and concentration parameter $c$ (describing the entropy/uncertainty of $m$) is defined as: 
\begin{equation}
f(x|m,c)=vM(x|m,c)=\frac{\exp(c\cos(x-m))}{2\pi I_0(c)}
\end{equation}
where $I_0(c)$ is zeroth order modified Bessel function of the first kind as the normalizing constant. Given the last univariate belief parameterized by von Mises distribution with parameter $\theta_{i-1}=\{m_{i-1},\ c_{i-1}\}$ and the sample $y$ from sender distribution with unknown data sample $x$ and known accuracy $\alpha$ describing the entropy/uncertainty of $y$,  Bayesian update for the receiver is deducted as:
\begin{equation}
 h(\{m_{i-1},c_{i-1}\},y,\alpha)=\{m_i,c_i \}, \text{where}
\end{equation}
\begin{equation}\label{eq:h_m}
m_i=\text{atan2}(\alpha\sin y+c_{i-1}\sin m_{i-1}, {\alpha\cos y+c_{i-1}\cos m_{i-1}})
\end{equation}
\begin{equation}\label{eq:h_c}
c_i =\sqrt{\alpha^2+c_{i-1}^2+2\alpha c_{i-1}\cos(y-m_{i-1})}
\end{equation}
The proof of the above equations can be found in \cref{apdx:bayesian_update_function}. The atan2 function refers to  2-argument arctangent. Independently conducting  Bayesian update for each dimension, we can obtain the Bayesian update distribution by marginalizing $\y$:
\begin{equation}
p_U(\vtheta'|\vtheta,\bold{x};\alpha)=\mathbb{E}_{p_S(\bold{y}|\bold{x};\alpha)}\delta(\vtheta'-h(\vtheta,\bold{y},\alpha))=\mathbb{E}_{vM(\bold{y}|\bold{x},\alpha)}\delta(\vtheta'-h(\vtheta,\bold{y},\alpha))
\end{equation} 
\begin{figure}
    \centering
    \vskip -0.15in
    \includegraphics[width=0.95\linewidth]{imgs/non_add.pdf}
    \caption{An intuitive illustration of non-additive accuracy Bayesian update on the torus. The lengths of arrows represent the uncertainty/entropy of the belief (\emph{e.g.}~$1/\sigma^2$ for Gaussian and $c$ for von Mises). The directions of the arrows represent the believed location (\emph{e.g.}~ $\mu$ for Gaussian and $m$ for von Mises).}
    \label{fig:non_add}
    \vskip -0.15in
\end{figure}
\textbf{Non-additive Accuracy} 
The additive accuracy is a nice property held with the Gaussian-formed sender distribution of the original BFN expressed as:
\begin{align}
\label{eq:standard_id}
    \update(\parsn{}'' \mid \parsn{}, \x; \alpha_a+\alpha_b) = \E_{\update(\parsn{}' \mid \parsn{}, \x; \alpha_a)} \update(\parsn{}'' \mid \parsn{}', \x; \alpha_b)
\end{align}
Such property is mainly derived based on the standard identity of Gaussian variable:
\begin{equation}
X \sim \mathcal{N}\left(\mu_X, \sigma_X^2\right), Y \sim \mathcal{N}\left(\mu_Y, \sigma_Y^2\right) \Longrightarrow X+Y \sim \mathcal{N}\left(\mu_X+\mu_Y, \sigma_X^2+\sigma_Y^2\right)
\end{equation}
The additive accuracy property makes it feasible to derive the Bayesian flow distribution $
p_F(\boldsymbol{\theta} \mid \mathbf{x} ; i)=p_U\left(\boldsymbol{\theta} \mid \boldsymbol{\theta}_0, \mathbf{x}, \sum_{k=1}^{i} \alpha_i \right)
$ for the simulation-free training of \cref{eq:loss_n}.
It should be noted that the standard identity in \cref{eq:standard_id} does not hold in the von Mises distribution. Hence there exists an important difference between the original Bayesian flow defined on Euclidean space and the Bayesian flow of circular data on $\mathbb{T}^D$ based on von Mises distribution. With prior $\btheta = \{\bold{0},\bold{0}\}$, we could formally represent the non-additive accuracy issue as:
% The additive accuracy property implies the fact that the "confidence" for the data sample after observing a series of the noisy samples with accuracy ${\alpha_1, \cdots, \alpha_i}$ could be  as the accuracy sum  which could be  
% Here we 
% Here we emphasize the specific property of BFN based on von Mises distribution.
% Note that 
% \begin{equation}
% \update(\parsn'' \mid \parsn, \x; \alpha_a+\alpha_b) \ne \E_{\update(\parsn' \mid \parsn, \x; \alpha_a)} \update(\parsn'' \mid \parsn', \x; \alpha_b)
% \end{equation}
% \oyyw{please check whether the below equation is better}
% \yuxuan{I fill somehow confusing on what is the update distribution with $\alpha$. }
% \begin{equation}
% \update(\parsn{}'' \mid \parsn{}, \x; \alpha_a+\alpha_b) \ne \E_{\update(\parsn{}' \mid \parsn{}, \x; \alpha_a)} \update(\parsn{}'' \mid \parsn{}', \x; \alpha_b)
% \end{equation}
% We give an intuitive visualization of such difference in \cref{fig:non_add}. The untenability of this property can materialize by considering the following case: with prior $\btheta = \{\bold{0},\bold{0}\}$, check the two-step Bayesian update distribution with $\alpha_a,\alpha_b$ and one-step Bayesian update with $\alpha=\alpha_a+\alpha_b$:
\begin{align}
\label{eq:nonadd}
     &\update(c'' \mid \parsn, \x; \alpha_a+\alpha_b)  = \delta(c-\alpha_a-\alpha_b)
     \ne  \mathbb{E}_{p_U(\parsn' \mid \parsn, \x; \alpha_a)}\update(c'' \mid \parsn', \x; \alpha_b) \nonumber \\&= \mathbb{E}_{vM(\bold{y}_b|\bold{x},\alpha_a)}\mathbb{E}_{vM(\bold{y}_a|\bold{x},\alpha_b)}\delta(c-||[\alpha_a \cos\y_a+\alpha_b\cos \y_b,\alpha_a \sin\y_a+\alpha_b\sin \y_b]^T||_2)
\end{align}
A more intuitive visualization could be found in \cref{fig:non_add}. This fundamental difference between periodic Bayesian flow and that of \citet{bfn} presents both theoretical and practical challenges, which we will explain and address in the following contents.

% This makes constructing Bayesian flow based on von Mises distribution intrinsically different from previous Bayesian flows (\citet{bfn}).

% Thus, we must reformulate the framework of Bayesian flow networks  accordingly. % and do necessary reformulations of BFN. 

% \yuxuan{overall I feel this part is complicated by using the language of update distribution. I would like to suggest simply use bayesian update, to provide intuitive explantion.}\hanlin{See the illustration in \cref{fig:non_add}}

% That introduces a cascade of problems, and we investigate the following issues: $(1)$ Accuracies between sender and receiver are not synchronized and need to be differentiated. $(2)$ There is no tractable Bayesian flow distribution for a one-step sample conditioned on a given time step $i$, and naively simulating the Bayesian flow results in computational overhead. $(3)$ It is difficult to control the entropy of the Bayesian flow. $(4)$ Accuracy is no longer a function of $t$ and becomes a distribution conditioned on $t$, which can be different across dimensions.
%\jj{Edited till here}

\textbf{Entropy Conditioning} As a common practice in generative models~\citep{ddpm,flowmatching,bfn}, timestep $t$ is widely used to distinguish among generation states by feeding the timestep information into the networks. However, this paper shows that for periodic Bayesian flow, the accumulated accuracy $\vc_i$ is more effective than time-based conditioning by informing the network about the entropy and certainty of the states $\parsnt{i}$. This stems from the intrinsic non-additive accuracy which makes the receiver's accumulated accuracy $c$ not bijective function of $t$, but a distribution conditioned on accumulated accuracies $\vc_i$ instead. Therefore, the entropy parameter $\vc$ is taken logarithm and fed into the network to describe the entropy of the input corrupted structure. We verify this consideration in \cref{sec:exp_ablation}. 
% \yuxuan{implement variant. traditionally, the timestep is widely used to distinguish the different states by putting the timestep embedding into the networks. citation of FM, diffusion, BFN. However, we find that conditioned on time in periodic flow could not provide extra benefits. To further boost the performance, we introduce a simple yet effective modification term entropy conditional. This is based on that the accumulated accuracy which represents the current uncertainty or entropy could be a better indicator to distinguish different states. + Describe how you do this. }



\textbf{Reformulations of BFN}. Recall the original update function with Gaussian sender distribution, after receiving noisy samples $\y_1,\y_2,\dots,\y_i$ with accuracies $\senderacc$, the accumulated accuracies of the receiver side could be analytically obtained by the additive property and it is consistent with the sender side.
% Since observing sample $\y$ with $\alpha_i$ can not result in exact accuracy increment $\alpha_i$ for receiver, the accuracies between sender and receiver are not synchronized which need to be differentiated. 
However, as previously mentioned, this does not apply to periodic Bayesian flow, and some of the notations in original BFN~\citep{bfn} need to be adjusted accordingly. We maintain the notations of sender side's one-step accuracy $\alpha$ and added accuracy $\beta$, and alter the notation of receiver's accuracy parameter as $c$, which is needed to be simulated by cascade of Bayesian updates. We emphasize that the receiver's accumulated accuracy $c$ is no longer a function of $t$ (differently from the Gaussian case), and it becomes a distribution conditioned on received accuracies $\senderacc$ from the sender. Therefore, we represent the Bayesian flow distribution of von Mises distribution as $p_F(\btheta|\x;\alpha_1,\alpha_2,\dots,\alpha_i)$. And the original simulation-free training with Bayesian flow distribution is no longer applicable in this scenario.
% Different from previous BFNs where the accumulated accuracy $\rho$ is not explicitly modeled, the accumulated accuracy parameter $c$ (visualized in \cref{fig:vmbf_vis}) needs to be explicitly modeled by feeding it to the network to avoid information loss.
% the randomaccuracy parameter $c$ (visualized in \cref{fig:vmbf_vis}) implies that there exists information in $c$ from the sender just like $m$, meaning that $c$ also should be fed into the network to avoid information loss. 
% We ablate this consideration in  \cref{sec:exp_ablation}. 

\textbf{Fast Sampling from Equivalent Bayesian Flow Distribution} Based on the above reformulations, the Bayesian flow distribution of von Mises distribution is reframed as: 
\begin{equation}\label{eq:flow_frac}
p_F(\btheta_i|\x;\alpha_1,\alpha_2,\dots,\alpha_i)=\E_{\update(\parsnt{1} \mid \parsnt{0}, \x ; \alphat{1})}\dots\E_{\update(\parsn_{i-1} \mid \parsnt{i-2}, \x; \alphat{i-1})} \update(\parsnt{i} | \parsnt{i-1},\x;\alphat{i} )
\end{equation}
Naively sampling from \cref{eq:flow_frac} requires slow auto-regressive iterated simulation, making training unaffordable. Noticing the mathematical properties of \cref{eq:h_m,eq:h_c}, we  transform \cref{eq:flow_frac} to the equivalent form:
\begin{equation}\label{eq:cirflow_equiv}
p_F(\vec{m}_i|\x;\alpha_1,\alpha_2,\dots,\alpha_i)=\E_{vM(\y_1|\x,\alpha_1)\dots vM(\y_i|\x,\alpha_i)} \delta(\vec{m}_i-\text{atan2}(\sum_{j=1}^i \alpha_j \cos \y_j,\sum_{j=1}^i \alpha_j \sin \y_j))
\end{equation}
\begin{equation}\label{eq:cirflow_equiv2}
p_F(\vec{c}_i|\x;\alpha_1,\alpha_2,\dots,\alpha_i)=\E_{vM(\y_1|\x,\alpha_1)\dots vM(\y_i|\x,\alpha_i)}  \delta(\vec{c}_i-||[\sum_{j=1}^i \alpha_j \cos \y_j,\sum_{j=1}^i \alpha_j \sin \y_j]^T||_2)
\end{equation}
which bypasses the computation of intermediate variables and allows pure tensor operations, with negligible computational overhead.
\begin{restatable}{proposition}{cirflowequiv}
The probability density function of Bayesian flow distribution defined by \cref{eq:cirflow_equiv,eq:cirflow_equiv2} is equivalent to the original definition in \cref{eq:flow_frac}. 
\end{restatable}
\textbf{Numerical Determination of Linear Entropy Sender Accuracy Schedule} ~Original BFN designs the accuracy schedule $\beta(t)$ to make the entropy of input distribution linearly decrease. As for crystal generation task, to ensure information coherence between modalities, we choose a sender accuracy schedule $\senderacc$ that makes the receiver's belief entropy $H(t_i)=H(p_I(\cdot|\vtheta_i))=H(p_I(\cdot|\vc_i))$ linearly decrease \emph{w.r.t.} time $t_i$, given the initial and final accuracy parameter $c(0)$ and $c(1)$. Due to the intractability of \cref{eq:vm_entropy}, we first use numerical binary search in $[0,c(1)]$ to determine the receiver's $c(t_i)$ for $i=1,\dots, n$ by solving the equation $H(c(t_i))=(1-t_i)H(c(0))+tH(c(1))$. Next, with $c(t_i)$, we conduct numerical binary search for each $\alpha_i$ in $[0,c(1)]$ by solving the equations $\E_{y\sim vM(x,\alpha_i)}[\sqrt{\alpha_i^2+c_{i-1}^2+2\alpha_i c_{i-1}\cos(y-m_{i-1})}]=c(t_i)$ from $i=1$ to $i=n$ for arbitrarily selected $x\in[-\pi,\pi)$.

After tackling all those issues, we have now arrived at a new BFN architecture for effectively modeling crystals. Such BFN can also be adapted to other type of data located in hyper-torus $\mathbb{T}^{D}$.

\subsection{Equivariant Bayesian Flow for Crystal}
With the above Bayesian flow designed for generative modeling of fractional coordinate $\vF$, we are able to build equivariant Bayesian flow for each modality of crystal. In this section, we first give an overview of the general training and sampling algorithm of \modelname (visualized in \cref{fig:framework}). Then, we describe the details of the Bayesian flow of every modality. The training and sampling algorithm can be found in \cref{alg:train} and \cref{alg:sampling}.

\textbf{Overview} Operating in the parameter space $\bthetaM=\{\bthetaA,\bthetaL,\bthetaF\}$, \modelname generates high-fidelity crystals through a joint BFN sampling process on the parameter of  atom type $\bthetaA$, lattice parameter $\vec{\theta}^L=\{\bmuL,\brhoL\}$, and the parameter of fractional coordinate matrix $\bthetaF=\{\bmF,\bcF\}$. We index the $n$-steps of the generation process in a discrete manner $i$, and denote the corresponding continuous notation $t_i=i/n$ from prior parameter $\thetaM_0$ to a considerably low variance parameter $\thetaM_n$ (\emph{i.e.} large $\vrho^L,\bmF$, and centered $\bthetaA$).

At training time, \modelname samples time $i\sim U\{1,n\}$ and $\bthetaM_{i-1}$ from the Bayesian flow distribution of each modality, serving as the input to the network. The network $\net$ outputs $\net(\parsnt{i-1}^\mathcal{M},t_{i-1})=\net(\parsnt{i-1}^A,\parsnt{i-1}^F,\parsnt{i-1}^L,t_{i-1})$ and conducts gradient descents on loss function \cref{eq:loss_n} for each modality. After proper training, the sender distribution $p_S$ can be approximated by the receiver distribution $p_R$. 

At inference time, from predefined $\thetaM_0$, we conduct transitions from $\thetaM_{i-1}$ to $\thetaM_{i}$ by: $(1)$ sampling $\y_i\sim p_R(\bold{y}|\thetaM_{i-1};t_i,\alpha_i)$ according to network prediction $\predM{i-1}$; and $(2)$ performing Bayesian update $h(\thetaM_{i-1},\y^\calM_{i-1},\alpha_i)$ for each dimension. 

% Alternatively, we complete this transition using the flow-back technique by sampling 
% $\thetaM_{i}$ from Bayesian flow distribution $\flow(\btheta^M_{i}|\predM{i-1};t_{i-1})$. 

% The training objective of $\net$ is to minimize the KL divergence between sender distribution and receiver distribution for every modality as defined in \cref{eq:loss_n} which is equivalent to optimizing the negative variational lower bound $\calL^{VLB}$ as discussed in \cref{sec:preliminaries}. 

%In the following part, we will present the Bayesian flow of each modality in detail.

\textbf{Bayesian Flow of Fractional Coordinate $\vF$}~The distribution of the prior parameter $\bthetaF_0$ is defined as:
\begin{equation}\label{eq:prior_frac}
    p(\bthetaF_0) \defeq \{vM(\vm_0^F|\vec{0}_{3\times N},\vec{0}_{3\times N}),\delta(\vc_0^F-\vec{0}_{3\times N})\} = \{U(\vec{0},\vec{1}),\delta(\vc_0^F-\vec{0}_{3\times N})\}
\end{equation}
Note that this prior distribution of $\vm_0^F$ is uniform over $[\vec{0},\vec{1})$, ensuring the periodic translation invariance property in \cref{De:pi}. The training objective is minimizing the KL divergence between sender and receiver distribution (deduction can be found in \cref{appd:cir_loss}): 
%\oyyw{replace $\vF$ with $\x$?} \hanlin{notations follow Preliminary?}
\begin{align}\label{loss_frac}
\calL_F = n \E_{i \sim \ui{n}, \flow(\parsn{}^F \mid \vF ; \senderacc)} \alpha_i\frac{I_1(\alpha_i)}{I_0(\alpha_i)}(1-\cos(\vF-\predF{i-1}))
\end{align}
where $I_0(x)$ and $I_1(x)$ are the zeroth and the first order of modified Bessel functions. The transition from $\bthetaF_{i-1}$ to $\bthetaF_{i}$ is the Bayesian update distribution based on network prediction:
\begin{equation}\label{eq:transi_frac}
    p(\btheta^F_{i}|\parsnt{i-1}^\calM)=\mathbb{E}_{vM(\bold{y}|\predF{i-1},\alpha_i)}\delta(\btheta^F_{i}-h(\btheta^F_{i-1},\bold{y},\alpha_i))
\end{equation}
\begin{restatable}{proposition}{fracinv}
With $\net_{F}$ as a periodic translation equivariant function namely $\net_F(\parsnt{}^A,w(\parsnt{}^F+\vt),\parsnt{}^L,t)=w(\net_F(\parsnt{}^A,\parsnt{}^F,\parsnt{}^L,t)+\vt), \forall\vt\in\R^3$, the marginal distribution of $p(\vF_n)$ defined by \cref{eq:prior_frac,eq:transi_frac} is periodic translation invariant. 
\end{restatable}
\textbf{Bayesian Flow of Lattice Parameter \texorpdfstring{$\boldsymbol{L}$}{}}   
Noting the lattice parameter $\bm{L}$ located in Euclidean space, we set prior as the parameter of a isotropic multivariate normal distribution $\btheta^L_0\defeq\{\vmu_0^L,\vrho_0^L\}=\{\bm{0}_{3\times3},\bm{1}_{3\times3}\}$
% \begin{equation}\label{eq:lattice_prior}
% \btheta^L_0\defeq\{\vmu_0^L,\vrho_0^L\}=\{\bm{0}_{3\times3},\bm{1}_{3\times3}\}
% \end{equation}
such that the prior distribution of the Markov process on $\vmu^L$ is the Dirac distribution $\delta(\vec{\mu_0}-\vec{0})$ and $\delta(\vec{\rho_0}-\vec{1})$, 
% \begin{equation}
%     p_I^L(\boldsymbol{L}|\btheta_0^L)=\mathcal{N}(\bm{L}|\bm{0},\bm{I})
% \end{equation}
which ensures O(3)-invariance of prior distribution of $\vL$. By Eq. 77 from \citet{bfn}, the Bayesian flow distribution of the lattice parameter $\bm{L}$ is: 
\begin{align}% =p_U(\bmuL|\btheta_0^L,\bm{L},\beta(t))
p_F^L(\bmuL|\bm{L};t) &=\mathcal{N}(\bmuL|\gamma(t)\bm{L},\gamma(t)(1-\gamma(t))\bm{I}) 
\end{align}
where $\gamma(t) = 1 - \sigma_1^{2t}$ and $\sigma_1$ is the predefined hyper-parameter controlling the variance of input distribution at $t=1$ under linear entropy accuracy schedule. The variance parameter $\vrho$ does not need to be modeled and fed to the network, since it is deterministic given the accuracy schedule. After sampling $\bmuL_i$ from $p_F^L$, the training objective is defined as minimizing KL divergence between sender and receiver distribution (based on Eq. 96 in \citet{bfn}):
\begin{align}
\mathcal{L}_{L} = \frac{n}{2}\left(1-\sigma_1^{2/n}\right)\E_{i \sim \ui{n}}\E_{\flow(\bmuL_{i-1} |\vL ; t_{i-1})}  \frac{\left\|\vL -\predL{i-1}\right\|^2}{\sigma_1^{2i/n}},\label{eq:lattice_loss}
\end{align}
where the prediction term $\predL{i-1}$ is the lattice parameter part of network output. After training, the generation process is defined as the Bayesian update distribution given network prediction:
\begin{equation}\label{eq:lattice_sampling}
    p(\bmuL_{i}|\parsnt{i-1}^\calM)=\update^L(\bmuL_{i}|\predL{i-1},\bmuL_{i-1};t_{i-1})
\end{equation}
    

% The final prediction of the lattice parameter is given by $\bmuL_n = \predL{n-1}$.
% \begin{equation}\label{eq:final_lattice}
%     \bmuL_n = \predL{n-1}
% \end{equation}

\begin{restatable}{proposition}{latticeinv}\label{prop:latticeinv}
With $\net_{L}$ as  O(3)-equivariant function namely $\net_L(\parsnt{}^A,\parsnt{}^F,\vQ\parsnt{}^L,t)=\vQ\net_L(\parsnt{}^A,\parsnt{}^F,\parsnt{}^L,t),\forall\vQ^T\vQ=\vI$, the marginal distribution of $p(\bmuL_n)$ defined by \cref{eq:lattice_sampling} is O(3)-invariant. 
\end{restatable}


\textbf{Bayesian Flow of Atom Types \texorpdfstring{$\boldsymbol{A}$}{}} 
Given that atom types are discrete random variables located in a simplex $\calS^K$, the prior parameter of $\boldsymbol{A}$ is the discrete uniform distribution over the vocabulary $\parsnt{0}^A \defeq \frac{1}{K}\vec{1}_{1\times N}$. 
% \begin{align}\label{eq:disc_input_prior}
% \parsnt{0}^A \defeq \frac{1}{K}\vec{1}_{1\times N}
% \end{align}
% \begin{align}
%     (\oh{j}{K})_k \defeq \delta_{j k}, \text{where }\oh{j}{K}\in \R^{K},\oh{\vA}{KD} \defeq \left(\oh{a_1}{K},\dots,\oh{a_N}{K}\right) \in \R^{K\times N}
% \end{align}
With the notation of the projection from the class index $j$ to the length $K$ one-hot vector $ (\oh{j}{K})_k \defeq \delta_{j k}, \text{where }\oh{j}{K}\in \R^{K},\oh{\vA}{KD} \defeq \left(\oh{a_1}{K},\dots,\oh{a_N}{K}\right) \in \R^{K\times N}$, the Bayesian flow distribution of atom types $\vA$ is derived in \citet{bfn}:
\begin{align}
\flow^{A}(\parsn^A \mid \vA; t) &= \E_{\N{\y \mid \beta^A(t)\left(K \oh{\vA}{K\times N} - \vec{1}_{K\times N}\right)}{\beta^A(t) K \vec{I}_{K\times N \times N}}} \delta\left(\parsn^A - \frac{e^{\y}\parsnt{0}^A}{\sum_{k=1}^K e^{\y_k}(\parsnt{0})_{k}^A}\right).
\end{align}
where $\beta^A(t)$ is the predefined accuracy schedule for atom types. Sampling $\btheta_i^A$ from $p_F^A$ as the training signal, the training objective is the $n$-step discrete-time loss for discrete variable \citep{bfn}: 
% \oyyw{can we simplify the next equation? Such as remove $K \times N, K \times N \times N$}
% \begin{align}
% &\calL_A = n\E_{i \sim U\{1,n\},\flow^A(\parsn^A \mid \vA ; t_{i-1}),\N{\y \mid \alphat{i}\left(K \oh{\vA}{KD} - \vec{1}_{K\times N}\right)}{\alphat{i} K \vec{I}_{K\times N \times N}}} \ln \N{\y \mid \alphat{i}\left(K \oh{\vA}{K\times N} - \vec{1}_{K\times N}\right)}{\alphat{i} K \vec{I}_{K\times N \times N}}\nonumber\\
% &\qquad\qquad\qquad-\sum_{d=1}^N \ln \left(\sum_{k=1}^K \out^{(d)}(k \mid \parsn^A; t_{i-1}) \N{\ydd{d} \mid \alphat{i}\left(K\oh{k}{K}- \vec{1}_{K\times N}\right)}{\alphat{i} K \vec{I}_{K\times N \times N}}\right)\label{discdisc_t_loss_exp}
% \end{align}
\begin{align}
&\calL_A = n\E_{i \sim U\{1,n\},\flow^A(\parsn^A \mid \vA ; t_{i-1}),\N{\y \mid \alphat{i}\left(K \oh{\vA}{KD} - \vec{1}\right)}{\alphat{i} K \vec{I}}} \ln \N{\y \mid \alphat{i}\left(K \oh{\vA}{K\times N} - \vec{1}\right)}{\alphat{i} K \vec{I}}\nonumber\\
&\qquad\qquad\qquad-\sum_{d=1}^N \ln \left(\sum_{k=1}^K \out^{(d)}(k \mid \parsn^A; t_{i-1}) \N{\ydd{d} \mid \alphat{i}\left(K\oh{k}{K}- \vec{1}\right)}{\alphat{i} K \vec{I}}\right)\label{discdisc_t_loss_exp}
\end{align}
where $\vec{I}\in \R^{K\times N \times N}$ and $\vec{1}\in\R^{K\times D}$. When sampling, the transition from $\bthetaA_{i-1}$ to $\bthetaA_{i}$ is derived as:
\begin{equation}
    p(\btheta^A_{i}|\parsnt{i-1}^\calM)=\update^A(\btheta^A_{i}|\btheta^A_{i-1},\predA{i-1};t_{i-1})
\end{equation}

The detailed training and sampling algorithm could be found in \cref{alg:train} and \cref{alg:sampling}.




\label{secOverview}
%

\section{Experiments}
\label{sec:exp}
Following the settings in Section \ref{sec:existing}, we evaluate \textit{NovelSum}'s correlation with the fine-tuned model performance across 53 IT datasets and compare it with previous diversity metrics. Additionally, we conduct a correlation analysis using Qwen-2.5-7B \cite{yang2024qwen2} as the backbone model, alongside previous LLaMA-3-8B experiments, to further demonstrate the metric's effectiveness across different scenarios. Qwen is used for both instruction tuning and deriving semantic embeddings. Due to resource constraints, we run each strategy on Qwen for two rounds, resulting in 25 datasets. 

\subsection{Main Results}

\begin{table*}[!t]
    \centering
    \resizebox{\linewidth}{!}{
    \begin{tabular}{lcccccccccc}
    \toprule
    \multirow{3}*{\textbf{Diversity Metrics}} & \multicolumn{10}{c}{\textbf{Data Selection Strategies}} \\
    \cmidrule(lr){2-11}
    & \multirow{2}*{\textbf{K-means}} & \multirow{2}*{\vtop{\hbox{\textbf{K-Center}}\vspace{1mm}\hbox{\textbf{-Greedy}}}}  & \multirow{2}*{\textbf{QDIT}} & \multirow{2}*{\vtop{\hbox{\textbf{Repr}}\vspace{1mm}\hbox{\textbf{Filter}}}} & \multicolumn{5}{c}{\textbf{Random}} & \multirow{2}{*}{\textbf{Duplicate}} \\ 
    \cmidrule(lr){6-10}
    & & & & & \textbf{$\mathcal{X}^{all}$} & ShareGPT & WizardLM & Alpaca & Dolly &  \\
    \midrule
    \rowcolor{gray!15} \multicolumn{11}{c}{\textit{LLaMA-3-8B}} \\
    Facility Loc. $_{\times10^5}$ & \cellcolor{BLUE!40} 2.99 & \cellcolor{ORANGE!10} 2.73 & \cellcolor{BLUE!40} 2.99 & \cellcolor{BLUE!20} 2.86 & \cellcolor{BLUE!40} 2.99 & \cellcolor{BLUE!0} 2.83 & \cellcolor{BLUE!30} 2.88 & \cellcolor{BLUE!0} 2.83 & \cellcolor{ORANGE!20} 2.59 & \cellcolor{ORANGE!30} 2.52 \\    
    DistSum$_{cosine}$  & \cellcolor{BLUE!30} 0.648 & \cellcolor{BLUE!60} 0.746 & \cellcolor{BLUE!0} 0.629 & \cellcolor{BLUE!50} 0.703 & \cellcolor{BLUE!10} 0.634 & \cellcolor{BLUE!40} 0.656 & \cellcolor{ORANGE!30} 0.578 & \cellcolor{ORANGE!10} 0.605 & \cellcolor{ORANGE!20} 0.603 & \cellcolor{BLUE!10} 0.634 \\
    Vendi Score $_{\times10^7}$ & \cellcolor{BLUE!30} 1.70 & \cellcolor{BLUE!60} 2.53 & \cellcolor{BLUE!10} 1.59 & \cellcolor{BLUE!50} 2.23 & \cellcolor{BLUE!20} 1.61 & \cellcolor{BLUE!30} 1.70 & \cellcolor{ORANGE!10} 1.44 & \cellcolor{ORANGE!20} 1.32 & \cellcolor{ORANGE!10} 1.44 & \cellcolor{ORANGE!30} 0.05 \\
    \textbf{NovelSum (Ours)} & \cellcolor{BLUE!60} 0.693 & \cellcolor{BLUE!50} 0.687 & \cellcolor{BLUE!30} 0.673 & \cellcolor{BLUE!20} 0.671 & \cellcolor{BLUE!40} 0.675 & \cellcolor{BLUE!10} 0.628 & \cellcolor{BLUE!0} 0.591 & \cellcolor{ORANGE!10} 0.572 & \cellcolor{ORANGE!20} 0.50 & \cellcolor{ORANGE!30} 0.461 \\
    \midrule    
    \textbf{Model Performance} & \cellcolor{BLUE!60}1.32 & \cellcolor{BLUE!50}1.31 & \cellcolor{BLUE!40}1.25 & \cellcolor{BLUE!30}1.05 & \cellcolor{BLUE!20}1.20 & \cellcolor{BLUE!10}0.83 & \cellcolor{BLUE!0}0.72 & \cellcolor{ORANGE!10}0.07 & \cellcolor{ORANGE!20}-0.14 & \cellcolor{ORANGE!30}-1.35 \\
    \midrule
    \midrule
    \rowcolor{gray!15} \multicolumn{11}{c}{\textit{Qwen-2.5-7B}} \\
    Facility Loc. $_{\times10^5}$ & \cellcolor{BLUE!40} 3.54 & \cellcolor{ORANGE!30} 3.42 & \cellcolor{BLUE!40} 3.54 & \cellcolor{ORANGE!20} 3.46 & \cellcolor{BLUE!40} 3.54 & \cellcolor{BLUE!30} 3.51 & \cellcolor{BLUE!10} 3.50 & \cellcolor{BLUE!10} 3.50 & \cellcolor{ORANGE!20} 3.46 & \cellcolor{BLUE!0} 3.48 \\ 
    DistSum$_{cosine}$ & \cellcolor{BLUE!30} 0.260 & \cellcolor{BLUE!60} 0.440 & \cellcolor{BLUE!0} 0.223 & \cellcolor{BLUE!50} 0.421 & \cellcolor{BLUE!10} 0.230 & \cellcolor{BLUE!40} 0.285 & \cellcolor{ORANGE!20} 0.211 & \cellcolor{ORANGE!30} 0.189 & \cellcolor{ORANGE!10} 0.221 & \cellcolor{BLUE!20} 0.243 \\
    Vendi Score $_{\times10^6}$ & \cellcolor{ORANGE!10} 1.60 & \cellcolor{BLUE!40} 3.09 & \cellcolor{BLUE!10} 2.60 & \cellcolor{BLUE!60} 7.15 & \cellcolor{ORANGE!20} 1.41 & \cellcolor{BLUE!50} 3.36 & \cellcolor{BLUE!20} 2.65 & \cellcolor{BLUE!0} 1.89 & \cellcolor{BLUE!30} 3.04 & \cellcolor{ORANGE!30} 0.20 \\
    \textbf{NovelSum (Ours)}  & \cellcolor{BLUE!40} 0.440 & \cellcolor{BLUE!60} 0.505 & \cellcolor{BLUE!20} 0.403 & \cellcolor{BLUE!50} 0.495 & \cellcolor{BLUE!30} 0.408 & \cellcolor{BLUE!10} 0.392 & \cellcolor{BLUE!0} 0.349 & \cellcolor{ORANGE!10} 0.336 & \cellcolor{ORANGE!20} 0.320 & \cellcolor{ORANGE!30} 0.309 \\
    \midrule
    \textbf{Model Performance} & \cellcolor{BLUE!30} 1.06 & \cellcolor{BLUE!60} 1.45 & \cellcolor{BLUE!40} 1.23 & \cellcolor{BLUE!50} 1.35 & \cellcolor{BLUE!20} 0.87 & \cellcolor{BLUE!10} 0.07 & \cellcolor{BLUE!0} -0.08 & \cellcolor{ORANGE!10} -0.38 & \cellcolor{ORANGE!30} -0.49 & \cellcolor{ORANGE!20} -0.43 \\
    \bottomrule
    \end{tabular}
    }
    \caption{Measuring the diversity of datasets selected by different strategies using \textit{NovelSum} and baseline metrics. Fine-tuned model performances (Eq. \ref{eq:perf}), based on MT-bench and AlpacaEval, are also included for cross reference. Darker \colorbox{BLUE!60}{blue} shades indicate higher values for each metric, while darker \colorbox{ORANGE!30}{orange} shades indicate lower values. While data selection strategies vary in performance on LLaMA-3-8B and Qwen-2.5-7B, \textit{NovelSum} consistently shows a stronger correlation with model performance than other metrics. More results are provided in Appendix \ref{app:results}.}
    \label{tbl:main}
    \vspace{-4mm}
\end{table*}


\begin{table}[t!]
\centering
\resizebox{\linewidth}{!}{
\begin{tabular}{lcccc}
\toprule
\multirow{2}*{\textbf{Diversity Metrics}} & \multicolumn{3}{c}{\textbf{LLaMA}} & \textbf{Qwen}\\
\cmidrule(lr){2-4} \cmidrule(lr){5-5} 
& \textbf{Pearson} & \textbf{Spearman} & \textbf{Avg.} & \textbf{Avg.} \\
\midrule
TTR & -0.38 & -0.16 & -0.27 & -0.30 \\
vocd-D & -0.43 & -0.17 & -0.30 & -0.31 \\
\midrule
Facility Loc. & 0.86 & 0.69 & 0.77 & 0.08 \\
Entropy & 0.93 & 0.80 & 0.86 & 0.63 \\
\midrule
LDD & 0.61 & 0.75 & 0.68 & 0.60 \\
KNN Distance & 0.59 & 0.80 & 0.70 & 0.67 \\
DistSum$_{cosine}$ & 0.85 & 0.67 & 0.76 & 0.51 \\
Vendi Score & 0.70 & 0.85 & 0.78 & 0.60 \\
DistSum$_{L2}$ & 0.86 & 0.76 & 0.81 & 0.51 \\
Cluster Inertia & 0.81 & 0.85 & 0.83 & 0.76 \\
Radius & 0.87 & 0.81 & 0.84 & 0.48 \\
\midrule
NovelSum & \textbf{0.98} & \textbf{0.95} & \textbf{0.97} & \textbf{0.90} \\
\bottomrule
\end{tabular}
}
\caption{Correlations between different metrics and model performance on LLaMA-3-8B and Qwen-2.5-7B.  “Avg.” denotes the average correlation (Eq. \ref{eq:cor}).}
\label{tbl:correlations}
\vspace{-2mm}
\end{table}

\paragraph{\textit{NovelSum} consistently achieves state-of-the-art correlation with model performance across various data selection strategies, backbone LLMs, and correlation measures.}
Table \ref{tbl:main} presents diversity measurement results on datasets constructed by mainstream data selection methods (based on $\mathcal{X}^{all}$), random selection from various sources, and duplicated samples (with only $m=100$ unique samples). 
Results from multiple runs are averaged for each strategy.
Although these strategies yield varying performance rankings across base models, \textit{NovelSum} consistently tracks changes in IT performance by accurately measuring dataset diversity. For instance, K-means achieves the best performance on LLaMA with the highest NovelSum score, while K-Center-Greedy excels on Qwen, also correlating with the highest NovelSum. Table \ref{tbl:correlations} shows the correlation coefficients between various metrics and model performance for both LLaMA and Qwen experiments, where \textit{NovelSum} achieves state-of-the-art correlation across different models and measures.

\paragraph{\textit{NovelSum} can provide valuable guidance for data engineering practices.}
As a reliable indicator of data diversity, \textit{NovelSum} can assess diversity at both the dataset and sample levels, directly guiding data selection and construction decisions. For example, Table \ref{tbl:main} shows that the combined data source $\mathcal{X}^{all}$ is a better choice for sampling diverse IT data than other sources. Moreover, \textit{NovelSum} can offer insights through comparative analyses, such as: (1) ShareGPT, which collects data from real internet users, exhibits greater diversity than Dolly, which relies on company employees, suggesting that IT samples from diverse sources enhance dataset diversity \cite{wang2024diversity-logD}; (2) In LLaMA experiments, random selection can outperform some mainstream strategies, aligning with prior work \cite{xia2024rethinking,diddee2024chasing}, highlighting gaps in current data selection methods for optimizing diversity.



\subsection{Ablation Study}


\textit{NovelSum} involves several flexible hyperparameters and variations. In our main experiments, \textit{NovelSum} uses cosine distance to compute $d(x_i, x_j)$ in Eq. \ref{eq:dad}. We set $\alpha = 1$, $\beta = 0.5$, and $K = 10$ nearest neighbors in Eq. \ref{eq:pws} and \ref{eq:dad}. Here, we conduct an ablation study to investigate the impact of these settings based on LLaMA-3-8B.

\begin{table}[ht!]
\centering
\resizebox{\linewidth}{!}{
\begin{tabular}{lccc}
\toprule
\textbf{Variants} & \textbf{Pearson} & \textbf{Spearman} & \textbf{Avg.} \\
\midrule
NovelSum & 0.98 & 0.96 & 0.97 \\
\midrule
\hspace{0.10cm} - Use $L2$ distance & 0.97 & 0.83 & 0.90\textsubscript{↓ 0.08} \\
\hspace{0.10cm} - $K=20$ & 0.98 & 0.96 & 0.97\textsubscript{↓ 0.00} \\
\hspace{0.10cm} - $\alpha=0$ (w/o proximity) & 0.79 & 0.31 & 0.55\textsubscript{↓ 0.42} \\
\hspace{0.10cm} - $\alpha=2$ & 0.73 & 0.88 & 0.81\textsubscript{↓ 0.16} \\
\hspace{0.10cm} - $\beta=0$ (w/o density) & 0.92 & 0.89 & 0.91\textsubscript{↓ 0.07} \\
\hspace{0.10cm} - $\beta=1$ & 0.90 & 0.62 & 0.76\textsubscript{↓ 0.21} \\
\bottomrule
\end{tabular}
}
\caption{Ablation Study for \textit{NovelSum}.}
\label{tbl:ablation}
\vspace{-2mm}
\end{table}

In Table \ref{tbl:ablation}, $\alpha=0$ removes the proximity weights, and $\beta=0$ eliminates the density multiplier. We observe that both $\alpha=0$ and $\beta=0$ significantly weaken the correlation, validating the benefits of the proximity-weighted sum and density-aware distance. Additionally, improper values for $\alpha$ and $\beta$ greatly reduce the metric's reliability, highlighting that \textit{NovelSum} strikes a delicate balance between distances and distribution. Replacing cosine distance with Euclidean distance and using more neighbors for density approximation have minimal impact, particularly on Pearson's correlation, demonstrating \textit{NovelSum}'s robustness to different distance measures.






\label{secEXP}

%
\section{Conclusion}
In this work, we propose a simple yet effective approach, called SMILE, for graph few-shot learning with fewer tasks. Specifically, we introduce a novel dual-level mixup strategy, including within-task and across-task mixup, for enriching the diversity of nodes within each task and the diversity of tasks. Also, we incorporate the degree-based prior information to learn expressive node embeddings. Theoretically, we prove that SMILE effectively enhances the model's generalization performance. Empirically, we conduct extensive experiments on multiple benchmarks and the results suggest that SMILE significantly outperforms other baselines, including both in-domain and cross-domain few-shot settings.
\label{secCon}

% \vspace{-2cm}
\section*{Acknowledgements}
This work was supported by the National Natural Science Foundation of China (No. T2322012, No. 62172218, No. 62032011), the Shenzhen Science and Technology Program (No. JCYJ20220818103401003, No. JCYJ20220530172403007), and the Guangdong Basic and Applied Basic Research Foundation (No. 2022A1515010170).

\noindent\textbf{Data Availability}
All synthetic datasets can be accessed at \emph{\textcolor{magenta}{https://github.com/yuxumin/PoinTr/blob/master/DATASET.md}}. All real-world data can be accessed at \emph{\textcolor{magenta}{https://github.com/xuelin-chen/pcl2pcl-gan-pub}}. The scene datasets are publicly available at \emph{\textcolor{magenta}{https://github.com/JinfengX/CasFusionNet}}.

{\small
    \bibliographystyle{spbasic}
    \bibliography{pcd}
}

\end{document}
