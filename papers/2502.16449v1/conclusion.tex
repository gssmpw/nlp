\chapter{Conclusions}\label{chap:conclusion}
This chapter synthesizes the primary findings outlined in Section~\ref{sec:findings} and delves into the limitations and prospective future directions detailed in Section~\ref{sec:limitations}. Finally, Section~\ref{sec:closing_remarks} offers reflective closing remarks on the contributions and implications of the research.
\section{Research Findings}\label{sec:findings}
\subsection{\textit{EMVLight}}
\textit{EMVLight}, introduced in Chapter~\ref{chap:emvlight}, demonstrated the feasibility and efficacy of integrating decentralized MARL to simultaneously address EMV routing and traffic signal pre-emption. It effectively tackles two key network-level challenges: (1) coupling dynamic EMV routing with traffic signal control, and (2) reducing EMV travel time while simultaneously mitigating overall congestion. Experimental results highlight that \textit{EMVLight} achieves a remarkable 42.6\% reduction in EMV travel time and a 23.5\% decrease in average travel time for all vehicles, significantly outperforming both conventional and RL-based traffic control methods in synthetic and real-world scenarios.

Many components contribute to the success of \textit{EMVLight}. The incorporation of emergency lane formation highlights the model's capacity to coordinate between primary pre-emption agents, secondary pre-emption agents, and normal agents to pre-clear links ahead of EMVs effectively. Experimental observations reveal that \textit{EMVLight} often navigates EMVs along longer routes that result in shorter overall travel times. The inclusion of a spatial discount factor, combined with the centralized training and decentralized execution scheme, facilitates optimal trajectory planning across the network during EMV traversal. Additionally, the definition of pressure was modified to align with the agent design, ensuring feasibility in achieving the framework's goals to reduce disturbance and overall traffic congestion. Another critical component is the use of fingerprints, which, as demonstrated in ablation studies, significantly stabilizes training and improves agent coordination. Furthermore, \textit{EMVLight} explores the trade-off between minimizing EMV travel time and mitigating overall congestion, delegating this decision to traffic managers based on operational priorities.

\subsection{MAPPO-DQJL}
Chapter~\ref{chap:mappo-dqjl} introduced MAPPO-DQJL, a novel intra-link coordination strategy leveraging the presence of CAVs. This framework dynamically forms lanes to prioritize EMV passage, utilizing V2X communication to coordinate CAVs and indirectly influence HDVs. Modeling the DQJL formation as a POMDP, the framework employs a transformer-based network design and the MAPPO algorithm to train agents to deliver the optimal DQJL establishment strategy for expedited EMV passage with reduced unnecessary maneuvers from non-EMVs.

Extensive experiments on a variety of simulated multi-lane roadway configurations demonstrated the framework's capacity to clear routes for EMVs while reducing overall traffic disturbances. EMVs changed lanes fewer than once on average across all experimental settings with the proposed methodology. By treating HDVs as agents rather than part of the environment, MAPPO-DQJL significantly enhances performance by enabling CAVs to learn and adapt to HDV behaviors, yielding optimal coordination strategies. The framework achieved up to a 39.8\% reduction in EMV passage time and a 55.7\% reduction in lane-change maneuvers for non-EMVs, underscoring its scalability and robustness in mixed traffic scenarios.

Notably, the framework's performance improved with higher CAV penetration rates, achieving near-optimal coordination when penetration reached 75\% or higher. In multi-lane configurations, MAPPO-DQJL demonstrated that EMVs could navigate segments with fewer than one lane change on average, minimizing disruptions to non-EMVs while expediting emergency responses. Furthermore, it was observed that the benefits of CAV coordination exhibit diminishing returns when penetration rates or traffic density reaches a critical mass, aligning with findings from other CAV applications in tasks similar to DQJL formation.

\subsection{NYC EMS Accessibility Study}
Chapter~\ref{chap:ems_accessibility} presented an intersection-aware EMS accessibility model for NYC, integrating road network characteristics, intersection density, and demographic data. The model accounts for delays occurring at intersections and provides valuable insights into travel times assigned to and from EMS facilities. Additionally, it explores the percentage of NYC residents covered within varying benchmark service times.

The study identified vulnerable areas in terms of EMS response times and further analyzed the demographic characteristics of these regions. By incorporating \textit{EMVLight} into the study, the model demonstrated its potential to reduce intersection delays and improve hospital accessibility.

Results indicated that implementing \textit{EMVLight} could enable over 95\% of NYC residents to access hospitals within benchmark travel times, offering actionable insights for urban planners and policymakers to enhance EMS equity and efficiency. Moreover, the study revealed that areas with high intersection densities continue to face delays due to congestion, underscoring the necessity for integrated signal optimization strategies. By analyzing borough-specific response times and population density distributions, the findings emphasize the critical importance of targeted interventions, such as increasing hospital density and optimizing resource allocation in underserved areas.
\section{Limitations and Future Directions}\label{sec:limitations}

\subsection{\textit{EMVLight}}
While \textit{EMVLight} effectively integrates routing and pre-emption, several limitations remain. The framework relies on accurate real-time traffic data, which may not always be available or reliable in real-world deployments. Future research could explore integrating predictive traffic models and robust communication systems to address data uncertainty. Additionally, the scalability of \textit{EMVLight} in larger urban networks with more complex traffic patterns warrants further investigation. 

A critical consideration for future work lies in the statistical robustness of our experimental results. While our current findings demonstrate improvements in travel times, the statistical significance of these improvements needs more rigorous validation. Future studies should determine the optimal number of experimental runs through power analysis, considering factors such as effect size and desired confidence levels. This would provide more statistically sound conclusions about the framework's performance improvements, especially regarding travel time reductions which might appear marginal in absolute terms but could have significant cumulative impacts on emergency response efficiency.

Non-trivial future directions for this study include, but are not limited to, the following. First, as an effort to bridge the gap between simulation and reality, we are motivated to develop more sophisticated traffic pattern models that better capture the complex dynamics of mixed traffic with EMVs. Second, we aim to address the challenge of navigating multiple EMVs simultaneously in the same traffic network, which raises both technical and ethical considerations in reward design for pre-emption agents. Third, a fundamental challenge in applying reinforcement learning to transportation systems lies in the extensive trial-and-error nature of the learning process. Future research should focus on developing safe learning frameworks that minimize negative impacts during the learning phase.

\subsection{MAPPO-DQJL}
Despite its success in coordinating DQJLs, MAPPO-DQJL faces challenges in handling extreme traffic densities and low CAV penetration rates. The reliance on centralized training may also limit scalability in real-time applications. Future work should focus on adaptive learning techniques to reduce computational overhead and enhance real-time adaptability.

A key area for enhancement lies in the representation of human-driven vehicles within the framework. Current models may oversimplify the stochastic nature of human driving behaviors and the complex decision-making processes of human drivers. Future work should explore incorporating more sophisticated driver behavior models that account for varying levels of aggression, risk tolerance, and decision-making patterns. Additionally, the game-theoretic nature of interactions between selfish human drivers, particularly during emergency scenarios, warrants deeper investigation. This could involve developing mechanism design approaches to incentivize cooperative behavior while acknowledging the inherently competitive nature of individual routing choices.

Experimental validation in real-world mixed autonomy environments would provide critical insights for deployment. Additionally, expanding the framework to incorporate multi-lane dynamic lane assignment strategies could further enhance its effectiveness, particularly in environments with asymmetric traffic flows or high vehicle heterogeneity.

Integrating \textit{\textit{EMVLight}} with MAPPO-DQJL presents a promising avenue for future research, as it combines the strengths of network-wide traffic signal coordination with dynamic intra-link lane-clearing strategies. By leveraging \textit{\textit{EMVLight}}'s decentralized MARL framework for real-time routing and signal optimization alongside MAPPO-DQJL's capability to manage mixed traffic at the lane level, a unified system could holistically address both macro-level and micro-level challenges of EMV passage. This integration could enable seamless coordination between intersections and road segments, dynamically adapting to traffic patterns while ensuring unobstructed EMV movement. Furthermore, the combined framework would be particularly effective in scenarios with varying CAV penetration rates, where coordination across intersections and individual lanes becomes critical. To achieve this, future work could focus on designing reward mechanisms that align global network efficiency with local lane-level optimization, ensuring that the dual objectives of reducing EMV travel time and minimizing disruptions to non-EMVs are met cohesively. Experimental validation in simulation environments and real-world mixed autonomy networks would be instrumental in demonstrating the practical feasibility and benefits of this integrated approach.

\subsection{NYC EMS Accessibility Study}
Chapter~\ref{chap:ems_accessibility} highlighted critical disparities but relied on static data and simplified assumptions about traffic dynamics and intersection delays. Future research could incorporate dynamic traffic simulations and real-time data feeds to refine accessibility models. Expanding the study to include other urban centers with varying demographics and infrastructure would provide a broader understanding of EMS accessibility challenges. Additionally, integrating equity-focused policies, such as targeted investments in underserved areas, could address systemic inequities in EMS coverage.

Another potential direction is to explore the impact of emerging technologies, such as autonomous ambulances or drone-based EMS support, on accessibility metrics. Coupling these innovations with intersection-aware frameworks could revolutionize emergency response strategies, particularly in geographically constrained regions.

\section{Closing Remarks}\label{sec:closing_remarks}
This dissertation presents a comprehensive approach to enhancing EMV passage and EMS accessibility in urban environments through innovative methodologies. The integration of MARL-based frameworks, such as \textit{EMVLight} and MAPPO-DQJL, alongside detailed accessibility studies, demonstrates significant potential to improve emergency response efficiency and equity. By addressing the limitations and exploring the proposed future directions, this research lays a robust foundation for advancing emergency traffic management and urban resilience. Furthermore, the findings underscore the transformative potential of integrating cutting-edge technologies with data-driven strategies to create safer, more equitable, and efficient urban transportation systems.
