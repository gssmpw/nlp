\chapter{Introduction}

In this chapter, we present the background and motivations underpinning this dissertation in Section~\ref{sec:background}, providing a comprehensive overview of the problem domain and its significance. Section~\ref{sec:challenges} identifies the key research challenges that this work aims to address, highlighting gaps in the existing literature and practical constraints. Finally, Section~\ref{sec:contributions} outlines the major contributions and structure of this dissertation.

\section{Background and Motivation}\label{sec:background}

Emergency Response Time (ERT) is a critical metric for assessing urban resiliency and public safety, reflecting a city's capacity to respond promptly and effectively to a diverse range of emergencies, including medical incidents, fire outbreaks, and criminal activities. Shorter ERTs are indicative of well-coordinated emergency response systems, efficient infrastructure, and sufficient resource allocation. Conversely, delays in ERT exacerbate the severity of incidents, increasing risks to human life and property.

Over the past decade, urban centers have faced mounting challenges in maintaining optimal ERT levels, primarily due to increasing traffic congestion. As urban populations grow and vehicular density escalates, the mobility of Emergency Vehicles (EMVs) is increasingly compromised. This trend has led to significant delays in emergency response, with profound implications for public health and safety. Prolonged ERTs correlate with higher mortality rates in critical medical cases, more extensive damage during fire incidents, and reduced effectiveness in deterring or containing criminal activities. Collectively, these factors undermine public trust in emergency management systems.

Taking New York City (NYC) as an illustrative case, Fig.~\ref{fig:ERT_NYC} presents the end-to-end ERT for all emergency incident types, comparing data from July 2014 to July 2024, as reported in the NYC 911 End-to-End Response Time dataset~\cite{NYC911Data}. The values annotated above each bar represent the average ERT (minutes), with the percentage increase over the ten-year period highlighted in red for each incident type. Notably, ERTs have increased significantly across all categories, with non-critical and non-life-threatening cases experiencing the most severe delays.

\begin{figure}[h]
    \centering
    \includegraphics[width=.75\linewidth]{Figures/NYC_911_Response_Times.png}
    \caption{NYC 911 End-to-End ERTs for July 2014 and July 2024, across all incident categories.}
    \label{fig:ERT_NYC}
\end{figure}

Among the various stages of emergency response, spanning from incident reporting to on-scene arrival, the increased travel time of EMVs accounts for over 50\% of the observed growth in ERT within NYC, as illustrated in Fig.~\ref{fig:EMV_travel_NYC}. This escalation is primarily attributed to worsening traffic congestion in NYC. In 2024, INRIX, a global traffic analytics service, identified NYC as the most congested city globally~\cite{INRIXScorecard}, with drivers enduring an average of 101 hours annually in traffic delays---a significant increase compared to the 91 hours recorded in 2014, marking a ten-hour rise over the decade.

The implications of delayed emergency responses are severe. For instance, a stroke victim loses approximately 2 million brain cells for every minute of delayed medical intervention~\cite{Time2006Jeffrey}, while survival rates during cardiac arrest decline by 7--10\% with each minute of delay~\cite{Heart2013}. Despite having designated rights-of-way on urban roads, EMVs face mounting challenges in navigating through increasingly congested streets to reach emergency scenes in a timely manner. 

Escalating traffic congestion not only impedes the mobility of EMVs but also undermines their capacity to meet critical response time benchmarks, thereby compounding risks to public health and safety. This alarming trend underscores the pressing need for innovative strategies to facilitate the mobility of EMVs, ensuring that life-saving interventions can be delivered without avoidable hindrance.

\begin{figure}[ht]
    \centering
    \includegraphics[width=.75\linewidth]{Figures/NYC_911_Travel_Times_Taller.png}
    \caption{NYC 911 EMV travel times for July 2014 and July 2024, across all incident categories.}
    \label{fig:EMV_travel_NYC}
\end{figure}

\section{Research Challenges}\label{sec:challenges}

\subsection{Coupling of EMV Navigation and Traffic Signal Pre-emption}

Efficient EMV passage in urban traffic networks presents two interrelated challenges: dynamically routing EMVs under time-dependent traffic conditions and coordinating traffic signal pre-emption to minimize delays for both EMVs and non-EMVs.

The first challenge arises from the inherently dynamic nature of urban traffic, where congestion levels fluctuate across road segments in real time. As an EMV progresses through the network, its routing decisions must be continuously updated to account for evolving traffic conditions. Traditional approaches, such as recalculating shortest paths at every intersection, are computationally expensive and fail to meet the real-time requirements of EMV passage. Therefore, an efficient and adaptive routing algorithm is essential to ensure timely and computationally feasible updates to the EMV's route.

The second challenge concerns the coordination of traffic signals to achieve dual objectives: prioritizing EMV passage while minimizing network-wide disruptions for non-EMVs. Conventional methods often focus exclusively on reducing EMV travel time, disregarding the broader impact on non-EMVs. Vehicles along an EMV's path are often required to stop or pull over, yet the absence of clear and coordinated guidance from traffic signals frequently results in unnecessary delays. Moreover, non-EMVs at adjacent intersections may experience additional disruptions due to the lack of coordination in addressing the cascading delays caused by EMV pre-emption. Addressing this challenge requires traffic signals to operate collaboratively across the network, balancing the urgent need for EMV prioritization with minimizing delays for non-EMVs.

Compounding these challenges are the stochastic and unpredictable characteristics of urban traffic, which introduce significant uncertainties into both routing and signal control decisions. These uncertainties are further exacerbated by the cascading disruptions caused by EMV pre-emption, which ripple through the traffic network and amplify congestion. Addressing these issues necessitates a holistic approach capable of dynamically adapting to real-time traffic conditions, coordinating actions across the network, and accounting for the inherent uncertainties of urban traffic environments. Chapter~\ref{chap:emvlight} aims to address these challenges.

\subsection{Intra-link Movement for EMV}

From a microscopic perspective, the intra-link traversal of EMVs poses significant challenges. While traffic laws mandate yielding the right-of-way to EMVs, and most drivers instinctively comply, their uncoordinated and unpredictable maneuvers often result in suboptimal outcomes, especially on congested roadways. This lack of coordination leads to inefficiencies and delays that undermine the effectiveness of emergency responses.

Despite advancements in vehicle-to-everything (V2X) technologies designed to enhance EMV mobility, sirens remain the primary tool relied upon in practice. However, traditional siren systems frequently fail to provide sufficient warning time to other drivers, particularly in heavily congested traffic conditions. Furthermore, ambiguity regarding which route should be cleared often causes confusion, delaying the EMV's progress. These uncertainties exacerbate delays and contribute to accident rates that are 4 to 17 times higher~\cite{Buchenscheit2009AVE}, along with increased collision severity~\cite{Yasmin2012Effects}. These factors not only hinder EMV traversal but also pose additional risks to traffic safety, highlighting the urgent need for effective solutions.

Current lane-clearing strategies in emergency scenarios predominantly rely on mixed-integer linear programming (MILP) formulations. While these approaches offer theoretical insights, they often oversimplify real-world traffic dynamics and fail to provide actionable decisions in real time. The reliance on computationally intensive models makes them unsuitable for dynamic, time-sensitive scenarios. Additionally, these methods struggle to account for the stochastic nature of traffic flows, unexpected driver behaviors, and the cascading disruptions caused by lane-clearing maneuvers.

The core challenge, therefore, is how to quickly establish a cleared and easy-to-maneuver route for EMV passage on a road segment while minimizing additional disturbances for non-EMVs. Addressing this challenge requires innovative strategies capable of capturing real-time traffic information, managing uncertainties, and delivering computationally efficient solutions that can infer optimal actions within milliseconds. Meeting this need is critical for ensuring timely and safe EMV traversal through congested urban traffic. Chapter~\ref{chap:mappo-dqjl} focuses on providing a solution.

\section{Dissertation Contributions and Summary}\label{sec:contributions}

This dissertation addresses the aforementioned challenges associated with efficient EMV passage in congested urban areas, adopting a comprehensive approach that spans network-level EMV route optimization to intra-link EMV passage strategies. By systematically reducing EMV travel time and mitigating downstream traffic disturbances, the proposed methods aim to enhance emergency response efficiency. Furthermore, a qualitative case study on the emergency accessibility of NYC is conducted to evaluate the potential of the proposed solutions in a real-world setting. The research topics explored in this dissertation represent some of the earliest contributions in EMV passage, addressing critical gaps between theoretical concepts and their practical applications in real-world scenarios.

\subsection{EMVLight}

Chapter~\ref{chap:emvlight} introduces \textit{EMVLight}, a decentralized multi-agent reinforcement learning (MARL) framework that addresses two critical challenges mentioned above: dynamically routing EMVs under time-dependent traffic conditions and coordinating traffic signal pre-emption to minimize delays for both EMVs and non-EMVs. Dynamic EMV routing is particularly challenging due to the evolving nature of road congestion, which requires real-time updates to routing decisions without the computational burden of recalculating shortest paths at every intersection. Similarly, traffic signal pre-emption must balance the need to prioritize EMVs along their routes while ensuring network-wide efficiency, a task that demands coordinated adjustments across the entire traffic network. To address these challenges, \textit{EMVLight} models each intersection as an autonomous agent capable of optimizing local traffic flow and coordinating with neighboring agents to facilitate EMV passage. The framework incorporates three key contributions: a mathematical model for emergency lane formation, which evaluates road segment capacity and dynamically reallocates lanes to enable full-speed EMV travel; a decentralized path-finding algorithm that leverages real-time traffic information to adapt EMV routes efficiently; and an integrated MARL approach that jointly optimizes EMV routing and traffic signal control through specialized agents with tailored reward functions. Experimental results demonstrate that \textit{EMVLight} achieves up to a $42.6\%$ reduction in EMV travel time and a $23.5\%$ decrease in the average travel time for all vehicles, significantly outperforming existing methods across both synthetic and real-world traffic networks for this problem.

\subsection{MAPPO-DQJL}

Chapter~\ref{chap:mappo-dqjl} introduces dynamic queue-jump lanes (DQJLs) as a novel intra-link coordination strategy designed to expedite EMV passage through congested road segments. Extending the traditional concept of queue-jump lanes (QJLs) used in bus operations, DQJLs dynamically form lanes during an EMV's approach or traversal, leveraging V2X communication technologies to coordinate connected autonomous vehicles (CAVs) while accounting for non-connected human-driven vehicles (HDVs). To address the inherent complexity of mixed traffic environments, we model DQJL formation as a partially observable Markov decision process (POMDP), which captures both the controllable behavior of CAVs and the stochastic, unpredictable nature of HDVs. To optimize the DQJL formation process, we develop a multi-agent proximal policy optimization algorithm (MAPPO-DQJL) that employs centralized training with decentralized execution, enabling CAVs to efficiently coordinate lane-clearing maneuvers while minimizing disruptions to all non-EMVs. This framework effectively balances the dual objectives of reducing EMV passage time and reducing disturbance to overall traffic, demonstrating scalability across varying CAV penetration rates and traffic densities in multilane roadways. Through extensive SUMO-based simulations, the proposed framework is validated, achieving up to 39.8\% reduction in EMV passage times and up to 55.7\% reduction in lane-changing maneuvers compared to baseline methods, with particularly pronounced benefits under scenarios of increasing CAV adoption. These contributions address critical gaps in current EMV passage strategies, providing a robust and scalable solution for integrating DQJLs into EMV management systems.

\subsection{EMS Accessibility Study of NYC}

Chapter~\ref{chap:ems_accessibility} presents a comprehensive intersection-aware EMS accessibility model to address the critical challenge of EMS accessibility in congested urban environments, focusing on NYC as a case study. By integrating road network characteristics, intersection density, and population demographics, the proposed model provides a granular evaluation of EMS accessibility, identifying vulnerable regions where response times exceed critical benchmarks. The study introduces a novel metric that incorporates intersection-induced delays into travel time calculations, capturing the complexities of urban traffic networks more realistically. Additionally, it highlights the disparities in EMS coverage across NYC boroughs, with significant accessibility gaps in Staten Island, Queens, and parts of Manhattan. The implementation of \textit{EMVLight}, introduced in Chapter~\ref{chap:emvlight}, is further explored to demonstrate its potential in reducing intersection delays, improving hospital accessibility, and ensuring that over 95\% of NYC residents are served within the benchmark response time. These contributions lay a robust foundation for advancing emergency traffic management strategies and guiding urban planning decisions aimed at equitable and efficient EMS accessibility.

\subsection{Dissertation Outline}

The remainder of this dissertation is organized as follows. Chapter~\ref{chap:emvlight} introduces \textit{EMVLight}, a multi-agent reinforcement learning framework for EMV decentralized routing and traffic signal control. Chapter~\ref{chap:mappo-dqjl} proposes MAPPO-DQJL, an innovative approach that facilitates intra-link EMV movements through cooperative and efficient lane-clearing strategies. Chapter~\ref{chap:ems_accessibility} presents a case study on EMS accessibility in New York City, providing valuable insights into the potential impacts of deploying \textit{EMVLight} in real-world settings. Finally, Chapter~\ref{chap:conclusion} concludes with a summary of research contributions, a discussion of research limitations, and potential directions for future work.
