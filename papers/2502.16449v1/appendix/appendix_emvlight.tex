\chapter{Appendix for Chapter 1}
In this appendix, we present the implementation details of the proposed EMVLight framework in Section~\ref{appendix_a}, the hyper-parameters selected for all RL-based methods in Section~\ref{appendix_b}, and the implementation of the emergency lane in SUMO in Section~\ref{appendix_c}.

\section{Implementation Details}\label{appendix_a}
% The architectures used for experiments are provided below.
\textbf{MDP time step.} Although MDP step length can be arbitrarily small enough for optimality, traffic signal phases should be maintained a minimum amount of time so that vehicles and pedestrians can safely cross the intersections. To avoid rapid switching between the phases, we set our MDP time step length to be 5 seconds.

% \BDedt{Also, due to safety concerns, once a signal phase has been initiated, it remains unchanged for certain minimum amount of time, e.g. 5 seconds. Therefore, to avoid rapid switching between the phases, we set our MDP time step length to be 5 seconds.}

\textbf{Implementation details for Non-emergency-capacitated/Emergency capacitated Synthetic \texorpdfstring{$\text{Grid}_{5 \times 5}$}{Grid}}
\begin{itemize}
    \item dimension of $s^t_{\mathcal{V}_i}$: $5 \times (8+8+4+2)=110$
    \item dimension of $\Tilde{s}^t_{\mathcal{V}_i}$: $5 \times (8+8+4+2)=110$
    \item dimension of $\pi^{t-1}_{\mathcal{N}_i}$: $4 \times 8 = 32$
    % \item Input for both policy and value networks:
    % observation as a tensor of $5 \times 22$ dimensions and neighbor policies as a tensor of $4 \times 8$ dimensions. The value network multiplies the neighbor policies with spatial discounted factor. 1 action dimension.
    \item Policy network $\pi_{\theta_i}(a_i^t|s^t_{\mathcal{V}_i}, \pi^{t-1}_{\mathcal{N}_i})$: 
    \texttt{concat[}$110 \xrightarrow[]{\textrm{FC}} 128$ReLu, $32 \xrightarrow[]{\textrm{FC}} 64$ReLu\texttt{]} $ \xrightarrow[]{} 64$LSTM $ \xrightarrow[]{\textrm{FC}}8$Softmax
    \item Value network $V_{\phi_i}(\Tilde{s}^t_{\mathcal{V}_i}, \pi^{t-1}_{\mathcal{N}_i})$: \texttt{concat[}$110 \xrightarrow[]{\textrm{FC}} 128$ReLu, $32 \xrightarrow[]{\textrm{FC}} 64$ReLu\texttt{]} $ \xrightarrow[]{} 64$LSTM $ \xrightarrow[]{\textrm{FC}}1$Linear
    \item Each link is $200m$. The free flow speed of the EMV is $12m/s$ and the free flow speed for non-EMVs is $6m/s$.
    \item Temporal discount factor $\gamma$ is $0.99$ and spatial discount factor $\alpha$ is $0.90$.
    \item Initial learning rates $\eta_\phi$ and $\eta_\theta$ are both 1e-3 and they decay linearly. Adam optimizer is used.
    \item MDP step length $\Delta t = 5s$ and for secondary pre-emption reward weight $\beta$ is $0.5$.
    \item Regularization coefficient is $0.01$.
\end{itemize}
\textbf{Implementation details for \texorpdfstring{$\text{Manhattan}_{16 \times 3}$}{Manhattan}}
The implementation is similar to the synthetic network implementation, with the following differences:
\begin{itemize}
    \item Initial learning rates $\eta_\phi$ and $\eta_\theta$ are both 5e-4.
    \item Since the avenues and streets are both one-directional, the number of actions of each agent are adjusted accordingly. 
    \item Avenues and streets length are based on real Manhattan block size with each block of $80m \times 274m$.
\end{itemize}

\textbf{Implementation details for \texorpdfstring{$\text{Hangzhou}_{4 \times 4}$}{Manhattan}}
The implementation is similar to the synthetic network implementation, with the following differences:
\begin{itemize}
    \item Initial learning rates $\eta_\phi$ and $\eta_\theta$ are both 5e-4.
    \item Streets length are based on real map of Hangzhou Gudang district.
\end{itemize}

\section{Hyperparameters}\label{appendix_b}
We provide the the choice of hyper-parameters for RL-based methods in Table.\ref{tab_RL_hyperparameters}.
\begin{table}[ht]
\centering
\fontsize{9.0pt}{10.0pt} \selectfont
\begin{tabular}{@{}ccccc@{}}
\toprule
Hyper-parameters & CDRL   & PL   & CL   & EMVLight   \\ \midrule
temporal discount  &   \multicolumn{4}{c}{0.99} \\
batch size      &  32     &  128    & 128     &    128        \\
buffer size     & 1e5      &  1e4    &  1e4    &   1e4         \\
sample size     &  2048     & 1000     & 1000     &  1000          \\
$\eta$ and decay  & 0.5\&0.975 & 0.8\&0.95     &  0.8\&0.95    & 0.8\&0.95           \\ \midrule
optimizer       &  Adam     & RMSprop     &  RMSprop    &   Adam         \\
Learning rate   &  0.00025     &  1e-3    & 1e-3     &   1e-3         \\ \midrule
\# Conv layers   &   1    &  -     &  3    &    -        \\
\# MLP layers    &   1    &   4   &  3    &      1      \\
\# MLP units     &   (168,168)    & (32,32)     &  (32,32)    &   (192,1)         \\
MLP Activation  & \multicolumn{4}{c}{ReLU}         \\
Initialization  & \multicolumn{4}{c}{Random Normal} \\
\midrule
step length $\Delta T$      &   \multicolumn{4}{c}{ 5 seconds}      \\
\bottomrule
\end{tabular}
\caption{Hyper-parameters selected for RL-based methods.}
\label{tab_RL_hyperparameters}
\end{table}

\section{Emergency Lane in SUMO}\label{appendix_c}
With increased lateral resolution after enabling the sublane model, SUMO is able to simulates the virtual lane, i.e. emergency lane, formation for emergency traffic. Vehicles can drive between lanes under this mode. At the same time, we define the EMV as a \emph{blue light device} to incorporate with the sublane model. Thus, we can observe that non-EMVs on the left would pull over towards the left side, i.e. latAlignment="left", and non-EMVs on the right would pull over towards the right side, i.e.,  latAlignment="right". When the EMV is traveling on the emergency lane, these vehicles do not perform lane changes and they do not move onto the emergency lane. After the EMV leaves this segments, these vehicles resume normal driving and move back with their previous lateral alignment. These two SUMO built-in modules together is able to achieve our proposed emergency lane. See Fig.~\ref{fig_emergency_lane_SUMO} below from \cite{bieker2018analysis}.
\begin{figure}[ht]
\includegraphics[width=0.8\linewidth]{images/emergency_lane.png}
\centering
\caption{A blue light device approaching an intersection with the sublane model activated. }
\label{fig_emergency_lane_SUMO}
\end{figure}

