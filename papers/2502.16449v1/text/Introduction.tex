\section{Introduction}
Emergency vehicles (EMVs) include ambulances, fire trucks, and police cars, which respond to critical events such as medical emergencies, fire disasters, and public security crisis. Emergency response time is the key indicator of a city's incidents management ability and resiliency. Reducing response time saves lives and prevents property losses. For instance, Berdowski et al.~\cite{berdowski2010global} indicated that the survivor rate from a sudden cardiac arrest without treatment drops 7\% - 10\% for every minute elapsed, and there is barely any chance to survive after 8 minutes. EMV travel time, the time interval for an EMV to travel from a rescue station to an incident site, accounts for a major portion of the emergency response time. However, overpopulation and urbanization have exacerbated road congestion, making it more challenging to reduce the average EMV travel time. Records \cite{end-to-end-response-times} have shown that even with a decline in average emergency response time, the average EMV travel time increased from 7.2 minutes in 2015 to 10.1 minutes in 2021 in New York City, an approximately 40\% increase over six years even accounting for post-Covid traffic conditions. Therefore, there is a severe urgency and significant benefit for shortening the average EMV travel time on increasingly crowded roads.
%
%Emergency vehicles (EMVs) are vehicles including ambulance, fire trucks and police cars, which respond to critical events such as medical emergencies, fire disasters and criminal activities.  
% \dz{TODO: update statistics and descriptions} According to New York City end-to-end response time record\cite{end-to-end-response-times}, it is reported that the average response time has increased from 7.2 minutes in 2011 to 9.4 minutes in 2021, which represent a 17\% increase of response time in NYC during the past decade. In the case of cardinal arrest, the survivor rate of a patient without treatment drops 7\% - 10\%. 
%Emergency response time is the key indicator of a city's emergency management capability. Reducing response time saves life and prevents property loss. For instance, the survivor rate from a sudden cardiac arrest without treatment drops 7\% - 10\% for every second elapsed, and there is barely any chance to survive after 8 minutes.
%EMV travel time, the time interval for an EMV to travel from a rescue station to an incident site, accounts for the majority of emergency response time. However, overpopulation and urbanization has been exacerbating the road congestion, making it more and more challenging to reduce the average EMV travel time.
%Records \cite{end-to-end-response-times} have shown that even with a decline in average emergency response time, the average EMV travel time increases from 7.2 minutes in 2015 to 10.1 minutes in 2021 in New York City.
%Needless to say, there is a severe urgency and huge benefits, to shorten the average EMV travel time on increasingly crowded roads.
% EMVs are expected to respond to an emergency incident as fast as possible since in certain medical emergencies a delayed response of one minute would increase the mortality rate by 1\%. 
% Strategies aiming to reduce the response time of EMVs has been a focus in the existing literature. 
% The safe, efficient, and fast emergency vehicle dispatching system improves the service level of the society.
% The response time to emergency sites is considered the key to save human lives and limit material damage. Dispatching emergency vehicle (EMV) to the scenes as soon as possible has become a major challenge for service providers and traffic managers nowadays because the roads are increasingly congested. 

Existing work has studied strategies to reduce the travel time of EMVs by route optimization and traffic signal pre-emption \cite{Lu2019Literature, humagain2020systematic}. Route optimization refers to the search for a time-based shortest path. The traffic network (e.g., city road map) is modeled as a graph with intersections as nodes and road segments between intersections as edges. Based on the time a vehicle needs to travel through each edge (road segment), route optimization calculates an optimal route with the minimal EMV travel time \cite{humagain2020systematic}. In addition, as the EMV needs to be as fast as possible, the law in most places requires non-EMVs to yield to emergency vehicles sounding sirens, regardless of the traffic signals at intersections \cite{de1991lights}. Even though this practice gives the right-of-way to EMVs, it poses safety risks for vehicles and pedestrians at intersections \cite{grant2017human}. To address this safety concern, existing methods \cite{nelson2000impact, qin2012control, humagain2020systematic, huang2015design} have also studied traffic signal pre-emption which refers to the process of deliberately altering the signal phases at each intersection to prioritize EMV passage.
%
% introduce the current method
% The travel time of an EMV refers to the time interval for the EMV to travel from a source location (e.g. hospital, fire station, police office) to a emergency site (e.g. a residential community where a fire outbreak happens). \dz{As EMV travel time accounts for a majority of EMV response time, 
% The time that a vehicle needs to travel through each edge (road segment) is based on history data. 
% The most straightforward pre-emption strategy is WALABI \cite{bieker2019modelling}, where the signal phases are altered to let an EMV pass each intersection without stop. 
% cons of the current method
%

However, a major challenge for adaptive EMV operation is the coupling between
EMV route optimization and traffic light pre-emption \cite{humagain2020systematic}. As the traffic condition constantly changes, static route optimization can potentially become suboptimal as an EMV travels through the network; i.e. the traffic is highly time-dependent and exhibits transient properties during a dispatch \cite{Coogan2015Compartmental}. Moreover, traffic signal pre-emption has a significant impact on the traffic flow, which would change the fastest route as well. Thus, the optimal route should be updated with real-time traffic flow information, i.e., the route optimization should be solved in a dynamic (time-dependent) way. As an optimal route can change as an EMV travels through the traffic network, the traffic signal pre-emption would need to adapt accordingly. In other words, the subproblems of dynamic route optimization and traffic signal pre-emption are coupled and should be solved ideally simultaneously in real-time. Existing approaches have limited consideration to this coupling.

% discuss the reduce traffic impact problem. 
% As traffic signal pre-emption prioritize the route of the EMV, the travel time of non-EMVs will be largely influenced. 
% our approach
%
In addition, most of the existing models on emergency vehicle service have a single objective of reducing the EMV travel time \cite{Haghani2004Simulation, haghani2003optimization, panahi2008gis, shaaban2019strategy}. As a result, their traffic signal control strategies have an undesirable effect of increasing the travel time of non-EMVs. Non-EMVs on the path of approaching EMVs would pull over or stop, and they usually do not receive clear guidance from traffic signals on resuming their trips \cite{hsiao2018preventing}, causing unnecessary delay. Non-EMVs elsewhere are also likely indirectly and negatively impacted if adjacent intersections lack coordination to address the incurred delay \cite{humagain2020systematic}. Therefore, traffic signal control strategies accommodating both EMVs and non-EMVs need to be recommended.

In this chapter, we aim to perform route optimization and traffic signal pre-emption to not only reduce EMV travel time but also to reduce the average travel time of non-EMVs. In particular, we address the following two key challenges:
\begin{enumerate}
    \item \textbf{How to dynamically route an EMV to a destination under time-dependent traffic conditions in a computationally efficient way?} As the congestion level of each road segment changes over time, the routing algorithm should be able to update the remaining route as the EMV passes each intersection. Running the shortest-path algorithm each time the EMV passes through an intersection is not efficient. A computationally efficient decentralized routing algorithm is desired. \label{challenge_1}
    \item \textbf{How to coordinate traffic signals to not only reduce EMV travel time but reduce the average travel time of non-EMVs as well?} To reduce EMV travel time, only the traffic signals along the route of the EMV need to be altered. However, to further reduce average non-EMV travel time, traffic signals in the whole traffic network need to cooperate. \label{challenge_2} 
\end{enumerate}

Reinforcement learning (RL), which gained significant traction in assorted domains of traffic signal control recently, has been extensively studied and proven effective for learning stochastic traffic conditions and dealing with randomness.
Thus, to tackle the above challenges, we propose \textbf{EMVLight}, a decentralized multi-agent reinforcement learning framework with a dynamic routing algorithm to control traffic signal phases for efficient EMV passage. \modi{In the proposed RL design, each intersection is an agent and each agent is responsible for deciding local traffic signal phases. Multiple agents coordinate to facilitate EMV passage as well as alleviate congestion.} Our experimental results demonstrate that EMVLight outperforms traditional traffic engineering methods and existing RL methods under two metrics - EMV travel time and the average travel time of all vehicles - on different traffic configurations. We extend the preliminary work \cite{Su2021EMVLight} by taking into account extra capacity of each road segments and the possibility of forming ``emergency lanes" for full speed EMV passage. In addition, we demonstrate EMVLight's performance on synthetic and real-world maps with extra capacities. We also present the difference in EMV routing between EMVLight and benchmark methods with an emphasize on the number of successfully formed emergency lanes.

Our contributions are threefold. First, we capture the emergency capacity in road segments for emergencies and incidents. We also propose a mathematical model to decide whether an emergency lane can be formed for full speed EMV passage based on emergency capacity and number of vehicles of a road segment.
Second, we incorporate a decentralized path-finding scheme for EMVs based on real time traffic information. Third, we propose to solve EMV routing and traffic signal control problems simultaneously in a multi-agent reinforcement learning framework. In particular, we set up different types of reinforcement learning agents based on the location of the EMV and design different rewards for each type. This leads to up to a $42.6\%$ reduction in EMV travel time as well as an $23.5\%$ shorter average travel time of all trips completed in the network compared with existing benchmark methods.


The rest of the paper is organized as follows. Section \ref{sec_related_work} reviews relevant literature. Section \ref{sec_preliminaries} introduces our definition of pressure and emergency capacity. Section \ref{sec_methodology} presents our EMVLight methodology, i.e.,  dynamic routing and reinforcement learning. Benchmark methods and experimental setup are presented in Section \ref{sec_experimentation}. Section \ref{sec_result} discussed and compared the performance of EMVLight and benchmark methods in terms of EMV travel time, average travel time of all vehicles as well as EMV route choices. We conclude in Section \ref{sec_conclusion} and share inspirations on future directions.


% Proven as a promising approach to reduce traffic congestion in urban traffic networks, reinforcement learning has gained increasing interest over the past few years.

% In addition to the purpose of reducing congestion, these multi-intersection communication and cooperation framework based on reinforcement learning are able to extend onto other  \emph{Intelligent Transportation System} (ITS) applications. 

% Among assorted ITS applications, \emph{emergency vehicle signal pre-emption} (EVSP) aims to facilitate the fast and safe passage of emergency vehicles


% As a crucial function of modern cities, emergency service vehicle \emph{response time} (RP) indicates the  
% The safe, efficient, and fast emergency vehicle dispatching system improves the service level of the society.
% The response time to emergency sites is considered the key to save human lives and limit material damage. Dispatching emergency vehicle (EMV) to the scenes as soon as possible has become a major challenge for service providers and traffic managers nowadays because the roads are increasingly congested. According to New York City end-to-end response time record\cite{end-to-end-response-times}, it is reported that the average response time has increased from 7.2 minutes in 2011 to 9.4 minutes in 2021, which represent a 17\% increase of response time in NYC during the past decade. In the case of cardinal arrest, the survivor rate of a patient without treatment drops 7\% - 10\%.
% Needless to say, there is an severe urgency to reduce the emergency response time in urban areas with increasingly congested road.

% Two major components affecting the emergency response time are the routing of the dispatched EMVs and the coordination by the traffic signals along the dispatching routes. Lack of real-time traffic flow information, the planned path may suffer from unanticipated and fluctuate traffic conditions and delay the traveling of EMVs on duty. Uncoordinated traffic signals worsens the situation by blocking the intersection which EMVs are trying to cross. 

% We recognize three key problems to be addressed for the proposed methodology to be effective:
% \begin{itemize}
%     \item Ensuring the EMVs can arrive at incident scenes within the shortest amount of time, how to efficiently relieve the traffic impacts, i.e. regional congestion, come along?
%     \item How to perform a centralized time-based dynamic shortest path search with the distributed traffic signal control?
%     \item How to facilitate the efficient passage 
% \end{itemize}

% \dz{
% Tentative outline:
% \begin{itemize}
%     \item Motivate the "reducing EV response time problem". Clearly specify the goal and its implication. Importance. Introduce subproblems -  1.routing, 2.EMSP, 3. reduce impact. 
%     \item Discuss the routing problem (view the traffic networks as graphs) and challenges (dynamics nature in traffic networks)
%     \item Discuss EMSP and traffic signal control. (This is something I didn't fully understand. I guess we need to use plain language to describe what is "pre-emption". It seems to me "signal pre-emption" is "traffic signal control to allow EVs take priority at intersections". Does it also indicate "secondary coordination"? And "pre-emption" seems to also indicate sensing technologies. Shall we also discuss the sensing technologies that allows "pre-emption" here?)
%     \item general traffic performance 
%     \item pose key questions and our contributions
%         \begin{itemize}
%             \item How to reduce EV travel time and at the same time reduce the impact of pre-emption (traffic signal control) on the traffic? Safety. 
%             \item How to dynamically route the EV to destination under time-sensitive traffic conditions in a computationally efficient way?
%             \item How to simulate link reservation? (Unclear about this one).
%         \end{itemize}
% \end{itemize}
% }