\section{Summary}\label{sec_conclusion}
In this paper, we introduced a decentralized reinforcement learning framework, EMVLight, to facilitate the efficient passage of EMVs and reduce traffic congestion at the same time.
Leveraging the multi-agent A2C framework, agents incorporate dynamic routing and cooperatively control traffic signals to reduce EMV travel time and average travel time of non-EMVs.
Our work considers the realistic settings of emergency capacitated road segments and traffic patterns before, during and after EMV passages. Extending Dijkstra's algorithm and embedding into the multi-class RL agent design, EMVLight fundamentally addresses the coupling challenge of EMV's dynamic routing and traffic signal control, filling the research gap on this particular task.
% Adopting the multi-agent A2C framework for the agents' communication, we manages to dynamically updating the shortest path for EMVs as well as reduces average travel time for non-EMVs.
Evaluated on both synthetic and real-world map, EMVLight shortens the EMV travel time and average travel time by up to $42.6\%$ and $23.5\%$ respectively, comparing with existing methods from traditional and learning-based traffic signal control for EMV-related managements. Both quantitative and qualitative assessments on EMVLight, including its EMV navigation as well as pre-clear and restore traffic conditions under emergency state, have concluded that EMVLight is a promising control scheme for such task.

Non-trivial future directions for this study including, but not limited to, the followings. First, the interactions among vehicles, especially under emergency, are extremely complicated. As an effort to close the gap between simulation and reality, we are motivated to extend current ETA estimation model to capture more realistic traffic patterns with EMVs so that agents are more responsive in the field tests. Second, we are looking forward to navigating multiple EMVs simultaneously in the same traffic network. The reward design for pre-emption agents (imagine two or more EMVs appears within one intersection) is worth a technical and ethical discussion. Last but not least, as one of the RL applications in the field of transportation, our method has required a massive number of updated iterations, even in the simulated environments, to achieve the desirable result. How to learn efficiently so that our trail-and-error attempts would not bring catastrophic impacts on real traffic is a critical question for all ITS RL applications.

