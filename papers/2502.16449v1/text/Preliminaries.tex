\section{Preliminaries}\label{sec_preliminaries}
In this section, we introduce relevant preliminary terms and definitions which facilitate the problem formulation.

 
\subsection{Traffic map, movement and signal phase}
\begin{figure}[h]
\includegraphics[width=0.65\linewidth]{images/fig_movements.jpeg}
\centering
\caption{Traffic movements of a four-way two-lane bidirectional intersection.}
\label{fig_movements}
\end{figure}
A traffic map can be represented by a graph $G(\mathcal{V}, \mathcal{E})$, with intersections as nodes and road segments between intersections as edges. We refer to a one-directional road segment between two intersections as a link. A link has a fixed number of lanes, denoted as $h(l)$ for lane $l$. Vehicles are allowed to switch lane between two intersections. Fig.~\ref{fig_movements} shows 8 links and each link has 2 lanes. 
    
% we make it clear it is per lane
% Traffic movements
A traffic movement $(l,m)$ is defined as the traffic traveling across an intersection from an incoming lane $l$ to an outgoing lane $m$. The intersection shown in Fig.~\ref{fig_movements} has 24 permissible traffic movements. As an example, vehicles on lane 1 are turning left, and vehicles on lane 2 may go straight or turn right. After turning into the link, vehicles will enter either lane. Thus, the incoming South link has the potential traffic movements of $\{(1, 7), (1, 8), (2, 5), (2, 6), (2, 3), (2, 4)\}$. The set of all permissible traffic movements of an intersection is denoted as $\mathcal{M}$.

%Traffic signal phase
A traffic signal phase is defined as the set of permissible traffic movements. 
As shown in Fig.~\ref{fig_phases}, an intersection with 4 links has 8 phases.
% During the normal traffic phases as shown in Fig.\ref{fig_movements}, lanes highlighted with the same color allow traffic movements in the opposite directions. 
% Only one pair of traffic movements is allowed anytime during the normal phases when there is no EMV, see Fig. \ref{fig_normal_phases}.
  \begin{figure}[t]
    \includegraphics[width=\linewidth]{images/fig_both_phases.png}
    \centering
  \caption{\emph{Top}: 8 signal phases; \emph{Left}: phase \#2 illustration; \emph{Right}: phase \#5 illustration.
%   Example of a normal traffic phase on the left and example of a pre-emption phase on the right. All traffic phases are presented at the bottom.
  }
  \label{fig_phases}
  \end{figure}
%   \begin{figure}[b]
%     \includegraphics[width=\linewidth]{LaTeX/images/fig_pressure_example.png}
%     \centering
%   \caption{Calculation of the lane pressure for Lane 2.}
%   \label{fig_pressure_example}
%   \end{figure}
%
% \subsection{Pre-emption phase} 
% When an EMV on duty approaches an intersection, however, both lanes along the direction of EMV traveling would have green lights to facilitate the intra-link passage of the EMV. We denote such phases as coordination phases.
%
\subsection{Pressure}
% Pressure of an incoming lane
The pressure of an incoming lane $l$ measures the unevenness of vehicle density between lane $l$ and corresponding out going lanes in permissible traffic movements. The vehicle density of a lane is $x(l)/x_{max}(l)$, where $x(l)$ is the number of vehicles on lane $l$ and $x_{max}(l)$ is the vehicle capacity on lane $l$, which is related to the length of a lane. Then the pressure of an incoming lane $l$ is 
\begin{equation} 
    w(l) = \left|\frac{x(l)}{x_{max}(l)} - \sum_{\{m|(l, m)\in \mathcal{M}\}}\frac{1}{h(m)}\frac{x(m)}{x_{max}(m)}\right|,
    \label{eqn:lane_pressure}
\end{equation}
where $h(m)$ is the number of lanes of the outgoing link which contains $m$. In Fig.~\ref{fig_movements}, $h(m)=2$ for all the outgoing lanes. An example for Eqn.~\eqref{eqn:lane_pressure} is shown in Fig.~\ref{fig_movements}.

Taking the intersection's traffic conditions shown in Fig.\ref{fig_movements} as an example, assuming the maximum capacity for each lane is 5 vehicles, we can calculate the lane pressure for lane 2 to be
\begin{equation*}
    w(2) = \left| \underbrace{\frac{1}{5}}_{\textrm{lane 2}} - \frac{1}{2}(\underbrace{\frac{1}{5} + \frac{2}{5}}_{\textrm{lane 3 and 4}}) - \frac{1}{2}(\underbrace{\frac{3}{5} + \frac{0}{5}}_{\textrm{lane 5 and 6}})\right| = \frac{2}{5}
\end{equation*}


% The pressure $P$ of an intersection is the average of the pressure of all incoming lanes, i.e. $P = \sum w(l)$. In the example shown in Fig.\ref{fig_movements}, the intersection pressure is $\frac{25}{10}$.

The pressure of an intersection indicates the unevenness of vehicle density between incoming and outgoing lanes in an intersection. Intuitively, reducing the pressure leads to more evenly distributed traffic, which indirectly reduce congestion and average travel time of vehicles. EMVLight defines pressure of an intersection in EMVLight as the average of the pressure of all incoming lanes,
\begin{equation*}\label{eq:EMVLight_pressure}
    P_{i} = \frac{1}{|\mathcal{I}_i|}\sum _{l\in \mathcal{I}_i} w(l),
\end{equation*}
where $\mathcal{I}_i$ represents the set of all incoming lanes of intersection $i$. According to such definition, the intersection pressure shown in Fig.\ref{fig_movements} is computed to be $\frac{25}{80}$.

% Here we present the key difference in pressure definition between our work and PressLight 
% \paragraph{Pressure in PressLight}
PressLight \cite{wei2019presslight} assumes that traffic movements are lane-to-lane, i.e., vehicles in one lane can only move into a particular lane in a link. Because of the lane-to-lane assumption, in PressLight, the pressure is defined per movement. PressLight defines the pressure of a movement as the difference of the vehicle density between an incoming lane $l$ and the outgoing lane $m$, i.e., 
\begin{equation*}
    w^{*}(l, m) = \frac{x(l)}{x_{max}(l)} - \frac{x(m)}{x_{max}(m)},
\end{equation*}
For instance, lane 2, shown in Fig.\ref{fig_movements}, carries $w^{*}(2, 3)$, $w^{*}(2, 4)$, $w^{*}(2, 5)$, $w^{*}(2, 6)$, $w^{*}(2, 7)$ and $w^{*}(2, 8)$. Taking the permissible traffic movement from lane 2 to lane 4 as an example, we can get the pressure for this movement as
\begin{equation*}
    w^{*}(2, 4) = \frac{1}{5} - \frac{2}{5} = -\frac{1}{5}
\end{equation*}

PressLight then defines the pressure of an intersection $i$ as the absolute value of the sum of pressure of movements of intersection $i$, i.e., 
When calculating the pressure intersection, PressLight has
\begin{equation*}\label{eq:PressLight_pressure}
    P^{*}_{i} = \left|\sum _{(l, m)\in \mathcal{M}_i} w^{*}(l, m)\right|,
\end{equation*}
where $\mathcal{M}_i$ is the set of permissible traffic movements of intersection $i$. According to PressLight's definition of intersection pressure, Fig.\ref{fig_movements} shows an intersection with pressure of 6.

\paragraph{Pressure in our work}
EMVLight assumes a lane-to-link style traffic movement as vehicles can enter either lane on the target link, see Fig.\ref{fig_movements}. 
Following lane pressure defined in \eqref{eqn:lane_pressure}, 


\paragraph{Comparison}
The first difference between the two definitions is that $w^{*}(l, m)$ can be both positive or negative, but $w(l)$ can only take positive values that measures the unevenness of the vehicle density in the incoming lane and that of the corresponding outgoing lanes. We take the absolute value since the direction of pressure is irrelevant here, and the goal of each agent is to minimize this unevenness. The second difference is that at the intersection level, $P^{*}_{i}$ takes a sum but $P_{i}$ takes an average. The average is more suitable for our purpose since it scales the pressure down and the unit penalty for normal agents would be relatively large as compared to rewards for pre-emption agents (Eqn. (3)).  This design puts the efficient passage of EMV vehicles at the top priority. Our experimentation results, presented in Sec.\ref{sec_result} indicate the proposed pressure design produces a more robust reward signal during training and outperforms PressLight in congestion reduction.

\subsection{Emergency capacity}

A roadway segment may have additional capacity, e.g. shoulder, parking lanes, bike lanes, dedicated to providing extra space that can be used under emergencies and incidents.
In the presence of EMVs, existing vehicles are allowed to pull-over or park on the shoulders temporarily, forming an \emph{emergency lane} for the emergency vehicle to pass. An emergency lane is an lane formed between original lanes dedicated for EMV passage, see Fig.\ref{fig_yielding} bottom. EMVs are assumed to travel freely on the emergency lane through the dense traffic. This is referred as an \emph{emergency yielding} and non-EMVs are experiencing an \emph{part-time shoulder use} \cite{part-time-shoulder-use}.

Adaptive traffic management strategies based on part-time shoulder use, such as dynamic hard shoulder running (D-HSR) \cite{Jiaqi2016Dynamic}, have proven beneficial and cost-effective for such scenarios. 

\begin{figure}[ht]
\includegraphics[width=0.85\linewidth]{images/fig_yielding.jpg}
\centering
\caption{A demonstration of an emergency lane for EMV passage. \emph{Top}: the EMV has to follow the congested queue due to insufficient emergency capacity;
\emph{Bottom}: the EMV is traversing on the yielded emergency lane due to sufficient emergency capacity.}
\label{fig_yielding}
\end{figure}

% Whether or not such an emergency lane for EMV passage can be established depends on several factors such as the segment's shoulder width, lane width, geometric clearance and other factors. 
We use an intuitive mathematical model to determine whether such an emergency lane for EMV passage can be established. 
First, we define \emph{emergency capacity} $C_{i}^{\textrm{EC}}$ of a link $i$ to be the additional capacity in the link for emergencies and incidents. The emergency capacity depends on the segment's shoulder width, lane width, geometric clearance and other factors. We say that a link is \emph{emergency-capacitated} if it has a nonzero emergency capacity.
% A larger emergency capacity enables an emergency lane for EMVs even when the road is more crowded. 
In order to express the maximum number of vehicles allowed in a link, for forming an emergency lane, we assume there are $n_i$ non-EMVs in the link $i$ with an average speed of $s_i$. We also assume that the normal capacity $k_i$ is evenly distributed among $l_i$ lanes so that each lane has a capacity of $k_i/l_i$. 
The overall capacity of the link is then $k_i + C_{i}^{\textrm{EC}}$. 
To form an emergency lane for EMV passage with maximum speed, all the non-EMVs need to move out of the emergency lane. Thus the maximum number of vehicles allowed is $k_i + C_{i}^{\textrm{EC}}- k_i/l_i$. As a result, the travel speed of the EMV is 
% Mathematically, the speed of an EMV passing through a particular segment $i$ can be determined. Without making oversimplified assumptions, we observe $n_{i}$ non-EMVs downstream with an average speed of $s_i$. 
% As an attribute of the segment, the normal capacity $k_i$ is uniformly distributed over $l_i$ lanes, resulting an average lane capacity of $\frac{k_i}{l_i}$ for this segment. Denoting $C_{i}^{\textrm{EC}}$ as the emergency capacity for this segment, we are able to approximate the travel speed of the EMV as
\begin{equation}\label{eqn:EMV_speed}
s_{i}^{\textrm{EMV}} = \begin{cases}
s_{f} & n_{i} \leq k_{i} + C_{i}^{\textrm{EC}} - \frac{k_i}{l_i},\\
s_{i} & \textrm{else,}
\end{cases}
\end{equation}
where $s_{f}$ represents the maximum speed allowed for EMVs.
% 10 <= 10 + 5 - 10 / 2
% 10 <= 10 + 4 - 10 / 2
% 


