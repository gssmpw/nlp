\section{Experimentation}\label{sec_experimentation}
%
In this section, we demonstrate our RL framework using Simulation of Urban MObility (SUMO) \cite{lopez2018microscopic}
% to simulate the dynamic routing of EMVs in congested traffic network.
SUMO is an open-source traffic simulator capable of simulating both microscopic and macroscopic traffic dynamics, suitable for capturing the EMV's impact on the regional traffic as well as monitoring the overall traffic flow. An RL-simulator training pipeline is established between the proposed RL framework and SUMO, i.e., the agents collect observations from SUMO and  preferred signal phases are fed back into SUMO. \modi{Notice that in order to reflect the proposed emergency lane in Fig.3, we enable the built-in \emph{Sublane model} \cite{sumo-sublane} and \emph{Blue light device} \cite{bluelight-device} to establish the emergency lane for EMV passage. Under this scenarios, non-EMVs pull over to the sides when the EMV is driving between, and they resume normal driving when the EMV leaves the segment, see \ref{appendix_c} for more details.
}


\subsection{Datasets and Maps Descriptions}
We conduct the following experiments based on both synthetic and real-world map. 
\paragraph{Synthetic $\text{Grid}_{5\times 5}$} We synthesize a $5 \times 5$ traffic grid, where intersections are connected with bi-directional links. Each link contains two lanes. We assume all the links have zero emergency capacity. We design 4 configurations of time-varying traffic flows, listed in Table \ref{tab_synthetic_configuration}. 
% As for the traffic flow, trip information including the origin, destination and departure time are generated upon the configurations. 
The origin (O) and destination (D) of the EMV are labelled in Fig.~\ref{fig_synthetic_map}.
The traffic for this map has a time span of 1200s. We dispatch the EMV at $t =600s$ to ensure the roads are compacted when it starts travel.
\begin{figure}[h]
    \centering
    \includegraphics[width=0.9\linewidth]{images/fig_synthetic_map_revised.jpeg}  
    \caption{\emph{Left}: the synthetic $\text{grid}_{5\times 5}$. Origin and destination for EMV are labeled. \emph{Right}: an intersection illustration in SUMO, the teal area are inductive loop detected area.}
  \label{fig_synthetic_map}
\end{figure}
%
\begin{table}[h]
\centering
\fontsize{10.0pt}{10.0pt} \selectfont
% \setlength\tabcolsep{4pt}
\begin{tabular}{@{}ccccc@{}}
\toprule[1pt]
\multicolumn{1}{c}{\multirow{2}{*}{Config}} & \multicolumn{2}{l}{Traffic Flow (veh/lane/hr)} & \multirow{2}{*}{Origin}                                                            & \multirow{2}{*}{Destination}                                                     \\ \cmidrule(lr){2-3}
\multicolumn{1}{c}{}                               & Non-peak                 & Peak                &                                                                                    &                                                                                  \\ 
\cmidrule{1-5}
1                                                  & 200                      & 240                 & \multirow{2}{*}{N,S} & \multirow{2}{*}{E,W} \\
\cmidrule{1-3}
2                                                  & 160                      & 320                 &                                                                                    &                                                                                  \\
\cmidrule{1-5}
3                                                  & 200                      & 240                 & \multicolumn{2}{c}{Randomly}                                                                                                               \\
\cmidrule{1-3}
4                                                  & 160                      & 320                 & \multicolumn{2}{c}{generated}     \\ \bottomrule[1pt]                                                                                                                                   
% \multirow{2}{*}{Randomly \\generated}
% \multicolumn{2}{c}{}
% \begin{tabular}[c]{@{}l@{}}N\\ S\end{tabular}
% \begin{tabular}[c]{@{}l@{}}E\\ W\end{tabular}
\end{tabular}
\caption{Configuration for Synthetic $\text{Grid}_{5\times 5}$. Peak flow is assigned from 400s to 800s and non-peak flow is assigned out of this period. For Config. 1 and 2, the vehicles enter the grid from North and South, and exit toward East and West.}
\label{tab_synthetic_configuration}
\end{table}
%
\paragraph{Emergency-capacitated (EC) Synthetic $\text{Grid}_{5\times 5}$}
This map adopts the same network layout as the Synthetic $\text{Grid}_{5\times 5}$ but with emergency-capacitated segments. As shown in Fig. \ref{fig_synthetic_map_reserved}, segments
% different emergency capacities on segments, see Fig. \ref{fig_synthetic_map_reserved}. Segments
approaching intersections highlighted by blue are emergency-capacitated with $C^{EC} = 0.2k$, with $k$ represents the normal vehicle capacity of this segment. All other segments are not emergency-capacitated.
\begin{figure}[h]
    \centering
    \includegraphics[width=0.9\linewidth]{images/fig_reserved.jpeg}
  \caption{Emergency-capacitated Synthetic $\text{Grid}_{5\times 5}$. Segments towards intersections highlighted by blue have emergency capacities.}
  \label{fig_synthetic_map_reserved}
\end{figure}

\paragraph{$\text{Manhattan}_{16\times 3}$}
This is a $16 \times 3$ traffic network extracted from Manhattan Hell's Kitchen area (Fig.~\ref{fig_manhattan}) and customized for demonstrating EMV passage. In this traffic network, intersections are connected
by 16 one-directional streets and 3 one-directional avenues. We assume each avenue contains four lanes and each street contains two lanes so that the right-of-way of EMVs and pre-emption can be demonstrated. 
We assume the emergency capacity for avenues and streets are $C^{EC}_{\textrm{avenue}} = 0.2k_{\textrm{avenue}}$ and $C^{EC}_{\textrm{street}} = 0.15k_{\textrm{street}}$, respectively.
% We also assume different emergency capacities for avenues and streets correspondingly.
The traffic flow for this map is generated from open-source NYC taxi data. Both the map and traffic flow data are publicly available.\footnote{https://traffic-signal-control.github.io/}
The origin and destination of EMV are set to be far away as shown in Fig.~\ref{fig_manhattan}. 
% Again, for a better demonstration purpose, we set the origin and destination as far as possible so the difference between methods are more obvious to see, see Figure. \ref{fig_manhattan}.


% \begin{figure}
% \begin{subfigure}{.5\textwidth}
%   \centering
%   \includegraphics[width=\linewidth]{images/fig_Manhattan_raw.jpeg}
%   \caption{Hell's Kitchen on Google Map}
% \end{subfigure}%
% \begin{subfigure}{.5\textwidth}
%   \centering
%   \includegraphics[width=\linewidth]{images/fig_manhattan_sumo.png}
%   \caption{Hell's Kitchen in SUMO simulator}
% \end{subfigure}
% \caption{$\textrm{Manhattan}_{16 \times 3}$: a 16-by-3 traffic network in Hell's Kitchen area. Origin and destination for the EMV dispatching are labeled.}
% \label{fig_manhattan}
% \end{figure}

\begin{figure}[h]
    \centering
    \includegraphics[width=\linewidth]{images/fig_Manhattan_maps.jpeg}
  \caption{$\textrm{Manhattan}_{16 \times 3}$: a 16-by-3 traffic network in Hell's Kitchen area. Origin and destination for the EMV dispatching are labeled. \emph{Left}: on Google Map; \emph{Right}: in SUMO simulator.}
  \label{fig_manhattan}
\end{figure}

\paragraph{$\textrm{Hangzhou}_{4\times 4}$}
An irregular $4 \times 4$ road network represents major avenues in Gudang sub-district in Hangzhou, China. All the road segments are bi-directional with two lanes in each direction. Both the map and traffic flow data are publicly available.
We set the origin and destination for EMV routing as shown in Fig.~\ref{fig_gudang}. The emergency capacity for each segment is set as $C^{EC} = 0.2k$.
\modi{Notice that this map only includes primary arterials and intersections because secondary arterials and intersections usually lack ITS infrastructure to assist EMVs' passage. They are also hard to monitor and control, considering Hangzhou Gudang district is a densely populated area. It is common to see double parking or merchants occupying road space on these secondary roads. The associated traffic flow dataset \cite{dataset} also excludes vehicle trajectories information on these secondary arterials for simplification.
}
\begin{figure}[h]
    \centering
    \includegraphics[width=\linewidth]{images/fig_Gudang_maps.jpeg}
  \caption{$\textrm{Hangzhou}_{4 \times 4}$: a 4-by-4 irregular and asymmetric network. Origin and destination for the EMV dispatching are labeled. \emph{Left}: on Google Map; \emph{Right}: in SUMO simulator.}
  \label{fig_gudang}
\end{figure}

% \begin{figure}
% \begin{subfigure}{.5\textwidth}
%   \centering
%   \includegraphics[width=\linewidth]{images/fig_Gudang.jpeg}
%   \caption{Gudang sub-district on Google Map}
%   \label{fig_gudang_map}
% \end{subfigure}%

% \begin{subfigure}{.5\textwidth}
%   \centering
%   \includegraphics[width=\linewidth]{images/fig_gudang_sumo.png}
%   \caption{Gudang sub-district in SUMO simulator}
%   \label{fig_gudang_sumo}
% \end{subfigure}
% \caption{$\textrm{Gudang}_{4 \times 4}$: a 4-by-4 irregular and asymmetric network. Origin and destination for the EMV dispatching are labeled.}
% \label{fig_gudang}
% \end{figure}

% \begin{table*}[t]
% \centering
% \fontsize{9.0pt}{10.0pt} \selectfont
% \setlength\tabcolsep{3.4pt}
% % \resizebox{\textwidth}{!}{%
% \begin{tabular}{@{}cccccc|ccccc@{}}
% \toprule[1pt]
% \multirow{2}{*}{Method
% % \footnote{All benchmark methods except Fixed Time without EMV are adopting \textbf{Walabi} as the pre-emption technique.}
% } & \multicolumn{5}{c|}{EMV Travel Time $T_{\textrm{EMV}}$ [s]}                                  & \multicolumn{5}{c}{Average Travel Time $T_{\textrm{avg}}$ [s]}                                             \\ \cmidrule(l){2-11} 
%                         & Config 1     & Config 2     & Config 3     & Config 4     & $\text{Manhattan}_{16\times 3}$    & Config 1        & Config 2        & Config 3        & Config 4        & $\text{Manhattan}_{16\times 3}$       \\ \midrule
% FT w/o EMV          & N/A          & N/A          & N/A          & N/A          & N/A          & 353.43          & 371.13          & 314.25          & 334.10          & 1649.64       \\ \midrule
% W + Static + FT             & 257.20         & 272.00          & 259.20          & 243.80          & 487.20          & 372.19          & 389.13          & 342.49          & 355.05          & 1811.03         \\
% W + Static + MP             & 255.00          & 269.00          & 261.20          & 245.40          & 461.80          & 349.38          & 352.54          & 307.91          & 322.68          & 708.13          \\
% W + Static + CDRL           & 281.20          & 286.20          & 289.80          & 277.80          & 492.20          & 503.35          & 524.26          & 488.12          & 509.55          & 2013.54         \\
% W + Static + PL             & 276.00          & 282.20          & 271.40          & 275.00          & 476.00          & 358.18          & 369.45          & 332.98          & 338.95          & 1410.76         \\ \midrule
% W + dynamic + FT                 & 229.60          & 231.20          & 228.60          & 227.20          & 442.20          & 370.09          & 393.40          & 330.13          & 345.50          & 1699.30         \\
% W + dynamic + MP                 & 226.20          & 234.60          & 224.20          & 217.60          & 438.80          & 345.45          & 348.43          & 313.26          & 325.72          & 721.32          \\
% W + dynamic + CDRL                & 273.40          & 269.60          & 281.00          & 270.80          & 450.20          & 514.29          & 536.78          & 502.12          & 542.63          & 1987.86         \\
% W + dynamic + PL                 & 251.20         & 257.80          & 247.00          & 268.80          & 436.20          & 359.31          & 342.59          & 340.11          & 349.20          & 1412.12         \\ \midrule
% EMVLight                & \textbf{198.60} & \textbf{192.20} & \textbf{199.20} & \textbf{196.80} & \textbf{391.80} & \textbf{322.40} & \textbf{318.76} & \textbf{301.90} & \textbf{321.02} & \textbf{681.23} \\ \bottomrule[1pt]
% \end{tabular}%
% % }
% \caption{Performance comparison of different methods evaluated in the four configurations of the synthetic traffic grid as well as Manhattan Map. For both metrics, the lower value indicates better performance. The lowest values are highlighted in bold. The average travel time of Manhattan map (1649.64) is retrieved from data. 
% }
% \label{tab_performance_review}
% \end{table*}





\subsection{Benchmark Methods}
Due to the lack of existing RL methods for efficient EMV passage, we select traditional methods and RL methods for each subproblem and combine them to set up benchmarks. 
% we integrate techniques from pre-emption, routing and traffic signal control together to serve as the benchmarks.

For traffic signal pre-emption, the most intuitive and widely-used approach is the idea of extending green light period for EMV passage at each intersection which results in a \emph{Green Wave} \cite{corman2009evaluation}. 
\textbf{Walabi (W)} \cite{bieker2019modelling} is 
an effective rule-based method that implemented Green Wave for EMVs in SUMO environment. We integrate Walabi with combinations of routing and traffic signal control strategies introduced below as benchmarks.
% As \textbf{Walabi} \hs{saves EMV travel time in the most obvious manner,} we incorporate it in each of the benchmark methods introduced below.
We first present two routing benchmarks. 
% \emph{Routing benchmarks:}
% as briefed in Section \ref{sec_related_work}. To be specific: 
\begin{itemize}
    \item \textbf{Static routing}: static routing is performed only when EMV starts to travel and the route remains fixed. We adopt A* search as the implementation of static routing since it is a powerful extension to the Dijkstra's shortest path algorithm and is used in many real-time applications. \footnote{Our implementation of A* search employs a Manhattan distance as the heuristic function.}
    % shortest path is predetermined based on the shortest distance. 
    % In highly symmetric maps such as a traffic grid, more than one distance-based shortest path may exist and one of these paths is randomly selected. 
    % The static path is found using Dijkstra's algorithm and remains fixed as the EMV travels.
    \item \textbf{Dynamic routing}: dynamic routing updates the route by taking into account real-time information of traffic conditions. The route is then updated by repeatedly running static routing algorithms. To set up the dynamic routing benchmark, we run A* every 50s as EMV travels. The update interval is set to 50s since running the full A* to update the route is not as efficient as our proposed dynamic Dijkstra's algorithm. 
    % To serve as a benchmark method, we enable the \textbf{A*} search to be performed based on the global traffic network information, namely all agents' states.
    % Considering the time complexity of \textbf{A*} search algorithm, we allow the EMV to perform an \textbf{A*} right before dispatching and every 50s passed by.
\end{itemize}

% For traffic signal control strategies on reducing congestion, we select both conventional transportation engineering benchmarks as well as RL benchmarks:
\emph{Traffic signal control benchmarks:}
\begin{itemize}
    \item \textbf{Fixed Time (FT)}: Cyclical fixed time traffic phases with random offset \cite{roess2004traffic} is a policy that split all phases with an predefined green ratio. The coordination between traffic signals are predefined so it is not updated based on real-time traffic. Because of its simplicity, it is the default strategy in real traffic signal control for steady traffic flow.
    \item \textbf{Max Pressure (MP)}: \cite{varaiya2013max} studies max pressure control and use it as the main criterion for selecting traffic signal phases. It defines pressure for each signal phases and aggressively selects the traffic signal phase with maximum pressure to smooth congestion. Hence the name Max Pressure. It is the state-of-the-art network-level signal control strategy that is not based on learning.
    \item \textbf{Coordinated Deep Reinforcement Learners (CDRL)}: CDRL \cite{van2016coordinated} is a Q-learning based coordinator which directly learns joint local value functions for adjacent intersections. It extends Q-learning from single-agent scenarios to multi-agent scenarios.
    It also employs transfer planning and max-plus coordination strategies for joint intersection coordination. 
    \item \textbf{PressLight (PL)}: PL \cite{wei2019presslight} is also a Q-learning based method for traffic signal coordination. It aims at optimizing the pressure at each intersection. However, it defines pressure for each intersection, which is slightly different from the definition in Max Pressure. Our definition of pressure Eqn.~\eqref{eqn:reward}. is also different from that in PL.
    \item \textbf{CoLight (CL)}: CoLight \cite{wei2019colight} uses a graph-attentional-network-based reinforcement learning method for large scale traffic signal control. It adjusts queue length with information from neighbor intersections.
\end{itemize}

% Adopting \textbf{Walabi} as the pre-emption method, we can comprise one method each from path finding and traffic signal control to assemble as an end-to-end benchmark.
% \subsection{Evaluation Metrics}
% \textbf{EMV travel time} reflects the shortest path finding ability of the methods. To measure the delay experienced by non-EMVs, we choose \textbf{average travel time}, according to existing related work to compare the congestion reducing capabilities. 
% Proven as the most effective metric in transportation engineering practices, it calculates the average time vehicles spent in the system.


\subsection{Metrics}
We evaluate performance of all strategies under two metrics: \emph{EMV travel time}, which reflects the routing and pre-emption ability, and \emph{average travel time}, which indicates the ability of traffic signal control for efficient vehicle passage. Vehicles which have completed their trips during the simulation interval are counted when calculating the average travel time. 


\section{Results and Discussion}\label{sec_result}

In this section, we demonstrate the performance of EMVLight and compare it against that of all benchmark methods on four experimentation maps. The results show a clear advantage of EMVLight under the two metrics. 
In addition, we illustrate the difference of underlying route selection by EMVLight and benchmark methods. We further conduct ablation studies to investigate the contribution of different components to EMVLight's performance.


\subsection{Performance Comparison}\label{sec:metrics_comparison}
To evaluate the performance of the proposed EMVLight and all benchmark methods, we conduct SUMO simulation  with five independent runs for each setting. Randomly generated seeds are used in learning-based methods. Means as well as standard deviations of the simulation results are reported for a full numerical assessment. The differences in simulation results for the same setting under independent SUMO runs are coming from configuration noise during generation, such as vehicles' lengths/accelerations/lane-changing eagerness, and, for RL-based methods, random seeds for initialization.
% For the purpose of fairly competing on both metrics, we weigh the EMV passage and traffic signal control tasks the same by default, which is reflected by $\beta = 0.5$ in Eqn.\eqref{eqn:reward2}. Increasing $\beta$ allows EMVLight to shift priority toward EMV passage, and decreasing $\beta$ asks EMVLight to strive for a better congestion management.

We provide implementation details of EMVLight on different experimentation settings in \ref{appendix_a}. Hyper-parameters choices for EMVLight and RL-based benchmarks are provided in \ref{appendix_b}.
\subsubsection{\texorpdfstring{Synthetic $\text{Grid}_{5\times 5}$ results}{Synthetic Grid results}}
% EMV Travel Time in Synthetic Map w/o emergency capacity
\begin{table}[h]
\centering
\fontsize{9.0pt}{10.0pt} \selectfont
\begin{tabular}{@{}ccccc@{}}
\toprule
\multirow{2}{*}{Method}           & \multicolumn{4}{c}{EMV Travel Time $T_{\textrm{EMV}}$ [s]}       \\ \cmidrule(l){2-5} 
                                  & Config 1 & Config 2 & Config 3 & Config 4 \\ \midrule
FT w/o EMV                        &    N/A      &    N/A      &    N/A      &    N/A      \\ \midrule
W + static + FT           &   258.18 $\pm$ 5.32   &  273.32  $\pm$ 9.74       &  256.40 $\pm$ 6.20      &   240.84  $\pm$ 4.43      \\
W + static + MP         &    260.22 $\pm$ 10.87      &    272.40 $\pm$ 10.92       &   265.74 $\pm$ 11.98       &    242.32 $\pm$ 9.48      \\
W + static + CDRL  &    269.42 $\pm$ 7.32      &   282.20 $\pm$ 5.28      &   276.14 $\pm$ 2.58       &   280.32 $\pm$ 4.82       \\
W + static + PL          &   270.68 $\pm$ 9.13       &  279.14 $\pm$ 9.22        &  281.42 $\pm$ 5.62        &   266.10 $\pm$ 8.32       \\
W + static + CL  &    255.72 $\pm$ 4.23      &   272.06 $\pm$ 8.13      &   270.22 $\pm$ 2.81       &  277.12 $\pm$ 6.10       \\
\midrule
W + dynamic + FT          &   229.38 $\pm$ 8.28       &    212.87 $\pm$ 3.17     &   218.46 $\pm$ 4.28       &    220.69 $\pm$ 7s.96      \\
W + dynamic + MP       &   220.48 $\pm$ 9.26      &  208.08 $\pm$ 12.90         &  212.46 $\pm$ 9.82        &   220.98 $\pm$ 10.62       \\
W + dynamic + CDRL &  239.84 $\pm$ 5.24        &   219.15 $\pm$ 8.26       &  211.86 $\pm$ 7.13        &   232.46 $\pm$ 10.16       \\
W + dynamic + PL          &   243.32 $\pm$ 13.86       &   244.82 $\pm$ 10.52       &  250.12 $\pm$ 8.13        &   255.02 $\pm$ 12.76       \\
W + dynamic + CL  &    220.12 $\pm$ 4.19      &  209.12 $\pm$ 4.76      &   224.00 $\pm$ 5.31       &   226.32 $\pm$ 4.13       \\
 \midrule
EMVLight    & \textbf{195.46} $\pm$ 7.48         &  \textbf{190.66} $\pm$ 8.28         &   \textbf{183.12} $\pm$ 6.43        &    \textbf{189.44} $\pm$ 8.32       \\ \bottomrule
\end{tabular}
\caption{EMV travel time in the four configurations of Synthetic $\text{Grid}_{5\times 5}$. Lower value indicates better performance and the lowest values are highlighted in bold.}
\label{tab_synthetic_emv}
\end{table}
% Average Travel Time in Synthetic Map w/o emergency capacity
\begin{table}[h]
\centering
\fontsize{9.0pt}{10.0pt} \selectfont
\begin{tabular}{@{}ccccc@{}}
\toprule
\multirow{2}{*}{Method}           & \multicolumn{4}{c}{Average Travel Time $T_{\textrm{avg}}$ [s]}       \\ \cmidrule(l){2-5} 
                                  & Config 1 & Config 2 & Config 3 & Config 4 \\ \midrule
FT w/o EMV                        &    353.43 $\pm$ 4.65    &    371.13  $\pm$ 4.58     &    314.25  $\pm$ 2.90    &    334.10 $\pm$ 3.73     \\ 
\midrule
W + static + FT           &   380.42 $\pm$ 13.35  &  395.17 $\pm$ 15.37      &  350.16 $\pm$ 13.66    &   363.90 $\pm$ 15.39     \\
W + static + MP         &  355.10  $\pm$ 15.36      &    362.09  $\pm$ 16.15     &   318.76  $\pm$ 14.90   &     330.69  $\pm$ 15.52   \\
W + static + CDRL  &   559.19 $\pm$ 3.60 & 540.81 $\pm$ 12.04  & 568.13 $\pm$ 13.25      &  568.13 $\pm$ 6.67       \\
W + static + PL  &   369.52 $\pm$ 8.72    &   372.32 $\pm$ 16.05     &   339.18  $\pm$ 7.17     &   339.12  $\pm$ 5.33     \\ 

W + static + CL  &   365.64 $\pm$ 14.08    &   380.13 $\pm$ 8.20     &   328.42  $\pm$ 17.52     &   333.74  $\pm$ 5.76     \\ 
\midrule
W + dynamic + FT          &  380.76  $\pm$ 10.70       &   404.81 $\pm$ 18.76      &   345.09  $\pm$ 11.60     &   358.90  $\pm$  15.27   \\
W + dynamic + MP       &   360.38 $\pm$ 10.31    &  365.10 $\pm$ 8.33       &   327.98  $\pm$ 18.90     &  351.62  $\pm$ 3.79      \\
W + dynamic + CDRL          & 565.38  $\pm$ 16.10      &   544.29  $\pm$ 19.23     &  598.73  $\pm$ 11.01      &    572.22  $\pm$ 13.94    \\
W + dynamic + PL &  373.17  $\pm$ 17.98      &   387.25 $\pm$ 13.98      &  349.12 $\pm$ 16.25     &   330.21  $\pm$ 17.23     \\ 
W + dynamic + CL &  359.14  $\pm$ 9.52      &   370.45 $\pm$ 4.02      &  320.64 $\pm$ 4.10     &   335.27  $\pm$ 7.62     \\ 
\midrule
EMVLight    &  \textbf{335.09}  $\pm$ 4.13    &   \textbf{333.28}  $\pm$ 8.81     & \textbf{307.90}   $\pm$ 3.89       &   \textbf{321.02} $\pm$ 5.87     \\ \bottomrule
\end{tabular}
\caption{Average travel time for all vehicles which have completed trips in the four configurations of Synthetic $\text{Grid}_{5\times 5}$. }
\label{tab_synthetic_avg}
\end{table}

Table~\ref{tab_synthetic_emv} and \ref{tab_synthetic_avg} present the experimental results on average EMV travel time and average travel time on Synthetic $\text{Grid}_{5\times 5}$. In terms of EMV travel time $T_{\textrm{EMV}}$, the dynamic routing benchmark performs better than static routing benchmarks. This is expected since dynamic routing considers the time-dependent nature of traffic conditions and update optimal route accordingly. The best learning and non-learning benchmark methods are dynamics routing with CoLight and Max Pressure, respectively. EMVLight further reduces EMV travel time by 16\% on average as compared to dynamic routing benchmarks. This advantage in performance can be attributed to the design of secondary pre-emption agents. This type of agents learns to ``reserve a link" by choosing signal phases that help clear the vehicles in the link to encourage high speed EMV passage (Eqn.~\eqref{eqn:reward}). 

As for average travel time $T_{\textrm{avg}}$, we first notice that the traditional pre-emption technique (W + static + FT) indeed increases the average travel time by around 10\% as compared to a traditional Fix Time strategy without EMV (denoted as ``FT w/o EMV" in Table \ref{tab_synthetic_avg}), thus decreasing the efficiency of vehicle passage. Different traffic signal control strategies have a direct impact on overall efficiency. Fixed Time is designed to handle steady traffic flow. Max Pressure, as a SOTA traditional method, outperforms Fix Time and, surprisingly, nearly outperforms all RL benchmarks in terms of overall efficiency. This shows that pressure is an effective indicator for reducing congestion and this is why we incorporate pressure in our reward design. Coordinate Learner performs the worst probably because its reward is not based on pressure. PressLight doesn't beat Max Pressure because it has a reward design that focuses on smoothing vehicle densities along a major direction, e.g. an arterial. Grid networks with the presence of EMV make PressLight less effective.
CoLight achieves similar performance as Max Pressure and is the best learning benchmark method.
% \hs{CoLight achieves a smaller $T_{\textrm{avg}}$ than other traffic signal control strategies, attributing to a shorter $T_{\textrm{EMV}}$ simultaneously since roads are overall less congested.} 
Our EMVLight improves its pressure-based reward design to encourage smoothing vehicle densities of all directions for each intersection. This enable us to achieve an advantage of 7.5\% over our best benchmarks (Max Pressure).

\subsubsection{Emergency-capacitated Synthetic \texorpdfstring{$\text{Grid}_{5\times 5}$}{Grid} results}

\begin{table}[h]
\centering
\fontsize{9.0pt}{10.0pt} \selectfont
\begin{tabular}{@{}ccccc@{}}
\toprule
\multirow{2}{*}{Method}           & \multicolumn{4}{c}{EMV Travel Time $T_{\textrm{EMV}}$ [s]}       \\ \cmidrule(l){2-5} 
                                  & Config 1 & Config 2 & Config 3 & Config 4 \\ \midrule
FT w/o EMV                        &    N/A      &    N/A      &    N/A      &    N/A      \\ \midrule
W + static + FT           &   254.04 $\pm$ 7.42   &  260.18  $\pm$ 12.03       &  252.12 $\pm$ 11.03      &   232.47  $\pm$ 12.23      \\

W + static + MP         &    233.76 $\pm$ 8.05      &    258.60 $\pm$ 9.06       &   252.74 $\pm$ 13.05      &    233.20 $\pm$ 8.96      \\

W + static + CDRL  &    240.10 $\pm$ 8.65      &   266.28 $\pm$ 8.54      &   258.10 $\pm$ 9.27       &   270.43 $\pm$ 6.18       \\

W + static + PressLight           &   265.28 $\pm$ 7.28       &  269.10 $\pm$ 6.65        &  270.18 $\pm$ 8.83        &   259.20 $\pm$ 7.13       \\

W + static + CoLight  &    250.82 $\pm$ 6.73      &   267.08 $\pm$ 10.21      &   266.12 $\pm$ 4.13       &  270.18 $\pm$ 8.12       \\

\midrule

W + dynamic + FT          &   210.28 $\pm$ 8.90       &    206.18 $\pm$ 7.19     &   210.28 $\pm$ 8.81       &    207.64 $\pm$ 10.02      \\

W + dynamic + MP       &   202.28 $\pm$ 8.54      &  203.20 $\pm$ 9.07         &  206.64 $\pm$ 7.98        &   210.86 $\pm$ 8.59       \\

W + dynamic + CDRL &  218.36 $\pm$ 8.12        &   209.28 $\pm$ 7.19       &  208.180$\pm$ 10.54        &   230.22 $\pm$ 9.22       \\

W + dynamic + PressLight          &   270.08 $\pm$ 10.20       &   238.10 $\pm$ 9.22       &  242.10 $\pm$ 6.98        &   248.24 $\pm$ 10.24       \\

W + dynamic + CoLight  &    216.04 $\pm$ 4.91      &  206.12 $\pm$ 6.27      &   219.26 $\pm$ 6.87      &   223.78 $\pm$ 5.10       \\

 \midrule
EMVLight    & \textbf{150.28} $\pm$ 7.48         &  \textbf{158.20} $\pm$ 6.28         &   \textbf{154.28} $\pm$ 4.19        &    \textbf{159.28} $\pm$ 6.03       \\ \bottomrule
\end{tabular}
\caption{EMV travel time in the four configurations of Synthetic $\text{Grid}_{5\times 5}$ with an emergency capacity co-efficient of 0.25.}
\label{tab_synthetic_ec_emv}
\end{table}

Table~\ref{tab_synthetic_ec_emv} shows $T_{\textrm{EMV}}$ of all the methods implemented on the  emergency-capacitated synthetic \texorpdfstring{$\text{Grid}_{5\times 5}$}{Grid} map. 
By comparing Table~\ref{tab_synthetic_emv} and Table~\ref{tab_synthetic_ec_emv}, we conclude that the emergency-capacitated map exhibits overall shorter $T_{\textrm{EMV}}$ in all configurations. 
% according to the patterns shown in Table~\ref{tab_synthetic_ec_emv}. 
For benchmark methods, the nonzero emergency capacity shorten $T_{\textrm{EMV}}$ by an average of approximately 12 seconds. 
In particular, dynamic routing-based methods benefit more from the additional emergency capacity, resulting in an average reduction of in 16.28 seconds $T_{\textrm{EMV}}$. This is due to the adaptive nature of dynamic navigation. 
By comparing EMVLight and benchmark methods in Table~\ref{tab_synthetic_ec_emv}, we observe that EMVLight reduce $T_{\textrm{EMV}}$ by up to 50 seconds (25\%) as compared to the best benchmark method in all configurations. 
% EMVLight dedicates an significant decrease in $T_{\textrm{EMV}}$ in all configurations within the emergency capacity settings. Emergency vehicles saves up to approximately 25\% traversing through this network. 
The substantial difference in $T_{\textrm{EMV}}$ reduction between benchmark methods and EMVLight is due to high success rate of emergency lane forming under coordination, which is investigated further in Section~\ref{subsec_route_selection}. 
% \dz{[do we need to define emergency yielding? or is it a well known concept in the field? I think at least we should mention the phrase in the caption of Figure 3.]}
% will be explained by routing behavior deviation among methods in \ref{subsec_route_selection}.

\begin{table}[h]
\centering
\fontsize{9.0pt}{10.0pt} \selectfont
\begin{tabular}{@{}ccccc@{}}
\toprule
\multirow{2}{*}{Method}           & \multicolumn{4}{c}{Average Travel Time $T_{\textrm{avg}}$ [s]}       \\ \cmidrule(l){2-5} 
                                  & Config 1 & Config 2 & Config 3 & Config 4 \\ \midrule
FT w/o EMV                        &    353.43 $\pm$ 4.65    &    371.13  $\pm$ 4.58     &    314.25  $\pm$ 2.90    &    334.10 $\pm$ 3.73     \\ 
\midrule

W + static + FT           &   395.28 $\pm$ 5.17  &  410.94 $\pm$ 9.16      &  365.82 $\pm$ 6.14    &   379.64 $\pm$ 6.21     \\

W + static + MP         &  370.52  $\pm$ 6.18      &    375.44  $\pm$ 6.48     &   331.62  $\pm$ 5.92   &     345.13  $\pm$ 8.63 \\

W + static + CDRL  &   575.28 $\pm$ 7.76 & 555.62 $\pm$ 10.04  & 574.91 $\pm$ 19.86      &  585.20 $\pm$ 7.53\\

W + static + PL  &   385.28 $\pm$ 12.09    &   380.83 $\pm$ 10.07     &   360.09  $\pm$ 11.62     &   369.72  $\pm$ 18.02     \\ 

W + static + CL  &   382.17 $\pm$ 6.02    &   380.13 $\pm$ 8.21     &   344.19  $\pm$ 16.02     &   352.07  $\pm$ 4.10     \\ 
\midrule

W + dynamic + FT          &  389.12  $\pm$ 18.21       &   411.98 $\pm$ 15.31      &   353.72  $\pm$ 9.09     &   367.74  $\pm$  16.82   \\

W + dynamic + MP       &   370.28 $\pm$ 12.51    &  362.82 $\pm$ 9.05      &   335.10  $\pm$ 9.16     &  360.02  $\pm$ 17.18     \\

W + dynamic + CDRL          & 575.05  $\pm$ 9.67      &   550.92  $\pm$ 14.06     &  609.26 $\pm$ 11.12      &    578.10  $\pm$ 12.09    \\

W + dynamic + PL &  380.29  $\pm$ 6.10      &   395.28 $\pm$ 5.62      &  359.16 $\pm$ 14.07     &   337.26  $\pm$ 4.96     \\ 

W + dynamic + CL &  366.14  $\pm$ 8.21      &   380.74 $\pm$ 15.84      &  330.44 $\pm$ 17.29     &   343.58  $\pm$ 14.27     \\ 
\midrule

EMVLight    &  \textbf{334.96}  $\pm$ 5.52    &   \textbf{336.18}  $\pm$ 17.09     & \textbf{309.10}   $\pm$ 15.56       &   \textbf{323.76} $\pm$ 17.24     \\ \bottomrule
\end{tabular}
\caption{Average travel time for all vehicles which have completed trips in the four configurations of Synthetic $\text{Grid}_{5\times 5}$ with an emergency capacity coefficient of 0.25. }
\label{tab_synthetic_ec_avg}
\end{table}

As the EMV travels faster and more emergency lanes are formed, the average travel time of non-emergency vehicles increases. Table~\ref{tab_synthetic_ec_avg} shows $T_{\textrm{avg}}$ with emergency capacity added. By comparing Table~\ref{tab_synthetic_avg} and Table~\ref{tab_synthetic_ec_avg}, we observe that with added emergency capacity, the average increase in $T_{\textrm{avg}}$ for non-learning-based and learning-based benchmarks are 12.04 seconds and 7.78 seconds, respectively. Learning-based methods lead to a smaller average increase since agents gradually learn to direct non-EMVs, which are interrupted by EMV passages, to resume their trips as soon as possible. As a result, potential congested queues would not be formed on these segments, effectively reducing the overall $T_{\textrm{avg}}$.


EMVLight, surprisingly, manages to achieve nearly no increase in $T_{\textrm{avg}}$ with added emergency capacity, even though more 
emergency lanes are formed as indicated by smaller $T_{\textrm{EMV}}$. This result shows that EMVLight is able to learn a strong traffic signal coordination strategy while navigating EMVs simultaneously.  The proposed multi-class agent design demonstrates EMVLight's capability of addressing the coupled problems of EMV routing and traffic signal control simultaneously. EMVLight manages to prepare segments for incoming EMVs by reducing the number of vehicles on those segments and restores the impacted traffic in a timely manner after the EMV passage.
% compared with its performance in non-emergency-capacitated counterpart, which again reinforces the traffic signal coordination capability for non-EMVs, especially for those affected by yielding to EMVs.
% \dz{[Again, I don't know how to explain the effect of agents intuitively here. Let's discuss.][coupling - explain good performance on Tavg]}

\subsubsection{\texorpdfstring{$\textrm{Manhattan}_{16 \times 3}$ results}{Manhattan results}} 
% Please add the following required packages to your document preamble:
% \usepackage{multirow}

\begin{table}[h]
\centering
\fontsize{10.0pt}{12.0pt} \selectfont
\begin{tabular}{ccc}
\hline
\multirow{2}{*}{Method} & \multicolumn{2}{c}{$\textrm{Manhattan}_{16 \times 3}$}               \\ \cline{2-3} 
                        & $T_{\textrm{EMV}}$     & $T_{\textrm{avg}}$     \\ \hline
FT w/o EMV              &        N/A              &  1649.64                    \\ \hline
W + static + FT         &    817.37 $\pm$ 17.40                  &      1816.43 $\pm$ 68.96                \\
W + static + MP         &      686.72    $\pm$ 19.23          &    917.52  $\pm$ 52.16                 \\
W + static + CDRL        &    702.62 $\pm$ 24.29          & 1247.67  $\pm$ 83.47  \\
W + static + PL         &      626.88    $\pm$ 24.82            &      992.06 $\pm$ 47.67                \\
W + static + CL         &      545.26 $\pm$ 30.21                &       855.28 $\pm$ 41.29               \\ \hline
W + dynamic + FT         &     820.54 $\pm$ 28.86                  &    1808.25 $\pm$ 68.04                  \\
W + dynamic + MP         &    632.68  $\pm$ 13.29                  &   921.18  $\pm$ 49.29                   \\
W + dynamic + CDRL        &      680.62  $\pm$ 20.17              &   1262.39 $\pm$ 60.09                   \\
W + dynamic + PL         & 521.42 $\pm$ 27.62 & 977.62 $\pm$ 53.45   \\
W + dynamic + CL         &    501.26 $\pm$ 28.71                  &  862.94 $\pm$ 45.19                  \\ \hline
EMVLight                &     \textbf{292.82} $\pm$ 16.23                  &         \textbf{782.13} $\pm$ 39.31             \\ \hline
\end{tabular}
\caption{$T_{\textrm{EMV}}$ and $T_{\textrm{avg}}$ for $\textrm{Manhattan}_{16 \times 3}$. The average travel time without the presence of EMVs (1649.64) is retrieved from data.}
\label{tab_Manhattan_results}
\end{table}

Table~\ref{tab_Manhattan_results} presents EMV travel time and average travel time of all the methods on the $\textrm{Manhattan}_{16 \times 3}$ map. In terms of $T_{\textrm{EMV}}$, dynamic routing benchmarks in general result in faster EMV passasge, as expected. 
% as expected, EMV navigation based on dynamic routing benchmarks travels faster than static routing. 
Compared with benchmark methods, EMVLight produces a considerably low average $T_{\textrm{EMV}}$ of 292.82 seconds, which is 38\% faster than the best benchmark (W+static+CL). 
As for $T_{\textrm{avg}}$, We have similar observation as in the synthetic maps that Max Pressure achieves a similar level of performance on reducing congestion as PressLight and beats CDRL by a solid margin of 25\%. 
CoLight stands out among benchmarks regarding both metrics. Particularly, CoLight shortens $T_{\text{avg}}$ by approximately one minute than Max Pressure strategies on this map.
% Regarding $T_{\textrm{avg}}$, we can also witness that Max Pressure achieves \textit{similar} level of congestion reliving performance with PressLight and \textit{beats} CDRL by a solid margin of 25\%. 

\begin{figure}[h!]
    \centering
    \includegraphics[width=\linewidth]{images/fig_emv_learning_curves.png}
  \caption{$T_{\textrm{EMV}}$ convergence by learning-based dynamic routing strategies on $\text{Manhattan}_{16\times3}$.}
  \label{fig_emv_learning_curves}
\end{figure}

\begin{figure}[h!]
    \centering
    \includegraphics[width=\linewidth]{images/fig_avg_learning_curves.png}
  \caption{$T_{\textrm{avg}}$ convergence by learning-based dynamic routing strategies on $\text{Manhattan}_{16\times3}$.}
  \label{fig_avg_learning_curves}
\end{figure}

Fig.~\ref{fig_emv_learning_curves} and Fig.~\ref{fig_avg_learning_curves} shows the learning curves of  $T_{\textrm{EMV}}$ and $T_{\textrm{avg}}$, respectively,  in the four RL methods. 
% The learning curves by RL benchmarks for $T_{\textrm{EMV}}$ and $T_{\textrm{avg}}$ are presented in Fig.\ref{fig_emv_learning_curves} and Fig.\ref{fig_avg_learning_curves}, respectively. 
From both figures, we observe that EMVLight has the fastest convergence - in 500 epochs for $T_{\textrm{EMV}}$ and in 1000 epoches for $T_{\textrm{avg}}$ - among all four methods. 
% Both figures illustrate that EMVLight converges to the lowest numerical values within 2000 epochs of training. 
% In addition, EMVLight is also the \textit{first} to converge, experiencing relatively minor fluctuations during learning as an evidence of its robustness. 
In Fig.~\ref{fig_emv_learning_curves}, both PressLight and CDRL struggle with converging to a stable $T_{\textrm{EMV}}$. 
In Fig.~\ref{fig_avg_learning_curves}, CDRL converges very slowly as compared to the other three methods. 
Both figures shows that CDRL behaves the worst since its DQN design hardly scales with an increasing number of intersections. 
These learning curves demonstrate the fast and stable learning of EMVLight. 
% At the same time, other RL benchmarks and EMVLight show learning stability after 750 episodes, striving for a low $T_{\textrm{avg}}$.
% The divergence in training results between EMVLight's $T_{\textrm{EMV}}$ and $T_{\textrm{EMV}}$ from compared benchmarks suggests EMVLight's overall coordination efficiency on simultaneous routing and signal control in this scenario.

\subsubsection{\texorpdfstring{$\textrm{Hangzhou}_{4 \times 4}$ results}{Hangzhou results}}
\begin{table}[h]
\centering
\fontsize{10.0pt}{12.0pt} \selectfont
\begin{tabular}{ccc}
\hline
\multirow{2}{*}{Method} & \multicolumn{2}{c}{$\textrm{Hangzhou}_{4 \times 4}$}               \\ \cline{2-3} 
                        & $T_{\textrm{EMV}}$     & $T_{\textrm{avg}}$     \\ \hline
FT w/o EMV              &        N/A              &   764.08                   \\ \hline
W + static + FT         & 466.19 $\pm$ 10.25 & 779.13 $\pm$ 12.90 \\
W + static + MP         & 377.20 $\pm$ 14.42 & 404.37 $\pm$ 8.12 \\
W + static + CDRL        & 409.56 $\pm$ 12.06 & 749.10 $\pm$ 10.02 \\
W + static + PL         & 380.82 $\pm$ 6.72 & 425.46 $\pm$ 9.74 \\
W + static + CL         & 368.20 $\pm$ 14.66 & 366.14 $\pm$ 8.25 \\ 
\hline
W + dynamic + FT         & 415.63 $\pm$ 9.03  & 783.89 $\pm$ 10.03\\
W + dynamic + MP         & 328.42 $\pm$ 12.28 & 410.25 $\pm$ 6.23 \\
W + dynamic + CDRL        & 401.08 $\pm$ 15.25 & 755.28 $\pm$ 12.82   \\
W + dynamic + PL         & 321.52 $\pm$ 14.58 & 431.27 $\pm$ 8.24\\
W + dynamic + CL         & 319.84 $\pm$ 11.09 & 370.20 $\pm$ 7.13\\ \hline
EMVLight                & \textbf{194.52} $\pm$ 9.65 & \textbf{331.42} $\pm$ 6.18
\\ \hline
\end{tabular}
\caption{$T_{\textrm{EMV}}$ and $T_{\textrm{avg}}$ for $\textrm{Gudang}_{4 \times 4}$. The average travel time without the presence of EMVs (764.08) is retrieved from data.}
\label{tab_Gudang_results}
\end{table}

Table \ref{tab_Gudang_results} presents $T_{\textrm{EMV}}$ and $T_{\textrm{avg}}$ of EMVLight and benchmark methods on $\textrm{Hangzhou}_{4 \times 4}$. EMVLight achieves the lowest $T_{\textrm{EMV}}$ of 194.52 seconds, \textit{beating} the best benchmark (W+dynamic+CL) by 115 seconds (37\%).
As for $T_{\textrm{avg}}$, EMVLight also has excellent performance, exhibiting a 10\% advantage over W+dynamic+CL, and a 20\% advantage over W+dynamic+MP.
Once again, Max Pressure \textit{outperforms} PressLight in terms of $T_{\textrm{avg}}$. This is consistent with our observations with other maps above, suggesting Max Pressure result in great traffic signal coordination strategies to reduce overall congestion.
% is capable of reducing general congestion

Based on results of all introduced experiments, Max Pressure has evinced the consistency to restrict congestion, particularly when restoring stagnant traffic after EMV passages. Consequently, Max Pressure accomplishes the comparable congestion reduction acquirement with, if not better than, RL-based strategies.

% However, lack of coordination for EMV navigation prevents Max Pressure from resulting in a lower $T_{\textrm{EMV}}$ than other methods. 

CDRL fails to learn an effective coordination strategy to manage congestion, where its $T_{\textrm{avg}}$ trivially differs from Fix Time strategy's $T_{\textrm{avg}}$.
CoLight surpasses other methods by shortening $T_{\textrm{avg}}$ by at least 40 seconds but it does not illustrate significant improvement in terms of $T_{\textrm{EMV}}$ from alternative benchmarks.
Learning curves in Fig.~\ref{fig_emv_Gudang_learning_curves} and \ref{fig_avg_Gudang_learning_curves} show that EMVLight converges in 500 epochs. 
% EMVLights remains the optimal control method by saving an additional 10\% of $T_{\textrm{avg}}$ from (W+dynamic+CL) and 20\% from (W+dynamic+MP).

\begin{figure}[h]
    \centering
    \includegraphics[width=\linewidth]{images/fig_emv_time_gudang.png}
  \caption{$T_{\textrm{EMV}}$ convergence by learning-based dynamic routing strategies on $\text{Hangzhou}_{4\times4}$.}
  \label{fig_emv_Gudang_learning_curves}
\end{figure}
\begin{figure}[h]
    \centering
    \includegraphics[width=\linewidth]{images/fig_avg_gudang_learning_curves.png}
  \caption{$T_{\textrm{avg}}$ convergence by learning-based dynamic routing strategies on $\text{Hangzhou}_{4\times4}$.}
  \label{fig_avg_Gudang_learning_curves}
\end{figure}


\subsection{Routing choices}
\label{subsec_route_selection}
In this section, we investigate the reason behind EMVLight's best performance in $T_{\textrm{EMV}}$ from a routing perspective. In particular, we show how EMVLight is able to leverage emergency capacity to achieve maximum speed passage. 
% The differences of $T_{\textrm{EMV}}$ in Table. \ref{tab_synthetic_emv} results from sequences of decisions on which directions to turn encountering intersections during EMVs' trips. 
% These decisions, given the traffic condition at the moment, determine the travel speed on the selected segment and impact the ultimate trip time for EMVs.
To understand different route choices between benchmark methods and EMVLight,
we analyze EMV routes on emergency-capacitated Synthetic $\textrm{Grid}_{5\times5}$ as well as $\textrm{Hangzhou}_{4\times4}$ to gain insight into the advantage of EMVLight.

\paragraph{Emergency-capacitated Synthetic \texorpdfstring{$\textrm{Grid}_{5\times5}$}{Grid} routes} 

Routes selected by EMVLight are demonstrated in Fig.~\ref{fig_routing_synthetic} for all four configurations. As this is a regular grid, first we notice that the length of all EMV routes are the Manhattan distance between the origin and destination. This is the shortest length possible to achieve successful EMV dispatch. 
The four routes confirm that EMVLight directs EMVs to enter the region in the east as soon as possible to leverage the extra emergency capacity for full speed passage. This results in 4 full speed links in Config 1 and 3 as well as 5 full speed links in Config 2 and 4.

We also present the route choices of W+static+MP and W+dynamic+MP in Config 1, of this map, as shown in Fig.~\ref{fig_route_choices}. 
By comparing these routes with EMVLight (Fig.~\ref{fig_routing_config1}), we can clearly see the benchmark method cannot actively leverage the emergency capacity, generating routes with only 1 full speed segment (Fig.~\ref{fig_route_choices_static}) and 2 full speed segment (Fig.~\ref{fig_route_choices_dynamics}). 
% \hs{Comparing with MP-based strategies (see Fig. \ref{fig_route_choices}), we reveal that neither benchmarks proactively guide EMVs into emergency-capacitated zone. Static routing alternative shows only one full speed segment for EMVs and dynamic routing experiences 2 full speed segments, elaborating the significant difference in $T_{\textrm{EMV}}$}.



% The route choices present that EMVLight, after training, is capable of guiding EMVs to immediately traverse into the emergency-capacitated ``zone'' to take the advantage of potential emergent lanes. 
% Notice that in Fig.\ref{fig_routing_config1} an emergent lane has not been established on an segment with emergency capacity. A plausible explanation is EMVLight faced an unanticipated overflow of non-EMVs which get pushed onto that particular segment. This unexpected overflow may come from the noise of the non-EMVs' routing model or an underestimation of emergency capacity of this particular segment. We consider this as a learning limitation and leave it as a future work. 

\begin{figure}
\centering
\subfloat[Config 1]{\label{fig_routing_config1} \includegraphics[width=0.20\textwidth]{images/fig_reserved_routing_1.png}}% The "%" masks the line break.
\hfill
\subfloat[Config 2]{\label{fig_routing_config2} \includegraphics[width=0.20\textwidth]{images/fig_reserved_routing_2.png}}%
\hfill
\subfloat[Config 3]{\label{fig_routing_config3} \includegraphics[width=0.20\textwidth]{images/fig_reserved_routing_3.png}}%
\hfill
\subfloat[config 4]{\label{fig_routing_config4} \includegraphics[width=0.20\textwidth]{images/fig_reserved_routing_4.png}}%
\caption{EMV's routing choice on Emergency-capacitated Synthetic $\text{Grid}_{5\times 5}$ based on EMVLight. Emergency lane established for EMV passage on segments highlighted in green, and not established on segments in red.}
\label{fig_routing_synthetic}
\end{figure}

\begin{figure}[h]
\centering
\begin{subfigure}{.5\textwidth}
  \centering
  \includegraphics[width=.4\linewidth]{images/fig_static_routing_capacitated.jpeg}
  \caption{W+static+MP}
  \label{fig_route_choices_static}
\end{subfigure}%
\begin{subfigure}{.5\textwidth}
  \centering
  \includegraphics[width=.4\linewidth]{images/fig_dynamic_routing_capacitated.jpeg}
  \caption{W+dynamic+MP}
  \label{fig_route_choices_dynamics}
\end{subfigure}
\caption{EMV's routing choice on config 1 of Emergency-capacitated Synthetic $\text{Grid}_{5\times 5}$ under MP-based benchmarks.}
\label{fig_route_choices}
\end{figure}

\paragraph{\texorpdfstring{$\textrm{Hangzhou}_{4 \times 4}$}{Hangzhou} routes} 
Since $\textrm{Hangzhou}_{4 \times 4}$ is an irregular grid where different links have different lengths, the routes optimized by different algorithms have different total lengths. This provides another perspective on evaluating routing performance of different models. 
Fig.~\ref{fig_gudang_routings} shows EMV routes given by W+static+MP, W+dynamic+MP and EMVLight in this grid.
% as an illustration of static/dynamic routing difference.
By comparing the total distances of the three routes, we find that the route chosen by EMVLight is the longest among the three models. However, EMVLight achieves the smallest EMV travel time on this route. This is because EMVLight is able to coordinate traffic signals to leverage the emergency capacity to let the EMV travel at its maximum speed on more than half of the route, indicated by the green segments. 

As for static routing (W+static+MP), it chooses a fixed route given the traffic conditions upon dispatching, favoring the shortest-in-distance route.
% It is not hard to notice that EMVLight chooses the longest-in-distance but shortest-in-time path for EMV passage. Static routing method chooses a fixed route given the traffic conditions upon dispatching, favoring the shortest-in-distance route.
The dynamic routing method (W+dynamic+MP) recalculate the time-based shortest path every 30 seconds as EMV travels. Most of the time, however, Max Pressure fails to reduce the number of vehicles on the upcoming links to enable an emergency lane. Fig.~\ref{SUBFIGURE LABEL 2} shows that the emergency lane is only formed in one of the six links.
% while it fails to effectively communicate with upcoming traffic intersections. 
EMVLight is able to further reduce EMV travel time partly because the time-based shortest path is updated in real time. More importantly, EMVLight is able to reduce the number of vehicles in upcoming links so that emergency lanes can be formed. This can be attributed to the design of primary and secondary preemption agents in EMVLight, which will be further examined in Sec.\ref{sec_ablation_reward}.

% \dz{[If it is possible, maybe we can provide evidence to this statement by running the ablation study (replace secondary-preemption agent by normal agents) on this map to confirm that it indeed outputs fewer green segments]}

% EMVLight, however, computes the time-based shortest path in real time and pro-actively communicate with upcoming intersections. 
% These intersections, after learning, manages to ``reserve'' links to establish an emergent lane for incoming EMVs. It is not hard to observe that EMV make the most of emergency capacities and traverse almost freely on 4 segments, out of 6 segments in total, and arrive at the destination within the shortest amount of time. 
\begin{figure}[ht]
\centering
\begin{subfigure}{.30\textwidth}
    \centering
    \includegraphics[width=.95\linewidth]{images/fig_gudang_static.png}  
    \caption{\bm{$5.5$}km, $277.20 \pm 14.42$s}
    \label{SUBFIGURE LABEL 1}
\end{subfigure}
\begin{subfigure}{.30\textwidth}
    \centering
    \includegraphics[width=.95\linewidth]{images/fig_gudang_A.png}  
    \caption{$5.9$km, $228.42\pm 12.28$s}
    \label{SUBFIGURE LABEL 2}
\end{subfigure}
\begin{subfigure}{.30\textwidth}
    \centering
    \includegraphics[width=.95\linewidth]{images/fig_gudang_emvlight.png}  
    \caption{$6.0$km, $\bm{194.52}\pm9.65$s}
    \label{fig_emvlight}
\end{subfigure}
\caption{The corresponding route selected by (a) W+static+MP, (b) W+dynamic+MP, (c) EMVLight on $\text{Hangzhou}_{4 \times 4}$. Distance and $T_{\textrm{EMV}}$ for the selected path are indicated. The lowest values are highlighted in bold.}
\label{fig_gudang_routings}
\end{figure}




% In terms of EMV travel time $T_{\textrm{EMV}}$, the dynamic routing benchmark performs better than static routing benchmarks. This is expected since dynamic routing considers the time-dependent nature of traffic conditions and update optimal route accordingly. EMVLight further reduces EMV travel time by 18\% in average as compared to dynamic routing benchmarks. This advantage in performance can be attributed to the design of secondary pre-emption agents. This type of agents learn to ``reserve a link" by choosing signal phases that help clear the vehicles in the link to encourage high speed EMV passage (Eqn.~\eqref{eqn:reward}).


% As for average travel time $T_{\textrm{avg}}$, we first notice that the traditional pre-emption technique (W+Static+FT) indeed increases the average travel time by around 10\% as compared to a traditional Fix Time strategy without EMV (denoted as ``FT w/o EMV" in Table \ref{tab_synthetic_avg}), thus decreasing the efficiency of vehicle passage. Different traffic signal control strategies have a direct impact on overall efficiency. Fixed Time is designed to handle steady traffic flow. Max Pressure, as a SOTA traditional method, outperforms Fix Time and, surprisingly, outperforms both RL benchmarks in terms of overall efficiency. This shows that pressure is an effective indicator for reducing congestion and this is why we incorporate pressure in our reward design. Coordinate Learner performs the worst probably because its reward is not based on pressure. PressLight doesn't beat Max Pressure because it has a reward design that focuses on smoothing vehicle densities along a major direction, e.g. an arterial. Grid networks with the presence of EMV make PressLight less effective. Our EMVLight improves its pressure-based reward design to encourage smoothing vehicle densities of all directions for each intersection. This enable us to achieve an advantage of 5\% over our best benchmarks (Max Pressure).

\subsection{Ablation Studies}

\subsubsection{Ablation studies on reward}\label{sec_ablation_reward}
We propose three types of agents and design their rewards (Eqn.~\eqref{eqn:reward}) based on our improved pressure definition and heuristics. 
In order to see how our improved pressure definition and proposed agent types influence the results, we propose three ablation studies:
\begin{enumerate}
    \item replacing our pressure definition by that defined in PressLight
    \modi{\item replacing our pressure definition by $
    P_{i} = \sum _{l\in \mathcal{I}_i} w(l)$}
    \item replacing secondary pre-emption agents with normal agents
    \item replacing primary pre-emption agents with normal agents
\end{enumerate}


\begin{table}[h]
\centering
\fontsize{10.0pt}{10.0pt} \selectfont
% \setlength\tabcolsep{4pt}
\begin{tabular}{@{}ccccc|c@{}}
\toprule[1pt]
Ablations                        & Ablation 1 & \modi{Ablation 2} & Ablation 3 & Ablation 4 & EMVLight\\ \midrule
$T_{\text{EMV}}$ [s]         & 205.20 $\pm$ 6.92   &  \modi{237.40 $\pm$ 10.08}  & 311.52   $\pm$ 5.18                   & 384.71 $\pm$ 8.52    & \textbf{194.52} $\pm$ 9.65       \\
$T_{\text{avg}}$ [s]  & 389.14 $\pm$ 8.40 & \modi{382.07 $\pm$ 6.92}  & 442.73 $\pm$ 6.65 & 444.15 $\pm$ 7.02  & \textbf{331.42} $\pm$ 6.18      \\ \bottomrule[1pt]
\end{tabular}
\caption{Ablation studies on pressure-based reward design and agent types. Experiments are conducted on $\textrm{Hangzhou}_{5 \times 5}$. The lowest value are highlighted in bold.}
\label{tab_ablation_reward}
\end{table}

\begin{figure}[h]
\centering
\begin{subfigure}{.5\textwidth}
  \centering
  \includegraphics[width=.6\linewidth]{images/fig_ablation_dynamic.jpeg}
  \caption{Without secondary agents}
  \label{fig_route_choices_ablation_secondary}
\end{subfigure}%
\begin{subfigure}{.5\textwidth}
  \centering
  \includegraphics[width=.6\linewidth]{images/fig_ablation_static.jpeg}
  \caption{Without primary agents}
  \label{fig_route_choices_ablation_primary}
\end{subfigure}
\caption{EMV's route choice on $\text{Hangzhou}_{4\times 4}$ with replaced primary (a) and secondary (b) agents.}
\label{fig_route_choices_ablation}
\end{figure}


Table \ref{tab_ablation_reward} shows the results of these ablations on the $\text{Hangzhou}_{4\times 4}$ map. We observe that PressLight-style pressure yields a slightly larger $T_{\textrm{EMV}}$ but significantly increases the $T_{\textrm{avg}}$. Without secondary pre-emption agents, $T_{\textrm{EMV}}$ increases by 60\% since almost no ``link reservation" happened. \modi{In Ablation 2, if we do not take the average while keep the sum of absolute value of lane pressures as the intersection pressure, we notice a slightly smaller $T_{\text{avg}}$ but a 15\% increase in $T_{\text{EMV}}$. EMVLight's pressure design outperforms the ones in Ablation 1 and 2 and has been proven as the most suitable pressure design for this particular task.} Moreover, without primary pre-emption agents, $T_{\textrm{EMV}}$ increases considerably, which again proves the importance of pre-emption. We can further confirm the importance of agent designs by inspecting the selected routes in the last two ablation studies. 
Fig. \ref{fig_route_choices_ablation_secondary} and \ref{fig_route_choices_ablation_primary} shows routes selected by EMVLights after replacing secondary agents and primary agents with normal agents, respectively. 
Even though the routes are similar as that in Fig.~\ref{fig_emvlight}, much fewer emergency lanes are successfully formed. This failure of utilizing emergency capacity lead to the significant increase in EMV travel time as shown in Table~\ref{tab_ablation_reward}.
As we can see from the routes, EMVs barely take advantage of emergency yielding during their trips.

% This is because agents now treat EMVs as normal vehicles and do not select the corresponding pre-emption phase, resulting in the EMV travel time slightly smaller than the average travel time.

\subsubsection{Ablation study on policy exchanging}
In multi-agent RL, fingerprint has been shown to stabilize training and enable faster convergence.
% is a common technique for faster and more stable training performance. 
In order to see how fingerprint affects training in EMVLight, we remove the fingerprint design, i.e., policy and value networks are changed from $\pi_{\theta_i}(a_i^t|s^t_{\mathcal{V}_i}, \pi^{t-1}_{\mathcal{N}_i})$ and  $V_{\phi_i}(\Tilde{s}^t_{\mathcal{V}_i}, \pi^{t-1}_{\mathcal{N}_i})$ to $\pi_{\theta_i}(a_i^t|s^t_{\mathcal{V}_i})$ and  $V_{\phi_i}(\Tilde{s}^t_{\mathcal{V}_i})$, respectively. 
Fig.~\ref{fig_FP_comparison} shows the influence of fingerprint on training. With fingerprint, the reward converges faster and suffers from less fluctuation, confirming the effectiveness of fingerprints, i.e. policy exchanging.
% We provide a comparison here, see Figure \ref{fig_FP_comparison}, on synthetic $\text{grid}_{5 \times 5}$ with configuration 1 to justify whether or not fingerprint facilitates our training.
\begin{figure}[ht]
    \centering
    \includegraphics[width=0.9\linewidth]{images/fig_fingerprint_comparison.jpg}
  \caption{Reward convergence with and without fingerprint. Experiments are conducted on Config 1 synthetic $\text{grid}_{5 \times 5}$.}
  \label{fig_FP_comparison}
\end{figure}
% As we can see, enabling policy sharing between agents indeed accelerate agents' learning. Training without fingerprint, although eventually reach the similar reward level, experiences more fluctuations in the learning curve.

% \subsection{Sensitivity test on emergency capacity}
% Comparison between Synthetic $\textrm{Grid}_{5\times5}$ and Emergency-capacitated Synthetic $\textrm{Grid}_{5\times5}$ reveals
\subsection{Sensitivity Analysis on \texorpdfstring{$\beta$}{beta}}
\modi{
The value of $\beta$'s in Eqn.~\eqref{eqn:reward2} represents the weights assigned to the two tasks. When $\beta = 1$, the secondary pre-emption agents will act like normal agents and only focus on congestion alleviation. When $\beta = 0$, the secondary pre-emption agents concentrate on emptying non-EMVs and reserve the whole segment for the approaching EMV. In order to illustrate the tradeoff, we conduct experimental runs on the emergency-capacitated synthetic $\text{Grid}_{5\times 5}$ with configuration 1 as the sensitivity analysis on $\beta$, see Table~\ref{tab:beta}. We perform 5 independent runs for each $\beta$'s value. 

\begin{table}[ht]
\fontsize{9.0pt}{10.0pt} \selectfont
\centering
\begin{tabular}{@{}cccccc@{}}
\toprule
           & 0 & 0.25 & 0.5 & 0.75 & 1 \\ \midrule
$T_{\text{EMV}}$ & $119.10 \pm 5.14$  & $134.40 \pm 5.83$     & $158.20 \pm 6.28$    & $172.60 \pm 5.71$   &  $185.20 \pm 4.19$ \\
$T_{\text{avg}}$ & $340.23 \pm 4.75$  &  $338.17 \pm 7.25$     & $334.92 \pm 5.52$    &  $331.81 \pm 9.28 $ & $327.10 \pm 8.18$  \\ \bottomrule
\end{tabular}
\caption{EMVLight's performance on Emergency-capacitated Synthetic $\text{Grid}_{5\times 5}$ with Configuration 1 given different values of $\beta$.}
\label{tab:beta}
\end{table}

As we can see from Table \ref{tab:beta}, $T_{\textrm{EMV}}$ obtains the minimum value when $\beta$ is 0 since the EMV's path has been cleared and reserved in advance. Comparing to $T_{\text{EMV}}$ when $\beta = 1$, we witness a $35\%$ reduction in $T_{\text{EMV}}$ when secondary agents effortlessly reserve the segments. Likewise, $T_{\text{avg}}$ hits the smallest value when secondary agents shift their responsibility onto congestion reduction entirely. Under this scenario, pre-emption only happens on the segment which EMV is currently traveling on and the EMV will travel with the slowest speed since more vehicles appear ahead of its road. Although every non-EMV saves about 10 seconds for the whole trip in this case, an ambulance might delay for more than one minute and a precious life might be lost.
We believe it would be the best for the traffic system administrator to decide $\beta$ given the emergency of the case.
}