Emergency Response Time (ERT) serves as a pivotal metric of urban resiliency and public safety, encapsulating a city’s capacity to respond promptly and effectively to medical, fire, and crime emergencies. Over the last decade, increasing urbanization and traffic congestion have severely exacerbated ERT, undermining emergency management systems and public trust. In New York City (NYC), average ERT for medical emergencies has surged by 72\% from 7.89 minutes in 2014 to 14.27 minutes in 2024, with over 50\% of delays attributable to prolonged Emergency Vehicle (EMV) travel times. These escalating delays have dire consequences: every minute of delay during a stroke can lead to the loss of 2 million brain cells, and survival rates for cardiac arrest diminish by 7--10\% per minute. This underscores the urgency of addressing EMV travel inefficiencies through advanced routing and traffic management strategies.

This dissertation responds to these critical challenges by advancing the state of EMV facilitation through three substantive contributions. First, it introduces \textit{EMVLight}, a decentralized multi-agent reinforcement learning (MARL) framework that dynamically integrates EMV routing and adaptive traffic signal pre-emption. \textit{EMVLight} leverages real-time traffic data and decentralized coordination between agents to optimize EMV travel paths while reducing disruptions to non-EMV traffic. Experimental results demonstrate a 42.6\% reduction in EMV travel times and a 23.5\% decrease in the average travel time for non-EMVs across synthetic and real-world traffic networks.

Second, the dissertation develops the \textit{Dynamic Queue-Jump Lane (DQJL)} system, which employs Multi-Agent Proximal Policy Optimization (MAPPO) to enable coordinated lane-clearing maneuvers in mixed-traffic environments comprising autonomous and human-driven vehicles. By dynamically forming queue-jump lanes in response to real-time traffic conditions, DQJL minimizes lane-change maneuvers and reduces EMV travel times by up to 40\%, with benefits amplified under higher autonomous vehicle penetration rates.

Third, the dissertation conducts an equity-focused evaluation of Emergency Medical Services (EMS) accessibility in NYC, integrating demographic, infrastructural, and traffic data to uncover disparities in response times across boroughs. Findings reveal systemic inequities, with Staten Island experiencing the longest delays due to sparse signalized intersections, while Manhattan, despite its dense network, faces severe congestion. The study proposes actionable interventions, including optimized EMS station placements, improved intersection configurations, and equitable resource allocation, to address these disparities.

Collectively, these contributions provide a robust foundation for enhancing EMV mobility, reducing response times, and ensuring equitable access to emergency services. The methodologies and insights presented herein offer significant implications for policymakers, transportation engineers, and urban planners, advancing the development of safer, more efficient, and resilient urban transportation systems.

\textbf{Keywords:} Emergency Response Time, Multi-Agent Reinforcement Learning, Traffic Signal Pre-emption, Dynamic Queue-Jump Lane, Emergency Medical Services, Urban Resiliency, Transportation Equity.