\section{Literature Review}
\label{sec:literature_review}
In this section, we discuss literature available on queue jump lane in Subsec.~\ref{sec:dqjl_literature}, and existing multi-agent deep reinforcement learning techniques on similar connected vehicle applications in Subsec. \ref{sec:marl_literature}.

\subsection{Queue Jump Lane for EMVs}
\label{sec:dqjl_literature}
Deep learning has recently demonstrated significant promise across various applications \cite{you2019ct,you2018structurally,lyu2018super,lyu2019super,you2019low,you2020unsupervised,guha2020deep}.
Queue jump lanes (QJLs) have never been used for EMV deployment and represent a novel operational strategy that leverages CAV technologies for EMV deployment. Although QJLs are relatively new, the literature already demonstrates their positive effects in reducing travel time variability, particularly when combined with transit signal priority (TSP). However, these studies are primarily based on moving-bottleneck models for buses \cite{Zhou2005Performance,Cesme2015Queue,cheng2017body,cheng2016hybrid}. We adapt this bus operation strategy for EMV deployment in our setting, given that EMVs typically travel faster than non-EMVs and can “preempt” non-EMV traffic due to their priority status.

At the same time, traditional siren technologies often fail to provide adequate warning time for other vehicles, and there is frequently a lack of clarity regarding which route should be avoided \cite{chen2021adaptive,you2021mrd,liu2021aligning}. This confusion, particularly under highly congested conditions, leads to increased delays, as mentioned earlier, and contributes to accident rates that are 4 to 17 times higher \cite{Buchenscheit2009AVE,cheng2016identification}, as well as increased collision severity \cite{Yasmin2012Effects,cheng2016random,li2019novel}, both of which can further delay EMV response times. A study by Savolainen et al.~\cite{Savolainen2010Effects}, utilizing data from Detroit, Michigan, confirmed the sensitivity of driver behaviors to various ITS communication methods for conveying EMV route information. 
Kohneh et al.~\cite{Kohneh2024twoways} adopts a hybrid non-dominated sorting genetic algorithm II-particle swarm optimization (NSGAII-PSO) approach to solve the bi-objective optimization problem which minimizes EMV travel time while maximizing safety.

Wu et al.~\cite{WU2020preclearing} proposed a lane pre-clearing strategy for normal road segments, employing cooperative driving with CAVs to maintain EMV speed while minimizing traffic disruption. Their approach formulates the problem as a bi-level optimization and employs an EMV sorting algorithm, effectively clearing lanes and identifying a linear relationship between optimal solutions and road density to improve EMV routing decisions. Hannoun et al.~\cite{Hannoun2019Facilitating} utilized mixed-integer linear programming (MILP) to develop a static path clearance control strategy for approaching EMVs. A follow-up work~\cite{hannoun2021sequential}, relying on V2X, extended the problem to larger distances by dividing roads into segments, allowing for a divide-and-conquer approach to optimization. These approaches simplify the road network into cells, approximating vehicle positions and velocities, limiting its precision and applicability to complex real-world traffic scenarios. Moreover, the computational complexity of MILP-based approaches makes them less suited for real-time applications in dynamic traffic settings. 

QJLs have not yet been studied as a dynamic control strategy, underscoring the need to account for uncertainties in realistic, non-deterministic traffic conditions, such as yielding to an approaching EMV \cite{liu2021auto,you2022megan,liu2022retrieve}. To address this gap, we introduce the concept of dynamic queue jump lanes (DQJLs), as illustrated in Fig.~\ref{fig:dqjl}. During the DQJL clearing process for an approaching EMV, non-EMVs are continuously monitored and instructed, enabling the rapid and safe establishment of DQJLs. Additionally, we explore the application of DQJLs in partially connected scenarios, where only a subset of vehicles are equipped with communication devices, allowing for partial communication between vehicles and infrastructure.

\begin{figure}[htbp]
    \centering
    \begin{subfigure}{0.7\textwidth}
        \centering
        \includegraphics[width=\linewidth]{Figures/dqjl_before.png}
        \caption{Actions planning.}
    \end{subfigure}

    \vspace{0.2cm} % Adds some vertical space between the figures

    \begin{subfigure}{0.7\textwidth}
        \centering
        \includegraphics[width=\linewidth]{Figures/dqjl_after.png}
        \caption{DQJL established.}
    \end{subfigure}
    \caption{Illustration of a DQJL formation: Green vehicles are CAVs and yellow ones are HDVs. HDVs yields and CAVs perform coordinated lane changes for a DQJL.}
    \label{fig:dqjl}
\end{figure}

A related task, referred to as corridor clearance, is investigated by Suo et al.~\cite{suo2024model,liu2020rethinking}, who propose a proximal policy optimization (PPO) method to enhance clearance between intersections. In their approach, the corridor is formed by designating a single CAV as a "splitting point", which blocks following traffic to allow the EMV to switch into a less congested lane. However, their study adopts a single-agent perspective, focusing solely on the interaction between a CAV and an EMV, thereby limiting the collaborative potential of multiple CAVs in the system. Additionally, the simulation was conducted on a simple two-lane road segment, failing to capture the more complex interactions between CAVs and background traffic in multi-lane environments.
This "splitting point" design, however, suggests that DQJLs do not always have to be a straight line. Instead, they can consist of different lanes that require minimal lane-change maneuvers by the EMV. While this might slightly increase the EMV's passage time, it minimizes disruption to the surrounding traffic, see Fig.~\ref{fig:lane_change_dqjl}.

\begin{figure}[h]
    \centering
    \begin{subfigure}{0.7\textwidth}
        \centering
        \includegraphics[width=\linewidth]{Figures/alt_dqjl_before.png}
        \caption{Actions planning.}
    \end{subfigure}

    \vspace{0.2cm} % Adds some vertical space between the figures

    \begin{subfigure}{0.7\textwidth}
        \centering
        \includegraphics[width=\linewidth]{Figures/alt_dqjl_after.png}
        \caption{DQJL established.}
    \end{subfigure}
    \caption{An alternative DQJL formation, where the EMV will change lane once.}
    \label{fig:lane_change_dqjl}
\end{figure}

\subsection{Multi-agent Reinforcement Learning for CAVs}\label{sec:marl_literature}
The proliferation of V2X communication technologies has significantly increased the presence of CAVs, driving extensive research into multi-agent reinforcement learning (MARL) frameworks for vehicular applications \cite{zhou2023survey,henao2023multiclass,liu2023medical,cai2023unsupervised}. In these systems, individual vehicles are modeled as autonomous agents collaborating toward shared objectives. This paradigm offers several advantages, including decentralized decision-making, scalability, adaptability to dynamic traffic conditions, and the capacity for collaborative learning among vehicles.

MARL in CAV scenarios presents a promising approach to addressing complex transportation challenges, particularly where traditional methods have shown limitations. Recent studies have applied MARL to various aspects of traffic management and vehicle coordination, highlighting its versatility and effectiveness. For instance, in the realm of congestion mitigation, Ha et al.~\cite{ha2020leveraging} developed a MARL-based control model using graph convolutional networks (GCN) and deterministic policy gradient (DDPG) techniques for highway scenarios, outperforming traditional speed harmonization methods even with low CAV penetration rates. Similarly, Nakka et al.~\cite{Nakka2022Merging} applied a decentralized multi-agent deterministic policy gradient (MADDPG) approach to highway merging, successfully eliminating stop-and-go traffic.

Intersection management has also been a key focus for MARL applications \cite{you2023rethinking,liu2023benchmarking,yan2024after,han2024hybrid,ma2024segment}. Guo et al.~\cite{guo2022coordination} employed an enhanced monotonic value function factorisation (QMIX) algorithm to coordinate CAVs at unsignalized intersections in mixed-autonomy traffic, while Antonio et al.~\cite{antonio2022multi} introduced an advanced autonomous intersection management (AIM) system using MARL and Curriculum through Self-Play, significantly surpassing traditional traffic light systems in performance.

In vehicle control, Zhou et al.~\cite{zhou2022multi} proposed an MA2C method for lane-changing decisions in mixed traffic environments, utilizing a multi-objective reward function. Shi et al.~\cite{SHI2021Connected} presented a cooperative longitudinal control strategy for CAVs in mixed traffic, improving both string stability and efficiency. Additionally, Chen et al.~\cite{Chen2023OnRamp} addressed mixed-traffic highway on-ramp merging with a MARL framework, enabling autonomous vehicles to collaborate and adapt to human-driven vehicles (HDVs).

Beyond traffic flow management, MARL has demonstrated potential in other CAV applications \cite{liang2024rescuing,cao2024multi,you2024calibrating}. Zhang et al.~\cite{ZHANG2022parking} developed a MARL framework for online parking assignment in mixed CV and non-connected vehicle (NCV) environments, achieving substantial improvements over conventional methods. In the domain of robotic driving simulation, Palanisamy et al.~\cite{Palanisamy2020Driving} introduced MACAD-Gym, a multi-agent simulation platform for CAVs, which supports research into deep reinforcement learning (DRL)-based algorithms in complex, multi-agent environments beyond geo-fenced operational design domains.

Parada et al.~\cite{parada2023safe} have proposed a MAPPO learning framework for autonomous vehicle maneuvering in the presence of EMVs. However, this study assumes a fully autonomous traffic setting and focuses primarily on collision risk avoidance. While effective for collision minimization, this design de-prioritizes the reduction of EMV passage time, which is a critical factor in emergency response. Such an approach may also not be feasible for near-term deployment in mixed autonomy environments.

These diverse applications of MARL in CAV contexts underscore its potential to revolutionize traffic management and vehicle coordination \cite{you2024mine,sun2024medical,huang2024cross}. For those interested in a more comprehensive review of MARL frameworks for CAV applications in mixed traffic environments, Yadav et al.~\cite{yadav2023comprehensive} and Dinneweth et al.~\cite{Dinneweth2022Survey} provide valuable insights into the current state of the art and emerging trends in this rapidly advancing field.

Despite the extensive research on MARL in CAV applications, there is currently no work that employs MARL and focuses on pre-clearing lanes for EMVs in heterogeneous traffic settings. This gap presents an exciting opportunity for this research to explore how MARL can be leveraged to optimize the formation and management of DQJLs, potentially resulting in more efficient and adaptive traffic management strategies in urban environments.