\section{Related Works}
The modeling of soft actuators, particularly in underwater environments, presents significant challenges due to the complexity of soft materials \cite{muhammad2014non}. Traditional physics-based or analytical methods often struggle to accurately capture the complex fluid-structure interactions between soft fins and the surrounding water \cite{singh2019dynamic}. This complexity is further compounded by the complex dynamics that arises from the interaction of flexible fins with water. Early studies on fin actuators based on specially designed oscillate motion with a mathematical model \cite{gkliva2018development}. However, recent advances in data-driven modeling, particularly through machine learning techniques, have yielded more accurate and flexible dynamic models \cite{lee2023data}, \cite{9652036}. These models, such as neural networks, can be trained on empirical data, allowing them to learn the complex, nonlinear behavior of soft actuators without the need for explicit mathematical models, significantly improving adaptability to diverse environmental conditions.

Reinforcement Learning (RL) has also emerged as an effective approach to develop controllers in systems where traditional control methods struggle, especially in complex environments \cite{tong2023survey}, \cite{6315769}. RL-based controllers allow systems to learn optimal policies through trial and error, guided by environment feedback \cite{singh2022reinforcement}. However, in the context of soft robotics, particularly soft underwater fin actuators, the highly complex dynamics pose significant challenges to the direct application of RL. To address these challenges, the integration of a surrogate models within RL frameworks has gained increasing attention, particularly in large-scale control problems such as plant energy management \cite{pinto2021data}, \cite{wang2021surrogate}. Surrogate models, often constructed using neural networks, approximate the behavior of complex systems by predicting outcomes based on precollected data \cite{hou2022dimensionality}. 

In this paper, we apply both surrogate modeling and RL techniques to develop a robust controller for soft-fin-actuated underwater robots. By integrating these approaches, we achieve optimal force generating motion for soft fin actuators.