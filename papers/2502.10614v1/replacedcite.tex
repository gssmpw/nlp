\section{Related Work}
The application of deep learning techniques, particularly CNNs, to medical imaging has been extensively explored in recent years. One of the pioneering studies in this domain was by Rajpurkar et al. ____, where a deep CNN model was developed to diagnose pneumonia from chest X-rays, outperforming radiologists in some cases. In a similar vein, ____ addressed multi-label classification problems in medical images, specifically in classifying chest X-rays into multiple disease categories. However, one of the most persistent challenges in this area is the inherent class imbalance in the datasets, where rare diseases are underrepresented, leading to biased predictions ____.

Recent work by ____ proposed an advanced multi-label classification framework designed specifically for medical image datasets, allowing for the simultaneous prediction of multiple diseases in a single X-ray image. This framework addressed the problem of missing data by employing advanced imputation techniques. Additionally, data augmentation and synthetic data generation have been explored to balance the dataset and improve model robustness ____. While these methods have been shown to enhance performance in some cases, they still face limitations, particularly in high-dimensional medical image datasets like ChestX-ray14.

In this study, we build upon these existing works by integrating class-weighted loss functions and leveraging PCA-based image compression to address dataset imbalance and reduce computational complexity. We also propose the inclusion of patient demographic information and additional metadata to improve the model's ability to generalize across diverse patient populations and imaging conditions.