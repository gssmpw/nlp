\clearpage
\appendix
\setcounter{page}{1}
\counterwithin{figure}{section}
\counterwithin{table}{section}
\counterwithin{equation}{section}

\maketitlesupplementary



\section{Principal Component Analysis}
Our analysis reveals that concepts frequently encountered in training data (e.g., "person") exhibit greater variation compared to concepts that are either less diverse or less common (e.g., "Van Gogh art" or "waterfalls"). We demonstrate this by analyzing the principal components for each concept through PCA visualization in Figure~\ref{fig:pca}. Notably, the 50th principal component for the "person" concept shows comparable variational magnitude to the 20th component of "waterfalls," highlighting the inherent variational differences across concepts. By discovering and uniformly sampling these variations, we effectively address the mode collapse problem in models, as shown in Figures~\ref{fig:conceptdiverse1} and~\ref{fig:conceptdiverse2}.
\begin{figure}[!htpb]
    \centering
    \includegraphics[width=.8\linewidth]{sup-figures-src/pca_values-cropped.pdf}
    \caption{Common concepts like ``person'' show higher variation in CLIP space compared to rarer concepts like ``waterfalls''. 50th PCA component of ``person'' matches the 20th component of ``waterfalls,'' indicating the latter's more limited variation.} 
    \label{fig:pca}
\end{figure}
\vspace{-2em}

\section{Effect of TimeStep during Inference}
The temporal application of sliders during inference significantly impacts both the precision and magnitude of image edit(Fig.~\ref{fig:timestep}) for SDXL-DMD SliderSpace. When sliders are applied at all timesteps during inference, we observe strong semantic and structural changes in the generated image. But applying the slider after a few steps helps preserve the image structure while still enabling controlled edits. This latter approach facilitates more precise editing, albeit with subtler semantic alterations that can be amplified by increasing the slider strength parameter.

% Add before the Interpretability & Control section

\subsection{Choice of Semantic Embeddings}
\label{sec:semantic}
While our primary implementation uses CLIP embeddings for semantic decomposition, SliderSpace is compatible with various semantic encoders. Our experiments with alternative embeddings like DINO-v2 and FaceNet demonstrate the framework's flexibility. As shown in Figure~\ref{fig:reb-concept}. DINO-v2 shows comparable overall performance to CLIP, with each encoder exhibiting different strengths across various concepts. For person-specific concepts, using FaceNet embeddings enables the discovery of fine-grained facial semantic directions as seen in Figure~\ref{fig:transfer}

The choice of encoder can be tailored to the target domain - CLIP for general concepts, DINO-v2 for certain visual attributes, and specialized encoders like FaceNet for domain-specific applications. This flexibility allows SliderSpace to adapt to different use cases while maintaining its core benefits of unsupervised discovery and semantic consistency.

\begin{figure*}[!ht]
    \centering
    \includegraphics[width=\linewidth]{rebuttal-figures-src/concept.pdf}
    \vspace{-1.5em}
    \caption{SliderSpace shows similar diversity and text alignment when using either Dino-V2 or CLIP embeddings for PCA analysis.}
    \vspace{-0.5em}
    \label{fig:reb-concept}
\end{figure*}


\subsection{Hyperparameter Analysis}
\label{sec:hyperparam}
We analyze the impact of two key hyperparameters in SliderSpace: the number of PCA directions and the LoRA rank. Our experiments reveal that increasing PCA directions improves both knowledge coverage and output diversity up to about 40 dimensions, after which returns diminish. With just 10 directions, SliderSpace matches the FID scores of 64 manually created Concept Sliders when evaluated against artistic style distributions. Regarding model architecture, we find that lower-rank adaptors (particularly rank-one) efficiently capture variations with a fixed training budget, outperforming higher-rank versions while maintaining better FID scores than Concept Sliders across different ranks.

This analysis guides our choice of using rank-one adapters with 40 PCA directions as the default configuration, offering an optimal balance between performance and computational efficiency.

\begin{table}[H]
\vspace{-3mm}
  \caption{Hyperparameters for main results.}
  
  \centering
    \small
  \begin{tabular}{l|c|c}

    \toprule 
    Model & Hyperparameter Type & Value\\
    \midrule
   \multirow{ 2}{*}{PINN} & network depth & 4\\
   & network width & 512 \\
    \midrule
   \multirow{ 2}{*}{QRes}& network depth & 4 \\
    & network width & 256 \\
    \midrule
    \multirow{ 3}{*}{KAN}   & network width  & [2,5,5,1] \\
     & grid size & 5\\
       & grid\_epsilon  & 1.0 \\
       \midrule
        \multirow{ 7}{*}{PINNsFormer}   & \# of encoder  & 1 \\
     & \# of decoder & 1\\
       & embedding size  & 32 \\
        & attention head  & 2 \\
                & MLP hidden width  & 512 \\
            & sequence length $k$  & 5 \\
            & sequence interval $\Delta t$  & 1e-4 \\
            \midrule
            \multirow{ 7}{*}{PINNMamba}   & \# of encoder  & 1 \\
       & embedding size  & 32 \\
        & $\Delta,B,C$ width & 8 \\
                & MLP hidden width  & 512 \\
            & sequence length $k$  & 7 \\
            & sequence interval $\Delta t$  & 1e-2 \\
                
       

    \bottomrule
  \end{tabular}
  \normalsize
  \label{tab:hyperpara}

\end{table}


\section{User Study}
We conducted user studies to evaluate SliderSpace's effectiveness through Amazon MTurk. For artistic evaluation (Sec~\ref{sec:art_exp}), participants compared two 9-image grids - one generated by SliderSpace using 3 random sliders per image, and another by our baselines. Both sets used identical base prompts: ``a building in a stunning landscape'' and ``a character in a scenic environment''. As shown in Fig~\ref{fig:user_art}, participants rated which grid exhibited greater artistic diversity and utility for art applications. For conceptual evaluation (Sec~\ref{sec:concept_exp}), participants compared image grids based on diversity, generative utility, and creativity (Fig~\ref{fig:user_concept}). Grid presentation order was randomized across all experiments.

\section{Qualitative Results}
\subsection{Art Exploration}
We identify the top-36 distinct art directions discovered by SDXL-DMD2 SliderSpace for the concept "artwork in the style of famous artist" in Figures~\ref{fig:artdmd1} and~\ref{fig:artdmd2}. Additionally, we showcase various combinations of SliderSpace samples in Figures~\ref{fig:artbuilding} and~\ref{fig:artcharacter}, where we randomly sample three sliders to generate images for both characters and buildings (used in our art experiments and user studies in Section~\ref{sec:art_exp}). The top 18 art styles discovered by SDXL-SliderSpace are presented in Figure~\ref{fig:artsdxl}.



\subsection{Diversity Enhancement}
We provide additional qualitative examples demonstrating how our generic diversity sliders mitigate mode collapse in distilled models. Our observations indicate that distilled models such as DMD2~\cite{dmd} tend to generate visually similar images for identical prompts, despite different random seed initializations. Through our trained diversity sliders, which are model-agnostic, we successfully counter mode collapse (Section~\ref{sec:diverse_exp}). As quantitatively validated in Table~\ref{tab:diverse}, the diversity SliderSpace significantly improves image variation, achieving FID scores comparable to the base model.

\subsection{Concept Decomposition}
We present qualitative examples of concept decomposition using the SDXL-DMD2~\cite{dmd} SliderSpace in Figures~\ref{fig:concept1}--\ref{fig:concept6}. Furthermore, we demonstrate SliderSpace's versatility across various models, including SDXL-Turbo~\cite{turbo} (Figures~\ref{fig:turbo1}, SDXL-Base~\cite{podell2023sdxl} (Figures~\ref{fig:artsdxl}), and the state-of-the-art transformer-based FLUX Schnell models (Figures~\ref{fig:intro} and~\ref{fig:flux}). We note that Claude3.5~\cite{claude} generated captions are not always accurate. For instance, in Figure~\ref{fig:concept1}, Claude annotates one of the sliders as ``Black Lab Technician'', but it is not visually distinct whether the slider is `lab technician' or a `scientist'.

\CatchFileDef{\resultsAblationsScaleSmall}{tables/results_ablations_scale_small.tex}{}
\CatchFileDef{\resultsAblationsSGDAll}{tables/results_ablations_sgd_all.tex}{}

\section{Ablations}
\label{sec:ablations}

We perform some additional experiments to shed further light on how Coupled Adam works. A model size of $N = 125\M$ and the dataset sizes $D \in \{ 5\B, 10\B, 20\B \}$ from the small-scale experiments (Sec.~\ref{sec:experiments_S}) are used, and each experiment is repeated $S = 3$ times with different seeds. 

\subsection{Scaled Coupled Adam}
\label{sec:ablation_different_learning_rate}

While coupling the second moment of the embedding gradients using the average in Eq.~(\ref{eq:optimizer_update_second_moment_avg}) is the canonical choice, one could also use a multiple of the average. We conduct additional experiments where the coupled second moment is scaled by powers of $2$:
\begin{equation}
\secondmomentavg \: \to \: 2^{-n} \cdot \secondmomentavg  \; ,
\label{eq:optimizer_update_second_moment_avg_scaled}
\end{equation}
with scaling exponents
$n \in \{ z \in \mathbb{Z} ~| -5 \leq z \leq 5 \}$.
Note that using a scaling exponent $n \neq 0$ is equivalent to using a different effective learning rate for the embeddings than for all the other parameters, via Eqs.~(\ref{eq:optimizer_update_second_moment_avg}) and (\ref{eq:adam_learning_rate}). In particular, a smaller scaling exponent $n$ corresponds to a smaller effective learning rate and vice versa. 
The results for $D=20\B$ are shown in Tab.~\ref{tab:results_ablations_scale}, 
and the dependency of the loss on the scaling exponent $n$ for that very dataset size is visualized in Fig.~\ref{fig:ablation_scale}.
\begin{figure*}[t]
\begin{minipage}{\textwidth}
  \begin{minipage}[b]{0.63\textwidth}
    \centering
    \scriptsize
    \begin{tabular}{c|rrrrrrrr}
    \toprule
    $n$ & $\Loss$ ($\downarrow$) & $\Acc$ ($\uparrow$) & $\ISO$ ($\uparrow$) & $\munorm$ ($\downarrow$) & $\munormrel$ ($\downarrow$) & $\rcos$ ($\uparrow$) & $\rho$ ($\uparrow$) & $\kappa$ ($\uparrow$) \\ 
    \midrule
    \resultsAblationsScaleSmall
    \bottomrule 
    \end{tabular}
    \captionof{table}{Results of our experiments with Scaled Coupled Adam, for $N=125\M$ and $D=20\B$. Values are highlighted in bold if they are significantly better than \textit{all} the other values in the same column, see the caption of Tab.~\ref{tab:results_S} for more details.}
    \label{tab:results_ablations_scale}
  \end{minipage}
  \hfill
  \begin{minipage}[b]{0.35\textwidth}
    \centering
    \includegraphics[scale=0.40]{figs/ablation_1_loss_20B.png}
    \captionof{figure}{Dependency of the loss on the scaling exponent $n$, see Eq.~(\ref{eq:optimizer_update_second_moment_avg_scaled}), for $N=125\M$ and $D=20\B$. The plot shows the difference to the loss obtained for $n = 0$.}
    \label{fig:ablation_scale}
  \end{minipage}
\end{minipage}
\end{figure*}
Results for other dataset sizes and plots for the other evaluation metrics can be found in App.~\ref{app:additional_results_scale}.
Our data shows that the loss reaches a minimum close to $n=0$, with a rather weak dependence on the scaling exponent in its vicinity. Nevertheless, for the smallest and largest scaling exponents studied, we find that the loss gets significantly worse.
Regarding downstream performance, we see indications of a similar pattern, although the statistical uncertainties are too large to draw definite conclusions.  
The semantic usefulness of the embedding vectors as measured by $\rcos$ seems to suffer from a scaling exponent $n < 0$. For the isotropy and the mean embedding, we observe the opposite behavior. They benefit from a smaller scaling exponent $n$ and the associated smaller embedding updates, with the effect being more pronounced the larger the training dataset size $D$.  
However, this also negatively affects the model performance. 
Hence, we conclude that, at least within the range of our experiments, the optimal setting is to have the same learning rate for the embedding parameters as for all the other model parameters, as implied by $n=0$ and Eq.~(\ref{eq:optimizer_update_second_moment_avg}).

\subsection{SGD}
\label{sec:ablation_sgd}

We train several models using SGD with momentum $\gamma = 0.9$ as the optimizer for the embeddings. Since Adam via the inverse square root of its second moment effectively scales the learning rate up by a factor comprising orders of magnitude (see Eq.~(\ref{eq:second_moment_global_factor})), we explicitly multiply the learning rate in SGD by a factor $f$ of comparable size\footnote{Note that the difference between momentum in SGD and the first moment in Adam also plays a role here.}. A hyperparameter search using 
$f \in \{ 100, 200, 300, 400, 500, 600 \}$
is performed to search for the optimum with respect to upstream performance (loss), see App.~\ref{app:additional_results_sgd} for details. It is found at $f = 300$ for $D \in \{ 5\B, 10\B \}$ and $f = 400$ for $D = 20\B$.
The respective optimal model is compared to its counterpart trained with Coupled Adam in Tab.~\ref{tab:results_ablations_sgd_all}.
\begin{table*}[htb]
\centering
\scriptsize
\begin{tabular}{ccc|rrrrrrrr}
\toprule
$D$ & $N$ & Optimizer & $\Loss$ ($\downarrow$) & $\Acc$ ($\uparrow$) & $\ISO$ ($\uparrow$) & $\munorm$ ($\downarrow$) & $\munormrel$ ($\downarrow$) & $\rcos$ ($\uparrow$) & $\rho$ ($\uparrow$) & $\kappa$ ($\uparrow$) \\ 
\midrule
\resultsAblationsSGDAll
\bottomrule 
\end{tabular}
\caption{Comparison of models whose embeddings were trained with SGD and Coupled Adam. The SGD models were obtained after hyperparameter search for the learning rate. The associated factor $f$ is specified in parentheses in the Optimizer column. Bold values indicate better results with statistical significance, see the caption of Tab.~\ref{tab:results_S} for more details.}
\label{tab:results_ablations_sgd_all}
\end{table*}
The results show that, although SGD is advantageous with respect to isotropy, the mean embedding shift and the condition number, Coupled Adam consistently achieves better results on all upstream and downstream task metrics, while having one less hyperparameter to fine-tune.


\begin{figure*}
    \centering
    \includegraphics[width=.8\linewidth]{sup-figures-src/concept_diverse_vangogh-cropped.pdf}
    \caption{We show a few possible variations possible with SliderSpace directions. For a given seed and prompt, users can sample different combinations of sliders from SliderSpace and generate unique and diverse outputs (all variations from a single prompt and seed). We show this for the concept ``Van Gogh'' SliderSpace on SDXL-DMD2~\cite{dmd}.} 
    \label{fig:conceptdiverse1}
\end{figure*}

\begin{figure*}
    \centering
    \includegraphics[width=.8\linewidth]{sup-figures-src/concept_diverse_toy-cropped.pdf}
    \caption{We show a few possible variations possible with SliderSpace directions. For a given seed and prompt, users can sample different combinations of sliders from SliderSpace and generate unique and diverse outputs (all variations from a single prompt and seed). We show this for the concept ``Toy'' SliderSpace on SDXL-DMD2~\cite{dmd}.} 
    \label{fig:conceptdiverse2}
\end{figure*}
 
\begin{figure*}
    \centering
    \includegraphics[width=\linewidth]{sup-figures-src/timestep_dmd-cropped.pdf}
    \caption{The choice of timestep at which sliders are applied can have an effect on the preciseness of the sliders. We show that when the sliders are applied to all the timesteps in inference, the images look different from the original models images for the same prompt and seed. But skipping the first timestep can lead to precise edits (similar observations as \citep{gandikota2023concept})} 
    \label{fig:timestep}
\end{figure*}


% \begin{figure*}[!htp]
%     \centering
%     \includegraphics[width=\linewidth]{sup-figures-src/slider_timestep-cropped.pdf}
%     \caption{The choice of timestep at which sliders are applied can have an effect on the preciseness of the sliders. We show that when the sliders are applied to all the timesteps in inference, the images look different from the original models images for the same prompt and seed. But skipping the first few timestep can lead to precise edits (similar observations as \citep{gandikota2023concept}). We show the results for SDXL SliderSpace~\cite{podell2023sdxl}} 
%     \label{fig:sdxltimestep}
% \end{figure*}



\begin{figure*}[!htbp]
    \centering
    \includegraphics[width=.8\linewidth]{sup-figures-src/artsliders_individual-cropped.pdf}
    \caption{We show the top 18 art directions that are discovered in the SDXL-DMD2~\cite{dmd} SliderSpace for the concept ``art''.} 
    \label{fig:artdmd1}
\end{figure*}

\begin{figure*}[!htbp]
    \centering
    \includegraphics[width=.8\linewidth]{sup-figures-src/artsliders_individual2-cropped.pdf}
    \caption{We show the top 18-36 art directions that are discovered in the SDXL-DMD2~\cite{dmd} SliderSpace for the concept ``art''.} 
    \label{fig:artdmd2}
\end{figure*}

\begin{figure*}[!htbp]
    \centering
    \includegraphics[width=.8\linewidth]{sup-figures-src/SDXL_artsliders_individual-cropped.pdf}
    \caption{We show the top 18 art directions that are discovered in the SDXL~\cite{podell2023sdxl} SliderSpace for the concept ``art''.} 
    \label{fig:artsdxl}
\end{figure*}

\begin{figure*}[!htbp]
    \centering
    \includegraphics[width=.8\linewidth]{sup-figures-src/artsliders_sample_building-cropped.pdf}
    \caption{We show samples from our art experiments ~\ref{sec:art_exp}. We sample random 3 sliders from the SDXL-DMD2~\cite{dmd} SliderSpace for the concept ``art'' and generate images for the prompt ``a building in a stunning landscape the style of a famous artist''.} 
    \label{fig:artbuilding}
\end{figure*}

\begin{figure*}[!htbp]
    \centering
    \includegraphics[width=.8\linewidth]{sup-figures-src/artsliders_sample_character-cropped.pdf}
    \caption{We show samples from our art experiments ~\ref{sec:art_exp}. We sample random 3 sliders from the SDXL-DMD2~\cite{dmd} SliderSpace for the concept ``art'' and generate images for the prompt ``a character in a scenic environment the style of a famous artist''.} 
    \label{fig:artcharacter}
\end{figure*}

\begin{figure*}
    \centering
    \includegraphics[width=.95\linewidth]{figures-src/diverse.pdf}
    \caption{SliderSpace can also decompose general broad visual variation of diffusion model's and can be used to overcome model collapse in distilled models. We randomly sample a sparse set of sliders and generate the sample showing higher variation than base distilled model.}
    \label{fig:diverse}
\end{figure*}



\begin{figure}[htp]
  \centering
   \includegraphics[width=\columnwidth]{Assets/userstudy_grid.pdf}
   
   \caption{\textbf{User study results.} Users preference percentage of our method compared to other methods in terms of text alignment, visual quality, and overall preference.
   }
   \label{fig:user_study}
\end{figure}

\begin{figure*}
    \centering
    \includegraphics[width=\linewidth]{sup-figures-src/dmd_concept_scientist-cropped.pdf}
    \caption{We show the SliderSpace discovered in SDXL-DMD2 4-step model~\cite{dmd} for the concept ``Scientist''} 
    \label{fig:concept1}
\end{figure*}

\begin{figure*}
    \centering
    \includegraphics[width=\linewidth]{sup-figures-src/dmd_concept_coral-cropped.pdf}
    \caption{We show the SliderSpace discovered in SDXL-DMD2 4-step model~\cite{dmd} for the concept ``Coral''} 
    \label{fig:concept2}
\end{figure*}

\begin{figure*}
    \centering
    \includegraphics[width=\linewidth]{sup-figures-src/dmd_concept_cowboy-cropped.pdf}
    \caption{We show the SliderSpace discovered in SDXL-DMD2 4-step model~\cite{dmd} for the concept ``Cowboy''} 
    \label{fig:concept3}
\end{figure*}

\begin{figure*}
    \centering
    \includegraphics[width=\linewidth]{sup-figures-src/dmd_concept_dragon-cropped.pdf}
    \caption{We show the SliderSpace discovered in SDXL-DMD2 4-step model~\cite{dmd} for the concept ``Dragon''} 
    \label{fig:concept4}
\end{figure*}

\begin{figure*}
    \centering
    \includegraphics[width=\linewidth]{sup-figures-src/dmd_concept_mask-cropped.pdf}
    \caption{We show the SliderSpace discovered in SDXL-DMD2 4-step model~\cite{dmd} for the concept ``Mask''} 
    \label{fig:concept5}
\end{figure*}

\begin{figure*}
    \centering
    \includegraphics[width=\linewidth]{sup-figures-src/dmd_concept_toy-cropped.pdf}
    \caption{We show the SliderSpace discovered in SDXL-DMD2 4-step model~\cite{dmd} for the concept ``Toy''} 
    \label{fig:concept6}
\end{figure*}

\begin{figure*}
    \centering
    \includegraphics[width=\linewidth]{sup-figures-src/turbo_image_edit-cropped.pdf}
    \caption{We show the SliderSpace discovered in SDXL-Turbo 4-step model~\cite{turbo} and how they can be used for precise control of image generation} 
    \label{fig:turbo1}
\end{figure*}

% \begin{figure*}
    \centering
    \includegraphics[width=\linewidth]{sup-figures-src/turbo_concept_bird-cropped.pdf}
    \caption{We show the SliderSpace discovered in SDXL-Turbo 4-step model~\cite{turbo} for the concept ``Bird''} 
    \label{fig:turbo2}
\end{figure*}

% \begin{figure*}
    \centering
    \includegraphics[width=\linewidth]{sup-figures-src/turbo_concept_castle-cropped.pdf}
    \caption{We show the SliderSpace discovered in SDXL-Turbo 4-step model~\cite{turbo} for the concept ``Castle''} 
    \label{fig:turbo3}
\end{figure*}

% \begin{figure*}
    \centering
    \includegraphics[width=\linewidth]{sup-figures-src/turbo_concept_temple-cropped.pdf}
    \caption{We show the SliderSpace discovered in SDXL-Turbo 4-step model~\cite{turbo} for the concept ``Temple''} 
    \label{fig:turbo4}
\end{figure*}

\begin{figure*}
    \centering
    \includegraphics[width=\linewidth]{sup-figures-src/flux_sliders-cropped.pdf}
    \caption{We show the SliderSpace discovered in FLUX Schnell model~\cite{flux} for concepts ``monster'' and ``dog''} 
    \label{fig:flux}
\end{figure*}
.




