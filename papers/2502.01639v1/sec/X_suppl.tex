\clearpage
\appendix
\setcounter{page}{1}
\counterwithin{figure}{section}
\counterwithin{table}{section}
\counterwithin{equation}{section}

\maketitlesupplementary



\section{Principal Component Analysis}
Our analysis reveals that concepts frequently encountered in training data (e.g., "person") exhibit greater variation compared to concepts that are either less diverse or less common (e.g., "Van Gogh art" or "waterfalls"). We demonstrate this by analyzing the principal components for each concept through PCA visualization in Figure~\ref{fig:pca}. Notably, the 50th principal component for the "person" concept shows comparable variational magnitude to the 20th component of "waterfalls," highlighting the inherent variational differences across concepts. By discovering and uniformly sampling these variations, we effectively address the mode collapse problem in models, as shown in Figures~\ref{fig:conceptdiverse1} and~\ref{fig:conceptdiverse2}.
\begin{figure}[!htpb]
    \centering
    \includegraphics[width=.8\linewidth]{sup-figures-src/pca_values-cropped.pdf}
    \caption{Common concepts like ``person'' show higher variation in CLIP space compared to rarer concepts like ``waterfalls''. 50th PCA component of ``person'' matches the 20th component of ``waterfalls,'' indicating the latter's more limited variation.} 
    \label{fig:pca}
\end{figure}
\vspace{-2em}

\section{Effect of TimeStep during Inference}
The temporal application of sliders during inference significantly impacts both the precision and magnitude of image edit(Fig.~\ref{fig:timestep}) for SDXL-DMD SliderSpace. When sliders are applied at all timesteps during inference, we observe strong semantic and structural changes in the generated image. But applying the slider after a few steps helps preserve the image structure while still enabling controlled edits. This latter approach facilitates more precise editing, albeit with subtler semantic alterations that can be amplified by increasing the slider strength parameter.

% Add before the Interpretability & Control section

\subsection{Choice of Semantic Embeddings}
\label{sec:semantic}
While our primary implementation uses CLIP embeddings for semantic decomposition, SliderSpace is compatible with various semantic encoders. Our experiments with alternative embeddings like DINO-v2 and FaceNet demonstrate the framework's flexibility. As shown in Figure~\ref{fig:reb-concept}. DINO-v2 shows comparable overall performance to CLIP, with each encoder exhibiting different strengths across various concepts. For person-specific concepts, using FaceNet embeddings enables the discovery of fine-grained facial semantic directions as seen in Figure~\ref{fig:transfer}

The choice of encoder can be tailored to the target domain - CLIP for general concepts, DINO-v2 for certain visual attributes, and specialized encoders like FaceNet for domain-specific applications. This flexibility allows SliderSpace to adapt to different use cases while maintaining its core benefits of unsupervised discovery and semantic consistency.

\subsection{Relations vs. Concepts}
We saw in Figure    \ref{fig:neuron_overlap} that the
storage of relations is generally well separated, but that
there are exceptions. We can view a relation as relating two
\textbf{concepts}, e.g.,     
\texttt{company\_ceo} relates instances of the ``subject'' concept
``company'' to instances of the ``object'' concept ``CEO''. From this
perspective, the exceptions in Figure
\ref{fig:neuron_overlap}, i.e., cases where a relation $r_1$
overlaps with a relation $r_2$, are 
generally cases where the concepts of $r_1$ and $r_2$  are
the same or overlap. For example, \texttt{company\_ceo}
and \texttt{company\_hq} have the same subject concept.

\enote{fy}{concept ? or argument ? or entity ?}
\enote{yl}{i guess here we need to be a a bit more consistent. I would interpret the "concept" as an abstract notion that many "entities" belong to, e.g., CEO. I am not sure if the "argument" here means something similar, or, it means the actual entity, e.g., "Jason Huang" (an actual CEO).}

To further explore this hypothesis empirically, 
we again use the method applied in
\secref{method} to relations, but now use it for
concepts; that is
we identify sets of
\textbf{concept-specific neurons}.
We group the LRE dataset 
triples by 
subject concept,
resulting in 11 different concepts. 
We create a set of triples with novel relations
such as ``can'' and ``has a'', balanced across
positive and negative
samples. This ensures that
the model's completion for a prompt like
(``Lincoln has a'') depends on the concept instance
(``Lincoln''), not on the relation (``has a'').


Figure \ref{fig:entity_neuron_overlap} shows the
overlap between relation neurons and concept neurons for 
13B. Most of the cells with large counts support our
hypothesis that 
the overlaps we observe are rooted in relations being
representationally associated  with their subject and
object concepts.
Clear examples include
\texttt{company\_ceo} and its subject concept
\texttt{company}; 
\texttt{company\_hq} and its object concept 
\texttt{city} (assuming that \texttt{hq} is a subcategory of
\texttt{city});
and \texttt{landmark\_continent} and its subject concept
\texttt{landmark}. There is little overlap of
\texttt{person} with relations like \texttt{person\_mother},
potentially because \texttt{person} is a more general and
semantically unspecific concept than the others. Note that
several concepts do not match a specific relation, e.g.,
\texttt{superhero}, and therefore are not strongly associated
with any relation. Recall that we picked the concepts
according to the availability of annotated data in LRE.
However, most
identified neurons
are only concept neurons or only relation neurons,
suggesting that
relational and conceptual representations are largely separate.

%general relational activation
%patterns are mostly disjunct of the concepts the subject
%stands in connection to.



%associated with a specific
%concept and those specialized in relations. Some overlap is
%observed in both models, particularly between the neuron
%experts of a concept and a relation applied on subjects from
%that concept, this can be for example observed between the
%concept \texttt{star} and the corresponding relation
%\texttt{star\_constellation}. Additionally, concepts more
%closely associated with the notion of location, such as
%\texttt{city} and \texttt{landmark}, exhibit overlap with
%location-related relations, including
%\texttt{landmark\_country} and \texttt{company\_hq}.  This
%pattern is comparatively more prominent in the 13b model,
%which aligns to the previously discussed drop of accuracy
%for location-based relations within this model, as shown on
%the right side in Figure \ref{fig:inter-relation}.

\enote{hs}{didn't udnerstand this. reintgrate?
This
shows the connection between \textbf{neuron versatility} and
the occurrence of a shared concept represented over a
certain group of neurons for different relations.}

\enote{lh}{the thought was to bring together findings from the section on neuron versatility and this section. In figure 4. there is accuracy drop in many relations around the notion of location upon an ablation of such a relation. There is a strong overlap between those relations (as shown in figure 13) and now we observe the same pattern around concepts concerned with locations here.}




%Introduction sentence & Motication

%- relations and concepts can be closely related
%  we saw it in our expoerimetn with realtions
%  now we can explain that
%  we see dditional evidence for that
%- person is a very general concept
%- we have concepts where indstances occur but not
%as intsance of tht concept
%- we subject contecpts vs object condtpetws (city)
%some realtion very speicfic to spdecif cocnpets
%an d they are stored togegther
%- star\_constellation

%and analyze their overlap
%with relation-specific neurons to determine potential
%dependencies between them.
%where the concepts of $r_1$ and $r_2$  are
%the same or overlap.




%Method


%The evaluation is performed on a test set derived from factually correct prompts in the LRE dataset.


%Overlap results 



\begin{figure}[tb]
    \centering
   \setlength{\abovecaptionskip}{-0.05cm}
   \setlength{\belowcaptionskip}{-0.5cm}
%\includegraphics[width=0.46\textwidth]{figs/entity_neurons/neuron_overlaps_top3000_llama-2-7b-hf.pdf}
\includegraphics[width=0.42\textwidth]{figs/entity_neurons/neuron_overlaps_top3000_13b_new.pdf}
    \caption{Overlap between the top 3000 neurons of
      relations and  concepts in the 13B model.}
    \label{fig:entity_neuron_overlap}
\end{figure}



\subsection{Hyperparameter Analysis}
\label{sec:hyperparam}
We analyze the impact of two key hyperparameters in SliderSpace: the number of PCA directions and the LoRA rank. Our experiments reveal that increasing PCA directions improves both knowledge coverage and output diversity up to about 40 dimensions, after which returns diminish. With just 10 directions, SliderSpace matches the FID scores of 64 manually created Concept Sliders when evaluated against artistic style distributions. Regarding model architecture, we find that lower-rank adaptors (particularly rank-one) efficiently capture variations with a fixed training budget, outperforming higher-rank versions while maintaining better FID scores than Concept Sliders across different ranks.

This analysis guides our choice of using rank-one adapters with 40 PCA directions as the default configuration, offering an optimal balance between performance and computational efficiency.

\begin{figure*}[!ht]
    \centering
    \includegraphics[width=\linewidth]{rebuttal-figures-src/hyperparams.pdf}
    \vspace{-1.5em}
    \caption{Concept Sliders Comparison \& Hyperparameter analysis: (Left) Impact of PCA directions: SliderSpace with 10 directions matches the FID of 64 Concept Sliders. More directions, upto 40, leads to improved FID. (Right) Effect of LoRA rank: Given a fixed training budget rank-one sliders are efficient than higher rank versions and outperforms Concept Sliders}
    \vspace{-0.3em}
    \label{fig:reb-hyperparam}
\end{figure*}



\section{User Study}
We conducted user studies to evaluate SliderSpace's effectiveness through Amazon MTurk. For artistic evaluation (Sec~\ref{sec:art_exp}), participants compared two 9-image grids - one generated by SliderSpace using 3 random sliders per image, and another by our baselines. Both sets used identical base prompts: ``a building in a stunning landscape'' and ``a character in a scenic environment''. As shown in Fig~\ref{fig:user_art}, participants rated which grid exhibited greater artistic diversity and utility for art applications. For conceptual evaluation (Sec~\ref{sec:concept_exp}), participants compared image grids based on diversity, generative utility, and creativity (Fig~\ref{fig:user_concept}). Grid presentation order was randomized across all experiments.

\section{Qualitative Results}
\subsection{Art Exploration}
We identify the top-36 distinct art directions discovered by SDXL-DMD2 SliderSpace for the concept "artwork in the style of famous artist" in Figures~\ref{fig:artdmd1} and~\ref{fig:artdmd2}. Additionally, we showcase various combinations of SliderSpace samples in Figures~\ref{fig:artbuilding} and~\ref{fig:artcharacter}, where we randomly sample three sliders to generate images for both characters and buildings (used in our art experiments and user studies in Section~\ref{sec:art_exp}). The top 18 art styles discovered by SDXL-SliderSpace are presented in Figure~\ref{fig:artsdxl}.



\subsection{Diversity Enhancement}
We provide additional qualitative examples demonstrating how our generic diversity sliders mitigate mode collapse in distilled models. Our observations indicate that distilled models such as DMD2~\cite{dmd} tend to generate visually similar images for identical prompts, despite different random seed initializations. Through our trained diversity sliders, which are model-agnostic, we successfully counter mode collapse (Section~\ref{sec:diverse_exp}). As quantitatively validated in Table~\ref{tab:diverse}, the diversity SliderSpace significantly improves image variation, achieving FID scores comparable to the base model.

\subsection{Concept Decomposition}
We present qualitative examples of concept decomposition using the SDXL-DMD2~\cite{dmd} SliderSpace in Figures~\ref{fig:concept1}--\ref{fig:concept6}. Furthermore, we demonstrate SliderSpace's versatility across various models, including SDXL-Turbo~\cite{turbo} (Figures~\ref{fig:turbo1}, SDXL-Base~\cite{podell2023sdxl} (Figures~\ref{fig:artsdxl}), and the state-of-the-art transformer-based FLUX Schnell models (Figures~\ref{fig:intro} and~\ref{fig:flux}). We note that Claude3.5~\cite{claude} generated captions are not always accurate. For instance, in Figure~\ref{fig:concept1}, Claude annotates one of the sliders as ``Black Lab Technician'', but it is not visually distinct whether the slider is `lab technician' or a `scientist'.

\section{Ablations}

\subsection{Number and Type of Calibration Images}

\begin{figure}[htbp]
    \vspace{-3mm}
    \centering
    \includegraphics[width=1.0\linewidth]{fig/ablations1.png}
    \caption{Ablation study on the number and type of calibration images used in SITR, showing their effect on (i) Classification accuracy for inter-sensor transfer, (ii) Classification accuracy for intra-sensor transfer, and (iii) Pose estimation error for inter-sensor transfer. }
    \label{fig:ablations_calibration}
    \vspace{-3mm}
\end{figure}

We conduct an ablation study to investigate the impact of the number and type of calibration images on the performance of \modelname. 
In the standard \modelname~setup, we press two objects—a ball and a cube corner—at nine locations roughly arranged in a 3x3 grid pattern across the sensor surface. 
To explore variations, we retrained \modelname~using different subsets of these calibration images and evaluated performance across all downstream tasks.

We test on five calibration configurations: No calibration images (0); Ball pressed at 4 corners (4); Ball pressed in a 3x3 grid (9); Ball and cube pressed at 4 corners ($8^*$); Ball and cube pressed in a 3x3 grid, which is the standard setup ($18^*$).
Fig. \ref{fig:ablations_calibration} illustrates how different numbers and types of calibration images impact \modelname's performance. We observe that increasing the number of calibration images increases performance across all tasks. However, the performance gains diminish as more images of the same object are added (as seen in the progression from cases (0) to (4) to (9)). Introducing a second calibration object with a distinct geometry, such as the cube (cases (4) to (8*)), results in a larger performance boost compared to simply adding more images of the same object (cases (4) to (9)). The effect of calibration images is particularly notable in the inter-sensor setting, where we see upwards of a 20\% increase in classification accuracy from case (0) to (18*). We choose case (18*) for SITR since increasing the number of calibration images does not incur additional inference costs, as calibration tokens are computed only once per sensor.

\subsection{Contrastive loss and temperature}

\begin{figure}[htbp]
    \vspace{-3mm}
    \centering
    \includegraphics[width=1.0\linewidth]{fig/ablations2.png}
    \caption{Ablation study examining the impact of SCL and varying contrastive temperature $\tau$ on SITR’s performance. Subplots (i) and (ii) show classification accuracy in inter-sensor and intra-sensor settings, respectively, while (iii) shows the effect on pose estimation RMSE.}
    \label{fig:ablations_temperature}
    \vspace{-3mm}
\end{figure}

We conduct an ablation study to assess the effect of SCL and varying contrastive temperatures $\tau$ on \modelname's performance. Specifically, we compared models with and without the SCL term and tested five contrastive temperatures: 0.25, 0.10, 0.07, 0.03, and 0.01.
No SCL corresponds to using only the normal map reconstruction loss during pre-training.
Results in Fig. \ref{fig:ablations_temperature} show that a contrastive temperature of 0.07 achieves the best classification performance in the intra-sensor setting, while 0.03 performs best for the inter-sensor setting. Lower or higher temperatures lead to reduced performance in both cases.
For the pose estimation task, the addition of SCL has a negligible impact on the RMSE. 
These results suggest that contrastive learning helps align features across sensors in classification tasks. However, in the pose estimation task, the model's performance is more dependent on the fine-grained geometry information from the contact surface. For SITR, we choose a temperature of 0.07 for its strong performance in the classification task. 

\begin{figure*}
    \centering
    \includegraphics[width=.8\linewidth]{sup-figures-src/concept_diverse_vangogh-cropped.pdf}
    \caption{We show a few possible variations possible with SliderSpace directions. For a given seed and prompt, users can sample different combinations of sliders from SliderSpace and generate unique and diverse outputs (all variations from a single prompt and seed). We show this for the concept ``Van Gogh'' SliderSpace on SDXL-DMD2~\cite{dmd}.} 
    \label{fig:conceptdiverse1}
\end{figure*}

\begin{figure*}
    \centering
    \includegraphics[width=.8\linewidth]{sup-figures-src/concept_diverse_toy-cropped.pdf}
    \caption{We show a few possible variations possible with SliderSpace directions. For a given seed and prompt, users can sample different combinations of sliders from SliderSpace and generate unique and diverse outputs (all variations from a single prompt and seed). We show this for the concept ``Toy'' SliderSpace on SDXL-DMD2~\cite{dmd}.} 
    \label{fig:conceptdiverse2}
\end{figure*}
 
\begin{figure*}
    \centering
    \includegraphics[width=\linewidth]{sup-figures-src/timestep_dmd-cropped.pdf}
    \caption{The choice of timestep at which sliders are applied can have an effect on the preciseness of the sliders. We show that when the sliders are applied to all the timesteps in inference, the images look different from the original models images for the same prompt and seed. But skipping the first timestep can lead to precise edits (similar observations as \citep{gandikota2023concept})} 
    \label{fig:timestep}
\end{figure*}


% \begin{figure*}[!htp]
%     \centering
%     \includegraphics[width=\linewidth]{sup-figures-src/slider_timestep-cropped.pdf}
%     \caption{The choice of timestep at which sliders are applied can have an effect on the preciseness of the sliders. We show that when the sliders are applied to all the timesteps in inference, the images look different from the original models images for the same prompt and seed. But skipping the first few timestep can lead to precise edits (similar observations as \citep{gandikota2023concept}). We show the results for SDXL SliderSpace~\cite{podell2023sdxl}} 
%     \label{fig:sdxltimestep}
% \end{figure*}



\begin{figure*}[!htbp]
    \centering
    \includegraphics[width=.8\linewidth]{sup-figures-src/artsliders_individual-cropped.pdf}
    \caption{We show the top 18 art directions that are discovered in the SDXL-DMD2~\cite{dmd} SliderSpace for the concept ``art''.} 
    \label{fig:artdmd1}
\end{figure*}

\begin{figure*}[!htbp]
    \centering
    \includegraphics[width=.8\linewidth]{sup-figures-src/artsliders_individual2-cropped.pdf}
    \caption{We show the top 18-36 art directions that are discovered in the SDXL-DMD2~\cite{dmd} SliderSpace for the concept ``art''.} 
    \label{fig:artdmd2}
\end{figure*}

\begin{figure*}[!htbp]
    \centering
    \includegraphics[width=.8\linewidth]{sup-figures-src/SDXL_artsliders_individual-cropped.pdf}
    \caption{We show the top 18 art directions that are discovered in the SDXL~\cite{podell2023sdxl} SliderSpace for the concept ``art''.} 
    \label{fig:artsdxl}
\end{figure*}

\begin{figure*}[!htbp]
    \centering
    \includegraphics[width=.8\linewidth]{sup-figures-src/artsliders_sample_building-cropped.pdf}
    \caption{We show samples from our art experiments ~\ref{sec:art_exp}. We sample random 3 sliders from the SDXL-DMD2~\cite{dmd} SliderSpace for the concept ``art'' and generate images for the prompt ``a building in a stunning landscape the style of a famous artist''.} 
    \label{fig:artbuilding}
\end{figure*}

\begin{figure*}[!htbp]
    \centering
    \includegraphics[width=.8\linewidth]{sup-figures-src/artsliders_sample_character-cropped.pdf}
    \caption{We show samples from our art experiments ~\ref{sec:art_exp}. We sample random 3 sliders from the SDXL-DMD2~\cite{dmd} SliderSpace for the concept ``art'' and generate images for the prompt ``a character in a scenic environment the style of a famous artist''.} 
    \label{fig:artcharacter}
\end{figure*}

\begin{figure*}
    \centering
    \includegraphics[width=.95\linewidth]{figures-src/diverse.pdf}
    \caption{SliderSpace can also decompose general broad visual variation of diffusion model's and can be used to overcome model collapse in distilled models. We randomly sample a sparse set of sliders and generate the sample showing higher variation than base distilled model.}
    \label{fig:diverse}
\end{figure*}



\section{User evaluation with frequent users of mobile ASR: Lab study and online survey }
To evaluate the usability of our approach, we decided to conduct an in-person lab evaluation of the SpeechCompass phone case and the speech-to-text application (described in Section~\ref{subsection:app}), with frequent users of mobile transcription technology. We first conducted a large-scale online pilot study to inform the design of the in-person lab evaluation, which we conducted with eight deaf or hard-of-hearing participants, set up to mimic a realistic conversation scenario. 

\begin{figure*}
  \centering
  \includegraphics[width=0.75\linewidth]{images/second_study.pdf}
  \caption{Participants' preferences for different visualization techniques in the online survey. A) Results indicating how valuable the specific indicator would be for the user. B) Preferences for the specific indicators for speech direction.} 
  \label{fig:user_preferences_online} 
\end{figure*}


\subsection{Large-scale, online survey (n=494)} In this survey, we use screenshots of our interactive UI prototypes to solicit initial user
feedback on the potential for our proposed approach, to guide the design of a more realistic in-person lab study.

The study was conducted using the same Google Surveys deployment and screening methodology as for the foundational study, detailed in Section 3. The participants were shown different UI renderings and were asked to rate them. The large-scale online survey could only show static images of the interfaces, due to limitations of the survey tool. Out of 985 respondents we focus our analysis on the 494 participants who use captioning technology multiple times per week or more frequently. 

As shown in Figure~\ref{fig:user_preferences_online}A, the colored text was found to be valuable by 60\% of participants. Glyph indicators for speech direction, which included arrow and circle+line indicators, were found valuable by 70\%. The Edge indicator and the mini map had a less positive reception. 

To better understand which glyph indicators were favored, we also asked targeted questions about them, as shown in Figure~\ref{fig:user_preferences_online}B. \emph{Circle + line} was preferred by 13.1\% more respondents than the \emph{highlight box} (45.1\% vs 32.0\%), and the \emph{arrow} was preferred by 21.9\% more respondents than the \emph{circle + line} (51.2\% vs 29.3\%).


\subsection{Lab study (n=8)}
\alex{explain and emphasize intention}
We recruited 8 participants from our institution who were frequent users of captioning technology. Five were female, three were male, and all were deaf or hard of hearing. One participant was 25--34 years old, four were 34--44, one was 45--53, and two were 65+ (we are only allowed to collect age ranges at our institution). 


% setup: https://docs.google.com/document/d/1akr5HVMgJb8Kd9KaEZJcdXn2S0IbHhd8JdBPTE0TiA0/edit?usp=sharing
The study took place in a quiet lab over approximately 60 minutes and used the phone-case prototype (Figure~\ref{fig:pcb_design}) with our mobile ASR application (Figure~\ref{fig:phone_interfaces}). First, the participant was introduced to the technology, prototype, and the purpose of the study. Then, the participant was asked to fill out a background survey, which included demographic questions and their current use and experienced challenges with transcription technology. Afterward, the participant was introduced to different visualization scenarios with the SpeechCompass application. The participant used the SpeechCompass transcription while sitting between the two experimenters, as they all sat around a small table with the SpeechCompass phone case in the center. In each of the seven conditions, which ran for 5 minutes, the experimenters sat across from each other and had short conversations about different topics. The participants were instructed to turn off hearing aid devices if they used any, and were asked to use the SoundCompass UI and transcript to follow the conversation. The experimenters' casual conversations included topics like weekend plans, hobbies, and the weather. The seven conditions, which used the ASR, diarization, and localization functionality for different visualization techniques, are shown in Figure~\ref{fig:ui_options} and presented with more UI context in Figure~\ref{fig:phone_interfaces}. The conditions were:
\begin{enumerate}
    \item \textbf{Transcription only}. The transcribed text is shown in white on a black background. 
    
    \item \textbf{Edge indicator}. A circle (``dot'') that moves around the edge of the screen to point to the currently active speaker. The color of the dot changes based on the direction. 
    
    \item \textbf{Arrow indicator}. A glyph using a colored arrow next to a white text block. The glyph points in the direction of the associated speech. 
    
    \item \textbf{Circle + line indicator}. A glyph using a circle with a directional line next to a white text block. The glyph points in the direction of the speech associated with the text. 
    
    \item \textbf{Mini map}. A colored circle with a smaller circle (``dot'') moves around its edge to point to the currently active speaker. The color of the dot changes based on the direction. 
    
    \item \textbf{Colored text}. The text is colored based on the direction that the associated speech was coming from. 
    
    \item \textbf{Everything on}. All indicators are turned on (except the Circle + line, as it couldn't be used simultaneously with the arrow). 
\end{enumerate}

%five isolated visualization techniques, baseline with just text transcription (no speaker information), and with all visualization turned one. Minimap was shown with an arrow, since we envisioned it would be combined with other techniques. 
After participants had completed all conditions, they filled out a form that asked them to rate how desirable each of the five visual indicator styles (\textit{Edge indicator}, \textit{Arrow}, \textit{Circle  + line}, \textit{Colored map}, and \textit{Colored text}) were on a 7-point Likert scale, from \emph{-3: Strongly dislike} to \emph{+3: Strongly like}. Finally, they were asked to rate the overall value of directional feedback to the transcription experience, how strongly they would recommend these features to users of mobile captioning, and whether they had any general free-form feedback about SpeechCompass. 

\begin{figure*}
  \centering
  \includegraphics[width=0.65\linewidth]{images/study_setup.png}
  \caption{Examples of seven visualization scenarios that participants experienced in the in-person study.} 
  \label{fig:ui_options} 
\end{figure*}

%After running the scenarios, participants filled out the second part of the survey, which asked them to rate each scenario and overall impression on a scale from -3:strongly dislike to +3:strongly like. Finally, the participants filled out free form feedback about the study. 

\begin{figure*}
  \centering
  \includegraphics[width=0.65\linewidth]{images/box_plot_in_person_study_results.png}
  \caption{Boxplots of results of the in-person study. A) Participants' preferences for different visualization techniques. B) Overall opinions about augmented mobile ASR application.\alex{love these plots -- maybe to B you could also add the question about multi-people conversations as the leftmost, since it is also on same scale?} } 
  \label{fig:user_preferences} 
\end{figure*}

\subsection{Results}
Mobile transcription apps (e.g., Android Live Transcribe) were the most used communication technology for the participants. Specifically, three used them multiple times per day, one used them daily, three used them multiple times per week, and one used them rarely. 

75\% of participants frequently experienced the scenario where multiple people would get mixed up in the transcript (two multiple times per day, two daily, two multiple times per week). All participants agreed that it was challenging to participate in conversations when speech was combined from multiple people. 
%Similarly to the online survey, we asked participants to select the biggest challenges they experienced in their use of transcription technology (same options as in Figure~\ref{fig: survey-challenges}). where the majority (6/8) selected \textit{"Have to look away from the person I am talking to"}.  
\\

A Kruskal-Wallis (KW) test found a significant effect
on participant preferences for visualization techniques (P=.014).
The post-hoc pair-wise analyses using the Wilcoxon test with Bonferroni correction did, however, not show statistical significance between any pairs.
Of the five visual indicator styles that participants experienced, \emph{Colored text} was the most well-received (mean ($\bar{x})=2.625$), as it was rated positively by all the participants. %, with six strong like (+3), one like (+2), and one slight like (+1). 
The \emph{Arrow} indicator was also well-received ($\bar{x}=1.125$), with six positive, one negative, and one neutral participant.
%(one strong like (+3), three like (+2) and one slight like (+1)) and one dislike (-1) and one neutral (0)). 
Several participants noted that \emph{Arrow} and \emph{Colored text} worked well together: \emph{"Arrows + color seem to be most easier way to indicate the direction." (P2)} and \emph{"The combination of the colored text with the arrow was the most effective for me." (P7)}.

The other indicator styles received more mixed feedback. The feedback for both \emph{Edge indicator} ($\bar{x}=0.25$) and \emph{Circle + line} ($\bar{x}=-0.125$) was split between four negative and four positive participants. 
Some participants were concerned that \emph{Edge indicator} was distracting and not sufficiently discreet: \textit{"I do prefer the tool be as discrete as possible and would perhaps choose to avoid bright colored things moving around since this would be eye-catching and this kind of attention is often undesired" (P3)} and 
\textit{"Indicator moving around the edge was distracting and causing a bit of eye strain" (P2)}.
On the other hand, another participant found this style particularly useful: \textit{"the color dot moving to the speaker direction worked REALLY well" (P1)}. 
For \emph{Circle + line}, some participants struggled with its legibility: \textit{"If the analog direction indicators were larger (and translucent, or set behind)" (P8)} and \textit{"The lines in a circle were a bit slower and not as accurate (buggy)" (P5)}.
The \emph{Mini map} was rated positively by five participants and negatively by three. The most favorable participant stated: \emph{"this is also great for environmental awareness for those with single-sided hearing or no hearing at all." (P3)} and a participant who disliked the \emph{Edge indicator} commented: \emph{"steady map in the corner worked a bit better (P5)"}.

Overall, all participants agreed with the value of directional feedback ($\bar{x}=2.88$, seven Strongly agree:+3 and one Agree:+2) and would recommend these features to other users of captioning technology ($\bar{x}=2.63$, five Strongly agree:+3 and three Agree:+2): \textit{"I really liked that almost immediately I could tell that there was a speaker change, so that as soon as the text started to show up, I could better contextualize that text as attributed to a new speaker." (P1)}, \textit{"I'm very happy to see this tool being developed, it's a great addition to other speech recognition tools!" (P3)}, and \textit{"This prototype is definitely a life changer and I strongly believe that it will improve the quality of access to communication with speakers for many users" (P6)}.

\subsection{Discussion}
Consistent with the large-scale survey, the value of the diarization and localization features was immediate to all users. The participants were asked if directional guidance would be valuable in their mobile transcription experience. All eight users agreed. Also, all eight users would recommend this feature to mobile captioning users. 

While the large-scale survey helped inform our testing and exclude conditions (e.g., \emph{Highlight box}), the lab study allowed us to more rigorously evaluate the techniques in a realistic scenario. This difference became significant for the \emph{Edge indicator} and \emph{Mini map}, where issues, such as discreetness and distracting aspects, became evident during live usage. 

The results suggest that the combination of \textit{Colored text} and \textit{Arrow} would meet the preferences of most users, thanks to the balance of directional encoding and clarity. The arrow has redundant benefits too, since colored text might not always be reliably visible depending on lighting and screen conditions (e.g., strong sunlight, or dim display) and might also not be usable for colorblind users. The mixed feedback for other techniques indicates that the interface may also benefit from mechanisms that would allow users to customize the visualization style. Such customization could also apply to rendering properties, such as color, transparency, and line thickness, as some participants found \textit{Circle + line} particularly difficult to interpret. In both the large-scale survey and the in-person lab study, the \textit{Arrow} was preferred over \textit{Circle + line}. Through more customization options and extended usage in their daily lives, participants will be able to provide more nuanced feedback about these techniques. 


% Edge indicator and mini map had a less positive reception. However, they were rated more positively than those in the in-person study. Since participants didn't experience the working prototype, the discreet and distracting aspects that were observed in the in-person study were not captured. 

% In both online and in-person study, the arrow directional glyph was preferred to circle+line.



% This dichotomy demonstrates that users should be given a way to customize their experience. For example, the edge indicator received strong likes and dislikes from different participants. 


% This indicates that the interface designers should make the directional glyphs as easy to read as possible.


% The results of the online survey followed what was observed in the in-person study. Edge indicator and mini map had a less positive reception. However, they were rated more positively than those in the in-person study. Since participants didn't experience the working prototype, the discreet and distracting aspects that were observed in the in-person study were not captured. In both online and in-person study, the arrow directional glyph was preferred to circle+line.

% As indicated in the survey, the value of the diarization and localization features was immediate to all users. The participants were asked if directional feedback is valuable in their mobile transcription experience. All eight users agreed. Also, all eight users would recommend this feature to mobile captioning users. 


% \textit{"I really liked that almost immediately I could tell that there was a speaker change, so that as soon as the text started to show up, I could better contextualize that text as attributed to a new speaker." (P1)}

% P3
% Arrows + color seem to be most easier way to indicate the direction.
% \emph{"Arrows + color seem to be most easier way to indicate the direction." (P2)}
% P4
% \textit{"I'm very happy to see this tool being developed, it's a great addition to other speech recognition tools!" (P3)
% }
% \textit{"it was great to see so many options being offered" (P3)
% }
% P6 
% \textit{"This prototype is definitely a life changer and I strongly beleve that it will improve the quality of access to communication with speakers for many users" (P6)}

% P8
% The combination of the colored text with the arrow was the most effective for me.

% \emph{"The combination of the colored text with the arrow was the most effective for me." (P7)}

\input{sup-figures/concept1}
\begin{figure*}
    \centering
    \includegraphics[width=\linewidth]{sup-figures-src/dmd_concept_coral-cropped.pdf}
    \caption{We show the SliderSpace discovered in SDXL-DMD2 4-step model~\cite{dmd} for the concept ``Coral''} 
    \label{fig:concept2}
\end{figure*}

\begin{figure*}
    \centering
    \includegraphics[width=\linewidth]{sup-figures-src/dmd_concept_cowboy-cropped.pdf}
    \caption{We show the SliderSpace discovered in SDXL-DMD2 4-step model~\cite{dmd} for the concept ``Cowboy''} 
    \label{fig:concept3}
\end{figure*}

\begin{figure*}
    \centering
    \includegraphics[width=\linewidth]{sup-figures-src/dmd_concept_dragon-cropped.pdf}
    \caption{We show the SliderSpace discovered in SDXL-DMD2 4-step model~\cite{dmd} for the concept ``Dragon''} 
    \label{fig:concept4}
\end{figure*}

\begin{figure*}
    \centering
    \includegraphics[width=\linewidth]{sup-figures-src/dmd_concept_mask-cropped.pdf}
    \caption{We show the SliderSpace discovered in SDXL-DMD2 4-step model~\cite{dmd} for the concept ``Mask''} 
    \label{fig:concept5}
\end{figure*}

\begin{figure*}
    \centering
    \includegraphics[width=\linewidth]{sup-figures-src/dmd_concept_toy-cropped.pdf}
    \caption{We show the SliderSpace discovered in SDXL-DMD2 4-step model~\cite{dmd} for the concept ``Toy''} 
    \label{fig:concept6}
\end{figure*}

\begin{figure*}
    \centering
    \includegraphics[width=\linewidth]{sup-figures-src/turbo_image_edit-cropped.pdf}
    \caption{We show the SliderSpace discovered in SDXL-Turbo 4-step model~\cite{turbo} and how they can be used for precise control of image generation} 
    \label{fig:turbo1}
\end{figure*}

% \begin{figure*}
    \centering
    \includegraphics[width=\linewidth]{sup-figures-src/turbo_concept_bird-cropped.pdf}
    \caption{We show the SliderSpace discovered in SDXL-Turbo 4-step model~\cite{turbo} for the concept ``Bird''} 
    \label{fig:turbo2}
\end{figure*}

% \input{author-kit-CVPR2025-v3-latex-/sup-figures/turbo_concept3}
% \begin{figure*}
    \centering
    \includegraphics[width=\linewidth]{sup-figures-src/turbo_concept_temple-cropped.pdf}
    \caption{We show the SliderSpace discovered in SDXL-Turbo 4-step model~\cite{turbo} for the concept ``Temple''} 
    \label{fig:turbo4}
\end{figure*}

\begin{figure*}
    \centering
    \includegraphics[width=\linewidth]{sup-figures-src/flux_sliders-cropped.pdf}
    \caption{We show the SliderSpace discovered in FLUX Schnell model~\cite{flux} for concepts ``monster'' and ``dog''} 
    \label{fig:flux}
\end{figure*}
.




