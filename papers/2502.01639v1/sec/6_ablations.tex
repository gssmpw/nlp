\section{Ablations}
\label{sec:ablations}
We analyze the key components of our method and validate their necessity: (1) the semantic orthogonality objective, (2) expanding diversity of training samples, and (3) CLIP embedding analysis. Figure~\ref{fig:ablations} shows qualitative examples and FID measures on art exploration experiments. In both the qualitative and quantitative experiments, we find that uniqueness criteria in Eqn~\ref{eq:contrast} is very important to get diverse discovery of SliderSpace. When we extract a naive-SliderSpace by training multiple sliders on a single concept using regular customization~\cite{kumari2022customdiffusion,ruiz2022dreambooth} loss and no contrastive objective, many redundant and junk directions appear, as shown in Fig.~\ref{fig:ablations}(a). This baseline is equivalent to Liu et al.~\cite{liu2023unsupervised}. Similarly, by applying our objective (Eq.~\ref{eq:contrast}) on diffusion output space $\Tilde{x}_{0,t}$ (Eq.~\ref{eq:x0}) rather than CLIP space, SliderSpace discovers directions that are more relevant in color and shape but not semantic variations, as shown in Fig.~\ref{fig:ablations}(b). This baseline is slider equivalent version of NoiseCLR~\cite{dalva2024noiseclr}. Finally, diversity expansion of training data (Fig.~\ref{fig:ablations} d,e), helps with expanding a diverse set of sliders. This can be used to improve the variation across sliders. We use SDXL for generating images in concept and art experiments. For diversity experiments, we use LLM prompt expansion as we compare against SDXL as baseline.
1. Distilling accelerated residuals from slow residuals
2. only one of them active -> either vla or m2r2