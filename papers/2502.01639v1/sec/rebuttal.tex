\documentclass[10pt,twocolumn,letterpaper]{article}
\usepackage[rebuttal]{cvpr}

% Include other packages here, before hyperref.
\usepackage{graphicx}
\usepackage{amsmath}
\usepackage{amssymb}
\usepackage{booktabs}

% Import additional packages in the preamble file, before hyperref
\newcommand{\CG}{\mathcal{G}\xspace}
\newcommand{\CV}{\mathcal{V}\xspace}
\newcommand{\CE}{\mathcal{E}\xspace}
\newcommand{\CA}{\mathcal{A}\xspace}
\newcommand{\CF}{\mathcal{F}\xspace}
\newcommand{\CR}{\mathcal{R}\xspace}
\newcommand{\CB}{\mathcal{B}\xspace}
\newcommand{\CX}{\mathcal{X}\xspace}
\newcommand{\CK}{\mathcal{K}\xspace}
\newcommand{\CM}{\mathcal{M}\xspace}
\newcommand{\CC}{\mathcal{C}\xspace}
\newcommand{\CL}{\mathcal{L}\xspace}
\newcommand{\CI}{\mathcal{I}\xspace}
\newcommand{\CQ}{\mathcal{Q}\xspace}
\newcommand{\CO}{\mathcal{O}\xspace}
\newcommand{\CP}{\mathcal{P}\xspace}
\newcommand{\CS}{\mathcal{S}\xspace}
\newcommand{\CT}{\mathcal{T}\xspace}
\newcommand{\CJ}{\mathcal{J}\xspace}
\usepackage[para]{footmisc}
\usepackage{subfig}
% \usepackage{subcaption}
% \usepackage{array}
% \usepackage{colortbl}




% \usepackage[pagebackref,breaklinks,colorlinks,citecolor=cvprblue]{hyperref}
% If you comment hyperref and then uncomment it, you should delete
% egpaper.aux before re-running latex.  (Or just hit 'q' on the first latex
% run, let it finish, and you should be clear).
\definecolor{cvprblue}{rgb}{0.21,0.49,0.74}
\usepackage[pagebackref,breaklinks,colorlinks,allcolors=cvprblue]{hyperref}

\usepackage[capitalize]{cleveref}
\crefname{section}{Sec.}{Secs.}
\Crefname{section}{Section}{Sections}
\Crefname{table}{Table}{Tables}
\crefname{table}{Tab.}{Tabs.}

% If you wish to avoid re-using figure, table, and equation numbers from
% the main paper, please uncomment the following and change the numbers
% appropriately.
%\setcounter{figure}{2}
%\setcounter{table}{1}
%\setcounter{equation}{2}

% If you wish to avoid re-using reference numbers from the main paper,
% please uncomment the following and change the counter value to the
% number of references you have in the main paper (here, 100).
%\makeatletter
%\apptocmd{\thebibliography}{\global\c@NAT@ctr 100\relax}{}{}
%\makeatother

\newcommand{\rone}[1]{\textcolor{blue}{(3WW8)}} 
\newcommand{\rtwo}[1]{\textcolor{purple}{(9cdw)}} 
\newcommand{\rthree}[1]{\textcolor{teal}{(FdWr)}} 
\makeatletter
\renewcommand{\paragraph}{%
  \@startsection{paragraph}{4}%
  {\z@}{1.25ex \@plus 1ex \@minus .2ex}{-1em}%
  {\normalfont\normalsize\bfseries}%
}
\makeatother

%%%%%%%%% PAPER ID  - PLEASE UPDATE
\def\paperID{7586} % *** Enter the Paper ID here
\def\confName{CVPR}
\def\confYear{2025}

\begin{document}

%%%%%%%%% TITLE - PLEASE UPDATE
\title{SliderSpace: Decomposing the Visual Capabilities of Diffusion Models}  % **** Enter the paper title here

\maketitle
\thispagestyle{empty}
\appendix

%%%%%%%%% BODY TEXT - ENTER YOUR RESPONSE BELOW
\small We thank the reviewers for their insightful feedback. We are encouraged that they found our work to be a significant advancement in generative models \rtwo{}, for discovering directions that reflect the true topology of the model's knowledge space \rtwo{}, and for its straightforward \rone{}, unsupervised approach in systematically uncovering a broader range of latent dimensions \rthree{}. We thank \rone{} \rtwo{} \rthree{} for noting that our framework maintains semantic consistency while providing controllable directions for image generation.

% \small We thank the reviewers for their insightful feedback. We are encouraged that they found our work to be a significant advance in generative models due to its innovative automated workflow \rone{}, for capturing the true topology of the model's knowledge space \rtwo{}, and for its improvement over prior work in systematic discovery of latent dimensions \rthree{}. We thank \rone{} \rtwo{} \rthree{} for noting that our framework provides fine-grained control while maintaining semantic consistency. 

%%%%%%%%% BODY TEXT - ENTER YOUR RESPONSE BELOW


\begin{figure*}[!ht]
    \centering
    \includegraphics[width=\linewidth]{rebuttal-figures-src/hyperparams.pdf}
    \vspace{-1.5em}
    \caption{Concept Sliders Comparison \& Hyperparameter analysis: (Left) Impact of PCA directions: SliderSpace with 10 directions matches the FID of 64 Concept Sliders. More directions, upto 40, leads to improved FID. (Right) Effect of LoRA rank: Given a fixed training budget rank-one sliders are efficient than higher rank versions and outperforms Concept Sliders}
    \vspace{-0.3em}
    \label{fig:reb-hyperparam}
\end{figure*}

\noindent\textbf{\rone{} How does unsupervised slider discovery compare to manual discovery with Concept Sliders? }
\textit{With the same number of sliders, SliderSpace captures a much larger portion of the model’s diversity distribution, covering more dimensions of the knowledge topology}. In Fig.~\ref{fig:reb-hyperparam}, we compare our method to supervised Concept Sliders by training 64 art-specific sliders per concept using Claude-generated training prompts. Samples produced by SliderSpace have significantly better / lower FID (relative to the manually discovered artistic styles of ParrotZone). Thank you for the suggestion, this comparison highlights the significant advantage of unsupervised vs supervised discovery and we will add this discussion to the paper.

\noindent\textbf{\rtwo{}
 Impact of PCA Directions and Rank on SliderSpace:} 
Our analysis in Fig~\ref{fig:reb-hyperparam} reveals that increasing PCA directions improves both knowledge coverage and output diversity, up to about 40 dimensions. Importantly, we find that lower-rank adaptors (particularly rank-one) efficiently capture these variations with a fixed training budget. We note that performance of SliderSpace outperforms Concept Sliders irrespective of rank.
%while minimizing computational overhead. (The dotted line in the plot compares to a Concept Slider baseline using Claude-generated prompts for ``art'' concepts; our method achieves comparable or better results with significantly less computational cost.)

\noindent\textbf{\rtwo{} \rthree{} Can SliderSpace use encodings other than CLIP?}
Yes. We show both Dino-v2 and FaceNet embeddings as alternatives. First, our findings in Fig~\ref{fig:reb-concept} show that Dino-v2 excels for some concepts, while CLIP outperforms in others. In general, both encoders achieve similar overall performance, though they exhibit different strengths. Second, using FaceNet embeddings, we train a ``person'' SliderSpace in Fig~\ref{fig:reb-person} that captures many facial semantic directions.  We will discuss on the generality of SliderSpace to multiple encoders in the paper, and the trade-offs selecting different encoders enable.
%We will discuss these encoder-specific capabilities and trade-offs in our paper.
%offering superior performance in fine-grained visual attribute detection, while CLIP performs better at higher-level semantic concepts. 
\begin{figure}[!ht]
    \centering
    \includegraphics[width=\linewidth]{rebuttal-figures-src/person.pdf}
    \vspace{-1.5em}
    \caption{SliderSpace directions for the ``person'' concept (top) successfully generalize to related ``police'' and ``athlete'' concepts. They also transfer to out-of-domain concepts like ``dog''}
    \vspace{-0.5em}
    \label{fig:reb-person}
\end{figure}

\subsection{Relations vs. Concepts}
We saw in Figure    \ref{fig:neuron_overlap} that the
storage of relations is generally well separated, but that
there are exceptions. We can view a relation as relating two
\textbf{concepts}, e.g.,     
\texttt{company\_ceo} relates instances of the ``subject'' concept
``company'' to instances of the ``object'' concept ``CEO''. From this
perspective, the exceptions in Figure
\ref{fig:neuron_overlap}, i.e., cases where a relation $r_1$
overlaps with a relation $r_2$, are 
generally cases where the concepts of $r_1$ and $r_2$  are
the same or overlap. For example, \texttt{company\_ceo}
and \texttt{company\_hq} have the same subject concept.

\enote{fy}{concept ? or argument ? or entity ?}
\enote{yl}{i guess here we need to be a a bit more consistent. I would interpret the "concept" as an abstract notion that many "entities" belong to, e.g., CEO. I am not sure if the "argument" here means something similar, or, it means the actual entity, e.g., "Jason Huang" (an actual CEO).}

To further explore this hypothesis empirically, 
we again use the method applied in
\secref{method} to relations, but now use it for
concepts; that is
we identify sets of
\textbf{concept-specific neurons}.
We group the LRE dataset 
triples by 
subject concept,
resulting in 11 different concepts. 
We create a set of triples with novel relations
such as ``can'' and ``has a'', balanced across
positive and negative
samples. This ensures that
the model's completion for a prompt like
(``Lincoln has a'') depends on the concept instance
(``Lincoln''), not on the relation (``has a'').


Figure \ref{fig:entity_neuron_overlap} shows the
overlap between relation neurons and concept neurons for 
13B. Most of the cells with large counts support our
hypothesis that 
the overlaps we observe are rooted in relations being
representationally associated  with their subject and
object concepts.
Clear examples include
\texttt{company\_ceo} and its subject concept
\texttt{company}; 
\texttt{company\_hq} and its object concept 
\texttt{city} (assuming that \texttt{hq} is a subcategory of
\texttt{city});
and \texttt{landmark\_continent} and its subject concept
\texttt{landmark}. There is little overlap of
\texttt{person} with relations like \texttt{person\_mother},
potentially because \texttt{person} is a more general and
semantically unspecific concept than the others. Note that
several concepts do not match a specific relation, e.g.,
\texttt{superhero}, and therefore are not strongly associated
with any relation. Recall that we picked the concepts
according to the availability of annotated data in LRE.
However, most
identified neurons
are only concept neurons or only relation neurons,
suggesting that
relational and conceptual representations are largely separate.

%general relational activation
%patterns are mostly disjunct of the concepts the subject
%stands in connection to.



%associated with a specific
%concept and those specialized in relations. Some overlap is
%observed in both models, particularly between the neuron
%experts of a concept and a relation applied on subjects from
%that concept, this can be for example observed between the
%concept \texttt{star} and the corresponding relation
%\texttt{star\_constellation}. Additionally, concepts more
%closely associated with the notion of location, such as
%\texttt{city} and \texttt{landmark}, exhibit overlap with
%location-related relations, including
%\texttt{landmark\_country} and \texttt{company\_hq}.  This
%pattern is comparatively more prominent in the 13b model,
%which aligns to the previously discussed drop of accuracy
%for location-based relations within this model, as shown on
%the right side in Figure \ref{fig:inter-relation}.

\enote{hs}{didn't udnerstand this. reintgrate?
This
shows the connection between \textbf{neuron versatility} and
the occurrence of a shared concept represented over a
certain group of neurons for different relations.}

\enote{lh}{the thought was to bring together findings from the section on neuron versatility and this section. In figure 4. there is accuracy drop in many relations around the notion of location upon an ablation of such a relation. There is a strong overlap between those relations (as shown in figure 13) and now we observe the same pattern around concepts concerned with locations here.}




%Introduction sentence & Motication

%- relations and concepts can be closely related
%  we saw it in our expoerimetn with realtions
%  now we can explain that
%  we see dditional evidence for that
%- person is a very general concept
%- we have concepts where indstances occur but not
%as intsance of tht concept
%- we subject contecpts vs object condtpetws (city)
%some realtion very speicfic to spdecif cocnpets
%an d they are stored togegther
%- star\_constellation

%and analyze their overlap
%with relation-specific neurons to determine potential
%dependencies between them.
%where the concepts of $r_1$ and $r_2$  are
%the same or overlap.




%Method


%The evaluation is performed on a test set derived from factually correct prompts in the LRE dataset.


%Overlap results 



\begin{figure}[tb]
    \centering
   \setlength{\abovecaptionskip}{-0.05cm}
   \setlength{\belowcaptionskip}{-0.5cm}
%\includegraphics[width=0.46\textwidth]{figs/entity_neurons/neuron_overlaps_top3000_llama-2-7b-hf.pdf}
\includegraphics[width=0.42\textwidth]{figs/entity_neurons/neuron_overlaps_top3000_13b_new.pdf}
    \caption{Overlap between the top 3000 neurons of
      relations and  concepts in the 13B model.}
    \label{fig:entity_neuron_overlap}
\end{figure}




\noindent\textbf{\rone{} \rthree{} Can sliders trained on one concept be transferred to other?}
Yes. To demonstrate transferability, we train SliderSpace on the general concept ``person'' and discover interpretable semantic directions controlling appearance, accessories, age, and expressions (Fig~\ref{fig:reb-person}). We also show more semantic directions in Appendix D.3. These learned directions successfully transfer to related concepts like ``police'' and ``athlete'' without additional training, enabling slider reuse across related domains. Surprisingly, many sliders even transfer to out-of-domain concepts like ``dog''. We will discuss these findings in the paper.
% On the other hand, training a SliderSpace specific to ``police'' is not expensive and reveals additional concept directions for that specialized distribution (see Appendix Figure D.10).

\noindent\textbf{\rthree{} Can we optimize the computational resources needed for discovering SliderSpace?}
This is an important practical consideration. We note that SliderSpace represents a significant improvement over Concept Sliders, which requires approximately 8 hours to train 64 direction-specific sliders. Our speed-up comes from parallel direction discovery. As suggested by reviewer, by carefully optimizing the pipeline, including reduced PCA sample size and improved VRAM management, we could reduce the end-to-end training time from 2 hours to \textit{65 minutes} on A100 GPU. Future work is warranted to optimize for interactive response times.

\noindent\textbf{\rone{} \rtwo{} \rthree{} Typos and manuscript updates:}
Thank you, we will incorporate all suggested fixes to typo. As suggested by reviewers, we will simplify the related work and background to accommodate additional analysis. The revised manuscript will include the rebuttal experiments: hyperparameter sweep, concept slider comparisons, and ablations of other encoders like Dino and FaceNet as recommended by reviewers. We will add refereces to StyleCLIP and StyleGAN-NADA to discuss the CLIP direction disentanglement through orthogonality.

\end{document}