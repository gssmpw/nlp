\documentclass{article}


% if you need to pass options to natbib, use, e.g.:
%     \PassOptionsToPackage{numbers, compress}{natbib}
% before loading neurips_2024


% ready for submission
% \usepackage{neurips_2024}


% to compile a preprint version, e.g., for submission to arXiv, add add the
% [preprint] option:
\usepackage[preprint]{neurips_2024}


% to compile a camera-ready version, add the [final] option, e.g.:
%     \usepackage[final]{neurips_2024}


% to avoid loading the natbib package, add option nonatbib:
%    \usepackage[nonatbib]{neurips_2024}


\usepackage[utf8]{inputenc} % allow utf-8 input
\usepackage[T1]{fontenc}    % use 8-bit T1 fonts
\usepackage{hyperref}       % hyperlinks
\usepackage{url}            % simple URL typesetting
\usepackage{booktabs}       % professional-quality tables
\usepackage{amsfonts}       % blackboard math symbols
\usepackage{nicefrac}       % compact symbols for 1/2, etc.
\usepackage{microtype}      % microtypography
\usepackage{xcolor}         % colors


% Recommended, but optional, packages for figures and better typesetting:
\usepackage{microtype}
\usepackage{graphicx}
\usepackage{subfigure}
\usepackage{booktabs} % for professional tables

\usepackage{amsmath} % 引入amsmath宏包
\usepackage{algorithm}
\usepackage{algorithmic}


% For theorems and such
\usepackage{amsmath}
\usepackage{amssymb}
\usepackage{mathtools}
\usepackage{amsthm}
\usepackage{multirow}
% if you use cleveref..
\usepackage[capitalize,noabbrev]{cleveref}
%%%%%%%%%%%%%%%%%%%%%%%%%%%%%%%%
% THEOREMS
%%%%%%%%%%%%%%%%%%%%%%%%%%%%%%%%
\theoremstyle{plain}
\newtheorem{theorem}{Theorem}[section]
\newtheorem{proposition}[theorem]{Proposition}
\newtheorem{lemma}[theorem]{Lemma}
\newtheorem{corollary}[theorem]{Corollary}
\theoremstyle{definition}
\newtheorem{definition}[theorem]{Definition}
\newtheorem{assumption}[theorem]{Assumption}
\theoremstyle{remark}
\newtheorem{remark}[theorem]{Remark}


\usepackage{multirow}


\title{Lipschitz Lifelong Monte Carlo Tree Search for Mastering Non-Stationary Tasks}


% The \author macro works with any number of authors. There are two commands
% used to separate the names and addresses of multiple authors: \And and \AND.
%
% Using \And between authors leaves it to LaTeX to determine where to break the
% lines. Using \AND forces a line break at that point. So, if LaTeX puts 3 of 4
% authors names on the first line, and the last on the second line, try using
% \AND instead of \And before the third author name.


\author{
    Zuyuan Zhang\\
    The George Washington University\\
    \texttt{zuyuan.zhang@gwu.edu}\\
    \And
    Tian Lan\\
    The George Washington University\\
    \texttt{tlan@gwu.edu}
}


\begin{document}


\maketitle


% Human gaze is a vital non-verbal cue that reveals attention and cognitive state. It makes 

Gaze estimation models are widely used in applications such as driver attention monitoring and human-computer interaction. While many methods for gaze estimation exist, they rely heavily on data-hungry deep learning to achieve high performance. This reliance often forces practitioners to harvest training data from unverified public datasets, outsource model training, or rely on pre-trained models. However, such practices expose gaze estimation models to backdoor attacks. In such attacks, adversaries inject backdoor triggers by poisoning the training data, creating a backdoor vulnerability: the model performs normally with benign inputs, but produces manipulated gaze directions when a specific trigger is present. This compromises the security of many gaze-based applications, such as causing the model to fail in tracking the driver's attention. To date, there is no defense that addresses backdoor attacks on gaze estimation models. In response, we introduce {\name}, the first solution designed to protect gaze estimation models from such attacks. Unlike classification models, defending gaze estimation poses unique challenges due to its continuous output space and globally activated backdoor behavior. By identifying distinctive characteristics of backdoored gaze estimation models, we develop a novel and effective approach to reverse-engineer the trigger function for reliable backdoor detection. Extensive evaluations in both digital and physical worlds demonstrate that {\name} effectively counters a range of backdoor attacks and outperforms seven state-of-the-art defenses adapted from classification models. 

\section{Introduction}
Monte Carlo Tree Search (MCTS) has demonstrated state-of-the-art performance in solving many challenging planning tasks, from playing the game of go~\cite{silver2016mastering} and chess to logistic planning~\cite{silver2017mastering}. It performs look-ahead searches based on Monte Carlo sampling of the actions to balance efficient exploration and optimized exploitation in the large search space. Recent efforts have focused on developing MCTS algorithms for real-world domains that require the elimination of certain standard assumptions. Examples include MuZero~\cite{schrittwieser2020mastering} that leverages the decoding of hidden states to avoid requiring the knowledge of the game dynamics; and MAzero~\cite{liu2024efficient} that performs multi-agent search through decentralized execution. However, existing work have not considered lifelong non-stationarity of task dynamics, which may manifest itself in many open world domains, as the task environment can often vary over time or across scenarios. It requires novel MCTS algorithms that can adapt in response, accumulate, and exploit knowledge throughout the learning process. 

We consider MCTS-based lifelong planning under non-stationarity. An agent faces a series of changing planning tasks -- e.g., with varying transition probabilities and rewards -- which are drawn sequentially throughout the operational lifetime. Transferring knowledge from prior experience to continually adapt Monte Carlo sampling of the actions and thus speed up searches in new tasks is a key question in this setting. We note that although continual and lifelong planning has been studied in reinforcement learning (RL) context, e.g., learning models of the non-stationary task environment~\cite{xie2020deep}, identifying reusable skills~\cite{lu2020reset}, or estimating Bayesian sampling posteriors~\cite{fu2022model}, such prior work are not applicable to MCTS. Monte Carlo action sampling in MCTS relies on Upper Confidence Tree (UCT) or polynomial Upper Confidence Tree (pUCT)~\cite{auger2013continuous,matsuzaki2018empirical} to balance exploration and exploitation in large search spaces. To the best of our knowledge, there has been not existing work analyzing the transfer of knowledge from past MCTS searches to new tasks, thus enabling adaptive the UCT/pUCT rules in lifelong MCTS.


This paper proposes LiZero for Lipschitz lifelong planning using MCTS. We quantify a novel concept that the amount of knowledge transferable from a source task to the UCT/pUCT rule of a new task depends on both the similarity between the tasks as well as the confidence of the knowledge. More precisely, by defining a distance metric between two MDPs, we refine the concentration argument and drive a new adaptive UCT bound (denoted as aUCT in this paper) for lifelong MCTS. The aUCT is shown to consist of two components -- relating to (i) the Lipschitz continuity between the two tasks and (ii) the confidence of knowledge due to the numbers of samples in Monte Carlo action sampling. Our results enable the development a novel LiZero algorithm that makes use of prior experience to run an adaptive MCTS by simulating/traversing from the root node and selecting actions according to the aUCT rule, until reaching a leaf node. We also analyze aUCT's acceleration factor in terms of improved sampling efficiency due to cross-task transfer. It is shown that smaller task distance and higher confidence can both lead to higher acceleration in aUCT.

To support practical deployment of LiZero in lifelong planning, we need efficient solutions to compute aUCT in an online fashion. To this end, we develop practical algorithms to estimate various terms in aUCT and especially the distance metric between two MDPs, from either available state-action samples using a data-driven approach or a parameterized distance using a model-based (deep learning) approach. We provide rigorous analysis on the sampling complexity of the data-driven approach, to ensure arbitrarily small error with high probability, by modeling a non-stationary policy update process by a filtration -- i.e., an increasing sequence of $\sigma$-algebras. For the model-based approach, we obtain an upper bound using a parameterized distance of the neural network models. These results enable effective LiZero application to open world tasks. %{\color{green} Discuss the evaluation.}
We evaluate LiZero on a series of learning tasks with varying transition probabilities and rewards. It is shown that LiZero significantly outperforms MCTS and lifelong RL baselines (e.g., ~\cite{winands2024monte,kocsis2006bandit,chengspeculative,Schrittwieser_2020,brafman2002r,lecarpentier2021lipschitz}) in terms of 
%better knowledge transfer and 
faster convergence to higher optimal rewards. Utilizing the knowledge of only a few source tasks, LiZero achieves 3$\sim$4x speedup with about $31\%$ higher early reward in the first half of the learning process.


Our key contributions are as follows. First, we study theoretically the transfer of past experience in MCTS and develop a novel aUCT rule, depending on both Lipschitz continuity between tasks and the confidence of knowledge in Monte Carlo action sampling. It is proven to provide positive acceleration in MCTS due to cross-task transfer. Second, we develop LiZero for lifelong MCTS planning, with efficient methods for online estimation of aUCT and analytical error bounds. Finally, LiZero achieves significant speed-up over MCTS and lifelong RL baselines in lifelong planning.

In this section, we introduce the details of our evaluation framework. We primarily evaluate one representative RAG system and two representative GraphRAG systems, as illustrated in Figure~\ref{fig:framework}.

% \jt{briefly intro figure 1 here}

\subsection{RAG}
\vspace{-0.1in}
We adopt a representative semantic similarity-based retrieval approach as our RAG method~\cite{karpukhin2020dense}. Specifically, we first split the text into chunks, each containing approximately 256 tokens. For indexing, we use OpenAI’s text-embedding-ada-002 model, which has demonstrated effectiveness across various tasks~\cite{nussbaum2024nomic}. For each query, we retrieve chunks with Top-10 similarity scores. To generate responses, we employ two open-source models of different sizes: Llama-3.1-8B-Instruct and Llama-3.1-70B-Instruct~\cite{dubey2024llama}.

For single-document tasks, we generate a separate RAG system for each document, ensuring that queries corresponding to a specific document are processed within its respective indexed chunk pool. For multi-document tasks, we use a shared RAG system by indexing all documents together.

\vspace{-0.1in}
\subsection{GraphRAG}

We select two representative GraphRAG methods for a comprehensive evaluation, as shown in Figure~\ref{fig:framework}, namely KG-based GraphRAG and Community-based GraphRAG.

In the KG-based GraphRAG (KG-GraphRAG)~\cite{Liu_LlamaIndex_2022}, a knowledge graph is first constructed from text chunks using LLMs through triplet extraction. When a query is received, its entities are extracted and matched to those in the constructed KG using LLMs. The retrieval process then traverses the graph from the matched entities and gathers triplets \textit{(head, relation, tail)} from their multi-hop neighbors as the retrieved content. Additionally, for each triplet, we can retrieve the corresponding text associated with it. We define two variants of KG-GraphRAG: {\bf (1)} {\it KG-GraphRAG (Triplets)}, which retrieves only the triplets, and {\bf (2)} {\it KG-GraphRAG (Triplets+Text)}, which retrieves both the triplets and their associated source text. We implement the KG-GraphRAG methods using LlamaIndex~\cite{Liu_LlamaIndex_2022}~\footnote{https://www.llamaindex.ai/}.


For the Community-based GraphRAG~\cite{edge2024local}, in addition to generating KGs using LLMs, hierarchical communities are constructed using graph community detection algorithms, as shown in Figure~\ref{fig:framework}. Each community is associated with a corresponding text summary or report, where lower-level communities contain detailed information from the original text. The higher-level communities further provide summaries of the lower-level communities. Due to the hierarchical community structure, there are two primary retrieval methods for retrieving relevant information given a query: {\bf Local Search and Global Search}.  In Local Search, entities, relations, their descriptions, and lower-level community reports are retrieved based on entity matching between the query's extracted entities and the constructed graph. We refer to this method as {\it Community-GraphRAG (Local)}. In Global Search, only high-level community summaries are retrieved based on semantic similarity to the query. We refer to this method as {\it Community-GraphRAG (Global)}. The Community-GraphRAG methods are implemented using Microsoft GraphRAG~\cite{edge2024local}\footnote{https://microsoft.github.io/graphrag}. 
% \yu{Would it be more clear if we can also visualize such second-level ablation in Figure 1, e.g., KG-GraphRAG(Triplets)/KG-GraphRAG(Triplets+Text)/Community-GraphRAG(Global)/Community-GraphRAG.}

To ensure a fair comparison, we adopt the same settings for both RAG and GraphRAG methods. This includes the chunking strategy, embedding model, and LLMs. We select two representative RAG tasks, i.e., Question Answering and Query-based Summarization, to evaluate RAG and GraphRAG simultaneously.

% \yu{missing something?}



\section{Our Proposed Solution}

\subsection{Deriving adaptive Upper Confidence Bound (aUCT)}

To derive the proposed aUCT rule, we consider set of $m$ past known MDPs $\mathcal{M}_1,\ldots,\mathcal{M}_m$ and their leaned search policies $\pi_1,\ldots\pi_m$. Let $S$ and $A$ be their state and action spaces, respectively\footnote{Without loss of generality, we assume that the MDPs have the same state and action spaces. Otherwise, we can consider the extended MDPs defined on the union of their state and action spaces.}, $N_i(s,a)$ be the visit count of MPD $\mathcal{M}_i$ to state-action pair $(s\in S,a\in A)$, $W(s,a)$ to denote its sampled return, and $Q^{N_i}_{\mathcal{M}_i}(s,a)=W_i(s,a)/N_i(s,a)$ be the learned estimate for Q-value of MDP $\mathcal{M}_i$. Our goal is to apply these knowledge toward learning a new MDP, denoted by $\mathcal{M}$. To this end, we derive a new Lipschitz upper confidence bound for $\mathcal{M}$, which utilizes and transfers the knowledge from past MDPs $\mathcal{M}_1,\ldots,\mathcal{M}_N$, thus obtaining an improved Monte Carlo action sampling strategy that limits the tree search on $\mathcal{M}$ to a smaller subsets of sampled actions. We use $N(s,a)$ to denote the visit count of the new MDP to $(s\in S,a\in A)$, $W(s,a)$ to denote the sampled return, and thus $Q^N_{\mathcal{M}}(s,a)=W(s,a)/N(s,a)$ to denote its current Q-value estimate. 

Our key idea in this paper is that an improved upper confidence bound for the new MDP $\mathcal{M}$ can be obtained by (i) analyzing the Lipschitz
continuity between the past and new MDPs with respect to the upper confidence bounds and (ii) taking into account the confidence and aleatory uncertainty of the learned Q-value estimates to determine to what extent the learned knowledge from each $\mathcal{M}_i$ is pertinent. Intuitively, the more similar $\mathcal{M}$ and $\mathcal{M}_i$ are and the more samples (and thus higher confidence) we have in the learned Q-value estimates, the less exploration we would need to perform for solving $\mathcal{M}$ through MCTS. Our analysis will lead to an improved upper confidence bound that guides the MCTS on the new MDP $\mathcal{M}$ over a much smaller subset of action samples, thus significantly improving the search performance. We start with introducing a definition of the distance between any two given MDPs, $\mathcal{M}=\langle R,P\rangle, \ {\mathcal{M}}^{\prime} = \langle {R}^{\prime},{P}^{\prime}\rangle$, with reward functions $R,R'$ and state transitions $P,P'$, respectively. We choose a positive scaling factor $\kappa>0$ to combine the distances with respect to transition probabilities and rewards. Proofs of all theorems and corollaries are presented in the appendix.


\begin{definition}
\label{def:gobal}
Give two MDPs $\mathcal{M}=\langle R,P\rangle, \ {\mathcal{M}}^{\prime} = \langle {R}^{\prime},{P}^{\prime}\rangle$, and a distribution for sampling the state transitions $\mathcal{U}:\mathcal{S}\times \mathcal{A} \times \mathcal{S}' \rightarrow[0,1]$,
we define the pseudometric between the MDPs
as:
\begin{equation}
\begin{aligned}
d(\mathcal{M},\mathcal{M}^{\prime}) &= \Delta R+ \kappa\cdot \Delta P \\
&= %\sum_{s\in\mathcal{S}}\sum_{a\in\mathcal{A}}
\mathbb{E}_{(s,a,s')\sim\mathcal{U}}
\left[|R_s^a-{R}'^{a}_s| + \kappa
|P_{ss^{\prime}}^{a}-{P'_{ss^{\prime}}}^{a} |\right].
\end{aligned} \nonumber
\end{equation}
\end{definition}
Here $d(\mathcal{M},\mathcal{M}^{\prime})$ is our definition of distance between two MDPs, $\mathcal{M}$ and $\mathcal{M}'$. We choose $\mathcal{U}$ to be uniform distribution for sampling the state transitions in this paper. In Section~\ref{sec:distance}, we discuss practical algorithms to estimate the distance metric between two MDPs, from either available state-action samples using a data-driven approach or a parameterized distance using a model-based (deep learning) approach. The sampling complexity and error bounds are also analyzed. 


Next, we prove the main result of this paper and show that the upper confidence bounds of $\mathcal{M}$ and $\mathcal{M}'$ is Lipschitz continuous with respect to distance $d(\mathcal{M},\mathcal{M}^{\prime})$. We obtain a new upper confidence bound for $\mathcal{M}$, by transfer the knowledge from the learned Q-value estimates $Q^{N'}_{\mathcal{M}'}(s,a)=W'(s,a)/N'(s,a)$ of MDP $\mathcal{M}'$. Obviously, the bound also depends on the confidence of learned Q-value estimates, relating to the visit counts $N(s,a)$ and $N'(s,a)$.



\begin{theorem}[Lipschitz aUCT Rule]
\label{Optimal_Q_Lipschitz}
Consider two MDPs \( M \) and \({M}'\) with visit count $N,N'$ and corresponding estimate Q-values $Q_M^{N}(s,a), Q_{M^\prime}^{N'}(s,a)$, respectively. With probability at least $(1-\delta)$ for some positive $\delta>0$, we have
\begin{equation}
\label{eqn:aUCT}
\begin{aligned}
    \left|Q_{\mathcal{M}}^{N}(s,a)-Q_{\mathcal{M}^\prime}^{N'}(s,a)\right|\leq L\cdot d(\mathcal{M},\mathcal{M}') + P(N,N')
\end{aligned} 
\end{equation}
where $L={1}/({1-\gamma})$ is a Lipschitz constant, $d(\mathcal{M},\mathcal{M}')$ is the distance between MDPs, and $P(N,N')$ is given by
\begin{equation}
\label{eqn:aUCT1}
P(N,N') = \frac{2R_{\max}}{1-\gamma}\sqrt{\frac{\ln(2/\delta)}{2\cdot {\rm min}(N,N')}}
\end{equation}
\end{theorem}
In the theorem above, we show that the estimate Q-values between two MDPs are bounded by two terms, i.e., a Lipschitz continuity term depending on the distance $d(\mathcal{M},\mathcal{M}')$ between the two environments and a confidence term depending on the number $N, N'$ of samples used to estimate the Q-values. The Lipschitz continuity term measures how much the learned knowledge of source MDP $\mathcal{M}$ is pertinent to the new MDP $\mathcal{M}'$, while the confidence terms $P(N,N')$ quantifies the sampling bias arising from statistical uncertainty due to limited sampling in MCTS. We note that as the number of samples $N$ goes to infinity, we have $Q_{\mathcal{M}}^{N}(s,a)\rightarrow Q_{\mathcal{M}}^{*}(s,a)$ in Theorem~3.2, approaching the true Q-value $Q_{\mathcal{M}}^{*}(s,a)$ of the new MDP. Our theorem effectively provides an upper confidence bound for the true  Q-value of the new MDP, based on knowledge transfer from the source MDP. We also note that as both numbers $N,N'$ goes to infinity, the confidence term becomes $P(N,N')\rightarrow 0$. Our theorem recovers the Lipschitz lifelong RL~\cite{lecarpentier2021lipschitzlifelongreinforcementlearning} as a special case of our results, with respect to the true Q-values of the two MDPs. 

We apply Theorem~3.2 to MCTS-based lifelong planning with a non-stationary series of $m$ tasks, $\mathcal{M}_1,\ldots,\mathcal{M}_m$. Our goal is to obtain an improved bound on the true Q-value of the new task $\mathcal{M}$ based on knowledge transfer. To this end, we independently apply the knowledge from each past MDP, i.e., $Q^{N_i}_{\mathcal{M}_i}(s,a)=W_i(s,a)/N_i(s,a)$, to the new MDP. By taking the minimum of these bounds and making $N\rightarrow \infty$, it provides a tightest upper bound on the true Q-value $Q_{\mathcal{M}}^{*}(s,a)$ of the new MDP, which is defined as our aUCT bound, as it adaptively transfers knowledge from past tasks to the new tasks in MCTS-based lifelong planning. The result is summarized in the following corollary.

\begin{corollary}[aUCT bound in lifelong planning]
\label{cor:MDPS} 
Given MDPs $\mathcal{M}_1,\ldots,\mathcal{M}_m$, the new MDP's true Q-value is bounded by $Q_{\mathcal{M}}^{*}(s,a)\le U_{\rm aUCT}$ with probability at least $(1-\delta)$. The aUCT bound $U_{\rm aUCT}$ is given by 
\begin{equation}
\begin{aligned}   
U_{\rm aUCT}(s,a) \triangleq \min_{1\leq i\leq m}  \Bigg[ Q_{M_i}^{N_i}(s,a) + L\cdot d(\mathcal{M},\mathcal{M}_i)  +  \frac{2R_{\max}}{1-\gamma} \sqrt{\frac{\ln(2/\delta)}{2N_i(s,a)}} \Bigg]
\end{aligned}
\end{equation}
\end{corollary}
Obtaining this corollary is straightforward from Theorem~3.2 by taking $N\rightarrow \infty$ and considering the tightest bound of all knowledge transfers. In the context of MCTS-based lifelong planning, the more knowledge we have from solving past tasks, the more likely we can easily plan a new task, as the aUCT bound $U_{\rm aUCT}(s,a)$ is taken over the minimum of all past tasks. The confidence of past knowledge, i.e., the statistical uncertainty due to sampling number $N_i$, also affects the knowledge transfer to the new task.


\subsection{Our Proposed LiZero Algorithm Using aUCT}

We use the derived aUCT to design a highly efficient LiZero algorithm for MCTS-based lifelong planning. The LiZero algorithm transfers knowledge from past known tasks by computing $U_{\rm aUCT}(s,a)$ in Corollary~3.3. It requires efficient estimate of the distance $d(\mathcal{M},\mathcal{M}_i)$ (as defined in Definition~3.1) between the source MDPs and the new (target) MDP. We will present practical algorithms for such distance estimate in the next section and present analysis on the sampling complexity and error bounds. We will first introduce our LiZero algorithm in this section. We note that, during MCTS, direct exploration/search in the new task $\mathcal{M}$ also produces new knowledge and leads to improved UCT bound of $\mathcal{M}$. Therefore, our proposed LiZero combines both knowledge transfer through $U_{\rm aUCT}(s,a)$ and knowledge from direct exploration/search in $\mathcal{M}$. 




The search in our proposed LiZero algorithm is divided into three stages, repeated for a certain number of simulations. First, each simulation starts from the internal root state and finishes when the simulation reaches a leaf node. Let $Q^N_{\mathcal{M}}(s,a)=W(s,a)/N(s,a)$ be the current estimate of the new MDP and $N(s)=\sum_{a\in \mathcal{A}} N(s,a)$ be the visit count to state $s\in\mathcal{S}$. For each simulated time-step, LiZero chooses an action $a$ by maximizing a combined upper confidence bound based on aUCT, i.e.,
\begin{equation}
a={\rm arg} \max_a \min \left[ \frac{W(s,a)}{N(s,a)} + C\sqrt{\frac{\ln N(s)}{N(s,a)}}, U_{\rm aUCT}(s,a)\right] \nonumber 
\end{equation}
In practice, we can also use the maximum possible return $R_{\max}/(1-\gamma)$ as an initial value of the search. Next, at the final time-step of the simulation, the reward and state are computed by a dynamics function. A new node, corresponding to the leaf state, is then added to the search tree. Finally, at the end of the simulation, the statistics along the trajectory are updated. Let $G$ be the accumulative (discounted) reward for state-action $(s,a)$ from the simulation. We update the statistics by:
\begin{eqnarray}
& & Q^{N+1}_{\mathcal{M}}(s,a) \coloneq \frac{N(s,a)\cdot Q^{N}_{\mathcal{M}}(s,a)+G}{N(s,a)+1}, \nonumber \\
& & N(s,a) \coloneq  N(s,a) +1. \nonumber
\end{eqnarray}


Intuitively, at the start of task $\mathcal{M}$'s MCTS, there are not sufficient samples available, and thus $U_{\rm aUCT}(s,a)$ serves as a tighter upper confidence bound than that resulted from the Monte Carlo actions sampling in $\mathcal{M}$. As more samples are obtained during the search process, the standard UCT bound is expected to become tighter than $U_{\rm aUCT}(s,a)$. The use of both bounds will ensure both efficient knowledge transfer and task-specific search. The pseudo-code of LiZero is provided in Appendix A.2.


For the proposed LiZero algorithm, we prove that it can result in accelerated convergence in MCTS. More precisely, we analyze the sampling complexity for the learned Q-value estimate $Q^N_{\mathcal{M}}(s,a)$ to converge to the true value $Q^{*}_{\mathcal{M}}(s,a)$, and demonstrate a strictly positive acceleration factor, compared to the standard UCT. The results are summarized in the following theorem.


\begin{theorem}
\label{the:converage}
To ensure the convergence in a finite state-action space, $\max_{(s,a)}|Q^{N}_{\mathcal{M}}(s,a)-Q_{\mathcal{M}}^{*}(s,a)|\leq \epsilon$ with probability \(1-\delta\), the number of samples required by standard UCT is 
\begin{equation}
\begin{aligned}
\tilde{O}\left(\frac{|\mathcal{S}|\cdot|\mathcal{A}|}{(1-\gamma)^3\epsilon^2}\ln\frac{1}{\delta}\right),
\end{aligned}
\end{equation}
while the proposed LiZero algorithm requires:
\begin{equation}
\begin{aligned}
    \tilde{O}\left(\frac{1}{\Gamma} \cdot \frac{|\mathcal{S}|\cdot|\mathcal{A}|}{(1-\gamma)^3\epsilon^2}\ln\frac{1}{\delta}\right),
\end{aligned}
\end{equation}
where $\Gamma> 1$ is an acceleration factor given by
\begin{equation}
\begin{aligned}
\Gamma =\frac
{\sum_{(s,a)\in \mathcal{S}_1\cup \mathcal{S}_0 } \frac{1}{(\Delta^{\mathcal{M}}_{(s,a)})^2}}
{\sum_{(s,a)\in\mathcal{S}_1} (1) + 
\sum_{(s,a)\in\mathcal{S}_{0}} \frac{1}{(\Delta^{\mathcal{M}}_{(s,a)})^2}},
\end{aligned}
\end{equation}
and \( \mathcal{S}_1 = \{(s, a) \mid \exists i : U_{\rm aUCT}(s, a) < Q^{*}_{\mathcal{M}}(s, a^{*})\} \) is a state-action set where $U_{\rm aUCT}$ of action $a$ is lower than the optimal return of $a^{*}$ in state $s$;
and $\Delta^{\mathcal{M}}_{(s,a)} \propto [Q_{\mathcal{M}}^{*}(s,a^{*}) - Q_{\mathcal{M}}^{*}(s,a)]$ is a normalized advantage in the range of $[0, 1]$.
\end{theorem}


The theorem shows that LiZero achieves a strictly improved acceleration $\Gamma>1$ with a reduced sampling complexity (by $1/\Gamma$), in terms of ensuring convergence to the optimal estimates, i.e., $\max_{(s,a)}|Q^{N}_{\mathcal{M}}(s,a)-Q_{\mathcal{M}}^{*}(s,a)|\leq \epsilon$ with probability \(1-\delta\). Since the normalized advantage $\Delta^{\mathcal{M}}_{(s,a)}$ is in $[0,1]$, we have $1/\Delta^{\mathcal{M}}_{(s,a)}\ge 1$. It is then easy to see that the value of $\Gamma$ depends on the cardinality $|\mathcal{S}_1|$ and the normalized advantage $\Delta^{\mathcal{M}}_{(s,a)}$. More precisely, LiZero achieves higher acceleration when (i) our $aUCT$ makes more actions $a$ less favorable, as $U_{\rm aUCT}(s, a) < Q^{*}_{\mathcal{M}}(s, a^{*})$ implies that the sub-optimality of action $a$ in $s$ can be more easily determined due to aUCT; or (ii) $aUCT$ helps establish tighter bounds in cases with a smaller advantage, which naturally requires more samples to distinguish the optimal actions -- since $\Gamma$ increases as the normalized advantage becomes smaller for $(s,a)\in \mathcal{S}_1$, while being larger for $(s,a)\in \mathcal{S}_0$. These explain LiZero's ability to achieve much higher acceleration and lower sampling complexity, resulted from significantly reduced search spaces. We will evaluate this acceleration/speedup through experiments in Section~\ref{sec:eval}.




\section{Estimaing aUCT in Practice}
\label{sec:distance}

To deploy LiZero in practice, we need to estimate aUCT, and in particular, the distance $d_{\mathcal{M}, \mathcal{M}_i}$ between two MDPS. Sampling all transitions based on a uniform distribution $\mathcal{U}$, as defined in Definition~3.1, is clearly too expensive. Thus, we develop efficient algorithms to estimate the distance metric, from either available state-action samples using a data-driven approach or a parameterized distance using a model-based (deep learning) approach. In this section, we also provide rigorous analysis on the sampling complexity and error bounds of the proposed algorithms for distance estimate. The results allow us to readily implement LiZero in practical environments. We will late evaluate the performance of different distance estimaters in Section~\ref{sec:eval} and present the numerical results.

More precisely, we first propose an algorithm to estimate the distance between two MDPs, $\mathcal{M}$ and $\mathcal{M}'$, using trajectory samples drawn from their search policies during MCTS and then making the use of importance sampling to mitigate the bias. We will start with analyzing a stationary search policy and then extend the results to a non-stationary policy update process, by modeling it as a filtration – i.e., an increasing sequence of $\sigma$-algebra. Next, since many practical problems are faced with extremely large or even continuous action and state spaces (i.e., $\mathcal{A}$ and $\mathcal{S}$), we further consider a model-based approach by learning neural network approximations of the MDPs -- denoted by parameter sets $\phi$ and $\phi'$, respectively -- and then computing an upper bound on the distance using a parameterized distance of the neural network models. Analysis on sampling complexity and error bounds are provided as theorems in this section. 



\subsection{Sample-based Distance Estimate}

During MCTS, transition samples are collected from the search to train a search policy $\pi$. It is easy to see that we can leverage these transition samples to estimate distance $d(\mathcal{M},\mathcal{M}')$ between two MDPs, as long as we address the bias arising from gap between search policy $\pi$ and desired sampling distribution $\mathcal{U}$ in the distance definition $d(\mathcal{M},\mathcal{M}')$. It also allows us to obtain a consistent estimate of MDP distance, without depending on the search policy that is updated during training. We note that this bias can be addressed by importance sampling. 

Let $\Delta X(s, a) = \Delta R_{s}^a + \kappa \Delta P_{s}^a$ be the distance metric for a given state-action pair $(s,a)$. We can rewrite the distance as $d(\mathcal{M},\mathcal{M}')=\mathbb{E}_{(s,a)\sim \mathcal{U}}[ \Delta X(s, a)]$. We denote $p_\mathcal{U}(s,a)$ as the probability (or density) of sampling $(s, a)$ according to distribution $\mathcal{U}$. Importance sampling implies:
\begin{equation}
\begin{aligned}
    \mathbb{E}_{(s,a)\sim \mathcal{U}} [\Delta X(s, a)] = \mathbb{E}_{(s,a)\sim \pi} \left[\frac{p_\mathcal{U}(s,a)}{\pi(s,a)}\cdot \Delta X(s, a)\right],
\end{aligned}
\end{equation}
which can be readily computed from the collected transition samples, following the search policy $\pi(s,a)$. Therefore, for a given set of samples $\{(s_i,a_i),\forall i=1,\ldots,n\}$ collected from a search policy $\pi(s,a)$, we can estimate the distance by the empirical mean:
\begin{equation}
\begin{aligned}
    \hat{d}_{1} = \frac{1}{n}\sum_{i=1}^{n} w_i \Delta X(s_i,a_i), \ {\rm with} \ w_i = \frac{\mathcal{U}(s_i,a_i)}{\pi(s_i,a_i)}
\end{aligned}
\end{equation}
where $w_i$ is the importance sampling weight.


As long as the state-action pairs with $\pi(s, a) > 0$ cover the support of $\mathcal{U}$, this estimator satisfies $\mathbb{E}[\hat{d}_{\mathcal{1}}] = d(\mathcal{M}, \mathcal{M}^{\prime})$, meaning it is unbiased.
Let $\alpha$ be the "coverage" of policy $\pi(s, a)$, i.e., $\pi(s, a) \geq \alpha > 0$, and $p_\mathcal{U}^{\max}$ be the maximum desired sampling probability.
We summarize this result in the following theorem and state the sampling complexity for estimator $\hat{d}_{1}$ to $\epsilon$-converge to $d(\mathcal{M}, \mathcal{M}^{\prime})$.




\begin{theorem}[Sampling Complexity under Stationarity]
\label{the:err_signal_policy}
Assume that for any $(s, a)$, the reward plus transition difference is bounded, i.e., $\Delta X(s, a) \in [0, b]$, and that there exists $\alpha$ such that $\pi(s, a) \geq \alpha > 0$.
When $n$ independent samples are used to estimate $\hat{d}_{1}$, we have
\begin{equation}
\begin{aligned}
\text{Pr}\{|\hat{d}_{1}-d(\mathcal{M},\mathcal{M}^{\prime})|\leq \epsilon\} \geq 1-\delta
\end{aligned}
\end{equation}
\end{theorem}
for any $\delta \in (0, 1)$, if the number of samples satisfy
\begin{equation}
\begin{aligned}
    n \geq \frac{1}{2\epsilon^2} b^2\left(\frac{p_\mathcal{U}^{\max}}{\alpha}\right)^2 \cdot \ln\left(\frac{2}{\delta}\right).
\end{aligned}
\end{equation}
Thus, we obtain a convergence guarantee in the sense of arbitrarily high probability $1-\delta$ and arbitrarily small error $\epsilon$, for estimating $d(\mathcal{M},\mathcal{M}^{\prime})$ using $\hat{d}_{1}$. $\hat{d}_{1}$ is unbiased and ensures convergence to the true distance as the number of samples is sufficiently large.

We note that in many practical settings, the search policy $\pi$ would not stick to a stationary distribution. In contrast, it is continuously updated in each iteration, resulting in a non-stationary sequence of policies over time, i.e., $\pi_1, \pi_2, \dots, \pi_k$. Thus, the transition samples $(s_k, a_k)$'s we obtain at each step $k$ for estimating the distance $d(\mathcal{M},\mathcal{M}^{\prime})$ are indeed drawn from a different $\pi_k$. We cannot assume that the samples follow a stationary distribution (nor that $\{\Delta X^w_k\}$ are i.i.d.) in importance sampling. To address this problem, we model the non-stationary process of policy updates as a filtration – i.e., an increasing sequence of $\sigma$-algebra. In particular, we make the following assumption: at the $k$-th sampling step, the environment is forcibly reset to a predetermined policy $\pi_k$ or independently draws a state from an external memory. This assumption is reasonable because, in many episodic learning scenarios, the environment is inherently divided into episodes: at the beginning of each episode, the state is reset to some initial distribution (e.g., the opening state in Atari games or the initial pose in MuJoCo). This naturally results in the ``reset" assumption. 

In this setup, the policy $\pi_{k}$ at step $k$ is determined by information at step $k-1$ or earlier. Consequently, once $\pi_k$ is fixed, the distribution (marginal) of $\Delta X^w_k = \frac{p_\mathcal{U}(s_k, a_k)}{\pi_{k}(s_k, a_k)}\Delta X(s_k, a_k)$ is also fixed. Therefore, we can establish the filtration $\{\mathcal{F}_k, k=1,2,\ldots\}$ as follows: 
\begin{equation}
\begin{aligned}
    \mathcal{F}_{k-1} = \sigma\{\pi_1,...,\pi_k,(s_1,a_1),...,(s_{k-1},a_{k-1})\},
\end{aligned}
\end{equation}
where $\sigma\{\cdot\}$ denotes the smallest $\sigma$-algebra generated by the random elements. Thus, we obtain: 
\begin{equation}
\label{eqn:Martingale}
\begin{aligned}
    \mathbb{E}[\Delta X_k|\mathcal{F}_{k-1}] &= \mathbb{E}_{(s_k,a_k)\sim \pi_k} \left[\frac{p_\mathcal{U}(s_k,a_k)}{\pi_k(s_k,a_k)}\cdot \Delta X(s_k, a_k)\right]\\
    & = \mathbb{E}_{(s_k,a_k)\sim \mathcal{U}} [\Delta X(s,a)] \\
    & = d(\mathcal{M},\mathcal{M}^{\prime})
\end{aligned}
\end{equation}
This allows us to obtain another empirical estimator $\hat{d}_{2}$ using the filtration model. We analyze the sampling complexity of $\hat{d}_{2}$ and summarise the results in the following theorem.
\begin{theorem}[Sampling Complexity under Non-Stationarity]
\label{the:err_multi_policy}
Under the same conditions as Theorem~\ref{the:err_signal_policy}
when $n$ independent samples are used to estimate $\hat{d}_{2}$, we have
\begin{equation}
\begin{aligned}
\text{Pr}\{|\hat{d}_{2}-d(\mathcal{M},\mathcal{M}^{\prime})|\leq \epsilon\} \geq 1-\delta
\end{aligned}
\end{equation}
for any $\delta \in (0, 1)$, if the number of samples satisfy
\begin{equation}
\begin{aligned}
    n \geq \frac{2}{\epsilon^2} b^2 \left(\frac{p_\mathcal{U}^{\max}}{\alpha}\right)^2\cdot \ln \left(\frac{2}{\delta}\right).
\end{aligned}
\end{equation}
\end{theorem}
It implies that more samples are needed considering the non-stationarity of policy update process for distance estimate.




\subsection{Model-based Distance Estimate}
When the action and state spaces, $\mathcal{A}$ and $\mathcal{S}$ are very large or even continuous, employing the sample based method will become increasingly expensive. Therefore, we propose a model-based approach to first approximate the dynamics of MDPs $\mathcal{M}$ and $\mathcal{M}'$ using two neural networks and then estimate $d(\mathcal{M},\mathcal{M}^{\prime})$ based on the parameterized distance between the neural networks. 


To this end, we need to establish a bound on $d(\mathcal{M},\mathcal{M}^{\prime})$ using the distance between their neural network parameters. 
We use a neural network $\Psi_{\phi}: \mathcal{S} \times \mathcal{A} \rightarrow \Delta(\mathcal{S})$ to model the MDP dynamics.
Many model-based learning algorithms, such as %{\color{green} cite some model-based algos}
PILCO~\cite{deisenroth2011pilco},MBPO~\cite{janner2019trust},PETS~\cite{chua2018deep},MuZero~\cite{schrittwieser2020mastering}, can be employed to learn the models of $\mathcal{M}$ and $\mathcal{M}'$.
Let $\phi$ be the neural network parameters of MDP $\mathcal{M}$ and $\phi'$ be the neural network parameters of MDP $\mathcal{M}'$. We define a distance in the parameter space:
\begin{equation}
\begin{aligned}
   \hat{d}_{para}  =  \rho(\phi,\phi') \geq 0,
\end{aligned}
\end{equation}
where \( \rho \) is a distance or divergence measure in the parameter space, such as the \( \ell_2 \)-norm, \( \ell_1 \)-norm, or certain kernel distances.
Intuitively, if $\phi$ and $\phi'$ are very close, it indicates that the two neural networks are similar in fitting the dynamics of the respective MDPs. It suggests that the two MDPs should have a small distance.
To provide a more rigorous characterization of this concept, we present the following theorem, which demonstrates that under proper assumptions, the distance $\hat{d}_{para}$ based on neural network parameters can serve as an upper bound for the desired $d(\mathcal{M},\mathcal{M}^{\prime})$. Let $\kappa={R_{\max}\gamma}/({1-\gamma})$ be a constant.


\begin{theorem}
\label{the:network}
If the neural networks modeling $\mathcal{M}$ and $\mathcal{M}'$ satisfy the Lipschitz condition, i.e., there exists a constant $L > 0$ such that $\forall (s, a)$,  
\( ||\Psi_{\phi}(s, a) - \Psi_{\phi'}(s, a)||_1 \leq L \cdot \rho(\phi, \phi'), \)
then we have:
\begin{equation}
\begin{aligned}
   d(\mathcal{M},\mathcal{M}^{\prime}) \le (1+\kappa)L \hat{d}_{\text{para}}.
\end{aligned}
\end{equation}
\end{theorem}
The theorem indicates that by learning neural networks to model the MDP dynamics, we can estimate the distance $d(\mathcal{M},\mathcal{M}^{\prime})$ by estimating the distance between the neural network parameters. This parameterized distance can be computed for event continuous action and state spaces. 


















\subsection{Datasets}

\name is evaluated on datasets and tasks in the biological and medical domains: cellular reprogramming experiments~(Figure~\ref{fig:data}a) and patient routine laboratory tests~(Figure~\ref{fig:data}b).

\xhdr{Cellular developmental trajectories} We introduce a novel benchmarking dataset, WOT. It is constructed using the Waddington-OT model, which simulates single-cell transcriptomic profiles of developmental time courses for individual cells \cite{schiebinger2019-ie}~(Figure~\ref{fig:data}a; Table~\ref{tab:data}). We also construct a paired counterfactual benchmarking dataset, WOT-CF (Table~\ref{tab:data}). We obtain condition embeddings of the activated transcription factors from ESM-2~\cite{lin2022language}. Refer to Appendix~\ref{appendix:cells} for further details.


\xhdr{Patient lab test trajectories}
%
We construct two real-world patient datasets of routine laboratory tests from eICU~\cite{pollard2018eicu} and MIMIC-IV~\cite{johnson2024mimic, johnson2023mimic, goldberger2000physiobank}~(Figure~\ref{fig:data}b; Table~\ref{tab:data}). In addition to a random split, we construct data splits with different levels of train/test split similarities using SPECTRA~\cite{ektefaie2024evaluating} to evaluate model generalizability~(Appendix Figure~\ref{fig:supp_spectra_cso}). For condition embeddings, we leverage pretrained embeddings of clinical codes from a clinical knowledge graph that integrates six existing databases of clinical vocabularies used in electronic health records~\cite{johnson2024unified}. Refer to Appendix~\ref{appendix:labs} for more details. 



\begin{figure}[ht]
\begin{center}
\centerline{\includegraphics[width=\columnwidth]{FIG/figure_domains.pdf}}
\caption{\name is evaluated on two real-world domains involving multivariate trajectories: \textbf{(a)}~cellular development and \textbf{(b)}~patient health. \textbf{(a)}~To study cellular development, fibroblast cells derived from mice can be artificially reprogrammed into various other cell states \textit{in vitro}. A cell's state is defined by its gene expression. Throughout reprogramming, a cell activates transcription factor (TF) genes at different time points to change its gene expression, thereby influencing its developmental trajectory. In this illustration, a mouse fibroblast is being reprogrammed over the span of 20 days~(D0-D20); color and shape represent cell state. On day 8, if the cell activates the Obox6 TF, the cell is on the path toward becoming an induced pluripotent stem cell (iPSC); whereas if it activates the Neurod4 TF, it is on the path toward becoming a neuron or astrocyte. \textbf{(b)}~The health of a human patient is often monitored through lab tests (e.g. blood sodium level, white blood cell count). The history of lab results across multiple patient visits (V1-V9) as well as candidate clinical interventions (e.g.,~medication) can be used to infer the most likely future trajectory of the patient's health. Illustrations from NIAID NIH BIOART Source~(see References). \nocite{fibroblast1,fibroblast2, fibroblast3, astrocyte, progenitor, petri, generic_immune, cajal, neuron, unidentified_offtarget, mouse, patient, syringe, wbc, hemoglobin, vial}}
\label{fig:data}
\end{center}
\vskip -0.35in
\end{figure}



\begin{table}[ht]
\caption{\textbf{Dataset statistics.} We construct three core datasets: WOT~(cellular developmental trajectories), eICU~(patient lab tests), and MIMIC-IV~(patient lab tests). We also construct a paired counterfactual cellular trajectories dataset, WOT-CF. $N$ is the number of sequences (i.e.,~cellular developmental trajectories, patient lab test trajectories), $V$ is the number of measured variables (i.e.,~gene expression, lab test), and $L$ is the length of the sequences.}
\label{tab:data}
\vskip 0.15in
\begin{center}
\begin{small}
\begin{tabular}
{lcccc}
\toprule
\textbf{Dataset}  & $N$ & $V$ & \textbf{Mean $L$} & \textbf{Max $L$} \\
\midrule
WOT      & $3,000$ & $1,480$ & $27.03 \pm 6.04$ & $37$ \\
WOT-CF   & $2,546$ & $1,480$ & $27.01 \pm 5.98$ & $37$ \\
eICU     & $108,346$ & $17$ & $20.27 \pm 25.23$ & $858$ \\
MIMIC-IV & $156,310$ & $16$ & $15.56 \pm 24.43$ & $949$ \\
\bottomrule
\end{tabular}
\end{small}
\end{center}
\vskip -0.15in
\end{table}




\subsection{Setup}


\xhdr{Metrics}
%
We use standard metrics (MAE, RMSE, and $R^2$) to quantify sequence editing performance.

\xhdr{Baselines}
%
We evaluate \name against a traditional multivariate time series algorithm, Vector Autoregression~(VAR) model~\cite{lutkepohl2005new}. As \name can leverage any type of sequence encoder, we benchmark against the state-of-the-art condition-guided counterfactual sequence generation setup with different sequential data encoders: Transformer~\cite{vaswani2017attention, narasimhan2024time, jing2024towards, zhang2023survey} and xLSTM~\cite{beck2024xlstm}. We further evaluate \name against a state-of-the-art time series foundation model, MOMENT~\cite{goswami2024moment}; specifically, we finetune an adapter for the $1024$-dimensional embeddings generated by the frozen \texttt{MOMENT-1-large} embedding model. 


\xhdr{Ablations}
%
To investigate the effectiveness of the learned temporal concepts, we evaluate against an ablated model, SimpleLinear, in which temporal concepts are simply all ones; in other words, temporal concepts are not learned nor meaningful. This ablation is inspired by traditional linear models that excel when $\mathbf{x}_{t_j} \simeq \mathbf{x}_{t_i}$~\cite{toner2024analysis, ahlmann2024deep}. We also evaluate different versions of \name with and without an FFN layer in the concept encoder~$E$~(Appendix~\ref{appendix:figures}).

\xhdr{Implementation details}
%
Models are trained on a single NVIDIA A100 or H100 GPU. All models have comparable number of parameters as their \name-based counterparts. Refer to Appendix~\ref{appendix:implementation} for details and hyperparameter selection.


\section{Conclusions}

We study theoretically the transfer of past experience in MCTS-based lifelong planning and develop a novel aUCT rule, depending on both Lipschitz continuity between tasks and the confidence of knowledge in Monte Carlo action sampling. The proposed aUCT is proven to provide positive acceleration in MCTS due to cross-task transfer and enable the development of a new lifelong MCTS algorithm, namely LiZero. We also present efficient methods for online estimation of aUCT and provide analysis on the sampling complexity and error bounds. LiZero is implemented on a non-stationary series of learning tasks with varying transition probabilities and rewards. It outperforms MCTS and lifelong RL baselines with 3$\sim$4x speed-up in solving
new tasks and about 31\% higher early reward.



\section*{Impact Statement}
This paper proposes a novel framework for applying Monte Carlo Tree Search (MCTS) in lifelong learning settings, addressing the challenges posed by non-stationary environments and dynamic game dynamics. By introducing the adaptive Upper Confidence Bound for Trees (aUCT) and leveraging insights from previous MDPs (Markov Decision Processes), our work significantly enhances the efficiency and adaptability of decision-making algorithms across evolving tasks.

The broader societal implications of this research include its potential to improve AI applications in robotics, automated systems, and other domains requiring dynamic decision-making under uncertainty. For instance, this framework could be used in autonomous systems to adaptively respond to changing environments, thereby improving safety and reliability. At the same time, it is crucial to acknowledge and mitigate potential risks, such as unintended biases or over-reliance on prior knowledge that may not fully represent novel situations.

Ethical considerations for this work focus on its use in high-stakes applications, such as healthcare, finance, or defense, where decision-making under uncertainty could have significant consequences. Developers and practitioners should implement safeguards to ensure responsible deployment, including comprehensive testing in diverse scenarios and establishing clear boundaries for its use.

By advancing the state of the art in continual learning and decision-making, this research contributes to the development of more adaptable and intelligent AI systems while highlighting the importance of ethical and responsible innovation in AI technologies.

\nocite{langley00}


{
\small
\bibliography{ref}
\bibliographystyle{unsrtnat} 

}



%%%%%%%%%%%%%%%%%%%%%%%%%%%%%%%%%%%%%%%%%%%%%%%%%%%%%%%%%%%%%%%%%%%%%%%%%%%%%%%
%%%%%%%%%%%%%%%%%%%%%%%%%%%%%%%%%%%%%%%%%%%%%%%%%%%%%%%%%%%%%%%%%%%%%%%%%%%%%%%
% APPENDIX
%%%%%%%%%%%%%%%%%%%%%%%%%%%%%%%%%%%%%%%%%%%%%%%%%%%%%%%%%%%%%%%%%%%%%%%%%%%%%%%
%%%%%%%%%%%%%%%%%%%%%%%%%%%%%%%%%%%%%%%%%%%%%%%%%%%%%%%%%%%%%%%%%%%%%%%%%%%%%%%
\newpage
\appendix
\onecolumn

\section{Appendix / supplemental material}

\subsection{Proof of Theorem~\ref{Optimal_Q_Lipschitz}}

\begin{proof}{Proof of Theorem~\ref{Optimal_Q_Lipschitz}}
Since in the MCTS UCB algorithm, the estimated Q-values are obtained through multiple simulations, we need to analyze how the differences in simulation results between two MDPs affect the estimated Q-values.

However, due to the randomness involved in the simulation process of the two MDPs:
\begin{itemize}
    \item \textbf{Transition randomness: }Due to different transition probabilities, the two MDPs may move to different next states even when starting from the same state and action.
    \item \textbf{Action selection randomness: }When using the UCB algorithm, action selection depends on the current statistical information, which in turn relies on the past simulation results.
\end{itemize}
The randomness mentioned above makes it impossible for us to compare two independent random simulation processes directly~\cite{qiao2024br,gao2024cooperative,riis2024mastering,chen2024survey,zhang2025network,yin2025predefined}.

To eliminate the impact of randomness, we need to construct a coupled simulation process for the two MDPs in the same probability space, allowing for a direct comparison between them.
Then we will incorporate the additional errors caused by randomness into the analysis as error terms.
For this purpose, we present the following assumptions.
\begin{assumption}
Let us temporarily assume that the actions selected in each simulation are the same for the two MDPs.
\begin{itemize}
    \item \textbf{Initial action consistency:} The simulation starts from the same state$s$
    \item  \textbf{Action selection consistency:} The same action $a$ is chosen in each state.
\end{itemize}
\end{assumption}
Note: This is a strong assumption and may not hold in practice. We will discuss its impact later.

Thus, we can obtain the difference in cumulative rewards between the two MDPs in a single simulation as:
\begin{equation}
\begin{aligned}
\Delta G = G_M - G_{M^{\prime}} = \sum\limits_{t=0}^{T}\gamma^{t}(R(s_t^{M},a_t)-R^{\prime}(s_t^{M^{\prime}},a_t))
\end{aligned}
\end{equation}
Where \(s_t^{M}\) and \(s_t^{M^{\prime}}\) are the states of the two MDPs at step \(t\), and \(a_t\) is the action selected at step \(t\).

So we can get
\begin{equation}
\left|Q_M^{n_1}(s,a)-Q_{M^\prime}^{n_2}(s,a)\right| = \left|\frac{1}{n_1}\sum_{i=1}^{n_1}G_{M,i} - \frac{1}{n_2}\sum_{i=1}^{n_2}G_{M,i}\right|\leq \bar{\Delta G} =\left|\frac{1}{n} \sum_{i=1}^{n}\Delta G_i\right|
\end{equation}
where $n = \min\{n_1,n_2\}$
To estimate the expectation and variance of \(\Delta G\), we need to analyze how the differences in the state sequences affect the cumulative rewards.

We present several settings for the state differences.
\begin{itemize}
    \item \textbf{Probability of state difference:} At each time step \(t\), the probability that the states of the two MDPs differ is denoted as \(p_t\).
    \item \textbf{Initial state is the same: }\(p_0 = 0\).
    \item \textbf{State difference propagation:} Due to differences in transition probabilities, state differences may accumulate in subsequent time steps.
\end{itemize}
Since the probability of state differences occurring at each step is difficult to calculate precisely, we can use the total variation distance to estimate the probability of transitioning to different states.
We present the definition of the total variation distance between the transition probabilities of the two MDPs and a recursive method for calculating the probability of state differences.

\begin{definition}
Under action \(a_t\), starting from state \(s_t\), the total variation distance between the transition probabilities of the two MDPs is:
\begin{equation}
\begin{aligned}
D_{TV}(P,P^{\prime}) = \frac{1}{2}\sum\limits_{s^{\prime}}|P(s^{\prime}|s_t,a_t)-P^{\prime}(s^{\prime}|s_t,a_t)|
\end{aligned}
\end{equation}
\end{definition}
Thus, starting from the same state \(s_t\) and action \(a_t\), the probability that the two MDPs transition to different next states is at most \(D_{TV}(P, P^{\prime}) \leq \frac{\Delta P}{2}\).

Thus, the probability of state differences occurring can be recursively expressed as:
\begin{equation}
\begin{aligned}
p_{t+1} \leq p_{t} + (1-p_{t})\cdot D_{TV}(P,P^{\prime}) \leq p_t + \frac{\Delta P}{2}
\end{aligned}
\end{equation}
So
\begin{equation}
\begin{aligned}
p_t \leq t \cdot \frac{\Delta P}{2}
\end{aligned}
\end{equation}

Thus, at each time step \(t\), the expected difference in cumulative rewards is:
\begin{equation}
\begin{aligned}
\mathbb{E}[|\Delta G|] 
&=\mathbb{E}[\sum_{t=0}^T\gamma^{t}(R(s_t^{M},a_t)-R^{\prime}(s_t^{M^{\prime}},a_t))] \\
& = \sum_{t=0}^T\gamma^{t}(\underbrace{\mathbb{E}[R(s_t^{M},a_t)-R^{\prime}(s_t^{M},a_t)]}_{\text{The impact of reward function differences}}+ \underbrace{\mathbb{E}[R^{\prime}(s_t^{M},a_t)-R^{\prime}(s_t^{M^{\prime}},a_t)]}_{\text{Reward differences caused by state differences}}) \\
& \leq \sum_{t=0}^T\gamma^{t}(\Delta R + 2R_{\max}\cdot p_{t})\\
& = \frac{\Delta R}{1-\gamma} + \sum_{t=0}^T\gamma^{t}\cdot 2R_{\max}\cdot t \cdot \frac{\Delta P}{2}\\
& = \frac{\Delta R}{1-\gamma} + R_{\max}\Delta P\sum_{t=0}^{T}t\gamma^t\\
& = \frac{\Delta R}{1-\gamma} + R_{\max}\Delta P\cdot \frac{\gamma}{(1-\gamma)^2}
\end{aligned}
\end{equation}

To estimate the variance of the cumulative reward difference, since the cumulative reward is bounded, its variance is also finite.
We can easily obtain
\begin{equation}
\begin{aligned}
|\Delta G| \leq G_{\max} = \frac{2R_{\max}}{1-\gamma}
\end{aligned}
\end{equation}

According to Hoeffding:

\begin{equation}
\begin{aligned}
P(|\bar{\Delta G} - \mathbb{E}[\bar{\Delta G}]|\geq \epsilon) \leq 2\exp(-\frac{2n\epsilon^2}{G_{\max}^2})
\end{aligned}
\end{equation}

Thus, with probability at least \(1 - \delta\), we have:
\begin{equation}
\begin{aligned}
    |\hat{Q}_M^n(s,a)-\hat{Q}_{M^\prime}^n(s,a)|
    &\leq \mathbb{E}[|\Delta\bar{G}|] + G_{\max}\sqrt{\frac{\ln(2/\delta)}{2n}}\\
    & = \frac{\Delta R}{1-\gamma} + R_{\max}\Delta P\cdot \frac{\gamma}{(1-\gamma)^2} + \frac{2R_{\max}}{1-\gamma}\sqrt{\frac{\ln(2/\delta)}{2n}}\\
    & = \frac{1}{1-\gamma}(\Delta R + \frac{R_{\max}\gamma}{1-\gamma}\Delta P) +     \frac{2R_{\max}}{1-\gamma}\sqrt{\frac{\ln(2/\delta)}{2n}}\\
    & = L(\Delta R + \kappa \Delta P) + L_2
\end{aligned}
\end{equation}
    
\end{proof}






\subsection{Proof of Theorem~\ref{the:converage}}

\begin{proof}{Proof of Theorem~\ref{the:converage}}
First, we consider the case of a single MDP and assume that we have a "universal" upper bound \( U(s, a) \geq Q_{M}^{*}(s, a) \).


\begin{lemma}
Since \( U(s, a) \geq Q_{M}^{*} \) holds for all \( (s, a) \), and initially \( Q(s, a) \leq U(s, a) \), for any update, \( Q(s, a) \) maintains \( Q(s, a) \leq U(s, a) \) and \( Q(s, a) \geq (\text{a non-negative expected estimate}) \).
\end{lemma}

The above two points illustrate
Since we update using \( Q(s, a) = \min\{\hat{Q}(s, a), U(s, a)\} \)
And since \( U(s, a) \geq Q^{*}(s, a) \), during all sampling processes, if \( \hat{Q}(s, a) \) overestimates \( Q^{*}(s, a) \) significantly, it will still be truncated by \( U(s, a) \), ensuring that \( Q(s, a) \leq U(s, a) \).
When \( \hat{Q}(s, a) \) gradually approaches \( Q^{*}(s, a) \), it will no longer be truncated. This does not hinder the convergence of \( Q \) to \( Q^{*} \).


\begin{theorem}[Convergence in a Single MDP]
If there are infinitely many samples for each state \(s\) and its available actions \(a\) (i.e., every branch in the MCTS search tree is "continuously" expanded), then the \(Q(s, a)\) generated by the above update formula almost surely converges to \(Q_{M}^{*}(s, a)\).
\end{theorem}




Now we aim to demonstrate that after completing certain MDPs (tasks) \(\bar{M}_1, \bar{M}_2, \dots, \bar{M}_m\), and then switching to a new MDP \(M\), the algorithm achieves faster convergence.

First, we analyze the classic scenario without upper bounds. In a finite state-action space, to achieve the desired outcome with high probability \(1-\delta\): 
\begin{equation}
\begin{aligned}
    \max_{(s,a)\in\mathcal{S}\times \mathcal{A}}|Q_{n}(s,a) - Q_{M}^{*}(s,a)| \leq \epsilon
\end{aligned}
\end{equation}

The standard UCT/UCB theory typically provides a time complexity of \( \tilde{O}\left(\frac{|\mathcal{S}||\mathcal{A}|}{(1-\gamma)^3 \epsilon^2} \ln\frac{1}{\delta}\right) \). To prove this theorem, we just need to analyze the acceleration factor $\Gamma$, comparing the sampling complexity of our aUCT and standard UCT.

More specifically, if we examine each specific \((s, a)\), the analysis often resembles that of multi-armed bandits: for "suboptimal" \((s, a)\), approximately \(\tilde{O}\left(\frac{1}{(\Delta^{M}_{(s,a)})^2}\ln\frac{1}{\delta}\right)\) samples are required.
Where \(\Delta^{M}_{(s,a)} = Q_{M}^{*}(s,a^{*}) - Q_{M}^{*}(s,a)\) is the value gap between the action and the optimal action. Summing up the exploration costs for all state-action pairs gives a total magnitude of \(\sum_{(s,a)} \frac{1}{(\Delta^{M}_{(s,a)})^2}\).

Now we introduce the case with upper bounds and analyze how to reduce the number of samples across different MDPs.

To quantitatively represent this acceleration, we divide the state-action pairs \((s, a)\) into two groups:
\begin{itemize}
    \item $\mathcal{S}_{1}:$ Upper bounds are sufficiently tight and are truncated to be lower than the optimal action from the very beginning.
    \begin{equation}
    \begin{aligned}
        \mathcal{S}_1 = \left\{ (s,a)|\exists i: U_{\bar{M}_i}(s,a)< Q_{M}^{0}(s,a)\right\}
    \end{aligned}
    \end{equation}
    \item $\mathcal{S}_{0}:$ The upper bounds are not "tight enough," i.e.,
    \begin{equation}
    \begin{aligned}
        \mathcal{S}_0 = \text{remaining actions}
    \end{aligned}
    \end{equation}
\end{itemize}


For \((s,a) \in \mathcal{S}_1\):

We treat each sampling as a multi-armed bandit. Let the true mean of the optimal arm be \(\mu^{*}\). For a certain arm \(j\), its true mean is known to satisfy \(\mu_j \leq U_j < \mu^{*}\).

Even if we truncate \(\hat{\mu}_n(j)\) at \(U_j\), the UCB algorithm's "optimistic estimate" for this arm at step \(n\) is still:
\begin{equation}
\begin{aligned}
    Q_{n}(j) = \min \left\{ \hat{\mu}_{n}(j), U_j \right\} + c\sqrt{\frac{\ln(n)}{N_j(n)}}
\end{aligned}
\end{equation}


\begin{equation}
\begin{aligned}
    U_j + c\sqrt{\frac{\ln(n)}{N_j(n)}} < \mu^{*}
\end{aligned}
\end{equation}

Let \(\Delta = \mu^* - U_j\). As long as:
\begin{equation}
\begin{aligned}
    \sqrt{\frac{\ln(n)}{N_j(n)}}\leq \frac{\Delta}{2c}
\end{aligned}
\end{equation}
From the above, it can be ensured that \(Q_n(j)\) cannot exceed \(\mu^{*} - \Delta/2\).
So
\begin{equation}
\begin{aligned}
    N_j(n) \geq \frac{4c^2\ln(n)}{\Delta^2}
\end{aligned}
\end{equation}
Where we obtain a sampling time complexity of \(\tilde{O}(\ln n)\).



For \((s,a) \in \mathcal{S}_0\), these \((s,a)\) cannot be pruned by "truncation." They still require multiple samples, as in classic UCT, to determine whether they are truly optimal. For any \((s,a) \in \mathcal{S}_0\), we still need approximately \( O\left(\frac{1}{(\Delta^{M}_{(s,a)})^2} \ln\frac{1}{\delta}\right) \) samples to distinguish that it is not as good as \((s,a^{*})\).
Thus, the sampling complexity of our algorithm is:  
\begin{eqnarray}
   X_{\rm aUCT} = \sum_{(s,a)\in \mathcal{S}_{1}}\tilde{O}\left(\ln n\right) + \sum_{(s,a)\in \mathcal{S}_{0}}\tilde{O}\left(\frac{1}{(\Delta^{M}_{(s,a)})^2} \ln\frac{1}{\delta}\right),
\end{eqnarray}
Using the fact that \( \tilde{O}(\ln n) \sim \tilde{O}(\ln\frac{1}{\delta}) \), we can rewrite this as
\begin{eqnarray}
\label{eq:aUCT}
   X_{\rm aUCT} = \sum_{(s,a)\in \mathcal{S}_{1}}\tilde{O}\left( \ln\frac{1}{\delta} \right) + \sum_{(s,a)\in \mathcal{S}_{0}}\tilde{O}\left(\frac{1}{(\Delta^{M}_{(s,a)})^2} \ln\frac{1}{\delta}\right).
\end{eqnarray}
In contrast, the sampling complexity of the standard UCT can be obtained using the same analysis, i.e.,
\begin{eqnarray}
\label{eq:UCT}
 X_{\rm UCT} =  \sum_{(s,a)\in \mathcal{S}_0 \cup \mathcal{S}_1}\tilde{O}\left(\frac{1}{(\Delta^{M}_{(s,a)})^2} \ln\frac{1}{\delta}\right).
\end{eqnarray}
Comparing the order bounds from Equation~(\ref{eq:UCT}) and Equation~(\ref{eq:aUCT}), we can find the acceleration factor $\Gamma$ as
\begin{equation}
\begin{aligned}
\Gamma =\frac
{\sum_{(s,a)\in \mathcal{S}_1\cup \mathcal{S}_0 } \frac{1}{(\Delta^{\mathcal{M}}_{(s,a)})^2}}
{\sum_{(s,a)\in\mathcal{S}_1} (1) + 
\sum_{(s,a)\in\mathcal{S}_{0}} \frac{1}{(\Delta^{\mathcal{M}}_{(s,a)})^2}},
\end{aligned}
\end{equation}
which is the desired result in the theorem.


\end{proof}





\subsection{Proof of Theorem~\ref{the:err_signal_policy}}

\begin{proof}{Proof of Theorem~\ref{the:err_signal_policy}}
First, we need to establish unbiasedness and boundedness.
For unbiasedness, we can derive:  
\begin{equation}
\begin{aligned}
    \mathbb{E}[X_i] =\mathbb{E}_{(s,a)\sim \pi} [\frac{\mathcal{U}(s,a)}{\pi(s,a)}\cdot \Delta X(s, a)]= \mathbb{E}_{(s,a)\sim \mathcal{U}} [\Delta X(s, a)] = d(M,M^{\prime})
\end{aligned}
\end{equation}
Therefore, $\mathbb{E}[\hat{d}_{\mathcal{U}}] = d(M, M^{\prime})$, meaning $\hat{d}_{\mathcal{U}}$ is an unbiased estimator.

\begin{equation}
\begin{aligned}
    w_i = \frac{\mathcal{U}(s_i,a_i)}{\pi(s_i,a_i)}\leq \frac{\mathcal{U}_{\max}}{\alpha}
\end{aligned}
\end{equation}
Where $\mathcal{U}{\max} = \max_{(s, a)} \mathcal{U}(s, a) = \frac{1}{|\mathcal{S}| \cdot |\mathcal{A}|}$.
So we can get:

\begin{equation}
\begin{aligned}
    X_i = w_i\Delta X(s_i,a_i)\leq (\frac{\mathcal{U}_{\max}}{\alpha}) b
\end{aligned}
\end{equation}

So we can get $X_i\in[0,C]$ where $C = \frac{\mathcal{U}_{\max}}{\alpha} b$.

Based on the above analysis, we have $\bar{X}_{N} = \frac{1}{N} \sum_{i=1}^{N} X_i = \hat{d}_{\mathcal{U}}$, $\mu = \mathbb{E}[X_i] = d(M,M^{\prime})$. According to Hoeffding's inequality, for $\bar{X}_{N} \in [0, C]$, we have:

\begin{equation}
\begin{aligned}
    \text{Pr}\{|\bar{X}_{N} - \mu|\geq \epsilon \} \leq 2\exp(-\frac{2 N \epsilon^2}{C^2})
\end{aligned}
\end{equation}
To achieve a confidence level of $\delta$, it requires:
\begin{equation}
\begin{aligned}
    2\exp(-\frac{2 N \epsilon^2}{C^2}) \leq \delta \Leftrightarrow \exp(-\frac{2 N \epsilon^2}{C^2}) \leq \frac{\delta}{2} \Leftrightarrow -\frac{2 N \epsilon^2}{C^2} \leq \ln \frac{\delta}{2}\Leftrightarrow \frac{2 N \epsilon^2}{C^2} \geq \ln \frac{2}{\delta} \Leftrightarrow N \geq \frac{C^2}{2\epsilon^2} \ln \frac{2}{\delta}
\end{aligned}
\end{equation}

We get if fulfilled:
\begin{equation}
\begin{aligned}
    N\geq \frac{1}{2\epsilon^2} (\frac{\mathcal{U}_{\max}}{\alpha} b)^2\ln \frac{2}{\delta}
\end{aligned}
\end{equation}
There is then a high probability error upper bound:
\begin{equation}
\begin{aligned}
    \text{Pr}\{|\hat{d}_{\mathcal{U}} - d(M,M^{\prime})|\leq \epsilon\}\geq 1-\delta
\end{aligned}
\end{equation}
\end{proof}



\subsection{Proof of Theorem~\ref{the:err_multi_policy}}


\begin{proof}{Proof of Theorem~\ref{the:err_multi_policy}}
Constructing a martingale difference, let:
\begin{equation}
\begin{aligned}
    S_n := \sum_{k=1}^{n}(X_k - d(M,M^{\prime})), Y_k := X_k-\mathbb{E}[X_k|\mathcal{F}_{k-1}]
\end{aligned}
\end{equation}
According to the martingale condition in formula~\ref{eqn:Martingale}, we know that \( Y_k = X_k - d(M, M^{\prime}) \), and \( S_n = \sum_{k=1}^n Y_k \) satisfies \( \mathbb{E}[Y_k | \mathcal{F}_{k-1}] = 0 \).
Thus, \(\{S_n, \mathcal{F}_n\}\) is a martingale process.

Since \(\pi_{k}(s, a) \geq \alpha \Rightarrow w_k \leq \frac{\mathcal{U}_{\max}}{\alpha}\), and \(\Delta X(s, a) \leq b \Rightarrow X_k = w_k \Delta X(s_k, a_k) \leq \frac{\mathcal{U}_{\max}}{\alpha} b =: C\). Therefore, we have:
\begin{equation}
\begin{aligned}
    |Y_k|\leq \max\{X_k,d(M,M^{\prime})\}\leq C
\end{aligned}
\end{equation}
According to the Azuma-Hoeffding inequality for bounded martingale differences, we have:
\begin{equation}
\begin{aligned}
    \text{Pr}\{|S_n|\geq t\}\leq 2\exp(-\frac{t^2}{2NC^2})
\end{aligned}
\end{equation}
Let \( t = N\epsilon \), then \( |S_n| \geq t \) is equivalent to \( \left|\sum_{k=1}^n X_k - N d(M, M^{\prime})\right| \geq N\epsilon \), that is:
\begin{equation}
\begin{aligned}
    |\hat{d}_{\mathcal{U}}^{(n)}-d(M,M^{\prime})|\geq \epsilon
\end{aligned}
\end{equation}
So:
\begin{equation}
\begin{aligned}
    \text{Pr}\{|\hat{d}_{\mathcal{U}}^{(N)}-d(M,M^{\prime})|\geq \epsilon\}\leq 2\exp(-\frac{N\epsilon^2}{2C^2})
\end{aligned}
\end{equation}
Thus, as long as \( N \geq \frac{2C^2}{\epsilon^2} \ln \frac{2}{\delta} \), we have \( \text{Pr}\{|\hat{d}_{\mathcal{U}}^{(N)} - d(M, M^{\prime})| \geq \epsilon\} \leq \delta \).
\end{proof}


\subsection{{Proof of Theorem~\ref{the:network}}}

\begin{proof}{Proof of Theorem~\ref{the:network}}
We decompose $d_{\mathcal{U}}$.
\begin{equation}
\begin{aligned}
    d_{\mathcal{U}}(M,M_i) & = \mathbb{E}_{(s,a)\sim \mathcal{U}}[\underbrace{|R_s^a-R_s^{a,(i)}|}_{\text{Reward difference}} + \underbrace{\kappa \sum_{s^{\prime}}|P_{ss^{\prime}}^a - P_{ss^{\prime}}^{a,(i)}|}_{\text{transition  difference}}]\\
    & \simeq \mathbb{E}_{(s,a)\sim \mathcal{U}}[|R_s^a-R_s^{a,(i)}| + \kappa ||\Psi_{\phi}(s,a) - \Psi_{\phi_i}(s,a)||_1]\\
    & \leq \mathbb{E}_{(s,a)\sim \mathcal{U}}[L_3\rho(\phi,\phi_i)  + \kappa L_3\rho(\phi,\phi_i)]\\
    &\leq L_3\rho(\phi,\phi_i)+ \kappa L_3\rho(\phi,\phi_i)\\
    & = (1+\kappa) L_3 \rho(\phi,\phi_i)\\
    & = (1+\kappa) L_3 \hat{d}_{para}(M,M_i)
\end{aligned}
\end{equation}
\end{proof}



%%%%%%%%%%%%%%%%%%%%%%%%%%%%%%%%%%%%%%%%%%%%%%%%%%%%%%%%%%%%%%%%%%%%%%%%%%%%%%%
%%%%%%%%%%%%%%%%%%%%%%%%%%%%%%%%%%%%%%%%%%%%%%%%%%%%%%%%%%%%%%%%%%%%%%%%%%%%%%%



\section{Pseudo-code}

\begin{algorithm}[ht]
\caption{UMCTS}
\label{alg:umcts}
\begin{algorithmic}[1]
\REQUIRE $\{\mathcal{M}_1,\dots,\mathcal{M}_M\}, \mathcal{U}, \kappa, L, L_2^{(i)}, \gamma, R_{\max}, C, T$
\FOR{$i = 1$ to $M$}
    \STATE Repeat sampling \( (s, a) \) from the uniform distribution \( \mathcal{U} \) to update \( R \) and \( P \).
   \FOR{$j = 1$ to $M$}
      \STATE $d(\mathcal{M}_i,\mathcal{M}_j)
      \gets \mathbb{E}_{(s,a,s') \sim \mathcal{U}}
      \bigl[\,
         |R_s^a - \overline{R}_s^a|
         + \kappa \,|P_{ss'}^a - \overline{P}_{ss'}^a|
      \bigr]$
   \ENDFOR

\vspace{5pt}
\STATE Initialize root node $s_0$, set $N(\cdot), N(\cdot,\cdot), W(\cdot,\cdot)$ to $0$
\FOR{$t = 1$ to $T$}
  \STATE \textbf{Selection}:
  \STATE \quad Set current node $s \leftarrow s_0$
  \WHILE{\text{child nodes of $s$ are fully expanded}}
    \STATE Choose $a = \underset{a}{\mathrm{argmax}}\;
    \bigl(Q(s,a)\bigr)$ \quad \text{// using Eq.\,(*) below}
    \STATE $s \leftarrow \text{child node after action $a$}$
  \ENDWHILE

  \STATE \textbf{Expansion}:
  \STATE \quad Expand one non-visited action $a_{\mathrm{new}}$ at $s$, 
    sample $s'$ from environment or model
  \STATE \quad Create new child node $s'$, set $N(s',\cdot)=0$, $W(s',\cdot)=0$

  \STATE \textbf{Simulation}:
  \STATE \quad Perform a (light) rollout or default policy from $s'$ to terminal or horizon
  \STATE \quad Receive cumulative reward $G$

  \STATE \textbf{Backpropagation}:
  \STATE \quad \text{Traverse back from $s'$ to $s_0$ along visited path}
  \FORALL{\text{visited state-action pairs } $(\tilde{s}, \tilde{a})$}
    \STATE $N(\tilde{s}) \,\leftarrow\, N(\tilde{s})+1$
    \STATE $N(\tilde{s},\tilde{a}) \,\leftarrow\, N(\tilde{s},\tilde{a})+1$
    \STATE $W(\tilde{s},\tilde{a}) \,\leftarrow\, W(\tilde{s},\tilde{a}) + G$
    \STATE \text{// Update $Q(\tilde{s},\tilde{a})$ with UMCTS bound:}
    \STATE $U_{\bar{\mathcal{M}}}(\tilde{s},\tilde{a}) \gets 
      Q_{\bar{\mathcal{M}}}^{*}(\tilde{s},\tilde{a}) 
      + L \cdot d(\mathcal{M},\bar{\mathcal{M}}) 
      + L_2^{(i)}$
    \STATE $U(\tilde{s},\tilde{a}) \gets 
      \min\bigl\{\frac{R_{\max}}{1-\gamma}\,,\,
                 U_{\bar{\mathcal{M}}}(\tilde{s},\tilde{a}),\,\dots\bigr\}$
    \STATE $Q(\tilde{s},\tilde{a}) \gets 
      \min\!\Bigl\{
        \dfrac{W(\tilde{s},\tilde{a})}{N(\tilde{s},\tilde{a})}
        + C\,\sqrt{\dfrac{\ln N(\tilde{s})}{N(\tilde{s},\tilde{a})}},\;
        U(\tilde{s},\tilde{a})
      \Bigr\}\quad (*)$
  \ENDFOR
\ENDFOR

\ENDFOR
\end{algorithmic}
\end{algorithm}






\begin{algorithm}[ht]
\caption{UMCTS with Importance Sampling}
\label{alg:umcts_is}
\begin{algorithmic}[1]
\REQUIRE Tasks $\{\mathcal{M}_1,\dots,\mathcal{M}_M\}$, each partially known; Uniform distribution $\mathcal{U}(s,a)$;Lipschitz constants $L, L_2^{(i)}$; Discount factor $\gamma$, maximum reward $R_{\max}$; Exploration constant $C$; Number of search iterations $T$;A (default) policy $\pi$ used in Simulation for importance sampling; 

\STATE \textbf{Function}~{Distance($\mathcal{M}, \bar{\mathcal{M}}, \pi$)}
\STATE ~~~~$\displaystyle \Delta X(s,a) \;\triangleq\; \Delta R_{s}^a \;+\;\kappa\,\Delta P_{s}^a$
\STATE \textbf{return} $\displaystyle
  \mathbb{E}_{(s,a)\sim \pi}
  \Bigl[
    \frac{\mathcal{U}(s,a)}{\pi(s,a)}
    \cdot
    \Delta X(s,a)
  \Bigr]$

\STATE \textbf{// For each task } $\mathcal{M}_i$
\FOR{$i = 1$ to $M$}
  \STATE Initialize root node $s_0$, set $N(\cdot)=0,\, N(\cdot,\cdot)=0,\,W(\cdot,\cdot)=0$
  \STATE \text{(Optionally maintain a buffer } $\mathcal{D}_i$ \text{ for storing samples }(s,a)\text{)}

  \FOR{$t = 1$ to $T$}
    %------------------------------------
    \STATE \textbf{Selection:}
    \STATE \quad $s \;\leftarrow\; s_0$
    \WHILE{all actions from $s$ are fully expanded \textbf{and} $s$ not terminal}
      \STATE $a \;\leftarrow\; \underset{a}{\mathrm{argmax}}\;\bigl(Q(s,a)\bigr)$ 
          \quad // UCB or UMCTS criterion
      \STATE $s \;\leftarrow\; \text{child node after action }a$
    \ENDWHILE

    %------------------------------------
    \STATE \textbf{Expansion:}
    \IF{\text{$s$ not terminal}}
      \STATE Choose one unvisited action $a_{\mathrm{new}}$ at $s$
      \STATE Sample next state $s' \sim P_i(\cdot \mid s,a_{\mathrm{new}})$  // from environment or model
      \STATE Create child node $s'$, set $N(s',\cdot)=0,\,W(s',\cdot)=0$
    \ENDIF

    %------------------------------------
    \STATE \textbf{Simulation:}
    \STATE \quad Initialize cumulative reward $G \leftarrow 0$
    \STATE \quad $s_{\mathrm{sim}} \leftarrow s'$
    \WHILE{$s_{\mathrm{sim}}$ is not terminal}
      \STATE Pick action $a_{\mathrm{sim}}$ by policy $\pi(\cdot \mid s_{\mathrm{sim}})$
      \STATE Observe reward $r_{\mathrm{sim}} = R_i(s_{\mathrm{sim}}, a_{\mathrm{sim}})$
      \STATE Observe next state $s_{\mathrm{next}} \sim P_i(\cdot \mid s_{\mathrm{sim}}, a_{\mathrm{sim}})$
      \STATE $G \leftarrow G + r_{\mathrm{sim}}$
      \STATE \text{// Update or record increments for } $ R_s^a,\,P_{s,s'}^a$
      \STATE \quad \(\Delta R_{s_{\mathrm{sim}}}^a \), \(\Delta P_{s_{\mathrm{sim}}}^a\) \(\leftarrow\) 
             (computed from new sample)
      \STATE \text{// Optionally store $(s_{\mathrm{sim}}, a_{\mathrm{sim}})$ in $\mathcal{D}_i$ for importance sampling}
      \STATE $s_{\mathrm{sim}} \leftarrow s_{\mathrm{next}}$
    \ENDWHILE

    %------------------------------------
    \STATE \textbf{Backpropagation:}
    \STATE \quad \text{Traverse from $s'$ back to $s_0$ along visited path}
    \FORALL{\text{visited pairs } $(\tilde{s}, \tilde{a})$}
      \STATE $N(\tilde{s}) \;\leftarrow\; N(\tilde{s}) + 1$
      \STATE $N(\tilde{s}, \tilde{a}) \;\leftarrow\; N(\tilde{s}, \tilde{a}) + 1$
      \STATE $W(\tilde{s}, \tilde{a}) \;\leftarrow\; W(\tilde{s}, \tilde{a}) + G$
      \STATE \text{/* Use the Lipschitz bound with distance estimation */}
      \STATE $d(\mathcal{M}_i,\bar{\mathcal{M}}) 
        \;\gets\; \mathrm{Distance}\bigl(\mathcal{M}_i,\bar{\mathcal{M}},\pi\bigr)$
      \STATE $U_{\bar{\mathcal{M}}}(\tilde{s}, \tilde{a})
         \;\gets\;Q_{\bar{\mathcal{M}}}^{*}(\tilde{s},\tilde{a})
         \;+\;L \cdot d(\mathcal{M}_i,\bar{\mathcal{M}})
         \;+\;L_2^{(i)}$
      \STATE $U(\tilde{s},\tilde{a})
         \;\gets\;\min\Bigl\{
           \dfrac{R_{\max}}{1-\gamma},\,
           U_{\bar{\mathcal{M}}}(\tilde{s},\tilde{a}),\dots
         \Bigr\}$
      \STATE \text{/* UMCTS update rule */}
      \STATE $Q(\tilde{s},\tilde{a})
         \;\gets\;\min\Bigl\{
           \dfrac{W(\tilde{s},\tilde{a})}{N(\tilde{s},\tilde{a})}
           + C\,\sqrt{\dfrac{\ln N(\tilde{s})}{N(\tilde{s},\tilde{a})}},
           \;U(\tilde{s},\tilde{a})
         \Bigr\} \quad (*)$
    \ENDFOR
  \ENDFOR
\ENDFOR

\end{algorithmic}
\end{algorithm}











\begin{algorithm}[ht]
\caption{UMCTS with Neural Network Environment Model}
\label{alg:umcts_nn}
\begin{algorithmic}[1]
\REQUIRE  MDPs $\{\mathcal{M}_1,\dots,\mathcal{M}_M\}$, each with trained neural network parameters $\{\phi_1,\dots,\phi_M\}$; A new MDP $M$ (partially known), with neural network $\Psi_{\phi}: \mathcal{S}\times \mathcal{A}\to \Delta(\mathcal{S})$; A distance function $\rho(\phi,\phi_i)\ge 0$ on parameter space (e.g., $\ell_2$-norm); Define $\hat{d}_{para}(M,M_i) = \rho(\phi,\phi_i)$; Lipschitz constants $L, L_2^{(i)}$, discount factor $\gamma$, $R_{\max}$, exploration constant $C$, iterations $T$; A default (simulation) policy $\pi$ for rollouts

\vspace{5pt}
\STATE \textbf{// For each task $M$ (with parameter $\phi$) run UMCTS}
\STATE Initialize root node $s_0$, counters $N(\cdot)=0,\,N(\cdot,\cdot)=0,\,W(\cdot,\cdot)=0$
\FOR{$t = 1$ to $T$}
  %------------------------------------------------
  \STATE \textbf{Selection}:
  \STATE \quad $s \leftarrow s_0$
  \WHILE{\text{all actions from } s \text{ are expanded \textbf{and} } s \text{ not terminal}}
    \STATE $a \;\leftarrow\;\underset{a}{\mathrm{argmax}}\;\bigl(Q(s,a)\bigr)$
    \STATE $s \;\leftarrow\;\text{child node after action }a$
  \ENDWHILE

  %------------------------------------------------
  \STATE \textbf{Expansion}:
  \IF{$s$ not terminal}
    \STATE \text{choose an unvisited action } $a_{\mathrm{new}}$
    \STATE \text{sample } $s' \sim \Psi_{\phi}(\cdot \mid s,a_{\mathrm{new}})$ \quad \text{// neural net predicts next state distribution}
    \STATE \text{create child node } s'
    \STATE $N(s',\cdot)\leftarrow 0,\;W(s',\cdot)\leftarrow 0$
  \ENDIF

  %------------------------------------------------
  \STATE \textbf{Simulation}:
  \STATE \quad $G \leftarrow 0$
  \STATE \quad $s_{\mathrm{sim}} \leftarrow s'$
  \WHILE{$s_{\mathrm{sim}}$ \text{ not terminal}}
    \STATE $a_{\mathrm{sim}} \leftarrow \text{sample from } \pi(\cdot \mid s_{\mathrm{sim}})$
    \STATE \text{// observe reward (possibly from real env or approximated by a learned reward model)}
    \STATE $r_{\mathrm{sim}} = R(s_{\mathrm{sim}}, a_{\mathrm{sim}})$
    \STATE $s_{\mathrm{next}} \sim \Psi_{\phi}(\cdot \mid s_{\mathrm{sim}}, a_{\mathrm{sim}})$

    \STATE $G \;\leftarrow\; G + r_{\mathrm{sim}}$

    \STATE \text{/* update $\phi$ via gradient (e.g. supervised/unsupervised RL objective) */}
    \STATE \quad $\phi \;\leftarrow\; \phi - \eta\,\nabla_{\phi} \mathcal{L}\bigl(\phi;(s_{\mathrm{sim}},a_{\mathrm{sim}},s_{\mathrm{next}})\bigr)$

    \STATE $s_{\mathrm{sim}} \;\leftarrow\; s_{\mathrm{next}}$
  \ENDWHILE

  %------------------------------------------------
  \STATE \textbf{Backpropagation}:
  \STATE \quad \text{traverse from $s'$ back to $s_0$}
  \FORALL{\text{visited state-action pairs } $(\tilde{s}, \tilde{a})$}
    \STATE $N(\tilde{s}) \;\leftarrow\; N(\tilde{s}) + 1$
    \STATE $N(\tilde{s},\tilde{a}) \;\leftarrow\; N(\tilde{s},\tilde{a}) + 1$
    \STATE $W(\tilde{s},\tilde{a}) \;\leftarrow\; W(\tilde{s},\tilde{a}) + G$

    \STATE \text{// parametric distance to previously trained model $\phi_i$}
    \STATE $\hat{d}_{para}(M,M_i) 
      \;\triangleq\;\rho(\phi,\phi_i)$

    \STATE \text{// Lipschitz-based upper bound}
    \STATE $U_{\bar{\mathcal{M}}}(\tilde{s},\tilde{a})
      \;\leftarrow\; 
      Q_{\bar{\mathcal{M}}}^{*}(\tilde{s},\tilde{a})
      \;+\;L\cdot \hat{d}_{para}(M,\bar{\mathcal{M}})
      \;+\;L_2^{(i)}$

    \STATE $U(\tilde{s},\tilde{a})
      \;\leftarrow\;\min\Bigl\{
        \frac{R_{\max}}{1-\gamma},\,
        U_{\bar{\mathcal{M}}}(\tilde{s},\tilde{a}),\dots
      \Bigr\}$

    \STATE \text{// UMCTS update rule}
    \STATE $Q(\tilde{s},\tilde{a})
      \;\leftarrow\;
      \min\Bigl\{
        \dfrac{W(\tilde{s},\tilde{a})}{N(\tilde{s},\tilde{a})}
        + C\sqrt{\dfrac{\ln N(\tilde{s})}{N(\tilde{s},\tilde{a})}},
        \;U(\tilde{s},\tilde{a})
      \Bigr\}
      \quad (*)$
  \ENDFOR

\ENDFOR
\end{algorithmic}
\end{algorithm}


\end{document}