%File: anonymous-submission-latex-2025.tex
\documentclass[letterpaper]{article} % DO NOT CHANGE THIS
\usepackage{aaai25}  % DO NOT CHANGE THIS
\usepackage{times}  % DO NOT CHANGE THIS
\usepackage{helvet}  % DO NOT CHANGE THIS
\usepackage{courier}  % DO NOT CHANGE THIS
\usepackage[hyphens]{url}  % DO NOT CHANGE THIS
\usepackage{graphicx} % DO NOT CHANGE THIS
\urlstyle{rm} % DO NOT CHANGE THIS
\def\UrlFont{\rm}  % DO NOT CHANGE THIS
\usepackage{natbib}  % DO NOT CHANGE THIS AND DO NOT ADD ANY OPTIONS TO IT
\usepackage{caption} % DO NOT CHANGE THIS AND DO NOT ADD ANY OPTIONS TO IT
\frenchspacing  % DO NOT CHANGE THIS
\setlength{\pdfpagewidth}{8.5in} % DO NOT CHANGE THIS
\setlength{\pdfpageheight}{11in} % DO NOT CHANGE THIS
%
% These are recommended to typeset algorithms but not required. See the subsubsection on algorithms. Remove them if you don't have algorithms in your paper.
 % For algorithmic environment

\usepackage{algorithm}
\usepackage{algorithmic}

%
% These are are recommended to typeset listings but not required. See the subsubsection on listing. Remove this block if you don't have listings in your paper.
\usepackage{newfloat}
\usepackage{listings}
\DeclareCaptionStyle{ruled}{labelfont=normalfont,labelsep=colon,strut=off} % DO NOT CHANGE THIS
\lstset{%
	basicstyle={\footnotesize\ttfamily},% footnotesize acceptable for monospace
	numbers=left,numberstyle=\footnotesize,xleftmargin=2em,% show line numbers, remove this entire line if you don't want the numbers.
	aboveskip=0pt,belowskip=0pt,%
	showstringspaces=false,tabsize=2,breaklines=true}
\floatstyle{ruled}
\newfloat{listing}{tb}{lst}{}
\floatname{listing}{Listing}
%
% Keep the \pdfinfo as shown here. There's no need
% for you to add the /Title and /Author tags.
\pdfinfo{
/TemplateVersion (2025.1)
}
\usepackage{multirow}
\usepackage{array}
\usepackage{graphicx} % Required for including images
\usepackage{subcaption} % Required for subfigures
% Adjusts the margins to allow more space
\usepackage{amsmath}  % For mathematical symbols and equations
\usepackage{amssymb}  % For additional symbols
%\usepackage{algpseudocode} % For algorithmic environment

\usepackage{pgfplots}
\pgfplotsset{compat=1.17}

% DISALLOWED PACKAGES
% \usepackage{authblk} -- This package is specifically forbidden
% \usepackage{balance} -- This package is specifically forbidden
% \usepackage{color (if used in text)
% \usepackage{CJK} -- This package is specifically forbidden
% \usepackage{float} -- This package is specifically forbidden
% \usepackage{flushend} -- This package is specifically forbidden
% \usepackage{fontenc} -- This package is specifically forbidden
% \usepackage{fullpage} -- This package is specifically forbidden
% \usepackage{geometry} -- This package is specifically forbidden
% \usepackage{grffile} -- This package is specifically forbidden
% \usepackage{hyperref} -- This package is specifically forbidden
% \usepackage{navigator} -- This package is specifically forbidden
% (or any other package that embeds links such as navigator or hyperref)
% \indentfirst} -- This package is specifically forbidden
% \layout} -- This package is specifically forbidden
% \multicol} -- This package is specifically forbidden
% \nameref} -- This package is specifically forbidden
% \usepackage{savetrees} -- This package is specifically forbidden
% \usepackage{setspace} -- This package is specifically forbidden
% \usepackage{stfloats} -- This package is specifically forbidden
% \usepackage{tabu} -- This package is specifically forbidden
% \usepackage{titlesec} -- This package is specifically forbidden
% \usepackage{tocbibind} -- This package is specifically forbidden
% \usepackage{ulem} -- This package is specifically forbidden
% \usepackage{wrapfig} -- This package is specifically forbidden
% DISALLOWED COMMANDS
% \nocopyright -- Your paper will not be published if you use this command
% \addtolength -- This command may not be used
% \balance -- This command may not be used
% \baselinestretch -- Your paper will not be published if you use this command
% \clearpage -- No page breaks of any kind may be used for the final version of your paper
% \columnsep -- This command may not be used
% \newpage -- No page breaks of any kind may be used for the final version of your paper
% \pagebreak -- No page breaks of any kind may be used for the final version of your paperr
% \pagestyle -- This command may not be used
% \tiny -- This is not an acceptable font size.
% \vspace{- -- No negative value may be used in proximity of a caption, figure, table, section, subsection, subsubsection, or reference
% \vskip{- -- No negative value may be used to alter spacing above or below a caption, figure, table, section, subsection, subsubsection, or reference
\captionsetup[figure]{skip=1pt} 

\setcounter{secnumdepth}{0} %May be changed to 1 or 2 if section numbers are desired.

% The file aaai25.sty is the style file for AAAI Press
% proceedings, working notes, and technical reports.
%

% Title

% Your title must be in mixed case, not sentence case.
% That means all verbs (including short verbs like be, is, using,and go),
% nouns, adverbs, adjectives should be capitalized, including both words in hyphenated terms, while
% articles, conjunctions, and prepositions are lower case unless they
% directly follow a colon or long dash
\title{DMPA: Model Poisoning Attacks on Decentralized Federated Learning for Model Differences}

\author {
    Chao Feng\textsuperscript{\rm 1}, 
    Yunlong Li\textsuperscript{\rm 1}, 
    Yuanzhe Gao\textsuperscript{\rm 1}, 
    Alberto Huertas Celdrán\textsuperscript{\rm 1}, 
    Jan von der Assen\textsuperscript{\rm 1},
    Gérôme Bovet\textsuperscript{\rm 2}, 
    Burkhard Stiller\textsuperscript{\rm 1}
}
\affiliations{
    \textsuperscript{\rm 1} Communication Systems Group, Department of Informatics, University of Zürich,\\
    Binzmühlestrasse 14, CH-8050 Zürich, Switzerland\\
    \textsuperscript{\rm 2} Cyber-Defence Campus, armasuisse Science \& Technology, CH-3602 Thun, Switzerland\\
    \{cfeng, huertas, vonderassen, stiller\}@ifi.uzh.ch, yuanzhe.gao@uzh.ch, yunlong.li@uzh.ch, gerome.bovet@armasuisse.ch
}

\iffalse
%Example, Multiple Authors, ->> remove \iffalse,\fi and place them surrounding AAAI title to use it
\title{My Publication Title --- Multiple Authors}
\author {
    % Authors
    First Author Name\textsuperscript{\rm 1},
    Second Author Name\textsuperscript{\rm 2},
    Third Author Name\textsuperscript{\rm 1}
}
\affiliations {
    % Affiliations
    \textsuperscript{\rm 1}Affiliation 1\\
    \textsuperscript{\rm 2}Affiliation 2\\
    firstAuthor@affiliation1.com, secondAuthor@affilation2.com, thirdAuthor@affiliation1.com
}
\fi


% REMOVE THIS: bibentry
% This is only needed to show inline citations in the guidelines document. You should not need it and can safely delete it.
\usepackage{bibentry}
% END REMOVE bibentry

\begin{document}

\maketitle

\begin{abstract}
% As a successful paradigm in the field of privacy protection, FL has received much attention in the field of model poisoning attacks. DFL abandons the structure of a single server in the traditional FL, improving the overall robustness and extensibility of the network. However, these features of DFL also open new doors for malicious participants to launch attacks. Currently, model poisoning attacks against FL are mostly focused on the traditional FL, while less research has been conducted on DFL. Even in existing DFL model poisoning attacks, the main strategy involves malicious participants only considering the effect of their own model parameters when making modifications. However, in DFL, the model parameters of each node are often different, which means that this attack method cannot effectively influence the performance of other benign models. In this work, a model poisoning attack method (DMPA) is proposed for DFL. This method calculates the differential characteristics of multiple malicious client models and obtains the most effective poisoning strategy that makes the effect of these models deteriorate together, thus realizing the collusive attack of multiple participants to offset the effect of inconsistent participant models in DFL. The effectiveness of this attack is also demonstrated in multiple datasets. The results show that the DMPA method generally outperforms the existing state-of-the-art FL model poisoning attack methods across various dimensions (with a reduction of xx-xx). 
Federated learning (FL) has garnered significant attention as a prominent privacy-preserving Machine Learning (ML) paradigm. Decentralized FL (DFL) eschews traditional FL's centralized server architecture, enhancing the system's robustness and scalability. However, these advantages of DFL also create new vulnerabilities for malicious participants to execute adversarial attacks, especially model poisoning attacks. In model poisoning attacks, malicious participants aim to diminish the performance of benign models by creating and disseminating the compromised model. Existing research on model poisoning attacks has predominantly concentrated on undermining global models within the Centralized FL (CFL) paradigm, while there needs to be more research in DFL. To fill the research gap, this paper proposes an innovative model poisoning attack called DMPA. This attack calculates the differential characteristics of multiple malicious client models and obtains the most effective poisoning strategy, thereby orchestrating a collusive attack by multiple participants. The effectiveness of this attack is validated across multiple datasets, with results indicating that the DMPA approach consistently surpasses existing state-of-the-art FL model poisoning attack strategies.

\end{abstract}

% Uncomment the following to link to your code, datasets, an extended version or similar.
%
% \begin{links}
%     \link{Code}{https://aaai.org/example/code}
%     \link{Datasets}{https://aaai.org/example/datasets}
%     \link{Extended version}{https://aaai.org/example/extended-version}
% \end{links}
\section{Introduction}
% \hspace{1em}In recent years, with the introduction of various privacy laws in different countries, the demand for privacy preservation is increasing. As an emerging multi-participant collaborative machine learning method, federated learning has attracted the attention of more and more researchers because its participants (or clients) do not need to share original data, but share model information, which helps to protect the privacy of local data\cite{yang2019federated}. The traditional Federated Learning architecture relies on a central server to send, receive, and aggregate model parameters from each participant, and is therefore called Centralized Federated Learning (CFL)\cite{alazab2021federated}. This client-server network architecture, while easy to implement, has significant drawbacks, including vulnerability to a single point of failure, and is not conducive to further optimizing the performance of federated learning. To overcome these limitations, Decentralized Federated Learning (DFL) came into being and has gradually gained wide attention and acceptance.

The advent of diverse privacy regulations has significantly increased the need for privacy preservation in Machine Learning (ML). Federated Learning (FL) emerges as a novel collaborative and privacy-preserving ML paradigm that has attracted substantial attention from both academic and industrial communities~\cite{mcmahan2017communication}. FL enables participants to share model parameters instead of raw data, thereby ensuring data privacy \cite{yang2019federated}. Traditional FL architectures rely on a central server to distribute, called Centralized FL (CFL), receive and aggregate model parameters from participants \cite{alazab2021federated}. However, this client-server architecture suffers from significant drawbacks, including susceptibility to a single point of failure and server-side bottleneck~\cite{Mart_nez_Beltr_n_2023}. To address these challenges, Decentralized FL (DFL) has been introduced.



% In DFL, the entire network is completely autonomous, and each client is independently responsible for establishing communication, passing, and aggregating model parameters with other clients \cite{lian2017can}. Each participant completes the following tasks in each round: training the local model, passing the model parameters to the directly connected clients, aggregating the received model parameters, and updating the local model accordingly. Different from CFL, each client in DFL can independently select any other participant to establish a two-way communication connection, so the topology of the overall network can be customized, such as fully connected, ring, star, etc., and even dynamic topology \cite{yuan2024decentralized}. This flexibility allows DFL to effectively mitigate the single point of failure problem in CFL and improve the robustness and scalability of the system. However, this decentralized nature of DFL also brings new security risks, making the network more vulnerable to various threats, such as poisoning attacks.

In DFL, each client independently manages the processes of establishing communication, sharing, and aggregating model parameters with other clients \cite{lian2017can}. During each iteration, participants execute several tasks: training their local models, transmitting model parameters to directly connected peers, aggregating received parameters, and updating their local models accordingly. Unlike CFL, clients in DFL have the autonomy to independently select any other participant to establish bidirectional communication connections, allowing for customizable overlay network topology, such as fully connected, ring, star, and even dynamic configurations \cite{yuan2024decentralized}. This adaptability enables DFL to effectively address the single point of failure issue inherent in CFL, thereby enhancing the system's robustness and scalability. Nevertheless, this decentralized nature introduces new security vulnerabilities, making DFL more susceptible to various threats, including model poisoning attacks.

Model poisoning attacks involve malicious clients directly altering their local model parameters to negatively influence the global model training by sending harmful updates \cite{feng2023sentinelaggregationfunctionsecure}. In FL, since participants can directly share and modify model parameters, model poisoning attacks are relatively more straightforward to execute and pose significant threats. Current model poisoning attacks primarily target CFL models. These adversarial clients transmit meticulously crafted local model updates to the central server during the training phase, leading to issues such as reduced accuracy of the global model \cite{cao2022mpaf}. These attacks presuppose that the attacker controls a substantial number of compromised real clients, which may include either hijacked clients or fabricated ones.

% Currently, the research on security issues for Federated Learning (FL) systems is very extensive and covers a variety of attack methods, among which Poisoning Attack is an important research direction. Poisoning attacks are mainly categorized into Data Poisoning and Model Poisoning. Data Poisoning Attacks usually refer to malicious clients disrupting the training process by injecting malicious data (e.g., backdoor data or labeling noise) into the local training set, which leads to biased learning of specific data by the model and ultimately misleads the overall model learning results of FL\cite{hallaji2022federated}. In contrast, model poisoning attacks involve malicious clients directly tampering with their local model parameters and influencing the global model training through malicious model updates. Because in federated learning, participants can directly share and modify model parameters, model poisoning attacks tend to be easier to implement and more threatening.


% Existing model poisoning attacks (MPA) mainly target centralized Federated Learning (CFL) models. The attack assumes that the attacker has access to a large number of compromised real clients, which could be legitimate clients that have been hijacked or fake clients. During the training process, these clients send carefully designed local model updates to the cloud server (i.e., the central server), resulting in problems such as a decline in the accuracy of the global model \cite{cao2022mpaf}. Most existing model poisoning attack method rely on tuning hyperparameters to amplify the effectiveness of malicious models. For example, the MIN-MAX method \cite{shejwalkar2021manipulating} generates a new malicious model by calculating the distance difference between the models. However, because these modifications are relatively drastic, they will significantly change the distribution of model weights, making these attacks easy to be detected and excluded by model aggregation methods \cite{pillutla2022robust}. However, if the gradient empirical variance between clients is high enough, an attacker can exploit this to launch a non-omniscient attack that operates within the range of the population variance. This attack uses mean and variance as the basis for model modification to carry out more covert model poisoning attacks \cite{baruch2019little} and gradually reduce the accuracy of federated learning. In addition, attackers can also hijack the network transmission process and modify it without obtaining a large amount of local data, which becomes a fast and effective attack mode, and greatly threatens the security of federated learning systems.

% In decentralized federated learning (DFL), due to poor network connectivity and long communication links, it is difficult for malicious participant to have a significant impact on the global network even if they carry out high-intensity attacks\cite{Mart_nez_Beltr_n_2023}. This makes DFL system more difficult to attack when facing model poisoning attacks. However, due to its nature, model poisoning attacks can still modify the model by hijacking the content of the communication. Such modifications based on model parameters or gradients are more common in reality than attack methods that require direct access to local data, and can lead to security issues in applications such as healthcare, intrusion detection, and autonomous vehicles\cite{issa2023blockchain}.

Nevertheless, the architectural distinctions between DFL and CFL, particularly in overlay network topology, imply that existing attacks targeting CFL can not be directly transferable to DFL. In DFL, the communication links are significantly longer compared to CFL, making it challenging for a malicious participant to impact the entire federation \cite{Mart_nez_Beltr_n_2023}. Additionally, in DFL, each node could decide whether to disseminate and receive models from neighboring nodes. This capability enables the potential blockage of malicious attacks at intermediate nodes, thereby diminishing the effectiveness of such attacks. Consequently, from an adversarial perspective, it is imperative to develop model poisoning attacks that are effective in DFL.

% As DFL gradually becomes a universal solution for privacy protection, it is of great practical significance to find an MPA that can be effectively implemented in a DFL environment. At present, the research on model poisoning attack of DFL is relatively scarce, which makes the security of DFL in a wide range of applications unknown risks. Existing DFL model poisoning attacks usually use Gaussian distribution of additive noise or other methods to modify model parameters at will. However, these modifications are often detected and excluded when the model is aggregated, resulting in a less effective attack. Current FL poisoning attack methods are mainly implemented by adjusting the Euclidian distance or Angle of the model, but these methods do not perform well in the network structure of DFL. Since malicious clients in DFL may not be directly connected to other clients, when calculating model distance or similarity, these methods may cause the features of the malicious model to be cleared, leaving only Euclidean distance or similarity information. The detection mechanism of some aggregation methods (such as Krum\cite{blanchard2017machine}) makes it possible that when benign clients outnumber malicious clients, the model parameters of the malicious clients may be excluded from the beginning and it is difficult to propagate to the entire network. Even so, the benign client continues to perform model updates for gradient descent. As a result, it is difficult for MPA on DFL to have a significant impact on benign clients which is not directly connected to malicious clients, which further increases the difficulty of attacking on DFL. Therefore, the current MPA research on DFL environment faces the following challenges: 1. How to keep the characteristics of malicious clients similar to those of other benign clients so that the numerical characteristics of model parameters in each round are stable without losing key information; 2. How to make the malicious client's model modification not excluded by the aggregation method, so as to continue to be accepted in the next round.

To this end, this paper investigates model poisoning attacks within DFL  and proposes the Decentralized Model Poisoning Attack (DMPA) based on an Angle Bias Vector to rectify reversed model parameters. Specifically, by determining the eigenvalues of the compromised benign parameters, the method extracts the corresponding eigenvectors to compute the angular deviation vector. This vector is derived by exploiting the discrepancies between models and ultimately adjusting the negative parameters, thereby generating an effective attack model across the majority of participants. The main contributions of this paper are: \textit{(i)} design and implementation of a model poisoning attack, called DMPA, to compromise the model robustness of DFL models; \textit{(ii)} an extensive series of experiments are conducted to evaluate the proposed DMPA and compared with selected state-of-the-art attack algorithms under diverse robustness aggregation functions, which encompass three benchmark datasets (MNIST, Fashion-MNIST, and CIFAR-10) and three varying overlay topologies (fully connected, star, and ring); and \textit{(iii)} critical insights into the robustness of the DFL model concerning both offensive and defensive points of view. The experiments demonstrate that the DMPA presented in this work exhibits a more potent attack capability and broader propagation potential, effectively compromising the DFL system.

The remainder of this paper is structured as follows. Section 2 provides an overview of the underlying security issues in DFL and discusses the concept of model poisoning attacks. Subsequently, Section 3 formalizes the problem. Section 4 delineates the design specifics of DMPA, the proposed model poisoning attack. Section 5 then evaluates DMPA's performance in comparison to selected related works. Finally, Section 6 offers a summary of the contributions made by this research and outlines potential avenues for future investigation.

% To this end, this paper investigates Model Poisoning attacks targeting decentralized Federated Learning (DFL), and proposes a Model Poisoning Attack method (DMPA) based on Angle bias vector to correct reverse model parameters. Specifically, by calculating the eigenvalues of the hijacked benign parameters, the method extracts the corresponding eigenvectors to calculate the angular deviation vector. This angular deviation vector is computed by exploiting the differences between the models and ultimately correcting the effects of the negative parameters to generate an attack model that is valid for most of the participants. The algorithm proposed in this paper shows effective attack capability in a wide range of topologies and significantly reduces the F1 score after aggregation, thus validating the effectiveness of the method.\\
% In summary, the main contributions of this paper are as follows:
% \begin{itemize}
%     \item A framework for implementing model poisoning attacks in DFL environment is proposed, which launches attacks after local training is completed, thus destroying the training process and final results of DFL.
%     \item A method was proposed to modify the negative model parameters by calculating the eigenvector projection corresponding to the maximum eigenvalue of the correlation matrix. The attack was carried out while retaining the model features, which solved the problem of feature disappearing after multiple rounds of training and enhanced the persistence of the attack.
%     \item On three publicly available benchmark datasets and three common DFL topologies, this paper evaluates the performance and effectiveness of three state-of-the-art model poisoning attack methods in DFL. The experimental results show that the proposed attack strategy solves the above problems, and a large number of experimental results prove that the DMPA method in this paper has stronger attack effect and wider propagation ability, and can effectively destroy the DFL system, which verifies the rationality and effectiveness of the proposed method.
% \end{itemize}



\section{Background and Related Work}
This section provides an introduction to the security problem in DFL and a summary of the current research on model poisoning attacks in FL.
\subsection{Security Problem in DFL}
Differing from traditional CFL, neighboring clients in DFL exchange local model parameters or gradients over a P2P network and construct consensus models independently. DFL solves the problem of risk associated with a single point of failure, enhances its scalability, and is well suited for applications in the Industrial Internet of Things (IIoT) \cite{tan2023collusive}. However, the decentralized nature of DFLs rather increases their risk of being exposed to malicious attacks, especially poisoning attacks~\cite{feng2023voyager}. Poisoning attacks, which aim to diminish the resilience of FL models, can be classified into data poisoning and model poisoning attacks. Data poisoning attacks typically involve malicious clients interfering with the training process by introducing harmful data (such as backdoor attacks or label flipping) into the local training dataset, thereby inducing biased learning outcomes \cite{hallaji2022federated}. Conversely, model poisoning attacks are characterized by malicious clients directly altering their local model parameters and affecting the global model training via harmful model updates. To improve the model robustness of FL and defend against the poisoning attacks, \cite{mcmahan2017communication} proposed the use of averaging model parameters to improve training efficiency. \cite{blanchard2017machine} introduced the Krum method, which excludes malicious updates by calculating the Euclidean distance of a multi-node model and selecting the model with the smallest distance. This method reduces the impact of malicious attacks and is now widely used for robust aggregation in DFL.
\cite{yin2018byzantine} developed two robust distributed learning algorithms for Byzantine errors and potential adversarial behavior, a robust distributed gradient descent algorithm based on Median and Trimmed Mean operations, respectively. These two methods are widely used in both CFL and DFL because they can be implemented with only one communication round and achieve optimal model robustness.

\subsection{Model Poisoning Attack}
Compared with data poisoning attacks, model poisoning attacks are easier to execute as they directly modify the shared model. Therefore, there has been research suggesting how to optimize the attack method to enhance the attack effectiveness. Existing model poisoning attacks mainly target CFL. The attack assumes that the attacker has access to a large number of compromised clients. During the training process, malicious clients send carefully designed local model updates to the server, resulting in problems such as a decline in the accuracy of the global model \cite{cao2022mpaf}. The MIN-MAX and MIN-SUM methods proposed by \cite{shejwalkar2021manipulating} respectively minimize the maximum and sum of the distance between toxic updates and all benign updates in CFL. These methods assume that malicious model updates are the sum of aggregated model updates and a fixed disturbance vector scaled by factors before the attack, and implement the attack by maximizing the distance between toxic updates and benign updates in the inverse direction of the global model. 
%These methods show remarkable results in traditional federated learning, and because they do not rely on training deep learning models, they require fewer computational resources, making them useful in FL's model poisoning attacks.
The "a little is enough" (LIE) attack \cite{baruch2019little} directly modifies the model parameters, which generates malicious model updates in each training round by calculating the average of the real model updates of the malicious client and perturbing them. %In a DFL system, it is assumed that an attacker has a set of malicious client models, and although these malicious clients may not be directly connected, they can still use these participants to implement model poisoning attacks. Therefore, this paper will try to apply the above methods to DFL system and study its attack logic to help explore the model poisoning attack in DFL.

Most model poisoning attacks rely on additional information from the central server, such as aggregation methods, global models, or even the training data utilized by nodes. This dependency renders external attacks impractical. Nevertheless, some attacks are engineered to maintain efficacy even in the absence of such additional knowledge. For instance, \cite{zhang2023denial} employs global historical data to build an estimator that forecasts the subsequent round of global models as a benign reference. Despite this, the approach necessitates substantial resources and encounters difficulties in accessing the global model within DFL systems, thereby limiting its practical application in DFL.

To conclude, while a substantial amount of research has been dedicated to optimizing model poisoning attacks against the FL, these efforts predominantly concentrate on the CFL paradigm, with minimal attention given to the DFL paradigm. Furthermore, the current attack methodologies exhibit limitations, such as inconsistent efficacy and susceptibility to detection by defensive mechanisms. To address these research gaps, this paper presents the design and implementation of a novel attack strategy tailored for DFL. This proposed attack demonstrates broad effectiveness across various datasets, diverse ML model architectures, and multiple types of DFL overlay topologies.


\section{Problem Setup}
\subsection{DFL Training Process}
\hspace{1em}In the DFL framework, consider a network of \( K \) clients, each denoted as \( k \in \{1, 2, \dots, K\} \). In each communication round, each client \( k \) maintains and updates a local model \( w_t^k \) according to an aggregation algorithm \( A(\cdot) \) defined by this framework. The communication between clients can be represented as a graph \( G = (\mathcal{V}, \mathcal{E}) \), where \( \mathcal{V} \) is the set of nodes (clients) and \( \mathcal{E} \) is the set of edges (communication links) between the nodes. The neighborhood of a client \( k \), denoted as \( \mathcal{N}_k \subseteq \mathcal{V} \), contains all clients that are directly connected to \( k \) via an edge in \( \mathcal{E} \). Each client updates its model by training it locally and aggregating models from its neighbors. The network topology, represented by \( G \), plays a key role in determining how information is shared and how global knowledge is propagated through the network.

A general process of DFL can be divided into the following stages: First, each client \( k \) initializes its local model parameters \( w_0^k \). Then, each client trains its local model to further optimize these parameters. Gradient descent is generally used as the core optimization strategy during local training. Specifically, client \( k \) performs several epochs of local training using its private dataset \( P_k \) to minimize its local loss function \( \mathcal{L}(w, P_k) \) as defined in  Equation \ref{eq:weight_update}.

\begin{equation}
w_{t}^k \leftarrow w_t^k - \eta \nabla \mathcal{L}(w_t^k, P_k)
\label{eq:weight_update}
\end{equation}

where the gradient \( \nabla \ell(w_t^k, P_k) \) reflects the change in the loss function \( \ell \) with respect to the model parameters \( w_t^k \). The learning rate \( \eta \) controls the step size of the parameter updates at each iteration. By iteratively reducing the loss function, gradient descent effectively guides the model parameters toward a more optimal direction.

After each client \( k \) completes the training of its local model, it interacts with its neighbors \( \mathcal{N}_k \) and transmits the parameters of the current round of local model training to them, as defined by the network topology \( G \). Simultaneously, client \( k \) also receives model updates \( w_t^j \) from its neighbors \( j \in \mathcal{N}_k \) and follows an aggregation algorithm \( A(\cdot) \) to form a new model \( w_{t+1}^k \). The aggeration process can be defined in  Equation \ref{eq:aggreagation_update}.

\begin{equation}
w_{t+1}^k = A\left(w_t^k, \{w_t^j : j \in \mathcal{N}_k\}\right)
\label{eq:aggreagation_update}
\end{equation}

By continuously repeating the above processes of local model training and aggregation until the predefined number of rounds is completed, DFL effectively achieves a distributed and collaborative model optimization across the entire network, enhancing both the robustness and privacy of the learning process.

Model poisoning attacks typically cause significant degradation in global model performance by maliciously manipulating model parameters. In the DFL framework, model poisoning attacks typically occur after the client has completed local model training. A malicious client deliberately tampers with its model parameters after training is complete and then sends these tainted parameters to its neighbors and participates in the model aggregation process. In this way, malicious clients are able to gradually contaminate the models of the entire federated learning system, ultimately leading to the degradation of the performance of the global model. The process of the DFL framework with malicious participants is illustrated in Figure\ref{fig:figure2}.

\begin{figure}[htbp]
    \centering
    \includegraphics[width=\columnwidth]{Figure_2.pdf} 
    \caption{DFL Process with Malicious Participants}
    \label{fig:figure2}
\end{figure}
%\vspace{-20pt}

\subsection{Threat Model}
\hspace{1em}\textbf{Adversary Objective.}
The primary goal of the attacker is to minimize the performance of the global model in the DFL system by manipulating the updates to the client models under its control. Specifically, the attacker hopes to mislead the learning process of the global model by introducing carefully crafted malicious updates to the model parameters, ultimately leading to significant degradation in model accuracy or deviation in behavior.

\textbf{Adversary Capability.} 
The attacker has the ability to control a certain percentage of clients and has full access to and modify the local model updates of these clients. The attacker is able to send maliciously modified model parameters to neighboring nodes and participate in model aggregation after each round of training. In addition, the attacker can continuously observe and adjust the attack strategy during the training process to improve the stealth and effectiveness of the attack.

\textbf{Adversary Knowledge.} 
The attacker only has model information of all malicious participants and has no knowledge of the internal information of other benign clients. This knowledge level assumption is very strict, but it is consistent with the conditions of a realistic attack, which reflects the harmfulness of the designed model poisoning attack discussed in next section.


\section{DMPA Approach}
In this section, the general framework for applying the DMPA method in DFL is described, followed by a discussion of the research issues concerning Model Poisoning Attacks in DFL. The section then presents solutions to the problems outlined in the first section.
\subsection{Method Overview}
% \hspace{1em}To successfully improve the effect of non-targeted attacks, DMPA's algorithm is a method to increase the loss of gradient descent and retain the numerical features of the model through feature calculation, thus affecting Benign Clients in DFL communication transmission.To damage the effect of Benign Clients' model prediction, this paper designs malicious model parameters in DFL, which solve the problems proposed by Section 1. Therefore, DMPA is proposed as a model poisoning attack applied in decentralized federated learning. It focuses on the differences in the numerical characteristics of malicious client models and leverages these differences to increase the loss of models, thereby using malicious clients to impact the benign clients. This method is aimed at the optimization attack of model gradient descent, so the main purpose of this method is to study how to extract features through different model parameter differences, to improve the loss of the client model. The DMPA method is shown in Figure \ref{fig:figureDMPA}. The method in this paper is different from other methods of model poisoning attacks in FL (\cite{shejwalkar2021manipulating},\cite{baruch2019little},\cite{zhang2023denial}). They aim at the model similarity (cosine similarity, Euclidean distance, etc.) calculated between malicious clients to generate model vectors similar in model similarity. 

This workd proposes DMPA, an advanced model poisoning attack specifically designed to target DFL models. As illustrated in Figure \ref{fig:figureDMPA}, DMPA employs a tripartite attack strategy. In contrast to existing model poisoning attacks in FL (e.g., \cite{shejwalkar2021manipulating, baruch2019little, zhang2023denial}), which determine the attack direction based on model similarity, DMPA leverages the maximum eigenvalue of the correlation matrix to identify the optimal poisoning direction. The primary objective is to maximize the training loss of benign models after model aggregation.


\begin{figure}[htbp]
    \centering
    \includegraphics[width=\columnwidth]{DMPA.pdf} 
    \caption{Overview of DMPA Attack Process}
    \label{fig:figureDMPA}
\end{figure}


% As shown in Figure \ref{fig:figureDMPA} , This work includes the modification calculation of model update, and the framework of attack before participants share their models with other linked clients, in order to increase the loss of benign clients every round. Specifically, this paper will follow the math in the next section to make local malicious updates.

% DMPA is a method of adding the projection of the gradient vector corresponding to the maximum eigenvalue of the correlation matrix extracted from each malicious model with the parameters of the inverse model so that it can improve the loss in gradient descent. The independent weight calculated when the correlation matrix is extracted from the model parameters representatives the influence degree of the model on gradient descent. At the final output, DMPA calculates the position of the largest value in the top 10\% of the original model vector. In order to prevent and effectively integrate attacks, DMPA has taken special measures to modify these locations. Specifically, uneven values are extracted from the modified vector and filled into the corresponding positions of the averaged vector, thereby maximizing the attack effect.

% Therefore, there are three steps for DMPA to realize the attack: (1) Find the correlation matrix according to the matrix difference,  and calculate the eigenvector of the maximum eigenvalue of the correlation matrix. Using this eigenvector to project on the model parameters when it is Not attacked to get the angular deviation; (2) Add this deviation with the reverse model parameters and average the modified model parameters; (3) Extract uneven values from the modified vector and use them to fill the corresponding positions in the average vector, thus maximizing the attack effect. Averaging the back vector and filling the original modified vector can prevent the attack effect from decreasing after the model is averaged,  and malicious users upload malicious parameters to their connected benign clients for aggregation. Algorithm 1 provides the attack pseudocode of DMPA,  which explains how malicious clients use DFL attack to decentralize the federated learning system.
% The other parts of this chapter will describe the details of DMPA theory.


As illustrated in Figure \ref{fig:figureDMPA}, DMPA is conducted the attack prior to the exchange of models among interconnected clients. There are three steps for DMPA to achieve the attack: (1) Find the correlation matrix according to the difference of the model parameter matrix, and calculate the eigenvectors of the maximum eigenvalue of the correlation matrix. Use this eigenvector to project on the model parameters to obtain the angular deviation; (2) Add this deviation with the inverse model parameters and the average adjusted model parameters; (3) Extract the uneven values from the average vector and use them to fill the corresponding positions in the average vector, thus maximizing the attack effect. Averaging the posterior vector and filling the original modified vector can prevent the attack effect from decreasing after the model is averaged, and malicious users upload malicious parameters to their connected benign clients for aggregation.



\begin{algorithm}[tb]
\caption{DMPA Attack Strategy}
\label{alg:algorithmDMPA}
\textbf{Input}: A matrix $\mathbf{U} \in \mathbb{R}^{d \times n}$ representing all updates, where each column $\mathbf{u}_i$ is a different update vector. \\
\textbf{Output}: The modified updates matrix $\mathbf{\mu}_{\text{new}}$. \\
\begin{algorithmic}[1] % The [1] here enables line numbering
\STATE Compute the mean vector $\mathbf{\mu} = \frac{1}{n} \sum_{j=1}^{n} \mathbf{u}_{i,j,:}$
\STATE Center the updates: $\mathbf{V} = \mathbf{U} - \mathbf{\mu}$
\STATE Compute the covariance matrix: $\mathbf{C} = \frac{1}{n-1} \mathbf{V} \mathbf{V}^\top$
\STATE Compute the standard deviation: $\mathbf{T} = \sqrt{\text{diag}(\mathbf{C})}$
\STATE Compute the correlation matrix: $\mathbf{Y} = \frac{\mathbf{C}}{\mathbf{T} \mathbf{T}^\top}$ \\
\hspace{1em} \textit{where} $\oslash$ \textit{denotes element-wise division}
\STATE Compute eigenvalues and eigenvectors of $\mathbf{Y}$: $(\lambda, \mathbf{V}) = \text{eig}(\mathbf{Y})$
\STATE Find the principal eigenvalue: $\lambda_{\max} = \max(\lambda)$ and corresponding eigenvector $\mathbf{y}_{\max}$
\STATE Compute the projection: $\mathbf{P} = (\mathbf{y}_{\max}^\top \mathbf{U}) \cdot \mathbf{y}_{\max}$
\STATE Modify the updates: $\mathbf{U}^{\text{new}}  = -\mathbf{U} + \mathbf{P}$
\STATE Compute the new mean vector: $\mathbf{\mu}^{\text{new}} = \frac{1}{d} \sum_{j=1}^{d} \mathbf{U}_{j,:}$

\FOR{$i = 1$ to $d$}
    \STATE Compute mask $\mathbf{m}_i = \text{select\_top\_k\_params}(\mathbf{u}_i^2, 10)$
    \STATE Update mean vector: $\mathbf{\mu}^{\text{new}} = \mathbf{m}_i \odot \mathbf{U}_i^{\text{new}} + (1 - \mathbf{m}_i) \odot \mathbf{\mu}^{\text{new}}$
\ENDFOR
%1.9修改后的向量比较可视化,cos,那几个一起比较一下,目的是证明我们这个修改是好的。
%2.补充和sota的比较的说明,好的比较显示为啥会有这样的效果,做一个可视化或者一些分析,;坏的用比较为啥会没有效果。
%3.针对DFL的分析,在稀疏网络的表现,讨论更加详细。解释为什么,可以用总结分析一下。找到规律。全链接和star是稀疏。看看能不能找到为啥。
%4.攻击的计算
\STATE \textbf{return} $\mathbf{\mu}^{\text{new}}$

\STATE \textbf{Function} $\text{select\_top\_k\_params}(\mathbf{vector}, k\%)$:
\STATE \hspace{1em}$\text{sorted\_indices} = \text{argsort}(\mathbf{vector}, \text{descending=True})$
\STATE \hspace{1em} Compute $k = \left\lfloor \frac{\text{length of } \text{sorted\_indices} \times k\%}{100} \right\rfloor$
\STATE \hspace{1em} Initialize mask $\mathbf{m} = \mathbf{0}$ (same shape as $\mathbf{vector}$)
\STATE \hspace{1em} Set top $k$ indices: $\mathbf{m}[\text{sorted\_indices}_{:k}] = 1$
\STATE \hspace{1em} \textbf{return} $\mathbf{m}$
\end{algorithmic}
\end{algorithm}

\subsection{Attack Strategy}
Algorithm 1 provides the attack pseudocode of DMPA,  which explains how malicious clients execute the DPMA attack in DFL. This work defines the model of the received malicious client as $\mathbf{U}$. All malicious client models are transformed into column vectors, represented as $\mathbf{U} = [u_1, u_2, \ldots]$. During the attack phase, a decentralized approach is employed, leading to the computation of a new deviation, denoted as $v_{ij}$, using Equation \ref{eq:pingjun}. These resulting relative offsets collectively constitute the matrix $\mathbf{V}$.

\begin{equation}
    v_{ij} = u_{ij} - \bar{u}_j
    \label{eq:pingjun}
\end{equation}
where $\bar{u}_j$ is the average of the $j$ th model parameter.

To ensure the rigor of this analysis, it is essential to compute the overall standard deviation and provide an unbiased estimate. This necessitates the derivation of a correlation matrix. As demonstrated in Equation \ref{eq:duli}, the correlation matrix $\mathbf{V}$ is determined using the matrix $\mathbf{C}$.

\begin{equation}
    \mathbf{C} = \frac{1}{n-1} \mathbf{V}^T \mathbf{V}
    \label{eq:duli}
\end{equation}

The matrix $\mathbf{C}$ can be regarded as a covariance matrix, in which each element represents the correlation between subscript corresponding vector groups. However, in this paper, when finding out the weight of each model for gradient rise, it is necessary to get the weight of each model for gradient rise under independent assumptions. Therefore, the diagonal elements of the matrix $\mathbf{C}$ are extracted and the square root is found. After dimension reduction, the value can reflect the independent weight vector of each model $\mathbf{T}$.

\begin{equation}
\mathbf{Y} = \frac{\mathbf{C}}{\mathbf{T} \mathbf{T}^\top}
\label{eq:zbtouying}
\end{equation}


To obtain values characterized by independent weight and correlation deviation, Equation \ref{eq:zbtouying} computes the outer product $\mathbf{Y}$ by multiplying the correlation matrix with the independent weight vector. Here, $\mathbf{Y}$ is derived through a linear combination of correlation deviation and independent weight. This vector provides a more accurate representation of the system weight for each parameter column.

\begin{equation}
\mathbf{P} = \left( \mathbf{y}_{\max}^T \mathbf{U} \right)  \cdot \mathbf{y}_{\max}
\label{eq:touying}
\end{equation}

\begin{equation}
\mathbf{\mu}^{\text{new}} = - \mathbf{U} +\mathbf{P}
\label{eq:output}
\end{equation}

Upon obtaining the independent weight, the eigenvector $\mathbf{y}_{\max}$ associated with the maximum eigenvalue of the weight is computed. This eigenvector is then projected onto the original vector to determine the angular deviation, as described by Equation \ref{eq:touying}. Subsequently, the original model parameters are adjusted by incorporating the angular deviation, resulting in the final update, as delineated by Equation \ref{eq:output}.

\section{Evaluation}
\subsection{Experimental Setup}

\textbf{Datasets.} The experiments use CIFAR-10 \cite{krizhevsky2009learning}, MNIST \cite{deng2012mnist} and Fashion-MNIST \cite{xiao2017fashion} as validation datasets, which are widely used in validating model performance. The experiments adopt an independent identically distributed (IID) data partitioning method, which ensures that the data have the same statistical properties. The $\alpha$ parameter is defaulted to 100 as set in (\cite{shejwalkar2021manipulating}, \cite{tan2023collusive}, \cite{li2024fedimp}).

\textbf{Machine learning models.} Three different model architectures are chosen to correspond to different datasets. The batch size is set to 64 for all clients, and the random seed for each client is set to its corresponding ID value. A simple convolutional neural network (CNN) model with Conv2d, BatchNorm2d, 5 Depthwise Conv2d, and fully connected (linear) layers is used for the CIFAR-10 dataset, a model with three fully connected (linear) layers is used for the MNIST dataset, and a simple CNN model with 2 Conv2d and 2 fully connected layers is used for the Fashion-MNIST dataset.

\textbf{Measurement metrics.} The model F1 score is used to assess the attack performance. A lower F1 score indicates a better attack method. All results are the average of the F1 scores of all benign client models after last round of training, which enables a clear assessment of the impact of the MPA attack on the clients in the whole DFL network.

\textbf{Baseline defenses and attacks.} Three of the most effective model poisoning attacks in FL are chosen to attack DFL, namely Lie \cite{baruch2019little}, Min-Sum and Min-Max \cite{shejwalkar2021manipulating}. In DFL, the modifications of Lie, Min-Max and Min-Sum assume that the malicious clients are colluding, and thus these models can be shared among malicious clients \cite{li2023plato}. In addition, four defence methods are chosen; FedAvg, Krum, Trimmed Mean and Median. it is assumed that the malicious client has no knowledge of the benign model and only knows the models of all malicious clients in the current round

\begin{itemize}
    \item \textbf{Lie}: Updates the weights by subtracting the product of the standard deviation of the malicious parameters and the calculated perturbation range from the mean weights\cite{baruch2019little}.
    \item \textbf{Min-Max}: Computes the malicious gradient such that its maximum distance from any other gradient is upper bounded by the maximum distance between any two benign gradients\cite{shejwalkar2021manipulating}.
    \item \textbf{Min-Sum}: Ensures that the sum of squared distances between the malicious gradient and all benign gradients is upper bounded by the sum of squared distances between any benign gradient and the other benign gradients\cite{shejwalkar2021manipulating}.
\end{itemize}




\begin{table*}[ht]
\centering
\caption{Average F1 Score of Benign Clients in DFL with a Configuration Consisting of 40\% Malicious Clients for the MNIST, Fashion-MNIST, and CIFAR-10 Datasets in Fully-Connected, Ring, and Star Topologies.}
\resizebox{\textwidth}{!}{
\begin{tabular}{|c|c|cccc|cccc|cccc|}
\hline
\multirow{3}{*}{Dataset}      & \multirow{3}{*}{Aggregation Rule} & \multicolumn{4}{c|}{Fully}                                            & \multicolumn{4}{c|}{Ring}                                             & \multicolumn{4}{c|}{Star}                                    \\ \cline{3-14} 
                              &                                   & LIE               & Min-Max           & Min-Sum  & DMPA              & LIE               & Min-Max           & Min-Sum  & DMPA              & LIE               & Min-Max  & Min-Sum  & DMPA              \\ \hline
\multirow{4}{*}{MNIST}        & Median                            & 0.919523          & \textbf{0.828144} & 0.861094 & 0.847919          & 0.864083          & 0.727043          & 0.820756 & \textbf{0.691755} & 0.820844           & 0.821185 & 0.830490  & \textbf{0.820525} \\ \cline{2-14} 
                              & Trimmed mean                       & 0.827509          & 0.825695          & 0.843681 & \textbf{0.801535} & 0.864083          & 0.727043          & 0.820756 & \textbf{0.691755} & 0.822594          & 0.819780  & 0.825661 & \textbf{0.813209} \\ \cline{2-14} 
                              & Krum                              & 0.874433          & 0.811460           & 0.885670  & \textbf{0.013312} & 0.873979          & 0.867974          & 0.840262 & \textbf{0.740244} & 0.819409          & 0.812599 & 0.822512 & \textbf{0.582298} \\ \cline{2-14} 
                              & Fed\_avg                          & 0.825292          & 0.826278          & 0.834523 & \textbf{0.802228} & 0.833553          & \textbf{0.64567}  & 0.827210  & 0.672756          & 0.821639          & 0.820869 & 0.825211 & \textbf{0.815387} \\ \hline
\multirow{4}{*}{CIFAR-10}    & Median                            & 0.410982          & 0.679403          & 0.706879 & \textbf{0.044529} & \textbf{0.280601} & 0.421047          & 0.446616 & 0.314470           & 0.548596          & 0.665249 & 0.652866 & \textbf{0.490103} \\ \cline{2-14} 
                              & Trimmed mean                       & 0.652813          & 0.734104          & 0.732013 & \textbf{0.016994} & \textbf{0.258510} & 0.400051          & 0.395156 & 0.305964          & 0.596924          & 0.636560  & 0.658856 & \textbf{0.193741} \\ \cline{2-14} 
                              & Krum                              & \textbf{0.230500} & 0.625591          & 0.625531 & 0.569555          & \textbf{0.372232}   & 0.545091          & 0.646702 & 0.433777          & \textbf{0.538895} & 0.600382 & 0.594952 & 0.569471          \\ \cline{2-14} 
                              & Fed\_avg                          & 0.674934          & 0.740961          & 0.737501 & \textbf{0.016994} & 0.346825          & 0.440712          & 0.449766 & \textbf{0.076015} & 0.617962          & 0.640634 & 0.652659 & \textbf{0.213768} \\ \hline
\multirow{4}{*}{Fashion-MNIST} & Median                            & 0.887029          & 0.892688          & 0.898451 & \textbf{0.881820}  & 0.843861          & 0.800340           & 0.885512 & \textbf{0.773666} & 0.883854          & 0.878469 & 0.887852 & \textbf{0.878103} \\ \cline{2-14} 
                              & Trimmed mean                       & 0.893127          & 0.885061          & 0.903730  & \textbf{0.863626} & 0.837478          & 0.800340           & 0.885512 & \textbf{0.773666} & 0.887656          & 0.878575 & 0.898533 & \textbf{0.874800}   \\ \cline{2-14} 
                              & Krum                              & 0.875101          & 0.862932          & 0.896126 & \textbf{0.058431} & 0.806988           & \textbf{0.453659} & 0.879798 & 0.767491          & 0.860227          & 0.864078 & 0.884784 & \textbf{0.597608} \\ \cline{2-14} 
                              & Fed\_avg                          & 0.895660          & 0.884624          & 0.897920  & \textbf{0.861161} & 0.880633          & \textbf{0.754967} & 0.894226 & 0.769304          & 0.890041          & 0.873614 & 0.892970  & \textbf{0.876514} \\ \hline
\end{tabular}}

\end{table*}


\textbf{DFL overlay topologies.} Three different DFL overlay topologies are employed in the experiments to evaluate the compliance of the proposed attack DPMA in different types of federation. 
\begin{itemize}
    \item \textbf{Fully:} Each node is directly connected to every other node.
    \item \textbf{Star:} All nodes are connected through a central node.
    \item \textbf{Ring:} Each node is connected to two neighboring nodes, forming a closed loop.
\end{itemize}

\textbf{Training environment.} 
The experiments are conducted using Python 3 and PyTorch, running on an NVIDIA T4 GPU. A total of 10 rounds are run, with each client performing 3 local training epochs in each round. The experiments involve 10 clients, and the DFL experimental programs for all nodes are executed sequentially.

\subsection{Experimental Results}
\textbf{Compare DMPA with state-of-art model poisoning attacks.} In this section, the attack method proposed in this study is compared with the current state-of-the-art model poisoning attack methods, including Lie\cite{baruch2019little} and Min-Max and Min-Sum \cite{shejwalkar2021manipulating}. Three topologies (Fully, Star, and Ring) are used for the comparison experiments, and IID data is used for the experiments. The experimental results are based on a configuration of 40\% malicious clients and 60\% benign clients, and calculate the average F1 scores in the final round (smaller F1 scores represent more effective attacks). The experimental results are shown in Table I. Bolded data indicates the best results.

Firstly, from the experimental results, when the proportion of malicious participants is 40\%. The results show that DMPA has a significant attack effect in DFL with different topologies, different aggregation methods, and different datasets, and its F1 scores are all the lowest, indicating that DMPA has the best attack effect, which fully verifies the effectiveness of the method on attack.

\textbf{Impact of overlay network topology.} In fully connected networks, DMPA shows extreme attackability in several experiments. For example, in experiments on the MNIST dataset corresponding to the Krum aggregation method, the CIFAR-10 dataset corresponding to the Median, Trimmed Mean and FedAvg aggregation methods, and the Fashion-MNIST dataset corresponding to the Krum aggregation method, DMPA achieves a score of less than 0.1. This shows that DMPA effectively influences the gradient descent process of 60\% benign clients and successfully fills the difference in gradient descent by increasing the loss. This shows that DMPA has been validated for its reasonable design, which can invalidate the gradient descent by increasing and filling in the gradient difference. In contrast, although the other three attack methods achieve the attack by influencing the direction and magnitude of the model update, most of the average F1 scores among the 60\% benign clients are greater than 0.8, indicating that their attack effects are more limited.

The overall F1 score of DMPA in the ring network is low and close to the lowest of the other attack methods. The F1 scores of DMPA are all less than 0.8, showing high stability. This may be due to the fact that in ring networks, 60\% of the benign clients are buffered to some extent due to the longer chain, mitigating the impact of the attack, which is a major advantage of DFL. As for the other methods, half of them have F1 scores above 0.8, indicating that these methods are not stable enough for model poisoning attacks in the ring structure and are prone to only changing the direction of the model and not effectively affecting other benign clients. This further demonstrates that DMPA exhibits stable and effective attacks that can generally affect benign clients in the DFL ring network.
In the Star network, DMPA also shows its advantages, further proving the effectiveness of the method in this paper. Since the results of CIFAR-10 are the most representative for studying the attack ratio among the three datasets, the CIFAR-10 dataset will be selected in the following to further investigate the impact of the malicious client ratio on the overall experimental results.


\textbf{Impact of increasing the percentage of malicious Participants.} Figure \ref{fig:cifar-attack} to Figure \ref{fig:fmnist-attack} shows the impact of MPA on the F1 score of benign clients after the last round of aggregation in datasets, as the rate of malicious nodes increases from 10\% to 60\% in the DFL environment. The orange of the dotted line is no attack in each aggregation. 

As can be seen, the DMPA method shows the best attack effect in many results to achieve effective attacks. Compared with other methods on different data sets, DMPA has the influence of attacks. However, in different datasets, this paper finds that the more complex the network is, the more effective its influence is. When more network layers are used (for example, CIFAR-10 uses more complex networks), its attack effectiveness is higher than other methods. Obviously, the higher the complexity of the network, the more influence it has. Overall, the method in this paper is more effective than other attack methods at present.

\begin{figure}[htbp]
    \centering
    \begin{subfigure}[b]{\columnwidth}
        \centering
        \includegraphics[width=\textwidth]{combined_cifar10.pdf}
        \caption{CIFAR-10 Dataset Attack Performance}
        \label{fig:cifar-attack}
    \end{subfigure}



    \begin{subfigure}[b]{\columnwidth}
        \centering
        \includegraphics[width=\textwidth]{combined_MNIST.pdf}
        \caption{MNIST Dataset Attack Performance}
        \label{fig:mnist-atack}
    \end{subfigure}


    \begin{subfigure}[b]{\columnwidth}
        \centering
        \includegraphics[width=\textwidth]{combined_FashionMNIST.pdf}
        \caption{Fashion-MNIST Dataset Attack Performance}
        \label{fig:fmnist-attack}
    \end{subfigure}
    \caption{Attack Performance in CIFAR-10, MNIST, and Fashion-MNIST with Different Malicious Clients Rate}
\end{figure}


\section{Conclusion and Future Work}

This study proposes DMPA, a general framework designed to execute model poisoning attacks within DFL systems. The DMPA framework leverages feature angle deviations to ascertain the most effective attack strategy. In contrast to prior attack methodologies that rely on imprecise or heuristic approaches, DMPA employs the model's numerical properties for its computations, thereby maintaining the model's numerical integrity across iterations. Empirical results indicate that DMPA achieves superior attack efficacy compared to existing state-of-the-art model poisoning techniques, underscoring its substantial practical relevance.

Future research is planned to explore both offensive and defensive dimensions. From an offensive standpoint, existing studies predominantly utilize data distributed in an IID manner, thereby simplifying the attack process. However, the complexity of attacks escalates when data distribution is non-IID. Consequently, future research will focus on devising strategies for effective assaults under non-IID conditions. From a defensive perspective, the proposed DPMA underscores the efficacy of using feature angle deviations as a potent attack vector, providing valuable insights for the development of robust mechanisms to protect against potential threats.


%File: formatting-instructions-latex-2025.tex
%release 2025.0
\documentclass[letterpaper]{article} % DO NOT CHANGE THIS
\usepackage{aaai25}  % DO NOT CHANGE THIS
\usepackage{times}  % DO NOT CHANGE THIS
\usepackage{helvet}  % DO NOT CHANGE THIS
\usepackage{courier}  % DO NOT CHANGE THIS
\usepackage[hyphens]{url}  % DO NOT CHANGE THIS
\usepackage{graphicx} % DO NOT CHANGE THIS
\urlstyle{rm} % DO NOT CHANGE THIS
\def\UrlFont{\rm}  % DO NOT CHANGE THIS
\usepackage{natbib}  % DO NOT CHANGE THIS AND DO NOT ADD ANY OPTIONS TO IT
% \usepackage{hyperref}
\usepackage{url}
\usepackage{xcolor}
\usepackage{amssymb}
% \hypersetup{
%     colorlinks=true,
%     linkcolor=blue,    % 设置文内链接的颜色
%     urlcolor=blue      % 设置网址链接的颜色
% }

\usepackage{caption} % DO NOT CHANGE THIS AND DO NOT ADD ANY OPTIONS TO IT
\frenchspacing  % DO NOT CHANGE THIS
\setlength{\pdfpagewidth}{8.5in}  % DO NOT CHANGE THIS
\setlength{\pdfpageheight}{11in}  % DO NOT CHANGE THIS
%
% These are recommended to typeset algorithms but not required. See the subsubsection on algorithms. Remove them if you don't have algorithms in your paper.
\usepackage{algorithm}
\usepackage{algorithmic}
\usepackage{float}
\usepackage{xcolor}
\usepackage{amsmath}
% comments
\newcommand{\yilin}[1]{{\color{red}[Yilin:#1]}}
\newcommand{\rui}[1]{{\color{green}[#1]}}
\newcommand{\ye}[1]{{\color{black}#1}}

%
% These are are recommended to typeset listings but not required. See the subsubsection on listing. Remove this block if you don't have listings in your paper.
\usepackage{newfloat}
\usepackage{listings}
\DeclareCaptionStyle{ruled}{labelfont=normalfont,labelsep=colon,strut=off} % DO NOT CHANGE THIS
\lstset{%
	basicstyle={\footnotesize\ttfamily},% footnotesize acceptable for monospace
	numbers=left,numberstyle=\footnotesize,xleftmargin=2em,% show line numbers, remove this entire line if you don't want the numbers.
	aboveskip=0pt,belowskip=0pt,%
	showstringspaces=false,tabsize=2,breaklines=true}
\floatstyle{ruled}
\newfloat{listing}{tb}{lst}{}
\floatname{listing}{Listing}
%
% Keep the \pdfinfo as shown here. There's no need
% for you to add the /Title and /Author tags.
\pdfinfo{
/TemplateVersion (2025.1)
}

% DISALLOWED PACKAGES
% \usepackage{authblk} -- This package is specifically forbidden
% \usepackage{balance} -- This package is specifically forbidden
% \usepackage{color (if used in text)
% \usepackage{CJK} -- This package is specifically forbidden
% \usepackage{float} -- This package is specifically forbidden
% \usepackage{flushend} -- This package is specifically forbidden
% \usepackage{fontenc} -- This package is specifically forbidden
% \usepackage{fullpage} -- This package is specifically forbidden
% \usepackage{geometry} -- This package is specifically forbidden
% \usepackage{grffile} -- This package is specifically forbidden
% \usepackage{hyperref} -- This package is specifically forbidden
% \usepackage{navigator} -- This package is specifically forbidden
% (or any other package that embeds links such as navigator or hyperref)
% \indentfirst} -- This package is specifically forbidden
% \layout} -- This package is specifically forbidden
% \multicol} -- This package is specifically forbidden
% \nameref} -- This package is specifically forbidden
% \usepackage{savetrees} -- This package is specifically forbidden
% \usepackage{setspace} -- This package is specifically forbidden
% \usepackage{stfloats} -- This package is specifically forbidden
% \usepackage{tabu} -- This package is specifically forbidden
% \usepackage{titlesec} -- This package is specifically forbidden
% \usepackage{tocbibind} -- This package is specifically forbidden
% \usepackage{ulem} -- This package is specifically forbidden
% \usepackage{wrapfig} -- This package is specifically forbidden
% DISALLOWED COMMANDS
% \nocopyright -- Your paper will not be published if you use this command
% \addtolength -- This command may not be used
% \balance -- This command may not be used
% \baselinestretch -- Your paper will not be published if you use this command
% \clearpage -- No page breaks of any kind may be used for the final version of your paper
% \columnsep -- This command may not be used
% \newpage -- No page breaks of any kind may be used for the final version of your paper
% \pagebreak -- No page breaks of any kind may be used for the final version of your paperr
% \pagestyle -- This command may not be used
% \tiny -- This is not an acceptable font size.
% \vspace{- -- No negative value may be used in proximity of a caption, figure, table, section, subsection, subsubsection, or reference
% \vskip{- -- No negative value may be used to alter spacing above or below a caption, figure, table, section, subsection, subsubsection, or reference

\setcounter{secnumdepth}{2} %May be changed to 1 or 2 if section numbers are desired.

% The file aaai25.sty is the style file for AAAI Press
% proceedings, working notes, and technical reports.
%

% Title

% Your title must be in mixed case, not sentence case.
% That means all verbs (including short verbs like be, is, using,and go),
% nouns, adverbs, adjectives should be capitalized, including both words in hyphenated terms, while
% articles, conjunctions, and prepositions are lower case unless they
% directly follow a colon or long dash
\title{SigStyle: Signature Style Transfer via Personalized Text-to-Image Models} 
\author{
    Ye Wang\textsuperscript{\rm 1},
    Tongyuan Bai\textsuperscript{\rm 1},
    Xuping Xie\textsuperscript{\rm 2},
    Zili Yi\textsuperscript{\rm 3},
    Yilin Wang\textsuperscript{\rm 4}\footnote{Corresponding authors},
    Rui Ma\textsuperscript{\rm 1,5}\footnotemark[1]
    }
% \cortext[cor1]{Corresponding author}
\affiliations{
    %Afiliations
    \textsuperscript{\rm 1} School of Artificial Intelligence, Jilin University\\
    \textsuperscript{\rm 2} College of Computer Science and Technology, Jilin University\\ 
    \textsuperscript{\rm 3} School of Intelligence Science and Technology, Nanjing University,
    \textsuperscript{\rm 4} Adobe\\
    \textsuperscript{\rm 5}  Engineering Research Center of Knowledge-Driven Human-Machine Intelligence, MOE, China\\
    \{yewang22, baity23, xiexp21\}@mails.jlu.edu.cn, yi@nju.edu.cn, yilwang@adobe.com, ruim@jlu.edu.cn
    \\
    Project Page:
    \url{https://wangyephd.github.io/projects/sigstyle.html}
    % If you have multiple authors and multiple affiliations
    % use superscripts in text and roman font to identify them.
    % For example,

    % Sunil Issar\textsuperscript{\rm 2}, 
    % J. Scott Penberthy\textsuperscript{\rm 3}, 
    % George Ferguson\textsuperscript{\rm 4},
    % Hans Guesgen\textsuperscript{\rm 5}
    % Note that the comma should be placed after the superscript

    
%
% See more examples next
}



%Example, Single Author, ->> remove \iffalse,\fi and place them surrounding AAAI title to use it
\iffalse
\title{My Publication Title --- Single Author}
\author {
    Author Name
}
\affiliations{
    Affiliation\\
    Affiliation Line 2\\
    name@example.com
}
\fi

\iffalse
%Example, Multiple Authors, ->> remove \iffalse,\fi and place them surrounding AAAI title to use it
\title{My Publication Title --- Multiple Authors}
\author {
    % Authors
    First Author Name\textsuperscript{\rm 1,\rm 2},
    Second Author Name\textsuperscript{\rm 2},
    Third Author Name\textsuperscript{\rm 1}
}
\affiliations {
    % Affiliations
    \textsuperscript{\rm 1}Affiliation 1\\
    \textsuperscript{\rm 2}Affiliation 2\\
    firstAuthor@affiliation1.com, secondAuthor@affilation2.com, thirdAuthor@affiliation1.com
}
\fi


% REMOVE THIS: bibentry
% This is only needed to show inline citations in the guidelines document. You should not need it and can safely delete it.
\usepackage{bibentry}
% \usepackage{float} % 提供 [H] 修饰符
\usepackage{cuted} % 支持跨栏内容
% \usepackage{caption}
% \captionsetup{skip=1.5pt} % 设置标题和图像之间的距离


\begin{document}

\maketitle

% \begin{strip}
%     \centering
%     \includegraphics[width=0.9\linewidth]{imgs/teaser.pdf}
%     \captionsetup{justification=raggedright, width=1\textwidth}  % 设置caption宽度和对齐
%     \captionof{figure}{Our method can achieve high-quality global style transfer (a) while keeping the signature style such as distinct and recognizable visual traits like geometric and structural patterns, color palettes and brush strokes etc. Also, our method is flexible and supports local style transfer (b), style-guided text-to-image generation (c), and texture transfer (d). Best viewed \\ in color.}
%     \label{teaser}
% \end{strip}

% \vspace{-30em}
% \twocolumn[{
%     \centering
%     \includegraphics[width=1\textwidth]{imgs/teaser.pdf}
%     \captionof{figure}{Our method can achieve high-quality global style transfer (a) while keeping the signature style such as distinct and recognizable visual traits like geometric and structural patterns, color palettes and brush strokes etc. Also, our method is flexible and supports local style transfer (b), style-guided text-to-image generation (c), and texture transfer (d). Best viewed in color.}
%     \label{teaser}
%     % \vspace{1em} % 调整标题下方留白
% }]
% \begin{center}
% \includegraphics [width=1\textwidth]{imgs/teaser.pdf}
% \captionof{figure}{Our method can achieve high-quality global style transfer (a) while keeping the signature style such as distinct and recognizable visual traits like geometric and structural patterns, color palettes and brush strokes etc. Also, our method is flexible and supports local style transfer (b), style-guided text-to-image generation (c), and texture transfer (d). Best viewed in color. }
% \label{teaser}
% \end{center}



\begin{figure*}[t]
    \centering
    \includegraphics[width=0.9\textwidth]{imgs/teaser.pdf}
    \caption{Our method can achieve high-quality global style transfer (a) while keeping the signature style such as distinct and recognizable visual traits like geometric and structural patterns, color palettes and brush strokes etc. Also, our method is flexible and supports local style transfer (b), style-guided text-to-image generation (c), and texture transfer (d). Best viewed \\ in color.}
    \label{teaser}
\end{figure*}
% \section{Method}



\begin{abstract}
Style transfer enables the seamless integration of artistic styles from a style image into a content image, resulting in visually striking and aesthetically enriched outputs. 
Despite numerous advances in this field, existing methods did not explicitly focus on the \textit{signature style}, which represents the distinct and recognizable visual traits of the image such as geometric and structural patterns, color palettes and brush strokes etc.
In this paper, we introduce SigStyle, a framework that leverages the semantic priors that embedded in a personalized text-to-image diffusion model to capture the signature style representation. This style capture process is powered by a hypernetwork that efficiently fine-tunes the diffusion model for any given single style image. Style transfer then is conceptualized as the reconstruction process of content image through learned style tokens from the personalized diffusion model. Additionally, to ensure the content consistency throughout the style transfer process, we introduce a time-aware attention swapping technique that incorporates content information from the original image into the early denoising steps of target image generation. Beyond enabling high-quality signature style transfer across a wide range of styles, SigStyle supports multiple interesting applications, such as local style transfer, texture transfer, style fusion and style-guided text-to-image generation. Quantitative and qualitative evaluations demonstrate our approach outperforms existing style transfer methods for recognizing and transferring the signature styles. 
\end{abstract}

% Uncomment the following to link to your code, datasets, an extended version or similar.
%
% \begin{links}
%     \link{Code}{https://aaai.org/example/code}
%     \link{Datasets}{https://aaai.org/example/datasets}
%     \link{Extended version}{https://aaai.org/example/extended-version}
% \end{links}

\section{Introduction}

% \ye{The following setcion }
Style transfer technology \cite{gatys2016image}, which seamlessly incorporates stylistic elements into content images to produce visually impactful results, has gained widespread attention in recent years due to its extensive applications in art design, photography, fashion and other fields. A critical aspect of style transfer is the preservation of the original style during the transfer process. Ensuring that these intricate details and artistic expressiveness are essential for achieving high-quality results, especially when dealing with complex styles such as the  shape and layout of artistic elements, and the patterns of brush strokes and lines. 


Early style transfer methods primarily include techniques relying on local region matching \cite{zhang2013style,wang2004efficient}, or use convolutional neural network (CNN) \cite{gatys2016image, gatys2017controlling, kolkin2019style} or feed-forward network \cite{deng2020arbitrary,huang2017arbitrary,liao2017visual,zhang2022exact} to achieve style transfer. 
With the rapid advancement of diffusion models, diffusion-based style transfer has significantly progressed. These methods can be categorized into two types: tuning-based and tuning-free. A representative tuning-based method is InST \cite{zhang2023inversion}, which introduces a textual inversion-based approach that maps a given single reference style image into a corresponding textual embedding. This textual embedding is then used as a condition to achieve style transfer for the content image. 
In contrast, tuning-free methods such as StyleID \cite{chung2024style}, DiffStyle \cite{jeong2023training}, InstantStyle \cite{wang2024instantstyle}, InstantStyle-Plus \cite{wang2024instantstyleplus}, and FreeTuner \cite{xu2024freetuner} merge style and content features extracted from the attention layers of Stable Diffusion \cite{rombach2022high} to achieve style transfer. These methods offer superior computational efficiency and only require a single forward pass without the need for additional tuning.

Despite the significant progress of the aforementioned methods, signature style transfer remains underexplored. Signature style refers to the unique and recognizable visual traits that defines a particular 
artistic style, such as geometric and structural patterns, color palettes, and brush strokes. 
%style, including geometric and structural patterns, color palettes and brush strokes etc. 
For example, as illustrated in the first row of Figure \ref{fig:complex_style}, the signature style of this image is defined by the structural arrangement and composition of numerous small images that together form the figure of a person. Additionally, the signature style of the image in the second row is characterized by geometric and structural patterns, as well as distinctive color palettes.
% The signature style is more challenging to preserve during the transfer process.
Although existing methods often succeed in transferring basic color information, they fail to capture and retain the essential artistic style from the reference images, including small image blocks, colorful ribbon-shaped lines and other intricate characteristics as shown in Figure \ref{fig:complex_style}.
This highlights a critical limitation: current methods struggle to achieve signature style transfer.


% 

% they still fall short in perfectly preserving complex styles, \ye{including lines, color textures, layouts, semantics, structures, and object elements}.


The main reason for the aforementioned issues is that existing methods insufficiently consider the distinct visual traits of style images.
Meanwhile, 
personalized text-to-image models such as Dreambooth \cite{ruiz2023dreambooth} can capture rich information about style concepts including structures, pose, texture, lines, and more. Inspired by this, we propose SigStyle, a novel framework that leverages rich priors embedded in a personalized text-to-image model to capture complex style for facilitating signature style transfer. However, existing customized text-to-image models often require multiple reference images for fine-tuning, making them unsuitable for style transfer tasks that depend on a single reference image. To address this limitation, we propose a hypernetwork-powered, style-aware fine-tuning mechanism that enables precise concept learning and accurate inversion of style attributes using just one style reference image.

%On the other hand, existing customized text-to-image models typically require multiple reference images for fine-tuning. It is not suitable for style transfer tasks that rely on a single reference image. Therefore, we introduce a hypernetwork-powered style-aware fine-tuning mechanism that enables precise concept learning and the inversion of style attributes given a single style reference image.

Specifically, we employ a lightweight hypernetwork to modulate and refine diffusion UNet weights. This strategy can not only ensure smoother updates of the parameters and reduces the likelihood of overfitting, but also effectively capture and represent the signature style attributes from a single reference image. Furthermore, unlike existing fine-tuning approaches \cite{zhang2023inversion, ruiz2023dreambooth, zhang2023inversion}, our method focuses on fine-tuning only the modules related to style attributes, instead of the entire diffusion network. Such fine-tuning mechanism not only enhances tuning efficiency, but also improves the style inversion accuracy. Based on this fine-tuning mechanism, we can represent a signature style as a special token *.
Subsequently, we define style transfer as the reconstruction process of the content image based on the style token learned from the customized diffusion model. Additionally, to maintain content consistency, we propose a time-aware attention swapping technique inspired by PhotoSwap \cite{gu2024photoswap} and SwapAnything \cite{gu2024swapanything}. This technique transfers content-related attention from the original image generation process to the target image generation during the early denoising timesteps, ensuring content consistency throughout the style transfer process.


As shown in Figure \ref{teaser}, our method achieves high-quality signature style transfer results across various complex style references. Moreover, our framework not only supports global style transfer, but also supports a broad range of applications, including local style transfer, texture transfer, style fusion, and style-guided text-to-image generation. Extensive experiments and evaluations further demonstrate the versatility and effectiveness of our method.

\begin{figure}[t]
    \centering
    \includegraphics[width=0.9\linewidth]{imgs/complex_style1.pdf}
    \caption{Signature style transfer comparison with SOTA methods on two complex style references.}
    \label{fig:complex_style}
\end{figure}


Our contributions can be summarized as follows:

\begin{itemize}
    \item We propose SigStyle, a novel framework that is the first to explicitly focusing on the challenging signature style transfer via utilizing the personalized text-to-image diffusion models.
    
    % . \yilin{most style transfer methods use single style image. We should emphasize we are the first work to consider the style preservation.}
    \item  We introduce a hypernetwork-powered style-aware fine-tuning mechanism that can learn the signature style attributes from only a single style image. This approach overcomes the limitation of customization methods that need multiple reference images, making it more suitable for style transfer tasks.
    % that not only eliminates severe overfitting issues associated with single-image fine-tuning but also enables precise learning and inversion of style attributes. Combined with a time-aware attention swapping technique, this allows efficient style transfer while maintaining content consistency.  \yilin{existing personalize text to image model can capture the richful information from the referece concepts including pose texture ans structures. But they often require a few images for finetuing which is not suitable for style transfer. Thus we propsoed hyper network.}
    \item Extensive experiments show that our method outperforms existing state-of-the-art methods on style transfer. Notably, our approach excels in preserving signature style details, highlighting its superior capability in signature style transfer. Moreover, the diverse applications emphasize the generalizability and versatility of our method.
    % \yilin{Demonstrate that we can keep the style very well others can't.}
\end{itemize}

\section{Related Work}



\paragraph{\textbf{Style Transfer.}}
% % InST, StyleDiffusion, DiffStyle,  Zero-shot contrastive loss for text-guided diffusion image style transfer DEA-Diff, 
% Neural style transfer is an example-guided image generation task that transfers the style of one image onto another while retaining the content
% of the original. 近年来,越来越多的工作采用diffusion model来实现 Neural Style Transfer。这些方法主要分为两个类别,tuning-based和tuning-free。作为代表工作,InST introduced a textual inversion-based approach, aiming to
% map a given style into corresponding textual embeddings,然而该方法对Style的反转并不准确,在应用反转后的style进行迁移,会改变原始图像的内容。StyleDiffusion [45] aimed to disentangle style and content by introducing CLIP-based style disentanglement loss for fine-tuning DM for style transfer. FreeTuner提出了compositional personalization,能够composing different subject and style concepts,然而组合式的定制化often overlooking the
% preservation of the content image’s structural integrity.

% tuning-free的方法能够在single-forward 过程中实现风格迁移,而无需微调。
% DEADiff [12] stands out by extracting
% disentangled representations of content and style using a paired dataset, facilitated by the Q-Former technique. InstantStyle [20], a recent innovation, employs
% block-specific injection techniques to implicitly achieve a decoupling of content
% and style, offering a nuanced approach to style transfer
% StyleID [1] operates by manipulating self-attention layers, proposing innovative techniques such as query preservation and initial latent AdaIN to maintain the integrity of the content

% 在这篇论文中,我们的目的是提出一个基于单图微调的高效style迁移方法,能够消除严重的过拟合问题,实现style的精准反转和无缝的风格迁移而保持原始的内容完整。

% % DiffStyle,DEADiff,StyleID,InstantStyle,InstantStyle-Plus,
Style transfer \cite{zhang2019multimodal,wang2020diversified,wang2020collaborative,park2019arbitrary,lu2019closed,li2018closed,li2017universal,lai2017deep,gatys2016image} applies the artistic style of one image to another while preserving the latter's content and structure. Early methods \cite{zhang2013style,wang2004efficient} relied on handcrafted features, while CNN-based approaches \cite{gatys2016image, gatys2017controlling, kolkin2019style} leveraged pre-trained networks to capture style pattern. Arbitrary style transfer methods \cite{deng2020arbitrary,huang2017arbitrary,liao2017visual,zhang2022exact} further improved this by using unified feed-forward models for flexible inputs.




In recent years, diffusion models have increasingly been employed for style transfer. These methods can be broadly categorized into two types: tuning-based methods and tuning-free methods. A representative example of the former is InST \cite{zhang2023inversion}, which introduces a text inversion-based approach, aiming to map a given style to its corresponding text token. 
% However, this method lacks precision in style inversion, often altering the content of the original image during style transfer. 
StyleDiffusion \cite{wang2023stylediffusion} refines diffusion models by introducing a CLIP-based style decoupling loss, effectively separating style from content. 
% FreeTuner \cite{xu2024freetuner} proposes compositional personalization, allowing the combination of different subject and style concepts. However, this approach often alters the structure and content of the original image.
Tuning-free methods achieve style transfer in a single forward process without model fine-tuning. DEADiff \cite{qi2024deadiff} utilizes the Q-Former \cite{li2023blip} and paired datasets to extract decoupled representations of content and style, facilitating style transfer. InstantStyle \cite{wang2024instantstyle} uses specific block injection techniques to implicitly decouple content and style for effective style transfer. Building on this, InstantStyle-Plus \cite{wang2024instantstyleplus} introduces ControlNet \cite{zhang2023adding} to further maintain the integrity of image content. StyleID \cite{chung2024style} adjusts self-attention layers and introduces novel techniques such as query preservation and initial latent AdaIN to maintain content integrity.

Nevertheless, these approaches struggle with achieving signature style transfer. In this paper, we address this challenge by utilizing a personalized text-to-image model to perform signature style transfer, generating visually compelling and aesthetically enhanced outputs.  


\paragraph{\textbf{Personalized Text-to-Image Generation.}} 

Recent studies \cite{ruiz2023dreambooth, gal2022image, ruiz2023hyperdreambooth, kumari2023multi} have increasingly focused on using visual exemplars as a foundation for image generation to address the inherent ambiguity and unpredictability of text-based prompts. This approach involves collecting multiple reference images to fine-tune diffusion models. 
However, for style transfer tasks that rely on a single style image, the aforementioned methods often cause severe overfitting when fine-tuning on this single image, significantly diminishing the effectiveness of the style transfer.
In contrast, our proposed hypernetwork-powered style aware fine-tuning method can overcome the limitations associated with fine-tuning on a single style image. It enables precise inversion of style attributes without the drawbacks of overfitting
% \yilin{will single image based personalized method work?}

% In addition to customizing the general essence of the reference image, some works have begun to explore attribute-level T2I customization \cite{voynov2023p+, zhang2023prospect, goel2023pair, yang2023paint} . However, these methods either require multiple reference images \cite{voynov2023p+} or extensive fine-tuning \cite{goel2023pair,yang2023paint, ye2023ip} across large datasets. 




\paragraph{\textbf{Parameter Efficient Fine Tuning (PEFT).}} PEFT represents an innovative approach in the refinement of deep learning models, emphasizing the adjustment of a subset of parameters rather than the entire model. These parameters are identified as either specific subsets from the originally trained model or a minimal number of newly introduced parameters during the fine-tuning phase. PEFT has been applied in text-to-image diffusion models \cite{saharia2022photorealistic,rombach2022high} through techniques such as LoRA \cite{ryu2023low} and adapter tuning \cite{mou2023t2i,ye2023ip,wei2023elite,chen2024subject,ma2023unified}. In this paper, we leverage a hypernetwork to adjust and refine a unique subset of pre-trained parameters.

\begin{figure*}[t]
    \centering
    \includegraphics[width=0.9\textwidth]{imgs/pipeline.pdf}
    \caption{The SigStyle framework. First, given a style image, we perform hypernetwork-powered style-aware fine-tuning for style inversion and represent the reference style as a special token * (see Figure \ref{fig:method}.a). In Figure \ref{fig:method}.b, the upper branch represents the reconstruction process of the content image, while the lower branch represents the generation process of the target image. When generating the target image using a pre-trained model and target text, we first use DDIM Inversion to map the content image into noise latents, which are then copied as the initial noise for generating the target image. Then, we adopt time-aware attention swapping to inject structural and content information during the first $k$ steps of the denoising process (see Figure b). In the subsequent $T-k$ steps, we proceed with the usual denoising process without any swapping. Finally, by decoding with VAE, we obtain the style-transferred image. }
    \label{fig:method}
\end{figure*}
\section{Method}



\subsection{Preliminaries}

\paragraph{\textbf{Stable Diffusion.}} 
Stable Diffusion \cite{rombach2022high} is a state-of-the-art text-to-image model that operates in a low-dimensional latent space. It encodes an image $x$ into a latent $z$ via a VAE encoder, adds noise $\epsilon$ at time step $t$ to obtain a noisy latent $z_t$, and uses a CLIP text encoder $\tau$ to incorporate textual prompts $c$ through cross-attention layers. A conditional UNet $\epsilon_\theta$ is then trained to predict the noise $\epsilon$, guided by the following training objective:
% Stable Diffusion \cite{rombach2022high}, a state-of-the-art text-to-image generation model, operates within a low-dimensional latent space. It begins by encoding an input image $x$ into a latent representation $z$ using a VAE encoder. Noise $\epsilon$ is then introduced at time step $t$ to create a noisy latent $z_t$. To guide the generation process with text conditions, Stable Diffusion incorporates a CLIP text encoder $\tau$ to encode textual prompts $c$, which are integrated into the cross-attention layers for interaction with the noisy latents. Finally, a conditional U-Net backbone $\epsilon_\theta$ is trained to predict the noise $\epsilon$. The training objectives is as follows:
\begin{equation}
    L_{SD}(\theta) := \mathbb{E}_{t,x_0,\epsilon} \left[ \lVert \epsilon - \epsilon_\theta(z_t, t, \tau(c)) \rVert^2 \right] \label{eq:LLDM}.
\end{equation}

% Classifier-free guidance \cite{ho2022classifier} is employed to steer the efficient inference of Stable Diffusion. It is designed to extrapolate the output of the model in the direction of $\epsilon_\theta(x_t \mid \tau(c))$
% and away from $\epsilon_\theta(x_t \mid \oslash)$ as follows:

% \begin{equation}
% \hat{\epsilon}=\epsilon_\theta\left(x_t \mid \oslash\right)+s \cdot\left(\epsilon_\theta\left(x_t \mid \tau(c)\right)-\epsilon_\theta\left(x_t \mid \oslash\right)\right)
% \end{equation}
% $\oslash$ is a null text, s is the guidance weight and increasing $s > 1$ 
% strengthens the effect of guidance.

\paragraph{\textbf{Attention Mechanisms in Diffusion.}}

The UNet-based diffusion model comprises self-attention and cross-attention modules. As demonstrated in \cite{hertz2022prompt, tumanyan2023plug, gu2024photoswap, gu2024swapanything}, self-attention maps capture image structure and identity-related information. The computation of self-attention maps is defined as follows:
\begin{equation}
S A=\operatorname{Softmax}\left(\frac{Q_s K_s^T}{\sqrt{d}}\right),
\end{equation}
where $Q_s$ and $K_s$ represent different projections of visual features, and $d$ is the dimension of feature used for scaling.

% \paragraph{\textbf{DDIM Inversion}}





\subsection{SigStyle}






\paragraph{\textbf{Pipeline Overview.}}
Given a style image $I_{s}$ and a content image $I_{c}$, SigStyle can seamlessly transfer the signature style to the content image while preserving the original content. The SigStyle pipeline is illustrated in Figure \ref{fig:method}. First, we use a hypernetwork to perform style-aware fine-tuning on diffusion model with a single style image and represent the target style using a special token $*$ (see Figure \ref{fig:method}.a).
Next, we employ DDIM Inversion to obtain the noise $z_{t}$ that can reconstruct the content image. Then, we extract the required self-attention maps $SA_{c}$ from the UNet, which preserve the structure and content information of the content image. Finally, during the generation of the target image that conditioned on the noise $z_{t}$ and the target text prompt $P_{t}$, we use the obtained self-attention maps to replace the corresponding ones in the first $k$ steps to maintain the content information (see Figure \ref{fig:method}.b ).

\paragraph{\textbf{Hypernetwork-powered Style-aware Fine-tuning.}} 
Object-level inversion \cite{ruiz2023dreambooth} typically requires fine-tuning the entire UNet, but such an approach is not suitable for style inversion. We hypothesize that style, as an attribute of the image, should be learned and understood by specific modules within the network. To verify this hypothesis, we conducted an in-depth analysis of the style attribute learning preferences of UNet.



% \paragraph{\textit{Style Learning Preferences Analysis of UNet.}}
\textit{Style Learning Preferences Analysis of UNet.}
Previous work \cite{voynov2023p+, zhang2023prospect} has also attempted to analyze the learning preferences of different layers, focusing mainly on shape and color. However, these studies did not accurately identify the specific layers responsible for learning style attributes. Therefore, we conducted a simple experiment using Stable Diffusion to analyze the learning preferences of different modules of UNet. As shown in Figure \ref{fig:style_tuning}, we first divided the UNet into two parts: the encoder and the decoder. we selected a style reference image, for example, \textit{The Starry Night}, and inputted default text prompts such as ``a photo in the style of *." The * is learnable token embedding. Subsequently, we separately fine-tuned the encoder and decoder modules of the UNet and utilized the fine-tuned model to generate the corresponding images. \ye{During fine-tuning, * is progressively refined to encapsulate the distinctive style patterns in the style image. This enables * to function as an abstract, high-level representation of the style, effectively acting as a trigger for the fine-tuned T2I model to generate stylized outputs. Importantly, the original style image is not required as input once the model has been fine-tuned, as * captures the essential elements of the style.}
During inference,
The text prompt used for inference is ``a dog in the style of *". 

\begin{figure}[t]
    \centering
    \includegraphics[width=0.8\linewidth]{imgs/style_tuning.pdf}
    \caption{Style learning preferences analysis of UNet's encoder and decoder.}
    \label{fig:style_tuning}
\end{figure}

We observe significant differences in the generated results when separately fine-tuning the encoder and decoder. Fine-tuning the encoder fails to produce images that match the style of the reference image, while fine-tuning the decoder results in images with similar styles. This experiment further validates our hypothesis that style attributes are learned by specific network modules, namely the decoder module.



Based on this insight, we propose a style-aware fine-tuning mechanism that tunes only the decoder module of the UNet to learn style attributes. However, this approach may suffer from severe overfitting issues, particularly when fine-tuning with a single image. To address this challenge, we use a hypernetwork to refine and modulate the network parameters, achieving smoother updates and thereby reducing the risk of overfitting.

\textit{Hypernetwork.}
The hypernetwork architecture is  shown in the Figure \ref{fig:hypernetwork}, we draw inspiration from E4T \cite{gal2023encoder} to design our hypernetwork.
The module takes as input a learnable constant $cons$ (default-initialized to 1) and the dimension information $[dim_{r}, dim_{c}]$ of the target weight parameters. It is then trained to predict weight offsets in the same dimensions as the target weight parameters. Here, $dim_{r}$ represents the number of rows of the target weight parameters, and $dim_{c}$ represents the number of columns.
In detail, the learnable constant passes through two linear layers, yielding outputs that are multiplied to derive the initial weight offset matrix. Row and column transformations are then applied to this matrix to obtain the final weight offset matrix $\Delta w$. As discussed in the literatures \cite{gal2023encoder,kumari2023multi,wei2023elite}, the weights of self-attention and cross-attention play a crucial role in the process of image customization.
Therefore, we utilize the hypernetwork as a weight offsets prediction module to modulate and guide the updates of attention-related weights within the decoder.
The high-level parameter update process is defined as follows:
\begin{figure}[t]
    \centering
    \includegraphics[width=0.8\linewidth]{imgs/hypernetwork.pdf}
    \caption{The architecture of hypernetwork.}
    \label{fig:hypernetwork}
\end{figure}
\begin{align}
    \Delta w &= hypernetwork(cons,dim_{r},dim_{c}),  \\
    w^*_{attn} &= w_{attn} + \lambda * \Delta w,
\end{align}
where $w_{attn}$ denotes the general term for attention-related parameters, including the query matrix, key matrix and value matrix for self-attention and cross-attention layers; $\lambda$ is a weight coefficient that is used to regulate the updating strength of parameters. During training, $\lambda$ is set to 1.0. 

\paragraph{Loss Function.} 
To guide the style inversion, we adapt the original noise prediction loss function to work with the text prompt for style learning:
\begin{equation}
    Loss(\theta) := \mathbb{E}_{t,x_0,\epsilon} \left[ \lVert \epsilon - \epsilon_\theta(x_t, t, \tau(P_s)) \rVert^2 \right]. 
    \label{eq:LLDM}
\end{equation}
Note that $\theta$ denotes the parameters of the decoder and hypernetwork, $t$ denotes the current time step, $\epsilon$ denotes the noise, $x_t$ represents the noise latent of style image at time $t$,  $\tau(P_s)$ represents the CLIP embedding of the input text prompt, i.e., ``a photo in the style of *.''



\begin{figure*}[t]
    \centering
    \includegraphics[width=0.85\textwidth]{imgs/compare.pdf}
    \caption{ Qualitative comparison with various SOTA image style transfer methods for global style transfer. }
    \label{fig:compare_other}
\end{figure*}


\paragraph{\textbf{Time-aware Attention Swapping.}}
To achieve style transfer, a straightforward approach could be using the noise of a content image as input and performing denoising with the pre-trained personalized diffusion model. While this method captures the target style, it often alters the original image's content. Previous work \cite{gu2024photoswap, gu2024swapanything} has shown that the self-attention feature maps in the early stages of denoising encode the image's content and structural information. Therefore, we first store the self-attention maps from the UNet during the content image reconstruction, treating them as content priors. During the generation of the target image, we replace the self-attention maps with these content priors. Since content and structural information are typically established in the early stages of denoising, this attention swapping operation is performed only within the first $k$ steps.



\paragraph{Implementation Details.}

We employ Stable Diffusion 1.4 \cite{rombach2022high} as our base model. We use BLIP-2 \cite{li2023blip} to generate the text prompt for the content image.
% , and the style prompt is in the form of ``a photo in the style of *". 
We apply random cropping and horizontal flipping to the input image to improve the robustness for style learning. Our model is trained on a single NVIDIA A6000 GPU with a batch size of 1 and a learning rate of 1e-6. 
The number of fine-tuning steps and time may vary slightly for each reference image, but on average, approximately 1500 steps are sufficient. We set $k$ = 25 for time-aware attention swapping. It takes about 10 seconds to generate a style-transferred image during the inference stage.



\section{Experiments}



% \begin{table*}[]


% \resizebox{\linewidth}{!}{
% \begin{tabular}{cccccccccc}
% \hline
% Methods & Ours & StyleID & InstantStyle & Diffuse-IT & InST & AesPA-Net & CAST & AdaAttN & ArtFlow \\ \hline
% % FID \downarrow   & \textbf{630}  & 470     & 221       & 213        & 89   & 322       & 583  & 379     & 414        \\ 
% LPIPS \downarrow   & 0.4923  & 0.5959     & 0.7226       & 0.7435        & 0.8199   & 0.5328       & 0.6485  & 0.4727     & 0.6922        \\ 
% User Study \uparrow   & \textbf{630}  & 470     & 221       & 213        & 89   & 322       & 583  & 379     & 414        \\ \hline
% \end{tabular}
% }
% \caption{Quantitative comparison with several state-of-the-art image style transfer methods.}
% \label{tab:user_study}
% \end{table*}

% \usepackage{graphicx}
\begin{table*}[]
\centering
\resizebox{0.9\textwidth}{!}{%
\begin{tabular}{cccccccccc}
\hline
Metrics\textbackslash{}Methods & Ours         & StyleID & InstantStyle & Diffuse-IT & InST & AesPA-Net & CAST           & AdaAttN & ArtFlow \\ \hline
Style Loss $\downarrow$   & \textbf{0.7641}  & 1.0902     & \underline{0.7653}       & 1.2312        & 0.9875   & 0.8766       & 0.7842  & 1.0288     & 0.7693        \\ 
LPIPS $\downarrow$   & \underline{0.5191}  & 0.5959     & 0.7226       & 0.7435        & 0.8199   & 0.5328       & 0.6485  & \textbf{0.4727}     & 0.6922        \\ 
\hline
Rank 1 (\%) $\uparrow$ & \textbf{9.1} & 5.06    & 2.76         & 2.85       & 1.01 & 2.20      & \underline{6.07}           & 3.49    & 3.49    \\
Top 3 (\%) $\uparrow$  & 14.54        & 13.61   & 6.16         & 6.25       & 2.48 & 10.40     & \textbf{17.75} & 13.71   & \underline{15.10}   \\ \hline
\end{tabular} 
}
\caption{Quantitative comparison with SOTA image style transfer methods.  The best results
are in \textbf{bold} while the second best results are marked with \underline{underline}.}
\label{tab:user_study}
\end{table*}

\subsection{Qualitative Comparison}
We compared multiple state-of-the-art methods to demonstrate the effectiveness of our approach. As shown in the third column of Figure \ref{fig:compare_other}, our method can well preserve and transfer the reference signature style. Specifically, our approach effectively transfers and preserves key signature style elements, such as the moon in the background in the first row and the small picture details in the second row of the style image, onto the content image.  
In contrast, other methods typically only transfer simple colors and fail to achieve high-quality  signature style transfer. Furthermore, for complex style images, our method generates more natural images compared to other methods, which often produce many artifacts (see the last four columns of the second row). Our method also maintains the structure of the content image well, whereas other tuning-based methods, such as InST \cite{zhang2023inversion} and DiffuseIT \cite{kwon2022diffusion}, alter the original content image's structure. This further demonstrates that our hypernetwork-powered style-aware tuning method can more precisely inverse style attributes, while the time-aware attention injection effectively preserves content information.


\subsection{Quantitative Comparison}
We randomly selected 10 content images and 15 style images, generating a total of 150 stylized images for each method by applying the Cartesian product to the content and style images. To evaluate content fidelity, we followed \cite{chung2024style} and employed the LPIPS metric \cite{zhang2018unreasonable}, which measures the similarity between the stylized image and the corresponding content image. \ye{ For style similarity, we adopted the Style Loss \cite{gatys2016image}, measuring the alignment between the stylized image and the corresponding style image.}

We compared our method with eight SOTA style transfer methods: StyleID, InstantStyle, Diffuse-IT, InST, AesPA-Net, CAST, AdaAttN, and ArtFlow. \ye{As presented in Table~\ref{tab:user_study}, our method achieves the second-lowest LPIPS score of 0.5191 and the lowest Style Loss of 0.7641, surpassing the majority of existing approaches.}
This result indicates that our approach effectively preserves content and style fidelity during signature style transfer. Furthermore, we also conducted a comprehensive user study. We randomly collected 12 content-style pairs, each producing nine images generated by the methods mentioned above. Participants were asked to select and rank the top three results from these nine randomly-arranged images based on two criteria: 1) the preservation of signature or key styles from the style image, including distinct and recognizable visual traits such as geometric patterns, color palettes, and brush strokes; 2) the preservation of content from the content image, including overall shape, structure, and semantics of the main object. A total of 1,152 votes were collected from 32 participants. As demonstrated in Table~\ref{tab:user_study}, we computed the proportion of first-place rankings (Rank 1) and the overall percentage of top three votes (Top 3) for each method. Our approach achieved the highest Rank 1 score (9.1\%) and a relatively high percentage of Top 3 votes (14.54\%). These results highlight that the style-transferred images generated by our method were preferred by users, showcasing its superior performance in signature style transfer.



% \setlength{\intextsep}{10pt}  % 调整为更小的间距

\ye{As shown in Table \ref{tab:time}, we compare our method with tuning-based methods like InST and Diffuse-IT, which require 20 and 18 minutes for tuning, respectively. Our method is more efficient, with 10 minutes of tuning and 10 seconds of inference. Although tuning-free methods like StyleID, InstantStyle, and CAST are faster (5-8 seconds for inference), they do not preserve style details well. Our method achieves a better balance between speed and style quality.}

\begin{table}[t]
\centering
\resizebox{0.9\linewidth}{!}{
\setlength{\tabcolsep}{2pt}
\begin{tabular}{c c c | c c c}
\hline
Methods & Tuning & Inference & Methods & Tuning & Inference \\
\hline
InST & 20 mins & 24 s & StylelD & no need & 7 s \\
Diffuse-IT & 18 mins & - & InstantStyle & no need & 5 s \\
Ours & 10 mins & 10 s & CAST & no need & 8 s \\
\hline
\end{tabular}
}
\caption{\ye{Comparison of time consumption in the fine-tuning and inference phases between ours and other methods.}}
\label{tab:time}
\end{table}

\begin{figure}[t]
    \centering
    \includegraphics[width=0.7\linewidth]{imgs/hypernetwork_ablation.pdf}
    \caption{The ablation study on hypernetwork.}
    \label{fig:ablation_hypernetwork}
\end{figure}


\subsection{Ablation Study}

\paragraph{Hypernetwork.}
As shown in Figure \ref{fig:ablation_hypernetwork}, our hypernetwork effectively facilitates the precise learning and inversion of the style. In contrast, when fine-tuning without the hypernetwork, the generated images match the target text but fail to retain the reference style. This further demonstrates the effectiveness of our proposed hypernetwork.




\begin{figure}[t]
    \centering
    \includegraphics[width=0.87\linewidth]{imgs/ablation_attention_swap.pdf}
    \caption{The ablation study on attention swapping.
    }
    \label{fig:ablation_attn_swap}
\end{figure}


\begin{figure}[t]
    \centering
    \includegraphics[width=0.87\linewidth]{imgs/local_transfer.pdf}
    \caption{Local style transfer results of our method. The transfer area is the foreground animal region by default.}
 
    \label{fig:local_transfer}
\end{figure}
% \vspace{-10pt}
\begin{figure}[t]
    \centering
    \includegraphics[width=0.77\linewidth]{imgs/texture_transfer.pdf}
    \caption{Cross-domain texture transfer of our method.
    }
    \label{fig:texture_transfer}
\end{figure}
% \vspace{-10pt}
\begin{figure}[!ht]
    \centering
    \includegraphics[width=0.77\linewidth]{imgs/2style_transfer.pdf}
    \caption{Style transfer based on multiple style references. 
    }
    \label{fig:2style_transfer}
\end{figure}

% \vspace{-15pt}
\begin{figure}[!ht]
    \centering
    \includegraphics[width=0.77\linewidth]{imgs/style_personlized.pdf}
    \caption{The style-guided text-to-image generation results.}
    \label{fig:style_guided}
\end{figure}





\paragraph{Attention Swapping.}
As shown in Figure \ref{fig:ablation_attn_swap}, when attention swapping is not used, the generated images only retain the target style but fail to maintain the image content. As the \(k\) value increases, the structure and other content-related information in the synthesized image become increasingly similar to the content image, while the style gradually diminishes. This demonstrates the effectiveness of the attention swapping technique in preserving content consistency.


% \vspace{-1.5cm}
% \paragraph{\textbf{Further Applications.}}
\subsection{Further Applications}
In addition to style transfer tasks, our method supports other applications including local style transfer, texture transfer, style fusion, and style-guided text-to-image generation.


% \paragraph{\textit{Local Style Transfer.}}
\paragraph{Local Style Transfer.}
Local style transfer applies style only to regions specified by a user-provided mask. Within the masked areas, we use SigStyle for style transfer, while denoising reconstruction is applied to non-masked areas to maintain consistency. Blending operations then integrate these regions seamlessly, producing a complete image and achieving local style transfer, as shown in Figure \ref{fig:local_transfer}.

% Local style transfer aims to apply style transfer only to the regions of interest specified by the user, while keeping other areas unchanged. This requires the user to provide a mask for the regions of interest in advance. Within the masked regions, we apply SigStyle for style transfer. For the areas outside the mask, we perform denoising reconstruction to ensure consistency. Finally, we perform blending operations to integrate these regions seamlessly, resulting in a complete image and achieving local style transfer, as shown in Figure \ref{fig:local_transfer}.

\paragraph{Texture Transfer.}
Texture, appearance, and style are interrelated concepts best learned by the same module, the UNet decoder. By replacing "style" with "appearance" in prompts while keeping inversion and transfer processes unchanged, a mask constrains the texture transfer region. Figure \ref{fig:texture_transfer} demonstrates high-quality cross-domain texture transfer, preserving the original image's pose, structure, identity, and other content.

% Texture, appearance and style are all visual abstract concepts with significant similarities. Therefore, these concepts should be learned by the same module, namely the UNet decoder. Based on this, we only need to modify the word "style" to "appearance" in the content and target text prompts, while keeping the inversion and transfer processes unchanged, and use a mask to constrain the texture transfer region. 
% Figure \ref{fig:texture_transfer} shows the results of high-quality cross-domain texture transfer, demonstrating our method can seamlessly transfer the reference texture onto the original image while preserving the original image's pose, structure, identity, and other content information.



\paragraph{Style Fusion.}
We can fuse multiple styles to create a new style for transfer and customized generation, leading to more intriguing results. This is achieved by fine-tuning with multiple style images. As shown in Figure \ref{fig:2style_transfer}, our method effectively transfers the fused style onto the content image. 

% For instance, we can generate an image with a sky that combines the styles of \textit{Rain Princess} and \textit{The Starry Night}, or create a scene featuring the sky from \textit{The Starry Night} with the Eiffel Tower in the style of \textit{The Scream}.

\paragraph{Style-guided Text-to-Image Generation.}
Our fine-tuning mechanism represents style as a special token *, enabling style-guided text-to-image generation. With a single style image, we can generate images guided by that style (see the first row of Figure \ref{fig:style_guided}). When using multiple style images, our method fuses them into a new style for more creative outputs (see the second row of Figure \ref{fig:style_guided}).





\section{Conclusion}In this paper, we presented SigStyle, a novel framework for high-quality signature style transfer using only a single style reference image. By introducing a hypernetwork-powered style-aware fine-tuning mechanism, our approach enhances the accuracy and efficiency of style inversion while addressing severe single-image overfitting issues. Additionally, our time-aware attention swapping technique ensures content consistency during the style transfer process. Extensive experiments show that SigStyle outperforms existing methods in preserving signature styles and supports various applications, including global and local style transfer, texture transfer, and style-guided text-to-image generation etc. 
% SigStyle encounters efficiency challenges, necessitating a distinct tuning process for each style.
% While achieving promising results, our SigStyle still has some limitations.
Currently, SigStyle still needs to fine-tune the diffusion model for each given style image during inference, which limits its deployment on resource-constrained devices.
How to further reduce the computation cost of the style learning and transferring process will be worthy to investigate.
Moreover, it will also be interesting to explore how to use more specified text prompt to guide more refined and more controllable style transfer.



\section*{Acknowledgments}
This work is supported in part by the National Natural Science Foundation of China (No. 62202199, No. 62406134) and the Science and Technology Development Plan of Jilin Province (No. 20230101071JC).

\bigskip


\bibliography{aaai25}

\end{document}


\end{document}
