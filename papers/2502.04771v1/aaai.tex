%File: anonymous-submission-latex-2025.tex
\documentclass[letterpaper]{article} % DO NOT CHANGE THIS
\usepackage{aaai25}  % DO NOT CHANGE THIS
\usepackage{times}  % DO NOT CHANGE THIS
\usepackage{helvet}  % DO NOT CHANGE THIS
\usepackage{courier}  % DO NOT CHANGE THIS
\usepackage[hyphens]{url}  % DO NOT CHANGE THIS
\usepackage{graphicx} % DO NOT CHANGE THIS
\urlstyle{rm} % DO NOT CHANGE THIS
\def\UrlFont{\rm}  % DO NOT CHANGE THIS
\usepackage{natbib}  % DO NOT CHANGE THIS AND DO NOT ADD ANY OPTIONS TO IT
\usepackage{caption} % DO NOT CHANGE THIS AND DO NOT ADD ANY OPTIONS TO IT
\frenchspacing  % DO NOT CHANGE THIS
\setlength{\pdfpagewidth}{8.5in} % DO NOT CHANGE THIS
\setlength{\pdfpageheight}{11in} % DO NOT CHANGE THIS
%
% These are recommended to typeset algorithms but not required. See the subsubsection on algorithms. Remove them if you don't have algorithms in your paper.
 % For algorithmic environment

\usepackage{algorithm}
\usepackage{algorithmic}

%
% These are are recommended to typeset listings but not required. See the subsubsection on listing. Remove this block if you don't have listings in your paper.
\usepackage{newfloat}
\usepackage{listings}
\DeclareCaptionStyle{ruled}{labelfont=normalfont,labelsep=colon,strut=off} % DO NOT CHANGE THIS
\lstset{%
	basicstyle={\footnotesize\ttfamily},% footnotesize acceptable for monospace
	numbers=left,numberstyle=\footnotesize,xleftmargin=2em,% show line numbers, remove this entire line if you don't want the numbers.
	aboveskip=0pt,belowskip=0pt,%
	showstringspaces=false,tabsize=2,breaklines=true}
\floatstyle{ruled}
\newfloat{listing}{tb}{lst}{}
\floatname{listing}{Listing}
%
% Keep the \pdfinfo as shown here. There's no need
% for you to add the /Title and /Author tags.
\pdfinfo{
/TemplateVersion (2025.1)
}
\usepackage{multirow}
\usepackage{array}
\usepackage{graphicx} % Required for including images
\usepackage{subcaption} % Required for subfigures
% Adjusts the margins to allow more space
\usepackage{amsmath}  % For mathematical symbols and equations
\usepackage{amssymb}  % For additional symbols
%\usepackage{algpseudocode} % For algorithmic environment

\usepackage{pgfplots}
\pgfplotsset{compat=1.17}

% DISALLOWED PACKAGES
% \usepackage{authblk} -- This package is specifically forbidden
% \usepackage{balance} -- This package is specifically forbidden
% \usepackage{color (if used in text)
% \usepackage{CJK} -- This package is specifically forbidden
% \usepackage{float} -- This package is specifically forbidden
% \usepackage{flushend} -- This package is specifically forbidden
% \usepackage{fontenc} -- This package is specifically forbidden
% \usepackage{fullpage} -- This package is specifically forbidden
% \usepackage{geometry} -- This package is specifically forbidden
% \usepackage{grffile} -- This package is specifically forbidden
% \usepackage{hyperref} -- This package is specifically forbidden
% \usepackage{navigator} -- This package is specifically forbidden
% (or any other package that embeds links such as navigator or hyperref)
% \indentfirst} -- This package is specifically forbidden
% \layout} -- This package is specifically forbidden
% \multicol} -- This package is specifically forbidden
% \nameref} -- This package is specifically forbidden
% \usepackage{savetrees} -- This package is specifically forbidden
% \usepackage{setspace} -- This package is specifically forbidden
% \usepackage{stfloats} -- This package is specifically forbidden
% \usepackage{tabu} -- This package is specifically forbidden
% \usepackage{titlesec} -- This package is specifically forbidden
% \usepackage{tocbibind} -- This package is specifically forbidden
% \usepackage{ulem} -- This package is specifically forbidden
% \usepackage{wrapfig} -- This package is specifically forbidden
% DISALLOWED COMMANDS
% \nocopyright -- Your paper will not be published if you use this command
% \addtolength -- This command may not be used
% \balance -- This command may not be used
% \baselinestretch -- Your paper will not be published if you use this command
% \clearpage -- No page breaks of any kind may be used for the final version of your paper
% \columnsep -- This command may not be used
% \newpage -- No page breaks of any kind may be used for the final version of your paper
% \pagebreak -- No page breaks of any kind may be used for the final version of your paperr
% \pagestyle -- This command may not be used
% \tiny -- This is not an acceptable font size.
% \vspace{- -- No negative value may be used in proximity of a caption, figure, table, section, subsection, subsubsection, or reference
% \vskip{- -- No negative value may be used to alter spacing above or below a caption, figure, table, section, subsection, subsubsection, or reference
\captionsetup[figure]{skip=1pt} 

\setcounter{secnumdepth}{0} %May be changed to 1 or 2 if section numbers are desired.

% The file aaai25.sty is the style file for AAAI Press
% proceedings, working notes, and technical reports.
%

% Title

% Your title must be in mixed case, not sentence case.
% That means all verbs (including short verbs like be, is, using,and go),
% nouns, adverbs, adjectives should be capitalized, including both words in hyphenated terms, while
% articles, conjunctions, and prepositions are lower case unless they
% directly follow a colon or long dash
\title{DMPA: Model Poisoning Attacks on Decentralized Federated Learning for Model Differences}

\author {
    Chao Feng\textsuperscript{\rm 1}, 
    Yunlong Li\textsuperscript{\rm 1}, 
    Yuanzhe Gao\textsuperscript{\rm 1}, 
    Alberto Huertas Celdrán\textsuperscript{\rm 1}, 
    Jan von der Assen\textsuperscript{\rm 1},
    Gérôme Bovet\textsuperscript{\rm 2}, 
    Burkhard Stiller\textsuperscript{\rm 1}
}
\affiliations{
    \textsuperscript{\rm 1} Communication Systems Group, Department of Informatics, University of Zürich,\\
    Binzmühlestrasse 14, CH-8050 Zürich, Switzerland\\
    \textsuperscript{\rm 2} Cyber-Defence Campus, armasuisse Science \& Technology, CH-3602 Thun, Switzerland\\
    \{cfeng, huertas, vonderassen, stiller\}@ifi.uzh.ch, yuanzhe.gao@uzh.ch, yunlong.li@uzh.ch, gerome.bovet@armasuisse.ch
}

\iffalse
%Example, Multiple Authors, ->> remove \iffalse,\fi and place them surrounding AAAI title to use it
\title{My Publication Title --- Multiple Authors}
\author {
    % Authors
    First Author Name\textsuperscript{\rm 1},
    Second Author Name\textsuperscript{\rm 2},
    Third Author Name\textsuperscript{\rm 1}
}
\affiliations {
    % Affiliations
    \textsuperscript{\rm 1}Affiliation 1\\
    \textsuperscript{\rm 2}Affiliation 2\\
    firstAuthor@affiliation1.com, secondAuthor@affilation2.com, thirdAuthor@affiliation1.com
}
\fi


% REMOVE THIS: bibentry
% This is only needed to show inline citations in the guidelines document. You should not need it and can safely delete it.
\usepackage{bibentry}
% END REMOVE bibentry

\begin{document}

\maketitle

\begin{abstract}
% As a successful paradigm in the field of privacy protection, FL has received much attention in the field of model poisoning attacks. DFL abandons the structure of a single server in the traditional FL, improving the overall robustness and extensibility of the network. However, these features of DFL also open new doors for malicious participants to launch attacks. Currently, model poisoning attacks against FL are mostly focused on the traditional FL, while less research has been conducted on DFL. Even in existing DFL model poisoning attacks, the main strategy involves malicious participants only considering the effect of their own model parameters when making modifications. However, in DFL, the model parameters of each node are often different, which means that this attack method cannot effectively influence the performance of other benign models. In this work, a model poisoning attack method (DMPA) is proposed for DFL. This method calculates the differential characteristics of multiple malicious client models and obtains the most effective poisoning strategy that makes the effect of these models deteriorate together, thus realizing the collusive attack of multiple participants to offset the effect of inconsistent participant models in DFL. The effectiveness of this attack is also demonstrated in multiple datasets. The results show that the DMPA method generally outperforms the existing state-of-the-art FL model poisoning attack methods across various dimensions (with a reduction of xx-xx). 
Federated learning (FL) has garnered significant attention as a prominent privacy-preserving Machine Learning (ML) paradigm. Decentralized FL (DFL) eschews traditional FL's centralized server architecture, enhancing the system's robustness and scalability. However, these advantages of DFL also create new vulnerabilities for malicious participants to execute adversarial attacks, especially model poisoning attacks. In model poisoning attacks, malicious participants aim to diminish the performance of benign models by creating and disseminating the compromised model. Existing research on model poisoning attacks has predominantly concentrated on undermining global models within the Centralized FL (CFL) paradigm, while there needs to be more research in DFL. To fill the research gap, this paper proposes an innovative model poisoning attack called DMPA. This attack calculates the differential characteristics of multiple malicious client models and obtains the most effective poisoning strategy, thereby orchestrating a collusive attack by multiple participants. The effectiveness of this attack is validated across multiple datasets, with results indicating that the DMPA approach consistently surpasses existing state-of-the-art FL model poisoning attack strategies.

\end{abstract}

% Uncomment the following to link to your code, datasets, an extended version or similar.
%
% \begin{links}
%     \link{Code}{https://aaai.org/example/code}
%     \link{Datasets}{https://aaai.org/example/datasets}
%     \link{Extended version}{https://aaai.org/example/extended-version}
% \end{links}
\section{Introduction}
% \hspace{1em}In recent years, with the introduction of various privacy laws in different countries, the demand for privacy preservation is increasing. As an emerging multi-participant collaborative machine learning method, federated learning has attracted the attention of more and more researchers because its participants (or clients) do not need to share original data, but share model information, which helps to protect the privacy of local data\cite{yang2019federated}. The traditional Federated Learning architecture relies on a central server to send, receive, and aggregate model parameters from each participant, and is therefore called Centralized Federated Learning (CFL)\cite{alazab2021federated}. This client-server network architecture, while easy to implement, has significant drawbacks, including vulnerability to a single point of failure, and is not conducive to further optimizing the performance of federated learning. To overcome these limitations, Decentralized Federated Learning (DFL) came into being and has gradually gained wide attention and acceptance.

The advent of diverse privacy regulations has significantly increased the need for privacy preservation in Machine Learning (ML). Federated Learning (FL) emerges as a novel collaborative and privacy-preserving ML paradigm that has attracted substantial attention from both academic and industrial communities~\cite{mcmahan2017communication}. FL enables participants to share model parameters instead of raw data, thereby ensuring data privacy \cite{yang2019federated}. Traditional FL architectures rely on a central server to distribute, called Centralized FL (CFL), receive and aggregate model parameters from participants \cite{alazab2021federated}. However, this client-server architecture suffers from significant drawbacks, including susceptibility to a single point of failure and server-side bottleneck~\cite{Mart_nez_Beltr_n_2023}. To address these challenges, Decentralized FL (DFL) has been introduced.



% In DFL, the entire network is completely autonomous, and each client is independently responsible for establishing communication, passing, and aggregating model parameters with other clients \cite{lian2017can}. Each participant completes the following tasks in each round: training the local model, passing the model parameters to the directly connected clients, aggregating the received model parameters, and updating the local model accordingly. Different from CFL, each client in DFL can independently select any other participant to establish a two-way communication connection, so the topology of the overall network can be customized, such as fully connected, ring, star, etc., and even dynamic topology \cite{yuan2024decentralized}. This flexibility allows DFL to effectively mitigate the single point of failure problem in CFL and improve the robustness and scalability of the system. However, this decentralized nature of DFL also brings new security risks, making the network more vulnerable to various threats, such as poisoning attacks.

In DFL, each client independently manages the processes of establishing communication, sharing, and aggregating model parameters with other clients \cite{lian2017can}. During each iteration, participants execute several tasks: training their local models, transmitting model parameters to directly connected peers, aggregating received parameters, and updating their local models accordingly. Unlike CFL, clients in DFL have the autonomy to independently select any other participant to establish bidirectional communication connections, allowing for customizable overlay network topology, such as fully connected, ring, star, and even dynamic configurations \cite{yuan2024decentralized}. This adaptability enables DFL to effectively address the single point of failure issue inherent in CFL, thereby enhancing the system's robustness and scalability. Nevertheless, this decentralized nature introduces new security vulnerabilities, making DFL more susceptible to various threats, including model poisoning attacks.

Model poisoning attacks involve malicious clients directly altering their local model parameters to negatively influence the global model training by sending harmful updates \cite{feng2023sentinelaggregationfunctionsecure}. In FL, since participants can directly share and modify model parameters, model poisoning attacks are relatively more straightforward to execute and pose significant threats. Current model poisoning attacks primarily target CFL models. These adversarial clients transmit meticulously crafted local model updates to the central server during the training phase, leading to issues such as reduced accuracy of the global model \cite{cao2022mpaf}. These attacks presuppose that the attacker controls a substantial number of compromised real clients, which may include either hijacked clients or fabricated ones.

% Currently, the research on security issues for Federated Learning (FL) systems is very extensive and covers a variety of attack methods, among which Poisoning Attack is an important research direction. Poisoning attacks are mainly categorized into Data Poisoning and Model Poisoning. Data Poisoning Attacks usually refer to malicious clients disrupting the training process by injecting malicious data (e.g., backdoor data or labeling noise) into the local training set, which leads to biased learning of specific data by the model and ultimately misleads the overall model learning results of FL\cite{hallaji2022federated}. In contrast, model poisoning attacks involve malicious clients directly tampering with their local model parameters and influencing the global model training through malicious model updates. Because in federated learning, participants can directly share and modify model parameters, model poisoning attacks tend to be easier to implement and more threatening.


% Existing model poisoning attacks (MPA) mainly target centralized Federated Learning (CFL) models. The attack assumes that the attacker has access to a large number of compromised real clients, which could be legitimate clients that have been hijacked or fake clients. During the training process, these clients send carefully designed local model updates to the cloud server (i.e., the central server), resulting in problems such as a decline in the accuracy of the global model \cite{cao2022mpaf}. Most existing model poisoning attack method rely on tuning hyperparameters to amplify the effectiveness of malicious models. For example, the MIN-MAX method \cite{shejwalkar2021manipulating} generates a new malicious model by calculating the distance difference between the models. However, because these modifications are relatively drastic, they will significantly change the distribution of model weights, making these attacks easy to be detected and excluded by model aggregation methods \cite{pillutla2022robust}. However, if the gradient empirical variance between clients is high enough, an attacker can exploit this to launch a non-omniscient attack that operates within the range of the population variance. This attack uses mean and variance as the basis for model modification to carry out more covert model poisoning attacks \cite{baruch2019little} and gradually reduce the accuracy of federated learning. In addition, attackers can also hijack the network transmission process and modify it without obtaining a large amount of local data, which becomes a fast and effective attack mode, and greatly threatens the security of federated learning systems.

% In decentralized federated learning (DFL), due to poor network connectivity and long communication links, it is difficult for malicious participant to have a significant impact on the global network even if they carry out high-intensity attacks\cite{Mart_nez_Beltr_n_2023}. This makes DFL system more difficult to attack when facing model poisoning attacks. However, due to its nature, model poisoning attacks can still modify the model by hijacking the content of the communication. Such modifications based on model parameters or gradients are more common in reality than attack methods that require direct access to local data, and can lead to security issues in applications such as healthcare, intrusion detection, and autonomous vehicles\cite{issa2023blockchain}.

Nevertheless, the architectural distinctions between DFL and CFL, particularly in overlay network topology, imply that existing attacks targeting CFL can not be directly transferable to DFL. In DFL, the communication links are significantly longer compared to CFL, making it challenging for a malicious participant to impact the entire federation \cite{Mart_nez_Beltr_n_2023}. Additionally, in DFL, each node could decide whether to disseminate and receive models from neighboring nodes. This capability enables the potential blockage of malicious attacks at intermediate nodes, thereby diminishing the effectiveness of such attacks. Consequently, from an adversarial perspective, it is imperative to develop model poisoning attacks that are effective in DFL.

% As DFL gradually becomes a universal solution for privacy protection, it is of great practical significance to find an MPA that can be effectively implemented in a DFL environment. At present, the research on model poisoning attack of DFL is relatively scarce, which makes the security of DFL in a wide range of applications unknown risks. Existing DFL model poisoning attacks usually use Gaussian distribution of additive noise or other methods to modify model parameters at will. However, these modifications are often detected and excluded when the model is aggregated, resulting in a less effective attack. Current FL poisoning attack methods are mainly implemented by adjusting the Euclidian distance or Angle of the model, but these methods do not perform well in the network structure of DFL. Since malicious clients in DFL may not be directly connected to other clients, when calculating model distance or similarity, these methods may cause the features of the malicious model to be cleared, leaving only Euclidean distance or similarity information. The detection mechanism of some aggregation methods (such as Krum\cite{blanchard2017machine}) makes it possible that when benign clients outnumber malicious clients, the model parameters of the malicious clients may be excluded from the beginning and it is difficult to propagate to the entire network. Even so, the benign client continues to perform model updates for gradient descent. As a result, it is difficult for MPA on DFL to have a significant impact on benign clients which is not directly connected to malicious clients, which further increases the difficulty of attacking on DFL. Therefore, the current MPA research on DFL environment faces the following challenges: 1. How to keep the characteristics of malicious clients similar to those of other benign clients so that the numerical characteristics of model parameters in each round are stable without losing key information; 2. How to make the malicious client's model modification not excluded by the aggregation method, so as to continue to be accepted in the next round.

To this end, this paper investigates model poisoning attacks within DFL  and proposes the Decentralized Model Poisoning Attack (DMPA) based on an Angle Bias Vector to rectify reversed model parameters. Specifically, by determining the eigenvalues of the compromised benign parameters, the method extracts the corresponding eigenvectors to compute the angular deviation vector. This vector is derived by exploiting the discrepancies between models and ultimately adjusting the negative parameters, thereby generating an effective attack model across the majority of participants. The main contributions of this paper are: \textit{(i)} design and implementation of a model poisoning attack, called DMPA, to compromise the model robustness of DFL models; \textit{(ii)} an extensive series of experiments are conducted to evaluate the proposed DMPA and compared with selected state-of-the-art attack algorithms under diverse robustness aggregation functions, which encompass three benchmark datasets (MNIST, Fashion-MNIST, and CIFAR-10) and three varying overlay topologies (fully connected, star, and ring); and \textit{(iii)} critical insights into the robustness of the DFL model concerning both offensive and defensive points of view. The experiments demonstrate that the DMPA presented in this work exhibits a more potent attack capability and broader propagation potential, effectively compromising the DFL system.

The remainder of this paper is structured as follows. Section 2 provides an overview of the underlying security issues in DFL and discusses the concept of model poisoning attacks. Subsequently, Section 3 formalizes the problem. Section 4 delineates the design specifics of DMPA, the proposed model poisoning attack. Section 5 then evaluates DMPA's performance in comparison to selected related works. Finally, Section 6 offers a summary of the contributions made by this research and outlines potential avenues for future investigation.

% To this end, this paper investigates Model Poisoning attacks targeting decentralized Federated Learning (DFL), and proposes a Model Poisoning Attack method (DMPA) based on Angle bias vector to correct reverse model parameters. Specifically, by calculating the eigenvalues of the hijacked benign parameters, the method extracts the corresponding eigenvectors to calculate the angular deviation vector. This angular deviation vector is computed by exploiting the differences between the models and ultimately correcting the effects of the negative parameters to generate an attack model that is valid for most of the participants. The algorithm proposed in this paper shows effective attack capability in a wide range of topologies and significantly reduces the F1 score after aggregation, thus validating the effectiveness of the method.\\
% In summary, the main contributions of this paper are as follows:
% \begin{itemize}
%     \item A framework for implementing model poisoning attacks in DFL environment is proposed, which launches attacks after local training is completed, thus destroying the training process and final results of DFL.
%     \item A method was proposed to modify the negative model parameters by calculating the eigenvector projection corresponding to the maximum eigenvalue of the correlation matrix. The attack was carried out while retaining the model features, which solved the problem of feature disappearing after multiple rounds of training and enhanced the persistence of the attack.
%     \item On three publicly available benchmark datasets and three common DFL topologies, this paper evaluates the performance and effectiveness of three state-of-the-art model poisoning attack methods in DFL. The experimental results show that the proposed attack strategy solves the above problems, and a large number of experimental results prove that the DMPA method in this paper has stronger attack effect and wider propagation ability, and can effectively destroy the DFL system, which verifies the rationality and effectiveness of the proposed method.
% \end{itemize}



\section{Background and Related Work}
This section provides an introduction to the security problem in DFL and a summary of the current research on model poisoning attacks in FL.
\subsection{Security Problem in DFL}
Differing from traditional CFL, neighboring clients in DFL exchange local model parameters or gradients over a P2P network and construct consensus models independently. DFL solves the problem of risk associated with a single point of failure, enhances its scalability, and is well suited for applications in the Industrial Internet of Things (IIoT) \cite{tan2023collusive}. However, the decentralized nature of DFLs rather increases their risk of being exposed to malicious attacks, especially poisoning attacks~\cite{feng2023voyager}. Poisoning attacks, which aim to diminish the resilience of FL models, can be classified into data poisoning and model poisoning attacks. Data poisoning attacks typically involve malicious clients interfering with the training process by introducing harmful data (such as backdoor attacks or label flipping) into the local training dataset, thereby inducing biased learning outcomes \cite{hallaji2022federated}. Conversely, model poisoning attacks are characterized by malicious clients directly altering their local model parameters and affecting the global model training via harmful model updates. To improve the model robustness of FL and defend against the poisoning attacks, \cite{mcmahan2017communication} proposed the use of averaging model parameters to improve training efficiency. \cite{blanchard2017machine} introduced the Krum method, which excludes malicious updates by calculating the Euclidean distance of a multi-node model and selecting the model with the smallest distance. This method reduces the impact of malicious attacks and is now widely used for robust aggregation in DFL.
\cite{yin2018byzantine} developed two robust distributed learning algorithms for Byzantine errors and potential adversarial behavior, a robust distributed gradient descent algorithm based on Median and Trimmed Mean operations, respectively. These two methods are widely used in both CFL and DFL because they can be implemented with only one communication round and achieve optimal model robustness.

\subsection{Model Poisoning Attack}
Compared with data poisoning attacks, model poisoning attacks are easier to execute as they directly modify the shared model. Therefore, there has been research suggesting how to optimize the attack method to enhance the attack effectiveness. Existing model poisoning attacks mainly target CFL. The attack assumes that the attacker has access to a large number of compromised clients. During the training process, malicious clients send carefully designed local model updates to the server, resulting in problems such as a decline in the accuracy of the global model \cite{cao2022mpaf}. The MIN-MAX and MIN-SUM methods proposed by \cite{shejwalkar2021manipulating} respectively minimize the maximum and sum of the distance between toxic updates and all benign updates in CFL. These methods assume that malicious model updates are the sum of aggregated model updates and a fixed disturbance vector scaled by factors before the attack, and implement the attack by maximizing the distance between toxic updates and benign updates in the inverse direction of the global model. 
%These methods show remarkable results in traditional federated learning, and because they do not rely on training deep learning models, they require fewer computational resources, making them useful in FL's model poisoning attacks.
The "a little is enough" (LIE) attack \cite{baruch2019little} directly modifies the model parameters, which generates malicious model updates in each training round by calculating the average of the real model updates of the malicious client and perturbing them. %In a DFL system, it is assumed that an attacker has a set of malicious client models, and although these malicious clients may not be directly connected, they can still use these participants to implement model poisoning attacks. Therefore, this paper will try to apply the above methods to DFL system and study its attack logic to help explore the model poisoning attack in DFL.

Most model poisoning attacks rely on additional information from the central server, such as aggregation methods, global models, or even the training data utilized by nodes. This dependency renders external attacks impractical. Nevertheless, some attacks are engineered to maintain efficacy even in the absence of such additional knowledge. For instance, \cite{zhang2023denial} employs global historical data to build an estimator that forecasts the subsequent round of global models as a benign reference. Despite this, the approach necessitates substantial resources and encounters difficulties in accessing the global model within DFL systems, thereby limiting its practical application in DFL.

To conclude, while a substantial amount of research has been dedicated to optimizing model poisoning attacks against the FL, these efforts predominantly concentrate on the CFL paradigm, with minimal attention given to the DFL paradigm. Furthermore, the current attack methodologies exhibit limitations, such as inconsistent efficacy and susceptibility to detection by defensive mechanisms. To address these research gaps, this paper presents the design and implementation of a novel attack strategy tailored for DFL. This proposed attack demonstrates broad effectiveness across various datasets, diverse ML model architectures, and multiple types of DFL overlay topologies.


\section{Problem Setup}
\subsection{DFL Training Process}
\hspace{1em}In the DFL framework, consider a network of \( K \) clients, each denoted as \( k \in \{1, 2, \dots, K\} \). In each communication round, each client \( k \) maintains and updates a local model \( w_t^k \) according to an aggregation algorithm \( A(\cdot) \) defined by this framework. The communication between clients can be represented as a graph \( G = (\mathcal{V}, \mathcal{E}) \), where \( \mathcal{V} \) is the set of nodes (clients) and \( \mathcal{E} \) is the set of edges (communication links) between the nodes. The neighborhood of a client \( k \), denoted as \( \mathcal{N}_k \subseteq \mathcal{V} \), contains all clients that are directly connected to \( k \) via an edge in \( \mathcal{E} \). Each client updates its model by training it locally and aggregating models from its neighbors. The network topology, represented by \( G \), plays a key role in determining how information is shared and how global knowledge is propagated through the network.

A general process of DFL can be divided into the following stages: First, each client \( k \) initializes its local model parameters \( w_0^k \). Then, each client trains its local model to further optimize these parameters. Gradient descent is generally used as the core optimization strategy during local training. Specifically, client \( k \) performs several epochs of local training using its private dataset \( P_k \) to minimize its local loss function \( \mathcal{L}(w, P_k) \) as defined in  Equation \ref{eq:weight_update}.

\begin{equation}
w_{t}^k \leftarrow w_t^k - \eta \nabla \mathcal{L}(w_t^k, P_k)
\label{eq:weight_update}
\end{equation}

where the gradient \( \nabla \ell(w_t^k, P_k) \) reflects the change in the loss function \( \ell \) with respect to the model parameters \( w_t^k \). The learning rate \( \eta \) controls the step size of the parameter updates at each iteration. By iteratively reducing the loss function, gradient descent effectively guides the model parameters toward a more optimal direction.

After each client \( k \) completes the training of its local model, it interacts with its neighbors \( \mathcal{N}_k \) and transmits the parameters of the current round of local model training to them, as defined by the network topology \( G \). Simultaneously, client \( k \) also receives model updates \( w_t^j \) from its neighbors \( j \in \mathcal{N}_k \) and follows an aggregation algorithm \( A(\cdot) \) to form a new model \( w_{t+1}^k \). The aggeration process can be defined in  Equation \ref{eq:aggreagation_update}.

\begin{equation}
w_{t+1}^k = A\left(w_t^k, \{w_t^j : j \in \mathcal{N}_k\}\right)
\label{eq:aggreagation_update}
\end{equation}

By continuously repeating the above processes of local model training and aggregation until the predefined number of rounds is completed, DFL effectively achieves a distributed and collaborative model optimization across the entire network, enhancing both the robustness and privacy of the learning process.

Model poisoning attacks typically cause significant degradation in global model performance by maliciously manipulating model parameters. In the DFL framework, model poisoning attacks typically occur after the client has completed local model training. A malicious client deliberately tampers with its model parameters after training is complete and then sends these tainted parameters to its neighbors and participates in the model aggregation process. In this way, malicious clients are able to gradually contaminate the models of the entire federated learning system, ultimately leading to the degradation of the performance of the global model. The process of the DFL framework with malicious participants is illustrated in Figure\ref{fig:figure2}.

\begin{figure}[htbp]
    \centering
    \includegraphics[width=\columnwidth]{Figure_2.pdf} 
    \caption{DFL Process with Malicious Participants}
    \label{fig:figure2}
\end{figure}
%\vspace{-20pt}

\subsection{Threat Model}
\hspace{1em}\textbf{Adversary Objective.}
The primary goal of the attacker is to minimize the performance of the global model in the DFL system by manipulating the updates to the client models under its control. Specifically, the attacker hopes to mislead the learning process of the global model by introducing carefully crafted malicious updates to the model parameters, ultimately leading to significant degradation in model accuracy or deviation in behavior.

\textbf{Adversary Capability.} 
The attacker has the ability to control a certain percentage of clients and has full access to and modify the local model updates of these clients. The attacker is able to send maliciously modified model parameters to neighboring nodes and participate in model aggregation after each round of training. In addition, the attacker can continuously observe and adjust the attack strategy during the training process to improve the stealth and effectiveness of the attack.

\textbf{Adversary Knowledge.} 
The attacker only has model information of all malicious participants and has no knowledge of the internal information of other benign clients. This knowledge level assumption is very strict, but it is consistent with the conditions of a realistic attack, which reflects the harmfulness of the designed model poisoning attack discussed in next section.


\section{DMPA Approach}
In this section, the general framework for applying the DMPA method in DFL is described, followed by a discussion of the research issues concerning Model Poisoning Attacks in DFL. The section then presents solutions to the problems outlined in the first section.
\subsection{Method Overview}
% \hspace{1em}To successfully improve the effect of non-targeted attacks, DMPA's algorithm is a method to increase the loss of gradient descent and retain the numerical features of the model through feature calculation, thus affecting Benign Clients in DFL communication transmission.To damage the effect of Benign Clients' model prediction, this paper designs malicious model parameters in DFL, which solve the problems proposed by Section 1. Therefore, DMPA is proposed as a model poisoning attack applied in decentralized federated learning. It focuses on the differences in the numerical characteristics of malicious client models and leverages these differences to increase the loss of models, thereby using malicious clients to impact the benign clients. This method is aimed at the optimization attack of model gradient descent, so the main purpose of this method is to study how to extract features through different model parameter differences, to improve the loss of the client model. The DMPA method is shown in Figure \ref{fig:figureDMPA}. The method in this paper is different from other methods of model poisoning attacks in FL (\cite{shejwalkar2021manipulating},\cite{baruch2019little},\cite{zhang2023denial}). They aim at the model similarity (cosine similarity, Euclidean distance, etc.) calculated between malicious clients to generate model vectors similar in model similarity. 

This workd proposes DMPA, an advanced model poisoning attack specifically designed to target DFL models. As illustrated in Figure \ref{fig:figureDMPA}, DMPA employs a tripartite attack strategy. In contrast to existing model poisoning attacks in FL (e.g., \cite{shejwalkar2021manipulating, baruch2019little, zhang2023denial}), which determine the attack direction based on model similarity, DMPA leverages the maximum eigenvalue of the correlation matrix to identify the optimal poisoning direction. The primary objective is to maximize the training loss of benign models after model aggregation.


\begin{figure}[htbp]
    \centering
    \includegraphics[width=\columnwidth]{DMPA.pdf} 
    \caption{Overview of DMPA Attack Process}
    \label{fig:figureDMPA}
\end{figure}


% As shown in Figure \ref{fig:figureDMPA} , This work includes the modification calculation of model update, and the framework of attack before participants share their models with other linked clients, in order to increase the loss of benign clients every round. Specifically, this paper will follow the math in the next section to make local malicious updates.

% DMPA is a method of adding the projection of the gradient vector corresponding to the maximum eigenvalue of the correlation matrix extracted from each malicious model with the parameters of the inverse model so that it can improve the loss in gradient descent. The independent weight calculated when the correlation matrix is extracted from the model parameters representatives the influence degree of the model on gradient descent. At the final output, DMPA calculates the position of the largest value in the top 10\% of the original model vector. In order to prevent and effectively integrate attacks, DMPA has taken special measures to modify these locations. Specifically, uneven values are extracted from the modified vector and filled into the corresponding positions of the averaged vector, thereby maximizing the attack effect.

% Therefore, there are three steps for DMPA to realize the attack: (1) Find the correlation matrix according to the matrix difference,  and calculate the eigenvector of the maximum eigenvalue of the correlation matrix. Using this eigenvector to project on the model parameters when it is Not attacked to get the angular deviation; (2) Add this deviation with the reverse model parameters and average the modified model parameters; (3) Extract uneven values from the modified vector and use them to fill the corresponding positions in the average vector, thus maximizing the attack effect. Averaging the back vector and filling the original modified vector can prevent the attack effect from decreasing after the model is averaged,  and malicious users upload malicious parameters to their connected benign clients for aggregation. Algorithm 1 provides the attack pseudocode of DMPA,  which explains how malicious clients use DFL attack to decentralize the federated learning system.
% The other parts of this chapter will describe the details of DMPA theory.


As illustrated in Figure \ref{fig:figureDMPA}, DMPA is conducted the attack prior to the exchange of models among interconnected clients. There are three steps for DMPA to achieve the attack: (1) Find the correlation matrix according to the difference of the model parameter matrix, and calculate the eigenvectors of the maximum eigenvalue of the correlation matrix. Use this eigenvector to project on the model parameters to obtain the angular deviation; (2) Add this deviation with the inverse model parameters and the average adjusted model parameters; (3) Extract the uneven values from the average vector and use them to fill the corresponding positions in the average vector, thus maximizing the attack effect. Averaging the posterior vector and filling the original modified vector can prevent the attack effect from decreasing after the model is averaged, and malicious users upload malicious parameters to their connected benign clients for aggregation.



\begin{algorithm}[tb]
\caption{DMPA Attack Strategy}
\label{alg:algorithmDMPA}
\textbf{Input}: A matrix $\mathbf{U} \in \mathbb{R}^{d \times n}$ representing all updates, where each column $\mathbf{u}_i$ is a different update vector. \\
\textbf{Output}: The modified updates matrix $\mathbf{\mu}_{\text{new}}$. \\
\begin{algorithmic}[1] % The [1] here enables line numbering
\STATE Compute the mean vector $\mathbf{\mu} = \frac{1}{n} \sum_{j=1}^{n} \mathbf{u}_{i,j,:}$
\STATE Center the updates: $\mathbf{V} = \mathbf{U} - \mathbf{\mu}$
\STATE Compute the covariance matrix: $\mathbf{C} = \frac{1}{n-1} \mathbf{V} \mathbf{V}^\top$
\STATE Compute the standard deviation: $\mathbf{T} = \sqrt{\text{diag}(\mathbf{C})}$
\STATE Compute the correlation matrix: $\mathbf{Y} = \frac{\mathbf{C}}{\mathbf{T} \mathbf{T}^\top}$ \\
\hspace{1em} \textit{where} $\oslash$ \textit{denotes element-wise division}
\STATE Compute eigenvalues and eigenvectors of $\mathbf{Y}$: $(\lambda, \mathbf{V}) = \text{eig}(\mathbf{Y})$
\STATE Find the principal eigenvalue: $\lambda_{\max} = \max(\lambda)$ and corresponding eigenvector $\mathbf{y}_{\max}$
\STATE Compute the projection: $\mathbf{P} = (\mathbf{y}_{\max}^\top \mathbf{U}) \cdot \mathbf{y}_{\max}$
\STATE Modify the updates: $\mathbf{U}^{\text{new}}  = -\mathbf{U} + \mathbf{P}$
\STATE Compute the new mean vector: $\mathbf{\mu}^{\text{new}} = \frac{1}{d} \sum_{j=1}^{d} \mathbf{U}_{j,:}$

\FOR{$i = 1$ to $d$}
    \STATE Compute mask $\mathbf{m}_i = \text{select\_top\_k\_params}(\mathbf{u}_i^2, 10)$
    \STATE Update mean vector: $\mathbf{\mu}^{\text{new}} = \mathbf{m}_i \odot \mathbf{U}_i^{\text{new}} + (1 - \mathbf{m}_i) \odot \mathbf{\mu}^{\text{new}}$
\ENDFOR
%1.9修改后的向量比较可视化,cos,那几个一起比较一下,目的是证明我们这个修改是好的。
%2.补充和sota的比较的说明,好的比较显示为啥会有这样的效果,做一个可视化或者一些分析,;坏的用比较为啥会没有效果。
%3.针对DFL的分析,在稀疏网络的表现,讨论更加详细。解释为什么,可以用总结分析一下。找到规律。全链接和star是稀疏。看看能不能找到为啥。
%4.攻击的计算
\STATE \textbf{return} $\mathbf{\mu}^{\text{new}}$

\STATE \textbf{Function} $\text{select\_top\_k\_params}(\mathbf{vector}, k\%)$:
\STATE \hspace{1em}$\text{sorted\_indices} = \text{argsort}(\mathbf{vector}, \text{descending=True})$
\STATE \hspace{1em} Compute $k = \left\lfloor \frac{\text{length of } \text{sorted\_indices} \times k\%}{100} \right\rfloor$
\STATE \hspace{1em} Initialize mask $\mathbf{m} = \mathbf{0}$ (same shape as $\mathbf{vector}$)
\STATE \hspace{1em} Set top $k$ indices: $\mathbf{m}[\text{sorted\_indices}_{:k}] = 1$
\STATE \hspace{1em} \textbf{return} $\mathbf{m}$
\end{algorithmic}
\end{algorithm}

\subsection{Attack Strategy}
Algorithm 1 provides the attack pseudocode of DMPA,  which explains how malicious clients execute the DPMA attack in DFL. This work defines the model of the received malicious client as $\mathbf{U}$. All malicious client models are transformed into column vectors, represented as $\mathbf{U} = [u_1, u_2, \ldots]$. During the attack phase, a decentralized approach is employed, leading to the computation of a new deviation, denoted as $v_{ij}$, using Equation \ref{eq:pingjun}. These resulting relative offsets collectively constitute the matrix $\mathbf{V}$.

\begin{equation}
    v_{ij} = u_{ij} - \bar{u}_j
    \label{eq:pingjun}
\end{equation}
where $\bar{u}_j$ is the average of the $j$ th model parameter.

To ensure the rigor of this analysis, it is essential to compute the overall standard deviation and provide an unbiased estimate. This necessitates the derivation of a correlation matrix. As demonstrated in Equation \ref{eq:duli}, the correlation matrix $\mathbf{V}$ is determined using the matrix $\mathbf{C}$.

\begin{equation}
    \mathbf{C} = \frac{1}{n-1} \mathbf{V}^T \mathbf{V}
    \label{eq:duli}
\end{equation}

The matrix $\mathbf{C}$ can be regarded as a covariance matrix, in which each element represents the correlation between subscript corresponding vector groups. However, in this paper, when finding out the weight of each model for gradient rise, it is necessary to get the weight of each model for gradient rise under independent assumptions. Therefore, the diagonal elements of the matrix $\mathbf{C}$ are extracted and the square root is found. After dimension reduction, the value can reflect the independent weight vector of each model $\mathbf{T}$.

\begin{equation}
\mathbf{Y} = \frac{\mathbf{C}}{\mathbf{T} \mathbf{T}^\top}
\label{eq:zbtouying}
\end{equation}


To obtain values characterized by independent weight and correlation deviation, Equation \ref{eq:zbtouying} computes the outer product $\mathbf{Y}$ by multiplying the correlation matrix with the independent weight vector. Here, $\mathbf{Y}$ is derived through a linear combination of correlation deviation and independent weight. This vector provides a more accurate representation of the system weight for each parameter column.

\begin{equation}
\mathbf{P} = \left( \mathbf{y}_{\max}^T \mathbf{U} \right)  \cdot \mathbf{y}_{\max}
\label{eq:touying}
\end{equation}

\begin{equation}
\mathbf{\mu}^{\text{new}} = - \mathbf{U} +\mathbf{P}
\label{eq:output}
\end{equation}

Upon obtaining the independent weight, the eigenvector $\mathbf{y}_{\max}$ associated with the maximum eigenvalue of the weight is computed. This eigenvector is then projected onto the original vector to determine the angular deviation, as described by Equation \ref{eq:touying}. Subsequently, the original model parameters are adjusted by incorporating the angular deviation, resulting in the final update, as delineated by Equation \ref{eq:output}.

\section{Evaluation}
\subsection{Experimental Setup}

\textbf{Datasets.} The experiments use CIFAR-10 \cite{krizhevsky2009learning}, MNIST \cite{deng2012mnist} and Fashion-MNIST \cite{xiao2017fashion} as validation datasets, which are widely used in validating model performance. The experiments adopt an independent identically distributed (IID) data partitioning method, which ensures that the data have the same statistical properties. The $\alpha$ parameter is defaulted to 100 as set in (\cite{shejwalkar2021manipulating}, \cite{tan2023collusive}, \cite{li2024fedimp}).

\textbf{Machine learning models.} Three different model architectures are chosen to correspond to different datasets. The batch size is set to 64 for all clients, and the random seed for each client is set to its corresponding ID value. A simple convolutional neural network (CNN) model with Conv2d, BatchNorm2d, 5 Depthwise Conv2d, and fully connected (linear) layers is used for the CIFAR-10 dataset, a model with three fully connected (linear) layers is used for the MNIST dataset, and a simple CNN model with 2 Conv2d and 2 fully connected layers is used for the Fashion-MNIST dataset.

\textbf{Measurement metrics.} The model F1 score is used to assess the attack performance. A lower F1 score indicates a better attack method. All results are the average of the F1 scores of all benign client models after last round of training, which enables a clear assessment of the impact of the MPA attack on the clients in the whole DFL network.

\textbf{Baseline defenses and attacks.} Three of the most effective model poisoning attacks in FL are chosen to attack DFL, namely Lie \cite{baruch2019little}, Min-Sum and Min-Max \cite{shejwalkar2021manipulating}. In DFL, the modifications of Lie, Min-Max and Min-Sum assume that the malicious clients are colluding, and thus these models can be shared among malicious clients \cite{li2023plato}. In addition, four defence methods are chosen; FedAvg, Krum, Trimmed Mean and Median. it is assumed that the malicious client has no knowledge of the benign model and only knows the models of all malicious clients in the current round

\begin{itemize}
    \item \textbf{Lie}: Updates the weights by subtracting the product of the standard deviation of the malicious parameters and the calculated perturbation range from the mean weights\cite{baruch2019little}.
    \item \textbf{Min-Max}: Computes the malicious gradient such that its maximum distance from any other gradient is upper bounded by the maximum distance between any two benign gradients\cite{shejwalkar2021manipulating}.
    \item \textbf{Min-Sum}: Ensures that the sum of squared distances between the malicious gradient and all benign gradients is upper bounded by the sum of squared distances between any benign gradient and the other benign gradients\cite{shejwalkar2021manipulating}.
\end{itemize}




\begin{table*}[ht]
\centering
\caption{Average F1 Score of Benign Clients in DFL with a Configuration Consisting of 40\% Malicious Clients for the MNIST, Fashion-MNIST, and CIFAR-10 Datasets in Fully-Connected, Ring, and Star Topologies.}
\resizebox{\textwidth}{!}{
\begin{tabular}{|c|c|cccc|cccc|cccc|}
\hline
\multirow{3}{*}{Dataset}      & \multirow{3}{*}{Aggregation Rule} & \multicolumn{4}{c|}{Fully}                                            & \multicolumn{4}{c|}{Ring}                                             & \multicolumn{4}{c|}{Star}                                    \\ \cline{3-14} 
                              &                                   & LIE               & Min-Max           & Min-Sum  & DMPA              & LIE               & Min-Max           & Min-Sum  & DMPA              & LIE               & Min-Max  & Min-Sum  & DMPA              \\ \hline
\multirow{4}{*}{MNIST}        & Median                            & 0.919523          & \textbf{0.828144} & 0.861094 & 0.847919          & 0.864083          & 0.727043          & 0.820756 & \textbf{0.691755} & 0.820844           & 0.821185 & 0.830490  & \textbf{0.820525} \\ \cline{2-14} 
                              & Trimmed mean                       & 0.827509          & 0.825695          & 0.843681 & \textbf{0.801535} & 0.864083          & 0.727043          & 0.820756 & \textbf{0.691755} & 0.822594          & 0.819780  & 0.825661 & \textbf{0.813209} \\ \cline{2-14} 
                              & Krum                              & 0.874433          & 0.811460           & 0.885670  & \textbf{0.013312} & 0.873979          & 0.867974          & 0.840262 & \textbf{0.740244} & 0.819409          & 0.812599 & 0.822512 & \textbf{0.582298} \\ \cline{2-14} 
                              & Fed\_avg                          & 0.825292          & 0.826278          & 0.834523 & \textbf{0.802228} & 0.833553          & \textbf{0.64567}  & 0.827210  & 0.672756          & 0.821639          & 0.820869 & 0.825211 & \textbf{0.815387} \\ \hline
\multirow{4}{*}{CIFAR-10}    & Median                            & 0.410982          & 0.679403          & 0.706879 & \textbf{0.044529} & \textbf{0.280601} & 0.421047          & 0.446616 & 0.314470           & 0.548596          & 0.665249 & 0.652866 & \textbf{0.490103} \\ \cline{2-14} 
                              & Trimmed mean                       & 0.652813          & 0.734104          & 0.732013 & \textbf{0.016994} & \textbf{0.258510} & 0.400051          & 0.395156 & 0.305964          & 0.596924          & 0.636560  & 0.658856 & \textbf{0.193741} \\ \cline{2-14} 
                              & Krum                              & \textbf{0.230500} & 0.625591          & 0.625531 & 0.569555          & \textbf{0.372232}   & 0.545091          & 0.646702 & 0.433777          & \textbf{0.538895} & 0.600382 & 0.594952 & 0.569471          \\ \cline{2-14} 
                              & Fed\_avg                          & 0.674934          & 0.740961          & 0.737501 & \textbf{0.016994} & 0.346825          & 0.440712          & 0.449766 & \textbf{0.076015} & 0.617962          & 0.640634 & 0.652659 & \textbf{0.213768} \\ \hline
\multirow{4}{*}{Fashion-MNIST} & Median                            & 0.887029          & 0.892688          & 0.898451 & \textbf{0.881820}  & 0.843861          & 0.800340           & 0.885512 & \textbf{0.773666} & 0.883854          & 0.878469 & 0.887852 & \textbf{0.878103} \\ \cline{2-14} 
                              & Trimmed mean                       & 0.893127          & 0.885061          & 0.903730  & \textbf{0.863626} & 0.837478          & 0.800340           & 0.885512 & \textbf{0.773666} & 0.887656          & 0.878575 & 0.898533 & \textbf{0.874800}   \\ \cline{2-14} 
                              & Krum                              & 0.875101          & 0.862932          & 0.896126 & \textbf{0.058431} & 0.806988           & \textbf{0.453659} & 0.879798 & 0.767491          & 0.860227          & 0.864078 & 0.884784 & \textbf{0.597608} \\ \cline{2-14} 
                              & Fed\_avg                          & 0.895660          & 0.884624          & 0.897920  & \textbf{0.861161} & 0.880633          & \textbf{0.754967} & 0.894226 & 0.769304          & 0.890041          & 0.873614 & 0.892970  & \textbf{0.876514} \\ \hline
\end{tabular}}

\end{table*}


\textbf{DFL overlay topologies.} Three different DFL overlay topologies are employed in the experiments to evaluate the compliance of the proposed attack DPMA in different types of federation. 
\begin{itemize}
    \item \textbf{Fully:} Each node is directly connected to every other node.
    \item \textbf{Star:} All nodes are connected through a central node.
    \item \textbf{Ring:} Each node is connected to two neighboring nodes, forming a closed loop.
\end{itemize}

\textbf{Training environment.} 
The experiments are conducted using Python 3 and PyTorch, running on an NVIDIA T4 GPU. A total of 10 rounds are run, with each client performing 3 local training epochs in each round. The experiments involve 10 clients, and the DFL experimental programs for all nodes are executed sequentially.

\subsection{Experimental Results}
\textbf{Compare DMPA with state-of-art model poisoning attacks.} In this section, the attack method proposed in this study is compared with the current state-of-the-art model poisoning attack methods, including Lie\cite{baruch2019little} and Min-Max and Min-Sum \cite{shejwalkar2021manipulating}. Three topologies (Fully, Star, and Ring) are used for the comparison experiments, and IID data is used for the experiments. The experimental results are based on a configuration of 40\% malicious clients and 60\% benign clients, and calculate the average F1 scores in the final round (smaller F1 scores represent more effective attacks). The experimental results are shown in Table I. Bolded data indicates the best results.

Firstly, from the experimental results, when the proportion of malicious participants is 40\%. The results show that DMPA has a significant attack effect in DFL with different topologies, different aggregation methods, and different datasets, and its F1 scores are all the lowest, indicating that DMPA has the best attack effect, which fully verifies the effectiveness of the method on attack.

\textbf{Impact of overlay network topology.} In fully connected networks, DMPA shows extreme attackability in several experiments. For example, in experiments on the MNIST dataset corresponding to the Krum aggregation method, the CIFAR-10 dataset corresponding to the Median, Trimmed Mean and FedAvg aggregation methods, and the Fashion-MNIST dataset corresponding to the Krum aggregation method, DMPA achieves a score of less than 0.1. This shows that DMPA effectively influences the gradient descent process of 60\% benign clients and successfully fills the difference in gradient descent by increasing the loss. This shows that DMPA has been validated for its reasonable design, which can invalidate the gradient descent by increasing and filling in the gradient difference. In contrast, although the other three attack methods achieve the attack by influencing the direction and magnitude of the model update, most of the average F1 scores among the 60\% benign clients are greater than 0.8, indicating that their attack effects are more limited.

The overall F1 score of DMPA in the ring network is low and close to the lowest of the other attack methods. The F1 scores of DMPA are all less than 0.8, showing high stability. This may be due to the fact that in ring networks, 60\% of the benign clients are buffered to some extent due to the longer chain, mitigating the impact of the attack, which is a major advantage of DFL. As for the other methods, half of them have F1 scores above 0.8, indicating that these methods are not stable enough for model poisoning attacks in the ring structure and are prone to only changing the direction of the model and not effectively affecting other benign clients. This further demonstrates that DMPA exhibits stable and effective attacks that can generally affect benign clients in the DFL ring network.
In the Star network, DMPA also shows its advantages, further proving the effectiveness of the method in this paper. Since the results of CIFAR-10 are the most representative for studying the attack ratio among the three datasets, the CIFAR-10 dataset will be selected in the following to further investigate the impact of the malicious client ratio on the overall experimental results.


\textbf{Impact of increasing the percentage of malicious Participants.} Figure \ref{fig:cifar-attack} to Figure \ref{fig:fmnist-attack} shows the impact of MPA on the F1 score of benign clients after the last round of aggregation in datasets, as the rate of malicious nodes increases from 10\% to 60\% in the DFL environment. The orange of the dotted line is no attack in each aggregation. 

As can be seen, the DMPA method shows the best attack effect in many results to achieve effective attacks. Compared with other methods on different data sets, DMPA has the influence of attacks. However, in different datasets, this paper finds that the more complex the network is, the more effective its influence is. When more network layers are used (for example, CIFAR-10 uses more complex networks), its attack effectiveness is higher than other methods. Obviously, the higher the complexity of the network, the more influence it has. Overall, the method in this paper is more effective than other attack methods at present.

\begin{figure}[htbp]
    \centering
    \begin{subfigure}[b]{\columnwidth}
        \centering
        \includegraphics[width=\textwidth]{combined_cifar10.pdf}
        \caption{CIFAR-10 Dataset Attack Performance}
        \label{fig:cifar-attack}
    \end{subfigure}



    \begin{subfigure}[b]{\columnwidth}
        \centering
        \includegraphics[width=\textwidth]{combined_MNIST.pdf}
        \caption{MNIST Dataset Attack Performance}
        \label{fig:mnist-atack}
    \end{subfigure}


    \begin{subfigure}[b]{\columnwidth}
        \centering
        \includegraphics[width=\textwidth]{combined_FashionMNIST.pdf}
        \caption{Fashion-MNIST Dataset Attack Performance}
        \label{fig:fmnist-attack}
    \end{subfigure}
    \caption{Attack Performance in CIFAR-10, MNIST, and Fashion-MNIST with Different Malicious Clients Rate}
\end{figure}


\section{Conclusion and Future Work}

This study proposes DMPA, a general framework designed to execute model poisoning attacks within DFL systems. The DMPA framework leverages feature angle deviations to ascertain the most effective attack strategy. In contrast to prior attack methodologies that rely on imprecise or heuristic approaches, DMPA employs the model's numerical properties for its computations, thereby maintaining the model's numerical integrity across iterations. Empirical results indicate that DMPA achieves superior attack efficacy compared to existing state-of-the-art model poisoning techniques, underscoring its substantial practical relevance.

Future research is planned to explore both offensive and defensive dimensions. From an offensive standpoint, existing studies predominantly utilize data distributed in an IID manner, thereby simplifying the attack process. However, the complexity of attacks escalates when data distribution is non-IID. Consequently, future research will focus on devising strategies for effective assaults under non-IID conditions. From a defensive perspective, the proposed DPMA underscores the efficacy of using feature angle deviations as a potent attack vector, providing valuable insights for the development of robust mechanisms to protect against potential threats.


%File: formatting-instructions-latex-2025.tex
%release 2025.0
\documentclass[letterpaper]{article} % DO NOT CHANGE THIS
\usepackage{aaai25}  % DO NOT CHANGE THIS
\usepackage{times}  % DO NOT CHANGE THIS
\usepackage{helvet}  % DO NOT CHANGE THIS
\usepackage{courier}  % DO NOT CHANGE THIS
\usepackage[hyphens]{url}  % DO NOT CHANGE THIS
\usepackage{graphicx} % DO NOT CHANGE THIS
\urlstyle{rm} % DO NOT CHANGE THIS
\def\UrlFont{\rm}  % DO NOT CHANGE THIS
\usepackage{natbib}  % DO NOT CHANGE THIS AND DO NOT ADD ANY OPTIONS TO IT
\usepackage{caption} % DO NOT CHANGE THIS AND DO NOT ADD ANY OPTIONS TO IT
\frenchspacing  % DO NOT CHANGE THIS
\setlength{\pdfpagewidth}{8.5in}  % DO NOT CHANGE THIS
\setlength{\pdfpageheight}{11in}  % DO NOT CHANGE THIS
%
% These are recommended to typeset algorithms but not required. See the subsubsection on algorithms. Remove them if you don't have algorithms in your paper.
\usepackage{algorithm}
\usepackage{algorithmic}

%
% These are are recommended to typeset listings but not required. See the subsubsection on listing. Remove this block if you don't have listings in your paper.
\usepackage{newfloat}
\usepackage{listings}
\DeclareCaptionStyle{ruled}{labelfont=normalfont,labelsep=colon,strut=off} % DO NOT CHANGE THIS
\lstset{%
	basicstyle={\footnotesize\ttfamily},% footnotesize acceptable for monospace
	numbers=left,numberstyle=\footnotesize,xleftmargin=2em,% show line numbers, remove this entire line if you don't want the numbers.
	aboveskip=0pt,belowskip=0pt,%
	showstringspaces=false,tabsize=2,breaklines=true}
\floatstyle{ruled}
\newfloat{listing}{tb}{lst}{}
\floatname{listing}{Listing}
%
% Keep the \pdfinfo as shown here. There's no need
% for you to add the /Title and /Author tags.
\pdfinfo{
/TemplateVersion (2025.1)
}
\newcommand{\bestoverall}[1]{\textbf{#1}}
\newcommand{\bestcolumn}[1]{\underline{#1}}
% DISALLOWED PACKAGES
% \usepackage{authblk} -- This package is specifically forbidden
% \usepackage{balance} -- This package is specifically forbidden
% \usepackage{color (if used in text)
% \usepackage{CJK} -- This package is specifically forbidden
% \usepackage{float} -- This package is specifically forbidden
% \usepackage{flushend} -- This package is specifically forbidden
% \usepackage{fontenc} -- This package is specifically forbidden
% \usepackage{fullpage} -- This package is specifically forbidden
% \usepackage{geometry} -- This package is specifically forbidden
% \usepackage{grffile} -- This package is specifically forbidden
% \usepackage{hyperref} -- This package is specifically forbidden
% \usepackage{navigator} -- This package is specifically forbidden
% (or any other package that embeds links such as navigator or hyperref)
% \indentfirst} -- This package is specifically forbidden
% \layout} -- This package is specifically forbidden
% \multicol} -- This package is specifically forbidden
% \nameref} -- This package is specifically forbidden
% \usepackage{savetrees} -- This package is specifically forbidden
% \usepackage{setspace} -- This package is specifically forbidden
% \usepackage{stfloats} -- This package is specifically forbidden
% \usepackage{tabu} -- This package is specifically forbidden
% \usepackage{titlesec} -- This package is specifically forbidden
% \usepackage{tocbibind} -- This package is specifically forbidden
% \usepackage{ulem} -- This package is specifically forbidden
% \usepackage{wrapfig} -- This package is specifically forbidden
% DISALLOWED COMMANDS
% \nocopyright -- Your paper will not be published if you use this command
% \addtolength -- This command may not be used
% \balance -- This command may not be used
% \baselinestretch -- Your paper will not be published if you use this command
% \clearpage -- No page breaks of any kind may be used for the final version of your paper
% \columnsep -- This command may not be used
% \newpage -- No page breaks of any kind may be used for the final version of your paper
% \pagebreak -- No page breaks of any kind may be used for the final version of your paperr
% \pagestyle -- This command may not be used
% \tiny -- This is not an acceptable font size.
% \vspace{- -- No negative value may be used in proximity of a caption, figure, table, section, subsection, subsubsection, or reference
% \vskip{- -- No negative value may be used to alter spacing above or below a caption, figure, table, section, subsection, subsubsection, or reference

\setcounter{secnumdepth}{0} %May be changed to 1 or 2 if section numbers are desired.

% The file aaai25.sty is the style file for AAAI Press
% proceedings, working notes, and technical reports.
%

% Title

% Your title must be in mixed case, not sentence case.
% That means all verbs (including short verbs like be, is, using,and go),
% nouns, adverbs, adjectives should be capitalized, including both words in hyphenated terms, while
% articles, conjunctions, and prepositions are lower case unless they
% directly follow a colon or long dash
\title{Multiple Distribution Shift - Aerial (MDS-A): A Dataset for Test-Time Error Detection and Model Adaptation}

\author {
    % Authors
    Noel Ngu\textsuperscript{\rm 1},
    Aditya Taparia\textsuperscript{\rm 1},
    Gerardo I. Simari\textsuperscript{\rm 2},
    Mario Leiva\textsuperscript{\rm 2},
    Jack Corcoran\textsuperscript{\rm 3},\\
    Ransalu Senanayake\textsuperscript{\rm 1},
    Paulo Shakarian\textsuperscript{\rm 1},
    Nathaniel D. Bastian\textsuperscript{\rm 4}
}
\affiliations {
    % Affiliations
    \textsuperscript{\rm 1}Arizona State University, Tempe, AZ USA\\
   \textsuperscript{\rm 2}Department of Computer Science and Engineering, Universidad Nacional del Sur and Institute for Computer Science and Engineering, 
Bahía Blanca, Argentina\\
    \textsuperscript{\rm 3}U.S. Department of Defense, Arlington, VA USA\\
    \textsuperscript{\rm 4}United States Military Academy, West Point, NY USA\\
    nngu2@asu.edu, ataparia@asu.edu, gis@cs.uns.edu.ar, mario.leiva@cs.uns.edu.ar, jack.fd.corcoran@gmail.com, ransalu@asu.edu, pshak02@asu.edu,  nathaniel.bastian@westpoint.edu
}

%Example, Single Author, ->> remove \iffalse,\fi and place them surrounding AAAI title to use it
\iffalse
\title{Multiple Distribution Shift - Aerial (MDS-A): A Dataset for Test-Time Error Detection and Model Adaptation}

\author {
    % Authors
    Noel Ngu\textsuperscript{\rm 1},
    Aditya Taparia\textsuperscript{\rm 1},
    Gerardo I. Simari\textsuperscript{\rm 2},
    Mario Leiva\textsuperscript{\rm 2},
    Jack Corcoran\textsuperscript{\rm 3},\\
    Ransalu Senanayake\textsuperscript{\rm 1},
    Paulo Shakarian\textsuperscript{\rm 1},
    Nathaniel D. Bastian\textsuperscript{\rm 4}
}
\affiliations {
    % Affiliations
    \textsuperscript{\rm 1}Arizona State University, Tempe, AZ USA\\
   \textsuperscript{\rm 2}Department of Computer Science and Engineering, Universidad Nacional del Sur and Institute for Computer Science and Engineering, 
Bahía Blanca, Argentina\\
    \textsuperscript{\rm 3}U.S. Department of Defense, Arlington, VA USA\\
    \textsuperscript{\rm 4}United States Military Academy, West Point, NY USA\\
    nngu2@asu.edu, ataparia@asu.edu, gis@cs.uns.edu.ar, mario.leiva@cs.uns.edu.ar, jack.fd.corcoran@gmail.com, ransalu@asu.edu, pshak02@asu.edu,  nathaniel.bastian@westpoint.edu
}
\fi

\iffalse
%Example, Multiple Authors, ->> remove \iffalse,\fi and place them surrounding AAAI title to use it
\title{Multiple Distribution Shift - Aerial (MDS-A): A Dataset for Test-Time Error Detection and Model Adaptation}

\author {
    % Authors
    Noel Ngu\textsuperscript{\rm 1},
    Aditya Taparia\textsuperscript{\rm 1},
    Gerardo I. Simari\textsuperscript{\rm 2},
    Mario Leiva\textsuperscript{\rm 2},
    Jack Corcoran\textsuperscript{\rm 3},\\
    Ransalu Senanayake\textsuperscript{\rm 1},
    Paulo Shakarian\textsuperscript{\rm 1},
    Nathaniel D. Bastian\textsuperscript{\rm 4}
}
\affiliations {
    % Affiliations
    \textsuperscript{\rm 1}Arizona State University, Tempe, AZ USA\\
   \textsuperscript{\rm 2}Department of Computer Science and Engineering, Universidad Nacional del Sur and Institute for Computer Science and Engineering, 
Bahía Blanca, Argentina\\
    \textsuperscript{\rm 3}U.S. Department of Defense, Arlington, VA USA\\
    \textsuperscript{\rm 4}United States Military Academy, West Point, NY USA\\
    nngu2@asu.edu, ataparia@asu.edu, gis@cs.uns.edu.ar, mario.leiva@cs.uns.edu.ar, jack.fd.corcoran@gmail.com, ransalu@asu.edu, pshak02@asu.edu,  nathaniel.bastian@westpoint.edu
}
\fi


% REMOVE THIS: bibentry
% This is only needed to show inline citations in the guidelines document. You should not need it and can safely delete it.
% END REMOVE bibentry
\def\mdsawebsite{https://lab-v2.github.io/mdsa-dataset-website}
\begin{document}

\maketitle

\begin{abstract}
Machine learning models assume that training and test samples are drawn from the same distribution. As such, significant differences between training and test distributions often lead to degradations in performance. We introduce Multiple Distribution Shift - Aerial (MDS-A) - a collection of inter-related datasets of the same aerial domain that are perturbed in different ways to better characterize the effects of out-of-distribution performance.  Specifically, MDS-A is a set of simulated aerial datasets collected under different weather conditions. We include six datasets under different simulated weather conditions along with six baseline object-detection models, as well as several test datasets that are a mix of weather conditions that we show have significant differences from the training data. In this paper, we present characterizations of MDS-A, provide performance results for the baseline machine learning models (on both their specific training datasets and the test data), as well as results of the baselines after employing recent knowledge-engineering error-detection techniques (EDR) thought to improve out-of-distribution performance.  The dataset is available at \mdsawebsite.
\end{abstract}

% Uncomment the following to link to your code, datasets, an extended version or similar.
%
% \begin{links}
%     \link{Code}{https://aaai.org/example/code}
%     \link{Datasets}{https://aaai.org/example/datasets}
%     \link{Extended version}{https://aaai.org/example/extended-version}
% \end{links}


\section{Introduction}
The robustness of models for object-detection remain a critical challenge when dealing with distributional shifts in real-world data. Distributional shifts in weather are especially important in aerial imagery since visibility and object-recognition can be heavily influenced by the weather.  Prior work on establishing benchmarks for out-of-distribution (OOD) object detection has largely focused on evaluating existing model performance during such a shift~\cite{Mao_Chen_Zhu_Chen_Su_Zhang_Xue_2023,Gardner_Popovic_Schmidt_2024}.  In this work, we present the Multiple Distribution Shift - Aerial (MDS-A) dataset - a collection of generated and labeled datasets with varying distribution differences and an associated set of baseline models.  To control experiments, we keep the baseline domain (aerial imagery) constant and perturb it in different ways to better characterize the effects of out-of-distribution performance. Specifically, MDS-A is a set of simulated aerial datasets taken under different weather conditions. We include six datasets under different simulated weather conditions along with six baseline object detection models as well as several test datasets that are a mix of weather conditions that we show have significant differences from the training data. In this paper, we present characterizations of MDS-A, provide performance results for the baseline models (on both in in-distribution and out-of-distribution test sets), as well as results of the baselines after employing recent knowledge-engineering error-detection techniques (error detection rules, or EDR~\cite{Kricheli_Vo_Datta_Ozgur_Shakarian_2024,Xi_Scaria_Bavikadi_Shakarian_2024,Lee_Ngu_Sahdev_Motaganahall_Chowdhury_Xi_Shakarian_2024,Shakarian_Simari_Bastian_2025}) thought to improve out-of-distribution performance. The rest of the paper is organized as follows.  First, we introduce the dataset, describing how it was constructed, and reporting on key statistics, importantly measures of distributional differences between the various training and testing sets.  Then, we describe how we trained a series of baseline models, and report on their performance both with and without error detection rules.  Finally, we discuss future research directions in the conclusion.

%This paper presents a novel aerial imagery dataset that was specifically designed to evaluate the impact of distributional shift on object-detection tasks in the context of weather.

%By using the AirSim simulator, we generate a set of aerial images that mimic various weather conditions. This dataset serves as a benchmark for evaluating object-detection models under distributional shifts related to weather. This dataset is publicly released, allowing researchers to access it for experimentation.


\section{Dataset}
In this section, we describe how we created the MDS-A dataset and report key statistics including measures of distributional differences.

\subsubsection{AirSim simulator}
To investigate the impact of distributional shift on aerial imagery in the context of weather conditions, we employed AirSim, an open-source simulator for drones, ground vehicles, cars, and other objects \cite{Shah_Dey_Lovett_Kapoor_2017}, to create a dataset of aerial imagery under various weather conditions. AirSim provides tools to capture images from different positions under different weather conditions by adjusting configurable parameters for effects such as dust, rain, fog, snow, and maple leaves. These parameters give us control over the intensity of various weather effects in the simulated scenes. Panel A and B in Figure \ref{fig:methodology_airsim} demonstrates how changing these parameters visually impact the images captured in AirSim.

\begin{figure*}[t]
\centering
\includegraphics[width=1\textwidth]{images/methodology_airsim.png} 
\caption{A) Image captured in AirSim with no weather effects applied along with a histogram showing the distribution of weather conditions of the dataset that it represents. B) Images captured in the same position in AirSim under different weather conditions: dust, fog, maple leaves, rain, snow-along with a histogram showing the distribution of weather conditions of the dataset that it represents. C) Images captured in the same position in AirSim with a mix of weather conditions applied along with a histogram showing the distribution of weather conditions of the dataset that it represents. D) A histogram showing the FID scores between the training sets and the 3 test sets. E) A histogram showing the FID score comparisions between the test sets.}
\label{fig:methodology_airsim}
\end{figure*}

\subsubsection{Data collection} 
For this study, the drone vehicle in AirSim was utilized to capture images from a top-down view at random positions within a simulated city environment. Given evidence that state-of-the-art object detection models are often susceptible to diverse weather conditions \cite{Pathiraja2024arxiv}, we configured AirSim with the following weather effects: rain, snow, fog, maple leaves, and dust. Using AirSim, images along with their bounding boxes were generated. Objects in the captured scenes were labeled by the research team to be classified into the following four categories: pedestrians, vehicles, nature, and construction. Each bounding box was assigned to exactly one of the categories. 

\begin{itemize}
    \item \textbf{Training sets} Multiple training sets were created, each focusing on a specific weather condition. For each dataset, the corresponding weather parameter for a weather condition (e.g.rain, snow, fog, maple leaves, or dust) are set randomly with a specific weather condition set to a particularly high value, while the other weather parameters were set to low values. As a result, we created distinct training datasets for the following conditions: Rain, Snow, Fog, Maple leaves, Dust. Panel B in Figure \ref{fig:methodology_airsim} shows the average intensities of each weather condition in each training set. In addition, a training set with no weather effects was created as well. The objective of these training sets are to enable the models to specialize in identifying objects under a single dominant weather condition.
    \item \textbf{Test sets} The test set was designed to evaluate the ability of models (trained on the training datasets) on a dataset that was created to simulate natural distributional shifts in weather. Unlike the training set, the test set consists of complex weather conditions where multiple weather conditions could be set to high values simultaneously, creating more challenging object-detection samples for the models. Panel C in Figure \ref{fig:methodology_airsim} shows the average intensities of each weather condition in the test set.
\end{itemize}

\subsubsection{Dataset Statistics} 
MDS-A consists of training sets that focus on a single weather condition, with each set containing 1000 images. The following training sets were generated: No-Effect Train Set, Dust Train Set, Fog Train Set, Maple-Leaves Train Set, Rain Train Set, Snow Train Set. The test sets, in contrast, feature a complex combination of weather conditions, also comprising 1000 images. 
Table \ref{table:dataset_statistics} shows some statistics regarding the number of images and the number of bounding boxes in each training set and test set.  
We note that with each of the six training sets, there is also a corresponding in-distribution hold-out set containing 100 images- this allows us to compare model in-distribution performance with out-of-distribution performance easily, all the while controlling for other factors.

\begin{table}
\centering
\begin{tabular}{|c|c|c|}
\hline
Name & Images & Bounding boxes \\
\hline
No-Effect Train Set & 1000 & 13320 \\
Dust Train Set & 1000 & 11257 \\
Fog Train Set & 1000 & 11099 \\
Maple-Leaves Train Set & 1000 & 12295 \\
Rain Train Set & 1000 & 11528 \\
Snow Train Set & 1000 & 11462 \\
\hline
No-Effect Test Set & 100 & 1267 \\
Dust Test Set & 100 & 1255 \\
Fog Test Set & 100 & 1406 \\
Maple-Leaves Test Set & 100 & 1077 \\
Rain Test Set & 100 & 1299 \\
Snow Test Set & 100 & 1368 \\
\hline
Test Set 1 & 1000 & 11117 \\
Test Set 2 & 1000 & 12466 \\
Test Set 3 & 1000 & 12558 \\
\hline
\end{tabular}
\caption{Statistics regarding the number of images and the number of bounding boxes in each training set and test set.}
\label{table:dataset_statistics}
\end{table}


Additionally, the Fréchet Inception Distance (FID) \cite{Heusel_Ramsauer_Unterthiner_Nessler_Hochreiter_2018} scores between the training sets and the test set are presented in Panel D in Figure \ref{fig:methodology_airsim}. These scores reflect the visual similarity between the training sets and the test set, providing a way to approximate the amount of distributional-shift between the training set and the test set. Higher FID scores, especially for conditions like Fog (73.3), suggests a larger distributional shift between the training set and the test set.

\subsubsection{Metadata Conditions} In addition to the datasets, we also provide additional meta conditions for each sample.  This information can be used to learn metacognitive models to identify potential errors. We use these in our baselines for error detection later in the paper. Examples of such conditions can be seen in Table \ref{tab:example_rls_Rules}


\begin{table}
\scriptsize
\centering
\begin{tabular}{|c|c|}
\hline
Rule & Meaning of Rule\\
\hline
$cond_{green}(w)$ &
Colors inside the bounding box has to be green \\

\hline 

$cond_{overlap}(w)$ & Pedestrians and vehicles should not overlap.
\\

\hline
\end{tabular}

\caption{Example EDCR Rule Learned for the MPSC Problem}
\label{tab:example_rls_Rules}
\end{table}

\section{Baseline Models and Associated Performance}
In addition to providing a dataset, we provide a series of baseline models, in addition to employing error detection rules \cite{Kricheli_Vo_Datta_Ozgur_Shakarian_2024,Xi_Scaria_Bavikadi_Shakarian_2024,Lee_Ngu_Sahdev_Motaganahall_Chowdhury_Xi_Shakarian_2024,Shakarian_Simari_Bastian_2025}.

% , hierarchical classification , and metal-price spike classification in time-series data \cite{Lee_Ngu_Sahdev_Motaganahall_Chowdhury_Xi_Shakarian_2024}. These works have demonstrated that EDCR not 
\subsubsection{Model Training}
In order to establish a baseline for model performance under distributional shifts in the context of weather conditions, object-detection models were trained on each training set. The baseline object detection model that was used was DeTR \cite{Carion_Massa_Synnaeve_Usunier_Kirillov_Zagoruyko_2020} with a ResNet-50 \cite{He_Zhang_Ren_Sun_2016} backbone. 
% Commenting this out so we fit in the page limit
% The models were trained with consistent hyperparameters across all experiments. These hyperparameters include:

% \begin{itemize}
%     \item Number of epochs: 500
%     \item Learning rate: 0.00005
%     \item Weight decay: 0.0001
%     \item Max gradient norm: 0.01
% \end{itemize}

% To enhance the robustness of the trained models, data augmentations such as perspective shifts, flipping, random changes in brightness and hue were applied to the data as well. 

The models were intentionally trained on a single training set without any mixes between training sets in order to emphasize different weather effects. These models were then evaluated on a more complex dataset aimed to emulate natural distributional shifts in weather conditions. 


\subsubsection{In-Distribution Model Performance}
Table \ref{tab:training_distribution} provides results of the baseline models on their corresponding in-distribution dataset, specifically the \textit{No Effect Test Set}, \textit{Dust Test Set}, \textit{Fog Test Set}, \textit{Maple-Leaves Test Set}, \textit{Rain Test Set}, and \textit{Snow Test Set} (see statistics in Table~\ref{table:dataset_statistics} for details).  Here we report precision, recall, and F1 (harmonic mean of precision and recall). We note that model performance is generally consistent across the various models.


\subsubsection{Performance of Baseline Models on Test Sets}
The baseline models were evaluated on out-of-distribution test sets to assess their robustness under complex weather conditions that differ from the distribution in which they were trained - this is to establish a baseline for out-of-distribution performance on the three test sets in MDS-A. Table \ref{tab:model_performance_on_test_set_1} shows an expected decline in precision, recall, and F1 compared to in-distribution results. 

\begin{table*}
\begin{center}
\begin{tabular}{|c|c|c|c|c|c|c|}
\hline
& & & & Precision & Recall & F1 \\
Model & Precision & Recall & F1 & (EDR) & (EDR) & (EDR) \\
\hline
\multicolumn{7}{|c|}{Test Set 1} \\
\hline
No Effect Model & 0.35 & 0.27 & 0.31 & \bestoverall{\bestcolumn{0.62}} & 0.25 & 0.36 \\
Snow Model & 0.59 & \bestoverall{\bestcolumn{0.55}} & \bestoverall{\bestcolumn{0.57}} & 0.61 & 0.50 & 0.55 \\
    Dust Model & 0.59 & 0.54 & \bestoverall{\bestcolumn{0.57}} & 0.61 & 0.49 & 0.54 \\
Maple Leaf Model & \bestcolumn{0.60} & \bestoverall{\bestcolumn{0.55}} & \bestoverall{\bestcolumn{0.57}} & 0.60 & \bestoverall{\bestcolumn{0.55}} & \bestoverall{\bestcolumn{0.57}} \\
Rain Model & \bestcolumn{0.60} & 0.54 & \bestoverall{\bestcolumn{0.57}} & 0.60 & 0.54 & \bestoverall{\bestcolumn{0.57}} \\
Fog Model & 0.56 & 0.53 & 0.55 & 0.56 & 0.53 & 0.55 \\
\hline
\multicolumn{7}{|c|}{Test Set 2} \\
\hline
No Effect Model & 0.16 & 0.14 & 0.15 & \bestoverall{\bestcolumn{0.54}} & 0.13 & 0.21 \\
Snow Model & 0.44 & \bestoverall{\bestcolumn{0.26}} & \bestoverall{\bestcolumn{0.32}} & 0.47 & \bestcolumn{0.25} & \bestoverall{\bestcolumn{0.32}} \\
Dust Model & 0.43 & 0.25 & \bestoverall{\bestcolumn{0.32}} & 0.46 & 0.24 & \bestoverall{\bestcolumn{0.32}} \\
Maple Leaf Model & 0.45 & 0.25 & \bestoverall{\bestcolumn{0.32}} & 0.45 & \bestcolumn{0.25} & \bestoverall{\bestcolumn{0.32}} \\
Rain & \bestcolumn{0.46} & 0.25 & \bestoverall{\bestcolumn{0.32}} & 0.46 & \bestcolumn{0.25} & \bestoverall{\bestcolumn{0.32}}\\
fog & 0.40 & 0.25 & 0.31 & 0.40 & \bestcolumn{0.25} & 0.31 \\
\hline
\multicolumn{7}{|c|}{Test Set 3} \\
\hline
No Effect Model & 0.50 & 0.35 & 0.41 & \bestoverall{\bestcolumn{0.65}} & 0.30 & 0.41 \\
Snow Model & \bestcolumn{0.63} & 0.52 & \bestoverall{\bestcolumn{0.57}} & \bestoverall{\bestcolumn{0.65}} & 0.49 & 0.56 \\
Dust Model & 0.58 & 0.47 & 0.52 & 0.60 & 0.43 & 0.50 \\
Maple Leaf Model & 0.61 & \bestoverall{\bestcolumn{0.53}} & \bestoverall{\bestcolumn{0.57}} & 0.61 & \bestoverall{\bestcolumn{0.53}} & \bestoverall{\bestcolumn{0.57}} \\
Rain Model & 0.57 & 0.47 & 0.52 & 0.57 & 0.47 & 0.52 \\
Fog Model & 0.55 & 0.42 & 0.48 & 0.55 & 0.42 & 0.48 \\
\hline
\end{tabular}
\end{center}
\caption{Table showing the before and after results of applying EDR. Underlined numbers indicates the best model. Bold numbers indicates the best performing model across both baseline and EDR.}
\label{tab:model_performance_on_test_set_1}
\end{table*}

\subsubsection{Models with Error Detection Rules}
To enhance the robustness of the baseline models, error detection rule learning (EDR) was applied using the DetRuleLearn algorithm \cite{Xi_Scaria_Bavikadi_Shakarian_2024} with the hyperparameter of $\epsilon$ set to 0.5. Note that the rules were trained on the same data as the models.v  The application of EDR showed improvements in Precision while mostly maintaining F1 across all test sets as shown in Table 3.  This is due to the fact that EDR rules produce detections that are essentially recognizing that the model will most likely produce an error - and hence the results are discarded - resulting in a reduction of recall but an increase in precision.  We note that the results of \cite{Xi_Scaria_Bavikadi_Shakarian_2024} associate recall reduction with the $\epsilon$ hyperparameter (which would be up to an $0.5$ reduction, see Theorem 2 in \cite{Xi_Scaria_Bavikadi_Shakarian_2024} ) - however it is noteworthy that the reduction in recall is much less than predicted by the theoretical guarantee.


\begin{table}
\begin{center}
\begin{tabular}{|c|c|c|c|}
\hline
Model & Precision & Recall & F1 \\
\hline
No Effect & 0.75 & 0.62 & 0.68  \\
Snow &  0.75 & 0.69 & 0.72  \\
Dust & 0.75 & 0.65 & 0.70  \\
Maple Leaves & 0.76 & 0.70 & 0.73  \\
Rain & 0.75 & 0.65 & 0.70 \\
Fog & 0.73 & 0.62 & 0.67 \\
\hline
\end{tabular}
\end{center}
\caption{Table showing the performance of the baseline models trained on different training sets on an in-distribution dataset that is distinct from the training set.}
\label{tab:training_distribution}
\end{table}



\section{Conclusion and Future Work}
In this paper, we introduced the Multiple Distribution Shift - Aerial (MDS-A) dataset, a collection of simulated aerial datasets made to investigate the impact of distributional shifts, in the context of weather conditions, on object-detection model performance. Using the AirSim simulator, we created training datasets under six distinct weather conditions—rain, snow, fog, maple leaves, dust, and no effects—and evaluated the performance of baseline object-detection models trained on each condition using a complex test set that combines multiple weather effects. We also provide a suite of baseline models and in this paper we report on their performance for both in-distribution and out-of-distribution datasets.  Additionally, we also provide a baseline using error detection rules, which mitigates the degradation of precision. As we intend this to be a challenge dataset, we released MDS-A and the associated baseline models at \mdsawebsite.

Recent advances in topics such as test time training~\cite{Liang_He_Tan_2025}, domain generalization~\cite{Zhou_Liu_Qiao_Xiang_Loy_2023}, and meta learning~\cite{Vanschoren_2018} are all potential candidates for improving performance. Further, this dataset allows the exploration of novel ensemble methods based on models trained on different distributions.

\section{Acknowledgments}
This research was supported by the Defense Advanced Research Projects Agency (DARPA) under Cooperative Agreement No. HR00112420370, the U.S. Army Combat Capabilities Development Command (DEVCOM) Army Research Office under Grant No. W911NF-24-1-0007, and the U.S. Army DEVCOM Army Research Lab under Support Agreement No. USMA 21050. The views expressed in this paper are those of the authors and do not reflect the official policy or position of the U.S. Military Academy, the U.S. Army, the U.S. Department of Defense, or the U.S. Government.

\bigskip
\noindent Thank you for reading these instructions carefully. We look forward to receiving your electronic files!

\bibliography{aaai25}

\end{document}


\end{document}
