%!TEX root = main.tex


\section{Conclusion}

We have presented a study of imperfect recall without
absentmindedness, through the lens of A-loss recall. Specifically, we
have given two methods to transform imperfect recall games to A-loss
recall games. Behavioral strategies for the original game can be
obtained by analyzing the transformed game. This investigation has
resulted in new polynomial-time solvable classes of one-player and
two-player games. We have also shown how to find a transformation
of minimal size. It would be interesting to see the influence of these
notions of $\salr$ and $\alr$-span, and the idea of using sequence
sets instead of games, in algorithms that use imperfect recall
abstractions.

In summary, in this work, we have laid the foundations to simplify imperfect
recall in terms of $\alr$. We do hope that this perspective leads to
further theoretical and experimental investigations.  


%%% Local Variables:
%%% mode: latex
%%% TeX-master: "main"
%%% End:
