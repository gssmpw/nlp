%!TEX root = main.tex


\section{Two-player games}\label{sec:two-player}
%!TEX root = ../main.tex

\begin{figure}
\begin{subfigure}{0.45\columnwidth}
\centering
\tikzset{
trisqangle/.style = {regular polygon,regular polygon sides=3,draw,inner sep = 2},
sq/.style = {circle,fill=cyan!10,draw,inner sep = 3},
term/.style = {circle,draw,inner sep = 1.5,fill=black},
circ/.style = {rectangle,fill=gray!20, draw, inner sep = 4}
}

\begin{tikzpicture}[scale=0.85]
\tikzstyle{level 1}=[level distance=9mm,sibling distance = 22mm]
\tikzstyle{level 2}=[level distance=7mm,sibling distance=10mm]
\tikzstyle{level 3}=[level distance=7mm,sibling distance=6mm]
\tikzstyle{level 4}=[level distance=7mm,sibling distance=5mm]

%node (ij) is the j th node in i th level

\begin{scope}[->, >=stealth]
\node (0) [sq] {}
child {
  node (00) [circ] {}
  child {
    node (000) [circ] {}
    child {
      node (0000) [term, label=below:{\scriptsize $z_1$}] {}
      edge from parent node [left] {\scriptsize $a$}
    }
    child {
      node (0001) [term, label=below:{\scriptsize $z_2$}] {}
      edge from parent node [right] {\scriptsize $\bar{a}$}
      }
    edge from parent node [left] {\scriptsize $b$}
  }
  child {
    node (001) [circ] {}
    child {
      node (0010) [term, label=below:{\scriptsize $z_3$}] {}
      edge from parent node [left] {\scriptsize $a$}
    }
    child {
      node (0011) [term, label=below:{\scriptsize $z_4$}] {}
      edge from parent node [right] {\scriptsize $\bar{a}$}
      }
    edge from parent node [right] {\scriptsize $\bar{b}$} 
  }
  edge from parent node [above] {\scriptsize $d$}
}
child {
  node (01) [circ] {}
   child {
     node (010) [circ] {}
     child {
      node (0100) [term, label=below:{\scriptsize $z_5$}] {}
      edge from parent node [left] {\scriptsize $a$}
    }
    child {
      node (0101) [term, label=below:{\scriptsize $z_6$}] {}
      edge from parent node [right] {\scriptsize $\bar{a}$}
      }
    edge from parent node [left] {\scriptsize $c$}
  }
  child {
    node (011) [circ] {}
    child {
      node (0110) [term, label=below:{\scriptsize $z_7$}] {}
      edge from parent node [left] {\scriptsize $a$}
    }
    child {
      node (0111) [term, label=below:{\scriptsize $z_8$}] {}
      edge from parent node [right] {\scriptsize $\bar{a}$}
      }
    edge from parent node [right] {\scriptsize $\bar{c}$} 
  }
  edge from parent node [above] {\scriptsize $\bar{d}$}
}
;
\node[fit=(0),dashed,thick,brown, draw, circle, inner sep=1pt] {};
 \node[fit=(00),dashed,thick,blue, draw, rectangle,inner sep=2pt] {};
  \node[fit=(01),dashed,thick,red, draw, rectangle,inner sep=2pt] {};
\end{scope}

\draw [dashed, thick, ForestGreen, in=150,out=30] (000) to (001);
\draw [dashed, thick, ForestGreen, in=150,out=30] (001) to (010);
\draw [dashed, thick, ForestGreen, in=150,out=30] (010) to (011);
%\draw [dashed, thick, blue, in=150,out=30] (000) to (001);
%\draw [dashed, thick, red, in=150,out=30] (010) to (011);

%\node [black] at (0,0.35) {\scriptsize $r$};
%\node [black] at (-1.5,-0.55) {\scriptsize $u_1$};
%\node [black] at (1.5, -0.55) {\scriptsize $u_2$};
%\node [black] at (-2, -1.6) {\scriptsize $u_3$};
%\node [black] at (-.25, -1.7) {\scriptsize $u_4$};
%
%\node [black] at (0.25, -1.7) {\scriptsize $u_5$};
%\node [black] at (2, -1.6) {\scriptsize $u_6$};

%obs labels
%\node [ForestGreen] at (0,-1.1) {\scriptsize $I_3$};
%\node [blue] at (-.55,-.9) {\scriptsize $I_1$};
%\node [red] at (.55,-.9) {\scriptsize $I_2$};


\end{tikzpicture}
\caption{$(\pfr,\salr)$ game}
\label{fig:2-p-shuffle-a}
\end{subfigure}
\begin{subfigure}{0.45\columnwidth}
\centering
\tikzset{
triangle/.style = {regular polygon,regular polygon sides=3,draw,inner sep = 2},
circ/.style = {circle,fill=cyan!10,draw,inner sep = 3},
term/.style = {circle,draw,inner sep = 1.5,fill=black},
sq/.style = {rectangle,fill=gray!20, draw, inner sep = 4}
}

\begin{tikzpicture}[scale=0.85]
\tikzstyle{level 1}=[level distance=9mm,sibling distance = 22mm]
\tikzstyle{level 2}=[level distance=7mm,sibling distance=10mm]
\tikzstyle{level 3}=[level distance=7mm,sibling distance=6mm]
\tikzstyle{level 4}=[level distance=7mm,sibling distance=5mm]

%node (ij) is the j th node in i th level

\begin{scope}[->, >=stealth]
\node (0) [circ] {}
child {
  node (00) [term, label=below:{}] {}
  edge from parent node [above] {\scriptsize $d$}
}
child {
  node (01) [term, label=below:{}] {}
  edge from parent node [above] {\scriptsize $\bar{d}$}
}
;
\node[fit=(0),dashed,thick,brown, draw, circle, inner sep=1pt] {};
 %\node[fit=(00),dashed,thick,blue, draw, circle,inner sep=1pt] {};
 % \node[fit=(01),dashed,thick,red, draw, circle,inner sep=1pt] {};
\end{scope}

%\draw [dashed, thick, ForestGreen, in=150,out=30] (000) to (001);
%\draw [dashed, thick, ForestGreen, in=150,out=30] (001) to (010);
%\draw [dashed, thick, ForestGreen, in=150,out=30] (010) to (011);
%\draw [dashed, thick, blue, in=150,out=30] (000) to (001);
%\draw [dashed, thick, red, in=150,out=30] (010) to (011);

%\node [black] at (0,0.35) {\scriptsize $r$};
%\node [black] at (-1.5,-0.55) {\scriptsize $u_1$};
%\node [black] at (1.5, -0.55) {\scriptsize $u_2$};
%\node [black] at (-2, -1.6) {\scriptsize $u_3$};
%\node [black] at (-.25, -1.7) {\scriptsize $u_4$};
%
%\node [black] at (0.25, -1.7) {\scriptsize $u_5$};
%\node [black] at (2, -1.6) {\scriptsize $u_6$};

%obs labels
%\node [ForestGreen] at (0,-1.1) {\scriptsize $I_3$};
%\node [blue] at (-.55,-.9) {\scriptsize $I_1$};
%\node [red] at (.55,-.9) {\scriptsize $I_2$};


\end{tikzpicture}
\caption{$\pfr$ game structure}
\label{fig:2-p-shuffle-b}
\end{subfigure}

\begin{subfigure}{\columnwidth}
\centering
\tikzset{
triangle/.style = {regular polygon,regular polygon sides=3,draw,inner sep = 2},
sq/.style = {circle,fill=cyan!10,draw,inner sep = 3},
term/.style = {circle,draw,inner sep = 1.5,fill=black},
circ/.style = {rectangle,fill=gray!20, draw, inner sep = 4}
}

\begin{tikzpicture}[scale=0.90]
\tikzstyle{level 1}=[level distance=7mm,sibling distance = 48mm]
\tikzstyle{level 2}=[level distance=5mm,sibling distance = 24mm]
\tikzstyle{level 3}=[level distance=9mm,sibling distance=12mm]
\tikzstyle{level 4}=[level distance=7mm,sibling distance=5mm]
\tikzstyle{level 5}=[level distance=9mm,sibling distance=4.5mm]

%node (ij) is the j th node in i th level

\begin{scope}[->, >=stealth]
\node  (r)[sq]{}
child{
node (l0) [circ] {}
child {
  node (l00) [triangle] {}
  child {
    node (l000) [circ] {}
    child {
      node (l0000) [term, label=below:{\scriptsize $2z_1$}] {}
      edge from parent node [left] {\scriptsize $b$}
    }
    child {
      node (l0001) [term, label=below:{\scriptsize $2z_3$}] {}
      edge from parent node [right] {\scriptsize $\bar{b}$}
      }
    edge from parent node [left,pos = 0.2] {\scriptsize $\frac{1}{2}$}
  }
  child {
    node (l001) [circ] {}
    child {
      node (l0010) [term, label=below:{\scriptsize $2z_5$}] {}
      edge from parent node [left] {\scriptsize $c$}
    }
    child {
      node (l0011) [term, label=below:{\scriptsize $2z_7$}] {}
      edge from parent node [right] {\scriptsize $\bar{c}$}
      }
    edge from parent node [right, pos=0.2] {\scriptsize $\frac{1}{2}$} 
  }
  edge from parent node [above] {\scriptsize $a$}
}
child {
  node (l01) [triangle] {}
   child {
     node (l010) [circ] {}
     child {
      node (l0100) [term, label=below:{\scriptsize $2z_2$}] {}
      edge from parent node [left] {\scriptsize $b$}
    }
    child {
      node (l0101) [term, label=below:{\scriptsize $2z_4$}] {}
      edge from parent node [right] {\scriptsize $\bar{b}$}
      }
    edge from parent node [left,pos = 0.2] {\scriptsize $\frac{1}{2}$}
  }
  child {
    node (l011) [circ] {}
    child {
      node (l0110) [term, label=below:{\scriptsize $2z_6$}] {}
      edge from parent node [left] {\scriptsize $c$}
    }
    child {
      node (l0111) [term, label=below:{\scriptsize $2z_8$}] {}
      edge from parent node [right] {\scriptsize $\bar{c}$}
      }
    edge from parent node [right, pos=0.2] {\scriptsize $\frac{1}{2}$} 
  }
  edge from parent node [above] {\scriptsize $\bar{a}$}
}
edge from parent node [above] {\scriptsize $d$} 
}
child{
node (r0) [circ] {}
child {
  node (r00) [triangle] {}
  child {
    node (r000) [circ] {}
    child {
      node (r0000) [term, label=below:{\scriptsize $2z_1$}] {}
      edge from parent node [left] {\scriptsize $b$}
    }
    child {
      node (r0001) [term, label=below:{\scriptsize $2z_3$}] {}
      edge from parent node [right] {\scriptsize $\bar{b}$}
      }
    edge from parent node [left,pos = 0.2] {\scriptsize $\frac{1}{2}$}
  }
  child {
    node (r001) [circ] {}
    child {
      node (r0010) [term, label=below:{\scriptsize $2z_5$}] {}
      edge from parent node [left] {\scriptsize $c$}
    }
    child {
      node (r0011) [term, label=below:{\scriptsize $2z_7$}] {}
      edge from parent node [right] {\scriptsize $\bar{c}$}
      }
    edge from parent node [right, pos=0.2] {\scriptsize $\frac{1}{2}$} 
  }
  edge from parent node [above] {\scriptsize $a$}
}
child {
  node (r01) [triangle] {}
   child {
     node (r010) [circ] {}
     child {
      node (r0100) [term, label=below:{\scriptsize $2z_2$}] {}
      edge from parent node [left] {\scriptsize $b$}
    }
    child {
      node (r0101) [term, label=below:{\scriptsize $2z_4$}] {}
      edge from parent node [right] {\scriptsize $\bar{b}$}
      }
    edge from parent node [left,pos = 0.2] {\scriptsize $\frac{1}{2}$}
  }
  child {
    node (r011) [circ] {}
    child {
      node (r0110) [term, label=below:{\scriptsize $2z_6$}] {}
      edge from parent node [left] {\scriptsize $c$}
    }
    child {
      node (r0111) [term, label=below:{\scriptsize $2z_8$}] {}
      edge from parent node [right] {\scriptsize $\bar{c}$}
      }
    edge from parent node [right, pos=0.2] {\scriptsize $\frac{1}{2}$} 
  }
  edge from parent node [above] {\scriptsize $\bar{a}$}
}
edge from parent node [above] {\scriptsize $\bar{d}$}  
}


;
\end{scope}

%\draw [dashed, thick, ForestGreen, in=150,out=30] (00) to (01);

%\node[fit=(0),dashed,thick,ForestGreen, draw, circle,inner sep=1pt] {};
\node[fit=(0),dashed,thick,brown, draw, circle, inner sep=1pt] {};

\draw [dashed, thick, ForestGreen, in=170,out=10] (l0) to (r0);
\draw [dashed, thick, blue, in=150,out=30] (l000) to (l010);
\draw [dashed, thick, red, in=150,out=30] (l001) to (l011);

\draw [dashed, thick, blue, in=150,out=30] (l010) to (r000);
\draw [dashed, thick, red, in=150,out=30] (l011) to (r001);

\draw [dashed, thick, blue, in=150,out=30] (r000) to (r010);
\draw [dashed, thick, red, in=150,out=30] (r001) to (r011);


%\node [black] at (0,0.45) {\scriptsize $r$};
%\node [black] at (-1.1,-0.45) {\scriptsize $u_1$};
%\node [black] at (1.1, -0.45) {\scriptsize $u_2$};
%\node [black] at (-2, -1.65) {\scriptsize $u_3$};
%\node [black] at (-.2, -1.65) {\scriptsize $u_4$};
%
%\node [black] at (0.25, -1.65) {\scriptsize $u_5$};
%\node [black] at (2, -1.65) {\scriptsize $u_6$};
%
%%obs labels
%\node [ForestGreen] at (0,-0.6) {\scriptsize $I_3$};
%\node [blue] at (-0.35,-1) {\scriptsize $I_1$};
%\node [red] at (0.35,-1) {\scriptsize $I_2$};

\end{tikzpicture}
\caption{$(\pfr,\alr)$ game}
\label{fig:2-p-shuffle-c}
\end{subfigure}%
\caption{Equivalent two player game obtained by composing $\alr$-spans of respective $\Max$, $\Min$ structures}
\label{fig:2-p-shuffle}
\end{figure}

For a two-player game structure $\Tt$ and the corresponding sequence set $S$, we can look at the projection of sequences on individual player actions, $S_{\Max}$ and $S_{\Min}$, consider their $\alr$-spans and knit them together. %One can show that the projections themselves correspond to one-player game structures $\Tt_{\Max}$ and $\Tt_{\Min}$.
Consider the 2-player game in \cref{fig:2-p-shuffle}. Projections of the sequence set in this game w.r.t individual players would correspond to game structures in \cref{fig:2-p-shuffle-b} (for $\Max$) and \cref{fig:shuffle-a}  (for $\Min$). The structure \cref{fig:2-p-shuffle-b}  already has $\pfr$. As seen before, \cref{fig:shuffle-b} is an $\salr$ witness for \cref{fig:shuffle-a}. We can plug in this $\salr$ witness to all leaf nodes of \cref{fig:2-p-shuffle-b} and get the structure \cref{fig:2-p-shuffle-c}. Information sets are maintained across all copies of \cref{fig:2-p-shuffle-c} as shown in the illustration. More generally, if $\Tt'_{\Max}$ is an $\alr$-span of $\Tt_{\Max}$, and $\Tt'_{\Min}$ is an $\alr$-span of $\Tt_{\Min}$, we can obtain an $(\alr, \alr)$ structure where all nodes of $\Max$ precede all nodes of $\Min$, and a copy of $\Tt'_{\Min}$ is attached to each leaf node of $\Max$, and information sets of $\Min$ are maintained across all copies of $\Tt'_{\Min}$, as shown in \cref{fig:2-p-shuffle}.  
We can assign suitable payoffs in a similar manner to \cref{prop:span-to-game} to get an equivalent game on this new game structure. 


\begin{restatable}{theorem}{twopPfrNam}\label{thm:2p-pfr-name}
Solving an $(\nam, \nam)$ game on structure $\Tt$ can be reduced to solving an $(\alr,\alr)$ game on a structure of size $|\Tt'_{\Max}||\Tt'_{\Min}|$ where $\Tt'_{\Max}$ and $\Tt'_{\Min}$ are minimal $\alr$-spans of $\Tt_{\Min}$.
\end{restatable}
Since $(\pfr, \alr)$-games can be solved in polynomial-time, \cref{thm:2p-pfr-name} and Corollary~\ref{cor:effic-solv-class} lead to new polynomial-time solvable classes.
\begin{corollary}\label{cor:2-effic-solv-class}
The maxmin value in a $(\pfr, \nam)$ game where SD of $\Tt_{\Min}$ is constant can be computed in polynomial time. As a conseqeunce, $(\pfr, \salr)$ games can be solved in polynomial time.
\end{corollary}



\endinput


%%% Local Variables:
%%% mode: latex
%%% TeX-master: "main"
%%% End:
