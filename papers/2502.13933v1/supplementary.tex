%!TEX root = main.tex

~\\~\\
\begin{center}
{\huge{Appendix}}
\end{center}




\section{Shuffled A-loss recall}

\shuffleSameLeafMonomials*
\begin{proof}
Let $L$ and $L'$ be set of leaf nodes of $\Tt$ and $\Tt'$ respectively. From definition of $\salr$, there is a bijection $f$ between the sets $\Hh(L)$ and $\Hh(L')$ such that for $s \in \Hh(L)$, $f(s)$ is a permutation of $s$. Since $\forall s \in \Hh(L),\mu(s) = \mu(f(s))$ it follows that $X(\Tt) = X(\Tt')$. 
\end{proof}


\alrSetStructEquiv*
\begin{proof}
First, starting with an $\alr$-game structure $\Tt$, we will show by induction on the size of $\Tt$, that $\Hh(L)$ has $\alr$. 

When $\Tt$ is a trivial leaf node, we have $\Hh(L) = \{ \epsilon \}$ which has $\alr$ by definition. Suppose the statement holds for all game structure of size at most $k$. 
Let $\Tt$ be a game structure of size $k+1$. First note, that any structure can be reduced to a structure where $\chance$ nodes do not have $\chance$ children, and with the same leaf histories $\Hh(L)$. This can be done by absorbing the $\chance$ children into the parent.

Now we can split into several cases and sub-cases.


Case 1: The root node $r$ of $\Tt$ is a $\chance$ node. When $r$ is a $\chance$ node, there can be two cases.

Case 1a: All children of $r$ (which are player nodes) are in one single information set $I$. In this case, one can construct a structure $\Tt'$ with first two levels of $\Tt$ swapped i.e. the root of $\Tt'$ is a player node in information set $I$ and each actions leading to $\chance$ nodes which again leads to the substructure from third level of $\Tt$. This way, $\Tt$ is reduced to a structure $\Tt'$ with player node at root and $\Hh(\Tt) = \Hh(\Tt')$. This falls in Case 2.

Case 1b: Not all children of $r$ are in single information sets. From definition of $\alr$, substructure rooted at two children of $r$ in two distinct information sets cannot share an information set. This implies, these are substructures of size at most $k$ and by inductive hypothesis, their set of leaf histories have $\alr$. Now since their leaf histories are mutually disconnected, this implies that $\Hh(L)$ has $\alr$.

Case 2: The root node $r$ is a player node.
In this case, for any action $a$ out of root node, and any two node in the sub-tree out of $a$, $a$ is the common prefix of histories of both node. Hence from definition of $\alr$, $\Tt$ has $\alr$ would imply for any $a$, the substructure rooted at each children must individually have $\alr$. By inductive hypothesis the sub-structure has size less than $k+1$ and hence the leaf histories of substructure has $\alr$ as well. Since $\Hh(L)$ is connected in this case, it follows that $\Hh(L)$ has $\alr$ as well. 

~\\
~\\ 
The other direction can be proved similarly, again by using induction on the size of $\Tt$ and breaking into similar cases.  
\end{proof}
\salrDisc*
\begin{proof}
Since $S$ is disconnected from definition of $\salr$ it follows that when $S$ has $\salr$ with $\salr$ witness $S'$, $S'$ is also disconnected, i.e. the maximal connected components of $S'$ are $\salr$ witnesses of maxmimal connected components of $S$. Hence each $S_i$ has $\salr$.

The other direction follows since the union of the $\salr$ witnesses for connected components of $S$ is an $\salr$ witness of $S$. 
\end{proof}
\commonact*
\begin{proof}
For the forward direction, let $S'$ be an $\salr$ witness of $S$. Since $S$ is connected, it follows that $S'$ is connected as well. Now from definition of $\alr$ sets, $\exists I$, such that all sequences in $S'$ starts with actions from $\act(I)$. This proves the first statement. 
Now again it follows from the definition of $\alr$ sets that for each $a \in \act(I)$, $S'_a$ has $\alr$. Moreover $S'_a$ is $\salr$ witness of $S_a$. This proves the second statement. 

For the other direction $\bigcup_a aS'_a$ will have $\alr$ by definition and hence is an $\salr$ witness of $S$, implying $S$ has $\salr$.   
\end{proof}
\salrPtime*
\begin{proof}
Suppose $S$ has $\salr$ and $I$ be an information set such that all sequences in $S$ has actions from $\act(I)$. Let $S'$ be an $\salr$ witness of $S$. We claim that for each $a \in \act(I)$, $S'_a$ will have $\alr$. 

For this we provide an equivalent formulation of $\alr$ sets. A set has $\alr$ iff every subset of size $2$ has $\alr$. This is true because $\Tt$ has $\alr$ iff every pair of leaf histories $s_1$ and $s_2$ of $\Tt$ satisfy the constraints imposed by $\alr$ condition, i.e. either (i) $s_1 = sa_1s'_1,~ s_2 = sa_2s'_2$ for $a_1 \neq a_2$ with $a_1,a_2 \in \act(I')$ for some $I'$ or (ii)  $s_1 = ss'_1,~ s_2 = ss'_2$ where $s'_1$, $s'_2$ are disconnected. Now it follows that if $s_1$ and $s_2$ share a common action $a$, the pair of sequences on removing this common $a$ from both will still satisfy the $\alr$ conditions. This implies that $S'_a$ will also have $\alr$ for each $a$. As a result for each $a$, $S_a$ will have $\salr$.

For the other direction $\bigcup_a aS'_a$ will have $\alr$ by definition and hence is an $\salr$ witness of $S$.  
\end{proof}
\onepShufflePtime*
\begin{proof}
Since $\salr$ witnesses are specific kinds of $\alr$-spans, using the construction in the proof of \cref{prop:span-to-game}, we can reduce the input game $G$ to a new game $G'$ with $\alr$ and with the same payoff polynomial. Since game structures with $\salr$ have SD $0$, this follows from \cref{cor:effic-solv-class}. 
\end{proof}
\section{Span}

\subsection{Existence of $\alr$-span}

%!TEX root = ../main.tex


\begin{figure}  
\centering
%--------------------1sy subfigure
\begin{subfigure}{0.5\columnwidth}
\centering
\tikzset{
triangle/.style = {regular polygon,regular polygon sides=3,draw,inner sep = 2},
circ/.style = {circle,fill=cyan!10,draw,inner sep = 3},
term/.style = {circle,draw,inner sep = 1.5,fill=black},
sq/.style = {rectangle,fill=gray!20, draw, inner sep = 4}
}

\begin{tikzpicture}[scale=0.7]
\tikzstyle{level 1}=[level distance=9mm,sibling distance = 14mm]
\tikzstyle{level 2}=[level distance=7mm,sibling distance=8mm]
\tikzstyle{level 3}=[level distance=7mm,sibling distance=6mm]
\tikzstyle{level 4}=[level distance=7mm,sibling distance=5mm]

%node (ij) is the j th node in i th level

\begin{scope}[->, >=stealth]
\node (0) [triangle] {}
child {
  node (00) [circ] {}
    child {
      node (000) [term, label=below:{}] {}
      edge from parent node [left] {\scriptsize $c$}
    }
    child {
      node (001) [term, label=below:{}] {}
      edge from parent node [right] {\scriptsize $\bar{c}$}
  }
  edge from parent node [above] {}
}
%------
child {
  node (01) [circ] {}
    child {
      node (010) [term, label=below:{}] {}
      edge from parent node [left] {\scriptsize $b$}
    }
    child {
      node (011) [term, label=below:{}] {}
      edge from parent node [right] {\scriptsize $\bar{b}$}
  }
  edge from parent node [above] {}
}
%---
child {
  node (02) [circ] {}
   child {
     node (020) [circ] {}
      child {
        node (0200) [circ] {}
          child {
            node (02000) [term, label=below:{}] {}
            edge from parent node [left] {\scriptsize $c$}
          }
          child {
            node (02001) [term, label=below:{}] {}
            edge from parent node [right] {\scriptsize $\bar{c}$}
          }
  %edge from parent node [above] {}
        edge from parent node [left] {\scriptsize $b$}
      }
      child {
        node (0201) [term, label=below:{}] {}
        edge from parent node [right] {\scriptsize $\bar{b}$}
      }
    edge from parent node [left] {\scriptsize $a$}
  }
  child {
    node (021) [term, label=below:{}] {}
    edge from parent node [right] {\scriptsize $\bar{a}$} 
  }
  edge from parent node [above] {}
}
%---
child {
  node (03) [circ] {}
    child {
      node (030) [term, label=below:{}] {}
      edge from parent node [left] {\scriptsize $a$}
    }
    child {
      node (031) [term, label=below:{}] {}
      edge from parent node [right] {\scriptsize $\bar{a}$}
  }
  edge from parent node [above] {}
}
;
\end{scope}

\draw [dashed, thick, red, in=160,out=20] (02) to (03);
\draw [dashed, thick, blue, in=160,out=20] (00) to (0200);
\draw [dashed, thick, ForestGreen, in=160,out=20] (01) to (020);

\end{tikzpicture}
\caption{}
\label{fig:app-alossSpan-a}
\end{subfigure}%
\begin{subfigure}{0.5\columnwidth}
\centering
\tikzset{
triangle/.style = {regular polygon,regular polygon sides=3,draw,inner sep = 2},
circ/.style = {circle,fill=cyan!10,draw,inner sep = 3},
term/.style = {circle,draw,inner sep = 1.5,fill=black},
sq/.style = {rectangle,fill=gray!20, draw, inner sep = 4}
}

\begin{tikzpicture}[scale=0.75]
\tikzstyle{level 1}=[level distance=9mm,sibling distance = 14mm]
\tikzstyle{level 2}=[level distance=7mm,sibling distance=8mm]
\tikzstyle{level 3}=[level distance=7mm,sibling distance=6mm]
\tikzstyle{level 4}=[level distance=7mm,sibling distance=5mm]

%node (ij) is the j th node in i th level

\begin{scope}[->, >=stealth]
\node (0) [triangle] {}
child {
  node (00) [circ] {}
    child {
      node (000) [term, label=below:{\scriptsize $1$} ] {}
      edge from parent node [left] {\scriptsize $c$}
    }
    child {
      node (001) [term, label=below:{\scriptsize $0$} ] {}
      edge from parent node [right] {\scriptsize $\bar{c}$}
  }
  edge from parent node [above] {\scriptsize $\frac{1}{4}$}
}
%------
child {
  node (01) [circ] {}
    child {
      node (010) [term, label=below:{\scriptsize $1$} ] {}
      edge from parent node [left] {\scriptsize $b$}
    }
    child {
      node (011) [term, label=below:{\scriptsize $0$} ] {}
      edge from parent node [right] {\scriptsize $\bar{b}$}
  }
  edge from parent node [left] {\scriptsize $\frac{1}{4}$}
}
%---
child {
  node (02) [circ] {}
   child {
     node (020) [circ] {}
      child {
        node (0200) [circ] {}
          child {
            node (02000) [term, label=below:{\scriptsize $1$} ] {}
            edge from parent node [left] {\scriptsize $c$}
          }
          child {
            node (02001) [term, label=below:{\scriptsize $0$} ] {}
            edge from parent node [right] {\scriptsize $\bar{c}$}
          }
  %edge from parent node [above] {}
        edge from parent node [left] {\scriptsize $b$}
      }
      child {
        node (0201) [term, label=below:{\scriptsize $0$} ] {}
        edge from parent node [right] {\scriptsize $\bar{b}$}
      }
    edge from parent node [left] {\scriptsize $a$}
  }
  child {
    node (021) [term, label=below:{\scriptsize $0$} ] {}
    edge from parent node [right] {\scriptsize $\bar{a}$} 
  }
  edge from parent node [right] {\scriptsize $\frac{1}{4}$}
}
%---
child {
  node (03) [circ] {}
    child {
      node (030) [term, label=below:{\scriptsize $1$} ] {}
      edge from parent node [left] {\scriptsize $a$}
    }
    child {
      node (031) [term, label=below:{\scriptsize $0$} ] {}
      edge from parent node [right] {\scriptsize $\bar{a}$}
  }
  edge from parent node [above] {\scriptsize $\frac{1}{4}$}
}
;
\end{scope}

\draw [dashed, thick, red, in=160,out=20] (02) to (03);
\draw [dashed, thick, blue, in=160,out=20] (00) to (0200);
\draw [dashed, thick, ForestGreen, in=160,out=20] (01) to (020);

\end{tikzpicture}
\caption{}
\label{fig:app-alossSpan-b}
\end{subfigure}%
%--------------------------2nd figure

\begin{subfigure}{.45\columnwidth}
\centering
\tikzset{
triangle/.style = {regular polygon,regular polygon sides=3,draw,inner sep = 2},
circ/.style = {circle,fill=cyan!10,draw,inner sep = 3},
term/.style = {circle,draw,inner sep = 1.5,fill=black},
sq/.style = {rectangle,fill=gray!20, draw, inner sep = 4}
}

\begin{tikzpicture}[scale=0.9]
\tikzstyle{level 1}=[level distance=7mm,sibling distance = 18mm]
\tikzstyle{level 2}=[level distance=5mm,sibling distance=9mm]
\tikzstyle{level 3}=[level distance=5mm,sibling distance=5mm]
\tikzstyle{level 4}=[level distance=7mm,sibling distance=2mm]


\begin{scope}[->, >=stealth]
\node (0) [circ]{}
child {
  node (00) [circ] {}
    child{
      node (000) [circ] {}
      child{
      node (0000) [term, label=below:{}] {}
      edge from parent node [left] {\scriptsize $c$}
      }
      child{
      node (0001) [term, label=below:{}] {}
      edge from parent node [right] {\scriptsize $\bar{c}$}
      } 
      edge from parent node [left] {\scriptsize $b$}
    }
    child{
      node (001) [circ] {}
      child{
      node (0010) [term, label=below:{}] {}
      edge from parent node [left] {\scriptsize $c$}
      }
      child{
      node (0011) [term, label=below:{}] {}
      edge from parent node [right] {\scriptsize $\bar{c}$}
      }
      edge from parent node [right] {\scriptsize $\bar{b}$}
    }
    edge from parent node [left] {\scriptsize $a$}
  }%--------------
  child {
    node (01) [circ] {}
    child{
      node (010) [circ] {}
      child{
      node (0100) [term, label=below:{}] {}
      edge from parent node [left] {\scriptsize $c$}
      }
      child{
      node (0101) [term, label=below:{}] {}
      edge from parent node [right] {\scriptsize $\bar{c}$}
      } 
      edge from parent node [left] {\scriptsize $b$}
    }
    child{
      node (011) [circ] {}
      child{
      node (0110) [term, label=below:{}] {}
      edge from parent node [left] {\scriptsize $c$}
      }
      child{
      node (0111) [term, label=below:{}] {}
      edge from parent node [right] {\scriptsize $\bar{c}$}
      }
      edge from parent node [right] {\scriptsize $\bar{b}$}
    }
    edge from parent node [right] {\scriptsize $\bar{a}$}
  }
;
\end{scope}

\node[fit=(0),dashed,thick,red, draw, circle,inner sep=1pt] {};
\draw [dashed, thick, ForestGreen, in=160,out=20] (00) to (01);
\draw [dashed, thick, blue, in=160,out=20] (000) to (001) to (010) to (011);
%\draw [dashed, thick, ForestGreen, in=190,out=-20] (001) to (011);

\end{tikzpicture}
\caption{}
\label{fig:app-alossSpan-c}
\end{subfigure}%~\\
\begin{subfigure}{.45\columnwidth}
\centering
\tikzset{
triangle/.style = {regular polygon,regular polygon sides=3,draw,inner sep = 2},
circ/.style = {circle,fill=cyan!10,draw,inner sep = 3},
term/.style = {circle,draw,inner sep = 1.5,fill=black},
sq/.style = {rectangle,fill=gray!20, draw, inner sep = 4}
}

\begin{tikzpicture}[scale=0.9]
\tikzstyle{level 1}=[level distance=7mm,sibling distance = 18mm]
\tikzstyle{level 2}=[level distance=5mm,sibling distance=9mm]
\tikzstyle{level 3}=[level distance=5mm,sibling distance=5mm]
\tikzstyle{level 4}=[level distance=7mm,sibling distance=2mm]


\begin{scope}[->, >=stealth]
\node (0) [circ]{}
child {
  node (00) [circ] {}
    child{
      node (000) [circ] {}
      child{
      node (0000) [term, label=below:{\scriptsize $1$} ] {}
      edge from parent node [left] {\scriptsize $c$}
      }
      child{
      node (0001) [term, label=below:{\scriptsize $\frac{1}{2}$} ] {}
      edge from parent node [right] {\scriptsize $\bar{c}$}
      } 
      edge from parent node [left] {\scriptsize $b$}
    }
    child{
      node (001) [circ] {}
      child{
      node (0010) [term, label=below:{\scriptsize $\frac{1}{2}$} ] {}
      edge from parent node [left] {\scriptsize $c$}
      }
      child{
      node (0011) [term, label=below:{\scriptsize $\frac{1}{4}$} ] {}
      edge from parent node [right] {\scriptsize $\bar{c}$}
      }
      edge from parent node [right] {\scriptsize $\bar{b}$}
    }
    edge from parent node [left] {\scriptsize $a$}
  }%--------------
  child {
    node (01) [circ] {}
    child{
      node (010) [circ] {}
      child{
      node (0100) [term, label=below:{\scriptsize $\frac{1}{2}$} ] {}
      edge from parent node [left] {\scriptsize $c$}
      }
      child{
      node (0101) [term, label=below:{\scriptsize $\frac{1}{4}$} ] {}
      edge from parent node [right] {\scriptsize $\bar{c}$}
      } 
      edge from parent node [left] {\scriptsize $b$}
    }
    child{
      node (011) [circ] {}
      child{
      node (0110) [term, label=below:{\scriptsize $\frac{1}{4}$} ] {}
      edge from parent node [left] {\scriptsize $c$}
      }
      child{
      node (0111) [term, label=below:{\scriptsize $0$} ] {}
      edge from parent node [right] {\scriptsize $\bar{c}$}
      }
      edge from parent node [right] {\scriptsize $\bar{b}$}
    }
    edge from parent node [right] {\scriptsize $\bar{a}$}
  }
;
\end{scope}

\node[fit=(0),dashed,thick,red, draw, circle,inner sep=1pt] {};
\draw [dashed, thick, ForestGreen, in=160,out=20] (00) to (01);
\draw [dashed, thick, blue, in=160,out=20] (000) to (001) to (010) to (011);
%\draw [dashed, thick, ForestGreen, in=190,out=-20] (001) to (011);

\end{tikzpicture}
\caption{}
\label{fig:app-alossSpan-d}
\end{subfigure}%~\\

%----------------
\caption{A-loss recall span}
\label{fig:app-alossSpan}
\end{figure}

\alrspan*

\begin{proof} Suppose $\Tt$ has information sets $\Ii = \{ I_1, \dots, \I_n \}$ and actions $A$. We will provide an explicit $\alr$-span  $\Tt'$ over the same information sets $\Ii$ and actions $A$.
An example of this construction for $n=3$ can be found in \cref{fig:app-alossSpan}, where the game structure in \cref{fig:app-alossSpan-c} is an $\alr$-span of the structure in \cref{fig:app-alossSpan-a} obtained in this manner. 
 
Let $S = \{\epsilon\} \cup  \{a_{i_1}\dots a_{i_k} \mid k \leq n, \forall j~ a_{i_j} \in \act(I_j) \}$. $\Tt'$ be the game structure over the vertex set $V' = \{u_s | s \in S\}$, with $v_{\epsilon}$ being the root node and the set of leaves $L' = \{ v_s \mid \lvert s \rvert = n \}$. 
The set of actions is $A$ where for $u_s \in V'\setm L'$, $u_s \xra{a} u_{sa}$. The set of information sets is $\Ii$ and for each $ 1 \leq i \leq n, I_i = \{ u_s \mid  \lvert s \rvert = i-1\}$.

We claim that $\Tt'$ is an $\alr$-span of $\Tt$. Firstly $\Tt'$ has $\alr$ because it follows from the recursive definition of $\alr$-sequence sets that $S$ has $\alr$.
Now let $\mu(s)$ be a monomial generated by some leaf sequence $s$ in $\Tt$. We will show that the leaf monomials of the set $\{ s' \in \Hh(L') \mid \act(s) \incl \act(s') \}$ generate the monomial $\mu(s)$ where $\act(s)$ is the set of actions in $s$. More particularly the expression $\sum\limits_{s' \in \Hh(L') \mid \act(s) \incl \act(s')} \mu(s')$ reduces to $\mu(s)$. This is because the previous expression can be re-written as

 $\mu(s)\prod\limits_{I \mid \act(I) \cap \act(s) = \emps}(\sum_{a \in \act(I)} x_a)$ which reduces to $\mu(s)$ by repeated applications of $\sum_{a \in \act{I}}x_a = 1$ for $I$ with $ \act(I) \cap \act(s) = \emps$. Hence $\Tt'$ is an $\alr$-span of $\Tt$.


~\\ 
\end{proof}

\spanToGame*

\begin{proof}
We will provide $\d'$ and $\Uu'$ to construct the game $G'$ on $\Tt'$. 

Firstly, we choose $\d'$ which at every $\chance$ node assigns an uniform distribution on its outgoing edges(any distribution with full support would work). We will factor out these $\chance$ probabilities later in the payoff function $\Uu'$.   
For a leaf $t \in L$, it follows from definition of span that a linear combination of $\mu(t')$ for $t' \in \Tt'$, reduces to $\mu(t)$. Let that combination be $\sum_{t' \in L'}c^t_{t'}\mu(t')$. Recall that $\prob_{\chance}(t')$ is the product of $\chance$ probabilities by $\d'$ on path to $t'$ in $\Tt'$.  Define $\Uu'(t') = \frac{\sum_{t \in L}\prob_{\chance}(t)c^t_{t'}\Uu(t)}{\prob_{\chance}(t')} $ and let $G'$ be the resulting game. Observe that payoff polynomial of $G'$ turns out to be 
$\sum\limits_{t' \in L'}\mu(t')\sum_{t \in L}\prob_{\chance}(t)c^t_{t'}\Uu(t)$. By rearranging terms, this is equivalent to $\sum\limits_{t \in L} (\prob_{\chance}(t)\Uu(t))\sum_{t' \in L'}c^t_{t'}\mu(t')$. It follows that this reduces to the payoff polynomial of $G$.
\end{proof}

\subsection{Minimal $\alr$-span }

\minSpanDisc*

\begin{proof}
Firstly we can see that $S'$ is indeed an $\alr$-span of $S$. Also for any distinct $i$ and $j$, since $S_i$ and $S_j$ are not connected, it follows that $S'_i$ and $S'j$ are not connected as well. 

Let us assume that there is an $\alr$-span $\hat{S}$ of $S$ such that $ \mid  \hat{S} \mid   <  \mid  S' \mid   = \sum\limits_i  \mid  S_i' \mid  $. For each $i$, let $\hat{S}_i$ be a minimal subset of $\hat{S}$ that spans $S_i$. Again from the similar previous argument, for any distinct $i$ and $j$, $\hat{S}_i$ and $\hat{S}_j$ are not connected. Since $ \mid  \hat{S} \mid  = \sum\limits_i  \mid  \hat{S}_i \mid  $, $\exists i$ such that $ \mid\hat{S}_i\mid   <  \mid S_i'\mid  $. But this  contradicts the minimality of $S'_i$ for each $S_i$. This completes the proof.  

\end{proof}

\spanFirstInfo*

\begin{proof}
Given a connected sequence set $S$, any minimal $\alr$-span $S'$ of $S$ is also connected. This is because if one could find a decomposition $S' = \uplus_i S_i'$, such that $S_i'$'s are disconnected, then the subsets of $S$ individually spanned by each $S'_i$ would be disconnected as well leading to a contradiction. 


Since an $\alr$-span is a set with $\alr$, it follows from definition of $\alr$-sets that for some $I$, all sequences in $S'$ must start with an action from $\act(I)$. 
\end{proof}



\spanAfterFixingInfo*

\begin{proof}
Suppose there is an $\alr$-span $\hat{S}$ of $S$ starting with $\act(I)$ such that $\lvert \hat{S} \rvert < \lvert S' \rvert$. 
Observe that $\hat{S} = \uplus_{a \in \act(I)}a\hat{S}_a$. 
We will show that for each $a$, $\hat{S}_a$ is an $\alr$-span of $S_a \cup S_{\bar{I}}$. Since $\lvert \hat{S} \rvert = \sum_{a} \lvert \hat{S}_a \rvert$ and $\lvert S' \rvert = \sum_{a} \lvert H'_a \rvert$ this would contradict the minimality of some $H'_a$. 


Since $\hat{S}$ is a span of $S$, for any $s \in S_a$, $\mu(as) = x_a\mu(s)$ is obtained by reduction from $\sum_{a \in \act(I)}x_af_a$ where $f_a$ is combination of monomials of sequences with $a$. Substituting $x_a$ with $0$ for every term containing $x_a$ we get an expression which reduces to $0$. This implies the expression $f_a$ reduces to $\mu(s)$ itself. Again for $s \in S_{\bar{I}}$ substituting for each $a \in \act(I)$, $x_a$ by $1$ we get an expression devoid of $x_a$ for $a \in \act(I)$ that reduces to $\mu(s)$. 
These together implies that $\hat{S}_a$ is an $\alr$-span of $S_a \cup S_{\bar{I}}$.
This completes the proof.  
\end{proof}

\spanFindFirstInfo*

\begin{proof}
We can show this by induction on the size of $\Ii$. This is trivially true when $\lvert \Ii \rvert = 1$. Let this be true when number of information sets is at most $k$. Now if we have a sequence set $S$ over $k+1$ information sets, since $S$ is connected it follows from \cref{lem:span-first-I} that in a minimal span $S'$ all sequences start with $\act(I')$ for some $I'$. Moreover for each $a \in \act(I')$, it follows from \cref{lem:span-after-fixing-an-I} that $S'_a$ is a minimal $\alr$ span of $S_a \cup S_{\bar{I'}}$. Now by inductive hypothesis since every $S_a \cup S_{\bar{I'}}$ has actions from $I$, we can assume that $S'_a$ starts with actions from $I$. Finally one can swap the order of actions in each sequence, so that actions from $I$ appear at the start. The resulting set will still have $\alr$ because every pair of sequences satisfy the conditions put down in the proof of \cref{lem:salr-common-act}. Since the resulting set still remains a span of $S$, this completes the proof.  
\end{proof}



We provide the algorithm for computing minimal $\alr$-span described in the main text. 
\begin{algorithm}[H]
\caption{Compute minimal $\alr$-span}
\label{algo:min-span}
\begin{algorithmic}[1]
\STATE \textbf{Input} : $S$ 
\STATE \textbf{Output}: minimal $\alr$-span $S'$ of $S$ 
\IF{$S$ is connected} 
\IF{$\exists I$ such that every $s \in S$ contains an action from
  $\act(I)$ \label{algo:min-span-connected}} \FOR {$a \in \act(I)$}
\STATE $S_a' \gets $ minimal $\alr$-span of $S_a$\ENDFOR
\RETURN $\bigcup_{a} aS_a'$ 
\ELSE 
\FOR {$I \in \Ii$}
\STATE $S_{\bar{I}} \gets \{ s \in S \mid s \text{ contains no action from } \act(I) \}$
\FOR {$a \in \act(I)$}
\STATE $S_a' \gets $ minimal $\alr$-span of $S_a \cup S_{\bar{I}}$
\ENDFOR
\STATE $\hat{S}_I \gets \bigcup_{a \in \act(I)} S'_a$
\ENDFOR
\RETURN $\hat{S}_I$ that satsfies $\min_{I \in \Ii}  \lvert \hat{S}_I \rvert $ \label{algo:min-span-all}
\ENDIF
\ELSE
\STATE $S = \uplus S_i$ where each $S_i$ is connected 
\STATE $S'_i \gets $ minimal $\alr$-span of $S_i$ \label{algo:min-span-disconnected}
\RETURN $\cup_i S'_i$
\ENDIF
\end{algorithmic} 
\end{algorithm}



As mentioned in the main text, at every step if $\epsilon$ is also in the set we get rid of it. This doesn't affect the computation of the minimal $\alr$-span. This is because for any $\alr$ structure $\Tt$, $X(\Tt)$ can always generate $\epsilon$. This can be show using the existence of ``strongly branching'' (see \cref{lem:strong-branch}) subsets in $\Hh(L_{\Tt})$. In \cref{algo:min-span}, at every recursive call for computing minimal $\alr$-span, the returned structure will have $\alr$ and hence will generate $\epsilon$ anyway. 

\spanLowerBound*

\begin{proof}
  For $n > 0$, let $\Ii_n = \{I_1,\dots,\I_n\}$, $\act(I_i) = \{a^i,b^i\}$ and $A_n = \bigcup_i \act(I_i)$. Define $S_n = \{a^ib^j \mid i < j\} \cup \{b^ia^j \mid i < j\} \cup \{a^ia^j \mid i < j\} \cup \{b^ib^j \mid i < j\} $. Let $\hat{S}_n =  A_n \cup
S_n$. We claim that the class of game structures with sequence set $\hat{S}_n$ has minimal $\alr$-span of size $\Omega(2^n)$. 


  Since $A_n \cup S_n$ is connected, according to \cref{lem:span-first-I}, a minimal span must start with actions from $\act(I)$ for some $I$. Now in order to choose an $I$ and apply \cref{lem:span-after-fixing-an-I}, 
we will observe that the set of monomials of $S_n$ is symmetric w.r.t each $I$ and the choice of $I$ doesn't make a difference. For a set $S$, observe that for any $I \in \Ii_n$, the leaf monomials of the set $\{s \in A_n \cup S_n  \mid  \act(I) \cap \act(s) = \emptyset\}$ and that of the set $S_{n-1} \cup A_{n-1}$ are same up to renaming of variables. 
Also, for any $I$ and for any $a \in \act(I)$, the leaf monomials of the set $\{ ss'  \mid  sas' \in A_n \cup S_n \} $ and the set $A_{n-1} \cup \{ \epsilon \}$ are again same upto renaming of variables. 
Hence for any choice of $I$ from $\Ii_n$, applying \cref{lem:span-after-fixing-an-I} yields a minimal $\alr$-span.


Following this argument if we pick $I_n$, since $(S_{n-1} \cup A_{n-1}) \cup (A_{n-1} \cup \{ \epsilon \}) = \hat{S}_{n-1}$ (ignoring the $\epsilon$) we are then left with finding the minimal $\alr$-span of $\hat{S}_{n-1}$. 
Let $T(n)$ be the size of a minimal $\alr$-span of $\hat{S}_n$. Then from \cref{lem:span-after-fixing-an-I} it follows $T(n) = 2T(n-1)$. 
Since for $n=1$, the minimal $\alr$-span has size $2$, it follows that a minimal $\alr$-span of $\hat{S}_n$ has size $\Omega(2^n)$. 




\end{proof}


\spanNP*

To prove \cref{thm:span-NP} we first define something called a \emph{strongly branching set} recursively.
The trivial set $S = \{ \epsilon \}$ is a strongly branching set. $S$ is a strongly branching set if there exists $I \in \Ii$ such that there is a partition of $S = \uplus_{a_i \in \act(I)} a_iS_i $ such that (i) $a_iS_i$s are non-empty (ii) each of $S_i$ are strongly branching sets.

We prove the following lemma, that we will use in our proof. 
\begin{lemma}\label{lem:strong-branch}
Let $S$ be a sequence set with $\alr$. Then $\sum\limits_{s \in S}\mu(s) = 1$  under strategy constraints iff $S$ is a strongly branching set.
\end{lemma}
\begin{proof}
First we will prove the backward direction by induction on the number of actions. Observe that any strongly branching set always has $\alr$, hence $\alr$ assumption is redundant in this case.  When $S = \{ \epsilon \}$ recall that $\mu( \epsilon ) = 1$. Now for any non-trivial strongly branching set $S$, from definition we have $S = \uplus_{a_i \in \act(I)} a_iS_i$ where $S_i$s are strongly branching. We have $\sum\limits_{s \in S}\mu(s) = \sum_{a_i} x_{a_i}(\sum\limits_{s' \in S_i}\mu(s'))$. By induction hypothesis the statement holds for each $S_i$ and hence this equals 1. Hence this extends to $S$ as well. 

For the the forward direction we will again use induction on the number of actions. The statement is true when $A = \emps$. Now suppose $S$ be a sequence set over non-trivial action set.

For any set $S'$, if $\sum\limits_{s \in S'}\mu(s)$ is a constant value then this value is at least 1 due to the fact that any $\mu(s)$ has either no constant terms or 1 in its full expansion.
Observe that $S$ is connected.Otherwise, if $S$ has at least two connected components $S'$ and $S''$, since $\mu(S')$ and $\mu(S'')$ have no variable in common, $\sum\limits_{s \in S}\mu(s)$ would become strictly more than 1.


 Since $S$ is connected and has $\alr$, it follows that $S= \uplus_{a_i \in \act(I)} a_iS_i $ for some $I$ where each $S_i$ has $\alr$. Let $Z_i = \sum\limits_{s' \in S_i}\mu(s') $. Then we have $\sum\limits_{s \in S}\mu(s) = \sum_{a_i} x_{a_i}Z_i = 1$, which implies each of $Z_i$s must equal 1. Hence by induction hypothesis each of $S_i$s are strongly branching sets and as a result $S$ is also a strongly branching set. 

\end{proof}
\begin{proof}[Proof of \cref{thm:span-NP}]
Given a structure $\Tt$, we will guess an $\alr$-span $\Tt'$ of size at most $k$ along with the linear combinations for each leaf monomial. Since from \cref{algo:min-span} we can get the minimal span and all the co-efficient needed lies in $\{0,1\}$, we can restrict to this kind of combinations. Hence, this is of size at most $k|\Tt|$.  

Firstly, checking if $\Tt'$ has $\alr$ can be done efficiently by traversing the tree from root.
Now, to verify if each for $s \in \Hh(L)$, the linear combination for generating $\mu(s)$, is correct, we just need to consider all $s' \in \Hh(L')$ that has all actions from $s$,
i.e. we consider the set $S_s = \{ s' \in \Hh(L') \mid \act(s) \incl \act(s') \}$ where $\act(s)$ is the set of actions in $s$. This is because monomials of sequences not containing all actions form $\act(s)$, will either vanish (thus not affecting the final outcome) or not reduce to $\mu(s)$.

 Hence, if $\sum \mu'$ is the candidate linear combination for $\mu$,  $\sum \mu'$ reduces to $\mu$ iff the $\sum \frac{\mu'}{\mu} = 1$. Let $\frac{s'}{s}$ denote the sequence $s'$ restricted to actions from $\act(s')\setminus \act(s)$. But since $S_s$ is assumed to have $\alr$, this set will also have $\alr$. Hence from \cref{lem:strong-branch}, we need to check if the set $\{\frac{s'}{s} | s' \in S_s\}$ has a strongly branching subset. This can be checked recursively in polynomial time.  
\end{proof}




\efficSolvClass*

\begin{proof}
Since $\Tt$ has SD $k$, the size of the minimal $\alr$-span $\Tt'$ of $\Tt$ is at most $|\Tt|^{k+1}$. Now given a game $G = (\Tt, \d, \Uu)$ on $\Tt$, consider the game $G' = (\Tt', \d', \Uu')$ obtained using the construction in \cref{prop:span-to-game}. It follows from \cref{prop:span-to-game} that solving $G$ can be reduced to solving $G'$. Since $\d'$ assigns uniform distribution at every chance nodes, probabilities in $\d'$ need at most $\Oo(\log(|\Tt|^{k+1}))$ bits. It also follows from \cref{algo:min-span} that the coefficients $c^t_{t'}$ in the proof of \cref{prop:span-to-game} can be either $1$ or $0$ for the minimal $\alr$-span. Hence each payoff in $\Uu'$ uses $\Oo(\log{(U|\Tt|^{k+1})})$ bits where $U$ is the maximum possible payoff in $\Uu$.
Since $k$ is constant, hence size of $G'$ is polynomial in the size of $G$. 
\end{proof}


\section{Extension to Two-player games}




\begin{lemma}
Let $\Tt$ be a two player game structure with sequence set $S$ and let $S_{\Max}$ and $S_{\Max}$ be the projection of $S$ onto actions of $\Max$ and $\Min$ respectively. Then, there exists game structures $\Tt_{\Min}$, $\Tt_{\Min}$ such that $S_{\Max}$(resp $S_{\Min}$) is the sequence set of $\Tt_{\Max}$(resp. $\Tt_{\Min}$).
\end{lemma}
\begin{proof}
Fix a behavioral strategy for $\Min$ and this makes the structure a one-player structure with $\Max$ player. The game structure of this game, $\Tt_{\Max}$ is a structure such that the sequence set of $\Tt_{\Max}$ is $S_{\Max}$. 
This works for $\Min$ in a similar way. 
\end{proof}
\twopPfrNam*
\begin{proof}
Given $\Tt_{\Max}'$ and $\Tt_{\Min}'$, the $\alr$ spans of $\Tt_{\Max}$ and $\Tt_{\Min}$, we construct $\Tt'$ as follows : 
the top part of $\Tt'$ is the game structure $\Tt_{\Max}'$ and to each leaf of $\Tt_{\Max}'$ a copy of $\Tt_{\Min}'$ is attached. Nodes in two different copies of $\Tt_{\Min}'$ that are in the same information set in $\Tt_{\Min}'$ are also in same information set of $\Tt'$. 

Now $\Tt'$ is a structure of size $\lvert \Tt_{\Max}' \rvert \lvert \Tt_{\Min}' \rvert$. We will show that $\Tt'$ spans $\Tt$, i.e. it satisfies the 2nd condition in the definition of $\alr$ span. Once we have shown this, the proof will follow using the same construction in \cref{prop:span-to-game}.

Any leaf history in $\Tt'$ will have a $\Max$ part followed by a $\Min$ part, i.e. for $t' \in L'$, $\mu(t') = \mu_{\Max}(t')\mu_{\Min}(t')$. This is true for $\Tt$ as well. Since $\Tt_{\Max}'$ and $\Tt_{\Min}'$ are $\alr$ spans of $\Tt_{\Max}$ and $\Tt_{\Min}$, for any $t \in L$, for $\mu_{\Max}(t)$ we have witness linear combination $\sum_{t'}c^t_{t'}\mu_{\Max}(t')$. The same holds for $\Min$. Hence we have $\mu(t) = \mu_{\Max}(t)\mu_{\Min}(t) = f_{\Max} f_{\Min}$ where $f_i$ is linear combination of monomials from $\Tt'_i$. Expanding this, it follows that $\sum_{t' \in L'}c^t_{t'}\mu(t')$ reduces to any $\mu(t)$. This completes the proof.     
\end{proof}


\endinput
\section*{Running example}

\begin{proposition}
For every $n > 2$, the following statements are true:
\begin{itemize}
    \item The game \textbf{I} has $\alr$.
    \item The game \textbf{I} doesn't have $\alr$, but has $\salr$.
    \item The game \textbf{III} doesn't have $\salr$ but has SD = 2. 
\end{itemize}  
\end{proposition}
\begin{proof}

\end{proof}

\endinput
\section*{(In)Expressiveness of $\pfr$-spans}

We show in \cref{thm:existence-alr-span} that any $\nam$-game structure has an $\alr$-span. We can naturally extend the definitions of $\salr$ and $\alr$-span to define shuffled-$\pfr$ and $\pfr$-spans. 
Then, a natural question to ask would be, if some class of imperfect recall game structures can be restructured into perfect recall game structures. We provide a definitive answer to this question by showing that this is not possible.
To show this, we first show that any $\nam$ game structure having a $\pfr$-span must have shuffled $\pfr$. Following which, one can observe that a $\nam$ game structure with imperfect recall cannot have shuffled-$\pfr$. \red{If any space is left, could provide one proof to why this is easy with example}
\begin{proposition}
Let $\Tt$ be a game structure with imperfect recall. Then $\Tt$ cannot have a $\pfr$-span. 
\end{proposition}
Similar to $\alr$-game trees we observe the recursive structure rulebooks from one-player $\pfr$ game trees. 
\begin{proposition}[Structure of $\pfr$-game trees]
Let $\Tt$ be a game tree with rulebook $S$.  Then $\Tt$ is $\pfr$-game tree if and only if exactly one of the following is tree
\begin{itemize}
  \item $S$ is disconnected and $S=\biguplus\limits_{i}S_i$ is a decomposition of $S$ into connected components where each $S_i$ is a  $\pfr$-set.
  \item $S$ is connected and for some $I \in \Ii$, $S=\biguplus\limits_{a_i \in Act(I)}a_iS_i$ such that each $S_i$ is a $\pfr$-set and $\forall i,j$ $S_i$ and $S_j$ are mutually disconnected. 
\end{itemize}
\end{proposition}

\begin{proposition}
Let $\Tt$ be game tree with rulebook $S$ such that $\Tt$ has some $\pfr$-span. Then there is a $\pfr$-game tree $\Tt'$ such that $\Tt'$ is a $\pfr$-shuffle of $\Tt$.
\end{proposition}

\begin{proposition}
Let $\Tt$ be game tree with rulebook $S$ such that $\Tt$ doesn't have $\pfr$. Then $\Tt$ cannot have a $\pfr$-shuffle. 
\end{proposition}

\begin{corollary}
A $(\nam,\nam)$-game tree $\Tt$ which is not $(\pfr,\nam)$ cannot have a $(\pfr,\nam)$-span. 
\end{corollary}
