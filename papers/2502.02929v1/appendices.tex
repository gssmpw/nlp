\documentclass[manuscript,review,anonymous]{acmart}
\PassOptionsToPackage{prologue,dvipsnames}{xcolor}
% note: Overleaf includes acmart.cls for you so you don't even need that
% it even includes the ACM-ReferenceFormat.* files, but I edited the .bst file to not warn about missing publisher/address for inproceedings

%\setcopyright{acmcopyright}
%\copyrightyear{2022}
%\acmYear{2022}
%\acmDOI{10.1145/1122445.1122456}

%\acmConference[Woodstock '18]{Woodstock '18: ACM Symposium on Neural  Gaze Detection}{June 03--05, 2018}{Woodstock, NY}
%\acmBooktitle{Woodstock '18: ACM Symposium on Neural Gaze Detection,  June 03--05, 2018, Woodstock, NY}
%\acmPrice{15.00}
%\acmISBN{978-1-4503-XXXX-X/18/06}

\usepackage[dvipsnames]{xcolor}
\usepackage{booktabs} % use booktabs instead of ugly regular tables
\usepackage{graphicx} % more figure options
\usepackage{enumerate}
\usepackage{enumitem}
\usepackage{subcaption}
\RequirePackage[l2tabu, orthodox]{nag} % checks for common LaTeX errors
\usepackage{microtype} % better typesetting
\usepackage[utf8]{inputenc} % lenient to utf-8 characters like smart quotes
%\usepackage{refcheck} % warns about unreferenced figures/tables that have labels. remove it before submitting
\usepackage[subtle]{savetrees} % denser formatting, you can comment this out
\usepackage{csquotes}
\usepackage{array}
\usepackage{multirow}
\usepackage{gensymb}


\interfootnotelinepenalty=10000 % split footnotes are ugly
\tolerance=400 % reduce how often words stick out into columns at the expense of word spacing

\graphicspath{{figures/}} % put all your figures in this folder
\newcommand{\note}[1]{{\color{blue}{{#1}}}}
\begin{document}

\newcommand{\nb}[1]{{\color{red}{$\Rightarrow$ ~\textbf{{#1}}}}} %nota bene

\newcommand{\cp}[1]{{\color{blue}{\enquote{{#1}}}}} %copy paste

\title{AudioMiXR: Spatial Audio Object Manipulation for Sound Design in Augmented Reality}


%% "authornote" and "authornotemark" commands
%% used to denote shared contribution to the research.
\author{\textbf{Brandon Woodard*}}
\affiliation{%
  \institution{Brown University}
  \country{United States}
}
\email{brandon\_woodard@brown.edu}

\author{\textbf{Margarita Geleta*}}
\affiliation{%
  \institution{UC Berkeley}
  \country{United States}
}
\email{geleta@berkeley.edu}

\author{\textbf{Andrea Fanelli}}
\affiliation{%
  \institution{Dolby Laboratories}
  \country{United States}
}
%\email{youremail@example.com}


\author{\textbf{Rhonda Wilson}}
\affiliation{%
  \institution{Dolby Laboratories}
  \country{United States}
}
%\email{youremail@example.com}

\renewcommand{\shortauthors}{Lastname, et al.}

\appendix

\section{Survey for professional audio mixers}\label{formative-interview}
Questions for professional audio mixers:
\begin{enumerate}[label=(\arabic*)]
\item Can you describe your general process when mixing audio tracks with spatial audio objects?
\item What tools or software do you typically use for spatial audio localization, mixing, and panning? How do these tools support your workflow?
\item How do you approach spatialization of audio objects within a 3D environment? What are your considerations when positioning audio objects spatially?

\item Can you discuss any specific techniques or strategies you employ to achieve optimal spatialization and panning effects in your projects?
\item In your experience, what are the main challenges or difficulties you encounter when working with spatial audio? How do you typically address these challenges?

\item How often do you collaborate with others (e.g., other sound engineers, content creators) during the spatial audio production process? Can you describe how collaboration influences your workflow and decision-making?

\item Have you encountered any notable successes or breakthroughs in your approach to spatial audio that you'd like to share? How did these impact your projects or clients?

\item Looking ahead, what advancements or improvements do you envision in spatial audio technology or tools that would benefit your work?

\item Can you describe a recent project where spatial audio played a critical role? What were the specific challenges and successes in that project?
\item How do you evaluate the effectiveness of spatial audio mixes? Are there specific criteria or metrics you use to assess quality?
\item What audio attributes do you think can benefit from spatialized interaction? (eg. in a audio panner to manipulate, volume, reverb, equalization etc.)
\item Finally, based on your experience, what advice would you give to someone new to spatial audio mixing and localization?
\end{enumerate}

\section{Demographic Questions}\label{demographic-questions}
\begin{enumerate}[label=(\arabic*)]
\item Age
\item Sex (M/F)
\item Glasses prescription
\item Experience with DAWs (1 = no experience, 2 = beginner, 3 = intermediate, 4 = advanced, 5 = expert)
\item Experience with Augmented reality prior to the experiment 1-5 (1 = no experience, 2 = beginner, 3 = intermediate, 4 = advanced, 5 = expert)
\end{enumerate}

\section{NASA TLX Questions for both non-experts and experts}\label{nasa-tlx}


\begin{enumerate}[label=(\arabic*)]
\item \textbf{Mental Demand}: Using a scale from 1 to 7, where 1 means very low workload and 7 means very high workload, please rate the mental workload involved in manipulating virtual audio objects in the XR system. How mentally demanding was the task?  (1 = very low, 7 = very high)

\item \textbf{Physical Demand}: How physically demanding was it to interact with virtual audio objects using free-hand gestures? How physically demanding was the task? (1 = very low, 7 = very high)

\item \textbf{Temporal Demand}: How hurried or rushed was the pace of the task? (1 = very low, 7 = very high)

\item \textbf{Performance}: How accurately were you able to position and manipulate virtual audio objects using free-hand gestures in the XR environment? How successful were you in accomplishing the task? (1 = perfect, 7 = failure)

\item \textbf{Effort}: How hard did you have to work to accomplish your level of performance? (1 = very low, 7 = very high)

\item \textbf{Frustration}: To what extent did you experience frustration or confusion while using the XR system to manipulate audio objects? How insecure, discouraged, irritated, stressed, and annoyed were you? (1 = very low, 7 = very high)

%% OTHER QUESTIONS
\item \textbf{Efficiency}: Rate the efficiency of completing tasks (e.g., moving, resizing, rotating) with virtual audio objects in the XR system. (1 = very low, 7 = very high)

\item \textbf{Overall Satisfaction}: On a scale from 1 to 7, how satisfied are you with the overall usability of the XR system for manipulating audio objects? (1 = very low, 7 = very high)
\end{enumerate}

\section{Short-answer survey}\label{short-answer-survey}
Questions for non-experts:
\begin{enumerate}[label=(\arabic*)]
\item How intuitive was the process of navigating and interacting with virtual audio objects using free-hand manipulation?
\item Did you encounter any difficulties or frustrations while attempting to manipulate the virtual audio objects? Please describe.
\item Were there any features or functionalities that were unclear or difficult to understand during your interaction with the system?
\item How comfortable did you feel using the XR system to manipulate audio objects in your physical environment?
\item What aspects of the system did you find most useful or enjoyable? Why?
\item Were there any specific improvements or additional features you would suggest to enhance the usability of the system?
\item (Experts only) Where can you see this interface being applied to your audio mixing workflow?
\end{enumerate}

\end{document}