\section{Related Work}
A growing body of literature has delved into the challenges older adults face in password management. To begin with, researchers have identified that password creation, updating, and recall present particular challenges for older adults due to cognitive limitations and usability concerns~\cite{stobert2014password,frik2019privacy,mentis2019upside}, often resulting in risky practices such as creating simple passwords~\cite{hargittai2013new,wash2016understanding,frik2019privacy} and reusing the same predictable passwords~\cite{grimes2010older,abela_consumer_2017, wei2024sok}. To address cognitive limitations and usability issues in password management among older adults, cybersecurity experts have long regarded digital password managers as a sure-fire way to streamline password management. However, password manager adoption rates are lowest among older adults \cite{ray2021older}. Reflecting concerns seen in broader populations~\cite{karole2011comparative}, Ray et al.~\cite{ray2021older} reported push-back from older adults exhibiting strong apprehension and mistrust toward cloud storage and cross-device synchronization. They often prefer direct control and tangible methods for password management, such as recording passwords in a notebook and storing it in a secure location. While the adoption of digital password managers may be encouraged through family support and the usability of built-in features like auto-fill, older adults often have incomplete mental models of how password managers work, which can evoke a sense of insecurity within this demographic. 


Despite the above evidence of coping mechanisms for password management among older adults internationally, password strategies specific to Irish older adults remain underexplored. To date, only two studies have provided limited qualitative insights into password management strategies, each reporting on the experiences of one older adult. Redahan~\cite{redahan2013older} interviews eight older adults, only one of which discussed passwords. This participant describes the frustrations of using a complex password system on an airline's website, resulting in a time-consuming experience due to frequent password resets and complex requirements. Flynn~\cite{flynn2023ireland} interviewed 20 older adults, with one participant discussing passwords and recalling an instance where a compromised social media password became a source of vulnerability.
Despite threats to other accounts, the participant assumed banking details were secure, indicating an overreliance on institutional safeguards rather than personal password management.


%Using Irish and Ireland-related terms in passwords introduces certain vulnerabilities. 

In terms of password creation, Murray and Malone found that Irish users frequently create weak passwords that are culturally and geographically specific, like ``dublin'', ``ireland'', ``munster'' and ``celtic'' \cite{murray2020convergence,murray2023adaptive}. These cultural references are strong indicators of demographic-based password choices.
Nedvěd \cite{nedvved2021careless} notes that the length of words in a language and the use of diacritics can make passwords more unique for example, Irish words are longer compared to English and the use of the Fada (accent) make Irish language (Gaeilge) passwords more secure and unique.
%, a trend that is noticeable among Irish users. 
%Another uniquely Irish password vulnerability is highlighted by 
%Nedvěd, who reports significant variations across Irish male (37.99\%), female (58.93\%) and unisex names (3.08\%)
%, posing as another indicator of password vulnerability 
%\cite{nedvved2021careless}. Given the limited pool of unisex Irish names, any user choosing a unisex name as part of their password might have a more predictable password pattern. This is problematic if attackers know that a user may select from a small list of unisex names.
An additional influence on Irish citizens' password habits may be rooted in Ireland's historical context. Garvey and Miller note that Ireland's journey to independence from British colonial rule, along with the subsequent years of social and technological development, has shaped attitudes toward technology \cite{garvey2021ageing}, which we posit may also extend to cybersecurity. As older adults shift from traditional religious frameworks to more individualized and socially integrated lives, they increasingly rely on digital platforms for community and well-being, bringing new concerns over data privacy and cybercrimes.
%A post-colonial skepticism toward authority may extend to caution with tech providers, as older adults seek control over personal data and selectively adopt technology and platforms they see fit. 

Technological caregiving from trusted individuals is often necessary, particularly for older adults with mild cognitive impairments~\cite{singh2007password,piper2016technological}. However, this support often involves a trade-off between convenience and security~\cite{latulipe2022unofficial}. Frik et al.~\cite{frik2019privacy} report that many older adults delegate password management to family members or technical assistants, reducing cognitive load but potentially introducing security risks if these helpers lack robust security practices. Mentis et al.~\cite{mentis2020illusion} found that while simplified password practices or shared access with caregivers can mitigate usability issues, these approaches may also heighten vulnerability to cyber threats. Mentis et al suggest however more inclusive approaches to foster meaningful participation of older adults in cybersecurity decisions. This is also reflected in Murthy et al.'s~\cite{murthy2021individually} study, emphasizing the need for designing security solutions that balance collective safety with individual empowerment, ensuring older adults can actively participate in managing their digital security and privacy. %They advocate for person-centered security designs which support autonomy without compromising security. 
Only 19.1\% of older adults in Ireland receive technological caregiving from family, below the EU average of 22.3\%~\cite{eurostat2024adulteducation}, this could be due to Ireland's high migration numbers.
In the 12 months leading up to April 2024, over 69,000 individuals emigrated from Ireland, an increase from 64,000 during the same period in 2023. This marks the highest level of emigration recorded since 2015~\cite{Emmigration}. Assuming that most of the Irish older adults rely on their now-emigrated children for help in cybersecurity, these migration patterns suggest that this demographic increasingly resorts to remote family support while also experiencing a sense of forced independence.


This paper builds on the call for region-specific research~\cite{herbert2023world}, by focusing on the unique password management knowledge, perceptions and behaviors of older adults in Ireland. By examining this demographic within a specific cultural and regional context, we aim to contribute to a more detailed understanding of how password management practices and challenges manifest and how education and outreach can best support the needs and requirements of this population. %within existing Western frameworks, \hl{which include Ireland's adult literacy program, Ireland's digital equality goals and the  European digital standards} \cite{DigComp, AdultLiteracy, digitalreport}.  
%Password misconceptions have been noted to vary globally \cite{herbert2023world}, the research above offers a distinct perspective within the Irish context, shedding light on factors that may shape password habits among Irish older adults.