\section{Introduction}
\label{sec:intro}


% Recently, transfer learning has emerged as a promising approach for medical semantic segmentation, primarily due to the persistent challenge of acquiring large amounts of labeled data necessary for training deep learning models effectively. 
For medical segmentation tasks, acquiring ground-truth labels is challenging because  detailed annotation of large 3D images is both time-consuming and requires expert input\cite{medical_deeplearning1, medical_deeplearning2}. To compensate the lack of labeled training samples, transfer learning has become a common approach for medical semantic segmentation \cite{medical_transfer1, medical_transfer2}, 
%, which refers to any learning strategy that 
leveraging the knowledge from a high resource domain to improve the performance on low-resource tasks\cite{pan2009survey}. 
%The hope is that this source domain is related to the target domain and thus transferring knowledge from the source can improve the performance within the target domain. 
% For vision applications, the most widely used approach is to pre-train a model on a source domain/task and then fine-tune that same model on the target domain/task. 
% In this approach, the knowledge that is transferred from the source to the target problem is in the form of the values of the network parameters. 
% Unfortunately, the effectiveness of transfer learning is not always guaranteed. Low task correlation, domain correlation, and the misapplication of the transfer method may lead to negative transfer\cite{negative-transfer,negative_transfer_2}. 
However, the effectiveness of transfer learning is not guaranteed: low task and domain correlation can impair the target domain performance. This phenomenon is known as  {\it negative transfer}\cite{negative_transfer_2}. 
% , along with improper application for specific scenarios\st{(such as using classification approach to address segmentation problem, using natural image method to tackle medical image problem)}, 
%Negative transfer may hinder the performance of the target, which is a long-standing and challenging issue in the transfer learning context. 
% i.e., introducing source domain knowledge in an undesirable manner decreases learning performance in the target domain, 
Due to the inherent heterogeneity in medical images, including differences in imaging modalities, contrast variations, and patient-specific anatomy\cite{medical_transfer1, medical_da},   negative transfer  is common in medical image segmentation\cite{negative-transfer}. 
% Therefore, studying NT facilitates more reliable and safer transfers in real-world applications.
% what to transfer,how to transfer, when to transfer. “when to transfer” examines whether the source data/knowledge should be transferred or not for a particular application scenario.

Most existing works address negative transfer by selecting more relevant source domain\cite{yicong, jin2024cross}, reweighting the source samples\cite{instance_level_3, negative_transfer_2}, and aligning distributions in the feature space\cite{feature_level_1}. The principle idea in these works lies in the notion of domain similarity \cite{MMD, KLD, correlation_coefficient} or some transferability metric across different tasks \cite{nce, hscore, leep, logme, otce} to quantify transfer performance.  
As these approaches have originated from a classification or regression context, they ignore the complex output structures in image-output tasks such as semantic segmentation. In segmentation, the risk of negative transfer of a given model does not uniformly distribute over the whole image. For example, some textures such as the white matter or the grey matter in the brain are shared across multiple domains, while some are unique to a specific domains such as the cortical protrusions and depressions across individual brains. This observation necessitates the need for a fine-grained transfer learning approach that could adapt to variable negative transfer risk at different image regions.


Recently, a few natural image segmentation works have incorporated the category-level and pixel-level transferability into loss function %to enhance the target performance by weighting the loss function by pixel-level transferability score\cite{Dong,10222912}.
by propagating the global category-wise transferability calculated by entropy criterion\cite{dong}, or weighting the loss function by transferability score during fine-tuning phase\cite{10222912}. However, they don't work effectively on medical segmentation tasks due to fewer semantic categories, tiny segmentation foreground, and severed class imbalance of medical images, compared to natural images\cite{medical_and_natural}.



%To measure negative transfer at specific image regions, previous works focus on estimating the domain similarity \cite{MMD, KLD, correlation_coefficient} or quantifying the global transfer hardness across different tasks \cite{nce, hscore, leep, logme, otce} to select beneficial source data or models to transfer. Transferability reveals how easy it is to transfer knowledge learned from a source task to a target task\cite{otce}. 
%To mitigate pixel-level negative transfer risk on medical segmentation tasks

To solve the aforementioned challenges, we propose a simple yet effective weighted fine-tuning approach that directs the model's attention towards regions with significant transfer risk, tailored to the medical semantic segmentation problem (Fig.~\ref{architecture}). Specifically, we introduce a pixel-level {\it transfer risk map}, which quantifies the transfer hardness for each pixel and the potential risks of negative transfer associated with them. Here we adopt LEEP (Log Expected Empirical Prediction)\cite{leep} metric as our transferability method for its high computational efficiency, simplicity, and superior performance\cite{10222912}. And to alleviate the adverse effects of class imbalance during the fine-tuning phase, we calculate loss values across all pixels but average them exclusively over the foreground pixels. Such a method effectively reduces the impact that a large amount of well-learned background pixels might have on biasing the loss value, which in turn expedites the refinement of model parameters. 


%The proposed method is designed to direct the model's attention towards regions with significant transfer risk, thus mitigating the negative transfer and enhancing the overall effectiveness of the fine-tuning process. The framework is shown in Fig.~\ref{architecture}.
% In the fine-tuning stage, the model can leverage the knowledge from source model more heavily in regions where transfer hardness is low. Conversely, in regions of high transfer hardness, the model prioritizes learning from the target data to refine these areas, thereby optimizing the overall performance. 

% \jy{To mitigate negative transfer, previous works focus on estimating the domain similarity \cite{MMD, KLD, correlation_coefficient} or quantifying the global transfer hardness across different tasks \cite{nce, hscore, leep, logme, otce} to select beneficial source data or models to transfer. Transferability reveals how easy it is to transfer knowledge learned from a source task to a target task\cite{otce}. Recent works address negative transfer by improving the transferability in dataset-level\cite{yicong, jin2024cross}, image-level\cite{instance_level_1, instance_level_2, instance_level_3, negative_transfer_2}, and feature-level\cite{feature_level_1, feature_level_2}. However, for a given source domain and task, the risk of negative transfer does not uniformly distribute over the whole image, as some textures such as \st{the white matter or the grey matter in the brain} are shared across multiple domains, while some are unique to a specific domain or task such as \st{the cortical protrusions and depressions across individual brains}. This necessitates the need for a pixel-wise approach that could adapt to variable negative transfer risk at different image regions. Recent works incorporate the pixel-level transferability to enhance the target performance by propagating the global category-wise transferability calculated by entropy criterion\cite{dong} or weighting the loss function by transferability score during fine-tuning phase\cite{10222912}. However, they don't work effectively on medical segmentation tasks due to fewer categories, tiny segmentation foreground, and severed class imbalance of medical images, compared to natural images.
% }

% \jy{Considering the issues mentioned above, this work focused on pixel-level transferability-guided weighted fine-tuning for medical segmentation tasks. To mitigate pixel-level negative transfer risk on medical segmentation tasks, we propose a pixel-level transfer risk map that meticulously quantifies the transfer hardness for each pixel and the potential risks of negative transfer associated with them. Here we adopt LEEP (Log Expected Empirical Prediction)\cite{leep} metric as our transferability method for its high computational efficiency, simplicity, and superior performance\cite{10222912}. However, these transferability metrics are designed for classification or regression tasks, which cannot be directly applied to semantic segmentation tasks. Inspired by \cite{10222912}, we adapt LEEP on segmentation tasks by calculating transferability scores over the sampled pixel-wise features.
% And to alleviate the adverse effects of class imbalance during the fine-tuning phase, we calculate loss values across all pixels but average them exclusively considering the non-zero pixels. Such a method effectively tempers the impact that a multitude of well-learned background pixels might have on diluting the loss value, which in turn expedites the refinement of model parameters. 
% }

%Transferability reveals how easy it is to transfer knowledge learned from a source task to a target task\cite{otce}, which can quantify the global transfer hardness accurately. 
% its estimation metrics aim to develop a measure (or a score) that can tell us, without training on the target dataset, how effectively they can transfer knowledge learned in the source model to the target task. 
% Feature statistics based transferability methods like MMD\cite{MMD}, KL-divergence\cite{KLD}, and correlation coefficient\cite{correlation_coefficient} estimate the domain similarity based on the original feature representation and its first- or high-order statistics, which is not suitable for real-world medical task without access to source data. Another category of transferability method is fine-tuning based, including LEEP\cite{leep}, LogME\cite{logme}, NCE\cite{nce} and H-score\cite{hscore},  considering the correlation between the predicted target label or the learned feature representations and the true target label. 

% make transferability computable
% They are predominantly applied to the source domain selection problem, aiming to a priori determine which domains are likely to yield superior transfer outcomes\cite{yicong, jin2024cross}. 
% 基于feature的方法需要source的数据,基于finetune的方法都基于分类任务

% Despite these empirical observations, little research work has been published to analyze or predict negative transfer. Third and most importantly, given limited or no labeled target data, how to detect and/or avoid negative transfer.

% Merely employing transferability methods for domain-level priori determine \cite{yicong, jin2024cross} often falls short. 
% Relying solely on transferability methods for domain-level priors \cite{yicong, jin2024cross} often proves inadequate. How to mitigate the risk of negative transfer from relevant source domains has become a considerable problem. Most works address negative transfer by improving the instance-level and feature-level transferability. Instance-level methods mitigate negative transfer by removing the irrelevant source samples\cite{instance_level_1} or reweighting the source samples\cite{instance_level_3, negative_transfer_2}, even some active learning methods\cite{instance_level_2}, which rely more on source data. Feature-level approaches usually focus on feature space decomposition\cite{feature_level_1}, or improving the transferability of feature representations\cite{feature_level_2}, which may lead to poor discriminability. However, for a given source domain and task, the risk of negative transfer to the target is not uniform over the entire image. This is not only caused by having fine-grained textures. Rather, it is because some textures such as the shape of the brain are shared across multiple domains, while some are unique to a domain or task such as specific brain matter. These requirements underscore the need for a meticulous focus on finer granularity during the transfer process, thereby facilitating the model's capacity to refine and fine-tune local regions. 
% Dong et al.\cite{Dong} quantify the adaptation contributions of semantic representations across domains via transferability information propagation from global category-wise prototypes calculated by entropy criterion, which is not suitable for medical segmentation tasks that typically involve fewer categories compared to natural images. Tan et al.\cite{10222912} proposed a transferability-weighted fine-tuning method to emphasize low-transferability regions, thereby enhancing the overall transfer accuracy for the target task, which doesn't work effectively on medical segmentation tasks due to their tiny segmentation foreground and severed class imbalance.

%In this work, we propose a LEEP based pixel-level transfer risk map that harnesses transferability evaluation techniques to meticulously quantify the transfer hardness for each pixel and the potential risks of negative transfer associated with them. Furthermore, we propose a novel weighted fine-tuning approach guided by the transfer risk map for medical segmentation tasks. 

% Extensive experiments on brain tumor and brain matter segmentation datasets demonstrate that our proposed fine-tuning consistently outperforms the vanilla fine-tuning in all transfer experiments, with 4.37\% gain on FeTS2021 and 1.81\% gain on iSeg-2019. A 2.9\% gain in Dice score under a few-shot scenario validates the robustness of our approach, highlighting its capacity to mitigate the impact of localized negative transfer and to facilitate model adaptation and refinement during fine-tuning. 
%\vspace{-3mm}
\subsection{Multi-destination active message format}
\label{section:message_format}
\vspace{-0.7cm}
\begin{figure}[h!]
	\scriptsize
        \centering
    % \hspace{-1cm}
    \includegraphics[width=1\columnwidth]{diagrams/message_format.pdf}
    \vspace{-0.4cm}
	\caption{Message format} 
	\label{fig:message_format}
	\vspace{-.3cm}
\end{figure}
\textit{Nexus Machine} extends the fundamental Active Message primitives to accommodate a multi-destination based routing mechanism. 
Fig.~\ref{fig:message_format} illustrates the message format: the first 12 bits specify intermediate destinations (\textit{R1}, \textit{R2}, \textit{R3}), based on our workload analysis. 
The next 4 bits contain the Program Counter (PC) for the next instruction (\textit{N\_PC}), followed by 4 bits for the \textit{Opcode}. 
A single bit (\textit{Res\_c}) indicates if the message carries a result. 
The subsequent 2 bits (\textit{Op1\_c} and \textit{Op2\_c}) identify whether \textit{Op1} and \textit{Op2} are addresses or values. 
Depending on \textit{Res\_c}, the \textit{Result} field contains the final result or its address, while the next 16 bits hold data for Operand1 (\textit{Op1}) and Operand2 (\textit{Op2}).

When a message arrives at a router, the first destination (\textit{R1}) is processed by the \textit{Route Computation} logic and then allocated to the appropriate output port. After reaching \textit{R1}, the message is handled by the \textit{Input Network Interface}, and the remaining destinations are cyclically rotated, making \textit{R2} the first and \textit{R3} the second. 

In the \textit{Nexus Machine}, a message is equivalent to a packet or flit (all messages are a single-flit packet).
\begin{comment}
\begin{figure*}[h!]
	\scriptsize
	\centering
	\includegraphics[width=\textwidth]{diagrams/architecture.pdf}
	\caption{\textit{Nexus Machine} microarchitecture. \textit{Nexus Machine} is a fabric of homogenous PEs interconnected by a mesh network for communicating Active messages, enhancing fabric utilization by executing messages en-route.} 
	\label{fig:detail_arch}
	%\vspace{-.5cm}
\end{figure*}
\end{comment}
%\vspace{-3mm}
\subsection{Nexus Machine Micro-architecture}
\begin{figure*}[h!]
	\scriptsize
	\centering
	\includegraphics[width=0.9\textwidth]{diagrams/architecture.pdf}
    \vspace{-.15cm}
	\caption{\textit{Nexus Machine} microarchitecture. A fabric of homogenous PEs interconnected by a mesh network for communicating Active Messages which carry instructions that can be launched en-route at any PE, enhancing fabric utilization and runtime.} 
    \vspace{-0.3cm}
 %\color{red}{\bf Peh: I suggest replacing (d) with one of the Compute Unit, cos it's a major component of Nexus and yet do not feature in any figure. We need to highlight to reviewers that our compute unit consists of ALU :)}} 
	\label{fig:detail_arch}
	%\vspace{-.5cm}
\end{figure*}
%\subsubsection{Top Level}
As presented in Fig.~\ref{fig:detail_arch}(a), the \textit{Nexus Machine}'s fabric comprises homogeneous processing elements (PEs) interconnected with a mesh network, with a global termination detector. Each PE is linked to four neighboring PEs in North, East, South, and West directions.
The off-chip memory is connected to the four PEs located along the left edge.


\subsubsection{Processing Elements (PEs).}
%As presented in figure~\ref{fig:detail_arch}(b), each PE combines a compute unit, a dynamic router for network connectivity with congestion control, a decode unit with local data memory, an Input Network Interface which contains an instruction memory for handling incoming AMs and an AM Network Interface unit for spawning new AMs. \\
As presented in Fig.~\ref{fig:detail_arch}(b), each PE combines a compute unit, a dynamic router for network connectivity with congestion control, a decode unit, and two Network Interface logic.
Specifically, \textit{Input Network Interface} unit is responsible for efficiently handling incoming AMs from the NoC, while the AM Network Interface unit initiates the injection of new messages into the NoC.

\textbf{Input Network Interface.}
%The Input Network Interface logic triggers the loading of subsequent instruction on AM arrival.
%The arrival of a Decode AM triggers loading of the data element 
The \textit{Input Network Interface} unit manages \textit{incoming AMs} to a PE.
Depending on the message, \\%it performs either of these two operations.\\
%(a) It either updates the instruction contained in the message based on the next Program Counter (N\_PC) value provided within the message body.\\
(a) If it pertains to an ALU operation, it is directed to the \textit{Compute Unit} for execution.\\
(b) Alternatively, in case of a memory operation, the message is forwarded to the \textit{Decode} unit. 
This unit initiates a load or store operation, utilizing the operand address information (\textit{Op1} or \textit{Op2}) contained in the message.\\
Once these operations are completed, the resulting \textit{output dynamic AM} is dispatched to the \textit{AM Network Interface} for injecting into the network.
%The message enters the network via the local input port, which feeds the \textit{Compute} unit.

\textbf{Compute Unit.}
The \textit{compute unit} within a PE can perform 16-bit arithmetic operations, logic operations, multiplication, and division on its ALU.

An incoming AM at the \textit{Input Network Interface} dispatches two operands, \textit{Op1} and \textit{Op2} along with the \textit{Opcode} field in the message to the compute unit.
After computation, it generates an output that is combined with the original AM, replacing the \textit{Op1} field in the message.
Finally, this modified AM is forwarded to the \textit{AM Network Interface} for injecting into the network.

%{\bf Peh: There needs to be detailed information on how an AM launches computation! This is the thesis of AM! For instance, what's the format of the AM, when it's received, which field is used to configure the ALU? How is the PC set? What happens in the beginning of execution? read config memory? are there registers? what happens if data operand is not present -- can that happen? stall? Lots of details needed here.}

\textbf{Decode Unit.}
The \textit{Decode Unit}, as shown in Fig.~\ref{fig:detail_arch}(e), can be flexibly configured to operate in dereference and streaming modes.
In \textbf{dereference mode}, the operand address field (\textit{Op1} or \textit{Op2}) in the message triggers the loading of a single element. This gets embedded into the output \textit{dynamic AM}.
Conversely, in \textbf{streaming mode}, the message initiates the loading of multiple elements from memory, generating multiple output AMs.
In this mode, the operand address is considered the base address, along with a count to access and load the elements from memory sequentially.
These two modes suffice for our benchmarks; however, our architecture allows for integration of additional modes if needed.

\textbf{Active Message (AM) Network Interface.}
The \textit{AM Network Interface logic} is responsible for injecting AMs into the network.
%The AM Network Interface logic consists of an AM Queue and a configuration memory.
%The AM Queue is a 1KB FIFO, initialized with 44-bit precompiled entries.
%The configuration memory is 16-bit wide, containing 8 configurations.
This module comprises two primary components: an \textit{AM Queue} and a \textit{configuration memory}. 
The \textit{AM Queue} is a 16KB FIFO initialized with 70-bit precompiled entries. 
The \textit{configuration memory}, 10-bit wide, accommodates 8 distinct configurations.

%Depending on the availability of the output dynamic AM from the \textit{Input Network Interface}, it either
It either performs these two operations, as shown in Fig.~\ref{fig:detail_arch}(b):
(1) If the output \textit{dynamic AM} is available from \textit{Input Network Interface}, the subsequent configuration is loaded from memory based on the \textit{N\_PC} field of the AM (see Fig.~\ref{fig:message_format}). 
This configuration is combined with the output \textit{dynamic AM} and forwarded into the injection port of the router.\\
(2) Alternatively, a \textit{static AM} is injected into the network to keep it occupied. 
This \textit{static AM} is the concatenation of the next precompiled entry from the \textit{AM Queue} with the first configuration loaded from memory.
The generation rate of \textit{static AMs} is determined by the backpressure signal at the router's injection port.

The highlighted blue fields in the message format (see Fig.~\ref{fig:message_format}) depict data from the configuration memory used to construct the subsequent dynamic AM, with fields \textit{Res\_c}, \textit{Op1\_c}, and \textit{Op2\_c} stored to prevent redundancy.
%The AM Network Interface logic consists of a 1KB AM Queue, a FIFO containing 44-bit pre-compiled entries.
%, alongside a 13-bit wide configuration register. 

%As shown in Figure~\ref{fig:detail_arch}(e), the output dynamic AMs from \textit{Input Network Interface} trigger loading the next subsequent configuration from the memory with the N\_PC field of the AM.
%These are further concatenated with the output dynamic AMs and pushed into the injection port of the router.
%The injection rate is managed by the backpressure signal at the injection port of the router.

%To keep the network occupied, static AMs are containing the first precompiled entry from AM queue
%As shown in Figure~\ref{fig:detail_arch}(e), it concatenates an AM Queue entry with the first configuration loaded from the memory to generate a static AM, which is subsequently pushed intothe injection port of the router. 

\subsubsection{Dynamic and Congestion Aware Routing.}
\textit{Nexus Machine} supports turn model routing~\cite{noc_peh}, with each router containing five input and five output ports.
%Specifically, these input ports correspond to AM, local, north, east, south and west, whereas output ports correspond to local and four directions.
Specifically, these input ports are designated for messages coming from \textit{AM Network Interface} unit, as well as north, east, south, and west directions, whereas output ports are designated for messages going to \textit{Input Network Interface} unit and four directions.
%The AM input port receives recently generated messages from the \textit{AM Network Interface unit}, while the local port handles messages coming from the \textit{Input Network Interface}. 
Each input port has a buffer comprising three registers to manage in-flight messages, accompanied by congestion control logic. \textit{Nexus Machine}'s design choice of employing only three registers is motivated by the goal of minimizing overall power consumption.

As presented in Fig.~\ref{fig:detail_arch}(c), each router contains a Route Computation Unit, Separable Allocator, and a Crossbar.

\textbf{Route Computation} logic considers the destination of messages from all the input ports. It compares it with the positional ID of the PE, and calculates the output port to be requested. This is sent as an input to the allocator.

%{\bf Peh: Separable alllocation is a well-known previously proposed technique... so there's no need to elaborate... just cite a NoCs textbook}
A toy example of \textbf{Separable Allocation} process is presented in Fig.~\ref{fig:detail_arch}(d)~\cite{noc_peh}.
\iffalse
The request matrix's rows correspond to input ports, and columns correspond to output ports.
The process consists of two stages of 6:1 and 5:1 fixed priority arbiters. The first stage prunes the matrix to ensure that each output port (or resource) receives requests from at most one input port (or requestor). Subsequently, the backpressure signal is applied to each output port, enabling congestion control, as explained below. The second stage further prunes the matrix to guarantee one grant per input port.
The allocator executes within a single cycle, marking granted requests as issued immediately to prevent them from bidding again.
\fi

\textbf{On/Off Congestion control} involves the transmission of a signal to the upstream router when the count of available buffers falls below a threshold, ensuring all in-flight messages will have buffers on arrival. Each of the five ports transmits an OFF signal when their corresponding available buffer space is reduced to 1, i.e., $T_{OFF} = 1$, and conversely, an ON signal when their buffer space reaches 2, i.e., $T_{ON} = 2$.

The output of the allocator is sent to a 6x5 \textbf{Crossbar}.
%, which forms many-to-many connections among internal and external datapaths.\\ Peh: A crossbar by definition forms many-to-many connections between its input and output ports, so no need to explain. 

\subsubsection{Off-chip Memory Datapath.}
Each off-chip memory port connects to a row of the PE array via an AXI bus, delivering a combined bandwidth of 1.28GBps. During data loading, data transfers from off-chip memory to the \textit{AM queues} and \textit{data memory} in each PE. 
The \textit{AM queues} are actively consumed during execution, effectively hiding data loading latency by performing it concurrently with the execution. 
However, data loading into \textit{data memories} occurs after tile execution is complete.

\subsubsection{Bit-vector Scanners.}
The first sparse operand is encoded in \textit{static AMs}. For subsequent sparse operands, bit-vector scanner hardware assists in efficient iteration, providing coordinates within compressed vectors as described in \cite{capstan}. \textit{Nexus Machine} integrates a modified version of this with its AXI bus controller to obtain these coordinates. It can vectorize 16 non-zeros within 128 elements, allowing it to handle matrices with densities exceeding 12\%.
\vspace{-0.2cm}
\subsection{Deadlock avoidance}
%\textcolor{blue}{\bf Peh: Write up a short blurb on how Nexus machine addresses various deadlock scenarios and explain design choice: (1) Within network: Flow control deadlocks addressed by bubble buffer [cite bubble flow control] instead of VCs so as to minimize buffering; Routing deadlocks addressed by turn model, so as to provide high throughput without complex adaptive routing hardware; (2) AM also introduces potential network-PE deadlocks -- addressed by compiler preventing such cyclic dependencies + runtime timeouts} 

Given the dynamic nature, \textit{Nexus Machine} can potentially encounter deadlock without careful design. We address various deadlock scenarios with these specific design choices: 
(1) To mitigate flow control deadlocks within the network, we adopt the bubble NoC~\cite{bubble_flow} approach over Virtual Channels (VCs), with the aim to minimize buffering. 
(2) Routing deadlocks are mitigated by using the turn model~\cite{noc_peh}, that ensures high throughput without the need for complex adaptive routing hardware.
(3) AMs can potentially create deadlocks between the network and PEs. 
These are effectively mitigated by the compiler through strategic data placement and runtime timeouts.
%\textit{Nexus Machine} currently uses a simple heuristic for data placement strategy within the compiler.
%\textcolor{red}{\bf Peh: Is the simple heuristic related to deadlocks? Cos the above sentence seems to contradict the earlier sentence. Elaborate on the heuristic and timeouts??}
Future research will explore more optimized data placement strategies.
Extensive experiments on brain tumor and brain matter segmentation datasets demonstrate that our proposed fine-tuning method achieves significantly enhanced performance when transferring knowledge between distinct modalities and tasks, with a 4.37\% gain in brain tumor segmentation dataset FeTS 2021 and a 1.81\% gain in brain matter segmentation dataset iSeg-2019, indicating that it indeed avoids negative transfer from diverse modalities and tasks while learning beneficial knowledge for segmentation across multiple modalities and tasks. In the few-shot scenario, our method also improved the baseline by 2.9\% on average, validating the robustness of our approach under different sample sizes.  
   
% Medical image segmentation enables the precise localization and quantification of regions of interest, serving as a critical step in diagnostic aid, automated diagnostic workflows, surgical planning, and treatment efficacy assessment. Despite significant progress in the field of automated segmentation, particularly with the advent of deep learning, there is still a formidable challenge with domain shift, according to the inherent variability present in medical images, including differences in imaging modalities (MRI, CT, Ultrasound, etc.), contrast variations, and patient-specific anatomy\cite{medical_transfer1, medical_da}. Transfer learning represents a powerful learning paradigm for mitigating the domain gap and enhancing the performance on target tasks with knowledge from related source tasks. Transferability reveals how easy it is to transfer knowledge learned from a source task to a target task. The most straightforward approach is to evaluate transferability based on the test accuracy of the target task after the transfer, which involves expensive computation in retraining neural networks. Recent studies including LEEP\cite{leep}, LogME\cite{logme}, NCE\cite{nce} and H-score\cite{hscore} have proposed different methods to estimate the knowledge transferability between source and target tasks for natural images. These transferability estimation metrics aim to develop a measure (or a score) that can tell us, without training on the target data set, how effectively these transfer learning algorithms can transfer knowledge learned in the source model to the target task, using the target dataset.
% Currently, the application of transferability theory estimation is predominantly concentrated on the source domain selection problem, aiming to a priori determine which domains are likely to yield superior transfer outcomes. Li et al.\cite{yicong} propose a prior knowledge guided and transferability based source selection framework, leveraging the prior knowledge and transferability estimation metrics to select the best source tasks for transfer learning on brain image segmentation tasks. Jin et al.\cite{jin2024cross} propose an MMD-based domain selection method, which adaptively selects the source domain with the least difference from the target subject in a BCI-MI dataset of 109 subjects. 

% Recent works in the natural images field have started to incorporate transferability and discriminability into their models, thereby contributing to the enhancement of the models’ performance. Dong et al.\cite{Dong} propose a novel Knowledge Aggregation-induced Transferability Perception Adaptation Network to explore where and how to capture transferable visual characterizations and semantic representations for unsupervised domain adaptation. They quantify the adaptation contributions of semantic representations across domains via transferability information propagation from global category-wise prototypes calculated by entropy criterion. A transferability-aware information bottleneck is developed to capture transferable appearance translation. Chen\cite{chen2020harmonizing} observe that some local regions of the whole image are more descriptive and dominant than others, they further enhance the local discriminability by proposing to compute local feature masks in both domains based on the shallow layer features for approximately guiding the semantic consistency in the following alignment, which can be seen as an attention-like module that captures the transferable regions in an unsupervised manner. Tan\cite{10222912} introduced an adaptation method that enables existing transferability metrics to be applied to semantic segmentation data and proposed a transferability-weighted fine-tuning method that emphasizes low-transferability regions, thereby enhancing the overall transfer accuracy for the target task. 

% Owing to the intrinsic disparities between medical and natural images, techniques from the natural image domain cannot be seamlessly adapted for medical imaging tasks. In the context of segmentation, a stark contrast exists where medical images allocate a tiny fraction of the image area as the segmentation foreground—encompassing lesions or specific tissue segments—resulting in a pronounced class imbalance that is particularly acute in medical image segmentation. Additionally, medical image processing necessitates a fine-grained capability for discerning local textures, as opposed to the coarse-grained object shape recognition typically emphasized in natural image segmentation. The minute anomalies within these textures can betray the signatures of underlying diseases, thus aiding in the model's segmentation accuracy. These requirements underscore the need for a meticulous focus on finer granularity during the transfer process, facilitating the model's capacity to meticulously refine and fine-tune local regions, consequently elevating the efficacy of the segmentation outcomes.


% \begin{itemize}
% \item[1)]A transferability guided method, a pixel-level transfer risk map. It harnesses transferability evaluation techniques to meticulously quantify the potential risks of negative transfer associated with them.
% \item[2)]A novel weighted fine-tuning approach guided by the transfer risk map for medical segmentation tasks. It directs the model's attention towards regions with significant transfer hardness, thus mitigating the potential risk of local negative transfer and enhancing the overall performance. 
% \item[3)] Extensive experiments on brain tumor and brain matter segmentation datasets demonstrate the effectiveness and robustness of our method. 
% % It validates the efficacy of our proposed approach in mitigating the risk of potential negative transfer.
% \end{itemize}
%\st{In summary, we propose a transferability-guided pixel-level weighted fine-tuning strategy with a tailored transfer risk map for medical segmentation tasks. The proposed pixel-level transfer risk map harnesses transferability evaluation techniques to meticulously quantify the potential risks of negative transfer associated with them. The weighted fine-tuning guided by the transfer risk map directs the model's attention towards regions with significant transfer hardness, thus mitigating the potential risk of local negative transfer and enhancing the overall performance. Extensive experiments have proved the superiority of our method compared with baseline methods.}