\section{Related Work}
\label{sec:RelatedWork}
IoT offers promising opportunities to optimize decision-making and improve efficiency. As part of intelligent transportation, many road hazard recognition and alarm techniques using IoT are introduced. Besides, the recent innovations of FL enable collaborative learning. Moreover, different privacy-preservation techniques are utilized to protect privacy. The related literature is summarized as follows.

\textbf{road hazard detection and alarm techniques.}
Deep learning algorithms are used for the identification of road hazards in many existing studies____ and have achieved promising results. 
Similarly,____ introduced CNN-based models for the classification of road hazards. Despite the success of deep learning models in processing visual data within cluttered real-world scenarios, current road damage detection systems primarily rely on visual input. For example,____ used a deep learning algorithm to identify road obstacles, thus mitigating road hazards.
In____, a threshold-edge-based algorithm was proposed to detect holes in roads and report them on Google Maps.

To our knowledge, no existing studies have explored the integration of multimodal data in this domain. In contrast, extensive research in other domains has demonstrated that leveraging data from various modalities can substantially enhance model performance____. A straightforward approach involves fusing features from different modalities to learn from the available data. 
____ utilized early (combining embeddings of initial layers) and late fusion (integrating final decisions) to achieve superior accuracy. The authors of ____ developed a classifier that is trained from both image and text data to monitor flooding incidents. 
Furthermore,____ introduced a multimodal deep learning system capable of recognizing damage to the infrastructures using image and text data retrieved from social media messages. Additionally,____ introduced a multimodal model with a cross-attention module for the categorization of crisis events.
____ proposed a a novel multimodal attentive framework combining graph neural networks and social context features to enhance disaster content classification by capturing both aligned and contrary patterns in text and visual data.

\textbf{IoT-based federated learning methods.} 
The concept of FedAvg was presented in____ in 2016 as a technique for developing a unified model by averaging multiple local models while ensuring the training data remains on its respective local devices. Nonetheless, a major obstacle limiting the widespread adoption of FL within the IoTs is the high communication expenses tied to frequent model updates exchanged between clients (i.e., road users) and the aggregator during the training process. Research conducted by____ suggests that the communication burden of an individual road user can reach a petabyte when training models on large datasets. Various solutions have been put forward to handle this challenge, which can be classified into two categories: reducing the frequency of communication and applying model compression techniques.

Methods aimed at reducing communication frequency seek to reduce the number of model transmissions. In____, each client refreshes its local model multiple times before transmitting it, rather than sending it after every iteration. Such strategies can considerably reduce both upload and download communication expenses. 
Conversely, compression techniques reduce communication costs by decreasing the model's size before transmission.
A prevalent method is quantization____, which reduces the model size by mapping updates to a limited range of possible values. SignSGD, introduced by____, is a quantization-based method that converts gradient updates into their binary sign, thereby significantly reducing the model size by 32 times while theoretically guaranteeing convergence on identically distributed data. Additionally, SignSGD conducts downstream data compression through majority voting to aggregate received binary updates. Other scholars have suggested stochastic gradient quantization techniques to reduce the size of upstream gradients without introducing bias, exemplified by quantized stochastic gradient descent (QSGD)____, which successfully quantifies model gradients while maintaining models' convergence.

In practical cases, the data gathered by individual users typically does not have an identical distribution (non-iid). The challenge of modeling non-iid data poses significant difficulties for deep learning algorithms. As indicated in____, models' performance when applying FL can decrease drastically (by more than 50\% ) when the models are trained on non-iid datasets. To tackle this issue,____ introduced a method that generates a slim data pool shared between all edge servers for training while____ utilized FedAvg and showed that it has some potential of dealing with non-iid data. Although this method partially mitigates the non-iid problem, achieving an identically distributed data setup on user devices remains challenging because of the unknown data distributions among clients. An alternative approach involves directly learning from non-iid data____.

The research in____ introduced a revised federated learning framework that organizes clients into hierarchical clusters according to the similarities of the local models to the global one, demonstrating higher convergence speed in non-iid data compared to conventional approaches. In____, a method called Asynchronous Online FL tailored for non-iid environments was designed, where users continuously gather data, and their local models are trained on the collected data. This strategy learns the interrelations between different devices through regularization and feature learning. Additionally,____ tackled the problem of catastrophic forgetting in FL by incorporating a penalty term into the model’s loss function, which incentivizes all local models to move toward a global optimum. Moreover, studies____ have established that variations in data distributions can obstruct the convergence of the model. Specifically, according to____, for FedAvg to achieve convergence to the global optimum in the context of non-iid data, the learning rate must be progressively reduced.
____ proposed an Intelligent-Optimization-Based Federated Learning (IOFL) framework, where the server directly searches for global model parameters using intelligent optimization algorithms, while clients only validate the model and return test accuracy. This approach fundamentally eliminates the impact of non-IID data on model performance. 
____ tackled Non-IID challenges in federated learning by generating privacy-preserving synthetic data that matches essential class-relevant features.


\begin{figure*}[t]
\vspace{-2mm}
\centering
\includegraphics[width=0.99\textwidth]{figures_journal/RoadFed.jpg}
\caption{An overview of the proposed RoadFed framework, including three key components (i.e., road users' devices, untrusted edges, and untrusted cloud.) and three key methodologies (i.e., MRHD, MFed, and MLDP).}
\label{fi:RoadFedFramework}
\vspace{-2mm}
\end{figure*}


\textbf{IoT-based privacy-preserving techniques:}
Numerous techniques have been put forward to safeguard privacy, including encryption methods, anonymization, and differential privacy. Encryption techniques are typically employed to thwart unauthorized access, data misusage, and model recovery attacks (i.e., deducing parameters of a model) originating from unreliable edges or cloud environments, as discussed in____ and____. Nonetheless, the substantial computational and communication expenses associated with these techniques limit their practicality in real-life scenarios. The approach presented in____ enhances user privacy using pseudonyms and pseudonym certificates within fog computing. Similarly,____ engaged a trusted third party that acts as an intermediary to ensure the anonymity of the user. However, studies such as____ and____ demonstrated that merely removing or disguising user identity information might not effectively secure privacy, as deanonymization attacks can exploit existing knowledge to reconstruct this information.

In contrast to anonymization and encryption techniques, differential privacy____ offers robust privacy assurances that can simultaneously protect user data and model efficacy. LDP is a type of DP that safeguards user data directly from personal devices like smartphones and smartwatches. Consequently, LDP can maintain user privacy without relying on a trusted intermediary (such as unreliable edge or cloud servers). The randomized response method was applied to encode values in____ and____ to facilitate local privacy protection. This strategy is straightforward to implement without incurring additional computation costs; however, it performs poorly with high dimensional data. The works of____ and____ employed Expectation Maximization (EM) based methodologies, which allocate the privacy budget across the values of individual features for preserving the privacy of local users’ data, addressing both two-attribute and multi-attribute scenarios. EM-based methods can cause high variance because of the allocation of the privacy budget, making them less suitable for high dimensional datasets. The authors of____ utilized transformation techniques to convert data into binary strings. The randomized response technique was then applied to generate these strings, with the nearest center being communicated differentially privately. 
% Although____ managed to safeguard the grouping information of users, it is not optimal for high dimensional data and incurs additional costs for selecting cluster centers.
____ introduced a two-layer federated learning framework with local differential privacy at the vehicle level ensures secure and privacy-preserving data sharing in VANETs without relying on trusted third parties.
____ enhances trajectory data utility under local differential privacy by adaptively allocating privacy budgets via water-filling theory and optimizing user segmentation through entropy-driven grouping.

To summarize, first, deep learning approaches have been extensively investigated for classifying road hazards, yielding promising outcomes. However, the majority rely solely on one modality (i.e., images or videos). Given the vast amounts of multimodal data generated daily, incorporating multimodal information could significantly enhance the performance of single-modality models. To our best knowledge, there is no existing research addressing multimodal learning for road hazard detection. Second, methods that reduce communication frequency in collaborative learning are both straightforward and effective in lowering communication costs; however, high expenses persist due to the large sizes of models. Last, there is currently no LDP solution capable of managing real-world data encompassing multiple modalities and attributes, which leads to considerable errors when handling high dimensional data. To address these challenges, this paper introduces a multimodal FL system aimed at road hazard detection and alerting with minimal data exchange while protecting privacy.