 \def\isarxiv{1} %%% for icml submission version, we comment this line

\ifdefined\isarxiv
\documentclass[11pt]{article}

\usepackage[numbers]{natbib}

\else


%%%%%%%% ICML 2025 EXAMPLE LATEX SUBMISSION FILE %%%%%%%%%%%%%%%%%

\documentclass{article}

% Recommended, but optional, packages for figures and better typesetting:
\usepackage{microtype}
\usepackage{graphicx}
\usepackage{subfig}
% \usepackage{subfigure}
\usepackage{booktabs} % for professional tables

% hyperref makes hyperlinks in the resulting PDF.
% If your build breaks (sometimes temporarily if a hyperlink spans a page)
% please comment out the following usepackage line and replace
% \usepackage{icml2025} with \usepackage[nohyperref]{icml2025} above.
\usepackage{hyperref}


% Attempt to make hyperref and algorithmic work together better:
% \newcommand{\theHalgorithm}{\arabic{algorithm}}

% Use the following line for the initial blind version submitted for review:
% \usepackage{icml2025}

% If accepted, instead use the following line for the camera-ready submission:
\usepackage{icml2025}

% For theorems and such
\usepackage{amsmath}
\usepackage{amssymb}
\usepackage{mathtools}
\usepackage{amsthm}

% if you use cleveref..
\usepackage[capitalize,noabbrev]{cleveref}

%%%%%%%%%%%%%%%%%%%%%%%%%%%%%%%%
% THEOREMS
%%%%%%%%%%%%%%%%%%%%%%%%%%%%%%%%
\theoremstyle{plain}
\newtheorem{theorem}{Theorem}[section]
\newtheorem{proposition}[theorem]{Proposition}
\newtheorem{lemma}[theorem]{Lemma}
\newtheorem{corollary}[theorem]{Corollary}
\theoremstyle{definition}
\newtheorem{definition}[theorem]{Definition}
\newtheorem{assumption}[theorem]{Assumption}
\theoremstyle{remark}
\newtheorem{remark}[theorem]{Remark}

% Todonotes is useful during development; simply uncomment the next line
%    and comment out the line below the next line to turn off comments
%\usepackage[disable,textsize=tiny]{todonotes}
\usepackage[textsize=tiny]{todonotes}


% The \icmltitle you define below is probably too long as a header.
% Therefore, a short form for the running title is supplied here:
\icmltitlerunning{Fundamental Limits of Prompt Tuning VAR: Universality, Capacity and Efficiency}




\fi


\usepackage{amsmath}
\usepackage{amsthm}
\usepackage{amssymb}
% \usepackage{algorithm}
\usepackage{subfig}
% \usepackage{algpseudocode}
\usepackage{graphicx}
\usepackage{grffile}
\usepackage{wrapfig,epsfig}
\usepackage{url}
\usepackage{xcolor}
\usepackage{epstopdf}


\usepackage{bbm}
\usepackage{dsfont}

 
\allowdisplaybreaks
 

\ifdefined\isarxiv

\let\C\relax
\usepackage{tikz}
\usepackage{hyperref}  %%% arxiv don't allow this.
\hypersetup{colorlinks=true,citecolor=blue,linkcolor=blue} %%% Zhao : maybe we should comment this in submission.
\usetikzlibrary{arrows}
\usepackage[margin=1in]{geometry}

\else

\usepackage{microtype}
\usepackage{hyperref}
\definecolor{mydarkblue}{rgb}{0,0.08,0.45}
\hypersetup{colorlinks=true, citecolor=mydarkblue,linkcolor=mydarkblue}
 

\fi
 
\graphicspath{{./figs/}}

\theoremstyle{plain}
\newtheorem{theorem}{Theorem}[section]
\newtheorem{lemma}[theorem]{Lemma}
\newtheorem{definition}[theorem]{Definition}
\newtheorem{notation}[theorem]{Notation}
% \newtheorem{proof}[theorem]{Proof}
\newtheorem{proposition}[theorem]{Proposition}
\newtheorem{corollary}[theorem]{Corollary}
\newtheorem{conjecture}[theorem]{Conjecture}
\newtheorem{assumption}[theorem]{Assumption}
\newtheorem{observation}[theorem]{Observation}
\newtheorem{fact}[theorem]{Fact}
\newtheorem{remark}[theorem]{Remark}
\newtheorem{claim}[theorem]{Claim}
\newtheorem{example}[theorem]{Example}
\newtheorem{problem}[theorem]{Problem}
\newtheorem{open}[theorem]{Open Problem}
\newtheorem{property}[theorem]{Property}
\newtheorem{hypothesis}[theorem]{Hypothesis}

\newcommand{\wh}{\widehat}
\newcommand{\wt}{\widetilde}
\newcommand{\ov}{\overline}
\newcommand{\N}{\mathcal{N}}
\newcommand{\R}{\mathbb{R}}
\newcommand{\RHS}{\mathrm{RHS}}
\newcommand{\LHS}{\mathrm{LHS}}
\renewcommand{\d}{\mathrm{d}}
\renewcommand{\i}{\mathbf{i}}
\renewcommand{\tilde}{\wt}
\renewcommand{\hat}{\wh}
\newcommand{\Tmat}{{\cal T}_{\mathrm{mat}}}
\newcommand{\SETH}{\mathsf{SETH}}
\newcommand{\A}{\mathsf{A}}
\newcommand{\VAR}{\mathrm{VAR}}
\newcommand{\X}{\mathsf{X}}
\newcommand{\Y}{\mathsf{Y}}
\newcommand{\Z}{\mathsf{Z}}
\newcommand{\F}{\mathsf{F}}
\newcommand{\V}{\mathsf{V}}

\DeclareMathOperator*{\E}{{\mathbb{E}}}
\DeclareMathOperator*{\var}{\mathrm{Var}}
% \DeclareMathOperator*{\Z}{\mathbb{Z}}
\DeclareMathOperator*{\C}{\mathbb{C}}
\DeclareMathOperator*{\D}{\mathcal{D}}
\DeclareMathOperator*{\median}{median}
\DeclareMathOperator*{\mean}{mean}
\DeclareMathOperator{\OPT}{OPT}
\DeclareMathOperator{\supp}{supp}
\DeclareMathOperator{\poly}{poly}

\DeclareMathOperator{\nnz}{nnz}
\DeclareMathOperator{\sparsity}{sparsity}
\DeclareMathOperator{\rank}{rank}
\DeclareMathOperator{\diag}{diag}
\DeclareMathOperator{\dist}{dist}
\DeclareMathOperator{\cost}{cost}
\DeclareMathOperator{\vect}{vec}
\DeclareMathOperator{\tr}{tr}
\DeclareMathOperator{\dis}{dis}
\DeclareMathOperator{\cts}{cts}
% \DeclareMathOperator{\X}{{\mathsf{X}}}



\makeatletter
\newcommand*{\RN}[1]{\expandafter\@slowromancap\romannumeral #1@}
\makeatletter
% \newcommand*{\RN}[1]{\expandafter\@slowromancap\romannumeral #1@}
% \makeatother
% \newcommand{\Zhao}[1]{{\color{red}[Zhao: #1]}}
% \newcommand{\Zhenmei}[1]{{\color{purple}[Zhenmei: #1]}}
% \newcommand{\Yifang}[1]{{\color{blue}[Yifang: #1]}} 
% \newcommand{\Xiaoyu}[1]{{\color{orange}[Xiaoyu: #1]}} %%%Change to intern name



\usepackage{lineno}
\def\linenumberfont{\normalfont\small}


% below is the command for this paper only Zhenmei Dec 21 2024 
% \newcommand{\f}{f}
% \newcommand{\h}{h}
% \renewcommand{\c}{c}
% \newcommand{\q}{q}
% \renewcommand{\L}{L}
% \renewcommand{\r}{r}
% \newcommand{\p}{p}


\newcommand{\f}{s}
\newcommand{\h}{v}
\renewcommand{\c}{\ensuremath{\ell}}
\newcommand{\q}{\ensuremath{\beta}}
\renewcommand{\L}{\ensuremath{\mathsf{Loss}}}
\renewcommand{\r}{\ensuremath{\rho}}
\newcommand{\p}{\ensuremath{\gamma}}


\begin{document}

\ifdefined\isarxiv

\date{}


\title{Universal Approximation of Visual Autoregressive Transformers}
\author{
Yifang Chen\thanks{\texttt{
yifangc@uchicago.edu}. The University of Chicago.}
\and
Xiaoyu Li\thanks{\texttt{
xiaoyu.li2@student.unsw.edu.au}. University of New South Wales.}
\and
Yingyu Liang\thanks{\texttt{
yingyul@hku.hk}. The University of Hong Kong. \texttt{
yliang@cs.wisc.edu}. University of Wisconsin-Madison.} 
\and
Zhenmei Shi\thanks{\texttt{
zhmeishi@cs.wisc.edu}. University of Wisconsin-Madison.}
\and 
Zhao Song\thanks{\texttt{ magic.linuxkde@gmail.com}. The Simons Institute for the Theory of Computing at UC Berkeley.}
}



\else


\twocolumn[
\icmltitle{Universal Approximation of Visual Autoregressive Transformers}

% It is OKAY to include author information, even for blind
% submissions: the style file will automatically remove it for you
% unless you've provided the [accepted] option to the icml2025
% package.

% List of affiliations: The first argument should be a (short)
% identifier you will use later to specify author affiliations
% Academic affiliations should list Department, University, City, Region, Country
% Industry affiliations should list Company, City, Region, Country

% You can specify symbols, otherwise they are numbered in order.
% Ideally, you should not use this facility. Affiliations will be numbered
% in order of appearance and this is the preferred way.
\icmlsetsymbol{equal}{*}

\begin{icmlauthorlist}
\icmlauthor{Firstname1 Lastname1}{equal,yyy}
\icmlauthor{Firstname2 Lastname2}{equal,yyy,comp}
\icmlauthor{Firstname3 Lastname3}{comp}
\icmlauthor{Firstname4 Lastname4}{sch}
\icmlauthor{Firstname5 Lastname5}{yyy}
\icmlauthor{Firstname6 Lastname6}{sch,yyy,comp}
\icmlauthor{Firstname7 Lastname7}{comp}
%\icmlauthor{}{sch}
\icmlauthor{Firstname8 Lastname8}{sch}
\icmlauthor{Firstname8 Lastname8}{yyy,comp}
%\icmlauthor{}{sch}
%\icmlauthor{}{sch}
\end{icmlauthorlist}

\icmlaffiliation{yyy}{Department of XXX, University of YYY, Location, Country}
\icmlaffiliation{comp}{Company Name, Location, Country}
\icmlaffiliation{sch}{School of ZZZ, Institute of WWW, Location, Country}

\icmlcorrespondingauthor{Firstname1 Lastname1}{first1.last1@xxx.edu}
\icmlcorrespondingauthor{Firstname2 Lastname2}{first2.last2@www.uk}

% You may provide any keywords that you
% find helpful for describing your paper; these are used to populate
% the "keywords" metadata in the PDF but will not be shown in the document
\icmlkeywords{Machine Learning, ICML}

\vskip 0.3in
]

% this must go after the closing bracket ] following \twocolumn[ ...

% This command actually creates the footnote in the first column
% listing the affiliations and the copyright notice.
% The command takes one argument, which is text to display at the start of the footnote.
% The \icmlEqualContribution command is standard text for equal contribution.
% Remove it (just {}) if you do not need this facility.

%\printAffiliationsAndNotice{}  % leave blank if no need to mention equal contribution
\printAffiliationsAndNotice{\icmlEqualContribution} % otherwise use the standard text.


\fi





\ifdefined\isarxiv
\begin{titlepage}
  \maketitle
  \begin{abstract}
\begin{abstract}

% Recent works to jointly reconstruct 3D human and object from a single RGB image, are mostly model-based, that fail to capture the fine details of the clothed human body and object surface. In this paper, we introduce ReCHOR, a novel, model-free, first-method to produce realistic clothed human-object reconstructions from a monocular view. This is extremely challenging due to human-object occlusions, diverse interactions and depth ambiguity, as it needs to infer both 3D spatial awareness and high resolution details. Our core idea is based on estimating neural implicit representations for human and object respectively by an attention-based neural implicit model that attends to pixel-aligned features from both the global human-object image for spatial awareness and  the local separate view of human and object images for high quality details. Additionally, the network is conditioned on semantic features from an initial estimated human-object pose prior and a generative diffusion model that inpaints occluded regions, thus enabling the retrieval of details from them.
% We also propose a synthetic dataset with rendered scenes of diverse, inter-occluded 3D human and object scans, to train our network. We evaluate our method on the synthetic and real world BEHAVE dataset. Our experiments show that our method outperforms the SOTA in achieving realistic clothed human-object reconstructions.
Recent approaches to jointly reconstruct 3D humans and objects from a single RGB image represent 3D shapes with template-based or coarse models, which fail to capture details of loose clothing on human bodies. In this paper, we introduce a novel implicit approach for jointly reconstructing realistic 3D clothed humans and objects from a monocular view. For the first time, we model both the human and the object with an implicit representation, allowing to capture more realistic details such as clothing. This task is extremely challenging due to human-object occlusions and the lack of 3D information in 2D images, often leading to poor detail reconstruction and depth ambiguity. To address these problems, we propose a novel attention-based neural implicit model that leverages image pixel alignment from both the input human-object image for a global understanding of the human-object scene and from local separate views of the human and object images to improve realism with, for example, clothing details. Additionally, the network is conditioned on semantic features derived from an estimated human-object pose prior, which provides 3D spatial information about the shared space of humans and objects. To handle human occlusion caused by objects, we use a generative diffusion model that inpaints the occluded regions, recovering otherwise lost details. For training and evaluation, we introduce a synthetic dataset featuring rendered scenes of inter-occluded 3D human scans and diverse objects. Extensive evaluation on both synthetic and real-world datasets demonstrates the superior quality of the proposed human-object reconstructions over competitive methods.
\end{abstract}

  \end{abstract}
  \thispagestyle{empty}
\end{titlepage}

{\hypersetup{linkcolor=black}
\tableofcontents
}
\newpage

\else

\begin{abstract}
\begin{abstract}

% Recent works to jointly reconstruct 3D human and object from a single RGB image, are mostly model-based, that fail to capture the fine details of the clothed human body and object surface. In this paper, we introduce ReCHOR, a novel, model-free, first-method to produce realistic clothed human-object reconstructions from a monocular view. This is extremely challenging due to human-object occlusions, diverse interactions and depth ambiguity, as it needs to infer both 3D spatial awareness and high resolution details. Our core idea is based on estimating neural implicit representations for human and object respectively by an attention-based neural implicit model that attends to pixel-aligned features from both the global human-object image for spatial awareness and  the local separate view of human and object images for high quality details. Additionally, the network is conditioned on semantic features from an initial estimated human-object pose prior and a generative diffusion model that inpaints occluded regions, thus enabling the retrieval of details from them.
% We also propose a synthetic dataset with rendered scenes of diverse, inter-occluded 3D human and object scans, to train our network. We evaluate our method on the synthetic and real world BEHAVE dataset. Our experiments show that our method outperforms the SOTA in achieving realistic clothed human-object reconstructions.
Recent approaches to jointly reconstruct 3D humans and objects from a single RGB image represent 3D shapes with template-based or coarse models, which fail to capture details of loose clothing on human bodies. In this paper, we introduce a novel implicit approach for jointly reconstructing realistic 3D clothed humans and objects from a monocular view. For the first time, we model both the human and the object with an implicit representation, allowing to capture more realistic details such as clothing. This task is extremely challenging due to human-object occlusions and the lack of 3D information in 2D images, often leading to poor detail reconstruction and depth ambiguity. To address these problems, we propose a novel attention-based neural implicit model that leverages image pixel alignment from both the input human-object image for a global understanding of the human-object scene and from local separate views of the human and object images to improve realism with, for example, clothing details. Additionally, the network is conditioned on semantic features derived from an estimated human-object pose prior, which provides 3D spatial information about the shared space of humans and objects. To handle human occlusion caused by objects, we use a generative diffusion model that inpaints the occluded regions, recovering otherwise lost details. For training and evaluation, we introduce a synthetic dataset featuring rendered scenes of inter-occluded 3D human scans and diverse objects. Extensive evaluation on both synthetic and real-world datasets demonstrates the superior quality of the proposed human-object reconstructions over competitive methods.
\end{abstract}
\end{abstract}

\fi


\section{Introduction}
\label{sec:intro}
% Image editing methods in diffusion models depend on user-defined control directions - users can unlock their creativity using these methods by specifying the desired manipulation through prompts~\cite{gandikota2023concept}, reference images~\cite{ruiz2022dreambooth, kumari2022customdiffusion, gal2022image, chen2024trainingfreeregionalpromptingdiffusion}, or attribute vectors~\cite{parmar2023zero,hertz2022prompt}. In this work, we ask a fundamentally different question: \emph{Can we automatically discover the underlying visual structure of a concept within diffusion model's knowledge?} %Rather than requiring user-specified controls, we aim to decompose the model's internal knowledge into meaningful directions.

% This question touches on a fundamental limitation in how we interact with diffusion models. Current control methods ~\cite{zhang2023addingconditionalcontroltexttoimage, gandikota2023concept, ye2023ipadaptertextcompatibleimage,ye2023ipadaptertextcompatibleimage, hertz2024stylealignedimagegeneration, li2023photomaker, shi2024instantbooth, chen2024trainingfreeregionalpromptingdiffusion} require users to specify their desired manipulations in advance, limiting interactive creativity. This contrasts with natural human artistic workflows, where creators dynamically explore creative ideas while jointly refining them toward meaningful artistic outcomes~\cite{hoffmann2016modeling}. This synergy between specification and exploration is not new to generative models. Early GAN architectures naturally developed disentangled latent spaces that enabled continuous\cite{harkonen2020ganspace,radford2015unsupervised, wu2021stylespace, shen2020interfacegan}, compositional control over generated images. Users could explore these spaces to discover interesting variations that would be difficult to describe in words~\cite{wu2021stylespace}, then combine them to achieve their creative goals~\cite{grabe2022towards}. 


% While diffusion models have largely superseded GANs in conditional image synthesis~\cite{dhariwal2021diffusion},  their underlying structure remains less understood. Diffusion models achieve remarkable diversity through high-dimensional latents, unlike GANs' compact latent spaces.  With a single prompt, diffusion models can generate radically different variations through different random initializations of input noise. We ask - Is it possible to discover interpretable structure within this vast space of variations?

Text-to-image diffusion models are capable of generating remarkable visual variations from a single prompt through different random initializations. However, this vast creative potential remains largely opaque to users---while we can generate diverse images, we lack understanding of the underlying structure of these variations. This presents a fundamental challenge: how can we discover and expose the latent visual capabilities encoded within these models?

\let\thefootnote\relax \footnote{$^{*}$Correspondence to \texttt{gandikota.ro@northeastern.edu}}

The challenge touches on a key limitation in how we interact with diffusion models today. Current control methods require users to explicitly specify their desired edits in advance through prompts~\cite{gandikota2023concept}, reference images~\cite{zhang2023addingconditionalcontroltexttoimage, chen2024trainingfreeregionalpromptingdiffusion, ruiz2022dreambooth,kumari2022customdiffusion, Ryu_lora, hu2021lora}, or attribute vectors~\cite{ye2023ipadaptertextcompatibleimage, hertz2024stylealignedimagegeneration, li2023photomaker, shi2024instantbooth,parmar2023zero,hertz2022prompt}. That contrasts sharply with natural human creative workflows, where artists dynamically explore creative ideas and jointly refine them toward meaningful artistic outcomes~\cite{hoffmann2016modeling}. The need for pre-specified controls creates a barrier between users and the full creative potential of these models.

Interestingly, earlier generative models like GANs~\cite{gans,karras2019style,brock2018large} naturally developed more interpretable internal structures. Their compact latent spaces often exhibited emergent disentanglement~\cite{harkonen2020ganspace,radford2015unsupervised, wu2021stylespace, shen2020interfacegan}, enabling continuous and compositional control over generated images. Users could explore these spaces to discover interesting variations that would be difficult to describe in words~\cite{wu2021stylespace}, then combine them to achieve their creative goals~\cite{grabe2022towards}.

Diffusion models have largely superseded GANs in conditional image synthesis~\cite{dhariwal2021diffusion}, achieving greater diversity through much higher-dimensional latents. And yet an understanding of the underlying structure of these larger latent spaces has remained elusive. In this work, we ask a fundamental question: \emph{Can we automatically discover the visual structure within a diffusion model's knowledge of a concept?} Rather than requiring user-specified controls, we aim to decompose the model's internal representations into expressive directions that users can explore and combine.

To address these needs, we present \textbf{SliderSpace}, a framework that brings systematic explorability to diffusion models. Given just a text prompt, SliderSpace discovers a canonical set of meaningful, diverse, and controllable directions within the model's knowledge of that concept. Each direction is implemented as a low-rank adapter~\cite{hu2021lora} that can be scaled and composed with others, allowing users to explore and smoothly combine different aspects of variation, as shown in Figure~\ref{fig:intro}.

We ground SliderSpace discovery in three key requirements for meaningful decomposition of a diffusion model's visual manifold: 
\begin{enumerate}
    \item \textbf{Unsupervised Discovery:} The decomposition process should emerge from the intrinsic structure of the model's learned representation, rather than being guided by predefined attributes. This ensures we capture the true topology of the model's knowledge space rather than projecting our assumptions onto it.
    
    \item \textbf{Semantic Orthogonality:} Each discovered control must represent a distinct semantic direction. This is enforced in a semantic feature space, like CLIP, where every slider has an orthogonal effect in embeddings. This prevents discovering multiple controls that create similar semantic effects, making the system more efficient and easier.
    
    \item \textbf{Distribution Consistency:} Directions must induce consistent transformations across both random seeds and prompt variations. 
\end{enumerate}

These requirements naturally lead to our proposed framework, which we formalize in Section~\ref{sec:method}. As we show in our experiments, SliderSpace is architecture-agnostic, working with both conventional U-Net based models like Stable Diffusion~\cite{rombach2022high, rombach2022sd20, podell2023sdxl, turbo, dmd} and recent transformer-based architectures like Flux~\cite{flux}.

We demonstrate the expressiveness of SliderSpace through three applications: First, we show how SliderSpace can decompose high-level concepts into diverse and expressive components, revealing the natural axes of variation in the model's understanding. Second, we explore artistic style variation, where SliderSpace discovers directions that match or exceed the diversity of manually curated artist lists while being judged more useful by human evaluators. Finally, we show how SliderSpace can help reverse the mode collapse commonly observed in distilled diffusion models, restoring diversity while maintaining generation speed.

Beyond providing practical creative control, SliderSpace opens new avenues for understanding and utilizing the latent capabilities of diffusion models. By mapping these models' visual potential into intuitive, composable directions, we take a step toward making their creative possibilities more accessible and interpretable to users.

% Image editing methods in diffusion models unlock the creativity of users. In this work we ask an alternate question: \emph{Can we organize and expose what of the diffusion model is already capable of?}.
% Existing methods for controlling image generation typically require users to manually specify edit directions for desired changes. This process is time-consuming, requires technical expertise, and limits the spontaneity of the creative process. For instance, if a user wants to adjust the smile of a generated person, they must explicitly request this edit, often through imprecise prompt engineering or model fine-tuning. This approach of predefined controls or manual specifications restricts users from fully exploring the latent capabilities of the model. There may be interesting stylistic variations or attributes that the model can generate, but users have no easy way to discover or utilize these.

% Natural visual disentanglement was an emergent property in the latent space of Generative Adversarial Models (GANs) \cite{harkonen2020ganspace,radford2015unsupervised, wu2021stylespace, shen2020interfacegan}. In particular, it has been observed that StyleGAN~\cite{karras2019style} stylespace neurons offer detailed control over many meaningful aspects of images that would be difficult to describe in words~\cite{wu2021stylespace}. However, diffusion models do not share such a compact latent space~\cite{park2023unsupervised}; and efforts to uncover such a space in the semantic embeddings of the text conditioning have met with limited success \nik{Nick - is there a specific citation you were thinking about?}.

% In this work we introduce \textbf{SliderSpace}, which takes a step towards uncovering an analogous low dimensional representation of diffusion models' visual breadth; in essence treating the diffusion model as many generators sharing parameters, where a particular generator is defined by a specific prompt. For a given prompt we sample many random seeds (and optionally prompt expansions using an LLM), generate the corresponding images, and apply an off the shelf feature extractor (in this work CLIP, but our method can be applied to any differentiable feature extractor). We use PCA to analyze these features, and for each of the leading $k$ principal components we train a LoRA \cite{} which causes the diffusion model to produces images which increase the feature magnitude along that component when passed back through the same feature extractor. This leads to a 'Slider' for each principal component, because each LoRA can be scaled and applied to the original diffusion model, continuously varying those visual features in the generated results (as measured, in our case, by CLIP).

% There are many other works that enhance the controllability of diffusion models. One common approach is enabling users to add spatial constraints to a generation either manually, or via a reference image \cite{zhang2023addingconditionalcontroltexttoimage, chen2024trainingfreeregionalpromptingdiffusion}, a second is leveraging more abstract embeddings (e.g. identity, style) extracted from a reference image \cite{ye2023ipadaptertextcompatibleimage, hertz2024stylealignedimagegeneration, li2023photomaker, shi2024instantbooth}, a third is finetuning a foundation model to better generate a concept important to the user \cite{ruiz2022dreambooth, kumari2022customdiffusion, Ryu_lora, hu2021lora}, and a fourth (most relevant to this work) is finding low-rank adaptors of the model based on a prompt or small training set which can be scaled to provide continous control over one aspect of generated image (e.g. night vs day, basic vs luxury, etc.) \cite{gandikota2023concept}. SliderSpace is complementary to all of these methods and offers something distinct. All of the other methods we are aware require the user (and / or model designer) to know in advance what type of control they want. In contrast SliderSpace assists users in discovering and controlling hidden capabilities present in the diffusion model's distribution of possible generations.

%We propose that truly intuitive creative control in a text-to-image model should meet three key criteria: \emph{discoverability}, \emph{intuitiveness}, and \emph{specificity}. The model should reveal controllable attributes that may not be immediately obvious, offer controls that are easy to understand and manipulate, and ensure each control affects a distinct attribute of the generated image.

% We demonstrate the utility and power of SliderSpace using three applications built on top of SDXL-DMD \cite{dmd}, because its fast generation speed lends itself well to the continuous control offered by SliderSpace.

% First, we study concept decomposition (Section \ref{sec:concept_exp}), where we learn sliders for a specific concept (e.g. 'monster', 'waterfall', 'car'). Through quantitative metrics of diversity and text alignment we demonstrate that the learned sliders dramatically boost the diversity of generations when randomly applied without harming text alignment; we also ask humans to qualitatively judge these results in a user study where they find the SliderSpace results to be more 'Diverse', 'Useful', and 'Creative' than our baselines.

% Second, we attempt to compare the automatic discoveries of SliderSpace to a large scale manual study of artistic styles (Section \ref{sec:art_exp}), open-sourced by ParrotZone \cite{parrotzone}. In this study SDXL was prompted with over 4300 artist names,  and based on visual inspection the cases of successful stylistic mimicry recorded. Quantitatively SliderSpace more closely matches the distribution of artistic variation discovered by ParrotZone than other baselines, and in our user studies was judged to be significantly more 'Diverse' and 'Useful' than the baselines. To our surprise humans even judged SliderSpace results to be slightly more 'Diverse' than the results generated by the manually discovered artist names of \cite{parrotzone}.

% Third, we attempt to use SliderSpace to reverse the mode collapse commonly observed in distilled few-step diffusion models relative to the original teacher model (Section \ref{sec:diverse_exp}). We quantitatively demonstrate that applying SliderSpace to SDXL-DMD leads to more closely matching the distribution of images by the original teacher, SDXL.

%Through extensive experiments on various state-of-the-art text-to-image models, we demonstrate that SliderSpace significantly enhances user control and creative expression in AI-assisted image generation tasks. Our method enables a range of applications, including concept decomposition and control, diversity improvement in generated images, customization dissection and edits, and the exploration of artistic styles inherent in the model.

% SliderSpace goes beyond providing a practical tool for enhanced creative control. By mapping the visual potential of diffusion models it can open new avenues for generative creativity and deepens our understanding of each model's hidden potential. %%% Section 1. Introduction
\section{Related Work}
\label{sec:related_work}

The original investigation \cite{gibson1979ecological} on the relationship between visual perception and human action defines \emph{affordance} as the opportunities for interaction with the surrounding environment. Behavioral studies on regular and cognitively impaired persons have shown evidence that perception results in both visual and motor signals in the human brain. An extended study \cite{anderson2002attentional} shows that visual attention to the spatial characteristics of the perceived objects initiates automatic motor signals for different actions. In computer vision, human affordance learning involves novel pose prediction such that the estimated pose represents a valid human action within the scene context. The task is fundamental to many problems requiring robust semantic reasoning about the environment, such as human motion synthesis \cite{wang2021scene} and scene-aware human pose generation \cite{wang2017binge, roy2016multi, zhang2022inpaint, yao2023scene}.

Earlier methods of affordance learning have explored knowledge mining \cite{zhu2014reasoning} and multimodal feature cues \cite{roy2016multi} to address the problem. In \cite{zhu2014reasoning}, the authors use a Markov Logic Network for constructing a knowledge base by extracting several object attributes from different image and metadata sources, which can perform various downstream visual inference tasks without any additional classifier, including zero-shot affordance prediction. In \cite{roy2016multi}, the authors use depth map, surface normals, and segmentation map as multimodal cues to train a multi-scale convolutional neural network (CNN) for scene-level semantic label assignment associated with specific human actions. In \cite{do2018affordancenet}, the authors design a multi-branch end-to-end CNN with two separate pathways for object detection and affordance label assignment to achieve high real-time inference throughput. Researchers \cite{chuang2018learning} have also explored socially imposed constraints for affordance learning. In \cite{chuang2018learning}, the authors propose a graph neural network (GNN) to propagate contextual scene information from egocentric views for action-object affordance reasoning.

Probabilistic modeling of scene-aware human motion generation also involves semantic reasoning of human interaction with the environment. Initial works on human motion synthesis have taken different architectural approaches, such as sequence-to-sequence models \cite{barsoum2018hp}, generative adversarial networks (GAN) \cite{barsoum2018hp, cai2018deep, yang2018pose}, graph convolutional networks (GCN) \cite{yan2019convolutional}, and variational autoencoders (VAE) \cite{guo2020action2motion}. However, these methods have mostly ignored the role of environmental semantics. Due to potential uncertainty in human motion, in a recent approach \cite{wang2021scene}, the authors address such motion synthesis with a GAN conditioned on scene attributes and motion trajectory to predict probable body pose dynamics.

One key challenge of human affordance generation in 2D scenes is the lack of large-scale datasets with rich pose annotations. In \cite{wang2017binge}, the authors compile the only public dataset of annotated human body poses in complex 2D indoor scenes by extracting frames from sitcom videos. Aiming to generate a contextually valid human affordance at a user-defined location, the authors propose sampling the scale and deformation parameters for an existing human pose template using a VAE conditioned on the localized image patches as scene context. In \cite{zhang2022inpaint}, the authors introduce a two-stage GAN architecture for achieving a similar goal by estimating the affine bounding box parameters to localize a probable human in the scene and then generating a potential body pose at that location. The method uses the input scene, corresponding depth, and segmentation maps as semantic guidance. In \cite{yao2023scene}, the authors propose a transformer-based approach with knowledge distillation for generating human affordances in 2D indoor scenes.


\section{Preliminary} \label{sec:preli}

In Section~\ref{sub:notation}, we introduce all the notations we used in our paper. Then, in Section~\ref{sub:flow_matching}, we show the basic facts about flow matching. In Section~\ref{sub:special_relativity}, we present the basic background of special relativity and define the relativistic force.

\subsection{Notations} \label{sub:notation}

For any positive integer $n$, we use $[n]$ to denote set $\{1,2,\cdots, n\}$. 
For two vectors $x \in \R^n$ and $y \in \R^n$, we use $\langle x, y \rangle$ to denote the inner product between $x,y$.
For a vector $v \in \R^n$, we use $\|v\|_2$ to denote the $\ell_2$-norm of $v$.
We use ${\bf 1}_n$ to denote a length-$n$ vector where all the entries are ones.
We use the symbol $ \perp $ to represent a component that is perpendicular to the direction of velocity, as exemplified by $ a_{\perp t} $, which denotes the perpendicular acceleration. Similarly, the symbol $ \parallel $ is employed to indicate a component parallel to the direction of velocity, such as $ f_{\parallel t} $, which represents the parallel force. We use $\dot{x}_t$ to denote $\frac{\d x_t}{\d t}$, and $\ddot{x}_t$ to denote $\frac{\d^2 x_t}{\d t^2}$.


\subsection{Flow Matching} \label{sub:flow_matching}

Flow Matching (FM) \cite{lcb+22,lgl22} is a generative modeling technique that constructs a smooth, invertible (i.e., diffeomorphic) mapping from a simple prior distribution to a complex target distribution. In FM, a time-dependent mapping $Z_t$ is defined to evolve according to an ordinary differential equation (ODE) driven by a vector field:
\begin{align*}
    \frac{\d x_t}{\d t} = V_t(x_t), \quad t \in [0, T].
\end{align*}
The goal is to ensure that, at the terminal time $T$, the ODE transforms a sample $x_0$ from a simple distribution (e.g., a Gaussian) into a sample $x_T$ from the target data distribution $\mathcal{D}$.

To achieve this, Flow Matching (FM) constructs a stochastic interpolation between a sample $x_1 \sim \mathcal{D}$ and a sample $x_0$ drawn from a known prior distribution, typically $\N(0,I)$. The interpolation is defined as
\begin{align*}
    x_t := \alpha_t x_1 + \sigma_t x_0, \quad t\in [0,T],
\end{align*}
where the time-dependent coefficients $\alpha_t$ and $\sigma_t$ are chosen so that
\begin{align*}
    \alpha_0 = 0,\quad \sigma_0 = 1,\quad \alpha_T = 1,\quad \sigma_T = 0.
\end{align*}
Thus, at $t=0$ the interpolated sample is purely the prior ($x_0$), and at $t=T$ it becomes a data sample ($x_1$).

The instantaneous change of $x$ is obtained by differentiating the interpolation:
\begin{align*}
    \frac{\d x_t}{\d t} = \frac{\d \alpha_t}{\d t} x_1 + \frac{\d \sigma_t}{\d t} x_0.
\end{align*}

The vector field is approximated by a neural network $V_t(x_t)$ with learnable parameters $\theta$. The FM training objective is then given by
\begin{align*}
    \mathcal{L}_\mathrm{FM}(\theta) := \E_{t\sim {\sf Uniform}[0,T], x_1 \sim \mathcal{D}} [\| V_t(x_t) - v_t(x_t) \|_2^2 ].
\end{align*}
This loss ensures that the learned velocity field $V_t(x_t)$ closely tracks the conditional dynamics $v_t(x_t)$ along the interpolation path.

After training, samples are generated by solving the ODE
\begin{align*}
    \frac{\d x_t}{\d t} = V_t(x_t),
\end{align*}
starting from an initial sample $x_0 \sim \N(0,I)$. Integrating this ODE from $t=0$ to $t=T$ yields a sample $x_T$ that approximates a draw from the target distribution. This ODE-based formulation offers a flexible and powerful framework for modeling complex data distributions while naturally incorporating conditional sampling.

\subsection{Background on Special Relativity} \label{sub:special_relativity}

We first introduce several essential ideas of special relativity \cite{e+05}.

\begin{definition}[Lorentz Factor]
\label{def:LorentzFactor}
According to special relativity~\cite{e+05}, the Lorentz factor at lab time $t$ is given by
\begin{align*}
\gamma_t := \frac{1}{\sqrt{1 - {\|v_t^{\rm lab}\|_2^2}/{c^2}}},
\end{align*}
where $v_t^{\rm lab}$ is the velocity at lab frame of reference, $c = 3 \times 10^8$ is the speed of light in vacuum.
\end{definition}

Then, we introduce the proper time of special relativity.

\begin{definition}[Proper Time]
\label{def:ProperTime}
The proper time is defined as the time interval measured in the rest frame of a moving object according to special relativity~\cite{e+05}. The differential form of the proper time is given by
\begin{align*}
    \d \tau = \frac{\d t}{\gamma_t},
\end{align*}
where $\d t$ is the time interval in the laboratory frame of reference, and $\gamma_t$ is the Lorentz factor at time lab time $t$ as defined in Definition~\ref{def:LorentzFactor}.
\end{definition}

Next, we define the force under special relativity here.

\begin{definition}[Relativistic Force]
\label{def:RelativisticForce}
In the framework of special relativity, the \emph{local force} (i.e., the force measured in the instantaneous rest frame of the particle) denoted as $f^{\rm local}$ has
\begin{align}
    f^{\rm local} := \frac{\d p^{\rm lab}}{\d \tau}, \label{eq:f_local}
\end{align}
where $p^{\rm lab}$ is the momentum at lab frame of reference, $\tau$ denotes the proper time defined in Definition~\ref{def:ProperTime}.

The momentum in the lab frame is defined as
\begin{align}
    p^{\rm lab} := m^{\rm lab} v_t^{\rm lab}, \label{eq:p}
\end{align}
where $m^{\rm lab}$ is the mass at lab frame of reference, and $v_t^{\rm lab}$ is the velocity at lab frame of reference.
\end{definition}

We state an equivalence lemma. Due to the space limitation, we delayed the proofs into the appendix.
\begin{lemma}[Equivalent Form of Relativistic Force, informal version of Lemma~\ref{lem:equiv_relativistic_force:formal}]\label{lem:equiv_relativistic_force:informal}
Let $p^{\rm lab}$ be the momentum defined in Eq.~\eqref{eq:p}, $\gamma_t$ be the Lorentz factor at lab time $t$ defined in Definition~\ref{def:LorentzFactor}, $\tau$ denotes the proper time, $v_t^{\rm lab} = \dot{x}_t$ denotes the velocity, 
$a_t^{\rm lab} = \ddot{x}_t$ denotes the acceleration.
The relativistic force, defined as the time derivative of the momentum in the lab frame, can be written as
\begin{align*}
f^{\rm local} =  m^{\rm lab}  (\gamma_t a_t^{\rm lab} + \gamma_t^3 \frac{ \langle v_t^{\rm lab}, a_t^{\rm lab} \rangle}{c^2} v_t^{\rm lab}).
\end{align*}

\end{lemma} 

\section{Any-Rank Single-Layer Attention is a Contextual Mapping Function}
\label{sec:context_map}
In this section, we show that Attention is a contextual mapping function. 
In Section~\ref{sec:cont_map}, we give the definition of contextual mapping.
In Section~\ref{sec:any_rank_attn_contmap}, we introduce any-rank single-layer attention as a contextual mapping function. 



\subsection{Contextual Mapping}\label{sec:cont_map}


{\bf Contextual Mapping.}
Let $X, Y \in \R^{n\times d}$ be the input embeddings and output label sequences, respectively. 
Let $X_{i} \in \R^{d}$ be the $i$-th token of each $X$ embedding sequence. 

\begin{definition}[Vocabulary, Definition 2.4 from~\cite{hwg+24} on Page 8]\label{def:vocab}
    We define the vocabulary.
    \begin{itemize}
        \item We define the $i$-th vocabulary set for $i \in [N]$ by $\mathcal{V}^{(i)}=\cup_{k \in[n]} X_k^{(i)} \subset \R^{d}$.
        \item We define the whole vocabulary set $\mathcal{V}$ as $\mathcal{V}=\cup_{i \in[N]} \mathcal{V}^{(i)} \subset \R^{d}$.
    \end{itemize}
\end{definition}
Note that while ``vocabulary'' typically refers to the tokens' codomain, here, it refers to the set of all tokens within a single sequence.
To facilitate our analysis, 
we introduce the idea of input token separation following~\cite{ks24,kkm22,ybr+20}.

\begin{definition}[Tokenwise Separateness, Definition 2.5 from~\cite{hwg+24} on Page 8] \label{def:token_seperate_new}
    We define the tokenwise separateness as follows.
    \begin{itemize}
        \item Let $X^{(1)}, \hdots, X^{(N)} \in \R^{n\times d}$ be embeddings. 
        \item Let $N$ be the number of sequences in the datasets.
        \item Let $n$ be the length of a sequence. i.e. $X^{(i)} \in \R^{n\times d}$
    \end{itemize}
    
    First, we state three conditions for $X^{(1)}, \hdots, X^{(N)}$ 
    \begin{itemize}
        \item [(i)] For any $i \in[N]$ and $k \in[n],\| X_k^{(i)}\|_2 > \gamma_{\min }$ holds. 
        \item [(ii)]
        For any $i \in[N]$ and $k \in[n], \|X_k^{(i)}\|_2<\gamma_{\max }$ holds.
        \item [(iii)]
        For any $i, j \in [N]$ and $k, l \in [n]$ if $X_k^{(i)} \neq X_ l^{(j)}, $ then $ \|X_k^{(i)}- X_l^{(j)}\|_2 > \delta$ holds.
    \end{itemize}
    Second, we define three types of separateness as follows, 
    \begin{itemize}
        \item {\bf Part 1.} If all conditions hold, then we call it  tokenwise $(\gamma_{\min }, \gamma_{\max }, \delta)$-separated
        \item {\bf Part 2.} If conditions (ii) and (iii) hold, then we denote this as $(\gamma, \delta)$-separateness.
        \item {\bf Part 3.} If only condition (iii) holds, then we denote it as $(\delta)$-separateness.
    \end{itemize} 
\end{definition}
To clarify condition (iii), we consider cases where there are repeated tokens between different input sequences. 
Next, we define contextual mapping. 
Contextual mapping describes a function's ability to capture the context of each input sequence as a whole and assign a unique ID to each input sequence.


\begin{definition} [$(\gamma,\delta)$-Contextual Mapping, Definition 2.6 from~\cite{hwg+24} on Page 8] \label{def:contextual_mapping_new}
A function $q:\R^{n\times d} \to \R^{n\times d}$ is said to be a $(\gamma, \delta)$-contextual mapping for a set of embeddings  $X^{(1)}, \hdots, X^{(N)} \in \R^{n\times d}$,  if the following conditions hold:
    \begin{itemize}
        \item {\bf Contextual Sensitivity $\gamma$.}
        For any $i \in[N]$ and $k \in[n],  \|q (X^{(i)})_k\|_2 < \gamma$ holds.
    
        \item {\bf Approximation Error $\delta$.}
        For any $i, j \in[N]$ and $k, l \in[n]$ such that $\mathcal{V}^{(i)} \neq \mathcal{V}^{(j)}$ or $X_k^{(i)} \neq X_l^{(j)}$, $\|q(X^{(i)})_k-q(X^{(j)})_l\|_2 > \delta$ holds.
    \end{itemize}
In addition, Note that $q (X^{(i)})$ for $i \in[N]$ is called a context ID of $X^{(i)}$.    
\end{definition}


\subsection{Any-Rank Single-Layer Attention is a Contextual Mapping Function}\label{sec:any_rank_attn_contmap}

Now we present the result showing that a softmax-based $1$-head, $1$-layer attention block with any-rank weight matrices is a contextual mapping.

\begin{lemma}[Any-Rank Attention as a $(\gamma, \delta)$-Contextual Mapping, Lemma 2.2 from~\cite{hwg+24} on Page 9]
\label{lem:contextual_map_self_attn_new}

    If the following conditions hold:
    \begin{itemize}
        \item Let $X^{(1)}, \hdots, X^{(N)} \in \R^{n \times d}$ be embeddings that are $(\gamma_{\min}, \gamma_{\max}, \epsilon)$-tokenwise separated, with the vocabulary set $\mathcal{V} = \cup_{i \in [N]} \mathcal{V}^{(i)} \subset \R^{d}$.
        \item $X_k^{(i)} \neq X_l^{(i)}$ for any $i \in [N]$ and $k, l \in [L]$.
        \item Let $
        \gamma = \gamma_{\max} + \frac{\epsilon}{4}$
        \item Let $
        \delta = \exp(-5 \epsilon^{-1} |{\cal V}|^4 d \kappa \gamma_{\max} \log L )$
        \item Let $\kappa := \gamma_{\max}/\gamma_{\min}$.
        \item Let $W^{(O)} \in \R^{d \times s}$ and $W_V, W_K, W_Q \in \R^{s \times d}$.
    \end{itemize}

    Then, we can show
    \begin{itemize}
        \item 1-layer, single-head attention mechanism serves as a $(\gamma, \delta)$-contextual mapping for the embeddings $X^{(1)}, \hdots, X^{(N)}$ with weight matrices $W^{(O)}$ and $W_V, W_K, W_Q$. 
    \end{itemize}
    
\end{lemma}


Lemma~\ref{lem:contextual_map_self_attn_new} indicates that any-rank self-attention function distinguishes input tokens $X_k^{(i)}=X_l^{(j)}$ such that $\mathcal{V}^{(i)} \neq \mathcal{V}^{(j)}$.
In other words, 
it distinguishes two identical tokens within a different context.
\section{Universality of VAR Transformer}\label{sec:var_mainresult}

In this section, we present our proof for the universality of the VAR Transformer.
In Section~\ref{sec:universality_2}, we used a universality result from a previous work.
In Section~\ref{sec:two_layer_pertu}, we analyze how the error behaves when two consecutive layers in our composition are each replaced by their respective approximations.
In Section~\ref{sec:pertu_recursive_one}, we present the scenario when one of the composited layers got replaced by a different function.
In Section~\ref{sec:pertu_recursive_all}, we present the scenario when all of the composited layers got replaced. 
In Section~\ref{sec:var_universality}, we present our proof for the universality of the VAR Transformer.


\subsection{Universality of \texorpdfstring{$\mathcal{T}_A^{1,1,4}$}{} with \texorpdfstring{$O((1/ \epsilon)^{d 
n})$}{} FFN Layers}
\label{sec:universality_2}

We used a universality result from~\cite{hwg+24}.

\begin{lemma}[$\tau \in \mathcal{T}^{1,1,4}_A$ Transformer is Universal Seq2Seq Approximator, Theorem 2.3 in~\cite{hwg+24} on Page 11]
\label{lem:PT_uni_multi_layer_FF2}
    If the following conditions hold: 
    \begin{itemize}
        \item Let $1 \leq p < \infty$ and $\epsilon > 0$.
        \item Let a transformer with one self-attention layer defined as $\tau \in \mathcal{T}^{1,1,4}_A$
    \end{itemize}

    Then, there exists
    \begin{itemize}
        \item a transformer $\tau$ with single self-attention layer, such that for any $\mathcal{L} \in \mathcal{F}_{C}$ there exists
    $ \| \tau ( \cdot), \mathcal{L}\|_\alpha \leq \epsilon$.
    \end{itemize}
\end{lemma}


\subsection{Two Layers Perturbation}\label{sec:two_layer_pertu}

In this section, we analyze how the error behaves when two consecutive layers in our composition are each replaced by their respective approximations. Specifically, we consider the composition $f_i \circ g_i$ and replace $g_i$ with an up interpolation function $\Phi_{\mathrm{up},i}$ and $f_i$ with a one-layer transformer $\tau_i$. We show that under appropriate Lipschitz and approximation assumptions, the overall error of the approximated two-layer composition can be controlled in terms of the individual approximation errors.

\begin{assumption}[Target Function Class]\label{as:function_class}
    We assume the following things:
    \begin{itemize}
        \item Let $f_1, \ldots, f_r$ be $r$ $K$-Lipschitz functions from $\R^{h_r \times w_r \times d}$ to $\R^{h_r \times w_r \times d}$.
        \item For each $i \in [r]$, let $g_i$ be a $K$-Lipschitz function from $\R^{h_{i-1} \times w_{i-1} \times d}$ to $\R^{h_i \times w_i \times d}$.
        \item We assume that for each $i \in [r]$, $g_i$ can be approximated by some up interpolation function $\phi_{\mathrm{up},i}$.
        \item We assume that the target function $f_{\mathrm{word2img}}:\R^{1 \times 1 \times d} \to \R^{h_r \times w_r \times d}$ satisfies
    \begin{align*}
        f_{\mathrm{word2img}} := f_r \circ g_r \cdots \circ f_1 \circ g_1.
    \end{align*}
    \end{itemize}
\end{assumption}

With the Assumption~\ref{as:function_class}, we present the two layers of perturbation as follows. 

\begin{lemma}[Two Layers Perturbation]
    Let $\phi_{{\rm up},i}$ be the up interpolation function defined in~\ref{def:up_inter_layer_one_step}. Let $f_r$ be $r$ $K$-Lipschitz functions from Assumption~\ref{as:function_class}. Let $g_i$ be $r$ $K$-Lipschitz functions from Assumption~\ref{as:function_class}. Let $\tau_i$ be the one-layer transformer defined in Eq. 2.4 from~\cite{hwg+24}. If the following conditions hold:
    \begin{itemize}
        \item $\|g_i - \Phi_{\mathrm{up},i}\| \leq \epsilon_{1,i}$ from Assumption~\ref{as:function_class}.
        \item $\|f_i - \tau_i\| \leq \epsilon_{2,i}$ from Theorem~\ref{lem:PT_uni_multi_layer_FF2}.
        \item $f_i$ is $K_{1,i}$-Lipschitz.
    \end{itemize}
    Then we have
    \begin{align*}
        \|f_i \circ g_i - \tau_i \circ \Phi_{\mathrm{up},i}\| \leq K_{1,i} \epsilon_{1,i} + \epsilon_{2,i}.
    \end{align*}
\end{lemma}
\begin{proof}
    We can show that
    \begin{align*}
         \| f_i \circ g_i -  \tau_i \circ \Phi_{\mathrm{up},i} \| 
         = &~ \|f_i \circ g_i- f_i \circ \Phi_{\mathrm{up},i} + f_i \circ \Phi_{\mathrm{up},i}- \tau_i \circ \Phi_{\mathrm{up},i}\| \\
         \leq &~ \|f_i \circ g_i- f_i \circ \Phi_{\mathrm{up},i}\| + \|f_i \circ \Phi_{\mathrm{up},i}- \tau_i \circ \Phi_{\mathrm{up},i}\| \\
         = &~ \|f_i \circ (g_i- \Phi_{\mathrm{up},i})\| + \|(f_i -\tau_i) \circ \Phi_{\mathrm{up},i}\| \\
         \leq &~ \|f_i \circ (g_i- \Phi_{\mathrm{up},i})\| + \|f_i -\tau_i \| \\
         \leq &~ K_{1,i} \epsilon_{1,i} + \epsilon_{2,i}
    \end{align*}
    where the first step follows from basic algebra, the second step follows from triangle inequality, the third and fourth steps follow from basic algebra, and the fifth step follows from our conditions.
\end{proof}


\subsection{Perturbation of Recursively Composting Functions that One Layer is Different}\label{sec:pertu_recursive_one}

In this section, we consider a scenario where we have a composition of many layers, but only one of the layers is replaced by a different function. This setting helps us see how a single local perturbation can propagate through subsequent layers in a multi-layer composition. The lemma below quantifies this propagation by leveraging Lipschitz continuity.

\begin{lemma}[Perturbation of Recursively Composting Functions, One Layer is Different]\label{lem:one_layer_perturbation}
if the following conditions hold
\begin{itemize}
    \item Assume $\|  u_j(w) - v_j(w) \| \leq \epsilon $ for any $w$.
    \item $v_i(x) \leq K_2 \cdot \| x \|$
\end{itemize}
    Fix $j$, we have 
    \begin{align*}
        \| \circ_{i=j+1}^{n+1} v_i  \circ_{i=1}^j u_i - \circ_{i=j}^n v_i  \circ_{i=0}^{j-1} u_i \| \leq K_2^{n-j} \cdot \epsilon
    \end{align*}
\end{lemma}

\begin{proof}
We define $u$
\begin{align*}
    w = \circ_{i=0}^{j-1} u_i (x)
\end{align*}

    We can show that for any $x$
    \begin{align*}
        \| \circ_{i=j+1}^{n+1} v_i  \circ_{i=1}^j u_i (x) - \circ_{i=j}^n v_i  \circ_{i=0}^{j-1} u_i (x) \| 
        = & ~  \| \circ_{i=j+1}^{n+1} v_i  u_j(w) - \circ_{i=j+1}^n v_i  ( v_j(w) ) \| \\
        = & ~ \| \circ_{i=j+1}^{n+1} v_i ( u_j(w) - v_j(w)  ) \| \\
        \leq & ~ K_2^{n-j} \cdot \epsilon
    \end{align*}
    where the first step follows from basic algebra, the second step follows from linearity, and the third step follows from lemma assumptions.
\end{proof}

\subsection{Perturbation of Recursively Composting Functions that All Layer are Different}\label{sec:pertu_recursive_all}

In this section, we extend the analysis to the most general scenario in which all layers in the composition are replaced by different functions. This captures the situation where each layer $u_i$ is approximated by some other function $v_i$. We derive a cumulative bound that sums the individual perturbations introduced at each layer.

\begin{lemma}[Perturbation of Recursively Compositing Functions, All Layers are Different]
    If the following conditions hold:
    \begin{itemize}
        \item Let $\circ_{i=1}^n u_i  = u_n \circ \cdots \circ u_1$
        \item Let $\circ_{i=1}$
        \item Let $u_0(x) = x$ which is identity mapping
        \item Let $v_{n+1}(x) = x$ which is identity mapping
    \end{itemize}
    Then 
    \begin{align*}
        \| \circ_{i=1}^n u_i - \circ_{i=1}^n v_i \| \leq\sum_{j=1}^{n} \| \circ_{i=j+1}^{n+1} v_i  \circ_{i=1}^j u_i - \circ_{i=j}^n v_i  \circ_{i=0}^{j-1} u_i \|
    \end{align*}
\end{lemma}
\begin{proof}
    We can show
    \begin{align*}
        \| \circ_{i=1}^n u_i - \circ_{i=1}^n v_i \| 
        = & ~ \| \sum_{j=1}^{n} ( \circ_{i=j+1}^{n+1} v_i  \circ_{i=1}^j u_i - \circ_{i=j}^n v_i  \circ_{i=0}^{j-1} u_i ) \| \\
        \leq & ~  
        \sum_{j=1}^{n} \| \circ_{i=j+1}^{n+1} v_i  \circ_{i=1}^j u_i - \circ_{i=j}^n v_i  \circ_{i=0}^{j-1} u_i \|
    \end{align*}
    where the first step follows from adding intermediate terms, and the last step follows from the triangle inequality.
    
Thus, we complete the proof.
\end{proof}

\subsection{The Universality of VAR Transformer}\label{sec:var_universality}

In this section, with the established error bounds for replacing individual or multiple layers with alternative functions, we now prove the main universality result for the VAR Transformer. In essence, we show that a properly constructed VAR Transformer can approximate the target function $f_\mathrm{word2img}$ (from Assumption~\ref{as:function_class}) with arbitrarily small errors under suitable Lipschitz and approximation assumptions on each layer.

\begin{theorem}[Universality of VAR Transformer]\label{thm:var_universality}
    Assume $K_2 > 2$.
    For $f_\mathrm{word2img}$ satisfies Assumption~\ref{as:function_class}, there exists a VAR Transformer $\tau_{\mathsf{VAR}}$ such that
    \begin{align*}
        \|\tau_{\mathsf{VAR}} - f_\mathrm{word2img} \| \leq K_2^n( K_{1,i} \epsilon_{1,i} + \epsilon_{2,i} ).
    \end{align*}
\end{theorem}
\begin{proof}
    We can show that
    \begin{align*}
        \|\tau_{\mathsf{VAR}} - f_\mathrm{word2img} \|  = &~ \| \circ_{i=1}^r (f_i \circ g_i) - \circ_{i=1}^r (\tau_i \circ \Phi_{\mathrm{up},i}) \| \\
        = &~ \sum_{j=1}^n K_2^{n-j} ( K_{1,i} \epsilon_{1,i} + \epsilon_{2,i} )\\
        = &~ \frac{K_2^n - 1}{K_2 - 1}( K_{1,i} \epsilon_{1,i} + \epsilon_{2,i} ) \\
        \leq &~ K_2^n( K_{1,i} \epsilon_{1,i} + \epsilon_{2,i} ) .
    \end{align*}
    where the first step follows from Definition~\ref{def:var_function_class}, the second step follows from Lemma~\ref{lem:one_layer_perturbation}, the third step follows from basic algebra, and the fourth step follows from the basic inequality.  
\end{proof}


\section{Universality of FlowAR}\label{sec:flowar_universality}

In this section, we show that the universality results established for the VAR Transformer can be extended to our FlowAR model. The key observation is that the same local perturbation bounds and Lipschitz assumptions used in the VAR Transformer setting also apply to FlowAR, with only minor changes. Specifically, each FlowAR layer $\Phi_{\mathrm{down}, i}$ can be analyzed in an analogous way to $\Phi_{\mathrm{up}, i}$, allowing us to derive a bound on the overall error of the composed FlowAR model.

\begin{corollary}
    Let $\phi_{{\rm down},i}$ be the down interpolation function of FlowAR (see Definition~\ref{def:down_sample_function}). Let $f_r$ be $r$ $K$-Lipschitz functions from Assumption~\ref{as:function_class}. Let $g_i$ be $r$ $K$-Lipschitz functions from Assumption~\ref{as:function_class}. Let $\tau_i$ be the one-layer transformer defined in Eq. 2.4 from~\cite{hwg+24}. 
    If the following conditions hold: 
    \begin{itemize}
        \item $\|g_i - \Phi_{\mathrm{down},i}\| \leq \epsilon_{1,i}$
        \item $\|f_i - \tau_i\| \leq \epsilon_{2,i}$
        \item $f_i$ is $K_{1,i}$-Lipschitz
        
    \end{itemize}
    Then we have
    \begin{align*}
        \|f_i \circ g_i - \tau_i \circ \Phi_{\mathrm{down},i}\| \leq K_{1,i} \epsilon_{1,i} + \epsilon_{2,i}.
    \end{align*}
\end{corollary}

The proof of this corollary mirrors the two-layer perturbation argument from the VAR Transformer, except each ``up'' interpolation function $\Phi_{\mathrm{up},i}$ is replaced by the corresponding ``down'' interpolation function $\Phi_{\mathrm{down},i}$. The same Lipschitz and approximation assumptions allow us to bound the difference between $f_i \circ g_i$ and $\tau_i \circ \Phi_{\mathrm{down},i}$.

\begin{corollary}\label{corollary:flowar_universality}
    There exists a FlowAR model such that
    \begin{align*}
        \| \tau_{\mathsf{FlowAR}} - f_{\mathrm{word2img}} \| \leq O(\epsilon).
    \end{align*}
\end{corollary}

The proof of Corollary~\ref{corollary:flowar_universality} follows the same high-level structure as our universality results for the VAR Transformer. By applying the local perturbation bound layer by layer and then summing the resulting errors, we obtain a global approximation guarantee that is $O(\epsilon)$. Hence, FlowAR, just like the VAR Transformer, can universally approximate the target function $f_{\mathrm{word2img}}$ under the given Lipschitz and approximation assumptions.
\section{Conclusion and future directions} \label{sec:conclusion}

In this paper we proposed a nested MLMC framework that offers important computational savings by performing most calculations in low precision and exploiting approximate random normal variables for the low precision path calculations. The low precision calculations could be performed in fixed precision on an FPGA for greater efficiency, and we suggested a procedure to optimise the bit-widths of every variable at each Monte Carlo level. This is an important improvement over previous mixed precision MLMC frameworks which held the lower precision fixed \cite{Rounding_error_oliver} or defined uniform bit-width at every level heuristically \cite{brugger2014mixed}. Our numerical results suggest that for the first levels our procedure reduces the cost at these levels by a factor 5 or 7. Hence the overall savings are significant since most paths are calculated on the first levels. Our approach would be even more efficient for the Milstein scheme because its higher order strong convergence leads to a greater proportion of the computational costs being on the coarsest levels.

The next stage of the research project will be to implement the RNG methods and the nested framework on FPGAs to determine the hardware requirements and confirm the extent of the computational savings. It would also be good to compare the performance benefits to using half-precision floating point arithmetic on GPUs or CPUs for the low-accuracy computations.





\ifdefined\isarxiv

\else
\bibliography{ref}
\bibliographystyle{icml2025}

\fi



\newpage
\onecolumn
\appendix
\begin{center}
	\textbf{\LARGE Appendix }
\end{center}





%%%% Cut-line between first 10 pages and appendix

{\bf Roadmap} In Section~\ref{sec:app_prelim}, we provide basic algebras that support our proofs. 
In Section~\ref{sec:app_var}, we provide other phases of the VAR Model. 
In Section~\ref{sec:training_of_flowar}, we give the definitions for training the FlowAR Model. 
In Section~\ref{sec:inference_of_flowar}, we give the definitions for the inference of the FlowAR Model.
In Section~\ref{sec:more_work}, we introduce more related work.

\section{Preliminary}\label{sec:app_prelim}

In this section, we introduce notations and basic facts that are used in our work. We first list some basic facts of matrix norm properties.

\subsection{Notations}
We denote the $\ell_p$ norm of a vector $x$ by $\| x \|_p$, i.e., $\|x\|_1 := \sum_{i=1}^n |x_i|$, $\| x \|_2 := (\sum_{i=1}^n x_i^2)^{1/2}$ and $\| x \|_{\infty} := \max_{i \in [n]} |x_i|$. For a vector $x \in \R^n$, $\exp(x) \in \R^n$ denotes a vector where $\exp(x)_i$ is $\exp(x_i)$ for all $i \in [n]$. For $n > k$, for any matrix $A \in \R^{n\times k}$, we denote the spectral norm of $A$ by $\| A \|$, i.e., $\| A \| := \sup_{x\in \R^k} \| Ax \|_2 / \| x \|_2$. We define the function norm as $\| f \|_\alpha :=  (\int \| f(X) \|_\alpha^\alpha \d X)^{1/\alpha}$ where $f$ is a function. We use $\sigma_{\min}(A)$ to denote the minimum singular value of $A$. Given two vectors $x, y \in \R^n$, we use $\langle x, y \rangle$ to denote $\sum_{i=1}^n x_iy_i$. Given two vectors $x, y \in \R^n$, we use $x \circ y$ to denote a vector that its $i$-th entry is $x_i y_i$ for all $i \in [n]$. We use $e_i \in \R^n$ to denote a vector where $i$-th entry is $1$, and all other entries are $0$. Let $x \in \R^n$ be a vector. We define $\diag (x) \in \R^{n \times n}$ as the diagonal matrix whose diagonal entries are given by $\diag(x)_{i, i} = x_i$ for $i = 1, \dots, n$, and all off-diagonal entries are zero. For a symmetric matrix $A \in \R^{n\times n}$, we say $A \succ 0$ (positive definite (PD)), if for all $x\in \R^n \setminus \{ {\bf 0}_n \}$, we have $x^\top A x > 0$. For a symmetric matrix $A \in \R^{n \times n}$, we say $A \succeq 0$ (positive semidefinite (PSD)), if for all $x \in \R^n$, we have $x^\top A x \geq 0$. The Taylor Series for $\exp(x)$ is $\exp(x) = \sum_{i=0}^{\infty} \frac{x^i}{i!}$. For a matrix $X \in \R^{n_1 n_2 \times d}$, we use $\X \in \R^{n_1 \times n_2 \times d}$ to denote its tensorization, and we only assume this for letters $X$ and $Y$.

\subsection{Basic Algebra}
In this section, we introduce the basic algebras used in our work.

\begin{fact}
Let $A$ denote the matrix. For each $i$, we use $A_{i,*}$ to denote the $i$-th row of $A$. For $j$, we use $A_{*,j}$ to denote the $j$-th column of $A$. 
We can show that
\begin{itemize}
    \item $\| A \| \leq \| A \|_F$
    \item $\| A \| \geq \| A_{i,*} \|_2$
    \item $\| A \| \geq \| A_{*,j} \|_2$
\end{itemize}
\end{fact}

Then, we introduce some useful inner product properties.

\begin{fact}
    For vectors $u,v,w \in \R^n$. We have 
    \begin{itemize}
        \item $\langle u, v \rangle = \langle u \circ v, {\bf 1}_n \rangle$
        \item $\langle u \circ v, w \rangle = \langle u \circ v \circ w, {\bf 1}_n \rangle$ 
        \item $\langle u, v \rangle = \langle v, u \rangle$
        \item $\langle u , v \rangle = u^\top v = v^\top u$
    \end{itemize}
\end{fact}

Now, we show more vector properties related to the hadamard products, inner products, and diagnoal matrices.
\begin{fact}
    For any vectors $u, v, w \in \R^n$, we have
    \begin{itemize}
        \item $u \circ v = v \circ u = \diag(u) \cdot v = \diag(v) \cdot u$
        \item $u^\top (v\circ w) = u^\top \diag(v) w$
        \item $u^\top (v\circ w) = v^\top (u \circ w) = w^\top (u \circ v)$
        \item $u^\top \diag(v) w = v^\top \diag(u) w = u^\top \diag(w) v$
        \item $\diag(u) \cdot \diag(v) \cdot {\bf 1}_n = \diag(u) v$
        \item $\diag(u \circ v) = \diag(u) \diag(v)$
        \item $\diag(u) + \diag(v) = \diag(u+v)$
    \end{itemize}
\end{fact}




\section{VAR Transformer Blocks}\label{sec:app_var}

In this section, we define the components in the VAR Transformer.

We first introduce the Softmax unit.

\begin{definition}[Softmax] \label{def:softmax}
    Let $z \in \R^{n}$. We define 
    $\mathsf{Softmax}: \R^{n} \to \R^{n}$ satisfying 
    \begin{align*}
        \mathsf{Softmax}(z):= \exp(z) / \langle \exp(z) , {\bf 1}_n \rangle.  
    \end{align*}
\end{definition}

Here, we define the attention matrix in the VAR Transformer as follows.

\begin{definition}[Attention Matrix]\label{def:attn_matrix}
    Let $W_Q, W_K \in \R^{d \times d}$ denote the model weights. Let $X \in \R^{n \times d}$ denote the representation of the length-$n$ input. Then, we define the attention matrix $A \in \R^{n \times n}$ by, For $i,j \in [n]$, 
    \begin{align*}
        A_{i,j} := & ~\exp( \underbrace{ X_{i,*} }_{1 \times d} \underbrace{ W_Q }_{d \times d} \underbrace{ W_K^\top }_{d \times d} \underbrace{ X_{j,*}^\top }_{d \times 1}).
    \end{align*}
\end{definition}

With the attention matrix, we now provide the definition for a single layer of Attention.

\begin{definition}[Single Attention Layer]\label{def:single_layer_transformer}
     Let $X \in \R^{n \times d}$ denote the representation of the length-$n$ sentence. Let $W_V \in \R^{d \times d}$ denote the model weights. As in the usual attention mechanism, the final goal is to output an $n \times d$ size matrix where $D:= \diag( A {\bf 1}_n) \in \R^{n \times n}$. Then, we define attention layer $\mathsf{Attn}$ as
    \begin{align*}
        \mathsf{Attn} (X) := & ~ D^{-1} A X W_V .
    \end{align*}
\end{definition}

Here we present the definition of the VAR Attention.

\begin{definition}[$\VAR$ Attention Layer]\label{def:single_layer_var_transformer}
    Let $r \geq 1$ be a positive integer. Let $h_r, w_r$ be two positive integers. 
     Let $\X \in \R^{h_r \times w_r \times d}$ denote the representation of the input token map. Let $W_V \in \R^{d \times d}$ denote the model weights. As in the usual attention mechanism, the final goal is to output an $n \times d$ size matrix where $D:= \diag( A {\bf 1}_n) \in \R^{h_rw_r \times h_rw_r}$. Then, we define attention layer $\mathsf{Attn}_r: \R^{h_rw_r \times d} \to \R^{h_rw_r \times d}$ as
    \begin{align*}
        \mathsf{Attn}_r (X) := & ~ D^{-1} A X W_V .
    \end{align*}
\end{definition}

We introduce the feed-forward layer in the VAR Transformer as follows.

\begin{definition}[Single Feed-Forward Layer]\label{def:ffn}
We define the FFN as follows:
    \begin{itemize}
        \item $X \in \R^{d\times L}$
        \item $k \in [n]$
        \item $c$ is the number of neurons
        \item $W^{(1)} \in \R^{c \times d}, W^{(2)} \in \R^{d \times c}$ are weight matrices
        \item $b^{(1)}\in \R^{ c}$, $ b^{(2)} \in \R^{d}$ are bias vectors.
        \item $\mathsf{FFN}: \R^{d\times L} \to \R^{d\times L}$ 
    \end{itemize}
    
    \begin{align*}
        \mathsf{FFN}(X)_{*,k} =
        & ~ \underbrace{X_{*,k}}_{d \times 1} +  \underbrace{W^{(2)}}_{d \times c} \mathsf{ReLU}( \underbrace{ W^{(1)} X_{*,k} }_{c\times 1} + \underbrace{b^{(1)}}_{c\times 1})  + \underbrace{b^{(2)}}_{d \times 1} 
    \end{align*}
    
\end{definition}

\subsection{Phase 2: Feature Map Reconstruction }\label{sec:phase_2}

In this section, we introduce the Phase Two of the VAR model.

\begin{definition}[Convolution Layer, Definition 3.9 from~\cite{kll+25} on Page 9]\label{def:conv_layer}
    The Convolution Layer is defined as follows:
    \begin{itemize}
        \item Let $h \in \mathbb{N}$ denote the height of the input and output feature map.
        \item Let $w \in \mathbb{N}$ denote the width of the input and output feature map.
        \item Let $c_{\rm in} \in \mathbb{N}$ denote the number of channels of the input feature map.
        \item Let $c_{\rm out} \in \mathbb{N}$ denote the number of channels of the output feature map.
        \item Let $X \in \R^{h \times w \times c_{\rm in}}$ denote the input feature map.
        
        \item For $l \in [c_{\rm out}]$, we use $K^l \in \R^{3 \times 3 \times c_{\rm in}}$ to denote the $l$-th convolution kernel.
        \item Let $p = 1$ denote the padding of the convolution layer.
        \item Let $s = 1$ denote the stride of the convolution kernel.
        \item Let $Y \in \R^{h \times w \times c_{\rm out}}$ denote the output feature map.
    \end{itemize}
    We use $\phi_{\rm conv}: \R^{h \times w \times c_{\rm in}} \to \R^{h \times w \times c_{\rm out}}$ to denote the convolution operation then we have $Y = \phi_{\rm conv}(X)$. Specifically, for $i \in [h], j \in [w], l \in [c_{\rm out}]$, we have
    \begin{align*}
        Y_{i,j,l} := \sum_{m=1}^3 \sum_{n=1}^3 \sum_{c = 1}^{c_{\rm in}} X_{i+m-1,j+n-1,c} \cdot K^l_{m,n,c} + b
    \end{align*}
\end{definition}

\begin{remark}
    Assumptions of kernel size, padding of the convolution layer, and stride of the convolution kernel are based on the specific implementation of \cite{tjy+24}.
\end{remark}

\subsection{Phase 3: VQ-VAE Decoder process}\label{sec:phase_3} 

In this section, we introduce Phase Three of the VAR model.

VAR will use the VQ-VAE Decoder Module to reconstruct the feature map generated in Section~\ref{sec:phase_2} into a new image. The Decoder of VQ-VAE has the following main modules \cite{kll+25}: (1) Resnet Blocks; (2) Attention Blocks; (3) Up Sample Blocks. We recommend readers to \cite{kll+25} for more details. 



\section{Training of FlowAR}\label{sec:training_of_flowar}

In this section, we introduce the training of FlowAR along with its definitions based on~\cite{ryh+24}.


We first introduce some notations used in FlowAR.
\subsection{Notations}\label{sub:notations}
For a matrix $X \in \R^{n_1 n_2 \times d}$, we use $\X \in \R^{n_1 \times n_2 \times d}$ to denote its tensorization, and we only assume this for letters $X, Y, Z, F, V$.

\subsection{Sample Function}
In this section, we introduce the sample functions used in FlowAR.

We first introduce the up sample function.

\begin{definition}[Up Sample Function]\label{def:up_sample_function}
    If the following conditions hold:
    \begin{itemize}
        \item Let $h, w \in \mathbb{N}$ denote the height and weight of latent $\X \in \R^{h \times w \times c}$.
        \item Let $r > 0$ denote a positive integer.
    \end{itemize}
    Then we define $\mathrm{Up}(\X,r) \in \R^{rh \times rw \times c}$ as the upsampling of latent $\X$ by a factor $r$.
\end{definition}

Then, we introduce the down sample function.
\begin{definition}[Down Sample Function]\label{def:down_sample_function}
    If the following conditions hold:
    \begin{itemize}
        \item  Let $h, w \in \mathbb{N}$ denote the height and weight of latent $\X  \in \R^{h \times w \times c}$.
        \item Let $r > 0$ denote a positive integer.
    \end{itemize}
    Then we define $\mathrm{Down}(\X ,r) \in \R^{\frac{h}{r} \times \frac{w}{r} \times c}$ as the downsampling of latent $\X $ by a factor $r$.
\end{definition}

\subsection{Linear Sample Function}
In this section, we present the linear sample function in FlowAR.

We first present the linear up sample function.
\begin{definition}[Linear Up Sample Function]\label{def:linear_up_sample_function}
    If the following conditions hold:
    \begin{itemize}
        \item Let $h, w \in \mathbb{N}$ denote the height and weight of latent $\X \in \R^{h \times w \times c}$. 
        \item Let $r > 0$ denote a positive integer.
        \item Let $\Phi_{\mathrm{up}} \in \R^{hw \times (rh \cdot rw)}$ be a matrix. 
    \end{itemize}
    Then we define the linear up sample function $\phi_{\mathrm{up}}(\cdot, \cdot)$ as it computes $\Y := \phi_{\mathrm{up}}(\X,r) \in \R^{rh \times rw \times c}$ such that the matrix version of $\X$ and $\Y$ satisfies
    \begin{align*}
        Y = \Phi_{\mathrm{up}}X \in \R^{(rh\cdot rw) \times c}.
    \end{align*}
\end{definition}

Then, we define linear down sample function as follows.
\begin{definition}[Linear Down Sample Function]\label{def:linear_down_sample_function}
    If the following conditions hold:
    \begin{itemize}
        \item Let $h, w \in \mathbb{N}$ denote the height and weight of latent $\X \in \R^{h \times w \times c}$.
        \item Let $r > 0$ denote a positive integer.
        \item Let $\Phi_{\mathrm{down}} \in \R^{((h/r) \cdot (w/r)) \times hw}$ be a matrix. 
    \end{itemize}
    Then we define the linear down sample function $\phi_{\mathrm{down}}(\X,r)$ as it computes $\Y := \phi_{\mathrm{down}}(\X,r) \in \R^{(h/r) \times (w/r) \times c}$ such that the matrix version of $\X$ and $\Y$ satisfies
    \begin{align*}
        Y = \Phi_{\mathrm{down}}X \in \R^{((h/r) \cdot (w/r)) \times c}.
    \end{align*}
\end{definition}


\subsection{VAE Tokenizer}\label{sub:vae_tokenizer}
In this section, we show the VAE Tokenizer.

\begin{definition}[VAE Tokenizer]\label{def:vae_tokenizer}
If the following conditions hold:
\begin{itemize}
    \item Let $\X \in \R^{h \times w \times c}$ denote a continuous latent representation generated by VAE.
    \item Let $K$ denote the total number of scales in FlowAR.
    \item Let $a$ be a positive integer.
    \item For $i \in [K]$, let $r_i := a^{K-i}$.
    \item For each $i\in [K]$, let $\phi_{\mathrm{down}, i}(\cdot , r_i) : \R^{h \times w \times c} \to \R^{(h / r_i) \times (w/r_i) \times c}$ denote the linear down sample function defined in Definition~\ref{def:linear_down_sample_function}.
\end{itemize}
For $i \in [K]$, we define the $i$-th token map generated by VAE Tokenizer be
\begin{align*}
    \Y^{i} := \phi_{\mathrm{down}, i}(\X, r_i) \in \R^{(h / r_i) \times (w/r_i) \times c},
\end{align*}
We define the output of VAE Tokenizer as follows:
    \begin{align*}
        \mathsf{Tokenizer}(\mathsf{X}) := \{\Y^{1},\Y^{2}, \dots,\Y^{K}\}.
    \end{align*}
\end{definition}
\begin{remark}
    In \cite{ryh+24}, they choose $a =2$ and hence for $i \in [n]$, $r_i := 2^{K-i}$.
\end{remark}

\subsection{Autoregressive Transformer}



Firstly, we give the definition of a single attention layer.
\begin{definition}[Single Attention Layer]\label{def:attn_layer}
    If the following conditions hold:
    \begin{itemize}
        \item Let $h,w \in \mathbb{N}$ denote the height and weight of latent $\X \in \R^{h \times w \times c}$.
        \item Let $W_Q, W_K, W_V \in \R^{c \times c}$ denote the weight matrix for query, key, and value, respectively.
    \end{itemize}
    Then we define the attention layer $\mathsf{Attn}(\cdot)$ as it computes  $\Y = \mathsf{Attn}(\X) \in \R^{h \times w \times c}$. For the matrix version, we first need to compute the attention matrix $A \in \R^{hw \times hw}$:
    \begin{align*}
        A_{i,j} := & ~\exp(  X_{i,*}   W_Q   W_K^\top   X_{j,*}^\top), \text{~~for~} i, j \in [hw].
    \end{align*}
    Then, we compute the output:
    \begin{align*}
        Y := D^{-1}AXW_V \in \R^{hw \times c}.
    \end{align*}
    where $D:=\diag(A {\bf 1}_n) \in \R^{hw \times hw}$.
\end{definition}


To move on, we present the definition of multilayer perceptron.
\begin{definition}[MLP layer]\label{def:mlp}
    If the following conditions hold:
    \begin{itemize}
        \item Let $h,w \in \mathbb{N}$ denote the height and weight of latent $\X \in \R^{h \times w \times c}$.
        \item Let $c$ denote the input dimension of latent $\X \in \R^{h \times w \times c}$.
        \item Let $d$ denote the dimension of the target output.
        \item Let $W \in \R^{c \times d}$ denote a weight matrix.
        \item Let $b \in \R^{1 \times d}$ denote a bias vector.
    \end{itemize}
    Then we define mlp layer as it computes $\Y := \mathsf{MLP}(X,c,d) \in \R^{h \times w \times d}$ such that the matrix version of $\X$ and $\Y$ astisfies, for each $j \in [hw]$,
    \begin{align*}
        Y_{j,:} = \underbrace{X_{j,:}}_{1\times c} \cdot \underbrace{W}_{c \times d} + \underbrace{b}_{1 \times d}
    \end{align*}
\end{definition}



We present the definition of layer-wise norm layer.
\begin{definition}[Layer-wise norm layer]\label{def:ln}
    Given a latent $\X \in \R^{h \times w \times c}$. We define the layer-wise as it computes $\Y := \mathsf{LN}(X) \in \R^{h\times w \times c}$ such that the matrix version of $\X$ and $\Y$ satisfies, for each $j \in [hw]$,
    \begin{align*}
        Y_{j,:} =  \frac{X_{j,:}-\mu_j}{\sqrt{\sigma_j^2}}
    \end{align*}
    where $\mu_j := \sum_{k=1}^c X_{j,k}/c$ and $\sigma_{j}^2 = \sum_{k=1}^c(X_{j,k}-\mu_j)^2/c$.
\end{definition}

\begin{definition}[Autoregressive Transformer]\label{def:ar_transformer}
    If the following conditions hold:
    \begin{itemize}
        \item Let $\X \in \R^{h \times w \times c}$ denote a continuous latent representation generated by VAE.
        \item Let $K$ denote the total number of scales in FlowAR.
        \item For $i \in [K]$, let $\Y_i \in \R^{(h / r_i) \times (w/r_i) \times c}$ be the $i$-th token map genereated by VAE Tokenizer defined in Definition~\ref{def:vae_tokenizer}.
        \item Let $a$ be a positive integer.
        \item For $i \in [K]$, let $r_i := a^{K-i}$.
        \item For $i \in [K-1]$, let $\phi_{\mathrm{up}, i}(\cdot, a):\R^{(h/r_i)\times(w/r_i)\times c} \to \R^{(h/r_{i+1})\times(w/r_{i+1})\times c}$ be the linear up sample function defined in Definition~\ref{def:linear_up_sample_function}.
        \item For $i \in [K]$, let $\mathsf{Attn}_i(\cdot):\R^{(\sum_{j=1}^i h/r_j)(\sum_{j=1}^i w/r_j) \times c} \to \R^{(\sum_{j=1}^i h/r_j)(\sum_{j=1}^i w/r_j) \times c}$ be the $i$-th attention layer defined in Definition~\ref{def:attn_layer}.
        \item For $i \in [K]$, let $\mathsf{FFN}_i(\cdot):\R^{(\sum_{j=1}^i h/r_j)(\sum_{j=1}^i w/r_j) \times c} \to \R^{(\sum_{j=1}^i h/r_j)(\sum_{j=1}^i w/r_j) \times c}$ be the $i$-th feed forward network defined in Definition~\ref{def:ffn}.
        \item Let $\Z_{\mathrm{init}} \in \R^{(h/r_1) \times (w/r_1) \times c}$ be the initial input denoting the class condition.
        \item Let $\Z^1 : = \mathsf{Z}_{\mathrm{init}} \in \R^{(h/r_1) \times (w/r_1) \times c}$.
        \item For $i \in [K] \setminus \{1\}$, Let $\Z^{i}$ be the reshape of the input sequence $\mathsf{Z}_{\mathrm{init}}, \phi_{\mathrm{up}, 1}(\Y^1, a), \ldots, \phi_{\mathrm{up}, i}(\Y^{i-1}, a)$ into the tensor of size $(\sum_{j=1}^i h /r_{j}) \times (\sum_{j=1}^i w /r_{j}) \times c$.
    \end{itemize}
    For $i \in [K]$, we define the Autoregressive transformer $\mathsf{TF}_i$ as 
    \begin{align*}
        \mathsf{TF}_i(\Z^i) = \mathsf{FFN}_i \circ \mathsf{Attn}_i (\Z^i) \in \R^{(\sum_{j=1}^{i} h /r_{j})(\sum_{j=1}^{i} w /r_{j}) \times c}.
    \end{align*}
    We denote $\wh{\Y}^i$ as the $i$-th block of size $(h/r_{i}) \times (w/r_{i}) \times c$ of the tensorization of $\mathsf{TF}_i(Z^i)$.
\end{definition}


\subsection{Flow Matching}
In this section, we introduce the flow matching definition.

\begin{definition}[Flow]\label{def:flow}
    If the following conditions hold:
    \begin{itemize}
        \item Let $\X \in \R^{h \times w \times c}$ denote a continuous latent representation generated by VAE.
        \item Let $K$ denote the total number of scales in FlowAR.
        \item For $i \in [K]$, let $\Y_i \in \R^{(h / r_i) \times (w/r_i) \times c}$ be the $i$-th token map genereated by VAE Tokenizer defined in Definition~\ref{def:vae_tokenizer}.
        \item For $i \in [K]$, let $\F^i_0 \in \R^{(h / r_i) \times (w/r_i) \times c}$ be a matrix where each entry is sampled from the standard Gaussian $\N(0,1)$.
    \end{itemize}
    We defined the interpolated input as follows:
    \begin{align*}
        \F^i_{t} := t \Y^i + (1-t)\F^i_0.
    \end{align*}
    The velocity flow is defined as
    \begin{align*}
        \V^i_t := \frac{\d \F^i_{t}}{\d t} = \Y^i -\F^i_0.
    \end{align*}
\end{definition}




Here, we define the architecture of the flow matching model defined in \cite{ryh+24}.
\begin{definition}[Flow Matching Architecture]\label{def:flow_matching_architecture}
    If the following conditions hold:
    \begin{itemize}

        \item Let $\X \in \R^{h \times w \times c}$ denote a continuous latent representation generated by VAE.
        \item Let $K$ denote the total number of scales in FlowAR.
        \item For $i \in [K]$, let $\Y_i \in \R^{(h / r_i) \times (w/r_i) \times c}$ be the $i$-th token map genereated by VAE Tokenizer defined in Definition~\ref{def:vae_tokenizer}.
        \item Let $i \in [K]$.
        \item Let $\wh{\Y}_i \in \R^{(h / r_i) \times (w/r_i) \times c}$ be the $i$-th block of the output of Autoregressive Transformer defined in Definition~\ref{def:ar_transformer}.
        \item Let $\F^i_t$ be the interpolated input defined in Definition~\ref{def:flow}.
        \item Let $\mathsf{Attn}_i(\cdot):\R^{(h / r_i) \times (w/r_i) \times c} \to \R^{(h / r_i) \times (w/r_i) \times c}$ be the $i$-th attention layer defined in Definition~\ref{def:attn_layer}.
        \item Let $\mathsf{MLP}_i(\cdot,c,d):\R^{(h / r_i) \times (w/r_i) \times c}  \to \R^{(h / r_i) \times (w/r_i) \times d}$ be the $i$-th attention layer defined in Definition~\ref{def:mlp}.
        \item Let $\mathsf{LN}_i(\cdot): \R^{(h / r_i) \times (w/r_i) \times c}  \to \R^{(h / r_i) \times (w/r_i) \times c}$ be the $i$-th layer-wise norm layer defined in Definition~\ref{def:ln}.
        \item Let $t_i \in [0,1]$ denote a time step.
    \end{itemize}
    Then we define the $i$-th flow matching model as $\mathsf{NN}_i(\F_t^i,\wh{\Y}_i,t_i): \R^{(h / r_i) \times (w/r_i) \times c} \times \R^{(h / r_i) \times (w/r_i) \times c} \times \R \to \R^{(h / r_i) \times (w/r_i) \times c}$. The tensor input needs to go through the following computational steps:
    \begin{itemize}
        \item {\bf Step 1:} Compute intermediate variables $\alpha_1, \alpha_2, \beta_1, \beta_2, \gamma_1, \gamma_2$. Specifically, we have
        \begin{align*}
            \alpha_1, \alpha_2, \beta_1, \beta_2, \gamma_1, \gamma_2 :=&~ \mathsf{MLP}_i(\wh{\Y}_i + t_i \cdot {\bf 1}_{(h / r_i) \times (w/r_i) \times c},c,6c)
        \end{align*}
        \item {\bf Step 2:} Compute intermediate variable $\wh{F_t}^{i'}$. Specifically, we have
        \begin{align*}
            \wh{\F_t}^{i'}:= \mathsf{Attn}_i (\gamma_1 \circ \mathsf{LN}(\F_t^i) + \beta_1) \circ \alpha_1
        \end{align*}
        where $\circ$ denotes the element-wise product for tensors.

        \item {\bf Step 3:} Compute final output $\F_t^{i''}$. Specifically, we have
        \begin{align*}
            \F_t^{i''} = \mathsf{MLP}_i(\gamma_2 \circ \mathsf{LN}(\wh{F_t}^{i'})+ \beta_2,c,c) \circ \alpha_2
        \end{align*}
        where $\circ$ denotes the element-wise product for tensors.
    \end{itemize}
\end{definition}



Then, we present our training objective.
\begin{definition}[Loss of FlowAR]
If the following conditions hold:
    \begin{itemize}
        \item Let $\X \in \R^{h \times w \times c}$ denote a continuous latent representation generated by VAE.
        \item Let $K$ denote the total number of scales in FlowAR.
        \item For $i \in [K]$, let $\Y_i \in \R^{(h / r_i) \times (w/r_i) \times c}$ be the $i$-th token map genereated by VAE Tokenizer defined in Definition~\ref{def:vae_tokenizer}.
        \item For $i \in [K]$, let $\wh{\Y}_i \in \R^{(h / r_i) \times (w/r_i) \times c}$ be the $i$-th block of the output of Autoregressive Transformer defined in Definition~\ref{def:ar_transformer}.
        \item For $i \in [K]$, let $\F^i_t$ be the interpolated input defined in Definition~\ref{def:flow}.
        \item For $i \in [K]$, let $\V^i_t$ be the velocity flow defined in Definition~\ref{def:flow}.
        \item For $i \in [K]$, let $\mathsf{NN}_i(\cdot,\cdot,\cdot):\R^{(h / r_i) \times (w/r_i) \times c} \times \R^{(h / r_i) \times (w/r_i) \times c} \times \R \to \R^{(h / r_i) \times (w/r_i) \times c}$ denote the $i$-th flow matching network defined in Definition~\ref{def:flow_matching_architecture}.
    \end{itemize}
    The loss function of FlowAR is 
    \begin{align*}
        L(\theta) = \sum_{i=1}^n \E_{t \sim \mathsf{Unif}[0,1]}\|\mathsf{NN}_i(\F^i_t, \wh{\Y}^{i}_t, t_i) - \V^i_t\|^2.
    \end{align*}
\end{definition}

\section{Inference of FlowAR}\label{sec:inference_of_flowar}
We define the architecture of FlowAR during the inference process as follows.
\begin{definition}[FlowAR Architecture in the Inference Pipeline]\label{def:flow_architecture_inference}
    If the following conditions hold:
    \begin{itemize}
        \item Let $K$ denote the total number of scales in FlowAR.
        \item Let $a$ be a positive integer.
        \item For $i \in [K]$, let $r_i := a^{K-i}$.
        \item For $i \in [K-1]$, let $\phi_{\mathrm{up}, i}(\cdot, a):\R^{(h/r_i)\times(w/r_i)\times c} \to \R^{(h/r_{i+1})\times(w/r_{i+1})\times c}$ be the linear up sample function defined in Definition~\ref{def:linear_up_sample_function}.
        \item For $i \in [K]$, let $\mathsf{Attn}_i(\cdot):\R^{(\sum_{j=1}^i h/r_j)(\sum_{j=1}^i w/r_j) \times c} \to \R^{(\sum_{j=1}^i h/r_j)(\sum_{j=1}^i w/r_j) \times c}$ be the $i$-th attention layer defined in Definition~\ref{def:attn_layer}.
        \item For $i \in [K]$, let $\mathsf{FFN}_i(\cdot):\R^{(\sum_{j=1}^i h/r_j)(\sum_{j=1}^i w/r_j) \times c} \to \R^{(\sum_{j=1}^i h/r_j)(\sum_{j=1}^i w/r_j) \times c}$ be the $i$-th feed forward network defined in Definition~\ref{def:ffn}.
        \item For $i \in [K]$, let $\mathsf{NN}_i(\cdot,\cdot,\cdot):\R^{(h / r_i) \times (w/r_i) \times c} \times \R^{(h / r_i) \times (w/r_i) \times c} \times \R \to \R^{(h / r_i) \times (w/r_i) \times c}$ denote the $i$-th flow matching network defined in Definition~\ref{def:flow_matching_architecture}.
        \item For $i \in [K]$, let $t_i \in [0,1]$ denote the time steps.
        \item For $i \in [K]$, let $\F^i_t$ be the interpolated input defined in Definition~\ref{def:flow}.
        \item Let $\Z_{\mathrm{init}} \in \R^{(h/r_1) \times (w/r_1) \times c}$ denote the initial input denoting the class condition.
        \item Let $\Z^1:=\Z_{\mathrm{init}} \in \R^{(h/r_1)\times(w/r_1) \times c}$.
    \end{itemize}
    Then, we define the architecture of FlowAR in the inference pipeline as follows:
    \begin{itemize}
        \item Layer 1: Given the initial token $\Z^1$, we compute
        \begin{align*}
            &~ s_1 = \mathsf{FFN}_1 \circ \mathsf{Attn}_1 (\Z^1) \in \R^{(h/r_1) \times (w/r_1) \times c}\\
            &~ \wh{s}_1 = \mathsf{NN}_1(F_t^1,s_1,t_1)
        \end{align*}
        \item Layer 2: Given the initial token $\Z^1$ and output of the first layer $\wh{s}_1$. Let $\Z^2$ be the reshape of the input sequence $\Z_{\mathrm{init}}, \phi_{\mathrm{up},1}(\wh{s}_1,a)$ into the tensor of size $(\sum_{i=1}^2 h/r_i) \times (\sum_{i=1}^2 w/r_i) \times c$. Then we compute
        \begin{align*}
            &~ s_2 = \mathsf{FFN}_2 \circ \mathsf{Attn}_2 (\Z^2)_{h/r_1:\sum_{i=1}^2 h/r_i,w/r_1 :\sum_{i=1}^2 w/r_i,0:c}\\
            &~ \wh{s}_2 = \mathsf{NN}_2(F_t^2,s_2,t_2)
        \end{align*}
        \item  Layer  $i \in [K]\setminus \{1,2\}$: Given the initial token $\Z^1$ and the output of the first $i-1$ layer $\wh{s}_1,\dots,\wh{s}_{i-1}$. Let $\Z^i$ be the reshape of the input sequence $\Z_{\mathrm{init}}, \phi_{\mathrm{up},1}(\wh{s}_1), \dots, \phi_{\mathrm{up},i-1}(\wh{s}_{i-1})$ into the tensor of size $(\sum_{j=1}^i h/r_j) \times (\sum_{j=1}^i w/r_j) \times c$. Then we compute
        \begin{align*}
            &~ s_i = \mathsf{FFN}_i \circ \mathsf{Attn}_i (\Z^i)_{\sum_{j=1}^{i-1} h/r_j : \sum_{j=1}^i h/r_j , \sum_{j=1}^{i-1} w/r_j : \sum_{j=1}^i w/r_j,0:c}\\
            &~ \wh{s}_i = \mathsf{NN}_i(F_t^i,s_i,t_i)
        \end{align*}
        Then the final output of FlowAR is $\wh{s}_K$.
    \end{itemize}
\end{definition}

\section{More Related Work}\label{sec:more_work}
In this section, we introduce more related work. 

\paragraph{Theoretical Machine Learning.}
Our work also takes inspiration from the following Machine Learning Theory work. Some works analyze the expressiveness of a neural network using the theory of circuit complexity~\cite{lls+25_gnn,kll+25_var_tc0,lls+24_rope_tensor_tc0,cll+24_mamba,cll+24_rope}. Some works optimize the algorithms that can accelerate the training of a neural network~\cite{llsz24,klsz24,dlms24,dswy22_coreset,haochen3,haochen4,dms23_spar,cll+25_deskreject,sy23,swyy23,lss+22,lsx+22,hst+22,hsw+22,hst+20,bsy23,dsy23,syyz23_weighted,gsy23_coin,gsy23_hyper,gsyz23,gswy23,syzz24,lsw+24,lsxy24,hsk+24,hlsl24}. Some works analyze neural networks via regressions~\cite{cll+24_icl,gms23,lsz23_exp,gsx23,ssz23_tradeoff,css+23,syyz23_ellinf,syz23,swy23,syz23_quantum,lls+25_grok}. Some works use reinforcement learning to optimize the neural networks~\cite{haochen1,haochen2,yunfan1,yunfan2,yunfan3,yunfan4,lswy23}. Some works optimize the attention mechanisms~\cite{sxy23,lls+24_conv}.


\paragraph{Accelerating Attention Mechanisms.}
The attention mechanism, with its quadratic computational complexity concerning context length, encounters increasing challenges as sequence lengths grow in modern large language models~\cite{gpto1,llama3_blog,claude3_pdf}. To address this limitation, polynomial kernel approximation methods \citep{aa22} have been introduced, leveraging low-rank approximations to efficiently approximate the attention matrix. These methods significantly enhance computation speed, allowing a single attention layer to perform both training and inference with nearly linear time complexity \citep{as23, as24b}. Moreover, these techniques can be extended to advanced attention mechanisms, such as tensor attention, while retaining almost linear time complexity for both training and inference \cite{as24_iclr}.~\cite{kll+25} provides an almost linear time algorithm to accelerate the inference of VAR Transformer. Other innovations include RoPE-based attention mechanisms~\cite{as24_rope,chl+24_rope} and differentially private cross-attention approaches~\cite{lssz24_dp}. Alternative strategies, such as the conv-basis method proposed in \cite{lls+24_conv}, present additional opportunities to accelerate attention computations, offering complementary solutions to this critical bottleneck. Additionally, various studies explore pruning-based methods to expedite attention mechanisms \cite{lls+24_prune,cls+24,llss24_sparse,ssz+25_prune,ssz+25_dit,hyw+23,whl+24,xhh+24,ssz+25_prune}.


\paragraph{Gradient Approximation.}
The low-rank approximation is a widely utilized approach for optimizing transformer training by reducing computational complexity \cite{lss+24,lssz24_tat,as24b,hwsl24,cls+24,lss+24_grad}. Building on the low-rank framework introduced in \cite{as23}, which initially focused on forward attention computation, \cite{as24b} extends this method to approximate attention gradients, effectively lowering the computational cost of gradient calculations. The study in \cite{lss+24} further expands this low-rank gradient approximation to multi-layer transformers, showing that backward computations in such architectures can achieve nearly linear time complexity. Additionally, \cite{lssz24_tat} generalizes the approach of \cite{as24b} to tensor-based attention models, utilizing forward computation results from \cite{as24_iclr} to enable efficient training of tensorized attention mechanisms. Lastly, \cite{hwsl24} applies low-rank approximation techniques during the training of Diffusion Transformers (DiTs), demonstrating the adaptability of these methods across various transformer-based architectures.




\ifdefined\isarxiv
%\section*{Acknowledgments}
\bibliographystyle{alpha}
\bibliography{ref}
\else


\fi

%%% some writing rules

%% Writing rule for creating tags.
%% Tags :
%% Theorem    \ref{thm:bla_bla}
%% Lemma      \ref{lem:bla_bla}
%% Claim      \ref{cla:bla_bla}
%% Corollary  \ref{cor:bla_bla}
%% Fact       \ref{fac:bla_bla}
%% Definition \ref{def:bla_bla}
%% Section    \ref{sec:bla_bla}
%% Subsection \ref{sub:bla_bla}
%% Equation   \ref{eq:bla_bla}



\end{document}



%%%%%%%%%%%%%%%%%%%%%%%%%%%%%%%%%%%%%%%%%%%%%%%%%%%%%%%%%%%%%%%%%%%%%%%%%%%%%%%%%%%%%%%%%%%%%%%%%%%%%%%%%%%%%%%%%%%%%%%%%%%%%%%%%%%%%%%%%%%%%%%%%%%%%%%%%%%%%%%%%%%%%%%%%%%%%%%%%%%%%%%%%%%%%%%%%%%%%%%%%%%%%%%%%%%%%%%%%%%%%%%%%%%%%%%%%%%%%%%%%%%%%%%%%%%%%%%%%%%%%%%%%%%%%%%%%%%%%%%%%%%%%%%%%%%%%%%%%%%%%%%%%%%%%%%%%%%%%%%%%%%%%%%%%%%%%%%%%%%%%%%%%%%%%%%%%%%%%%%%%%%%%%%%%%%%%%%%%%%%%%%%%%%%%%%%%%%%%%%%%%%%%%%%%%%%%%%%%%%%%%%%%%%%%%%%%%%%%%%%%%%%%%%%%%%%%%%%%%%%%%
