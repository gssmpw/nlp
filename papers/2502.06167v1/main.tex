 \def\isarxiv{1} %%% for icml submission version, we comment this line

\ifdefined\isarxiv
\documentclass[11pt]{article}

\usepackage[numbers]{natbib}

\else


%%%%%%%% ICML 2025 EXAMPLE LATEX SUBMISSION FILE %%%%%%%%%%%%%%%%%

\documentclass{article}

% Recommended, but optional, packages for figures and better typesetting:
\usepackage{microtype}
\usepackage{graphicx}
\usepackage{subfig}
% \usepackage{subfigure}
\usepackage{booktabs} % for professional tables

% hyperref makes hyperlinks in the resulting PDF.
% If your build breaks (sometimes temporarily if a hyperlink spans a page)
% please comment out the following usepackage line and replace
% \usepackage{icml2025} with \usepackage[nohyperref]{icml2025} above.
\usepackage{hyperref}


% Attempt to make hyperref and algorithmic work together better:
% \newcommand{\theHalgorithm}{\arabic{algorithm}}

% Use the following line for the initial blind version submitted for review:
% \usepackage{icml2025}

% If accepted, instead use the following line for the camera-ready submission:
\usepackage{icml2025}

% For theorems and such
\usepackage{amsmath}
\usepackage{amssymb}
\usepackage{mathtools}
\usepackage{amsthm}

% if you use cleveref..
\usepackage[capitalize,noabbrev]{cleveref}

%%%%%%%%%%%%%%%%%%%%%%%%%%%%%%%%
% THEOREMS
%%%%%%%%%%%%%%%%%%%%%%%%%%%%%%%%
\theoremstyle{plain}
\newtheorem{theorem}{Theorem}[section]
\newtheorem{proposition}[theorem]{Proposition}
\newtheorem{lemma}[theorem]{Lemma}
\newtheorem{corollary}[theorem]{Corollary}
\theoremstyle{definition}
\newtheorem{definition}[theorem]{Definition}
\newtheorem{assumption}[theorem]{Assumption}
\theoremstyle{remark}
\newtheorem{remark}[theorem]{Remark}

% Todonotes is useful during development; simply uncomment the next line
%    and comment out the line below the next line to turn off comments
%\usepackage[disable,textsize=tiny]{todonotes}
\usepackage[textsize=tiny]{todonotes}


% The \icmltitle you define below is probably too long as a header.
% Therefore, a short form for the running title is supplied here:
\icmltitlerunning{Fundamental Limits of Prompt Tuning VAR: Universality, Capacity and Efficiency}




\fi


\usepackage{amsmath}
\usepackage{amsthm}
\usepackage{amssymb}
% \usepackage{algorithm}
\usepackage{subfig}
% \usepackage{algpseudocode}
\usepackage{graphicx}
\usepackage{grffile}
\usepackage{wrapfig,epsfig}
\usepackage{url}
\usepackage{xcolor}
\usepackage{epstopdf}


\usepackage{bbm}
\usepackage{dsfont}

 
\allowdisplaybreaks
 

\ifdefined\isarxiv

\let\C\relax
\usepackage{tikz}
\usepackage{hyperref}  %%% arxiv don't allow this.
\hypersetup{colorlinks=true,citecolor=blue,linkcolor=blue} %%% Zhao : maybe we should comment this in submission.
\usetikzlibrary{arrows}
\usepackage[margin=1in]{geometry}

\else

\usepackage{microtype}
\usepackage{hyperref}
\definecolor{mydarkblue}{rgb}{0,0.08,0.45}
\hypersetup{colorlinks=true, citecolor=mydarkblue,linkcolor=mydarkblue}
 

\fi
 
\graphicspath{{./figs/}}

\theoremstyle{plain}
\newtheorem{theorem}{Theorem}[section]
\newtheorem{lemma}[theorem]{Lemma}
\newtheorem{definition}[theorem]{Definition}
\newtheorem{notation}[theorem]{Notation}
% \newtheorem{proof}[theorem]{Proof}
\newtheorem{proposition}[theorem]{Proposition}
\newtheorem{corollary}[theorem]{Corollary}
\newtheorem{conjecture}[theorem]{Conjecture}
\newtheorem{assumption}[theorem]{Assumption}
\newtheorem{observation}[theorem]{Observation}
\newtheorem{fact}[theorem]{Fact}
\newtheorem{remark}[theorem]{Remark}
\newtheorem{claim}[theorem]{Claim}
\newtheorem{example}[theorem]{Example}
\newtheorem{problem}[theorem]{Problem}
\newtheorem{open}[theorem]{Open Problem}
\newtheorem{property}[theorem]{Property}
\newtheorem{hypothesis}[theorem]{Hypothesis}

\newcommand{\wh}{\widehat}
\newcommand{\wt}{\widetilde}
\newcommand{\ov}{\overline}
\newcommand{\N}{\mathcal{N}}
\newcommand{\R}{\mathbb{R}}
\newcommand{\RHS}{\mathrm{RHS}}
\newcommand{\LHS}{\mathrm{LHS}}
\renewcommand{\d}{\mathrm{d}}
\renewcommand{\i}{\mathbf{i}}
\renewcommand{\tilde}{\wt}
\renewcommand{\hat}{\wh}
\newcommand{\Tmat}{{\cal T}_{\mathrm{mat}}}
\newcommand{\SETH}{\mathsf{SETH}}
\newcommand{\A}{\mathsf{A}}
\newcommand{\VAR}{\mathrm{VAR}}
\newcommand{\X}{\mathsf{X}}
\newcommand{\Y}{\mathsf{Y}}
\newcommand{\Z}{\mathsf{Z}}
\newcommand{\F}{\mathsf{F}}
\newcommand{\V}{\mathsf{V}}

\DeclareMathOperator*{\E}{{\mathbb{E}}}
\DeclareMathOperator*{\var}{\mathrm{Var}}
% \DeclareMathOperator*{\Z}{\mathbb{Z}}
\DeclareMathOperator*{\C}{\mathbb{C}}
\DeclareMathOperator*{\D}{\mathcal{D}}
\DeclareMathOperator*{\median}{median}
\DeclareMathOperator*{\mean}{mean}
\DeclareMathOperator{\OPT}{OPT}
\DeclareMathOperator{\supp}{supp}
\DeclareMathOperator{\poly}{poly}

\DeclareMathOperator{\nnz}{nnz}
\DeclareMathOperator{\sparsity}{sparsity}
\DeclareMathOperator{\rank}{rank}
\DeclareMathOperator{\diag}{diag}
\DeclareMathOperator{\dist}{dist}
\DeclareMathOperator{\cost}{cost}
\DeclareMathOperator{\vect}{vec}
\DeclareMathOperator{\tr}{tr}
\DeclareMathOperator{\dis}{dis}
\DeclareMathOperator{\cts}{cts}
% \DeclareMathOperator{\X}{{\mathsf{X}}}



\makeatletter
\newcommand*{\RN}[1]{\expandafter\@slowromancap\romannumeral #1@}
\makeatletter
% \newcommand*{\RN}[1]{\expandafter\@slowromancap\romannumeral #1@}
% \makeatother
% \newcommand{\Zhao}[1]{{\color{red}[Zhao: #1]}}
% \newcommand{\Zhenmei}[1]{{\color{purple}[Zhenmei: #1]}}
% \newcommand{\Yifang}[1]{{\color{blue}[Yifang: #1]}} 
% \newcommand{\Xiaoyu}[1]{{\color{orange}[Xiaoyu: #1]}} %%%Change to intern name



\usepackage{lineno}
\def\linenumberfont{\normalfont\small}


% below is the command for this paper only Zhenmei Dec 21 2024 
% \newcommand{\f}{f}
% \newcommand{\h}{h}
% \renewcommand{\c}{c}
% \newcommand{\q}{q}
% \renewcommand{\L}{L}
% \renewcommand{\r}{r}
% \newcommand{\p}{p}


\newcommand{\f}{s}
\newcommand{\h}{v}
\renewcommand{\c}{\ensuremath{\ell}}
\newcommand{\q}{\ensuremath{\beta}}
\renewcommand{\L}{\ensuremath{\mathsf{Loss}}}
\renewcommand{\r}{\ensuremath{\rho}}
\newcommand{\p}{\ensuremath{\gamma}}


\begin{document}

\ifdefined\isarxiv

\date{}


\title{Universal Approximation of Visual Autoregressive Transformers}
\author{
Yifang Chen\thanks{\texttt{
yifangc@uchicago.edu}. The University of Chicago.}
\and
Xiaoyu Li\thanks{\texttt{
xiaoyu.li2@student.unsw.edu.au}. University of New South Wales.}
\and
Yingyu Liang\thanks{\texttt{
yingyul@hku.hk}. The University of Hong Kong. \texttt{
yliang@cs.wisc.edu}. University of Wisconsin-Madison.} 
\and
Zhenmei Shi\thanks{\texttt{
zhmeishi@cs.wisc.edu}. University of Wisconsin-Madison.}
\and 
Zhao Song\thanks{\texttt{ magic.linuxkde@gmail.com}. The Simons Institute for the Theory of Computing at UC Berkeley.}
}



\else


\twocolumn[
\icmltitle{Universal Approximation of Visual Autoregressive Transformers}

% It is OKAY to include author information, even for blind
% submissions: the style file will automatically remove it for you
% unless you've provided the [accepted] option to the icml2025
% package.

% List of affiliations: The first argument should be a (short)
% identifier you will use later to specify author affiliations
% Academic affiliations should list Department, University, City, Region, Country
% Industry affiliations should list Company, City, Region, Country

% You can specify symbols, otherwise they are numbered in order.
% Ideally, you should not use this facility. Affiliations will be numbered
% in order of appearance and this is the preferred way.
\icmlsetsymbol{equal}{*}

\begin{icmlauthorlist}
\icmlauthor{Firstname1 Lastname1}{equal,yyy}
\icmlauthor{Firstname2 Lastname2}{equal,yyy,comp}
\icmlauthor{Firstname3 Lastname3}{comp}
\icmlauthor{Firstname4 Lastname4}{sch}
\icmlauthor{Firstname5 Lastname5}{yyy}
\icmlauthor{Firstname6 Lastname6}{sch,yyy,comp}
\icmlauthor{Firstname7 Lastname7}{comp}
%\icmlauthor{}{sch}
\icmlauthor{Firstname8 Lastname8}{sch}
\icmlauthor{Firstname8 Lastname8}{yyy,comp}
%\icmlauthor{}{sch}
%\icmlauthor{}{sch}
\end{icmlauthorlist}

\icmlaffiliation{yyy}{Department of XXX, University of YYY, Location, Country}
\icmlaffiliation{comp}{Company Name, Location, Country}
\icmlaffiliation{sch}{School of ZZZ, Institute of WWW, Location, Country}

\icmlcorrespondingauthor{Firstname1 Lastname1}{first1.last1@xxx.edu}
\icmlcorrespondingauthor{Firstname2 Lastname2}{first2.last2@www.uk}

% You may provide any keywords that you
% find helpful for describing your paper; these are used to populate
% the "keywords" metadata in the PDF but will not be shown in the document
\icmlkeywords{Machine Learning, ICML}

\vskip 0.3in
]

% this must go after the closing bracket ] following \twocolumn[ ...

% This command actually creates the footnote in the first column
% listing the affiliations and the copyright notice.
% The command takes one argument, which is text to display at the start of the footnote.
% The \icmlEqualContribution command is standard text for equal contribution.
% Remove it (just {}) if you do not need this facility.

%\printAffiliationsAndNotice{}  % leave blank if no need to mention equal contribution
\printAffiliationsAndNotice{\icmlEqualContribution} % otherwise use the standard text.


\fi





\ifdefined\isarxiv
\begin{titlepage}
  \maketitle
  \begin{abstract}
\begin{abstract}


The choice of representation for geographic location significantly impacts the accuracy of models for a broad range of geospatial tasks, including fine-grained species classification, population density estimation, and biome classification. Recent works like SatCLIP and GeoCLIP learn such representations by contrastively aligning geolocation with co-located images. While these methods work exceptionally well, in this paper, we posit that the current training strategies fail to fully capture the important visual features. We provide an information theoretic perspective on why the resulting embeddings from these methods discard crucial visual information that is important for many downstream tasks. To solve this problem, we propose a novel retrieval-augmented strategy called RANGE. We build our method on the intuition that the visual features of a location can be estimated by combining the visual features from multiple similar-looking locations. We evaluate our method across a wide variety of tasks. Our results show that RANGE outperforms the existing state-of-the-art models with significant margins in most tasks. We show gains of up to 13.1\% on classification tasks and 0.145 $R^2$ on regression tasks. All our code and models will be made available at: \href{https://github.com/mvrl/RANGE}{https://github.com/mvrl/RANGE}.

\end{abstract}



  \end{abstract}
  \thispagestyle{empty}
\end{titlepage}

{\hypersetup{linkcolor=black}
\tableofcontents
}
\newpage

\else

\begin{abstract}
\begin{abstract}


The choice of representation for geographic location significantly impacts the accuracy of models for a broad range of geospatial tasks, including fine-grained species classification, population density estimation, and biome classification. Recent works like SatCLIP and GeoCLIP learn such representations by contrastively aligning geolocation with co-located images. While these methods work exceptionally well, in this paper, we posit that the current training strategies fail to fully capture the important visual features. We provide an information theoretic perspective on why the resulting embeddings from these methods discard crucial visual information that is important for many downstream tasks. To solve this problem, we propose a novel retrieval-augmented strategy called RANGE. We build our method on the intuition that the visual features of a location can be estimated by combining the visual features from multiple similar-looking locations. We evaluate our method across a wide variety of tasks. Our results show that RANGE outperforms the existing state-of-the-art models with significant margins in most tasks. We show gains of up to 13.1\% on classification tasks and 0.145 $R^2$ on regression tasks. All our code and models will be made available at: \href{https://github.com/mvrl/RANGE}{https://github.com/mvrl/RANGE}.

\end{abstract}


\end{abstract}

\fi


\section{Introduction}
Backdoor attacks pose a concealed yet profound security risk to machine learning (ML) models, for which the adversaries can inject a stealth backdoor into the model during training, enabling them to illicitly control the model's output upon encountering predefined inputs. These attacks can even occur without the knowledge of developers or end-users, thereby undermining the trust in ML systems. As ML becomes more deeply embedded in critical sectors like finance, healthcare, and autonomous driving \citep{he2016deep, liu2020computing, tournier2019mrtrix3, adjabi2020past}, the potential damage from backdoor attacks grows, underscoring the emergency for developing robust defense mechanisms against backdoor attacks.

To address the threat of backdoor attacks, researchers have developed a variety of strategies \cite{liu2018fine,wu2021adversarial,wang2019neural,zeng2022adversarial,zhu2023neural,Zhu_2023_ICCV, wei2024shared,wei2024d3}, aimed at purifying backdoors within victim models. These methods are designed to integrate with current deployment workflows seamlessly and have demonstrated significant success in mitigating the effects of backdoor triggers \cite{wubackdoorbench, wu2023defenses, wu2024backdoorbench,dunnett2024countering}.  However, most state-of-the-art (SOTA) backdoor purification methods operate under the assumption that a small clean dataset, often referred to as \textbf{auxiliary dataset}, is available for purification. Such an assumption poses practical challenges, especially in scenarios where data is scarce. To tackle this challenge, efforts have been made to reduce the size of the required auxiliary dataset~\cite{chai2022oneshot,li2023reconstructive, Zhu_2023_ICCV} and even explore dataset-free purification techniques~\cite{zheng2022data,hong2023revisiting,lin2024fusing}. Although these approaches offer some improvements, recent evaluations \cite{dunnett2024countering, wu2024backdoorbench} continue to highlight the importance of sufficient auxiliary data for achieving robust defenses against backdoor attacks.

While significant progress has been made in reducing the size of auxiliary datasets, an equally critical yet underexplored question remains: \emph{how does the nature of the auxiliary dataset affect purification effectiveness?} In  real-world  applications, auxiliary datasets can vary widely, encompassing in-distribution data, synthetic data, or external data from different sources. Understanding how each type of auxiliary dataset influences the purification effectiveness is vital for selecting or constructing the most suitable auxiliary dataset and the corresponding technique. For instance, when multiple datasets are available, understanding how different datasets contribute to purification can guide defenders in selecting or crafting the most appropriate dataset. Conversely, when only limited auxiliary data is accessible, knowing which purification technique works best under those constraints is critical. Therefore, there is an urgent need for a thorough investigation into the impact of auxiliary datasets on purification effectiveness to guide defenders in  enhancing the security of ML systems. 

In this paper, we systematically investigate the critical role of auxiliary datasets in backdoor purification, aiming to bridge the gap between idealized and practical purification scenarios.  Specifically, we first construct a diverse set of auxiliary datasets to emulate real-world conditions, as summarized in Table~\ref{overall}. These datasets include in-distribution data, synthetic data, and external data from other sources. Through an evaluation of SOTA backdoor purification methods across these datasets, we uncover several critical insights: \textbf{1)} In-distribution datasets, particularly those carefully filtered from the original training data of the victim model, effectively preserve the model’s utility for its intended tasks but may fall short in eliminating backdoors. \textbf{2)} Incorporating OOD datasets can help the model forget backdoors but also bring the risk of forgetting critical learned knowledge, significantly degrading its overall performance. Building on these findings, we propose Guided Input Calibration (GIC), a novel technique that enhances backdoor purification by adaptively transforming auxiliary data to better align with the victim model’s learned representations. By leveraging the victim model itself to guide this transformation, GIC optimizes the purification process, striking a balance between preserving model utility and mitigating backdoor threats. Extensive experiments demonstrate that GIC significantly improves the effectiveness of backdoor purification across diverse auxiliary datasets, providing a practical and robust defense solution.

Our main contributions are threefold:
\textbf{1) Impact analysis of auxiliary datasets:} We take the \textbf{first step}  in systematically investigating how different types of auxiliary datasets influence backdoor purification effectiveness. Our findings provide novel insights and serve as a foundation for future research on optimizing dataset selection and construction for enhanced backdoor defense.
%
\textbf{2) Compilation and evaluation of diverse auxiliary datasets:}  We have compiled and rigorously evaluated a diverse set of auxiliary datasets using SOTA purification methods, making our datasets and code publicly available to facilitate and support future research on practical backdoor defense strategies.
%
\textbf{3) Introduction of GIC:} We introduce GIC, the \textbf{first} dedicated solution designed to align auxiliary datasets with the model’s learned representations, significantly enhancing backdoor mitigation across various dataset types. Our approach sets a new benchmark for practical and effective backdoor defense.


 %%% Section 1. Introduction
\section{Related Work}

\subsection{Large 3D Reconstruction Models}
Recently, generalized feed-forward models for 3D reconstruction from sparse input views have garnered considerable attention due to their applicability in heavily under-constrained scenarios. The Large Reconstruction Model (LRM)~\cite{hong2023lrm} uses a transformer-based encoder-decoder pipeline to infer a NeRF reconstruction from just a single image. Newer iterations have shifted the focus towards generating 3D Gaussian representations from four input images~\cite{tang2025lgm, xu2024grm, zhang2025gslrm, charatan2024pixelsplat, chen2025mvsplat, liu2025mvsgaussian}, showing remarkable novel view synthesis results. The paradigm of transformer-based sparse 3D reconstruction has also successfully been applied to lifting monocular videos to 4D~\cite{ren2024l4gm}. \\
Yet, none of the existing works in the domain have studied the use-case of inferring \textit{animatable} 3D representations from sparse input images, which is the focus of our work. To this end, we build on top of the Large Gaussian Reconstruction Model (GRM)~\cite{xu2024grm}.

\subsection{3D-aware Portrait Animation}
A different line of work focuses on animating portraits in a 3D-aware manner.
MegaPortraits~\cite{drobyshev2022megaportraits} builds a 3D Volume given a source and driving image, and renders the animated source actor via orthographic projection with subsequent 2D neural rendering.
3D morphable models (3DMMs)~\cite{blanz19993dmm} are extensively used to obtain more interpretable control over the portrait animation. For example, StyleRig~\cite{tewari2020stylerig} demonstrates how a 3DMM can be used to control the data generated from a pre-trained StyleGAN~\cite{karras2019stylegan} network. ROME~\cite{khakhulin2022rome} predicts vertex offsets and texture of a FLAME~\cite{li2017flame} mesh from the input image.
A TriPlane representation is inferred and animated via FLAME~\cite{li2017flame} in multiple methods like Portrait4D~\cite{deng2024portrait4d}, Portrait4D-v2~\cite{deng2024portrait4dv2}, and GPAvatar~\cite{chu2024gpavatar}.
Others, such as VOODOO 3D~\cite{tran2024voodoo3d} and VOODOO XP~\cite{tran2024voodooxp}, learn their own expression encoder to drive the source person in a more detailed manner. \\
All of the aforementioned methods require nothing more than a single image of a person to animate it. This allows them to train on large monocular video datasets to infer a very generic motion prior that even translates to paintings or cartoon characters. However, due to their task formulation, these methods mostly focus on image synthesis from a frontal camera, often trading 3D consistency for better image quality by using 2D screen-space neural renderers. In contrast, our work aims to produce a truthful and complete 3D avatar representation from the input images that can be viewed from any angle.  

\subsection{Photo-realistic 3D Face Models}
The increasing availability of large-scale multi-view face datasets~\cite{kirschstein2023nersemble, ava256, pan2024renderme360, yang2020facescape} has enabled building photo-realistic 3D face models that learn a detailed prior over both geometry and appearance of human faces. HeadNeRF~\cite{hong2022headnerf} conditions a Neural Radiance Field (NeRF)~\cite{mildenhall2021nerf} on identity, expression, albedo, and illumination codes. VRMM~\cite{yang2024vrmm} builds a high-quality and relightable 3D face model using volumetric primitives~\cite{lombardi2021mvp}. One2Avatar~\cite{yu2024one2avatar} extends a 3DMM by anchoring a radiance field to its surface. More recently, GPHM~\cite{xu2025gphm} and HeadGAP~\cite{zheng2024headgap} have adopted 3D Gaussians to build a photo-realistic 3D face model. \\
Photo-realistic 3D face models learn a powerful prior over human facial appearance and geometry, which can be fitted to a single or multiple images of a person, effectively inferring a 3D head avatar. However, the fitting procedure itself is non-trivial and often requires expensive test-time optimization, impeding casual use-cases on consumer-grade devices. While this limitation may be circumvented by learning a generalized encoder that maps images into the 3D face model's latent space, another fundamental limitation remains. Even with more multi-view face datasets being published, the number of available training subjects rarely exceeds the thousands, making it hard to truly learn the full distibution of human facial appearance. Instead, our approach avoids generalizing over the identity axis by conditioning on some images of a person, and only generalizes over the expression axis for which plenty of data is available. 

A similar motivation has inspired recent work on codec avatars where a generalized network infers an animatable 3D representation given a registered mesh of a person~\cite{cao2022authentic, li2024uravatar}.
The resulting avatars exhibit excellent quality at the cost of several minutes of video capture per subject and expensive test-time optimization.
For example, URAvatar~\cite{li2024uravatar} finetunes their network on the given video recording for 3 hours on 8 A100 GPUs, making inference on consumer-grade devices impossible. In contrast, our approach directly regresses the final 3D head avatar from just four input images without the need for expensive test-time fine-tuning.




\section{Preliminary} \label{sec:preliminary}
In this section, we first introduce the notations in Section~\ref{sec:notations}. Then, we present the general problem formulation in Section~\ref{sec:problem_formulation}.

\subsection{Notations}\label{sec:notations}
For any positive integer $n$, we use $[n]$ to denote the set $\{1, 2, \ldots, n\}$. We use $\mathbb{N}_+$ to represent the set of all positive integers. For two sets $\mathcal{B}$ and $\mathcal{C}$, we denote the set difference as $\mathcal{B} \setminus \mathcal{C}:=\{x\in \mathcal{B}:x\notin\mathcal{C}\}$. For a vector $x \in \mathbb{R}^d$, $\Diag(d)$ denotes a diagonal matrix $X \in \mathbb{R}^{d \times d}$, where the diagonal entries satisfy $X_{i,i} = x_i$ for all $i \in [d]$, and all off-diagonal entries are zero. We use $\mathbf{1}_n$ to denote an $n$-dimensional column vector with all entries equal to one.


\begin{figure}[!ht]
    \centering
    \includegraphics[width=0.95\linewidth]{our_research.pdf}
    \caption{
    Our research objective. This figure presents the goal of our study: creating a more equitable desk-rejection system. Consider Professor A, who has carelessly submitted numerous papers exceeding the submission limit, collaborating with another senior researcher (Professor B) with many submissions, and a young student with only one paper. Our proposed system prioritizes desk-rejecting papers from authors with a large number of submissions first, thereby increasing the student’s chances of having their paper accepted. This approach aims to mitigate the disparity in the impact of desk rejections and promote fairness.
    }
    \label{fig:our_research}
\end{figure}

\subsection{Problem Formulation} \label{sec:problem_formulation}

In this section, we further introduce the actual problem we will investigate in this paper, where we begin with introducing the definition for three kinds of authors that will appear later in our discussion. 


\begin{definition}[Submission Limit Problem]\label{def:submit_limit_problem}
    Let $\mathcal{A} = \{a_1, a_2, \dots, a_n\}$ denote the set of $n$ authors, and let $\mathcal{P} = \{p_1, p_2, \dots, p_m\}$ denote the set of $m$ papers. Each author $a_i \in \mathcal{A}$ has a subset of papers $P_i \subseteq \mathcal{P}$, and each paper $p_j \in \mathcal{P}$ is authored by a subset of authors $A_j \subseteq \mathcal{A}$. For each author, $a_i \in \mathcal{A}$, let $C_i$ denote the set of all coauthors of $a_i$ and let $x \in \mathbb{N}_+$ denote the maximum number of papers each author can submit. 

    The goal is to find a subset $S \subseteq \mathcal{P}$ of papers (to keep) such that for every $a_i\in\mathcal{A},$
    \begin{align*}
        \underbrace{|\{p_j \in  S : a_i \in A_j\}|}_{\#\mathrm{remained~papers~of~author}~a_i} \leq x.  
    \end{align*}
    or equivalently find a subset $\ov{S} \subseteq \mathcal{P}$ of papers (to reject) such that for every $a_i \in \mathcal{A}$,
        \begin{align*}
        |P_i| - \underbrace{|\{j \in  \ov{S} : i \in A_j\}|}_{\#{\mathrm{rejected~papers~of~author}~}a_i} \leq x.  
    \end{align*}
\end{definition}

We now present several fundamental facts related to Definition~\ref{def:submit_limit_problem}, which can be easily verified through basic set theory. 
\begin{fact}
    For any author $a_i \in \mathcal{A}$ and paper $p_j \in \mathcal{P}$, $a_i \in A_j$ if and only if $p_j \in P_i$.
\end{fact}

\begin{fact}
    For each author $a_i \in \mathcal{A}$, the number of papers submitted by the author can be formulated as:
    \begin{align*}
        |P_i| = |\{p_j \in \mathcal{P} : a_i \in A_j\}|.
    \end{align*}
\end{fact}

\begin{fact}
    For each paper $j \in [m]$, the number of authors of this paper can be formulated as:
    \begin{align*}
        |A_j| = |\{a_i \in \mathcal{A} : p_j \in P_i\}|.
    \end{align*}
\end{fact}

\begin{fact}
    For each author $a_i \in \mathcal{A}$, the set of coauthors for author $a_i$ can be formulated as:
    \begin{align*}
        C_i = (\bigcup_{p_j \in P_i} A_j) \setminus \{ a_i \}.
    \end{align*}
\end{fact}



\section{Any-Rank Single-Layer Attention is a Contextual Mapping Function}
\label{sec:context_map}
In this section, we show that Attention is a contextual mapping function. 
In Section~\ref{sec:cont_map}, we give the definition of contextual mapping.
In Section~\ref{sec:any_rank_attn_contmap}, we introduce any-rank single-layer attention as a contextual mapping function. 



\subsection{Contextual Mapping}\label{sec:cont_map}


{\bf Contextual Mapping.}
Let $X, Y \in \R^{n\times d}$ be the input embeddings and output label sequences, respectively. 
Let $X_{i} \in \R^{d}$ be the $i$-th token of each $X$ embedding sequence. 

\begin{definition}[Vocabulary, Definition 2.4 from~\cite{hwg+24} on Page 8]\label{def:vocab}
    We define the vocabulary.
    \begin{itemize}
        \item We define the $i$-th vocabulary set for $i \in [N]$ by $\mathcal{V}^{(i)}=\cup_{k \in[n]} X_k^{(i)} \subset \R^{d}$.
        \item We define the whole vocabulary set $\mathcal{V}$ as $\mathcal{V}=\cup_{i \in[N]} \mathcal{V}^{(i)} \subset \R^{d}$.
    \end{itemize}
\end{definition}
Note that while ``vocabulary'' typically refers to the tokens' codomain, here, it refers to the set of all tokens within a single sequence.
To facilitate our analysis, 
we introduce the idea of input token separation following~\cite{ks24,kkm22,ybr+20}.

\begin{definition}[Tokenwise Separateness, Definition 2.5 from~\cite{hwg+24} on Page 8] \label{def:token_seperate_new}
    We define the tokenwise separateness as follows.
    \begin{itemize}
        \item Let $X^{(1)}, \hdots, X^{(N)} \in \R^{n\times d}$ be embeddings. 
        \item Let $N$ be the number of sequences in the datasets.
        \item Let $n$ be the length of a sequence. i.e. $X^{(i)} \in \R^{n\times d}$
    \end{itemize}
    
    First, we state three conditions for $X^{(1)}, \hdots, X^{(N)}$ 
    \begin{itemize}
        \item [(i)] For any $i \in[N]$ and $k \in[n],\| X_k^{(i)}\|_2 > \gamma_{\min }$ holds. 
        \item [(ii)]
        For any $i \in[N]$ and $k \in[n], \|X_k^{(i)}\|_2<\gamma_{\max }$ holds.
        \item [(iii)]
        For any $i, j \in [N]$ and $k, l \in [n]$ if $X_k^{(i)} \neq X_ l^{(j)}, $ then $ \|X_k^{(i)}- X_l^{(j)}\|_2 > \delta$ holds.
    \end{itemize}
    Second, we define three types of separateness as follows, 
    \begin{itemize}
        \item {\bf Part 1.} If all conditions hold, then we call it  tokenwise $(\gamma_{\min }, \gamma_{\max }, \delta)$-separated
        \item {\bf Part 2.} If conditions (ii) and (iii) hold, then we denote this as $(\gamma, \delta)$-separateness.
        \item {\bf Part 3.} If only condition (iii) holds, then we denote it as $(\delta)$-separateness.
    \end{itemize} 
\end{definition}
To clarify condition (iii), we consider cases where there are repeated tokens between different input sequences. 
Next, we define contextual mapping. 
Contextual mapping describes a function's ability to capture the context of each input sequence as a whole and assign a unique ID to each input sequence.


\begin{definition} [$(\gamma,\delta)$-Contextual Mapping, Definition 2.6 from~\cite{hwg+24} on Page 8] \label{def:contextual_mapping_new}
A function $q:\R^{n\times d} \to \R^{n\times d}$ is said to be a $(\gamma, \delta)$-contextual mapping for a set of embeddings  $X^{(1)}, \hdots, X^{(N)} \in \R^{n\times d}$,  if the following conditions hold:
    \begin{itemize}
        \item {\bf Contextual Sensitivity $\gamma$.}
        For any $i \in[N]$ and $k \in[n],  \|q (X^{(i)})_k\|_2 < \gamma$ holds.
    
        \item {\bf Approximation Error $\delta$.}
        For any $i, j \in[N]$ and $k, l \in[n]$ such that $\mathcal{V}^{(i)} \neq \mathcal{V}^{(j)}$ or $X_k^{(i)} \neq X_l^{(j)}$, $\|q(X^{(i)})_k-q(X^{(j)})_l\|_2 > \delta$ holds.
    \end{itemize}
In addition, Note that $q (X^{(i)})$ for $i \in[N]$ is called a context ID of $X^{(i)}$.    
\end{definition}


\subsection{Any-Rank Single-Layer Attention is a Contextual Mapping Function}\label{sec:any_rank_attn_contmap}

Now we present the result showing that a softmax-based $1$-head, $1$-layer attention block with any-rank weight matrices is a contextual mapping.

\begin{lemma}[Any-Rank Attention as a $(\gamma, \delta)$-Contextual Mapping, Lemma 2.2 from~\cite{hwg+24} on Page 9]
\label{lem:contextual_map_self_attn_new}

    If the following conditions hold:
    \begin{itemize}
        \item Let $X^{(1)}, \hdots, X^{(N)} \in \R^{n \times d}$ be embeddings that are $(\gamma_{\min}, \gamma_{\max}, \epsilon)$-tokenwise separated, with the vocabulary set $\mathcal{V} = \cup_{i \in [N]} \mathcal{V}^{(i)} \subset \R^{d}$.
        \item $X_k^{(i)} \neq X_l^{(i)}$ for any $i \in [N]$ and $k, l \in [L]$.
        \item Let $
        \gamma = \gamma_{\max} + \frac{\epsilon}{4}$
        \item Let $
        \delta = \exp(-5 \epsilon^{-1} |{\cal V}|^4 d \kappa \gamma_{\max} \log L )$
        \item Let $\kappa := \gamma_{\max}/\gamma_{\min}$.
        \item Let $W^{(O)} \in \R^{d \times s}$ and $W_V, W_K, W_Q \in \R^{s \times d}$.
    \end{itemize}

    Then, we can show
    \begin{itemize}
        \item 1-layer, single-head attention mechanism serves as a $(\gamma, \delta)$-contextual mapping for the embeddings $X^{(1)}, \hdots, X^{(N)}$ with weight matrices $W^{(O)}$ and $W_V, W_K, W_Q$. 
    \end{itemize}
    
\end{lemma}


Lemma~\ref{lem:contextual_map_self_attn_new} indicates that any-rank self-attention function distinguishes input tokens $X_k^{(i)}=X_l^{(j)}$ such that $\mathcal{V}^{(i)} \neq \mathcal{V}^{(j)}$.
In other words, 
it distinguishes two identical tokens within a different context.
\section{Universality of VAR Transformer}\label{sec:var_mainresult}

In this section, we present our proof for the universality of the VAR Transformer.
In Section~\ref{sec:universality_2}, we used a universality result from a previous work.
In Section~\ref{sec:two_layer_pertu}, we analyze how the error behaves when two consecutive layers in our composition are each replaced by their respective approximations.
In Section~\ref{sec:pertu_recursive_one}, we present the scenario when one of the composited layers got replaced by a different function.
In Section~\ref{sec:pertu_recursive_all}, we present the scenario when all of the composited layers got replaced. 
In Section~\ref{sec:var_universality}, we present our proof for the universality of the VAR Transformer.


\subsection{Universality of \texorpdfstring{$\mathcal{T}_A^{1,1,4}$}{} with \texorpdfstring{$O((1/ \epsilon)^{d 
n})$}{} FFN Layers}
\label{sec:universality_2}

We used a universality result from~\cite{hwg+24}.

\begin{lemma}[$\tau \in \mathcal{T}^{1,1,4}_A$ Transformer is Universal Seq2Seq Approximator, Theorem 2.3 in~\cite{hwg+24} on Page 11]
\label{lem:PT_uni_multi_layer_FF2}
    If the following conditions hold: 
    \begin{itemize}
        \item Let $1 \leq p < \infty$ and $\epsilon > 0$.
        \item Let a transformer with one self-attention layer defined as $\tau \in \mathcal{T}^{1,1,4}_A$
    \end{itemize}

    Then, there exists
    \begin{itemize}
        \item a transformer $\tau$ with single self-attention layer, such that for any $\mathcal{L} \in \mathcal{F}_{C}$ there exists
    $ \| \tau ( \cdot), \mathcal{L}\|_\alpha \leq \epsilon$.
    \end{itemize}
\end{lemma}


\subsection{Two Layers Perturbation}\label{sec:two_layer_pertu}

In this section, we analyze how the error behaves when two consecutive layers in our composition are each replaced by their respective approximations. Specifically, we consider the composition $f_i \circ g_i$ and replace $g_i$ with an up interpolation function $\Phi_{\mathrm{up},i}$ and $f_i$ with a one-layer transformer $\tau_i$. We show that under appropriate Lipschitz and approximation assumptions, the overall error of the approximated two-layer composition can be controlled in terms of the individual approximation errors.

\begin{assumption}[Target Function Class]\label{as:function_class}
    We assume the following things:
    \begin{itemize}
        \item Let $f_1, \ldots, f_r$ be $r$ $K$-Lipschitz functions from $\R^{h_r \times w_r \times d}$ to $\R^{h_r \times w_r \times d}$.
        \item For each $i \in [r]$, let $g_i$ be a $K$-Lipschitz function from $\R^{h_{i-1} \times w_{i-1} \times d}$ to $\R^{h_i \times w_i \times d}$.
        \item We assume that for each $i \in [r]$, $g_i$ can be approximated by some up interpolation function $\phi_{\mathrm{up},i}$.
        \item We assume that the target function $f_{\mathrm{word2img}}:\R^{1 \times 1 \times d} \to \R^{h_r \times w_r \times d}$ satisfies
    \begin{align*}
        f_{\mathrm{word2img}} := f_r \circ g_r \cdots \circ f_1 \circ g_1.
    \end{align*}
    \end{itemize}
\end{assumption}

With the Assumption~\ref{as:function_class}, we present the two layers of perturbation as follows. 

\begin{lemma}[Two Layers Perturbation]
    Let $\phi_{{\rm up},i}$ be the up interpolation function defined in~\ref{def:up_inter_layer_one_step}. Let $f_r$ be $r$ $K$-Lipschitz functions from Assumption~\ref{as:function_class}. Let $g_i$ be $r$ $K$-Lipschitz functions from Assumption~\ref{as:function_class}. Let $\tau_i$ be the one-layer transformer defined in Eq. 2.4 from~\cite{hwg+24}. If the following conditions hold:
    \begin{itemize}
        \item $\|g_i - \Phi_{\mathrm{up},i}\| \leq \epsilon_{1,i}$ from Assumption~\ref{as:function_class}.
        \item $\|f_i - \tau_i\| \leq \epsilon_{2,i}$ from Theorem~\ref{lem:PT_uni_multi_layer_FF2}.
        \item $f_i$ is $K_{1,i}$-Lipschitz.
    \end{itemize}
    Then we have
    \begin{align*}
        \|f_i \circ g_i - \tau_i \circ \Phi_{\mathrm{up},i}\| \leq K_{1,i} \epsilon_{1,i} + \epsilon_{2,i}.
    \end{align*}
\end{lemma}
\begin{proof}
    We can show that
    \begin{align*}
         \| f_i \circ g_i -  \tau_i \circ \Phi_{\mathrm{up},i} \| 
         = &~ \|f_i \circ g_i- f_i \circ \Phi_{\mathrm{up},i} + f_i \circ \Phi_{\mathrm{up},i}- \tau_i \circ \Phi_{\mathrm{up},i}\| \\
         \leq &~ \|f_i \circ g_i- f_i \circ \Phi_{\mathrm{up},i}\| + \|f_i \circ \Phi_{\mathrm{up},i}- \tau_i \circ \Phi_{\mathrm{up},i}\| \\
         = &~ \|f_i \circ (g_i- \Phi_{\mathrm{up},i})\| + \|(f_i -\tau_i) \circ \Phi_{\mathrm{up},i}\| \\
         \leq &~ \|f_i \circ (g_i- \Phi_{\mathrm{up},i})\| + \|f_i -\tau_i \| \\
         \leq &~ K_{1,i} \epsilon_{1,i} + \epsilon_{2,i}
    \end{align*}
    where the first step follows from basic algebra, the second step follows from triangle inequality, the third and fourth steps follow from basic algebra, and the fifth step follows from our conditions.
\end{proof}


\subsection{Perturbation of Recursively Composting Functions that One Layer is Different}\label{sec:pertu_recursive_one}

In this section, we consider a scenario where we have a composition of many layers, but only one of the layers is replaced by a different function. This setting helps us see how a single local perturbation can propagate through subsequent layers in a multi-layer composition. The lemma below quantifies this propagation by leveraging Lipschitz continuity.

\begin{lemma}[Perturbation of Recursively Composting Functions, One Layer is Different]\label{lem:one_layer_perturbation}
if the following conditions hold
\begin{itemize}
    \item Assume $\|  u_j(w) - v_j(w) \| \leq \epsilon $ for any $w$.
    \item $v_i(x) \leq K_2 \cdot \| x \|$
\end{itemize}
    Fix $j$, we have 
    \begin{align*}
        \| \circ_{i=j+1}^{n+1} v_i  \circ_{i=1}^j u_i - \circ_{i=j}^n v_i  \circ_{i=0}^{j-1} u_i \| \leq K_2^{n-j} \cdot \epsilon
    \end{align*}
\end{lemma}

\begin{proof}
We define $u$
\begin{align*}
    w = \circ_{i=0}^{j-1} u_i (x)
\end{align*}

    We can show that for any $x$
    \begin{align*}
        \| \circ_{i=j+1}^{n+1} v_i  \circ_{i=1}^j u_i (x) - \circ_{i=j}^n v_i  \circ_{i=0}^{j-1} u_i (x) \| 
        = & ~  \| \circ_{i=j+1}^{n+1} v_i  u_j(w) - \circ_{i=j+1}^n v_i  ( v_j(w) ) \| \\
        = & ~ \| \circ_{i=j+1}^{n+1} v_i ( u_j(w) - v_j(w)  ) \| \\
        \leq & ~ K_2^{n-j} \cdot \epsilon
    \end{align*}
    where the first step follows from basic algebra, the second step follows from linearity, and the third step follows from lemma assumptions.
\end{proof}

\subsection{Perturbation of Recursively Composting Functions that All Layer are Different}\label{sec:pertu_recursive_all}

In this section, we extend the analysis to the most general scenario in which all layers in the composition are replaced by different functions. This captures the situation where each layer $u_i$ is approximated by some other function $v_i$. We derive a cumulative bound that sums the individual perturbations introduced at each layer.

\begin{lemma}[Perturbation of Recursively Compositing Functions, All Layers are Different]
    If the following conditions hold:
    \begin{itemize}
        \item Let $\circ_{i=1}^n u_i  = u_n \circ \cdots \circ u_1$
        \item Let $\circ_{i=1}$
        \item Let $u_0(x) = x$ which is identity mapping
        \item Let $v_{n+1}(x) = x$ which is identity mapping
    \end{itemize}
    Then 
    \begin{align*}
        \| \circ_{i=1}^n u_i - \circ_{i=1}^n v_i \| \leq\sum_{j=1}^{n} \| \circ_{i=j+1}^{n+1} v_i  \circ_{i=1}^j u_i - \circ_{i=j}^n v_i  \circ_{i=0}^{j-1} u_i \|
    \end{align*}
\end{lemma}
\begin{proof}
    We can show
    \begin{align*}
        \| \circ_{i=1}^n u_i - \circ_{i=1}^n v_i \| 
        = & ~ \| \sum_{j=1}^{n} ( \circ_{i=j+1}^{n+1} v_i  \circ_{i=1}^j u_i - \circ_{i=j}^n v_i  \circ_{i=0}^{j-1} u_i ) \| \\
        \leq & ~  
        \sum_{j=1}^{n} \| \circ_{i=j+1}^{n+1} v_i  \circ_{i=1}^j u_i - \circ_{i=j}^n v_i  \circ_{i=0}^{j-1} u_i \|
    \end{align*}
    where the first step follows from adding intermediate terms, and the last step follows from the triangle inequality.
    
Thus, we complete the proof.
\end{proof}

\subsection{The Universality of VAR Transformer}\label{sec:var_universality}

In this section, with the established error bounds for replacing individual or multiple layers with alternative functions, we now prove the main universality result for the VAR Transformer. In essence, we show that a properly constructed VAR Transformer can approximate the target function $f_\mathrm{word2img}$ (from Assumption~\ref{as:function_class}) with arbitrarily small errors under suitable Lipschitz and approximation assumptions on each layer.

\begin{theorem}[Universality of VAR Transformer]\label{thm:var_universality}
    Assume $K_2 > 2$.
    For $f_\mathrm{word2img}$ satisfies Assumption~\ref{as:function_class}, there exists a VAR Transformer $\tau_{\mathsf{VAR}}$ such that
    \begin{align*}
        \|\tau_{\mathsf{VAR}} - f_\mathrm{word2img} \| \leq K_2^n( K_{1,i} \epsilon_{1,i} + \epsilon_{2,i} ).
    \end{align*}
\end{theorem}
\begin{proof}
    We can show that
    \begin{align*}
        \|\tau_{\mathsf{VAR}} - f_\mathrm{word2img} \|  = &~ \| \circ_{i=1}^r (f_i \circ g_i) - \circ_{i=1}^r (\tau_i \circ \Phi_{\mathrm{up},i}) \| \\
        = &~ \sum_{j=1}^n K_2^{n-j} ( K_{1,i} \epsilon_{1,i} + \epsilon_{2,i} )\\
        = &~ \frac{K_2^n - 1}{K_2 - 1}( K_{1,i} \epsilon_{1,i} + \epsilon_{2,i} ) \\
        \leq &~ K_2^n( K_{1,i} \epsilon_{1,i} + \epsilon_{2,i} ) .
    \end{align*}
    where the first step follows from Definition~\ref{def:var_function_class}, the second step follows from Lemma~\ref{lem:one_layer_perturbation}, the third step follows from basic algebra, and the fourth step follows from the basic inequality.  
\end{proof}


\section{Universality of FlowAR}\label{sec:flowar_universality}

In this section, we show that the universality results established for the VAR Transformer can be extended to our FlowAR model. The key observation is that the same local perturbation bounds and Lipschitz assumptions used in the VAR Transformer setting also apply to FlowAR, with only minor changes. Specifically, each FlowAR layer $\Phi_{\mathrm{down}, i}$ can be analyzed in an analogous way to $\Phi_{\mathrm{up}, i}$, allowing us to derive a bound on the overall error of the composed FlowAR model.

\begin{corollary}
    Let $\phi_{{\rm down},i}$ be the down interpolation function of FlowAR (see Definition~\ref{def:down_sample_function}). Let $f_r$ be $r$ $K$-Lipschitz functions from Assumption~\ref{as:function_class}. Let $g_i$ be $r$ $K$-Lipschitz functions from Assumption~\ref{as:function_class}. Let $\tau_i$ be the one-layer transformer defined in Eq. 2.4 from~\cite{hwg+24}. 
    If the following conditions hold: 
    \begin{itemize}
        \item $\|g_i - \Phi_{\mathrm{down},i}\| \leq \epsilon_{1,i}$
        \item $\|f_i - \tau_i\| \leq \epsilon_{2,i}$
        \item $f_i$ is $K_{1,i}$-Lipschitz
        
    \end{itemize}
    Then we have
    \begin{align*}
        \|f_i \circ g_i - \tau_i \circ \Phi_{\mathrm{down},i}\| \leq K_{1,i} \epsilon_{1,i} + \epsilon_{2,i}.
    \end{align*}
\end{corollary}

The proof of this corollary mirrors the two-layer perturbation argument from the VAR Transformer, except each ``up'' interpolation function $\Phi_{\mathrm{up},i}$ is replaced by the corresponding ``down'' interpolation function $\Phi_{\mathrm{down},i}$. The same Lipschitz and approximation assumptions allow us to bound the difference between $f_i \circ g_i$ and $\tau_i \circ \Phi_{\mathrm{down},i}$.

\begin{corollary}\label{corollary:flowar_universality}
    There exists a FlowAR model such that
    \begin{align*}
        \| \tau_{\mathsf{FlowAR}} - f_{\mathrm{word2img}} \| \leq O(\epsilon).
    \end{align*}
\end{corollary}

The proof of Corollary~\ref{corollary:flowar_universality} follows the same high-level structure as our universality results for the VAR Transformer. By applying the local perturbation bound layer by layer and then summing the resulting errors, we obtain a global approximation guarantee that is $O(\epsilon)$. Hence, FlowAR, just like the VAR Transformer, can universally approximate the target function $f_{\mathrm{word2img}}$ under the given Lipschitz and approximation assumptions.
\section{Discussion}\label{sec:discussion}



\subsection{From Interactive Prompting to Interactive Multi-modal Prompting}
The rapid advancements of large pre-trained generative models including large language models and text-to-image generation models, have inspired many HCI researchers to develop interactive tools to support users in crafting appropriate prompts.
% Studies on this topic in last two years' HCI conferences are predominantly focused on helping users refine single-modality textual prompts.
Many previous studies are focused on helping users refine single-modality textual prompts.
However, for many real-world applications concerning data beyond text modality, such as multi-modal AI and embodied intelligence, information from other modalities is essential in constructing sophisticated multi-modal prompts that fully convey users' instruction.
This demand inspires some researchers to develop multimodal prompting interactions to facilitate generation tasks ranging from visual modality image generation~\cite{wang2024promptcharm, promptpaint} to textual modality story generation~\cite{chung2022tale}.
% Some previous studies contributed relevant findings on this topic. 
Specifically, for the image generation task, recent studies have contributed some relevant findings on multi-modal prompting.
For example, PromptCharm~\cite{wang2024promptcharm} discovers the importance of multimodal feedback in refining initial text-based prompting in diffusion models.
However, the multi-modal interactions in PromptCharm are mainly focused on the feedback empowered the inpainting function, instead of supporting initial multimodal sketch-prompt control. 

\begin{figure*}[t]
    \centering
    \includegraphics[width=0.9\textwidth]{src/img/novice_expert.pdf}
    \vspace{-2mm}
    \caption{The comparison between novice and expert participants in painting reveals that experts produce more accurate and fine-grained sketches, resulting in closer alignment with reference images in close-ended tasks. Conversely, in open-ended tasks, expert fine-grained strokes fail to generate precise results due to \tool's lack of control at the thin stroke level.}
    \Description{The comparison between novice and expert participants in painting reveals that experts produce more accurate and fine-grained sketches, resulting in closer alignment with reference images in close-ended tasks. Novice users create rougher sketches with less accuracy in shape. Conversely, in open-ended tasks, expert fine-grained strokes fail to generate precise results due to \tool's lack of control at the thin stroke level, while novice users' broader strokes yield results more aligned with their sketches.}
    \label{fig:novice_expert}
    % \vspace{-3mm}
\end{figure*}


% In particular, in the initial control input, users are unable to explicitly specify multi-modal generation intents.
In another example, PromptPaint~\cite{promptpaint} stresses the importance of paint-medium-like interactions and introduces Prompt stencil functions that allow users to perform fine-grained controls with localized image generation. 
However, insufficient spatial control (\eg, PromptPaint only allows for single-object prompt stencil at a time) and unstable models can still leave some users feeling the uncertainty of AI and a varying degree of ownership of the generated artwork~\cite{promptpaint}.
% As a result, the gap between intuitive multi-modal or paint-medium-like control and the current prompting interface still exists, which requires further research on multi-modal prompting interactions.
From this perspective, our work seeks to further enhance multi-object spatial-semantic prompting control by users' natural sketching.
However, there are still some challenges to be resolved, such as consistent multi-object generation in multiple rounds to increase stability and improved understanding of user sketches.   


% \new{
% From this perspective, our work is a step forward in this direction by allowing multi-object spatial-semantic prompting control by users' natural sketching, which considers the interplay between multiple sketch regions.
% % To further advance the multi-modal prompting experience, there are some aspects we identify to be important.
% % One of the important aspects is enhancing the consistency and stability of multiple rounds of generation to reduce the uncertainty and loss of control on users' part.
% % For this purpose, we need to develop techniques to incorporate consistent generation~\cite{tewel2024training} into multi-modal prompting framework.}
% % Another important aspect is improving generative models' understanding of the implicit user intents \new{implied by the paint-medium-like or sketch-based input (\eg, sketch of two people with their hands slightly overlapping indicates holding hand without needing explicit prompt).
% % This can facilitate more natural control and alleviate users' effort in tuning the textual prompt.
% % In addition, it can increase users' sense of ownership as the generated results can be more aligned with their sketching intents.
% }
% For example, when users draw sketches of two people with their hands slightly overlapping, current region-based models cannot automatically infer users' implicit intention that the two people are holding hands.
% Instead, they still require users to explicitly specify in the prompt such relationship.
% \tool addresses this through sketch-aware prompt recommendation to fill in the necessary semantic information, alleviating users' workload.
% However, some users want the generative AI in the future to be able to directly infer this natural implicit intentions from the sketches without additional prompting since prompt recommendation can still be unstable sometimes.


% \new{
% Besides visual generation, 
% }
% For example, one of the important aspect is referring~\cite{he2024multi}, linking specific text semantics with specific spatial object, which is partly what we do in our sketch-aware prompt recommendation.
% Analogously, in natural communication between humans, text or audio alone often cannot suffice in expressing the speakers' intentions, and speakers often need to refer to an existing spatial object or draw out an illustration of her ideas for better explanation.
% Philosophically, we HCI researchers are mostly concerned about the human-end experience in human-AI communications.
% However, studies on prompting is unique in that we should not just care about the human-end interaction, but also make sure that AI can really get what the human means and produce intention-aligned output.
% Such consideration can drastically impact the design of prompting interactions in human-AI collaboration applications.
% On this note, although studies on multi-modal interactions is a well-established topic in HCI community, it remains a challenging problem what kind of multi-modal information is really effective in helping humans convey their ideas to current and next generation large AI models.




\subsection{Novice Performance vs. Expert Performance}\label{sec:nVe}
In this section we discuss the performance difference between novice and expert regarding experience in painting and prompting.
First, regarding painting skills, some participants with experience (4/12) preferred to draw accurate and fine-grained shapes at the beginning. 
All novice users (5/12) draw rough and less accurate shapes, while some participants with basic painting skills (3/12) also favored sketching rough areas of objects, as exemplified in Figure~\ref{fig:novice_expert}.
The experienced participants using fine-grained strokes (4/12, none of whom were experienced in prompting) achieved higher IoU scores (0.557) in the close-ended task (0.535) when using \tool. 
This is because their sketches were closer in shape and location to the reference, making the single object decomposition result more accurate.
Also, experienced participants are better at arranging spatial location and size of objects than novice participants.
However, some experienced participants (3/12) have mentioned that the fine-grained stroke sometimes makes them frustrated.
As P1's comment for his result in open-ended task: "\emph{It seems it cannot understand thin strokes; even if the shape is accurate, it can only generate content roughly around the area, especially when there is overlapping.}" 
This suggests that while \tool\ provides rough control to produce reasonably fine results from less accurate sketches for novice users, it may disappoint experienced users seeking more precise control through finer strokes. 
As shown in the last column in Figure~\ref{fig:novice_expert}, the dragon hovering in the sky was wrongly turned into a standing large dragon by \tool.

Second, regarding prompting skills, 3 out of 12 participants had one or more years of experience in T2I prompting. These participants used more modifiers than others during both T2I and R2I tasks.
Their performance in the T2I (0.335) and R2I (0.469) tasks showed higher scores than the average T2I (0.314) and R2I (0.418), but there was no performance improvement with \tool\ between their results (0.508) and the overall average score (0.528). 
This indicates that \tool\ can assist novice users in prompting, enabling them to produce satisfactory images similar to those created by users with prompting expertise.



\subsection{Applicability of \tool}
The feedback from user study highlighted several potential applications for our system. 
Three participants (P2, P6, P8) mentioned its possible use in commercial advertising design, emphasizing the importance of controllability for such work. 
They noted that the system's flexibility allows designers to quickly experiment with different settings.
Some participants (N = 3) also mentioned its potential for digital asset creation, particularly for game asset design. 
P7, a game mod developer, found the system highly useful for mod development. 
He explained: "\emph{Mods often require a series of images with a consistent theme and specific spatial requirements. 
For example, in a sacrifice scene, how the objects are arranged is closely tied to the mod's background. It would be difficult for a developer without professional skills, but with this system, it is possible to quickly construct such images}."
A few participants expressed similar thoughts regarding its use in scene construction, such as in film production. 
An interesting suggestion came from participant P4, who proposed its application in crime scene description. 
She pointed out that witnesses are often not skilled artists, and typically describe crime scenes verbally while someone else illustrates their account. 
With this system, witnesses could more easily express what they saw themselves, potentially producing depictions closer to the real events. "\emph{Details like object locations and distances from buildings can be easily conveyed using the system}," she added.

% \subsection{Model Understanding of Users' Implicit Intents}
% In region-sketch-based control of generative models, a significant gap between interaction design and actual implementation is the model's failure in understanding users' naturally expressed intentions.
% For example, when users draw sketches of two people with their hands slightly overlapping, current region-based models cannot automatically infer users' implicit intention that the two people are holding hands.
% Instead, they still require users to explicitly specify in the prompt such relationship.
% \tool addresses this through sketch-aware prompt recommendation to fill in the necessary semantic information, alleviating users' workload.
% However, some users want the generative AI in the future to be able to directly infer this natural implicit intentions from the sketches without additional prompting since prompt recommendation can still be unstable sometimes.
% This problem reflects a more general dilemma, which ubiquitously exists in all forms of conditioned control for generative models such as canny or scribble control.
% This is because all the control models are trained on pairs of explicit control signal and target image, which is lacking further interpretation or customization of the user intentions behind the seemingly straightforward input.
% For another example, the generative models cannot understand what abstraction level the user has in mind for her personal scribbles.
% Such problems leave more challenges to be addressed by future human-AI co-creation research.
% One possible direction is fine-tuning the conditioned models on individual user's conditioned control data to provide more customized interpretation. 

% \subsection{Balance between recommendation and autonomy}
% AIGC tools are a typical example of 
\subsection{Progressive Sketching}
Currently \tool is mainly aimed at novice users who are only capable of creating very rough sketches by themselves.
However, more accomplished painters or even professional artists typically have a coarse-to-fine creative process. 
Such a process is most evident in painting styles like traditional oil painting or digital impasto painting, where artists first quickly lay down large color patches to outline the most primitive proportion and structure of visual elements.
After that, the artists will progressively add layers of finer color strokes to the canvas to gradually refine the painting to an exquisite piece of artwork.
One participant in our user study (P1) , as a professional painter, has mentioned a similar point "\emph{
I think it is useful for laying out the big picture, give some inspirations for the initial drawing stage}."
Therefore, rough sketch also plays a part in the professional artists' creation process, yet it is more challenging to integrate AI into this more complex coarse-to-fine procedure.
Particularly, artists would like to preserve some of their finer strokes in later progression, not just the shape of the initial sketch.
In addition, instead of requiring the tool to generate a finished piece of artwork, some artists may prefer a model that can generate another more accurate sketch based on the initial one, and leave the final coloring and refining to the artists themselves.
To accommodate these diverse progressive sketching requirements, a more advanced sketch-based AI-assisted creation tool should be developed that can seamlessly enable artist intervention at any stage of the sketch and maximally preserve their creative intents to the finest level. 

\subsection{Ethical Issues}
Intellectual property and unethical misuse are two potential ethical concerns of AI-assisted creative tools, particularly those targeting novice users.
In terms of intellectual property, \tool hands over to novice users more control, giving them a higher sense of ownership of the creation.
However, the question still remains: how much contribution from the user's part constitutes full authorship of the artwork?
As \tool still relies on backbone generative models which may be trained on uncopyrighted data largely responsible for turning the sketch into finished artwork, we should design some mechanisms to circumvent this risk.
For example, we can allow artists to upload backbone models trained on their own artworks to integrate with our sketch control.
Regarding unethical misuse, \tool makes fine-grained spatial control more accessible to novice users, who may maliciously generate inappropriate content such as more realistic deepfake with specific postures they want or other explicit content.
To address this issue, we plan to incorporate a more sophisticated filtering mechanism that can detect and screen unethical content with more complex spatial-semantic conditions. 
% In the future, we plan to enable artists to upload their own style model

% \subsection{From interactive prompting to interactive spatial prompting}


\subsection{Limitations and Future work}

    \textbf{User Study Design}. Our open-ended task assesses the usability of \tool's system features in general use cases. To further examine aspects such as creativity and controllability across different methods, the open-ended task could be improved by incorporating baselines to provide more insightful comparative analysis. 
    Besides, in close-ended tasks, while the fixing order of tool usage prevents prior knowledge leakage, it might introduce learning effects. In our study, we include practice sessions for the three systems before the formal task to mitigate these effects. In the future, utilizing parallel tests (\textit{e.g.} different content with the same difficulty) or adding a control group could further reduce the learning effects.

    \textbf{Failure Cases}. There are certain failure cases with \tool that can limit its usability. 
    Firstly, when there are three or more objects with similar semantics, objects may still be missing despite prompt recommendations. 
    Secondly, if an object's stroke is thin, \tool may incorrectly interpret it as a full area, as demonstrated in the expert results of the open-ended task in Figure~\ref{fig:novice_expert}. 
    Finally, sometimes inclusion relationships (\textit{e.g.} inside) between objects cannot be generated correctly, partially due to biases in the base model that lack training samples with such relationship. 

    \textbf{More support for single object adjustment}.
    Participants (N=4) suggested that additional control features should be introduced, beyond just adjusting size and location. They noted that when objects overlap, they cannot freely control which object appears on top or which should be covered, and overlapping areas are currently not allowed.
    They proposed adding features such as layer control and depth control within the single-object mask manipulation. Currently, the system assigns layers based on color order, but future versions should allow users to adjust the layer of each object freely, while considering weighted prompts for overlapping areas.

    \textbf{More customized generation ability}.
    Our current system is built around a single model $ColorfulXL-Lightning$, which limits its ability to fully support the diverse creative needs of users. Feedback from participants has indicated a strong desire for more flexibility in style and personalization, such as integrating fine-tuned models that cater to specific artistic styles or individual preferences. 
    This limitation restricts the ability to adapt to varied creative intents across different users and contexts.
    In future iterations, we plan to address this by embedding a model selection feature, allowing users to choose from a variety of pre-trained or custom fine-tuned models that better align with their stylistic preferences. 
    
    \textbf{Integrate other model functions}.
    Our current system is compatible with many existing tools, such as Promptist~\cite{hao2024optimizing} and Magic Prompt, allowing users to iteratively generate prompts for single objects. However, the integration of these functions is somewhat limited in scope, and users may benefit from a broader range of interactive options, especially for more complex generation tasks. Additionally, for multimodal large models, users can currently explore using affordable or open-source models like Qwen2-VL~\cite{qwen} and InternVL2-Llama3~\cite{llama}, which have demonstrated solid inference performance in our tests. While GPT-4o remains a leading choice, alternative models also offer competitive results.
    Moving forward, we aim to integrate more multimodal large models into the system, giving users the flexibility to choose the models that best fit their needs. 
    


\section{Conclusion}\label{sec:conclusion}
In this paper, we present \tool, an interactive system designed to help novice users create high-quality, fine-grained images that align with their intentions based on rough sketches. 
The system first refines the user's initial prompt into a complete and coherent one that matches the rough sketch, ensuring the generated results are both stable, coherent and high quality.
To further support users in achieving fine-grained alignment between the generated image and their creative intent without requiring professional skills, we introduce a decompose-and-recompose strategy. 
This allows users to select desired, refined object shapes for individual decomposed objects and then recombine them, providing flexible mask manipulation for precise spatial control.
The framework operates through a coarse-to-fine process, enabling iterative and fine-grained control that is not possible with traditional end-to-end generation methods. 
Our user study demonstrates that \tool offers novice users enhanced flexibility in control and fine-grained alignment between their intentions and the generated images.



\ifdefined\isarxiv

\else
\bibliography{ref}
\bibliographystyle{icml2025}

\fi



\newpage
\onecolumn
\appendix
\begin{center}
	\textbf{\LARGE Appendix }
\end{center}





%%%% Cut-line between first 10 pages and appendix

{\bf Roadmap} In Section~\ref{sec:app_prelim}, we provide basic algebras that support our proofs. 
In Section~\ref{sec:app_var}, we provide other phases of the VAR Model. 
In Section~\ref{sec:training_of_flowar}, we give the definitions for training the FlowAR Model. 
In Section~\ref{sec:inference_of_flowar}, we give the definitions for the inference of the FlowAR Model.
In Section~\ref{sec:more_work}, we introduce more related work.

\section{Preliminary}\label{sec:app_prelim}

In this section, we introduce notations and basic facts that are used in our work. We first list some basic facts of matrix norm properties.

\subsection{Notations}
We denote the $\ell_p$ norm of a vector $x$ by $\| x \|_p$, i.e., $\|x\|_1 := \sum_{i=1}^n |x_i|$, $\| x \|_2 := (\sum_{i=1}^n x_i^2)^{1/2}$ and $\| x \|_{\infty} := \max_{i \in [n]} |x_i|$. For a vector $x \in \R^n$, $\exp(x) \in \R^n$ denotes a vector where $\exp(x)_i$ is $\exp(x_i)$ for all $i \in [n]$. For $n > k$, for any matrix $A \in \R^{n\times k}$, we denote the spectral norm of $A$ by $\| A \|$, i.e., $\| A \| := \sup_{x\in \R^k} \| Ax \|_2 / \| x \|_2$. We define the function norm as $\| f \|_\alpha :=  (\int \| f(X) \|_\alpha^\alpha \d X)^{1/\alpha}$ where $f$ is a function. We use $\sigma_{\min}(A)$ to denote the minimum singular value of $A$. Given two vectors $x, y \in \R^n$, we use $\langle x, y \rangle$ to denote $\sum_{i=1}^n x_iy_i$. Given two vectors $x, y \in \R^n$, we use $x \circ y$ to denote a vector that its $i$-th entry is $x_i y_i$ for all $i \in [n]$. We use $e_i \in \R^n$ to denote a vector where $i$-th entry is $1$, and all other entries are $0$. Let $x \in \R^n$ be a vector. We define $\diag (x) \in \R^{n \times n}$ as the diagonal matrix whose diagonal entries are given by $\diag(x)_{i, i} = x_i$ for $i = 1, \dots, n$, and all off-diagonal entries are zero. For a symmetric matrix $A \in \R^{n\times n}$, we say $A \succ 0$ (positive definite (PD)), if for all $x\in \R^n \setminus \{ {\bf 0}_n \}$, we have $x^\top A x > 0$. For a symmetric matrix $A \in \R^{n \times n}$, we say $A \succeq 0$ (positive semidefinite (PSD)), if for all $x \in \R^n$, we have $x^\top A x \geq 0$. The Taylor Series for $\exp(x)$ is $\exp(x) = \sum_{i=0}^{\infty} \frac{x^i}{i!}$. For a matrix $X \in \R^{n_1 n_2 \times d}$, we use $\X \in \R^{n_1 \times n_2 \times d}$ to denote its tensorization, and we only assume this for letters $X$ and $Y$.

\subsection{Basic Algebra}
In this section, we introduce the basic algebras used in our work.

\begin{fact}
Let $A$ denote the matrix. For each $i$, we use $A_{i,*}$ to denote the $i$-th row of $A$. For $j$, we use $A_{*,j}$ to denote the $j$-th column of $A$. 
We can show that
\begin{itemize}
    \item $\| A \| \leq \| A \|_F$
    \item $\| A \| \geq \| A_{i,*} \|_2$
    \item $\| A \| \geq \| A_{*,j} \|_2$
\end{itemize}
\end{fact}

Then, we introduce some useful inner product properties.

\begin{fact}
    For vectors $u,v,w \in \R^n$. We have 
    \begin{itemize}
        \item $\langle u, v \rangle = \langle u \circ v, {\bf 1}_n \rangle$
        \item $\langle u \circ v, w \rangle = \langle u \circ v \circ w, {\bf 1}_n \rangle$ 
        \item $\langle u, v \rangle = \langle v, u \rangle$
        \item $\langle u , v \rangle = u^\top v = v^\top u$
    \end{itemize}
\end{fact}

Now, we show more vector properties related to the hadamard products, inner products, and diagnoal matrices.
\begin{fact}
    For any vectors $u, v, w \in \R^n$, we have
    \begin{itemize}
        \item $u \circ v = v \circ u = \diag(u) \cdot v = \diag(v) \cdot u$
        \item $u^\top (v\circ w) = u^\top \diag(v) w$
        \item $u^\top (v\circ w) = v^\top (u \circ w) = w^\top (u \circ v)$
        \item $u^\top \diag(v) w = v^\top \diag(u) w = u^\top \diag(w) v$
        \item $\diag(u) \cdot \diag(v) \cdot {\bf 1}_n = \diag(u) v$
        \item $\diag(u \circ v) = \diag(u) \diag(v)$
        \item $\diag(u) + \diag(v) = \diag(u+v)$
    \end{itemize}
\end{fact}




\section{VAR Transformer Blocks}\label{sec:app_var}

In this section, we define the components in the VAR Transformer.

We first introduce the Softmax unit.

\begin{definition}[Softmax] \label{def:softmax}
    Let $z \in \R^{n}$. We define 
    $\mathsf{Softmax}: \R^{n} \to \R^{n}$ satisfying 
    \begin{align*}
        \mathsf{Softmax}(z):= \exp(z) / \langle \exp(z) , {\bf 1}_n \rangle.  
    \end{align*}
\end{definition}

Here, we define the attention matrix in the VAR Transformer as follows.

\begin{definition}[Attention Matrix]\label{def:attn_matrix}
    Let $W_Q, W_K \in \R^{d \times d}$ denote the model weights. Let $X \in \R^{n \times d}$ denote the representation of the length-$n$ input. Then, we define the attention matrix $A \in \R^{n \times n}$ by, For $i,j \in [n]$, 
    \begin{align*}
        A_{i,j} := & ~\exp( \underbrace{ X_{i,*} }_{1 \times d} \underbrace{ W_Q }_{d \times d} \underbrace{ W_K^\top }_{d \times d} \underbrace{ X_{j,*}^\top }_{d \times 1}).
    \end{align*}
\end{definition}

With the attention matrix, we now provide the definition for a single layer of Attention.

\begin{definition}[Single Attention Layer]\label{def:single_layer_transformer}
     Let $X \in \R^{n \times d}$ denote the representation of the length-$n$ sentence. Let $W_V \in \R^{d \times d}$ denote the model weights. As in the usual attention mechanism, the final goal is to output an $n \times d$ size matrix where $D:= \diag( A {\bf 1}_n) \in \R^{n \times n}$. Then, we define attention layer $\mathsf{Attn}$ as
    \begin{align*}
        \mathsf{Attn} (X) := & ~ D^{-1} A X W_V .
    \end{align*}
\end{definition}

Here we present the definition of the VAR Attention.

\begin{definition}[$\VAR$ Attention Layer]\label{def:single_layer_var_transformer}
    Let $r \geq 1$ be a positive integer. Let $h_r, w_r$ be two positive integers. 
     Let $\X \in \R^{h_r \times w_r \times d}$ denote the representation of the input token map. Let $W_V \in \R^{d \times d}$ denote the model weights. As in the usual attention mechanism, the final goal is to output an $n \times d$ size matrix where $D:= \diag( A {\bf 1}_n) \in \R^{h_rw_r \times h_rw_r}$. Then, we define attention layer $\mathsf{Attn}_r: \R^{h_rw_r \times d} \to \R^{h_rw_r \times d}$ as
    \begin{align*}
        \mathsf{Attn}_r (X) := & ~ D^{-1} A X W_V .
    \end{align*}
\end{definition}

We introduce the feed-forward layer in the VAR Transformer as follows.

\begin{definition}[Single Feed-Forward Layer]\label{def:ffn}
We define the FFN as follows:
    \begin{itemize}
        \item $X \in \R^{d\times L}$
        \item $k \in [n]$
        \item $c$ is the number of neurons
        \item $W^{(1)} \in \R^{c \times d}, W^{(2)} \in \R^{d \times c}$ are weight matrices
        \item $b^{(1)}\in \R^{ c}$, $ b^{(2)} \in \R^{d}$ are bias vectors.
        \item $\mathsf{FFN}: \R^{d\times L} \to \R^{d\times L}$ 
    \end{itemize}
    
    \begin{align*}
        \mathsf{FFN}(X)_{*,k} =
        & ~ \underbrace{X_{*,k}}_{d \times 1} +  \underbrace{W^{(2)}}_{d \times c} \mathsf{ReLU}( \underbrace{ W^{(1)} X_{*,k} }_{c\times 1} + \underbrace{b^{(1)}}_{c\times 1})  + \underbrace{b^{(2)}}_{d \times 1} 
    \end{align*}
    
\end{definition}

\subsection{Phase 2: Feature Map Reconstruction }\label{sec:phase_2}

In this section, we introduce the Phase Two of the VAR model.

\begin{definition}[Convolution Layer, Definition 3.9 from~\cite{kll+25} on Page 9]\label{def:conv_layer}
    The Convolution Layer is defined as follows:
    \begin{itemize}
        \item Let $h \in \mathbb{N}$ denote the height of the input and output feature map.
        \item Let $w \in \mathbb{N}$ denote the width of the input and output feature map.
        \item Let $c_{\rm in} \in \mathbb{N}$ denote the number of channels of the input feature map.
        \item Let $c_{\rm out} \in \mathbb{N}$ denote the number of channels of the output feature map.
        \item Let $X \in \R^{h \times w \times c_{\rm in}}$ denote the input feature map.
        
        \item For $l \in [c_{\rm out}]$, we use $K^l \in \R^{3 \times 3 \times c_{\rm in}}$ to denote the $l$-th convolution kernel.
        \item Let $p = 1$ denote the padding of the convolution layer.
        \item Let $s = 1$ denote the stride of the convolution kernel.
        \item Let $Y \in \R^{h \times w \times c_{\rm out}}$ denote the output feature map.
    \end{itemize}
    We use $\phi_{\rm conv}: \R^{h \times w \times c_{\rm in}} \to \R^{h \times w \times c_{\rm out}}$ to denote the convolution operation then we have $Y = \phi_{\rm conv}(X)$. Specifically, for $i \in [h], j \in [w], l \in [c_{\rm out}]$, we have
    \begin{align*}
        Y_{i,j,l} := \sum_{m=1}^3 \sum_{n=1}^3 \sum_{c = 1}^{c_{\rm in}} X_{i+m-1,j+n-1,c} \cdot K^l_{m,n,c} + b
    \end{align*}
\end{definition}

\begin{remark}
    Assumptions of kernel size, padding of the convolution layer, and stride of the convolution kernel are based on the specific implementation of \cite{tjy+24}.
\end{remark}

\subsection{Phase 3: VQ-VAE Decoder process}\label{sec:phase_3} 

In this section, we introduce Phase Three of the VAR model.

VAR will use the VQ-VAE Decoder Module to reconstruct the feature map generated in Section~\ref{sec:phase_2} into a new image. The Decoder of VQ-VAE has the following main modules \cite{kll+25}: (1) Resnet Blocks; (2) Attention Blocks; (3) Up Sample Blocks. We recommend readers to \cite{kll+25} for more details. 



\section{Training of FlowAR}\label{sec:training_of_flowar}

In this section, we introduce the training of FlowAR along with its definitions based on~\cite{ryh+24}.


We first introduce some notations used in FlowAR.
\subsection{Notations}\label{sub:notations}
For a matrix $X \in \R^{n_1 n_2 \times d}$, we use $\X \in \R^{n_1 \times n_2 \times d}$ to denote its tensorization, and we only assume this for letters $X, Y, Z, F, V$.

\subsection{Sample Function}
In this section, we introduce the sample functions used in FlowAR.

We first introduce the up sample function.

\begin{definition}[Up Sample Function]\label{def:up_sample_function}
    If the following conditions hold:
    \begin{itemize}
        \item Let $h, w \in \mathbb{N}$ denote the height and weight of latent $\X \in \R^{h \times w \times c}$.
        \item Let $r > 0$ denote a positive integer.
    \end{itemize}
    Then we define $\mathrm{Up}(\X,r) \in \R^{rh \times rw \times c}$ as the upsampling of latent $\X$ by a factor $r$.
\end{definition}

Then, we introduce the down sample function.
\begin{definition}[Down Sample Function]\label{def:down_sample_function}
    If the following conditions hold:
    \begin{itemize}
        \item  Let $h, w \in \mathbb{N}$ denote the height and weight of latent $\X  \in \R^{h \times w \times c}$.
        \item Let $r > 0$ denote a positive integer.
    \end{itemize}
    Then we define $\mathrm{Down}(\X ,r) \in \R^{\frac{h}{r} \times \frac{w}{r} \times c}$ as the downsampling of latent $\X $ by a factor $r$.
\end{definition}

\subsection{Linear Sample Function}
In this section, we present the linear sample function in FlowAR.

We first present the linear up sample function.
\begin{definition}[Linear Up Sample Function]\label{def:linear_up_sample_function}
    If the following conditions hold:
    \begin{itemize}
        \item Let $h, w \in \mathbb{N}$ denote the height and weight of latent $\X \in \R^{h \times w \times c}$. 
        \item Let $r > 0$ denote a positive integer.
        \item Let $\Phi_{\mathrm{up}} \in \R^{hw \times (rh \cdot rw)}$ be a matrix. 
    \end{itemize}
    Then we define the linear up sample function $\phi_{\mathrm{up}}(\cdot, \cdot)$ as it computes $\Y := \phi_{\mathrm{up}}(\X,r) \in \R^{rh \times rw \times c}$ such that the matrix version of $\X$ and $\Y$ satisfies
    \begin{align*}
        Y = \Phi_{\mathrm{up}}X \in \R^{(rh\cdot rw) \times c}.
    \end{align*}
\end{definition}

Then, we define linear down sample function as follows.
\begin{definition}[Linear Down Sample Function]\label{def:linear_down_sample_function}
    If the following conditions hold:
    \begin{itemize}
        \item Let $h, w \in \mathbb{N}$ denote the height and weight of latent $\X \in \R^{h \times w \times c}$.
        \item Let $r > 0$ denote a positive integer.
        \item Let $\Phi_{\mathrm{down}} \in \R^{((h/r) \cdot (w/r)) \times hw}$ be a matrix. 
    \end{itemize}
    Then we define the linear down sample function $\phi_{\mathrm{down}}(\X,r)$ as it computes $\Y := \phi_{\mathrm{down}}(\X,r) \in \R^{(h/r) \times (w/r) \times c}$ such that the matrix version of $\X$ and $\Y$ satisfies
    \begin{align*}
        Y = \Phi_{\mathrm{down}}X \in \R^{((h/r) \cdot (w/r)) \times c}.
    \end{align*}
\end{definition}


\subsection{VAE Tokenizer}\label{sub:vae_tokenizer}
In this section, we show the VAE Tokenizer.

\begin{definition}[VAE Tokenizer]\label{def:vae_tokenizer}
If the following conditions hold:
\begin{itemize}
    \item Let $\X \in \R^{h \times w \times c}$ denote a continuous latent representation generated by VAE.
    \item Let $K$ denote the total number of scales in FlowAR.
    \item Let $a$ be a positive integer.
    \item For $i \in [K]$, let $r_i := a^{K-i}$.
    \item For each $i\in [K]$, let $\phi_{\mathrm{down}, i}(\cdot , r_i) : \R^{h \times w \times c} \to \R^{(h / r_i) \times (w/r_i) \times c}$ denote the linear down sample function defined in Definition~\ref{def:linear_down_sample_function}.
\end{itemize}
For $i \in [K]$, we define the $i$-th token map generated by VAE Tokenizer be
\begin{align*}
    \Y^{i} := \phi_{\mathrm{down}, i}(\X, r_i) \in \R^{(h / r_i) \times (w/r_i) \times c},
\end{align*}
We define the output of VAE Tokenizer as follows:
    \begin{align*}
        \mathsf{Tokenizer}(\mathsf{X}) := \{\Y^{1},\Y^{2}, \dots,\Y^{K}\}.
    \end{align*}
\end{definition}
\begin{remark}
    In \cite{ryh+24}, they choose $a =2$ and hence for $i \in [n]$, $r_i := 2^{K-i}$.
\end{remark}

\subsection{Autoregressive Transformer}



Firstly, we give the definition of a single attention layer.
\begin{definition}[Single Attention Layer]\label{def:attn_layer}
    If the following conditions hold:
    \begin{itemize}
        \item Let $h,w \in \mathbb{N}$ denote the height and weight of latent $\X \in \R^{h \times w \times c}$.
        \item Let $W_Q, W_K, W_V \in \R^{c \times c}$ denote the weight matrix for query, key, and value, respectively.
    \end{itemize}
    Then we define the attention layer $\mathsf{Attn}(\cdot)$ as it computes  $\Y = \mathsf{Attn}(\X) \in \R^{h \times w \times c}$. For the matrix version, we first need to compute the attention matrix $A \in \R^{hw \times hw}$:
    \begin{align*}
        A_{i,j} := & ~\exp(  X_{i,*}   W_Q   W_K^\top   X_{j,*}^\top), \text{~~for~} i, j \in [hw].
    \end{align*}
    Then, we compute the output:
    \begin{align*}
        Y := D^{-1}AXW_V \in \R^{hw \times c}.
    \end{align*}
    where $D:=\diag(A {\bf 1}_n) \in \R^{hw \times hw}$.
\end{definition}


To move on, we present the definition of multilayer perceptron.
\begin{definition}[MLP layer]\label{def:mlp}
    If the following conditions hold:
    \begin{itemize}
        \item Let $h,w \in \mathbb{N}$ denote the height and weight of latent $\X \in \R^{h \times w \times c}$.
        \item Let $c$ denote the input dimension of latent $\X \in \R^{h \times w \times c}$.
        \item Let $d$ denote the dimension of the target output.
        \item Let $W \in \R^{c \times d}$ denote a weight matrix.
        \item Let $b \in \R^{1 \times d}$ denote a bias vector.
    \end{itemize}
    Then we define mlp layer as it computes $\Y := \mathsf{MLP}(X,c,d) \in \R^{h \times w \times d}$ such that the matrix version of $\X$ and $\Y$ astisfies, for each $j \in [hw]$,
    \begin{align*}
        Y_{j,:} = \underbrace{X_{j,:}}_{1\times c} \cdot \underbrace{W}_{c \times d} + \underbrace{b}_{1 \times d}
    \end{align*}
\end{definition}



We present the definition of layer-wise norm layer.
\begin{definition}[Layer-wise norm layer]\label{def:ln}
    Given a latent $\X \in \R^{h \times w \times c}$. We define the layer-wise as it computes $\Y := \mathsf{LN}(X) \in \R^{h\times w \times c}$ such that the matrix version of $\X$ and $\Y$ satisfies, for each $j \in [hw]$,
    \begin{align*}
        Y_{j,:} =  \frac{X_{j,:}-\mu_j}{\sqrt{\sigma_j^2}}
    \end{align*}
    where $\mu_j := \sum_{k=1}^c X_{j,k}/c$ and $\sigma_{j}^2 = \sum_{k=1}^c(X_{j,k}-\mu_j)^2/c$.
\end{definition}

\begin{definition}[Autoregressive Transformer]\label{def:ar_transformer}
    If the following conditions hold:
    \begin{itemize}
        \item Let $\X \in \R^{h \times w \times c}$ denote a continuous latent representation generated by VAE.
        \item Let $K$ denote the total number of scales in FlowAR.
        \item For $i \in [K]$, let $\Y_i \in \R^{(h / r_i) \times (w/r_i) \times c}$ be the $i$-th token map genereated by VAE Tokenizer defined in Definition~\ref{def:vae_tokenizer}.
        \item Let $a$ be a positive integer.
        \item For $i \in [K]$, let $r_i := a^{K-i}$.
        \item For $i \in [K-1]$, let $\phi_{\mathrm{up}, i}(\cdot, a):\R^{(h/r_i)\times(w/r_i)\times c} \to \R^{(h/r_{i+1})\times(w/r_{i+1})\times c}$ be the linear up sample function defined in Definition~\ref{def:linear_up_sample_function}.
        \item For $i \in [K]$, let $\mathsf{Attn}_i(\cdot):\R^{(\sum_{j=1}^i h/r_j)(\sum_{j=1}^i w/r_j) \times c} \to \R^{(\sum_{j=1}^i h/r_j)(\sum_{j=1}^i w/r_j) \times c}$ be the $i$-th attention layer defined in Definition~\ref{def:attn_layer}.
        \item For $i \in [K]$, let $\mathsf{FFN}_i(\cdot):\R^{(\sum_{j=1}^i h/r_j)(\sum_{j=1}^i w/r_j) \times c} \to \R^{(\sum_{j=1}^i h/r_j)(\sum_{j=1}^i w/r_j) \times c}$ be the $i$-th feed forward network defined in Definition~\ref{def:ffn}.
        \item Let $\Z_{\mathrm{init}} \in \R^{(h/r_1) \times (w/r_1) \times c}$ be the initial input denoting the class condition.
        \item Let $\Z^1 : = \mathsf{Z}_{\mathrm{init}} \in \R^{(h/r_1) \times (w/r_1) \times c}$.
        \item For $i \in [K] \setminus \{1\}$, Let $\Z^{i}$ be the reshape of the input sequence $\mathsf{Z}_{\mathrm{init}}, \phi_{\mathrm{up}, 1}(\Y^1, a), \ldots, \phi_{\mathrm{up}, i}(\Y^{i-1}, a)$ into the tensor of size $(\sum_{j=1}^i h /r_{j}) \times (\sum_{j=1}^i w /r_{j}) \times c$.
    \end{itemize}
    For $i \in [K]$, we define the Autoregressive transformer $\mathsf{TF}_i$ as 
    \begin{align*}
        \mathsf{TF}_i(\Z^i) = \mathsf{FFN}_i \circ \mathsf{Attn}_i (\Z^i) \in \R^{(\sum_{j=1}^{i} h /r_{j})(\sum_{j=1}^{i} w /r_{j}) \times c}.
    \end{align*}
    We denote $\wh{\Y}^i$ as the $i$-th block of size $(h/r_{i}) \times (w/r_{i}) \times c$ of the tensorization of $\mathsf{TF}_i(Z^i)$.
\end{definition}


\subsection{Flow Matching}
In this section, we introduce the flow matching definition.

\begin{definition}[Flow]\label{def:flow}
    If the following conditions hold:
    \begin{itemize}
        \item Let $\X \in \R^{h \times w \times c}$ denote a continuous latent representation generated by VAE.
        \item Let $K$ denote the total number of scales in FlowAR.
        \item For $i \in [K]$, let $\Y_i \in \R^{(h / r_i) \times (w/r_i) \times c}$ be the $i$-th token map genereated by VAE Tokenizer defined in Definition~\ref{def:vae_tokenizer}.
        \item For $i \in [K]$, let $\F^i_0 \in \R^{(h / r_i) \times (w/r_i) \times c}$ be a matrix where each entry is sampled from the standard Gaussian $\N(0,1)$.
    \end{itemize}
    We defined the interpolated input as follows:
    \begin{align*}
        \F^i_{t} := t \Y^i + (1-t)\F^i_0.
    \end{align*}
    The velocity flow is defined as
    \begin{align*}
        \V^i_t := \frac{\d \F^i_{t}}{\d t} = \Y^i -\F^i_0.
    \end{align*}
\end{definition}




Here, we define the architecture of the flow matching model defined in \cite{ryh+24}.
\begin{definition}[Flow Matching Architecture]\label{def:flow_matching_architecture}
    If the following conditions hold:
    \begin{itemize}

        \item Let $\X \in \R^{h \times w \times c}$ denote a continuous latent representation generated by VAE.
        \item Let $K$ denote the total number of scales in FlowAR.
        \item For $i \in [K]$, let $\Y_i \in \R^{(h / r_i) \times (w/r_i) \times c}$ be the $i$-th token map genereated by VAE Tokenizer defined in Definition~\ref{def:vae_tokenizer}.
        \item Let $i \in [K]$.
        \item Let $\wh{\Y}_i \in \R^{(h / r_i) \times (w/r_i) \times c}$ be the $i$-th block of the output of Autoregressive Transformer defined in Definition~\ref{def:ar_transformer}.
        \item Let $\F^i_t$ be the interpolated input defined in Definition~\ref{def:flow}.
        \item Let $\mathsf{Attn}_i(\cdot):\R^{(h / r_i) \times (w/r_i) \times c} \to \R^{(h / r_i) \times (w/r_i) \times c}$ be the $i$-th attention layer defined in Definition~\ref{def:attn_layer}.
        \item Let $\mathsf{MLP}_i(\cdot,c,d):\R^{(h / r_i) \times (w/r_i) \times c}  \to \R^{(h / r_i) \times (w/r_i) \times d}$ be the $i$-th attention layer defined in Definition~\ref{def:mlp}.
        \item Let $\mathsf{LN}_i(\cdot): \R^{(h / r_i) \times (w/r_i) \times c}  \to \R^{(h / r_i) \times (w/r_i) \times c}$ be the $i$-th layer-wise norm layer defined in Definition~\ref{def:ln}.
        \item Let $t_i \in [0,1]$ denote a time step.
    \end{itemize}
    Then we define the $i$-th flow matching model as $\mathsf{NN}_i(\F_t^i,\wh{\Y}_i,t_i): \R^{(h / r_i) \times (w/r_i) \times c} \times \R^{(h / r_i) \times (w/r_i) \times c} \times \R \to \R^{(h / r_i) \times (w/r_i) \times c}$. The tensor input needs to go through the following computational steps:
    \begin{itemize}
        \item {\bf Step 1:} Compute intermediate variables $\alpha_1, \alpha_2, \beta_1, \beta_2, \gamma_1, \gamma_2$. Specifically, we have
        \begin{align*}
            \alpha_1, \alpha_2, \beta_1, \beta_2, \gamma_1, \gamma_2 :=&~ \mathsf{MLP}_i(\wh{\Y}_i + t_i \cdot {\bf 1}_{(h / r_i) \times (w/r_i) \times c},c,6c)
        \end{align*}
        \item {\bf Step 2:} Compute intermediate variable $\wh{F_t}^{i'}$. Specifically, we have
        \begin{align*}
            \wh{\F_t}^{i'}:= \mathsf{Attn}_i (\gamma_1 \circ \mathsf{LN}(\F_t^i) + \beta_1) \circ \alpha_1
        \end{align*}
        where $\circ$ denotes the element-wise product for tensors.

        \item {\bf Step 3:} Compute final output $\F_t^{i''}$. Specifically, we have
        \begin{align*}
            \F_t^{i''} = \mathsf{MLP}_i(\gamma_2 \circ \mathsf{LN}(\wh{F_t}^{i'})+ \beta_2,c,c) \circ \alpha_2
        \end{align*}
        where $\circ$ denotes the element-wise product for tensors.
    \end{itemize}
\end{definition}



Then, we present our training objective.
\begin{definition}[Loss of FlowAR]
If the following conditions hold:
    \begin{itemize}
        \item Let $\X \in \R^{h \times w \times c}$ denote a continuous latent representation generated by VAE.
        \item Let $K$ denote the total number of scales in FlowAR.
        \item For $i \in [K]$, let $\Y_i \in \R^{(h / r_i) \times (w/r_i) \times c}$ be the $i$-th token map genereated by VAE Tokenizer defined in Definition~\ref{def:vae_tokenizer}.
        \item For $i \in [K]$, let $\wh{\Y}_i \in \R^{(h / r_i) \times (w/r_i) \times c}$ be the $i$-th block of the output of Autoregressive Transformer defined in Definition~\ref{def:ar_transformer}.
        \item For $i \in [K]$, let $\F^i_t$ be the interpolated input defined in Definition~\ref{def:flow}.
        \item For $i \in [K]$, let $\V^i_t$ be the velocity flow defined in Definition~\ref{def:flow}.
        \item For $i \in [K]$, let $\mathsf{NN}_i(\cdot,\cdot,\cdot):\R^{(h / r_i) \times (w/r_i) \times c} \times \R^{(h / r_i) \times (w/r_i) \times c} \times \R \to \R^{(h / r_i) \times (w/r_i) \times c}$ denote the $i$-th flow matching network defined in Definition~\ref{def:flow_matching_architecture}.
    \end{itemize}
    The loss function of FlowAR is 
    \begin{align*}
        L(\theta) = \sum_{i=1}^n \E_{t \sim \mathsf{Unif}[0,1]}\|\mathsf{NN}_i(\F^i_t, \wh{\Y}^{i}_t, t_i) - \V^i_t\|^2.
    \end{align*}
\end{definition}

\section{Inference of FlowAR}\label{sec:inference_of_flowar}
We define the architecture of FlowAR during the inference process as follows.
\begin{definition}[FlowAR Architecture in the Inference Pipeline]\label{def:flow_architecture_inference}
    If the following conditions hold:
    \begin{itemize}
        \item Let $K$ denote the total number of scales in FlowAR.
        \item Let $a$ be a positive integer.
        \item For $i \in [K]$, let $r_i := a^{K-i}$.
        \item For $i \in [K-1]$, let $\phi_{\mathrm{up}, i}(\cdot, a):\R^{(h/r_i)\times(w/r_i)\times c} \to \R^{(h/r_{i+1})\times(w/r_{i+1})\times c}$ be the linear up sample function defined in Definition~\ref{def:linear_up_sample_function}.
        \item For $i \in [K]$, let $\mathsf{Attn}_i(\cdot):\R^{(\sum_{j=1}^i h/r_j)(\sum_{j=1}^i w/r_j) \times c} \to \R^{(\sum_{j=1}^i h/r_j)(\sum_{j=1}^i w/r_j) \times c}$ be the $i$-th attention layer defined in Definition~\ref{def:attn_layer}.
        \item For $i \in [K]$, let $\mathsf{FFN}_i(\cdot):\R^{(\sum_{j=1}^i h/r_j)(\sum_{j=1}^i w/r_j) \times c} \to \R^{(\sum_{j=1}^i h/r_j)(\sum_{j=1}^i w/r_j) \times c}$ be the $i$-th feed forward network defined in Definition~\ref{def:ffn}.
        \item For $i \in [K]$, let $\mathsf{NN}_i(\cdot,\cdot,\cdot):\R^{(h / r_i) \times (w/r_i) \times c} \times \R^{(h / r_i) \times (w/r_i) \times c} \times \R \to \R^{(h / r_i) \times (w/r_i) \times c}$ denote the $i$-th flow matching network defined in Definition~\ref{def:flow_matching_architecture}.
        \item For $i \in [K]$, let $t_i \in [0,1]$ denote the time steps.
        \item For $i \in [K]$, let $\F^i_t$ be the interpolated input defined in Definition~\ref{def:flow}.
        \item Let $\Z_{\mathrm{init}} \in \R^{(h/r_1) \times (w/r_1) \times c}$ denote the initial input denoting the class condition.
        \item Let $\Z^1:=\Z_{\mathrm{init}} \in \R^{(h/r_1)\times(w/r_1) \times c}$.
    \end{itemize}
    Then, we define the architecture of FlowAR in the inference pipeline as follows:
    \begin{itemize}
        \item Layer 1: Given the initial token $\Z^1$, we compute
        \begin{align*}
            &~ s_1 = \mathsf{FFN}_1 \circ \mathsf{Attn}_1 (\Z^1) \in \R^{(h/r_1) \times (w/r_1) \times c}\\
            &~ \wh{s}_1 = \mathsf{NN}_1(F_t^1,s_1,t_1)
        \end{align*}
        \item Layer 2: Given the initial token $\Z^1$ and output of the first layer $\wh{s}_1$. Let $\Z^2$ be the reshape of the input sequence $\Z_{\mathrm{init}}, \phi_{\mathrm{up},1}(\wh{s}_1,a)$ into the tensor of size $(\sum_{i=1}^2 h/r_i) \times (\sum_{i=1}^2 w/r_i) \times c$. Then we compute
        \begin{align*}
            &~ s_2 = \mathsf{FFN}_2 \circ \mathsf{Attn}_2 (\Z^2)_{h/r_1:\sum_{i=1}^2 h/r_i,w/r_1 :\sum_{i=1}^2 w/r_i,0:c}\\
            &~ \wh{s}_2 = \mathsf{NN}_2(F_t^2,s_2,t_2)
        \end{align*}
        \item  Layer  $i \in [K]\setminus \{1,2\}$: Given the initial token $\Z^1$ and the output of the first $i-1$ layer $\wh{s}_1,\dots,\wh{s}_{i-1}$. Let $\Z^i$ be the reshape of the input sequence $\Z_{\mathrm{init}}, \phi_{\mathrm{up},1}(\wh{s}_1), \dots, \phi_{\mathrm{up},i-1}(\wh{s}_{i-1})$ into the tensor of size $(\sum_{j=1}^i h/r_j) \times (\sum_{j=1}^i w/r_j) \times c$. Then we compute
        \begin{align*}
            &~ s_i = \mathsf{FFN}_i \circ \mathsf{Attn}_i (\Z^i)_{\sum_{j=1}^{i-1} h/r_j : \sum_{j=1}^i h/r_j , \sum_{j=1}^{i-1} w/r_j : \sum_{j=1}^i w/r_j,0:c}\\
            &~ \wh{s}_i = \mathsf{NN}_i(F_t^i,s_i,t_i)
        \end{align*}
        Then the final output of FlowAR is $\wh{s}_K$.
    \end{itemize}
\end{definition}

\section{More Related Work}\label{sec:more_work}
In this section, we introduce more related work. 

\paragraph{Theoretical Machine Learning.}
Our work also takes inspiration from the following Machine Learning Theory work. Some works analyze the expressiveness of a neural network using the theory of circuit complexity~\cite{lls+25_gnn,kll+25_var_tc0,lls+24_rope_tensor_tc0,cll+24_mamba,cll+24_rope}. Some works optimize the algorithms that can accelerate the training of a neural network~\cite{llsz24,klsz24,dlms24,dswy22_coreset,haochen3,haochen4,dms23_spar,cll+25_deskreject,sy23,swyy23,lss+22,lsx+22,hst+22,hsw+22,hst+20,bsy23,dsy23,syyz23_weighted,gsy23_coin,gsy23_hyper,gsyz23,gswy23,syzz24,lsw+24,lsxy24,hsk+24,hlsl24}. Some works analyze neural networks via regressions~\cite{cll+24_icl,gms23,lsz23_exp,gsx23,ssz23_tradeoff,css+23,syyz23_ellinf,syz23,swy23,syz23_quantum,lls+25_grok}. Some works use reinforcement learning to optimize the neural networks~\cite{haochen1,haochen2,yunfan1,yunfan2,yunfan3,yunfan4,lswy23}. Some works optimize the attention mechanisms~\cite{sxy23,lls+24_conv}.


\paragraph{Accelerating Attention Mechanisms.}
The attention mechanism, with its quadratic computational complexity concerning context length, encounters increasing challenges as sequence lengths grow in modern large language models~\cite{gpto1,llama3_blog,claude3_pdf}. To address this limitation, polynomial kernel approximation methods \citep{aa22} have been introduced, leveraging low-rank approximations to efficiently approximate the attention matrix. These methods significantly enhance computation speed, allowing a single attention layer to perform both training and inference with nearly linear time complexity \citep{as23, as24b}. Moreover, these techniques can be extended to advanced attention mechanisms, such as tensor attention, while retaining almost linear time complexity for both training and inference \cite{as24_iclr}.~\cite{kll+25} provides an almost linear time algorithm to accelerate the inference of VAR Transformer. Other innovations include RoPE-based attention mechanisms~\cite{as24_rope,chl+24_rope} and differentially private cross-attention approaches~\cite{lssz24_dp}. Alternative strategies, such as the conv-basis method proposed in \cite{lls+24_conv}, present additional opportunities to accelerate attention computations, offering complementary solutions to this critical bottleneck. Additionally, various studies explore pruning-based methods to expedite attention mechanisms \cite{lls+24_prune,cls+24,llss24_sparse,ssz+25_prune,ssz+25_dit,hyw+23,whl+24,xhh+24,ssz+25_prune}.


\paragraph{Gradient Approximation.}
The low-rank approximation is a widely utilized approach for optimizing transformer training by reducing computational complexity \cite{lss+24,lssz24_tat,as24b,hwsl24,cls+24,lss+24_grad}. Building on the low-rank framework introduced in \cite{as23}, which initially focused on forward attention computation, \cite{as24b} extends this method to approximate attention gradients, effectively lowering the computational cost of gradient calculations. The study in \cite{lss+24} further expands this low-rank gradient approximation to multi-layer transformers, showing that backward computations in such architectures can achieve nearly linear time complexity. Additionally, \cite{lssz24_tat} generalizes the approach of \cite{as24b} to tensor-based attention models, utilizing forward computation results from \cite{as24_iclr} to enable efficient training of tensorized attention mechanisms. Lastly, \cite{hwsl24} applies low-rank approximation techniques during the training of Diffusion Transformers (DiTs), demonstrating the adaptability of these methods across various transformer-based architectures.




\ifdefined\isarxiv
%\section*{Acknowledgments}
\bibliographystyle{alpha}
\bibliography{ref}
\else


\fi

%%% some writing rules

%% Writing rule for creating tags.
%% Tags :
%% Theorem    \ref{thm:bla_bla}
%% Lemma      \ref{lem:bla_bla}
%% Claim      \ref{cla:bla_bla}
%% Corollary  \ref{cor:bla_bla}
%% Fact       \ref{fac:bla_bla}
%% Definition \ref{def:bla_bla}
%% Section    \ref{sec:bla_bla}
%% Subsection \ref{sub:bla_bla}
%% Equation   \ref{eq:bla_bla}



\end{document}



%%%%%%%%%%%%%%%%%%%%%%%%%%%%%%%%%%%%%%%%%%%%%%%%%%%%%%%%%%%%%%%%%%%%%%%%%%%%%%%%%%%%%%%%%%%%%%%%%%%%%%%%%%%%%%%%%%%%%%%%%%%%%%%%%%%%%%%%%%%%%%%%%%%%%%%%%%%%%%%%%%%%%%%%%%%%%%%%%%%%%%%%%%%%%%%%%%%%%%%%%%%%%%%%%%%%%%%%%%%%%%%%%%%%%%%%%%%%%%%%%%%%%%%%%%%%%%%%%%%%%%%%%%%%%%%%%%%%%%%%%%%%%%%%%%%%%%%%%%%%%%%%%%%%%%%%%%%%%%%%%%%%%%%%%%%%%%%%%%%%%%%%%%%%%%%%%%%%%%%%%%%%%%%%%%%%%%%%%%%%%%%%%%%%%%%%%%%%%%%%%%%%%%%%%%%%%%%%%%%%%%%%%%%%%%%%%%%%%%%%%%%%%%%%%%%%%%%%%%%%%%
