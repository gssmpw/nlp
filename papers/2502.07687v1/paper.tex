% !TEX program = pdflatex

\documentclass{article}
\usepackage{natbib}
\usepackage[utf8]{inputenc}
\usepackage{graphicx}
\usepackage{amsmath}
\usepackage{amssymb}
\usepackage{mathrsfs}
\usepackage{stmaryrd}
\usepackage{array}
\usepackage{float}
\usepackage[table]{xcolor}
\usepackage{tabularx}
\usepackage{booktabs}
\usepackage{textcomp}
\usepackage{fancyhdr}
\usepackage{booktabs}
\usepackage{setspace}
\usepackage{multirow}
\usepackage{endnotes}
\usepackage{url}
\usepackage{authblk}
\usepackage[T2A]{fontenc}
\usepackage[russian,english]{babel}

\definecolor{darkgreen}{rgb}{0,0.5,0}
\definecolor{darkred}{rgb}{0.6,0.0,0}
\definecolor{teal}{rgb}{0.00, 0.62, 0.45}
\definecolor{coral}{rgb}{0.83, 0.37, 0.00}
\newcommand{\usc}{\underline{\hspace{0.4cm}}\ }
\newcommand{\angles}[1]{\langle {#1} \rangle\ }
\newcolumntype{L}{>{\raggedright\arraybackslash}X}
\newcommand{\colorA}      [1]{\textcolor{Maroon}      {#1}}
\newcommand{\colorB}      [1]{\textcolor{NavyBlue}      {#1}}
\newcommand{\heatmap}[1]{%
    \ifdim#1pt<50pt\cellcolor{gray!20}#1\%\else%
    \ifdim#1pt>72pt\cellcolor{blue!50}#1\%\else%
    \ifdim#1pt>60pt\cellcolor{blue!20}#1\%\else%
    \ifdim#1pt>50pt\cellcolor{blue!10}#1\%\fi\fi\fi\fi}
\newcommand{\emptycell}{\cellcolor{white} }


\usepackage{expex}

\begin{document}


\author[1]{Imry Ziv}
\author[2]{Nur Lan}
\author[2]{Emmanuel Chemla}
\author[1]{Roni Katzir}


\affil[1]{Tel Aviv University
}
\affil[2]{École Normale Supérieure}


\title{Large Language Models as Proxies for Theories of Human Linguistic Cognition}


\maketitle              %


\begin{abstract}
We consider the possible role of current large language models (LLMs) in the study of human linguistic cognition. We focus on the use of such models as proxies for theories 
of cognition that are relatively linguistically-neutral in their representations and learning but differ from current LLMs in key ways. 
We illustrate this potential use of LLMs as proxies for theories of cognition in the context of two kinds of questions:  (a) whether the target theory accounts for the acquisition of a given pattern from a given corpus; and (b) whether the target theory makes a given typologically-attested pattern easier to acquire than another, typologically-unattested pattern. For each of the two questions we show, building on recent literature, how current LLMs can potentially be of help, but we note that at present this help is quite limited.
\end{abstract}

\noindent \textbf{Keywords:} Learning, Poverty of the stimulus, Linguistic typology, Large language models

\section{Introduction}
\label{sec:intro}

The question of whether current large language models (LLMs) might be relevant scientifically for the study of human linguistic cognition (HLC) is often framed in terms of a perceived threat that LLMs pose to generative linguistics. On the one hand, some researchers have suggested that these models discredit prevailing views within  linguistics \citep{Piantadosi:2023}, while others have suggested that LLMs can teach us nothing about HLC (\citealp{chomsky2023falsepromise}, \citealp{Moro:2023}).

Here we chart a more collaborative path forward for the integration of LLMs into the study of HLC. While the idea that current LLMs are themselves good theories of HLC is untenable (as reviewed below), we note that LLMs are a potentially useful tool that can inform the evaluation of theories of HLC. In this capacity, LLMs can serve as proxies for explicit theories of HLC that are different from LLMs in important ways but are still relatively unbiased with respect to linguistic representations and learning. We take it that much of the recent literature on the potential cognitive relevance of LLMs takes current models to support relatively linguistically neutral theories of this kind. To the extent that this use of LLMs as proxies is justified, they can assist in comparing such relatively unbiased theories of HLC with more traditional theories of HLC coming out of theoretical linguistics, where the combination of representations and learning is often highly biased in favor of linguistic patterns.



We start in section \ref{sec:bg} by briefly reviewing some considerations pertaining to the role of LLMs in studying HLC. We start with a reminder that the study of HLC, like the rest of empirical science, proceeds through inference to the best explanation, with explicit theories competing in light of known observations. We review the recent idea that LLMs themselves can serve as better explanations than generative theories and then summarize some of the reasons noted in the literature to reject this idea. We then consider the idea mentioned above that the best explanation is that HLC is quite different from current LLMs but still involves representations and learning that are relatively linguistically neutral. It is the potential of LLMs to serve as proxies for such a relatively linguistically-neutral theory that will preoccupy us in the remainder of the paper, and in particular the question of whether evidence from LLMs can bear on the question of whether the linguistically-neutral theory is a better explanation than generative approaches. In section \ref{sec:aps} we ask – building on \citealt{WilcoxFutrellLevy:2023} and \citealt{LanChemlaKatzir:2024a}, among others –  whether evidence from LLMs suggests that a relatively linguistically-neutral theory of HLC might succeed in acquiring various linguistic patterns from a developmentally-realistic corpus. In section \ref{sec:typ} we ask – building on \citealt{KalliniPapadimitriouFutrellMahowaldPotts:2024} and others – whether evidence from LLMs suggests that a relatively linguistically-neutral theory of HLC makes attested languages easier to acquire than various unattested variants of those languages. In both cases, we will see that the evidence from LLMs gives no reason to think that a linguistically-neutral theory of HLC could account for the data, which in turn means that current LLMs contribute little to the ability of such linguistically-neutral theories to challenge generative theories. 


\section{Background: LLMs, HLC, and the inference to the best explanation}
\label{sec:bg}

Being part of empirical science, the study of HLC proceeds through inference to the best explanation: explicit theories are evaluated in light of the observations, and those theories that explain the data best are preferred to the other contenders. The theories coming out of generative linguistics over the past 70 years or so are routinely evaluated in this way. These theories are often strongly linguistically biased, in the sense that they make it relatively easy to represent and learn knowledge of the kind found across languages and hard or impossible to represent or learn many other patterns.
Let us use $H_1$ to refer to some concrete instantiation of the generative approach. We will return to the biases embodied by the architecture and representations of $H_1$ in the following sections.






More recently, \citet{Piantadosi:2023} has suggested that current LLMs %
should be treated as explicit theories of HLC that should be taken seriously as part of this evaluation. Following \citet{FoxKatzir:2024} we refer to this new approach to HLC as the \textit{LLM Theory}, and we discuss it in Section \ref{sec:llm:th}, through a concrete instantiation we will call $H_2$. Perhaps in light of the glaring inadequacies of the LLM Theory, some recent literature, including \citealt{WilcoxFutrellLevy:2023} and \citealt{MahowaldIvanovaBlankKanwisherTenenbaumFedorenko:2024}, has advocated for a revision of generative linguistics on the basis of LLMs but without committing to the LLM Theory. We refer to this idea as the \textit{Proxy View}: 

\ex \label{} The Proxy View of LLMs in HLC: current LLMs are not themselves good theories of HLC, but their performance is indicative of the success of some other theory of HLC, presumably more linguistically-neutral than the generative approach, and can inform the inference to the best explanation.
\xe


Since the literature that promotes the Proxy View generally does not make its supported theory explicit, our ability to evaluate it at present is limited. For concreteness, we will outline what such a target theory might look like, a theory that we will refer to as $H_3$ and that addresses at least some of the most immediate inadequacies of the LLM Theory but is still more linguistically neutral than generative proposals. But we can of course have no certainty that this is what \citet{WilcoxFutrellLevy:2023}, \citet{MahowaldIvanovaBlankKanwisherTenenbaumFedorenko:2024}, and others have in mind. We will therefore focus on methodological considerations that pertain to the evaluation of the target theory once it is provided and show how LLMs might bear on this evaluation. We show that, at least at present, LLMs lend no support to the Proxy View.


\subsection{The LLM Theory and its inadequacy}
\label{sec:llm:th}


As mentioned, \citet{Piantadosi:2023} argues for the LLM Theory of HLC. That is, he suggests that current LLMs, taken as real theories of HLC, are a better explanation of the known observations than anything coming out of the generative tradition. This claim is incorrect, for reasons discussed in detail in the literature and rehearsed briefly here. Following \citealt{Katzir:2023} and \citealt{FoxKatzir:2024} we structure our brief discussion of the failure of the LLM Theory around three key issues: (a)~competence vs.\ performance, (b)~correctness vs.\ probability, and (c)~learning and representations.%
\footnote{See  \citealt{KodnerPayneHeinz:2023}, \citealt{MoroGrecoCappa:2023}, \citealt{RawskiBaumont:2023}, and others for further critical discussion of the LLM Theory.}

\textbf{Competence vs.\ performance.}  Humans sometimes struggle with sentences that, given more time and focus, they ultimately accept. Familiar examples include center embedding, as in "The mouse that the cat that the dog chased bit died". Humans also sometimes initially accept sentences that, upon further thought, they reject. Familiar examples include variants of sentences such as the above but in which the verbs and the nouns are not balanced, as in "The mouse that the cat that the dog chased died". Other examples include so-called agreement attraction, as in "The keys to the cabinet is on the table". 

The best explanation to these and other examples, going back to \citealt{Yngve:1960} and \citealt{MillerChomsky:1963}, distinguishes linguistic competence from performance. The competence of English speakers, for example, licenses center-embedded structures regardless of depth, and it requires agreement between verbs and subjects regardless of intervening nouns. Performance factors get in the way and make center embedding and agreement across intervening nouns difficult and error prone. Among other things, the explanation in terms of a distinction between competence and performance accounts for the way center embedding can become easier or harder given various manipulations (tiredness, noise, parallel tasks, etc.).

Differently from common generative proposals, the design and behavior of current LLMs does not suggest a meaningful distinction between competence and performance of the kind discovered in humans: there is no reason to think that more (or less) time and working memory would make them change their behavior on center embedding or agreement. This leaves current LLMs with no good explanation for the increasing difficulty of sentences with increasing levels of center embedding or for the other observations that have been taken to support the distinction between competence and performance. 




\textbf{Correctness vs.\ likelihood.}  Humans recognize some sentences as correct but unlikely (one example again is center embedding as in the example above), and they recognize some other sentences as incorrect but likely (again, unbalanced variants of center embedding and agreement attraction, among other examples).
LLMs provide straightforward information regarding their representation of likelihood. But it remains unclear whether they even have a distinct representation of correctness.\footnote{A potentially important area of research concerns understanding the inner workings of LLMs, sometimes relating the discovered dynamics to linguistic patterns. See \citealt{Lakretz:2019} and \citealt{Ravfogel:2021}, among others. It is conceivable that such investigations will eventually uncover a notion of correctness within LLMs.} %

\textbf{Learning and representations.} The inductive leaps that humans make, sometimes based on very little data, suggest specific representations and biases. This is the essence of so-called \textit{arguments from the poverty of the stimulus} (APS). A particularly well-known APS, originally outlined by \citet{Chomsky:1971} and discussed in detail in much later work, concerns constituency. This is often illustrated with the formation of yes-no questions, as in (\ref{sai}). The generalization that English-learning children reach is that such questions are formed by fronting the auxiliary that is structurally highest, as in (\ref{sai:good}). They do not seem to seriously consider an alternative generalization that says that the fronted auxiliary is the one that is first linearly (which would yield the incorrect (\ref{sai:bad})), even though the data that children are typically exposed to has been argued to be unhelpful in making the choice.\footnote{But see \citealt{{Reali:Christiansen:2005}} for further discussion of the accessibility of subject-aux inversion to acquisition from data alone.} 
\pex \label{sai}
    \a[] \label{sai:main} [The boy who is singing] can dance.
    \a[] \label{sai:good} Can [the boy who is singing] \usc\ dance?
    \a[] \label{sai:bad} *Is [the boy who \usc\ singing] can dance?
\xe

Generative theories of HLC typically derive this inductive leap by assuming that children are born with a combination of representations and learning biases that favor the representation of constituency-based dependencies over dependencies that rely on linear order (in some approaches by preventing the latter from being represented in the first place). Current LLMs, on the other hand, seem to provide no basis for this inductive leap. In fact, recent work by \citet{LanGeyerChemlaKatzir:2022,LanChemlaKatzir:2024b} casts doubt on the ability of networks trained by current methods to even acquire the basic notion of constituency, let alone to explain the preference for constituency-based generalizations over linear-based ones. Not surprisingly, then, \citet{yedetore:2023} find evidence suggesting that at least some neural networks fail to acquire the correct constituency-based generalization from their training data. We will return to APS in section \ref{sec:aps}, where we will consider the possible role of LLMs in reasoning about the inductive leaps predicted by non-LLM theories of HLC.

Evidence about the representations and learning biases of humans also comes from the highly skewed distribution of linguistic patterns cross-linguistically. We can again consider constituency (following \citealt{FoxKatzir:2024}), though we emphasize that a very wide range of cross-linguistic patterns have been discussed in the linguistic literature. We just mentioned that constituency is part of the best explanation of the competence of English speakers, noting its role in the formation of yes-no questions. Constituency is central to the explanation of many other aspects of English, including center-embedding and agreement mentioned earlier, wh-movement discussed in section \ref{sec:aps} below, the computation of meaning, and more. But it seems clear that this is not just an accident of actual English: constituency is part of the best explanation of almost all languages that have been studied closely, which suggests that humans are born with representations and learning biases that favor constituency. Again, the representations and learning biases of current LLMs might not even be able to support the acquisition of constituency, let alone explain its cross-linguistic prevalence. We will return to cross-linguistic evidence in section \ref{sec:typ}, where we will consider the possible role of LLMs in reasoning about the typological patterns predicted by non-LLMs theories of HLC. 


\subsection{LLMs as proxies for good learners}
But if the LLM Theory is discredicted, does this mean that LLMs are irrelevant to the scientific study of HLC? As mentioned above, the Proxy View explains why the answer is no: in principle LLMs can serve as proxies for theories of a third kind, theories that involve relatively linguistically-neutral representations and learning (and are different in this regard from generative theories) but that are not vulnerable to the most immediate arguments against the LLM Theory.

Here we consider two such uses of LLMs as proxies. Following \citealt{LanChemlaKatzir:2024a} we use LLMs as tools for reasoning about whether a relatively linguistically-neutral theory of HLC predicts a given inductive leap based on a given amount of linguistic exposure. And following \citealt{KalliniPapadimitriouFutrellMahowaldPotts:2024} we use LLMs as tools for reasoning about whether a relatively linguistically-neutral theory of HLC derives a learning asymmetry that might be part of an explanation of a typological asymmetry. 

At this point we should mention again an immediate obstacle to the proxy use of LLMs: while recent literature, including  \citet{WilcoxFutrellLevy:2023} and \citet{MahowaldIvanovaBlankKanwisherTenenbaumFedorenko:2024}, suggests that the performance of LLMs supports linguistically-neutral theories of HLC, this literature does not provide much by way of specifics. Until such specifics are provided, expressions of hope about the undisclosed theories remain just that. 

Still, we can outline some methodological considerations regarding the possible role of current LLMs in evaluating a linguistically-neutral theory of HLC once such a theory is made sufficiently explicit. 

Let us illustrate with a concrete target theory, though we emphasize that this is for presentational purposes only and that we obviously cannot tell whether this target theory is at all close to what the literature supporting the Proxy View envisions. Consider then the following theory of HLC, which we will call $H_3$. 
According to $H_3$, the syntax is written and understood as Multiple Context-Free Grammar: any MCFG can be written, using the formalism of \citet{SekiMatsumuraFujiiKasami:1991}, and no other grammar can be written.  The semantics in $H_3$ is model-theoretic and is computed compositionally based on the syntax, while the syntax is blind to most of the semantics. Learning according to this theory follows the principle of Minimum Description Length (MDL; \citealp{Rissanen:1978}). And processing relies on a parser with access to very little working memory and that can output not just parses (for grammatical inputs) but also probabilities.

$H_3$ is quite different from the LLM Theory and does not share the inadequacies of that theory reviewed above. For one thing, $H_3$ has a meaningful distinction between competence and performance: by relying on a limited amount of working memory, the parser will run into trouble with some of the same grammatically correct sentences that humans struggle with such as those with center embedding. $H_3$ also has a meaningful distinction between correct and probable. As to representations and learning, the combination of (a) MCFG as the representational framework for syntax and (b) MDL as the learning criterion will allow the learner to represent and learn constituency and perhaps also constituency-based dependencies. And the combination of those representations with compositional semantics can go a long way toward explaining the centrality of constituency across languages. As to modularity, this is baked into the assumptions mentioned earlier about the information available to the syntax. But while $H_3$ is very different from the LLM Theory, it is also more linguistically-neutral than common generative theories of HLC such as $H_1$. The representational system of $H_1$ would mostly likely include a specific preparation for grammars that involve linguistic patterns attested across langauges, such as constraints on movement commonly known as syntactic islands (\citealt{Ross:1967}), and including the phenomena we discuss in the following section. While $H_3$ makes it possible to state such grammars, it does not postulate anything specific to these patterns inside the theory, and does not provide further explicit biases beyond those present in a constituency-based syntax and model-theoretic semantics. It is therefore conceivable that the performance of LLMs might be informative to some extent about what can be learned and under what conditions under $H_3$, though adequately justifying the Proxy View for $H_3$ would require careful argumentation that goes beyond the present outline. 

\begin{table}
\centering
\small
\renewcommand{\arraystretch}{1.5}
\begin{tabular}{|p{5.4cm}|c|c|c|}
\hline
\textbf{Criterion} & \textbf{Generative~Linguistics} & \textbf{LLM Theory} & \textbf{Proxy View} \\
\hline
\textbf{Linguistic Bias} & \cellcolor{green!25} Strong & \cellcolor{red!25} Weak & \cellcolor{green!25} Intermediate \\
\hline
\textbf{Competence vs. Performance} & \cellcolor{green!25} $\checkmark$ & \cellcolor{red!25} $\times$ & \cellcolor{green!25} $\checkmark$ \\
\hline
\textbf{Grammaticality $\neq$ Likelihood} & \cellcolor{green!25} $\checkmark$ & \cellcolor{red!25} $\times$ & \cellcolor{green!25} $\checkmark$ \\
\hline
\textbf{Explanation for learning assymetries} &
\cellcolor{green!25} $\checkmark$ &
\cellcolor{red!25} $\times$ &
\cellcolor{green!25} $\checkmark$ \\
\hline
\textbf{Concrete Instantiation} &
\cellcolor{green!25} $H_1 $&
\cellcolor{red!25} 
$H_2$ &
\cellcolor{green!25} $H_3$ \\
\hline
\end{tabular}

\caption{Comparison of the three theories outlined above on key issues, following~\citealt{FoxKatzir:2024}.} 
\label{tab:theories}
\end{table}






As mentioned, we have no way of telling whether $H_3$ is the theory of HLC that works such as \citealt{WilcoxFutrellLevy:2023} and \citealt{MahowaldIvanovaBlankKanwisherTenenbaumFedorenko:2024} seem to have tried to promote on the basis of LLM performance. None of our methodological discussion below depends in any way on the specific properties of $H_3$: this theory can serve as a concrete example of a relatively linguistically-neutral theory of HLC if having such an example is helpful, but the reader should feel perfectly free to ignore it. It should also be clear that any serious attempt to promote $H_3$ - or any other theory of HLC - as the best explanation would need to examine it in light of the whole body of work within theoretical linguists since the 1950s and that linguists have relied on to argue for theories of HLC that incorporate strong linguistic biases. Since our goal here is merely to illustrate the potential use of LLMs as proxies for linguistically-neutral theories, we will proceed as if this task has already been carried out.


\section{Large Language Models and alignment with the stimulus}
\label{sec:aps}

Consider then a theory of HLC - $H_3$ above if one so wishes, or something else -that is neutral with respect to some linguistic property $PROP$ for which generative theories include a strong linguistic bias. A potential argument against this relatively linguistically-neutral theory and in favor of a generative theory might take the form of an APS: if humans show knowledge of $PROP$ after exposure to a corpus that seems too impoverished to support the learning of $PROP$ by a linguistically-neutral learner, this will support the strongly biased generative theory. The problem is that it is generally very hard to determine what a given learner might acquire from a given corpus, even for a theory such as $H_{3}$ for which both the representations and the learning mechanism are explicitly provided. 

LLMs might be able to help. We can train them on a suitable corpus and then probe their knowledge. Note that this is not entirely straightforward, for various reasons discussed in \citealt{LanChemlaKatzir:2024a}. Most immediately, while we might be able to directly inspect a grammar within a theory of HLC such as $H_3$ and ask whether it has knowledge of $PROP$, we cannot do the same with current LLMs, whose inner workings are extremely complex and remain mostly opaque, as discussed in Section~\ref{sec:llm:th}. Instead of inspection, we might turn to behavioral measures such as acceptability judgments in minimal pairs where one member satisfies $PROP$ and is acceptable and a minimal variant violates it and is unacceptable, a methodology that has been used in \citealt{MarvinLinzen:2018}, \citealt{Hu:2020} and \citealt{WilcoxLevyFutrell:2019}, among others. But while the theory of HLC under consideration might offer such judgments, current LLMs provide likelihood assessments, which are an entirely different thing, as mentioned above. However, if we focus on areas where acceptability and likelihood are reasonably well aligned, we might be justified in looking at whether the LLM assigns a higher probability to the $PROP$-satisfying member of the pair than to the $PROP$-violating one. If this probabilistic preference is sufficiently strong, we might consider the fact that the LLM has acquired this preference as reason to think that the learner under our target theory of HLC would acquire the actual $PROP$. This might be so even if the probabilistic preference of the LLM reflects a flawed approximation of $PROP$ rather than knowledge of $PROP$ in any meaningful sense.

Following \citeauthor{WilcoxFutrellLevy:2023} and \citeauthor{LanChemlaKatzir:2024a}, our success criterion considers the model successful if and only if it assigns a higher probability to the grammatical member of the minimal pair. Specifically, we evaluate probabilities on a critical region token whose preceding context is the same in both the grammatical and the ungrammatical sentence (see Table~\ref{tab:phenomena}, where critical regions are indicated with underscores). Note that this is an extremely lenient notion of success: the model is considered successful even if the preference for the grammatical member of the minimal pair is extremely small, and even if ungrammatical sentences are considered more likely than some (less related) grammatical sentences.

\subsection{Methods}

\begin{table}[htbp]
\begin{tabularx}{\textwidth}{@{}X r@{ }l r@{ }l@{}}
    \toprule
    \textbf{Model} & \multicolumn{2}{l}{\textbf{Train Dataset Size}} & \multicolumn{2}{l}{\textbf{Human Equivalent}} \\
    \midrule
    \textbf{CHILDES LSTM,\ Transformer} & 8.6  & million tokens    & 10      & months \\
    \textbf{BabyLM 10M}                & 10   & million tokens    & 1       & year \\
    \textbf{Wikipedia Transformer}     & 90   & million tokens    & 8       & years \\
    \textbf{BabyLM 100M}               & 100  & million tokens    & 9       & years \\
    \textbf{bert-based-uncased}        & $\approx$ 3.5 & billion tokens    & 320     & years \\
    \textbf{GPT2}                      & $\approx$ 8   & billion tokens    & 730     & years \\ 
    \textbf{llama3.2-3b}               & $\approx$ 9   & trillion tokens   & 821,250 & years \\
    \bottomrule
    \end{tabularx}
    \caption{Training data size of the eight language models considered here, and
    the human linguistic experience equivalent to these data sizes, following \citeauthor{HartRisley:1995}.}
    \label{tab:model-training-sizes}
\end{table}


Let us illustrate   this kind of reasoning with concrete examples. To do so, we use an empirical setup that consists of eight models of different architectures and training schemes (see Section~\ref{sec:appendix-aps}).\footnote{All experimental material can be found in https://osf.io/5zh6q/.} The models were trained on datasets that roughly correspond to different timeframes of linguistic experience, ranging from 10 months of human experience (CHILDES) through eight years of linguistic experience (English Wikipedia subset) to 320,730 or even 821,250 years of linguistic experience (BERT base uncased, GPT-2 and Llama 3.2-3b respectively).\footnote{More concretely, we use two models from \citeauthor{yedetore:2023}, an LSTM and a Transformer, that were trained on the CHILDES corpus of child-directed speech; a Transformer we trained on a subset of English Wikipedia, for which we used one of the large Transformer architectures used in \citeauthor{yedetore:2023}; the \textit{base-strict} and \textit{base-strict-small} versions of the BabyLM LLMs  from \citet{conll-2023-babylm}, trained on the 100M and 10M BabyLM datasets correspondingly; the \textit{BERT base uncased} version of BERT (\citealt{DBLP:journals/corr/abs-1810-04805}), trained on the BooksCorpus and English Wikipedia datasets; OpenAI’s GPT-2 (\citealt{radford2019language}); Meta's \textit{Llama 3.2-3b} LLAMA model (released September 2024). See Section~\ref{sec:appendix-aps} for further technical detail.} The models were probed on their knowledge of three linguistic phenomena using the success criterion introduced above (see Table~\ref{tab:phenomena}). The linguistic test cases presented in the sections below show that all models, despite being trained on multitudes of data, perform around chance level and worse on the lenient success criterion. This result is not optimistic for LLMs as a proxy for an adequate theory of HLC.


\begin{table}[h]
\centering
\small
\renewcommand{\arraystretch}{1.3} %
\setlength{\tabcolsep}{8pt} %
\begin{tabular}{p{2.5cm} p{7.5cm}}  
\toprule
\textbf{Phenomenon} & \textbf{Example (\textcolor{teal}{Grammatical}/\textcolor{coral}{*Ungrammatical})} \\ 
\midrule

\multirow{2}{=}{Across-the-board movement (ATB)}  
& \textcolor{teal}{Which boy did you say that Kim hated and Mary loved \underline{yesterday}?} \\  
& \textcolor{coral}{* Which boy did you say that Kim hated and Mary loved \underline{Ann} yesterday?} \\  
\midrule

\multirow{2}{=}{Parasitic gaps (PG)}  
& \textcolor{teal}{I know who John's talking to is going to annoy \underline{soon}.} \\  
& \textcolor{coral}{* I know who John's talking to is going to annoy \underline{you} soon.} \\  
\midrule

\multirow{4}{=}{That-trace effects (TTE)}  
& \textcolor{teal}{Who did you say that \underline{Sue} loves?} \\  
& \textcolor{teal}{Who did you say \underline{loves} Sue?} \\  
& \textcolor{teal}{Who did you say \underline{Sue} loves?} \\  
& \textcolor{coral}{* Who did you say that \underline{loves} Sue?} \\  
\bottomrule
\end{tabular}
\caption{Examples of linguistic phenomena with grammatical and ungrammatical sentences. Model probabilities are compared on the underscored critical regions.}
\label{tab:phenomena}
\end{table}

\subsection{Test Case: Across-the-board movement and parasitic gaps}
\label{sec:atb}

Above we mentioned displacement phenomena, illustrating with the position of the auxiliary in yes-no questions in English. Displacement phenomena have featured in recent discussions of the role of LLMs in evaluating APSs, both for the placement of the auxiliary (see, e.g., \citealt{yedetore:2023}) and for wh-movement (see \citealt{ChowdhuryZamparelli:2018}, \citealt{WilcoxFutrellLevy:2023}, and \citealt{LanChemlaKatzir:2024a}). The latter will concern us in the current two subsections. Here is a simple illustration:
\ex \label{ex:wh:book} [Which book] did you say that Mary read \usc\ last week?
\xe

The common analysis of wh-movement in generative theories involves a constituent that is attached in one position and is then re-attached higher up in the structure. In (\ref{ex:wh:book}), the constituent \textit{[which book]} is attached within the embedded clause as a sister to \textit{Mary} and is then re-attached at or near the root of the matrix clause, which among other things leads to it being pronounced in the beginning of the sentence. 

Some work within generative linguistics (\citealt{PearlSprouse:2013}, \citealt{Philips:2011}) suggests that a relatively linguistically-neutral theory cannot account for the full knowledge of wh-movement given the kind of data that children are exposed to, an instance of the APS. This is not a consensus view, however, and some recent work such as \citeauthor{WilcoxFutrellLevy:2023} suggests that LLMs undermine this APS and that these models show that a linguistically neutral theory would, in fact, acquire a full knowledge of wh-movement from a developmentally-realistic corpus. Following \citeauthor{LanChemlaKatzir:2024a} let us show how LLMs (used as proxies for an undisclosed linguistically-neutral theory of HLC) might bear on the APS under consideration and why they currently do nothing to undermine it.

Before proceeding, let us reiterate that it is hard to tell whether LLMs are good proxies for the relatively linguistically-neutral theory of HLC that is envisioned in works such as \citet{WilcoxFutrellLevy:2023} in the absence of at least some specifics about that theory. But if they are, and if we focus on cases where correctness and likelihood are reasonably well-aligned, we could reason that if the LLM overwhelmingly assigns much higher probability to the correct member of each minimal pair, this indicates that there was sufficient information in the training data for the target linguistically-neutral theory to have acquired the relevant knowledge. And if the LLM does not perform as well, this suggests that the target theory would also struggle. The former possibility weakens the APS, and the latter strengthens it.

Following \citet{LanChemlaKatzir:2024a} we focus on a family of wh-movement configurations where there is complex interaction between two gaps, in a way that seemingly violates island constraints on syntactic movement (\citealt{Ross:1967}). The first such phenomenon is \textit{across-the-board (ATB) movement} in coordinate structures. While extraction from a single coordinate is ungrammatical, as in \ref{atb:badleftextraction} and \ref{atb:badrightextraction}, it becomes grammatical when extracting from both coordinates, as in \ref{atb:good}:
\pex \label{atb}
    \a[] \label{atb:badleftextraction} *[Which book] did you say that [Kim wrote \usc\ last year] and [Mary read a new novel yesterday]?
    \a[] \label{atb:badrightextraction} *[Which book] did you say that [Kim wrote a new novel last year] and [Mary read \usc\ yesterday]?
    \a[] \label{atb:good} [Which book] did you say that [Kim wrote \usc\ last year] and [Mary read \usc\ yesterday]?
\xe


The second nuance of wh-movement we test is when extraction from an island is made possible by the presence of another gap downstream, a phenomenon known as a \textit{parasitic gap (PG)}: 
\pex \label{pg}
    \a[] \label{pg:bad} *I know who [John's talking to \usc] is going to annoy you soon.
    \a[] \label{pg:good} I know who [John's talking to \usc] is going to annoy \usc\ soon.
\xe


We test four models from Table~\ref{tab:model-training-sizes} on their knowledge of across-the-board movement and parasitic gaps using the minimal pair method. We compare sentences that both have an upstream filler - like \textit{which book} in Example~\ref{atb}, or \textit{who} in Example~\ref{pg} - and differ in whether they are gapped accordingly.\footnote{The paradigm sentences in \cite{LanChemlaKatzir:2024a} were generated by template for ATB and PG. Here we unify the templates for both phenomena so that they use the same lexical choices where relevant (e.g., proper names), and by adding more lexical choices.
From each template, represented by a context-free grammar, we sampled 10,000 sentence pairs and ran them through each model to get the relevant probability values (see Appendix~\ref{sec:appendix-typ}).} As is shown in Figure~\ref{fig:atb-pg-raw}, all prefer the ungrammatical over the grammatical members of the minimal pairs in the vast majority of cases. This failure suggests that the linguistically-neutral theory for which the LLMs are taken to be useful proxies would not acquire wh-movement from a developmentally-realistic corpus. 

\begin{figure}[t]
    \includegraphics[width=0.9\textwidth]{aps-atb-raw.png} \\
    \includegraphics[width=0.9\textwidth]{aps-pg-raw.png}
    \caption{Model accuracy on ATB and PG datasets averaged over five experiment seeds. Accuracy is measured as the ratio of cases where the model assigns a higher probability to the grammatical sentence continuation.}
    \label{fig:atb-pg-raw}
\end{figure}


\subsection{Test Case: that-trace effects}
\label{sec:tte}

Another property of wh-movement in English is that when the moved wh-phrase is a subject, the clause from which it moves cannot have an overt complementizer such as `that':

\pex \label{ex:tte}
    \a[] \label{tte:pthatptrace} *Who did you say that \usc\ \underline{loves} Sue?\textsubscript{(+that, +trace)}
    \a[] \label{tte:pthatmtrace} Who did you say that \underline{Sue} loves \usc?\textsubscript{(+that, -trace)}
    \a[] \label{tte:mthatptrace} Who did you say \usc\ \underline{loves} Sue?\textsubscript{(-that, +trace)}
    \a[] \label{tte:mthatmtrace} Who did you say that \underline{Sue} loves \usc?\textsubscript{(-that, -trace)}
\xe


This asymmetry, known as that-trace effects (TTE), has been thoroughly discussed in theoretical linguistics (see \citealt{Perlmutter:1968}, \citealt{ChomskyLasnik:1977}, and \citealt{Phillips:2013a}, among others). It has been shown that extraction of any kind from clauses with \textit{that} is vanishingly rare in adult-directed corpora, and even rarer in child-directed speech \citet{Phillips:2013a}\footnote{Out of 11,308 analyzed utterances from child-directed speech, there were only two instances of non-subject extraction with \textit{that} and zero instances of subject extraction with \textit{that}, contrasting with 159 instances of object extraction without \textit{that}, and 13 instances of subject extraction without \textit{that}.}. This distribution is claimed (\citealt{ChomskyLasnik:1977}, \citealt{Phillips:2013b}) to be insufficient to warrant acquisition of the paradigm above. We set aside the important theoretical discussion of what bias present in the architecture or representations of $H_1$ underlies the TTE according to generative linguistics.\footnote{See \citet{Sobin:1987}, \citet{Lohndal:2009}, \citet{Boskovic:2016} and \citet{Pesetksy:2017}, among others, for discussion.} Instead, we focus on its status as an APS, which warrants evaluating the performance of the linguistically-neutral LLM contender on it. 

Similarly to the previous case, we test eight LLM models (see Table~\ref{tab:model-training-sizes}). The four-way asymmetry warrants two criteria that constitute knowledge of the TTE: first, given an overt complementizer upstream, the model should prefer an overt subject to a gap in the embedded clause. This preference should translate into probabilities so that $P(loves)$ in \ref{tte:pthatptrace} should be smaller than $P(Sue)$ in \ref{tte:pthatmtrace}. Second, given a gapped sentence downstream ("... \usc\ loves Sue"), the model should prefer a prefix without an overt complementizer. That is, $P(loves)$ in \ref{tte:pthatptrace} is expected to be smaller than $P(loves)$ in \ref{tte:mthatptrace}. This way, we measure the dispreference for an overt complementizer in the presence of a moved subject along both axes - that of the complementizer and that of the gap.

Results for both success criteria are presented in Figure~\ref{fig:tte-raw}. Under both criteria, the performance of the models hovers around chance. While it is unclear how to interpret these results, we see nothing in the performance of the models that suggests that the target hypothesis will account for the learning of that-trace effects from similar training corpora.

\begin{figure}[h]
    \includegraphics[width=0.8\linewidth]{aps-tte-raw-old-criterion.png} \\[1ex] %
    \includegraphics[width=0.8\linewidth]{aps-tte-raw-new-criterion.png}
    \caption{Model accuracy values for the $P(+that, +trace) < P(+that, -trace)$ criterion (top) and $P(+that, +trace) < P(-that, +trace)$ criterion (bottom), over samples of 10,000 sentence pairs from the TTE test set. Results are averaged over five seeds. The dark plot represents model training sizes.}
    \label{fig:tte-raw}
\end{figure}

\section{Large Language Models and cross-linguistic evidence}
\label{sec:typ}

Consider again the comparison of a strongly linguistically-biased generative theory and a more linguistically-neutral alternative. Suppose that the strong linguistic biases of the generative theory can be shown to help derive a cross-linguistic pattern, perhaps by making it easier to represent and learn the languages that adhere to the pattern than those that do not. In the linguistically-neutral theory it can be much harder to see directly whether the attested patterns are predicted to be easier to learn than the unattested ones. One might try to evaluate this matter empirically, by simulating learning of languages of the attested and the unattested kind given the linguistically-neutral theory, but this can be hard to do. Consequently, it can be hard to see whether the theory can explain the cross-linguistic pattern under consideration and whether it can compete with the generative theory. Again, however, if an LLM is a good proxy for the theory, the problem is avoided to some extent. One can examine the ease with which the LLM learns to approximate both languages that follow the pattern and languages that violate it. If the LLM approximates the former more easily than the latter, this can suggest that the target theory might be able to account for the cross-linguistic pattern (subject to various further assumptions about how relative ease of learning derives the typology, and to an adequate definition of how ease-of-learning via LLM is to be measured). If, on the other hand, some of the pattern-violating languages are easier to learn than some of the pattern-satisfying ones, the linguistically-neutral theory would need further mechanisms to derive the typology. 

We should emphasize that the inferences from this argument are particularly weak. On the one hand, even if the linguistically-neutral theory derives a learning asymmetry between pattern-satisfying and pattern-defying languages but the difference is very small, this might not suffice to account for a robust typological pattern. On the other hand, even if the linguistically-neutral theory does not derive a learning asymmetry, it might be able to account for it in terms of other factors, such as communicative pressure. In order to overcome these limitations one could try to situate the learning component within a model of cultural evolution, which allows for the amplification of small asymmetries over generations and also factors in communicative pressures and other considerations. We believe that such a model is the proper context for the use ease of learning by LLMs to evaluate typological asymmetries, but following \citet{KalliniPapadimitriouFutrellMahowaldPotts:2024}, whose setup we use here, we leave such a model for future work and only assess currently available models.\footnote{\citet{KalliniPapadimitriouFutrellMahowaldPotts:2024} contextualize their work differently. We will not attempt to determine whether the present work is in line with their goals.} 


\subsection{Ease of learning by LLM as a theory of HLC?}

\citet{KalliniPapadimitriouFutrellMahowaldPotts:2024}\ compare GPT-2 when trained on an English corpus and when trained on corpora that are derived from the original corpus through various perturbations, each corresponding to an unattested language. They find that the original English corpus was easier for GPT-2 than the derived corpora of unattested languages when ease-of-learning is measured through test set perplexities through time (see Figure~\ref{fig:il-partial-reverse}). In doing so, \citet{KalliniPapadimitriouFutrellMahowaldPotts:2024} make use of LLMs as proxies in the context of explaining the cross-linguistic skew that rules out the unattested perturbations of English, even though this theory is never made explicit in the paper.
If a model like GPT-2 is indeed a good proxy for this contender, then we may say that what  \citet{KalliniPapadimitriouFutrellMahowaldPotts:2024} have shown to hold for English should also hold cross-linguistically, as presented in the introduction: all unattested languages should be harder to learn than attested ones. 

In the following subsections we consider a few languages for which the evaluation is simple given current resources and test them with the perturbations used by \citet{KalliniPapadimitriouFutrellMahowaldPotts:2024} and one additional perturbation, listed in Table~\ref{tab:perturbations}. For each perturbation, we create perturbed versions and baseline versions for each of the four languages - English, Italian, Russian and Hebrew, based on datasets from \citet{GulordavaBojanowskiGraveLinzenBaroni:2018}. We then train a transformer LLM based on the architecture and training scheme from \citet{yedetore:2023}, and consider the validation set perplexities as training progresses.\footnote{The model was trained for 48 hours and validation perplexity was computed every
200 training batches. The 48 hours of training amounted to roughly 1.5 epochs per dataset, with batch size 10, such that each epoch amounted to an average of 181,552 batches, see Appendix~\ref{sec:appendix-typ} for further details.} We find we find several different languages of an unattested kind that are easier for the LLM to approximate than a corresponding attested language, as measured by the perplexity metric used by \citet{KalliniPapadimitriouFutrellMahowaldPotts:2024}. From the perspective of the Proxy View, this means that the target, relatively linguistically-neutral theory fails to receive support from the LLM: if the target hypothesis can explain the typological pattern, this is not detected by the LLM experiment.

\begin{table}[h!] 
    \renewcommand{\arraystretch}{1.5}  %
    \begin{tabular}{p{2cm}p{7cm}}  %
        \toprule
        Perturbation & \textcolor{teal}{Attested} / \textcolor{coral}{Perturbed} \\ 
        \midrule
        \textit{partial-reverse} & 
        \textcolor{teal}{Colorless\textsubscript{0} green\textsubscript{1} \texttt{<rev>}\textsubscript{2} ideas\textsubscript{3} sleep\textsubscript{4} furiously\textsubscript{5}.} 
        \newline
        \textcolor{coral}{Colorless\textsubscript{0} green\textsubscript{1} \texttt{<rev>}\textsubscript{2} furiously\textsubscript{5} sleep\textsubscript{4} ideas\textsubscript{3}.} \\

        \midrule
        \textit{full-reverse} & 
        \textcolor{teal}{Colorless\textsubscript{0} green\textsubscript{1} \texttt{<rev>}\textsubscript{2} ideas\textsubscript{3} sleep\textsubscript{4} furiously\textsubscript{5}.} 
        \newline
        \textcolor{coral}{Furiously\textsubscript{5} sleep\textsubscript{4} ideas\textsubscript{3} \texttt{<rev>}\textsubscript{2} green\textsubscript{1} colorless\textsubscript{0}.} \\
        \midrule
        \textit{switch-indices} & 
        \textcolor{teal}{Colorless\textsubscript{0} green\textsubscript{1} ideas\textsubscript{2} sleep\textsubscript{3} furiously\textsubscript{4}.} 
        \newline 
        \textcolor{coral}{Ideas\textsubscript{2} green\textsubscript{1} colorless\textsubscript{0} sleep\textsubscript{3} furiously\textsubscript{4}.} \\
        \midrule
        \textit{token-hop} & 
        \textcolor{teal}{They\textsubscript{0} were\textsubscript{1} sleeping\textsubscript{2} \textbf{v\textsubscript{3}} next\textsubscript{4} to\textsubscript{5} the\textsubscript{6} colorless\textsubscript{7} green\textsubscript{8} ideas\textsubscript{9}.} 
        \newline
        \textcolor{coral}{They\textsubscript{0} were\textsubscript{1} sleeping\textsubscript{2} next\textsubscript{3} to\textsubscript{4} the\textsubscript{5} \textbf{v\textsubscript{6}} colorless\textsubscript{7} green\textsubscript{8} ideas\textsubscript{9}.} \\
        \bottomrule
    \end{tabular}
\label{tab:perturbations}
\caption{Perturbation test cases. Ease-of-learning is evaluated for the attested and perturbed versions of English, Italian, Hebrew and Russian.}
\end{table}


\subsection{Test case: \textit{partial-reverse}}


We demonstrate the ease-of-learning method in our first test case - a cross-linguistic reproduction of the \textit{partial-reverse} perturbation from \citealt{MitchellBowers:2020} and \citealt{KalliniPapadimitriouFutrellMahowaldPotts:2024}.  Each sentence is reversed starting from a randomly chosen index in the input sentence, in which a special marker token \textit{\textless rev\textgreater} is inserted:

\pex \label{ex:partial-reverse}
    \a \textbf{Baseline:} \label{partial-reverse-baseline} Colorless$_{0}$ green$_{1}$ \texttt{<rev>}$_{2}$ ideas$_{3}$ sleep$_{4}$ furiously$_{5}$.
    \a \textbf{Perturbed:} \label{partial-reverse-perturbed} Colorless$_{0}$ green$_{1}$ \texttt{<rev>}$_{2}$ furiously$_{5}$ sleep$_{4}$ ideas$_{3}$.
\xe



The target, relatively linguistically-neutral theory in this case might be a version of a generative theory in which there is a parameter for reverse. Any grammar that is statable within the theory now comes in two variants. If the parameter is set to 0, the grammar works as in the original generative theory. But if the parameter is set to 1, every derivation ends with reversing the output starting from a random point. This may or may not be the target theory that \citeauthor{KalliniPapadimitriouFutrellMahowaldPotts:2024} have in mind; other target theories are of course imaginable, and \citeauthor{KalliniPapadimitriouFutrellMahowaldPotts:2024} do not comment explicitly about this. We note that, like \citeauthor{KalliniPapadimitriouFutrellMahowaldPotts:2024}'s other perturbations, reversal seems quite far away from the typological questions that generative linguistics have usually concerned themselves with. Here we stay with \citeauthor{KalliniPapadimitriouFutrellMahowaldPotts:2024}'s perturbations and setup simply in order to illustrate the relevant methodological considerations, but a proper exploration of the Proxy View would need to look at those cross-linguistic patterns that have informed linguistic research.  


\begin{figure}[h]
\includegraphics[width=\linewidth]{il-partial-reverse-en-it-ru-cl-view.png}
\caption{Validation perplexity during training for English, Italian, and Russian and their \textit{partial-reverse} perturbations. The results indicate that $\Pi(\textit{attested}) < \Pi(\textit{partial-reverse})$.}

\label{fig:il-partial-reverse}
\end{figure}

We find that validation perplexities follow an attested-unattested divide: the partial-reverse perturbed versions of English, Italian and Russian are harder to learn than their attested counterparts (see Fig.~\ref{fig:il-partial-reverse}).\footnote{The \textit{partial-reverse} languages are compared to a modified baseline of English in which the marker \textit{\textless rev\textgreater} is inserted randomly without reversing, to control for the effect for additional textual material in the sentence on perplexities. Since we use the same seed for both perturbations, the markers are inserted in the same indices (see example \ref{ex:partial-reverse}).}



This, as discussed above, could be part of an argument that linguistic biases are not needed to rule out partial-reverse languages, but crucially only if there were a serious contender for the best explanation that allows for such languages in the first place, and could be compared theoretically with the restrictive generative account. However, other perturbations paint a different picture under the cross-lingustic view.


\subsection{Test case: \textit{full-reverse}}

 We now consider the perturbation $\Pi=$ \textit{full-reverse} from \citealt{KalliniPapadimitriouFutrellMahowaldPotts:2024}, where each sentence is reversed in its entirety and a special marker token is randomly inserted: 

 \pex \label{ex:full-reverse}
 \a \textbf{Baseline:} \label{full-reverse-baseline} Colorless$_{0}$ green$_{1}$ \texttt{<rev>}$_{2}$ ideas$_{3}$ sleep$_{4}$ furiously$_{5}$.
 \a \textbf{Perturbed:} \label{full-reverse-perturbed} Furiously$_{5}$ sleep$_{4}$ ideas$_{3}$ \texttt{<rev>}$_{2}$ green$_{1}$ colorless$_{0}$.
\xe


 We find that the perturbed versions of Italian, Russian and Hebrew are projected as easier than attested English (see Fig.~\ref{fig:il-full-reverse}), even though the perturbation is considered humanly impossible. In this case, ease of learning according to \citet{KalliniPapadimitriouFutrellMahowaldPotts:2024}’s perplexity metric does not provide support for the idea that the target theory can explain away the typological asymmetry through asymmetries of ease of learning. As mentioned above, this conclusion is weak. It is possible that the target theory is the best explanation and that the typology is explained with the help of factors other than ease of learning. It is also very possible that \citeauthor{KalliniPapadimitriouFutrellMahowaldPotts:2024}'s perplexity metric is a poor way to evaluate ease of approximation by the LLM. All we can conclude at present is that proponents of the Proxy View have work to do in light of the LLM's performance in this case, and that previous results only provide a limited picture.


\begin{figure}

\includegraphics[width=\linewidth]{il-full-reverse-en-he-it-ru.png}
\caption{Validation perplexity during training for attested (baseline) and \textit{full-reverse} versions of English, Russian, Italian, and Hebrew.}
\label{fig:il-full-reverse}
\end{figure}


\subsection{Test case: \textit{switch-indices}}

We turn to another asymmetry that the Proxy View does not at present explain. Consider $\Pi=$ \textit{switch-indices}, where the tokens at index 0 and index 2 in every sentence are switched:


\pex \label{ex:switch-indices}
    \a \textbf{Baseline:} \label{switch-indices-baseline} Colorless$_{0}$ green$_{1}$ ideas$_{2}$ sleep$_{3}$ furiously$_{4}$.
    \a \textbf{Perturbed:} \label{switch-indices-perturbed} Ideas$_{2}$ green$_{1}$ colorless$_{0}$ sleep$_{3}$ furiously$_{4}$.
\xe

One can again imagine various target theories that allow for languages like English as well as their switch-index variants. And again the question would be whether such a target theory can account for the typological observation that such switch-index variants are unattested.

If the target theory can account for the typological asymmetry, this is again not detected through the perplexity curve of the LLM proxy: each language is wrongly projected as harder than its perturbed version, and the perplexity delta between each two attested languages is significantly larger (see Fig. \ref{fig:il-switch-indices}). In all cases, there is an unattested language that is projected as easier than an attested language.


\begin{figure}[t]
\includegraphics[width=1\linewidth]{il-switch-indices-en-he-it-ru.png}

\caption{Validation perplexity during training for the attested (baseline) and \textit{switch-indices} versions of English, Italian, Hebrew and Russian.}
\label{fig:il-switch-indices}

\end{figure}


\subsection{Test case: \textit{token-hop}, \textit{no-hop}}

We look at one final typological asymmetry, based on the $\Pi=\textit{token-hop}$ perturbation employed by  \citet{KalliniPapadimitriouFutrellMahowaldPotts:2024}.  For each of English, Italian, and Russian, we create a perturbed variant by inserting a new marker token three tokens after each verb:\footnote{New marker tokens were randomly generated as single-character tokens that do not already exist in the model's vocabulary (English: "v", Italian: "v", Russian: "\textit{\textcyrillic{Ч}}").}
\pex \label{ex:token-hop-no-hop}
    \a \textbf{no-hop (baseline):} \label{no-hop-baseline} They$_0$ were$_1$ sleeping$_2$ \textbf{v$_3$} next$_4$ to$_5$ the$_6$ colorless$_7$ green$_8$ ideas$_9$.
    \a \textbf{token-hop (perturbed):} \label{token-hop-perturbed} They$_0$ were$_1$ sleeping$_2$ next$_3$ to$_4$ the$_5$ \textbf{v$_6$} colorless$_7$ green$_8$ ideas$_9$.
\xe

Since additional textual material is inserted, in a way that might confound perplexities over time, we compare the perplexities achieved by our model on \textit{token-hop} datasets to \textit{no-hop} datasets, in which the marker token is inserted exactly after each verb. This achieves control for the effect of the marker token itself, such that difference in perplexities can be attributed solely to the effect of the token-counting generalization. The \textit{no-hop} version adheres to attested generalizations, as it can be thought of as post-verbal clitic that marks for parts of speech. Learning \textit{token-hop} on the other hand crucially requires counting tokens, an ability that is particularly non-humanlike, both because human languages have no atomic concept that corresponds to tokens, and because counting rules violate the strictly hierarchal nature of linguistic generalizations across languages. Although the distance of the token from its associated verb has an effect on perplexity, we find that the \textit{no-hop} version of English is still harder to learn than \textit{token-hop} Italian and Russian (see Fig.~\ref{fig:il-token-hop}).  Unperturbed English datasets are consistently projected as harder than \textit{token-hop} perturbed versions of Italian and Russian, despite a clear typological skew against counting rules. The same point is made with the validation perplexities of \textit{token-hop} Italian and \textit{no-hop} Russian patterning very similarly. The best explanation of linguistic typology should predict a clear divide between unattested languages that crucially require counting linguistic elements and attested languages. Ease-of-learning via LLM does not explain this clear typological skew.

\begin{figure}
\includegraphics[width=\linewidth]{il-token-hop-no-hop-en-it-ru.png}
\caption{Validation perplexity during training for the \textit{no-hop} (baseline) and \textit{token-hop} versions of English, Italian, and Russian.}
\label{fig:il-token-hop}

\end{figure}

\section{Conclusion}
\label{sec:conc}

LLMs have been presented as a challenge to more traditional approaches to the study of HLC, and in particular to generative linguistics. A blunt version of this idea is the LLM Theory, promoted by \citet{Piantadosi:2023}, which maintains that current LLMs are themselves good theories of HLC. Current LLMs can, of course, be viewed as theories of HLC, but this treatment does these models no favors, as the literature was quick to note. 

A more cautious approach is the Proxy View, promoted by \citet{WilcoxFutrellLevy:2023} and  \citet{MahowaldIvanovaBlankKanwisherTenenbaumFedorenko:2024}, which maintains that current LLMs are good proxies - as far as various behavioral measures are concerned - for good theories of HLC that are more linguistically-neutral than generative theories. No commitment to shared essentials between LLMs and the relevant good theories is implied. The Proxy View is of course perfectly coherent, but differently from the LLM Theory it is too vague about the theories of HLC that it champions to be of much direct use within empirical science. Empirical science is based on inference to the best explanation and proceeds through competition between reasonably well-understood theories, and saying just that one's favorite theory learns roughly as well as an LLM is insufficient for this purpose. One would like to see enough detail at least for a proper comparison of the relevant theory to generative linguistics in light of the discoveries and considerations in the literature of the past 70 years or so. More than anything else, the present paper is a plea for supporters of the Proxy View to provide this kind of detail and evaluation for the theory they have in mind.

Still, already at this stage we can outline an empirical evaluation of the Proxy View on various kinds of test cases, and here we focused on two: alignment with the stimulus and cross-linguistic variation. The results were not encouraging for the Proxy View. The LLMs failed to even approximate key pieces of knowledge that humans have, even when the models were trained on much larger corpora than children are exposed to. The LLMs also had an easier time approximating various typologically-unattested languages than actual human languages, at least with respect to a success criterion used in \citet{KalliniPapadimitriouFutrellMahowaldPotts:2024}.
\pagebreak

\appendix 
\section{Section~\ref{sec:aps} - Appendix}
\label{sec:appendix-aps}

\subsection{General Notes}
In all experiments, probabilities are evaluated on the critical region in boldface (as in the example sentences in Tables~\ref{tab:cfg-atb}, \ref{tab:cfg-pg}, \ref{tab:cfg-tte}). For models listed as unidirectional, the probability value takes into consideration the left context of the evaluated token, whereas for bidirectional models, the probability value factors in both the left and the right context. The bidirectional models we consider were pretrained with masked language modeling (\textit{fill-mask}). 

In all generated sentences, the critical region unambiguously disambiguates the sentence as an instance of the phenomenon under consideration (there is no other grammatical reading). To promote lexical variation in target regions, possible target tokens in the CFGs were chosen at random from a POS-tagged English Wikipedia dump. We filtered out potential target tokens that were not part of the vocabulary of one the models considered in our setup, which are detailed in Table~\ref{tab:models-verbose}.
\begin{table}[h!]
    \centering
    \begin{tabular}{|l|l|l|l|}
    \hline
    \textbf{Model Name} & \textbf{Training Size} & \textbf{Directionality} & \textbf{Source} \\
    \hline
    CHILDES LSTM & 8.6M & Uni & \citealt{yedetore:2023} \\
    CHILDES Transformer & 20M & Uni & \citealt{yedetore:2023} \\
    BabyLM 10M (\textit{strict-small}) & 10M & Bi & \citealt{Babylm:proceedings}  \\
    Wikipedia Transformer & 15M & Uni & Trained by authors \\
    BabyLM 100M & 100M & Bi & \citealt{Babylm:proceedings} \\
    BERT Base Uncased & $\approx$3.5B & Bi & \citealt{Devlin:2018} \\
    GPT2 & $\approx$8B & Uni & \citealt{Radford:2019} \\
    Llama 3.2-3b & $\approx$9T & Uni & \citealt{Touvron:2023} \\
    \hline
    \end{tabular}
    \caption{Models used in the experiments in Section~\ref{sec:atb}.}
    \label{tab:models-verbose}
    \end{table}
    


\subsection{ATB and PG}
For the ATB and PG experiments, we replicate the experimental setup from \citealt{LanChemlaKatzir:2024a} with a wider variety of test sentences to increase robustness. We also unify the templates for both phenomena
so that they use the same lexical choices where relevant (e.g., proper names). From each template, represented by a context free grammar, we generated tuples of sentences that correspond to the \textit{(+filler,+gap), (+filler,-gap)} conditions (see CFGs and example sentences in Tables~\ref{tab:cfg-atb} and~\ref{tab:cfg-pg}). We then evaluated the success criterion $P(+filler,-gap) < P(+filler,+gap)$ on a sample of 10,000 tuples for each phenomenon. For ATB, sentence structure was crucially changed such that the two conjuncts do not agree in number, to prevent a problematic non-ATB reading present in some of the \citeauthor{LanChemlaKatzir:2024b} test sentences. Underlined words alternate according to the $\pm filler$ condition; words in bold mark the position where the $\pm gap$ condition becomes evident and probability is measured. We evaluate the performance of four models from Table~\ref{tab:models-verbose}. We do not consider bidirectional models for the PG and ATB experiments because they are highly sensitive to whether the token is the last token in the sentence, which in the ATB and PG case is one of the minimal differences between the conditions we consider.


\begin{table}[h!]
    \begin{tabular}{p{9cm}}
        \hline
        ATB Grammar \\
        \hline
        \\
$S \rightarrow \angles{PREAMBLE} \angles{\pm F} \angles{LINK} \angles{\pm G} $
\\
$\langle PREAMBLE \rangle \rightarrow \ \textit{I know}$
\\
$\angles{+F} \rightarrow \ \underline{\textit{{which}}\  \angles{PLNOUN}} \angles{VP1}\ \angles{ADV1}$
\\
$\angles{+G} \rightarrow \angles{LINK}\ \angles{VP2}\ \angles{\textbf{ADV2}}$
\\
$\angles{-G} \rightarrow \angles{LINK}\ \angles{VP2}\ \angles{\textbf{OBJ}}\ \angles{ADV2}$
\\
$\angles{LINK} \rightarrow \text{`and is going to'}$
 \\ 

$\angles{PLNOUN} \rightarrow \text{'boys'}\ |\
\text{'girls'}\ |\ \text{'people'}$

$\angles{ADV1} \rightarrow \text{`recently'}\ |\ \text{`lately'}$
 \\ 
$\angles{ADV2} \rightarrow \text{`soon'}\ |\ \text{`today'}\ |\ \text{`now'}$
 \\ 
$\angles{VP1} \rightarrow \angles{VP1\_SIMPLE}\ |\ \angles{VP1\_COMPLEX}$
 \\ 
$\angles{VP1\_SIMPLE} \rightarrow \text{`met'}\ |\ \text{`saw'}$
 \\ 
$\angles{VP2} \rightarrow \angles{VP2\_SIMPLE}\ |\ \angles{VP2\_COMPLEX}$
 \\ 
$\angles{VP2\_SIMPLE} \rightarrow \ \text{`hug'}\ |\ \text{`slap'}\ |\ \text{`kiss'}$
 \\ 
 $\angles{OBJ} \rightarrow \text{`you'}\ |\ \text{`us'}\ |\ \text{`Kim'}$
\\
$\cdots$
\\

$\Rightarrow$ I know \underline{which boys} John met recently and is going to hug \textbf{soon}. $_{(+filler, +gap)}$
\\
$\Rightarrow$ *I know \underline{which boys} John met recently and is going to hug \textbf{you} soon. $_{(+filler, -gap)}$
\\
\\
\hline
\hline
 \\
    \end{tabular}
    \caption{
    Excerpt from the context-free grammar used to generate Across-the-Board sentences for the experiments in Section~\ref{sec:aps}.}
    \label{tab:cfg-atb}
\end{table}



\begin{table}[h!]
\begin{tabular}{p{9cm}}
\hline
PG Grammar \\
\hline
\\
\\
$S \rightarrow \angles{PREAMBLE} \angles{\pm F} \angles{\pm G} $
\\
$\langle PREAMBLE \rangle \rightarrow \ \textit{I know}$
\\
$\angles{+F} \rightarrow \ \textit{\underline{who}}\  \langle NAME1 \rangle \angles{GEN} \angles{NP} $
\\
$\angles{-F} \rightarrow \ \textit{\underline{that}}\  \langle NAME1 \rangle \angles{GEN} \angles{NP} \angles{\underline{NAME2}} $
\\
$\angles{+G} \rightarrow \angles{LINK}\angles{V} \angles{\textbf{ADV}}$
\\
$\angles{-G} \rightarrow \angles{LINK}\angles{V} \angles{\textbf{OBJ}}\angles{ADV} $
\\
$\angles{GEN} \rightarrow \ \textit{'s} $
\\
$\angles{NP} \rightarrow \angles{NP\_SIMPLE}\ |\ \angles{NP\_COMPLEX}$
 \\ 
$\angles{NP\_SIMPLE} \rightarrow \angles{GERUND}$
 \\ 
$\angles{NP\_COMPLEX} \rightarrow \angles{N\_EMBEDDED}\ \text{`to'}\ \angles{V\_EMBEDDED}$
 \\ 
$\angles{LINK} \rightarrow \text{`is about to'}\ |\ \text{`is likely to'}\ |\ \text{`is going to'}\ |\ \text{`is expected to'}$
 \\ 
$\angles{V} \rightarrow \text{`bother'}\ |\ \text{`annoy'}\ |\ \text{`disturb'}$
 \\ 
 $\angles{OBJ} \rightarrow \text{`you'}\ |\ \text{`us'}\ |\ \text{`Kim'}$
\\
$\angles{GERUND} \rightarrow \text{`talking to'}\ |\ \text{`dancing with'}\ |\ \text{`playing with'}$
 \\ 
$\angles{N\_EMBEDDED} \rightarrow \text{`decision'}\ |\ \text{`intent'}\ |\ \text{`effort'}\ |\ \text{`attempt'}\ |\ \text{`failure'}$
 \\ 
$\angles{V\_EMBEDDED} \rightarrow \text{`talk to'}\ |\ \text{`call'}\ |\ \text{`meet'}\ |\ \text{`dance with'}\ |\ \text{`play with'}$
 \\ 
$\angles{ADV} \rightarrow \text{`soon'}\ |\ \text{`eventually'}$
\\
$\cdots$
\\

$\Rightarrow$ I know \underline{who} John's talking to is going to annoy \textbf{soon}. $_{(+filler, +gap)}$
\\
$\Rightarrow$ * I know \underline{who} John's talking to is going to annoy \textbf{you} soon. $_{(+filler, -gap)}$
\\
\\
\hline
\hline
 \\
\end{tabular}
\caption{Excerpt from the context-free grammar used to generate Parasitic Gap sentences for the experiments in Section~\ref{sec:aps}.}
\label{tab:cfg-pg}
\end{table}


\subsection{TTE}

To create test sets, we generate 1,178,496 quadruplets, out of which
we randomly sample 10,000 quadruplets per experiment. Paradigms are varied lexically in several ways: we first create paradigms that differ
structurally from each other, to control for the possibility that our results are an
artifact of the specific sentence structure introduced in the example sentences in Table~\ref{tab:cfg-tte}. We
also introduce lexical variation to all non-target elements, such as the main subject,
the embedding verb, the wh-element and the verb inside the wh-clause. Most
importantly, lexical choices for target verbs and nouns are varied using 30 target nouns and 30 target verbs. We perform five inference experiments on five such samples and consider average accuracy over all experiments. Quadruplets are evaluated with the two success criteria introduced in Section~\ref{sec:tte}.

\begin{table}[h!]
\begin{tabular}{p{12.5cm}}\hline
TTE Grammar \\
\hline
\\
\\
$S \rightarrow \angles{PREAMBLE} \angles{F} \angles{EMBEDDING\_VERB} \angles{COMP} \angles{TARGET\_SUBJ} \angles{TARGET\_VERB}$
\\
$\angles{PREAMBLE} \rightarrow \text{'He knows'}\ |\ \text{'The boy asked'}\ |\ \text{'The girl knew'} \ldots
$
\\
$\angles{F} \rightarrow \text{who}\ |\ \varepsilon$
\\
$\angles{EMBEDDING\_VERB} \rightarrow \text{`think'}\ |\ \text{`believe'}\ |\ \text{`say'}$
\\
$\angles{COMP} \rightarrow \text{that}\ |\ \varepsilon$
\\
$\angles{TARGET\_SUBJ} \rightarrow \text{`people'}\ |\ \text{`children'}\ |\ \text{`parents'}\ |\ \text{`scientists'}\ |\ \text{`kids'} \ldots$  
\\
$\angles{TARGET\_VERB} \rightarrow \text{'are'}\ |\ \text{'have'}\ |\ \text{'ate'}\ |\ \text{'got'}\ |\ \text{'saw'}\ |\ \text{'met'} \ldots$
\\
$\cdots$
\\


$\Rightarrow$ He knows who you think that \textbf{ate}
monsters $_{(+filler, +gap)}$
\\
$\Rightarrow$ * He knows who you think that \textbf{monsters} ate $_{(+filler, -gap)}$
\\
$\Rightarrow$ * He knows you think \textbf{ate} monsters $_{(-filler, +gap)}$
\\
$\Rightarrow$ He knows you think \textbf{monsters} ate $_{(-filler, -gap)}$
\\
\\
\hline
\hline

\end{tabular}
\caption{Excerpt from the context-free grammar used to generate That-Trace-Effect sentences for the experiments in Section~\ref{sec:aps}.}
\label{tab:cfg-tte}
\end{table}


\section{Section~\ref{sec:typ} - Appendix}
\label{sec:appendix-typ}

\subsection{Dataset Creation}

Our baseline datasets are based on Wikipedia dumps in English, Italian, Hebrew and Russian from \citet{GulordavaBojanowskiGraveLinzenBaroni:2018}, which extracted 90M token subsets for each language, and switched all tokens that do not belong to the top 50K most frequent tokens into $\angles{unk}$ tokens (see \citealt{GulordavaBojanowskiGraveLinzenBaroni:2018}). For each baseline dataset we create perturbed \textit{train}, \textit{test} and \textit{validation} datasets (see Table~\ref{tab:perturbations}), similarly to \citealt{KalliniPapadimitriouFutrellMahowaldPotts:2024}, such that each perturbed dataset has the same number of tokens as its corresponding baseline dataset. 

The \textit{no-hop} and \textit{token-hop} perturbations require POS tagging, which we performed using the \href{https://spacy.io/models}{Spacy} Python library. Since the POS tagging accuracy is low for Hebrew, we did not perform \textit{no-hop} and \textit{token-hop} experiments on it. New marker tokens were randomly generated as single-character tokens that do not already exist in the model's vocabulary (English: "v", Italian: "v", Russian: "\textit{\textcyrillic{Ч}}"). We used the following Spacy NLP pipelines to perform POS tagging and identify verbs: 
\begin{enumerate}
    \item "en\_core\_web\_sm", accuracy of POS tagger: 0.97
    \item "it\_core\_news\_sm", accuracy of POS tagger: 0.97
    \item "ru\_core\_news\_sm", accuracy of POS tagger: 0.99
\end{enumerate}
We chose to replace \citet{KalliniPapadimitriouFutrellMahowaldPotts:2024}'s 3rd person agreement \textit{token-hop} with a POS marker because of the lower accuracy of morphological taggers on languages that are not English and cross-linguistic differences in verb morphology.



\subsection{Model Training and Evaluation}

We use the optimal Transformer architecture chosen in a hyperparameter search conducted by \citealt{yedetore:2023}, which has four layers, a hidden and embedding size of 800, a batch size of 10, a dropout rate of 0.2 and a learning rate of 5. Every 200 batches, we evaluate the average perplexity per token of the current trained model on a held out validation set. We set a 48-hour training threshold, during which each model processed a different number of batches. For consistency in comparison graphs, we use the smallest batch count achieved and compare all models up to that point. Table~\ref{tab:typ-batches} details the number of training batches achieved by the Transformer model on each perturbed dataset.

\begin{table}[h!]
\
\renewcommand{\arraystretch}{1.3}  %

\begin{tabular}{|p{1.7cm}|p{2cm}|p{3cm}|p{1.5cm}|}
\hline

\textbf{Language} & \textbf{Perturbation} & \textbf{Size of Validation Set (batches)} & \textbf{Training batches} \\
\hline
\multirow{2}{*}{English} & \textit{no-perturb} & 138430 & 137600 \\
                         & \textit{full-reverse}   & 143518 & 117400 \\
                         & \textit{partial-reverse}   &  143517 & 131800 \\
                         & \textit{switch-indices}   & 166116 & 205516 \\
                        & \textit{no-hop}   & 181502 &  189102 \\
                         & \textit{token-hop} &  181502   & 142200 \\
\hline
\multirow{2}{*}{Italian} & \textit{no-perturb} & 138589 & 137200 \\
                         & \textit{full-reverse}   & 143840 & 154200 \\
                         & \textit{partial-reverse}   & 143506 & 152000 \\
                         & \textit{switch-indices}   & 166307 & 213507 \\
                         & \textit{no-hop}   & 181552 & 146400 \\
                         & \textit{token-hop}  & 181552 & 191752 \\
\hline
\multirow{2}{*}{Russian} & \textit{no-perturb} & 138602 & 136200 \\
                         & \textit{full-reverse}   & 145993 & 153600 \\
                         & \textit{partial-reverse}   & 145984 & 125400 \\
                         & \textit{switch-indices}   & 166323 & 213323 \\
                         & \textit{no-hop}   & 178771 & 148800 \\
                         & \textit{token-hop}   & 178771 & 192571 \\

\hline
\multirow{2}{*}{Hebrew} & \textit{no-perturb}& 137773 & 138000 \\
                         & \textit{full-reverse}   & 142884 & 154800 \\
                         & \textit{partial-reverse}   & 171460 & 203860 \\
                         & \textit{switch-indices}   & 165328 & 210128 \\
                         & \textit{no-hop}   & - & - \\
                         & \textit{token-hop}   & - & - \\
\hline
\end{tabular}
\caption{Number of training batches achieved in 48 hours of training on our perturbed datasets from Section~\ref{sec:typ}.}
\label{tab:typ-batches}
\end{table}



\clearpage

\bibliographystyle{custom} %
\documentclass{MITstyle}

%\usepackage[table]{xcolor}
\usepackage{chngcntr}
\usepackage{hyperref}
\usepackage{microtype}

\title{A Lightweight and Extensible Cell Segmentation and Classification Model for Whole Slide Images}

\author{Nikita Shvetsov~$^{1, }$\footnote{Correspondence e-mail: nikita.shvetsov@uit.no}, Thomas K. Kilvaer~$^{2, 3}$, Masoud Tafavvoghi~$^{4}$, Anders Sildnes~$^{1}$, \\ Kajsa Møllersen~$^{4}$, Lill-Tove Rasmussen Busund~$^{5, 6}$, Lars Ailo Bongo~$^{1}$ \\
%
\vspace{1em} % Space between authors and afilliations
%
\normalfont{\small $^{1}$Department of Computer Science, UiT The Arctic University of Norway}\\
\normalfont{\small $^{2}$Department of Oncology, University Hospital of North Norway}\\
\normalfont{\small $^{3}$Department of Clinical Medicine, UiT The Arctic University of Norway}\\
\normalfont{\small $^{4}$Department of Community Medicine, UiT The Arctic University of Norway}\\
\normalfont{\small $^{5}$Department of Medical Biology, UiT The Arctic University of Norway} \\
\normalfont{\small $^{6}$Department of Clinical Pathology, University Hospital of North Norway} %\vspace{2em}
}

\begin{document}
\maketitle

\section*{Abstract}

% \begin{abstract}
% Developing clinically useful cell-level analysis tools in digital pathology remains challenging due to limitations in dataset granularity, inconsistent annotations, computational demands of advanced models, and difficulties in integrating new technologies into clinical workflows. To address these challenges, we propose a multi-faceted solution that enhances data quality, model performance, and usability to create a lightweight and extensible cell segmentation and classification model.

% First, we update data labels by employing a cross-relabeling process that refines the labels of two existing datasets, PanNuke and MoNuSAC, to create a new unified dataset with enhanced granularity, encompassing seven distinct cell types. Second, we leverage the H-Optimus foundation model as a fixed encoder to improve feature representation for simultaneous cell segmentation and classification tasks. Third, to address the computational demands of foundation models, we employ knowledge distillation to reduce model size and complexity while maintaining comparable performance. Finally, to facilitate integration into clinical workflows, we integrate the distilled model into the QuPath software, a widely used open-source platform in digital pathology.

% Our results demonstrate improvements in cell segmentation and classification performance using the H‑Optimus-based model compared to a CNN-based model. Specifically, the average $R^2$ improved from 0.575 to 0.871, and the average $PQ$ score improved from 0.450 to 0.492, indicating better alignment with actual cell counts and enhanced segmentation and classification quality. Furthermore, the distilled student model maintains performance comparable to the larger foundation model while reducing the parameter count by a factor of 48.
% Overall, by reducing computational complexity and integrating it into existing workflows, the proposed approach may significantly impact diagnostic processes, reduce the workload of pathologists, and contribute to improved patient outcomes. Though our approach shows potential enhancements in efficiency and usability of cell segmentation and classification models in digital pathology, extensive validation is needed to deploy these models in clinical practice.
% \end{abstract}

%%% shortened abstract
\begin{abstract}
Developing clinically useful cell-level analysis tools in digital pathology remains challenging due to limitations in dataset granularity, inconsistent annotations, high computational demands, and difficulties integrating new technologies into workflows. To address these issues, we propose a solution that enhances data quality, model performance, and usability by creating a lightweight, extensible cell segmentation and classification model. 

First, we update data labels through cross-relabeling to refine annotations of PanNuke and MoNuSAC, producing a unified dataset with seven distinct cell types. Second, we leverage the H-Optimus foundation model as a fixed encoder to improve feature representation for simultaneous segmentation and classification tasks. Third, to address foundation models' computational demands, we distill knowledge to reduce model size and complexity while maintaining comparable performance. Finally, we integrate the distilled model into QuPath, a widely used open-source digital pathology platform. 

Results demonstrate improved segmentation and classification performance using the H-Optimus-based model compared to a CNN-based model. Specifically, average $R^2$ improved from 0.575 to 0.871, and average $PQ$ score improved from 0.450 to 0.492, indicating better alignment with actual cell counts and enhanced segmentation quality. The distilled model maintains comparable performance while reducing parameter count by a factor of 48. By reducing computational complexity and integrating into workflows, this approach may significantly impact diagnostics, reduce pathologist workload, and improve outcomes. Although the method shows promise, extensive validation is necessary prior to clinical deployment.
\end{abstract}
\clearpage

\section{Introduction}
In digital pathology, accurate segmentation and classification of cells are crucial for many diagnostic, prognostic, and predictive analyses \cite{Jaber_Beziaeva_etal._2019,Lin_Pan_etal._2022,Park_Ock_etal._2022,Shen_Choi_etal._2024}. Nowadays, developments in computational pathology offer multiple solutions \cite{H._Qu_P._Wu_etal._2020,Javed_Mahmood_etal._2020} to utilize cell-level datasets to train machine learning models that solve these problems. The quality and specificity of training datasets are critical for robust and accurate models. Adhering to the principle of "garbage in, garbage out", it is essential to ensure that these datasets are extensively and accurately labeled with distinct classes that reflect the diverse biological characteristics of different cell types. Unfortunately, the number of open-source datasets comprising such high-quality annotations is limited. Existing cell segmentation datasets \cite{Gamper_Koohbanani_etal._2019,Graham_Vu_etal._2019,Verma_Kumar_etal._2021} may offer extensive annotations for certain cell types while providing more general labels for others. For example, in PanNuke, which is one of the largest open-source datasets comprising labeled cells, various types of morphologically and functionally different inflammatory cells like macrophages and lymphocytes are clustered in a broad "inflammatory" class. Consequently, these classes are frequently omitted from analyses or aggregated into broader meta-classes \cite{Gamper_Koohbanani_etal._2020} and likely interfere with other cell classes included in the dataset. This and similar inconsistencies in annotation granularity limit the ability of machine learning models to learn the comprehensive and nuanced features necessary for accurate cell segmentation and classification. To address these challenges, methods for refining and standardizing dataset annotations are essential to enhance the quality of training data.

A complementary approach to mitigate the absence of high-quality training data is the use of foundation models. Foundation models as encoders are defined as large-scale, versatile networks pre-trained on vast, diverse datasets using self-supervised learning, contrasting with convolutional neural network (CNN) pre-trained encoders that rely on supervised learning with labeled data. In practice, foundation models leverage enormous amounts of weakly or unlabeled data from millions of whole slide images (WSIs) and employ self-attention mechanisms to capture long-range dependencies and global context \cite{Chen_Ding_etal._2024,Saillard_Jenatton_etal._2024,Vorontsov_Bozkurt_etal._2024,Xu_Usuyama_etal._2024}. As a consequence, foundation models are able to produce transferable feature representations across different cell types and tissue environments. The feature representations can be leveraged by decoder networks to produce segmentation masks and pixel-level classifications. Because foundation models have comprehensive feature representations, they can be effectively fine-tuned using much smaller amounts of cell-level data compared to the large datasets needed to train models from scratch. Furthermore, foundation models incorporate adversarial training elements or contrastive learning \cite{Chen_Ding_etal._2024,Xu_Usuyama_etal._2024}, enhancing their resilience and adaptability by exposing them to challenging and varied scenarios during training. This may result in more generalizable models, often making them well-suited for diverse and complex tasks in digital pathology.

Despite the inherent advantages of foundation models, their deployment for practical use faces its own obstacles. In particular, they require substantial computational power, financial investments and rigorous testing to ensure reliability and efficacy for a given task \cite{Akkus_Dangott_etal._2022,Dragomir_Cocuz_etal._2022,Go_2022,Jafri_Farooqui_etal._2024}. Moreover, while foundation models enhance feature representation and performance, they depend on the quality of available annotations for decoder fine-tuning and, like any other model, cannot resolve existing inconsistencies or ambiguities in data labels. Therefore, there remains a critical need for solutions that address both data quality and practical deployment considerations.
Further, integrating new technologies into existing clinical workflows often encounters resistance, as it necessitates adjustments to established diagnostic processes. So, there is a need to develop solutions that could be integrated into current practices, minimizing the burden on medical professionals to adopt new tools \cite{King_Williams_etal._2023}.

Existing solutions \cite{Goldsborough_Philps_etal._2024,Hörst_Rempe_etal._2024}, while addressing some aspects of these challenges, fall short in providing a comprehensive approach. To address the data quality and clinical deployment issues, we propose a multi-faceted solution that encompasses data refinement, model optimization, and integration with existing pathology tools (\hyperref[fig:fig1]{Figure 1}). The outcome is a lightweight cell segmentation and classification model that can be integrated into digital pathology workflows for practical clinical use.

\begin{figure}[h!]
    \centering
    \includegraphics[width=\textwidth, height=0.82\textheight, keepaspectratio]{images/Figure_1.pdf}
    \caption{Overview of the proposed solution, including 1) Data refinement using cross-relabeling, 2) Teacher model development and fine tuning, 3) Student model optimization with knowledge distillation and 4) Student model and QuPath integration}
    \label{fig:fig1}
\end{figure}
\clearpage

Our approach begins with preparing the data for the fine-tuning and training of the machine learning models. We create a refined dataset, acquired via cross-relabeling two cell-level datasets, enhancing annotation specificity and consistency of the labeled data. Subsequently, we create a cell segmentation and classification model based on the foundation model. We leverage the foundation model as a fixed encoder and fine-tune a decoder using the refined dataset to improve generalization across diverse tissue- and cell types.
To ensure that the model remains lightweight and deployable in a possibly resource-constrained environment, we employ knowledge distillation to approximate the functionality of the foundation model. Finally, to facilitate the practical application of our model in digital pathology workflows, we integrate it with the QuPath \cite{Bankhead_Loughrey_etal._2017} application. Each methodological component contributes to the overarching goal of enhancing model performance, generalizability, and usability in clinical settings.

The primary contributions of this paper are:
\begin{enumerate}
    \item \textit{Data labels refinement through cross-relabeling:}
    
    We propose a new method for refining labels of cell-level datasets through cross-relabeling. This method employs classification models to re-label broad and ambiguous instances, resulting in a more diverse dataset. Our evaluation demonstrates that these classification models achieve high accuracy on test subsets, indicating the reliability of the method for label refinement.

    \item \textit{Enhanced model performance via foundation models:}
    
    We employ a foundation model as a feature extractor for the cell segmentation and classification task. In comparison with training a CNN model from scratch, the foundation model backbone only needs fine-tuning, which significantly reduces training time, computational resources and data requirements. We show that using a foundation model encoder leads to better performance in cell segmentation and classification networks than using a CNN-based encoder. This improvement may enable the model to generalize more effectively across various tissue types and imaging methods.
    
    \item \textit{Model optimization through knowledge distillation:}
    
    We show that a smaller student model trained using knowledge distillation on the refined dataset obtained via our cross-relabeling approach from a foundation model achieves comparable performance in cell segmentation and quantification tasks. As a result, this model is more suitable for deployment in environments without high-performance computing resources.
    
    \item \textit{Integration with QuPath:}
    
    We integrate the distilled cell segmentation and classification model into QuPath, a widely used open-source digital pathology platform, to accelerate clinical adaptation by enabling pathologists to more easily incorporate advanced computational tools into their existing workflows.
\end{enumerate}

Through these methodological steps, we aim to bridge the gap between advanced machine learning techniques and practical clinical applications, making accurate and efficient digital pathology accessible in a broader range of healthcare settings.

\section{Refining Existing Datasets Using Cross-Relabeling}
To address the limitations of sparse and ambiguous labeling of cell-level datasets, we propose a generalizable cross-relabeling strategy that can be applied to any dataset containing broadly categorized or imprecisely labeled cell types. This approach involves training and subsequently leveraging classification models to refine broad categories into more specific or biologically relevant classes.
When applied to cell-level data, the methodology includes extracting individual cell images from the dataset patches, preprocessing these images to standardize the size and accommodate partial cells, and then training deep learning classifiers capable of distinguishing between the finer cell subtypes within the coarser categories. 
To illustrate our approach, we focus on the PanNuke \cite{Gamper_Koohbanani_etal._2020, Gamper_Koohbanani_etal._2019} and MoNuSAC \cite{Verma_Kumar_etal._2021} datasets that we have used to train models for cell quantification in our previous works \cite{Shvetsov_Grønnesby_etal._2022,Shvetsov_Sildnes_etal._2024}. We find that for better cell differentiation we have to introduce more granular labels. PanNuke includes a broad classification of "inflammatory" cells, encompassing lymphocytes, macrophages, and neutrophils. Each cell type differs significantly in structure, function, and clinical relevance. Conversely, MoNuSAC uses the label "epithelial" for a class that comprises both benign epithelial cells and malignant neoplastic cells. This practice makes it challenging to differentiate between benign and malignant epithelial cells in the dataset, which is a critical distinction when identifying tumor areas within tissue samples. To address these issues, we implement a cross-relabeling strategy as shown in \hyperref[fig:fig2]{Figure 2}. The key components are two classification models: one is trained on singular cell images from PanNuke data to classify the epithelial meta-class into epithelial and neoplastic classes. The other is trained on MoNuSAC to refine the inflammatory class into lymphocytes, neutrophils, and macrophages.

\begin{figure}[h!]
    \centering
    \includegraphics[width=\textwidth]{images/Figure_2.pdf}
    \caption{Refined dataset generation via cross relabeling}
    \label{fig:fig2}
\end{figure}

The refining approach consists of three consecutive steps. The first is the preprocessing step, in which we extract individual cells from both datasets (\hyperref[fig:fig3]{Figure 3}). The specifics of PanNuke and MoNuSAC patch preparation before cell preprocessing are provided in \hyperref[chap:S1]{Appendix S1}.

\begin{figure}[h!]
    \centering
    \includegraphics[width=\textwidth]{images/Figure_3.pdf}
    \caption{Cell instances preprocessing including (1) cell map extraction, (2) bounding box delineation, (3) adjusting cell boxes and (4) cropping and resizing of cell images}
    \label{fig:fig3}
\end{figure}

During preprocessing, we extract cell type maps from the ground truth label mask and calculate bounding boxes around each cell instance. To accommodate partial cells at patch borders, a common issue in cropped patch images, we employ mirror padding and extend the field of view of the cell label by 15 pixels to capture adjacent cells. We then crop and resize the identified regions to $64 \times 64$ pixels using bicubic interpolation.

The preprocessed PanNuke dataset comprises 68,031 neoplastic and 23,207 epithelial cell images, while MoNuSAC comprises  33,104 lymphocytes, 1,252 neutrophils, and 1,695 macrophages, which we subsequently use in training cell classification models and classifying the cell image data \hyperref[fig:S2]{Appendix Figure S2 (1)}. 

The next step is to train two distinct ResNet50-based classifiers tailored to address the specific labeling challenges inherent in each dataset. We use ResNet50 for classification models due to its proven effectiveness for image classification tasks in histopathology \cite{pan2022reviewmachinelearningapproaches}, and its compatibility with small images. For the PanNuke dataset, we design the classifier, trained on MoNuSAC data, to disaggregate the heterogeneous "inflammatory" cell category into distinct subtypes: lymphocytes, macrophages, and neutrophils. Similarly, for the MoNuSAC dataset, the classifier is trained on PanNuke data and distinguishes between benign and malignant epithelial cells within the overarching "epithelial" label. By applying these targeted classifiers to their respective datasets, we assign more specific labels to individual cell instances, thus enabling us to create a unified dataset.
To ensure a balanced representation of classes, we train both models on datasets that had been equalized to match the size of the least represented class. Thus, we obtain datasets comprising 23,207 samples per class for PanNuke and 1,252 samples per class for MoNuSAC data. Next, we partition both of them into training (70\%), validation (20\%), and testing (10\%) subsets. To mitigate the risk of overfitting, we use a single dropout layer with a rate of p=0.5 in both models and data augmentation using randomized color perturbations, rotation, and horizontal and vertical flipping. We employ AdamW optimizer and the cross-entropy loss function for the training criterion.

To evaluate the two trained models, we measure the classification accuracy on the respective test subsets. The accuracies on the test subset for both classifiers are presented in \hyperref[tab:1]{Table 1}. The PanNuke model achieves an average accuracy of 93.57\%, with higher accuracy for neoplastic cells (96.06\%) compared to epithelial cells (86.26\%). The confusion matrix in Figure A3.1 shows that the model predominantly distinguishes accurately between epithelial and neoplastic tissues, with a substantial number of correct classifications and relatively few misclassifications. The MoNuSAC model demonstrates an average accuracy of 98.92\%, excelling in classifying lymphocytes (99.67\%) and macrophages (94.12\%), with lower performance for neutrophils (85.71\%). The confusion matrix in Figure A3.2 shows that the model identifies lymphocytes and performs reasonably well with macrophages and neutrophils.

\begin{table}[h!]
\renewcommand{\arraystretch}{1.5}
  \centering
  \caption{Cell classification results for PanNuke and MoNuSAC trained models (CI 95\%).}
  \label{tab:1}
  \begin{tabular}{|l|c|c|}
   \hline
   %\rowcolor{gray!30}
    Accuracy               & PanNuke model              & MoNuSAC model              \\
    \hline
    Average      & 0.936 (0.931--0.941)         & 0.989 (0.986--0.993)        \\
    \hline
    Neoplastic   & 0.961 (0.956--0.965)         & -                          \\
    \hline
    Epithelial   & 0.863 (0.849--0.877)         & -                          \\
    \hline
    Lymphocytes  & -                          & 0.997 (0.995--0.999)        \\
    \hline
    Neutrophils  & -                          & 0.857 (0.796--0.918)        \\
    \hline
    Macrophages  & -                          & 0.941 (0.906--0.976)        \\
    \hline
  \end{tabular}
\end{table}

Finally, during the last step, we use the model trained on PanNuke data for epithelial cells in MoNuSAC and the model trained on MoNuSAC for the inflammatory cells class in PanNuke. Specifically, we use classifier models to relabel epithelial cells in MoNuSAC and inflammatory cells in PanNuke data. Then we combine cells with refined labels and the rest of the cells in both datasets to create a refined dataset (\hyperref[fig:S2]{Appendix Figure S2 (2)}). The process of relabeling cells and visualizing them on a patch is shown in \hyperref[fig:fig4]{Figure 4}. The cell counts in the refined dataset are provided in \hyperref[tab:S4]{Appendix Table S4}.

\begin{figure}[h!]
    \centering
    \includegraphics[width=\textwidth, height=0.42\textheight, keepaspectratio]{images/Figure_4.pdf}
    \caption{Cell relabeling procedure for epithelial and inflammatory cell classes}
    \label{fig:fig4}
\end{figure}

%\hfill

Relabeling and combining datasets have been explored in a prior study \cite{Parulekar_Kanwat_etal._2023}, where consecutive fine-tuning on multiple datasets was employed to account for hierarchical class label structures. While the method presented in \cite{Parulekar_Kanwat_etal._2023} is intuitive, it often lacks consistency and requires multiple fine-tuning runs, which can be cumbersome and time-consuming. 
In contrast, cross-relabeling simplifies this process by using specialized classification models tailored to each dataset's specific labeling challenges. This approach provides better transparency and produces a unified dataset encompassing seven distinct cell types across multiple tissue samples, enhancing data diversity for further model training or fine-tuning.

Despite these improvements, cross-relabeling does not entirely resolve issues related to poor labeling quality or the amount of labeled data. Specifically, our results show lower accuracies persist for underrepresented classes, such as macrophages, which may stem from a limited sample availability and intrinsic challenges in distinguishing these cells based solely on H\&E staining. Furthermore, while our method enhances label specificity, it relies on the initial quality of the broad labels; thus, any fundamental inaccuracies in the original annotations can propagate through the relabeling process. Addressing the overall problem of limited data labels may require integrating additional data sources or utilizing complementary immunohistochemical staining methods.
Although the reported performance metrics are obtained from evaluations on the native test sets of each dataset, it is important to note that the primary application of these classifiers is to perform cross-relabeling, where a model trained on one dataset (e.g., PanNuke) is applied to another (e.g., MoNuSAC) and vice versa. We acknowledge that a more systematic evaluation of cross-dataset generalization is needed and could be performed in future work.

Overall, the refined dataset produced by our approach can enhance the supervised training or fine-tuning of cell segmentation and classification models, especially those that utilize pre-trained foundation models to improve feature extraction robustness. In addition, these models can detect nuanced classes that enable researchers to conduct more detailed analyses of biological processes in computational pathology.

\section{Foundation models for robust cell segmentation and classification}

Accurate cell segmentation and classification in digital pathology are hindered by limited labeled data and the fact that conventional CNNs are unable to capture global contextual information due to their local receptive field constraints \cite{Gheflati_Rivaz_2022,Yang_Marcus_etal.}. Traditional approaches in cell quantification have predominantly relied on CNN encoders, such as ResNet50, given their proven effectiveness in semantic segmentation tasks \cite{Deshmane_2023,Graham_Vu_etal._2019,Mukasheva_Koishiyeva_etal._2024,Stringer_Wang_etal._2021}. However, approaches that include fine-tuning of pretrained CNNs, data augmentation, and stain normalization to partially increase data variability and address staining differences often fail to achieve the necessary generalization and robustness across diverse tissue types and staining conditions \cite{G._Wang_W._Li_etal._2018,Gao_Bagci_etal._2018,Karim_El_Khoury_Martin_Fockedey_etal._2021}.

To overcome these challenges, we leverage an encoder-decoder network that uses a foundation model as the encoder and a CNN upsampling decoder (\hyperref[fig:fig5]{Figure 5}) for simultaneous cell segmentation and classification in 2D patches extracted from WSIs. Foundation models with transformer-based architectures are viable alternatives to CNN-based encoders \cite{Shamshad_Khan_etal._2023,Sourget_2023}. They enable the creation of more advanced architectures that can decode or transform learned features more effectively \cite{Chen_Duan_etal._2023,Cheng_Misra_etal._2022,Xie_Wang_etal._2021}.

\begin{figure}[h!]
    \centering
    \includegraphics[width=\textwidth]{images/Figure_5.pdf}
    \caption{UNETR-like model with foundational model as backbone}
    \label{fig:fig5}
\end{figure}

By utilizing a transformer-based encoder, we incorporate global contextual information into the feature extraction process, which is a key advantage of such architectures \cite{Chen_Lu_etal._2021}. This foundation model integration facilitates accurate pixel-wise segmentation and classification without the need for extensive encoder training, thereby potentially improving generalization across varied cellular structures and tissue types.
In our implementation, we employ a modified UNETR \cite{Hatamizadeh_Tang_etal._2021} architecture that combines a vision transformer (ViT) \cite{Dosovitskiy_Beyer_etal._2021} encoder with a CNN-based decoder. The encoder utilizes the pretrained H-Optimus foundation model, which contains 1.1 billion parameters and is trained on over 500,000 H\&E stained WSIs \cite{Saillard_Jenatton_etal._2024}. We extract outputs from four evenly spaced transformer blocks $Z_i$, where $i \in [1, 14, 26, 38]$, to serve as residual connections for the CNN decoder. We select these blocks based on our observation that features from non-adjacent levels of the encoder lead to better overall performance on the test subset.

The CNN decoder upsamples the feature representations, acquired from the transformer blocks, to generate an intermediate vector that is handled by two task-specific layers that generate cell segmentation and classification masks. The first task-specific layer is the ‘Cellpose head’,  which is used to delineate cell instances. The layer generates horizontal and vertical gradient maps to form vector fields that are refined through gradient tracking in a post-processing step using the Cellpose algorithm \cite{Stringer_Wang_etal._2021}, known for its efficacy in cell segmentation tasks and generalizability across multiple domains \cite{Pachitariu_Stringer_2022,Stringer_Pachitariu_2024}. The second task-specific layer is the "Cell type head", which assigns labels to individual pixels. In the post-processing step, we determine the output classification label of each segmented cell instance by majority voting over the labeled pixels that comprise the cell in the segmentation map.

To evaluate model performance and measure the impact of adding a foundation model as backbone, we compare it to a ResNet50-based model. ResNet50 is a widely used solution for encoders in segmentation architectures in the medical domain \cite{Deshmane_2023,Graham_Vu_etal._2019,Mukasheva_Koishiyeva_etal._2024,Stringer_Wang_etal._2021}. For the H-Optimus-based model, we utilize frozen weights for the encoder and only fine-tune the decoder to take advantage of the extensive pre-training of the foundation model. For the ResNet50-based model we start with ImageNet \cite{Deng_Dong_etal.} weights and train both encoder and decoder parts. Hyperparameters for the training step are set to be identical, where possible, for comparable evaluation. 
For this evaluation, we deliberately use the PanNuke dataset to provide a standardized and controlled comparison between the H‑Optimus and ResNet50-based models (\hyperref[fig:S2]{Appendix Figure S2 (3)}). Specifically, we use two of the default PanNuke dataset splits (66\%) for training and validation, and reserve the third split (33\%) for testing.

To address the challenge of cell class imbalance in the PanNuke dataset, which is a common characteristic in most cell-level H\&E patch datasets, both models’ training processes employ a weighted loss function comprising cross-entropy and focal loss \cite{Lin_Goyal_etal._2018}. The focal loss component is adjusted with coefficients derived from each cell class' instance frequency, emphasizing learning from underrepresented classes and enhancing the model's sensitivity to rare but significant cellular patterns. The cross-entropy loss is augmented with spectral decoupling regularization \cite{Pezeshki_Kaba_etal._2021,Pohjonen_Stürenberg_etal._2022} and spatially varying label smoothing \cite{Islam_Glocker_2021}, which potentially stabilizes training and improves generalization in case of complex tissue morphologies. For optimization, we employ the \textit{AdamW} \cite{Loshchilov_Hutter_2019} to counter unbalanced class scenarios, with cosine annealing learning rate scheduler.

We utilize the scikit-learn library \cite{Van_der_Walt_Schönberger_etal._2014} and HoVer-Net \cite{Graham_Vu_etal._2019} implementations of $R^2$ (the coefficient of determination) and $PQ$ (panoptic quality) to evaluate our experiments. Complete mathematical formulations and detailed explanations of these metrics are provided in \hyperref[chap:S5]{Appendix S5}. To compute confidence intervals, we use nonparametric bootstrapping, where after calculating the metric on the full sample, we generated 1000 bootstrap replicates by resampling with replacement and then determined the 95\% confidence intervals as the 2.5th and 97.5th percentiles of the resulting empirical distribution.

%\hfill

The model comparisons are summarized in \hyperref[tab:2]{Table 2}. The H‑Optimus-based model achieves higher $R^2$ across all cell classes compared to the ResNet50-based model, which means that its predictions are more closely aligned with the PanNuke cell counts, indicating a stronger correlation with the observed data. Notably, the improvement of $R^2_{dead}$ may be an indicator of better global contextual representations provided by the foundation model backbone. In terms of segmentation and classification quality combined, measured by the PQ score, the H‑Optimus-based model demonstrates notable improvements across most cell classes. Overall, the average $R^2$ improved from 0.575 to 0.871, while the average $PQ$ score improved from 0.450 to 0.492, demonstrating better performance of the H-Optimus-based model.

\begin{table}[h!]
\renewcommand{\arraystretch}{1.5}
  \centering
  \caption{Cell quantification metrics for baseline and proposed models (CI 95\%).}
  \label{tab:2}
  \begin{tabular}{|l|c|c|}
    \hline
    %\rowcolor{gray!30}
    Metric             & Resnet50-based            & H-optimus-based              \\
    \hline
    $R^2_{neoplastic}$    & 0.681 (0.576--0.769)       & \textbf{0.941 (0.917--0.960)} \\
    \hline
    $R^2_{inflammatory}$  & 0.863 (0.778--0.903)       & \textbf{0.949 (0.918--0.966)} \\
    \hline
    $R^2_{connective}$    & 0.600 (0.488--0.698)       & 0.609 (0.436--0.772)          \\
    \hline
    $R^2_{dead}$          & 0.097 (-11.389--0.669)     & 0.925 (0.404--0.982)          \\
    \hline
    $R^2_{epithelial}$    & 0.635 (0.490--0.747)       & \textbf{0.930 (0.886--0.964)} \\
    \hline
    $PQ_{neoplastic}$       & 0.517 (0.499--0.535)       & \textbf{0.589 (0.575--0.604)} \\
    \hline
    $PQ_{inflammatory}$     & 0.455 (0.429--0.482)       & \textbf{0.528 (0.507--0.549)} \\
    \hline
    $PQ_{connective}$       & 0.416 (0.400--0.431)       & \textbf{0.451 (0.436--0.465)} \\
    \hline
    $PQ_{dead}$             & 0.374 (0.342--0.408)       & 0.292 (0.209--0.365)          \\
    \hline
    $PQ_{epithelial}$       & 0.488 (0.460--0.519)       & \textbf{0.599 (0.579--0.618)} \\
    \hline
  \end{tabular}
\end{table}

Our results  show that integrating the H‑Optimus foundation model within the UNETR architecture enhances the model's ability to segment and classify cells across diverse tissues from PanNuke data. The pretrained transformer encoder provides robust feature representations, resulting in higher average $R^2$ and $PQ$ scores compared to the CNN-based model. This leads to more reliable cell quantification and more accurate downstream analysis. Additionally, the streamlined fine-tuning process reduces computational overhead and training time, making the model more adaptable for new data.

Despite these advancements, the foundation model-based approach does not fully resolve all challenges related to cell segmentation and classification. We observe lower metric scores for underrepresented classes in the training data. Furthermore, foundation models typically encompass billions of parameters, resulting in substantial computational and memory requirements. It therefore poses challenges for deployment in resource-constrained environments, limiting their practical applicability in certain clinical settings.

\section{Model optimization via Knowledge Distillation}

To address the limitations posed by the extensive size of foundation models, we implement knowledge distillation — a model compression technique that leverages the teacher-student paradigm \cite{Hinton_Vinyals_etal._2015}. By training a smaller, more efficient student model to replicate the output of a larger, pre-trained teacher model, we retain performance while significantly reducing the model's complexity and resource requirements (\hyperref[fig:fig6]{Figure 6}).

\begin{figure}[h!]
    \centering
    \includegraphics[width=\textwidth, height=0.45\textheight, keepaspectratio]{images/Figure_6.pdf}
    \caption{Knowledge distillation framework for training a student model using a pre-trained teacher}
    \label{fig:fig6}
\end{figure}

We employ knowledge distillation to compress the H‑Optimus-based teacher model into a more efficient student model. The teacher model is the modified UNETR architecture with the H‑Optimus foundation model described in the previous chapter. The student model is based on a UNet architecture augmented with residual connections and incorporates a smaller ViT encoder with 9 million parameters \cite{Steiner_Kolesnikov_etal._2022,Wightman_2019}. 

First, we fine-tune the teacher model using the refined dataset from the cross-relabeling procedure (Section 2). Initially we train the decoder of the teacher model while keeping the encoder weights frozen. We split the refined dataset into train (70\%), validation (20\%) and test (10\%) subsets (\hyperref[fig:S2]{Appendix Figure S2 (4)}). During fine-tuning, we use the train and validation subsets, while leaving the test subset for model evaluation. We set the training procedure and model hyperparameters to be identical to those that were used to demonstrate the utility of foundation models for the simultaneous cell segmentation and classification task.

Next, we perform knowledge distillation from teacher to student using the refined dataset used to fine-tune the teacher model. The student model is trained to replicate the teacher model's outputs. We utilize a specialized loss function that aligns the student's predicted probability distribution with the teacher's, incorporating the teacher's class probability distribution derived from the output. Following the methodology of Hinton et al. \cite{Hinton_Vinyals_etal._2015}, we experiment with various hyperparameter settings for the temperature ($T$) and the balancing coefficients ($\alpha$ and $\beta$) in the loss function. We vary $T$ from 1 to 20 and adjust $\alpha$ and $\beta$ to balance the distillation and student losses. Through iterative tuning and evaluation, we identify that setting $T=14$, $\alpha=0.3$, and $\beta=0.7$ yields a configuration that converges and closely approximates the teacher model's performance during training.

Finally, we assess the performance of both models using the $R^2$ and $PQ$ (defined in \hyperref[chap:S5]{Appendix S5}) on the test set of the refined dataset (\hyperref[tab:3]{Table 3}). We observe that the 95\% confidence intervals overlap for most cell types, so we cannot claim statistically significant performance differences between the teacher and student models. One exception appears in the neoplastic class. The teacher model produces an $R^2$ of 0.919, while the student model shows an $R^2$ of 0.852. In addition, the student model achieves higher $PQ$ values for the neoplastic and connective classes, though the confidence intervals show overlap.

\begin{table}[h!]
\renewcommand{\arraystretch}{1.5}
  \centering
  \caption{Cell quantification metrics for teacher and distilled student models (CI 95\%).}
  \label{tab:3}
  \begin{tabular}{|l|c|c|}
    \hline
    %\rowcolor{gray!30}
    Metric & Teacher & Student \\
    \hline
    $R^2_{neoplastic}$    & \textbf{0.919} (0.898--0.939) & 0.852 (0.800--0.891) \\
    \hline
    $R^2_{lymphocyte}$    & 0.969 (0.956--0.977)         & 0.969 (0.956--0.978) \\
    \hline
    $R^2_{connective}$    & 0.694 (0.548--0.809)         & 0.618 (0.469--0.741) \\
    \hline
    $R^2_{dead}$          & 0.755 (0.400--0.908)         & 0.424 (0.100--0.731) \\
    \hline
    $R^2_{epithelial}$    & 0.922 (0.870--0.958)         & 0.843 (0.738--0.917) \\
    \hline
    $R^2_{macrophage}$    & 0.384 (-0.369--0.724)        & 0.704 (0.352--0.859) \\
    \hline
    $R^2_{neutrofil}$     & 0.854 (0.578--0.929)         & 0.833 (0.502--0.925) \\
    \hline
    $PQ_{neoplastic}$       & 0.581 (0.569--0.593)         & 0.601 (0.588--0.613) \\
    \hline
    $PQ_{lymphocyte}$       & 0.536 (0.520--0.553)         & 0.563 (0.544--0.579) \\
    \hline
    $PQ_{connective}$       & 0.436 (0.421--0.451)         & 0.457 (0.441--0.474) \\
    \hline
    $PQ_{dead}$             & 0.272 (0.235--0.315)         & 0.279 (0.201--0.369) \\
    \hline
    $PQ_{epithelial}$       & 0.522 (0.500--0.545)         & 0.530 (0.506--0.555) \\
    \hline
    $PQ_{macrophage}$       & 0.524 (0.459--0.588)         & 0.474 (0.405--0.543) \\
    \hline
    $PQ_{neutrofil}$        & 0.541 (0.490--0.592)         & 0.565 (0.522--0.607) \\
    \hline
  \end{tabular}
\end{table}


We further decompose the $PQ$ metric into its $SQ$ and $DQ$ components (\hyperref[tab:S6]{Appendix Table S6}). Both models produce nearly identical $SQ$ values, which indicates that they predict instance boundaries with similar precision. Although the student model shows some improvement in $DQ$ scores for certain classes, the confidence intervals overlap and do not confirm a statistically significant difference.

We observe that the student and teacher models yield comparable detection performance despite the student model using a much smaller and simpler architecture. A model with fewer parameters reduces the risk of overfitting when training data are scarce relative to the model’s complexity \cite{Farias_Ludermir_etal._2022}. The knowledge distillation process also encourages the student model to focus on the most generalizable detection features learned from the teacher. These factors enable the student model to achieve similar detection performance across different cell types.

Additionally, considering the model sizes reported in \hyperref[tab:4]{Table 4}, the distilled model achieves a significant reduction compared to the teacher model, with a 48-fold decrease in parameter count and a 5.5-fold reduction in on-disk size. In inference mode, the teacher model requires 16 GB of VRAM for a batch size of 32, while the distilled model only needs 3 GB of VRAM for the same batch size. These reductions make the distilled model significantly more practical for fine-tuning and deployment in resource-constrained environments.

\begin{table}[h!]
\renewcommand{\arraystretch}{1.5}
  \centering
  \caption{Parameter counts and size of teacher and distilled model}
  \label{tab:4}
  \adjustbox{max width=\textwidth}{%
  \begin{tabular}{|l|c|c|c|}
    \hline
    %\rowcolor{gray!30}
    Metric & H-optimus-based (Teacher) & mobileViT-based (Student) & Magnitude of difference \\
    \hline
    Parameters count       & 1,158,917,906   & \textbf{24,093,393}   & \textbf{48x}  \\
    \hline
    Estimated Total Size (MB) & 87,912       & \textbf{15,935}    & \textbf{5.5x} \\
    \hline
  \end{tabular}%
}
\end{table}

%\hfill

With recent advancements in complex network architectures and the use of pretrained encoders to achieve state-of-the-art performance \cite{Baumann_Dislich_etal._2024,Hörst_Rempe_etal._2024} in cell segmentation and classification tasks, model size, computational complexity, and processing times have increased. This limits the scalability and accessibility of these models. As we demonstrate, this may be mitigated using knowledge distillation. Studies in the field of natural language processing have demonstrated the efficacy of knowledge distillation in retaining the capabilities of the teacher model while achieving significant reductions in size and complexity \cite{Huangpu_Gao_2024,Sun_Yu_etal.}. 

We demonstrate the feasibility of knowledge distillation in digital pathology, specifically for cell segmentation and classification tasks. Moreover, we achieve this performance while also significantly reducing the parameter count. In addressing the challenge of knowledge transfer, we found that distillation from a transformer-based model to a smaller transformer is more straightforward than attempting to map transformer features to CNN blocks. In our experiments, using a CNN-based network as a student results in worse cell quantification performance due to the structural constraints of CNN feature space dimensions. 

Although our primary approach relies on a transformer-based student model that performs well, it can be further optimized to incorporate advantages from CNN architectures. For example, employing alternative techniques such as using ViT adapters \cite{Chen_Duan_etal._2023} or $1 \times 1$ convolutions to adjust feature map sizes may be beneficial for harnessing CNN advantages like enhanced local feature extraction. Moreover, if additional performance improvements are desired, the process can be further enhanced by applying supplementary knowledge distillation techniques, such as self-distillation \cite{Zhang_Song_etal._2019} or online distillation \cite{Houyon_Cioppa_etal._2023}.

Despite these promising results, further validation on independent datasets is necessary to fully understand the model's limitations. Underrepresented classes may pose challenges when addressing complex cases. Pathologists need to validate these models to adopt them in clinical settings. While the distilled models are smaller and more deployable, a technological gap persists because pathologists traditionally rely on established methods for inspecting WSIs and diagnosing diseases. Addressing the complexities involved in deploying models for inference and supporting pathologists in adopting new tools is essential for integrating these models into clinical workflows.

\section{Model integration with QuPath}
Digital pathology tools with graphical user interfaces are essential for visualizing and analyzing WSIs. To make our student model useful in clinical pathology workflows, it needs to be integrated into a tool that enables inspecting regions, creating annotations, and providing quantitative analyses of biomarkers. Therefore, we integrate the trained student model from the previous chapter into the QuPath open‑source platform \cite{Bankhead_Loughrey_etal._2017}. QuPath provides the required annotation, visualization, and analysis tools to interpret complex histological data, including workflows for cell segmentation, classification, and quantification (\hyperref[fig:fig7]{Figure 7}). 

\begin{figure}[h!]
    \centering
    \includegraphics[width=\textwidth]{images/Figure_7.pdf}
    \caption{Visualization of model-generated cell quantification annotations (left) and the corresponding unannotated slide (right) in QuPath}
    \label{fig:fig7}
\end{figure}

To identify the regions in a WSI critical for prognosticating tumor development, such as specific tumor areas or border regions without overlapping healthy tissue, the pathologist uses QuPath to outline these regions. Then, the pathologist initiates a cell segmentation and classification script through the QuPath interface for the selected regions. The resulting annotations and quantified cell information are then directly overlaid onto the WSI in the QuPath interface. Additional design and implementation details are in \hyperref[chap:S7]{Appendix S7}. 

Two common approaches for integrating deep learning models into QuPath are Java‑based native QuPath extensions \cite{Goldsborough_Philps_etal._2024} and the execution of RESTful API requests to a model server coupled with handling the response via an extension, as demonstrated in the application of cell segmentation models applied to immunofluorescence images \cite{Sugawara_2023}. While the community is actively working on these integration strategies, there is currently no universal solution that fully addresses all integration and performance requirements.

Extensions may offer better integration with QuPath, allowing slightly improved performance and more widespread usage of the built-in QuPath models, but they lack the flexibility to customize models and modify their behavior. For example, the newest version of QuPath includes models such as StarDist \cite{Weigert_Schmidt} and InstanSeg \cite{Goldsborough_Philps_etal._2024} that can perform cell segmentation. Both models pose limitations when applied to simultaneous cell segmentation and classification. StarDist performs well only on convex, round shapes by design, whereas some neoplastic, inflammatory, and connective cells exhibit complex and non-convex shapes. InstanSeg provides only semantic segmentation without assigning classes to the segmented cells.

%\hfill

In contrast, our approach offers an alternative integration strategy. It utilizes the paquo library to directly interact with QuPath’s internal application programming interface from within Python. This enables data exchange and processing without the need for intermediate conversion steps and provides greater control over model customization, retraining, and the incorporation of custom processing steps.

The integration of our custom model with QuPath underscores its potential to significantly enhance the diagnostic process by reducing the time burden on pathologists and enabling them to focus on more complex interpretative tasks using familiar software. Leveraging a tool that is already well-established among pathologists increases the likelihood of its adoption into daily clinical workflows. The quantitative data generated through the automated workflow is critical for both clinical decision-making and research, facilitating more accurate biomarker analysis, enabling robust statistical evaluations, and supporting hypothesis generation and testing. Additionally, by streamlining cell segmentation and classification, the tool enhances the scalability and reproducibility of pathological assessments, ultimately contributing to improved diagnostic accuracy and patient outcomes.

\section{Conclusion and future work}

In this study, we address critical challenges in digital pathology and tackle the usability and deployment issues of the developed models in standard computing environments without the need for high-performance computing systems. Our multi-faceted approach encompasses data refinement through cross-relabeling, leveraging foundation models for robust cell segmentation and classification, optimizing model performance via knowledge distillation, and integrating the optimized model into the QuPath software for practical application. This approach is used to construct a capable, versatile, and adjustable model for cell segmentation and classification, with enhanced performance and usability.

\begin{sloppypar}
While our approach shows potential in the field of computational pathology, certain limitations persist. 
For example, our implementation currently exhibits lower performance in detecting macrophages. 
This serves as an instance of the broader challenge of accurately identifying complex cell types. In order to address this issue, extending our approach to incorporate additional data sources, exploring alternative modeling approaches, and integrating other imaging modalities such as immunohistochemical staining may help improve detection accuracy. Moreover, although the distilled model reduces computational demands, integrating advanced deep learning models into clinical practice requires addressing technological gaps and potential resistance to adopting new tools within established diagnostic processes.
\end{sloppypar}

Future work could focus on several key areas to refine the proposed approach and facilitate its adoption in clinical environments. Enhancing the cell-relabeling process with additional datasets \cite{Graham_Jahanifar_etal._2021} could improve the representation of underrepresented cell types and enhance overall model performance. Also, incorporating additional data sources, such as multi-modal imaging or complementary staining methods, may address limitations related to cell type differentiation and class imbalance. Exploring other foundation models \cite{Vorontsov_Bozkurt_etal._2024,Zimmermann_Vorontsov_etal._2024} or introducing additional modalities \cite{Ding_Wagner_etal._2024,Vaidya_Zhang_etal._2025} may provide alternative architectures better suited to specific tasks or offer improved efficiency. Implementing more complex knowledge distillation techniques \cite{Houyon_Cioppa_etal._2023,Zhang_Song_etal._2019} could further optimize the model's performance and adaptability. Additionally, deeper integration with QuPath or other digital pathology software could provide pathologists more control over cell quantification analysis directly within the QuPath interface, thereby increasing accessibility and usability. Such enhancements would not only refine model performance but also ensure greater adaptability and scalability within various clinical environments. Finally, extensive validation of the model by pathologists and benchmarking against independent datasets are essential steps toward establishing the model's reliability and fostering confidence in its clinical utility.

\section*{Acknowledgments} 
This work was funded in part by the Research Council of Norway grant no. 309439 SFI Visual Intelligence, and the North Norwegian Health Authority grant no. HNF1521-20.

\bibliographystyle{IEEEtran}
\begin{sloppypar}
\begin{thebibliography}{99}

\bibitem{chaplot2020neural} Chaplot, Devendra Singh, et al. "Neural topological slam for visual navigation." Proceedings of the IEEE/CVF conference on computer vision and pattern recognition. 2020.

\bibitem{maksymets2021thda} Maksymets, Oleksandr, et al. "Thda: Treasure hunt data augmentation for semantic navigation." Proceedings of the IEEE/CVF International Conference on Computer Vision. 2021.

\bibitem{mezghan2022memory} Mezghan, Lina, et al. "Memory-augmented reinforcement learning for image-goal navigation." 2022 IEEE/RSJ International Conference on Intelligent Robots and Systems (IROS). IEEE, 2022.

\bibitem{al2022zero} Al-Halah, Ziad, Santhosh Kumar Ramakrishnan, and Kristen Grauman. "Zero experience required: Plug \& play modular transfer learning for semantic visual navigation." Proceedings of the IEEE/CVF Conference on Computer Vision and Pattern Recognition. 2022.

\bibitem{ye2021auxiliary} Ye, Joel, et al. "Auxiliary tasks and exploration enable objectgoal navigation." Proceedings of the IEEE/CVF international conference on computer vision. 2021.

\bibitem{chaplot2020object} Chaplot, Devendra Singh, et al. "Object goal navigation using goal-oriented semantic exploration." Advances in Neural Information Processing Systems 33 (2020)

\bibitem{ramakrishnan2022poni} Ramakrishnan, Santhosh Kumar, et al. "Poni: Potential functions for objectgoal navigation with interaction-free learning." Proceedings of the IEEE/CVF Conference on Computer Vision and Pattern Recognition. 2022.

\bibitem{ramrakhya2022habitat} Ramrakhya, Ram, et al. "Habitat-web: Learning embodied object-search strategies from human demonstrations at scale." Proceedings of the IEEE/CVF Conference on Computer Vision and Pattern Recognition. 2022.

\bibitem{mousavian2019visual} Mousavian, Arsalan, et al. "Visual representations for semantic target driven navigation." 2019 International Conference on Robotics and Automation (ICRA). IEEE, 2019.

\bibitem{dhariwal2021diffusion} Dhariwal, Prafulla, and Alexander Nichol. "Diffusion models beat gans on image synthesis." Advances in neural information processing systems 34 (2021)

\bibitem{ho2022classifier} Ho, Jonathan, and Tim Salimans. "Classifier-free diffusion guidance." arXiv preprint arXiv:2207.12598 (2022).

\bibitem{nichol2021glide} Nichol, Alex, et al. "Glide: Towards photorealistic image generation and editing with text-guided diffusion models." arXiv preprint arXiv:2112.10741 (2021)

\bibitem{brooks2023instructpix2pix} Brooks, Tim, Aleksander Holynski, and Alexei A. Efros. "Instructpix2pix: Learning to follow image editing instructions." Proceedings of the IEEE/CVF Conference on Computer Vision and Pattern Recognition. 2023.

\bibitem{fu2023guiding} Fu, Tsu-Jui, et al. "Guiding instruction-based image editing via multimodal large language models." arXiv preprint arXiv:2309.17102 (2023).

\bibitem{geng2024instructdiffusion} Geng, Zigang, et al. "Instructdiffusion: A generalist modeling interface for vision tasks." Proceedings of the IEEE/CVF Conference on Computer Vision and Pattern Recognition. 2024.

\bibitem{zhou2024minedreamer} Zhou, Enshen, et al. "Minedreamer: Learning to follow instructions via chain-of-imagination for simulated-world control." arXiv preprint arXiv:2403.12037 (2024).

\bibitem{zhou2023esc} Zhou, Kaiwen, et al. "Esc: Exploration with soft commonsense constraints for zero-shot object navigation." International Conference on Machine Learning. PMLR, 2023.

\bibitem{yu2023l3mvn} Yu, Bangguo, Hamidreza Kasaei, and Ming Cao. "L3mvn: Leveraging large language models for visual target navigation." 2023 IEEE/RSJ International Conference on Intelligent Robots and Systems (IROS). IEEE, 2023.

\bibitem{gadre2023cows} Gadre, Samir Yitzhak, et al. "Cows on pasture: Baselines and benchmarks for language-driven zero-shot object navigation." Proceedings of the IEEE/CVF Conference on Computer Vision and Pattern Recognition. 2023.

\bibitem{shah2023navigation} Shah, Dhruv, et al. "Navigation with large language models: Semantic guesswork as a heuristic for planning." Conference on Robot Learning. PMLR, 2023.

\bibitem{cai2024bridging} Cai, Wenzhe, et al. "Bridging zero-shot object navigation and foundation models through pixel-guided navigation skill." 2024 IEEE International Conference on Robotics and Automation (ICRA). IEEE, 2024.

\bibitem{yu2023co} Yu, Bangguo, Hamidreza Kasaei, and Ming Cao. "Co-NavGPT: Multi-robot cooperative visual semantic navigation using large language models." arXiv preprint arXiv:2310.07937 (2023).

\bibitem{wu2024voronav} Wu, Pengying, et al. "Voronav: Voronoi-based zero-shot object navigation with large language model." arXiv preprint arXiv:2401.02695 (2024).

\bibitem{qin2023mp5} Qin, Yiran, et al. "Mp5: A multi-modal open-ended embodied system in minecraft via active perception." arXiv preprint arXiv:2312.07472 (2023).

\bibitem{du2024learning} Du, Yilun, et al. "Learning universal policies via text-guided video generation." Advances in Neural Information Processing Systems 36 (2024).

\bibitem{ajay2024compositional} Ajay, Anurag, et al. "Compositional foundation models for hierarchical planning." Advances in Neural Information Processing Systems 36 (2024).

\bibitem{liang2024skilldiffuser} Liang, Zhixuan, et al. "Skilldiffuser: Interpretable hierarchical planning via skill abstractions in diffusion-based task execution." Proceedings of the IEEE/CVF Conference on Computer Vision and Pattern Recognition. 2024.

\bibitem{heusel2017gans} Heusel, Martin, et al. "Gans trained by a two time-scale update rule converge to a local nash equilibrium." Advances in neural information processing systems 30 (2017).

\bibitem{zhang2018unreasonable} Zhang, Richard, et al. "The unreasonable effectiveness of deep features as a perceptual metric." Proceedings of the IEEE conference on computer vision and pattern recognition. 2018.

\bibitem{brown2020language} Brown, Tom B. "Language models are few-shot learners." arXiv preprint arXiv:2005.14165 (2020).

\bibitem{podell2023sdxl} Podell, Dustin, et al. "Sdxl: Improving latent diffusion models for high-resolution image synthesis." arXiv preprint arXiv:2307.01952 (2023).

\bibitem{brohan2022rt} Brohan, Anthony, et al. "Rt-1: Robotics transformer for real-world control at scale." arXiv preprint arXiv:2212.06817 (2022).

\bibitem{brohan2023rt} Brohan, Anthony, et al. "Rt-2: Vision-language-action models transfer web knowledge to robotic control." arXiv preprint arXiv:2307.15818 (2023).

\bibitem{li2024manipllm} Li, Xiaoqi, et al. "Manipllm: Embodied multimodal large language model for object-centric robotic manipulation." Proceedings of the IEEE/CVF Conference on Computer Vision and Pattern Recognition. 2024.

\bibitem{shah2023vint} Shah, Dhruv, et al. "ViNT: A foundation model for visual navigation." arXiv preprint arXiv:2306.14846 (2023).

\bibitem{liu2024visual} Liu, Haotian, et al. "Visual instruction tuning." Advances in neural information processing systems 36 (2024).

\bibitem{hu2021lora} Hu, Edward J., et al. "Lora: Low-rank adaptation of large language models." arXiv preprint arXiv:2106.09685 (2021).

\bibitem{qin2023supfusion} Qin, Yiran, et al. "SupFusion: Supervised LiDAR-camera fusion for 3D object detection." Proceedings of the IEEE/CVF International Conference on Computer Vision. 2023.

\bibitem{qin2024worldsimbench} Qin, Yiran, et al. "Worldsimbench: Towards video generation models as world simulators." arXiv preprint arXiv:2410.18072 (2024).

\bibitem{yu2025gamefactory} Yu, Jiwen, et al. "GameFactory: Creating New Games with Generative Interactive Videos." arXiv preprint arXiv:2501.08325 (2025).

\bibitem{zhou2024code} Zhou, Enshen, et al. "Code-as-Monitor: Constraint-aware Visual Programming for Reactive and Proactive Robotic Failure Detection." arXiv preprint arXiv:2412.04455 (2024).

\bibitem{zhang2024ad} Zhang, Zaibin, et al. "AD-H: Autonomous Driving with Hierarchical Agents." arXiv preprint arXiv:2406.03474 (2024).

\bibitem{wang2024toward} Wang, Chaoqun, et al. "Toward Accurate Camera-based 3D Object Detection via Cascade Depth Estimation and Calibration." arXiv preprint arXiv:2402.04883 (2024).

\bibitem{huang2024story3d} Huang, Yuzhou, et al. "Story3d-agent: Exploring 3d storytelling visualization with large language models." arXiv preprint arXiv:2408.11801 (2024).

\bibitem{savinov2018semi} Savinov, Nikolay, Alexey Dosovitskiy, and Vladlen Koltun. "Semi-parametric topological memory for navigation." arXiv preprint arXiv:1803.00653 (2018).

\bibitem{majumdar2022zson} Majumdar, Arjun, et al. "Zson: Zero-shot object-goal navigation using multimodal goal embeddings." Advances in Neural Information Processing Systems 35 (2022): 32340-32352.

\bibitem{yadav2023offline} Yadav, Karmesh, et al. "Offline visual representation learning for embodied navigation." Workshop on Reincarnating Reinforcement Learning at ICLR 2023. 2023.

\bibitem{yadav2023ovrl} Yadav, Karmesh, et al. "Ovrl-v2: A simple state-of-art baseline for imagenav and objectnav." arXiv preprint arXiv:2303.07798 (2023).

\bibitem{sun2024fgprompt} Sun, Xinyu, et al. "FGPrompt: fine-grained goal prompting for image-goal navigation." Advances in Neural Information Processing Systems 36 (2024).

\bibitem{zhu2017target} Zhu, Yuke, et al. "Target-driven visual navigation in indoor scenes using deep reinforcement learning." 2017 IEEE international conference on robotics and automation (ICRA). IEEE, 2017.

\bibitem{koh2024generating} Koh, Jing Yu, Daniel Fried, and Russ R. Salakhutdinov. "Generating images with multimodal language models." Advances in Neural Information Processing Systems 36 (2024).

\bibitem{krantz2022instance} Krantz, Jacob, et al. "Instance-specific image goal navigation: Training embodied agents to find object instances." arXiv preprint arXiv:2211.15876 (2022).

\bibitem{schulman2017proximal} Schulman, John, et al. "Proximal policy optimization algorithms." arXiv preprint arXiv:1707.06347 (2017).

\bibitem{anderson2018evaluation} Anderson, Peter, et al. "On evaluation of embodied navigation agents." arXiv preprint arXiv:1807.06757 (2018).

\bibitem{lin2024navcot} Lin, Bingqian, et al. "NavCoT: Boosting LLM-Based Vision-and-Language Navigation via Learning Disentangled Reasoning." arXiv preprint arXiv:2403.07376 (2024).

\bibitem{NavGPT} Zhou, Gengze, Yicong Hong, and Qi Wu. "Navgpt: Explicit reasoning in vision-and-language navigation with large language models." Proceedings of the AAAI Conference on Artificial Intelligence.

\bibitem{hahn2021no} Hahn, Meera, et al. "No rl, no simulation: Learning to navigate without navigating." Advances in Neural Information Processing Systems 34 (2021): 26661-26673.

\bibitem{li2025t2isafety} Li, Lijun, et al. "T2ISafety: Benchmark for Assessing Fairness, Toxicity, and Privacy in Image Generation." arXiv preprint arXiv:2501.12612 (2025).

\bibitem{an2024agfsync} An, Jingkun, et al. "AGFSync: Leveraging AI-Generated Feedback for Preference Optimization in Text-to-Image Generation." arXiv preprint arXiv:2403.13352 (2024).


\end{thebibliography}
\end{sloppypar}

\clearpage
\beginsupplement
\section*{Appendix}
\renewcommand{\thesubsection}{S\arabic{subsection}}

\subsection{\label{chap:S1}PanNuke and MoNuSAC preprocessing}
The PanNuke dataset comprises a set of 7,901 RGB patches, each with dimensions of $256 \times 256$ pixels, which we set as the standard patch size for our analysis. In contrast, the MoNuSAC dataset encompasses 294 images of heterogeneous dimensions. To standardize the MoNuSAC images with our experiments, we implement a standardization protocol. Specifically, for images exceeding the dimensions of $256 \times 256$ pixels, we segment them into equal-sized patches and apply mirror padding to the remaining portions to avoid information loss at the peripherals. Patches with dimensions less than $128 \times 128$ pixels are excluded from the dataset due to the insufficient resolution to capture relevant cellular details. For patches where either dimension falls between 128 and 256 pixels, we employ upsampling to achieve the standard patch size. As a result, we obtain a total of 2,823 RGB patches derived from the MoNuSAC dataset for subsequent analysis. For additional details on the MoNuSAC data preparation process, refer to the source code \cite{Shvetsov_2025a}.
\clearpage

\subsection{\label{chap:S2}Data usage for the methodology}

\counterwithin{figure}{subsection}
\renewcommand{\thefigure}{S\arabic{subsection}}

\begin{figure}[h!]
    \centering
    \includegraphics[width=\textwidth, height=0.85\textheight, keepaspectratio]{images/A2.pdf}
    \caption{Overview of the methodology for cross-labeling, dataset refinement, and model comparison. (1) Cross-relabeling - training and testing cell classification models, (2) Cross-relabeling - using cell classification models to create refined dataset, (3) Fine-tuning and training models for comparison, (4) Student knowledge distillation with refined dataset}
    \label{fig:S2}
\end{figure}
\clearpage

\subsection{\label{chap:S3}Confusion matrices for classification models}
\counterwithin{figure}{subsection}
\renewcommand{\thefigure}{S\arabic{subsection}.\arabic{figure}}

\begin{figure}[h!]
    \centering
    \includegraphics[width=\textwidth, height=0.4\textheight, keepaspectratio]{images/A3_1.pdf}
    \caption{Confusion matrix for PanNuke trained model}
    \label{fig:S3.1}
\end{figure}

\begin{figure}[h!]
    \centering
    \includegraphics[width=\textwidth, height=0.4\textheight, keepaspectratio]{images/A3_2.pdf}
    \caption{Confusion matrix for MoNuSAC trained model}
    \label{fig:S3.2}
\end{figure}

\clearpage

\subsection{\label{chap:S4}Datasets cell counts}

\counterwithin{table}{subsection}
\renewcommand{\thetable}{S\arabic{subsection}}

\begin{table}[h!]
\renewcommand{\arraystretch}{2.0}
\centering
\caption{\label{tab:S4}Cell counts for PanNuke, MoNuSAC and refined datasets. Numbers in parentheses indicate preprocessed cell counts for cell classifier models training and testing.}
%\adjustbox{max width=\textwidth}{%
\begin{tabular}{|l|c|c|c|}
\hline
%\rowcolor{gray!30}
Cell type & PanNuke & MoNuSAC & Refined \\
\hline
Neoplastic & 77,403 (68,031) & - & 105,451 \\
\hline
Epithelial & 26,572 (23,207) & - & 29,926 \\
\hline
Epithelial (benign and malignant) & - & 31,402 & - \\
\hline
Inflammatory & 32,276 & - & - \\
\hline
Lymphocytes & - & 37,045 (33,104) & 65,275 \\
\hline
Neutrophils & - & 1,355 (1,252) & 3,833 \\
\hline
Macrophage & - & 1,842 (1,695) & 3,410 \\
\hline
Dead & 2,908 & - & 2,908 \\
\hline
Connective & 50,585 & - & 50,585 \\
\hline
\end{tabular}
%
%}
\end{table}



\clearpage

\subsection{\label{chap:S5}Definition of validation metrics}
\counterwithin{equation}{subsection}
\renewcommand{\theequation}{\arabic{equation}}

\subsubsection{\label{chap:S5.1}R\textsuperscript{2}}
The coefficient of determination, denoted as $R^2$, is a statistical measure that represents the proportion of variance in the dependent variable that is predictable from the independent variables. In the context of cell quantification in pathology, $R^2$ is used to assess how well the predicted quantities of different cell types in a patch align with the actual quantities observed in the ground truth data, with higher values representing more accurate quantification. $R^2$ is defined as
\begin{equation*}
R^2 = 1 - \frac{\sum_{i=1}^n (y_i - \hat{y}_i)^2}{\sum_{i=1}^n (y_i - \bar{y})^2},
\end{equation*}
where $y_i$ represents the actual number of cells of a specific type in the $i$-th image, $\hat{y}_i$ represents the predicted number of cells of that type in the $i$-th image, $\bar{y}$ is the mean of the actual numbers across all images, and $n$ is the total number of images in the dataset.

The $R^2$ metric has a range of $(-\infty, 1]$. An $R^2$ of 1 indicates perfect prediction, where all predicted values exactly match the actual values. An $R^2$ of 0 suggests that the model explains none of the variability of the response data around its mean. If $R^2$ is negative, it indicates that the model performs worse than a model that simply predicts the mean of the actual values for all observations.

\subsubsection{\label{chap:S5.2}PQ}
Panoptic Quality ($PQ$) is a comprehensive metric used to evaluate the performance of segmentation models in tasks that require both instance segmentation and classification. $PQ$ provides a single score that encapsulates both the detection accuracy (i.e., how many objects were correctly identified) and the segmentation quality (i.e., how accurately the objects' boundaries were delineated). This metric is particularly useful in multiclass scenarios where each pixel is classified into distinct categories, such as different cell types in pathology images.

$PQ$ is calculated as the product of two terms: Detection Quality ($DQ$) and Segmentation Quality ($SQ$). It can be expressed as
\begin{equation*}
PQ = DQ \cdot SQ,
\end{equation*}
where
\begin{equation*}
DQ = \frac{TP}{TP + 0.5\, FP + 0.5\, FN},
\end{equation*}
\begin{equation*}
SQ = \frac{\sum_{(p, g) \in \mathcal{M}} IoU(p, g)}{TP}.
\end{equation*}
In these formulas, $TP$ denotes the number of correctly matched instances between ground truth and prediction, $FP$ denotes the predicted instances that have no corresponding ground truth, $FN$ denotes the ground truth instances that were not detected, $IoU(p, g)$ is the Intersection over Union for a pair of matched instances $p$ (prediction) and $g$ (ground truth), and $\mathcal{M}$ is the set of matched pairs.

The $PQ$ metric is calculated for each class and is averaged across classes to provide a global performance measure.

The $PQ$ score has a range of $[0, 1.0]$, where a higher score indicates better performance in both detecting and segmenting the instances correctly. A $PQ$ of 1 signifies perfect identification and segmentation of all instances, whereas a $PQ$ of 0 indicates that no instances were correctly identified and segmented.

\clearpage

\subsection{\label{chap:S6}Segmentation and Detection quality metrics for teacher and student models}

\begin{table}[h!]
\renewcommand{\arraystretch}{2.0}
\centering
\caption{Segmentation and detection quality for student and teacher models (CI 95\%)}
\label{tab:S6}
%\adjustbox{max width=\textwidth}{%
\begin{tabular}{|l|c|c|}
\hline
%\rowcolor{gray!30}
Metric & Teacher & Student \\
\hline
$SQ_{neoplastic}$ & 0.819 (0.815--0.823) & 0.824 (0.819--0.828) \\
\hline
$SQ_{lymphocyte}$ & 0.795 (0.788--0.802) & 0.790 (0.783--0.796) \\
\hline
$SQ_{connective}$ & 0.770 (0.762--0.776) & 0.780 (0.772--0.786) \\
\hline
$SQ_{dead}$ & 0.659 (0.623--0.688) & 0.657 (0.624--0.695) \\
\hline
$SQ_{epithelial}$ & 0.780 (0.770--0.790) & 0.788 (0.779--0.797) \\
\hline
$SQ_{macrophage}$ & 0.788 (0.760--0.810) & 0.757 (0.730--0.783) \\
\hline
$SQ_{neutrofil}$ & 0.782 (0.761--0.801) & 0.775 (0.759--0.792) \\
\hline
$DQ_{neoplastic}$ & 0.706 (0.692--0.719) & 0.727 (0.712--0.741) \\
\hline
$DQ_{lymphocyte}$ & 0.675 (0.656--0.698) & 0.713 (0.691--0.734) \\
\hline
$DQ_{connective}$ & 0.566 (0.546--0.584) & 0.583 (0.565--0.602) \\
\hline
$DQ_{dead}$ & 0.410 (0.361--0.465) & 0.435 (0.306--0.561) \\
\hline
$DQ_{epithelial}$ & 0.668 (0.639--0.694) & 0.673 (0.644--0.702) \\
\hline
$DQ_{macrophage}$ & 0.657 (0.583--0.727) & 0.615 (0.531--0.703) \\
\hline
$DQ_{neutrofil}$ & 0.691 (0.625--0.753) & 0.729 (0.679--0.778) \\
\hline
\end{tabular}
%
%}
\end{table}

\clearpage

\subsection{\label{chap:S7}QuPath integration method}
We adopt an integration strategy leveraging the paquo \cite{Bayer_AG} library, a Python package that enables direct interaction with QuPath’s internal API, thereby facilitating seamless data exchange without intermediate conversion steps. The data processing pipeline (\hyperref[fig:S7]{Appendix Figure S7}) begins with the acquisition of WSIs and their associated annotations from QuPath, which are represented as Shapely \cite{Gillies_Wel_etal._2024} polygons. Utilizing paquo, we directly read, create, and modify these annotations and detections within a QuPath project in the Python environment. Images are then cropped using these polygons and processed by cell segmentation and classification models employing standard vision processing toolkits such as OpenCV, pyvips, and PyTorch. Additionally, QuPath employs Groovy scripts to initiate a Python process that starts the entire pipeline from QuPath graphical interface: fetching polygons, extracting images from them, and running deep learning model inference on the cropped images. 
The results are returned to QuPath, leveraging paquo's Python bindings to manipulate QuPath data while minimizing the computational overhead typically associated with cross-environment communication.

\counterwithin{figure}{subsection}
\renewcommand{\thefigure}{S\arabic{subsection}}

\begin{figure}[h!]
    \centering
    \includegraphics[width=\textwidth]{images/A7.pdf}
    \caption{QuPath integration workflow using Python environment}
    \label{fig:S7}
\end{figure}

Compared to traditional workflows that involve exporting annotations as GeoJSON, classifying them in Python, and reimporting them into QuPath, our approach offers several advantages. We eliminate the need to switch between programming languages, providing a cohesive and streamlined development process entirely within QuPath software and removing the necessity to use other tools. Meanwhile, we avoid storing annotations as intermediate JSON files unless required for external use or archiving. By conducting the entire inference and post-processing workflow within the Python environment, we leverage the power and flexibility of Python libraries for image processing and machine learning. This approach also enables adjustments to any set of labels and models, thereby improving its applicability.

%\hfill

The distilled model and QuPath integration code are packaged into a Docker container, enabling streamlined execution with the Docker engine. Detailed integration code and deployment instructions can be found in the GitHub repository \cite{Shvetsov_2025b}.

Despite these benefits, we acknowledge that the paquo library is a proof‑of‑concept project in its early development stage and has not been tested across all versions of QuPath.

\clearpage

\subsection{\label{chap:S8}Data and code availability statement}
All datasets, models, and code used in this study are publicly available and can be obtained from the repositories listed below. 
The PanNuke \cite{Gamper_Koohbanani_etal._2019} and MoNuSAC \cite{Verma_Kumar_etal._2021} datasets are publicly accessible, and download information along with detailed descriptions can be found in their respective articles. Preprocessing scripts for PanNuke and MoNuSAC data, as well as individual cell extraction scripts, are available on GitHub \cite{Shvetsov_2025a}. The H-Optimus foundation model used in our experiments can be downloaded from the HuggingFace repository \cite{hoptimus2024}, and model information is available on GitHub \cite{Saillard_Jenatton_etal._2024}. In addition, the integration code for QuPath and the distilled model packaged in a Docker container are provided in the repository \cite{Shvetsov_2025b}, and paquo Python library is available from the authors GitHub repository \cite{Bayer_AG}.
\clearpage

\end{document}


\end{document}
