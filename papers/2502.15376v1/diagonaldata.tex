There is a equivalance relation among samples $W_k$: $W_k \sim \Omega_k\Tilde{W}_x\Omega_k^\dagger$, $\forall \Omega_k\in U(n)$, $\forall k$. Namely the equivalent classes of fluxes is a subset of $(U(n)/\text{Ad})^{N_\text{site}}$.
By the isomorphism
\begin{equation}
    U(n)/\text{Ad}\cong U(1)^n/S_n,
\end{equation}
we could generate plaquettes $W_k$ as diagonal matrices, i.e. $W_k=\text{diag} \{e^{i\theta_k^1},\dots, e^{i\theta_k^N}\}$.\\
Notice that each link appears exactly twice in all plaquettes, once in itself, and once inversed. For example, $U_{k}^x$ appears in itself in $W_k$ and inversed in $W_{k-\hat{y}}$. Then we have:
$$\prod_k \det W_k = \prod_{\mu, k} \det U_k^\mu (\det U_k^\mu)^{-1}=1$$
Specifically, since $\sum_k\im(\log(\det W_k)) = \im(\log(\prod\det W_k)) \mod 2\pi$, the discrete Chern number $\Tilde{C}$ is an integer.
\begin{proposition}\label{chern}
    $\Tilde{C}=\frac{1}{2\pi}\sum_xF_x= n\in \Zb$. 
\end{proposition}
Then the necessary condition for a set of plaquettes to be generated from some links is:
\begin{equation}\prod_{k}\prod_{\lambda} e^{i\theta_k^\lambda}=e^{i\sum_k\sum_\lambda \theta_k^\lambda}=1,\label{necessary}
\end{equation}

On the other hand, given any $W_k$ that is diagonal per site, suppose it is generated by diagonal links $U_k^\mu=\text{diag} \{e^{i\tau_{k,\mu}^{1}},\dots, e^{i\tau_{k,\mu}^N}\}$. Then for each index $\lambda$ we have the following equations:
\begin{equation}
    \prod e^{i\tau_{k,x}^\lambda}e^{i\tau_{k+\hat{x},y}^\lambda}e^{-i\tau_{k+\hat{y},x}^\lambda}e^{-i\tau_{k,y}^\lambda}=1, \forall k
\end{equation}
This implies a necessary condition for $W_k$ to be generated from diagonal links is that, for any $\lambda$, $\sum \theta_k^\lambda=0$. We omit the subscript $\lambda$ for now. \\
Recall that $k$ is the flattened index of $(i,j)$, which could have the possible form $k=N_\text{site}i+j$. If we further flatten the index $(k,\mu)$ as $k$ for $\mu=x$, $k+N_\text{site}$ for $\mu=y$, then the equations become linear:
\begin{equation}
    \tau_{k}+\tau_{(k+N_\text{site}+1)\,\text{mod}\, 2N_\text{site}}-\tau_{(k+N_x)\, \text{mod}\, N_\text{site}}-\tau_{k+N_\text{site}}=\theta_{\hat{x}}, \forall k
\end{equation}
Which is just:
\begin{equation}
 \left(\begin{array}{cccccccccc}
      1&&-1&& & -1&1&&&\\
      &\ddots&&\ddots& &&\ddots&\ddots&&\\
      -1&&\ddots&&-1 &&&\ddots&\ddots&\\
      &\ddots&&\ddots& &&&&\ddots&1\\
      &&-1&&1& 1&&&&-1\\
 \end{array}\right)^T\left(\begin{array}{c}\tau_0\\\tau_1\\\vdots\\\tau_{2N_\text{site}-1}\end{array}\right)=
\left(\begin{array}{c}\theta_0\\\theta_1\\\vdots\\\theta_{N_\text{site}-1}\end{array}\right)\label{eq:linear_equation}
\end{equation}
The coefficient matrix has rank $N_\text{site}-1$, and it is solvable iff. $\sum_k\theta_{k}=0$, and that is exactly what the necessary condition specifies.
Therefore, the fluxes $W_k$ can be generated from diagonal $U_k^\mu$ if and only if
\begin{equation}
\forall \lambda,\  \prod_k  e^{i\theta_k^\gamma}=1.
\label{sufficient}
\end{equation}
This determines a submanifold $M'$ in $M=\{m\in U(1)^{N\times N_{\text{site}}}: m \text{ satisfies }\eqref{necessary}\}$ with codimension $N-1$. 
With the natural metric on $U(N)^{N_\text{site}}\supset M$,  defined as $d(g,h)=\|\psi_k^\lambda\|_2$,\ where $\psi_k^\lambda$ are phase angles of eigenvalues of $gh^{-1}$, $M'$ is a $\pi\sqrt{\frac{N}{N_\text{site}}}$-net of $M$.
For each channel $\lambda$, suppose $\sum_k \theta_{k}^\lambda = \phi_\lambda$, $\phi_\lambda \in [-\pi, \pi)$. Let the new $\theta$ be $\Tilde{\theta}_{k}^\gamma = \theta_{k}^\lambda + -\phi_k/N_\text{site}$. 
Then $$d(W, \Tilde{W})\leq \sqrt{\sum_{k,\lambda} \left(\frac{1}{N_\text{site}}\right)^2\phi_k^2}\leq \pi \sqrt{\frac{N}{N_\text{site}}}.$$As the number of sites gets larger (the grid gets more refined), the net gets denser. We can further extend the sufficient condition by considering the permutations, since the permutation matrices are also unitary and their actions on fluxes are adjoint.

We now propose the diagonal data generation scheme:
\begin{enumerate}
    \item Generate label $F_k\in [-\pi, \pi)$, such that $\sum F_k = 2\pi n$.
    \item If only zero samples: check if $\sum F_k=0$.
    \item For every $k$ but the last one, generate $(\phi_k)_x$ such that $\sum_\lambda\phi_k^\lambda=F_k$.
    \item For every $k$ but the last one, let $W_k$ be $\text{diag} \{e^{i\theta_k^1},\dots, e^{i\theta_k^N}\}$.
    \item Let the last $W_{\hat{k}}$ be $\prod_{k\neq \hat{k}} W_{k}^{-1}$.
\end{enumerate}
The last product will not cause confusion since diagonal matrix multiplication is commutative.

It could also go the other way around: generate the fluxes first, then find a solution to \eqref{eq:linear_equation} to get the links. This way, we could operate directly on the distribution of eigenvalues, thus customizing the data generation process. Furthermore, the diagonal dataset reduces the computation cost significantly for training.

For validation, we show in Figure~\ref{fig:diag-loss} the loss curves and in Table~\ref{tab:diag-acc} accuracies of evaluation on nontrivial, general (non-diagonal) datasets, of a training run on a diagonal, trivial dataset.
\begin{figure}
    \centering
    \includegraphics[width=1\linewidth]{Figures_arxiv/loss_curve_generalization.pdf}
    \caption{Global Loss and Standard Deviation Loss curve of the baseline model, trained on a diagonal, trivial dataset, to learn the Chern number on a $5^2$ grid, with $4$ filled bands.}
    \label{fig:diag-loss}
\end{figure}
 \begin{table}[]
        \centering
        \caption{Accuracy of the same run in Figure~\ref{fig:diag-loss}, evaluated on 
 non-diagonal, non-trivial data on a $5\times 5$ grid, with $4$ filled bands. }
        \label{tab:diag-acc}
        \vspace{10pt}
        \begin{tabular}{ccccc}
            \toprule
             Seeds&  No.1&No.2&No.3&No.4 \\
             \midrule
             Accuracy& $92.7\%$& $94.3\%$& $95.4\%$& $93.8\%$\\
             \bottomrule
        \end{tabular}

    \end{table}