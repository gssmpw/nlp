 \title{Supporting Information for ``A Neural Operator-Based Emulator for Regional Shallow Water Dynamics"}

\authors{Peter Rivera-Casillas\affil{1,2}\thanks{peter.g.rivera-casillas@erdc.dren.mil}, Sourav Dutta\affil{3}, Shukai Cai\affil{3}, Mark Loveland\affil{1,3},
Kamaljyoti Nath\affil{4},
Khemraj Shukla\affil{4},
Corey Trahan\affil{1}, Jonghyun Lee\affil{2}, Matthew Farthing\affil{5}, Clint Dawson\affil{3}}

\affiliation{1}{Information Technology Laboratory, U.S. Army Engineer Research and Development Center, MS, USA}
\affiliation{2}{Civil, Environmental and Construction Engineering \& Water Resources Research Center, University of Hawai’i at Manoa, HI, USA}
\affiliation{3}{Oden Institute for Computational Engineering and Sciences, The University of Texas at Austin, TX, USA}
\affiliation{4}{Division of Applied Mathematics, Brown University, Providence, RI, USA}
\affiliation{5}{Coastal and Hydraulics Laboratory, U.S. Army Engineer Research and Development Center, MS, USA}

\vspace{1cm}
\noindent\textbf{Contents of this file}

\begin{enumerate}
\item Figures S1 to S6

\end{enumerate}

\clearpage

\noindent\textbf{Introduction}

This supporting information document provides supplementary results to reinforce the findings presented in the results and discussion section of the main paper. Figures S1 to S3 show these comparisons of MITONet predictions and ADCIRC simulation output at the three sensors over a $10$-day forecast period for three unseen test values of the bottom friction coefficient, $r$. Figures S4 to S6 provide visualizations to compare the snapshots of MITONet predictions and the ADCIRC simulation outputs on day $60$, following 55 days of autoregressive predictions, over the full domain as well as zoomed-in over the inlet area, for three unseen test values of $r$. 

\clearpage

\begin{figure}
    \centering
    \includegraphics[width=\textwidth]{box_signals_0025_2.pdf}
    \caption{Columns 1 to 3 compare the predictions of MITONet with the simulation results of ADCIRC at three different sensor locations (see Fig. \ref{fig:sensors}) from day 50 to day 60 using test $r = 0.0025$ for $H$, $U$, and $V$ (Rows 1, 2 and 3)}
    \label{fig:box_signals1}
\end{figure}

\begin{figure}
    \centering
    \includegraphics[width=\textwidth]{box_signals_015_2.pdf}
    \caption{Columns 1 to 3 compare the predictions of MITONet with the simulation results of ADCIRC at three different sensor locations (see Fig. \ref{fig:sensors}) from day 50 to day 60 using test $r = 0.015$ for $H$, $U$, and $V$ (Rows 1, 2 and 3)}
    \label{fig:box_signals2}
\end{figure}

\begin{figure}
    \centering
    \includegraphics[width=\textwidth]{box_signals_1_2.pdf}
    \caption{Columns 1 to 3 compare the predictions of MITONet with the simulation results of ADCIRC at three different sensor locations (see Fig. \ref{fig:sensors}) from day 50 to day 60 using test $r = 0.1$ for $H$, $U$, and $V$ (Rows 1, 2 and 3)}
    \label{fig:box_signals3}
\end{figure}

\begin{figure}
    \centering
    \includegraphics[width=\textwidth]{snaps_0025.pdf}
    \caption{Full-domain and zoomed-in snapshots of ADCIRC (Rows 1 and 3) and MITONet (Rows 2 and 4) for variables $H$, $U$, and $V$ (Columns 1, 2, and 3). These results correspond to $r = 0.0025$ on day 60, following 55 days of autoregressive predictions.}
    \label{fig:rmse_snaps_0025_2}
\end{figure}

\begin{figure}
    \centering
    \includegraphics[width=\textwidth]{snaps_015.pdf}
    \caption{Full-domain and zoomed-in snapshots of ADCIRC (Rows 1 and 3) and MITONet (Rows 2 and 4) for variables $H$, $U$, and $V$ (Columns 1, 2, and 3). These results correspond to $r = 0.015$ on day 60, following 55 days of autoregressive predictions.}
    \label{fig:rmse_snaps_015_2}
\end{figure}

\begin{figure}
    \centering
    \includegraphics[width=\textwidth]{snaps_1.pdf}
    \caption{Full-domain and zoomed-in snapshots of ADCIRC (Rows 1 and 3) and MITONet (Rows 2 and 4) for variables $H$, $U$, and $V$ (Columns 1, 2, and 3). These results correspond to $r = 0.1$ on day 60, following 55 days of autoregressive predictions.}
    \label{fig:rmse_snaps_1_2}
\end{figure}
