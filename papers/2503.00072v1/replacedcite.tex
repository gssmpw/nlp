\section{Related work}
\label{sec:LR}
\subsection{MOOC recommendation}
% Various types of RSs have been used in the context of MOOCs such as collaborative, knowledge- and content-based filtering. Among these different types, CF RSs have been extensively applied, either individually or in combination with other RSs, since these types of RSs do not require item or user meta-data to come-up with recommendations and therefore can rely solely on the logs of learners____. Among numerous studies which applied CF for MOOC recommendations, nearest neighbors-____ and matrix factorization____-based approaches are the most popular ones. Few studies considered time-information in forming their MOOC RS. In____, learners' dwelling time in the MOOC page in edX is considered to provide personalized recommendations. In another study on the edX MOOCs, time-augmented Recurrent Neural Network (RNN) has been applied to consider the amount of time that a learner spent on each course pages in order to provide personalized recommendations. While this simple way of applying time-information enhance RSs performance. to our knowledge, time-to-event data, i.e., time to completion and time-to-dropout, have not been used to generate more informed recommendations.

Various types of RSs have been utilized in the context of MOOCs, including collaborative, knowledge-based, and content-based filtering. Among these, CF RSs have been extensively applied, either individually or in combination with other types, since they do not require item or user metadata to generate recommendations and can rely solely on learners' logs____. Numerous studies have applied CF for MOOC recommendations, with nearest neighbors____ and matrix factorization____ approaches being the most popular. Although time-related information provides relevant insights into learners' preferences and needs in MOOCs, few studies have incorporated this data into their MOOC recommendations. For instance, one study used learners' dwell time on the MOOC page in edX\footnote{\url{https://www.edx.org/}} to provide personalized recommendations____. Another similar study____ applied a time-augmented Recurrent Neural Network (RNN) to consider the amount of time learners spent on each course page for making personalized recommendations in edX. In our previous study____, we demonstrated that SA can improve the performance of a specific RS, namely Bayesian Personalized Ranking (BPR), when the predictions of a SA method, trained based on time to dropouts, are embedded in the BPR algorithm. While SA based on time-to-dropout improved the quality of recommendations, it has only used in a specific algorithm, namely BPR.

While using time information has proven to have a positive effect on RS performance, to our knowledge, time-to-event data, such as time-to-completion and time-to-dropout, have not been utilized to provide more informed recommendations, specifically in CF RSs.

\subsection{Time-to-event prediction in MOOCs}
The task of dropout prediction in the context of MOOCs has been mainly modeled as a classification task____. While in these studies the task was predicting the event of dropout, the authors ignored the time information in their predictions. SA can be used to incorporate the time information in modeling dropout in MOOCs and there are some promising examples in the literature. The authors in____ used SA, specifically Cox proportional hazards method, to model dropout risk in the context of MOOCs and unveil social and behavioral features impacting the outcome. Xie____ utilized survival analysis to examine the hazard function of dropout, employing the learner's course viewing duration on a course in MOOCs. Labrador et al.____ specified the fundamental factors attached to learners' dropout in an online MOOC platform using Cox proportional hazard regression. Wintermute et al.____ applied a Weibull survival function to model the certificate rates of learners in a MOOCs platform, assuming that learners “survive” in a course for a particular time before stochastically dropping out. In____ a more sophisticated SA deep learning approach was proposed to tackle volatility and sparsity of the data, that moderately outperformed the Cox model. Masci et al.____ applied shared frailty Cox models to model dropout of students who enrolled in engineering programs. 

% While SA has been applied to model dropout in MOOCs, to the best of our knowledge, it hasn't been used to model user preferences and needs in MOOC recommendations. The research gap that we aim to fill is to investigate the merits of SA to model time-to-dropout and time-to-completion and use it to enhance performance of typical CF RSs, which are the most common type of RSs, in the context of MOOCs. Although SA has been used to model dropout in MOOCs, it has not been employed to understand user preferences and needs within MOOC recommendations. Our research aims to bridge this gap by exploring the benefits of using SA to model time-to-dropout and time-to-completion in MOOCs. We will leverage this information to enhance the performance of CF RSs, which are the most prevalent type of RSs, in the context of MOOCs.

Although SA has been applied to model dropout in MOOCs, to the best of our knowledge, it hasn't been used to model user preferences and needs in MOOC recommendations. The research gap that we aim to fill is to investigate the merits of SA to model time-to-events in the context of MOOCs, specifically time-to-dropout and time-to-completion, and use it to enhance the performance of typical CF RSs.