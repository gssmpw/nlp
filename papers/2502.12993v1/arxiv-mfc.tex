\documentclass[11pt]{article}
\usepackage[bottom=1in,left=1in, right=1in, top=1in]{geometry}
\geometry{letterpaper}            
\usepackage{authblk}
\usepackage{algorithm}
\usepackage[noend]{algpseudocode}
\usepackage{tabularx}
\usepackage{enumitem}
\usepackage{graphicx}
\usepackage{subcaption}
\usepackage{amsmath}
\usepackage{amssymb}
\usepackage{amsfonts}
\usepackage{cite}
\usepackage{hyperref}
\usepackage{booktabs}
\usepackage[T1]{fontenc}

\hypersetup{
	colorlinks=true, %set true if you want colored links
	linktoc=all,     %set to all if you want both sections and subsections linked
	linkcolor=blue,  %choose some color if you want links to stand out
	citecolor = blue
}

\usepackage{titlesec}
\titlespacing{\paragraph}{0pt}{5pt}{5pt}


\newcommand{\CG}{\mathcal{G}\xspace}
\newcommand{\CV}{\mathcal{V}\xspace}
\newcommand{\CE}{\mathcal{E}\xspace}
\newcommand{\CA}{\mathcal{A}\xspace}
\newcommand{\CF}{\mathcal{F}\xspace}
\newcommand{\CR}{\mathcal{R}\xspace}
\newcommand{\CB}{\mathcal{B}\xspace}
\newcommand{\CX}{\mathcal{X}\xspace}
\newcommand{\CK}{\mathcal{K}\xspace}
\newcommand{\CM}{\mathcal{M}\xspace}
\newcommand{\CC}{\mathcal{C}\xspace}
\newcommand{\CL}{\mathcal{L}\xspace}
\newcommand{\CI}{\mathcal{I}\xspace}
\newcommand{\CQ}{\mathcal{Q}\xspace}
\newcommand{\CO}{\mathcal{O}\xspace}
\newcommand{\CP}{\mathcal{P}\xspace}
\newcommand{\CS}{\mathcal{S}\xspace}
\newcommand{\CT}{\mathcal{T}\xspace}
\newcommand{\CJ}{\mathcal{J}\xspace}
\usepackage[para]{footmisc}
\usepackage{subfig}
% \usepackage{subcaption}
% \usepackage{array}
% \usepackage{colortbl}


\newcommand{\mfc}{\textsc{Metric Forest Completion}}

%\title{Approximate Completion Strategies and Learning-Augmented Algorithms for Metric Minimum Spanning Trees}
%\title{Minimum Spanning Trees in Arbitrary Metric Spaces: Approximate Completion Strategies and Learning-Augmented Algorithms}
%\title{Spanning Tree Completion in Arbitrary Metric Spaces}
%\title{The metric minimum spanning tree completion problem}
% \title{Minimum Spanning Trees in Arbitrary Metrics: Approximate Tree Completion and Learning-Augmented Algorithms\thanks{Any acknowledgements here}}
\title{Approximate Tree Completion and Learning-Augmented Algorithms for Metric Minimum Spanning Trees\thanks{This work was performed under the auspices of the U.S. Department of Energy by Lawrence Livermore National Laboratory under Contract DE-AC52-07NA27344 (LLNL-ABS-XXXXXX), and was supported by LLNL LDRD project 24-ERD-024.}}

\author[1]{Nate Veldt}
\author[1]{Thomas Stanley}
\author[2]{Benjamin W. Priest}
\author[2]{Trevor Steil}
\author[2]{Keita Iwabuchi}
\author[2]{T.S.~Jayram}
\author[2]{Geoffrey Sanders}

\affil[1]{Department of Computer Science and Engineering, Texas A\&M University}
\affil[2]{Center for Applied Scientific Computing, Lawrence Livermore National Laboratory}
\date{}

\begin{document}
	
	\maketitle
\begin{abstract}
Finding a minimum spanning tree (MST) for $n$ points 
in an arbitrary metric space is a fundamental primitive for hierarchical clustering and many other ML tasks, but this takes $\Omega(n^2)$ time to even approximate.
We introduce a framework for metric MSTs that first (1) finds a forest of disconnected components using practical heuristics, and then (2) finds a small weight set of edges to connect disjoint components of the forest into a spanning tree. 
We prove that optimally solving the second step still takes $\Omega(n^2)$ time, but we provide a subquadratic 2.62-approximation algorithm. In the spirit of learning-augmented algorithms, we then show that if the forest found in step (1) overlaps with an optimal MST, we can approximate the original MST problem in subquadratic time, where the approximation factor depends on a measure of overlap. In practice, we find nearly optimal spanning trees for a wide range of metrics, while being orders of magnitude faster than exact algorithms.
\end{abstract}
	

\section{Introduction}
\label{sec:intro}

\begin{figure*}[tb]
    \centering
    \includegraphics[width=0.848\linewidth]{figs/circuitnn.pdf} 
    \caption{Illustration of differentiable CircuitNN. CircuitNN is designed based on differentiable NAND gates. After DAS is guided by PI and PO pairs of the truth table, CircuitNN can get the precise circuit architecture logic equivalent to the truth table.}
    \label{fig:circuitnn}
\end{figure*}

% 1. Describe the importance of logic synthesis
% 2. Existing Problems
% (a) Neural Architecture Search: Unstable, Predefined Setting, etc.
% (b) Circuit Generation: Probabilistic Model, Logic Equivalence

With the rapid advancement of technology, the scale of integrated circuits (ICs) has expanded exponentially. 
This expansion has introduced significant challenges in chip manufacturing, particularly concerning power and area metrics.
A primary objective in IC design is achieving the same circuit function with fewer transistors, thereby reducing power usage and area occupancy.

Logic synthesis~\cite{hachtel2005logicsynth}, a critical step in electronic design automation (EDA), transforms behavioral-level circuit designs into optimized gate-level circuits, ultimately yielding the final IC layout. 
The primary goal of logic synthesis is to identify the physical implementation with the fewest gates for a given circuit function. 
This task constitutes a challenging NP-hard combinatorial optimization problem. 
Current logic synthesis tools~\cite{brayton2010abc, wolf2013yosys} rely on human-designed heuristics, often leading to sub-optimal outcomes.

Differentiable architecture search (DAS) techniques~\cite{liu2018darts, chu2020darts} offer novel perspectives on addressing challenges in this problem.
Circuit functions can be represented through truth tables, which map binary inputs to their corresponding outputs. 
Truth tables provide a precise representation of input-output relationships, ensuring the design of functionally equivalent circuits.
Inspired by this, researchers~\cite{deepmind2024ai4sys, wang2024tnet} have begun exploring the application of DAS to synthesize circuits directly from truth tables.
Specifically, \citet{deepmind2024ai4sys} proposed CircuitNN, a framework that learns differentiable connection structures with logic gates, enabling the automatic generation of logic circuits from truth tables.
This approach significantly reduces the complexity of traditional circuit generation. 
Building on this, \citet{wang2024tnet} introduced T-Net, a triangle-shaped variant of CircuitNN, incorporating regularization techniques to enhance the efficiency of DAS.

Despite these advancements, several challenges remain. 
The computational complexity of DAS grows quadratically with the number of gates, posing scalability issues.
Although triangle-shaped architecture~\cite{wang2024tnet} partially mitigates this problem, redundancy persists. 
%Additionally, DAS is susceptible to converging to local optima, limiting the ability to search architectures that satisfy the given truth tables~\cite{liu2018darts}. 
%Furthermore, hyperparameters (network depth and layer width) require extensive searches, introducing complexity and prolonging the synthesis process. 
Additionally, DAS is susceptible to converging to local optima~\cite{liu2018darts} and hyperparameters (network depth and layer width) require extensive searches. 
The challenges arise from the vast search space in DAS. 
% Even with predefined settings for CircuitNN, finding a configuration that meets the truth table requires extensive trial and error during the DAS process. 
Intuitively, limiting the search space through predefined parameters (network depth, gates per layer, and connection probabilities) can significantly reduce the complexity.

Recent advances~\cite{openai2023gpt4, abramson2024alphafold3, esser2024sd3, li2024mar} in conditional generative models have demonstrated remarkable performance across language, vision, and graph generation tasks. 
Motivated by these developments, we propose a novel approach to circuit generation that generates preliminary circuit structures to guide DAS in generating refined circuits matching specified truth tables. 
Firstly, we introduce CircuitVQ, a tokenizer with a discrete codebook for circuit tokenization. 
Built upon our Circuit AutoEncoder framework~\cite{hou2022graphmae,li2023maskgae,wu2025mgvga}, CircuitVQ is trained through a circuit reconstruction task. 
Specifically, the CircuitVQ encoder encodes input circuits into discrete tokens using a learnable codebook, while the decoder reconstructs the circuit adjacency matrix based on these tokens.
Subsequently, the CircuitVQ encoder serves as a circuit tokenizer for CircuitAR pretraining, which employs a masked autoregressive modeling paradigm~\cite{chang2022maskgit, li2023mage}. 
In this process, the discrete codes function as supervision signals. 
After training, CircuitAR can generate discrete tokens progressively, which can be decoded into initial circuit structures by the decoder of the CircuitVQ. 
These prior insights can guide DAS in producing refined circuits that match the target truth tables precisely.

Our key contributions can be summarized as follows:
\begin{itemize}
\item We introduce CircuitVQ, a circuit tokenizer that facilitates graph autoregressive modeling for circuit generation, based on our Circuit AutoEncoder framework;
\item Develop CircuitAR, a model trained using masked autoregressive modeling, which generates initial circuit structures conditioned on given truth tables;
\item Propose a refinement framework that integrates differentiable architecture search to produce functionally equivalent circuits guided by target truth tables;
\item Comprehensive experiments demonstrating the scalability and capability emergence of our CircuitAR and the superior performance of the proposed circuit generation approach.
\end{itemize}

% Motivation
% (a) Diffusion (Vision, Graph), Autoregressive (Language, Vision)
% (b) Circuit Generation for Predefined Setting
% (c) Neural Architecture Search for Strict Logic Equivalence

% Contribution
% (a) Circuit Tokenizer (new transformer arch, training strategy)
% (b) CircuitAR (train and gen strategies, post-ar strategy)
% (c) Extensive Evaluation including BitD (Bit Distance) for Scalability

	
\section{Notation and Preliminaries}\label{sec:prelims}
This section fixes the notation and relevant notions for fair division of goods; the notation specific to division of chores is relegated to Section \ref{sec:chores}. 
 
\paragraph{Fair Division Instances.} A {fair division instance} is given by a tuple $\langle [n], [m], \{v_i\}_{i=1}^n \rangle$, where $[n]=\{1,2,.\dots,n\}$ is the set of $n\in\mathbb{Z}_+$ agents, $[m]=\{1,2, \dots, m\}$ the set of $m\in \mathbb{Z}_+$ indivisible goods, and for each agent $i\in[n]$, the set function $v_i: 2^{[m]} \to \mathbb{R}_+$ denotes the valuation of agent $i$ over subsets of goods. Specifically, $v_i(S) \in \mathbb{R}_+$ denotes the value that agent $i$ derives from the subset $S \subseteq [m]$ of goods. For subsets $S \subseteq [m]$ and $g \in [m]$, we will write $S + g$ to denote the union $S \cup \{ g\}$. 

A valuation $v_i$ is said to be monotone if the inclusion of goods into any subset does not decrease its value, under $v_i$, i.e., $v_i(S)\leq v_i(T)$ for every pair of subsets $S \subseteq T \subseteq[m]$. We will assume throughout that the agents' valuations are monotone and normalized: $v_i(\emptyset)=0$ for all agents $i$. 

We will additionally consider instances with identically ordered valuations. Here, we have an indexing of the $m$ goods, $\{g_1, \ldots g_m\}$, such that for each pair of goods $g_s, g_t$, with index $s < t$, and all agents $i \in [n]$, the inequality $v_i(S + g_s) \geq  v_i(S + g_t)$ holds for each subset $S \subset [m]$ that does not contain $g_s$ and $g_t$; see Example \ref{ex:sqrt-ordered} in Section \ref{subsec:additive-ordered}. 

This work also establishes improved bounds for the specific case of additive valuations. Recall that a valuation $v_i$ is said to be additive if, for every subset $S\subseteq[m]$ of goods, $v_i(S)=\sum_{g\in S} v_i(\{g\})$. We will use the shorthand $v_i(g)$---instead of $v_i(\{g\}) \in \mathbb{R}_+$---to denote agent $i$'s value for any good $g \in [m]$.  


\paragraph{Allocations and Multi-Allocations.} An allocation $\calB=(B_1,B_2,\ldots, B_n)$ of the goods among the $n$ agents is a partition of $[m]$ into $n$ pairwise disjoint subsets $B_1,\ldots, B_n \subseteq [m]$. Here, the subset of goods $B_i$ is assigned to agent $i \in [n]$ and is referred to as $i$'s bundle. In addition, write $\Pi_n([m])$ to denote the collection of all $n$-partitions of $[m]$. Note that for any allocation $\calB =(B_1,\ldots, B_n)$ we have, by definition, $\cup_{i=1}^n B_i = [m]$ and $B_i \cap B_j = \emptyset$, for all $i \neq j$, and hence $\calB \in \Pi_n([m])$.

 
A \textit{multi-allocation} is a tuple $\calA=(A_1,A_2\dots,A_n)$ of $n$ subsets, wherein subset $A_i \subseteq [m]$ denotes the bundle assigned to agent $i$. In contrast to allocations, in a multi-allocation, we do not require that the assigned bundles $A_i$ are pairwise disjoint and that they partition $[m]$.\footnote{Note that $A_i$s are still subsets of goods and not multisets.} Hence, in a multi-allocation, a single good may be present in multiple bundles or even in none. 

Though, when in a multi-allocation $\calA$, each good $g$ is assigned to exactly one agent, we refer to $\calA$ as an {\it exact allocation}; this is to emphasize that the bundles of such a multi-allocation do partition $[m]$. 

We associate with each bundle $A_i \subseteq [m]$ an $m$-dimensional characteristic vector $\rmchar(A_i) \in \{0,1\}^m$. For each good $g\in [m]$, the $g$th component of the characteristic vector---denoted as $\rmchar(A_i)_g$---is equal to one if $g \in A_i$, otherwise the $g$th component is zero. That is, 
\begin{align*}
\rmchar(A_i)_g \coloneqq \begin{cases}
    1 & \text{if } g\in A_i \\
    0 & \text{otherwise}.
\end{cases}
\end{align*}

For any multi-allocation $\calA=(A_1, \ldots, A_n)$, we will use $\chi^\calA \in \mathbb{Z}^m_+$ to denote the vector sum of the characteristic vectors of its bundles, $\chi^\calA \coloneqq \sum_{i=1}^n\rmchar(A_i)$. We will refer to $\chi^\calA$ as the \textit{characteristic vector} of the multi-allocation $\calA$. When there is no ambiguity, we will omit the notational dependence in the superscript and simply write $\chi$ for $\chi^\calA$.

Note that for any good $g\in [m]$ and multi-allocation $\calA$, the $g^{th}$ component of the characteristic vector $\chi^\calA_g$ is equal to the number of bundles in $\calA$ that contain $g$. Conceptually, we think of this setting as one in which $\chi^A_g$ identical copies of the good $g$ are assigned among different agents. 

Write $\ellone{\chi^\calA}$ and $\ellinfty{\chi^\calA}$ to denote the $\ell_1$ and $\ell_\infty$ norm, respectively, of the characteristic vector. Hence,  $\ellone{\chi^\calA} = \sum_{g=1}^m \chi^\calA_g$ and $\ellinfty{\chi^\calA} = \max_{g\in[m]} \chi^\calA_g$. It is relevant to note that $\ellone{\chi^\calA}$ captures the total number of goods, with copies, assigned among the agents,  $\ellone{\chi^\calA} = \sum_{i=1}^n |A_i|$. Further, $\ellinfty{\chi^\calA}$ captures the maximum number of copies of any one good $g$ assigned under $\calA$.

In particular, if $\calA$ is an {\it exact} allocation, then $\chi^\calA$ is equal to the all-ones vector and we have $\ellone{\chi^\calA} =m$ and $\ellinfty{\chi^\calA} =1$.
 
\noindent
The shared-based fairness criterion we study in this work is defined using maximin shares; these shares are defined next.
\begin{definition}[Maximin Share (MMS)]\label{def:mms}
    Given any fair division instance $\langle [n], [m], \{v_i\}_{i=1}^n \rangle$ with goods, the {maximin share}, $\mu_i \in \mathbb{R}_+$, of each agent $i \in [n]$ is defined as 
    \begin{align*}
    \mu_i \coloneqq  \max_{(X_1,\dots, X_n) \in \Pi_n([m])} \ \ \min_{j\in[n]} v_i(X_{j}).
    \end{align*}
Further, for each agent $i$, let $\calM^i=(M^i_1, M^i_2, \ldots, M^i_n) \in \Pi_n([m])$ denote an {MMS-inducing partition}:
\begin{align*}
\calM^i \in \argmax_{(X_1,\dots, X_n) \in \Pi_n([m])} \ \ \min_{j\in[n]} v_i(X_{j})
\end{align*}
\end{definition}

Note that in Definition \ref{def:mms} the maximum is taken over all $n$-partitions of $[m]$. Also, by definition, the partition $\calM^i =(M^i_1, \ldots, M^i_n)$ satisfies $v_i(M^i_j) \geq \mu_i$, for each index $j \in [n]$. 

\paragraph{Fair Multi-Allocations.} A multi-allocation $\calA=(A_1,\dots,A_n)$ is said to be an \emph{MMS multi-allocation} (i.e., it is deemed to be fair) if under it each agent receives a bundle of value at least its maximin share:  $v_i(A_i)\geq \mu_i$ for all agents $i \in [n]$.
 
To establish existential guarantees for MMS multi-allocations $\calA$, we will assume that, for all the agents, we are given the MMS-inducing partitions $\calM^i$, which in turn are guaranteed to exist (see Definition \ref{def:mms}).  

\section{Metric Forest Completion}
\label{sec:mfc}
We now formalize our \mfc{} (MFC) framework\footnote{This is distinct from two other MST-related concepts that also use the acronym MFC; see Section~\ref{sec:related} for details.}
which assumes access to an initial forest that is then grown into a full spanning tree.

 \subsection{Formalizing the MFC problem}
\label{sec:definemfc}
As a starting point for the metric MST problem on $(\mathcal{X},d)$, we assume access to a partitioning $\mathcal{P} = \{P_1, P_2, \hdots , P_t\}$ where $\mathcal{X} = \bigcup_{i = 1}^t P_i$ and $P_i \cap P_j = \emptyset$ for $i \neq j$. For each component $P_i$ we have a partition spanning tree $T_i = (P_i, E_{T_i})$ for that component.
% , which may or may not be an optimal MST for $P_i$. 
See Figure~\ref{fig:warmstart} for an illustration. Let $G_t = (\mathcal{X}, E_t)$ represent the union of these trees, which has the same node set as $G_\mathcal{X}$, and edge set $E_t = \bigcup_{i = 1}^t E_{T_i}$. Each set $P_i$ for $i \in [t]$ defines a group of points in $\mathcal{X}$ as well as a connected component of $G_t$. We refer to this as the \emph{initial forest} for MFC. To provide intuition, the initial forest can be viewed as a proxy for the forest obtained at an intermediate step of Kruskal's or Boruvka's algorithm (see Figure~\ref{fig:partial_MST_1}). While this serves as a useful analogy, we stress that the partitioning will typically be obtained using much cheaper methods and will not satisfy any formal approximation guarantees. Section~\ref{sec:initialforest} covers practical considerations about strategies, runtimes, and quality measures for an initial forest. For now we simply assume it is given as a ``good enough'' starting point, that will ideally overlap, even if not perfectly, with some true MST (see Figure~\ref{fig:overlap}). 
%The goal is to efficiently grow or \emph{complete} this partial tree into a spanning tree for $G_\mathcal{X}$. 

%Our theoretical analysis requires very few assumptions about the partitioning $\mathcal{P}$ and partition spanning trees (initial forest) $\{T_i \colon i = 1,2, \hdots t\}$ that are given as input to the \mfc{} problem. 
%In particular, we need not assume that $T_i$ is a minimum spanning tree or even a good spanning tree of $P_i$. Similarly, the components are not required to be sets of points that are close together in the metric space in order for the problem to be well-defined. Nevertheless, \textit{in practice} the hope is that $\mathcal{P}$ identifies groups of nearby points in $\mathcal{X}$ and that $T_i$ is a reasonably good spanning tree for $P_i$ for each $i \in [t] = \{1,2, \hdots, t\}$. Appendix~\ref{sec:initialsubtree} covers two practical strategies and corresponding runtimes for finding an initial forest. In summary these are:
%
%\textit{Strategy 1: $k$-centering.} Apply a fast $k$-centering heuristic to $\mathcal{X}$ to form $\mathcal{P}$, e.g., using a simple 2-approximation~\cite{gonzalez1985clustering} or fast distributed methods~\cite{malkomes2015fast,mcclintock2016efficient}.
%
%\textit{Strategy 2: $k$-NN graph.}  Compute an approximate $k$-nearest neighbor graph of $\mathcal{X}$ for a reasonably small $k$, e.g., via the scalable $k$-NN descent algorithm~\cite{dong2011efficient} or its distributed generalization~\cite{iwabuchi2023towards}.
%
%Similar but more restrictive strategies have been used by previous heuristics for Euclidean MSTs (see Appendix~\ref{sec:initialsubtree}).  

%\textbf{Defining the MFC problem.}
% Given an initial forest, the goal of 
\mfc{} seeks to connect the initial forest into a spanning tree for $G_\mathcal{X}$. Let $P(x) \in \mathcal{P}$ denote the component that $x \in \mathcal{X}$ belongs to. The set of inter-component edges is
\begin{equation*}
	\label{eq:intercomponent}
	\mathcal{I} = \{ (x, y) \in \mathcal{X} \times \mathcal{X} \colon P(x) \neq P(y) \}.
\end{equation*}
We wish to find a minimum weight set of edges $M \subseteq \mathcal{I}$ so that $M\cup E_t$ defines a connected graph on $\mathcal{X}$. If $M$ satisfies these constraints we say it is a valid \emph{completion set} and that $M$ \emph{completes} $\mathcal{P}$. 
The MFC problem can then be written as
\begin{equation}
	\label{eq:mfc}
	\begin{array}{ll}
		\minimize & w_\mathcal{X}(M) + w_\mathcal{X}(E_t)\\
		\text{subject to} & M \emph{ completes } \mathcal{P}.
	\end{array}
\end{equation}
Let $M^*$ denote an optimal completion set. The graph $T^* = (\mathcal{X}, E_t \cup M^*)$ is then guaranteed to be a tree (see Figure~\ref{fig:opt_mfc}); if not we could remove edges to decrease the weight while still spanning $\mathcal{X}$. If the initial forest $G_t$ is in fact contained in some optimal spanning tree of $G_\mathcal{X}$ (which would be the case if it were obtained by running a few iterations of Kruskal's or Boruvka's algorithm), then solving MFC would produce an MST of $G_\mathcal{X}$. In practice this will typically not be the case, but the problem remains well-defined regardless of any assumptions about the quality of the initial forest.

\begin{figure}[t]
	\centering
	\begin{subfigure}[b]{0.32\textwidth}
		\centering
		\includegraphics[width=\textwidth]{figures/true_MST.pdf}
		\caption{True metric MST}
		\label{fig:truemst}
	\end{subfigure}
	\begin{subfigure}[b]{0.32\textwidth}
		\centering
		\includegraphics[width=\textwidth]{figures/warm-start.pdf}
		\caption{Partial spanning tree ($\mathcal{P}$; $\{T_i\})$}
		\label{fig:warmstart}
	\end{subfigure}
	\begin{subfigure}[b]{0.32\textwidth}
		\centering
		\includegraphics[width=\textwidth]{figures/true_MST_clusters.pdf}
		\caption{True MST overlap with $\mathcal{P}$}
		\label{fig:overlap}
	\end{subfigure}
	\begin{subfigure}[b]{0.32\textwidth}
		\centering
		\includegraphics[width=\textwidth]{figures/opt_mstc.pdf}
		\caption{MFC solution ($M^*$ in orange)}
		\label{fig:opt_mfc}
	\end{subfigure}
	\begin{subfigure}[b]{0.32\textwidth}
		\centering
		\includegraphics[width=\textwidth]{figures/componentgraph.pdf}
		\caption{Coarsened graph $G_\mathcal{P}$ w.r.t.\ $w^*$}
		\label{fig:coarsenedgraph}
	\end{subfigure}
	\begin{subfigure}[b]{0.32\textwidth}
		\centering
		\includegraphics[width=\textwidth]{figures/component_mst.pdf}
		\caption{MST of $G_\mathcal{P}$ w.r.t.\ $w^*$}
		\label{fig:mstcoarsened}
	\end{subfigure}
	\caption{(a) We display an optimal metric MST for a toy example with $|\mathcal{X}| = 75$ points. Our framework and algorithm apply to general metric spaces, but for visualization purposes our figures focus on 2-dimensional Euclidean space. (b) The \mfc{} problem is given a partitioning $\mathcal{P}$ and spanning trees $\{T_i\}$ for components of the partition. For this illustration we used a $k$-means algorithm with $k = 5$ computed optimal spanning trees of components using the naive approach. (c)~The true MST overlaps significantly with the initial partial spanning tree, but its induced subgraph on each component is not necessarily connected. For this example, the $\gamma$-overlap (see Section~\ref{sec:learningaugmented}) is $\gamma \leq 1.12$. (d)~The optimal completion set $M^*$ is shown in orange; combining it with the spanning trees of the partial spanning tree produces a spanning tree for all of $\mathcal{X}$. (e)~The coarsened graph $G_\mathcal{P}$ has a node $v_i$ for each component $P_i \in \mathcal{P}$. Solving $O(t^2)$ bichromatic closest pair problems identifies the closest pair of points between each pair of clusters, defining an optimal weight function $w^*$ on $G_\mathcal{P}$. (f)~Finding the minimum-weight completion set $M^*$ amounts to finding the MST of $G_\mathcal{P}$ with respect to weight function $w^*$. }
	\label{fig:three_figures}
\end{figure}


The objective function in Eq.~\eqref{eq:mfc} includes the weight of the initial forest $w_\mathcal{X}(E_t)$. Although this is constant with respect to $M$ and does not affect optimal solutions, there are several reasons to incorporate this term explicitly. Most importantly, our ultimate goal is to obtain a good spanning tree for all of $G_\mathcal{X}$, and thus the weight of the full spanning tree (i.e., the objective in Eq.~\ref{eq:mfc}) is a more natural measure. Considering the weight of the full spanning tree also makes more sense in the context of our learning-augmented algorithm analysis, where the goal is to approximate the original metric MST problem on $G_\mathcal{X}$, under different assumptions about the initial forest. We note finally that excluding the term $w_\mathcal{X}(E_t)$ rules out the possibility of any meaningful approximation results. 
We prove the following result using a reduction from BCP to MFC, combined with a slight variation of a simple lower bound for monochromatic closest pair that was shown in Section 9 of Indyk~\cite{indyk1999sublinear}.
\begin{theorem}
\label{thm:hard}
Every optimal algorithm for MFC has $\Omega(n^2)$ query complexity. Furthermore, for any multiplicative factor $p \geq 1$ (not necessarily a constant), any algorithm that finds a set $M \subseteq \mathcal{I}$ that is feasible for~\eqref{eq:mfc} and satisfies $w_\mathcal{X}(M) \leq p \cdot w_{\mathcal{X}}(M^*)$ requires $\Omega(n^2)$ queries.
\end{theorem}
\begin{proof}
	Let $\mathcal{X}$ be a set of $n$ points that are partitioned into two sets $P_1$ and $P_2$ of size $n/2$. Define a distance function $d$ such that $d(a,b) = 1$ for a randomly chosen pair $(a,b) \in P_1 \times P_2$, and such that $d(x,y) = 2p$ for all other pairs $(x,y) \in {\mathcal{X} \choose 2} \backslash \{(a,b)\}$. Note that this $d$ is a metric. The MFC problem on this instance is identical to solving BCP on $P_1$ and $P_2$. The unique optimal solution is exactly $M^* = (a,b)$, and no other choice of $M \subseteq \mathcal{I}$ comes within a factor $p$ of this solution. Thus, any $p$-approximation algorithm must find the pair $(a,b)$ with distance 1 among a collection of $\Omega(n^2)$ pairs, where all other pairs have distance $2p$. This requires $\Omega(n^2)$ queries.
\end{proof}
Theorem~\ref{thm:hard} shows that it is impossible in general to find optimal solutions for MFC, or multiplicative approximations for $w_\mathcal{X}(M^*)$, in $o(n^2)$ time. However, this does not rule out the possibility of approximating the more relevant objective $w_\mathcal{X}(M) + w_\mathcal{X}(E_t)$.

\paragraph{The MFC coarsened graph.} The  MFC problem is equivalent to finding a minimum spanning tree in a \textit{coarsened graph} $G_\mathcal{P} = (V_\mathcal{P}, E_\mathcal{P})$ with node set $V_\mathcal{P} = \{v_1, v_2, \hdots, v_t\}$ where $v_i$ represents component $P_i$. We refer to $v_i$ as the $i$th \textit{component node}. This graph is complete: $E_\mathcal{P}$ includes all pairs of component nodes. Figure~\ref{fig:coarsenedgraph} provides an illustration of the coarsened graph. Finding an MFC solution $M^* \subseteq \mathcal{I}$ is equivalent to finding an MST in $G_\mathcal{P}$ (see Figure~\ref{fig:mstcoarsened}) with respect to the weight function $w^* \colon E_\mathcal{P} \rightarrow \mathbb{R}^+$ defined for every $i, j \in \{1,2, \hdots, t\}$ by
\begin{equation}
	\label{eq:wstar}
	w^*_{ij} = w^*(v_i, v_j) = d(P_i, P_j) = \min_{x \in P_i, y\in P_j} d(x,y).
	%, \quad \text{ for every $P_i, P_j \in \mathcal{P} \times \mathcal{P}.$}
\end{equation}
Computing $w_{ij}^*$ exactly requires solving a bichromatic closest pair problem over sets $P_i$ and $P_j$. A straightforward approach for computing $w_{ij}^*$ is to check all $|P_i|\cdot |P_j|$ pairs of points in $P_i \times P_j$. The number of distance queries needed to apply this simple strategy to form all of $w^*$ is
\begin{equation*} 
	\frac{1}{2}\sum_{i = 1}^t |P_i| \cdot (n - |P_i|) = \frac{n^2}{2} - \frac{1}{2} \sum_{i = 1}^t |P_i|^2.
\end{equation*}
In a worst-case scenario where component sizes are balanced, we would need
$\Omega\left({t \choose 2} \frac{n}{t} \frac{n}{t}\right) = \Omega(n^2)$ 
queries, which is not surprising given Theorem~\ref{thm:hard}. Nevertheless, this notion of a coarsened graph will be very useful in developing approximation algorithms for MFC.

\subsection{Computing an initial forest}
\label{sec:initialforest}
Our theoretical analysis requires very few assumptions about the partitioning $\mathcal{P}$ and partition spanning trees $\{T_i \colon i = 1,2, \hdots t\}$ that are given as input to the \mfc{} problem. In particular, we need not assume that $T_i$ is a minimum spanning tree or even a good spanning tree of $P_i$. Similarly, the components are not required to be sets of points that are close together in the metric space in order for the problem to be well-defined. Nevertheless, \textit{in practice} the hope is that $\mathcal{P}$ identifies groups of nearby points in $\mathcal{X}$ and that $T_i$ is a reasonably good spanning tree for $P_i$ for each $i \in [t] = \{1,2, \hdots, t\}$. 

In order for our approach to be meaningful for large-scale metric MST problems, we must be able to obtain a reasonably good initial forest without this dominating our overall algorithmic pipeline.
Here we discuss two specific strategies for initial forest computations, both of which are fast, easy to parallelize, and motivated by techniques that are already being used in practice in large-scale clustering pipelines. In particular, there are already a number of existing heuristics for finding MSTs and hierarchical clusters that rely in some way on partitioning an initial dataset and then connecting or merging components~\cite{zhong2015fast,jothi2018fast,mishra2020efficient,chen2013clustering}. See Section~\ref{sec:related} for more details. These typically focus only on point cloud data, do not apply to arbitrary metric spaces, and do not come with any type of approximation guarantee. Nevertheless, they provide examples of fast heuristics for large-scale metric spanning tree problems, and serve as motivation for our more general strategies.

\paragraph{Strategy 1: Components of a $k$-NN graph.}
A natural way to obtain an initial forest for $G_\mathcal{X}$ is to compute an approximate $k$-nearest neighbor graph for a reasonably small $k$, which can be accomplished with the $k$-NN descent algorithm~\cite{dong2011efficient} or a recent distributed generalization of this method~\cite{iwabuchi2023towards}. This efficiently connects a large number of points using small-weight edges. The $k$-NN graph will often be disconnected, and we can use the set of connected components as our initial components $\mathcal{P} = \{P_1, P_2, \hdots , P_t\}$. The exact number of components will depend on the distribution of the data and the number of nearest neighbors computed. For larger values of $k$, the $k$-NN graph is more expensive to compute, but then there are fewer components to connect, so there are trade-offs to consider. The components of the $k$-NN graph will typically not be trees but will be sparse (each node has at most $O(k)$ edges), so we can use classical MST algorithms to find spanning trees for all components in $\sum_{i = 1}^t \tilde{O}(k \cdot |P_i|) = \tilde{O}(kn)$ time. The exact runtime of the $k$-NN descent algorithm depends on various parameters settings, but prior work reports an empirical runtime of $O(n^{1.14})$~\cite{dong2011efficient}, with strong empirical performance across a range of different metrics and dataset sizes. We remark that $k$-NN computations have already been used elsewhere as subroutines for large-scale Euclidean MST computations~\cite{almansoori2024fast,chen2013clustering}.

\paragraph{Strategy 2: Fast clustering heuristics.}
Another approach is to form components of the initial forest $\mathcal{P} = \{P_1, P_2, \hdots, P_t\}$ by applying a fast clustering heuristic to $\mathcal{X}$ such as a distributed $k$-center algorithm~\cite{malkomes2015fast,mcclintock2016efficient}. Even the simple sequential greedy 2-approximation algorithm for $k$-center can produce an approximate clustering using $O(nk)$ queries~\cite{gonzalez1985clustering}. Approximate or exact minimum spanning trees for each $P_i$ can be found in parallel. The remaining step is to find a good way to connect the forest. Similar approaches that partition the initial dataset using $k$-means clustering also exist~\cite{zhong2015fast,jothi2018fast}, though this inherently forms clusters based on Euclidean distances. For all of these clustering-based approaches, the number of components $t$ for the initial forest is easy to control since it exactly corresponds to the number of clusters $k$. Smaller $k$ leads to larger clusters, and hence finding an MST of each $P_i$ is more expensive. However, there are then fewer components to connect, so there is again a trade-off to consider. 
%
We remark that it may seem counterintuitive to use a clustering method as a subroutine for finding an MST, since one of the main reasons to compute an MST is to perform clustering. We stress that Strategy 2 uses a cheap and fast clustering method that identifies sets of points that are somewhat close in the metric space, without focusing on whether they are good clusters for a downstream application. This speeds up the search for a good spanning tree, which can be used as one step of a more sophisticated hierarchical clustering pipeline. \\


\subsection{MFC as a learning-augmented framework}
\label{sec:learningaugmented}
Our approach fits the framework of learning-augmented algorithms in that the initial forest can be viewed as a prediction for a partial metric MST, such as the forest obtained by running several iterations of a classical MST algorithm. 
In an ideal setting, the initial forest would be a subset of an optimal MST. If so, then an optimal solution to MFC would produce an optimal metric MST. We relax this by introducing a more general way to measure how much an optimal MST ``overlaps'' with initial forest components. 
Let $\mathcal{T}_\mathcal{X}$ represent the set of minimum spanning trees of $G_\mathcal{X}$, and $T \in \mathcal{T}_\mathcal{X}$ denote an arbitrary MST.
For components $\mathcal{P} =  \{P_1, P_2, \hdots, P_t\}$, let $T(P_i)$ denote the induced subgraph of $T$ on $P_i$, and let $T(\mathcal{P}) = \bigcup_{i = 1}^t T(P_i)$ denote the edges of $T$ contained inside components of $\mathcal{P}$. 
% Then, $w(T(\mathcal{P}))$ is the total weight that an optimal MST $T$ places inside components of $\mathcal{P}$. 
Larger values of $w_\mathcal{X}(T(\mathcal{P}))$ indicate better initial forests, since this means an optimal MST places a larger weight of edges inside these components. 
% Given components $\mathcal{P} =  \{P_1, P_2, \hdots, P_t\}$ and corresponding spanning trees $\{T_1, T_2, \hdots, T_t$\}, 
We define the $\gamma$-overlap for the initial forest to be the ratio
\begin{equation}
\label{eq:gamma}
\gamma(\mathcal{P})  = \frac{w_\mathcal{X}(E_t)}{\max_{T \in \mathcal{T}_\mathcal{X}} w_\mathcal{X}(T(\mathcal{P}))}.
\end{equation}
This measures the weight of edges that the initial forest places inside $\mathcal{P}$, relative to the weight of edges an optimal MST places inside $\mathcal{P}$. 
When $\mathcal{P}$ is clear from context we will simply write $\gamma$. 
Lower ratios for $\gamma$ are better. 
In the denominator, we maximize $w_\mathcal{X}(T(\mathcal{P}))$ over all optimal spanning trees since any MST of $G_\mathcal{X}$ is equally good for our purposes; hence we are free to focus on the MST that overlaps most with $\mathcal{P}$. The optimality of $T$ implies that $w_\mathcal{X}(T(P_i)) \leq w_\mathcal{X}(T_i)$ for every $i \in \{1,2, \hdots, t\}$, which in turn implies that $\gamma(\mathcal{P}) \geq 1$ always. Even if $T_i$ is a minimum spanning tree for $P_i$, it is possible to have $w_\mathcal{X}(T(P_i)) < w_\mathcal{X}(T_i)$ since the induced subgraph $T(P_i)$ is not necessarily connected (see Figure~\ref{fig:overlap} for an example). We achieve the lower bound $\gamma = 1$ exactly in the idealized setting where the initial forest is contained in some optimal metric MST. In practice, we expect the clusters $P_i$ and spanning trees $T_i$ to be imperfect in the sense that connecting them will likely not provide an optimal MST for $G_X$. However, for an initial forest where each $P_i$ is a set of nearby points and $T_i$ is a reasonably good spanning tree for $P_i$, we would expect $\gamma$ to be larger than 1 but still not too large. Figure~\ref{fig:overlap} provides an example where we can certify that $\gamma \leq 1.12$ by comparing against one optimal MST. We later show experimentally that we can quickly obtain initial forests with small $\gamma$-overlap for a wide range of datasets and metrics. 

\paragraph{Learning-augmented algorithm guarantees.}
 We typically would not compute the ratio $\gamma$ for large-scale applications as this would be even more computationally expensive than solving the original metric MST problem. However, this serves as a theoretical measure of initial forest quality when proving approximation guarantees, following the standard approach in the analysis of learning-augmented algorithms. 
In the worst case, no algorithm making $o(n^2)$ queries can obtain a constant factor approximation for metric MST in arbitrary metric spaces~\cite{indyk1999sublinear}.  Following the standard goal of learning-augmented algorithms, we will improve on this worst-case setting when the initial forest is good; otherwise we will recover the worst-case behavior. Formally, we will consider the initial forest to be good when $\gamma$ is finite and the number of components is $t = o(n)$. For this setting, Section~\ref{sec:algs} will present a learning-augmented $(2\gamma+1)$-approximation algorithm for metric MST with subquadratic complexity. There are two ways to see that this is no worse than the worst-case guarantee using no initial forest, depending on whether we focus on worst-case runtimes or wort-case approximation factors. From the perspective of runtimes, if $t = \Omega(n)$ we can default to the worst-case quadratic complexity algorithms that provide an optimal MST. From the perspective of approximation factors, we certainly do no worse than the worst-case approximation factor, which is infinite for subquadratic algorithms. We will state our approximation results and runtimes more precisely in Section~\ref{sec:algs}.



\section{Approximate Completion Algorithm}
\label{sec:algs}
\begin{algorithm}[tb]
	\caption{\textsf{MFC-Approx}}
	\label{alg:mfcapprox}
	\begin{algorithmic}[5]
		\State{\bfseries Input:} $\mathcal{X} = \{x_1, x_2, \hdots , x_n\}$, components $\mathcal{P} = \{P_1, P_2, \hdots, P_t\}$, spanning trees $\{T_1, T_2, \hdots, T_t\}$
		\State {\bfseries Output:} Spanning tree for implicit metric graph of $\mathcal{X}$
		\For{$i = 1, 2, \hdots t$}
		\State Select arbitrary component representative $s_i \in P_i$
		\EndFor
		\For{$i = 1, 2, \hdots t-1$}
		\For{$j = i+1, \hdots , t$}
		\State $w_{i \rightarrow j} = \min_{x_i \in P_i} d(x_i, s_j)$  \quad \hfill \texttt{ // closest a $P_i$ node comes to $s_j$}
		\State $w_{j \rightarrow i} = \min_{x_j \in P_j} d(x_j, s_i)$ \quad \hfill  \texttt{ // closest a $P_j$ node comes to $s_i$}
		\State $\hat{w}_{ij} = \min \{w_{i \rightarrow j}, w_{j \rightarrow i}\}$ \quad  \hfill \texttt{// set weight for edge $(v_i, v_j)$} 
		\EndFor
		\EndFor
		\State $\hat{T}_{\mathcal{P}} = \textsc{OptimalMST}(\{\hat{w}_{ij}\}_{i,j \in [t]})$ \hfill \texttt{ // find optimal MST on complete $t$-node graph}
		\State Return spanning tree $\hat{T}$ of $G_\mathcal{X}$ by combining $\bigcup_{i=1}^t T_i$ with edges from $\hat{T}_{\mathcal{P}}$.
	\end{algorithmic}
\end{algorithm}

We now present an algorithm that approximates MFC to within a factor $c < 2.62$. We also prove it can be viewed as a learning-augmented algorithm for metric MST, where the approximation factor depends on the $\gamma$-overlap of the initial forest. 
%Pseudocode for our algorithm is provided in the appendix. Here in the main text we give a full description of the algorithm along with visual aids in Figure~\ref{fig:mfcapprox}. Due to space constraints, proofs are relegated to the appendix.



\begin{figure*}[t]
	\centering
	\begin{subfigure}[b]{0.33\textwidth}
		\centering
		\includegraphics[width=\textwidth]{figures/wijhat.pdf}
		%        \vspace{-10pt}
		\caption{Computing $\hat{w}_{ij}$}
		\label{fig:wijhat}
	\end{subfigure}\hfill
	\begin{subfigure}[b]{0.33\textwidth}
		\centering
		\includegraphics[width=\textwidth]{figures/componentsgraphhat.pdf}
		%        \vspace{-10pt}
		\caption{$G_\mathcal{P}$ w.r.t.\ $\hat{w}$}
		\label{fig:componentsgraphhat}
	\end{subfigure}\hfill
	\begin{subfigure}[b]{0.33\textwidth}
		\centering
		\includegraphics[width=\textwidth]{figures/approx_mstc.pdf}
		%         \vspace{-10pt}
		\caption{\textsf{MFC-Approx} output}
		\label{fig:approx_mfc}
	\end{subfigure}
	\caption{(a) Finding the minimum distance between components $P_i$ and $P_j$ (dashed line) is an expensive bichromatic closest pair problem. \textsf{MFC-Approx} instead performs a cheaper nearest neighbor query for a \emph{representative} point in each component ($s_i$ and $s_j$, shown as stars). The algorithm finds the closest point to each representative from the opposite cluster, then takes the minimum of the two distances. 
		(b) Applying this to each pair of components produces a weight function $\hat{w}$ for the coarsened graph $G_\mathcal{P}$. Finding an MST of $G_\mathcal{P}$ with respect to $\hat{w}$ yields (c) a 2.62-approximation for MFC.}
	\label{fig:mfcapprox}
\end{figure*}


%\subsection{Algorithm description} 
Pseodocode for our method, which we call \textsf{MFC-Approx}, is shown in Algorithm~\ref{alg:mfcapprox}. This algorithm starts by choosing an arbitrary point $s_i \in P_i$ for each $i \in \{1,2, \hdots, t\}$ to be the component's \emph{representative} (starred nodes in Figure~\ref{fig:mfcapprox}). 
The algorithm computes the distance between every point $x \in \mathcal{X}$ and all of the other representatives. For every pair of distinct components $i$ and $j$ we compute the weights:
\begin{align}
	w_{i \rightarrow j} &= \min_{x_i \in P_i} d(x_i, s_j)  	&\text{(the closest $P_i$ node to $s_j$)} \\
	w_{j \rightarrow i} &= \min_{x_j \in P_j} d(x_j, s_i)  &\text{(the closest $P_j$ node to $s_i$).} 
\end{align}
We then define the approximate edge weight between component nodes $v_i$ and $v_j$ to be:
\begin{align}
	\label{eq:approxweight}
	\hat{w}_{ij} = \min \{w_{i \rightarrow j}, w_{j \rightarrow i}\}.
\end{align}
This upper bounds the minimum distance $w_{ij}^*$ between the two components (Figure~\ref{fig:wijhat}).
Computing this for all pairs of components creates a new weight function $\hat{w}$ for the
coarsened graph $G_\mathcal{P}$ (Figure~\ref{fig:componentsgraphhat}). The algorithm keeps track of the points in $\mathcal{X}$ that define these edge weights in $G_\mathcal{P}$. It then computes an MST in $G_\mathcal{P}$ with respect to $\hat{w}$, then identifies the corresponding edges in $G_\mathcal{X}$, to produce a feasible solution for MFC (Figure~\ref{fig:approx_mfc}). 





\subsection{Bounded and unbounded edges in the coarsensed graph}
A key step of our approximation analysis is to separate edges of the coarsened graph $G_\mathcal{P} = (V_\mathcal{P},E_\mathcal{P})$ into two categories, depending on the relationship between $\hat{w}$ and $w^*$.
% \begin{definition}
Formally, for an arbitrary constant $\beta \geq 1$, we say edge $(v_i,v_j) \in E_\mathcal{P}$ is \textit{$\beta$-bounded} if $\hat{w}_{ij}  \leq \beta w^*_{ij}$, otherwise it is \textit{$\beta$-unbounded}. 
If all edges in $G_\mathcal{P}$ were $\beta$-bounded, finding an MST in $G_\mathcal{P}$ with respect to $\hat{w}$ would produce a $\beta$-approximation for the MST problem in $G_\mathcal{P}$ with respect to $w^*$. In general we cannot guarantee all edges will be $\beta$-bounded, as this would imply Algorithm~\ref{alg:mfcapprox} is a subquadratic $\beta$-approximation algorithm for MFC, contradicting Theorem~\ref{thm:hard}. Nevertheless, any $\beta$-bounded edge that Algorithm~\ref{alg:mfcapprox} includes in its MST of $G_\mathcal{P}$ is easy to bound in terms of the optimal edge weights $w^*$. 
%
If we include a $\beta$-unbounded edge in our MST of $G_\mathcal{P}$, we can no longer bound its weight in terms of an optimal solution to MFC. However, the following lemma shows that its weight can be bounded in terms of the weight of the initial forest.
\begin{lemma}
	\label{lem:unboundededges}
	Let $P_i$ and $P_j$ be an arbitrary pair of components and let $\beta > 1$. If $\hat{w}_{ij} > \beta w^*_{ij}$, then 
	\begin{align}
		\hat{w}_{ij} <
		%		\frac{\beta}{\beta - 1} \min \{\alpha_i, \alpha_j\} \leq 
		\frac{\beta}{\beta - 1}  \min\{w_\mathcal{X}(T_i), w_\mathcal{X}(T_j)\}.
	\end{align}
\end{lemma}
\begin{proof}
	For each $i \in \{1, 2, \hdots, t\}$, we denote the maximum distance between a point in $P_i$ and its component representative $s_i$ by $\alpha_i = \max_{x \in P_i} d(x, s_i)$.
	Let $x_i^* \in P_i$ and $x_j^* \in P_j$ be points satisfying $d(x_{i}^*, x_{j}^*) = d(P_i, P_j) = w_{ij}^*$.
	We use the (reverse) triangle inequality and the definition of $\alpha_i$ to see that:
	\begin{align*}
		d(x_i^*, x_j^*) = \min_{x_j \in P_j} d(x_i^*, x_j) \geq  \min_{x_j \in P_j} d(s_i, x_j) - d(s_i, x_i^*) \geq  \min_{x_j \in P_j} d(s_i, x_j) - \alpha_i = w_{j \rightarrow i} - \alpha_i \geq \hat{w}_{ij} - \alpha_i.
	\end{align*}
	Similarly we can show that 
	\begin{align*}
		d(x_i^*, x_j^*) = \min_{x_i \in P_i} d(x_i, x_j^*) \geq  \min_{x_i \in P_i} d(s_j, x_i) - d(s_j, x_j^*)  \geq  \min_{x_i \in P_i} d(s_j, x_i) - \alpha_j = w_{i \rightarrow j} - \alpha_j \geq \hat{w}_{ij} - \alpha_j.
	\end{align*}
	In other words, we have the bound $w^*_{ij} \geq \hat{w}_{ij} - \min\{\alpha_i, \alpha_j\}$. Combining this with the assumption that $\hat{w}_{ij} > \beta w_{ij}^*$ gives
	\begin{align*}
		\hat{w}_{ij} &> \beta w_{ij}^* \geq \beta\hat{w}_{ij} - \beta\min \{\alpha_i, \alpha_j\} \implies \frac{\beta}{\beta -1 } \min \{\alpha_i, \alpha_j\}> \hat{w}_{ij}.
	\end{align*}
	The proof follows from the observation that $\alpha_i \leq w_\mathcal{X}(T_i)$. To see why, note that there exists some ${x} \in P_i$ such that $d({x}, s_i) = \alpha_i$. Since $T_i$ is a spanning tree of $P_i$, it must contain a path from $s_i$ to ${x}$ with sum of edge weights at least $\alpha_i$.
\end{proof}
Our main approximation guarantees rely on Lemma~\ref{lem:unboundededges}, as well as two other simple supporting observations. This first amounts to the observation that a tree has arboricity and degeneracy 1.
\begin{observation}
	\label{lem:treeorient}
	If $T = (V,E_T)$ is an undirected tree, there is a way to orient edges in such a way that every node has at most one outgoing edge.
\end{observation}
\begin{proof}
	The proof is constructive. Define an iterative algorithm that removes a minimum degree node at each step and deletes all its incident edges. Orient the deleted edges so that they start at the node that was removed. Note that a tree always contains a node of degree 1, and removing such a node leads to another tree with one fewer node. Thus, this procedure will orient edges of the original graph in such a way that each node has at most one outgoing edge.
\end{proof}
The other supporting result deals with MSTs in a graph that includes edges of weight zero.
\begin{observation}
	\label{obs:Z}
	Let $w^{(1)} \colon E \rightarrow \mathbb{R}^+$ and $w^{(2)} \colon E \rightarrow \mathbb{R}^+$ be two nonnegative weight functions for an undirected graph $G = (V,E)$. Assume there exists an edge set $Z \subseteq E$ such that 
	\begin{equation*}
		w^{(1)}(i,j) = w^{(2)}(i,j) = 0 \text{ for every $(i,j) \in Z$}.
	\end{equation*}
	Then there exist spanning trees $M_1$ and $M_2$ for $G$ such that $M_i$ is an MST for $G$ with respect to $w^{(i)}$ for $i \in \{1,2\}$, and $M_1 \cap Z = M_2 \cap Z$.
\end{observation}
\begin{proof}
	The proof is constructive. Recall that Kruskal's algorithm finds an MST by ordering edges by weight (starting with the smallest and breaking ties arbitrarily in the ordering) and then greedily adds each edge a growing spanning tree if and only if it connects two previously disconnected components. 
	Fix an arbitrary ordering $\sigma_Z$ of edges in $Z$. When applying Kruskal's algorithm to find minimum spanning trees of $G$ with respect to $w^{(1)}$ and $w^{(2)}$, we can choose orderings for these functions that exactly coincide for the first $|Z|$ edges visited. Namely, we place edges in $Z$ first, using the order given by $\sigma_Z$. 
	% The remaining edges $E\backslash Z$ may be ordered differently in the orderings for $w^{(1)}$ and $w^{(2)}$. 
	The first $|Z|$ steps of Kruskal's algorithm will be identical when building MSTs with respect to $w^{(1)}$ and $w^{(2)}$. Thus, if $M_1$ and $M_2$ are the spanning trees obtained for $w^{(1)}$ and $w^{(2)}$ respectively using this approach, we know these trees will include the same set of edges from $Z$ and discard the same set of edges from $Z$, i.e., $M_1\cap Z = M_2 \cap Z.$
\end{proof}


\subsection{Main approximation guarantees}
Let $T^*_\mathcal{P}$ represent an MST of $G_\mathcal{P}$ with respect to the optimal weight function $w^* \colon E_\mathcal{P} \rightarrow \mathbb{R}^+$ and $\hat{T}_\mathcal{P}$ represent an MST of $G_\mathcal{P}$ with respect to the approximate weight function $\hat{w} \colon E_\mathcal{P} \rightarrow \mathbb{R}^+$. The edges of $T^*_\mathcal{P}$ map to a set of edges $M^*$ in $G_\mathcal{X}$ that optimally solves the metric MST completion problem, and the edges in $\hat{T}_\mathcal{P}$ map to an edge set $\hat{M}$. The weight of these edges is given by:
\begin{align*}
	w_\mathcal{X}(M^*) &= w^*(T^*_\mathcal{P}) \\
	w_\mathcal{X}(\hat{M}) &= \hat{w}(\hat{T}_\mathcal{P}).
\end{align*}
Let $T^*$ be the spanning tree of $G_\mathcal{X}$ defined by combining $\bigcup_{i = 1}^t T_i$ with $M^*$ and $\hat{T}$ be the spanning tree (returned by Algorithm~\ref{alg:mfcapprox}) that combines $\bigcup_{i = 1}^t T_i$ with $\hat{M}$. These have weights given by
\begin{align}
	\label{eq:Tstar}
	w_\mathcal{X}(T^*) &= w^*(T_\mathcal{P}^*) +  \sum_{i = 1}^t w_\mathcal{X}(T_i) \\
	\label{eq:That}
	w_\mathcal{X}(\hat{T}) &= \hat{w}(\hat{T}_\mathcal{P}) +  \sum_{i = 1}^t w_\mathcal{X}(T_i) .
\end{align}
We are now ready to prove the approximation guarantee for \textsf{MFC-Approx}.
\begin{theorem}
	\label{thm:main}
	The spanning tree $\hat{T}$ returned by Algorithm~\ref{alg:mfcapprox} satisfies
	\begin{equation*}
		w_\mathcal{X}(T^*) \leq w_\mathcal{X}(\hat{T}) \leq \beta w_\mathcal{X}(T^*)
	\end{equation*}
	for $\beta = (3 + \sqrt{5})/2 < 2.62$. 
\end{theorem}
\begin{proof}
	For our analysis we consider two hypothetical weight functions $w^*_0$ and $\hat{w}_0$ for $G_\mathcal{P} = (V_\mathcal{P}, E_\mathcal{P})$, defined by zeroing out the $\beta$-unbounded edges in $w^*$ and $\hat{w}$:
	\begin{align*}
		w^*_0(v_i, v_j) &= \begin{cases} 
			w^*_{ij} & \text{ if $(v_i,v_j)$ is $\beta$-bounded, i.e., $\hat{w}_{ij} \leq \beta w^*_{ij}$} \\
			0 & \text{ otherwise} 
		\end{cases}\\
		\hat{w}_0(v_i, v_j) &= \begin{cases} 
			\hat{w}_{ij} & \text{ if $(v_i,v_j)$ is $\beta$-bounded, i.e., $\hat{w}_{ij} \leq \beta w^*_{ij}$} \\
			0 & \text{ otherwise}. 
		\end{cases}
	\end{align*}
	By Observation~\ref{obs:Z}, there exist spanning trees $T_0^*$ and $\hat{T}_0$ for $G_\mathcal{P}$ that are optimal with respect to $w_0^*$ and $\hat{w}_0$, respectively, which contain the same exact set of $\beta$-unbounded edges. Let $U$ represent this set of $\beta$-unbounded edges in $\hat{T}_0$ and $T^*_0$. Let $B^*$ be the set of $\beta$-bounded edges in $T^*_0$ and $\hat{B}$ be the set of $\beta$-bounded edges in $\hat{T}_0$. Because $\hat{T}_\mathcal{P}$ is an MST with respect to $\hat{w}$ we know that:
	\begin{align}
		\label{eq:hats}
		\hat{w}(\hat{T}_\mathcal{P}) &\leq \hat{w}(\hat{T}_0) = \hat{w}(U) + \hat{w}(\hat{B}).
	\end{align}
	We will use this to upper bound the weight of $\hat{T}_\mathcal{P}$ in terms of $T^*$. First we claim that
	\begin{equation}
		\label{eq:Lhatbound}
		\hat{w}(\hat{B}) \leq \beta w^*(T_\mathcal{P}^*).
	\end{equation}
	This follows from the following sequence of inequalities:
	\begin{align*}
		\hat{w}(\hat{B}) 
		&= \hat{w}_0(\hat{B}) & \text{ since $\hat{w}$ and $\hat{w}_0$ coincide on $\beta$-bounded edges} \\
		&= \hat{w}_0(\hat{B}) + \hat{w}_0(U) & \text{ since $\hat{w}_0$ is zero on $\beta$-unbounded edges} \\
		&= \hat{w}_0(\hat{T}_0)  & \text{ since $\hat{T}_0 = \hat{B} \cup U$} \\
		& \leq \hat{w}_0(T_0^*) & \text{ since $\hat{T}_0$ is optimal for $\hat{w}_0$} \\
		& = \hat{w}_0(B^*) & \text{ since $\hat{w}_0$ is zero on $\beta$-unbounded edges} \\
		& \leq \beta w^*_0(B^*) & \text{ since $\hat{w}_0 \leq \beta w^*_0$ on $\beta$-bounded edges}\\
		%			&= w^*_0(B^*) + w^*_0(U) & \text{since $w_0^*$ is zero on $\beta$-unbounded}  \\
		&= \beta w^*_0(T_0^*) & \text{since $T_0^* = B^* \cup U$ and $w^*_0(U) = 0$}  \\
		&\leq \beta w^*_0(T_\mathcal{P}^*) & \text{since $T_0^*$ is optimal for $w_0^*$}  \\
		& \leq \beta w^*(T_\mathcal{P}^*) & \text{ since $w^*_0 \leq w^*$ for all edges.}
	\end{align*}	
	Next we bound $\hat{w}(U)$. From Lemma~\ref{lem:unboundededges}, we know that $\hat{w}_{ij} \leq {\beta}/({\beta-1}) \min \{w_\mathcal{X}(T_i), w_\mathcal{X}(T_j)\}$ for every $(v_i, v_j) \in U$. Because $\hat{T}_\mathcal{P}$ is a tree on $G_\mathcal{P}$, we know by Observation~\ref{lem:treeorient} that we can orient its edges in such a way that each node in $V_\mathcal{P} = \{v_1, v_2, \hdots, v_t\}$ has at most one outgoing edge. 
	We can therefore assign each $(v_i, v_j) \in U$ to one of its nodes in such a way that each node in $V_\mathcal{P}$ is assigned at most one edge from $U$. Assume without loss of generality that we write edges in such a way that edge $(v_i, v_j) \in U$ is assigned to node $v_i$. Thus,
	\begin{equation}
		\label{eq:small}
		\hat{w}(U) = \sum_{(v_i, v_j) \in U} \hat{w}_{ij} \leq \sum_{(v_i, v_j) \in U} \frac{\beta}{\beta-1} w_\mathcal{X}(T_i) \leq \frac{\beta}{\beta-1} \sum_{i = 1}^t w_\mathcal{X}(T_i).
	\end{equation}
	%
	Combining these gives our final bound
	\begin{align*}
		w_\mathcal{X}(\hat{T}) 
		&= \hat{w}(\hat{T}_\mathcal{P}) + \sum_{i = 1}^t w_\mathcal{X}(T_i)  & \text{by Eq.~\eqref{eq:That}}\\
		&\leq \hat{w}(\hat{B}) + \hat{w}(U) +  \sum_{i = 1}^t w_\mathcal{X}(T_i)  & \text{by Eq.~\eqref{eq:hats}}\\
		& \leq \beta w^*(T^*_\mathcal{P}) + \left(\frac{\beta}{\beta-1} + 1\right) \sum_{i = 1}^t w_\mathcal{X}(T_i) & \text{by Eqs.~\eqref{eq:Lhatbound} and~\eqref{eq:small}}\\
		&\leq \max \left\{ \beta, \frac{\beta}{\beta - 1} + 1 \right\} \left(w^*(T_\mathcal{P}^*) +  \sum_{i = 1}^k w_\mathcal{X}(T_i) \right)  \\
		%			&\leq \max \left\{ \beta, 1 + \frac{\beta}{\beta - 1} \right\} \left(w^*(T_\mathcal{P}^*) +  \sum_{i = 1}^k w_\mathcal{X}(T_i) \right)  \\
		&= \beta w_\mathcal{X}(T^*) &\text{ by Eq.~\eqref{eq:Tstar} and our choice of $\beta$.}
	\end{align*}
	For the last step that we have specifically chosen $\beta = (3+\sqrt{5})/2$ to ensure that $\beta = 1 + \beta/(\beta-1)$, as this leads to the best approximation guarantee using the above inequalities. 
\end{proof}

Using a similar proof technique as Theorem~\ref{thm:main} we obtain the following result, showing that Algorithm~\ref{alg:mfcapprox} is a learning-augmented algorithm for metric MST whose performance depends on the $\gamma$-overlap of the initial forest. 
\begin{theorem}
	\label{thm:learning}
	Let $G_\mathcal{X}$ be an implicit metric graph and $\mathcal{P}$ be an initial partitioning with $\gamma$-overlap $\gamma = \gamma(\mathcal{P})$. Algorithm~\ref{alg:mfcapprox} returns a spanning tree of $\hat{T}$ of $G_\mathcal{X}$ that satisfies
	\begin{equation}
		w_\mathcal{X}(T_\mathcal{X}) \leq w_\mathcal{X}(\hat{T}) \leq \beta w_\mathcal{X}(T_\mathcal{X})
	\end{equation}
	where $T_\mathcal{X}$ is an MST of $G_\mathcal{X}$ and $\beta = \frac{1}{2}\left(2\gamma + 1 + \sqrt{4\gamma + 1} \right) \leq 2\gamma+ 1$.
\end{theorem}
\begin{proof}
	We use the same terminology and notation as in the proof of Theorem~\ref{thm:main}. The only difference is that we do not necessarily use $\beta = (3+ \sqrt{5})/2$. For an arbitrary $\beta \geq 1$, we can still prove in the same way that
	\begin{align}
		\label{eq:start}
		w_\mathcal{X}(\hat{T}) \leq  \beta w^*(T^*_\mathcal{P}) + \left(\frac{\beta}{\beta-1} + 1\right) \sum_{i = 1}^t w_\mathcal{X}(T_i).
	\end{align}
	The $\gamma$-overlap of the initial forest implies there exists an MST $T_\mathcal{X}$ of $G_\mathcal{X}$ satisfying:
	\begin{equation}
		\label{eq:ieratio}
		\sum_{i = 1}^t w_\mathcal{X}(T_i) = \gamma w_\mathcal{X}(I_\mathcal{X}),
	\end{equation}
	where $I_\mathcal{X}$ is the set of edges of $T_\mathcal{X}$ inside components $\mathcal{P}$ of the initial forest. Let $B_\mathcal{X}$ be the set of edges in $T_\mathcal{X}$ that cross between components, so that $w_\mathcal{X}(T_\mathcal{X}) = w_\mathcal{X}(B_\mathcal{X}) + w_\mathcal{X}(I_\mathcal{X})$. Since $T_\mathcal{X}$ is a spanning tree, $B_\mathcal{X}$ must contain a path between every pair of components, meaning that $B_\mathcal{X}$ corresponds to a spanning subgraph of the coarsened graph $G_\mathcal{P}$. Since $T_\mathcal{P}^*$ defines an MST of $G_\mathcal{P}$ with respect to $w^*$, which captures the minimum distances between pairs of components, we know
	\begin{equation}
		w^*(T_\mathcal{P}^*) \leq w_\mathcal{X}(B_\mathcal{X}).
	\end{equation}
	Putting the pieces together we see that
	\begin{equation}
		w_\mathcal{X}(\hat{T}) \leq  \beta w_\mathcal{X}(B_\mathcal{X}) + \left( 1 + \frac{\beta}{\beta-1}\right) \gamma w_\mathcal{X}(I_\mathcal{X}) \leq \max \left\{\beta, \gamma\left(1 + \frac{\beta}{\beta -1}\right)  \right\} w_\mathcal{X}(T_\mathcal{X}).
	\end{equation}
	This will hold for any choice of $\beta \geq 1$. In order to prove the smallest approximation guarantee, we choose $\beta$ satisfying:
	\begin{equation*}
		\beta = \gamma\left(1 + \frac{\beta}{\beta -1}\right).
	\end{equation*}
	The solution for this equation under constraint $\beta \geq 1$ and $\gamma \geq 1$ is 
	\begin{equation*}
		\beta = \frac{1}{2}\left(2\gamma + 1 + \sqrt{4\gamma + 1} \right) \leq 2\gamma+ 1.
	\end{equation*}
\end{proof}


\subsection{Runtime analysis and practical considerations}
Algorithm~\ref{alg:mfcapprox} finds the distance between each point in $\mathcal{X}$ and each of the $t$ component representatives, for a total of $O(nt)$ distance queries. It then finds an MST of a dense graph with ${t \choose 2}$ edges, which has runtime and space requirements of $\tilde{O}(t^2)$. Thus, the algorithm has subquadratic memory and query complexity as long as $t = o(n)$. The runtime is $\tilde{O}(nt\texttt{Q}_\mathcal{X} + t^2)$ where $\texttt{Q}_\mathcal{X}$ is the complexity for one distance query in $\mathcal{X}$, which also is subquadratic as long as $t\texttt{Q}_\mathcal{X} = o(n)$. In settings where $\texttt{Q}_\mathcal{X} = \tilde{O}(1)$, the memory, runtime, and query complexity are all subquadratic as long as $t = o(n)$.


\paragraph{Full runtime using $k$-center initialization.} The practical utility of our full MST pipeline also depends on the time it takes to find an initial forest, which depends on various design choices and trade-offs when using any strategy. For intuition we provide a rough complexity analysis for the $k$-center strategy assuming an idealized case of balanced clusters. The simple $2$-approximation for $k$-center chooses an arbitrary first cluster center, and chooses the $i$th cluster center to be the point with maximum distance from the first $i - 1$ centers~\cite{gonzalez1985clustering}. This requires $O(nt)$ distance queries. We can use the cluster centers as the component representatives for \textsf{MFC-Approx}, which allows us to compute $\hat{w}$ without any additional queries. If clusters are balanced in size, we can compute minimum spanning trees for all clusters using $O(n^2/t)$ queries and memory and a runtime of $\tilde{O}(\texttt{Q}_\mathcal{X}n^2/t)$, simply by querying all inner-cluster edges and running a standard MST algorithm. In this balanced-cluster case, combining the initial forest complexity with the complexity of our MFC algorithm, the entire pipeline for finding a spanning tree takes $\tilde{O}(\texttt{Q}_\mathcal{X}(n^2/t + nt) + t^2)$ time. This is minimized by choosing $t = \sqrt{n}$ clusters, leading to a complexity that grows as $n^{1.5}$. For unbalanced clusters, one must consider different trade-offs for cluster-balancing strategies, which could be beneficial for runtime but may affect initial cluster quality.
We could also improve the runtime at the expense of initial forest quality by not computing an exact MST for each cluster. For example, we could recursively apply our entire MFC framework to find a spanning tree of each cluster. 


\paragraph{Practical improvements.} There are several ways to relax our MFC framework to make our approach faster while still satisfying strong approximation guarantees. For metrics with high query complexity, we can use approximate queries with only minor degradation in approximation guarantees. For example, for high-dimensional Euclidean distance we can apply Johnson-Lindenstrauss transformations to reduce the query complexity while approximately maintaining distances. As another relaxation, we can replace the exact nearest neighbor search subroutine in \textsf{MFC-Approx} with an approximate nearest neighbor search. If for some $\varepsilon > 0$ we find a $(1+\varepsilon)$-approximate nearest neighbor in each component for every component representative $s_i$, this will make our approximation guarantees worse by at most a factor $(1+\varepsilon)$. There are also numerous opportunities for parallelization, such as parallelizing distance queries and MST computation for components. 

We can also incorporate heuristics to improve the spanning tree quality of our algorithm with little effect on runtime. As a specific example, when approximating the distance between components $P_i$ and $P_j$ of the initial forest, we could compute $\tilde{x}_i = \argmin_{x \in P_i} d(x,s_j)$ and $\tilde{x}_j = \argmin_{x \in P_j} d(x,s_i)$ and then use the following weight for the coarsened graph:
\begin{equation*}
	\tilde{w}_{ij} = \min \{d(\tilde{x}_i, s_j),d(\tilde{x}_j, s_i), d(\tilde{x}_j,\tilde{x}_i)\}.
\end{equation*}
This differs from Algorithm~\ref{alg:mfcapprox} only in that it additionally checks the distance between $d(\tilde{x}_j,\tilde{x}_i)$ to see if this provides an even closer pair of points between $P_i$ and $P_j$. Although this does not always improve results, it can never be worse in terms of approximations. Figure~\ref{fig:wijhat} provides an example where this strategy would find the optimal distance $w_{ij}^*$, which is strictly better than $\hat{w}_{ij}$. An interesting future direction is to implement this and also explore other heuristics that could improve the practical performance of our method without affecting our theoretical guarantees.



\section{Related Work} \label{sec:related}

% \textbf{Adversarial Attack}
\textbf{Attacks on SLAM.} 
%With the rise of machine learning, 
The robustness of computer vision systems is being actively investigated. With the emergence of adversarial images in the digital domain by adding optimized noise directly to images~\cite{szegedy2013intriguing,carlini2017towards}, researchers find that such attacks also exist physically in the real world \cite{eykholt2018robust,song2018physical,zhao2019seeing}. To fill the gap between attacks in the digital and physical worlds, recent studies have demonstrated that attacks on real-world computer vision systems are practical \cite{eykholt2018robust,li2019adversarial,man2020ghostimage,sharif2016accessorize,zhao2019seeing,zhou2018invisible}. However, attacks on traditional computer vision methods such as SLAM are relatively less explored. \cite{yoshida2022adversarial} proposes an attack against the scan matching algorithm in LiDAR-based SLAM, while most SLAMs in AR/VR devices rely on different sensors like RGB/depth cameras and IMUs. \cite{ikram2022perceptual} and \cite{chen2024adversary} mislead visual SLAM by poisoning the images with special patterns, and \cite{wang2021can} causes the camera to fail using infrared light. In our work, we demonstrate attacks on Visual-Inertial SLAM (VI-SLAM) by perturbing the IMU readings, rather than cameras, and showing its impact on XR user experience. 

\textbf{Acoustic Injection Attacks.} Among various physical attacks, acoustic injection attacks are attractive due to their low cost. Son~\etal~\cite{son2015rocking} were the first to introduce acoustic attacks on MEMS gyroscopes, demonstrating how these attacks could lead to sensor denial-of-service and result in drone crashes. WALNUT~\cite{trippel2017walnut} expanded on this by developing output biasing and control attacks that enable precise manipulation of MEMS accelerometer outputs using modulated sound waves. Wang et al.~\cite{wang2017sonic} demonstrated a sonic gun, showcasing the vulnerability of various smart devices (\eg drones and self-balancing vehicles) to acoustic attacks. Tu et al. \cite{tu2018injected} designed side-swing and switching attacks to alter the outputs of MEMS gyroscopes and accelerometers. Furthermore, Ji et al. \cite{ji2021poltergeist} fool the object detectors by applying acoustic attack to the image stabilizers commonly used in modern cameras. However, none of the existing works study the relationship between the acoustic injections and SLAM outputs on recent XR devices. 

% \zijian{Do we need one session about security in AR/VR?}
% \yicheng{TODO}
%\jiasi{cite the AIVR paper (UMass Amherst?) paper is we have not already. They add IMU perturbation but w/o SLAM, iirc} \yicheng{Cited}

\textbf{XR Security and Privacy.} 
%Security and privacy concerns in XR systems have gained significant attention. 
For single-user XR systems, researchers have demonstrated various side-channel attacks to extract sensitive information (\eg keystrokes) through video feeds~\cite{ling2019know}, head movements~\cite{nair2023unique, slocum2023going}, architectural hints~\cite{zhang2023its,shang2020arspy}, power usage~\cite{li2024dangers}, and EM side-channel leakages~\cite{al2021vr}. In multi-user XR systems, Su et al.~\cite{su2024remote} use avatar motion data to infer keystrokes in shared VR environments. Slocum et al.~\cite{slocum2024doesn} reveal vulnerabilities in the shared state frameworks of multi-user AR. Similarly, Lebeck et al.~\cite{lebeck2017securing} highlight risks like deceptive virtual objects and emphasize access control for managing shared physical and virtual spaces. Ruth et al.~\cite{ruth2019secure} further propose a secure multi-user AR framework focusing on content sharing and permissions.
Chandio et al.~\cite{chandio2024stealthy} %introduced a multi-modal spatiotemporal attack that 
simultaneously manipulated visual and inertial sensors to disrupt XR pose estimation. However, their study evaluated the attack using offline datasets and assumed the attacker's capability to manipulate IMU data streams through acoustic means, without real experiments. Ours is the first to demonstrate acoustic injection attacks on recent XR devices, like the Hololens 2, in the real world.
 


	
\section{Experiments: Planning outperforms Heuristics}
\label{sec:experiment}

We begin our empirical demonstrations by showcasing the effectiveness of our planning framework on both synthetic and real datasets. We focus on the simplest planning algorithm, 1-step lookaheads (Algorithm~\ref{alg:complete}), and show that even basic planning can hold great promise. 
We illustrate our framework using two uncertainty quantification modules---GPs and 
\ensembles/ \ensembleplus. 

Throughout this section, we focus on evaluating the mean squared error of 
a regression model $\model$,  and develop adaptive policies that minimize uncertainty on $g(f)$ defined in~\eqref{eqn:l2-g-f}.
When GPs provide a valid model of uncertainty, 
our experiments show that our planning framework significantly outperforms other baselines. 
We further demonstrate that our conceptual framework extends to deep learning-based uncertainty quantification methods such as  \ensembleplus while highlighting computational challenges that need to be resolved in order to scale our ideas. 
For simplicity, we assume a naive predictor, i.e., $\psi(\cdot) \equiv 0$. However, we emphasize that this problem is just as complex as if we were using a sophisticated model $\psi(.)$. The performance gap between the algorithms 
primarily depends
on the level  of uncertainty in our prior beliefs.

To evaluate the performance of our algorithm, we benchmark it against several baselines. 
%Active learning baselines use an acquisition function $\ac$ to select points that have the highest   function value: $X\opt_t \in \argmax_{X \in \xpoolj{t}} \ac({X})$ at every step $t$. These methods may also need an UQ module, which we simply use the same UQ module as in our algorithm, and it  outputs $V(X)$ that measures the the uncertainty of each point $X \in \xpoolj{t}$.
Our first set of baselines are from active learning~\citep{AggarwalKoGuHaPh14}:
\\ % \noindent\textbf{Active Learning Heuristics:} 
\textbf{(1)} 
\textsf{Uncertainty Sampling (Static):}  In this approach, we query the samples for which the model is least certain about. Specifically, we estimate the variance of the latent output $f(X)$ for each $X \in \xpool$ using the UQ module and select the top-$K$ points with the highest uncertainty. \\
\textbf{(2)} \textsf{Uncertainty Sampling (Sequential):} This is a greedy heuristic that sequentially selects the points with the highest uncertainty within a batch, while updating the posterior beliefs using pseudo labels from the current posterior state. Unlike \textsf{Uncertainty Sampling (Static)}, this method takes into account the information gained from each point within batch, and hence tries to diversify the selected points within a batch. 

 
We also compare our approach to the  \textbf{(3)} \textsf{Random Sampling}, which selects each batch uniformly at random from the pool. Additionally, we compare solving the planning problem using  \textsf{REINFORCE}-based policy gradients with   $\mathsf{Smoothed\text{-}Autodiff}$ policy gradients.\footnote{Our code repository is available at
  \url{https://github.com/namkoong-lab/adaptive-labeling}.}
%Detailed experimental setups are provided in Section \ref{sec:details-experiments}.

%We repeat all experiments with 10 random seeds.




\begin{figure}[t]
\centering
\begin{minipage}[b]{0.49\textwidth}
\centering
\includegraphics[width=\textwidth, height=5cm]{figures/original_scale/Var_of_l_2_loss.pdf}
\caption{(Synthetic data) Variance of mean squared loss evaluated through the posterior belief $\mu_t$ at each horizon $t$. This is the objective that policy gradient methods like \textsf{REINFORCE} and $\ouralgo$ optimizes. 1-step lookaheads are surprisingly effective even in long horizons.}
\label{fig:var-l2-sim}
\end{minipage}
\hfill
\begin{minipage}[b]{0.49\textwidth}
\centering \includegraphics[width=\textwidth, height=5cm]{figures/original_scale/Error_of_estimated_model_l_2_loss.pdf}
\caption{(Synthetic data) Error between MSE calculated based on collected data $\mc{D}^{0:T}$ vs. population oracle MSE over $\mc{D}_{\rm eval} \sim P_X$. Reducing uncertainty over posteriors directly leads to better OOD evaluations. 1-step lookaheads significantly outperform active learning heuristics in small horizons.}
\label{fig:mean-l2-sim}
\end{minipage}
%\caption{Simulated data for GPs}
%\label{fig:both_plots}
\end{figure}

\subsection{Planning with Gaussian processes}
\label{sec:experiment-plan-GP}
We now briefly describe the data generation process for the GP experiments,  deferring a more detailed discussion of the dataset generation to Section~\ref{sec:details-experiments}. 
We use both the synthetic data and the real data to test our methodology.
For the \emph{simulated data},  we construct a setting where the general population is distributed across \emph{51 non-overlapping clusters} while the initial labeled data $\dtrain$ just comes from one cluster. In contrast, both $\dpool \defeq (\xpool,\ypool),\deval \defeq (\xeval,\yeval)$ are generated   from all the clusters. 
We begin with a low-dimensional scenario, generating a one-dimensional regression setting using a GP. %Gaussian Process (GP).
Although the data-generating process is not known to the algorithms,  we assume that the GP hyperparameters are known to all the algorithms
to ensure fair comparisons. This can be viewed as a setting where our prior is well-specified, allowing us to isolate the effects
of different policy optimization approaches
 without any concerns about the misspecified priors. We select $10$ batches, each of size $K=5$ across $T = 10$ time horizons.

To examine the robustness of our method against the distributional assumptions made  in the simulated case, we then move to a real dataset where the correct prior is not known. We simulate selection bias from the eICU dataset~\citep{PollardJoRaCeMaBa18}, which contains real-world patient data with in-hospital mortality outcomes. 
We conduct a $k$-means clustering to generate 51 clusters and then select data from those clusters. We view this to be a credible replication of practice, as severe distribution shifts are common due to selection bias in clinical labels.  To convert the binary mortality labels into a regression setting, we train a  random forest classifier and fit a GP on predicted scores, which serves as the UQ module for all the algorithms. As before, the task is to select 10 batches, each consisting of 5 samples, across 10 time horizons.

 In Figures~\ref{fig:var-l2-sim} and~\ref{fig:mean-l2-sim}, we present results for the simulated data. 
Figure~\ref{fig:var-l2-sim} shows the variance of $\ell_2$ loss, and Figure~\ref{fig:mean-l2-sim} presents the error in the estimated $\ell_2$ loss using $\mu_t$ (relative to true $\ell_2$ loss, that is unknown to the algorithm). 
As we can see from these plots, our method one-step lookahead  gives substantial improvements  over active learning baselines and random sampling. In addition,
compared to the one-step lookahead planning approach using \textsf{REINFORCE}-based policy gradients, 
we observe that $\mathsf{Smoothed\text{-}Autodiff}$-based policy gradients provide significantly more robust performance over all horizons.

In Figures~\ref{fig:var-l2-real}~and~\ref{fig:mean-l2-real}, we observe similar findings on the eICU data. We see that planning policies (\textsf{REINFORCE} and $\mathsf{Smoothed\text{-}Autodiff}$) consistently outperform other heuristics by a large margin.  Active learning baselines perform poorly in these small-horizon batched problems and can sometimes be even worse than the random search baselines.  Overall, our results show the importance of careful planning in adaptive labeling for reliable model evaluation. 

We offer some intuition as to why one-step lookahead planning may outperform other heuristic algorithms. 
 First,  \textsf{Uncertainty sampling (Static)} while myopically selects the
 top-$K$ inputs with the highest uncertainty, it fails to consider 
the overlap in information content among the ``best” instances; see \citep{AggarwalKoGuHaPh14} for more details. 
In other words,  it might acquire points from the same region with high uncertainty while failing to induce diversity among the batch.
Although \textsf{Uncertainty Sampling (Sequential)} somewhat addresses the issue of information overlap, a significant drawback of 
this algorithm
is the disconnect between the objective we aim to optimize and the algorithm. For example, it might sample from a region with high uncertainty but very low density. 

\begin{figure}[t]
\centering
\begin{minipage}[b]{0.48\textwidth}
\centering
\includegraphics[width=\textwidth, height=5cm]{figures/original_scale/Var_of_l_2_loss_real.pdf}
\caption{(Real-world eICU data) Variance of mean squared loss evaluated through the posterior belief $\mu_t$ at each horizon $t$. Even 1-step lookaheads are extremely effective planners, and auto-differentiation-based pathwise policy gradients provide a reliable optimization algorithm based on low-variance gradient estimates.}
\label{fig:var-l2-real}
\end{minipage}
\hfill
\begin{minipage}[b]{0.48\textwidth}
\centering \includegraphics[width=\textwidth, height=5cm]{figures/original_scale/Error_of_estimated_model_l_2_loss_real.pdf}
\caption{(Real-world eICU data) Error between MSE calculated based on collected data $\mc{D}^{0:T}$ vs. population oracle MSE over $\mc{D}_{\rm eval} \sim P_X$. Reducing uncertainty over posteriors directly leads to better OOD evaluations. Our method significantly outperforms active learning-based heuristics, and random sampling.}
\label{fig:mean-l2-real}
\end{minipage}
%\caption{Real data for GPs}
\end{figure}
 
%\vspace{-1.5cm}
% \begin{wrapfigure}{r}{.32\columnwidth}
%   \vspace{-.5cm} 
%   \centering
% \includegraphics[scale=.29]{figures/Var of l2l_2 loss.pdf}
%   \vspace{-0.2cm}
%   \caption{Results of GP}
% \label{fig:var-l2-gp}
%   \vspace{-0.1cm}
% \end{wrapfigure}


% Attempts have been made  in the past to address these  drawbacks heuristically  (see \citep{AggarwalKoGuHaPh14}). We give a unified computational framework while approaching the problem in a more principled manner and solving it more optimally.




\subsection{Planning with  neural network-based uncertainty quantification methods ($\ensembleplus$)}


We now provide a proof-of-concept that shows the generalizability of our conceptual framework  to the deep learning-based UQ modules, specifically focusing on $\ensembleplus$ due to their previously observed superior performance~\citep{OsbandWenAsDwIbLuRo23}. Recall that implementing our framework with deep learning-based UQ modules  requires us to retrain the model across multiple possible random actions $\bm{a}(\theta)$ sampled from the current policy $\pi_\theta$.
This requires significant computational resources, in sharp contrast to the GPs where the posteriors are in closed form and can be readily updated and differentiated. 

Due to the computational constraints, we test $\ensembleplus$ on a toy setting to demonstrate the generalizability of our framework. We consider a setting where the general population consists of four clusters, while the initial labeled data only comes from one cluster. Again we generate data using GPs.  The task is to select a batch of 2 points in one horizon. We detail the $\ensembleplus$ architecture in Section \ref{sec:details-experiments}, and we assume prior uncertainty to be large (depends on the scaling of the prior generating functions). 
The results are summarized in the Table~\ref{tab:UQ_ensemble}.

% \begin{table}[H]
% \vspace{-10pt}
% \caption{Performance under \ensembleplus as UQ module}
%     \centering
%     \begin{tabular}{|m{3cm}|m{2.5cm}|m{2cm}|} 
%     \hline
%       Algorithm   & Variance of $\loss_2$ loss estimate & Error of $\loss_2$ loss estimate  \\ \hline Random Sampling 
%          & $1710.9 \pm 1352.1$ & $8.67\pm6.62$ 
%       \\ \hline \ouralgo & $1.30 \pm 0.68$ & $0.91\pm0.25$ \\ \hline
%     \end{tabular}
%     \label{tab:UQ_ensemble}
%     %\vspace{-10pt}
% \end{table}




\begin{table}[h]
\vspace{-10pt}
\caption{Performance under \ensembleplus as the UQ module}
\centering
\begin{tabular}{|l|l|l|}
\hline
Algorithm   & Variance of $\loss_2$ loss estimate & Error of $\loss_2$ loss estimate  \\
\hline
\textsf{Random sampling} & 7129.8 $\pm$ 1027.0 & 136.2 $\pm$ 8.28 \\ \hline
\textsf{Uncertainty sampling (Static)} & 10852 $\pm$ 0.0 & 162.156 $\pm$ 0.0 \\ \hline
\textsf{Uncertainty sampling (Sequential)} & 8585.5 $\pm$ 898.9 & 144 $\pm$ 6.93 \\ \hline
\textsf{REINFORCE} & 1697.1 $\pm$ 0.0 & 45.27 $\pm$ 0.0 \\ \hline
\ouralgo & 1697.1 $\pm$ 0.0 & 45.27 $\pm$ 0.0 \\ \hline
\end{tabular}
%\caption{Comparison of different algorithms based on variance   and   error in $\ell_2$ loss estimation with Ensemble $+$ as the UQ module. Our results demonstrate that {\ouralgo} and REINFORCE outperformthe other active learning based heuristics, confirming the benefits of our MDP formulation for the adaptive labeling problem, as also demonstrated in Section 4.\\
%\footnotesize{Experimental details: We use Gaussian Processes as our data generating process, GP parameters are the same as in Section D.3.  The task is to select a batch of 2 points along one horizon.The marginal distribution $p_X$ has 4 \textit{non-overlapping} clusters. Initial data comes from one cluster, while pool and evaluation points comes from all the clusters. We have $20$ initial labeled data points, $10$ pool points, and $252$ evaluation points.  Training procedures are similar to the one in Section D.3.} }
\label{tab:UQ_ensemble}
\end{table}



% We faced  issues in scaling up these experiments which will be our focus in the future. 





% \begin{itemize}
%     \item Posteriors should be consistent. Two dimensions: even with less training,  
%     \item the inference should be  fast enough
% \end{itemize}


% Potential research directions for uncertainty quantification

% In this section we consider a simple setting We consider a simpler setting and 


% For synthetic dataset generation, we use ...... For real datasets, we use ...... We compare our methodolgy to several baselines ()    This Section is structured as follows:
% \begin{itemize}
%     \item \textbf{GPs, square loss objective} (Section \ref{}): 
%     %the broad aim of the experiments  in this section is to isolate the performance of our methodology without any concerns for the inefficiencies induced due to a mis-specified prior or imperfect posterior inference. To accomplish this we generate synthetic datasets using GPs (detailed later). We use the well specified prior (GPs - with same hyperparameter setting) as our UQ module.   
%      As GPs provide differentaible posterior inference - any errors induced due to imperfect posterior updates are also isolated. We note that under this setting
%      \item In Section\ref{} we demonstrate why our methodology performs better than other baselines - by devising various synthetic experiments ()
%     \item  \textbf{UQ Benchmarking }(Section \ref{}): Before diving into the experiments using $\ensembleplus$ and ENNs,  we showcase our benchmarking experiments in Section \ref{}. We use real datasets We observe that ENNs perform better
%      \item \textbf{Ensemble $+$}, objective: recall, accuracy
%     \item \textbf{ENN}, objective: recall, accuracy
% \end{itemize}




% In Section {}, we test 
% \subsection{Experimental details}

% \begin{itemize}
%     \item UQ methodologies - GPs, ENNs
%     \item Objectives - Recall,  ATE
%     \item Datasets - ATE-synthetic datasets, Recall-synthetic, real datasets
%     \item Baselines - 
%     \begin{itemize}
%         \item Random sampling
%         \item Active learning - Uncertainty based sampling - In regression setting almost all of the 
%         \item Myopic greedy - Greedy Batch based sampling
%         \item Policy Gradient
%     \end{itemize}
    
% \end{itemize}

% \subsection{Experiments}
%     \begin{itemize}
%     \item GPs with square loss
%     \item Benchmarking ENN
%         \item ENNs with ATE
%         \item ENNs with Recall
%     \end{itemize}

% \subsection{Benefits over other algorithms - intuition and experiments}

%Active learning - Myopic greedy / Don't rely on the objective rather some entropy version.


%%% Local Variables:
%%% mode: latex
%%% TeX-master: "main"
%%% End:



\section{Conclusions and Discussion}
We have presented a new framework for finding spanning trees in arbitrary metric spaces, which is highly scalable and grounded in rigorous approximation guarantees. 
This framework is based on completing an initial forest, which can be obtained efficiently using  practical heuristics. This paper focuses on serial implementations and theoretical guarantees, as our framework already provides many advantages in this setting. At the same time, our work is strongly motivated by massive-scale clustering applications that require high-performance computing capabilities, and the algorithm we developed is highly parallelizable. A natural direction for further research is to develop parallel versions of our algorithm that can be run on a much larger scale. There are also many remaining questions in the serial setting. One direction is to try to improve on the $(\sqrt{5} + 3)/2$-approximation guarantee while still using subquadratic time, or obtain an approximation with better dependence on the $\gamma$-overlap parameter. Another direction is to prove lower bounds for the best possible approximation guarantees for subquadratic algorithms. There are also many opportunities to explore more efficient and practical methods for obtaining the initial forest that serves as the input to MFC.


	
%	\section*{Acknowledgments}
	
	\bibliographystyle{plain}
	\bibliography{mst-bib}

    \appendix

\end{document}
