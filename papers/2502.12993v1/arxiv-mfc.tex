\documentclass[11pt]{article}
\usepackage[bottom=1in,left=1in, right=1in, top=1in]{geometry}
\geometry{letterpaper}            
\usepackage{authblk}
\usepackage{algorithm}
\usepackage[noend]{algpseudocode}
\usepackage{tabularx}
\usepackage{enumitem}
\usepackage{graphicx}
\usepackage{subcaption}
\usepackage{amsmath}
\usepackage{amssymb}
\usepackage{amsfonts}
\usepackage{cite}
\usepackage{hyperref}
\usepackage{booktabs}
\usepackage[T1]{fontenc}

\hypersetup{
	colorlinks=true, %set true if you want colored links
	linktoc=all,     %set to all if you want both sections and subsections linked
	linkcolor=blue,  %choose some color if you want links to stand out
	citecolor = blue
}

\usepackage{titlesec}
\titlespacing{\paragraph}{0pt}{5pt}{5pt}


%
% --- inline annotations
%
\newcommand{\red}[1]{{\color{red}#1}}
\newcommand{\todo}[1]{{\color{red}#1}}
\newcommand{\TODO}[1]{\textbf{\color{red}[TODO: #1]}}
% --- disable by uncommenting  
% \renewcommand{\TODO}[1]{}
% \renewcommand{\todo}[1]{#1}



\newcommand{\VLM}{LVLM\xspace} 
\newcommand{\ours}{PeKit\xspace}
\newcommand{\yollava}{Yo’LLaVA\xspace}

\newcommand{\thisismy}{This-Is-My-Img\xspace}
\newcommand{\myparagraph}[1]{\noindent\textbf{#1}}
\newcommand{\vdoro}[1]{{\color[rgb]{0.4, 0.18, 0.78} {[V] #1}}}
% --- disable by uncommenting  
% \renewcommand{\TODO}[1]{}
% \renewcommand{\todo}[1]{#1}
\usepackage{slashbox}
% Vectors
\newcommand{\bB}{\mathcal{B}}
\newcommand{\bw}{\mathbf{w}}
\newcommand{\bs}{\mathbf{s}}
\newcommand{\bo}{\mathbf{o}}
\newcommand{\bn}{\mathbf{n}}
\newcommand{\bc}{\mathbf{c}}
\newcommand{\bp}{\mathbf{p}}
\newcommand{\bS}{\mathbf{S}}
\newcommand{\bk}{\mathbf{k}}
\newcommand{\bmu}{\boldsymbol{\mu}}
\newcommand{\bx}{\mathbf{x}}
\newcommand{\bg}{\mathbf{g}}
\newcommand{\be}{\mathbf{e}}
\newcommand{\bX}{\mathbf{X}}
\newcommand{\by}{\mathbf{y}}
\newcommand{\bv}{\mathbf{v}}
\newcommand{\bz}{\mathbf{z}}
\newcommand{\bq}{\mathbf{q}}
\newcommand{\bff}{\mathbf{f}}
\newcommand{\bu}{\mathbf{u}}
\newcommand{\bh}{\mathbf{h}}
\newcommand{\bb}{\mathbf{b}}

\newcommand{\rone}{\textcolor{green}{R1}}
\newcommand{\rtwo}{\textcolor{orange}{R2}}
\newcommand{\rthree}{\textcolor{red}{R3}}
\usepackage{amsmath}
%\usepackage{arydshln}
\DeclareMathOperator{\similarity}{sim}
\DeclareMathOperator{\AvgPool}{AvgPool}

\newcommand{\argmax}{\mathop{\mathrm{argmax}}}     


\newcommand{\mfc}{\textsc{Metric Forest Completion}}

%\title{Approximate Completion Strategies and Learning-Augmented Algorithms for Metric Minimum Spanning Trees}
%\title{Minimum Spanning Trees in Arbitrary Metric Spaces: Approximate Completion Strategies and Learning-Augmented Algorithms}
%\title{Spanning Tree Completion in Arbitrary Metric Spaces}
%\title{The metric minimum spanning tree completion problem}
% \title{Minimum Spanning Trees in Arbitrary Metrics: Approximate Tree Completion and Learning-Augmented Algorithms\thanks{Any acknowledgements here}}
\title{Approximate Tree Completion and Learning-Augmented Algorithms for Metric Minimum Spanning Trees\thanks{This work was performed under the auspices of the U.S. Department of Energy by Lawrence Livermore National Laboratory under Contract DE-AC52-07NA27344 (LLNL-ABS-XXXXXX), and was supported by LLNL LDRD project 24-ERD-024.}}

\author[1]{Nate Veldt}
\author[1]{Thomas Stanley}
\author[2]{Benjamin W. Priest}
\author[2]{Trevor Steil}
\author[2]{Keita Iwabuchi}
\author[2]{T.S.~Jayram}
\author[2]{Geoffrey Sanders}

\affil[1]{Department of Computer Science and Engineering, Texas A\&M University}
\affil[2]{Center for Applied Scientific Computing, Lawrence Livermore National Laboratory}
\date{}

\begin{document}
	
	\maketitle
\begin{abstract}
Finding a minimum spanning tree (MST) for $n$ points 
in an arbitrary metric space is a fundamental primitive for hierarchical clustering and many other ML tasks, but this takes $\Omega(n^2)$ time to even approximate.
We introduce a framework for metric MSTs that first (1) finds a forest of disconnected components using practical heuristics, and then (2) finds a small weight set of edges to connect disjoint components of the forest into a spanning tree. 
We prove that optimally solving the second step still takes $\Omega(n^2)$ time, but we provide a subquadratic 2.62-approximation algorithm. In the spirit of learning-augmented algorithms, we then show that if the forest found in step (1) overlaps with an optimal MST, we can approximate the original MST problem in subquadratic time, where the approximation factor depends on a measure of overlap. In practice, we find nearly optimal spanning trees for a wide range of metrics, while being orders of magnitude faster than exact algorithms.
\end{abstract}
	

\section{Introduction}


\begin{figure}[t]
\centering
\includegraphics[width=0.6\columnwidth]{figures/evaluation_desiderata_V5.pdf}
\vspace{-0.5cm}
\caption{\systemName is a platform for conducting realistic evaluations of code LLMs, collecting human preferences of coding models with real users, real tasks, and in realistic environments, aimed at addressing the limitations of existing evaluations.
}
\label{fig:motivation}
\end{figure}

\begin{figure*}[t]
\centering
\includegraphics[width=\textwidth]{figures/system_design_v2.png}
\caption{We introduce \systemName, a VSCode extension to collect human preferences of code directly in a developer's IDE. \systemName enables developers to use code completions from various models. The system comprises a) the interface in the user's IDE which presents paired completions to users (left), b) a sampling strategy that picks model pairs to reduce latency (right, top), and c) a prompting scheme that allows diverse LLMs to perform code completions with high fidelity.
Users can select between the top completion (green box) using \texttt{tab} or the bottom completion (blue box) using \texttt{shift+tab}.}
\label{fig:overview}
\end{figure*}

As model capabilities improve, large language models (LLMs) are increasingly integrated into user environments and workflows.
For example, software developers code with AI in integrated developer environments (IDEs)~\citep{peng2023impact}, doctors rely on notes generated through ambient listening~\citep{oberst2024science}, and lawyers consider case evidence identified by electronic discovery systems~\citep{yang2024beyond}.
Increasing deployment of models in productivity tools demands evaluation that more closely reflects real-world circumstances~\citep{hutchinson2022evaluation, saxon2024benchmarks, kapoor2024ai}.
While newer benchmarks and live platforms incorporate human feedback to capture real-world usage, they almost exclusively focus on evaluating LLMs in chat conversations~\citep{zheng2023judging,dubois2023alpacafarm,chiang2024chatbot, kirk2024the}.
Model evaluation must move beyond chat-based interactions and into specialized user environments.



 

In this work, we focus on evaluating LLM-based coding assistants. 
Despite the popularity of these tools---millions of developers use Github Copilot~\citep{Copilot}---existing
evaluations of the coding capabilities of new models exhibit multiple limitations (Figure~\ref{fig:motivation}, bottom).
Traditional ML benchmarks evaluate LLM capabilities by measuring how well a model can complete static, interview-style coding tasks~\citep{chen2021evaluating,austin2021program,jain2024livecodebench, white2024livebench} and lack \emph{real users}. 
User studies recruit real users to evaluate the effectiveness of LLMs as coding assistants, but are often limited to simple programming tasks as opposed to \emph{real tasks}~\citep{vaithilingam2022expectation,ross2023programmer, mozannar2024realhumaneval}.
Recent efforts to collect human feedback such as Chatbot Arena~\citep{chiang2024chatbot} are still removed from a \emph{realistic environment}, resulting in users and data that deviate from typical software development processes.
We introduce \systemName to address these limitations (Figure~\ref{fig:motivation}, top), and we describe our three main contributions below.


\textbf{We deploy \systemName in-the-wild to collect human preferences on code.} 
\systemName is a Visual Studio Code extension, collecting preferences directly in a developer's IDE within their actual workflow (Figure~\ref{fig:overview}).
\systemName provides developers with code completions, akin to the type of support provided by Github Copilot~\citep{Copilot}. 
Over the past 3 months, \systemName has served over~\completions suggestions from 10 state-of-the-art LLMs, 
gathering \sampleCount~votes from \userCount~users.
To collect user preferences,
\systemName presents a novel interface that shows users paired code completions from two different LLMs, which are determined based on a sampling strategy that aims to 
mitigate latency while preserving coverage across model comparisons.
Additionally, we devise a prompting scheme that allows a diverse set of models to perform code completions with high fidelity.
See Section~\ref{sec:system} and Section~\ref{sec:deployment} for details about system design and deployment respectively.



\textbf{We construct a leaderboard of user preferences and find notable differences from existing static benchmarks and human preference leaderboards.}
In general, we observe that smaller models seem to overperform in static benchmarks compared to our leaderboard, while performance among larger models is mixed (Section~\ref{sec:leaderboard_calculation}).
We attribute these differences to the fact that \systemName is exposed to users and tasks that differ drastically from code evaluations in the past. 
Our data spans 103 programming languages and 24 natural languages as well as a variety of real-world applications and code structures, while static benchmarks tend to focus on a specific programming and natural language and task (e.g. coding competition problems).
Additionally, while all of \systemName interactions contain code contexts and the majority involve infilling tasks, a much smaller fraction of Chatbot Arena's coding tasks contain code context, with infilling tasks appearing even more rarely. 
We analyze our data in depth in Section~\ref{subsec:comparison}.



\textbf{We derive new insights into user preferences of code by analyzing \systemName's diverse and distinct data distribution.}
We compare user preferences across different stratifications of input data (e.g., common versus rare languages) and observe which affect observed preferences most (Section~\ref{sec:analysis}).
For example, while user preferences stay relatively consistent across various programming languages, they differ drastically between different task categories (e.g. frontend/backend versus algorithm design).
We also observe variations in user preference due to different features related to code structure 
(e.g., context length and completion patterns).
We open-source \systemName and release a curated subset of code contexts.
Altogether, our results highlight the necessity of model evaluation in realistic and domain-specific settings.





	
\section{Preliminaries} \label{sec:prelims}
Before diving into the technical results, we state the basic graph notations used throughout the paper and recap the new non-standard definitions we have introduced throughout \Cref{sec:overview}.

\paragraph{Graphs.}
Throughout we consider directed simple graphs $G = (V, E)$, where $E \subseteq V^2$, with $n = |V|$ nodes and $m = |E|$ edges. The edges of the graph can be associated with some value: a length $\ell(e)$ or a capacity/cost $c(e)$, all of which we require to be positive. For any $U \subseteq V$, we write $\overline U = V \setminus U$. Let $G[U]$ be the subgraph induced by $U$. We denote with $\delta^{+}(U)$ the set of edges that have their starting point in $U$ and endpoint in~$\overline U$. We define $\delta^{-}(U)$ symmetrically. We also sometimes write $c(S) = \sum_{e \in S} c(e)$ (for a set of edges $S$) or $c(U, W) = \sum_{e \in E \cap (U \times W)} c(e)$ and $c(U) = c(U, U)$ (for sets of nodes $U, W$).

The distance between two nodes $v$ and $u$ is written $d_G(v,u)$ (throughout we consider only the \emph{length} functions to be relevant for distances). We may omit the subscript if it is clear from the context. The diameter of the graph is the maximum distance between any pair of nodes. For a subgraph $G'$ of $G$ we occasionally say that~$G'$ has \emph{weak diameter} $D$ if for all pairs of nodes $u, v$ in~$G'$, we have $d_G(u, v), d_G(v, u) \leq D$. A strongly connected component in a directed graph $G$ is a subgraph where for every pair of nodes $v,u$ there is a path from $v$ to $u$ and vise versa. Finally, for a radius $r \geq 0$ we write $B^+(v, r) = \set{x \in V : d_G(v, x) \leq r}$ and $B^-(v, r) = \set{y \in V : d_G(y, v) \leq r}$.


\paragraph{Polynomial Bounds.}
For graphs with edge lengths (or capacities), we assume that they are positive and the maximum edge length is bounded by $\poly(n)$. This is only for the sake of simplicity in \cref{sec:ldd-expander,sec:ldd-deterministic} (where in the more general case that all edge lengths are bounded by some threshold $W$ some logarithmic factors in $n$ become $\log (nW)$ instead), and is not necessary for our strongest LDD developed in \cref{sec:ldd-fast}.

\paragraph{Expander Graphs.}
Let $G = (V, E, \ell, c)$ be a directed graph with positive edge capacities $c$ and positive unit edge lengths $\ell$. We define the \emph{volume $\vol(U)$} by
\begin{equation*}
	\vol(U) = c(U, V) = \sum_{e \in E \cap (U \times V)} c(e),
\end{equation*}
and set $\minvol(U) = \min\set{\vol(U), \vol(\overline U)}$ where $\overline U = V \setminus U$. A node set $U$ naturally corresponds to a cut $(U, \overline U)$. The \emph{sparsity} (or \emph{conductance}) of $U$ is defined by
\begin{equation*}
	\phi(U) = \frac{c(U, \overline U)}{\minvol(U)}.
\end{equation*}
In the special cases that $U = \emptyset$ we set $\phi(U) = 1$ and in the special case that $U \neq \emptyset$ but $\vol(U) = 0$, we set $\phi(U) = 0$.
We say that $U$ is \emph{$\phi$-sparse} if $\phi(U) \leq \phi$. We say that a directed graph is a $\phi$-expander if it does not contain a $\phi$-sparse cut $U \subseteq V$. 
We define the \emph{lopsided sparsity} of $U$ as
\begin{equation*}
	\psi(U) = \frac{c(U, \overline U)}{\minvol(U) \cdot \log \frac{\vol(V)}{\minvol(U)}},
\end{equation*}
(with similar special cases), and we similarly say that $U$ is \emph{$\psi$-lopsided sparse} if $\psi(U) \leq \psi$. Finally, we call a graph a \emph{$\psi$-lopsided expander} if it does not contain a $\psi$-lopsided sparse cut $U \subseteq V$.


\section{Metric Forest Completion}
\label{sec:mfc}
We now formalize our \mfc{} (MFC) framework\footnote{This is distinct from two other MST-related concepts that also use the acronym MFC; see Section~\ref{sec:related} for details.}
which assumes access to an initial forest that is then grown into a full spanning tree.

 \subsection{Formalizing the MFC problem}
\label{sec:definemfc}
As a starting point for the metric MST problem on $(\mathcal{X},d)$, we assume access to a partitioning $\mathcal{P} = \{P_1, P_2, \hdots , P_t\}$ where $\mathcal{X} = \bigcup_{i = 1}^t P_i$ and $P_i \cap P_j = \emptyset$ for $i \neq j$. For each component $P_i$ we have a partition spanning tree $T_i = (P_i, E_{T_i})$ for that component.
% , which may or may not be an optimal MST for $P_i$. 
See Figure~\ref{fig:warmstart} for an illustration. Let $G_t = (\mathcal{X}, E_t)$ represent the union of these trees, which has the same node set as $G_\mathcal{X}$, and edge set $E_t = \bigcup_{i = 1}^t E_{T_i}$. Each set $P_i$ for $i \in [t]$ defines a group of points in $\mathcal{X}$ as well as a connected component of $G_t$. We refer to this as the \emph{initial forest} for MFC. To provide intuition, the initial forest can be viewed as a proxy for the forest obtained at an intermediate step of Kruskal's or Boruvka's algorithm (see Figure~\ref{fig:partial_MST_1}). While this serves as a useful analogy, we stress that the partitioning will typically be obtained using much cheaper methods and will not satisfy any formal approximation guarantees. Section~\ref{sec:initialforest} covers practical considerations about strategies, runtimes, and quality measures for an initial forest. For now we simply assume it is given as a ``good enough'' starting point, that will ideally overlap, even if not perfectly, with some true MST (see Figure~\ref{fig:overlap}). 
%The goal is to efficiently grow or \emph{complete} this partial tree into a spanning tree for $G_\mathcal{X}$. 

%Our theoretical analysis requires very few assumptions about the partitioning $\mathcal{P}$ and partition spanning trees (initial forest) $\{T_i \colon i = 1,2, \hdots t\}$ that are given as input to the \mfc{} problem. 
%In particular, we need not assume that $T_i$ is a minimum spanning tree or even a good spanning tree of $P_i$. Similarly, the components are not required to be sets of points that are close together in the metric space in order for the problem to be well-defined. Nevertheless, \textit{in practice} the hope is that $\mathcal{P}$ identifies groups of nearby points in $\mathcal{X}$ and that $T_i$ is a reasonably good spanning tree for $P_i$ for each $i \in [t] = \{1,2, \hdots, t\}$. Appendix~\ref{sec:initialsubtree} covers two practical strategies and corresponding runtimes for finding an initial forest. In summary these are:
%
%\textit{Strategy 1: $k$-centering.} Apply a fast $k$-centering heuristic to $\mathcal{X}$ to form $\mathcal{P}$, e.g., using a simple 2-approximation~\cite{gonzalez1985clustering} or fast distributed methods~\cite{malkomes2015fast,mcclintock2016efficient}.
%
%\textit{Strategy 2: $k$-NN graph.}  Compute an approximate $k$-nearest neighbor graph of $\mathcal{X}$ for a reasonably small $k$, e.g., via the scalable $k$-NN descent algorithm~\cite{dong2011efficient} or its distributed generalization~\cite{iwabuchi2023towards}.
%
%Similar but more restrictive strategies have been used by previous heuristics for Euclidean MSTs (see Appendix~\ref{sec:initialsubtree}).  

%\textbf{Defining the MFC problem.}
% Given an initial forest, the goal of 
\mfc{} seeks to connect the initial forest into a spanning tree for $G_\mathcal{X}$. Let $P(x) \in \mathcal{P}$ denote the component that $x \in \mathcal{X}$ belongs to. The set of inter-component edges is
\begin{equation*}
	\label{eq:intercomponent}
	\mathcal{I} = \{ (x, y) \in \mathcal{X} \times \mathcal{X} \colon P(x) \neq P(y) \}.
\end{equation*}
We wish to find a minimum weight set of edges $M \subseteq \mathcal{I}$ so that $M\cup E_t$ defines a connected graph on $\mathcal{X}$. If $M$ satisfies these constraints we say it is a valid \emph{completion set} and that $M$ \emph{completes} $\mathcal{P}$. 
The MFC problem can then be written as
\begin{equation}
	\label{eq:mfc}
	\begin{array}{ll}
		\minimize & w_\mathcal{X}(M) + w_\mathcal{X}(E_t)\\
		\text{subject to} & M \emph{ completes } \mathcal{P}.
	\end{array}
\end{equation}
Let $M^*$ denote an optimal completion set. The graph $T^* = (\mathcal{X}, E_t \cup M^*)$ is then guaranteed to be a tree (see Figure~\ref{fig:opt_mfc}); if not we could remove edges to decrease the weight while still spanning $\mathcal{X}$. If the initial forest $G_t$ is in fact contained in some optimal spanning tree of $G_\mathcal{X}$ (which would be the case if it were obtained by running a few iterations of Kruskal's or Boruvka's algorithm), then solving MFC would produce an MST of $G_\mathcal{X}$. In practice this will typically not be the case, but the problem remains well-defined regardless of any assumptions about the quality of the initial forest.

\begin{figure}[t]
	\centering
	\begin{subfigure}[b]{0.32\textwidth}
		\centering
		\includegraphics[width=\textwidth]{figures/true_MST.pdf}
		\caption{True metric MST}
		\label{fig:truemst}
	\end{subfigure}
	\begin{subfigure}[b]{0.32\textwidth}
		\centering
		\includegraphics[width=\textwidth]{figures/warm-start.pdf}
		\caption{Partial spanning tree ($\mathcal{P}$; $\{T_i\})$}
		\label{fig:warmstart}
	\end{subfigure}
	\begin{subfigure}[b]{0.32\textwidth}
		\centering
		\includegraphics[width=\textwidth]{figures/true_MST_clusters.pdf}
		\caption{True MST overlap with $\mathcal{P}$}
		\label{fig:overlap}
	\end{subfigure}
	\begin{subfigure}[b]{0.32\textwidth}
		\centering
		\includegraphics[width=\textwidth]{figures/opt_mstc.pdf}
		\caption{MFC solution ($M^*$ in orange)}
		\label{fig:opt_mfc}
	\end{subfigure}
	\begin{subfigure}[b]{0.32\textwidth}
		\centering
		\includegraphics[width=\textwidth]{figures/componentgraph.pdf}
		\caption{Coarsened graph $G_\mathcal{P}$ w.r.t.\ $w^*$}
		\label{fig:coarsenedgraph}
	\end{subfigure}
	\begin{subfigure}[b]{0.32\textwidth}
		\centering
		\includegraphics[width=\textwidth]{figures/component_mst.pdf}
		\caption{MST of $G_\mathcal{P}$ w.r.t.\ $w^*$}
		\label{fig:mstcoarsened}
	\end{subfigure}
	\caption{(a) We display an optimal metric MST for a toy example with $|\mathcal{X}| = 75$ points. Our framework and algorithm apply to general metric spaces, but for visualization purposes our figures focus on 2-dimensional Euclidean space. (b) The \mfc{} problem is given a partitioning $\mathcal{P}$ and spanning trees $\{T_i\}$ for components of the partition. For this illustration we used a $k$-means algorithm with $k = 5$ computed optimal spanning trees of components using the naive approach. (c)~The true MST overlaps significantly with the initial partial spanning tree, but its induced subgraph on each component is not necessarily connected. For this example, the $\gamma$-overlap (see Section~\ref{sec:learningaugmented}) is $\gamma \leq 1.12$. (d)~The optimal completion set $M^*$ is shown in orange; combining it with the spanning trees of the partial spanning tree produces a spanning tree for all of $\mathcal{X}$. (e)~The coarsened graph $G_\mathcal{P}$ has a node $v_i$ for each component $P_i \in \mathcal{P}$. Solving $O(t^2)$ bichromatic closest pair problems identifies the closest pair of points between each pair of clusters, defining an optimal weight function $w^*$ on $G_\mathcal{P}$. (f)~Finding the minimum-weight completion set $M^*$ amounts to finding the MST of $G_\mathcal{P}$ with respect to weight function $w^*$. }
	\label{fig:three_figures}
\end{figure}


The objective function in Eq.~\eqref{eq:mfc} includes the weight of the initial forest $w_\mathcal{X}(E_t)$. Although this is constant with respect to $M$ and does not affect optimal solutions, there are several reasons to incorporate this term explicitly. Most importantly, our ultimate goal is to obtain a good spanning tree for all of $G_\mathcal{X}$, and thus the weight of the full spanning tree (i.e., the objective in Eq.~\ref{eq:mfc}) is a more natural measure. Considering the weight of the full spanning tree also makes more sense in the context of our learning-augmented algorithm analysis, where the goal is to approximate the original metric MST problem on $G_\mathcal{X}$, under different assumptions about the initial forest. We note finally that excluding the term $w_\mathcal{X}(E_t)$ rules out the possibility of any meaningful approximation results. 
We prove the following result using a reduction from BCP to MFC, combined with a slight variation of a simple lower bound for monochromatic closest pair that was shown in Section 9 of Indyk~\cite{indyk1999sublinear}.
\begin{theorem}
\label{thm:hard}
Every optimal algorithm for MFC has $\Omega(n^2)$ query complexity. Furthermore, for any multiplicative factor $p \geq 1$ (not necessarily a constant), any algorithm that finds a set $M \subseteq \mathcal{I}$ that is feasible for~\eqref{eq:mfc} and satisfies $w_\mathcal{X}(M) \leq p \cdot w_{\mathcal{X}}(M^*)$ requires $\Omega(n^2)$ queries.
\end{theorem}
\begin{proof}
	Let $\mathcal{X}$ be a set of $n$ points that are partitioned into two sets $P_1$ and $P_2$ of size $n/2$. Define a distance function $d$ such that $d(a,b) = 1$ for a randomly chosen pair $(a,b) \in P_1 \times P_2$, and such that $d(x,y) = 2p$ for all other pairs $(x,y) \in {\mathcal{X} \choose 2} \backslash \{(a,b)\}$. Note that this $d$ is a metric. The MFC problem on this instance is identical to solving BCP on $P_1$ and $P_2$. The unique optimal solution is exactly $M^* = (a,b)$, and no other choice of $M \subseteq \mathcal{I}$ comes within a factor $p$ of this solution. Thus, any $p$-approximation algorithm must find the pair $(a,b)$ with distance 1 among a collection of $\Omega(n^2)$ pairs, where all other pairs have distance $2p$. This requires $\Omega(n^2)$ queries.
\end{proof}
Theorem~\ref{thm:hard} shows that it is impossible in general to find optimal solutions for MFC, or multiplicative approximations for $w_\mathcal{X}(M^*)$, in $o(n^2)$ time. However, this does not rule out the possibility of approximating the more relevant objective $w_\mathcal{X}(M) + w_\mathcal{X}(E_t)$.

\paragraph{The MFC coarsened graph.} The  MFC problem is equivalent to finding a minimum spanning tree in a \textit{coarsened graph} $G_\mathcal{P} = (V_\mathcal{P}, E_\mathcal{P})$ with node set $V_\mathcal{P} = \{v_1, v_2, \hdots, v_t\}$ where $v_i$ represents component $P_i$. We refer to $v_i$ as the $i$th \textit{component node}. This graph is complete: $E_\mathcal{P}$ includes all pairs of component nodes. Figure~\ref{fig:coarsenedgraph} provides an illustration of the coarsened graph. Finding an MFC solution $M^* \subseteq \mathcal{I}$ is equivalent to finding an MST in $G_\mathcal{P}$ (see Figure~\ref{fig:mstcoarsened}) with respect to the weight function $w^* \colon E_\mathcal{P} \rightarrow \mathbb{R}^+$ defined for every $i, j \in \{1,2, \hdots, t\}$ by
\begin{equation}
	\label{eq:wstar}
	w^*_{ij} = w^*(v_i, v_j) = d(P_i, P_j) = \min_{x \in P_i, y\in P_j} d(x,y).
	%, \quad \text{ for every $P_i, P_j \in \mathcal{P} \times \mathcal{P}.$}
\end{equation}
Computing $w_{ij}^*$ exactly requires solving a bichromatic closest pair problem over sets $P_i$ and $P_j$. A straightforward approach for computing $w_{ij}^*$ is to check all $|P_i|\cdot |P_j|$ pairs of points in $P_i \times P_j$. The number of distance queries needed to apply this simple strategy to form all of $w^*$ is
\begin{equation*} 
	\frac{1}{2}\sum_{i = 1}^t |P_i| \cdot (n - |P_i|) = \frac{n^2}{2} - \frac{1}{2} \sum_{i = 1}^t |P_i|^2.
\end{equation*}
In a worst-case scenario where component sizes are balanced, we would need
$\Omega\left({t \choose 2} \frac{n}{t} \frac{n}{t}\right) = \Omega(n^2)$ 
queries, which is not surprising given Theorem~\ref{thm:hard}. Nevertheless, this notion of a coarsened graph will be very useful in developing approximation algorithms for MFC.

\subsection{Computing an initial forest}
\label{sec:initialforest}
Our theoretical analysis requires very few assumptions about the partitioning $\mathcal{P}$ and partition spanning trees $\{T_i \colon i = 1,2, \hdots t\}$ that are given as input to the \mfc{} problem. In particular, we need not assume that $T_i$ is a minimum spanning tree or even a good spanning tree of $P_i$. Similarly, the components are not required to be sets of points that are close together in the metric space in order for the problem to be well-defined. Nevertheless, \textit{in practice} the hope is that $\mathcal{P}$ identifies groups of nearby points in $\mathcal{X}$ and that $T_i$ is a reasonably good spanning tree for $P_i$ for each $i \in [t] = \{1,2, \hdots, t\}$. 

In order for our approach to be meaningful for large-scale metric MST problems, we must be able to obtain a reasonably good initial forest without this dominating our overall algorithmic pipeline.
Here we discuss two specific strategies for initial forest computations, both of which are fast, easy to parallelize, and motivated by techniques that are already being used in practice in large-scale clustering pipelines. In particular, there are already a number of existing heuristics for finding MSTs and hierarchical clusters that rely in some way on partitioning an initial dataset and then connecting or merging components~\cite{zhong2015fast,jothi2018fast,mishra2020efficient,chen2013clustering}. See Section~\ref{sec:related} for more details. These typically focus only on point cloud data, do not apply to arbitrary metric spaces, and do not come with any type of approximation guarantee. Nevertheless, they provide examples of fast heuristics for large-scale metric spanning tree problems, and serve as motivation for our more general strategies.

\paragraph{Strategy 1: Components of a $k$-NN graph.}
A natural way to obtain an initial forest for $G_\mathcal{X}$ is to compute an approximate $k$-nearest neighbor graph for a reasonably small $k$, which can be accomplished with the $k$-NN descent algorithm~\cite{dong2011efficient} or a recent distributed generalization of this method~\cite{iwabuchi2023towards}. This efficiently connects a large number of points using small-weight edges. The $k$-NN graph will often be disconnected, and we can use the set of connected components as our initial components $\mathcal{P} = \{P_1, P_2, \hdots , P_t\}$. The exact number of components will depend on the distribution of the data and the number of nearest neighbors computed. For larger values of $k$, the $k$-NN graph is more expensive to compute, but then there are fewer components to connect, so there are trade-offs to consider. The components of the $k$-NN graph will typically not be trees but will be sparse (each node has at most $O(k)$ edges), so we can use classical MST algorithms to find spanning trees for all components in $\sum_{i = 1}^t \tilde{O}(k \cdot |P_i|) = \tilde{O}(kn)$ time. The exact runtime of the $k$-NN descent algorithm depends on various parameters settings, but prior work reports an empirical runtime of $O(n^{1.14})$~\cite{dong2011efficient}, with strong empirical performance across a range of different metrics and dataset sizes. We remark that $k$-NN computations have already been used elsewhere as subroutines for large-scale Euclidean MST computations~\cite{almansoori2024fast,chen2013clustering}.

\paragraph{Strategy 2: Fast clustering heuristics.}
Another approach is to form components of the initial forest $\mathcal{P} = \{P_1, P_2, \hdots, P_t\}$ by applying a fast clustering heuristic to $\mathcal{X}$ such as a distributed $k$-center algorithm~\cite{malkomes2015fast,mcclintock2016efficient}. Even the simple sequential greedy 2-approximation algorithm for $k$-center can produce an approximate clustering using $O(nk)$ queries~\cite{gonzalez1985clustering}. Approximate or exact minimum spanning trees for each $P_i$ can be found in parallel. The remaining step is to find a good way to connect the forest. Similar approaches that partition the initial dataset using $k$-means clustering also exist~\cite{zhong2015fast,jothi2018fast}, though this inherently forms clusters based on Euclidean distances. For all of these clustering-based approaches, the number of components $t$ for the initial forest is easy to control since it exactly corresponds to the number of clusters $k$. Smaller $k$ leads to larger clusters, and hence finding an MST of each $P_i$ is more expensive. However, there are then fewer components to connect, so there is again a trade-off to consider. 
%
We remark that it may seem counterintuitive to use a clustering method as a subroutine for finding an MST, since one of the main reasons to compute an MST is to perform clustering. We stress that Strategy 2 uses a cheap and fast clustering method that identifies sets of points that are somewhat close in the metric space, without focusing on whether they are good clusters for a downstream application. This speeds up the search for a good spanning tree, which can be used as one step of a more sophisticated hierarchical clustering pipeline. \\


\subsection{MFC as a learning-augmented framework}
\label{sec:learningaugmented}
Our approach fits the framework of learning-augmented algorithms in that the initial forest can be viewed as a prediction for a partial metric MST, such as the forest obtained by running several iterations of a classical MST algorithm. 
In an ideal setting, the initial forest would be a subset of an optimal MST. If so, then an optimal solution to MFC would produce an optimal metric MST. We relax this by introducing a more general way to measure how much an optimal MST ``overlaps'' with initial forest components. 
Let $\mathcal{T}_\mathcal{X}$ represent the set of minimum spanning trees of $G_\mathcal{X}$, and $T \in \mathcal{T}_\mathcal{X}$ denote an arbitrary MST.
For components $\mathcal{P} =  \{P_1, P_2, \hdots, P_t\}$, let $T(P_i)$ denote the induced subgraph of $T$ on $P_i$, and let $T(\mathcal{P}) = \bigcup_{i = 1}^t T(P_i)$ denote the edges of $T$ contained inside components of $\mathcal{P}$. 
% Then, $w(T(\mathcal{P}))$ is the total weight that an optimal MST $T$ places inside components of $\mathcal{P}$. 
Larger values of $w_\mathcal{X}(T(\mathcal{P}))$ indicate better initial forests, since this means an optimal MST places a larger weight of edges inside these components. 
% Given components $\mathcal{P} =  \{P_1, P_2, \hdots, P_t\}$ and corresponding spanning trees $\{T_1, T_2, \hdots, T_t$\}, 
We define the $\gamma$-overlap for the initial forest to be the ratio
\begin{equation}
\label{eq:gamma}
\gamma(\mathcal{P})  = \frac{w_\mathcal{X}(E_t)}{\max_{T \in \mathcal{T}_\mathcal{X}} w_\mathcal{X}(T(\mathcal{P}))}.
\end{equation}
This measures the weight of edges that the initial forest places inside $\mathcal{P}$, relative to the weight of edges an optimal MST places inside $\mathcal{P}$. 
When $\mathcal{P}$ is clear from context we will simply write $\gamma$. 
Lower ratios for $\gamma$ are better. 
In the denominator, we maximize $w_\mathcal{X}(T(\mathcal{P}))$ over all optimal spanning trees since any MST of $G_\mathcal{X}$ is equally good for our purposes; hence we are free to focus on the MST that overlaps most with $\mathcal{P}$. The optimality of $T$ implies that $w_\mathcal{X}(T(P_i)) \leq w_\mathcal{X}(T_i)$ for every $i \in \{1,2, \hdots, t\}$, which in turn implies that $\gamma(\mathcal{P}) \geq 1$ always. Even if $T_i$ is a minimum spanning tree for $P_i$, it is possible to have $w_\mathcal{X}(T(P_i)) < w_\mathcal{X}(T_i)$ since the induced subgraph $T(P_i)$ is not necessarily connected (see Figure~\ref{fig:overlap} for an example). We achieve the lower bound $\gamma = 1$ exactly in the idealized setting where the initial forest is contained in some optimal metric MST. In practice, we expect the clusters $P_i$ and spanning trees $T_i$ to be imperfect in the sense that connecting them will likely not provide an optimal MST for $G_X$. However, for an initial forest where each $P_i$ is a set of nearby points and $T_i$ is a reasonably good spanning tree for $P_i$, we would expect $\gamma$ to be larger than 1 but still not too large. Figure~\ref{fig:overlap} provides an example where we can certify that $\gamma \leq 1.12$ by comparing against one optimal MST. We later show experimentally that we can quickly obtain initial forests with small $\gamma$-overlap for a wide range of datasets and metrics. 

\paragraph{Learning-augmented algorithm guarantees.}
 We typically would not compute the ratio $\gamma$ for large-scale applications as this would be even more computationally expensive than solving the original metric MST problem. However, this serves as a theoretical measure of initial forest quality when proving approximation guarantees, following the standard approach in the analysis of learning-augmented algorithms. 
In the worst case, no algorithm making $o(n^2)$ queries can obtain a constant factor approximation for metric MST in arbitrary metric spaces~\cite{indyk1999sublinear}.  Following the standard goal of learning-augmented algorithms, we will improve on this worst-case setting when the initial forest is good; otherwise we will recover the worst-case behavior. Formally, we will consider the initial forest to be good when $\gamma$ is finite and the number of components is $t = o(n)$. For this setting, Section~\ref{sec:algs} will present a learning-augmented $(2\gamma+1)$-approximation algorithm for metric MST with subquadratic complexity. There are two ways to see that this is no worse than the worst-case guarantee using no initial forest, depending on whether we focus on worst-case runtimes or wort-case approximation factors. From the perspective of runtimes, if $t = \Omega(n)$ we can default to the worst-case quadratic complexity algorithms that provide an optimal MST. From the perspective of approximation factors, we certainly do no worse than the worst-case approximation factor, which is infinite for subquadratic algorithms. We will state our approximation results and runtimes more precisely in Section~\ref{sec:algs}.



\section{Approximate Completion Algorithm}
\label{sec:algs}
\begin{algorithm}[tb]
	\caption{\textsf{MFC-Approx}}
	\label{alg:mfcapprox}
	\begin{algorithmic}[5]
		\State{\bfseries Input:} $\mathcal{X} = \{x_1, x_2, \hdots , x_n\}$, components $\mathcal{P} = \{P_1, P_2, \hdots, P_t\}$, spanning trees $\{T_1, T_2, \hdots, T_t\}$
		\State {\bfseries Output:} Spanning tree for implicit metric graph of $\mathcal{X}$
		\For{$i = 1, 2, \hdots t$}
		\State Select arbitrary component representative $s_i \in P_i$
		\EndFor
		\For{$i = 1, 2, \hdots t-1$}
		\For{$j = i+1, \hdots , t$}
		\State $w_{i \rightarrow j} = \min_{x_i \in P_i} d(x_i, s_j)$  \quad \hfill \texttt{ // closest a $P_i$ node comes to $s_j$}
		\State $w_{j \rightarrow i} = \min_{x_j \in P_j} d(x_j, s_i)$ \quad \hfill  \texttt{ // closest a $P_j$ node comes to $s_i$}
		\State $\hat{w}_{ij} = \min \{w_{i \rightarrow j}, w_{j \rightarrow i}\}$ \quad  \hfill \texttt{// set weight for edge $(v_i, v_j)$} 
		\EndFor
		\EndFor
		\State $\hat{T}_{\mathcal{P}} = \textsc{OptimalMST}(\{\hat{w}_{ij}\}_{i,j \in [t]})$ \hfill \texttt{ // find optimal MST on complete $t$-node graph}
		\State Return spanning tree $\hat{T}$ of $G_\mathcal{X}$ by combining $\bigcup_{i=1}^t T_i$ with edges from $\hat{T}_{\mathcal{P}}$.
	\end{algorithmic}
\end{algorithm}

We now present an algorithm that approximates MFC to within a factor $c < 2.62$. We also prove it can be viewed as a learning-augmented algorithm for metric MST, where the approximation factor depends on the $\gamma$-overlap of the initial forest. 
%Pseudocode for our algorithm is provided in the appendix. Here in the main text we give a full description of the algorithm along with visual aids in Figure~\ref{fig:mfcapprox}. Due to space constraints, proofs are relegated to the appendix.



\begin{figure*}[t]
	\centering
	\begin{subfigure}[b]{0.33\textwidth}
		\centering
		\includegraphics[width=\textwidth]{figures/wijhat.pdf}
		%        \vspace{-10pt}
		\caption{Computing $\hat{w}_{ij}$}
		\label{fig:wijhat}
	\end{subfigure}\hfill
	\begin{subfigure}[b]{0.33\textwidth}
		\centering
		\includegraphics[width=\textwidth]{figures/componentsgraphhat.pdf}
		%        \vspace{-10pt}
		\caption{$G_\mathcal{P}$ w.r.t.\ $\hat{w}$}
		\label{fig:componentsgraphhat}
	\end{subfigure}\hfill
	\begin{subfigure}[b]{0.33\textwidth}
		\centering
		\includegraphics[width=\textwidth]{figures/approx_mstc.pdf}
		%         \vspace{-10pt}
		\caption{\textsf{MFC-Approx} output}
		\label{fig:approx_mfc}
	\end{subfigure}
	\caption{(a) Finding the minimum distance between components $P_i$ and $P_j$ (dashed line) is an expensive bichromatic closest pair problem. \textsf{MFC-Approx} instead performs a cheaper nearest neighbor query for a \emph{representative} point in each component ($s_i$ and $s_j$, shown as stars). The algorithm finds the closest point to each representative from the opposite cluster, then takes the minimum of the two distances. 
		(b) Applying this to each pair of components produces a weight function $\hat{w}$ for the coarsened graph $G_\mathcal{P}$. Finding an MST of $G_\mathcal{P}$ with respect to $\hat{w}$ yields (c) a 2.62-approximation for MFC.}
	\label{fig:mfcapprox}
\end{figure*}


%\subsection{Algorithm description} 
Pseodocode for our method, which we call \textsf{MFC-Approx}, is shown in Algorithm~\ref{alg:mfcapprox}. This algorithm starts by choosing an arbitrary point $s_i \in P_i$ for each $i \in \{1,2, \hdots, t\}$ to be the component's \emph{representative} (starred nodes in Figure~\ref{fig:mfcapprox}). 
The algorithm computes the distance between every point $x \in \mathcal{X}$ and all of the other representatives. For every pair of distinct components $i$ and $j$ we compute the weights:
\begin{align}
	w_{i \rightarrow j} &= \min_{x_i \in P_i} d(x_i, s_j)  	&\text{(the closest $P_i$ node to $s_j$)} \\
	w_{j \rightarrow i} &= \min_{x_j \in P_j} d(x_j, s_i)  &\text{(the closest $P_j$ node to $s_i$).} 
\end{align}
We then define the approximate edge weight between component nodes $v_i$ and $v_j$ to be:
\begin{align}
	\label{eq:approxweight}
	\hat{w}_{ij} = \min \{w_{i \rightarrow j}, w_{j \rightarrow i}\}.
\end{align}
This upper bounds the minimum distance $w_{ij}^*$ between the two components (Figure~\ref{fig:wijhat}).
Computing this for all pairs of components creates a new weight function $\hat{w}$ for the
coarsened graph $G_\mathcal{P}$ (Figure~\ref{fig:componentsgraphhat}). The algorithm keeps track of the points in $\mathcal{X}$ that define these edge weights in $G_\mathcal{P}$. It then computes an MST in $G_\mathcal{P}$ with respect to $\hat{w}$, then identifies the corresponding edges in $G_\mathcal{X}$, to produce a feasible solution for MFC (Figure~\ref{fig:approx_mfc}). 





\subsection{Bounded and unbounded edges in the coarsensed graph}
A key step of our approximation analysis is to separate edges of the coarsened graph $G_\mathcal{P} = (V_\mathcal{P},E_\mathcal{P})$ into two categories, depending on the relationship between $\hat{w}$ and $w^*$.
% \begin{definition}
Formally, for an arbitrary constant $\beta \geq 1$, we say edge $(v_i,v_j) \in E_\mathcal{P}$ is \textit{$\beta$-bounded} if $\hat{w}_{ij}  \leq \beta w^*_{ij}$, otherwise it is \textit{$\beta$-unbounded}. 
If all edges in $G_\mathcal{P}$ were $\beta$-bounded, finding an MST in $G_\mathcal{P}$ with respect to $\hat{w}$ would produce a $\beta$-approximation for the MST problem in $G_\mathcal{P}$ with respect to $w^*$. In general we cannot guarantee all edges will be $\beta$-bounded, as this would imply Algorithm~\ref{alg:mfcapprox} is a subquadratic $\beta$-approximation algorithm for MFC, contradicting Theorem~\ref{thm:hard}. Nevertheless, any $\beta$-bounded edge that Algorithm~\ref{alg:mfcapprox} includes in its MST of $G_\mathcal{P}$ is easy to bound in terms of the optimal edge weights $w^*$. 
%
If we include a $\beta$-unbounded edge in our MST of $G_\mathcal{P}$, we can no longer bound its weight in terms of an optimal solution to MFC. However, the following lemma shows that its weight can be bounded in terms of the weight of the initial forest.
\begin{lemma}
	\label{lem:unboundededges}
	Let $P_i$ and $P_j$ be an arbitrary pair of components and let $\beta > 1$. If $\hat{w}_{ij} > \beta w^*_{ij}$, then 
	\begin{align}
		\hat{w}_{ij} <
		%		\frac{\beta}{\beta - 1} \min \{\alpha_i, \alpha_j\} \leq 
		\frac{\beta}{\beta - 1}  \min\{w_\mathcal{X}(T_i), w_\mathcal{X}(T_j)\}.
	\end{align}
\end{lemma}
\begin{proof}
	For each $i \in \{1, 2, \hdots, t\}$, we denote the maximum distance between a point in $P_i$ and its component representative $s_i$ by $\alpha_i = \max_{x \in P_i} d(x, s_i)$.
	Let $x_i^* \in P_i$ and $x_j^* \in P_j$ be points satisfying $d(x_{i}^*, x_{j}^*) = d(P_i, P_j) = w_{ij}^*$.
	We use the (reverse) triangle inequality and the definition of $\alpha_i$ to see that:
	\begin{align*}
		d(x_i^*, x_j^*) = \min_{x_j \in P_j} d(x_i^*, x_j) \geq  \min_{x_j \in P_j} d(s_i, x_j) - d(s_i, x_i^*) \geq  \min_{x_j \in P_j} d(s_i, x_j) - \alpha_i = w_{j \rightarrow i} - \alpha_i \geq \hat{w}_{ij} - \alpha_i.
	\end{align*}
	Similarly we can show that 
	\begin{align*}
		d(x_i^*, x_j^*) = \min_{x_i \in P_i} d(x_i, x_j^*) \geq  \min_{x_i \in P_i} d(s_j, x_i) - d(s_j, x_j^*)  \geq  \min_{x_i \in P_i} d(s_j, x_i) - \alpha_j = w_{i \rightarrow j} - \alpha_j \geq \hat{w}_{ij} - \alpha_j.
	\end{align*}
	In other words, we have the bound $w^*_{ij} \geq \hat{w}_{ij} - \min\{\alpha_i, \alpha_j\}$. Combining this with the assumption that $\hat{w}_{ij} > \beta w_{ij}^*$ gives
	\begin{align*}
		\hat{w}_{ij} &> \beta w_{ij}^* \geq \beta\hat{w}_{ij} - \beta\min \{\alpha_i, \alpha_j\} \implies \frac{\beta}{\beta -1 } \min \{\alpha_i, \alpha_j\}> \hat{w}_{ij}.
	\end{align*}
	The proof follows from the observation that $\alpha_i \leq w_\mathcal{X}(T_i)$. To see why, note that there exists some ${x} \in P_i$ such that $d({x}, s_i) = \alpha_i$. Since $T_i$ is a spanning tree of $P_i$, it must contain a path from $s_i$ to ${x}$ with sum of edge weights at least $\alpha_i$.
\end{proof}
Our main approximation guarantees rely on Lemma~\ref{lem:unboundededges}, as well as two other simple supporting observations. This first amounts to the observation that a tree has arboricity and degeneracy 1.
\begin{observation}
	\label{lem:treeorient}
	If $T = (V,E_T)$ is an undirected tree, there is a way to orient edges in such a way that every node has at most one outgoing edge.
\end{observation}
\begin{proof}
	The proof is constructive. Define an iterative algorithm that removes a minimum degree node at each step and deletes all its incident edges. Orient the deleted edges so that they start at the node that was removed. Note that a tree always contains a node of degree 1, and removing such a node leads to another tree with one fewer node. Thus, this procedure will orient edges of the original graph in such a way that each node has at most one outgoing edge.
\end{proof}
The other supporting result deals with MSTs in a graph that includes edges of weight zero.
\begin{observation}
	\label{obs:Z}
	Let $w^{(1)} \colon E \rightarrow \mathbb{R}^+$ and $w^{(2)} \colon E \rightarrow \mathbb{R}^+$ be two nonnegative weight functions for an undirected graph $G = (V,E)$. Assume there exists an edge set $Z \subseteq E$ such that 
	\begin{equation*}
		w^{(1)}(i,j) = w^{(2)}(i,j) = 0 \text{ for every $(i,j) \in Z$}.
	\end{equation*}
	Then there exist spanning trees $M_1$ and $M_2$ for $G$ such that $M_i$ is an MST for $G$ with respect to $w^{(i)}$ for $i \in \{1,2\}$, and $M_1 \cap Z = M_2 \cap Z$.
\end{observation}
\begin{proof}
	The proof is constructive. Recall that Kruskal's algorithm finds an MST by ordering edges by weight (starting with the smallest and breaking ties arbitrarily in the ordering) and then greedily adds each edge a growing spanning tree if and only if it connects two previously disconnected components. 
	Fix an arbitrary ordering $\sigma_Z$ of edges in $Z$. When applying Kruskal's algorithm to find minimum spanning trees of $G$ with respect to $w^{(1)}$ and $w^{(2)}$, we can choose orderings for these functions that exactly coincide for the first $|Z|$ edges visited. Namely, we place edges in $Z$ first, using the order given by $\sigma_Z$. 
	% The remaining edges $E\backslash Z$ may be ordered differently in the orderings for $w^{(1)}$ and $w^{(2)}$. 
	The first $|Z|$ steps of Kruskal's algorithm will be identical when building MSTs with respect to $w^{(1)}$ and $w^{(2)}$. Thus, if $M_1$ and $M_2$ are the spanning trees obtained for $w^{(1)}$ and $w^{(2)}$ respectively using this approach, we know these trees will include the same set of edges from $Z$ and discard the same set of edges from $Z$, i.e., $M_1\cap Z = M_2 \cap Z.$
\end{proof}


\subsection{Main approximation guarantees}
Let $T^*_\mathcal{P}$ represent an MST of $G_\mathcal{P}$ with respect to the optimal weight function $w^* \colon E_\mathcal{P} \rightarrow \mathbb{R}^+$ and $\hat{T}_\mathcal{P}$ represent an MST of $G_\mathcal{P}$ with respect to the approximate weight function $\hat{w} \colon E_\mathcal{P} \rightarrow \mathbb{R}^+$. The edges of $T^*_\mathcal{P}$ map to a set of edges $M^*$ in $G_\mathcal{X}$ that optimally solves the metric MST completion problem, and the edges in $\hat{T}_\mathcal{P}$ map to an edge set $\hat{M}$. The weight of these edges is given by:
\begin{align*}
	w_\mathcal{X}(M^*) &= w^*(T^*_\mathcal{P}) \\
	w_\mathcal{X}(\hat{M}) &= \hat{w}(\hat{T}_\mathcal{P}).
\end{align*}
Let $T^*$ be the spanning tree of $G_\mathcal{X}$ defined by combining $\bigcup_{i = 1}^t T_i$ with $M^*$ and $\hat{T}$ be the spanning tree (returned by Algorithm~\ref{alg:mfcapprox}) that combines $\bigcup_{i = 1}^t T_i$ with $\hat{M}$. These have weights given by
\begin{align}
	\label{eq:Tstar}
	w_\mathcal{X}(T^*) &= w^*(T_\mathcal{P}^*) +  \sum_{i = 1}^t w_\mathcal{X}(T_i) \\
	\label{eq:That}
	w_\mathcal{X}(\hat{T}) &= \hat{w}(\hat{T}_\mathcal{P}) +  \sum_{i = 1}^t w_\mathcal{X}(T_i) .
\end{align}
We are now ready to prove the approximation guarantee for \textsf{MFC-Approx}.
\begin{theorem}
	\label{thm:main}
	The spanning tree $\hat{T}$ returned by Algorithm~\ref{alg:mfcapprox} satisfies
	\begin{equation*}
		w_\mathcal{X}(T^*) \leq w_\mathcal{X}(\hat{T}) \leq \beta w_\mathcal{X}(T^*)
	\end{equation*}
	for $\beta = (3 + \sqrt{5})/2 < 2.62$. 
\end{theorem}
\begin{proof}
	For our analysis we consider two hypothetical weight functions $w^*_0$ and $\hat{w}_0$ for $G_\mathcal{P} = (V_\mathcal{P}, E_\mathcal{P})$, defined by zeroing out the $\beta$-unbounded edges in $w^*$ and $\hat{w}$:
	\begin{align*}
		w^*_0(v_i, v_j) &= \begin{cases} 
			w^*_{ij} & \text{ if $(v_i,v_j)$ is $\beta$-bounded, i.e., $\hat{w}_{ij} \leq \beta w^*_{ij}$} \\
			0 & \text{ otherwise} 
		\end{cases}\\
		\hat{w}_0(v_i, v_j) &= \begin{cases} 
			\hat{w}_{ij} & \text{ if $(v_i,v_j)$ is $\beta$-bounded, i.e., $\hat{w}_{ij} \leq \beta w^*_{ij}$} \\
			0 & \text{ otherwise}. 
		\end{cases}
	\end{align*}
	By Observation~\ref{obs:Z}, there exist spanning trees $T_0^*$ and $\hat{T}_0$ for $G_\mathcal{P}$ that are optimal with respect to $w_0^*$ and $\hat{w}_0$, respectively, which contain the same exact set of $\beta$-unbounded edges. Let $U$ represent this set of $\beta$-unbounded edges in $\hat{T}_0$ and $T^*_0$. Let $B^*$ be the set of $\beta$-bounded edges in $T^*_0$ and $\hat{B}$ be the set of $\beta$-bounded edges in $\hat{T}_0$. Because $\hat{T}_\mathcal{P}$ is an MST with respect to $\hat{w}$ we know that:
	\begin{align}
		\label{eq:hats}
		\hat{w}(\hat{T}_\mathcal{P}) &\leq \hat{w}(\hat{T}_0) = \hat{w}(U) + \hat{w}(\hat{B}).
	\end{align}
	We will use this to upper bound the weight of $\hat{T}_\mathcal{P}$ in terms of $T^*$. First we claim that
	\begin{equation}
		\label{eq:Lhatbound}
		\hat{w}(\hat{B}) \leq \beta w^*(T_\mathcal{P}^*).
	\end{equation}
	This follows from the following sequence of inequalities:
	\begin{align*}
		\hat{w}(\hat{B}) 
		&= \hat{w}_0(\hat{B}) & \text{ since $\hat{w}$ and $\hat{w}_0$ coincide on $\beta$-bounded edges} \\
		&= \hat{w}_0(\hat{B}) + \hat{w}_0(U) & \text{ since $\hat{w}_0$ is zero on $\beta$-unbounded edges} \\
		&= \hat{w}_0(\hat{T}_0)  & \text{ since $\hat{T}_0 = \hat{B} \cup U$} \\
		& \leq \hat{w}_0(T_0^*) & \text{ since $\hat{T}_0$ is optimal for $\hat{w}_0$} \\
		& = \hat{w}_0(B^*) & \text{ since $\hat{w}_0$ is zero on $\beta$-unbounded edges} \\
		& \leq \beta w^*_0(B^*) & \text{ since $\hat{w}_0 \leq \beta w^*_0$ on $\beta$-bounded edges}\\
		%			&= w^*_0(B^*) + w^*_0(U) & \text{since $w_0^*$ is zero on $\beta$-unbounded}  \\
		&= \beta w^*_0(T_0^*) & \text{since $T_0^* = B^* \cup U$ and $w^*_0(U) = 0$}  \\
		&\leq \beta w^*_0(T_\mathcal{P}^*) & \text{since $T_0^*$ is optimal for $w_0^*$}  \\
		& \leq \beta w^*(T_\mathcal{P}^*) & \text{ since $w^*_0 \leq w^*$ for all edges.}
	\end{align*}	
	Next we bound $\hat{w}(U)$. From Lemma~\ref{lem:unboundededges}, we know that $\hat{w}_{ij} \leq {\beta}/({\beta-1}) \min \{w_\mathcal{X}(T_i), w_\mathcal{X}(T_j)\}$ for every $(v_i, v_j) \in U$. Because $\hat{T}_\mathcal{P}$ is a tree on $G_\mathcal{P}$, we know by Observation~\ref{lem:treeorient} that we can orient its edges in such a way that each node in $V_\mathcal{P} = \{v_1, v_2, \hdots, v_t\}$ has at most one outgoing edge. 
	We can therefore assign each $(v_i, v_j) \in U$ to one of its nodes in such a way that each node in $V_\mathcal{P}$ is assigned at most one edge from $U$. Assume without loss of generality that we write edges in such a way that edge $(v_i, v_j) \in U$ is assigned to node $v_i$. Thus,
	\begin{equation}
		\label{eq:small}
		\hat{w}(U) = \sum_{(v_i, v_j) \in U} \hat{w}_{ij} \leq \sum_{(v_i, v_j) \in U} \frac{\beta}{\beta-1} w_\mathcal{X}(T_i) \leq \frac{\beta}{\beta-1} \sum_{i = 1}^t w_\mathcal{X}(T_i).
	\end{equation}
	%
	Combining these gives our final bound
	\begin{align*}
		w_\mathcal{X}(\hat{T}) 
		&= \hat{w}(\hat{T}_\mathcal{P}) + \sum_{i = 1}^t w_\mathcal{X}(T_i)  & \text{by Eq.~\eqref{eq:That}}\\
		&\leq \hat{w}(\hat{B}) + \hat{w}(U) +  \sum_{i = 1}^t w_\mathcal{X}(T_i)  & \text{by Eq.~\eqref{eq:hats}}\\
		& \leq \beta w^*(T^*_\mathcal{P}) + \left(\frac{\beta}{\beta-1} + 1\right) \sum_{i = 1}^t w_\mathcal{X}(T_i) & \text{by Eqs.~\eqref{eq:Lhatbound} and~\eqref{eq:small}}\\
		&\leq \max \left\{ \beta, \frac{\beta}{\beta - 1} + 1 \right\} \left(w^*(T_\mathcal{P}^*) +  \sum_{i = 1}^k w_\mathcal{X}(T_i) \right)  \\
		%			&\leq \max \left\{ \beta, 1 + \frac{\beta}{\beta - 1} \right\} \left(w^*(T_\mathcal{P}^*) +  \sum_{i = 1}^k w_\mathcal{X}(T_i) \right)  \\
		&= \beta w_\mathcal{X}(T^*) &\text{ by Eq.~\eqref{eq:Tstar} and our choice of $\beta$.}
	\end{align*}
	For the last step that we have specifically chosen $\beta = (3+\sqrt{5})/2$ to ensure that $\beta = 1 + \beta/(\beta-1)$, as this leads to the best approximation guarantee using the above inequalities. 
\end{proof}

Using a similar proof technique as Theorem~\ref{thm:main} we obtain the following result, showing that Algorithm~\ref{alg:mfcapprox} is a learning-augmented algorithm for metric MST whose performance depends on the $\gamma$-overlap of the initial forest. 
\begin{theorem}
	\label{thm:learning}
	Let $G_\mathcal{X}$ be an implicit metric graph and $\mathcal{P}$ be an initial partitioning with $\gamma$-overlap $\gamma = \gamma(\mathcal{P})$. Algorithm~\ref{alg:mfcapprox} returns a spanning tree of $\hat{T}$ of $G_\mathcal{X}$ that satisfies
	\begin{equation}
		w_\mathcal{X}(T_\mathcal{X}) \leq w_\mathcal{X}(\hat{T}) \leq \beta w_\mathcal{X}(T_\mathcal{X})
	\end{equation}
	where $T_\mathcal{X}$ is an MST of $G_\mathcal{X}$ and $\beta = \frac{1}{2}\left(2\gamma + 1 + \sqrt{4\gamma + 1} \right) \leq 2\gamma+ 1$.
\end{theorem}
\begin{proof}
	We use the same terminology and notation as in the proof of Theorem~\ref{thm:main}. The only difference is that we do not necessarily use $\beta = (3+ \sqrt{5})/2$. For an arbitrary $\beta \geq 1$, we can still prove in the same way that
	\begin{align}
		\label{eq:start}
		w_\mathcal{X}(\hat{T}) \leq  \beta w^*(T^*_\mathcal{P}) + \left(\frac{\beta}{\beta-1} + 1\right) \sum_{i = 1}^t w_\mathcal{X}(T_i).
	\end{align}
	The $\gamma$-overlap of the initial forest implies there exists an MST $T_\mathcal{X}$ of $G_\mathcal{X}$ satisfying:
	\begin{equation}
		\label{eq:ieratio}
		\sum_{i = 1}^t w_\mathcal{X}(T_i) = \gamma w_\mathcal{X}(I_\mathcal{X}),
	\end{equation}
	where $I_\mathcal{X}$ is the set of edges of $T_\mathcal{X}$ inside components $\mathcal{P}$ of the initial forest. Let $B_\mathcal{X}$ be the set of edges in $T_\mathcal{X}$ that cross between components, so that $w_\mathcal{X}(T_\mathcal{X}) = w_\mathcal{X}(B_\mathcal{X}) + w_\mathcal{X}(I_\mathcal{X})$. Since $T_\mathcal{X}$ is a spanning tree, $B_\mathcal{X}$ must contain a path between every pair of components, meaning that $B_\mathcal{X}$ corresponds to a spanning subgraph of the coarsened graph $G_\mathcal{P}$. Since $T_\mathcal{P}^*$ defines an MST of $G_\mathcal{P}$ with respect to $w^*$, which captures the minimum distances between pairs of components, we know
	\begin{equation}
		w^*(T_\mathcal{P}^*) \leq w_\mathcal{X}(B_\mathcal{X}).
	\end{equation}
	Putting the pieces together we see that
	\begin{equation}
		w_\mathcal{X}(\hat{T}) \leq  \beta w_\mathcal{X}(B_\mathcal{X}) + \left( 1 + \frac{\beta}{\beta-1}\right) \gamma w_\mathcal{X}(I_\mathcal{X}) \leq \max \left\{\beta, \gamma\left(1 + \frac{\beta}{\beta -1}\right)  \right\} w_\mathcal{X}(T_\mathcal{X}).
	\end{equation}
	This will hold for any choice of $\beta \geq 1$. In order to prove the smallest approximation guarantee, we choose $\beta$ satisfying:
	\begin{equation*}
		\beta = \gamma\left(1 + \frac{\beta}{\beta -1}\right).
	\end{equation*}
	The solution for this equation under constraint $\beta \geq 1$ and $\gamma \geq 1$ is 
	\begin{equation*}
		\beta = \frac{1}{2}\left(2\gamma + 1 + \sqrt{4\gamma + 1} \right) \leq 2\gamma+ 1.
	\end{equation*}
\end{proof}


\subsection{Runtime analysis and practical considerations}
Algorithm~\ref{alg:mfcapprox} finds the distance between each point in $\mathcal{X}$ and each of the $t$ component representatives, for a total of $O(nt)$ distance queries. It then finds an MST of a dense graph with ${t \choose 2}$ edges, which has runtime and space requirements of $\tilde{O}(t^2)$. Thus, the algorithm has subquadratic memory and query complexity as long as $t = o(n)$. The runtime is $\tilde{O}(nt\texttt{Q}_\mathcal{X} + t^2)$ where $\texttt{Q}_\mathcal{X}$ is the complexity for one distance query in $\mathcal{X}$, which also is subquadratic as long as $t\texttt{Q}_\mathcal{X} = o(n)$. In settings where $\texttt{Q}_\mathcal{X} = \tilde{O}(1)$, the memory, runtime, and query complexity are all subquadratic as long as $t = o(n)$.


\paragraph{Full runtime using $k$-center initialization.} The practical utility of our full MST pipeline also depends on the time it takes to find an initial forest, which depends on various design choices and trade-offs when using any strategy. For intuition we provide a rough complexity analysis for the $k$-center strategy assuming an idealized case of balanced clusters. The simple $2$-approximation for $k$-center chooses an arbitrary first cluster center, and chooses the $i$th cluster center to be the point with maximum distance from the first $i - 1$ centers~\cite{gonzalez1985clustering}. This requires $O(nt)$ distance queries. We can use the cluster centers as the component representatives for \textsf{MFC-Approx}, which allows us to compute $\hat{w}$ without any additional queries. If clusters are balanced in size, we can compute minimum spanning trees for all clusters using $O(n^2/t)$ queries and memory and a runtime of $\tilde{O}(\texttt{Q}_\mathcal{X}n^2/t)$, simply by querying all inner-cluster edges and running a standard MST algorithm. In this balanced-cluster case, combining the initial forest complexity with the complexity of our MFC algorithm, the entire pipeline for finding a spanning tree takes $\tilde{O}(\texttt{Q}_\mathcal{X}(n^2/t + nt) + t^2)$ time. This is minimized by choosing $t = \sqrt{n}$ clusters, leading to a complexity that grows as $n^{1.5}$. For unbalanced clusters, one must consider different trade-offs for cluster-balancing strategies, which could be beneficial for runtime but may affect initial cluster quality.
We could also improve the runtime at the expense of initial forest quality by not computing an exact MST for each cluster. For example, we could recursively apply our entire MFC framework to find a spanning tree of each cluster. 


\paragraph{Practical improvements.} There are several ways to relax our MFC framework to make our approach faster while still satisfying strong approximation guarantees. For metrics with high query complexity, we can use approximate queries with only minor degradation in approximation guarantees. For example, for high-dimensional Euclidean distance we can apply Johnson-Lindenstrauss transformations to reduce the query complexity while approximately maintaining distances. As another relaxation, we can replace the exact nearest neighbor search subroutine in \textsf{MFC-Approx} with an approximate nearest neighbor search. If for some $\varepsilon > 0$ we find a $(1+\varepsilon)$-approximate nearest neighbor in each component for every component representative $s_i$, this will make our approximation guarantees worse by at most a factor $(1+\varepsilon)$. There are also numerous opportunities for parallelization, such as parallelizing distance queries and MST computation for components. 

We can also incorporate heuristics to improve the spanning tree quality of our algorithm with little effect on runtime. As a specific example, when approximating the distance between components $P_i$ and $P_j$ of the initial forest, we could compute $\tilde{x}_i = \argmin_{x \in P_i} d(x,s_j)$ and $\tilde{x}_j = \argmin_{x \in P_j} d(x,s_i)$ and then use the following weight for the coarsened graph:
\begin{equation*}
	\tilde{w}_{ij} = \min \{d(\tilde{x}_i, s_j),d(\tilde{x}_j, s_i), d(\tilde{x}_j,\tilde{x}_i)\}.
\end{equation*}
This differs from Algorithm~\ref{alg:mfcapprox} only in that it additionally checks the distance between $d(\tilde{x}_j,\tilde{x}_i)$ to see if this provides an even closer pair of points between $P_i$ and $P_j$. Although this does not always improve results, it can never be worse in terms of approximations. Figure~\ref{fig:wijhat} provides an example where this strategy would find the optimal distance $w_{ij}^*$, which is strictly better than $\hat{w}_{ij}$. An interesting future direction is to implement this and also explore other heuristics that could improve the practical performance of our method without affecting our theoretical guarantees.


\putsec{related}{Related Work}

\noindent \textbf{Efficient Radiance Field Rendering.}
%
The introduction of Neural Radiance Fields (NeRF)~\cite{mil:sri20} has
generated significant interest in efficient 3D scene representation and
rendering for radiance fields.
%
Over the past years, there has been a large amount of research aimed at
accelerating NeRFs through algorithmic or software
optimizations~\cite{mul:eva22,fri:yu22,che:fun23,sun:sun22}, and the
development of hardware
accelerators~\cite{lee:cho23,li:li23,son:wen23,mub:kan23,fen:liu24}.
%
The state-of-the-art method, 3D Gaussian splatting~\cite{ker:kop23}, has
further fueled interest in accelerating radiance field
rendering~\cite{rad:ste24,lee:lee24,nie:stu24,lee:rho24,ham:mel24} as it
employs rasterization primitives that can be rendered much faster than NeRFs.
%
However, previous research focused on software graphics rendering on
programmable cores or building dedicated hardware accelerators. In contrast,
\name{} investigates the potential of efficient radiance field rendering while
utilizing fixed-function units in graphics hardware.
%
To our knowledge, this is the first work that assesses the performance
implications of rendering Gaussian-based radiance fields on the hardware
graphics pipeline with software and hardware optimizations.

%%%%%%%%%%%%%%%%%%%%%%%%%%%%%%%%%%%%%%%%%%%%%%%%%%%%%%%%%%%%%%%%%%%%%%%%%%
\myparagraph{Enhancing Graphics Rendering Hardware.}
%
The performance advantage of executing graphics rendering on either
programmable shader cores or fixed-function units varies depending on the
rendering methods and hardware designs.
%
Previous studies have explored the performance implication of graphics hardware
design by developing simulation infrastructures for graphics
workloads~\cite{bar:gon06,gub:aam19,tin:sax23,arn:par13}.
%
Additionally, several studies have aimed to improve the performance of
special-purpose hardware such as ray tracing units in graphics
hardware~\cite{cho:now23,liu:cha21} and proposed hardware accelerators for
graphics applications~\cite{lu:hua17,ram:gri09}.
%
In contrast to these works, which primarily evaluate traditional graphics
workloads, our work focuses on improving the performance of volume rendering
workloads, such as Gaussian splatting, which require blending a huge number of
fragments per pixel.

%%%%%%%%%%%%%%%%%%%%%%%%%%%%%%%%%%%%%%%%%%%%%%%%%%%%%%%%%%%%%%%%%%%%%%%%%%
%
In the context of multi-sample anti-aliasing, prior work proposed reducing the
amount of redundant shading by merging fragments from adjacent triangles in a
mesh at the quad granularity~\cite{fat:bou10}.
%
While both our work and quad-fragment merging (QFM)~\cite{fat:bou10} aim to
reduce operations by merging quads, our proposed technique differs from QFM in
many aspects.
%
Our method aims to blend \emph{overlapping primitives} along the depth
direction and applies to quads from any primitive. In contrast, QFM merges quad
fragments from small (e.g., pixel-sized) triangles that \emph{share} an edge
(i.e., \emph{connected}, \emph{non-overlapping} triangles).
%
As such, QFM is not applicable to the scenes consisting of a number of
unconnected transparent triangles, such as those in 3D Gaussian splatting.
%
In addition, our method computes the \emph{exact} color for each pixel by
offloading blending operations from ROPs to shader units, whereas QFM
\emph{approximates} pixel colors by using the color from one triangle when
multiple triangles are merged into a single quad.


	
\section{Experiments}
\label{sec:experiments}
The experiments are designed to address two key research questions.
First, \textbf{RQ1} evaluates whether the average $L_2$-norm of the counterfactual perturbation vectors ($\overline{||\perturb||}$) decreases as the model overfits the data, thereby providing further empirical validation for our hypothesis.
Second, \textbf{RQ2} evaluates the ability of the proposed counterfactual regularized loss, as defined in (\ref{eq:regularized_loss2}), to mitigate overfitting when compared to existing regularization techniques.

% The experiments are designed to address three key research questions. First, \textbf{RQ1} investigates whether the mean perturbation vector norm decreases as the model overfits the data, aiming to further validate our intuition. Second, \textbf{RQ2} explores whether the mean perturbation vector norm can be effectively leveraged as a regularization term during training, offering insights into its potential role in mitigating overfitting. Finally, \textbf{RQ3} examines whether our counterfactual regularizer enables the model to achieve superior performance compared to existing regularization methods, thus highlighting its practical advantage.

\subsection{Experimental Setup}
\textbf{\textit{Datasets, Models, and Tasks.}}
The experiments are conducted on three datasets: \textit{Water Potability}~\cite{kadiwal2020waterpotability}, \textit{Phomene}~\cite{phomene}, and \textit{CIFAR-10}~\cite{krizhevsky2009learning}. For \textit{Water Potability} and \textit{Phomene}, we randomly select $80\%$ of the samples for the training set, and the remaining $20\%$ for the test set, \textit{CIFAR-10} comes already split. Furthermore, we consider the following models: Logistic Regression, Multi-Layer Perceptron (MLP) with 100 and 30 neurons on each hidden layer, and PreactResNet-18~\cite{he2016cvecvv} as a Convolutional Neural Network (CNN) architecture.
We focus on binary classification tasks and leave the extension to multiclass scenarios for future work. However, for datasets that are inherently multiclass, we transform the problem into a binary classification task by selecting two classes, aligning with our assumption.

\smallskip
\noindent\textbf{\textit{Evaluation Measures.}} To characterize the degree of overfitting, we use the test loss, as it serves as a reliable indicator of the model's generalization capability to unseen data. Additionally, we evaluate the predictive performance of each model using the test accuracy.

\smallskip
\noindent\textbf{\textit{Baselines.}} We compare CF-Reg with the following regularization techniques: L1 (``Lasso''), L2 (``Ridge''), and Dropout.

\smallskip
\noindent\textbf{\textit{Configurations.}}
For each model, we adopt specific configurations as follows.
\begin{itemize}
\item \textit{Logistic Regression:} To induce overfitting in the model, we artificially increase the dimensionality of the data beyond the number of training samples by applying a polynomial feature expansion. This approach ensures that the model has enough capacity to overfit the training data, allowing us to analyze the impact of our counterfactual regularizer. The degree of the polynomial is chosen as the smallest degree that makes the number of features greater than the number of data.
\item \textit{Neural Networks (MLP and CNN):} To take advantage of the closed-form solution for computing the optimal perturbation vector as defined in (\ref{eq:opt-delta}), we use a local linear approximation of the neural network models. Hence, given an instance $\inst_i$, we consider the (optimal) counterfactual not with respect to $\model$ but with respect to:
\begin{equation}
\label{eq:taylor}
    \model^{lin}(\inst) = \model(\inst_i) + \nabla_{\inst}\model(\inst_i)(\inst - \inst_i),
\end{equation}
where $\model^{lin}$ represents the first-order Taylor approximation of $\model$ at $\inst_i$.
Note that this step is unnecessary for Logistic Regression, as it is inherently a linear model.
\end{itemize}

\smallskip
\noindent \textbf{\textit{Implementation Details.}} We run all experiments on a machine equipped with an AMD Ryzen 9 7900 12-Core Processor and an NVIDIA GeForce RTX 4090 GPU. Our implementation is based on the PyTorch Lightning framework. We use stochastic gradient descent as the optimizer with a learning rate of $\eta = 0.001$ and no weight decay. We use a batch size of $128$. The training and test steps are conducted for $6000$ epochs on the \textit{Water Potability} and \textit{Phoneme} datasets, while for the \textit{CIFAR-10} dataset, they are performed for $200$ epochs.
Finally, the contribution $w_i^{\varepsilon}$ of each training point $\inst_i$ is uniformly set as $w_i^{\varepsilon} = 1~\forall i\in \{1,\ldots,m\}$.

The source code implementation for our experiments is available at the following GitHub repository: \url{https://anonymous.4open.science/r/COCE-80B4/README.md} 

\subsection{RQ1: Counterfactual Perturbation vs. Overfitting}
To address \textbf{RQ1}, we analyze the relationship between the test loss and the average $L_2$-norm of the counterfactual perturbation vectors ($\overline{||\perturb||}$) over training epochs.

In particular, Figure~\ref{fig:delta_loss_epochs} depicts the evolution of $\overline{||\perturb||}$ alongside the test loss for an MLP trained \textit{without} regularization on the \textit{Water Potability} dataset. 
\begin{figure}[ht]
    \centering
    \includegraphics[width=0.85\linewidth]{img/delta_loss_epochs.png}
    \caption{The average counterfactual perturbation vector $\overline{||\perturb||}$ (left $y$-axis) and the cross-entropy test loss (right $y$-axis) over training epochs ($x$-axis) for an MLP trained on the \textit{Water Potability} dataset \textit{without} regularization.}
    \label{fig:delta_loss_epochs}
\end{figure}

The plot shows a clear trend as the model starts to overfit the data (evidenced by an increase in test loss). 
Notably, $\overline{||\perturb||}$ begins to decrease, which aligns with the hypothesis that the average distance to the optimal counterfactual example gets smaller as the model's decision boundary becomes increasingly adherent to the training data.

It is worth noting that this trend is heavily influenced by the choice of the counterfactual generator model. In particular, the relationship between $\overline{||\perturb||}$ and the degree of overfitting may become even more pronounced when leveraging more accurate counterfactual generators. However, these models often come at the cost of higher computational complexity, and their exploration is left to future work.

Nonetheless, we expect that $\overline{||\perturb||}$ will eventually stabilize at a plateau, as the average $L_2$-norm of the optimal counterfactual perturbations cannot vanish to zero.

% Additionally, the choice of employing the score-based counterfactual explanation framework to generate counterfactuals was driven to promote computational efficiency.

% Future enhancements to the framework may involve adopting models capable of generating more precise counterfactuals. While such approaches may yield to performance improvements, they are likely to come at the cost of increased computational complexity.


\subsection{RQ2: Counterfactual Regularization Performance}
To answer \textbf{RQ2}, we evaluate the effectiveness of the proposed counterfactual regularization (CF-Reg) by comparing its performance against existing baselines: unregularized training loss (No-Reg), L1 regularization (L1-Reg), L2 regularization (L2-Reg), and Dropout.
Specifically, for each model and dataset combination, Table~\ref{tab:regularization_comparison} presents the mean value and standard deviation of test accuracy achieved by each method across 5 random initialization. 

The table illustrates that our regularization technique consistently delivers better results than existing methods across all evaluated scenarios, except for one case -- i.e., Logistic Regression on the \textit{Phomene} dataset. 
However, this setting exhibits an unusual pattern, as the highest model accuracy is achieved without any regularization. Even in this case, CF-Reg still surpasses other regularization baselines.

From the results above, we derive the following key insights. First, CF-Reg proves to be effective across various model types, ranging from simple linear models (Logistic Regression) to deep architectures like MLPs and CNNs, and across diverse datasets, including both tabular and image data. 
Second, CF-Reg's strong performance on the \textit{Water} dataset with Logistic Regression suggests that its benefits may be more pronounced when applied to simpler models. However, the unexpected outcome on the \textit{Phoneme} dataset calls for further investigation into this phenomenon.


\begin{table*}[h!]
    \centering
    \caption{Mean value and standard deviation of test accuracy across 5 random initializations for different model, dataset, and regularization method. The best results are highlighted in \textbf{bold}.}
    \label{tab:regularization_comparison}
    \begin{tabular}{|c|c|c|c|c|c|c|}
        \hline
        \textbf{Model} & \textbf{Dataset} & \textbf{No-Reg} & \textbf{L1-Reg} & \textbf{L2-Reg} & \textbf{Dropout} & \textbf{CF-Reg (ours)} \\ \hline
        Logistic Regression   & \textit{Water}   & $0.6595 \pm 0.0038$   & $0.6729 \pm 0.0056$   & $0.6756 \pm 0.0046$  & N/A    & $\mathbf{0.6918 \pm 0.0036}$                     \\ \hline
        MLP   & \textit{Water}   & $0.6756 \pm 0.0042$   & $0.6790 \pm 0.0058$   & $0.6790 \pm 0.0023$  & $0.6750 \pm 0.0036$    & $\mathbf{0.6802 \pm 0.0046}$                    \\ \hline
%        MLP   & \textit{Adult}   & $0.8404 \pm 0.0010$   & $\mathbf{0.8495 \pm 0.0007}$   & $0.8489 \pm 0.0014$  & $\mathbf{0.8495 \pm 0.0016}$     & $0.8449 \pm 0.0019$                    \\ \hline
        Logistic Regression   & \textit{Phomene}   & $\mathbf{0.8148 \pm 0.0020}$   & $0.8041 \pm 0.0028$   & $0.7835 \pm 0.0176$  & N/A    & $0.8098 \pm 0.0055$                     \\ \hline
        MLP   & \textit{Phomene}   & $0.8677 \pm 0.0033$   & $0.8374 \pm 0.0080$   & $0.8673 \pm 0.0045$  & $0.8672 \pm 0.0042$     & $\mathbf{0.8718 \pm 0.0040}$                    \\ \hline
        CNN   & \textit{CIFAR-10} & $0.6670 \pm 0.0233$   & $0.6229 \pm 0.0850$   & $0.7348 \pm 0.0365$   & N/A    & $\mathbf{0.7427 \pm 0.0571}$                     \\ \hline
    \end{tabular}
\end{table*}

\begin{table*}[htb!]
    \centering
    \caption{Hyperparameter configurations utilized for the generation of Table \ref{tab:regularization_comparison}. For our regularization the hyperparameters are reported as $\mathbf{\alpha/\beta}$.}
    \label{tab:performance_parameters}
    \begin{tabular}{|c|c|c|c|c|c|c|}
        \hline
        \textbf{Model} & \textbf{Dataset} & \textbf{No-Reg} & \textbf{L1-Reg} & \textbf{L2-Reg} & \textbf{Dropout} & \textbf{CF-Reg (ours)} \\ \hline
        Logistic Regression   & \textit{Water}   & N/A   & $0.0093$   & $0.6927$  & N/A    & $0.3791/1.0355$                     \\ \hline
        MLP   & \textit{Water}   & N/A   & $0.0007$   & $0.0022$  & $0.0002$    & $0.2567/1.9775$                    \\ \hline
        Logistic Regression   &
        \textit{Phomene}   & N/A   & $0.0097$   & $0.7979$  & N/A    & $0.0571/1.8516$                     \\ \hline
        MLP   & \textit{Phomene}   & N/A   & $0.0007$   & $4.24\cdot10^{-5}$  & $0.0015$    & $0.0516/2.2700$                    \\ \hline
       % MLP   & \textit{Adult}   & N/A   & $0.0018$   & $0.0018$  & $0.0601$     & $0.0764/2.2068$                    \\ \hline
        CNN   & \textit{CIFAR-10} & N/A   & $0.0050$   & $0.0864$ & N/A    & $0.3018/
        2.1502$                     \\ \hline
    \end{tabular}
\end{table*}

\begin{table*}[htb!]
    \centering
    \caption{Mean value and standard deviation of training time across 5 different runs. The reported time (in seconds) corresponds to the generation of each entry in Table \ref{tab:regularization_comparison}. Times are }
    \label{tab:times}
    \begin{tabular}{|c|c|c|c|c|c|c|}
        \hline
        \textbf{Model} & \textbf{Dataset} & \textbf{No-Reg} & \textbf{L1-Reg} & \textbf{L2-Reg} & \textbf{Dropout} & \textbf{CF-Reg (ours)} \\ \hline
        Logistic Regression   & \textit{Water}   & $222.98 \pm 1.07$   & $239.94 \pm 2.59$   & $241.60 \pm 1.88$  & N/A    & $251.50 \pm 1.93$                     \\ \hline
        MLP   & \textit{Water}   & $225.71 \pm 3.85$   & $250.13 \pm 4.44$   & $255.78 \pm 2.38$  & $237.83 \pm 3.45$    & $266.48 \pm 3.46$                    \\ \hline
        Logistic Regression   & \textit{Phomene}   & $266.39 \pm 0.82$ & $367.52 \pm 6.85$   & $361.69 \pm 4.04$  & N/A   & $310.48 \pm 0.76$                    \\ \hline
        MLP   &
        \textit{Phomene} & $335.62 \pm 1.77$   & $390.86 \pm 2.11$   & $393.96 \pm 1.95$ & $363.51 \pm 5.07$    & $403.14 \pm 1.92$                     \\ \hline
       % MLP   & \textit{Adult}   & N/A   & $0.0018$   & $0.0018$  & $0.0601$     & $0.0764/2.2068$                    \\ \hline
        CNN   & \textit{CIFAR-10} & $370.09 \pm 0.18$   & $395.71 \pm 0.55$   & $401.38 \pm 0.16$ & N/A    & $1287.8 \pm 0.26$                     \\ \hline
    \end{tabular}
\end{table*}

\subsection{Feasibility of our Method}
A crucial requirement for any regularization technique is that it should impose minimal impact on the overall training process.
In this respect, CF-Reg introduces an overhead that depends on the time required to find the optimal counterfactual example for each training instance. 
As such, the more sophisticated the counterfactual generator model probed during training the higher would be the time required. However, a more advanced counterfactual generator might provide a more effective regularization. We discuss this trade-off in more details in Section~\ref{sec:discussion}.

Table~\ref{tab:times} presents the average training time ($\pm$ standard deviation) for each model and dataset combination listed in Table~\ref{tab:regularization_comparison}.
We can observe that the higher accuracy achieved by CF-Reg using the score-based counterfactual generator comes with only minimal overhead. However, when applied to deep neural networks with many hidden layers, such as \textit{PreactResNet-18}, the forward derivative computation required for the linearization of the network introduces a more noticeable computational cost, explaining the longer training times in the table.

\subsection{Hyperparameter Sensitivity Analysis}
The proposed counterfactual regularization technique relies on two key hyperparameters: $\alpha$ and $\beta$. The former is intrinsic to the loss formulation defined in (\ref{eq:cf-train}), while the latter is closely tied to the choice of the score-based counterfactual explanation method used.

Figure~\ref{fig:test_alpha_beta} illustrates how the test accuracy of an MLP trained on the \textit{Water Potability} dataset changes for different combinations of $\alpha$ and $\beta$.

\begin{figure}[ht]
    \centering
    \includegraphics[width=0.85\linewidth]{img/test_acc_alpha_beta.png}
    \caption{The test accuracy of an MLP trained on the \textit{Water Potability} dataset, evaluated while varying the weight of our counterfactual regularizer ($\alpha$) for different values of $\beta$.}
    \label{fig:test_alpha_beta}
\end{figure}

We observe that, for a fixed $\beta$, increasing the weight of our counterfactual regularizer ($\alpha$) can slightly improve test accuracy until a sudden drop is noticed for $\alpha > 0.1$.
This behavior was expected, as the impact of our penalty, like any regularization term, can be disruptive if not properly controlled.

Moreover, this finding further demonstrates that our regularization method, CF-Reg, is inherently data-driven. Therefore, it requires specific fine-tuning based on the combination of the model and dataset at hand.


\section{Conclusions and Discussion}
We have presented a new framework for finding spanning trees in arbitrary metric spaces, which is highly scalable and grounded in rigorous approximation guarantees. 
This framework is based on completing an initial forest, which can be obtained efficiently using  practical heuristics. This paper focuses on serial implementations and theoretical guarantees, as our framework already provides many advantages in this setting. At the same time, our work is strongly motivated by massive-scale clustering applications that require high-performance computing capabilities, and the algorithm we developed is highly parallelizable. A natural direction for further research is to develop parallel versions of our algorithm that can be run on a much larger scale. There are also many remaining questions in the serial setting. One direction is to try to improve on the $(\sqrt{5} + 3)/2$-approximation guarantee while still using subquadratic time, or obtain an approximation with better dependence on the $\gamma$-overlap parameter. Another direction is to prove lower bounds for the best possible approximation guarantees for subquadratic algorithms. There are also many opportunities to explore more efficient and practical methods for obtaining the initial forest that serves as the input to MFC.


	
%	\section*{Acknowledgments}
	
	\bibliographystyle{plain}
	\bibliography{mst-bib}

    \appendix

\end{document}
