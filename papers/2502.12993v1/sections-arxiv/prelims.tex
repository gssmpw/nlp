\section{Technical Preliminaries}
\label{sec:prelims}
We cover several technical preliminaries to set the stage for our new MFC problem and algorithms. 

\subsection{Graph notation and minimum spanning trees.}
For an undirected graph $G = (V,E)$ and weight function $w \colon E \rightarrow \mathbb{R}^+$, we denote the weight of an edge $e = (u,v)$ as $w(u,v)$, $w_{uv}$, or $w_e$. 
% Every weight function $w$ can be extended to apply to entire sets of edges, so that 
The weight of an edge set $F \subseteq E$ is denoted by 
 \begin{equation}
 	\label{eq:edgesum}
 	w(F) = \sum_{e \in F} w_e.
 \end{equation}
% The weight of all edges in $G$ is denoted by 
We often use $w(G) = w(E)$
to denote the weight of all edges in $G = (V,E)$. 

\paragraph{Minimum spanning trees.} 
The minimum spanning tree problem on $G$ with respect to weight function $w$ seeks a spanning tree $T = (V,E_T)$ of $G$ that minimizes $w(E_T)$. When $w$ is clear from context we will refer to $T$ simply as an MST of $G$. We do often consider multiple spanning trees of the same graph, each optimal for a different weight function. In these cases we  explicitly state the weight function associated with an MST.

There are many well-known greedy algorithms for optimally constructing an MST~\cite{kruskal1956shortest, prim1957shortest,boruvka1926jistem}. 
We review Kruskal's~\cite{kruskal1956shortest} and Boruvka's~\cite{boruvka1926jistem} algorithms as their mechanics are relevant for understanding our approach and results. These methods first place all nodes into singleton components. Each iteration of Kruskal's algorithm identifies a minimum weight edge that connects two nodes in different components, and adds it to a forest of edges (initialized to the empty set at the outset of the algorithm) that is guaranteed to be part of some MST. This proceeds until all components of the forest have been merged into one tree. The method is equivalent to ordering all edges based on weight and visiting them in order, greedily adding the $i$th edge to the spanning tree if and only if its endpoints are in different components. Boruvka's algorithm~\cite{boruvka1926jistem} is similar, but instead identifies a minimum weight edge incident to  \emph{each} component, adding all of them to the growing forest. Both algorithms can be implemented to run in $O(m \log n)$ time where $n = |V|$ and $m = |E|$. Faster algorithms, often based on one of these algorithms, have been developed. The fastest deterministic algorithm has a runtime of $O(m \cdot \alpha(m,n))$ where $\alpha$ is the inverse of the Ackerman function~\cite{chazelle2000minimum}. For this paper it suffices to know that finding an MST takes $\tilde{O}(m)$ time where $\tilde{O}$ hides factors that are logarithmic (or smaller) in $n$. 


\subsection{The metric MST problem}
Throughout the paper we let $(\mathcal{X},d)$ be a finite metric space where $\mathcal{X} = \{x_1, x_2, \hdots , x_n\}$ is a set of data points and $d(x_i, x_j)$ is the distance between the $i$th and $j$th points.  
In order to be a metric space, $d$ must satisfy the triangle inequality: $d(x_{i}, x_{j}) \leq d(x_{i}, x_{k}) + d(x_{k}, x_{j})$ for all triplets $i,j,k \in [n] = \{1,2, \hdots n\}$. Aside from this we make no formal assumptions about the points in $\mathcal{X}$ or the distance function $d$.
The space is associated with a complete graph $G_\mathcal{X} = (\mathcal{X}, E_\mathcal{X})$ where we treat $\mathcal{X}$ as a node set and the edge set $E_\mathcal{X} = {\mathcal{X} \choose 2}$ includes all pairs of nodes. 
The weight function $w_\mathcal{X}$ for $G_\mathcal{X}$ is defined by $w_\mathcal{X}(i,j) = d(x_i, x_j)$. This graph $G_\mathcal{X}$ is implicit; in order to know the weight of an edge we must query the distance function $d$. We make no assumptions regarding the complexity of querying distances. Rather, we will give the complexity of our algorithms both in terms of the runtime and number of queries. 

The \emph{metric} minimum spanning tree problem is simply the MST problem applied to $G_\mathcal{X}$ (see Figure~\ref{fig:metric_mst}). This can be solved by querying the distance function $O(n^2)$ times to form $G_\mathcal{X}$, and then applying an existing MST algorithm. 
We show it is more practical to implicitly deal with $G_\mathcal{X}$ via edge queries, while solving different types of nearest neighbor problems over subsets of $\mathcal{X}$.
\begin{figure}[t]
	\centering
	\begin{subfigure}[b]{0.32\textwidth}
		\centering
		\includegraphics[width=\textwidth]{figures/initial_points.pdf}
		\caption{Initial space $\mathcal{X}$}
		\label{fig:opt_mstc}
	\end{subfigure}
	\begin{subfigure}[b]{0.33\textwidth}
		\centering
		\includegraphics[width=\textwidth]{figures/partial_MST_1.pdf}
		\caption{Forest from Kruskal's}
		\label{fig:partial_MST_1}
	\end{subfigure}
	\begin{subfigure}[b]{0.33\textwidth}
		\centering
		\includegraphics[width=\textwidth]{figures/true_MST_1.pdf}
		\caption{Full metric MST}
		\label{fig:true_MST_1}
	\end{subfigure}
	\caption{(a)~Consider a simple example of a finite metric space $(\mathcal{X},d)$: 75 points in $\mathbb{R}^2$ equipped with Euclidean distance. $G_\mathcal{X}$ is an implicit complete graph obtained by computing distances between all pairs of points. (b)~Kruskal's algorithm iteratively merges components in a growing forest. Here we display the forest at an intermediate step. Knowing the minimum distance between two different components requires solving a bichromatic closest pair problem. (c)~Continuing until all components are merged produces a metric minimum spanning tree.}
	\label{fig:metric_mst}
\end{figure}

\paragraph{Metric MSTs and bichromatic closest pairs.} 
The \emph{bichromatic closest pair problem} (BCP) is a particularly relevant computational primitive for metric MSTs. The input to BCP is two sets of points $A$ and $B$ in a metric space. The goal is to find a pair of opposite-set points (one from $A$ and one from $B$) with smallest distance. One can implicitly apply a classical MST algorithm such as Kruskal's or Boruvka's algorithm to $G_\mathcal{X}$ by repeatedly solving BCP problems. In each step, these algorithms must identify one or more disconnected components in an intermediate forest (see Figure~\ref{fig:partial_MST_1}) to join via minimum weight edges. Finding a minimum weight edge between two components exactly corresponds to a BCP problem. This connection between metric MSTs and BCP is well known and has been leveraged in many prior works on metric MSTs~\cite{agarwal1990euclidean,callahan1993faster,narasimhan2001geometric,chatterjee2010geometric}. 





