\begin{figure*}[!tbp]% [H] is so declass\'e!
\centering
    \begin{subfigure}{0.33\textwidth}
    \centering    \includegraphics[width=0.85\textwidth]{Media/depth_scaling_eval_perplexity.png}
    % \caption{}
    \label{fig:depth_scaling_eval_perplexity}
    \end{subfigure}\hfill
    \centering
    \begin{subfigure}{0.33\textwidth}
    \centering    \includegraphics[width=0.85\textwidth]{Media/depth_scaling_closedQA.png}
    % \caption{}
    \label{fig:depth_scaling_closedQA}
    \end{subfigure}\hfill
    \centering
    \begin{subfigure}{0.33\textwidth}
\centering    
    \includegraphics[width=0.85\textwidth]{Media/depth_scaling_openQA.png}
    % \caption{}
    \label{fig:depth_scaling_openQA}
    \end{subfigure}\hfill
\centering
    \begin{subfigure}{0.33\textwidth}
    \centering    \includegraphics[width=0.85\textwidth]{Media/depth_scaling_eval_token_acc.png}
    % \caption{}
    \label{fig:depth_scaling_eval_token_acc}
    \end{subfigure}\hfill
\centering
    \begin{subfigure}{0.33\textwidth}
    \centering    \includegraphics[width=0.9\textwidth]{Media/depth_scaling_mwp.png}
    % \caption{}
    \label{fig:depth_scaling_mwp}
    \end{subfigure}\hfill
\centering
    \begin{subfigure}{0.33\textwidth}
    \centering    \includegraphics[width=0.9\textwidth]{Media/depth_scaling_primitives.png}
    % \caption{}
    \label{fig:depth_scaling_primitives}
    \end{subfigure}
    \vspace{-0.1in}
    \caption{\looseness-1\new{Scaling behavior for various task group as the effective depth increases. The blue curve shows how performance scales as the number of loops increases, without increasing parameters, using models of the form \loopy{4}{D/4} for various values of $D$. The orange curve visualizes the scaling behavior of \loopy{D}{1} which increases the depth by adding fresh parameters. For reasoning primitives, the looped model scales as well, or even better, than the baseline despite having $D/4$ fewer parameters.}}
    \label{fig:depth_scaling}
\end{figure*}
