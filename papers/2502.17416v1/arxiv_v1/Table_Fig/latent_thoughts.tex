\begin{figure*}[!tbp]% [H] is so declass\'e!
\centering
    \begin{subfigure}{0.5\textwidth}
    \centering    \includegraphics[width=0.55\textwidth]{Media/cot_iteration.png}
    % \caption{}
    \label{fig:cot_iteration}
    \end{subfigure}\hfill
    \centering
    \begin{subfigure}{0.5\textwidth}
    \centering    \includegraphics[width=0.55\textwidth]{Media/latent_thoughts_iteration.png}
    % \caption{}
    \label{fig:latent_thoughts_iteration}
    \end{subfigure}
    \vspace{-0.1in}
    \caption{\looseness-1\new{{\bf Left.} Chain-of-thought reasoning can be viewed as a looped model, where each iteration produces one new thoughts token. The new tokens are highlighted in red. {\bf Right.} A looped model can instead generate multiple {\em latent thoughts} in parallel and, in theory, can simulate CoT reasoning my masking the updates appropriately (see \Cref{thm:cot_informal})}}
    \label{fig:latent_thoughts}
\end{figure*}
