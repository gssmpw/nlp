% WACV 2025 Paper Template
% based on the WACV 2024 template, which is
% based on the CVPR 2023 template (https://media.icml.cc/Conferences/CVPR2023/cvpr2023-author_kit-v1_1-1.zip) with 2-track changes from the WACV 2023 template (https://github.com/wacv-pcs/WACV-2023-Author-Kit)
% based on the CVPR template provided by Ming-Ming Cheng (https://github.com/MCG-NKU/CVPR_Template)
% modified and extended by Stefan Roth (stefan.roth@NOSPAMtu-darmstadt.de)

\documentclass[10pt,twocolumn,letterpaper]{article}

%%%%%%%%% PAPER TYPE  - PLEASE UPDATE FOR FINAL VERSION
% \usepackage[review,algorithms]{wacv}      % To produce the REVIEW version for the algorithms track
%\usepackage[review,applications]{wacv}      % To produce the REVIEW version for the applications track
\usepackage{wacv}              % To produce the CAMERA-READY version
%\usepackage[pagenumbers]{wacv} % To force page numbers, e.g. for an arXiv version

% Include other packages here, before hyperref.
\usepackage{graphicx}
\usepackage{amsmath}
\usepackage{amssymb}
\usepackage{booktabs}
\usepackage{adjustbox}
\usepackage{float}


\usepackage{algpseudocode}
\usepackage[linesnumbered,ruled,vlined]{algorithm2e}

% It is strongly recommended to use hyperref, especially for the review version.
% hyperref with option pagebackref eases the reviewers' job.
% Please disable hyperref *only* if you encounter grave issues, e.g. with the
% file validation for the camera-ready version.
%
% If you comment hyperref and then uncomment it, you should delete
% ReviewTempalte.aux before re-running LaTeX.
% (Or just hit 'q' on the first LaTeX run, let it finish, and you
%  should be clear).
\usepackage[pagebackref,breaklinks,colorlinks]{hyperref}


% Support for easy cross-referencing
\usepackage[capitalize]{cleveref}
\crefname{section}{Sec.}{Secs.}
\Crefname{section}{Section}{Sections}
\Crefname{table}{Table}{Tables}
\crefname{table}{Tab.}{Tabs.}


%%%%%%%%% PAPER ID  - PLEASE UPDATE
\def\wacvPaperID{1734} % *** Enter the WACV Paper ID here
\def\confName{WACV}
\def\confYear{2025}


\begin{document}

%%%%%%%%% TITLE - PLEASE UPDATE
\title{Feature Space Perturbation: A Panacea to Enhanced Transferability Estimation}

% \author{Prafful Kumar Khoba\\
% UQ–IITD Research Academy\\
% Delhi, India\\
% {\tt\small qiz228274@iitd.ac.in}
% % For a paper whose authors are all at the same institution,
% % omit the following lines up until the closing ``}''.
% % Additional authors and addresses can be added with ``\and'',
% % just like the second author.
% % To save space, use either the email address or home page, not both
% \and
% Zijian Wang\\
% The University of Queensland\\
% Brisbane, Australia\\
% {\tt\small zijian.wang@uq.edu.au}
% \and
% Chetan Arora\\
% IIT Delhi\\
% Delhi, India\\
% {\tt\small chetan@cse.iitd.ac.in}
% \and
% Mahsa Baktashmotlagh\\
% The University of Queensland\\
% Brisbane, Australia\\
% {\tt\small m.baktashmotlagh@uq.edu.au}}

\author{
Prafful Kumar Khoba$^{1}$ \quad Zijian Wang$^{2}$ \quad Chetan Arora$^{3}$ \quad Mahsa Baktashmotlagh$^{2}$ \\
$^{1}$UQ–IITD Research Academy, New Delhi, India \\
$^{2}$The University of Queensland, Brisbane, Australia \\
$^{3}$Indian Institute of Technology Delhi, New Delhi, India \\
{\tt\small qiz228274@iitd.ac.in, zijian.wang@uq.edu.au, chetan@cse.iitd.ac.in, m.baktashmotlagh@uq.edu.au}
}

\maketitle

\begin{figure*}[!h]
  \centering
  \includegraphics[width=\textwidth]{images/Teaser.pdf}
  \caption{Illustration of our feature perturbation method for transferability estimation. (a) provides a flowchart outlining the process of enhancing transferability estimation, with bold elements representing our perturbation steps. The remaining part of the flowchart illustrates the process traditionally employed in existing transferability estimation work. (b) shows initial embeddings with significant inter-class separation and compact intra-class clustering, typical of supervised models. (c) displays embeddings after our feature perturbation.
  (d) and (e) present actual correlation charts and weighted Kendall correlation coefficient $\tau_w$ on the pets dataset. Correlation chart depicts the predicted rankings versus actual rankings before and after perturbations, where each symbol in these charts represents a model. The shift from lower to higher correlation values highlights the improved accuracy of model rankings after applying our perturbation method.}
  \label{fig:teaser}
  \vspace{-1em}
\end{figure*}


%%%%%%%%% ABSTRACT
\begin{abstract}
    Leveraging a transferability estimation metric facilitates the non-trivial challenge of selecting the optimal model for the downstream task from a pool of pre-trained models. Most existing metrics primarily focus on identifying the statistical relationship between feature embeddings and the corresponding labels within the target dataset, but overlook crucial aspect of model robustness. This oversight may limit their effectiveness in accurately ranking pre-trained models. To address this limitation, we introduce a feature perturbation method that enhances the transferability estimation process by systematically altering the feature space. Our method includes a Spread operation that increases intra-class variability, adding complexity within classes, and an Attract operation that minimizes the distances between different classes, thereby blurring the class boundaries.
   Through extensive experimentation, we demonstrate the efficacy of our feature perturbation method in providing a more precise and robust estimation of model transferability. Notably, the existing LogMe method exhibited a significant improvement, showing a 28.84$\%$ increase in performance after applying our feature perturbation method. The implementation is available at \url{https://github.com/prafful-kumar/enhancing_TE.git}
\end{abstract}

\section{Introduction}


Transfer learning enables the application of knowledge acquired from one task to enhance performance on another with only minimal additional training, typically through fine-tuning \cite{zhuang2020comprehensive}. At the core of this approach is the challenge of transferability estimation, which aims to predict how well the pre-trained models adapt when applied to new, target datasets without fine-tuning. However, the main challenge lies in determining the most appropriate model for a specific task without individually fine-tuning each candidate model, which can be a resource-intensive and time-consuming task, especially for large target datasets.

To address this issue, researchers have proposed various metrics \cite{nguyen2020leep, tran2019transferability, li2021ranking,you2021logme,shao2022not,DBLP:conf/iccv/WangLZHB23,Pandy2022TransferabilityEstimation, zhang2024model} based on the distinctive properties of the pre-trained model, known as the transferability estimation metric. The core idea behind many of these metrics is to establish a statistical relationship between the feature embedding of pre-trained models and the labels of the samples, thereby generating a transferability score for each model. Based on the score, these metrics aim to predict the actual rank of these models on the target dataset, with the goal of achieving a high correlation between the predicted rank and the actual rank, as shown in Fig. \ref{fig:teaser}(a). However, while these metrics effectively measure adaptability, they often overlook how models handle disruptions in the structure of the embeddings, which is crucial for assessing their robustness. To fill this gap, we propose a novel approach that perturbs both the intra-class (through the spread operation) and inter-class (through the attract operation) structures of the embeddings.
By introducing this targeted perturbation, we aim to provide a more comprehensive and realistic assessment of a model's resilience and adaptability. This method not only improves the accuracy of transferability estimations but also ensures that the selected models are genuinely capable of performing reliably in diverse and dynamic environments.


To illustrate how robustness to perturbations impacts transferability estimation, consider a toy example depicted in Fig. \ref{fig:teaser}.
The figure visually illustrates how embeddings respond to perturbations and its impact on model transferability. Initially, the embeddings display clear inter-class separation and compact intra-class configurations Fig. \ref{fig:teaser}(a), typical of supervised models' structured embeddings. After our feature perturbation method is applied Fig. \ref{fig:teaser}(b), these embeddings show expanded intra-class distributions and reduced inter-class separations, highlighting how perturbations disrupt the standard embedding structure. As traditional metrics rely adaptability of the features, they tend to produce lower transferability scores post-perturbation for all the models. Yet, the degree of score reduction varies significantly among the models. This is evident on the x-axis of the correlation chart for the pets dataset in Fig. \ref{fig:teaser}(d,e), as referenced from the results section \ref{sect:result}. For instance, if one model’s score drops drastically compared to others, it indicates a lower robustness to perturbations. This substantial decrease suggests that the model’s embeddings are overly sensitive to changes, potentially undermining its performance when applied to new datasets.
This leads to a more accurate relative ranking of these models. %These improvements are detailed in the experimental section \ref{sec:exp}, highlighting how perturbations refine the correlation between predicted and actual model rankings.


While most of the prior works focus on estimating the transferability of models under the vanilla fine-tuning schema, we argue that it may not be sufficient to address a broader use case. Recent literature~\cite{kirichenko2022dfr, kumar2022fine} highlights that while vanilla fine-tuning can achieve higher in-distribution (ID) test accuracy, it sacrifices the out-of-distribution (OOD) robustness compared with alternative fine-tuning strategy (\textit{i.e.}, linear probing). The trade-off between ID and OOD test accuracy urges transferability estimation metrics to consider a wider range of fine-tuning schemes. Therefore, in this work, we assess the transferability estimation metric for three distinct fine-tuning strategies: \textit{vanilla fine-tuning}, which updates all model parameters; \textit{last block fine-tuning (LBFT)}, which updates only the parameters of the last block and final linear layer; and \textit{linear fine-tuning (LFT)}, which focuses on updating the last fully connected layer. In summary, our contribution is threefold:

% Note that we avoid applying our perturbation technique to self-supervised models. This is because self-supervised models do not rely on labeled data to learn representations \cite{sela, simclr-v1, simclr-v2, byol, infomin}. Whereas our feature perturbation method, which relies on class label information, may disrupt the meaningful structures they learn by erasing valuable geometric configurations of the embeddings. Unlike supervised models, which follow a discriminative pattern, self-supervised models do not, as shown in Fig. \ref{fig:ss_vs_s}. This complexity makes it difficult to create an effective perturbation strategy for self-supervised models. As a result, our research is dedicated to improving transferability estimation for supervised models only. However, we have developed LDA-based metrics for self-supervised models that surpass many of the previous transferability estimation metrics. In summary, our contribution is threefold:

\begin{itemize}
    \item  
    Our feature perturbation method effectively perturbs the feature embedding spaces of pre-trained supervised models, significantly improving the rank estimation of these pre-trained models by transferability estimation metric. This approach led to a notable 28.84$\%$ performance boost in the LogMe \cite{you2021logme} method.
    \item 
    Our findings highlight that existing metrics are primarily effective for vanilla fine-tuning, pointing out the necessity for more adaptable solutions for diverse fine-tuning techniques. Our feature perturbation method significantly enhances transferability estimation across key fine-tuning strategies, specifically vanilla and LBFT. Its effectiveness across these strategies and compatibility with all current transferability estimation metrics underscores its broad applicability and versatility. 
    \item 
   Unlike supervised models, self-supervised models do not rely on labeled data and are particularly sensitive to disruptions in their geometric embedding structures. Therefore, we avoid applying our class-based perturbation strategy to these models. Instead, we have developed a Linear Discriminant Analysis \cite{balakrishnama1998linear} (LDA)-based metric, specifically tailored for self-supervised models, that significantly outperforms traditional baselines across all fine-tuning variants.

\end{itemize}
% Our analysis demonstrates the superior performance of simpler architectures, like Linear Discriminant Analysis (LDA) \cite{balakrishnama1998linear}, over state-of-the-art methods for self-supervised models across all fine-tuning variants. 


\section{Related Work}

% \subsection{Transferability estimation techniques}

Prior work can be categories into label-based method and source-embedding based method. Log Expected Empirical Predictor (LEEP) \cite{nguyen2020leep} and Negative Conditional Entropy (NCE) \cite{tran2019transferability} are two transferability estimation metrics that rely on information from both the source and target label sets. % One drawback of label-comparison based methods is their sensitivity to changes in the model's head (the last layer), which can affect the output scores of these metrics. Instead of using the head of the pre-trained model, source-embedding based approaches tend to use the feature extractor for creating transferability estimation metric.
Passing target data through a pre-trained model's feature extractor yields target embeddings for source-embedding methods. These methods include $\mathcal{N}$LEEP \cite{li2021ranking}, LogMe \cite{you2021logme}, SFDA \cite{shao2022not}, NCTI \cite{DBLP:conf/iccv/WangLZHB23}, and GBC \cite{Pandy2022TransferabilityEstimation}.


$\mathcal{N}$LEEP \cite{li2021ranking} is build to remove the drawbacks of LEEP, i.e., it does not uses source head. It utilizes Principal Component Analysis (PCA) \cite{wold1987principal} to reduce the dimensionality of the data, followed by fitting a Gaussian Mixture Model (GMM) \cite{reynolds2009gaussian} to the target data embedding. LogMe \cite{you2021logme} assesses model transferability by modeling each target label as a linear model with Gaussian noise using maximum evidence to evaluate how well pre-trained model features fit target labels. GBC \cite{Pandy2022TransferabilityEstimation} metric evaluates how much classes in the target data's embedding space overlap; a higher GBC value indicates greater overlap. SFDA, presented in \cite{shao2022not}, is a transferability metric which employs Fisher Discriminant Analysis (FDA) \cite{mika1999fisher} and introduces ConfMix. FDA works by finding a linear data transformation that maximizes the separation between classes using within-class scatter and between-class scatter followed by applying ConfMix, a mechanism to generate challenging negative samples. NCTI \cite{DBLP:conf/iccv/WangLZHB23} metric quantifies the gap between a model's embedding and the ideal neural collapse embedding. 

% A notable attempt by Li et al. \cite{li2023exploring} introduces the Potential Energy Decline (PED) approach, a physics-inspired method aimed at better capturing the dynamics of model adaptability by modeling the interactive forces that influence the fine-tuning process. \cite{menta2024active} enhances transferability estimation by selecting the most informative subset from the target dataset using the pre-trained model's internal and output representations. However, none of the existing works adequately address the aspect of robustness in transferability estimation. Our study introduces a vital perspective, crucial for developing more reliable transferability estimation metrics that effectively incorporate both adaptability and robustness.

Li et al. \cite{li2023exploring} proposed Potential Energy Decline (PED), a physics-inspired method modeling interactive forces to capture model adaptability during fine-tuning. Menta et al. \cite{menta2024active} improved transferability estimation by selecting informative subsets of the target dataset using pre-trained model representations. However, existing methods overlook robustness in transferability estimation. Our study addresses this gap by introducing a framework that integrates adaptability and robustness for more reliable transferability metrics.
\section{Methodology}

In this section, we outline the problem definition, our proposed method, transferability estimation metrics, and the evaluation criteria for model selection. For the sake of simplicity, we use image classification as our primary task throughout the paper.

\subsection{Preliminaries}

\textbf{Problem definition:}
Let $\mathcal{T} = \{X,Y\}$ represent the target dataset, and $\{\phi_l\}_{l=1}^L$ denote $L$ the pre-trained feature-extractors, with $l$ being the index. Using transferability estimation, given by $\mathcal{M}$, our goal is to rank these pre-trained models based on their performance on a given target dataset. We assess the transferability of each pre-trained model using a specific metric $\mathcal{M}$ that produces a numerical score, denoted as $T_l$. Essentially, a higher score for $T_l$ implies that the model $\phi_l$ is more likely to perform effectively on the given target dataset. 

\noindent\textbf{Ground truth:}
To establish a reliable basis for comparing pre-trained models, a process of fine-tuning each model on the target dataset is conducted, accompanied by a thorough exploration of different hyperparameters (i.e., learning rate and weight decay). Test accuracy obtained from this fine-tuning is then utilized as a `ground truth' for the ranking of these models. The fine-tuning performance of each model is notated as $\{G_l\}_{l=1}^L$, serving as a benchmark for evaluating the model ranking by transferability estimation metric.


\subsection{Proposed Approach}
\label{sec:SA}


Source-embedding based transferability estimation metrics\cite{shao2022not, DBLP:conf/iccv/WangLZHB23,li2021ranking,Pandy2022TransferabilityEstimation,you2021logme} leverage features extracted from pre-trained models to estimate their adaptability to a downstream task. Our proposed feature perturbation method systematically tests the robustness of model embeddings by applying controlled perturbations. Recognizing that the embedding structure varies with the feature extractor and the target dataset, our method avoids a one-size-fits-all perturbation. Instead, it dynamically adjusts the perturbation magnitude to preserve the meaningful structure of embeddings. We achieve this through two operations called spread and attract (SA), which are described below:



\noindent\textbf{Spread operation:} One of the desirable properties of features in supervised learning is high intra-class compactness \cite{bengio2013representation}. The Spread operation deliberately perturbs the internal structure of each class by increasing the intra-class variance, effectively decreasing intra-class compactness. This perturbation pushes examples that were previously near the centroid—and thus easier to classify—further away, increasing their variability and making them more challenging to classify correctly. Specifically, for every class, we compute the centroid of the class \( \mathbf{C}_u\) as given by Eq. \ref{eq:centroid} and displace each data point uniformly away from its centroid, given by:



\begin{equation}
\mathbf{\hat{X}}_{\text{spread}, u} = \mathbf{\hat{X}}_u + \left( \frac{\mathbf{\hat{X}}_u - \mathbf{C}_u}{\|\mathbf{\hat{X}}_u - \mathbf{C}_u\|_2} \right)
\label{eq:spread}
\end{equation}


\begin{equation}
\mathbf{C}_u = \frac{1}{n_u} \sum_{i=1}^{n_u} \mathbf{\hat{X}}_{u,i}
\label{eq:centroid}
\end{equation}

where \(\mathbf{\hat{X}}_{\text{spread}, u}\) represents the feature embeddings for class \(u\) after the spread operation, \(\mathbf{C}_u\) is the centroid of class \(u\), \(\mathbf{\hat{X}}_{u}\) is the reduced-dimensionality embedding of class \(u\) obtained after applying PCA on the extracted feature embeddings, and \( n_u \) is the number of samples in class \( u \).


\noindent\textbf{Attract operation:}  Another desirable property of features in supervised learning is high inter-class separability\cite{bengio2013representation}. Supervised models for classification tasks generally learn clear boundaries between classes in feature space. The attract operation aims to diminish this separation by adjusting class embeddings according to the distances between their centroids and variance, thereby perturbing the feature space to create closer inter-class proximities. This operation tests the resilience of the model's embeddings in scenarios where class boundaries are intentionally blurred.
The attract operation is formulated as:

\begin{equation}
\mathbf{\hat{X}}_{\text{attract}_{u}} = \mathbf{\hat{X}}_{\text{spread}_{u}} + \alpha \cdot \mathbf{Disp}_{uv}
\label{eq:attract}
\end{equation}

where, \(\mathbf{\hat{X}}_{\text{attract}_{u}}\) is the feature embedding of the class \textit{u} after applying the attract method to \(\mathbf{\hat{X}}_{\text{spread}_{u}}\), 
$\alpha$ is a hyper-parameter that modulates the magnitude of the displacement towards the other class centroids, and $\mathbf{Disp}_{uv}$ is: 

\begin{equation}
\mathbf{Disp}_{uv} = \sum_{v \neq u} \left( \frac{\mathbf{D}_{uv}}{\|\mathbf{D}_{uv}\|_2} \right) \cdot \left( \|\mathbf{D}_{uv}\|_2 - (\sigma \cdot R_u + \sigma \cdot R_v) \right) \\ \nonumber
\label{eq:displacement}
\end{equation} 

\begin{equation}
R_u = \sqrt{\frac{1}{n_u} \sum_{i=1}^{n_u} \| \mathbf{X}_{u,i} - \mathbf{C}_u \|^2}
\label{eq:standard_deviation}
\end{equation}

Here, $\mathbf{D}_{uv} = \mathbf{C}_u - \mathbf{C}_v$ is the distance vector between the centroids \(\mathbf{C}\) of two different classes, \(u\) and \(v\). $\sigma$ is a hyper-parameter that adjusts the sensitivity of the displacement to the variability within class embeddings, as measured by $R_u$ and $R_v$. 
Attract operation ensures that the perturbations are proportionate to the natural variability within and between the classes. This approach helps to preserve the meaningful structure of the embeddings while testing their resilience against perturbations.



\begin{algorithm} [!h]
\caption{SA algorithm}
\label{alg:SA}
\KwIn{Target dataset $\mathcal{T}=\{X,Y\}$, pre-trained models $\{\phi_l\}_{l=1}^L$, hyper-parameters $\sigma$, $\alpha$, transferability metric $\mathcal{M}$}
\KwOut{Transferability scores $T_l$ for each model}
\For{$l=1$ \textbf{to} $L$}{
    $\hat{X} \leftarrow$ PCA($\phi_l(X)$) \\
    \For{each class $k$ in $Y$}{
        $C_k \leftarrow$ Compute centroid of $\hat{X}_k$ \\
        \For{each point $X_{k,i}$ in class $k$}{
            $D_{k,i} \leftarrow \frac{X_{k,i} - C_k}{\|X_{k,i} - C_k\|_2}$ \\
            $X_{\text{spread}_{k,i}} \leftarrow X_{k,i} + D_{k,i}$ \tcp{Spread}
        }
    }
    \For{each class pair $(u, v)$}{
        $D_{uv} \leftarrow C_u - C_v$ \\
        $R_u \leftarrow \| \text{Std}(\hat{X}_u) \|_2$, $R_v \leftarrow \| \text{Std}(\hat{X}_v) \|_2$ \\
        \For{each $X_{\text{spread}_{u,i}}$ in class $u$}{
            $Disp_{uv,i} \leftarrow \sum_{v \neq u} \frac{D_{uv}}{\|D_{uv}\|_2} \cdot (\|D_{uv}\|_2 - (\sigma \cdot R_u + \sigma \cdot R_v))$ \\
            $X_{\text{attract}_{u,i}} \leftarrow X_{\text{spread}_{u,i}} + \alpha \cdot Disp_{uv,i}$ \tcp{Attract}
        }
    }
    $T_l \leftarrow \mathcal{M}(X_{\text{attract}}, Y)$
}
Rank models based on $T_l$.
\end{algorithm}


 \noindent\textbf{The importance of controlled perturbation:}
Fig. \ref{fig:hyp}(b) demonstrates an optimal level of perturbation that preserves the necessary balance and structural integrity required for accurate transferability estimation. Conversely, Fig. \ref{fig:hyp}(c) shows how suboptimal perturbation settings can result in excessive alterations, complicating the feature space and hindering transferability assessment. These examples underline the crucial role of carefully calibrated perturbations in ensuring reliable model evaluations.


\begin{figure}[!h]
  \centering
  \includegraphics[width=0.90\columnwidth]{images/new_hyperparameter_variation.pdf}
  \vspace{-2ex}
  \caption{Demonstrating the importance of controlled perturbation in feature space manipulation, using a toy example. (a) Represents the initial target embedding. (b) depict an appropriate amount of feature perturbation, while (c) demonstrate excessive levels of feature perturbation.}
  \label{fig:hyp}
  \vspace{-1ex}
\end{figure}

\subsection{Overall Objective}

Our feature perturbation method systematically alters the feature embedding representation for target dataset $\mathcal{T}=\{X,Y\}$ using $\{\phi_l\}_{l=1}^L$ feature extractor of the pre-trained model. We then compute the transferability score for each model $\{T_l\}_{l=1}^L$ using the metric $\mathcal{M}$, which assesses the perturbed embeddings $\mathbf{\hat{X}}_{\text{attract}}$ against the target labels $\mathbf{Y}$, as shown in the equation below. 
\begin{equation}
\label{eq:transferability_estimation}
    T_l = \mathcal{M}(\mathbf{\hat{X}}_{\text{attract}}, \mathbf{Y})
\end{equation}
For each feature extractor $\phi_l$, the algorithm extracts feature embeddings and applies PCA to get a reduced-dimensionality representation, $\hat{X}$. This step is crucial for managing computational complexity and focusing on the most informative features of the embeddings. The algorithm for our perturbation method is given in Algorithm \ref{alg:SA}. 
% Essentially, a higher score for $T_l$ implies that the model $\phi_l$ is more likely to perform effectively on the given target dataset.


% \textcolor{red}{Choose one of the algorithm (see code editor!)}

% \begin{algorithm} [!t]
% \caption{Option 2: Detailed Feature perturbation algorithm}
% \label{alg:SA}
% \KwIn{Target dataset $\mathcal{T}=\{X,Y\}$; Pre-trained models $\{\phi_l\}_{l=1}^L$; Hyper-parameter: Difficulty enhance parameters $\sigma$; TE metric $\mathcal{M}$;}
% \KwOut{$T_l$ metric score for each model}
% \For{$l=1$ {\bfseries to} $L$}{
%     Extract feature embedding from model $\phi_l$ and apply PCA to get $\hat{X}$\\
%     Apply Spread operation: \\
%     \For{each class $k$ in $Y$}{
%         Calculate the class centroid $C_k$: $C_k = \frac{1}{N_k} \sum_{i=1}^{N_k} \hat{X}_{k,i}$ \\
%         Apply Spread operation to class $k$: \\
%         \For{each point $X_{k,i}$ in class $k$}{
%             Calculate normalized direction vector $D_{k,i}$: 
%             $D_{k,i} = \frac{\hat{X}_{k,i} - C_k}{\|\hat{X}_{k,i} - C_k\|_2}$ \\
%             Update embedding: 
%             $\hat{X}_{\text{spread}_{k,i}} = \hat{X}_{k,i} + D_{k,i}$ \\
%         }
%     }
  
%     % Combine all spread points into $X_{\text{spread}}$ \\
%     Apply Attract operation: \\
%     \For{each pair of different classes $(u, v)$}{
%         % Calculate standard deviation $R_u, R_v = \sqrt{\text{var\(\hat{X}_spread\)}_u}, \sqrt{\text{var\(\hat{X}_spread\)}_v}$ \\
%         Calculate $L2$-norm of standard deviation $(SD)$ for each class 
%          $R_u = \| SD_u \|_2, \quad R_v = \| SD_v \|_2$  \\
%         % Spread $R_u = \|\hat{X}_u\|_2, \quad R_v = \|\hat{X}_v\|_2$ \\
%         % Calculate standard deviation $R_u, R_v = \sqrt{\frac{1}{N_u}\sum_{i=1}^{N_u} \|\hat{X}_{\text{spread}_{u,i}} - C_u\|_2^2}, \sqrt{\frac{1}{N_v}\sum_{i=1}^{N_v} \|\hat{X}_{\text{spread}_{v,i}} - C_v\|_2^2}$ \\

%         Calculate distance vector $D_{uv}$: $D_{uv} = C_u - C_v$ \\
%         \For{each point $\hat{X}_{\text{spread}_{u,i}}$ in class $u$}{
%             Calculate attraction displacement $Disp_{uv,i}$: $Disp_{uv,i} = \sum_{v \neq u} \left( \frac{D_{uv}}{\|D_{uv}\|_2} \right) \cdot \left( \|D_{uv}\|_2 - (\sigma \cdot R_u + \sigma \cdot R_v) \right)$ \\
%             Update embedding: $\hat{X}_{\text{attract}_{u,i}} = \hat{X}_{\text{spread}_{u,i}} + \alpha \cdot Disp_{uv,i}$ \\
%         }
%     }
%     $T_l = \mathcal{M}(\hat{X}_{\text{attract}},Y)$
  
% }
% Rank the models based on the $T_l$ metric score.
% \end{algorithm}


% \vspace{-1ex}

% \textcolor{red}{can we make this para better?}
% The combined effects of the spread and attract operations create a perturbed feature space that rigorously tests the models' ability to maintain distinct and organized class representations under altered conditions. This serves as a rigorous test of their robustness. Models that continue to exhibit clear, discernible features despite these perturbations demonstrate a high level of resilience and are likely to receive higher rankings, thereby improving the accuracy of transferability estimations. Moreover, our feature perturbation method not only enhances the transferability estimation process by rigorously testing and proving the endurance of model embeddings under stress but also ensures that models identified for their robustness are truly capable of adapting to diverse and challenging scenarios. This approach not only refines the evaluation metrics but also significantly boosts confidence in the practical application of these models.


% vanilla supervised
\begin{table*}[t!]\setlength\tabcolsep{5pt}
\footnotesize
\centering
\caption{Performance comparison (average weighted Kendall $\tau_w$) between original and enhanced frameworks for vanilla fine-tuning on supervised models. For a pair of rows, the original metric is presented first followed by the corresponding enhanced metric. Best results are highlighted in bold. Our framework consistently outperforms the original framework across all metrics.} 
\label{tab:sup_fine-tune}
\begin{adjustbox}{max width=\textwidth}
\begin{tabular}{l|ccccccccccc|c}
    \toprule
    Method          & Aircraft & Caltech-101 & Cars  & CIFAR10 & CIFAR100 & DTD   & Flowers & Food-101 & Pets  & Sun & VOC & Average \\ 
    \midrule
    $\mathcal{N}$LEEP\cite{li2021ranking}           & -0.449   & \textbf{0.769}       & 0.602 & 0.783   & 0.717    & 0.796 & 0.295   & 0.581    & 0.511 & \textbf{0.944} & 0.710 & 0.569 \\
    \textbf{SA} + $\mathcal{N}$LEEP& \textbf{0.236}   & 0.626       & \textbf{0.763} & \textbf{0.910}    & \textbf{0.843}    & \textbf{0.836} & \textbf{0.435}   & \textbf{0.657}   & \textbf{0.829} & 0.790 & \textbf{0.828}   & \textbf{0.704} \\
    \hline
    LogME\cite{you2021logme}           & 0.439    & 0.497       & \textbf{0.605} & 0.852   & 0.725    & 0.700 & 0.147   & 0.385    & 0.411 & 0.511 & 0.695 & 0.542 \\
    \textbf{SA} + LogME & \textbf{0.442}    & \textbf{0.655}       & 0.603 & \textbf{0.924}  & \textbf{0.855}    & \textbf{0.784} & \textbf{0.743}   & \textbf{0.665}    & \textbf{0.447} & \textbf{0.788} & \textbf{0.782} & \textbf{0.698} \\
    \hline
    GBC\cite{Pandy2022TransferabilityEstimation}             & \textbf{0.423}    & 0.213       & \textbf{0.617} & 0.735   & 0.664    & 0.703 & 0.214   & 0.548    & 0.514 & 0.271 & 0.743 & 0.513 \\
    \textbf{SA} + GBC   & -0.110    & \textbf{0.473}       & 0.591 & \textbf{0.928}   & \textbf{0.789}    & \textbf{0.713} & \textbf{0.510}  & \textbf{0.711}   & \textbf{0.651} & \textbf{0.803} & \textbf{0.787} & \textbf{0.622} \\
    \hline
    SFDA\cite{shao2022not}            & \textbf{0.614}    & \textbf{0.615}       & 0.574 & \textbf{0.949}   & 0.866    & 0.575 & 0.492   & \textbf{0.815}    & 0.545 & 0.558 & 0.671 & 0.661 \\
    \textbf{SA} + SFDA  & 0.414    & 0.598       & \textbf{0.801} & 0.901   & \textbf{0.908}    & \textbf{0.816} & \textbf{0.736}   & 0.681    & \textbf{0.865} & \textbf{0.790} & \textbf{0.816} & \textbf{0.756} \\
    \hline
    NCTI\cite{DBLP:conf/iccv/WangLZHB23}            & 0.496     & \textbf{0.492}      & 0.662 & 0.843   & \textbf{0.879}    & 0.616 & 0.541   & \textbf{0.773}    & \textbf{0.867} & 0.756 & 0.741 & 0.697 \\
    \textbf{SA} + NCTI  & \textbf{0.872}   & 0.483       & \textbf{0.805} & \textbf{0.843}   & 0.878    & \textbf{0.776} & \textbf{0.714}   & 0.619    & 0.856 & \textbf{0.790} & \textbf{0.800}   & \textbf{0.767} \\
    \bottomrule
\end{tabular}
\end{adjustbox}
\end{table*}

% LBFT supervised
\begin{table*}[h]\setlength\tabcolsep{5pt}
\footnotesize
\centering
\caption{Performance comparison (average weighted Kendall $\tau_w$) between original and enhanced frameworks for LBFT on supervised models. For a pair of rows, the original metric is presented first followed by the corresponding enhanced metric. The optimal outcomes are emphasized in bold. Our framework consistently exceeds the performance of the original framework across all metrics.} 
\label{tab:sup_conv-ft}
\begin{adjustbox}{max width=\textwidth}
\begin{tabular}{l|ccccccccccc|c}
    \toprule
    Method & Aircraft & Caltech-101 & Cars & CIFAR10 & CIFAR100 & DTD & Flowers & Food-101 & Pets & Sun & VOC & Average \\ \midrule
    $\mathcal{N}$LEEP\cite{li2021ranking} & -0.415 & 0.370 & 0.159 & 0.732 & 0.803 & 0.593 & -0.035 & 0.667 & 0.505 & \textbf{0.691} & 0.512 & 0.417 \\
    \textbf{SA} + $\mathcal{N}$LEEP & \textbf{0.305} & \textbf{0.735} & \textbf{0.848} & \textbf{0.757} & \textbf{0.853} & \textbf{0.629} & \textbf{0.020} & \textbf{0.742} & \textbf{0.529} & 0.641 & \textbf{0.777} & \textbf{0.621} \\ 
    \hline
    LogME\cite{you2021logme} & \textbf{0.386} & \textbf{0.577} & 0.453 & \textbf{0.789} & 0.640 & \textbf{0.715} & 0.286 & 0.690 & 0.192 & \textbf{0.627} & 0.222 & 0.507 \\
    \textbf{SA} + LogME & 0.223 & 0.260 & \textbf{0.633} & 0.739 & \textbf{0.831} & 0.548 & \textbf{0.458} & \textbf{0.783} & \textbf{0.233} & 0.622 & \textbf{0.707} & \textbf{0.548} \\
    \hline
     GBC\cite{Pandy2022TransferabilityEstimation} & \textbf{0.676} & 0.076 & 0.476 & 0.631 & 0.751 & 0.612 &\textbf{ 0.176} & \textbf{0.790} & \textbf{0.349} & 0.395 & 0.210 & 0.467 \\
    \textbf{SA} + GBC & -0.164 & \textbf{0.541} & \textbf{0.626} & \textbf{0.656} & \textbf{0.767} & \textbf{0.650} & 0.133 & 0.749 & 0.149 & \textbf{0.641} & \textbf{0.777} & \textbf{0.502} \\
    \hline
    SFDA\cite{shao2022not} & \textbf{0.395} & 0.432 & 0.324 & 0.702 & 0.671 & 0.585 & \textbf{0.414} & 0.553 & 0.372 & 0.393 & 0.142 & 0.453 \\
    \textbf{SA} + SFDA & 0.174 & \textbf{0.754} & \textbf{0.864} & \textbf{0.765} & \textbf{0.775} & \textbf{0.768} & 0.361 & \textbf{0.708} & \textbf{0.698} & \textbf{0.610} & \textbf{0.764} & \textbf{0.658} \\
    \hline
    NCTI\cite{DBLP:conf/iccv/WangLZHB23} & 0.366 &0.441& 0.447 & 0.728 & 0.760 & 0.395 & 0.150 & 0.637 & \textbf{0.766} & \textbf{0.848} & 0.388 & 0.539 \\
    \textbf{SA} + NCTI & \textbf{0.620} & \textbf{0.652} & \textbf{0.942} & \textbf{0.739} &\textbf{ 0.789} & \textbf{0.568} & \textbf{0.415} & \textbf{0.730} & 0.685 & 0.641 & \textbf{0.792} & \textbf{0.688} \\ 
    \bottomrule
\end{tabular}
\end{adjustbox}
\end{table*}


 % linear probing supervised
\begin{table*}[h]\setlength\tabcolsep{5pt}
\footnotesize
\centering
\caption{Comparison (average weighted Kendall $\tau_w$) between original and enhanced frameworks for LFT on supervised models. In each pair of rows, the original metric is listed first, followed by the corresponding enhanced metric. The superior results are emphasized in bold.} 
\label{tab:sup_lp}
\begin{adjustbox}{max width=\textwidth}
\begin{tabular}{l|ccccccccccc|c}
    \toprule
    Method & Aircraft & Caltech-101 & Cars & CIFAR10 & CIFAR100 & DTD & Flowers & Food-101 & Pets & Sun & VOC & Average \\ \midrule
    $\mathcal{N}$LEEP\cite{li2021ranking} & 0.422 & 0.685 & \textbf{0.747} & 0.558 & 0.509 & \textbf{0.790} & \textbf{0.378} & 0.512 & \textbf{0.717} & \textbf{0.746} & 0.698 &\textbf{ 0.615} \\
    \textbf{SA} + $\mathcal{N}$LEEP & \textbf{0.615} & \textbf{0.760} & 0.328 & \textbf{0.831} & \textbf{0.705} & 0.635 & 0.285 & \textbf{0.615} & 0.677 & 0.244 & \textbf{0.824} & 0.592 \\
    \hline
    LogME\cite{you2021logme} & \textbf{0.127} & 0.247 & \textbf{0.144} & 0.490 & 0.359 & \textbf{0.721} & 0.170 & 0.229 & -0.128 & 0.198 & 0.470 & 0.275 \\
    \textbf{SA} + LogME & -0.129 & \textbf{0.363} & 0.073 & \textbf{0.907} & \textbf{0.883} & 0.634 & \textbf{0.465} & \textbf{0.625} & \textbf{0.178} & \textbf{0.262} & \textbf{0.689} & \textbf{0.450} \\
    \hline
     GBC\cite{Pandy2022TransferabilityEstimation} & \textbf{-0.048} & 0.266 & \textbf{0.124} & 0.359 & 0.373 & 0.440 & 0.121 & 0.330 & \textbf{0.209} & 0.228 & 0.531 & 0.267 \\
    \textbf{SA} + GBC & -0.308 & \textbf{0.628} & 0.044 & \textbf{0.935} & \textbf{0.818} & \textbf{0.758} & \textbf{0.304} & \textbf{0.694} & 0.096 & \textbf{0.244} & \textbf{0.824} & \textbf{0.457} \\
    \hline
    SFDA\cite{shao2022not} & 0.320 & 0.475 & \textbf{0.508} & 0.611 & 0.558 & 0.380 & \textbf{0.429} & \textbf{0.709} & 0.119 & \textbf{0.574} & 0.432 & 0.465 \\
    \textbf{SA} + SFDA & \textbf{0.585} & \textbf{0.934} & 0.381 & \textbf{0.882} & \textbf{0.905} & \textbf{0.855} & 0.307 & 0.684 & \textbf{0.751} & 0.251 & \textbf{0.812} & \textbf{0.667} \\
    \hline
    NCTI\cite{DBLP:conf/iccv/WangLZHB23} & \textbf{0.656} & 0.775 & \textbf{0.572} & 0.627 & 0.636 & 0.526 & \textbf{0.589} & \textbf{0.675} & 0.431 &\textbf{ 0.521} & 0.667 & 0.607 \\
    \textbf{SA} + NCTI & 0.182 & \textbf{0.833} & 0.381 & \textbf{0.907} & \textbf{0.840} & \textbf{0.738} & 0.423 & 0.635 & \textbf{0.706} & 0.244 & \textbf{0.839} & \textbf{0.611} \\
    \bottomrule
\end{tabular}
\end{adjustbox}
\end{table*}

\section{Experiments}
\label{sec:exp}

This section is organized into five distinct parts to evaluate the proposed feature perturbation method. Section \ref{sect:setup} outlines the experimental setup; Section \ref{sect:result} presents the results; Section \ref{sect:ablation} covers the ablation study; Section \ref{sect:hyperparameter} examines hyper-parameter sensitivity; Section \ref{sect:time} analyzes time complexity; and section \ref{sect:self-supervised} presents the result on self-supervised architectures.

\subsection{Experiment Setup}
\label{sect:setup}

\noindent\textbf{Datasets.} Our study utilizes a diverse collection of datasets commonly used in transferability estimation research \cite{shao2022not}. The collection contains fine-grained object classification dataset (\textit{i.e.}, Oxford-102 Flowers \cite{nilsback2008automated}, Food-101 \cite{bossard2014food}, Stanford Cars \cite{cars}, FGVC Aircraft \cite{maji2013fine}, Oxford-IIIT Pets \cite{parkhi2012cats}),
coarse-grained object classification dataset (\textit{i.e.}, Caltech-101 \cite{fei2004learning}, Cifar-10 \cite{cifar}, Cifar-100 \cite{cifar}, Voc2007 \cite{pascal-voc-2007} ), 
one scene classification dataset (\textit{i.e.}, Sun397 \cite{xiao2010sun}), and 
one texture classification dataset (\textit{i.e.}, DTD \cite{cimpoi2014describing}). These datasets provide a broad spectrum of challenges under various scenarios.


\subsection{Experimental Results}
\label{sect:result}

\noindent\textbf{Overview.} To evaluate the performance of transferability estimation metrics, we initiate the model pool by following the existing work~\cite{shao2022not}. Specifically, the model pool consists of 11 ImageNet pre-trained architectures, including InceptionV1 \cite{goingdeeperwithconvolutions}, InceptionV3 \cite{rethinkingtheinceptionarchitectureforcv}, ResNet50 \cite{deepresidulalearningforimagerecognition}, ResNet101 \cite{deepresidulalearningforimagerecognition}, ResNet152 \cite{deepresidulalearningforimagerecognition}, DenseNet121 \cite{denselyconnectedconvolutionalnetworks}, DenseNet169 \cite{denselyconnectedconvolutionalnetworks}, DenseNet201 \cite{denselyconnectedconvolutionalnetworks}, MobileNetV2 \cite{mobilenetv2invertedresidualsandlinearbottlenecks}, and NASNet-A Mobile \cite{mnasnetplatformawareneuralarchitecturesearchformobile}. 


\noindent\textbf{Ground truth and correlation measurement.}
We follow the grid search described in \cite{shao2022not}, which selects the learning rates from $\{10^{-1}, 10^{-2}, 10^{-3}, 10^{-4}\}$ and weight decay parameters from $\{10^{-6}, 10^{-5}, 10^{-4}, 10^{-3}\}$. To ensure the robustness and reliability of our evaluation, we execute the code using five distinct seeds for each experiment and then take the average of target accuracy. For evaluation, we use weighted Kendall \cite{kendall1938new} correlation coefficient $\tau_w$ because it assigns weights to the concordant and discordant pairs based on their positions in the ranking; A high positive $\tau_w$ value indicates a strong correlation, while a negative value implies an inverse correlation.


\noindent\textbf{Performance comparison on vanilla fine-tuning.} In this part, we consider vanilla fine-tuning, which updates all parameters during the training process. We report the experimental results in Table \ref{tab:sup_fine-tune}. As can be seen from the table, SA feature perturbation method demonstrates notable improvements in transferability estimation across all previous metrics. In particular, our proposed method achieves a relative improvement of 23.86 \% on $\mathcal{N}$LEEP, 28.84\% on LogMe, 21.27\% on GBC, 14.46\% on SFDA, 10.04\% on NCTI, and with the LogMe metric benefiting the most from this approach. The table reports an average improvement of 19.69\% on previous SOTA transferability estimation methods, indicating the effectiveness of our approach.

% \textcolor{red}{should we remove this para. This figure and para can develop some question in the minds of reviewers}
Meanwhile, our perturbation strategy demonstrates variably beneficial outcomes in a few datasets. For example, on Aircraft, our methods can only improve three out of five baseline metrics. We argue that embedding structure for a few datasets contains class overlap causing mixed improvements for various metrics. To understand the intrinsic difference of these datasets, we randomly sample three classes and visualize the feature distribution of (a) variably beneficial datasets and (b) consistent beneficial datasets in Fig. \ref{fig:result_embed}. One can see from the figure that the class-conditional feature distribution largely overlaps with each other. A similar pattern can be seen in other datasets like Caltech-101 and Cars. Given the well-separated feature embeddings in set (b), our perturbation method has a more pronounced effect in enhancing transferability estimation. This improvement is evident in the column corresponding to the dataset in Table \ref{tab:sup_fine-tune}. %Additional results and detailed analyses are provided in the supplementary material.

\begin{figure}[!t]
  \centering
  \includegraphics[width=\columnwidth]{images/result_viz_updated_color.pdf}
  \caption{Visualization of ResNet50 target embeddings before feature perturbation (best viewed in color): (a) Represents datasets exhibiting mixed improvement for various metrics, shown in Table \ref{tab:sup_fine-tune}. The presence of class overlap in (a) contributes to the varied performance across metrics. In contrast, (b) depicts datasets demonstrating consistent improvement across all other evaluated metrics, facilitated by well-separated embeddings. This distinction underscores the role of embedding structure in the estimation.}
  \label{fig:result_embed}
  \vspace{-1ex}
  %\vspace{-2em}
\end{figure}


\noindent\textbf{Performance comparison on fine-tuning variants.} In this work, we study the performance of transferability estimation metrics under two alternative fine-tuning strategies, namely LBFT and LFT. LBFT updates the last convolutional block and the final linear layer of the network, and LFT only updates the final linear layer with all other parameters frozen in the training process. More details can be found in the supplementary material. 

Table \ref{tab:sup_conv-ft} shows the ranking performance of the models trained under the LBFT strategy.  Although the order of rank of the estimation metrics remains largely consistent between vanilla fine-tuning and LBFT, the overall ranking correlation drops from 0.59 to 0.47. This suggests that previous metrics could be vulnerable when applied to LBFT. Following SA feature perturbation, there is a substantial performance boost of 27.47\% for LBFT, proving the effectiveness of our feature perturbation method in enhancing accurate transferability estimation specifically for LBFT scenarios.

Compared to vanilla fine-tuning, LFT can be deployed faster, with improved performance on out-of-distribution (OOD) test samples. The ranking performance under the LFT strategy is shown in Table.~\ref{tab:sup_lp}. While a strategy like LFT does not change the structure of the feature space during training, our approach can still achieve a predominantly beneficial result. SA feature perturbation strategy improves three out of five comparison metrics, scoring 0.142 weighted Kendall $\tau_w$ increase. The results reflect that our approach is suitable to adopt in the LFT use case.

\begin{table*}[!tbh]
\setlength\tabcolsep{5pt}
\footnotesize
\centering
\caption{Performance comparison (average weighted Kendall $\tau_w$) for vanilla fine-tuning on self-supervised models.  In each column, the best results are highlighted in bold. Remarkably, LDA achieves the highest overall average weighted Kendall’s $\tau_w$ score.} 
\label{tab:ss_vanilla-ft}
\begin{adjustbox}{max width=\textwidth}
\begin{tabular}{l|ccccccccccc|c}
    \toprule
    Method & Aircraft & Caltech-101 & Cars & CIFAR10 & CIFAR100 & DTD & Flowers & Food-101 & Pets & Sun397 & VOC & Average \\ 
    \midrule
    $\mathcal{N}$LEEP\cite{li2021ranking} & -0.029 & 0.631 & 0.358 & 0.074 & 0.276 & 0.641 & 0.585 & 0.544 & \textbf{0.836} & 0.735 & -0.076 & 0.416 \\
    LogME\cite{you2021logme} & 0.223 & 0.387 & 0.387 & 0.295 & -0.028 & 0.627 & 0.718 & 0.570 & 0.704 & 0.217 & 0.121 & 0.384 \\
    GBC\cite{Pandy2022TransferabilityEstimation} & 0.070 & 0.417 & 0.464 & -0.054 & 0.237 & 0.317 & 0.701 & 0.729 & 0.484 & 0.539 & 0.161 & 0.370 \\
    SFDA\cite{shao2022not} & \textbf{0.254} & 0.526 & 0.553 & 0.619 & 0.548 & 0.815 & \textbf{0.847} & 0.685 & 0.556 & 0.732 & 0.532 & 0.606 \\
    NCTI\cite{DBLP:conf/iccv/WangLZHB23} & 0.035 & 0.643 & \textbf{0.724} & 0.546 & 0.533 & 0.715 & 0.705 & 0.892 & 0.767 & 0.697 & 0.547 & 0.619 \\
    LDA & 0.058 & \textbf{0.708} & 0.720 & \textbf{0.707} & \textbf{0.692} & \textbf{0.913} & 0.779 & \textbf{0.944} & 0.540 & \textbf{0.892} & \textbf{0.723} & \textbf{0.698} \\
    \bottomrule
\end{tabular}
\end{adjustbox}
\end{table*} 


% selfsup linear probe
\begin{table*}[!tbh] 
    \setlength\tabcolsep{5pt}
    \footnotesize
    \centering
    \caption{Performance comparison (average weighted Kendall $\tau_w$) for LFT on self-supervised models. 
    The highest performing $\tau_w$ value in each column are highlighted in bold. LDA achieves the highest overall average weighted Kendall $\tau_w$ score.} 
    \label{tab:ss_lp}
    \begin{adjustbox}{max width=\textwidth}
    \begin{tabular}{l|ccccccccccc|c}
        \toprule
        Method & Aircraft & Caltech-101 & Cars & CIFAR10 & CIFAR100 & DTD & Flowers & Food-101 & Pets & Sun397 & VOC & Average \\
        \midrule
        $\mathcal{N}$LEEP\cite{li2021ranking} & 0.446 & 0.654 & 0.537 & -0.044 & 0.210 & 0.668 & 0.633 & 0.519 & 0.608 & 0.275 & 0.126 & 0.421 \\
        LogME\cite{you2021logme} & 0.569 & 0.350 & 0.629 & -0.083 & -0.250 & 0.655 & 0.653 & 0.518 & 0.487 & -0.178 & 0.037 & 0.308 \\
        GBC\cite{Pandy2022TransferabilityEstimation} & 0.498 & 0.446 & 0.717 & -0.093 & 0.147 & 0.504 & 0.702 & 0.590 & 0.422 & 0.219 & 0.320 & 0.407 \\
        SFDA\cite{shao2022not} & 0.685 & 0.582 & 0.813 & 0.275 & 0.138 & 0.717 & 0.705 & 0.693 & 0.681 & 0.426 & 0.633 & 0.577 \\
        NCTI\cite{DBLP:conf/iccv/WangLZHB23} & 0.842 & 0.661 & \textbf{0.917} & 0.275 & 0.473 & 0.699 & 0.683 & 0.846 & \textbf{0.846} & 0.308 & 0.670 & 0.656 \\
        LDA & \textbf{0.903} & \textbf{0.764} & 0.800 & \textbf{0.598} & \textbf{0.497} & \textbf{0.807} & \textbf{0.845} & \textbf{0.867} & 0.748 & \textbf{0.656} & \textbf{0.823} & \textbf{0.755} \\
        \bottomrule
    \end{tabular}
    \end{adjustbox} \vspace{-1ex}
\end{table*}


\subsection{Ablation Study}
\label{sect:ablation}

To validate the effectiveness of each component in the proposed methods, we conduct an ablation study of the Spread and Attract operation on four different baseline transferability estimation metrics (\textit{i.e.,} NCTI, SFDA, LogMe, and GBC). We show the experimental results in Fig. \ref{fig:ablation}. From the figure, we can observe that the application of both the Spread and Attract operations brings improvement in ranking correlations. On average, the Spread operation yields a 10.57\% improvement, while the Attract operation provides a 14.57\% improvement. The Attract operation's superior performance can be attributed to its sophisticated approach that accounts for both the intra-class coherence and the inter-class separations before perturbation. Specifically, as defined in Eq \ref{eq:attract}, the Attract operation applies changes to the embeddings after considering the distance between class centroids (\(\mathbf{D}_{uv}\)) and the L2 norm of standard deviations within the classes (\(R_u\) and \(R_v\)). This allows for perturbations that are informed by a holistic view of the entire feature space, maintaining a delicate balance between disrupting and preserving the structural integrity essential for accurate class differentiation.
Moreover, the most substantial enhancement is observed when both operations are applied sequentially: first applying the Spread operation followed by the Attract operation. This sequential application leads to an average improvement of 17.82\% over the originally obtained weighted Kendall coefficient. This demonstrates that the synergistic effect of applying both operations sequentially is significantly greater than the impact of each operation when applied independently.


\begin{figure}
    \centering
    \includegraphics[width=0.9\columnwidth]{images/improvement-separate.pdf}
    \vspace{-2ex}
    \caption{This figure demonstrates a bar chart that illustrates the performance improvement of various operations of feature perturbation over the original baseline. Each metric is represented by four bars, corresponding to different operations: Original, Spread, Attract, and Combined Spread-Attract, illustrating that the combined approach significantly outperform others.}
    \label{fig:ablation}
    \vspace{-3ex}
\end{figure}

\begin{figure}
    \centering
    \includegraphics[width=\columnwidth]{images/sensitivity-Page-2.png}
    \vspace{-4ex}
    \caption{Hyper-parameter sensitivity analysis: The left figure showcases consistent performance ($\tau_w$) across a wide range of $\alpha$ values, at optimum $\sigma$. This observation indicates an insensitivity to hyper-parameter changes. On the other hand, the right figure illustrates limited variance in performance ($\tau_w$) across a broad spectrum of $\sigma$ values, ranging from 0.5 to 0.9, at optimum $\alpha$. %The minimal performance fluctuation observed across diverse parameter ranges underscores the algorithm's resilience, thereby emphasizing its robustness and reliability.
    } 
    \label{fig:sensitivity}
    \vspace{-3ex}
\end{figure}


\subsection{Hyper-parameter Sensitivity Analysis}
\label{sect:hyperparameter}

In this section, we assess the impact of key hyperparameters, namely $\sigma$ and $\alpha$, which govern the magnitude of perturbation within our method. The sensitivity analysis is conducted on two best-performed transferability estimation metrics (\textit{i.e.,} SFDA and NCTI) after applying the proposed SA feature perturbation technique. Fig. \ref{fig:sensitivity} shows the ranking correlation under different $\sigma$ and $\alpha$. The minimal variations in performance across a wide range of hyper-parameters highlight the algorithm's resilience, showcasing its robustness and reliability. We note that we fix the value of one hyper-parameter to tune the other one. From the figure, we can see that a similar performance trend is presented for both metrics. Specifically, the optimal performance is achieved at $\alpha$ = 0.005 and $\sigma$ = 0.6 for metrics. The analysis results demonstrate that the recommended hyper-parameters are versatile. 




\subsection{Time Complexity}
\label{sect:time}

Our feature perturbation strategy improves ranking correlation while maintaining the same level of time complexity, as shown in Fig. \ref{fig:time}. The impact of our feature perturbation method on the time complexity of various transferability estimation metrics has yielded mixed outcomes. While some metrics have experienced a reduction in time complexity, others have seen an increase, reflecting the effect of our feature perturbation method on different evaluation approaches.
Metrics that rely heavily on the dimensionality of features such as $\mathcal{N}$LEEP, benefit from our method's dimensionality reduction, which reduces processing time. However, metrics that are less dependent on feature dimensionality do not experience the same reductions in time complexity. In fact, the introduction of SA perturbation adds computational steps, slightly increasing overall time requirements.


\begin{figure}
    \centering
    \includegraphics[width=0.9\columnwidth]{images/time_separate.pdf}
        \vspace{-2ex}
    \caption{This figure compare the time complexity before and after applying feature perturbation techniques to vanilla fine-tuning.} 
    \label{fig:time}
    \vspace{-1ex}
\end{figure}

\subsection{Performance on Self-supervised Models}
\label{sect:self-supervised}

To evaluate the transferability estimation on the self-supervised models, we construct the pool with ResNet50 \cite{deepresidulalearningforimagerecognition} pretrained on various self-supervised methods, spanning BYOL \cite{byol}, Infomin \cite{infomin}, PCL-v1 \cite{pcl}, PCL-v2 \cite{pcl}, Sela-v2 \cite{sela}, InsDis \cite{indis}, SimCLR-v1 \cite{simclr-v1}, SimCLR-v2 \cite{simclr-v2}, MoCo-v1 \cite{moco-v1}, MoCo-v2 \cite{moco-v2}, DeepCluster-v2 \cite{deepcluster}, and SWAV \cite{swav}. In self-supervised tasks, we develop a baseline method, which leverages LDA to predict class probabilities for each sample and accumulate the probability of the correct class corresponding to samples as the transferability score. Further details can be found in supplementary material.


LDA-based metric achieves impressive results on transferability estimation of self-supervised models as shown in Table. \ref{tab:ss_vanilla-ft} and Table. \ref{tab:ss_lp}. For both vanilla fine-tuning (Table. \ref{tab:ss_vanilla-ft}) and LFT (Table. \ref{tab:ss_lp}), LDA-based metric demonstrates a 12.7\% and 15.06\% higher average weighted Kendall coefficient compared to the best performing SOTA. The superior performance of the LDA-based metric compared to SOTA metrics across all fine-tuning types emphasizes a key insight: previous metrics have been primarily designed with supervised models in mind, overlooking the unique characteristics and requirements of self-supervised models.  To understand why the LDA-based metric can outperform SOTA on self-supervised estimation tasks, we study the difference in feature geometry generated by the self-supervised and supervised models. The t-SNE feature visualization can be found in Fig. \ref{fig:ss_vs_s}. Without semantic supervision, the self-supervised models present a more heterogeneous feature space than that of the supervised models. This finding also indicates that when estimating the transferability of models with diverse, less discriminative feature space, the existing transferability estimation metric could be vulnerable. On the other hand, the capability of the LDA metric to reflect the transferability of less discriminative feature spaces demonstrates a foundation for developing transferability estimation matrics for self-supervised models.



\begin{figure}[t!]
  \centering
  \includegraphics[width=\columnwidth]{images/updated_embedding.pdf}
  \caption{Comparison of CIFAR-10 embedding structures: (a) illustrates the embedding structure derived from a self-supervised model, while (b) depicts the embedding structure from a supervised model. Self-supervised models, learning without explicit label guidance, tend to capture more abstract relationships in the data, leading to embeddings with diverse patterns. In contrast, supervised models emphasize class separation, leading to a similar embedding structure for different models.}
  \label{fig:ss_vs_s}
  \vspace{-2ex}
\end{figure}


\section{Conclusions and Discussions}

The introduction and evaluation of our feature perturbation method represent a significant advancement in the field of transferability estimation. Through a comprehensive analysis, we have observed that our feature perturbation method not only enhances the accuracy of existing transferability metrics across various fine-tuning methods but also introduces a vital aspect of robustness evaluation. This additional layer of analysis provides more precise rankings by assessing the resilience of model embeddings to perturbations, ensuring that the best model is robust and transferable to new, target datasets.
Specifically, our method has shown to significantly improve metrics such as $\mathcal{N}$LEEP, LogMe, GBC, SFDA, and NCTI with varied effects on time complexity, indicating its capacity to optimize computational efficiency in certain scenarios.


We provide insights into the disparities in embedding structures between self-supervised and supervised models, emphasizing the need for carefully tailored transferability estimation metrics for both model types. Our results reveal that an LDA-based metric outperforms SOTA across all fine-tuning variants for self-supervised tasks. This highlights an opportunity for the community to develop more adaptable and accurate transferability estimation metrics. 

\section*{Acknowledgment}

This work was supported in part by the Australian Research Council (FT230100426).

%%%%%%%%% REFERENCES
{\small
\bibliographystyle{ieee_fullname}
\bibliography{egbib}
}

\clearpage  % Ensures supplementary material starts on a new page

\appendix
\section*{Supplementary Material}
% \documentclass[twoside]{article}

% \usepackage{aistats2025}
% If your paper is accepted, change the options for the package
% aistats2025 as follows:
%
%\usepackage[accepted]{aistats2025}
%
% This option will print headings for the title of your paper and
% headings for the authors names, plus a copyright note at the end of
% the first column of the first page.

% If you set papersize explicitly, activate the following three lines:
%\special{papersize = 8.5in, 11in}
%\setlength{\pdfpageheight}{11in}
%\setlength{\pdfpagewidth}{8.5in}

% If you use natbib package, activate the following three lines:
%\usepackage[round]{natbib}
%\renewcommand{\bibname}{References}
%\renewcommand{\bibsection}{\subsubsection*{\bibname}}

% If you use BibTeX in apalike style, activate the following line:
%\bibliographystyle{apalike}

% \begin{document}

% If your paper is accepted and the title of your paper is very long,
% the style will print as headings an error message. Use the following
% command to supply a shorter title of your paper so that it can be
% used as headings.
%
%\runningtitle{I use this title instead because the last one was very long}

% If your paper is accepted and the number of authors is large, the
% style will print as headings an error message. Use the following
% command to supply a shorter version of the authors names so that
% they can be used as headings (for example, use only the surnames)
%
%\runningauthor{Surname 1, Surname 2, Surname 3, ...., Surname n}

% Supplementary material: To improve readability, you must use a single-column format for the supplementary material.
\onecolumn
\appendix
\aistatstitle{From Deep Additive Kernel Learning to Last-Layer \\ Bayesian Neural Networks via Induced Prior Approximation: \\
Supplementary Materials}

\section{SPARSE CHOLESKY DECOMPOSITION}
\label{sec:sparse chol decompose}
In this section, we present the algorithm for constructing the induced grids $\mathbf{U}$ as defined in \cref{eq:GPlayer} by using sorted dyadic points, and obtaining the sparse Choleksy decomposition of the Laplace kernel in one dimension, as proposed in \citep{ding2024sparse}.

A set of one-dimensional level-$L$ dyadic points $\Xv_L$ in increasing order over the interval $[0,1]$ is defined as:
\begin{align}
    \Xv_{L}:= \left\{ \frac{1}{2^{L}}, \frac{2}{2^{L}}, \frac{3}{2^{L}}, \ldots, \frac{2^{L}-1}{2^{L}} \right\}.
\end{align}
However, this increasing order does not yield a sparse representation of the Markov kernel $k(\cdot,\cdot)$ on the points $\Xv_L$, i.e., Cholesky decomposition of the covariance matrix $k(\Xv_L, \Xv_L)$ is not sparse. To achieve a sparse hierarchical expansion, we first sort the dyadic points $\Xv_L$ according to their levels.

\paragraph{Sorted Dyadic Points}
For level-$\ell$ dyadic points $\Xv_{\ell}$ where $ \ell=1,\ldots,L$, we first define the set $\rho(\ell)$ consisting of odd numbers as follows:
\begin{align}
    \rho(\ell) = \left\{ 1,3,5,\ldots,2^{\ell}-1 \right\}.
\end{align}
Next, we define the sorted incremental set $\Dv_{\ell}$ (with $\Xv_{0}:= \varnothing$) as:
\begin{align}
    \Dv_{\ell} = 
    \left\{ \frac{i}{2^{\ell}}: i\in \rho(\ell) \right\} = \Xv_{\ell} - \Xv_{\ell-1}, \quad  \ell=1,\ldots L.
\end{align}
Thus, the level-$L$ dyadic points $\Xv_L$ can be decomposed into disjoint incremental sets $\{ \Dv_{\ell} \}_{\ell=1}^{L}$:
\begin{align}
    \Xv_{L} = \cup_{\ell=1}^{L} \Dv_{\ell}, \quad \Dv_{i} \cap \Dv_{j} = \varnothing \text{ for $i\neq j$}.
\end{align}
Therefore, we can define the sorted level-$L$ dyadic points using these incremental sets as:
\begin{align}\label{eq:sorted dyadic}
    \Xv_{L}^{\text{sort}}:= \left\{ \Dv_1,\Dv_2, \ldots, \Dv_{L} \right\} 
    = \left\{ \frac{i \in \rho(\ell) }{2^{\ell}}, \ell=1,\ldots,L \right\}.
\end{align}
For example, the sorted level-3 dyadic points are given by:
\begin{align}
    \Xv_{3}^{\text{sort}} 
    = \bigg\{ 
    \begingroup
        \color{blue}
        \underbracket{
            \color{black}
            \frac{1}{2^1}
        }_{\color{blue}
            \Dv_1
        }
    \endgroup
    , 
    \begingroup
        \color{blue}
        \underbracket{
            \color{black}
            \frac{1}{2^2}, \frac{3}{2^2}
        }_{\color{blue}
            \Dv_2
        }
    \endgroup
    ,
    \begingroup
        \color{blue}
        \underbracket{
            \color{black}
            \frac{1}{2^3}, \frac{3}{2^3}, \frac{5}{2^3}, \frac{7}{2^3}
        }_{\color{blue}
            \Dv_3
        }
    \endgroup
     \bigg\}.
\end{align}

\paragraph{Algorithm}
We now present the algorithm for computing the inverse of the upper triangular Cholesky factor $[ \Lv_{\Xv_{L}^{\text{sort}}}^{\top} ]^{-1}$ of the covariance matrix $k(\Xv_{L}^{\text{sort}}, \Xv_{L}^{\text{sort}})$ in \Cref{alg:cholesky}, where $\Lv_{\Xv_{L}^{\text{sort}}} \Lv_{\Xv_{L}^{\text{sort}}}^{\top} = k(\Xv_{L}^{\text{sort}}, \Xv_{L}^{\text{sort}})$.. The corresponding proof can be found in \citep{ding2024sparse}. The output of \Cref{alg:cholesky} is a sparse matrix with $\Oc(3 \cdot (2^{L}-1))$ nonzero entries. Since each iteration of the for-loop only requires solving a $3 \times 3$ linear system, which costs $\Oc(3^3)$ time, the total computational complexity of \Cref{alg:cholesky} is $\Oc(2^L-1)$. This implies that the complexity of computing $\left[ \Lv_{\Uv}^{\top} \right]^{-1}$ in \cref{eq:GPlayer} is $\Oc(M)$ when $\Uv$, the induced grid of size $M$, consists of sorted dyadic points as defined in \cref{eq:sorted dyadic}.

\begin{algorithm}[hbt!]
\caption{Computation of the inverse Cholesky factor for the Markov kernel $k(\cdot, \cdot)$ on sorted one-dimensional level-$L$ dyadic points $\Xv_L^{\text{sort}}$.}
\label{alg:cholesky}
\setstretch{0.99} % set the line spacing to 0.99
\begin{algorithmic}[1]
    \STATE {\bfseries Input:} Markov kernel $k(\cdot,\cdot)$, sorted level-$L$ dyadic points $\Xv_{L}^{\text{sort}}$
    \STATE {\bfseries Output:} inverse of the upper triangular Cholesky factor $\Rv:= [ \Lv_{\Xv_{L}^{\text{sort}}}^{\top} ]^{-1}$, s.t. $\Lv_{\Xv_{L}^{\text{sort}}} \Lv_{\Xv_{L}^{\text{sort}}}^{\top} = k(\Xv_{L}^{\text{sort}}, \Xv_{L}^{\text{sort}})$
    \STATE Initialize $\Rv \leftarrow \text{zeros($2^L-1$,$2^L-1$)}$;
    \STATE Define $k(\pm \infty, \cdot) = k(\cdot, \pm \infty) = 0$;
    \FOR{$\ell=1$ {\bfseries to} $L$}
        \FOR{$i \in \rho(\ell)=\{1,3,\ldots,2^{\ell}-1\}$}
            \STATE $x_{\text{mid}} := \frac{i}{2^{\ell}}$;\quad
            $x_{\text{left}}:=\frac{i-1}{2^{\ell}}$ {\bfseries if} $i>1$ {\bfseries else} $-\infty$;\quad
            $x_{\text{right}}:=\frac{i+1}{2^{\ell}}$ {\bfseries if} $i<2^{\ell}-1$ {\bfseries else} $+\infty$;
            \STATE Get $i_{\text{mid}}$, $i_{\text{left}}$, $i_{\text{right}}$, the indices of the points $x_{\text{mid}}$, $x_{\text{left}}$, $x_{\text{right}}$ in the sorted set $\Xv_{L}^{\text{sort}}$ respectively;
            \STATE Get the coefficients $c_1$, $c_2$, $c_3$ by solving the following linear system:
            \begin{align}
                \begin{bmatrix}
                     & k(x_{\text{left}}, x_{\text{left}})
                     & k(x_{\text{left}}, x_{\text{mid}})
                     & k(x_{\text{left}}, x_{\text{right}}) \\
                     & k(x_{\text{mid}}, x_{\text{left}})
                     & k(x_{\text{mid}}, x_{\text{mid}})
                     & k(x_{\text{mid}}, x_{\text{right}}) \\
                     & k(x_{\text{right}}, x_{\text{left}})
                     & k(x_{\text{right}}, x_{\text{mid}})
                    &k(x_{\text{right}}, x_{\text{right}})
                \end{bmatrix}
                \begin{bmatrix}
                    c1\\
                    c2\\
                    c3
                \end{bmatrix}=
                \begin{bmatrix}
                    0\\
                    1\\
                    0
                \end{bmatrix}.
            \end{align}
            \STATE $[c_1,c_2,c_3] := [c_1,c_2,c_3] / \sqrt{c_2}$;
            \STATE {\bfseries if} $x_{\text{left}} \neq - \infty$, 
            {\bfseries then} $\Rv[i_{\text{left}} ,i_{\text{mid}}] = c_1$; \quad
            {\bfseries if} $x_{\text{right}} \neq + \infty$, 
            {\bfseries then} $\Rv[i_{\text{right}} ,i_{\text{mid}}] = c_3$;
            \STATE $\Rv[i_{\text{mid}} ,i_{\text{mid}}] = c_2$;
        \ENDFOR
    \ENDFOR
\end{algorithmic}
\end{algorithm}


\section{REPARAMETERIZATION OF KERNEL LENGTHSCALES}
\label{sec:theo}
Considering the additive Laplace kernel with fixed lengthscale $\tilde{\theta}$ for all base kernels, applying linear projections $\left\{ \wv_{p}^{\top}\xv \right\}_{p=1}^{P}$ on inputs $\xv\in \Rb^D$ will give:
\begin{align}
    &\sum_{p=1}^{P}\sigma^2_p k_p\left( \wv^{\top}_{p}\xv,\wv^{\top}_{p}\xv^{\prime} \right)\nonumber \\
    = & \sum_{p=1}^{P} \sigma^2_p\exp \left( -  \frac{\sum_{d=1}^{D} \left| w_{p,d}\left( x_{d}-x_{d}^{\prime} \right) \right|}{\tilde{\theta}} \right)\nonumber \\
    = & \sum_{p=1}^{P} \prod_{d=1}^{D} \sigma^2_p\exp \left( - \frac{\left| x_{d}-x_{d}^{\prime} \right|}{\tilde{\theta} / \left| w_{p,d}\right| } \right)\nonumber \\
    = & \sum_{p=1}^{P} \prod_{d=1}^{D} \sigma^2_p\exp \left( - \frac{\left| x_{d}-x_{d}^{\prime} \right|}{\theta_{p,d}} \right),
\end{align}
This still leads to an additive Laplace kernel but with adaptive lengthscale $\theta_{p,d}$ for base kernels. The resulting kernel also retains \emph{sparse} Cholesky decomposition by the properties of Markov kernels so that the complexity of inference is $\Oc(M)$.

\section{INFERENCE OF PREDICTIVE DISTRIBUTION}
\label{sec:uq of inference}
Given an input $\xv \in \Rb^D$, the prediction of the DAK model can be written in the following equation according to \cref{eq:DAK prediction}: 
\begin{align}
    \tilde{f}_{\xv}
    &= \sum_{p=1}^{P}
    \sigma_p \Big(
        \phi(h_{\psi}^{[p]}(\xv)) \zv_p
    \Big) + \mu \nonumber\\
    &= \sum_{p=1}^{P}
    \sigma_p \Big(
        \bm{\phi}_{p}^{\top} \zv_p
    \Big) + \mu,
\end{align}
where $\bm{\phi}_{p}^{\top}:=\phi(h_{\psi}^{[p]}(\xv)) \in \Rb^{1 \times M}$
% , $\mu_p:=\mu_p(h_{\psi}^{[p]}(\xv)) \in \Rb$
. We assume the variational distribution over the independent Gaussian weights $\zv_p \sim \Nc(\bm{m}_{\zv_p}, \Sv_{\zv_p})$ and the bias $\mu \sim \Nc(m_{\mu}, \sigma_{\mu}^2)$. Then it's straighforward to deduce that
\begin{align}
    \bm{\phi}_{p}^{\top} \zv_p + \mu 
    &\sim
    \Nc\left(
    \bm{\phi}_{p}^{\top} \bm{m}_{\zv_p} + m_{\mu},\hspace{0.2em}
    \bm{\phi}_{p}^{\top} \Sv_{\zv_p} \bm{\phi}_{p} + \sigma_{\mu}^2
    \right), \\
    \sigma_p \left(
    \bm{\phi}_{p}^{\top} \zv_p 
    \right) + \mu
    & \sim
    \Nc\left(
    \sigma_p ( \bm{\phi}_{p}^{\top} \bm{m}_{\zv_p} )+ m_{\mu} ,\hspace{0.2em}
    \sigma_p^2( \bm{\phi}_{p}^{\top} \Sv_{\zv_p} \bm{\phi}_{p}) + \sigma_{\mu}^2
    \right), \\
    \tilde{f}_{\xv} = 
    \sum_{p=1}^{P}
    \sigma_p \left(
    \bm{\phi}_{p}^{\top} \zv_p
    \right) + \mu
    & \sim
    \Nc\left(
    \sum_{p=1}^{P}
    \sigma_p ( \bm{\phi}_{p}^{\top} \bm{m}_{\zv_p}) + m_{\mu} ,\hspace{0.2em}
    \sum_{p=1}^{P}
    \sigma_p^2( \bm{\phi}_{p}^{\top} \Sv_{\zv_p} \bm{\phi}_{p} ) + \sigma_{\mu}^2
    \right).
\end{align}
Therefore, we obtain the predictive distribution of the $\tilde{f}(\xv)$ at the point $\xv \in \Rb^D$ and its mean and variance are given by:
\begin{subequations}
\label{eq:dak inference closed form}
\begin{align}
    \Eb\left[ \tilde{f}_{\xv} \right]
        = \sum_{p=1}^{P}
        \sigma_p ( \bm{\phi}_{p}^{\top} \bm{m}_{\zv_p}) + m_{\mu},
\end{align}
\begin{align}
    \text{Var}\left[ \tilde{f}_{\xv} \right]
        =\sum_{p=1}^{P}
        \sigma_p^2( \bm{\phi}_{p}^{\top} \Sv_{\zv_p} \bm{\phi}_{p}) + \sigma_{\mu}^2.
\end{align}
\end{subequations}
% \begin{subequations}
% \label{eq:dak inference closed form}
%     \begin{align}
%         \Eb\left[ \tilde{f}(\xv) \right]
%         = \sum_{p=1}^{P}
%         \sigma_p ( \bm{\phi}_{p}^{\top} \bm{m}_{\zv_p} + m_{\mu_p} ),
%     \end{align}
%     \begin{align}
%         \text{Var}\left[ \tilde{f}(\xv) \right]
%         =\sum_{p=1}^{P}
%         \sigma_p^2( \bm{\phi}_{p}^{\top} \Sigma_{\zv_p} \bm{\phi}_{p} + \sigma_{\mu_p}^2).
%     \end{align}
% \end{subequations}


\section{TRAINING OF VARIATIONAL INFERENCE}
\label{sec:training}
Given the dataset $\mathcal{D}=\{ \Xv, \yv \}$ where $\Xv:=\{ \xv_i \}_{i=1}^N$, $\yv=(y_1,\ldots,y_N)^{\top}$, $\xv_i \in \Rb^D$, $y_i\in\Rb$, the prediction $\tilde{f}_{\Xv}\in \Rb^N$ of DAK is given by all the parameters $\bm{\theta}=\left\{ \psi, \bm{\sigma} \right\}$, $\bm{\eta}=\left\{ \{ \mv_{\zv_{p}},\Sv_{\zv_{p}}\}_{p=1}^{P} , \{m_{\mu},\sigma_{\mu} \} \right\}$ according to \cref{eq:DAK prediction}:
\begin{align}
    \tilde{f}_{\Xv}:= \tilde{f}(\Xv; \bm{\theta}, \bm{\eta})
    = \sum_{p=1}^{P}
    \sigma_p \Big(
        \phi(h_{\psi}^{[p]}(\Xv)) \zv_p
    \Big) + \mu,
\end{align}
where $\zv_{p} \sim \mathcal{N} (\bm{m}_{\zv_p} ,\Sv_{\zv_p})$, $p=1,\ldots,P$, and $\mu \sim \mathcal{N} ( m_{\mu},\sigma^2_{\mu} )$ are variational variables $\Theta_{\text{var}}$ parameterized by $\bm{\eta}$. The variational distribution is denoted by $q_{\bm{\eta}}(\Theta_{\text{var}})= q(\mu)\prod_{p=1}^{P} q(\zv_{p}) = \Nc ( m_{\mu} ,\sigma_{\mu}^2 )\prod_{p=1}^{P} 
\Nc ( \bm{m}_{\zv_p} ,\Sv_{\zv_p} )$, and the variational prior is denoted by $p(\Theta_{\text{var}})$.

We consider the KL divergence between $q_{\bm{\eta}}(\Theta_{\text{var}})$ and the true posterior $p(\Theta_{\text{var}}\vert \yv, \Xv, \bm{\theta})$:
\begin{align}
& \qquad \text{KL} \left[ q_{\bm{\eta}}(\Theta_{\text{var}}) \| p(\Theta_{\text{var}} \vert \yv,\Xv, \bm{\theta} ) \right] \nonumber \\
= & \int q_{\bm{\eta}}(\Theta_{\text{var}} )\log \frac{q_{\bm{\eta}}(\Theta_{\text{var}} )}{p(\Theta_{\text{var}} \vert \yv,\Xv,\bm{\theta} )} d\Theta_{\text{var}} \nonumber \\
= & \int q_{\bm{\eta}}(\Theta_{\text{var}} )\log \frac{q_{\bm{\eta}}(\Theta_{\text{var}} )p(\yv \vert \Xv,\bm{\theta})}{p(\yv \vert \Xv,\bm{\theta} ,\Theta_{\text{var}} )p(\Theta_{\text{var}} )} d\Theta_{\text{var}} \nonumber \\
= & \int q_{\bm{\eta}}(\Theta_{\text{var}} )\log \frac{q_{\bm{\eta}}(\Theta_{\text{var}} )}{p(\Theta_{\text{var}} )} d\Theta_{\text{var}} -\int q_{\bm{\eta}}(\Theta_{\text{var}} )\log p(\yv \vert \tilde{f}_{\Xv} )d\Theta_{\text{var}} +\log p(\yv\vert \Xv,\bm{\theta}).
\end{align}
Using the fact that $\text{KL}[\cdot \| \cdot] \geq 0$, we have
\begin{align}
\label{eq:variational lower bound}
    \log p(\yv\vert \Xv,\bm{\theta}) & \geq \int q_{\bm{\eta}}(\Theta_{\text{var}} )\log p(\yv \vert \tilde{f}_{\Xv} )d\Theta_{\text{var}} - \text{KL} \left[ q_{\bm{\eta}}(\Theta_{\text{var}} ) \| p(\Theta_{\text{var}}) \right] \nonumber \\
    & = \Eb_{q_{\bm{\eta}}(\Theta_{\text{var}} )} \left[ \log p(\yv \vert \tilde{f}_{\Xv} ) \right] - \text{KL} \left[ q_{\bm{\eta}}(\Theta_{\text{var}} ) \| p(\Theta_{\text{var}}) \right].
\end{align}

\paragraph{Full-training.}
Firstly, we present the joint training of $\bm{\theta}$ and $\bm{\eta}$. The most common approach optimizes the marginal log-likelihood (the left-hand side of \cref{eq:variational lower bound}):
\begin{align}
    \bm{\theta}^{\ast} &=\argmax_{\bm{\theta}} \log p(\yv\vert \Xv,\bm{\theta} ) \\
    &= \argmax_{\bm{\theta}} \log \int p\left( y\vert X,\bm{\theta},\Theta_{\text{var}} \right) p(\Theta_{\text{var}})d\Theta_{\text{var}},
\end{align}
which involves intractable integral in some tasks such as classification. Instead, we optimize the variational lower bound (the right-hand side of \cref{eq:variational lower bound}):
\begin{align}
    \Theta^{\ast} := \argmax_{\bm{\theta},\bm{\eta}} \mathcal{L}(\bm{\theta},\bm{\eta}) =\argmax_{\bm{\theta},\bm{\eta}}\left\{ E_{q_{\bm{\eta}}(\Theta_{\text{var}} )}\left[ \log p(\yv|\tilde{f}_{\Xv} ) \right] -\text{KL} \left[ q_{\bm{\eta}}(\Theta_{\text{var}} )\| p(\Theta_{\text{var}} ) \right] \right\}.
\end{align}

\paragraph{Fine-tuning.}
An alternative training approach is to firstly pre-train the deterministic parameters of feature extractor by standard neural network training, with mean squared error for regression or cross-entropy for classification as the loss function, and then fine-tune the last layer additive GP with fixed features. The objective function is identical to \cref{eq:elbo}, but $\bm{\theta}$ is learned during the pre-training step and is no longer optimized during fine-tuning.


\section{ELBO}%{DERIVATION OF ELBO}
\label{sec:elbo}
\subsection{Assumptions}
Consider the model $y_i = \tilde{f}(\xv_i) + \epsilon_i$ with the i.i.d. noise $\epsilon_i \overset{\text{i.i.d.}}{\sim} \Nc(0, \sigma_{f}^2)$ and $\tilde{f} : \Rb^D \rightarrow \Rb$ is defined in \cref{eq:DAK prediction}. The training dataset is $\mathcal{D} = \{ \Xv, \yv \}$ where $\Xv:=\{ \xv_i \}_{i=1}^N$, $\yv=(y_1,\ldots,y_N)^{\top}$, $\xv_i \in \Rb^D$, $y_i\in\Rb$. $\Theta_{\text{var}}:= \{ \mu ,\{ \zv_{p}\}_{p=1}^{P} \}$ are the variational random variables consisting of Gaussian weights and bias of $P$ units, $\psi$ are the parameters of the NN, $\bm{\sigma}:=(\sigma_1, \ldots, \sigma_p)^{\top}$ are the scale parameters of base GP layers. The variational distributions are $q(\mu)=\Nc(m_{\mu}, \sigma_{\mu}^2)$, $q(\zv_p)=\Nc(\bm{m}_{\zv_p}, \Sv_{\zv_p})$ and the variational priors are $p(\mu)=\Nc(\check{m}_{\mu} ,\check{\sigma}^2_{\mu})$, $p(\zv_p)=\Nc(\check{\bm{m}}_{\zv_p} ,\check{\Sv}_{\zv_p})$. Note that $\Sv_{\zv_p}\in\Rb^{M \times M}$ is a diagonal covariance matrix due to the independence of $\zv_p$, $M$ is the number of inducing points $\Uv$ defined in \cref{eq:GPlayer}, and $\bm{m}_{\zv_p} \in \Rb^M$, $m_{\mu} \in \Rb$, $\sigma_{\mu}^2 \in \Rb$. We derive the ELBO in VI to learn the preditive posterior over the variational variables $\Theta_{\text{var}}:= \{ \mu ,\{ \zv_{p}\}_{p=1}^{P} \}$ parameterized by $\bm{\eta}:=\left\{ \{ \mv_{\zv_{p}},\Sv_{\zv_{p}}\}_{p=1}^{P} , \{m_{\mu},\sigma_{\mu} \} \right\}$, and optimize the deterministic parameters $\bm{\theta}:=\{\psi, \bm{\sigma}\}$.

\subsection{Expected Log Likelihood}
\paragraph{Closed Form}
The \emph{expected log likelihood}, which is the first term in ELBO defined in \cref{eq:elbo}, is given by 
\begin{align}
    {\Eb}_{q_{\bm{\eta}}(\Theta_{\text{var}})} \left[ \log \text{Pr} (\yv \vert \tilde{f}_{\Xv} ) \right]
    &= {\Eb}_{q_{\bm{\eta}}(\Theta_{\text{var}})} \left[ 
    \log \prod_{i=1}^{N} 
    p (y_i \vert \tilde{f}_{\xv_i} )
    \right] \nonumber\\
    &= \sum_{i=1}^{N} 
    {\Eb}_{q_{\bm{\eta}}(\Theta_{\text{var}})} \left[ 
    \log
    p (y_i \vert \tilde{f}_{\xv_i} )
    \right] \nonumber\\
    &= \sum_{i=1}^{N} 
    {\Eb}_{q_{\bm{\eta}}(\Theta_{\text{var}})} \left[ 
    \log
    \Nc( \tilde{f}_i,\hspace{0.2em} \sigma_{f}^2 )
    \right] \nonumber\\
    &= \sum_{i=1}^{N} 
    {\Eb}_{q_{\bm{\eta}}(\Theta_{\text{var}})} \left[ 
    \log \left(
    (2\pi \sigma_{f}^2)^{-\frac{1}{2}}
    \exp\left\{  
        -\frac{ (y_i - \tilde{f}_i)^2 }{2 \sigma_{f}^2}
    \right\}
    \right)
    \right] \nonumber\\
    &= \sum_{i=1}^{N} 
    {\Eb}_{q_{\bm{\eta}}(\Theta_{\text{var}})} \left[
    -\frac{1}{2} \log(2\pi) 
    - \frac{1}{2}\log(\sigma_{f}^2)
    - \frac{1}{2 \sigma_{f}^2}
    (y_i - \tilde{f}_i)^2
    \right] \nonumber\\
    &= - \frac{N}{2} \log(2\pi)
    - \frac{N}{2} \log(\sigma_{f}^2)
    - \frac{1}{2 \sigma_{f}^2}
    \sum_{i=1}^{N}
    {\Eb}_{q_{\bm{\eta}}(\Theta_{\text{var}})} \left[
    (y_i - \tilde{f}_i)^2
    \right] \nonumber\\
    &= - \frac{N}{2} \log(2\pi)
    - \frac{N}{2} \log(\sigma_{f}^2)
    - \frac{1}{2 \sigma_{f}^2}
    \sum_{i=1}^{N} \left(
    \left({\Eb}_{q(\Theta_{\text{var}})} \left[
    (y_i - \tilde{f}_i)
    \right] \right)^2
    + \text{Var}_{q(\Theta_{\text{var}})} \left[
    (y_i - \tilde{f}_i)
    \right]
    \right) \label{eq:evidence halfway},
\end{align}
where
\begin{align}
    \tilde{f}_i
    % \mu_{\tilde{f}_i} &:= \tilde{f}(\xv_i;\Theta_{\text{var}}, \Theta_{\text{det}} ) \nonumber\\
    &= \sum_{p=1}^{P} \sigma_p \Big(
    \begingroup
        \color{blue}
        \underbracket{
            \color{black}
            \phi(h_{\psi}^{[p]}(\xv_i))
        }_{\color{blue}
            :=\bm{\phi}_{i,p}^{\top} \in \Rb^{1 \times M}
        }
    \endgroup
    \zv_p
    \Big)
    + \mu
    % \begingroup
    %     \color{blue}
    %     \underbracket{
    %         \color{black}
    %         \mu_{p}(h_{\psi}^{[p]}(\xv_i))
    %     }_{\color{blue}
    %         :=\mu_{i,p} \in \Rb
    %     }
    % \endgroup 
    \nonumber\\
    &= \sum_{p=1}^{P} \sigma_p \left(
    \bm{\phi}_{i,p}^{\top} \zv_p 
    \right) + \mu.
\end{align}
Recall that the variational assumptions $q(\zv_p)=\Nc(\bm{m}_{\zv_p}, \Sv_{\zv_p})$ and $q(\mu)=\Nc(m_{\mu}, \sigma_{\mu}^2)$, we can infer that
\begin{align}
    \bm{\phi}_{i,p}^{\top} \zv_p + \mu 
    &\sim
    \Nc\left(
    \bm{\phi}_{i,p}^{\top} \bm{m}_{\zv_p} + m_{\mu},\hspace{0.2em}
    \bm{\phi}_{i,p}^{\top} \Sv_{\zv_p} \bm{\phi}_{i,p} + \sigma_{\mu}^2
    \right), \\
    \sigma_p \left(
    \bm{\phi}_{i,p}^{\top} \zv_p 
    \right) + \mu
    & \sim
    \Nc\left(
    \sigma_p ( \bm{\phi}_{i,p}^{\top} \bm{m}_{\zv_p} ) + m_{\mu},\hspace{0.2em}
    \sigma_p^2( \bm{\phi}_{i,p}^{\top} \Sv_{\zv_p} \bm{\phi}_{i,p} ) + \sigma_{\mu}^2
    \right), \\
    \tilde{f}_i = 
    \sum_{p=1}^{P}
    \sigma_p \left(
    \bm{\phi}_{i,p}^{\top} \zv_p 
    \right)+ \mu
    & \sim
    \Nc\left(
    \sum_{p=1}^{P}
    \sigma_p ( \bm{\phi}_{i,p}^{\top} \bm{m}_{\zv_p} )+ m_{\mu},\hspace{0.2em}
    \sum_{p=1}^{P}
    \sigma_p^2( \bm{\phi}_{i,p}^{\top} \Sv_{\zv_p} \bm{\phi}_{i,p} ) + \sigma_{\mu}^2
    \right), \\
    y_i - \tilde{f}_i
    & \sim 
    \Nc\left(
    y_i - 
    \sum_{p=1}^{P}
    \sigma_p ( \bm{\phi}_{i,p}^{\top} \bm{m}_{\zv_p} ) -m_{\mu},\hspace{0.2em}
    \sum_{p=1}^{P}
    \sigma_p^2( \bm{\phi}_{i,p}^{\top} \Sv_{\zv_p} \bm{\phi}_{i,p} ) + \sigma_{\mu}^2
    \right).
\end{align}
Therefore, 
\begin{subequations}\label{eq:exp and var in evidence}
    \begin{align}
        \left({\Eb}_{q(\Theta_{\text{var}})} \left[
        (y_i - \tilde{f}_i)
        \right] \right)^2
        = \left(
         y_i - 
        \sum_{p=1}^{P}
        \sigma_p ( \bm{\phi}_{i,p}^{\top} \bm{m}_{\zv_p} ) -m_{\mu}
        \right)^2,
    \end{align}
    \begin{align}
        \text{Var}_{q(\Theta_{\text{var}})}
        \left[
        (y_i - \tilde{f}_i)
        \right]
        = \sum_{p=1}^{P}
        \sigma_p^2( \bm{\phi}_{i,p}^{\top} \Sv_{\zv_p} \bm{\phi}_{i,p} ) + \sigma_{\mu}^2.
    \end{align}
\end{subequations}
By applying \cref{eq:exp and var in evidence} to \cref{eq:evidence halfway}, we derive the analytical formula for the expected evidence, expressed as
\begin{align}
    {\Eb}_{q_{\bm{\eta}}(\Theta_{\text{var}})} \left[ \log \text{Pr} (\yv \vert \tilde{f}_{\Xv} ) \right]
    &= - \frac{N}{2} \log(2\pi)
    - \frac{N}{2} \log(\sigma_{f}^2) \nonumber\\
    &- \frac{1}{2 \sigma_{f}^2}
    \sum_{i=1}^{N} \left(
        \Big(
         y_i - 
        \sum_{p=1}^{P}
        \sigma_p ( \bm{\phi}_{i,p}^{\top} \bm{m}_{\zv_p} ) -m_{\mu}
        \Big)^2
        + \sum_{p=1}^{P}
        \sigma_p^2( \bm{\phi}_{i,p}^{\top} \Sv_{\zv_p} \bm{\phi}_{i,p} )+ \sigma_{\mu}^2
    \right). \label{eq:evidence final}
\end{align}

\paragraph{Monte Carlo Approximation}
For comparison, we provide the equation for computing the Monte Carlo estimate of the ELBO in the paragraph that follows.
\begin{align}
    {\Eb}_{q_{\bm{\eta}}(\Theta_{\text{var}})} \left[ \log \text{Pr} (\yv \vert \tilde{f}_{\Xv} ) \right]
    % &= {\Eb}_{q(\Theta)} \left[ 
    % \log \prod_{i=1}^{N} 
    % p (y_i \vert \xv_i,\Theta, \psi, \bm{\sigma})
    % \right] \nonumber\\
    &= \sum_{i=1}^{N} 
    {\Eb}_{q_{\bm{\eta}}(\Theta_{\text{var}} )} \left[ 
    \log
    p (y_i \vert \tilde{f}_{\xv_i} )
    \right] \nonumber\\
    & \approx \sum_{i=1}^{N}
    \frac{1}{S}
     \sum_{s=1}^{S}
    \log
    p (y_i \vert \xv_i,\tilde{\Theta}^{(s)}_{\text{var}}, \bm{\theta} ) \nonumber\\
    &= \frac{1}{S} \sum_{i=1}^{N} 
    \sum_{s=1}^{S} 
    \log
    \Nc(y_i \left\vert\right. \tilde{f}_{i}^{(s)},\hspace{0.2em} \sigma_{f}^2 )
    \nonumber\\
    &= \frac{1}{S} \sum_{i=1}^{N} 
    \sum_{s=1}^{S} 
    \log \left(
    (2\pi \sigma_{f}^2)^{-\frac{1}{2}}
    \exp\left\{  
        -\frac{ (y_i - \tilde{f}_{i}^{(s)})^2 }{2 \sigma_{f}^2}
    \right\}
    \right)
    \nonumber\\
    &= \frac{1}{S} \sum_{i=1}^{N} 
    \sum_{s=1}^{S} \left(
    -\frac{1}{2} \log(2\pi) 
    - \frac{1}{2}\log(\sigma_{f}^2)
    - \frac{1}{2 \sigma_{f}^2}
    (y_i - \tilde{f}_{i}^{(s)})^2
    \right) \nonumber\\
    &= - \frac{N}{2} \log(2\pi)
    - \frac{N}{2} \log(\sigma_{f}^2)
    - \frac{1}{2 \sigma_{f}^2}
    \sum_{i=1}^{N}
    \frac{1}{S} \sum_{s=1}^{S}
    (y_i - \tilde{f}_{i}^{(s)})^2, \label{eq:evidence halfway mc approx}
\end{align}
where $S$ is the number of Monte Carlo samples, $\{  \tilde{\mu}^{(s)} ,\{ \tilde{\zv}_{p}^{(s)} \}_{p=1}^{P} \} := \tilde{\Theta}^{(s)}_{\text{var}}$ are the $s$-th Monte Carlo samplings over the variational parameters $\Theta_{\text{var}}$ and $\tilde{\Theta}^{(s)}_{\text{var}} \sim q_{\bm{\eta}}(\Theta_{\text{var}})$, $\tilde{f}_{i}^{(s)}$ is given as follows:
\begin{align}
    \tilde{f}_{i}^{(s)} &:= \tilde{f}(\xv_i;\tilde{\Theta}^{(s)}_{\text{var}},\bm{\theta} ) \nonumber\\
    &= \sum_{p=1}^{P} \sigma_p \Big(
    \begingroup
        \color{blue}
        \underbracket{
            \color{black}
            \phi(h_{\psi}^{[p]}(\xv_i))
        }_{\color{blue}
            :=\bm{\phi}_{i,p}^{\top} \in \Rb^{1 \times M}
        }
    \endgroup
    \tilde{\zv}_p^{(s)} 
    \Big) + \tilde{\mu}^{(s)} \nonumber\\
    &= \sum_{p=1}^{P} \sigma_p \left(
    \bm{\phi}_{i,p}^{\top} \tilde{\zv}_p^{(s)} 
    \right)+ \tilde{\mu}^{(s)}. \label{eq:mc approx mean}
\end{align}
Therefore, we plug \cref{eq:mc approx mean} into \cref{eq:evidence halfway mc approx} and get the the Monte Carlo estimate of the ELBO written in the following formula:
\begin{align}
    {\Eb}_{q_{\bm{\eta}}(\Theta_{\text{var}})} \left[ \log \text{Pr} (\yv \vert \tilde{f}_{\Xv} ) \right]
    &\approx
    - \frac{N}{2} \log(2\pi)
    - \frac{N}{2} \log(\sigma_{f}^2)
    - \frac{1}{2 \sigma_{f}^2}
    \sum_{i=1}^{N}
    \frac{1}{S} \sum_{s=1}^{S}
    \Big(y_i - 
    \sum_{p=1}^{P} \sigma_p \left(
    \bm{\phi}_{i,p}^{\top} \tilde{\zv}_p^{(s)} 
    \Big)- \tilde{\mu}^{(s)}
    \right)^2, \label{eq:evidence final mc approx} \\
    \tilde{\zv}_p^{(s)} &\sim \Nc(\bm{m}_{\zv_p}, \Sv_{\zv_p}),\qquad
    \tilde{\mu}^{(s)} \sim \Nc(m_{\mu}, \sigma_{\mu}^2).
\end{align}


\subsection{KL Divergence}
Since we place Gaussian assumptions over the variational parameters $\Theta_{\text{var}}$,  the \emph{KL divergence}, which is the second term in ELBO defined in \cref{eq:elbo}, is then given by
\begin{align}
    \text{KL} \left[ q(\Theta_{\text{var}} ) \| p(\Theta_{\text{var}}) \right]
    &= \text{KL} \left[ q( \mu ,\{ \zv_{p}\}_{p=1}^{P} ) \Vert p( \mu ,\{ \zv_{p}\}_{p=1}^{P}) \right] \nonumber\\
    & =  
    \text{KL} \left[ q(\mu) \Vert p(\mu) \right] 
    + \sum_{p=1}^{P} 
    \text{KL} \left[ q(\zv_{p}) \Vert p(\zv_{p}) \right],
\end{align}

\begin{align}
     \text{KL} \left[ q(\mu) \Vert p(\mu) \right]
     = \frac{1}{2} \left(
     \frac{\sigma_{\mu}^2}{\check{\sigma}_{\mu}^2} 
     + \frac{(m_{\mu} - \check{m}_{\mu})^2}{\check{\sigma}_{\mu}^2} 
     -\log\left( \frac{\sigma_{\mu}^2}{\check{\sigma}_{\mu}^2} \right)
     -1
     \right),
\end{align}

\begin{align}
    \text{KL} \left[ q(\zv_{p}) \Vert p(\zv_{p}) \right]
    = \frac{1}{2} \sum_{i=1}^{M} \left(
     \frac{[\Sv_{\zv_p}]_{ii}}{[\check{\Sv}_{\zv_p}]_{ii}} 
     + \frac{([\bm{m}_{\zv_p}]_{i} - [\check{\bm{m}}_{\zv_p}]_i)^2}{[\check{\Sv}_{\zv_p}]_{ii}}
     -\log\left( 
     \frac{[\Sv_{\zv_p}]_{ii}}{[\check{\Sv}_{\zv_p}]_{ii}}  
     \right)
     -1
     \right),
\end{align}
where $[\Sv_{\zv_p}]_{ii}$ is the $(i,i)$-th element of the diagonal covariance matrix $\Sv_{\zv_p} \in \Rb^{M \times M}$, $[\bm{m}_{\zv_p}]_{i}$ is the $i$-th element of the mean vector $\bm{m}_{\zv_p} \in \Rb^M$, the approximated posteriors are $q(\mu)=\Nc(m_{\mu}, \sigma_{\mu}^2)$, $q(\zv_p)=\Nc(\bm{m}_{\zv_p}, \Sv_{\zv_p})$ and the priors are $p(\mu)=\Nc(\check{m}_{\mu} ,\check{\sigma}^2_{\mu})$, $p(\zv_p)=\Nc(\check{\bm{m}}_{\zv_p} ,\check{\Sv}_{\zv_p})$.

% \subsection{Performance Comparison}
% \label{sec:toy exp compare}
% We compare the perforamce of computing the ELBO in \cref{eq:elbo} by using closed form in \cref{eq:evidence final} and using Monte Carlo approximation in \cref{eq:evidence final mc approx} in a toy example.
% \textcolor{red}{Table or Figure to add if time available}


\subsection{Limitations of the Closed-Form ELBO}

The closed-form ELBO is only applicable to regression problems. In classification, applying the softmax function to $\tilde{f}(\xv;\bm{\theta}, \bm{\eta})$ results in a non-analytic predictive distribution, meaning the ELBO must still be computed via Monte Carlo sampling during training. Similarly, the closed-form expressions for the predictive mean and variance, as provided in \cref{eq:dak inference closed form} in \Cref{sec:uq of inference}, are not applicable to classification but only apply to regression problems.


\section{COMPUTATIONAL COMPLEXITY}
\label{sec:complexity}
In this section, we discuss the computational complexity of various DKL models compared to the proposed DAK method, focusing on the GP layer as the most computationally demanding component. \Cref{tab:complexity supp} shows the computational complexity of our model compared to other state-of-the-art GP and DKL methods.

\begin{table}[ht]
    \caption{Computational complexity of the DKL models for $N$ training points. The reported training complexity is for one iteration. $\hat{M}$ is the number of inducing points in SVGP and KISS-GP, while $M$ is the size of induced grids in DAK, $M < \hat{M}$. $S$ is the number of Monte Carlo samples, $B$ is the size of mini-batch, $D_w$ is the dimension of the NN outputs in DKL, $P$ is the dimension of the outputs after applying linear transformations to the NN outputs in the proposed DAK model. DAK-MC refers to the DAK model using Monte Carlo approximation, while DAK-CF refers to the DAK model using closed-form inference and ELBO.}
    \centering
    \begin{tabular}{lcc}
    \toprule[1pt]
                  & \textbf{Inference}       & \textbf{Training} (per iteration) \\
    \midrule[0.5pt]
    NN + SVGP     & $\Oc(\hat{M}^2 N)$    & $\Oc( S D_w MB + \hat{M}^3)$ \\
    NN + KISS-GP  & $\Oc(D_w \hat{M}^{1+\frac{1}{D_w}})$  & $\Oc(S D_w MB + D_w \hat{M}^{\frac{3}{D_w}})$ \\
    DAK-MC (ours) & $\Oc(SM)$       & $\Oc(SPMB + PM)$   \\
    DAK-CF (ours) & $\Oc(M)$        & $\Oc(PMB + PM)$    \\
    \bottomrule[1pt]
    \end{tabular}
    \label{tab:complexity supp}
\end{table}

\paragraph{Inference Complexity.}
In inference based on induced approximation, computing the multiplication of the inverse of the covariance matrix $k(\Uv, \Uv)$ and a vector takes $\Oc(\hat{M}^2N)$ time for $\hat{M}$ inducing points $\Uv$ and $N$ training points when using SVGP. This cost is reduced by KISS-GP to $\Oc(D \hat{M}^{1+\frac{1}{D}})$ by decomposing the covariance matrix into a Kronecker product of $D$ one-dimensional covariance matrices of the inducing points: $k(\Uv, \Uv) = \bigotimes_{d=1}^{D} k(\Uv^{[d]}, \Uv^{[d]})$. Despite the significant reduction on complexity, it requires inducing points $\Uv$ arranged on a Cartesian grid of size $\hat{M} = \prod_{d=1}^{D} \hat{M}_d$, where $\hat{M}_d$ is the number of inducing points in the $d$-th dimension. In high-dimensional spaces, fixing $\hat{M}$ leads to very small $\hat{M}_d$ per dimension, which can degrade model performance. To address this, we propose the DAK model via sparse finite-rank approximation, which employs an additive Laplace kernel for GPs. The inverse Cholesky factor $\Lv_{\Uv}^{\top}$ for one-dimensional induced grids $\Uv$ of size $M$, where $M < \hat{M}$, as defined in \cref{eq:GPlayer}, is sparse and can be computed in $\Oc(M)$ time.

\paragraph{Training Complexity.}
In training, VI requires computing the ELBO as described in \cref{eq:elbo}, which consists of two terms: the \emph{expected log likelihood} and the \emph{KL divergence} between the variational distributions and priors. 

1) The \emph{expected log likelihood} is usually approximated via Monte Carlo sampling at a cost of $\Oc(S N_{\Theta} N)$, where $S$ is the number of Monte Carlo samples, $N_{\Theta}$ is the total number of variational parameters $\Theta_{\text{var}}$, and $N$ is the number of training points. This complexity can be reduced to $\Oc(S N_{\Theta} B)$ by applying stochastic variational inference with a mini-batch of size $B \ll N$. For DKL models using SVGP and KISS-GP, $\Theta_{\text{var}}$ are inducing variables, and the expectation does not have a closed form, requiring Monte Carlo sampling. In contrast, in the proposed DAK model, $\Theta_{\text{var}}= \{ \{ \zv_{p}\}_{p=1}^{P}, \mu \}$ consists of independent Gaussian weights $\zv_p\in \Rb^M$ and bias $\mu$. This allows us to derive an analytical form for this term, as shown in \cref{eq:evidence final} in \Cref{sec:elbo}, reducing the computational cost to $\Oc(N_{\Theta} B) = \Oc(PM B)$ when using a mini-batch of size $B$.

2) The \emph{KL divergence} between two Gaussian distributions can be computed in closed form. This leads to a linear time complexity of $\Oc(N_{\Theta})$ if the parameters $\Theta_{\text{var}}$ are independent, or cubic time $\Oc(N_{\Theta}^3)$ if they are fully correlated. In SVGP and KISS-GP, $\Theta_{\text{var}}$ represents fully correlated Gaussian distributed inducing variables, so computing the KL divergence takes $\Oc(\hat{M}^3)$ for SVGP. In KISS-GP, this can be reduced to $\Oc(D \hat{M}^{\frac{3}{D}})$ using fast eigendecomposition of Kronecker matrices. In the DAK model, the weights $\{\zv_p\}_{p=1}^{P}$ as defined in \cref{eq:GPlayer} are independent Gaussian random variables, allowing the KL divergence to be computed in $\Oc(N_{\Theta}) = \Oc(PM)$ time, where $P$ is the number of base GP layers.


\section{ADDITIONAL DISCUSSIONS}

Although interpretability is one advantage of additive models, the main motivation for replacing a GP layer with an additive GP layer in our work is to handle high-dimensional data. When the input dimension is low, it is reasonable that GPs are superior to additive GPs since the additive kernel is an approximated and restrictive kernel. However, when the input dimension increases, the computational complexity grows considerably even in GPs with sparse approximation. For example, in DKL, the output dimension of NN encoder is usually chosen as small as 2, while in pixel data experiments, DKL cannot handle the computation associated with the dimensionality when the output dimension of ResNet is 512 or more. Although DKL is superior in low-dimensional and simple cases, we view additive structure as a necessary component to achieve scalability and good performance with high-dimensional data.

\subsection{Why choosing the induced grids instead of learning the inducing points?}

From an approximation accuracy point of view, there are two separate strategies to increase the accuracy. The first one is to learn the inducing point locations. The second one, however, is to simply increase the number of inducing points on a pre-specified finer grid. The second method is much easier to implement and has a theoretical guarantee by the GP regression theory: as the inducing points become dense in the input region, the approximation will become exact. In contrast, the first approach does not have such a favorable theoretical guarantee. 

The second approach would become difficult to use for many existing methodologies as in general the computational cost would scale as $\mathcal{O}(M^3)$ with $M$ inducing points, which is particularly problematic in high dimensions. 
% The first approach can be viewed as a compromise in those situations, and that is why many existing methods chose to learn the locations of the inducing points instead.
This difficulty is resolved by additive GPs, since approximating an additive GP boils down to approximating one dimensional GPs, which can be accomplished by using a set of pre-specified inducing points on a fine grid in 1-D. One major benefit of the proposed methodology is that the computation now scales at $\mathcal{O}(M)$, enabled by the Markov kernel and the additive kernel. Therefore, a large number of inducing points can be used in an efficient way. 

The proposed method also has several additional benefits: 1) It can decouple to some extent the neural network component and GP component by avoiding learning the inducing points, which may help reduce overfitting/overconfidence; 2) The equivalence to BNN holds exactly with the fixed inducing points, whereas for learned inducing points, this BNN equivalence breaks down, and the proposed computation/training framework would not be possible to carry through; 3) It can simplify the overall optimization since there is no need to learn the inducing points.

\subsection{Limitations and future directions}

Generally, a finer grid will lead to better approximations, but the number of parameters to be trained will also increase. Therefore, there is a trade-off between the accuracy and the computational cost that we can afford. This current work is using a specific Laplace kernel, which can utilize sparse Cholesky decomposition. More general kernels may result in more computational complexity but better representation power of the model. In addition, the current variational family is restricted under mean-field assumptions. A more general variational family, e.g. full/low-rank covariance, may lead to superior performance in some applications. 


\section{EXPERIMENTAL DETAILS}
\label{sec:expdetail}
In this section, we provide additional details regarding the experiments.

\subsection{Benchmarks for Regression}
\label{subsec:regression supp}
\paragraph{Experiment Setup}
For all models, the NN architecture is a fully connected NN with rectified linear unit (ReLU) activation function \citep{nair2010rectified} and two hidden layers containing 64 and 32 neurons, respectively, structured as $D \rightarrow 64 \rightarrow 32 \rightarrow D_w$, where $D$ is the input feature size (also the size of input $\Xv$) and $D_w$ is the output feature size. The models are evaluated with $D_w=16$, 64, and 256, respectively. The number of Monte Carlo samples is set to 8 during training and 20 during inference.

The NN is a deterministic model, and we use the negative Gaussian log-likelihood as the loss function to quantify the uncertainty of the NN outputs and compute the NLPD.

For NN+SVGP, the inducing points are set to the size of 64 in $D_w$ dimension. We implement the \texttt{ApproximateGP} model in GPyTorch \citep{gardner2018gpytorch}, defining the inducing variables as variational parameters, and use \texttt{VariationalELBO} in GPyTorch to perform variational inference and compute the loss.

SV-DKL is originally designed for classification, so for a fair comparison in regression tasks, we modify it by first applying a linear embedding layer $\Wv: \Rb^{D_w} \rightarrow \Rb^P$ with $P=16$ and normalizing the outputs to the interval $[0,1]$ for each base GP, similar to the DAK model. To adapt the additive GP layer for regression, we remove the softmax function from the model in eq. (1) of \citep{wilson2016stochastic}. Given training data $\{ \xv_i, \yv_i \}_{i=1}^{N}$, the model is modified as follows:
\begin{align}
    p(\yv_i \vert \fv_i, A) = \mathcal{A}(\fv_i)^{\top} \yv_i
\end{align}
where $\fv_i \in \Rb^P$ is a vector of independent GPs followed by a linear mixing layer $\mathcal{A}(\fv_i) = A \fv_i$, with $A \in \Rb^{C \times P}$ as the transformation matrix. Here, $C=1$ for single-task regression. For each $p$-th GP ($1 \leq p \leq P$) in the additive GP layer, the corresponding inducing variables $\uv_p$ are set to the size of 64 and treated as variational parameters for training. We use the \texttt{GridInterpolationVariationalStrategy} model with \texttt{LMCVariationalStrategy} in GPyTorch to perform KISS-GP with variational inducing variables, augmented by a linear mixing layer.

For AV-DKL, the inducing points are set to size of 64 in $D_{w}$ dimension. We implement the AV-DKL model based on the source code~\cite{matias2024amortized}.

Both DAK-MC and DAK-CF use the same additive GP layer size as SV-DKL, with $P=16$, and employ fixed induced grids $\Uv = \{1/8, 2/8, \ldots, 7/8\}$ of size 7 for each base GP, which is much smaller than that of SV-DKL.

\paragraph{Metrics}
Let $\{\xv_t, y_t\}_{t=1}^{T}$ represent a test dataset of size $T$, where $\mu_t$ and $\sigma_t^2$ are the predictive mean and variance. We evaluate model performance using two common metrics: Root Mean Squared Error (RMSE) and Negative Log Predictive Density (NLPD).

RMSE is widely used to assess the accuracy of predictions, measuring how far predictions deviate from the true target values. It is calculated as:
\begin{align}
    \text{RMSE} = \sqrt{ \frac{1}{T} \sum_{t=1}^{T}(y_t - \mu_t)^2 }.
\end{align}

NLPD is a standard probabilistic metric for evaluating the quality of a model's uncertainty quantification. It represents the negative log likelihood of the test data given the predictive distribution. For GPs, NLPD is calculated as:
\begin{align}
    \text{NLPD}
    &= - \sum_{t=1}^{T} \log p(y_t = \mu_t \vert \xv_t) \\
    &= \frac{1}{T}
    \sum_{t=1}^{T} \Big[
    \frac{(y_t - \mu_t)^2}{2\sigma_t^2} + \frac{1}{2} \log(2\pi \sigma_t^2)
    \Big].
\end{align}
Both RMSE and NLPD are widely used in the GP regression literature, where smaller values indicate better model performance.

\paragraph{Computing Infrastructure}
The experiments for regression were run on Macbook Pro M1 with 8 cores and 16GB RAM.

\subsection{Benchmarks for Classification}
\label{subsec:classification supp}
We use PyTorch \citep{paszke2019pytorch} baseline of NN models, GPyTorch \citep{gardner2018gpytorch} baseline of SVGP and SV-DKL models. In classification tasks, we apply a softmax likelihood to normalize the output digits to probability distributions. The NN is a deterministic model trained via negative log-likelihood loss, while DKL and DAK models are trained via ELBO loss. The setting of all training tasks are described in \Cref{tab:model classification} and \Cref{tab:optimizer classification}.

SVGP is originally designed for single-output regression. To make it fit for multi-output classification, we used \texttt{IndependentMultitaskVariationalStrategy} in GPyTorch to implement the multi-task \texttt{ApproximateGP} model, and use \texttt{VariationalELBO} with \texttt{SoftmaxLikelihood} in GPyTorch to perform variational inference and compute the loss. 

For SV-DKL, we employed the same \texttt{VariationalELBO} with \texttt{SoftmaxLikelihood} as the variational loss objective. \texttt{GridInterpolationVariationalStrategy} is applied within \texttt{IndependentMultitaskVariationalStrategy} to perform additive KISS-GP approximation. For each KISS-GP unit, we used $64$ variational inducing points initialized on a grid of size $[-1,1]$. 

For DAK, we implemented DAK-MC using Monte Carlo estimation given the intractable softmax likelihood. We employed fixed induced grids $\Uv=\{ -31/32, -30/32, \ldots, 30/32, 31/32 \}$ of size 63 for each base GP component.

\begin{table}[ht]
\caption{Model architectures for image classification on MNIST, CIFAR-10 and CIFAR-100.}
\centering
\resizebox{0.7\linewidth}{!}{
\begin{tabular}{l|l|ccc}
\toprule[1pt]
Model                   & Hyper-parameter          & MNIST       & CIFAR-10    & CIFAR-100   \\
\midrule[0.5pt]
\multirow{4}{*}{NN+SVGP}   & Feature extractor        & CNN         & ResNet-18   & ResNet-34   \\
                        & NN out features $D_w$         & 128         & 512         & 512         \\
                        & Embedding features $P$               & 16          & 64          & 128         \\
                        & \# inducing points $\hat{M}$      & 512         & 512         & 512         \\
                        & \# epochs       & 20         & 200         & 200         \\
                        & Training strategy      & Full-training         & Full-training         & Fine-tuning         \\
\midrule[0.5pt]
\multirow{5}{*}{SV-DKL} & Feature extractor        & CNN         & ResNet-18   & ResNet-34   \\
                        & NN out features $D_w$         & 128         & 512         & 512         \\
                        & Embedding features $P$               & 16          & 64          & 128         \\
                        & \# inducing points $\hat{M}$      & 64          & 64          & 64          \\
                        & Grid bounds              & {[}-1,1{]} & {[}-1,1{]} & {[}-1,1{]} \\
                        & \# epochs       & 20         & 200         & 200         \\
                        & Training strategy       & Full-training         & Full-training         & Fine-tuning         \\
\midrule[0.5pt]
\multirow{4}{*}{DAK}    & Feature extractor        & CNN         & ResNet-18   & ResNet-34   \\
                        & NN out features $D_w$         & 128         & 512         & 512         \\
                        & Embedding features $P$               & 16          & 64          & 128         \\
                        & \# induced interpolation $M$ & 63          & 63          & 63         \\
                        & \# epochs       & 20         & 200         & 200         \\
                        & Training strategy      & Full-training         & Full-training         & Full-training         \\
\bottomrule[1pt]
\end{tabular}

}
\label{tab:model classification}
\end{table}

\paragraph{MNIST} We used a CNN implemented in PyTorch as the feature extractor: \texttt{Conv2d}(1,32,3) $\rightarrow$ \texttt{Conv2d}(32,64,3) $\rightarrow$ \texttt{MaxPool2d}(2) $\rightarrow$ \texttt{Dropout}(0.25) $\rightarrow$ \texttt{Linear}(9216,128) $\rightarrow$ \texttt{Dropout}(0.5). To make a fair comparison, for both SV-DKL and DAK, we applied an embedding module through a linear layer that transform $128$ output features into $P=16$ base GP channels. 

\paragraph{CIFAR-10} We used a ResNet-18 as the feature extractor followed by a linear embedding layer that compressed the $512$ output features into $P=64$ base GP channels. 

\paragraph{CIFAR-100} We used a pretrained ResNet-34 as the feature extractor for SV-DKL and fine-tuned GP output layers since SV-DKL struggled to fit using full-training. For proposed DAK, we used full-training. The number of base GP channels is selected as $P=128$. 

\begin{table}[ht]
\caption{Details of training optimizer for image classification on MNIST, CIFAR-10 and CIFAR-100.}
\centering
\resizebox{0.7\linewidth}{!}{

\begin{tabular}{l|ccc}
\toprule[1pt]
Optimization      & MNIST                                                             & CIFAR-10                                                                                                  & CIFAR-100                                                                                                 \\
\midrule[0.5pt]
Optimizer         & Adadelta                                                          & SGD                                                                                                       & SGD                                                                                                       \\
Initial lr.       & 1.0                                                               & 0.1                                                                                                       & 0.1                                                                                                       \\
Weight decay      & 0.0001                                                            & 0.0001                                                                                                    & 0.0001                                                                                                    \\
Scheduler         & StepLR                                                            & CosineAnnealingLR                                                                                         & CosineAnnealingLR                                                                                         \\
\midrule[0.5pt]
Data Augmentation & MNIST                                                             & CIFAR-10                                                                                                  & CIFAR-100                                                                                                 \\
\midrule[0.5pt]
RandomCrop        & -                                                                 & size=32, padding=4                                                                                        & size=32, padding=4                                                                                        \\
HorizontalFlip    & -                                                                 & p=0.5                                                                                                     & p=0.5                                                                                                     \\
% Normalization     & \begin{tabular}[c]{@{}l@{}}mean=0.1307,\\ std=0.3081\end{tabular} & \begin{tabular}[c]{@{}l@{}}mean={[}0.4914,0.4822,0.4465{]},\\ std={[}0.2023,0.1994,0.2010{]}\end{tabular} & \begin{tabular}[c]{@{}l@{}}mean={[}0.5071,0.4867,0.4408{]},\\ std={[}0.2675,0.2565,0.2761{]}\end{tabular} \\
\bottomrule[1pt]
\end{tabular}
}
\label{tab:optimizer classification}
\end{table}

\paragraph{Additional Benchmark.}  \citet{matias2024amortized} proposed Amortized Variational DKL (AV-DKL), which is a variant SV-DKL using amortization network to compute the inducing locations and variational parameters, thus attenuating the overcorrelation of NN extracted features. AV-DKL is included as the additional benchmark for classification tasks in \Cref{tab:img avdkl}. The training recipe is the same with SV-DKL. 


\begin{table*}[ht]
\caption{\small{Accuracy, NLL, ECE for AV-DKL, SV-DKL, DAK-MC on CIFAR-10/100 averaged over 3 runs. CIFAR-10 uses ResNet-18 with 64 features extracted; CIFAR-100 uses ResNet-34 with 512 features. The best results are highlighted in \textbf{bold}; the second best results are highlighted by \underline{underline}.}}
\centering
\vspace{-0.1cm}
\resizebox{\linewidth}{!}{%
\begin{tabular}{rccclccc}
\toprule[1pt]
\multicolumn{1}{l}{} & \multicolumn{3}{c}{Batch size: 128}  &  & \multicolumn{3}{c}{Batch size: 1024} \\ \cline{2-4} \cline{6-8} \vspace{-8pt} \\
\multicolumn{1}{l}{} & AV-DKL & SV-DKL & \cellcolor{Gray} DAK-MC &   & AV-DKL  & SV-DKL & \cellcolor{Gray} DAK-MC \\ 
\midrule[1pt]
CIFAR-10 - Acc. (\%) $\uparrow$    & \underline{94.23 $\pm$ 0.65}  & 93.44 $\pm$ 0.28    &  \cellcolor{Gray} \textbf{94.81 $\pm$ 0.13}   &     &  \textbf{93.32} $\pm$ \textbf{0.13}        & 90.22 $\pm$ 1.42       & \cellcolor{Gray} \underline{93.02 $\pm$ 0.18}        \\
NLL $\downarrow$     & 0.352 $\pm$ 0.084    & \underline{0.312 $\pm$ 0.033}       &  \cellcolor{Gray} \textbf{0.256} $\pm$ \textbf{0.014}     &      & \underline{0.439 $\pm$ 0.022}         & 0.485 $\pm$ 0.061       & \cellcolor{Gray} \textbf{0.345 $\pm$ 0.001}    \\
ECE $\downarrow$      & 0.048 $\pm$ 0.006    & \underline{0.046 $\pm$ 0.003}       &  \cellcolor{Gray} \textbf{0.039 $\pm$ 0.002}          &     & \underline{0.054 $\pm$ 0.001}       & 0.060 $\pm$ 0.004       & \cellcolor{Gray} \textbf{0.052 $\pm$ 0.001}           \\
\midrule[1pt]
CIFAR-100 -  Acc. (\%) $\uparrow$    & \textbf{77.47 $\pm$ 0.19}  & 74.52 $\pm$ 0.13       & \cellcolor{Gray}  \underline{76.75 $\pm$ 0.18}     &     &  \textbf{77.07 $\pm$ 0.10}        & 66.54 $\pm$ 0.74       & \cellcolor{Gray} \underline{70.38 $\pm$ 1.25}        \\
NLL $\downarrow$     & 1.787 $\pm$ 0.011    & \underline{1.041 $\pm$ 0.007}       & \cellcolor{Gray}  \textbf{1.001 $\pm$ 0.027}     &      & 2.326 $\pm$ 0.030    & \underline{1.738 $\pm$  0.058}      & \cellcolor{Gray} \textbf{1.203 $\pm$ 0.040}        \\
ECE $\downarrow$      & 0.166 $\pm$ 0.002    & \underline{0.049 $\pm$ 0.002}       & \cellcolor{Gray}  \textbf{0.041 $\pm$ 0.004}        &     & 0.175 $\pm$ 0.001         & \underline{0.148 $\pm$ 0.007}       &\cellcolor{Gray}  \textbf{0.056 $\pm$ 0.006}           \\
\bottomrule[1pt]
\end{tabular}
}
\vspace{-0.2cm}
\label{tab:img avdkl}
\end{table*}

\paragraph{Metrics} 
We evaluate model performance using four common metrics: Top-1 accuracy, ELBO, Negative Log Likelihood (NLL), and Expected Calibration Error (ECE). 

ECE is a metric used to quantify the degree of ``calibration'' of a probabilistic model in UQ, specifically for classification problems. It is defined as the weighted average of the absolute difference between the model's predicted probability (confidence) and the actual outcome (accuracy) over several bins of predicted probability. Mathematically, ECE is given by:
\begin{align}
    \text{ECE} =\sum_{m=1}^{M} \frac{\left| B_{m} \right|}{n} \left| \text{acc} (B_{m})-\text{conf} (B_{m}) \right|,
\end{align}
where $M$ is the number of bins into which the confidence values are partitioned, $B_m$ is the set of indices of samples whose predicted confidence falls into the $m$-th bin, $n$ is the total number of samples.

\paragraph{Computing Infrastructure}
The experiments for classification were run on a Linux machine with NVIDIA RTX4080 GPU, and 32GB of RAM.




\subsection{Additional Tables and Figures}
\label{sec:additional exp results}

\paragraph{Choices of learning rates.}
We evaluate the choices of learning rates on 1D regression examples. DKL requires a separate tuning of the learning rate of the GP covariance parameters, which differs from the learning rate of the NN feature extractor. In \Cref{fig:dkl lr}, we choose the learning rate of the NN feature extractor as $0.01$, while the learning rate of the GP covariance is set to different values. (a)-(c) show that different learning rates of covariance in DKL result in different predictive posterior. In particular, although the training losses for DKL in both (a) and (b) are minimal, the regressions do not fit well. On the other hand, DAK does not need a distinct recipe for tuning GP covariances because of the BNN interpretation. Furthermore, the poor posterior is indicated by the higher training loss, as illustrated in (d)-(f).

\begin{figure}[ht]
\centering
\subfloat[$\begin{gathered}\text{DKL: last-layer lr} =0.01.\\ \text{Training loss:} -0.21.\end{gathered}$]{\includegraphics[width=.3\textwidth]{toy_dkl_lr_01.pdf}}
\subfloat[$\begin{gathered}\text{DKL: last-layer lr} =0.001.\\ \text{Training loss: } -0.07.\end{gathered}$]{\includegraphics[width=.3\textwidth]{toy_dkl_lr_001.pdf}}
\subfloat[$\begin{gathered}\text{DKL: last-layer lr} =0.0001.\\ \text{Training loss: } 0.22.\end{gathered}$]{\includegraphics[width=.3\textwidth]{toy_dkl_lr_0001.pdf}}

\subfloat[$\begin{gathered}\text{DAK: last-layer lr} =0.1.\\ \text{Training loss: } 0.10.\end{gathered}$]{\includegraphics[width=.3\textwidth]{toy_dak_lr_1.pdf}}
\subfloat[$\begin{gathered}\text{DAK: last-layer lr} =0.01.\\ \text{Training loss: } 0.10.\end{gathered}$]{\includegraphics[width=.3\textwidth]{toy_dak_lr_01.pdf}}
\subfloat[$\begin{gathered}\text{DAK: last-layer lr} =0.001.\\ \text{Training loss: } 0.22.\end{gathered}$]{\includegraphics[width=.3\textwidth]{toy_dak_lr_001.pdf}}

\caption{Results on 1D regression with different last-layer learning rates. The learning rate of NN feature extractor is set as $0.01$. (a)--(f) shows the regression fits and corresponding training losses. DAK fits for the same learning rate strategy with NN feature extractor (lr=0.01), while DKL requires a separate tuning for last-layer learning rate of GPs. Additionally, a better training loss does not necessarily prevent overfitting for DKL.}
\label{fig:dkl lr}
\end{figure}


\paragraph{Learning curves.} We plot the learning curves of CIFAR-10/100 in \Cref{fig:cifar10 curves} and \ref{fig:cifar100 curves}. The learning curves of SVDKL in \Cref{fig:cifar10 curves} is more unstable, with many significant spikes, and the convergence is slower than DAK. Futhermore, SVDKL struggles to fit with full-training in CIFAR-100, and a pretrained feature extractor is used in CIFAR-100. Therefore, the learning curves of SVDKL look smoothing, but DAK fits well with full-training in CIFAR-100.


\begin{figure}[ht]
\centering
\subfloat[Test Error (\%).]{\includegraphics[width=.3\textwidth]{CIFAR_10_test_error.pdf}}
\subfloat[Test NLL.]{\includegraphics[width=.3\textwidth]{CIFAR_10_nll.pdf}}
\subfloat[ELBO.]{\includegraphics[width=.3\textwidth]{CIFAR_10_elbo.pdf}}
\caption{Test errors, test NLLs, ELBOs of NN, SVDKL, and DAK curves with batch size of 128/1024 for CIFAR-10 averaged on 3 runs. DAK outperforms SVDKL on both test error and NLL along the training epochs. Additionally, SVDKL degrades more and struggles to fit when the batch size becomes larger.}
\label{fig:cifar10 curves}
\end{figure}

\begin{figure}[ht]
\centering
\subfloat[Test Error (\%).]{\includegraphics[width=.3\textwidth]{CIFAR_100_test_error.pdf}}
\subfloat[Test NLL.]{\includegraphics[width=.3\textwidth]{CIFAR_100_nll.pdf}}
\subfloat[ELBO.]{\includegraphics[width=.3\textwidth]{CIFAR_100_elbo.pdf}}
\caption{Test errors, test NLLs, ELBOs of NN, SVDKL, and DAK curves with batch size of 128/1024 for CIFAR-100 averaged on 3 runs. DAK trained NN and last-layer additive GPs jointly, while SVDKL used the pre-trained NN and fine-tuned the last-layer GP since SVDKL struggles to fit using full-training. DAK outperforms SVDKL on both test error and NLL along the training epochs. SVDKL struggled to fit in high-dimensional multitask cases, indicating the necessity of pre-training in SVDKL. However, DAK fitted well with high dimensionality and large batch sizes.}
\label{fig:cifar100 curves}
\end{figure}







% \end{document}



\end{document}
