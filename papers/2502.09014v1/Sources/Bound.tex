\section{Shortlist vs. No Shortlist}
\label{sec: compare}
How much does a contest designer benefit from implementing a shortlist? In this section, we address this question by comparing the optimal contest designs with and without a shortlist under a linear cost function. Specifically, we focus on three types of contests:
%How much can a contest designer benefit from the shortlist option? In this section, we answer this question by comparing the optimal contest design with and without a shortlist, when the cost function is linear. In this section, we mainly focus on three kinds of contests:
\begin{enumerate}
    \item The $n$-contestant winner-take-all contest – This is the optimal contest without a shortlist for both maximum individual effort and total effort \cite{MS01,CHS19}.
    \item The two-contestant winner-take-all contest – This is the optimal contest with a shortlist for the maximum individual effort (by Theorem~\ref{thm:ExpostHighestEffort}).
    \item The complete simple contest with a shortlist size of $m^*$ and $m^*-1$ prizes - This is the optimal contest with a shortlist for the total effort (by Proposition~\ref{thm:ConpleteSimpleContest}).
\end{enumerate}

For any ability distribution $F$ and $n$ initial registered contestants, let $S^{(1)}(m,n,l)$ and $S(m,n,l)$ denote the maximum individual effort and total effort, respectively, in a simple contest with a shortlist size of $m$ and $l\leq m$ prizes. To quantify the gap in objectives, we establish bounds on $S^{(1)}(m,n,l)$ and $S(m,n,l)$ for the three contest types discussed above.

Before deriving bounds on the objectives, we first express the maximum individual effort and total effort objectives in terms of quantiles. Given any ability distribution $F$, recall that the quantile is defined as $q:=1-F(x)$ and its reverse function $v(q):=F^{-1}(1-q)=x$. Using this notation, we obtain the following result.

\begin{lemma}[Quantile Representation for Effort]\label{lem:QuantileRep}
    % By using quantile $q:=1-F(x)$ and its reverse function $v(q):=F^{-1}(1-q)=x$, Ex-ante total effort of a simple contest expresses as:
    % \[
    % S(m,n, l)= n\int_0^1|v'(q)|\int_0^qG_{(m,l)}(t)\,dt\,dq,
    % \]where $l$ is the number of prizes, $G_{(m,l)}(t)=\frac{\binom{n-1}{l}(1-t)^{n-l-1}t^{l-1}}{\sum_{j=1}^{m}\binom{n-1}{j-1}(1-t)^{n-j}t^{j-1}}\int_{0}^{t}\sum_{j=1}^{m}\binom{n-1}{j-1}p^{j-1}(1-p)^{n-j}\,dp$. We use $H_{(m,l)}(q):=\int_0^qG_{(m,l)}(t)\,dt$ to denote the distribution-free part.  
    By using quantile $q:=1-F(x)$ and its reverse function $v(q):=F^{-1}(1-q)=x$, the ex-ante maximum individual effort of a simple contest is:
    \[
    S^{(1)}(m,n, l)= n\int_0^1|v'(q)|\int_0^qG^{(1)}_{(m,l)}(t)\,dt\,dq,
    \]
    where $l$ is the number of prizes and $G_{l,m}^{(1)}(t):=\frac{\binom{n-1}{l}(1-t)^{n-l-1}t^{l-1}}{\sum_{j=1}^{m}\binom{n-1}{j-1}(1-t)^{n-j}t^{j-1}}\int_{0}^{t}(1-p)^{n-1}\,dp.$
    
    And the ex-ante total effort of a simple contest expresses as:
    \[
    S(m,n, l)= n\int_0^1|v'(q)|\int_0^qG_{(m,l)}(t)\,dt\,dq,
    \]similarly, $G_{(m,l)}(t):=\frac{\binom{n-1}{l}(1-t)^{n-l-1}t^{l-1}}{\sum_{j=1}^{m}\binom{n-1}{j-1}(1-t)^{n-j}t^{j-1}}\int_{0}^{t}\sum_{j=1}^{m}\binom{n-1}{j-1}p^{j-1}(1-p)^{n-j}\,dp$, and we use $H(q):=\int_0^qG(t)\,dt$ to denote the distribution-free part.  
\end{lemma}

Based on this representation, we establish the ranges for both the maximum individual effort and the total effort achieved by the $n$-contestant winner-take-all contest and the $2$-contestant winner-take-all contest, respectively, as stated in the following two lemmas.
% Having this representation, we provide the ranges of both the maximum individual effort  and the total effort achieved by the $n$-contestant winner-take-all contest and the $2$-contestant winner-take-all contest, respectively, shown in the following two lemmata. 

\begin{lemma}\label{lem:bound on n,1}
    For any ability distribution, the $n$-contestant winner-take-all contest achieves a maximum individual effort of $S^{(1)}(n, n,1) = \Theta(1)$ and a total effort of $S(n, n,1) = \Theta(1)$.
\end{lemma}
    
\begin{lemma}\label{lem:bound on 2,1}
    For any ability distribution, the $2$-contestant winner-take-all contest achieves a maximum individual effort of $S^{(1)}(2, n,1) = \Theta(\log n)$ and a total effort of $S(2, n,1) = \Theta(\log n)$.
\end{lemma}

On the other hand, using the Beta representation for total effort (Lemma \ref{lem:betaRepTotalEffort}), we can derive the range of total effort achieved by a complete simple contest. 

\begin{lemma}\label{lem:bound on m,m-1}
    Fixed any ability distribution $F$, the optimal complete simple contest with a shortlist size $m^*$ achieves a total effort of $S(m^*,n, m^*-1) = \Theta(n)$, where $m^*$ depends on the distribution $F$.
\end{lemma}

Based on these three lemmas, we compare the optimal contests with and without a shortlist in terms of two types of objectives.

\noindent \textbf{Maximum Individual Effort.}
% For the maximum individual effort objective, we only focus on comparing the $2$-contestant winner-take-all contest with $n$-contestant winner-take-all contest, in term of the maximum individual effort they can achieve, since they are the optimal among all feasible ability distributions. Based on Lemma \label{lem:bound on n,1} and \label{lem:bound on 2,1}, we can obtain the following theorem. 
For the maximum individual effort objective, we only focus on comparing the two-contestant winner-take-all contest with the $n$-contestant winner-take-all contest in terms of the maximum individual effort they can achieve, as these are optimal for all feasible ability distributions. Based on Lemmas \ref{lem:bound on n,1} and \ref{lem:bound on 2,1}, we derive the following theorem.

\begin{theorem}\label{thm: 2,1 vs n,1 max effort}
    For any ability distribution, under the maximum individual effort objective, the two-contestant winner-take-all contest results in $\Theta(\log n)$ times the maximum individual effort of the optimal contest without a shortlist. Specifically, we have:
    $$\frac{S^{(1)}(2,n,1)}{S^{(1)}(n,n,1)} = \Theta(\log n).$$
\end{theorem}

\noindent \textbf{Total Effort.}
For the total effort objective, while the optimal contest with a shortlist is a complete simple contest as shown by Theorem \ref{thm:ConpleteSimpleContest}, the optimal shortlist size depends on the ability distribution. This means the optimal contests vary with different ability functions. To make a meaningful comparison, we focus on a fixed ability distribution $F$ and examine the gap in total effort between the optimal contest with a shortlist under $F$ and the $n$-contestant winner-take-all contest. Based on Lemma \ref{lem:bound on n,1} and \ref{lem:bound on m,m-1}, the comparison of these two contests is summarized in Theorem \ref{thm:TotalOPTVAN}.

\begin{theorem}\label{thm:TotalOPTVAN}
    Fixed any ability distribution $F$, under the total effort objective, the optimal contest with a shortlist can achieve $\Theta(n)$ times the total effort compared to the optimal contest without a shortlist. Specifically,
    $$\frac{S(m^*,n,m^*-1)}{S(n,n,1)} = \Theta(n),$$
    where $m^*$ is the optimal shortlist size for the ability distribution $F$. 
\end{theorem}

Additionally, with respect to total effort, we still can compare the fully shortlisted contest (the 2-contestant winner-take-all contest) with the optimal contest without a shortlist (the $n$-contestant winner-take-all contest), as presented in the following proposition.
\begin{proposition}\label{prop:TotalTWOVAN}
    For any ability distribution, under the total effort objective, the two-contestant winner-take-all contest results in $\Theta(\log n)$ times the total effort of the optimal contest without a shortlist. Specifically, we have:
    $$\frac{S^{(1)}(2,n,1)}{S^{(1)}(n,n,1)} = \Theta(\log n).$$
\end{proposition}



%We discusses the maximum individual effort objective and the total effort objective in Subsection~\ref{subsec:ApproxMaxi} and \ref{subsec:ApproxTotal}, respectively.

% \subsection{Maximum individual Effort}\label{subsec:ApproxMaxi}
% When the objective is the maximum individual effort, we have known that the optimal contest with a shortlist is a two-contestant winner-take-all contest and the optimal contest without a shortlist is a $n$-contestant winner-take-all contest. In the following, we will compare the maximum individual efforts induced by these two contests. 

% Before analyzing the gap between the optimal contests with and without a shortlist, we first rewrite the objective function by the quantile. Given any ability distribution $F$, recall that the quantile $q:=1-F(x)$ and its reverse function $v(q):=F^{-1}(1-q)=x$ and use $S^{(1)}(m,n,l)$ to represent the maximum individual effort obtained by a simple contest with the shortlist size $m$ and $l\leq m$ prizes. Now, we get the following lemma to represent the maximum individual effort objective. 
% \begin{lemma}
%     {\color{red} Add the representation of objective, refer to Lemma A.10}
% \end{lemma}

% Having this representation, we can find that the ratio between these contests is 

% \[
%     \begin{aligned}
%         \frac{S^{(1)}(2,n,1)}{S^{(1)}(n,n,1)} & = \frac{n\int_0^1|v'(q)|H^{(1)}_{(2,1)}(q)\,dq}{n\int_0^1|v'(q)|H^{(1)}_{(n,1)}(q)\,dq} 
%     \end{aligned}
% \]


% \begin{lemma}
%     {\color{red} Add the relationship between $H^{(1)}_{(2,1)}(q)$ and $H^{(1)}_{(n,1)}(q)$ }
% \end{lemma}

% Obtaining the relationship between relationship between $H^{(1)}_{(2,1)}(q)$ and $H^{(1)}_{(n,1)}(q)$, we can get the final ratio. 
% \begin{theorem}
%     For any ability level distribution, under the total effort objective, the 2-contestant winner-take-all contest gives $\Theta(\log n)$ times the effort of the optimal contest without a shortlist, i.e.:
%     $$\frac{S^{(1)}(2,n,1)}{S^{(1)}(n,n,1)} = \Theta(\log n).$$
% \end{theorem}

% \subsection{Total Effort}\label{subsec:ApproxTotal}
% In this subsection, we care about the total effort objective. First, we know that the optimal contest without a shortlist is still a $n$-contestant winner-take-all contest \cite{CHS19}. The optimal contest with a shortlist is a complete simple contest shown by Theorem \ref{thm:ConpleteSimpleContest}, but the optimal shortlist size depends on the ability distribution, which means that the optimal contests are different under different ability functions. Therefore, to make the comparison meaningful, we focus on that fixed any ability distribution $F$, consider the gap of total effort between the current optimal contest with a shortlist under distribution $F$ with the $n$-contestant winner-take-all contest. 


% \begin{lemma}
%     $S(n ,n,1) = \Theta(1).$
% \end{lemma}
    
% \begin{lemma}
%     $S(2,n,1) = \Theta(\log n).$
% \end{lemma}
    
% \begin{lemma}
%     $S(m^*,n, m^*-1) = \Theta(n).$
% \end{lemma}


