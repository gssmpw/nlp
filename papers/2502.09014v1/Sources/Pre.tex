\section{Model and Preliminaries}
\label{sec:pre}

In this section, we formally introduce the contest model with a shortlist. %, which is according with the model in \cite{SSYJ24}. 
Consider a single-contest setting with $n$ registered contestants. Each contestant $i \in [n]=\{1,2,\cdots, n\}$ has a private bounded ability level, denoted by $0<x_i<+\infty$, which is independently and identically drawn from a publicly known distribution  $F(\cdot)$ with a continuous probability density function $f(\cdot)>0$. For convenience, we also represent ability using the quantile $q_i := 1-F(x_i)$ which follows a uniform distribution on $[0, 1]$, i.e., $U[0,1]$. The inverse function, $v(q_i):=F^{-1}(1-q_i)=x_i$,  is strictly decreasing, meaning that a lower quantile corresponds to a higher ability level.

As mentioned in Section \ref{sec:intro}, we consider a contest with a shortlist. Initially, the contest designer determines the size of the shortlist, i.e., the number of admitted contestants, $2\leq m \leq n$, and establishes the prize structure $\vec{V}=(V_1, V_2, \cdots, V_m)$ within the budget $B$, where  $V_1 \geq V_2\geq  \cdots \geq V_m$ and $\sum_{i=1}^p V_i \leq B$. Denote by $\mathcal{C}=(m,\vec{V})$ the designer's decision variable, i.e., the contest configuration. Then, the designer then discerns the abilities of the registered contestants\footnote{In practice, ``ability'' can be reflected in various ways, such as applicants' resumes, athletes' historical scores, or other relevant data, prior to the contest. We assume the designer is able to assess and discern the contestants' abilities.},
selecting the top $m$ contestants based on ability into the shortlist, while eliminating the others. Note that the designer determines the shortlist size and prize structure before knowing the exact abilities of the contestants. In other words, these decisions are based solely on the number of registered contestants $n$ and the prior ability distribution $F$. 

Next, each admitted contestant $i$, i.e., whose ability $x_i$ ranks in the top $m$, is informed and then strategically chooses an effort level $e_i$ to compete for prizes. The cost of exerting effort is given by $g(e_i)/x_i$, where $g(\cdot): \mathbb{R}_{\geq 0} \rightarrow \mathbb{R}_{\geq 0}$ is a strictly increasing, continuous and differentiable function with $g(0)=0$. Intuitively, for the same effort level $e$, a contestant with higher ability incurs a lower cost than one with lower ability. After effort levels are chosen, the contest designer allocates prizes based on a rank-order rule: the contestant with the highest effort receives the first prize $V_1$, the second-highest effort earns $V_2$, and so on. 

In summary, our model follows these sequential steps:
\begin{enumerate}
    \item The contest designer determines the number of admitted contestants, $m$, and sets the prize structure $\vec{V}$ within the budget $B$.
    \item The top $m$ contestants based on ability are selected, while the others are eliminated.
    \item The remaining $m$ contestants exert effort to compete for prizes.
    \item Prizes are awarded based on effort, in non-descending order.
\end{enumerate}

Given our contest model, we now define a contestant's utility. Since eliminated contestants exert no effort and receive no prize, their utility is zero.
Let $\mathbf{e}= (e_1, e_2,\cdots, e_m)$ denote the effort profile of all admitted contestants. For an admitted contestant $i$, the utility is given by the prize earned minus the incurred cost:
$$
    u_i(\mathbf{e}) = V_{Rank(i,\mathbf{e})}-\frac{g(e_i)}{x_i},
$$
where $Rank(i,\mathbf{e})$ represents contestant $i$'s rank based on the effort profile $e$. Each contestant aims to maximize utility by choosing an effort level.

For an admitted contestant $i$, the exerted effort $e_i$ depends only on her ability $x_i$ and her strategy function $b_i(x): \mathbb{R} \rightarrow \mathbb{R}$, i.e., $e_i=b_i(x_i)$. We assume that $b_i(x)$ is monotone non-decreasing, meaning that a contestant with higher ability will not exert lower effort.
In this paper, suppose all admitted contestants follow the same strategy function, i.e., the strategy is symmetric: $b_i(x)=b_j(x)$ for any $x>0$ and any two admitted contestants $i$ and $j$. Since our model is under the incomplete information setting, we adopt the symmetric Bayesian Nash equilibrium as the solution concept. 
\begin{definition}\label{def:sBNE}
    An effort function constitutes a symmetric Bayesian Nash equilibrium (sBNE) if and only if, for any admitted contestant, choosing this function maximizes expected utility when all others do the same. Specifically, an effort function $b^*(x)$ is an sBNE if and only if, for any admitted contestant $i$,
$$
    b^*(x_i) \in \arg\max_{e_i} \sum_{j=1}^m V_j \cdot \text{Pr}_{ij}(e_i) -\frac{g(e_i)}{x_i},
$$
    where $\text{Pr}_{ij}(e_i)$ represents the probability that contestant $i$ is ranked at $j$ in the contest, given that all other admitted contestants also adopt $b^*(x)$. 
\end{definition}

Lastly, the contest designer aims to optimize an objective function by establishing the contest configuration $\mathcal{C}=(m, \vec{w})$, including the shortlist size and prize structure. In this work, we focus on two objective functions: maximum individual effort and total effort. Specifically, given a contest configuration $\mathcal{C}$, the maximum individual effort objective is defined as
$$
    \text{ME}(\mathcal{C},F)= \mathbb{E}_{x_1, x_2,\cdots, x_n \sim F}\big[e_{(1)}\big],
$$
where $e_{(1)}$ is the highest effort among all contestants under the sBNE. The total effort objective is defined as
$$
    \text{TE}(\mathcal{C},F)= \mathbb{E}_{x_1, x_2,\cdots, x_n \sim F}\big[\sum_{i=1}^n e_{i}\big],
$$
where $e_i$ is contestant $i$'s exerted effort under the sBNE, assuming that eliminated contestants exert zero effort. 

% In this paper, we focus on comparing two contests and adopt the concept of the approximation ratio to quantify the gap in their objectives. Consider two contests, $\mathcal{C}$ and $\mathcal{C}'$, where the performance of contest $\mathcal{C}$ is superior to that of $\mathcal{C}'$. Formally, for any ability distribution $F$, we define the approximation ratio between contest $\mathcal{C}$ and $\mathcal{C}'$ with respect to the maximum individual effort objective (and similarly for the total effort objective) as:
% $$
%     \text{APR}=\sup_{F\in\mathcal{F}} \Big\{\frac{\text{ME}(\mathcal{C},F)}{\text{ME}(\mathcal{C}',F)}
%      \Big\}.
% $$
%In the following, we consider how to design the optimal contest configuration to optimize these two objectives under the sBNE, respectively.

