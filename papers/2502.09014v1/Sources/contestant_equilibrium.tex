\section{Contestant Equilibrium}
\label{sec:playerSBNE}
In this section, we fully characterize the unique symmetric Bayesian Nash equilibrium of admitted contestants for any shortlist size $m$ and any prize structure $\vec{V}$. A key step in achieving this characterization is to explicitly represent each contestant's utility, allowing us to derive and solve the first-order condition.
Unlike traditional contest settings without a shortlist, where the probability of contestant $i$ is ranked at $j$ can be directly computed using the prior ability distribution, our setting requires calculating these probabilities based on posterior beliefs about the abilities of other admitted contestants. This necessitates a detailed understanding of the posterior ability distribution. Therefore, we first describe the closed form of posterior beliefs in Subsection \ref{subsec:Posterior Beliefs}. Then, in Subsection \ref{subsec:Equilibrium Efforts}, we provide a complete characterization of contestant equilibrium efforts.

\subsection{Posterior Beliefs}
\label{subsec:Posterior Beliefs}

Given a contest with a shortlist size of $m$ and a prize structure $\vec{V}$, we first update an admitted contestant's posterior belief about the ability levels of other admitted contestants. This posterior belief, represented as the joint posterior distribution, is derived using Bayes' rule, as stated in the following proposition.

% To find the subjective probability of getting each rank for any players, we need to first derive their posterior beliefs on other contestants' ability level after updating according to Bayesian Rule. 

\begin{proposition}[Posterior Beliefs]\label{prop:posteriorBeliefs}
    For any admitted contestant (w.l.o.g., labeled as Contestant 1), the joint posterior probability density function of the ability levels of the other admitted contestants is given by:
    % For any selected observer (her player index is re-indexed as $1$), the joint posterior probability density function of other players' ability level perceived by her is:
    \[
        \beta_1(\mathbf{x}) =   
        \begin{cases} 
        \frac{\binom{n-1}{m-1}F^{n-m}(x^{(1)})\prod_{i=2}^{m}f(x_i)}{J(F,n,m,x_1)} & \text{if } x^{(1)} \leq x_1, \\
        \frac{\binom{n-1}{m-1}F^{n-m}(x_1)\prod_{i=2}^{m}f(x_i)}{J(F,n,m,x_1)} & \text{if } x^{(1)} > x_1,
        \end{cases}
    \]
    where $x_1$ is the contestant 1's ability level and $x^{(1)}:=\min_{j\in [m]\setminus \{1\}}x_i$ is the lowest ability level of other admitted contestants. The normalization denominator is defined as $J(F(\cdot),n,m,x) := \binom{n-1}{m-1}F^{n-m}(x)(1-F(x))^{m-1}+\binom{n-1}{m-1}(m-1)\int_0^{F(x)}t^{n-m}(1-t)^{m-2} \, dt$.
\end{proposition}

We further marginalize this belief to obtain the posterior ability distribution of a specific admitted contestant.

% To see the how shortlist affects the belief of admitted players, we marginalize this belief, thereby to see the posterior ability level distribution the observer holds for a specific rival.

\begin{corollary}[Marginal Posterior Beliefs]\label{prop:marginalBelief}
For any admitted contestant (labeled as Contestant 1), the posterior probability density function of another admitted contestant's ability (labeled as Contestant 2) is:

% For any selected observer (her player index is re-indexed as $1$), the posterior probability density function of another player's ability level (this player's index is re-indexed as $2$) perceived by her is:
\[
\beta_1(z) =   
\begin{cases} 
\frac{\binom{n-1}{m-1}\left [ F^{n-m}(z)(1-F(z))^{m-2}+(m-2)\int_0^{F(z)}t^{n-m}(1-t)^{m-3}\, dt \right ]f(z)}{J(F,n,m,x_1)} & \text{if } z \leq x_1, \\
\frac{\binom{n-1}{m-1}\left [  F^{n-m}(x_1)(1-F(x_1))^{m-2} +(m-2)\int_0^{F(x_1)} t^{n-m}(1-t)^{m-3}\, dt \right ] f(z)}{J(F,n,m,x_1)} & \text{if } z > x_1.
\end{cases}
\]

The posterior cumulative distribution function is then expressed as:
\[
\Pr_{\beta_1}(X_2\leq z) =   
\begin{cases} 
\frac{\binom{n-1}{m-1} \left [ B_{F(z)}(n-m+1,m-1) + (m-2)\int_{0}^{z} B_{F(t)}(n-m+1, m-2) \, dF(t) \right ]}{J(F,n,m,x_1)} & \text{if } z \leq x_1, \\
\Pr_{\beta_1}(X_2\leq x_1) + \\ \quad\frac{\binom{n-1}{m-1}(F(z)-F(x_1))\left [ F^{n-m}(x_1)(1-F(x_1))^{m-2}+(m-2)B_{F(x_1)}(n-m+1, m-2)\right ]}{J(F,n,m,x_1)} & \text{if } z > x_1.
\end{cases}
\]
where $B_{x}(a,b):=\int_{0}^{x} t^{a-1}(1-t)^{b-1} \, dt$ denotes an incomplete beta function.

Note that both $\beta_1(\cdot)$ and $\Pr_{\beta_1}(\cdot)$ are continuous.
%Also, for the selected observer, her marginal posterior belief for different opponents are identical but not independent.  
\end{corollary}

Using the marginal posterior belief, we derive two types of stochastic dominance to illustrate the effect of a shortlist.
% We then derive two types of stochastic dominance using this marginal posterior belief.

\begin{proposition}[Stochastic Dominance of Posterior over Prior]\label{prop:stoDomPos}
    Based on the prior p.d.f. \(f(x)\) and posterior p.d.f. \(\beta_i(x)\), let \(\Pr_{f}(X_j \leq z)\) and \(\Pr_{\beta_i}(X_j \leq z)\) denote the probabilities that contestant \(i\) believes that the ability level of contestant \(j \neq i\) is at most $z$ before and after receiving the admitted signal, respectively. Then, for all $z$, we have:
    %Let \(\Pr_{f}(X_j \leq z)\) denote the probability that player \(i\), according to the prior distribution \(F(x)\), believes the ability level of player \(j \neq i\) lower than \(z\) before the promotion signal is received. Similarly, let \(\Pr_{\beta_i}(X_j \leq z)\) denote the probability that player \(i\), according to the posterior belief \(\beta_i(x)\), believes the ability level of player \(j \neq i\) is lower than \(z\) after the promotion signal is received. Then, for all \(z \geq 0\), we have:
    \[
    \Pr_{\beta_i}(X_j \leq z) \leq \Pr_{f}(X_j \leq z).
    \]
\end{proposition} 

\begin{proposition}[Posterior of Higher Ability Stochastically Dominates Lower Ability]\label{prop:StoDomAbi}
    For two admitted contestants, say $i$ and $j$, if contestant $i$ has a higher ability than contestant $j$ (i.e., $x_i > x_j$), then contestant $i$'s posterior belief about the ability of any other admitted contestant  $k$ first-order stochastically dominates that of contestant $j$. Formally, for all $z\geq 0$, we have:
    % For two promoted players, say $i,j$, if player $i$'s ability level is higher than player $j$'s, i.e., $x_i > x_j$, then player $i$'s posterior belief of the ability level of any other promoted player $k$ first-order stochastically dominates  player $j$'s, i.e., for all $z\geq 0$, we have:
    \[
    \Pr_{\beta_i}(X_k \leq z) \leq \Pr_{\beta_j}(X_k \leq z).
    \]
\end{proposition}


A direct implication of stochastic dominance is higher expectation\footnote{Formally, if random variables $X$ an $Y$ satisfy $\Pr(X\leq z) \leq \Pr(Y\leq z)$ for all $z$, i.e., $Y$ is first-order stochastically dominated by $X$, then $\mathbb{E}[X] \geq \mathbb{E}[Y]$ holds.}. Therefore, the above two propositions imply that after the shortlist is applied, each admitted contestant perceives their opponents as stronger. Moreover, the stronger a contestant is, the more competitive they perceive their opponents to be.


% A direct implication of stochastic dominance is higher expectation\footnote{Formally, if random variables $X,Y$ satisfies $\Pr(X\leq z) \leq \Pr(Y\leq z)$ for all $z$, i.e., $Y$ is first-order stochastically dominated by $X$, then $\mathbb{E}[X] \geq \mathbb{E}[Y]$ holds.}. Therefore the above two propositions shows that, after shortlist, the observer perceives every admitted opponent as stronger, also, the stronger herself, the more competitive she thinks of her rivals.

\begin{example}
    Figure~\ref{fig:beliefs} shows the belief change of contestant $x_1$ resulting from shortlist. We rewrite the posterior belief as $\beta_1(z)=q_{x_1}(z)f(z)$, then the factor $q_{x_1}(z)$ contains the information brings by the admission signal. When a contestant gets admitted, her belief for low ability is discounted, and the factor is also smaller for lower ability, then the belief for ability higher than hers is adjusted by a constant factor accordingly, as show in Figure~\ref{fig:beliefs-a},~\ref{fig:beliefs-b}. The cumulative probability functions are plotted in Figure~\ref{fig:beliefs-c}. An opponent is viewed as stronger after shortlist (Proposition~\ref{prop:stoDomPos}), and stronger contestant also sees her opponent as stronger (Proposition~\ref{prop:StoDomAbi}).
        \begin{figure}[h]
        \centering
    \begin{subfigure}[ht]{0.30\textwidth}
        \centering
        \includegraphics[width=\textwidth]{figure/Post1.pdf}
        \subcaption{Factor $q_{x_i}$}
        \label{fig:beliefs-a}
        \end{subfigure}%\hfill
    \begin{subfigure}[ht]{0.30\textwidth}
        \centering
        \includegraphics[width=\textwidth]{figure/Post2.pdf}
        \subcaption{PDF}
        \label{fig:beliefs-b}
        \end{subfigure}%\hfill
    \begin{subfigure}[ht]{0.30\textwidth}
        \centering
        \includegraphics[width=\textwidth]{figure/Post3.pdf}
        \subcaption{CDF}
        \label{fig:beliefs-c}
        \end{subfigure}
    \caption{Posterior beliefs of player $x_1$ ($n = 5,m=2,F(x)=x^2$).}
    \label{fig:beliefs}
    \end{figure}
\end{example}
Although each contestant believes that every opponent becomes stronger after the shortlist, a counterintuitive result emerges: if we consider all opponents together, the contest environment actually becomes less competitive. This is because the number of opponents is the primary factor influencing competitiveness, and its reduction makes the contestant believe that her probability of achieving a certain rank is higher. Formally, we present the following proposition:
% However, this is only part of the story. In fact, the opponents as a whole is perceived as less competitive after shortlist. Although every admitted player is assumed as stronger, the decline of rival numbers is more compelling, making the observer believe that the ability level of each ranking in her rivals has a lower chance to surpass hers. Formally, we have the following proposition:

\begin{proposition}[Threatens of Opponents Decrease after Shortlist]\label{prop:ThreatenDesc}
For any contestant (labeled as 1) and any $l < m$, she perceives the opponent with the $l^{\text{th}}$ highest ability as weaker after the shortlist, in comparison to herself:
% For any selected observer (her player index is re-indexed as $1$), she sees the $l^{\text{th}}$ strongest opponent weaker after shortlist (in comparison to herself), for any $l < m$, i.e.:
\[
\Pr_{\beta_1}(X_{(l)} > x_1) \leq \Pr_{f}(X_{(l)}>x_1),
\]
where $X_{(l)}$ is the $l^{\text{th}}$ highest ability among her opponents, $[m]\backslash\{1\}$. 

Moreover, the fewer opponents that are admitted, the weaker they are perceived by the this contestant compared to herself. In other words, $\Pr_{\beta_1}(X_{(l)} > x_1)$ increase with $m$. 
% Moreover, the less rivals get promoted, the weaker they are deemed by the observer compared to herself, i.e., $\Pr_{\beta_1}(X_{(l)} > x_1)$ increase with $m$. 
\end{proposition}
% remarks and examples
% \begin{example}
%     In Figure~\ref{fig:threatens}, when more contestants are allowed into the contest, the contestant $x_1$ believes her opponents as a whole get stronger, thus her chances of getting high rank decreases dramatically.
%     \begin{figure}[h]
%     \begin{subfigure}[ht]{0.30\textwidth}
%         \centering
%         \includegraphics[width=\textwidth]{figure/Threaten1.pdf}
%         \subcaption{$F(x)=x$}
%         \label{fig:threatens-a}
%         \end{subfigure}
%     %\hfill
%               % 子图 (b)
%     \begin{subfigure}[ht]{0.30\textwidth}
%         \centering
%         \includegraphics[width=\textwidth]{figure/Threaten2.pdf}
%         \subcaption{$F(x)=x^2$}
%         \label{fig:threatens-b}
%         \end{subfigure}
%     %\hfill
%     \begin{subfigure}[ht]{0.30\textwidth}
%         \centering
%         \includegraphics[width=\textwidth]{figure/Threaten3.pdf}
%         \subcaption{$F(x)=1-e^{-x}$}
%         \label{fig:threatens-c}
%         \end{subfigure}
%     \caption{Player $x_1$'s beliefs on surpassing $l^{\text{th}}$ strongest opponent ($n = 8, x_1=0.4$).}
%     \label{fig:threatens}
%     \end{figure}
% \end{example}

It is important to note that, for a given contestant, although the marginal posterior belief is identical for all her opponents, they are not independent. The intuition is straightforward: knowing that a contestant weaker than herself has been admitted to the game will lower her expectations for another admitted contestant. Therefore, in general, $\beta_1(\mathbf{x}) \neq \prod_{i=2}^{m} \beta_1(x_i)$, and the joint probability distribution is required to fully characterize the posterior beliefs.

% It is noteworthy that, for a given observer, although this marginal posterior belief is identical for all of her rivals, they are actually dependent. The intuition is clear, since knowing a player weaker than herself also admitted into the game lowers her expectation for a third player. Thus, $\beta_1(\mathbf{x}) \neq \prod_{i=2}^{m} \beta_1(x_i)$ in general, the joint probability is needed to fully characterize posterior beliefs. 

\subsection{Equilibrium Efforts}
\label{subsec:Equilibrium Efforts}
%[Definition of the symmetric BNE]
With the detailed characterization of posterior beliefs in place, we can now calculate the symmetric Bayesian Nash equilibrium. To facilitate the analysis, we first introduce some symbols. Let \( b: x_i \mapsto e_i \) be a strictly increasing strategy function, and let its inverse function be \( \gamma(\cdot) \). For simplicity, we use \( \gamma_i \) For simplicity, we use \( \gamma(e_i) \), which represents the corresponding ability \( x_i \) for an effort  \( e_i \) under the function $b(\cdot)$. Thus, the decision problem for contestant \(i\) to exert effort  \(e_i\) is equivalent to reporting an ability \(\gamma_i\).


%Under a symmetric Bayesian Nash Equilibrium, player \(i\) assumes that all advancing opponents determine their effort levels based on a strictly increasing function \( b: x_j \mapsto e_j \). Let the inverse of this function be denoted as \( \gamma(\cdot) \). For simplicity, we use \( \gamma_i \) as shorthand for \( \gamma(e_i) \), which represents the corresponding ability level \( x_i \) under such function $b(\cdot)$ when the effort \( e_i \) is exerted. The decision problem of player \(i\) to exert an effort level \(e_i\) then is equivalent to report a ability level \(\gamma_i\).
Now, we can calculate the probabilities that the admitted contestant $i$ exerts $e_i$ (which may not necessarily be $b(x_i)$) and is ranked at position $l$, based on her posterior beliefs.

%The subjective probability of getting each rank depends on the posterior beliefs on opponents' ability level, then, we can derive the subjective probabilities from Proposition~\ref{prop:posteriorBeliefs}.

\begin{proposition}[Subjective Probability]\label{prop:winProb}
% When promoted player $i$ decides to exert effort $e_i$ given her ability $x_i$, her perceived probability of getting rank $l$, denoted as $P_{(i,l)}(\gamma_i \mid x_i)$, is:
When admitted contestant $i$ exerts effort $e_i$ given her ability $x_i$, her perceived probability of obtaining rank $l$, denoted as $P_{(i,l)}(\gamma_i \mid x_i)$, is:

\begin{enumerate}
    \item When $\gamma_i \leq x_i$,
\[
P_{(i,l)} =   
\begin{cases} 
\frac{\binom{n-1}{l-1}F(\gamma_i)^{n-l}(1-F(\gamma_i))^{l-1}}{J(F,n,m,x_i)} & \text{if } l < m, \\
\frac{\binom{n-1}{m-1}\left [ F^{n-m}(x_i)(1-F(x_i))^{m-1}+(m-1)\int_{F(\gamma_i)}^{F(x_i)}t^{n-m}(1-t)^{m-2} \, dt \right ]}{J(F,n,m,x_i)} & \text{if } l = m.
\end{cases}
\]
    \item When $\gamma_i > x_i$,
\[
P_{(i,l)} =   
\begin{cases} 
\frac{\binom{n-1}{m-1}\binom{m-1}{l-1}(1-F(\gamma_i))^{l-1}\left [F^{n-m}(x_i)(F(\gamma_i)-F(x_i))^{m-l}+(m-l)\int_{0}^{F(x_i)}t^{n-m}(F(\gamma_i)-t)^{m-l-1}\, dt\right]}{J(F,n,m,x_i)} & \text{if } l < m, \\
\frac{\binom{n-1}{m-1}F^{n-m}(x_i)(1-F(\gamma_i))^{m-1}}{J(F,n,m,x_i)} & \text{if } l = m,
\end{cases}
\]
\end{enumerate} 
where $P_{(i,l)}$ is $P_{(i,l)}(\gamma_i \mid x_i)$ in short and $\gamma_i = b^{-1}(e_i)$ and $P_{(i,l)}(\gamma_i \mid x_i)$ is a continuous function of $\gamma_i$. 
\end{proposition}
\begin{remark}\label{rmk:subjectiveProb}
    Under the sBNE (i.e., when the observer truthfully reports her ability level, $\gamma_i=x_i$), her subjective probability of obtaining rank $l<m$ is given by $\frac{\binom{n-1}{l-1}F(x_i)^{n-l}(1-F(x_i))^{l-1}}{\sum_{j=1}^{m}\binom{n-1}{j-1}F^{n-j}(x_i)(1-F(x_i))^{j-1}}$ (since $J(F, n, m, x_i) = \sum_{j=1}^{m}\binom{n-1}{j-1}F^{n-j}(x_i)(1-F(x_i))^{j-1}$, as shown in Lemma~\ref{lem:normal}), it is exactly the prior probability of achieving rank $l$, given that the contestant is admitted (i.e., ranks among the top $m$). Notably, this subjective probability decreases as the shortlist size $m$ increases. 

    This further highlights the signaling effect of shortlisting. As the shortlist size decreases, admission becomes more difficult, leading admitted contestants to perceive themselves as relatively more competitive (as established in Proposition~\ref{prop:ThreatenDesc}). Consequently, they believe they have a higher probability of securing a better rank.
    
    %Under the sBNE, (i.e., the observer truthfully reports her ability level, $\gamma_i=x_i$), her subjective probability of getting each rank $l<m$ becomes $\frac{\binom{n-1}{l-1}F(x_i)^{n-l}(1-F(x_i))^{l-1}}{\sum_{j=1}^{m}\binom{n-1}{j-1}F^{n-j}(x_i)(1-F(x_i))^{j-1}}$ (Since $J(F, n, m, x_i) = \sum_{j=1}^{m}\binom{n-1}{j-1}F^{n-j}(x_i)(1-F(x_i))^{j-1}$, as shown in Lemma~\ref{lem:normal}), which is exactly the prior probability of getting rank $l$ conditioned on her getting admitted into the contest (i.e.,  to rank among top $m$ players). The subjective probability decreases as shortlist capacity $m$ grows.
    
    % This further shows the signaling effect of shortlisting. If the capacity becomes smaller, it is harder to qualify, then the advanced players are convinced that they are relatively more competitive (as seen from Proposition~\ref{prop:ThreatenDesc}), and therefore they have a better chance of obtaining a good rank. 
\end{remark}

Having detailed probabilities for each rank, we can now express the expected utility for each admitted contestant. Next, we move to derive the symmetric strategy function under equilibrium, thereby characterizing the sBNE of contestants. First, we integrate the first-order condition of the utility function to establish a necessary condition for the equilibrium strategy function $b(\cdot)$. Then, by verifying its non-negativity and monotonicity, we confirm the validity of the derived symmetric strategy. Consequently, we conclude that the sBNE exists and is unique, given as follows.


%Next, we move to derive the symmetric strategy under equilibrium, thereby revealing the sBNE of contestants. First, we integrate the first-order condition of equilibrium to recover $b(\cdot)$, therefore getting the expression that the equilibrium strategy must have. Then, by checking the non-negativity and monotonicity, we verify that the derived symmetric strategy is valid. Thus, we conclude that the sBNE exists and it is unique, the symmetric stategy of which states as follows:

\begin{theorem}[Unique sBNE of Contestants]\label{thm:contestantSBNE}
    For any ability distribution $F$, any size of shortlist $m$ and any prize structure $\vec{V}$, the unique symmetric Bayesian Nash equilibrium exists and can be expresses as:
    %Under symmetric Bayesian Nash Equilibrium, each promoted player $i \in [m]$ exerts effort according to a shared strictly increasing non-negative function $b(\cdot)$ of her ability level $x_i$, which uniquely expresses as:
    \[
        b^*(x) = g^{-1}\left(\int_{0}^{x}\frac{\sum_{l=1}^{m-1}\binom{n-1}{l-1}(n-l)(V_l-V_{l+1})F^{n-l-1}(t)(1-F(t))^{l-1}f(t)}{J(F,n,m,t)} t\, dt \right), 
    \]
    where $g(\cdot)$ is the cost function.
\end{theorem}
\begin{remark}\label{rmk:PrizeGap}
    From the expression of equilibrium effort, it becomes clear that contestants are incentivized by the gap between consecutive prizes, $(V_{l}-V_{l+1})$,  rather than the absolute value of the prizes themselves. Furthermore, if the cost function $g(\cdot)$ is linear, the contributions of these gaps to effort exertion remain independent of one another.
    % From the expression of equilibrium effort, we can see that it is not the absolute value of prizes but the gap between consecutive prizes $(V_{l}-V_{l+1})$s that incentive contestants to exert more effort. Moreover, if the cost function $g(\cdot)$ is linear, contribution of the gaps are independent.  
\end{remark}