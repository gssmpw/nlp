\section{Introduction}
\label{sec:intro}

% Contest is a classic topic of mechanism design in economics, where one contest designer sets prize structure and winning rule to achieve some desirable objectives and several contestants invest irreversible and costly efforts to compete for these prizes under this contest configuration. Since contest is commonly used to depict the competitive scenario, it has a broad application in reality, like sports, programming contest, resource allocation and so on. 
% Contest theory refers to a series of theories for the better understanding and informed design of contests, mainly focusing on the insights into guidelines of how to allocate prizes, what contestant behavior may arise in equilibrium and comparison of performance among different contests. 

% In reality, there always exits shortlists for the competitive scenarios. In detail, the holders do not admit all registered candidates into the final contest, and usually only pick up a part of candidates into a shortlist from the registered list and eliminate the others. For example, in the Olympic Games, only top-ranked athletes qualify for the final competition due to scheduling constraints. Similarly, in the International Collegiate Programming Contest, the offline finals have limited venue capacity, so only teams meeting specific requirements advance. A comparable scenario occurs in corporate recruitment, where only applicants who surpass a certain threshold in their resumes receive interview opportunities. Therefore, does a shortlist really benefit for the holder? or what does a shortlist affect the the competitive setting?

In real-world competitive scenarios, shortlists are commonly used. Specifically, organizers do not admit all registered candidates into the final contest but instead select a subset of candidates to form a shortlist, eliminating others. For example, in the Olympic Games, only top-ranked athletes qualify for the final competition due to scheduling constraints. Similarly, in the International Collegiate Programming Contest, the offline finals have limited venue capacity, so only teams meeting specific requirements advance. A similar scenario is observed in corporate recruitment, where only applicants who surpass a certain threshold in their resumes are invited for interviews.
This raises an important question: does a shortlist benefit the organizer, and how does it affect the competitive results?

To answer this question, we use the setting of a ``contest'' as a basic competitive (or game) model and examine the impact of a shortlist on contest design. Contests are a fundamental topic in mechanism design within economics, where a contest designer determines the prize structure and winning rules to achieve specific objectives. Contestants, in turn, invest irreversible and costly efforts to compete under these rules. Since contests effectively model competitive scenarios, they have widespread real-world applications, including sports, programming competitions, resource allocation and so on. A rank-order contest is a common contest format in which prizes are allocated strictly based on contestants' performance\footnote{The term “performance” is an abstract concept that varies by context. For example, in some cases, it may refer to measurable metrics such as race completion time in the Olympic 100-meter sprint or bid prices in an auction.}. Specifically, the contestant with the best performance wins the first prize, the second-best wins the second prize, and so on.
%Contest theory comprises a set of frameworks aimed at improving the understanding and design of contests. It primarily focuses on optimal prize allocation, predicting contestant behavior in equilibrium, and comparing the performance of different contest structures.


%From a theoretical perspective, research on rank-order contests primarily focuses on characterizing contestant equilibrium in performance and optimizing contest design to achieve various objectives, such as maximizing total effort \cite{MS01}, maximizing individual effort \cite{CHS19}, or other criteria \cite{AS09,EGG21}. %However, these studies typically assume that all contestants have an opportunity to compete. In practice, this assumption cannot be guaranteed, since contest organizers often need to filter part of participants and only admit the rest to compete for prizes, due to constraints like time and space limitations. 
%For example, in the Olympic Games, only top-ranked athletes qualify for the final competition due to scheduling constraints. Similarly, in the International Collegiate Programming Contest, the offline finals have limited venue capacity, so only teams meeting specific requirements advance. A comparable scenario occurs in corporate recruitment, where only applicants who surpass a certain threshold in their resumes receive interview opportunities. Given these real-world constraints, contest designers must often pre-screen contestants, allowing only a select group to compete for prizes.

From a theoretical perspective, research on rank-order contests primarily focuses on characterizing contestant equilibrium in performance and optimizing contest design to achieve various objectives, such as the total effort \cite{MS01}, the maximum individual effort \cite{CHS19}, or other criteria \cite{AS09,EGG21}.
Broadly speaking, the existence of a shortlist can incentivize admitted contestants to exert more effort, as they anticipate facing stronger competition. However, eliminating too many contestants may reduce motivation, as fewer opponents make it easier to secure a desirable prize. Thus, determining the optimal shortlist size is a crucial problem.
To address this, several key questions must be considered sequentially. First, we need to understand how the shortlist size affects contestants' performance under any given prize structure. Once this relationship is established, the next step is to jointly determine the optimal shortlist size and prize structure to achieve various objectives. This is particularly challenging for the total effort objective, as there is a tradeoff between the number of admitted contestants and the incentive effect. Intuitively, reducing the number of contestants increases individual effort, but total effort may not follow a simple monotonic pattern. Finally, it is essential to analyze how the presence of a shortlist impacts the designer’s objective by comparing the optimal contests with and without a shortlist.

% Rank-order contest is a kind of mainstream contest form, where the prize allocation strictly by the contestants' performance\footnote{The term ``performance'' is an abstract concept to illustrate the rule of rank-order contest. For example, in certain scenarios, it can be realized as the corresponding metrics, like the spending time in the Olympic 100-meter or the bid in an auction.} in the contest. Specifically, the contestant with the best performance wines the first prize, the second one wins the second prize and so on.
% From the theoretical perspective, current investigation on rank-order contests concentrates on the characterization of contestant equilibrium on performance and the optimal contest design with respect to variety of objectives, e.g., the total effort \cite{MS01}, the maximum individual effort \cite{CHS19} or others \cite{AS09,EGG21}. However, these investigations are all based on the assumption that all contestants have a chance to perform in the contest. In reality, the contest holder has to filter part of registered contestants before the formal contest, due to realistic reasons like the limitation on time or space. For example, in Olympics games, only the top athletes (depending on their periodic scores) around the world can enter the final competition, due to the limited schedule. Consider another example of international collegiate programming contest. Since the final is held offline and the space of stadium is limited, only the teams that satisfy some requirements are eligible and can enter the offline competition. The situation of company recruitment is similar. Not all applicants can obtain an opportunity of interview unless the resumes can exceed a certain level. In view of this point, due to the finite resource in reality, the contest designer usually does not allow all registered contestants to perform in the contest and should eliminate part of contestants in advance. Thus, only the contestants in the shortlist can perform to compete for prizes. 

% Roughly speaking, the existence of shortlist can incentive admitted  contestants to impose more effort into contest, since they are  aware that the opponents may be more competitive. On the other hand, eliminating too many contestants also may discourage the contestants' enthusiasm, because they only need to beats less opponents to get a desirable prize. Therefore, how to appropriately decide the size of shortlist is a meaningful problem. To answer this problem, we should deal with the following questions sequentially. Firstly, we should figure out the detailed effect of shortlist on contestants' performance, for any prize structure. If we get this, secondly, how to pairwise decide the size of shortlist and prize structure to optimize different objectives. This is particularly difficult to the total effort objective, since there is a tradeoff between this number of admitted contestants and incentive effect. Intuitively, the smaller this number is, the effort will be greater, but the total effort may not be monotone. Last but not least, how a shortlist affects the designer's objective, i.e., the difference between the optimal contests with a shortlist to that without shortlist.

%Therefore, when there exist a shortlist of contestants, how to design the optimal contest, including the number of shortlist and prize structure, is valuable for theoretical investigation and practical application. Hierarchically, what the contestants in shortlist will perform to reach an equilibrium state; Then, what the optimal contest will be under the objectives of the maximum individual effort and the total effort, respectively; 

\subsection{Main Results}

\noindent \textbf{Model.} 
%\paragraph{Model}
To address the aforementioned questions, we first construct a model of a contest with a shortlist. Consider a rank-order contest with $n \geq 2$ initially registered contestants. Each contestant’s ability, $x_i$, is private information drawn from a publicly known distribution $F(x)$. We assume that the contest designer is aware of all contestants' abilities upon registration. This assumption is reasonable, as designers can assess abilities based on submitted registration materials, such as competition records, CVs, or asset certificates.
The contest proceeds in two stages. First, the designer selects the number of shortlisted contestants, $2 \leq m\leq n$, and determines the prize structure  $\vec{V}=(V_1,V_2, \cdots, V_m) \in \mathbb{R}_{1\times m}$, specifying the rewards for each rank, within a budget $B$. The designer then ranks all $n$ contestants by ability in non-increasing order, admitting the top $m$ contestants to the next round while eliminating the remaining $n-m$ contestants. In the second stage, the shortlisted contestants observe the prize structure and recognize their eligibility. They then exert costly effort to compete for the prizes, with their effort levels influenced by their abilities. Each admitted contestant aims to maximize the utility (which is equal to the obtained prize minus the cost incurred by effort) by choosing the level of effort exerted.
The contest designer's decision variables consist of two components: the number of shortlisted contestants $m$ and the prize structure $\vec{V}$, to optimize two kinds of objectives, the maximum individual effort and the total effort.

%\noindent \textbf{Main Results.} 
Based on this contest-theoretical model, our main results are divided into three parts:
\subsubsection{Posterior distribution and equilibrium effort}
Our first result characterizes contestant equilibrium under any shortlist and prize structure. We use the symmetric Bayesian Nash equilibrium (sBNE) to define the contest’s solution, and can fully characterize the equilibrium effort function. %, as stated in Theorem \ref{thm:contestantSBNE}.

\begin{theorem}[Unique sBNE of Contestants (Sketch)]
    For any ability distribution $F$, any size of shortlist $m$ and any prize structure $\vec{V}$, the unique symmetric Bayesian Nash equilibrium exists and can be expressed in a closed-form.
\end{theorem}


The sBNE is typically derived using the first-order condition of a contestant's expected utility function. A key component of this function is the probability of ranking in each position, which, in traditional rank-order contests, is determined by the prior distribution of abilities. However, in our model, the elimination stage alters contestants' beliefs, i.e., admitted contestants receive a signal of eligibility and update their beliefs about opponents from the prior to a posterior distribution.
Therefore, the main challenge in deriving this equilibrium is accurately computing an admitted contestant’s posterior belief about opponents (see Propositions~\ref{prop:posteriorBeliefs} and Corollary~\ref{prop:marginalBelief} for details). On the other hand, since contestants with different abilities form distinct posterior beliefs, calculating the sBNE in our model is significantly more complex due to the non-identical posterior distributions, compared to the traditional settings.

% The symmetric BNE is commonly derived using the first-order condition of the contestant's expected utility function. In the expected utility function, one key component is the probabilities that one contestant ranks in each position, which is calculated by the prior distribution of ability in traditional rank-order contests. However, in our model, due to the existence of the elimination stage, admitted contestants receive the signal of eligibility and alter beliefs on their opponents from the prior to a posterior distribution. Therefore, the main challenge on obtaining this conclusion is to clearly calculate the admitted contestant's posterior belief on the opponents (see Proposition \ref{prop:posteriorBeliefs} and \ref{prop:marginalBelief} for details). Note that since the different contestants with different abilities will have the different posterior beliefs, it is much complicated to calculate the sBNE by using the non-identical posterior distributions in our model. 

% The main challenge on obtaining this conclusion is to calculate the admitted contestant's posterior belief on the opponents, using it represent the expected utility of each admitted contestant. 

%  To get sBNE, we need to first figure out the contestants' posterior distribution. Additionally, 

% alters contestants' beliefs about their opponents, requiring an update from the prior to a posterior distribution. 

% calculate the admitted contestants' beliefs on opponents. 
%  a contestant's expected utility is directly determined by the prior distribution, and the BNE can be derived using the first-order condition of the utility function. However, in our model, the elimination stage alters contestants' beliefs about their opponents, requiring an update from the prior to a posterior distribution.
%{\color{red} Add more about the challenge on posterior distribution?} To address this, we first derive a closed-form expression for the posterior distribution of opponents' abilities.

%Note that \cite{SSYJ24} follows a similar approach to computing equilibrium effort. However, their calculation of the posterior distribution contains an error, leading much of their subsequent analysis to be based on incorrect results. We correct this mistake and derive a different representation of the symmetric Bayesian Nash equilibrium, which we elaborate on in Section \ref{sec:playerSBNE}. 


% The challenge for the characterization of BNE is that, initially, each contestant regard the distribution $F(a)$ as a prior. After elimination, each eligible contestant receives the ``signal'' of entry and will update $F(a)$ to a posterior distribution, depending on her own ability $a_i$, since it is intuitive that the rest contestants are more competitive. With the help of posterior distribution, we can represent the utility of an eligible contestant by the order statistics 

\subsubsection{Optimal Contest Design}
With a detailed characterization of contestant equilibrium, we next address the problem of the optimal contest design under two common objectives: the maximum individual effort and the total effort. In the traditional rank-order contest design problem, determining the optimal prize structure is already complex due to its high-dimension. In our work, we not only determine the optimal size of shortlist but also design the optimal prize structure within a fixed budget, making the problem more complex. However, we provide a fundamentally different characterization of the optimal contest, as detailed below.
% Having the detailed characterization on contestant equilibrium, we consider the problem of the optimal contest design under two popular kinds of objectives, the maximum individual effort and the total effort. In the traditional rank-order contest design problem, only designing  prize structure is very complicated, since it is a high-dimension variable. In our work, we not only decide on the optimal number of eligible contestants, but also design the optimal prize structure within the budget, which makes our problems much complex. 

%In \cite{SSYJ24}, the authors found that the presence of a shortlist discourages all contestants, leading to lower effort exertion. This suggests that a winner-take-all contest without a shortlist is optimal for both the maximum individual effort and total effort, as supported by \cite{MS01,CHS19}. However, their conclusion is based on an incorrect representation of equilibrium effort. In our work, we provide a fundamentally different characterization of the optimal contest, as detailed below.

Before presenting the detailed description of the optimal contests, we introduce a technical guideline to simplify our analysis. Specifically, we focus on a special type of contest, referred to as a ``simple contest'', where the designer allocates the budget equally across the prizes. Interestingly, we identify two sufficient conditions (see as Proposition \ref{prop:DesignGuideline}) under which the optimal contest is exactly a simple contest. These conditions allow us to reduce the prize structure's dimensionality from $m-1$ to one. First, when the objective is to maximize the effort of a certain rank, the optimal contest is always a simple contest, regardless of the ability distribution or cost function. Second, a simple contest is optimal for any linear combination of contestants' efforts only when the cost function is linear\footnote{A linear cost function is commonly used in both theoretical and practical contexts to characterize direct output in contests. For example, in an all-pay auction—often considered a type of contest—the payment follows a linear cost structure. In a political election setting, the cost can be interpreted as the investment in competition. Additionally, several studies \cite{MS01, AS09, DV09} have examined contests with linear cost functions.}.



% In \cite{SSYJ24}, they found that the existence of shortlist discourage every contestant, i.e., exert less effort, implying that a winner-take-all contest without a shortlist is the optimal \cite{MS01,CHS19} for the objectives of both the maximum individual effort and the total effort. However, this is obtained based on a wrong representation of equilibrium effort. In our work, we get a totally different characterization of the optimal contest, shown as follows.

% Before giving the detailed description of the optimal contests, we propose a technical guideline to simplify our analysis. Generally speaking, we focus on a special kind of contests, call simple contest, where designer divides the budget equally to set prizes. Interestingly, we find two sufficient conditions such that the optimal contests are exactly a simple contest. It helps us decrease the dimension of prize structure from $m-1$ to one. One is that when the objective is the maximum individual effort, for any ability distribution and any cost function, the optimal contest is a simple contest. The other one is that only if the cost function is a linear function, a simple contest is optimal for any objective of linear combination of contestants' efforts. 

\noindent \textbf{Maximum Individual Effort.}
%Unlike the characterization in \cite{SSYJ24}, our findings are entirely opposite. 
Based on our equilibrium effort function, we determine that the optimal contest for maximizing the maximum individual effort admits only two contestants and allocates the entire budget to a single prize, i.e., a two-contestant, winner-take-all contest.
%Different from the characterization in \cite{SSYJ24}, our result is totally opposite. We find that, based on our form of equilibrium effort function, the optimal contest to maximize the maximum individual effort is to admit only two contestants enter the second stage and  set one prize with all budget, i.e., two-contestants the winner-take-all contest. 

\begin{theorem}[Optimal Contest for the Maximum Individual Effort Objective]
    The optimal contest that maximizes the maximum individual effort is a two-contestant winner-take-all contest, where $m=2$ and $V_1=B$.
\end{theorem}

The intuition behind this result is twofold. First, for a given number of contestants, a larger first prize incentivizes greater effort. Second, for a fixed prize structure, admitting an additional contestant surprisingly reduces the expected maximum individual effort. This occurs because with more competitors, the expected prize for each contestant decreases, leading to lower effort to minimize costs.
Combining these insights, we conclude that a two-contestant, winner-take-all contest is optimal.

% The intuition behind it is that, one the one hand, for the same number of admitted contestants, the greater the first prize is, the more effort contestants will exert. On the other hand, for the same prize structure, admitting one more contestant will weaken the expected maximum individual effort, which is a little counter-intuitive. However, it is true since when the prize structure is fixed, more competitors will decrease the expected prize one can get. It leads to that everyone will exert a less effort to decrease the cost.
% Combining these two observations, we can conclude that a two-contestant winner-take-all contest is the optimal.


\noindent \textbf{Total effort.}
%Similar to \cite{SSYJ24}, 
We examine the optimal contest design with a linear cost function to maximize total effort. %However, our result is also different.

\begin{theorem}[Optimal Contest for the Total Effort Objective]
    For any ability distribution $F$, the optimal contest for maximizing the total effort can be described as:
    \begin{enumerate}
        \item The number of prize $l^*$ is equal to the shortlist size $m^*$ minus one, i.e., $l^*=m^*(n)-1$.
        \item The budget is equally divided into these prizes, i.e, $V_1=V_2=\cdots= V_{l^*}=B/l^*$.
        \item The optimal shortlist is proportional to $n$, i.e., $m^*(n)=kn$, where $k$ is the solution to 
        \[
            \int_k^1 F^{-1}(1-q)(\frac{1}{q}-(2k-k^2)\frac{1}{q^2})dq=0.
        \]
    \end{enumerate}
    %is a complete simple contest, where the number of prizes equals the shortlist size minus one, and the budget is equally divided into these prizes.
\end{theorem}

As stated in the guidelines, when the cost function is linear, the optimal contest for any linear combination of contestants' efforts is a simple contest, particularly for the total effort objective. This reduces the design process to determining only the number of prizes and the shortlist size.
Furthermore, we find that the optimal number of prizes equals the shortlist size minus one. The reasoning is twofold. First, a single zero-prize is necessary to incentivize higher effort (see Corollary \ref{coro:Consolation}), as without it, some contestants may exert no effort while still receiving a non-negative prize. Second, having more than one zero-prize is unnecessary (see Corollary \ref{coro:EmptyPrize}). If the prize structure is fixed, admitting an additional contestant increases competition but lowers the probability of any contestant winning a prize. Since effort is costly, contestants will adjust by exerting less effort to balance the reduced expected rewards.
Ultimately, the optimal contest is a complete simple contest. This further simplifies the designer's decision-making from two variables to one, requiring only the selection of the optimal shortlist size.

% As mentioned in guideline, when the cost function is linear, the optimal contest for any linear combination of contestants' efforts is a simple contest, of course for the objective of total effort. It makes us only need to decide the number of prizes and the size of shortlist. Furthermore, we find that the number of prizes is exactly equal to the number of admitted contestants minus one. The main reasons are: on the one hand, one zero-prize is needed to spur more exerted effort (refer to Corollary \ref{coro:EmptyPrize}), because, if not, some contestants may exert zero effort, but always can get a non-negative prize. 
% On the other hand, setting more than one zero-prizes is not necessary.  Intuitively, if we fix a prize structure, admitting one more contestant into shortlist improves the competitiveness and decreases the probabilities that one admitted contestant obtains a certain prize. Since every contestant exerts a costly effort, she will also decrease the effort to balance the loss on the expected prizes. All in all, the optimal contest is a complete simple contest.
% It means that we further decrease the dimension of designer's decision variable from two to one, and only consider the optimal size of shortlist. 

However, the optimal shortlist size heavily depends on the ability distribution. Due to the tradeoff between shortlist size and exerted effort, total effort does not vary monotonically with this size. To derive this result, we first express the total effort objective using a Beta distribution (see Lemma \ref{lem:betaRepTotalEffort}). This representation allows us to obtain an asymptotic form of the objective, enabling the analysis of the optimal shortlist size. In other words, we can design an efficient algorithm to find the optimal shortlist size for any ability distribution. 

In general, we also prove a uniformly upper bound on the optimal size, that is, for any ability distribution, the optimal size does not exceed $0.3162n$, as $n$ goes to infinity and this upper bound is tight achieved by a specific distribution.
\begin{theorem}[Tight Upper Bound for the Optimal Shortlist Size]%\label{thm:UniversalBound}
    For an arbitrary ability distribution $F$, when $n\rightarrow +\infty$, the optimal shortlist size that maximizes ex-ante total effort, denoted by $m^{*}$, has the following linear upper bound with respect to $n$:
    $$ \lim_{n \rightarrow \infty} \frac{m^*(n)}{n} \leq \bar{k},$$
    where $\bar{k} \approx 31.62\%$ is the solution to the equation $\ln k=(2-k)(k-1)$.

    Moreover, there exists a distribution such that $m^*(n)/n=\bar{k}$, meaning the bound is tight. 
\end{theorem}
% \begin{theorem}[Tight Upper Bound for the Optimal Shortlist Size ]
%     For the optimal shortlist size, let $k_n = \sup_{F} \frac{m^*(n)}{n}$ denote the upper bound of $k$ across all ability distributions. We establish that a uniformly tight upper bound of $k^*=0.3162$. Specifically, it has $k^*=\lim_{n\rightarrow +\infty} k_n$, where $k^*$ is the solution to the equation $\ln k=(2-k)(k-1)$.
    
%     % , when $n\rightarrow \infty$, the optimal shortlist size $m^{*}$ that maximizes the total effort satisfies the following linear upper bound with respect to $n$, :
%     % $$ \lim_{n \rightarrow \infty} \frac{m^*(n)}{n} \leq k^*,$$
%     % where $k^* \approx 31.62\%$ is the solution to the equation $\ln k=(2-k)(k-1)$.

%     % Furthermore, there exists a distribution for which $m^*(n)/n=k^*$, proving that this bound is tight. 
% \end{theorem}




\subsubsection{Comparison with the optimal contest without shortlist.}
In addition to the equilibrium analysis and the optimal contest design for different objectives, we also compare the performance of the optimal contest with a shortlist to that without one 

In previous results, when the cost function is linear, the optimal contest without a shortlist is a winner-take-all contest for both the maximum effort objective \cite{CHS19} and the total effort objective \cite{MS01}. Denote by $\mathcal{C}^{(1,n)}$ the $n$-contestant winner-take-all contest. For the maximum individual effort objective, we have shown that the optimal contest with a shortlist is a two-player winner-take-all contest, denoted as $\mathcal{C}^{(1,2)}$. For the total effort objective, although we know the optimal contest is a complete simple contest, the optimal size cannot be uniformly determined and highly depends on the ability distribution. For any ability distribution $F$, let $\mathcal{C}^{(m^*(F,n)-1, m^*(F,n))}$ denote the optimal complete simple contest with a shortlist. 

To compare these contests, we analyze the outputs across three types of contests for any ability distribution: (i) the optimal contest without a shortlist, $\mathcal{C}^{(1,n)}$, achieves $\Theta(1)$ of the maximum individual and total effort; (ii) a two-contestant winner-take-all contest, $\mathcal{C}^{(1,2)}$,  yields $\Theta(\log n)$ of the maximum individual and total effort;  and (iii) the optimal contest with a shortlist, $\mathcal{C}^{(m^*(F,n)-1, m^*(F,n))}$, reaches $\Theta(n)$ in total effort. These effort bounds not only establish the theorem but also highlight the impact of shortlist on the total effort.

% derive this result, we analyze the total effort across three types of contests for any ability distribution: (i) the optimal contest without a shortlist, which achieves $\Theta(1)$, (ii) a two-contestant winner-take-all contest, which yield $\Theta(\log n)$, and (iii) the optimal contest with a shortlist, which reaches $\Theta(n)$. These effort bounds not only establish the theorem but also highlight the impact of shortlist on the total effort.

For the maximum individual effort objective, we define the ratio of maximum individual effort as the gap between $\mathcal{C}^{(1,2)}$ and $\mathcal{C}^{(1,n)}$, and show that this ratio is $\Theta(\log n)$ for any distribution.

\begin{theorem}
    For any ability distribution $F$, under the maximum individual effort objective, $\mathcal{C}^{(1,2)}$ results in $\Theta(\log n)$ times the maximum individual effort of $\mathcal{C}^{(1,n)}$. Specifically,  
    $$
        \frac{\text{ME}(\mathcal{C}^{(1,2)},F)}{\text{ME}(\mathcal{C}^{(1,n)},F)}
        = \Theta(\log n),
    $$
    where $\text{ME}(\cdot)$ represents the maximum individual effort.
    %Moreover, there exists a distribution where the gap is exactly $\Omega(\log n)$, implying that this ratio is tight.
\end{theorem}

%{\color{red} Add intuition on the worst instance?}

%However, for the total effort objective, although we know the optimal contest is a complete simple contest, the optimal size cannot be uniformly determined and highly depends on the ability distribution. For any ability distribution $F$, let $\mathcal{SC}^{(m^*(F,n)-1, m^*(F,n))}$ denote the optimal complete simple contest with a shortlist. 
For the total effort objective, since the optimal contests may vary under different ability distributions, we first fix a specific distribution and then compare the optimal contests under it. Our findings can be summarized as follows.

\begin{theorem}
    Fixed any ability distribution $F$, under the total effort objective, the optimal contest $\mathcal{C}^{(m^*(F,n)-1, m^*(F,n))},F)$ can achieve $\Theta(n)$ times the total effort compared to $\mathcal{C}^{(1,n)}$. Specifically,
    $$
        \frac{\text{TE}(\mathcal{C}^{(m^*(F,n)-1, m^*(F,n))},F)}{\text{TE}(\mathcal{C}^{(1,n)},F)}
         = \Theta(n),
    $$
    where $\text{TE}(\cdot)$ is the total effort and  $m^*(F,n)$ is the optimal shortlist size for the ability distribution $F$. 
    %Moreover, there exists a distribution such that the gap is exactly $\Omega(n)$, implying that this ratio is tight.
\end{theorem}




\subsection{Related Literature}
% \subsubsection{Relevance and difference from \cite{SSYJ24}}
% Since \cite{SSYJ24} and we both consider a similar rank-order contest model with a shortlist, we briefly discuss the similarities and differences with their work. Firstly, \cite{SSYJ24} characterize the contestants' sBNE by updating posterior beliefs about opponents' ability distributions. However, when calculating the posterior distributions and sBNE, they made critical errors on conditions of posterior probabilities, which led to an incorrect representation of the sBNE. We correct the mistakes and derive a different form of sBNE.
% Secondly, due to this erroneous characterization of the symmetric BNE, \cite{SSYJ24} conclude that no shortlist can maximize equilibrium effort, and further suggest that a winner-take-all contest without a shortlist is optimal for maximizing both the maximum individual effort and total effort when the cost function is linear. In contrast, our results on optimal contest design are entirely different. For the maximum individual effort objective, we show that the optimal contest is a two-contestant winner-take-all contest for any cost function. For the total effort objective, the optimal contest is a simple contest with only one zero prize when the cost function is linear, but the optimal number of admitted contestants depends on the ability distribution.
% In addition to these results, we compare the performance of optimal contests with and without a shortlist for both objectives and prove the tight approximation ratios.

%Since \cite{SSYJ24} and we consider the similar rank-order contest model with a shortlist. Herein, we briefly discuss the relevance and difference from \cite{SSYJ24}. Firstly, \cite{SSYJ24} characterize the contestant's symmetric BNE by updating a posterior beliefs on opponents' ability distributions, but they made a mistake during the calculation of posterior distributions, leading to a wrong representation of symmetric BNE. We correct this mistake and obtain a different form of symmetric BNE. Second, due to the mistake characterization of symmetric BNE, \cite{SSYJ24} show that no shortlist will maximize the equilibrium effort, further implying that a winner-take-all contest without a shortlist is the optimal contest to maximize both the maximum individual effort and the total effort, when the cost function is linear. However, we get a totally different results on the optimal contest design. For the objective of the maximum individual effort, we show that the optimal contest is a two-contestant winner-take-all contest for any cost function. For the objective of total effort, the optimal contest is a simple contest with only one prize being zero, when the cost function is linear form, but the optimal number of admitted is not fixed, which highly depends on the ability distribution. Besides these common-focused results, we additionally compare the performance between the optimal contests with and without a shortlist, in term of these two objectives, and prove the tight approximation ratios. 

\subsubsection{Further related works}
Our paper belongs to the field of single contest design, specifically focusing on rank-order contests (also known as all-pay contests). Most papers in this field concentrate on characterizing contestant equilibrium and designing optimal prize structures for various objectives.
In the early period, \cite{GH88} identify the optimal contests for both identical and non-identical contestants based on their abilities, aiming to maximize total effort. \cite{MS01} extend this work by considering cases where contestants' abilities are drawn from a publicly known distribution, and they design the optimal contests for linear, concave, and convex cost functions.
\cite{KG03} study procurement contests in both symmetric and asymmetric settings and propose the corresponding optimal contest designs. \cite{AS09} analyze crowdsourcing contests with a large number of participants, characterizing the asymptotically optimal prize structure to maximize the effort of the top contestants.
\cite{GK16} introduce contests with simple contestants who can strategically choose whether to participate, designing the optimal prize structure to maximize the total ability of participants. \cite{LLWZ18} and \cite{LL23} explore the optimal rank-order contest designs allowing negative prizes.
\cite{CHS19} propose an optimal crowdsourcing contest catering for the maximum effort objective, and \cite{EGG21} study the designer's threshold objective, deriving the optimal contest structures. \cite{G23} investigate the impact of prize structure and ability distributions on contestants' equilibrium efforts. 
%\cite{SSYJ24} consider winner-take-all contest in the setting of contest with a shortlist, but it is not the optimal contest.
%\cite{SSYJ24} consider incorporating entry restriction into contests, and find that it lowers equilibrium effort of all contestants.
\cite{SSYJ24} consider incorporating entry restriction into contests, but do not find the contestant equilibrium and the optimal contest design. 
Additionally, several studies \cite{BKV96, BK98, BKV12} analyze the equilibria of all-pay auctions.
Our work considers a rank-order contest with a shortlist, introducing an elimination stage before effort is exerted, and investigates the optimal contest design for maximizing both the maximum individual effort and total effort, which confirms the positive effect of shortlist for contest designers. 

%In the initial period, \cite{GH88} show the optimal contests for identical case and non-identical case, with respect to the ability of contestants, when the designer aims to maximize the total effort. \cite{MS01} extend the above result and consider the case that the abilities of contestants are drawn from a publicly known distribution. They design the optimal contests for linear, concave and convex cost functions, respectively. \cite{KG03} investigate a procurement contest in symmetric and asymmetric cases and give the corresponding optimal contests. \cite{AS09} consider a crowdsourcing contest with a large number of participants and characterize the asymptotically optimal prize structure, aiming to maximize the summation of several top contestants' effort. \cite{GK16} propose the contest with simple contestants who only strategically join the contest or not, and design the optimal prize structure to maximize the total ability of participants. \cite{LLWZ18} and \cite{LL23} consider the optimal rank-order contest design with allowing the negative prizes.
%\cite{CHS19} propose the optimal crowdsourcing contest with the methodology of auction theory. \cite{EGG21} study the threshold objective of designer, and come up with the optimal contests. \cite{G23} investigates how the prize structure and ability distributions influence contestants’ equilibrium efforts. Besides, there are several works \cite{BKV96,BK98,BKV12} analyzing the equilibria of all-pay auction. Our work considers a rank-order contest setting with a shortlist, where an elimination stage is incurred before exerting effort, and investigates the optimal contest design with respect to the maximum individual effort and total effort. 

In addition, due to the presence of a shortlist, our paper is related to the concepts of elimination and signaling in contests. The seminal paper by \cite{MS06} divides contestants into two sub-contests, where the winners of each sub-contest compete again in the final contest. \cite{FL12} investigate the optimal design of multi-stage Tullock contests \cite{T08}. \cite{LMZ18} introduce information disclosure policies in all-pay contests and compare different disclosure strategies. \cite{LSA19} summarize the intersection between contest design and information disclosure. \cite{MPS21} study all-pay sequential elimination contests in a complete information setting. \cite{FW22} incorporate information disclosure and bias in a two-stage Tullock contest. \cite{R24} examines a multi-stage all-pay contest, where each stage eliminates the contestant with the lowest effort, and analyzes contestants' strategies at each stage.
Additionally, some studies \cite{CKZ17, C24, KZZ24} apply the framework of Bayesian persuasion to contests and investigate optimal strategies for information disclosure.
In contrast to these papers, our contest model introduces an elimination stage via a shortlist. After elimination, contestants receive a signal indicating whether they are admitted, which influences their beliefs about the ability distributions of their opponents. Moreover, while several papers aim to improve contest performance through an elimination stage, the methods discussed in those works tend to be quite sophisticated. In contrast, our model is much simpler—requiring only the addition of a shortlist initially—yet it achieves significantly better results in terms of both maximum individual effort and total effort compared to traditional optimal contests.
% In contrast to these papers, our contest model includes an elimination stage by a shortlist. After elimination, contestants receive a signal indicating whether they are admitted, which influences their beliefs about the ability distributions of their opponents. Additionally, even though there are several papers aiming to improve the performance of contests by an elimination stage,  the elimination methods mentioned before are rather sophisticated. Nevertheless, our model is very simple, only needing to add a shortlist initially, and can also achieve much better on the maximum individual effort and total effort, compared to the traditional optimal contests.  


% In addition, due to the existence of shortlist, our paper is related to elimination and signaling in the contest. The seminal paper \cite{MS06} divides contestants into two sub-contests and the winners in sub-contests compete again in the final contest. \cite{FL12} investigate the optimal design of multi-stage Tullock \cite{T08} contest. \cite{LMZ18} introduce information disclosure policies to all-pay contests and compare different disclosure policies. \cite{LSA19} summarize the intersection between contest design and information disclosure. \cite{MPS21} study all-pay sequential elimination contests in complete information setting. \cite{FW22} take information disclosure and bias into a two-stage Tullock contest. \cite{R24} focuses on a multi-stage all-pay contest, where each stage eliminates the contestant with the lowest effort, and analyzes the contestants' strategies in each stage. Additionally, some literature \cite{CKZ17,C24,KZZ24} incorporate the framework of Basyesian persuasion into contest, and study the optimal strategies of information disclosure. Different from above papers, our contest model consists of an elimination stage based on the contestants' abilities. After elimination, contestants receive a signal about whether they are admitted and affect their beliefs on the ability distributions of opponents.

% \subsection{Roadmap}
% In Section \ref{sec:pre}, we formally build up our model and provide the necessary preliminaries. In Section \ref{sec:playerSBNE}, we investigate the update of the posterior ability distribution and fully characterize the symmetric Bayesian Nash equilibrium. In Section \ref{sec:optimal design}, we derive the optimal contests for both the maximum individual effort and total effort objectives. In Section \ref{sec: compare}, we compare the optimal contest with a shortlist to that without one, for both objectives. In Section \ref{sec:practicalApp}, we introduce how to apply our technique of contest design into reality. Finally, in Section \ref{sec:conclusion}, we summarize our results and propose several directions for future research.
% % In Section \ref{sec:pre}, we elaborate our model and provide necessary preliminaries. In Section \ref{sec:playerSBNE}, we briefly investigate the update of posterior ability distribution and further fully characterize the symmetric Bayesian Nash equilibrium. In Section \ref{sec:optimal design}, we show the optimal contests for the objectives of the maximum individual effort and total effort, respectively. In Section \ref{sec: compare}, we compare the optimal contest with a shortlist to that without one, with respect to two objectives. In Section \ref{sec:conclusion}, we summarize our results and propose several future research directions. 
 
