\section{Conclusion and Future Works}
\label{sec:conclusion}


In this work, we study the optimal design of a rank-order contest with a shortlist for two objectives: the maximum individual effort and total effort. The designer must determine both the shortlist size and the prize structure, with only shortlisted contestants exerting costly effort to compete for prizes.
First, we fully characterize the unique symmetric Bayesian Nash equilibrium of admitted contestants. Next, we provide a detailed characterization of the optimal contests with a shortlist for both objectives. Finally, we compare the performance of optimal contests with and without a shortlist, establishing asymptotically tight bounds.

Several interesting directions remain for future research. First, when there are no restrictions on the cost function, the optimal contest with a shortlist remains unknown for the total effort. Second, investigating the approximation ratio in the general case, particularly for general cost functions, is a valuable problem. Lastly, How do contestant equilibria and optimal contest design, if there multiple contests with a shortlist?

% In this work, we consider the optimal rank-order contest design with a shortlist for two kinds of objectives, the maximum individual effort and the total effort, where the designer should decide not only the shortlist size, but also the prize structure, and only the contestants in the shortlist can exert a costly effort to compete for prizes. Based on this contest setting, we first fully characterize the unique symmetric Bayesian Nash equilibrium of admitted contestants. %, which highly depends on the contestants' posterior beliefs. 
% Secondly, the optimal contests with a shortlist are characterized in detail for two kinds of objectives. Finally, we compare the optimal contests with a shortlist and without a shortlist under two objectives, and the asymptotically tight bounds are given.   

% There are several interesting directions for future research. First, when there is no restriction on cost function, the optimal contest with a shortlist is unknown. Second, exploring the approximation ratio in general case is a worthy problem to be investigated, particularly for general cost function. Last but not least, what the economy will be, if there are multiple contests with a shortlist? How about the contestant equilibrium and the optimal contests?

% This paper investigates a rank-order contest model with a shortlist, where the contest designer first eliminates part of contestants with low abilities and then allows the remaining contestants to compete for prizes. The designer's decision variables include the number of admitted contestants and the prize structure, subject to a fixed budget. Each admitted contestant strategically exerts costly effort to maximize her utility.
% We begin by updating the contestants' beliefs about their opponents and fully characterizing the Bayesian Nash equilibrium for any number of admitted contestants and any prize structure.
% Next, we focus on the optimal contest design for two objectives: maximum individual effort and total effort. For the first objective, we show that a two-contestant winner-take-all contest—where only two contestants are admitted and the entire budget is allocated to a single prize—is always optimal for any ability distribution and cost function. For the second objective, when the cost function is linear, the optimal contest is one in which the budget is equally divided and the number of prizes is equal to the number of admitted contestants minus one. We also prove that the optimal number of admitted contestants is less than 31.62\% of the initial number under any ability distribution.
% Finally, we compare the performance of the optimal contest with and without a shortlist. Under a linear cost function, for the objective of maximum individual effort, the approximation ratio is $O(\log n)$, where $n$ is the number of initial contestants. For the objective of total effort, the approximation ratio is $O(n)$.