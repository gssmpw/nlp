\section{Optimal Contest Design}
\label{sec:optimal design}

In this section, we discuss the optimal contest design under different objectives. In Subsection~\ref{subsec:GeneralGuideline}, we identify a guideline (i.e., two sufficient conditions) such that the optimal contest is a form of simple contest.  In Subsection~\ref{subsec:HighestEffort}, we focus on the maximum individual effort objective, showing that the two-contestant winner-take-all contest is optimal. In Subsection~\ref{subsec:TotalEffort}, we analyze the total effort objective in detail and find that: (a) the optimal contest is a complete simple contest, i.e, the number of prizes is exactly the shortlist size minus one; (b) the optimal shortlist size for a given distribution grows asymptotically linearly with $n$; (c) the optimal shortlist size is no more than $31.62\%\,n$ for any distribution asymptotically.

%In this section, we discuss the optimal contest design under different objectives. In Subsection~\ref{subsec:GeneralGuideline}, we identify that the optimal contest under linear objectives must be a simple contest. Then, we focus on the two most common linear objectives. Section~\ref{subsec:HighestEffort} discusses the maximum individual effort objective, where we find the 2-contestant winner-take-all contest is optimal. In Section~\ref{subsec:TotalEffort}, we analyze the total effort objective in depth, and find out that: (a) The optimal contest is a complete simple contest; (b) The optimal shortlist size of a given distribution grows asymptotically linear with $n$; (c) The optimal size is no more than $31.62\%\,n$ for any distribution asymptotically.

\subsection{General Guideline}\label{subsec:GeneralGuideline}

For a given set of contestants $[n]$ and an ability distribution $F$, the designer aims to determine the shortlist size $2\leq m \leq n$ and and a prize structure $\vec{V}=\{V_1,\ldots,V_m\}$ that satisfies the rank-order property $V_1\geq V_2 \geq\ldots\geq V_m \geq 0$ and the budget constraint $\sum_{l=1}^mV_l \leq B$, with the goal of maximizing the ex-ante effort-based objective under equilibrium. Formally, 

% For a given list of contestants $[n]$ and the common ability distribution $F(x)$, the designer's problem is to decide a shortlist size $2\leq m \leq n$ and a prize structure $\vec{V}=\{V_1,\ldots,V_m\}$ that satisfies rank-order property $V_1\geq V_2 \geq\ldots\geq V_m \geq 0$ and budget constraint $\sum_{l=1}^mV_l \leq B$, such that her ex-ante effort objective is maximized under the equilibrium, formally: 
\[
    \begin{aligned}
        \mathop{\arg \max}_{m,\vec{V}} \quad & \mathbb{E}_{\vec{x} \sim F^n} [ \text{OBJ}(b(x_{(1)}), \ldots,b(x_{(m)}))] \\
        s.t. \quad & \sum_{l=1}^{m} V_l \leq B \\
            & V_l  \geq V_{l+1} \geq 0,
    \end{aligned}
\]
where $x_{(i)}$ denotes the $i^\text{th}$ highest realized ability level, and $\text{OBJ}(\cdot)$ represents the component-wise non-decreasing objective function of the efforts, arranged in decreasing order. %from the $m$ admitted contestants.

To solve this problem, we first establish two key properties of the equilibrium effort from the designer's perspective, which will help reduce the space of decision variables.

% To solve this problem, we begin by showing two useful properties of the equilibrium effort from the perspective of designer, which will help us to reduce the design space.

\begin{corollary}[No Consolation Prize Should be Set]\label{coro:Consolation}
    For any contest that allocates a non-zero prize for the last place, i.e., \(V_m > 0\), setting \( V_m = 0\) increases the equilibrium effort of every contestant.
\end{corollary}

Recall from Remark~\ref{rmk:PrizeGap} that the gap between consecutive prizes motivates contestants. Since a consolation prize makes prizes for higher ranks less attractive, it negatively impacts the effort objectives.
% \begin{remark}
%     Recall from Remark~\ref{rmk:PrizeGap} that the gap between consecutive prizes motivate contestants, since consolation prize make prizes for higher ranks less attractive, it is bad for effort objectives.
% \end{remark}

\begin{corollary}[Empty Prizes Discourage contestants]\label{coro:EmptyPrize}
For a given list of contestant ability $\vec{x}$ nd a prize structure with $k$ non-zero prize, i.e., \( V_1 \geq V_2 \geq ... \geq V_k > V_{k+1} = 0 \), extending the shortlist from $k+1$ to $m$ contestants decreases the equilibrium efforts of all contestants in the former shortlist.
\end{corollary}

From an equilibrium perspective, a contestant perceives herself as less competitive as the contest becomes more crowded (Proposition~\ref{prop:ThreatenDesc}). Consequently, her subjective probability of winning any non-empty prizes decreases when more contestants are admitted (Remark~\ref{rmk:subjectiveProb}). This results in a lower expected payoff, which discourages contestants from exerting more effort.
% \begin{remark}
%     From the perspective of equilibrium, a contestant feels herself less competitive as the contest gets more crowded (Proposition~\ref{prop:ThreatenDesc}), therefore her subjective probability of getting every non-empty prizes decreases when allowing more contestants into the contest (Remark~\ref{rmk:subjectiveProb}). Thus expected payoff also decreases, which discourages contestants from exerting more effort. 
% \end{remark}

Before further discussion, we extend the definition of a Simple Contest \cite{EGG21} to the shortlist setting, a special type of contest that is crucial for effort-based objectives.

\begin{definition}[Simple Contest]
    A contest with $n$ participants and a shortlist size $m$ is called a simple contest if all of its non-zero prizes are equal. Furthermore, if all the prizes in the contest are equal, meaning either $V_i = 0$ for all $i$ or $V_1 = \ldots = V_m \neq 0$, then the contest is referred to as a trivial simple contest.
    %if a simple contest has at lease one non-zero prize and allocates zero prize for the last place, i.e., $\exists l < m, l \in \mathbb{N}$ such that $V_1 = \ldots = V_l > V_{l+1} = 0$, it is a non-trivial simple contest.  [trivial] 
\end{definition}

Note that, in a trivial simple contest, every contestant will exert zero effort in the contest. Additionally, winner-take-all contest, i.e., $V_1=B,$, is a special case of non-trivial simple contest.
% \begin{remark}\label{rmk:TrivialZero}
%     Every contestant exerts zero effort in a trivial simple contest. 
% \end{remark}
% \begin{example}
%     Winner-take-all contests, which is the optimal contest under [Paper] \cite{}, i.e., $V_1=B,2\leq m \leq n$, are a special type of non-trivial simple contest. \textcolor{red}{[Citation]}
% \end{example}

In fact, in contest settings where the designer has the option to create a shortlist, the optimal contest design remains a simple contest for the linear combination of efforts objective, as long as the cost function is linear or only the effort of one certain rank is concerned. %This is formally stated as follows:

% In fact, in the contest setting where the designer has a shortlist option, the optimal contest design is still a simple contest for linear effort objectives, as long as the cost function is linear, or the effort of exactly one position is concerned, which is formally stated as follows:

\begin{proposition}[General Design Guideline]\label{prop:DesignGuideline}
    If the designer's objective is a non-negative, non-zero linear combination of the contestants' effort under equilibrium, denoted by $\vec{e}=(b(x_{(1)}), \ldots,b(x_{(m)}))$ (with contestants re-indexed according to their rankings), either ex-ante (in expectation) or ex-post (for a specific ability level profile of registered contestants), i.e., $u_d=\mathbb{E}_{\vec{x}}[\vec{c}\cdot e(\vec{x})]$ or $u_d(\vec{x})=\vec{c}\cdot e(\vec{x})$, then the optimal contest that maximizes the designer's objective will be a non-trivial simple contest if any of the following conditions are satisfied:
    % If the designer's objective is a non-negative and non-zero linear combination of qualified contestants' effort under the equilibrium, $\vec{e}=(b(x_{(1)}, \ldots,b(x_{(m)}))$ (where contestants are re-indexed according to their rankings), either ex-ante (in expectation) or ex-post (for a concrete ability level profile of registered contestants), i.e., $u_d=\mathbb{E}_{\vec{x}}[\vec{c}\cdot e(\vec{x})]$ or $u_d(\vec{x})=\vec{c}\cdot e(\vec{x})$,  then if any one of the following conditions are satisfied: 
    \begin{enumerate}
        \item The designer only cares about the effort of exactly one ranking, i.e., $c_i>0$ and $\forall \,l\neq i$, $c_l=0$.
        \item The cost function is linear, i.e., $g(e)=ke$ for some $k>0$.
    \end{enumerate}
    % then the optimal contest that maximize the designer's utility must be a non-trivial simple contest.
\end{proposition}

Recall that the difference between consecutive prizes, $(V_l-V_{l+1})$ incentivizes contestants' efforts (Remark~\ref{rmk:PrizeGap}). The above conditions ensure that the designer's objective is linear with respect to the gaps $(V_l-V_{l+1})$. Therefore, we can identify the most profitable gap (the one with the highest weight after adjustments) and allocate the entire budget to enlarging that gap. This results in an optimal simple contest. Additionally, this approach provides an $O(n^2)$ algorithm for finding the optimal contest.

We now focus on two specific types of linear objectives, which are commonly used in practice and have been extensively studied in the literature: the maximum individual effort and total effort.

\subsection{Maximum Individual Effort}\label{subsec:HighestEffort}

Under the maximum individual effort objective, the designer aims to maximize the expected effort of the strongest contestant in equilibrium, i.e., $\mathbb{E}_{x \sim X_{(1)}}[b(x)]$. As a result, the general design guideline (Proposition~\ref{prop:DesignGuideline}) indicates that the optimal contest is a simple contest when the goal is to maximize individual effort. This reduces the design problem to finding the best combination of shortlist size $m$ and the number of prizes $l$. 
On one hand, when the prize structure is fixed, there is no need to admit more than $l+1$ contestants, as additional competition diminishes the strongest contestant's enthusiasm (Corollary~\ref{coro:EmptyPrize}), which implies that the optimal contest satisfies $m=l+1$. On the other hand, intuitively, as more contestants are admitted, the expected prize awarded to the strongest contestant decreases (Proposition~\ref{prop:ThreatenDesc}), leading to a reduction in the effort exerted. 
In summary, a two-contestant winner-take-all contest (i.e., $m=2$ and $l=1$) is the optimal design for maximizing individual effort. Since this analysis holds for any realization of ability levels, we can conclude the following theorem:



% The general design guideline (Proposition~\ref{prop:DesignGuideline}) has shown that the optimal contest is a simple contest, then the decision problem becomes to find the best combination of size $m$ and prize number $l$. When prizes are fixed, there is no need to admit more than $l+1$ contestants, since this extra competition discourages the strongest contestant (Corollary~\ref{coro:EmptyPrize}). Then, remaining candidates are contests with $m$ position and $m-1$ equal prizes. Intuitively, the more contestants are admitted, the less probable that the strongest contestant end up with nothing (getting the last place), we can then further prove that her effort is decreasing with $m$, resulting in the optimal contest with $m=2,l=1$. Since this analysis holds for any realization of ability level, we arrive at the following theorem:

\begin{theorem}[Optimal Contest for the Ex-post Maximum Individual Effort]\label{thm:ExpostHighestEffort}
    The optimal contest that maximizes the ex-post maximum individual effort is a two-contestant winner-take-all contest, where $m=2$ and $V_1=B$.

    The resulting maximum ex-post individual effort is as follows:
    \[
    g^{-1}\left ( B \int_{0}^{x_{1}} \frac{(n-1)f(t)t}{F(t)+(n-1)(1-F(t))} \, dt \right ),
    \]
    where $x_1$ denotes the highest realized ability level among all contestants.
\end{theorem}

Thus, even for the the ex-ante maximum individual effort, the optimality of the two-contestant winner-take-all contest still holds.

\begin{corollary}[Optimal Contest for the Ex-ante Maximum Individual Effort]\label{coro:OptimalMaximumEffort}
    The optimal contest that maximizes the ex-ante maximum individual effort is a two-contestant winner-take-all contest, where $m=2$ and $V_1=B$. $m=2$ and $V_1=B$.

    The resulting maximum ex-ante individual effort is as follows:
    \[
    \mathbb{E}_{x\sim X_{(1)}} \left [ g^{-1}\left ( B \int_{0}^{x} \frac{(n-1)f(t)t}{F(t)+(n-1)(1-F(t))} \, dt \right ) \right ],
    \]
    where random variable $X_{(1)}$ denotes the highest realized ability level among all contestants.
\end{corollary}

\subsection{Total Effort}\label{subsec:TotalEffort}

Under the total effort objective, the designer aims to maximize the expected total effort of all admitted contestants in equilibrium, i.e., $\mathbb{E}_{X_{(1)}, \ldots, X_{(m)}}[\sum_{j=1}^{m}b(x_{(j)})]$.
In this subsection, we assume a linear cost function, $g(e)=ke$ with $k>0$ and show that the optimal contest as follows.
\begin{theorem}[Optimal Contest for the Total Effort Objective]\label{thm: opt for total}
    For any ability distribution $F$, the optimal contest for maximizing the total effort can be described as:
    \begin{enumerate}
        \item The number of prize $l^*$ is equal to the shortlist size $m^*$ minus one, i.e., $l^*=m^*(n)-1$.
        \item The budget is equally divided into these prizes, i.e, $V_1=V_2=\cdots= V_{l^*}=B/l^*$.
        \item The optimal shortlist is proportional to $n$, i.e., $m^*(n)=kn$, where $k$ is the solution to 
        \[
            \int_k^1 F^{-1}(1-q)(\frac{1}{q}-(2k-k^2)\frac{1}{q^2})dq=0.
        \]
    \end{enumerate}
\end{theorem}

To show the optimality of the above characterization, we first define the contests satisfy condition (1) and (2) as a complete simple contest. 
\begin{definition}[Complete Simple Contest]
    A simple contest is complete, if it is a non-trivial simple contest with shortlist size $m$ that has exactly $m-1$ prizes, i.e., $V_1 = \ldots =V_{m-1}>V_m=0$.
\end{definition}

Then, we show that the optimal contest maximizing the total effort is exactly a complete simple contest. By general design guideline, the optimal contest in this case is a simple contest. Moreover, due to linearity, the budget appears as a constant factor in the objective under simple contests and therefore does not affect optimality. To facilitate discussion, we set $B=1$ as a standing assumption.

% Under the total effort objective, the designer seeks to maximize the expected total effort of all admitted contestants in equilibrium, i.e., $\mathbb{E}_{X_{(1)}, \ldots, X_{(m)}}[\sum_{j=1}^{m}b(x_{(j)})]$.  In this subsection, we assume that the cost function is linear, i.e., $g(e)=ke, k>0$. Then, by general design guideline, the optimal contest is a simple contest. Also, by linearity, the budget will appear in the objective as a constant factor under simple contests, therefore will not affect optimality, we then let $B=1$ as a standing assumption to facilitate discussion.  

Using the equilibrium effort expression from Theorem~\ref{thm:contestantSBNE}, the ex-ante total effort of a simple contest with shortlist size $m$ and $l$ equal prizes, denoted by $S(m,n,l)$, is given by:
\[
\mathbb{E}_{X_{(1)}, \ldots, X_{(m)}} \left [\sum_{i=1}^{m}\int_{0}^{x_{(i)}}\frac{\binom{n-1}{l}F^{n-l-1}(t)(1-F(t))^{l}}{\sum_{j=1}^{m}\binom{n-1}{j-1}F^{n-j}(t)(1-F(t))^{j-1}}\frac{f(t)}{1-F(t)} t\, dt \right ],
\]
where $X_{(i)}$ is the $i^{\text{th}}-$highest ability level among all contestants, and $x_{(i)}$ is its realization. 


% Expanding from the equilibrium effort expression (Theorem~\ref{thm:contestantSBNE}), the ex-ante total effort of a simple contest with shortlist size $m$ and $l$ equal prizes, denote by $S(m,n,l)$, is given as:
% \[
% \mathbb{E}_{X_{(1)}, \ldots, X_{(m)}} \left [\sum_{i=1}^{m}\int_{0}^{x_{(i)}}\frac{\binom{n-1}{l}F^{n-l-1}(t)(1-F(t))^{l}}{\sum_{j=1}^{m}\binom{n-1}{j-1}F^{n-j}(t)(1-F(t))^{j-1}}\frac{f(t)}{1-F(t)} t\, dt \right ],
% \]where $X_{(i)}$ is the $i^{\text{th}}$ highest ability level among all contestants and $x_{(i)}$ is its realization. 

% Before further characterizing the optimal contest, we introduce the following definition:


% \begin{example}
%     The optimal contest design under maximum individual effort target, i.e., a 2-contestant winner-take-all contest (Corollary~\ref{coro:OptimalMaximumEffort}), is a complete simple contest.
% \end{example}

In Subsection~\ref{subsec:HighestEffort}, we used Corollary~\ref{coro:EmptyPrize} to show that, when prizes are fixed, admitting more contestants decreases the maximum individual effort. This monotonicity can extend to the total effort. When more contestants are allowed into the contest, the effort of the previously admitted contestants decreases, as their subjective probability of winning declines (Remark~\ref{rmk:subjectiveProb}). On the other hand, the effort contributed by the newcomers cannot compensate for this, as their ability levels are lower. Thus, we have:

\begin{proposition}\label{thm:ConpleteSimpleContest}
    The optimal contest with a shortlist that maximizes total effort is a complete simple contest.
\end{proposition}

% In previous Section~\ref{subsec:HighestEffort}, we use Corollary~\ref{coro:EmptyPrize} to show that when prizes are fixed, admitting more contestants will lower the maximum individual effort. This monotonicity generalizes to the total effort. When more contestants are allowed into the contest, the effort from former qualified contestants decrease since their subjective winning probability declines (Remark~\ref{rmk:subjectiveProb}). On the other hand, effort contributed from the new comers can not compensate for that, since their ability level is lower. Then we have:



The decision problem now reduces to determining the optimal shortlist size $m$ (with the number of prizes being $m-1$ accordingly). We explore additional properties of the optimal contest through asymptotic analysis. First, we rewrite the total effort objective using the beta distribution:
\begin{lemma}[Beta Representation for Total Effort]\label{lem:betaRepTotalEffort}
    The ex-ante total effort in a complete simple contest, denoted by $S(m,n)$ (abbreviated from $S(m,n,m-1)$ when no confusion arises), can be expressed using the beta distribution $\beta(x,a,b)$, as follows:
    \[
    \begin{aligned}
        S(m,n) = & 
        \int_0^1F^{-1}(q)\beta(q,n-m+1,m)\,dq \\
        & +\int_0^1F^{-1}(q)\frac{q}{1-q} \frac{m}{n-m}\frac{\beta(q,n-m,m)}{\int_0^q\beta(x,n-m,m)\,dx}\int_q^1\beta(x,n-m,m+1)\,dx\, dq,
    \end{aligned}
    \] where $\beta(x,a,b) =x^{a-1}(1-x)^{b-1}B(a,b)^{-1}$ is the probability density function of the beta distribution, parametrized by positive integers $a$ and $b$, and $B(a,b)=\frac{\Gamma(a)\Gamma(b)}{\Gamma(a+b)}$ is the beta function.
\end{lemma}

The concentration behavior of the beta distribution\footnote{In general, $\beta(q,\alpha,\beta)$ concentrates around its maximum point $\mu=\frac{\alpha-1}{\alpha+\beta-2}$ as $\alpha$ and $\beta$ go large.} allows us to asymptotically approximate the integration terms in the expression (Lemmas~\ref{lem:1} and~\ref{lem:2}). Thus, the representation simplifies to:
\begin{lemma}[Asymptotic Expression for Total Effort]\label{lem:AsyRep}
    The ex-ante total effort in a complete simple contest has the following asymptotic expression with respect to $n$:
    $$\lim_{n\rightarrow +\infty}\frac{S(m,n)}{n}=\int_0^{1-k}F^{-1}(q)\frac{q}{1-q}\frac{k}{1-k}(\frac{1}{q}-\frac{k}{q(1-q)})dq,$$
    where $k=m/n$, and the convergence rate is independent of $k$. 
\end{lemma}

Following this expression, we treat the admitting ratio $k$ as a decision variable and solve the first-order condition. This leads to the asymptotic optimal size, which is proportional to $n$.

\begin{proposition}[Optimal size is Asymptotically Linear]\label{thm:OptAsmLinear}
The ex-ante total effort in a complete simple contest converges to a function of $k=m/n$ as $n \rightarrow \infty$, and the convergence rate is independent of 
$k$. Therefore, the optimal shortlist size grows asymptotically linearly with $n$. Formally, there exists a $k^*\in(0,1)$ such that:
$$\lim_{n\rightarrow \infty}m^*(n)/n=k^*,$$
where $k^*$ is the solution of the following equation:
\[
\int_k^1 F^{-1}(1-q)(\frac{1}{q}-(2k-k^2)\frac{1}{q^2})dq=0.
\]
\end{proposition}

Proposition~\ref{thm:OptAsmLinear} provides a general way to solve the optimal size for any given distribution, i.e., to solve the equation of first-order condition, as can be seen from following examples. 

\begin{example}[Optimal size for Uniform Distributions]\label{exam:OptimalUniform}
    For a uniform distribution $U[0,b]$ that starting at $0$, i.e., $F(x)=x/b$, then $F^{-1}(1-q)=b(1-q)$. Following Theorem~\ref{thm:OptAsmLinear}, the optimal ratio $k^*$ satisfies:\(\int_k^1 b(1-q)(\frac{1}{q}-(2k-k^2)\frac{1}{q^2})dq=0 \).
    This equation simplifies to $\ln k=1+\frac{4-2k}{k^2-2k-1}$, which can be solved numerically, obtaining $k^* \approx 15.07\%.$
\end{example}
\begin{example}[Optimal size for Square Function Distribution]\label{exam:OptimalSquare}
    For square function distribution, i.e., $F(x) = x^2, x\in[0,1]$, then $F^{-1}(1-q) = (1-q)^{1/a}$. The first-order condition $\sqrt{1-k}(k-4)+(k^2-2k+2)\ln(1+\sqrt{(1-k)})+1/2 \cdot(k^2-2k-2)\ln k =0$ gives that $k^*\approx 20.67\%$. 
    %Sqrt[1 - k] (-4 + k) + (2 - (-2 + k) k) Log[1 + Sqrt[1 - k]] + 1/2 (-2 + (-2 + k) k) Log[k] == 0
\end{example}
\begin{example}[Optimal size for Exponential Distributions]\label{exam:OptimalExp}
    For exponential distribution, i.e., $F(x)=1-e^{-\lambda x}$, then $F^{-1}(1-q)=-\frac{1}{\lambda}\ln q$. The first-order condition $\frac{1}{2\lambda}((\ln k)^2+(4-2k)\ln k+2(k-2)(k-1))= 0$ gives that $k^*\approx 9.70\%$, which is independent of the parameter $\lambda$.
\end{example}

%Example~\ref{exam:OptimalUniform},~\ref{exam:OptimalSquare},~\ref{exam:OptimalExp}. 

In this way, we have fully characterized the optimal contest with a shortlist for the total effort objective. A natural question arises: Is there an upper bound for the optimal admitting ratio? In other words, can the contest designer eliminate a portion of contestants before knowing anything about the ability distribution, while still seeking to maximize total effort? Surprisingly, the answer is yes. Formally, we state the following theorem:

\begin{theorem}[Tight Upper Bound for the Optimal size]\label{thm:UniversalBound}
    For an arbitrary ability distribution $F$, when $n\rightarrow +\infty$, the optimal shortlist size that maximizes ex-ante total effort, denoted by $m^{*}$, has the following linear upper bound with respect to $n$:
    $$ \lim_{n \rightarrow \infty} \frac{m^*(n)}{n} \leq \bar{k},$$
    where $\bar{k} \approx 31.62\%$ is the solution to the equation $\ln k=(2-k)(k-1)$.

    Moreover, there exists a distribution such that $m^*(n)/n=\bar{k}$, meaning the bound is tight. 
\end{theorem}

Interestingly, with no assumptions on the ability distribution, the designer can eliminate up to approximately 68.38\% of the contestants without sacrificing the optimal contest that maximizes total effort. Less competitive contestants contribute little effort, but their presence significantly discourages stronger contestants, as they are perceived to be as competitive as the others. By eliminating weaker contestants, the stronger contestants have more room to fully compete, which increases the total effort and allows the designer to fully leverage their informational advantage.
% \begin{remark}
%     Astonishingly, with no assumption on ability  distribution, the designer can at most eliminate approximately $68.38\%$ of the contestants without missing the optimal contest that maximizes total effort! Less competitive contestants contribute little effort, but their existence greatly discourages stronger contestants, since they are perceived as strong as other contestants. Eliminating weaker contestants leaves room for the stronger to fully compete, raises the total effort, and therefore realizes the information advantage of the designer. 
%     %Even though our proof is asymptotic, numerical result shows that this linear trend actually fits pretty well even for small $m$, as shown in Figure~\ref{fig:universal-a}. 
% \end{remark}
    
%\caption{Comparison of model's self-assessed performance (average implicit reward policy model given to its own roll-outs) and real performance (EX) on Bird development set (Pass@1) during DPO training.}

% \section{Shortlist V.S. Vanilla}
% \label{sec: compare}
% How much can a contest designer benefit from the shortlist option? In this section, we answer this question by comparing the optimal contest design with and without a shortlist, when the cost function is linear. We discusses the maximum individual effort objective and the total effort objective in Subsection~\ref{subsec:ApproxMaxi} and \ref{subsec:ApproxTotal}, respectively.

% Before analyzing the gap between the optimal contests with and without a shortlist, we first rewrite the objective function by the quantile.  
% \subsection{Maximum individual Effort}\label{subsec:ApproxMaxi}

% \begin{lemma}
    
% \end{lemma}

% \begin{lemma}
    
% \end{lemma}

% \begin{theorem}
%     For any ability level distribution, under the total effort objective, the 2-contestant winner-take-all contest gives $\Theta(\log n)$ times the effort of the optimal contest without a shortlist, i.e.:
%     $$\frac{S(2,n,1)}{S(n,n,1)} = \Theta(\log n).$$
% \end{theorem}

% \subsection{Total Effort}\label{subsec:ApproxTotal}

% \begin{lemma}
%     $S(n ,n,1) = \Theta(1).$
% \end{lemma}
    
% \begin{lemma}
%     $S(2,n,1) = \Theta(\log n).$
% \end{lemma}
    
% \begin{lemma}
%     $S(m^*,n, m^*-1) = \Theta(n).$
% \end{lemma}

% \begin{proposition}\label{prop:TotalTWOVAN}
%     For any ability level distribution, under the total effort objective, the 2-contestant winner-take-all contest gives $\Theta(\log n)$ times the effort of the optimal contest without a shortlist, i.e.:
%     $$\frac{S(2,n,1)}{S(n,n,1)} = \Theta(\log n).$$
% \end{proposition}

% \begin{theorem}\label{thm:TotalOPTVAN}
%     For any ability level distribution, under the total effort objective, the optimal contest with a shortlist option results in $\Theta(n)$ times the effort of the optimal contest without a shortlist, i.e.:
%     $$\frac{S(m^*,n,m^*-1)}{S(n,n,1)} = \Theta(n).$$
% \end{theorem}

% \section{Towards Practical Applications}

% In previous sections, we use asymptotic analysis to characterize and solve the optimal contest design, which has a potential application value. In this section, we further provide numerical results to support that our findings fit well even when $n$ is small, stepping towards practical applications.

% In this section, we focus on the total effort objective, which is most widely used in practice.

% \noindent \textbf{Finding the Optimal Contest Design.} The optimal contest is a complete simple contest (Theorem~\ref{thm:ConpleteSimpleContest}), and the corresponding shortlist size is linear to $n$, the slope $k$ is determined by the distribution and can be identified easily by solving an equation (Theorem~\ref{thm:OptAsmLinear}, asymptotic).

% Numerical results show that this linear trend is clear starting from very small $n$, and the asymptotic ratio gives a close prediction. Thus, the optimal contest of any distribution can be find efficiently.

% \begin{figure}[htb]
% \begin{subfigure}[ht]{0.30\textwidth}
%     \centering
%     \includegraphics[width=\textwidth]{figure/AsmUniTex.pdf}
%     \subcaption{$F(x)=x$}
%     \label{fig:disopt-a}
%     \end{subfigure}
% %\hfill
%           % 子图 (b)
% \begin{subfigure}[ht]{0.30\textwidth}
%     \centering
%     \includegraphics[width=\textwidth]{figure/AsmSquareTex.pdf}
%     \subcaption{$F(x)=x^2$}
%     \label{fig:disopt-b}
%     \end{subfigure}
% %\hfill
% \begin{subfigure}[ht]{0.30\textwidth}
%     \centering
%     \includegraphics[width=\textwidth]{figure/AsmExpTex.pdf}
%     \subcaption{$F(x)=1-e^{-x}$}
%     \label{fig:disopt-c}
%     \end{subfigure}

% \caption{The actual optimal size and $m^*$ predicted by asymptotic relation.}
% \label{fig:DistributionOpt}
% \end{figure}
% \noindent \textbf{Universal Upper Bound of Optimal size.} There is no distribution such that the optimal shortlist size is larger than $31.62\%n$ (Theorem~\ref{thm:UniversalBound}, asymptotic).

% We devise an $O(n)$ algorithm to find the supremum of optimal size over all distributions for any given $n$ (Proposition~\ref{prop:SupM}), numerical result shows that the asymptotic linear trend holds for very small $n$, therefore designers can eliminate near $68\%$ of the contestants without any worries. 

% \begin{figure}[h]
%     \centering
%     \includegraphics[width=0.33\textwidth]{figure/UpperBound.pdf}
%     \caption{The upper bound of optimal $m$.}
%     \label{fig:universal}
% \end{figure}

% \noindent \textbf{Performance Enhancement.} For any distribution, the 2-contestant winner-take-all contest is $\Theta(\log n)$ times better than best achievable performance without a shortlist (Proposition~\ref{prop:TotalTWOVAN}), and for the optimal contest with a shortlist, it is $\Theta(n)$ times better than that (Theorem~\ref{thm:TotalOPTVAN}, asymptotic). 

% Numerical results shows that the asymptotic approximation ratio is clear even in small $n$, and the performance of the asymptotic optimal contest is almost the same as the actual optimal contest. This indicates that our algorithm produces an contest design that is almost-optimal and well-performing at small scale, with asymptotic optimality and approximation ratio guaranteed. 



% \begin{figure}[h]
% \begin{subfigure}[ht]{0.30\textwidth}
%     \centering
%     \includegraphics[width=\textwidth]{figure/EffortUniTex.pdf}
%     \subcaption{$F(x)=x$}
%     \label{fig:disopt-a}
%     \end{subfigure}
% %\hfill
%           % 子图 (b)
% \begin{subfigure}[ht]{0.30\textwidth}
%     \centering
%     \includegraphics[width=\textwidth]{figure/EffortSqureTex.pdf}
%     \subcaption{$F(x)=x^2$}
%     \label{fig:disopt-b}
%     \end{subfigure}
% %\hfill
% \begin{subfigure}[ht]{0.30\textwidth}
%     \centering
%     \includegraphics[width=\textwidth]{figure/EffortExpTex.pdf}
%     \subcaption{$F(x)=1-e^{-x}$}
%     \label{fig:disopt-c}
%     \end{subfigure}
    
% \caption{dont know what to say here}
% \label{fig:DistributionOpt}
% \end{figure}

%     Finally, we provide a practice guideline that condenses our findings.

% \noindent \textbf{Contest Design Cheatsheet.} A complete simple contest with no more than $31.62\%$ of contestant admitted. Small scale? YES $\to$ Find optimal size by enumeration in $O(n)$, get $\Theta(n)$ / Use 2-contestant winner-take-all, get $\Theta(\log n)$. NO $\to$ Distribution known? NO $\to$ Admit $31.62\%n$, then get $\Theta(n)$. YES $\to$ Solve  asymptotic slope $k \leq 31.62\%$, admit $kn$, and get almost-optimal performance, $\Theta(n)$.

% % How can we numerically find the worst distribution that reaching the highest optimal size? We rely on another representation of total effort, which help us to somehow decouple the contribution of contest structure itself and the distribution.

% % \begin{lemma}[Quantile Representation for Total Effort]\label{lem:QuantileRep}
% %     By using quantile $q:=1-F(x)$ and its reverse function $v(q):=F^{-1}(1-q)=x$, Ex-ante total effort of a simple contest expresses as:
% %     \[
% %     S(m,n, l)= n\int_0^1|v'(q)|\int_0^qG_{(m,l)}(t)\,dt\,dq,
% %     \]where $l$ is the number of prizes, $G_{(m,l)}(t)=\frac{\binom{n-1}{l}(1-t)^{n-l-1}t^{l-1}}{\sum_{j=1}^{m}\binom{n-1}{j-1}(1-t)^{n-j}t^{j-1}}\int_{0}^{t}\sum_{j=1}^{m}\binom{n-1}{j-1}p^{j-1}(1-p)^{n-j}\,dp$. We use $H_{(m,l)}(q):=\int_0^qG_{(m,l)}(t)\,dt$ to denote the distribution-free part.  
% % \end{lemma}

% % Since Theorem~\ref{thm:ConpleteSimpleContest} states that optimal contest is a complete simple contest, i.e., $l = m-1$, we therefore omit $l$ in the following discussion.

% % In this representation, total effort becomes the integration of the multiplication of a function $|v'(q)|$ determined by ability level distribution, and a function $H_{(m,l)}(q)=\int_0^qG_{(m,l)}(t)\,dt$ that is completely decided by the contest structure.

% % Let us focus on the distribution-free part. We can plot $H_{m}(q)$ as a function of $q\in[0,1]$ for $m = 2,\ldots,n$. The example of $n=10$ is shown in Figure~\ref{fig:universal-b}. In this case, we can see that for some $m$ (e.g., $m=3$), $H_{(m)}(q)>H_{(m+1)}(q)$ holds point-wise, thus, $S(m,n) > S(m+1,n)$ stands true for arbitrary distributions, indicating that $m+1$ is a strictly dominated choice. 

% % Actually, it can be shown that $H_{(m)}(q)/H_{(m')}(q)$ is decreasing with $q$ for $m'>m$ (Lemma~\ref{lem:FracDesc}), then $H_{(m)}(1) > H_{(m')}(1)$ is a suffice and necessary condition for $H_{(m)}(q)>H_{(m')}(q)$ point-wise. On the other hand, if $H_{(m)}(1) < H_{(m')}(1)$, then there exists unique $q' \in(0,1)$ such that $H_{(m)}(q')=H_{(m')}(q')$ and $H_{(m)}(q)<H_{(m')}(q)$ afterwards (e.g., the $q'$ for $m=2$ and $m'=3$ is marked with asterisk in Figure~\ref{fig:universal-b}). Since $v'(q)$ can be any positive function, we can always construct a distribution that satisfies $v'(q)=1$ when $q\geq q'$ and $v'(q)=\epsilon$ elsewhere such that $S(m,n) < S(m',n)$ (See an example distribution that make $m'=3$ better than $m=2$ in Figure~\ref{fig:universal-c}, where we let $q' \approx 0.859$, $\epsilon=0.01$, and $S(2,10) \approx 0.481 < 0.489 \approx S(3,10)$.).

% % Therefore, we can find the $m$ that maximize $H_{m}(1)$. For $m'>m$, we have $H_{(m)}(q) > H_{(m')}(q)$ point-wise, so the optimal size can not be more than $m$. For $m'<m$, we have $H_{(m')}(1) < H_{(m)}(1)$, then we can still construct a distribution that satisfies $v'(q)=1$ when $q \geq \max\{\vec{q'}\}$ and $v'(q) = \epsilon$ elsewhere such that $S(m,n) > S(m',n)$ for all $m'<m$, hence we find an instance making $m$ the optimal size. We then conclude that $m$ is the tight upper bound for optimal size for given $n$, as desired.

% % \begin{remark}
% %     The insight from the construction of worst case distribution (e.g., Figure~\ref{fig:universal-c}) is, when almost all of the population are concentrated near the strongest end of ability level, it tends to need larger shortlist size to reach optimality. On the other hand, if highest ability only takes up a little probability mass, or equivalently, $|v'(q)|$ is much higher when $q$ is small, it tends to obtain optimality with fewer contestants. [Mathematical insight]. An uniform distribution, i.e., $|v'(q)|=1$, whose probability mass is evenly distributed, is right in the middle, with $k^*\approx15\%$, as shown in Example~\ref{exam:OptimalUniform}. 
% % \end{remark}

% % \begin{proposition}[Optimal size for Exponential Distribution]\label{prop:OptCapExp}
% %     For exponential distribution, i.e., $F(x)=1-e^{-\lambda x}$ and $f(x)=\lambda e^{-\lambda x}$, when $n \rightarrow \infty$ it holds that:
% %     \[
% %     \lim_{n \rightarrow \infty} m^*(n)/n = 9.70\%,
% %     \]which is independent of the parameter $\lambda$. 
% % \end{proposition}

% % \begin{proposition}[Optimal size for Power-Law Distribution]\label{prop:OptCapPowerLaw}
% %     For power-law distribution $f(x)=cx^{-\alpha-1}, x\ge \delta$ that parametrized by $\alpha \in(0,1]$ and $\delta > 0$. When $n \rightarrow \infty$, $m^*=2$. 
% % \end{proposition}
% % \begin{remark}
% %     Actually, the power-law distribution does not always achieve optimality at extremely small values of \( m \). When \(\alpha > 1\), \( S'(0) \rightarrow +\infty \), and \( S''(k) \) is initially negative and then positive. Consequently, \( S'(k) \) first decreases from \( +\infty \), then increases, eventually reaching \( S'(1) = 0 \). Thus, there exists a point in the interval \( (0,1) \) where \( S'(k) = 0 \), at which \( S \) attains its maximum value. From the condition \( S'(k) = 0 \), it follows that the optimal solution \( k \) satisfies the equation:
% % \[
% % \ln \frac{\alpha+(1-k)^2}{\alpha-(1-k)} + \frac{1}{\alpha}\ln k = 0.
% % \]

% % For large values of \(\alpha\), the term \(\ln \frac{\alpha+(1-k)^2}{\alpha-(1-k)}\) can be approximated as \(\ln \left(1+ \frac{(1-k)+(1-k)^2}{\alpha-(1-k)}\right) \simeq \frac{(1-k)(2-k)}{\alpha}\). Therefore, the equation simplifies to:
% % \[
% % \frac{(1-k)(2-k)}{\alpha} + \frac{1}{\alpha}\ln k = 0.
% % \]

% % The solution to this equation is approximately \( k_2 \approx 31.65\% \), which reaches the worst ratio given by Theorem~\ref{thm:UniversalBound}. 
% % \end{remark}

% %\subsection*{Hello} this is a subsection

