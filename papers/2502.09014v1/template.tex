\documentclass{article}



\usepackage{arxiv}

\usepackage[utf8]{inputenc} % allow utf-8 input
\usepackage[T1]{fontenc}    % use 8-bit T1 fonts
\usepackage{hyperref}       % hyperlinks
\usepackage{url}            % simple URL typesetting
\usepackage{booktabs}       % professional-quality tables
\usepackage{amsfonts}       % blackboard math symbols
\usepackage{nicefrac}       % compact symbols for 1/2, etc.
\usepackage{microtype}      % microtypography
\usepackage{lipsum}		% Can be removed after putting your text content
\usepackage{graphicx}
\usepackage{natbib}
\usepackage{doi}

\usepackage{amsmath}
\usepackage{amsthm}
\usepackage{subcaption}

% 定义定理环境
\newtheorem{theorem}{Theorem}  % [section] 表示定理编号与章节相关
\newtheorem{lemma}{Lemma}      % 引理与定理共享编号
\newtheorem{proposition}{Proposition}
\newtheorem{corollary}{Corollary}
\newtheorem{definition}{Definition}

% 定义备注和例子环境
\newtheorem{remark}{Remark}     % 备注
\newtheorem{example}{Example}    % 例子


%\title{A template for the \emph{arxiv} style}
%\title{Less is More: Optimal Contest Design with a Shortlist}
\title{Optimal Contest Design with Entry Restriction}


%\date{September 9, 1985}	% Here you can change the date presented in the paper title
\date{} 					% Or removing it

% \author{ \href{https://orcid.org/0000-0000-0000-0000}{\includegraphics[scale=0.06]{orcid.pdf}\hspace{1mm}David S.~Hippocampus}\thanks{Use footnote for providing further
% 		information about author (webpage, alternative
% 		address)---\emph{not} for acknowledging funding agencies.} \\
% 	Department of Computer Science\\
% 	Cranberry-Lemon University\\
% 	Pittsburgh, PA 15213 \\
% 	\texttt{hippo@cs.cranberry-lemon.edu} \\
% 	%% examples of more authors
% 	\And
% 	\href{https://orcid.org/0000-0000-0000-0000}{\includegraphics[scale=0.06]{orcid.pdf}\hspace{1mm}Elias D.~Striatum} \\
% 	Department of Electrical Engineering\\
% 	Mount-Sheikh University\\
% 	Santa Narimana, Levand \\
% 	\texttt{stariate@ee.mount-sheikh.edu} \\
% 	%% \AND
% 	%% Coauthor \\
% 	%% Affiliation \\
% 	%% Address \\
% 	%% \texttt{email} \\
% 	%% \And
% 	%% Coauthor \\
% 	%% Affiliation \\
% 	%% Address \\
% 	%% \texttt{email} \\
% 	%% \And
% 	%% Coauthor \\
% 	%% Affiliation \\
% 	%% Address \\
% 	%% \texttt{email} \\
% }

\author{ Hanbing Liu\\
	Gaoling School of Artificial Intelligence\\
	Renmin University of China\\
	Beijing, China\\
	\texttt{liuhanbing@ruc.edu.cn} \\
	%% examples of more authors
	\And
	Ningyuan Li \\
	CFCS, School of Computer Science\\
	Peking University\\
	Beijing, China \\
	\texttt{liningyuan@pku.edu.cn}
    \AND
    Weian Li \\
	School of Software \\
	Shandong University \\
	Shandong, China \\
	\texttt{weian.li@sdu.edu.cn}
    \And
    Qi Qi\thanks{Corresponding author.} \\
	Gaoling School of Artificial Intelligence \\
	Renmin University of China\\
	Beijing, China \\
	\texttt{qi.qi@ruc.edu.cn} \\
    \And
    Changyuan Yu \\
    %\phantom{h}    \\
	Baidu Inc. \\
	Beijing, China \\
	\texttt{yuchangyuan@baidu.com}
}

% Uncomment to remove the date
%\date{}

% Uncomment to override  the `A preprint' in the header
%\renewcommand{\headeright}{Technical Report}
%\renewcommand{\undertitle}{Technical Report}
%\renewcommand{\shorttitle}{\textit{arXiv} Template}


%%% Add PDF metadata to help others organize their library
%%% Once the PDF is generated, you can check the metadata with
%%% $ pdfinfo template.pdf
% \hypersetup{
% pdftitle={A template for the arxiv style},
% pdfsubject={q-bio.NC, q-bio.QM},
% pdfauthor={David S.~Hippocampus, Elias D.~Striatum},
% pdfkeywords={First keyword, Second keyword, More},
% }

\begin{document}
\maketitle

% \begin{abstract}
% 	%\lipsum[1]
% This paper examines the optimal design of contests with $n$ contestants, where the designer determines both the number of admitted contestants in a shortlist and the prize structure under a fixed budget. We fully characterize the contestant's symmetric Bayesian Nash equilibrium and design the optimal contests for two objectives: the maximum individual effort and the total effort.
% For the maximum individual effort objective, the optimal contest is remarkably simple: a two-contestant winner-take-all contest, where only two contestants are shortlisted, and a single prize is awarded. For the total effort objective, the optimal contest admits $m=kn$ contestants and distributes the budget equally among the top $m-1$ contestants. 
% We also present a general algorithm to determine the optimal $k$ for any ability distribution (e.g., $k=0.09$ for exponential distribution and $k=0.15$ for the uniform distribution), which has practical applications. Additionally, we establish a tight upper bound of $\bar{k}=0.3162$ for any distribution.
% Finally, we compare total effort across different contest configurations: the optimal contest without a shortlist achieves $\Theta(1)$, a two-contestant winner-take-all contest yields $\Theta(\log n)$, and the optimal contest with a shortlist reaches $\Theta(n)$. This demonstrates the significant advantage of incorporating a shortlist.
% \end{abstract}

\begin{abstract}
    This paper explores the design of contests involving \( n \) contestants, focusing on how the designer decides on the number of contestants allowed and the prize structure with a fixed budget. We characterize the unique symmetric Bayesian Nash equilibrium of contestants and find the optimal contests design for the maximum individual effort objective and the total effort objective.
\end{abstract}

%\keywords{Contest Design \and Shortlist \and Contestant Equilibrium \and Tight Bounds}
\section{Introduction}
Implicit Neural Representations (INRs), which fit the target function using only input coordinates, have recently gained significant attention.
%
By leveraging the powerful fitting capability of Multilayer Perceptrons (MLPs), INRs can implicitly represent the target function without requiring their analytical expressions. 
%
The versatility of MLPs allows INRs to be applied in various fields, including inverse graphics~\citep{mildenhall2021nerf, barron2023zip, martin2021nerf}, image super-resolution~\citep{chen2021learning, yuan2022sobolev, gao2023implicit}, 
image generation~\citep{skorokhodov2021adversarial}, and more~\citep{chen2021nerv, strumpler2022implicit, shue20233d}.
%
\begin{figure}
    \includegraphics[width=0.5\textwidth]{Image/Fig2.pdf}
    \caption{As illustrated at the circled blue regions and green regions, it can be observed that even with well-chosen standard deviation/scale, as experimented in \autoref{figure:combined}, the results are still unsatisfactory. However, using our proposed method, the noise is significantly alleviated while further enhancing the high-frequency details.}
    \label{fig:var}
    \vspace{-10pt}
\end{figure}

\begin{figure*}[!ht]
    \centering
    \begin{minipage}[b]{0.25\textwidth}
        \centering
        \includegraphics[width=1.\textwidth]{Image/fig_cropped.pdf} % 替换为你的小图文件
        \label{figure:small_image}
        \vspace{-20pt}
    \end{minipage}%
    \hfill
    \begin{minipage}[b]{0.75\textwidth}
        \centering
        \includegraphics[width=1.\textwidth]{Image/psnr_trends_rff_pe_simplified.pdf} % 替换为你的大图文件
        \vspace{-20pt}
        \label{figure:large_image}
        
    \end{minipage}
    \caption{We test the performance of MLPs with Random Fourier Features (RFF) and MLPs with Positional Encoding (PE) on a 1024-resolution image to better distinguish between high- and low-frequency regions, as demonstrated on the left-hand side of this figure. We find that the performance of MLPs+RFF degrades rapidly with increasing standard deviation compared with MLPs+PE. Since positional encoding is deterministic, scale=512 can be considered to have standard deviation around 121.}
    \label{figure:combined}
    \vspace{-10pt}
\end{figure*}
Varying the sampling standard deviation/scale may lead to degradation results, as shown in \autoref{figure:combined}.
%
However, MLPs face a significant challenge known as the spectral bias, where low-frequency signals are typically favored during training~\citep{rahaman2019spectral}. 
A common solution is to map coordinates into the frequency domain using Fourier features, such as Random Fourier Features and Positional Encoding, which can be understood as manually set high-frequency correspondence prior to accelerating the learning of high-frequency targets.~\citep{tancik2020fourier}. 
This embeddings widely applied to the INRs for novel view synthesis~\citep{mildenhall2021nerf,barron2021mip}, dynamic scene reconstruction~\citep{pumarola2021d}, object tracking~\citep{wang2023tracking}, and medical imaging~\citep{corona2022mednerf}.
% \begin{figure}[!h]
%     \centering
%     \includegraphics[width=1.\textwidth]{Image/psnr_trends_rff_pe_simplified.pdf}
%     \caption{This figure shows the change of PSNR on the whole, low-frequency region, and high-frequency region of the image fitting by using two Fourier Features Embedding with varying scale of variance: (Right) Positional Encoding (PE) (Left) Random Fourier Features (RFF). Both PE and RFF will degrade the low-frequency regions of the target image when variance increases.}
%     \vspace{-20pt} 
%     \label{figure:stats}
% \end{figure}


Although many INRs' downstream application scenarios use this encoding type, it has certain limitations when applied to specific tasks.
%
It depends heavily on two key hyperparameters: the sampling standard deviation/scale (available sampling range of frequencies) and the number of samples.
%
Even with a proper choice of sampling standard deviation/scale, the output remains unsatisfactory, as shown in \autoref{fig:var}: Noisy low-frequency regions and degraded high-frequency regions persist with well chosen sampling standard deviation/scale with the grid-searched standard deviation/scale, which may potentially affect the performance of the downstream applications resulting in noisy or coarse output.
%
However, limited research has contributed to explaining the reason and finding a proper frequency embeddings for input~\citep{landgraf2022pins, yuce2022structured}.

In this paper, we aim to offer a potential explanation for the high-frequency noise and propose an effective solution to the inherent drawbacks of Fourier feature embeddings for INRs.
%
Firstly, we hypothesize that the noisy output arises from the interaction between Fourier feature embeddings and multi-layer perceptrons (MLPs). We argue that these two elements can enhance each other's representation capabilities when combined. However, this combination also introduces the inherent properties of the Fourier series into the MLPs.
%
To support our hypothesis, we propose a simple theorem stating that the unsampled frequency components of the embeddings establish a lower bound on the expected performance. This underpins our hypothesis, as the primary fitting error in finitely sampled Fourier series originates from these unsampled frequencies.

Inspired by the analysis of noisy output and the properties of Fourier series expansion, we propose an approach to address this issue by enabling INRs to adaptively filter out unnecessary high-frequency components in low-frequency regions while enriching the input frequencies of the embeddings if possible.
%
To achieve this, we employ bias-free (additive term-free) MLPs. These MLPs function as adaptive linear filters due to their strictly linear and scale-invariant properties~\citep{mohan2019robust}, which preserves the input pattern through each activation layer and potentially enhances the expressive capability of the embeddings.
%
Moreover, by viewing the learning rate of the proposed filter and INRs as a dynamically balancing problem, we introduce a custom line-search algorithm to adjust the learning rate during training. This algorithm tackles an optimization problem to approximate a global minimum solution. Integrating these approaches leads to significant performance improvements in both low-frequency and high-frequency regions, as demonstrated in the comparison shown in \autoref{fig:var}.
%
Finally, to evaluate the performance of the proposed method, we test it on various INRs tasks and compare it with state-of-the-art models, including BACON~\citep{lindell2022bacon}, SIREN~\citep{sitzmann2020implicit}, GAUSS~\citep{ramasinghe2022beyond} and WIRE~\citep{saragadam2023wire}. 
The experimental results prove that our approach enables MLPs to capture finer details via Fourier Features while effectively reducing high-frequency noise without causing oversmoothness.
%
To summarize, the following are the main contributions of this work:
\begin{itemize}
    \item From the perspective of Fourier features embeddings and MLPs, we hypothesize that the representation capacity of their combination is also the combination of their strengths and limitations. A simple lemma offers partial validation of this hypothesis.

    
    \item  We propose a method that employs a bias-free MLP as an adaptive linear filter to suppress unnecessary high frequencies. Additionally, a custom line-search algorithm is introduced to dynamically optimize the learning rate, achieving a balance between the filter and INRs modules.

    \item To validate our approach, we conduct extensive experiments across a variety of tasks, including image regression, 3D shape regression, and inverse graphics. These experiments demonstrate the effectiveness of our method in significantly reducing noisy outputs while avoiding the common issue of excessive smoothing.
\end{itemize}



\documentclass[11pt]{article}

\usepackage{amsmath}
\usepackage{amssymb}
\usepackage{amsthm}
\usepackage{graphicx}
\usepackage{enumerate}
\usepackage{theoremref}
%you can add more packages using the same code above
\usepackage{spconf}

\usepackage{caption}
\usepackage{bm}
%\usepackage{natbib}
\usepackage[dvipsnames]{xcolor}
\usepackage{caption}
\usepackage[font=small,labelfont=bf]{caption}
\usepackage{subcaption}
\usepackage{algorithmic}
\usepackage{algorithm}
\usepackage[hidelinks]{hyperref}
\usepackage{cleveref}
\usepackage{todonotes}
%\usepackage{subfigure}
\usepackage{subcaption}
\usepackage{titlesec}
\usepackage{setspace}

%------------------
 
\newtheorem{theorem}{Theorem}[section]
\newtheorem{proposition}[theorem]{Proposition}
\newtheorem{lemma}[theorem]{Lemma}
\newtheorem{corollary}[theorem]{Corollary}

\newtheorem{question}[theorem]{Question}
\newtheorem{remark}[theorem]{Remark}
\newtheorem{conjecture}{Conjecture}
\theoremstyle{definition}
\newtheorem{definition}{Definition}
\newtheorem*{example}{Example}

\graphicspath{{fig/}}

\def\fro{\textnormal{F}}
\def\circ{\textnormal{circ}}
\def\diag{\textnormal{diag}}
\def\ifft{\textnormal{ifft}}
\def\fft{\textnormal{fft}}
\def\reshape{\textnormal{resh}}
\newcommand{\lx}[1]{\textcolor{red}{#1 }}
 
%------------------

%Everything before begin document is called the pre-amble and sets out how the document will look
%It is recommended you don't touch the pre-amble until you are familiar with LateX
%\titlespacing*{\subsection}{0pt}{10pt}{5pt}

\setstretch{0.9}
\begin{document}

\title{Three-dimensional  Signal Processing: A New Approach in Dynamical Sampling via Tensor Products}%{Dynamical Sampling in Multidimensional Signal Recovery}
\name{Yisen Wang$^{1}$, HanQin Cai$^{2}$,  Longxiu Huang$^{1}$}

\address{$^{1}$ Michigan State University\\$^{2}$ University of Central Florida
}
% \name{Xia Li, Longxiu Huang, Deanna Needell}
% \address{University of California, Los Angeles}
\date{}
\maketitle

\begin{abstract}
The dynamical sampling problem is centered around reconstructing signals that evolve over time according to a dynamical process, from spatial-temporal samples that may be noisy. This topic has been thoroughly explored for one-dimensional signals. Multidimensional signal recovery has also been studied, but primarily in scenarios where the driving operator is a convolution operator. In this work, we shift our focus to the dynamical sampling problem in the context of three-dimensional signal recovery, where the evolution system can be characterized by tensor products. Specifically, we provide a necessary condition for the sampling set that ensures successful recovery of the three-dimensional signal. Furthermore, we reformulate the reconstruction problem as an optimization task, which can be solved efficiently.  To demonstrate the effectiveness of our approach, we include some straightforward numerical simulations that showcase the reconstruction performance.
\end{abstract}


 
\section{Introduction}
The concept of dynamical sampling, first introduced in the works of \cite{lu2009spatial}, %lu2011localization, ranieri2011sampling, hormati2009distributed}, 
addresses the challenge of compensating for spatial sampling deficiencies by leveraging the temporal evolution of data during recovery \cite{aldroubi2013dynamical}. This method utilizes the time-dependent nature of signal evolution, driven by external forces, to enhance the quality of the collected samples, % \cite{huang2021robust}, 
setting it apart from traditional static sampling.

Dynamical sampling allows for efficient data acquisition by sampling only a subset of data points in time and space. This is particularly useful in systems where collecting data from every point in space or at every moment in time is either costly or impractical. For instance, in large sensor networks or medical imaging, dynamical sampling can reduce the number of measurements needed while still allowing accurate reconstruction of the full signal.

Dynamical sampling has been extensively explored for one-dimensional signals, with studies like \cite{aldroubi2017dynamical, aldroubi2019frames, aldroubi2017iterative,huang2024robust}.  Particularly, for the scenario where the evolution operator is represented by a matrix \(A \in \mathbb{C}^{d \times d}\) and the signal, \(f \in \mathbb{C}^d\), is to be recovered,  \cite{aldroubi2017dynamical} provides necessary and/or sufficient conditions on the sampling set of indices \(\Omega \subseteq \{1, 2, \ldots, d\}\) and  the numbers $\{\ell_i\}_{i\in\Omega}$ such that $f\in\mathbb{C}^d$ can be recovered from the samples $\{A^jf(i):i\in\Omega,j=0,\cdots,\ell_i-1\}$. 

Although one-dimensional signals have been extensively studied, research on multi-dimensional signals remains relatively limited.  However, in industrial applications, the observed time-varying signals often involve multiple variables, highlighting the critical importance of studying multi-dimensional dynamical sampling. For example, in sensor networks used for environmental monitoring or industrial processes, data such as temperature, pressure, and humidity are collected over time across various spatial locations, forming a three-dimensional tensor \cite{vairamani2013}. Each dimension can represent spatial coordinates and time, highlighting the complexity of the data. To date, the primary research in this area has focused on signals evolving under convolution-driven operators in multi-dimensional settings \cite{aceska2015multidimensional}. This underscores a significant gap in the literature and highlights the need for further investigation into multi-dimensional dynamical sampling. %Dynamical sampling provides an efficient, scalable, and flexible framework for acquiring and processing signals in systems that evolve over time. Its ability to adapt to multi-dimensional data and time-varying processes makes it an essential tool in many modern applications, from industrial monitoring to medical imaging and beyond.

%Our work mainly focuses on reconstructing the initial signal. In signal recovery or reconstruction problems, particularly in dynamical sampling, the initial signal plays a key role. When trying to recover a full signal from sparse measurements, knowledge of the initial signal can dramatically improve the accuracy of reconstruction. This is because the initial signal often contains critical information about the structure and features of the underlying signal, which helps to "guide" the recovery process.




In this work, we explore the dynamical sampling problem where the initial signal \( \mathcal{F} \) is in \( \mathbb{C}^{m \times p \times n} \), and evolves over time driven by the t-product of the tensor \( \mathcal{A} \in \mathbb{C}^{m \times m \times n} \).  Specifically,  the signal at time \( t \) is transformed according to:
\begin{equation}\label{eqn:evolution rule}
    \mathcal{F}_t = \mathcal{A}^t \ast \mathcal{F},
\end{equation}
where \( \ast \) denotes the t-product between two tensors \cite{kilmer2011factorization}, and \( \mathcal{A}^t \) represents the \( t \)-th power of \( \mathcal{A} \) under the t-product. We consider the spatio-temporal sampling data represented by the set \( \Psi = \{\mathcal{F}_t(i,j,k) : (i,j,k) \in \Omega\subseteq[m]\times[p]\times[n], t \in \{0\}\cup[T-1]\} \) with $[m]=\{1,2,\cdots, m\}$. The objective of this study is to identify the conditions on \( \Omega \) and \( T \) necessary to guarantee the reconstruction of the initial signal \( \mathcal{F} \) from \( \Psi \), and to formulate the reconstruction of \( \mathcal{F}\) as an optimization problem that can be efficiently solved.
 
    \subsection{Contributions}
Our main contributions are as follows:
\begin{itemize}
    \item We have established a necessary condition on $\Omega$ for the successful recovery of the initial three-dimensional signal from the given samples.
    \item We have transformed the reconstruction of the three-dimensional signal $\mathcal{F}$ from spatio-temporal samples into $p$ independent optimization problems.
    \item We have conducted several experiments to demonstrate the effectiveness of our method, determine the optimal total sampling time 
$T$, and verify our conjecture.
\end{itemize}






\section{Preliminaries}
In this section, we introduce the mathematical notations and concepts required for our study, with a focus on the t-product and other tensor operations.
\begin{definition}[t-product] The t-product of tensors $\mathcal{T}_1\in\mathbb{C}^{m\times p\times n}$ and  $\mathcal{T}_2\in\mathbb{R}^{p\times q\times n}$ is denoted by $\mathcal{T}_1\ast\mathcal{T}_2=:\mathcal{T}\in\mathbb{C}^{m\times q\times n}$  and can be defined by the following steps:\\
    \begin{itemize}
    
\item  $\widehat {\mathcal{T}_1 }= \fft(\mathcal{T}_1,[], 3)$, $\widehat {\mathcal{T}_2 }= \fft(\mathcal{T}_2,[], 3)$ 
\item $\widehat {\mathcal{T}} (:,:,k)$=$\widehat {\mathcal{T}_1} (:,:,k)\widehat {\mathcal{T}_2} (:,:,k)$
\item $\mathcal{T}:=\mathcal{T}_1\ast \mathcal{T}_2=\ifft(\widehat{\mathcal{T}},[],3)$.
\end{itemize}
\end{definition}
Apart from t-product, we also involve other products between the tensors for the signal recovery.
\begin{definition}\label{def:others}
Other products between tensor:
\begin{itemize}
    \item  Element-wise tensor product $\odot$: $\mathcal{T}=\mathcal{T}_1\odot \mathcal{T}_2$ for $\mathcal{T}, \mathcal{T}_1, \mathcal{T}_2 \in \mathbb{C}^{m\times p\times n}$, with $[\mathcal{T}]_{i,j,k}=[\mathcal{T}_1]_{i,j,k}[\mathcal{T}_2]_{i,j,k}$.
    \item  Tube-wise circular convolution $\circledast$: $\mathcal{T}=\mathcal{T}_1\circledast \mathcal{T}_2$ for $\mathcal{T}, \mathcal{T}_1, \mathcal{T}_2 \in \mathbb{C}^{m\times p\times n}$, with $[\mathcal{T}]_{i,j,:}=[\mathcal{T}_1]_{i,j,:}\ast [\mathcal{T}_2]_{i,j,:}$.
    \item  Frontal-slice-wise product $\bigtriangleup$: $\mathcal{T}=\mathcal{T}_1\bigtriangleup \mathcal{T}_2$ for $\mathcal{T}_1 \in \mathbb{C}^{m\times n\times p}$, $\mathcal{T}_2 \in \mathbb{C}^{n\times s\times p}$, $\mathcal{T} \in \mathbb{C}^{m\times s\times p}$, with $[\mathcal{T}]_{:,:,k}=[\mathcal{T}_1]_{:,:,k}[\mathcal{T}_2]_{:,:,k}$.
\end{itemize}
\end{definition}

\section{Main results}
\subsection{Necessary condition}
We have initially focused on the sampling set \(\Omega\) structured in a lattice form, specifically \(\Omega = I \times J \times [n]\), where \(I \subseteq [m]\) and \(J \subseteq [p]\). In this configuration, we present the following result:

\begin{theorem}
   Suppose \( \mathcal{F} \in \mathbb{C}^{m \times p \times n} \) and \( \mathcal{A} \in \mathbb{C}^{m \times m \times n} \). And suppose that the signal at time \( t \) follows the transformation specified in \eqref{eqn:evolution rule}. Then, the recovery of \( \mathcal{F} \) from \( \Psi \), with \( \Omega = I \times J \times [n] \) where \( I \subseteq [m] \) and \( J \subseteq [p] \), is not possible if \( J \neq [p] \).
\end{theorem}
\begin{proof}
    Given that \(\Omega = I \times J \times [n]\), the samples at time \(t\) can be represented as:
\begin{equation}\label{eqn:samplesatt}
    \mathcal{Y}_t = [\mathcal{I}_m]_{I,:,:} \ast \mathcal{F}_t \ast [\mathcal{I}_p]_{:,J,:}
\end{equation}
where \(\mathcal{I}_m\) is the \(m \times m \times n\) identity tensor \cite{kilmer2011factorization}. Utilizing the properties of the t-product and applying the discrete Fourier transformation on the third dimension of both sides of \eqref{eqn:samplesatt}, we obtain:
\begin{equation}\label{eqn:samplesInfrequency}
    [\widehat{\mathcal{Y}}_t]_{:,:,k} = [\mathbb{I}_m]_{I,:} \widehat{\mathcal{A}}_{:,:,k}^t [\widehat{\mathcal{F}}]_{:,J,k}
\end{equation}
where $\mathbb{I}_m$ stands for the $m\times m$ identity matrix. 
Given that the Fourier transformation is a unitary transformation, reconstructing \(\mathcal{F}\) from \(\Psi\) is equivalent to reconstructing \(\widehat{\mathcal{F}}\) from \(\widehat{\Psi} = \{\widehat{\mathcal{Y}}_t : t \in 0 \cup [T-1]\}\).

From \eqref{eqn:samplesInfrequency}, it is evident that the reconstructions of \([\widehat{\mathcal{F}}]_{:,j,k}\) are independent for different \(j \in J\). This implies that if \(J \neq [p]\), there will be some \(j \in [p] \setminus J\) for which \([\widehat{\mathcal{F}}]_{:,j,k}\) cannot be reconstructed from \(\widehat{\Psi}\). Consequently, \(\widehat{\mathcal{F}}\) cannot be fully reconstructed from \(\widehat{\Psi}\). The result of this theorem is thus established.
\end{proof}
Based on this result, we propose the following conjecture, which we intend to explore in our future work:
\begin {conjecture}\label{conj1}
A necessary condition for the successful recovery of \( \mathcal{F} \) from \( \Psi \) is that \( \bigcup_{(i,j,k)\in\Omega} \{j\} = [p] \).
\end {conjecture}
Although we do not provide a formal theoretical proof for this conjecture here, we conducted simulations to test our hypothesis. The results indicate that losing any vertical index from the second dimension leads to a failure in recovery, thereby supporting the conjecture.
\subsection{Method development for signal recovery}\label{sec:alg}
We now focus on reconstructing $\mathcal{F}$ from $\Psi$. If the samples sufficiently guarantee the reconstruction of $\mathcal{F}$, then the following optimization problem will yield a unique solution:
\begin{equation}\label{eqn:DS2opt}
\min_{\mathcal{X}}\sum_{t=0}^{T-1}\sum_{(i,j,k)\in\Omega}\|[\mathcal{A}^t\ast \mathcal{X}]_{i,j,k}-[\mathcal{F}_t]_{i,j,k}\|_{\fro}^2.
\end{equation}
For clarity, we introduce the following definitions:
Let $\mathcal{P}_{\Omega}(\cdot)$ denote the projection of a tensor onto the observed set $\Omega$ such that
\[
[\mathcal{P}_{\Omega}(\mathcal{T})]_{i,j,k}=\begin{cases}
    \mathcal{T}_{i,j,k}, & \text{if } (i,j,k)\in\Omega\\
    0, & \text{otherwise}
\end{cases}.
\]
According to \Cref{def:others}, we can reformulate \eqref{eqn:DS2opt} as:
\begin{equation}\label{eqn:optProj}
\min_{\mathcal{X}}\sum_{t=0}^{T-1} \|\mathcal{P}_{\Omega}(\mathcal{A}^t\ast \mathcal{X})-\mathcal{P}_{\Omega}(\mathcal{F}_t)\|_{\fro}^2.
\end{equation}
Consider that
\begin{equation}
\begin{aligned}
\mathcal{P}_{\Omega}(\mathcal{F}_t) &= \mathcal{P}_{\Omega}\odot\mathcal{F}_t \\
&= \ifft(\widehat{\mathcal{P}_{\Omega}}\circledast\widehat{\mathcal{F}_t},[],3)/n.
\end{aligned}
\end{equation}
Drawing on the properties of the t-product and inspired by \cite{liu2019low}, we can convert the least squares minimization problem \eqref{eqn:optProj} into a frequency domain version:
 \begin{equation}\label{eqn:optfrequency}
    \min_{\widehat{\mathcal{X}}\in\mathbb{C}^{m\times p\times n}}\sum_{t=0}^{T-1} \|\widehat{\mathcal{P}_{\Omega}} \circledast(\widehat{\mathcal{A}}^t\bigtriangleup \widehat{\mathcal{X}})/n-\widehat{\mathcal{P}_{\Omega}(\mathcal{F}_t)}\|_{\fro}^2. 
 \end{equation}
This problem can be decomposed into $p$ separate subproblems, one for each $j\in[p]$, where we solve:
\begin{equation}\label{eqn:optfrequency-sub}
\min_{[\widehat{\mathcal{X}}]_{:,j,:} }\sum_{t=0}^{T-1} \|[\widehat{\mathcal{P}_{\Omega}}]_{:,j,:} \circledast(\widehat{\mathcal{A}}^t\bigtriangleup [\widehat{\mathcal{X}}]_{:,j,:})/n-[\widehat{\mathcal{P}_{\Omega}(\mathcal{F}_t)}]_{:,j,:}\|_{\fro}^2.
\end{equation}
The goal is to achieve a minimal value of zero for each subproblem. To facilitate this, we construct the following system for each $t$ and $j$:
\begin{equation}\label{eq:optfre-eq}
A_3(j)A_1(t)x(j)=b(j,t)
\end{equation}
where 
\begin{equation}
x(j)=\begin{bmatrix}[\widehat{\mathcal{X}}]_{:,j,1};\cdots;[\widehat{\mathcal{X}}]_{:,j,n}\end{bmatrix}\in\mathbb{C}^{mn\times 1},
\end{equation}
\begin{equation}
b(j,t)=\begin{bmatrix}[\widehat{\mathcal{P}}_{\Omega}(\mathcal{F}_t)]_{:,j,1};\cdots;[\widehat{\mathcal{P}}_{\Omega}(\mathcal{F}_t)]_{:,j,n}\end{bmatrix}\in\mathbb{C}^{mn\times 1},
\end{equation}
and $A_1(t)$ and $A_3(j)$ are defined as:
\begin{equation}
A_1(t)=\begin{bmatrix}
    [\widehat{\mathcal{A}}]_{:,:,1}^t&&\\
    &\ddots&\\
    &&[\widehat{\mathcal{A}}]_{:,:,n}^t
\end{bmatrix}\in\mathbb{C}^{mn\times mn},
\end{equation}
\resizebox{0.5\textwidth}{!}{$
A_3(j)=\begin{bmatrix}
    \diag([\mathcal{A}_2(j)]_{1,1,:})& \diag([\mathcal{A}_2(j)]_{1,2,:})&\cdots& \diag([\mathcal{A}_2(j)]_{1,n,:})\\
    \diag([\mathcal{A}_2(j)]_{2,1,:})& \diag([\mathcal{A}_2(j)]_{2,2,:})&\cdots& \diag([\mathcal{A}_2(j)]_{2,n,:})\\
    \vdots&\vdots&\ddots&\vdots\\
    \diag([\mathcal{A}_2(j)]_{n,1,:})& \diag([\mathcal{A}_2(j)]_{n,2,:})&\cdots& \diag([\mathcal{A}_2(j)]_{n,n,:})
\end{bmatrix}$}
with $[\mathcal{A}_2(j)]_{:,:,\ell}=\circ([\widehat{\mathcal{P}_{\Omega}}]_{\ell,j,:})$. Solving this system will enable us to recover $x(j)$. Once all $x(j)$ values are obtained, they are combined and reshaped into a tensor $\widehat{\mathcal{X}}_{\textnormal{app}}$ of size $m\times n\times p$. The estimation of $\mathcal{F}$ is then set as $\mathcal{X}=\ifft(\widehat{\mathcal{X}}_{\textnormal{app}},[],3)$.

  
\subsection{Experiments}

To evaluate the performance of our proposed method, we conducted several simulations aimed at recovering the initial signal. The experiments are structured in three parts: the first part evaluates the overall recovery performance and the point-wise recovery of our algorithm, the second part focuses on determining the optimal value of the parameter $T$, which is crucial for effective signal recovery, the third part verifies the conjecture, demonstrating that to fully recover the signal, the union of the second dimension in our dataset must equal to $p$, i.e., the second dimension of the initial signal.
\subsubsection{Datasets}
To generate the synthetic datasets, we first create a random tensor of size \(20 \times 15 \times 5\) as the initial signal \(\mathcal{F}\) and another tensor of size \(20 \times 20 \times 5\) as the driven operator \(\mathcal{A}\). Using these, we produce signals at different times \(t\) by applying the operation \(\mathcal{F}_{t} = \mathcal{A}^t * \mathcal{F}\).  
Next, we generate another tensor \(\mathcal{P}_{\Omega} \in \{0,1\}^{20 \times 15 \times 5}\) of the same size as \(\mathcal{F}\) to denote the sampled locations. The entries of \(\mathcal{P}_{\Omega}\) are generated using a Bernoulli distribution, where an entry of \(1\) indicates the presence of a sample and \(0\) indicates its absence. The probability of \(1\) in \(\mathcal{P}_{\Omega}\) is set to the sampling rate \(\alpha\). 
Finally, we extract the spatio-temporal samples by generating \(\mathcal{P}_{\Omega}(\mathcal{F}_t)\) at various time points \(t \in \{0\} \cup [T-1]\), capturing samples across different locations and times.


 
\subsubsection{Recovery accuracy}
In the first set of experiments, we evaluated recovery accuracy. We set the maximum sampling time \( T \) to 5 and repeated the experiment 10 times to assess the stability of the method. As shown in \Cref{fig:recovery performance}, when the sampling rate reached 40\%, the relative error decreased to approximately \( 10^{-12} \), demonstrating the effectiveness of our approach.
\begin{figure}[ht]
    \centering
    \includegraphics[width=0.68\linewidth]{fig/newer_2.png}
    \caption{Relative error v.s. sampling rate $\alpha$: we repeated this experiment 10 times, calculating both the mean value and the standard deviation of the results. As shown by the shadow, the standard deviation is relatively small, indicating that our method consistently performs well across different trials.}
    \label{fig:recovery performance}
\end{figure}


 
\begin{figure}[ht]
    \centering
    \includegraphics[width=0.68\linewidth]{fig/tfig2.png}
    \caption{This figure shows the  point-wise gap between  
        the reconstructed signal and the ground truth signal, there are $20\times 15\times 5=1500$ sample points in total, we compared each point from the constructed tensor with the ground truth tensor.}
    \label{fig:point-wise recovery}
\end{figure}
 It is important to note that the product of \( T \) and the sampling rate \(\alpha\) provides a measure that reflects the overall sampling size.   If this product is less than 1, recovery is likely to fail, as the sample set lacks sufficient information to reconstruct the initial signal. Additionally, as observed in \Cref{fig:point-wise recovery}, a sampling rate of 40\% enables successful recovery of all points, further underscoring the robustness of our method.
 
 
\subsubsection{Optimal maximum sampling time T}
The parameter \( T \) is a critical hyperparameter, especially when samples are affected by noise. To investigate the effect of \( T \) on recovery performance, we conducted a series of experiments, varying \( T \) from 1 to 15 while keeping the sampling rate $\alpha$ fixed at 40\%. Additive Gaussian noises with mean 0 and variance \(\sigma^2\) were applied to the samples, i.e., \(\varepsilon \sim \mathcal{N}(0, \sigma^2)\).

\begin{figure*}[th]
    \centering
    \begin{subfigure}{0.32\textwidth}
        \centering
        \includegraphics[width=\textwidth]{fig/Opttt.png} 
        \caption{$\alpha=0.2$}
    \end{subfigure}
    \begin{subfigure}{0.32\textwidth}
        \centering
        \includegraphics[width=\textwidth]{fig/npc.png} 
        \caption{$\alpha=0.5$}
    \end{subfigure}
    
    
    \caption{Relative error v.s. maximum sampling time  $T$   under different noise levels and sampling rates}
    \label{fig:Optimal T}
\end{figure*}
As shown in \Cref{fig:Optimal T}, increasing \( T \) does not always enhance recovery performance; rather, an optimal value of \( T \) exists. For \( T > 10 \), we calculate the condition number \(\kappa(j)\) of the matrix 
{\footnotesize
\[
\begin{bmatrix}
    (A_3(j)A_1(0))^\top & (A_3(j)A_1(1))^\top & \cdots & (A_3(j)A_1(T))^\top
\end{bmatrix}^\top
\]}
and define \( K = \max_{j} \kappa(j) \) as the condition number of the entire system. We observed that \( K \) became excessively large (on the order of \(10^{11}\)), which caused instability in the linear system \eqref{eq:optfre-eq} and led to a significant increase in relative error, as shown in \Cref{fig:condition number}. This indicates that selecting an appropriate value for \( T \) is crucial for achieving stable and accurate recovery.



 \begin{figure}[ht]
    \centering
     \includegraphics[width=0.6\linewidth]{fig/cond.png}
     \caption{Condition number for different T}
     \label{fig:condition number}
 \end{figure} 
 \begin{figure}[ht]
    \centering
    \includegraphics[width=0.6\linewidth]{fig/Vcoo_2.png }
    \caption{Undersampled second dimension}
    \label{fig:verify conj1}

    \vspace{0.5cm} 

    \includegraphics[width=0.6\linewidth]{fig/ffp.png} 
    \caption{Undersampled first and third dimension}
    \label{fig:verify conje}
\end{figure}

  %\begin{table}[h!]
% %\centering
% %\begin{tabular}{|c|c|c|c|c|c|c|c|c|c|c|c|c|c|}
% %\hline
%  % & T=2 & T=3 &T=4 & T=5& T=6 & T=7 & T=8 &T=9 & T=10 &T=11 & T=12 &T=13 &T=14 &T=15 \\ \hline
% %Condition number   &$2.2 \times 10^16$    & $409$  & $2638$ & $1.4 \time 10^7$ & $5.5 \times 10^8$  & $1.2 \times 10^9$ & $1.6 \times 10^10$ & $1.8 \times 10^12$ & $6.5 \times 10^12$& $8.6 \times 10^13$ & $6.8 \times 10^14$\\ \hline

% % \begin{table}[h!]
% % \centering
% % \begin{tabular}{|c|c|c|c|c|c|c|c|c|c|c|c|c|c|}
% % \hline
% %         & Column 2 & Column 3 & Column 4 & Column 5 & Column 6 & Column 7 & Column 8 & Column 9 & Column 10 & Column 11 & Column 12 & Column 13 & Column 14 \\ \hline
% % Data 1  & Data 2   & Data 3   & Data 4   & Data 5   & Data 6   & Data 7   & Data 8   & Data 9   & Data 10   & Data 11   & Data 12   & Data 13   & Data 14   \\ \hline

% % \caption{2x14 Table with Blank First Cell}
% % \label{tab:2x14_blank_first}
% % \end{tabular}
% % \end{table}





\subsubsection{Verifying the conjecture}
\label{sec 234}
In this section, we conducted simulations to evaluate \Cref{conj1}. The initial signal is a tensor of size \( 20 \times 15 \times 5 \). To test the conjecture, we first set the sampling rate to \( \alpha = 1 \), generating the sampling location set \( \Omega = \bigcup_{j=1}^{15} \Omega_{j} \), where \( \Omega_{j} \) represents the locations corresponding to the second index equal to \( j \). In each experiment, one \( \Omega_{j} \) was excluded by varying \( j \) from 1 to 15, resulting in the sampling location set \( \Omega^{j} = \Omega \setminus \Omega_{j} \). This configuration led to a sampling rate of 93\%. However, as shown in \Cref{fig:verify conj1}, the relative error remained above 0.2, indicating that the initial signal could not be fully recovered.

To further explore this, we generated a new sampling location set with \( \alpha = 0.5 \), and similarly adjusted the set by excluding all locations where the first (or third) index equals \( j \), with \( j \) varying from 1 to 20 (or from 1 to 5 for the third index). As shown in \Cref{fig:verify conje}, excluding locations where the first or third index takes these values did not affect the recovery, and the relative error reduced to \( 10^{-11} \).

\small




\bibliographystyle{plain}%{plainnat}
\bibliography{ref}
\end{document}


\section{Contestant Equilibrium}
\label{sec:playerSBNE}
In this section, we fully characterize the unique symmetric Bayesian Nash equilibrium of admitted contestants for any shortlist size $m$ and any prize structure $\vec{V}$. A key step in achieving this characterization is to explicitly represent each contestant's utility, allowing us to derive and solve the first-order condition.
Unlike traditional contest settings without a shortlist, where the probability of contestant $i$ is ranked at $j$ can be directly computed using the prior ability distribution, our setting requires calculating these probabilities based on posterior beliefs about the abilities of other admitted contestants. This necessitates a detailed understanding of the posterior ability distribution. Therefore, we first describe the closed form of posterior beliefs in Subsection \ref{subsec:Posterior Beliefs}. Then, in Subsection \ref{subsec:Equilibrium Efforts}, we provide a complete characterization of contestant equilibrium efforts.

\subsection{Posterior Beliefs}
\label{subsec:Posterior Beliefs}

Given a contest with a shortlist size of $m$ and a prize structure $\vec{V}$, we first update an admitted contestant's posterior belief about the ability levels of other admitted contestants. This posterior belief, represented as the joint posterior distribution, is derived using Bayes' rule, as stated in the following proposition.

% To find the subjective probability of getting each rank for any players, we need to first derive their posterior beliefs on other contestants' ability level after updating according to Bayesian Rule. 

\begin{proposition}[Posterior Beliefs]\label{prop:posteriorBeliefs}
    For any admitted contestant (w.l.o.g., labeled as Contestant 1), the joint posterior probability density function of the ability levels of the other admitted contestants is given by:
    % For any selected observer (her player index is re-indexed as $1$), the joint posterior probability density function of other players' ability level perceived by her is:
    \[
        \beta_1(\mathbf{x}) =   
        \begin{cases} 
        \frac{\binom{n-1}{m-1}F^{n-m}(x^{(1)})\prod_{i=2}^{m}f(x_i)}{J(F,n,m,x_1)} & \text{if } x^{(1)} \leq x_1, \\
        \frac{\binom{n-1}{m-1}F^{n-m}(x_1)\prod_{i=2}^{m}f(x_i)}{J(F,n,m,x_1)} & \text{if } x^{(1)} > x_1,
        \end{cases}
    \]
    where $x_1$ is the contestant 1's ability level and $x^{(1)}:=\min_{j\in [m]\setminus \{1\}}x_i$ is the lowest ability level of other admitted contestants. The normalization denominator is defined as $J(F(\cdot),n,m,x) := \binom{n-1}{m-1}F^{n-m}(x)(1-F(x))^{m-1}+\binom{n-1}{m-1}(m-1)\int_0^{F(x)}t^{n-m}(1-t)^{m-2} \, dt$.
\end{proposition}

We further marginalize this belief to obtain the posterior ability distribution of a specific admitted contestant.

% To see the how shortlist affects the belief of admitted players, we marginalize this belief, thereby to see the posterior ability level distribution the observer holds for a specific rival.

\begin{corollary}[Marginal Posterior Beliefs]\label{prop:marginalBelief}
For any admitted contestant (labeled as Contestant 1), the posterior probability density function of another admitted contestant's ability (labeled as Contestant 2) is:

% For any selected observer (her player index is re-indexed as $1$), the posterior probability density function of another player's ability level (this player's index is re-indexed as $2$) perceived by her is:
\[
\beta_1(z) =   
\begin{cases} 
\frac{\binom{n-1}{m-1}\left [ F^{n-m}(z)(1-F(z))^{m-2}+(m-2)\int_0^{F(z)}t^{n-m}(1-t)^{m-3}\, dt \right ]f(z)}{J(F,n,m,x_1)} & \text{if } z \leq x_1, \\
\frac{\binom{n-1}{m-1}\left [  F^{n-m}(x_1)(1-F(x_1))^{m-2} +(m-2)\int_0^{F(x_1)} t^{n-m}(1-t)^{m-3}\, dt \right ] f(z)}{J(F,n,m,x_1)} & \text{if } z > x_1.
\end{cases}
\]

The posterior cumulative distribution function is then expressed as:
\[
\Pr_{\beta_1}(X_2\leq z) =   
\begin{cases} 
\frac{\binom{n-1}{m-1} \left [ B_{F(z)}(n-m+1,m-1) + (m-2)\int_{0}^{z} B_{F(t)}(n-m+1, m-2) \, dF(t) \right ]}{J(F,n,m,x_1)} & \text{if } z \leq x_1, \\
\Pr_{\beta_1}(X_2\leq x_1) + \\ \quad\frac{\binom{n-1}{m-1}(F(z)-F(x_1))\left [ F^{n-m}(x_1)(1-F(x_1))^{m-2}+(m-2)B_{F(x_1)}(n-m+1, m-2)\right ]}{J(F,n,m,x_1)} & \text{if } z > x_1.
\end{cases}
\]
where $B_{x}(a,b):=\int_{0}^{x} t^{a-1}(1-t)^{b-1} \, dt$ denotes an incomplete beta function.

Note that both $\beta_1(\cdot)$ and $\Pr_{\beta_1}(\cdot)$ are continuous.
%Also, for the selected observer, her marginal posterior belief for different opponents are identical but not independent.  
\end{corollary}

Using the marginal posterior belief, we derive two types of stochastic dominance to illustrate the effect of a shortlist.
% We then derive two types of stochastic dominance using this marginal posterior belief.

\begin{proposition}[Stochastic Dominance of Posterior over Prior]\label{prop:stoDomPos}
    Based on the prior p.d.f. \(f(x)\) and posterior p.d.f. \(\beta_i(x)\), let \(\Pr_{f}(X_j \leq z)\) and \(\Pr_{\beta_i}(X_j \leq z)\) denote the probabilities that contestant \(i\) believes that the ability level of contestant \(j \neq i\) is at most $z$ before and after receiving the admitted signal, respectively. Then, for all $z$, we have:
    %Let \(\Pr_{f}(X_j \leq z)\) denote the probability that player \(i\), according to the prior distribution \(F(x)\), believes the ability level of player \(j \neq i\) lower than \(z\) before the promotion signal is received. Similarly, let \(\Pr_{\beta_i}(X_j \leq z)\) denote the probability that player \(i\), according to the posterior belief \(\beta_i(x)\), believes the ability level of player \(j \neq i\) is lower than \(z\) after the promotion signal is received. Then, for all \(z \geq 0\), we have:
    \[
    \Pr_{\beta_i}(X_j \leq z) \leq \Pr_{f}(X_j \leq z).
    \]
\end{proposition} 

\begin{proposition}[Posterior of Higher Ability Stochastically Dominates Lower Ability]\label{prop:StoDomAbi}
    For two admitted contestants, say $i$ and $j$, if contestant $i$ has a higher ability than contestant $j$ (i.e., $x_i > x_j$), then contestant $i$'s posterior belief about the ability of any other admitted contestant  $k$ first-order stochastically dominates that of contestant $j$. Formally, for all $z\geq 0$, we have:
    % For two promoted players, say $i,j$, if player $i$'s ability level is higher than player $j$'s, i.e., $x_i > x_j$, then player $i$'s posterior belief of the ability level of any other promoted player $k$ first-order stochastically dominates  player $j$'s, i.e., for all $z\geq 0$, we have:
    \[
    \Pr_{\beta_i}(X_k \leq z) \leq \Pr_{\beta_j}(X_k \leq z).
    \]
\end{proposition}


A direct implication of stochastic dominance is higher expectation\footnote{Formally, if random variables $X$ an $Y$ satisfy $\Pr(X\leq z) \leq \Pr(Y\leq z)$ for all $z$, i.e., $Y$ is first-order stochastically dominated by $X$, then $\mathbb{E}[X] \geq \mathbb{E}[Y]$ holds.}. Therefore, the above two propositions imply that after the shortlist is applied, each admitted contestant perceives their opponents as stronger. Moreover, the stronger a contestant is, the more competitive they perceive their opponents to be.


% A direct implication of stochastic dominance is higher expectation\footnote{Formally, if random variables $X,Y$ satisfies $\Pr(X\leq z) \leq \Pr(Y\leq z)$ for all $z$, i.e., $Y$ is first-order stochastically dominated by $X$, then $\mathbb{E}[X] \geq \mathbb{E}[Y]$ holds.}. Therefore the above two propositions shows that, after shortlist, the observer perceives every admitted opponent as stronger, also, the stronger herself, the more competitive she thinks of her rivals.

\begin{example}
    Figure~\ref{fig:beliefs} shows the belief change of contestant $x_1$ resulting from shortlist. We rewrite the posterior belief as $\beta_1(z)=q_{x_1}(z)f(z)$, then the factor $q_{x_1}(z)$ contains the information brings by the admission signal. When a contestant gets admitted, her belief for low ability is discounted, and the factor is also smaller for lower ability, then the belief for ability higher than hers is adjusted by a constant factor accordingly, as show in Figure~\ref{fig:beliefs-a},~\ref{fig:beliefs-b}. The cumulative probability functions are plotted in Figure~\ref{fig:beliefs-c}. An opponent is viewed as stronger after shortlist (Proposition~\ref{prop:stoDomPos}), and stronger contestant also sees her opponent as stronger (Proposition~\ref{prop:StoDomAbi}).
        \begin{figure}[h]
        \centering
    \begin{subfigure}[ht]{0.30\textwidth}
        \centering
        \includegraphics[width=\textwidth]{figure/Post1.pdf}
        \subcaption{Factor $q_{x_i}$}
        \label{fig:beliefs-a}
        \end{subfigure}%\hfill
    \begin{subfigure}[ht]{0.30\textwidth}
        \centering
        \includegraphics[width=\textwidth]{figure/Post2.pdf}
        \subcaption{PDF}
        \label{fig:beliefs-b}
        \end{subfigure}%\hfill
    \begin{subfigure}[ht]{0.30\textwidth}
        \centering
        \includegraphics[width=\textwidth]{figure/Post3.pdf}
        \subcaption{CDF}
        \label{fig:beliefs-c}
        \end{subfigure}
    \caption{Posterior beliefs of player $x_1$ ($n = 5,m=2,F(x)=x^2$).}
    \label{fig:beliefs}
    \end{figure}
\end{example}
Although each contestant believes that every opponent becomes stronger after the shortlist, a counterintuitive result emerges: if we consider all opponents together, the contest environment actually becomes less competitive. This is because the number of opponents is the primary factor influencing competitiveness, and its reduction makes the contestant believe that her probability of achieving a certain rank is higher. Formally, we present the following proposition:
% However, this is only part of the story. In fact, the opponents as a whole is perceived as less competitive after shortlist. Although every admitted player is assumed as stronger, the decline of rival numbers is more compelling, making the observer believe that the ability level of each ranking in her rivals has a lower chance to surpass hers. Formally, we have the following proposition:

\begin{proposition}[Threatens of Opponents Decrease after Shortlist]\label{prop:ThreatenDesc}
For any contestant (labeled as 1) and any $l < m$, she perceives the opponent with the $l^{\text{th}}$ highest ability as weaker after the shortlist, in comparison to herself:
% For any selected observer (her player index is re-indexed as $1$), she sees the $l^{\text{th}}$ strongest opponent weaker after shortlist (in comparison to herself), for any $l < m$, i.e.:
\[
\Pr_{\beta_1}(X_{(l)} > x_1) \leq \Pr_{f}(X_{(l)}>x_1),
\]
where $X_{(l)}$ is the $l^{\text{th}}$ highest ability among her opponents, $[m]\backslash\{1\}$. 

Moreover, the fewer opponents that are admitted, the weaker they are perceived by the this contestant compared to herself. In other words, $\Pr_{\beta_1}(X_{(l)} > x_1)$ increase with $m$. 
% Moreover, the less rivals get promoted, the weaker they are deemed by the observer compared to herself, i.e., $\Pr_{\beta_1}(X_{(l)} > x_1)$ increase with $m$. 
\end{proposition}
% remarks and examples
% \begin{example}
%     In Figure~\ref{fig:threatens}, when more contestants are allowed into the contest, the contestant $x_1$ believes her opponents as a whole get stronger, thus her chances of getting high rank decreases dramatically.
%     \begin{figure}[h]
%     \begin{subfigure}[ht]{0.30\textwidth}
%         \centering
%         \includegraphics[width=\textwidth]{figure/Threaten1.pdf}
%         \subcaption{$F(x)=x$}
%         \label{fig:threatens-a}
%         \end{subfigure}
%     %\hfill
%               % 子图 (b)
%     \begin{subfigure}[ht]{0.30\textwidth}
%         \centering
%         \includegraphics[width=\textwidth]{figure/Threaten2.pdf}
%         \subcaption{$F(x)=x^2$}
%         \label{fig:threatens-b}
%         \end{subfigure}
%     %\hfill
%     \begin{subfigure}[ht]{0.30\textwidth}
%         \centering
%         \includegraphics[width=\textwidth]{figure/Threaten3.pdf}
%         \subcaption{$F(x)=1-e^{-x}$}
%         \label{fig:threatens-c}
%         \end{subfigure}
%     \caption{Player $x_1$'s beliefs on surpassing $l^{\text{th}}$ strongest opponent ($n = 8, x_1=0.4$).}
%     \label{fig:threatens}
%     \end{figure}
% \end{example}

It is important to note that, for a given contestant, although the marginal posterior belief is identical for all her opponents, they are not independent. The intuition is straightforward: knowing that a contestant weaker than herself has been admitted to the game will lower her expectations for another admitted contestant. Therefore, in general, $\beta_1(\mathbf{x}) \neq \prod_{i=2}^{m} \beta_1(x_i)$, and the joint probability distribution is required to fully characterize the posterior beliefs.

% It is noteworthy that, for a given observer, although this marginal posterior belief is identical for all of her rivals, they are actually dependent. The intuition is clear, since knowing a player weaker than herself also admitted into the game lowers her expectation for a third player. Thus, $\beta_1(\mathbf{x}) \neq \prod_{i=2}^{m} \beta_1(x_i)$ in general, the joint probability is needed to fully characterize posterior beliefs. 

\subsection{Equilibrium Efforts}
\label{subsec:Equilibrium Efforts}
%[Definition of the symmetric BNE]
With the detailed characterization of posterior beliefs in place, we can now calculate the symmetric Bayesian Nash equilibrium. To facilitate the analysis, we first introduce some symbols. Let \( b: x_i \mapsto e_i \) be a strictly increasing strategy function, and let its inverse function be \( \gamma(\cdot) \). For simplicity, we use \( \gamma_i \) For simplicity, we use \( \gamma(e_i) \), which represents the corresponding ability \( x_i \) for an effort  \( e_i \) under the function $b(\cdot)$. Thus, the decision problem for contestant \(i\) to exert effort  \(e_i\) is equivalent to reporting an ability \(\gamma_i\).


%Under a symmetric Bayesian Nash Equilibrium, player \(i\) assumes that all advancing opponents determine their effort levels based on a strictly increasing function \( b: x_j \mapsto e_j \). Let the inverse of this function be denoted as \( \gamma(\cdot) \). For simplicity, we use \( \gamma_i \) as shorthand for \( \gamma(e_i) \), which represents the corresponding ability level \( x_i \) under such function $b(\cdot)$ when the effort \( e_i \) is exerted. The decision problem of player \(i\) to exert an effort level \(e_i\) then is equivalent to report a ability level \(\gamma_i\).
Now, we can calculate the probabilities that the admitted contestant $i$ exerts $e_i$ (which may not necessarily be $b(x_i)$) and is ranked at position $l$, based on her posterior beliefs.

%The subjective probability of getting each rank depends on the posterior beliefs on opponents' ability level, then, we can derive the subjective probabilities from Proposition~\ref{prop:posteriorBeliefs}.

\begin{proposition}[Subjective Probability]\label{prop:winProb}
% When promoted player $i$ decides to exert effort $e_i$ given her ability $x_i$, her perceived probability of getting rank $l$, denoted as $P_{(i,l)}(\gamma_i \mid x_i)$, is:
When admitted contestant $i$ exerts effort $e_i$ given her ability $x_i$, her perceived probability of obtaining rank $l$, denoted as $P_{(i,l)}(\gamma_i \mid x_i)$, is:

\begin{enumerate}
    \item When $\gamma_i \leq x_i$,
\[
P_{(i,l)} =   
\begin{cases} 
\frac{\binom{n-1}{l-1}F(\gamma_i)^{n-l}(1-F(\gamma_i))^{l-1}}{J(F,n,m,x_i)} & \text{if } l < m, \\
\frac{\binom{n-1}{m-1}\left [ F^{n-m}(x_i)(1-F(x_i))^{m-1}+(m-1)\int_{F(\gamma_i)}^{F(x_i)}t^{n-m}(1-t)^{m-2} \, dt \right ]}{J(F,n,m,x_i)} & \text{if } l = m.
\end{cases}
\]
    \item When $\gamma_i > x_i$,
\[
P_{(i,l)} =   
\begin{cases} 
\frac{\binom{n-1}{m-1}\binom{m-1}{l-1}(1-F(\gamma_i))^{l-1}\left [F^{n-m}(x_i)(F(\gamma_i)-F(x_i))^{m-l}+(m-l)\int_{0}^{F(x_i)}t^{n-m}(F(\gamma_i)-t)^{m-l-1}\, dt\right]}{J(F,n,m,x_i)} & \text{if } l < m, \\
\frac{\binom{n-1}{m-1}F^{n-m}(x_i)(1-F(\gamma_i))^{m-1}}{J(F,n,m,x_i)} & \text{if } l = m,
\end{cases}
\]
\end{enumerate} 
where $P_{(i,l)}$ is $P_{(i,l)}(\gamma_i \mid x_i)$ in short and $\gamma_i = b^{-1}(e_i)$ and $P_{(i,l)}(\gamma_i \mid x_i)$ is a continuous function of $\gamma_i$. 
\end{proposition}
\begin{remark}\label{rmk:subjectiveProb}
    Under the sBNE (i.e., when the observer truthfully reports her ability level, $\gamma_i=x_i$), her subjective probability of obtaining rank $l<m$ is given by $\frac{\binom{n-1}{l-1}F(x_i)^{n-l}(1-F(x_i))^{l-1}}{\sum_{j=1}^{m}\binom{n-1}{j-1}F^{n-j}(x_i)(1-F(x_i))^{j-1}}$ (since $J(F, n, m, x_i) = \sum_{j=1}^{m}\binom{n-1}{j-1}F^{n-j}(x_i)(1-F(x_i))^{j-1}$, as shown in Lemma~\ref{lem:normal}), it is exactly the prior probability of achieving rank $l$, given that the contestant is admitted (i.e., ranks among the top $m$). Notably, this subjective probability decreases as the shortlist size $m$ increases. 

    This further highlights the signaling effect of shortlisting. As the shortlist size decreases, admission becomes more difficult, leading admitted contestants to perceive themselves as relatively more competitive (as established in Proposition~\ref{prop:ThreatenDesc}). Consequently, they believe they have a higher probability of securing a better rank.
    
    %Under the sBNE, (i.e., the observer truthfully reports her ability level, $\gamma_i=x_i$), her subjective probability of getting each rank $l<m$ becomes $\frac{\binom{n-1}{l-1}F(x_i)^{n-l}(1-F(x_i))^{l-1}}{\sum_{j=1}^{m}\binom{n-1}{j-1}F^{n-j}(x_i)(1-F(x_i))^{j-1}}$ (Since $J(F, n, m, x_i) = \sum_{j=1}^{m}\binom{n-1}{j-1}F^{n-j}(x_i)(1-F(x_i))^{j-1}$, as shown in Lemma~\ref{lem:normal}), which is exactly the prior probability of getting rank $l$ conditioned on her getting admitted into the contest (i.e.,  to rank among top $m$ players). The subjective probability decreases as shortlist capacity $m$ grows.
    
    % This further shows the signaling effect of shortlisting. If the capacity becomes smaller, it is harder to qualify, then the advanced players are convinced that they are relatively more competitive (as seen from Proposition~\ref{prop:ThreatenDesc}), and therefore they have a better chance of obtaining a good rank. 
\end{remark}

Having detailed probabilities for each rank, we can now express the expected utility for each admitted contestant. Next, we move to derive the symmetric strategy function under equilibrium, thereby characterizing the sBNE of contestants. First, we integrate the first-order condition of the utility function to establish a necessary condition for the equilibrium strategy function $b(\cdot)$. Then, by verifying its non-negativity and monotonicity, we confirm the validity of the derived symmetric strategy. Consequently, we conclude that the sBNE exists and is unique, given as follows.


%Next, we move to derive the symmetric strategy under equilibrium, thereby revealing the sBNE of contestants. First, we integrate the first-order condition of equilibrium to recover $b(\cdot)$, therefore getting the expression that the equilibrium strategy must have. Then, by checking the non-negativity and monotonicity, we verify that the derived symmetric strategy is valid. Thus, we conclude that the sBNE exists and it is unique, the symmetric stategy of which states as follows:

\begin{theorem}[Unique sBNE of Contestants]\label{thm:contestantSBNE}
    For any ability distribution $F$, any size of shortlist $m$ and any prize structure $\vec{V}$, the unique symmetric Bayesian Nash equilibrium exists and can be expresses as:
    %Under symmetric Bayesian Nash Equilibrium, each promoted player $i \in [m]$ exerts effort according to a shared strictly increasing non-negative function $b(\cdot)$ of her ability level $x_i$, which uniquely expresses as:
    \[
        b^*(x) = g^{-1}\left(\int_{0}^{x}\frac{\sum_{l=1}^{m-1}\binom{n-1}{l-1}(n-l)(V_l-V_{l+1})F^{n-l-1}(t)(1-F(t))^{l-1}f(t)}{J(F,n,m,t)} t\, dt \right), 
    \]
    where $g(\cdot)$ is the cost function.
\end{theorem}
\begin{remark}\label{rmk:PrizeGap}
    From the expression of equilibrium effort, it becomes clear that contestants are incentivized by the gap between consecutive prizes, $(V_{l}-V_{l+1})$,  rather than the absolute value of the prizes themselves. Furthermore, if the cost function $g(\cdot)$ is linear, the contributions of these gaps to effort exertion remain independent of one another.
    % From the expression of equilibrium effort, we can see that it is not the absolute value of prizes but the gap between consecutive prizes $(V_{l}-V_{l+1})$s that incentive contestants to exert more effort. Moreover, if the cost function $g(\cdot)$ is linear, contribution of the gaps are independent.  
\end{remark}

\section{Optimal Contest Design}
\label{sec:optimal design}

In this section, we discuss the optimal contest design under different objectives. In Subsection~\ref{subsec:GeneralGuideline}, we identify a guideline (i.e., two sufficient conditions) such that the optimal contest is a form of simple contest.  In Subsection~\ref{subsec:HighestEffort}, we focus on the maximum individual effort objective, showing that the two-contestant winner-take-all contest is optimal. In Subsection~\ref{subsec:TotalEffort}, we analyze the total effort objective in detail and find that: (a) the optimal contest is a complete simple contest, i.e, the number of prizes is exactly the shortlist size minus one; (b) the optimal shortlist size for a given distribution grows asymptotically linearly with $n$; (c) the optimal shortlist size is no more than $31.62\%\,n$ for any distribution asymptotically.

%In this section, we discuss the optimal contest design under different objectives. In Subsection~\ref{subsec:GeneralGuideline}, we identify that the optimal contest under linear objectives must be a simple contest. Then, we focus on the two most common linear objectives. Section~\ref{subsec:HighestEffort} discusses the maximum individual effort objective, where we find the 2-contestant winner-take-all contest is optimal. In Section~\ref{subsec:TotalEffort}, we analyze the total effort objective in depth, and find out that: (a) The optimal contest is a complete simple contest; (b) The optimal shortlist size of a given distribution grows asymptotically linear with $n$; (c) The optimal size is no more than $31.62\%\,n$ for any distribution asymptotically.

\subsection{General Guideline}\label{subsec:GeneralGuideline}

For a given set of contestants $[n]$ and an ability distribution $F$, the designer aims to determine the shortlist size $2\leq m \leq n$ and and a prize structure $\vec{V}=\{V_1,\ldots,V_m\}$ that satisfies the rank-order property $V_1\geq V_2 \geq\ldots\geq V_m \geq 0$ and the budget constraint $\sum_{l=1}^mV_l \leq B$, with the goal of maximizing the ex-ante effort-based objective under equilibrium. Formally, 

% For a given list of contestants $[n]$ and the common ability distribution $F(x)$, the designer's problem is to decide a shortlist size $2\leq m \leq n$ and a prize structure $\vec{V}=\{V_1,\ldots,V_m\}$ that satisfies rank-order property $V_1\geq V_2 \geq\ldots\geq V_m \geq 0$ and budget constraint $\sum_{l=1}^mV_l \leq B$, such that her ex-ante effort objective is maximized under the equilibrium, formally: 
\[
    \begin{aligned}
        \mathop{\arg \max}_{m,\vec{V}} \quad & \mathbb{E}_{\vec{x} \sim F^n} [ \text{OBJ}(b(x_{(1)}), \ldots,b(x_{(m)}))] \\
        s.t. \quad & \sum_{l=1}^{m} V_l \leq B \\
            & V_l  \geq V_{l+1} \geq 0,
    \end{aligned}
\]
where $x_{(i)}$ denotes the $i^\text{th}$ highest realized ability level, and $\text{OBJ}(\cdot)$ represents the component-wise non-decreasing objective function of the efforts, arranged in decreasing order. %from the $m$ admitted contestants.

To solve this problem, we first establish two key properties of the equilibrium effort from the designer's perspective, which will help reduce the space of decision variables.

% To solve this problem, we begin by showing two useful properties of the equilibrium effort from the perspective of designer, which will help us to reduce the design space.

\begin{corollary}[No Consolation Prize Should be Set]\label{coro:Consolation}
    For any contest that allocates a non-zero prize for the last place, i.e., \(V_m > 0\), setting \( V_m = 0\) increases the equilibrium effort of every contestant.
\end{corollary}

Recall from Remark~\ref{rmk:PrizeGap} that the gap between consecutive prizes motivates contestants. Since a consolation prize makes prizes for higher ranks less attractive, it negatively impacts the effort objectives.
% \begin{remark}
%     Recall from Remark~\ref{rmk:PrizeGap} that the gap between consecutive prizes motivate contestants, since consolation prize make prizes for higher ranks less attractive, it is bad for effort objectives.
% \end{remark}

\begin{corollary}[Empty Prizes Discourage contestants]\label{coro:EmptyPrize}
For a given list of contestant ability $\vec{x}$ nd a prize structure with $k$ non-zero prize, i.e., \( V_1 \geq V_2 \geq ... \geq V_k > V_{k+1} = 0 \), extending the shortlist from $k+1$ to $m$ contestants decreases the equilibrium efforts of all contestants in the former shortlist.
\end{corollary}

From an equilibrium perspective, a contestant perceives herself as less competitive as the contest becomes more crowded (Proposition~\ref{prop:ThreatenDesc}). Consequently, her subjective probability of winning any non-empty prizes decreases when more contestants are admitted (Remark~\ref{rmk:subjectiveProb}). This results in a lower expected payoff, which discourages contestants from exerting more effort.
% \begin{remark}
%     From the perspective of equilibrium, a contestant feels herself less competitive as the contest gets more crowded (Proposition~\ref{prop:ThreatenDesc}), therefore her subjective probability of getting every non-empty prizes decreases when allowing more contestants into the contest (Remark~\ref{rmk:subjectiveProb}). Thus expected payoff also decreases, which discourages contestants from exerting more effort. 
% \end{remark}

Before further discussion, we extend the definition of a Simple Contest \cite{EGG21} to the shortlist setting, a special type of contest that is crucial for effort-based objectives.

\begin{definition}[Simple Contest]
    A contest with $n$ participants and a shortlist size $m$ is called a simple contest if all of its non-zero prizes are equal. Furthermore, if all the prizes in the contest are equal, meaning either $V_i = 0$ for all $i$ or $V_1 = \ldots = V_m \neq 0$, then the contest is referred to as a trivial simple contest.
    %if a simple contest has at lease one non-zero prize and allocates zero prize for the last place, i.e., $\exists l < m, l \in \mathbb{N}$ such that $V_1 = \ldots = V_l > V_{l+1} = 0$, it is a non-trivial simple contest.  [trivial] 
\end{definition}

Note that, in a trivial simple contest, every contestant will exert zero effort in the contest. Additionally, winner-take-all contest, i.e., $V_1=B,$, is a special case of non-trivial simple contest.
% \begin{remark}\label{rmk:TrivialZero}
%     Every contestant exerts zero effort in a trivial simple contest. 
% \end{remark}
% \begin{example}
%     Winner-take-all contests, which is the optimal contest under [Paper] \cite{}, i.e., $V_1=B,2\leq m \leq n$, are a special type of non-trivial simple contest. \textcolor{red}{[Citation]}
% \end{example}

In fact, in contest settings where the designer has the option to create a shortlist, the optimal contest design remains a simple contest for the linear combination of efforts objective, as long as the cost function is linear or only the effort of one certain rank is concerned. %This is formally stated as follows:

% In fact, in the contest setting where the designer has a shortlist option, the optimal contest design is still a simple contest for linear effort objectives, as long as the cost function is linear, or the effort of exactly one position is concerned, which is formally stated as follows:

\begin{proposition}[General Design Guideline]\label{prop:DesignGuideline}
    If the designer's objective is a non-negative, non-zero linear combination of the contestants' effort under equilibrium, denoted by $\vec{e}=(b(x_{(1)}), \ldots,b(x_{(m)}))$ (with contestants re-indexed according to their rankings), either ex-ante (in expectation) or ex-post (for a specific ability level profile of registered contestants), i.e., $u_d=\mathbb{E}_{\vec{x}}[\vec{c}\cdot e(\vec{x})]$ or $u_d(\vec{x})=\vec{c}\cdot e(\vec{x})$, then the optimal contest that maximizes the designer's objective will be a non-trivial simple contest if any of the following conditions are satisfied:
    % If the designer's objective is a non-negative and non-zero linear combination of qualified contestants' effort under the equilibrium, $\vec{e}=(b(x_{(1)}, \ldots,b(x_{(m)}))$ (where contestants are re-indexed according to their rankings), either ex-ante (in expectation) or ex-post (for a concrete ability level profile of registered contestants), i.e., $u_d=\mathbb{E}_{\vec{x}}[\vec{c}\cdot e(\vec{x})]$ or $u_d(\vec{x})=\vec{c}\cdot e(\vec{x})$,  then if any one of the following conditions are satisfied: 
    \begin{enumerate}
        \item The designer only cares about the effort of exactly one ranking, i.e., $c_i>0$ and $\forall \,l\neq i$, $c_l=0$.
        \item The cost function is linear, i.e., $g(e)=ke$ for some $k>0$.
    \end{enumerate}
    % then the optimal contest that maximize the designer's utility must be a non-trivial simple contest.
\end{proposition}

Recall that the difference between consecutive prizes, $(V_l-V_{l+1})$ incentivizes contestants' efforts (Remark~\ref{rmk:PrizeGap}). The above conditions ensure that the designer's objective is linear with respect to the gaps $(V_l-V_{l+1})$. Therefore, we can identify the most profitable gap (the one with the highest weight after adjustments) and allocate the entire budget to enlarging that gap. This results in an optimal simple contest. Additionally, this approach provides an $O(n^2)$ algorithm for finding the optimal contest.

We now focus on two specific types of linear objectives, which are commonly used in practice and have been extensively studied in the literature: the maximum individual effort and total effort.

\subsection{Maximum Individual Effort}\label{subsec:HighestEffort}

Under the maximum individual effort objective, the designer aims to maximize the expected effort of the strongest contestant in equilibrium, i.e., $\mathbb{E}_{x \sim X_{(1)}}[b(x)]$. As a result, the general design guideline (Proposition~\ref{prop:DesignGuideline}) indicates that the optimal contest is a simple contest when the goal is to maximize individual effort. This reduces the design problem to finding the best combination of shortlist size $m$ and the number of prizes $l$. 
On one hand, when the prize structure is fixed, there is no need to admit more than $l+1$ contestants, as additional competition diminishes the strongest contestant's enthusiasm (Corollary~\ref{coro:EmptyPrize}), which implies that the optimal contest satisfies $m=l+1$. On the other hand, intuitively, as more contestants are admitted, the expected prize awarded to the strongest contestant decreases (Proposition~\ref{prop:ThreatenDesc}), leading to a reduction in the effort exerted. 
In summary, a two-contestant winner-take-all contest (i.e., $m=2$ and $l=1$) is the optimal design for maximizing individual effort. Since this analysis holds for any realization of ability levels, we can conclude the following theorem:



% The general design guideline (Proposition~\ref{prop:DesignGuideline}) has shown that the optimal contest is a simple contest, then the decision problem becomes to find the best combination of size $m$ and prize number $l$. When prizes are fixed, there is no need to admit more than $l+1$ contestants, since this extra competition discourages the strongest contestant (Corollary~\ref{coro:EmptyPrize}). Then, remaining candidates are contests with $m$ position and $m-1$ equal prizes. Intuitively, the more contestants are admitted, the less probable that the strongest contestant end up with nothing (getting the last place), we can then further prove that her effort is decreasing with $m$, resulting in the optimal contest with $m=2,l=1$. Since this analysis holds for any realization of ability level, we arrive at the following theorem:

\begin{theorem}[Optimal Contest for the Ex-post Maximum Individual Effort]\label{thm:ExpostHighestEffort}
    The optimal contest that maximizes the ex-post maximum individual effort is a two-contestant winner-take-all contest, where $m=2$ and $V_1=B$.

    The resulting maximum ex-post individual effort is as follows:
    \[
    g^{-1}\left ( B \int_{0}^{x_{1}} \frac{(n-1)f(t)t}{F(t)+(n-1)(1-F(t))} \, dt \right ),
    \]
    where $x_1$ denotes the highest realized ability level among all contestants.
\end{theorem}

Thus, even for the the ex-ante maximum individual effort, the optimality of the two-contestant winner-take-all contest still holds.

\begin{corollary}[Optimal Contest for the Ex-ante Maximum Individual Effort]\label{coro:OptimalMaximumEffort}
    The optimal contest that maximizes the ex-ante maximum individual effort is a two-contestant winner-take-all contest, where $m=2$ and $V_1=B$. $m=2$ and $V_1=B$.

    The resulting maximum ex-ante individual effort is as follows:
    \[
    \mathbb{E}_{x\sim X_{(1)}} \left [ g^{-1}\left ( B \int_{0}^{x} \frac{(n-1)f(t)t}{F(t)+(n-1)(1-F(t))} \, dt \right ) \right ],
    \]
    where random variable $X_{(1)}$ denotes the highest realized ability level among all contestants.
\end{corollary}

\subsection{Total Effort}\label{subsec:TotalEffort}

Under the total effort objective, the designer aims to maximize the expected total effort of all admitted contestants in equilibrium, i.e., $\mathbb{E}_{X_{(1)}, \ldots, X_{(m)}}[\sum_{j=1}^{m}b(x_{(j)})]$.
In this subsection, we assume a linear cost function, $g(e)=ke$ with $k>0$ and show that the optimal contest as follows.
\begin{theorem}[Optimal Contest for the Total Effort Objective]\label{thm: opt for total}
    For any ability distribution $F$, the optimal contest for maximizing the total effort can be described as:
    \begin{enumerate}
        \item The number of prize $l^*$ is equal to the shortlist size $m^*$ minus one, i.e., $l^*=m^*(n)-1$.
        \item The budget is equally divided into these prizes, i.e, $V_1=V_2=\cdots= V_{l^*}=B/l^*$.
        \item The optimal shortlist is proportional to $n$, i.e., $m^*(n)=kn$, where $k$ is the solution to 
        \[
            \int_k^1 F^{-1}(1-q)(\frac{1}{q}-(2k-k^2)\frac{1}{q^2})dq=0.
        \]
    \end{enumerate}
\end{theorem}

To show the optimality of the above characterization, we first define the contests satisfy condition (1) and (2) as a complete simple contest. 
\begin{definition}[Complete Simple Contest]
    A simple contest is complete, if it is a non-trivial simple contest with shortlist size $m$ that has exactly $m-1$ prizes, i.e., $V_1 = \ldots =V_{m-1}>V_m=0$.
\end{definition}

Then, we show that the optimal contest maximizing the total effort is exactly a complete simple contest. By general design guideline, the optimal contest in this case is a simple contest. Moreover, due to linearity, the budget appears as a constant factor in the objective under simple contests and therefore does not affect optimality. To facilitate discussion, we set $B=1$ as a standing assumption.

% Under the total effort objective, the designer seeks to maximize the expected total effort of all admitted contestants in equilibrium, i.e., $\mathbb{E}_{X_{(1)}, \ldots, X_{(m)}}[\sum_{j=1}^{m}b(x_{(j)})]$.  In this subsection, we assume that the cost function is linear, i.e., $g(e)=ke, k>0$. Then, by general design guideline, the optimal contest is a simple contest. Also, by linearity, the budget will appear in the objective as a constant factor under simple contests, therefore will not affect optimality, we then let $B=1$ as a standing assumption to facilitate discussion.  

Using the equilibrium effort expression from Theorem~\ref{thm:contestantSBNE}, the ex-ante total effort of a simple contest with shortlist size $m$ and $l$ equal prizes, denoted by $S(m,n,l)$, is given by:
\[
\mathbb{E}_{X_{(1)}, \ldots, X_{(m)}} \left [\sum_{i=1}^{m}\int_{0}^{x_{(i)}}\frac{\binom{n-1}{l}F^{n-l-1}(t)(1-F(t))^{l}}{\sum_{j=1}^{m}\binom{n-1}{j-1}F^{n-j}(t)(1-F(t))^{j-1}}\frac{f(t)}{1-F(t)} t\, dt \right ],
\]
where $X_{(i)}$ is the $i^{\text{th}}-$highest ability level among all contestants, and $x_{(i)}$ is its realization. 


% Expanding from the equilibrium effort expression (Theorem~\ref{thm:contestantSBNE}), the ex-ante total effort of a simple contest with shortlist size $m$ and $l$ equal prizes, denote by $S(m,n,l)$, is given as:
% \[
% \mathbb{E}_{X_{(1)}, \ldots, X_{(m)}} \left [\sum_{i=1}^{m}\int_{0}^{x_{(i)}}\frac{\binom{n-1}{l}F^{n-l-1}(t)(1-F(t))^{l}}{\sum_{j=1}^{m}\binom{n-1}{j-1}F^{n-j}(t)(1-F(t))^{j-1}}\frac{f(t)}{1-F(t)} t\, dt \right ],
% \]where $X_{(i)}$ is the $i^{\text{th}}$ highest ability level among all contestants and $x_{(i)}$ is its realization. 

% Before further characterizing the optimal contest, we introduce the following definition:


% \begin{example}
%     The optimal contest design under maximum individual effort target, i.e., a 2-contestant winner-take-all contest (Corollary~\ref{coro:OptimalMaximumEffort}), is a complete simple contest.
% \end{example}

In Subsection~\ref{subsec:HighestEffort}, we used Corollary~\ref{coro:EmptyPrize} to show that, when prizes are fixed, admitting more contestants decreases the maximum individual effort. This monotonicity can extend to the total effort. When more contestants are allowed into the contest, the effort of the previously admitted contestants decreases, as their subjective probability of winning declines (Remark~\ref{rmk:subjectiveProb}). On the other hand, the effort contributed by the newcomers cannot compensate for this, as their ability levels are lower. Thus, we have:

\begin{proposition}\label{thm:ConpleteSimpleContest}
    The optimal contest with a shortlist that maximizes total effort is a complete simple contest.
\end{proposition}

% In previous Section~\ref{subsec:HighestEffort}, we use Corollary~\ref{coro:EmptyPrize} to show that when prizes are fixed, admitting more contestants will lower the maximum individual effort. This monotonicity generalizes to the total effort. When more contestants are allowed into the contest, the effort from former qualified contestants decrease since their subjective winning probability declines (Remark~\ref{rmk:subjectiveProb}). On the other hand, effort contributed from the new comers can not compensate for that, since their ability level is lower. Then we have:



The decision problem now reduces to determining the optimal shortlist size $m$ (with the number of prizes being $m-1$ accordingly). We explore additional properties of the optimal contest through asymptotic analysis. First, we rewrite the total effort objective using the beta distribution:
\begin{lemma}[Beta Representation for Total Effort]\label{lem:betaRepTotalEffort}
    The ex-ante total effort in a complete simple contest, denoted by $S(m,n)$ (abbreviated from $S(m,n,m-1)$ when no confusion arises), can be expressed using the beta distribution $\beta(x,a,b)$, as follows:
    \[
    \begin{aligned}
        S(m,n) = & 
        \int_0^1F^{-1}(q)\beta(q,n-m+1,m)\,dq \\
        & +\int_0^1F^{-1}(q)\frac{q}{1-q} \frac{m}{n-m}\frac{\beta(q,n-m,m)}{\int_0^q\beta(x,n-m,m)\,dx}\int_q^1\beta(x,n-m,m+1)\,dx\, dq,
    \end{aligned}
    \] where $\beta(x,a,b) =x^{a-1}(1-x)^{b-1}B(a,b)^{-1}$ is the probability density function of the beta distribution, parametrized by positive integers $a$ and $b$, and $B(a,b)=\frac{\Gamma(a)\Gamma(b)}{\Gamma(a+b)}$ is the beta function.
\end{lemma}

The concentration behavior of the beta distribution\footnote{In general, $\beta(q,\alpha,\beta)$ concentrates around its maximum point $\mu=\frac{\alpha-1}{\alpha+\beta-2}$ as $\alpha$ and $\beta$ go large.} allows us to asymptotically approximate the integration terms in the expression (Lemmas~\ref{lem:1} and~\ref{lem:2}). Thus, the representation simplifies to:
\begin{lemma}[Asymptotic Expression for Total Effort]\label{lem:AsyRep}
    The ex-ante total effort in a complete simple contest has the following asymptotic expression with respect to $n$:
    $$\lim_{n\rightarrow +\infty}\frac{S(m,n)}{n}=\int_0^{1-k}F^{-1}(q)\frac{q}{1-q}\frac{k}{1-k}(\frac{1}{q}-\frac{k}{q(1-q)})dq,$$
    where $k=m/n$, and the convergence rate is independent of $k$. 
\end{lemma}

Following this expression, we treat the admitting ratio $k$ as a decision variable and solve the first-order condition. This leads to the asymptotic optimal size, which is proportional to $n$.

\begin{proposition}[Optimal size is Asymptotically Linear]\label{thm:OptAsmLinear}
The ex-ante total effort in a complete simple contest converges to a function of $k=m/n$ as $n \rightarrow \infty$, and the convergence rate is independent of 
$k$. Therefore, the optimal shortlist size grows asymptotically linearly with $n$. Formally, there exists a $k^*\in(0,1)$ such that:
$$\lim_{n\rightarrow \infty}m^*(n)/n=k^*,$$
where $k^*$ is the solution of the following equation:
\[
\int_k^1 F^{-1}(1-q)(\frac{1}{q}-(2k-k^2)\frac{1}{q^2})dq=0.
\]
\end{proposition}

Proposition~\ref{thm:OptAsmLinear} provides a general way to solve the optimal size for any given distribution, i.e., to solve the equation of first-order condition, as can be seen from following examples. 

\begin{example}[Optimal size for Uniform Distributions]\label{exam:OptimalUniform}
    For a uniform distribution $U[0,b]$ that starting at $0$, i.e., $F(x)=x/b$, then $F^{-1}(1-q)=b(1-q)$. Following Theorem~\ref{thm:OptAsmLinear}, the optimal ratio $k^*$ satisfies:\(\int_k^1 b(1-q)(\frac{1}{q}-(2k-k^2)\frac{1}{q^2})dq=0 \).
    This equation simplifies to $\ln k=1+\frac{4-2k}{k^2-2k-1}$, which can be solved numerically, obtaining $k^* \approx 15.07\%.$
\end{example}
\begin{example}[Optimal size for Square Function Distribution]\label{exam:OptimalSquare}
    For square function distribution, i.e., $F(x) = x^2, x\in[0,1]$, then $F^{-1}(1-q) = (1-q)^{1/a}$. The first-order condition $\sqrt{1-k}(k-4)+(k^2-2k+2)\ln(1+\sqrt{(1-k)})+1/2 \cdot(k^2-2k-2)\ln k =0$ gives that $k^*\approx 20.67\%$. 
    %Sqrt[1 - k] (-4 + k) + (2 - (-2 + k) k) Log[1 + Sqrt[1 - k]] + 1/2 (-2 + (-2 + k) k) Log[k] == 0
\end{example}
\begin{example}[Optimal size for Exponential Distributions]\label{exam:OptimalExp}
    For exponential distribution, i.e., $F(x)=1-e^{-\lambda x}$, then $F^{-1}(1-q)=-\frac{1}{\lambda}\ln q$. The first-order condition $\frac{1}{2\lambda}((\ln k)^2+(4-2k)\ln k+2(k-2)(k-1))= 0$ gives that $k^*\approx 9.70\%$, which is independent of the parameter $\lambda$.
\end{example}

%Example~\ref{exam:OptimalUniform},~\ref{exam:OptimalSquare},~\ref{exam:OptimalExp}. 

In this way, we have fully characterized the optimal contest with a shortlist for the total effort objective. A natural question arises: Is there an upper bound for the optimal admitting ratio? In other words, can the contest designer eliminate a portion of contestants before knowing anything about the ability distribution, while still seeking to maximize total effort? Surprisingly, the answer is yes. Formally, we state the following theorem:

\begin{theorem}[Tight Upper Bound for the Optimal size]\label{thm:UniversalBound}
    For an arbitrary ability distribution $F$, when $n\rightarrow +\infty$, the optimal shortlist size that maximizes ex-ante total effort, denoted by $m^{*}$, has the following linear upper bound with respect to $n$:
    $$ \lim_{n \rightarrow \infty} \frac{m^*(n)}{n} \leq \bar{k},$$
    where $\bar{k} \approx 31.62\%$ is the solution to the equation $\ln k=(2-k)(k-1)$.

    Moreover, there exists a distribution such that $m^*(n)/n=\bar{k}$, meaning the bound is tight. 
\end{theorem}

Interestingly, with no assumptions on the ability distribution, the designer can eliminate up to approximately 68.38\% of the contestants without sacrificing the optimal contest that maximizes total effort. Less competitive contestants contribute little effort, but their presence significantly discourages stronger contestants, as they are perceived to be as competitive as the others. By eliminating weaker contestants, the stronger contestants have more room to fully compete, which increases the total effort and allows the designer to fully leverage their informational advantage.
% \begin{remark}
%     Astonishingly, with no assumption on ability  distribution, the designer can at most eliminate approximately $68.38\%$ of the contestants without missing the optimal contest that maximizes total effort! Less competitive contestants contribute little effort, but their existence greatly discourages stronger contestants, since they are perceived as strong as other contestants. Eliminating weaker contestants leaves room for the stronger to fully compete, raises the total effort, and therefore realizes the information advantage of the designer. 
%     %Even though our proof is asymptotic, numerical result shows that this linear trend actually fits pretty well even for small $m$, as shown in Figure~\ref{fig:universal-a}. 
% \end{remark}
    
%\caption{Comparison of model's self-assessed performance (average implicit reward policy model given to its own roll-outs) and real performance (EX) on Bird development set (Pass@1) during DPO training.}

% \section{Shortlist V.S. Vanilla}
% \label{sec: compare}
% How much can a contest designer benefit from the shortlist option? In this section, we answer this question by comparing the optimal contest design with and without a shortlist, when the cost function is linear. We discusses the maximum individual effort objective and the total effort objective in Subsection~\ref{subsec:ApproxMaxi} and \ref{subsec:ApproxTotal}, respectively.

% Before analyzing the gap between the optimal contests with and without a shortlist, we first rewrite the objective function by the quantile.  
% \subsection{Maximum individual Effort}\label{subsec:ApproxMaxi}

% \begin{lemma}
    
% \end{lemma}

% \begin{lemma}
    
% \end{lemma}

% \begin{theorem}
%     For any ability level distribution, under the total effort objective, the 2-contestant winner-take-all contest gives $\Theta(\log n)$ times the effort of the optimal contest without a shortlist, i.e.:
%     $$\frac{S(2,n,1)}{S(n,n,1)} = \Theta(\log n).$$
% \end{theorem}

% \subsection{Total Effort}\label{subsec:ApproxTotal}

% \begin{lemma}
%     $S(n ,n,1) = \Theta(1).$
% \end{lemma}
    
% \begin{lemma}
%     $S(2,n,1) = \Theta(\log n).$
% \end{lemma}
    
% \begin{lemma}
%     $S(m^*,n, m^*-1) = \Theta(n).$
% \end{lemma}

% \begin{proposition}\label{prop:TotalTWOVAN}
%     For any ability level distribution, under the total effort objective, the 2-contestant winner-take-all contest gives $\Theta(\log n)$ times the effort of the optimal contest without a shortlist, i.e.:
%     $$\frac{S(2,n,1)}{S(n,n,1)} = \Theta(\log n).$$
% \end{proposition}

% \begin{theorem}\label{thm:TotalOPTVAN}
%     For any ability level distribution, under the total effort objective, the optimal contest with a shortlist option results in $\Theta(n)$ times the effort of the optimal contest without a shortlist, i.e.:
%     $$\frac{S(m^*,n,m^*-1)}{S(n,n,1)} = \Theta(n).$$
% \end{theorem}

% \section{Towards Practical Applications}

% In previous sections, we use asymptotic analysis to characterize and solve the optimal contest design, which has a potential application value. In this section, we further provide numerical results to support that our findings fit well even when $n$ is small, stepping towards practical applications.

% In this section, we focus on the total effort objective, which is most widely used in practice.

% \noindent \textbf{Finding the Optimal Contest Design.} The optimal contest is a complete simple contest (Theorem~\ref{thm:ConpleteSimpleContest}), and the corresponding shortlist size is linear to $n$, the slope $k$ is determined by the distribution and can be identified easily by solving an equation (Theorem~\ref{thm:OptAsmLinear}, asymptotic).

% Numerical results show that this linear trend is clear starting from very small $n$, and the asymptotic ratio gives a close prediction. Thus, the optimal contest of any distribution can be find efficiently.

% \begin{figure}[htb]
% \begin{subfigure}[ht]{0.30\textwidth}
%     \centering
%     \includegraphics[width=\textwidth]{figure/AsmUniTex.pdf}
%     \subcaption{$F(x)=x$}
%     \label{fig:disopt-a}
%     \end{subfigure}
% %\hfill
%           % 子图 (b)
% \begin{subfigure}[ht]{0.30\textwidth}
%     \centering
%     \includegraphics[width=\textwidth]{figure/AsmSquareTex.pdf}
%     \subcaption{$F(x)=x^2$}
%     \label{fig:disopt-b}
%     \end{subfigure}
% %\hfill
% \begin{subfigure}[ht]{0.30\textwidth}
%     \centering
%     \includegraphics[width=\textwidth]{figure/AsmExpTex.pdf}
%     \subcaption{$F(x)=1-e^{-x}$}
%     \label{fig:disopt-c}
%     \end{subfigure}

% \caption{The actual optimal size and $m^*$ predicted by asymptotic relation.}
% \label{fig:DistributionOpt}
% \end{figure}
% \noindent \textbf{Universal Upper Bound of Optimal size.} There is no distribution such that the optimal shortlist size is larger than $31.62\%n$ (Theorem~\ref{thm:UniversalBound}, asymptotic).

% We devise an $O(n)$ algorithm to find the supremum of optimal size over all distributions for any given $n$ (Proposition~\ref{prop:SupM}), numerical result shows that the asymptotic linear trend holds for very small $n$, therefore designers can eliminate near $68\%$ of the contestants without any worries. 

% \begin{figure}[h]
%     \centering
%     \includegraphics[width=0.33\textwidth]{figure/UpperBound.pdf}
%     \caption{The upper bound of optimal $m$.}
%     \label{fig:universal}
% \end{figure}

% \noindent \textbf{Performance Enhancement.} For any distribution, the 2-contestant winner-take-all contest is $\Theta(\log n)$ times better than best achievable performance without a shortlist (Proposition~\ref{prop:TotalTWOVAN}), and for the optimal contest with a shortlist, it is $\Theta(n)$ times better than that (Theorem~\ref{thm:TotalOPTVAN}, asymptotic). 

% Numerical results shows that the asymptotic approximation ratio is clear even in small $n$, and the performance of the asymptotic optimal contest is almost the same as the actual optimal contest. This indicates that our algorithm produces an contest design that is almost-optimal and well-performing at small scale, with asymptotic optimality and approximation ratio guaranteed. 



% \begin{figure}[h]
% \begin{subfigure}[ht]{0.30\textwidth}
%     \centering
%     \includegraphics[width=\textwidth]{figure/EffortUniTex.pdf}
%     \subcaption{$F(x)=x$}
%     \label{fig:disopt-a}
%     \end{subfigure}
% %\hfill
%           % 子图 (b)
% \begin{subfigure}[ht]{0.30\textwidth}
%     \centering
%     \includegraphics[width=\textwidth]{figure/EffortSqureTex.pdf}
%     \subcaption{$F(x)=x^2$}
%     \label{fig:disopt-b}
%     \end{subfigure}
% %\hfill
% \begin{subfigure}[ht]{0.30\textwidth}
%     \centering
%     \includegraphics[width=\textwidth]{figure/EffortExpTex.pdf}
%     \subcaption{$F(x)=1-e^{-x}$}
%     \label{fig:disopt-c}
%     \end{subfigure}
    
% \caption{dont know what to say here}
% \label{fig:DistributionOpt}
% \end{figure}

%     Finally, we provide a practice guideline that condenses our findings.

% \noindent \textbf{Contest Design Cheatsheet.} A complete simple contest with no more than $31.62\%$ of contestant admitted. Small scale? YES $\to$ Find optimal size by enumeration in $O(n)$, get $\Theta(n)$ / Use 2-contestant winner-take-all, get $\Theta(\log n)$. NO $\to$ Distribution known? NO $\to$ Admit $31.62\%n$, then get $\Theta(n)$. YES $\to$ Solve  asymptotic slope $k \leq 31.62\%$, admit $kn$, and get almost-optimal performance, $\Theta(n)$.

% % How can we numerically find the worst distribution that reaching the highest optimal size? We rely on another representation of total effort, which help us to somehow decouple the contribution of contest structure itself and the distribution.

% % \begin{lemma}[Quantile Representation for Total Effort]\label{lem:QuantileRep}
% %     By using quantile $q:=1-F(x)$ and its reverse function $v(q):=F^{-1}(1-q)=x$, Ex-ante total effort of a simple contest expresses as:
% %     \[
% %     S(m,n, l)= n\int_0^1|v'(q)|\int_0^qG_{(m,l)}(t)\,dt\,dq,
% %     \]where $l$ is the number of prizes, $G_{(m,l)}(t)=\frac{\binom{n-1}{l}(1-t)^{n-l-1}t^{l-1}}{\sum_{j=1}^{m}\binom{n-1}{j-1}(1-t)^{n-j}t^{j-1}}\int_{0}^{t}\sum_{j=1}^{m}\binom{n-1}{j-1}p^{j-1}(1-p)^{n-j}\,dp$. We use $H_{(m,l)}(q):=\int_0^qG_{(m,l)}(t)\,dt$ to denote the distribution-free part.  
% % \end{lemma}

% % Since Theorem~\ref{thm:ConpleteSimpleContest} states that optimal contest is a complete simple contest, i.e., $l = m-1$, we therefore omit $l$ in the following discussion.

% % In this representation, total effort becomes the integration of the multiplication of a function $|v'(q)|$ determined by ability level distribution, and a function $H_{(m,l)}(q)=\int_0^qG_{(m,l)}(t)\,dt$ that is completely decided by the contest structure.

% % Let us focus on the distribution-free part. We can plot $H_{m}(q)$ as a function of $q\in[0,1]$ for $m = 2,\ldots,n$. The example of $n=10$ is shown in Figure~\ref{fig:universal-b}. In this case, we can see that for some $m$ (e.g., $m=3$), $H_{(m)}(q)>H_{(m+1)}(q)$ holds point-wise, thus, $S(m,n) > S(m+1,n)$ stands true for arbitrary distributions, indicating that $m+1$ is a strictly dominated choice. 

% % Actually, it can be shown that $H_{(m)}(q)/H_{(m')}(q)$ is decreasing with $q$ for $m'>m$ (Lemma~\ref{lem:FracDesc}), then $H_{(m)}(1) > H_{(m')}(1)$ is a suffice and necessary condition for $H_{(m)}(q)>H_{(m')}(q)$ point-wise. On the other hand, if $H_{(m)}(1) < H_{(m')}(1)$, then there exists unique $q' \in(0,1)$ such that $H_{(m)}(q')=H_{(m')}(q')$ and $H_{(m)}(q)<H_{(m')}(q)$ afterwards (e.g., the $q'$ for $m=2$ and $m'=3$ is marked with asterisk in Figure~\ref{fig:universal-b}). Since $v'(q)$ can be any positive function, we can always construct a distribution that satisfies $v'(q)=1$ when $q\geq q'$ and $v'(q)=\epsilon$ elsewhere such that $S(m,n) < S(m',n)$ (See an example distribution that make $m'=3$ better than $m=2$ in Figure~\ref{fig:universal-c}, where we let $q' \approx 0.859$, $\epsilon=0.01$, and $S(2,10) \approx 0.481 < 0.489 \approx S(3,10)$.).

% % Therefore, we can find the $m$ that maximize $H_{m}(1)$. For $m'>m$, we have $H_{(m)}(q) > H_{(m')}(q)$ point-wise, so the optimal size can not be more than $m$. For $m'<m$, we have $H_{(m')}(1) < H_{(m)}(1)$, then we can still construct a distribution that satisfies $v'(q)=1$ when $q \geq \max\{\vec{q'}\}$ and $v'(q) = \epsilon$ elsewhere such that $S(m,n) > S(m',n)$ for all $m'<m$, hence we find an instance making $m$ the optimal size. We then conclude that $m$ is the tight upper bound for optimal size for given $n$, as desired.

% % \begin{remark}
% %     The insight from the construction of worst case distribution (e.g., Figure~\ref{fig:universal-c}) is, when almost all of the population are concentrated near the strongest end of ability level, it tends to need larger shortlist size to reach optimality. On the other hand, if highest ability only takes up a little probability mass, or equivalently, $|v'(q)|$ is much higher when $q$ is small, it tends to obtain optimality with fewer contestants. [Mathematical insight]. An uniform distribution, i.e., $|v'(q)|=1$, whose probability mass is evenly distributed, is right in the middle, with $k^*\approx15\%$, as shown in Example~\ref{exam:OptimalUniform}. 
% % \end{remark}

% % \begin{proposition}[Optimal size for Exponential Distribution]\label{prop:OptCapExp}
% %     For exponential distribution, i.e., $F(x)=1-e^{-\lambda x}$ and $f(x)=\lambda e^{-\lambda x}$, when $n \rightarrow \infty$ it holds that:
% %     \[
% %     \lim_{n \rightarrow \infty} m^*(n)/n = 9.70\%,
% %     \]which is independent of the parameter $\lambda$. 
% % \end{proposition}

% % \begin{proposition}[Optimal size for Power-Law Distribution]\label{prop:OptCapPowerLaw}
% %     For power-law distribution $f(x)=cx^{-\alpha-1}, x\ge \delta$ that parametrized by $\alpha \in(0,1]$ and $\delta > 0$. When $n \rightarrow \infty$, $m^*=2$. 
% % \end{proposition}
% % \begin{remark}
% %     Actually, the power-law distribution does not always achieve optimality at extremely small values of \( m \). When \(\alpha > 1\), \( S'(0) \rightarrow +\infty \), and \( S''(k) \) is initially negative and then positive. Consequently, \( S'(k) \) first decreases from \( +\infty \), then increases, eventually reaching \( S'(1) = 0 \). Thus, there exists a point in the interval \( (0,1) \) where \( S'(k) = 0 \), at which \( S \) attains its maximum value. From the condition \( S'(k) = 0 \), it follows that the optimal solution \( k \) satisfies the equation:
% % \[
% % \ln \frac{\alpha+(1-k)^2}{\alpha-(1-k)} + \frac{1}{\alpha}\ln k = 0.
% % \]

% % For large values of \(\alpha\), the term \(\ln \frac{\alpha+(1-k)^2}{\alpha-(1-k)}\) can be approximated as \(\ln \left(1+ \frac{(1-k)+(1-k)^2}{\alpha-(1-k)}\right) \simeq \frac{(1-k)(2-k)}{\alpha}\). Therefore, the equation simplifies to:
% % \[
% % \frac{(1-k)(2-k)}{\alpha} + \frac{1}{\alpha}\ln k = 0.
% % \]

% % The solution to this equation is approximately \( k_2 \approx 31.65\% \), which reaches the worst ratio given by Theorem~\ref{thm:UniversalBound}. 
% % \end{remark}

% %\subsection*{Hello} this is a subsection



\section{Shortlist vs. No Shortlist}
\label{sec: compare}
How much does a contest designer benefit from implementing a shortlist? In this section, we address this question by comparing the optimal contest designs with and without a shortlist under a linear cost function. Specifically, we focus on three types of contests:
%How much can a contest designer benefit from the shortlist option? In this section, we answer this question by comparing the optimal contest design with and without a shortlist, when the cost function is linear. In this section, we mainly focus on three kinds of contests:
\begin{enumerate}
    \item The $n$-contestant winner-take-all contest – This is the optimal contest without a shortlist for both maximum individual effort and total effort \cite{MS01,CHS19}.
    \item The two-contestant winner-take-all contest – This is the optimal contest with a shortlist for the maximum individual effort (by Theorem~\ref{thm:ExpostHighestEffort}).
    \item The complete simple contest with a shortlist size of $m^*$ and $m^*-1$ prizes - This is the optimal contest with a shortlist for the total effort (by Proposition~\ref{thm:ConpleteSimpleContest}).
\end{enumerate}

For any ability distribution $F$ and $n$ initial registered contestants, let $S^{(1)}(m,n,l)$ and $S(m,n,l)$ denote the maximum individual effort and total effort, respectively, in a simple contest with a shortlist size of $m$ and $l\leq m$ prizes. To quantify the gap in objectives, we establish bounds on $S^{(1)}(m,n,l)$ and $S(m,n,l)$ for the three contest types discussed above.

Before deriving bounds on the objectives, we first express the maximum individual effort and total effort objectives in terms of quantiles. Given any ability distribution $F$, recall that the quantile is defined as $q:=1-F(x)$ and its reverse function $v(q):=F^{-1}(1-q)=x$. Using this notation, we obtain the following result.

\begin{lemma}[Quantile Representation for Effort]\label{lem:QuantileRep}
    % By using quantile $q:=1-F(x)$ and its reverse function $v(q):=F^{-1}(1-q)=x$, Ex-ante total effort of a simple contest expresses as:
    % \[
    % S(m,n, l)= n\int_0^1|v'(q)|\int_0^qG_{(m,l)}(t)\,dt\,dq,
    % \]where $l$ is the number of prizes, $G_{(m,l)}(t)=\frac{\binom{n-1}{l}(1-t)^{n-l-1}t^{l-1}}{\sum_{j=1}^{m}\binom{n-1}{j-1}(1-t)^{n-j}t^{j-1}}\int_{0}^{t}\sum_{j=1}^{m}\binom{n-1}{j-1}p^{j-1}(1-p)^{n-j}\,dp$. We use $H_{(m,l)}(q):=\int_0^qG_{(m,l)}(t)\,dt$ to denote the distribution-free part.  
    By using quantile $q:=1-F(x)$ and its reverse function $v(q):=F^{-1}(1-q)=x$, the ex-ante maximum individual effort of a simple contest is:
    \[
    S^{(1)}(m,n, l)= n\int_0^1|v'(q)|\int_0^qG^{(1)}_{(m,l)}(t)\,dt\,dq,
    \]
    where $l$ is the number of prizes and $G_{l,m}^{(1)}(t):=\frac{\binom{n-1}{l}(1-t)^{n-l-1}t^{l-1}}{\sum_{j=1}^{m}\binom{n-1}{j-1}(1-t)^{n-j}t^{j-1}}\int_{0}^{t}(1-p)^{n-1}\,dp.$
    
    And the ex-ante total effort of a simple contest expresses as:
    \[
    S(m,n, l)= n\int_0^1|v'(q)|\int_0^qG_{(m,l)}(t)\,dt\,dq,
    \]similarly, $G_{(m,l)}(t):=\frac{\binom{n-1}{l}(1-t)^{n-l-1}t^{l-1}}{\sum_{j=1}^{m}\binom{n-1}{j-1}(1-t)^{n-j}t^{j-1}}\int_{0}^{t}\sum_{j=1}^{m}\binom{n-1}{j-1}p^{j-1}(1-p)^{n-j}\,dp$, and we use $H(q):=\int_0^qG(t)\,dt$ to denote the distribution-free part.  
\end{lemma}

Based on this representation, we establish the ranges for both the maximum individual effort and the total effort achieved by the $n$-contestant winner-take-all contest and the $2$-contestant winner-take-all contest, respectively, as stated in the following two lemmas.
% Having this representation, we provide the ranges of both the maximum individual effort  and the total effort achieved by the $n$-contestant winner-take-all contest and the $2$-contestant winner-take-all contest, respectively, shown in the following two lemmata. 

\begin{lemma}\label{lem:bound on n,1}
    For any ability distribution, the $n$-contestant winner-take-all contest achieves a maximum individual effort of $S^{(1)}(n, n,1) = \Theta(1)$ and a total effort of $S(n, n,1) = \Theta(1)$.
\end{lemma}
    
\begin{lemma}\label{lem:bound on 2,1}
    For any ability distribution, the $2$-contestant winner-take-all contest achieves a maximum individual effort of $S^{(1)}(2, n,1) = \Theta(\log n)$ and a total effort of $S(2, n,1) = \Theta(\log n)$.
\end{lemma}

On the other hand, using the Beta representation for total effort (Lemma \ref{lem:betaRepTotalEffort}), we can derive the range of total effort achieved by a complete simple contest. 

\begin{lemma}\label{lem:bound on m,m-1}
    Fixed any ability distribution $F$, the optimal complete simple contest with a shortlist size $m^*$ achieves a total effort of $S(m^*,n, m^*-1) = \Theta(n)$, where $m^*$ depends on the distribution $F$.
\end{lemma}

Based on these three lemmas, we compare the optimal contests with and without a shortlist in terms of two types of objectives.

\noindent \textbf{Maximum Individual Effort.}
% For the maximum individual effort objective, we only focus on comparing the $2$-contestant winner-take-all contest with $n$-contestant winner-take-all contest, in term of the maximum individual effort they can achieve, since they are the optimal among all feasible ability distributions. Based on Lemma \label{lem:bound on n,1} and \label{lem:bound on 2,1}, we can obtain the following theorem. 
For the maximum individual effort objective, we only focus on comparing the two-contestant winner-take-all contest with the $n$-contestant winner-take-all contest in terms of the maximum individual effort they can achieve, as these are optimal for all feasible ability distributions. Based on Lemmas \ref{lem:bound on n,1} and \ref{lem:bound on 2,1}, we derive the following theorem.

\begin{theorem}\label{thm: 2,1 vs n,1 max effort}
    For any ability distribution, under the maximum individual effort objective, the two-contestant winner-take-all contest results in $\Theta(\log n)$ times the maximum individual effort of the optimal contest without a shortlist. Specifically, we have:
    $$\frac{S^{(1)}(2,n,1)}{S^{(1)}(n,n,1)} = \Theta(\log n).$$
\end{theorem}

\noindent \textbf{Total Effort.}
For the total effort objective, while the optimal contest with a shortlist is a complete simple contest as shown by Theorem \ref{thm:ConpleteSimpleContest}, the optimal shortlist size depends on the ability distribution. This means the optimal contests vary with different ability functions. To make a meaningful comparison, we focus on a fixed ability distribution $F$ and examine the gap in total effort between the optimal contest with a shortlist under $F$ and the $n$-contestant winner-take-all contest. Based on Lemma \ref{lem:bound on n,1} and \ref{lem:bound on m,m-1}, the comparison of these two contests is summarized in Theorem \ref{thm:TotalOPTVAN}.

\begin{theorem}\label{thm:TotalOPTVAN}
    Fixed any ability distribution $F$, under the total effort objective, the optimal contest with a shortlist can achieve $\Theta(n)$ times the total effort compared to the optimal contest without a shortlist. Specifically,
    $$\frac{S(m^*,n,m^*-1)}{S(n,n,1)} = \Theta(n),$$
    where $m^*$ is the optimal shortlist size for the ability distribution $F$. 
\end{theorem}

Additionally, with respect to total effort, we still can compare the fully shortlisted contest (the 2-contestant winner-take-all contest) with the optimal contest without a shortlist (the $n$-contestant winner-take-all contest), as presented in the following proposition.
\begin{proposition}\label{prop:TotalTWOVAN}
    For any ability distribution, under the total effort objective, the two-contestant winner-take-all contest results in $\Theta(\log n)$ times the total effort of the optimal contest without a shortlist. Specifically, we have:
    $$\frac{S^{(1)}(2,n,1)}{S^{(1)}(n,n,1)} = \Theta(\log n).$$
\end{proposition}



%We discusses the maximum individual effort objective and the total effort objective in Subsection~\ref{subsec:ApproxMaxi} and \ref{subsec:ApproxTotal}, respectively.

% \subsection{Maximum individual Effort}\label{subsec:ApproxMaxi}
% When the objective is the maximum individual effort, we have known that the optimal contest with a shortlist is a two-contestant winner-take-all contest and the optimal contest without a shortlist is a $n$-contestant winner-take-all contest. In the following, we will compare the maximum individual efforts induced by these two contests. 

% Before analyzing the gap between the optimal contests with and without a shortlist, we first rewrite the objective function by the quantile. Given any ability distribution $F$, recall that the quantile $q:=1-F(x)$ and its reverse function $v(q):=F^{-1}(1-q)=x$ and use $S^{(1)}(m,n,l)$ to represent the maximum individual effort obtained by a simple contest with the shortlist size $m$ and $l\leq m$ prizes. Now, we get the following lemma to represent the maximum individual effort objective. 
% \begin{lemma}
%     {\color{red} Add the representation of objective, refer to Lemma A.10}
% \end{lemma}

% Having this representation, we can find that the ratio between these contests is 

% \[
%     \begin{aligned}
%         \frac{S^{(1)}(2,n,1)}{S^{(1)}(n,n,1)} & = \frac{n\int_0^1|v'(q)|H^{(1)}_{(2,1)}(q)\,dq}{n\int_0^1|v'(q)|H^{(1)}_{(n,1)}(q)\,dq} 
%     \end{aligned}
% \]


% \begin{lemma}
%     {\color{red} Add the relationship between $H^{(1)}_{(2,1)}(q)$ and $H^{(1)}_{(n,1)}(q)$ }
% \end{lemma}

% Obtaining the relationship between relationship between $H^{(1)}_{(2,1)}(q)$ and $H^{(1)}_{(n,1)}(q)$, we can get the final ratio. 
% \begin{theorem}
%     For any ability level distribution, under the total effort objective, the 2-contestant winner-take-all contest gives $\Theta(\log n)$ times the effort of the optimal contest without a shortlist, i.e.:
%     $$\frac{S^{(1)}(2,n,1)}{S^{(1)}(n,n,1)} = \Theta(\log n).$$
% \end{theorem}

% \subsection{Total Effort}\label{subsec:ApproxTotal}
% In this subsection, we care about the total effort objective. First, we know that the optimal contest without a shortlist is still a $n$-contestant winner-take-all contest \cite{CHS19}. The optimal contest with a shortlist is a complete simple contest shown by Theorem \ref{thm:ConpleteSimpleContest}, but the optimal shortlist size depends on the ability distribution, which means that the optimal contests are different under different ability functions. Therefore, to make the comparison meaningful, we focus on that fixed any ability distribution $F$, consider the gap of total effort between the current optimal contest with a shortlist under distribution $F$ with the $n$-contestant winner-take-all contest. 


% \begin{lemma}
%     $S(n ,n,1) = \Theta(1).$
% \end{lemma}
    
% \begin{lemma}
%     $S(2,n,1) = \Theta(\log n).$
% \end{lemma}
    
% \begin{lemma}
%     $S(m^*,n, m^*-1) = \Theta(n).$
% \end{lemma}




\section{Towards Practical Applications}\label{sec:practicalApp}

In previous sections, we use asymptotic analysis to characterize and solve the optimal contests, which has a potential application value. In this section, we further provide numerical results to support that our findings fit well even for small $n$, bringing them closer to practical implementation.
In this section, we focus on the total effort objective, the most widely used in practice.

\noindent \textbf{Finding the Optimal Contest.} The optimal contest is a complete simple contest (Proposition~\ref{thm:ConpleteSimpleContest}), and the corresponding shortlist size grows linearly with  $n$. The slope $k$ is determined by the ability distribution and can be identified by solving an equation (Theorem~\ref{thm:OptAsmLinear}, asymptotic).

Figure~\ref{fig:DistributionOpt} illustrates that this linear trend emerges even for very small values of  $n$, with the asymptotic ratio providing a close prediction. Therefore, the optimal contest for any given distribution can be determined efficiently

\begin{figure}[htb]
\centering
\begin{subfigure}[ht]{0.30\textwidth}
    \centering
    \includegraphics[width=\textwidth]{figure/AsmUniTex.pdf}
    \subcaption{$F(x)=x$}
    \label{fig:disopt-a}
    \end{subfigure}
%\hfill
          % 子图 (b)
\begin{subfigure}[ht]{0.30\textwidth}
    \centering
    \includegraphics[width=\textwidth]{figure/AsmSquareTex.pdf}
    \subcaption{$F(x)=x^2$}
    \label{fig:disopt-b}
    \end{subfigure}
%\hfill
\begin{subfigure}[ht]{0.30\textwidth}
    \centering
    \includegraphics[width=\textwidth]{figure/AsmExpTex.pdf}
    \subcaption{$F(x)=1-e^{-x}$}
    \label{fig:disopt-c}
    \end{subfigure}

\caption{The actual optimal size and $m^*$ predicted by asymptotic relation.}
\label{fig:DistributionOpt}
\end{figure}
\noindent \textbf{Universal Upper Bound of the Optimal Size.} There is no distribution such that the optimal shortlist size is larger than $31.62\%n$ (Theorem~\ref{thm:UniversalBound}, asymptotic).
We propose an $O(n)$ algorithm to determine the supremum of the optimal shortlist size across all distributions for any given $n$ (Proposition~\ref{prop:SupM}). Numerical results in Figure~\ref{fig:universal1} confirm that the asymptotic linear trend holds even for very small $n$. Therefore, contest designers can confidently eliminate nearly 68\% of contestants without compromising optimality.


\begin{figure}
    \centering
    \begin{minipage}{0.30\linewidth}
        \centering
        \includegraphics[width=\textwidth]{figure/UpperBound.pdf}
        \caption{Supremum of optimal $m$.}
        \label{fig:universal1}
    \end{minipage}
    %\hfill
    \begin{minipage}{0.55\linewidth}
        \centering
        \includegraphics[width=\textwidth]{figure/flow_chart.pdf}
        \caption{A flow chart for contest design in practice.}
        \label{fig:flowchart}
    \end{minipage}
\end{figure}


% \begin{figure}[h]
%     \centering
%     \includegraphics[width=0.33\textwidth]{figure/UpperBound.pdf}
%     \caption{The upper bound of optimal $m$.}
%     \label{fig:universal}
% \end{figure}

\noindent \textbf{Performance Enhancement.} For any distribution, the two-player winner-take-all contest is $\Theta(\log n)$ times better (Proposition~\ref{prop:TotalTWOVAN}) and the optimal contest with a shortlist, it is $\Theta(n)$ times better (Theorem~\ref{thm:TotalOPTVAN}, asymptotic), compared to the optimal one without a shortlist. Moreover, even when the distribution is unknown, the designer can still attain a $\Theta(n)$ improvement simply by shortlisting to 31.62\% of the contestants, compared to having no shortlist (Corollary~\ref{coro:shortlistAlways}).

In Figure~\ref{fig:DistributionOpt1}, numerical results demonstrate that the asymptotic approximation ratio holds even for small $n$, and the performance of the asymptotic optimal contest is nearly identical to that of the actual optimal contest. This indicates that our algorithm yields a contest design that is not only near-optimal and highly effective at small scales but also guarantees asymptotic optimality and a strong approximation ratio. 

\begin{figure}[h]
\begin{subfigure}[ht]{0.30\textwidth}
    \centering
    \includegraphics[width=\textwidth]{figure/EffortUniTex.pdf}
    \subcaption{$F(x)=x$}
    \label{fig:disopt-a1}
    \end{subfigure}
%\hfill
          % 子图 (b)
\begin{subfigure}[ht]{0.30\textwidth}
    \centering
    \includegraphics[width=\textwidth]{figure/EffortSqureTex.pdf}
    \subcaption{$F(x)=x^2$}
    \label{fig:disopt-b1}
    \end{subfigure}
%\hfill
\begin{subfigure}[ht]{0.30\textwidth}
    \centering
    \includegraphics[width=\textwidth]{figure/EffortExpTex.pdf}
    \subcaption{$F(x)=1-e^{-x}$}
    \label{fig:disopt-c1}
    \end{subfigure}
    
\caption{Total effort performance of different contest designs.}
\label{fig:DistributionOpt1}
\end{figure}

Finally, we summarize our findings into a practical guideline, as illustrated in Figure~\ref{fig:flowchart}.

\noindent \textbf{Contest Design Cheatsheet.} 
%A complete simple contest with no more than $31.62\%$ of contestant admitted. Small scale? YES $\to$ Find optimal capacity by enumeration in $O(n)$, get $\Theta(n)$ / Use 2-player winner-take-all, get $\Theta(\log n)$. NO $\to$ Distribution known? NO $\to$ Admit $31.62\%n$, then get $\Theta(n)$. YES $\to$ Solve  asymptotic slope $k \leq 31.62\%$, admit $kn$, and get almost-optimal performance, $\Theta(n)$.
Is Distribution known? YES $\to$ Solve for the asymptotic slope $k \leq 31.62\%$, admit $kn$, and achieve almost-optimal performance, get $\Theta(n)$. NO $\to$ Is $n$ small? YES $\to$ Use 2-Contestant winner-take-all, get $\Theta(\log n)$. NO $\to$ Admit $31.62\%n$, get $\Theta(n)$. 

% How can we numerically find the worst distribution that reaching the highest optimal capacity? We rely on another representation of total effort, which help us to somehow decouple the contribution of contest structure itself and the distribution.

% \begin{lemma}[Quantile Representation for Total Effort]\label{lem:QuantileRep}
%     By using quantile $q:=1-F(x)$ and its reverse function $v(q):=F^{-1}(1-q)=x$, Ex-ante total effort of a simple contest expresses as:
%     \[
%     S(m,n, l)= n\int_0^1|v'(q)|\int_0^qG_{(m,l)}(t)\,dt\,dq,
%     \]where $l$ is the number of prizes, $G_{(m,l)}(t)=\frac{\binom{n-1}{l}(1-t)^{n-l-1}t^{l-1}}{\sum_{j=1}^{m}\binom{n-1}{j-1}(1-t)^{n-j}t^{j-1}}\int_{0}^{t}\sum_{j=1}^{m}\binom{n-1}{j-1}p^{j-1}(1-p)^{n-j}\,dp$. We use $H_{(m,l)}(q):=\int_0^qG_{(m,l)}(t)\,dt$ to denote the distribution-free part.  
% \end{lemma}

% Since Theorem~\ref{thm:ConpleteSimpleContest} states that optimal contest is a complete simple contest, i.e., $l = m-1$, we therefore omit $l$ in the following discussion.

% In this representation, total effort becomes the integration of the multiplication of a function $|v'(q)|$ determined by ability level distribution, and a function $H_{(m,l)}(q)=\int_0^qG_{(m,l)}(t)\,dt$ that is completely decided by the contest structure.

% Let us focus on the distribution-free part. We can plot $H_{m}(q)$ as a function of $q\in[0,1]$ for $m = 2,\ldots,n$. The example of $n=10$ is shown in Figure~\ref{fig:universal-b}. In this case, we can see that for some $m$ (e.g., $m=3$), $H_{(m)}(q)>H_{(m+1)}(q)$ holds point-wise, thus, $S(m,n) > S(m+1,n)$ stands true for arbitrary distributions, indicating that $m+1$ is a strictly dominated choice. 

% Actually, it can be shown that $H_{(m)}(q)/H_{(m')}(q)$ is decreasing with $q$ for $m'>m$ (Lemma~\ref{lem:FracDesc}), then $H_{(m)}(1) > H_{(m')}(1)$ is a suffice and necessary condition for $H_{(m)}(q)>H_{(m')}(q)$ point-wise. On the other hand, if $H_{(m)}(1) < H_{(m')}(1)$, then there exists unique $q' \in(0,1)$ such that $H_{(m)}(q')=H_{(m')}(q')$ and $H_{(m)}(q)<H_{(m')}(q)$ afterwards (e.g., the $q'$ for $m=2$ and $m'=3$ is marked with asterisk in Figure~\ref{fig:universal-b}). Since $v'(q)$ can be any positive function, we can always construct a distribution that satisfies $v'(q)=1$ when $q\geq q'$ and $v'(q)=\epsilon$ elsewhere such that $S(m,n) < S(m',n)$ (See an example distribution that make $m'=3$ better than $m=2$ in Figure~\ref{fig:universal-c}, where we let $q' \approx 0.859$, $\epsilon=0.01$, and $S(2,10) \approx 0.481 < 0.489 \approx S(3,10)$.).

% Therefore, we can find the $m$ that maximize $H_{m}(1)$. For $m'>m$, we have $H_{(m)}(q) > H_{(m')}(q)$ point-wise, so the optimal capacity can not be more than $m$. For $m'<m$, we have $H_{(m')}(1) < H_{(m)}(1)$, then we can still construct a distribution that satisfies $v'(q)=1$ when $q \geq \max\{\vec{q'}\}$ and $v'(q) = \epsilon$ elsewhere such that $S(m,n) > S(m',n)$ for all $m'<m$, hence we find an instance making $m$ the optimal capacity. We then conclude that $m$ is the tight upper bound for optimal capacity for given $n$, as desired.

% \begin{remark}
%     The insight from the construction of worst case distribution (e.g., Figure~\ref{fig:universal-c}) is, when almost all of the population are concentrated near the strongest end of ability level, it tends to need larger shortlist capacity to reach optimality. On the other hand, if highest ability only takes up a little probability mass, or equivalently, $|v'(q)|$ is much higher when $q$ is small, it tends to obtain optimality with fewer players. [Mathematical insight]. An uniform distribution, i.e., $|v'(q)|=1$, whose probability mass is evenly distributed, is right in the middle, with $k^*\approx15\%$, as shown in Example~\ref{exam:OptimalUniform}. 
% \end{remark}

% \begin{proposition}[Optimal Capacity for Exponential Distribution]\label{prop:OptCapExp}
%     For exponential distribution, i.e., $F(x)=1-e^{-\lambda x}$ and $f(x)=\lambda e^{-\lambda x}$, when $n \rightarrow \infty$ it holds that:
%     \[
%     \lim_{n \rightarrow \infty} m^*(n)/n = 9.70\%,
%     \]which is independent of the parameter $\lambda$. 
% \end{proposition}

% \begin{proposition}[Optimal Capacity for Power-Law Distribution]\label{prop:OptCapPowerLaw}
%     For power-law distribution $f(x)=cx^{-\alpha-1}, x\ge \delta$ that parametrized by $\alpha \in(0,1]$ and $\delta > 0$. When $n \rightarrow \infty$, $m^*=2$. 
% \end{proposition}
% \begin{remark}
%     Actually, the power-law distribution does not always achieve optimality at extremely small values of \( m \). When \(\alpha > 1\), \( S'(0) \rightarrow +\infty \), and \( S''(k) \) is initially negative and then positive. Consequently, \( S'(k) \) first decreases from \( +\infty \), then increases, eventually reaching \( S'(1) = 0 \). Thus, there exists a point in the interval \( (0,1) \) where \( S'(k) = 0 \), at which \( S \) attains its maximum value. From the condition \( S'(k) = 0 \), it follows that the optimal solution \( k \) satisfies the equation:
% \[
% \ln \frac{\alpha+(1-k)^2}{\alpha-(1-k)} + \frac{1}{\alpha}\ln k = 0.
% \]

% For large values of \(\alpha\), the term \(\ln \frac{\alpha+(1-k)^2}{\alpha-(1-k)}\) can be approximated as \(\ln \left(1+ \frac{(1-k)+(1-k)^2}{\alpha-(1-k)}\right) \simeq \frac{(1-k)(2-k)}{\alpha}\). Therefore, the equation simplifies to:
% \[
% \frac{(1-k)(2-k)}{\alpha} + \frac{1}{\alpha}\ln k = 0.
% \]

% The solution to this equation is approximately \( k_2 \approx 31.65\% \), which reaches the worst ratio given by Theorem~\ref{thm:UniversalBound}. 
% \end{remark}

%\subsection*{Hello} this is a subsection


\section{Conclusion}
In this work, we propose a simple yet effective approach, called SMILE, for graph few-shot learning with fewer tasks. Specifically, we introduce a novel dual-level mixup strategy, including within-task and across-task mixup, for enriching the diversity of nodes within each task and the diversity of tasks. Also, we incorporate the degree-based prior information to learn expressive node embeddings. Theoretically, we prove that SMILE effectively enhances the model's generalization performance. Empirically, we conduct extensive experiments on multiple benchmarks and the results suggest that SMILE significantly outperforms other baselines, including both in-domain and cross-domain few-shot settings.

\bibliographystyle{unsrtnat}
\bibliography{sample-bibliography}

\newpage
%\bibliographystyle{ACM-Reference-Format}
%\bibliography{sample-bibliography}

\appendix
\section{Proofs}
\label{sec:appendix}



%\section{Introduction}
%\lipsum[2]
%\lipsum[3]


% \section{Headings: first level}
% \label{sec:headings}

% \lipsum[4] See Section \ref{sec:headings}.

% \subsection{Headings: second level}
% \lipsum[5]
% \begin{equation}
% 	\xi _{ij}(t)=P(x_{t}=i,x_{t+1}=j|y,v,w;\theta)= {\frac {\alpha _{i}(t)a^{w_t}_{ij}\beta _{j}(t+1)b^{v_{t+1}}_{j}(y_{t+1})}{\sum _{i=1}^{N} \sum _{j=1}^{N} \alpha _{i}(t)a^{w_t}_{ij}\beta _{j}(t+1)b^{v_{t+1}}_{j}(y_{t+1})}}
% \end{equation}

% \subsubsection{Headings: third level}
% \lipsum[6]

% \paragraph{Paragraph}
% \lipsum[7]



% \section{Examples of citations, figures, tables, references}
% \label{sec:others}

% \subsection{Citations}
% Citations use \verb+natbib+. The documentation may be found at
% \begin{center}
% 	\url{http://mirrors.ctan.org/macros/latex/contrib/natbib/natnotes.pdf}
% \end{center}

% Here is an example usage of the two main commands (\verb+citet+ and \verb+citep+): Some people thought a thing \citep{kour2014real, hadash2018estimate} but other people thought something else \citep{kour2014fast}. Many people have speculated that if we knew exactly why \citet{kour2014fast} thought this\dots

% \subsection{Figures}
% \lipsum[10]
% See Figure \ref{fig:fig1}. Here is how you add footnotes. \footnote{Sample of the first footnote.}
% \lipsum[11]

% \begin{figure}
% 	\centering
% 	\fbox{\rule[-.5cm]{4cm}{4cm} \rule[-.5cm]{4cm}{0cm}}
% 	\caption{Sample figure caption.}
% 	\label{fig:fig1}
% \end{figure}

% \subsection{Tables}
% See awesome Table~\ref{tab:table}.

% The documentation for \verb+booktabs+ (`Publication quality tables in LaTeX') is available from:
% \begin{center}
% 	\url{https://www.ctan.org/pkg/booktabs}
% \end{center}


% \begin{table}
% 	\caption{Sample table title}
% 	\centering
% 	\begin{tabular}{lll}
% 		\toprule
% 		\multicolumn{2}{c}{Part}                   \\
% 		\cmidrule(r){1-2}
% 		Name     & Description     & Size ($\mu$m) \\
% 		\midrule
% 		Dendrite & Input terminal  & $\sim$100     \\
% 		Axon     & Output terminal & $\sim$10      \\
% 		Soma     & Cell body       & up to $10^6$  \\
% 		\bottomrule
% 	\end{tabular}
% 	\label{tab:table}
% \end{table}

% \subsection{Lists}
% \begin{itemize}
% 	\item Lorem ipsum dolor sit amet
% 	\item consectetur adipiscing elit.
% 	\item Aliquam dignissim blandit est, in dictum tortor gravida eget. In ac rutrum magna.
% \end{itemize}


%\bibliographystyle{unsrtnat}
%\bibliography{sample-bibliography}  %%% Uncomment this line and comment out the ``thebibliography'' section below to use the external .bib file (using bibtex) .


%%% Uncomment this section and comment out the \bibliography{references} line above to use inline references.
% \begin{thebibliography}{1}

% 	\bibitem{kour2014real}
% 	George Kour and Raid Saabne.
% 	\newblock Real-time segmentation of on-line handwritten arabic script.
% 	\newblock In {\em Frontiers in Handwriting Recognition (ICFHR), 2014 14th
% 			International Conference on}, pages 417--422. IEEE, 2014.

% 	\bibitem{kour2014fast}
% 	George Kour and Raid Saabne.
% 	\newblock Fast classification of handwritten on-line arabic characters.
% 	\newblock In {\em Soft Computing and Pattern Recognition (SoCPaR), 2014 6th
% 			International Conference of}, pages 312--318. IEEE, 2014.

% 	\bibitem{hadash2018estimate}
% 	Guy Hadash, Einat Kermany, Boaz Carmeli, Ofer Lavi, George Kour, and Alon
% 	Jacovi.
% 	\newblock Estimate and replace: A novel approach to integrating deep neural
% 	networks with existing applications.
% 	\newblock {\em arXiv preprint arXiv:1804.09028}, 2018.

% \end{thebibliography}


\end{document}
