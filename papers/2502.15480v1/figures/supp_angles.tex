\begin{figure*}[t]  %
  \centering  %
  \small  %
  \newcommand{\mywidthc}{0.02\textwidth}  %
  \newcommand{\mywidthx}{0.45\textwidth}  %
  \newcolumntype{C}{ >{\centering\arraybackslash} m{\mywidthc} } %
  \newcolumntype{X}{ >{\centering\arraybackslash} m{\mywidthx} } %
  \setlength\tabcolsep{1pt} %
  \def\arraystretch{1} %
  \begin{tabular}{XX}
    \includegraphics[width=\mywidthx]{figures/images/view_light_angles.pdf} &
    \includegraphics[width=\mywidthx]{figures/images/rusinkiewicz_angles.pdf}\\
    View-Light Angles & Rusinkiewicz Angles
  \end{tabular}
  \caption{Visualization of the view-light and the Rusinkiewicz angles on the left and on the right, respectively. Shown are view and light direction, $\view$ and $\light$ in a local coordinate system given by surface normal $n$, a surface tangent $t$ (which is arbitrary in the case of isotropic BRDFs) and the binormal $b=n\times t$. The view-light angles are simply the polar angles of $\view$ and $\light$, while the Rusinkiewicz angles are given in terms of the half angle $h=\frac{\view + \light}{\|\view + \light\|}$ and a ``difference'' vector $d$ which is the light direction $\light$ in a coordinate system with $h$ as the normal.}
\label{fig:angles}
\end{figure*}