\begin{figure}[t]  %
  \centering  %
  \small  %
  \newcommand{\mywidthx}{0.16\textwidth}  %
  \newcommand{\mywidtht}{0.035\textwidth}
  \newcommand{\myheightx}{0.15\textwidth}  %
  \newcolumntype{X}{ >{\centering\arraybackslash} m{\mywidthx} } %
  \newcolumntype{T}{ >{\centering\arraybackslash} m{\mywidtht} } %
  
  \setlength\tabcolsep{0pt} %

  \setlength{\extrarowheight}{1.5pt}

  \newcommand{\spyWidth}{1.1cm}
  \newcommand{\spyHeight}{0.75cm}
  \newcommand{\spyMagnification}{1.8}
  \newcommand{\spyBoxX}{1.3}
  \newcommand{\spyBoxY}{0.95}
  \newcommand{\spyMagnificationBoxX}{0.15}
  \newcommand{\spyMagnificationBoxY}{2.25}
  \newcommand{\spyColor}{green}

  \newcommand{\spyTwoWidth}{0.8cm}
  \newcommand{\spyTwoHeight}{1.0cm}
  \newcommand{\spyTwoMagnification}{1.7}
  \newcommand{\spyTwoBoxX}{2.0}
  \newcommand{\spyTwoBoxY}{1.55}
  \newcommand{\spyTwoMagnificationBoxX}{2.0}
  \newcommand{\spyTwoMagnificationBoxY}{0.5}
  \newcommand{\spyTwoColor}{orange}

  \newcommand{\spyThreeWidth}{0.9cm}
  \newcommand{\spyThreeHeight}{1.1cm}
  \newcommand{\spyThreeMagnification}{1.7}
  \newcommand{\spyThreeBoxX}{1.2}
  \newcommand{\spyThreeBoxY}{1.0}
  \newcommand{\spyThreeMagnificationBoxX}{1.8}
  \newcommand{\spyThreeMagnificationBoxY}{2.1}
  \newcommand{\spyThreeColor}{cyan}

  \newcommand{\manualRowSpacing}{0.8cm}
  
  \def\arraystretch{1} %
  \begin{tabular}{XXX}


\begin{tikzpicture}[spy using outlines={rectangle,connect spies}]
\node[anchor=south west,inner sep=0]  at (0,0) {\includegraphics[width=\mywidthx, height=\myheightx, keepaspectratio]{figures/renderings/teaser/spot_merl_chrome_steel_disney_view_02_1_rendering.jpg}};
\spy[color=\spyColor,width=\spyWidth,height=\spyHeight, magnification=\spyMagnification] on (\spyBoxX,\spyBoxY) in node [right] at (\spyMagnificationBoxX,\spyMagnificationBoxY);
\spy[color=\spyTwoColor,width=\spyTwoWidth,height=\spyTwoHeight, magnification=\spyTwoMagnification] on (\spyTwoBoxX,\spyTwoBoxY) in node [right] at (\spyTwoMagnificationBoxX,\spyTwoMagnificationBoxY);
\end{tikzpicture} 
 &
\begin{tikzpicture}[spy using outlines={rectangle,connect spies}]
\node[anchor=south west,inner sep=0]  at (0,0) {\includegraphics[width=\mywidthx, height=\myheightx, keepaspectratio]{figures/renderings/teaser/spot_merl_chrome_steel_add_shared_view_02_1_rendering.jpg}};
\spy[color=\spyColor,width=\spyWidth,height=\spyHeight, magnification=\spyMagnification] on (\spyBoxX,\spyBoxY) in node [right] at (\spyMagnificationBoxX,\spyMagnificationBoxY);
\spy[color=\spyTwoColor,width=\spyTwoWidth,height=\spyTwoHeight, magnification=\spyTwoMagnification] on (\spyTwoBoxX,\spyTwoBoxY) in node [right] at (\spyTwoMagnificationBoxX,\spyTwoMagnificationBoxY);
\end{tikzpicture} 
 &
\begin{tikzpicture}[spy using outlines={rectangle,connect spies}]
\node[anchor=south west,inner sep=0]  at (0,0) {\includegraphics[width=\mywidthx, height=\myheightx, keepaspectratio]{figures/renderings/teaser/spot_merl_chrome_steel_view_02_1_gt.jpg}};
\spy[color=\spyColor,width=\spyWidth,height=\spyHeight, magnification=\spyMagnification] on (\spyBoxX,\spyBoxY) in node [right] at (\spyMagnificationBoxX,\spyMagnificationBoxY);
\spy[color=\spyTwoColor,width=\spyTwoWidth,height=\spyTwoHeight, magnification=\spyTwoMagnification] on (\spyTwoBoxX,\spyTwoBoxY) in node [right] at (\spyTwoMagnificationBoxX,\spyTwoMagnificationBoxY);
\end{tikzpicture} 
\\%[\manualRowSpacing]
\begin{tikzpicture}[spy using outlines={rectangle,connect spies}]
\node[anchor=south west,inner sep=0]  at (0,0) {\includegraphics[width=\mywidthx, height=\myheightx, keepaspectratio]{figures/renderings/teaser/pot2PNG_disney_view_02_1_rendering.jpg}};
\spy[color=\spyThreeColor,width=\spyThreeWidth,height=\spyThreeHeight, magnification=\spyThreeMagnification] on (\spyThreeBoxX,\spyThreeBoxY) in node [right] at (\spyThreeMagnificationBoxX,\spyThreeMagnificationBoxY);
\end{tikzpicture} 
 &
\begin{tikzpicture}[spy using outlines={rectangle,connect spies}]
\node[anchor=south west,inner sep=0]  at (0,0) {\includegraphics[width=\mywidthx, height=\myheightx, keepaspectratio]{figures/renderings/teaser/pot2PNG_add_shared_view_02_1_rendering.jpg}};
\spy[color=\spyThreeColor,width=\spyThreeWidth,height=\spyThreeHeight, magnification=\spyThreeMagnification] on (\spyThreeBoxX,\spyThreeBoxY) in node [right] at (\spyThreeMagnificationBoxX,\spyThreeMagnificationBoxY);
\end{tikzpicture} 
 &
\begin{tikzpicture}[spy using outlines={rectangle,connect spies}]
\node[anchor=south west,inner sep=0]  at (0,0) {\includegraphics[width=\mywidthx, height=\myheightx, keepaspectratio]{figures/renderings/teaser/pot2PNG_view_02_1_gt.jpg}};
\spy[color=\spyThreeColor,width=\spyThreeWidth,height=\spyThreeHeight, magnification=\spyThreeMagnification] on (\spyThreeBoxX,\spyThreeBoxY) in node [right] at (\spyThreeMagnificationBoxX,\spyThreeMagnificationBoxY);
\end{tikzpicture} 
\\[-0.3cm]
Based on Parametric Model (Disney)		%
 &
Purely Neural (Additive Shared)		%
 &
\gt		%
\\

  \end{tabular}
  \caption{
  In this work, we present a thorough comparison of neural approaches for BRDF parametrization, including methods based on parametric models like the Disney BRDF \cite{burley2012physically} and purely neural approaches. We show that while purely neural approaches have advantages for highly specular materials (top row: semi-synthetic, chrome steel from MERL \cite{matusik2003MERL}), we find much less difference between the methods for the DiLiGenT-MV real-world dataset \cite{Li2020DiLiGentMVDataset} (bottom row: pot2 from DiLiGenT-MV).
  }
\label{fig:teaser}
\end{figure}