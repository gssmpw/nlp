\begin{table*}[t]  %
\centering  %
  \footnotesize
    \begin{tabular}{l||ccccc|cccc}
& \multicolumn{5}{c|}{\merlc} & \multicolumn{4}{c}{\diligentc} \\


 & $\Delta$RMSE$^{\sqrt[3]{}}$ & $\Delta$\psnr & $\Delta$\dssim & $\Delta$\lpips & $\Delta$\flip & $\Delta$\psnr & $\Delta$\dssim & $\Delta$\lpips & $\Delta$\flip \\ \hline \hline
Additive Separate (albedo clamp.) 		%
&   \textcolor{red}{+0.90} &   \textcolor{red}{-2.25} &   \textcolor{red}{+0.54} &   \textcolor{red}{+0.84} &   \textcolor{red}{+0.90} &  \textcolor{greenValid}{+0.05} &  0.00 &  \textcolor{greenValid}{-0.02}  &  \textcolor{red}{+0.01} \\
Additive Separate (scalar spec.) 		%
&   \textcolor{red}{+0.41} &   \textcolor{red}{-3.32} &   \textcolor{red}{+0.10} &   \textcolor{red}{+0.34} &   \textcolor{red}{+0.74} &  \textcolor{red}{-0.54} &  \textcolor{red}{+0.01} &  \textcolor{red}{+0.04}  &  \textcolor{red}{+0.32} \\


\end{tabular}
  
\caption{
Effect of using the albedo clamping and a scalar specular term proposed in NeRFactor  \cite{Zhang2021NeRFactor} for the \emph{additive separate} architecture. Shown are the differences to the results in
\iftoggle{arxiv}{\cref{tab:quantitative}}{Tab.~1} in the main text, which reports the reconstruction quality of the additive separate model with neither of the two. We see, that the albedo clamping reduces the reconstruction quality, in particular for the MERL-based data. The clamping prohibits the model from predicting an albedo close to zero, which is necessary, however, for the metallic materials contained in this dataset. See also \cref{sec:supp:qualitative_diffuse_specular} and in particular in \cref{fig:supp_diff_spec_synth_2}. Similarly, the scalar specular term reduces the reconstruction quality for both datasets. We find, that for certain materials, an RGB specularity is necessary for a faithful reconstruction, see \cref{fig:supp:nerfactor_cow_scalar_spec}. RMSE$^{\sqrt[3]{}}$, DSSIM, LPIPS and \FLIP are scaled by 100.
}
\label{tab:supp:changes_quantitative_nerfactor}
\end{table*}