\newcommand\stringNLayersDirs{Number layers view/light direction}
\newcommand\stringReci{Reciprocity Mapping}
\newcommand{\spaceBetweenExps}{0mm}
\newcommand{\spaceAfterMetrics}{0mm}

\begin{table*}[t]  %
\centering  %
  \footnotesize
  \vspace{-0.15cm}
    \begin{tabular}{l||ccccc|cccc}
& \multicolumn{5}{c|}{\merlc} & \multicolumn{4}{c}{\diligentc} \\


 & $\Delta$RMSE$^{\sqrt[3]{}}$ & $\Delta$\psnr & $\Delta$\dssim & $\Delta$\lpips & $\Delta$\flip & $\Delta$\psnr & $\Delta$\dssim & $\Delta$\lpips & $\Delta$\flip \\ \hline \hline
Single MLP$^\ast$ (NOL dirs $6\rightarrow 3$) 		%
&   \textcolor{red}{+0.01} &   \textcolor{red}{-0.15} &   \textcolor{red}{+0.01} &   \textcolor{red}{+0.01} &   \textcolor{red}{+0.02} &  \textcolor{greenValid}{+0.51} &  \textcolor{greenValid}{-0.06} &  \textcolor{greenValid}{-0.10}  &  \textcolor{greenValid}{-0.09} \\
Additive Sep.$^\ast$ (NOL dirs $4\rightarrow 2$) 		%
&   \textcolor{red}{+0.02} &   \textcolor{red}{-0.09} &   \textcolor{red}{+0.01} &   \textcolor{red}{+0.01} &   \textcolor{red}{+0.02} &  \textcolor{greenValid}{+0.26} &  \textcolor{greenValid}{-0.01} &  \textcolor{red}{+0.02}  &  \textcolor{red}{+0.01} \\
Additive Shared$^\ast$ (NOL dirs $2\rightarrow 4$) 		%
&   \textcolor{greenValid}{-0.09} &   \textcolor{greenValid}{+0.71} &   \textcolor{greenValid}{-0.01} &   \textcolor{greenValid}{-0.05} &   \textcolor{greenValid}{-0.25} &  \textcolor{red}{-0.37} &  \textcolor{red}{+0.01} &  \textcolor{red}{+0.01}  &  \textcolor{red}{+0.02} \\


\end{tabular}
\vspace{-0.15cm}
  
  \caption{
  Quantitative changes for varying the number of layers (NOL) for the directions. Shown is the difference to the results in \cref{tab:quantitative}; RMSE$^{\sqrt[3]{}}$, DSSIM, LPIPS and \FLIP are scaled by 100. The experiments confirm that reducing the NOL for the directions tends to increase the reconstruction quality for the real-world data while simultaneously decreasing it slightly for the semi-synthetic data. Increasing the NOL has the opposite effect. We hypothesize, that more layers for the directions enable the model to learn the more complex reflection patterns in the MERL data, while simultaneously making the model less robust to noise that might be contained in the real-world data.
  \vspace{-0.15cm}
  }
\vspace{-0.1cm}
\label{tab:changes_quantitative_NOL}
\end{table*}