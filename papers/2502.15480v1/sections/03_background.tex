\section{Background: BRDF and Rendering}
\label{sec:background}

The \emph{bidirectional reflectance distribution function} (BRDF) describes how a surface reflects incoming light. More precisely, the BRDF $\brdf(\point, \light, \view)$ describes the ratio of the radiance reflected in the viewing direction $\view\in\Sphere^2$ to the irradiance incident from the light direction $\light\in\Sphere^2$ at the position $\point$. 
In this work, we focus on \emph{spatially varying} BRDF (SVBRDF) where $\brdf$ depends on $\point$. For simplicity, however, we use BRDF and SVBRDF interchangeably.

The \emph{rendering equation} 
integrates the reflections of the incident irradiances $\LightIntIn\geq 0$ over the upper hemisphere $\Hemi$ centered around the surface normal to obtain the total radiance $\LightIntOut(\point, \view)$ at position $\point$ in the viewing direction $\view$,
\begin{equation}\label{eq:rendering_eq}
    \LightIntOut(\point, \view) = \int_{\Hemi} \brdf(\point, \light, \view) \LightIntIn(\point, \light) \cos\theta_\light \,\mathrm{d}\light.
\end{equation}
A plausible and physically realistic BRDF needs to fulfill three properties:
\emph{positivity} (\cref{eq:brdf_positivity}), \emph{Helmholtz reciprocity} (\cref{eq:brdf_reciprocity}), and \emph{energy conservation} (\cref{eq:brdf_energy_conservation}),
\begin{align}
    \brdf(\point, \light, \view)\geq 0,\qquad &\forall \point,\, \forall \light,\, \forall \view, \label{eq:brdf_positivity} \\
    \brdf(\point, \light, \view) = \brdf(\point, \view, \light),\qquad &\forall \point,\, \forall \light,\, \forall \view, \label{eq:brdf_reciprocity} \\
    \int_{\Hemi} \brdf(\point, \light, \view)\cos{\theta_{\view}} \,\mathrm{d}\view \le 1,\qquad &\forall \point,\, \forall \light. \label{eq:brdf_energy_conservation}
\end{align}
The first two properties are fulfilled by most state-of-the-art BRDF models~\cite{lafortune1994using,torrance1967theory,burley2012physically}. However, only very few models fulfill energy conservation by construction~\cite{lafortune1994using}. While for special algorithms like (bidirectional) path-tracing, reciprocity and energy conservation ensure convergence, in most cases, it is sufficient to fulfill them only approximately without noticeable artifacts \cite{akenine2019realTimeRendering}.


Mainly two physical processes are responsible for the reflection of light. %
These are often modeled as two separate terms within the BRDF: Surface reflection and subsurface scattering, often called specular and diffuse term.
In the first case, %
light is directly mirrored at the %
surface
and creates a %
view-dependent %
reflection lobe.
In the second case, %
light enters the %
surface and is scattered and partially absorbed %
until a fraction of light is re-emitted. 
While in general, %
subsurface scattering is not purely uniform, \ie, not Lambertian, but depending on the viewing direction~\cite{Oren94OrenNayar}, most models still assume a uniform diffuse reflection.
Light %
being reflected at the surface is %
absent for subsurface scattering~\cite{akenine2019realTimeRendering}, \ie, one can disjointly split the %
light used for surface reflection and for subsurface scattering.

A common special case are isotropic BRDFs which we consider in this paper.
In this case the reflectance at a point does not change if the object is rotated around 
the normal, or in other words, if the relative angle between the light and the view direction remains the same. 
In that case, three angles are sufficient for the BRDF parameterization.
