\section{Introduction}
The physical principles underlying the interaction of light with matter are diverse and complex. The electromagnetic waves are reflected and refracted at material interfaces, and they are scattered and absorbed within materials. These principles need to be taken into account to obtain realistic rendering results. The standard approach to describe an opaque surface is the \emph{bidirectional reflectance distribution function} (BRDF), which for a given light direction describes the amount of light reflected in a specific view direction.

Realistic BRDF modelling is a long-standing research question and countless methods have been proposed. A popular class of approaches aims to explicitly replicate the physical behavior. Some works are mainly phenomenological \cite{Blinn77,Phong75}, others, particularly the microfacet approaches, carefully model the physical principles \cite{CookTorranceT82,trowbridge1975average,walter2007microfacet,smith1967geometrical,schlick1994inexpensive}. While these works rely on a thorough analysis of the underlying physics, they only include a subset of the governing phenomena, limiting them to the modeling choices. 


\begin{figure}
\centering
\begin{subfigure}[b]{0.3\textwidth}
    \centering
    \includegraphics[width=\textwidth]{figures/macarons_1.png}
    \caption{MACARONS (simple scene).}
    \label{fig:teaser_macarons}
\end{subfigure}
\hfill
\begin{subfigure}[b]{0.3\textwidth}
    \centering
    \includegraphics[width=\textwidth]{figures/ours_1_3.png}
    \caption{Our NBP (simple scene).}
    \label{fig:teaser_ours_1}
\end{subfigure}
\hfill
\begin{subfigure}[b]{0.3\textwidth}
    \centering
    \includegraphics[width=0.7\textwidth]{figures/ours_2_1.jpg}
    \caption{Our NBP (hard scene).}
    \label{fig:teaser_ours_2}
\end{subfigure}

\caption{
Reconstruction results and trajectories of MACARONS~\citep{guedon2023macarons} and our NBP model. 
\cite{guedon2023macarons} fails to fully map the environment in simple scenes (a), while our NBP model manages to capture the full scene (b), even in much more complex geometry (c).} 
\label{fig:teaser}
\vspace{-1em}
\end{figure}


% \begin{figure}
%     \centering
%     \begin{tabular}{ccc}
%     \adjustbox{valign=c}{\includegraphics[height=2.7cm]{figures/macarons_1.png}} &
%     \adjustbox{valign=c}{\includegraphics[height=2.7cm]{figures/ours_1_3.png}} &
%     \adjustbox{valign=c}{\includegraphics[height=2.7cm]{figures/ours_2_1.jpg}} \\
%     MACARONS trajectory and &
%     Our trajectory and &
%     Our results for a much\\
%     the resulting reconstruction&
%     the resulting reconstruction&
%     more complex scene\\
%     \end{tabular}
%     \caption{\textbf{Left:} Even in relatively simple scenes, state-of-the-art methods~\citep{guedon2023macarons} can fail to fully map the environment, while our NextBestPath method manages to capture the full scene~(\textbf{middle}), even in much more complex geometry (\textbf{right}).} 
%     \label{fig:teaser}
% \end{figure}

Recently, neural fields have become 
popular for BRDF modeling, allowing for a continuous and resolution-free representation. While several works ultimately rely on parametric physical models, for which the neural field is used to predict the parameters \cite{bi2020neuralReflectanceFields,srinivasan2021nerv,Boss2021NERD,Zhang22IRON,Deschaintre2018SingleImageSVBRDFCaptureDeepNN,Henzler2021GenerativeModellingBRDFFlashIms,Guo2020MaterialGAN,Zhang21PhySG,Zhang2022ModellingIndirIlluminationInvRendering,Brahimi24SuperVol,brahimi24SparseViewsNearLight}, other approaches use neural networks to directly predict the value of the BRDF \cite{sztrajman2021neural,Fan2022NeuralLayeredBRDF,Zhang2021NeRFactor,Sarkar23LitNerf}. Although all neural BRDFs yield impressive results, a thorough comparison between the different approaches is lacking in the literature.

\paragraph{Contribution}

In this work, we 
address this shortcoming and 
perform an exhaustive comparison of different approaches for neural BRDF modeling. In contrast to 
the setting in 
most previous works, we deliberately estimate the reflectance for \emph{given} geometry and \emph{calibrated} light to avoid cross-influences from the joint estimation of shape, material and lighting conditions. This ensures that the capabilities of the approaches can be compared as unobscured as possible. %
The results suggest advantages for purely neural methods over approaches based on parametric models in particular for highly specular materials.

Moreover, we propose two extensions that can be added to existing purely neural approaches: A novel input mapping that ensures reciprocity of the BRDF by construction as well as an enhancement for split-based methods that aims at representing the physical behavior more faithfully. 

We summarize our main contributions as follows:
\begin{enumerate}
    \item We perform an extensive comparison of different neural BRDF approaches, including neural fields for parameters of different physical models and direct BRDF prediction by the neural network.
    \item We analyze the energy and reciprocity constraint for all methods. The results show that neural approaches do not seem to learn reciprocity from data. We propose a modification to ensure reciprocity by construction.
    \item We introduce a novel splitting scheme for the diffuse and specular part of additive neural BRDF models and show that it improves existing methods.
\end{enumerate}


Please see \href{https://florianhofherr.github.io/neural-brdfs}{florianhofherr.github.io/neural-brdfs} and the supplementary material for additional architecture and evaluation details as well as more experiments.
