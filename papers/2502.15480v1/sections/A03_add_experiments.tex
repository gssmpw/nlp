\section{Additional Experimental Results}
\label{sec:supp:add_experimental_results}

In this section, we present several additional results. \cref{sec:supp:quantitative_reciprocity} and \cref{sec:supp:arch_changes_view_light_exps} give further results and details for the experiments on the reciprocity approach and the number of layers for the directions from 
\iftoggle{arxiv}{\cref{sec:analysis_brdf_models}}{Sec.~6.2}
in the main text. In \cref{sec:supp:exps_angle_param}, we analyze the influence of the angle parametrization on the reconstruction quality. \cref{sec:supp:changes_nerfactor} justifies our changes to the NeRFactor architecture, which is the basis of our additive separate architecture.
\cref{sec:supp:failure_cases} gives insight to failure cases for the purely neural additive approaches that we observed occasionally.
While \cref{sec:supp:qualitative_diffuse_specular} presents a qualitative analysis of the diffuse and specular parts of the models based on a split of the reflectance
\cref{sec:supp:spat_var_recon_brdf} uses the albedo maps to analyze how well the models can represent the spatially uniform BRDFs of the semi-synthetic dataset.
Finally, \cref{sec:supp:additional_experiments_comp} provides additional qualitative and quantitative results for both datasets, comparing the various neural BRDF approaches.


\subsection{Quantitative Results for the Reciprocity Approaches}
\label{sec:supp:quantitative_reciprocity}

\begin{table*}[t]  %
\centering  %
  \footnotesize
    \begin{tabular}{l||ccccc|cccc}
& \multicolumn{5}{c|}{\merlc} & \multicolumn{4}{c}{\diligentc} \\


 & $\Delta$RMSE$^{\sqrt[3]{}}$ & $\Delta$\psnr & $\Delta$\dssim & $\Delta$\lpips & $\Delta$\flip & $\Delta$\psnr & $\Delta$\dssim & $\Delta$\lpips & $\Delta$\flip \\ \hline \hline
Single MLP (rnd. in-swap) 		%
&   \textcolor{greenValid}{-0.08} &   \textcolor{greenValid}{+0.04} &   \textcolor{red}{+0.02} &   \textcolor{red}{+0.02} &   \textcolor{red}{+0.02} &  \textcolor{greenValid}{+0.06} &  \textcolor{greenValid}{-0.01} &  \textcolor{greenValid}{-0.01}  &  \textcolor{greenValid}{-0.02} \\
Single MLP (ours) 		%
&   \textcolor{greenValid}{-0.08} &   \textcolor{red}{-0.10} &   \textcolor{red}{+0.02} &   \textcolor{red}{+0.01} &   \textcolor{red}{+0.03} &  \textcolor{red}{-0.30} &  \textcolor{red}{+0.03} &  \textcolor{red}{+0.06}  &  \textcolor{red}{+0.07} \\
Additive Separate (rnd. in-swap) 		%
&   \textcolor{greenValid}{-0.07} &   \textcolor{greenValid}{+0.04} &   \textcolor{red}{+0.02} &   \textcolor{red}{+0.01} &   \textcolor{red}{+0.02} &  0.00 &  0.00 &  0.00  &  \textcolor{red}{+0.01} \\
Additive Separate (ours) 		%
&   \textcolor{greenValid}{-0.07} &   \textcolor{red}{-0.05} &   \textcolor{red}{+0.02} &   \textcolor{red}{+0.02} &   \textcolor{red}{+0.04} &  \textcolor{red}{-0.02} &  0.00 &  \textcolor{red}{+0.01}  &  \textcolor{red}{+0.01} \\
Additive Shared (rnd. in-swap) 		%
&   \textcolor{greenValid}{-0.05} &   \textcolor{red}{-0.10} &   \textcolor{red}{+0.04} &   \textcolor{red}{+0.16} &   \textcolor{greenValid}{-0.12} &  \textcolor{red}{-0.15} &  \textcolor{red}{+0.01} &  \textcolor{red}{+0.01}  &  \textcolor{red}{+0.01} \\
Additive Shared (ours) 		%
&   \textcolor{greenValid}{-0.01} &   \textcolor{red}{-0.21} &   \textcolor{red}{+0.06} &   \textcolor{red}{+0.24} &   \textcolor{greenValid}{-0.05} &  \textcolor{red}{-0.13} &  0.00 &  0.00  &  \textcolor{red}{+0.01} \\


\end{tabular}
  
\caption{
Effect of the reciprocity strategies on the reconstruction quality. Shown are the differences to the results in
\iftoggle{arxiv}{\cref{tab:quantitative}}{Tab.~1} in the main text, which reports the reconstruction quality of the architectures without any reciprocity strategy. With a few exceptions, we see that the influence of both strategies on the results is fairly marginal, with a tendency for slightly worse results. RMSE$^{\sqrt[3]{}}$, DSSIM, LPIPS and \FLIP are scaled by 100.
}
\label{tab:changes_quantitative_reci}
\end{table*}

As shown in 
\iftoggle{arxiv}{\cref{sec:analysis_brdf_models}}{Sec.~6.2}
in the main text, the random input swap of view and light direction applied during training, as described in LitNerf \cite{Sarkar23LitNerf}, can reduce the RMSE of the reciprocity constraint. However, this strategy does not provide a guarantee that the reciprocity is actually fulfilled. In contrast, our input mapping, as proposed in
\iftoggle{arxiv}{\cref{sec:reciMapping}}{Sec.~4.3},
ensures that the reciprocity constraint is exactly fulfilled by construction.
In \cref{tab:changes_quantitative_reci}, we analyze the effect of both strategies on the reconstruction quality by comparing against the results without any reciprocity strategy. Overall, we observe only a minimal difference with a minor tendency for slightly worse results as a trade-off for the fulfilled reciprocity.
One exception is the single MLP architecture, where, for the real-world examples, the random input swap seems to have a more noticeable advantage over our strategy.


\subsection{Architecture Changes Experiments: Number of Layers for View/Light Direction}
\label{sec:supp:arch_changes_view_light_exps}

In
\iftoggle{arxiv}{\cref{sec:analysis_brdf_models}}{Sec.~6.2}
in the main text, we analyzed the influence of the number of layers (NOL) for the directions on the reconstruction quality. In the following, we provide detailed architectural changes for the different models.

\begin{itemize}
    \item For the single MLP architecture, we feed the directions in at layer 4 instead of 1, decreasing the NOL for the directions from 6 to 3.
    \item For the additive separate architecture, we reduce the NOL of the specular MLP from 4 to 2. Also, we remove the input skip.
    \item For the additive shared architecture, we reduce the NOL of the shared MLP from 5 to 3 while increasing the NOL of the diffuse and specular MLPs (and therefore NOL for the directions) by 2, respectively.
\end{itemize}


\subsection{Influence of the Angle Parametrization}
\label{sec:supp:exps_angle_param}

\begin{table*}[t]  %
\centering  %
  \footnotesize
    \begin{tabular}{l||ccccc|cccc}
& \multicolumn{5}{c|}{\merlc} & \multicolumn{4}{c}{\diligentc} \\


 & $\Delta$RMSE$^{\sqrt[3]{}}$ & $\Delta$\psnr & $\Delta$\dssim & $\Delta$\lpips & $\Delta$\flip & $\Delta$\psnr & $\Delta$\dssim & $\Delta$\lpips & $\Delta$\flip \\ \hline \hline
Single MLP (view-light) 		%
&   \textcolor{red}{+0.11} &   \textcolor{red}{-0.86} &   \textcolor{red}{+0.05} &   \textcolor{red}{+0.14} &   \textcolor{red}{+0.15} &  \textcolor{red}{-0.10} &  \textcolor{red}{+0.01} &  \textcolor{red}{+0.01}  &  \textcolor{red}{+0.01} \\
Additive Separate (view-light) 		%
&   \textcolor{red}{+0.11} &   \textcolor{red}{-0.85} &   \textcolor{red}{+0.04} &   \textcolor{red}{+0.12} &   \textcolor{red}{+0.16} &  \textcolor{red}{-0.03} &  \textcolor{red}{+0.01} &  \textcolor{greenValid}{-0.02}  &  \textcolor{red}{+0.01} \\
Additive Shared (view-light) 		%
&   \textcolor{red}{+0.17} &   \textcolor{red}{-1.01} &   \textcolor{red}{+0.05} &   \textcolor{red}{+0.09} &   \textcolor{red}{+0.07} &  \textcolor{red}{-0.07} &  \textcolor{red}{+0.01} &  \textcolor{greenValid}{-0.01}  &  0.00 \\


\end{tabular}
  
\caption{
Effect of using the view-light angles instead of the Rusinkiewicz angles as a parametrization of the directions. Shown are the differences to the results in
\iftoggle{arxiv}{\cref{tab:quantitative}}{Tab.~1} in the main text, which reports the reconstruction quality of the architectures with the Rusinkiewicz angles. The results show that overall, the view-light parametrization reduces the reduction quality compared to the Rusinkiewicz parametrization. The difference is more significant for the semi-synthetic MERL-based dataset. The reason is that this data contains more highly specular materials with complex reflective patterns. The alignment of the specular peaks with the coordinate axes provided by the Rusinkiewicz angles seems to provide a significant benefit in this case. RMSE$^{\sqrt[3]{}}$, DSSIM, LPIPS and \FLIP are scaled by 100.
}
\label{tab:changes_quantitative_view_light}
\end{table*}

Following previous work \cite{Zhang2021NeRFactor, sztrajman2021neural}, we use the Rusinkiewicz angles \cite{rusinkiewicz1998new} to parametrize the directions as inputs for the MLPs; see \cref{sec:supp:angle_definitions} for a review and discussion. As noted by Rusinkiewicz, this parametrization aligns the specular peaks better with the coordinate axes, which benefits learning highly specular materials. The results in \cref{tab:changes_quantitative_view_light} confirm that using the view-light angles as parametrization for purely neural BRDF models reduces the reconstruction quality compared to the Rusinkiewicz angles. Moreover, we see that, indeed, this effect is more prominent for the MERL-based semi-synthetic dataset, which contains more highly specular materials.


\subsection{Changes from the NeRFactor Architecture}
\label{sec:supp:changes_nerfactor}

\begin{table*}[t]  %
\centering  %
  \footnotesize
    \begin{tabular}{l||ccccc|cccc}
& \multicolumn{5}{c|}{\merlc} & \multicolumn{4}{c}{\diligentc} \\


 & $\Delta$RMSE$^{\sqrt[3]{}}$ & $\Delta$\psnr & $\Delta$\dssim & $\Delta$\lpips & $\Delta$\flip & $\Delta$\psnr & $\Delta$\dssim & $\Delta$\lpips & $\Delta$\flip \\ \hline \hline
Additive Separate (albedo clamp.) 		%
&   \textcolor{red}{+0.90} &   \textcolor{red}{-2.25} &   \textcolor{red}{+0.54} &   \textcolor{red}{+0.84} &   \textcolor{red}{+0.90} &  \textcolor{greenValid}{+0.05} &  0.00 &  \textcolor{greenValid}{-0.02}  &  \textcolor{red}{+0.01} \\
Additive Separate (scalar spec.) 		%
&   \textcolor{red}{+0.41} &   \textcolor{red}{-3.32} &   \textcolor{red}{+0.10} &   \textcolor{red}{+0.34} &   \textcolor{red}{+0.74} &  \textcolor{red}{-0.54} &  \textcolor{red}{+0.01} &  \textcolor{red}{+0.04}  &  \textcolor{red}{+0.32} \\


\end{tabular}
  
\caption{
Effect of using the albedo clamping and a scalar specular term proposed in NeRFactor  \cite{Zhang2021NeRFactor} for the \emph{additive separate} architecture. Shown are the differences to the results in
\iftoggle{arxiv}{\cref{tab:quantitative}}{Tab.~1} in the main text, which reports the reconstruction quality of the additive separate model with neither of the two. We see, that the albedo clamping reduces the reconstruction quality, in particular for the MERL-based data. The clamping prohibits the model from predicting an albedo close to zero, which is necessary, however, for the metallic materials contained in this dataset. See also \cref{sec:supp:qualitative_diffuse_specular} and in particular in \cref{fig:supp_diff_spec_synth_2}. Similarly, the scalar specular term reduces the reconstruction quality for both datasets. We find, that for certain materials, an RGB specularity is necessary for a faithful reconstruction, see \cref{fig:supp:nerfactor_cow_scalar_spec}. RMSE$^{\sqrt[3]{}}$, DSSIM, LPIPS and \FLIP are scaled by 100.
}
\label{tab:supp:changes_quantitative_nerfactor}
\end{table*}
\begin{figure}
    \centering
    
    \begin{tabular}{cc}
    
     \includegraphics[width=0.38\linewidth]{figures/renderings/nerfactor_arch/cowPNG_add_split_scalar_spec_view_02_1_rendering.jpg}
    &
    \includegraphics[width=0.38\linewidth]{figures/renderings/nerfactor_arch/cowPNG_view_02_1_gt.jpg} \\
    Scalar Specularity 
    &
    GT
    \end{tabular}    
    
\caption{
    Qualitative BRDF reconstruction for the cow object from the  DiLiGenT-MV dataset \cite{Li2020DiLiGentMVDataset} for the additive separate architecture with a scalar specular term (as suggested in NeRFactor \cite{Zhang2021NeRFactor}). The results show, that a scalar specular term is unable to reconstruct the reflectance of this object and creates a spurious glow. This indicates that for some materials, a specular term with 3 channels is necessary to yield high-quality reconstructions.
}
\label{fig:supp:nerfactor_cow_scalar_spec}
\end{figure}

While our additive separate architecture is based on NeRFactor \cite{Zhang2021NeRFactor}, we made two changes, which improved the results for our data significantly. In this section, we present the comparison to justify these adjustments.

As a first change, we remove the albedo clamping. The original work empirically constrains the diffuse reflection (\ie the albedo) to $[0.03, 0.8]$. Since we work in linear space, we transform these values from sRGB to linear space, which yields the range $[0.0023, 0.6038]$. The results in \cref{tab:supp:changes_quantitative_nerfactor} show that we obtain significantly worse results for the semi-synthetic MERL-based data. The reason lies in the metallic materials contained in this dataset. Metals show almost no subsurface scattering due to the free electrons \cite{akenine2019realTimeRendering}. As demonstrated in \cref{sec:supp:qualitative_diffuse_specular} and visible in particular in \cref{fig:supp_diff_spec_synth_2}, all models replicate this behavior and show almost no diffuse contribution. However, the lower bound on the diffuse part imposed by the albedo clamping prohibits a negligible contribution of the albedo, and we observe a dark gray base color for the diffuse renderings. This leads to the decrease in construction quality. While we see slight improvements with the albedo clamping for non-metal materials and the real-world data, we still decided to remove it due to the significant performance decrease for metallic materials.

As a second change, we use an RGB specular term instead of a scalar one. While the original work assumes, that all color information can be handeled by the albedo network, we find, that for certain materials, a colored specular part is necessary for good reconstructions. The most extreme example we observed is the cow object from the DiLiGenT-MV dataset \cite{Li2020DiLiGentMVDataset} as shown in \cref{fig:supp:nerfactor_cow_scalar_spec}. We can clearly see, that an approach with a scalar specular term yields an inaccurate glow effect that is not observed for models with an RGB specular term (\cf
the rendering for the additive separate architecture in
\iftoggle{arxiv}{\cref{fig:evaluation_renderings}}{Fig.~3}
in the main text and \cref{fig:supp_diff_spec_real} in the appendix).
While the effect is less prominent for other materials, \cref{tab:supp:changes_quantitative_nerfactor} confirms that a scalar specular term instead of an RGB one yields systematically worse results for both datasets. Recall that we observe a similar effect on the cow for the FMBRDF model \cite{ichikawa2023fresnel}, which also employs a scalar specular term; \cf
\iftoggle{arxiv}{\cref{sec:comparison_brdf_models}}{Sec.~6.1}
in the main text.

\subsection{Failure Cases}
\label{sec:supp:failure_cases}

\begin{figure}
    \centering
    
    \begin{tabular}{ccc}
    
     \includegraphics[width=0.31\linewidth]{figures/renderings/failure_softplus/scaling_1.jpg}
    &
    \includegraphics[width=0.31\linewidth]{figures/renderings/failure_softplus/scaling_p5.jpg}
    &
    \includegraphics[width=0.31\linewidth]{figures/renderings/failure_softplus/gt.jpg} \\
    No Scaling
    &
    Output Scaling 0.5
    &
    GT
    \end{tabular}    
    
\caption{
    Failure case for the additive shared architecture. For additive purely neural methods with softplus output nonlinearity, we observed occasional failures like this for very shiny materials. As described in \cref{sec:supp:softplusScaling}, scaling the output of the softplus function solves this issue.
}
\label{fig:supp:failure_softplus}
\end{figure}

Although all methods show quite stable convergence with the chosen parameters, we observed occasional issues with very shiny materials for the additive purely neural models that employ a softplus output nonlinearity. \cref{fig:supp:failure_softplus} shows a failure example for the additive shared architecture. As described in \cref{sec:supp:softplusScaling}, scaling the output of the softplus function solves this issue.



\subsection{Qualitative Results Diffuse and Specular Split}
\label{sec:supp:qualitative_diffuse_specular}

\begin{figure*}[t]  %
  \centering  %
  \footnotesize
  \newcommand{\mywidthc}{0.02\textwidth}  %
  \newcommand{\mywidthx}{0.10\textwidth}  %
  \newcommand{\mywidthw}{0.008\textwidth}  %
  \newcommand{\myheightx}{0.15\textwidth}  %
  \newcommand{\mywidtht}{0.04\textwidth}  %
  \newcolumntype{C}{ >{\centering\arraybackslash} m{\mywidthc} } %
  \newcolumntype{X}{ >{\centering\arraybackslash} m{\mywidthx} } %
  \newcolumntype{W}{ >{\centering\arraybackslash} m{\mywidthw} } %
  \newcolumntype{T}{ >{\centering\arraybackslash} m{\mywidtht} } %

  \newcommand{\heightcolorbar}{0.10\textwidth}  %
  \newcommand{\xposOne}{-0.95}
  \newcommand{\yposOne}{0.35}
  \newcommand{\xposTwo}{0.35}
  \newcommand{\yposTwo}{0.8}

  \newcommand{\fontsizePSNR}{\ssmall}
  
  \setlength\tabcolsep{0pt} %

  \setlength{\extrarowheight}{1.25pt}
  
  \def\arraystretch{0.8} %
  \begin{tabular}{TTXXXXXXXXWX}

\multirow{2}{*}{\rotatebox{90}{\parbox{3cm}{\centering \diligentc \\ cow}}} 
&
\rotatebox{90}{\centering \emph{diffuse}} 
&
\includegraphics[width=\mywidthx, height=\myheightx, keepaspectratio]{figures/renderings/appendix_diff_spec/cowPNG_phong_view_02_1_diffuse.jpg}
&
\includegraphics[width=\mywidthx, height=\myheightx, keepaspectratio]{figures/renderings/appendix_diff_spec/cowPNG_micro_view_02_1_diffuse.jpg}
&
\includegraphics[width=\mywidthx, height=\myheightx, keepaspectratio]{figures/renderings/appendix_diff_spec/cowPNG_fmbrdf_view_02_1_diffuse.jpg}
&
\includegraphics[width=\mywidthx, height=\myheightx, keepaspectratio]{figures/renderings/appendix_diff_spec/cowPNG_disney_view_02_1_diffuse.jpg}
&
\includegraphics[width=\mywidthx, height=\myheightx, keepaspectratio]{figures/renderings/appendix_diff_spec/cowPNG_add_sep_view_02_1_diffuse.jpg}
&
\includegraphics[width=\mywidthx, height=\myheightx, keepaspectratio]{figures/renderings/appendix_diff_spec/cowPNG_add_shared_view_02_1_diffuse.jpg}
&
\includegraphics[width=\mywidthx, height=\myheightx, keepaspectratio]{figures/renderings/appendix_diff_spec/cowPNG_add_sep_reduce_view_02_1_diffuse.jpg}
&
\includegraphics[width=\mywidthx, height=\myheightx, keepaspectratio]{figures/renderings/appendix_diff_spec/cowPNG_add_shared_reduce_view_02_1_diffuse.jpg}
&
&
\\
&
\rotatebox{90}{\centering \emph{specular}} 
&
\includegraphics[width=\mywidthx, height=\myheightx, keepaspectratio]{figures/renderings/appendix_diff_spec/cowPNG_phong_view_02_1_specular.jpg}
&
\includegraphics[width=\mywidthx, height=\myheightx, keepaspectratio]{figures/renderings/appendix_diff_spec/cowPNG_micro_view_02_1_specular.jpg}
&
\includegraphics[width=\mywidthx, height=\myheightx, keepaspectratio]{figures/renderings/appendix_diff_spec/cowPNG_fmbrdf_view_02_1_specular.jpg}
&
\includegraphics[width=\mywidthx, height=\myheightx, keepaspectratio]{figures/renderings/appendix_diff_spec/cowPNG_disney_view_02_1_specular.jpg}
&
\includegraphics[width=\mywidthx, height=\myheightx, keepaspectratio]{figures/renderings/appendix_diff_spec/cowPNG_add_sep_view_02_1_specular.jpg}
&
\includegraphics[width=\mywidthx, height=\myheightx, keepaspectratio]{figures/renderings/appendix_diff_spec/cowPNG_add_shared_view_02_1_specular.jpg}
&
\includegraphics[width=\mywidthx, height=\myheightx, keepaspectratio]{figures/renderings/appendix_diff_spec/cowPNG_add_sep_reduce_view_02_1_specular.jpg}
&
\includegraphics[width=\mywidthx, height=\myheightx, keepaspectratio]{figures/renderings/appendix_diff_spec/cowPNG_add_shared_reduce_view_02_1_specular.jpg}
&
&
\\
&
\rotatebox{90}{\centering \emph{added}} 
&
\includegraphics[width=\mywidthx, height=\myheightx, keepaspectratio]{figures/renderings/appendix_diff_spec/cowPNG_phong_view_02_1_rendering.jpg}
&
\includegraphics[width=\mywidthx, height=\myheightx, keepaspectratio]{figures/renderings/appendix_diff_spec/cowPNG_micro_view_02_1_rendering.jpg}
&
\includegraphics[width=\mywidthx, height=\myheightx, keepaspectratio]{figures/renderings/appendix_diff_spec/cowPNG_fmbrdf_view_02_1_rendering.jpg}
&
\includegraphics[width=\mywidthx, height=\myheightx, keepaspectratio]{figures/renderings/appendix_diff_spec/cowPNG_disney_view_02_1_rendering.jpg}
&
\includegraphics[width=\mywidthx, height=\myheightx, keepaspectratio]{figures/renderings/appendix_diff_spec/cowPNG_add_sep_view_02_1_rendering.jpg}
&
\includegraphics[width=\mywidthx, height=\myheightx, keepaspectratio]{figures/renderings/appendix_diff_spec/cowPNG_add_shared_view_02_1_rendering.jpg}
&
\includegraphics[width=\mywidthx, height=\myheightx, keepaspectratio]{figures/renderings/appendix_diff_spec/cowPNG_add_sep_reduce_view_02_1_rendering.jpg}
&
\includegraphics[width=\mywidthx, height=\myheightx, keepaspectratio]{figures/renderings/appendix_diff_spec/cowPNG_add_shared_reduce_view_02_1_rendering.jpg}
&
&
\includegraphics[width=\mywidthx, height=\myheightx, keepaspectratio]{figures/renderings/appendix_diff_spec/cowPNG_view_02_1_gt.jpg}\\
\hline\hline\\
\multirow{2}{*}{\rotatebox{90}{\parbox{3cm}{\centering \diligentc \\ pot2}}} 
&
\rotatebox{90}{\centering \emph{diffuse}} 
&
\includegraphics[width=\mywidthx, height=\myheightx, keepaspectratio]{figures/renderings/appendix_diff_spec/pot2PNG_phong_view_02_1_diffuse.jpg}
&
\includegraphics[width=\mywidthx, height=\myheightx, keepaspectratio]{figures/renderings/appendix_diff_spec/pot2PNG_micro_view_02_1_diffuse.jpg}
&
\includegraphics[width=\mywidthx, height=\myheightx, keepaspectratio]{figures/renderings/appendix_diff_spec/pot2PNG_fmbrdf_view_02_1_diffuse.jpg}
&
\includegraphics[width=\mywidthx, height=\myheightx, keepaspectratio]{figures/renderings/appendix_diff_spec/pot2PNG_disney_view_02_1_diffuse.jpg}
&
\includegraphics[width=\mywidthx, height=\myheightx, keepaspectratio]{figures/renderings/appendix_diff_spec/pot2PNG_add_sep_view_02_1_diffuse.jpg}
&
\includegraphics[width=\mywidthx, height=\myheightx, keepaspectratio]{figures/renderings/appendix_diff_spec/pot2PNG_add_shared_view_02_1_diffuse.jpg}
&
\includegraphics[width=\mywidthx, height=\myheightx, keepaspectratio]{figures/renderings/appendix_diff_spec/pot2PNG_add_sep_reduce_view_02_1_diffuse.jpg}
&
\includegraphics[width=\mywidthx, height=\myheightx, keepaspectratio]{figures/renderings/appendix_diff_spec/pot2PNG_add_shared_reduce_view_02_1_diffuse.jpg}
&
&
\\
&
\rotatebox{90}{\centering \emph{specular}} 
&
\includegraphics[width=\mywidthx, height=\myheightx, keepaspectratio]{figures/renderings/appendix_diff_spec/pot2PNG_phong_view_02_1_specular.jpg}
&
\includegraphics[width=\mywidthx, height=\myheightx, keepaspectratio]{figures/renderings/appendix_diff_spec/pot2PNG_micro_view_02_1_specular.jpg}
&
\includegraphics[width=\mywidthx, height=\myheightx, keepaspectratio]{figures/renderings/appendix_diff_spec/pot2PNG_fmbrdf_view_02_1_specular.jpg}
&
\includegraphics[width=\mywidthx, height=\myheightx, keepaspectratio]{figures/renderings/appendix_diff_spec/pot2PNG_disney_view_02_1_specular.jpg}
&
\includegraphics[width=\mywidthx, height=\myheightx, keepaspectratio]{figures/renderings/appendix_diff_spec/pot2PNG_add_sep_view_02_1_specular.jpg}
&
\includegraphics[width=\mywidthx, height=\myheightx, keepaspectratio]{figures/renderings/appendix_diff_spec/pot2PNG_add_shared_view_02_1_specular.jpg}
&
\includegraphics[width=\mywidthx, height=\myheightx, keepaspectratio]{figures/renderings/appendix_diff_spec/pot2PNG_add_sep_reduce_view_02_1_specular.jpg}
&
\includegraphics[width=\mywidthx, height=\myheightx, keepaspectratio]{figures/renderings/appendix_diff_spec/pot2PNG_add_shared_reduce_view_02_1_specular.jpg}
&
&
\\
&
\rotatebox{90}{\centering \emph{added}} 
&
\includegraphics[width=\mywidthx, height=\myheightx, keepaspectratio]{figures/renderings/appendix_diff_spec/pot2PNG_phong_view_02_1_rendering.jpg}
&
\includegraphics[width=\mywidthx, height=\myheightx, keepaspectratio]{figures/renderings/appendix_diff_spec/pot2PNG_micro_view_02_1_rendering.jpg}
&
\includegraphics[width=\mywidthx, height=\myheightx, keepaspectratio]{figures/renderings/appendix_diff_spec/pot2PNG_fmbrdf_view_02_1_rendering.jpg}
&
\includegraphics[width=\mywidthx, height=\myheightx, keepaspectratio]{figures/renderings/appendix_diff_spec/pot2PNG_disney_view_02_1_rendering.jpg}
&
\includegraphics[width=\mywidthx, height=\myheightx, keepaspectratio]{figures/renderings/appendix_diff_spec/pot2PNG_add_sep_view_02_1_rendering.jpg}
&
\includegraphics[width=\mywidthx, height=\myheightx, keepaspectratio]{figures/renderings/appendix_diff_spec/pot2PNG_add_shared_view_02_1_rendering.jpg}
&
\includegraphics[width=\mywidthx, height=\myheightx, keepaspectratio]{figures/renderings/appendix_diff_spec/pot2PNG_add_sep_reduce_view_02_1_rendering.jpg}
&
\includegraphics[width=\mywidthx, height=\myheightx, keepaspectratio]{figures/renderings/appendix_diff_spec/pot2PNG_add_shared_reduce_view_02_1_rendering.jpg}
&
&
\includegraphics[width=\mywidthx, height=\myheightx, keepaspectratio]{figures/renderings/appendix_diff_spec/pot2PNG_view_02_1_gt.jpg}\\
\hline\hline\\
\multirow{2}{*}{\rotatebox{90}{\parbox{3cm}{\centering \diligentc \\ buddha}}} 
&
\rotatebox{90}{\centering \emph{diffuse}} 
&
\includegraphics[width=\mywidthx, height=\myheightx, keepaspectratio]{figures/renderings/appendix_diff_spec/buddhaPNG_phong_view_02_1_diffuse.jpg}
&
\includegraphics[width=\mywidthx, height=\myheightx, keepaspectratio]{figures/renderings/appendix_diff_spec/buddhaPNG_micro_view_02_1_diffuse.jpg}
&
\includegraphics[width=\mywidthx, height=\myheightx, keepaspectratio]{figures/renderings/appendix_diff_spec/buddhaPNG_fmbrdf_view_02_1_diffuse.jpg}
&
\includegraphics[width=\mywidthx, height=\myheightx, keepaspectratio]{figures/renderings/appendix_diff_spec/buddhaPNG_disney_view_02_1_diffuse.jpg}
&
\includegraphics[width=\mywidthx, height=\myheightx, keepaspectratio]{figures/renderings/appendix_diff_spec/buddhaPNG_add_sep_view_02_1_diffuse.jpg}
&
\includegraphics[width=\mywidthx, height=\myheightx, keepaspectratio]{figures/renderings/appendix_diff_spec/buddhaPNG_add_shared_view_02_1_diffuse.jpg}
&
\includegraphics[width=\mywidthx, height=\myheightx, keepaspectratio]{figures/renderings/appendix_diff_spec/buddhaPNG_add_sep_reduce_view_02_1_diffuse.jpg}
&
\includegraphics[width=\mywidthx, height=\myheightx, keepaspectratio]{figures/renderings/appendix_diff_spec/buddhaPNG_add_shared_reduce_view_02_1_diffuse.jpg}
&
&
\\
&
\rotatebox{90}{\centering \emph{specular}} 
&
\includegraphics[width=\mywidthx, height=\myheightx, keepaspectratio]{figures/renderings/appendix_diff_spec/buddhaPNG_phong_view_02_1_specular.jpg}
&
\includegraphics[width=\mywidthx, height=\myheightx, keepaspectratio]{figures/renderings/appendix_diff_spec/buddhaPNG_micro_view_02_1_specular.jpg}
&
\includegraphics[width=\mywidthx, height=\myheightx, keepaspectratio]{figures/renderings/appendix_diff_spec/buddhaPNG_fmbrdf_view_02_1_specular.jpg}
&
\includegraphics[width=\mywidthx, height=\myheightx, keepaspectratio]{figures/renderings/appendix_diff_spec/buddhaPNG_disney_view_02_1_specular.jpg}
&
\includegraphics[width=\mywidthx, height=\myheightx, keepaspectratio]{figures/renderings/appendix_diff_spec/buddhaPNG_add_sep_view_02_1_specular.jpg}
&
\includegraphics[width=\mywidthx, height=\myheightx, keepaspectratio]{figures/renderings/appendix_diff_spec/buddhaPNG_add_shared_view_02_1_specular.jpg}
&
\includegraphics[width=\mywidthx, height=\myheightx, keepaspectratio]{figures/renderings/appendix_diff_spec/buddhaPNG_add_sep_reduce_view_02_1_specular.jpg}
&
\includegraphics[width=\mywidthx, height=\myheightx, keepaspectratio]{figures/renderings/appendix_diff_spec/buddhaPNG_add_shared_reduce_view_02_1_specular.jpg}
&
&
\\
&
\rotatebox{90}{\centering \emph{added}} 
&
\includegraphics[width=\mywidthx, height=\myheightx, keepaspectratio]{figures/renderings/appendix_diff_spec/buddhaPNG_phong_view_02_1_rendering.jpg}
&
\includegraphics[width=\mywidthx, height=\myheightx, keepaspectratio]{figures/renderings/appendix_diff_spec/buddhaPNG_micro_view_02_1_rendering.jpg}
&
\includegraphics[width=\mywidthx, height=\myheightx, keepaspectratio]{figures/renderings/appendix_diff_spec/buddhaPNG_fmbrdf_view_02_1_rendering.jpg}
&
\includegraphics[width=\mywidthx, height=\myheightx, keepaspectratio]{figures/renderings/appendix_diff_spec/buddhaPNG_disney_view_02_1_rendering.jpg}
&
\includegraphics[width=\mywidthx, height=\myheightx, keepaspectratio]{figures/renderings/appendix_diff_spec/buddhaPNG_add_sep_view_02_1_rendering.jpg}
&
\includegraphics[width=\mywidthx, height=\myheightx, keepaspectratio]{figures/renderings/appendix_diff_spec/buddhaPNG_add_shared_view_02_1_rendering.jpg}
&
\includegraphics[width=\mywidthx, height=\myheightx, keepaspectratio]{figures/renderings/appendix_diff_spec/buddhaPNG_add_sep_reduce_view_02_1_rendering.jpg}
&
\includegraphics[width=\mywidthx, height=\myheightx, keepaspectratio]{figures/renderings/appendix_diff_spec/buddhaPNG_add_shared_reduce_view_02_1_rendering.jpg}
&
&
\includegraphics[width=\mywidthx, height=\myheightx, keepaspectratio]{figures/renderings/appendix_diff_spec/buddhaPNG_view_02_1_gt.jpg}\\
\hline\hline\\[-0.2cm]
&
 & \cellcolor{cellParamBased}\rpc		%
 & \cellcolor{cellParamBased}\tsc		%
 & \cellcolor{cellParamBased}\fmbrdfc		%
 & \cellcolor{cellParamBased}\disneyc		%
 & \cellcolor{celPurelyNeural}Add Sep		%
 & \cellcolor{celPurelyNeural}Add Shared		%
 & \cellcolor{celPurelyNeural}Add Sep (enh.)		%
 & \cellcolor{celPurelyNeural}Add Shared (enh.)		%
& %
 & \gt
  \end{tabular}
\caption{
Renderings of the diffuse and the specular parts separately for all additive models. Note that for the models with the enhanced additive strategy (\emph{enh.}), the diffuse part is already weighted with $\xi$. Also shown are the combined rendering (\emph{added}) and the ground truth image (\emph{GT}). 
All models show a reasonable split into diffuse albedo and specular parts. For the vanilla methods of the purely neural category (Additive Separate and Additive Shared), we observe the tendency to represent more appearance in the specular part, which, as can be seen by the other models, does not seem necessary. Our enhancement for the additive split (\emph{enh.}) as introduced in 
\iftoggle{arxiv}{\cref{sec:enhancingAddSplit}}{Sec.~4.4}
and in particular the regularizers discussed in 
\iftoggle{arxiv}{\cref{sec:enhancingAddSplit}}{Sec.~4.4} and \cref{sec:supp:regularizers_enhanced} seem to improve on the disentanglement of diffuse and specular components.
Note that for the cow object, a colored (RGB) specular component seems necessary to reconstruct its appearance. The FMBRDF model \cite{ichikawa2023fresnel}, which is based on a scalar specular term, shows an unnatural glow. Note that we observed similar behavior for the scalar specular term proposed by NeRFactor \cite{Zhang2021NeRFactor}, as discussed in \cref{sec:supp:changes_nerfactor}.
}
\label{fig:supp_diff_spec_real}
\end{figure*}
\begin{figure*}[t]  %
  \centering  %
  \footnotesize
  \newcommand{\mywidthc}{0.02\textwidth}  %
  \newcommand{\mywidthx}{0.10\textwidth}  %
  \newcommand{\mywidthw}{0.008\textwidth}  %
  \newcommand{\myheightx}{0.15\textwidth}  %
  \newcommand{\mywidtht}{0.04\textwidth}  %
  \newcolumntype{C}{ >{\centering\arraybackslash} m{\mywidthc} } %
  \newcolumntype{X}{ >{\centering\arraybackslash} m{\mywidthx} } %
  \newcolumntype{W}{ >{\centering\arraybackslash} m{\mywidthw} } %
  \newcolumntype{T}{ >{\centering\arraybackslash} m{\mywidtht} } %

  \newcommand{\heightcolorbar}{0.10\textwidth}  %
  \newcommand{\xposOne}{-0.95}
  \newcommand{\yposOne}{0.35}
  \newcommand{\xposTwo}{0.35}
  \newcommand{\yposTwo}{0.8}

  \newcommand{\fontsizePSNR}{\ssmall}
  
  \setlength\tabcolsep{0pt} %

  \setlength{\extrarowheight}{1.25pt}
  
  \def\arraystretch{0.8} %
  \begin{tabular}{TTXXXXXXXXWX}

\multirow{2}{*}{\rotatebox{90}{\parbox{3cm}{\centering \merlc \\ blue acrylic}}} 
&
\rotatebox{90}{\centering \emph{diffuse}} 
&
\includegraphics[width=\mywidthx, height=\myheightx, keepaspectratio]{figures/renderings/appendix_diff_spec/dragon_merl_blue_acrylic_phong_view_02_1_diffuse.jpg}
&
\includegraphics[width=\mywidthx, height=\myheightx, keepaspectratio]{figures/renderings/appendix_diff_spec/dragon_merl_blue_acrylic_micro_view_02_1_diffuse.jpg}
&
\includegraphics[width=\mywidthx, height=\myheightx, keepaspectratio]{figures/renderings/appendix_diff_spec/dragon_merl_blue_acrylic_fmbrdf_view_02_1_diffuse.jpg}
&
\includegraphics[width=\mywidthx, height=\myheightx, keepaspectratio]{figures/renderings/appendix_diff_spec/dragon_merl_blue_acrylic_disney_view_02_1_diffuse.jpg}
&
\includegraphics[width=\mywidthx, height=\myheightx, keepaspectratio]{figures/renderings/appendix_diff_spec/dragon_merl_blue_acrylic_add_sep_view_02_1_diffuse.jpg}
&
\includegraphics[width=\mywidthx, height=\myheightx, keepaspectratio]{figures/renderings/appendix_diff_spec/dragon_merl_blue_acrylic_add_shared_view_02_1_diffuse.jpg}
&
\includegraphics[width=\mywidthx, height=\myheightx, keepaspectratio]{figures/renderings/appendix_diff_spec/dragon_merl_blue_acrylic_add_sep_reduce_view_02_1_diffuse.jpg}
&
\includegraphics[width=\mywidthx, height=\myheightx, keepaspectratio]{figures/renderings/appendix_diff_spec/dragon_merl_blue_acrylic_add_shared_reduce_view_02_1_diffuse.jpg}
&
&
\\
&
\rotatebox{90}{\centering \emph{specular}} 
&
\includegraphics[width=\mywidthx, height=\myheightx, keepaspectratio]{figures/renderings/appendix_diff_spec/dragon_merl_blue_acrylic_phong_view_02_1_specular.jpg}
&
\includegraphics[width=\mywidthx, height=\myheightx, keepaspectratio]{figures/renderings/appendix_diff_spec/dragon_merl_blue_acrylic_micro_view_02_1_specular.jpg}
&
\includegraphics[width=\mywidthx, height=\myheightx, keepaspectratio]{figures/renderings/appendix_diff_spec/dragon_merl_blue_acrylic_fmbrdf_view_02_1_specular.jpg}
&
\includegraphics[width=\mywidthx, height=\myheightx, keepaspectratio]{figures/renderings/appendix_diff_spec/dragon_merl_blue_acrylic_disney_view_02_1_specular.jpg}
&
\includegraphics[width=\mywidthx, height=\myheightx, keepaspectratio]{figures/renderings/appendix_diff_spec/dragon_merl_blue_acrylic_add_sep_view_02_1_specular.jpg}
&
\includegraphics[width=\mywidthx, height=\myheightx, keepaspectratio]{figures/renderings/appendix_diff_spec/dragon_merl_blue_acrylic_add_shared_view_02_1_specular.jpg}
&
\includegraphics[width=\mywidthx, height=\myheightx, keepaspectratio]{figures/renderings/appendix_diff_spec/dragon_merl_blue_acrylic_add_sep_reduce_view_02_1_specular.jpg}
&
\includegraphics[width=\mywidthx, height=\myheightx, keepaspectratio]{figures/renderings/appendix_diff_spec/dragon_merl_blue_acrylic_add_shared_reduce_view_02_1_specular.jpg}
&
&
\\
&
\rotatebox{90}{\centering \emph{added}} 
&
\includegraphics[width=\mywidthx, height=\myheightx, keepaspectratio]{figures/renderings/appendix_diff_spec/dragon_merl_blue_acrylic_phong_view_02_1_rendering.jpg}
&
\includegraphics[width=\mywidthx, height=\myheightx, keepaspectratio]{figures/renderings/appendix_diff_spec/dragon_merl_blue_acrylic_micro_view_02_1_rendering.jpg}
&
\includegraphics[width=\mywidthx, height=\myheightx, keepaspectratio]{figures/renderings/appendix_diff_spec/dragon_merl_blue_acrylic_fmbrdf_view_02_1_rendering.jpg}
&
\includegraphics[width=\mywidthx, height=\myheightx, keepaspectratio]{figures/renderings/appendix_diff_spec/dragon_merl_blue_acrylic_disney_view_02_1_rendering.jpg}
&
\includegraphics[width=\mywidthx, height=\myheightx, keepaspectratio]{figures/renderings/appendix_diff_spec/dragon_merl_blue_acrylic_add_sep_view_02_1_rendering.jpg}
&
\includegraphics[width=\mywidthx, height=\myheightx, keepaspectratio]{figures/renderings/appendix_diff_spec/dragon_merl_blue_acrylic_add_shared_view_02_1_rendering.jpg}
&
\includegraphics[width=\mywidthx, height=\myheightx, keepaspectratio]{figures/renderings/appendix_diff_spec/dragon_merl_blue_acrylic_add_sep_reduce_view_02_1_rendering.jpg}
&
\includegraphics[width=\mywidthx, height=\myheightx, keepaspectratio]{figures/renderings/appendix_diff_spec/dragon_merl_blue_acrylic_add_shared_reduce_view_02_1_rendering.jpg}
&
&
\includegraphics[width=\mywidthx, height=\myheightx, keepaspectratio]{figures/renderings/appendix_diff_spec/dragon_merl_blue_acrylic_view_02_1_gt.jpg}\\
\hline\hline\\
\multirow{2}{*}{\rotatebox{90}{\parbox{3cm}{\centering \merlc \\ alumina oxide}}} 
&
\rotatebox{90}{\centering \emph{diffuse}} 
&
\includegraphics[width=\mywidthx, height=\myheightx, keepaspectratio]{figures/renderings/appendix_diff_spec/armadillo_merl_alumina_oxide_phong_view_02_1_diffuse.jpg}
&
\includegraphics[width=\mywidthx, height=\myheightx, keepaspectratio]{figures/renderings/appendix_diff_spec/armadillo_merl_alumina_oxide_micro_view_02_1_diffuse.jpg}
&
\includegraphics[width=\mywidthx, height=\myheightx, keepaspectratio]{figures/renderings/appendix_diff_spec/armadillo_merl_alumina_oxide_fmbrdf_view_02_1_diffuse.jpg}
&
\includegraphics[width=\mywidthx, height=\myheightx, keepaspectratio]{figures/renderings/appendix_diff_spec/armadillo_merl_alumina_oxide_disney_view_02_1_diffuse.jpg}
&
\includegraphics[width=\mywidthx, height=\myheightx, keepaspectratio]{figures/renderings/appendix_diff_spec/armadillo_merl_alumina_oxide_add_sep_view_02_1_diffuse.jpg}
&
\includegraphics[width=\mywidthx, height=\myheightx, keepaspectratio]{figures/renderings/appendix_diff_spec/armadillo_merl_alumina_oxide_add_shared_view_02_1_diffuse.jpg}
&
\includegraphics[width=\mywidthx, height=\myheightx, keepaspectratio]{figures/renderings/appendix_diff_spec/armadillo_merl_alumina_oxide_add_sep_reduce_view_02_1_diffuse.jpg}
&
\includegraphics[width=\mywidthx, height=\myheightx, keepaspectratio]{figures/renderings/appendix_diff_spec/armadillo_merl_alumina_oxide_add_shared_reduce_view_02_1_diffuse.jpg}
&
&
\\
&
\rotatebox{90}{\centering \emph{specular}} 
&
\includegraphics[width=\mywidthx, height=\myheightx, keepaspectratio]{figures/renderings/appendix_diff_spec/armadillo_merl_alumina_oxide_phong_view_02_1_specular.jpg}
&
\includegraphics[width=\mywidthx, height=\myheightx, keepaspectratio]{figures/renderings/appendix_diff_spec/armadillo_merl_alumina_oxide_micro_view_02_1_specular.jpg}
&
\includegraphics[width=\mywidthx, height=\myheightx, keepaspectratio]{figures/renderings/appendix_diff_spec/armadillo_merl_alumina_oxide_fmbrdf_view_02_1_specular.jpg}
&
\includegraphics[width=\mywidthx, height=\myheightx, keepaspectratio]{figures/renderings/appendix_diff_spec/armadillo_merl_alumina_oxide_disney_view_02_1_specular.jpg}
&
\includegraphics[width=\mywidthx, height=\myheightx, keepaspectratio]{figures/renderings/appendix_diff_spec/armadillo_merl_alumina_oxide_add_sep_view_02_1_specular.jpg}
&
\includegraphics[width=\mywidthx, height=\myheightx, keepaspectratio]{figures/renderings/appendix_diff_spec/armadillo_merl_alumina_oxide_add_shared_view_02_1_specular.jpg}
&
\includegraphics[width=\mywidthx, height=\myheightx, keepaspectratio]{figures/renderings/appendix_diff_spec/armadillo_merl_alumina_oxide_add_sep_reduce_view_02_1_specular.jpg}
&
\includegraphics[width=\mywidthx, height=\myheightx, keepaspectratio]{figures/renderings/appendix_diff_spec/armadillo_merl_alumina_oxide_add_shared_reduce_view_02_1_specular.jpg}
&
&
\\
&
\rotatebox{90}{\centering \emph{added}} 
&
\includegraphics[width=\mywidthx, height=\myheightx, keepaspectratio]{figures/renderings/appendix_diff_spec/armadillo_merl_alumina_oxide_phong_view_02_1_rendering.jpg}
&
\includegraphics[width=\mywidthx, height=\myheightx, keepaspectratio]{figures/renderings/appendix_diff_spec/armadillo_merl_alumina_oxide_micro_view_02_1_rendering.jpg}
&
\includegraphics[width=\mywidthx, height=\myheightx, keepaspectratio]{figures/renderings/appendix_diff_spec/armadillo_merl_alumina_oxide_fmbrdf_view_02_1_rendering.jpg}
&
\includegraphics[width=\mywidthx, height=\myheightx, keepaspectratio]{figures/renderings/appendix_diff_spec/armadillo_merl_alumina_oxide_disney_view_02_1_rendering.jpg}
&
\includegraphics[width=\mywidthx, height=\myheightx, keepaspectratio]{figures/renderings/appendix_diff_spec/armadillo_merl_alumina_oxide_add_sep_view_02_1_rendering.jpg}
&
\includegraphics[width=\mywidthx, height=\myheightx, keepaspectratio]{figures/renderings/appendix_diff_spec/armadillo_merl_alumina_oxide_add_shared_view_02_1_rendering.jpg}
&
\includegraphics[width=\mywidthx, height=\myheightx, keepaspectratio]{figures/renderings/appendix_diff_spec/armadillo_merl_alumina_oxide_add_sep_reduce_view_02_1_rendering.jpg}
&
\includegraphics[width=\mywidthx, height=\myheightx, keepaspectratio]{figures/renderings/appendix_diff_spec/armadillo_merl_alumina_oxide_add_shared_reduce_view_02_1_rendering.jpg}
&
&
\includegraphics[width=\mywidthx, height=\myheightx, keepaspectratio]{figures/renderings/appendix_diff_spec/armadillo_merl_alumina_oxide_view_02_1_gt.jpg}\\
\hline\hline\\
\multirow{2}{*}{\rotatebox{90}{\parbox{3cm}{\centering \merlc \\ green acrylic}}} 
&
\rotatebox{90}{\centering \emph{diffuse}} 
&
\includegraphics[width=\mywidthx, height=\myheightx, keepaspectratio]{figures/renderings/appendix_diff_spec/horse_merl_green_acrylic_phong_view_02_1_diffuse.jpg}
&
\includegraphics[width=\mywidthx, height=\myheightx, keepaspectratio]{figures/renderings/appendix_diff_spec/horse_merl_green_acrylic_micro_view_02_1_diffuse.jpg}
&
\includegraphics[width=\mywidthx, height=\myheightx, keepaspectratio]{figures/renderings/appendix_diff_spec/horse_merl_green_acrylic_fmbrdf_view_02_1_diffuse.jpg}
&
\includegraphics[width=\mywidthx, height=\myheightx, keepaspectratio]{figures/renderings/appendix_diff_spec/horse_merl_green_acrylic_disney_view_02_1_diffuse.jpg}
&
\includegraphics[width=\mywidthx, height=\myheightx, keepaspectratio]{figures/renderings/appendix_diff_spec/horse_merl_green_acrylic_add_sep_view_02_1_diffuse.jpg}
&
\includegraphics[width=\mywidthx, height=\myheightx, keepaspectratio]{figures/renderings/appendix_diff_spec/horse_merl_green_acrylic_add_shared_view_02_1_diffuse.jpg}
&
\includegraphics[width=\mywidthx, height=\myheightx, keepaspectratio]{figures/renderings/appendix_diff_spec/horse_merl_green_acrylic_add_sep_reduce_view_02_1_diffuse.jpg}
&
\includegraphics[width=\mywidthx, height=\myheightx, keepaspectratio]{figures/renderings/appendix_diff_spec/horse_merl_green_acrylic_add_shared_reduce_view_02_1_diffuse.jpg}
&
&
\\
&
\rotatebox{90}{\centering \emph{specular}} 
&
\includegraphics[width=\mywidthx, height=\myheightx, keepaspectratio]{figures/renderings/appendix_diff_spec/horse_merl_green_acrylic_phong_view_02_1_specular.jpg}
&
\includegraphics[width=\mywidthx, height=\myheightx, keepaspectratio]{figures/renderings/appendix_diff_spec/horse_merl_green_acrylic_micro_view_02_1_specular.jpg}
&
\includegraphics[width=\mywidthx, height=\myheightx, keepaspectratio]{figures/renderings/appendix_diff_spec/horse_merl_green_acrylic_fmbrdf_view_02_1_specular.jpg}
&
\includegraphics[width=\mywidthx, height=\myheightx, keepaspectratio]{figures/renderings/appendix_diff_spec/horse_merl_green_acrylic_disney_view_02_1_specular.jpg}
&
\includegraphics[width=\mywidthx, height=\myheightx, keepaspectratio]{figures/renderings/appendix_diff_spec/horse_merl_green_acrylic_add_sep_view_02_1_specular.jpg}
&
\includegraphics[width=\mywidthx, height=\myheightx, keepaspectratio]{figures/renderings/appendix_diff_spec/horse_merl_green_acrylic_add_shared_view_02_1_specular.jpg}
&
\includegraphics[width=\mywidthx, height=\myheightx, keepaspectratio]{figures/renderings/appendix_diff_spec/horse_merl_green_acrylic_add_sep_reduce_view_02_1_specular.jpg}
&
\includegraphics[width=\mywidthx, height=\myheightx, keepaspectratio]{figures/renderings/appendix_diff_spec/horse_merl_green_acrylic_add_shared_reduce_view_02_1_specular.jpg}
&
&
\\
&
\rotatebox{90}{\centering \emph{added}} 
&
\includegraphics[width=\mywidthx, height=\myheightx, keepaspectratio]{figures/renderings/appendix_diff_spec/horse_merl_green_acrylic_phong_view_02_1_rendering.jpg}
&
\includegraphics[width=\mywidthx, height=\myheightx, keepaspectratio]{figures/renderings/appendix_diff_spec/horse_merl_green_acrylic_micro_view_02_1_rendering.jpg}
&
\includegraphics[width=\mywidthx, height=\myheightx, keepaspectratio]{figures/renderings/appendix_diff_spec/horse_merl_green_acrylic_fmbrdf_view_02_1_rendering.jpg}
&
\includegraphics[width=\mywidthx, height=\myheightx, keepaspectratio]{figures/renderings/appendix_diff_spec/horse_merl_green_acrylic_disney_view_02_1_rendering.jpg}
&
\includegraphics[width=\mywidthx, height=\myheightx, keepaspectratio]{figures/renderings/appendix_diff_spec/horse_merl_green_acrylic_add_sep_view_02_1_rendering.jpg}
&
\includegraphics[width=\mywidthx, height=\myheightx, keepaspectratio]{figures/renderings/appendix_diff_spec/horse_merl_green_acrylic_add_shared_view_02_1_rendering.jpg}
&
\includegraphics[width=\mywidthx, height=\myheightx, keepaspectratio]{figures/renderings/appendix_diff_spec/horse_merl_green_acrylic_add_sep_reduce_view_02_1_rendering.jpg}
&
\includegraphics[width=\mywidthx, height=\myheightx, keepaspectratio]{figures/renderings/appendix_diff_spec/horse_merl_green_acrylic_add_shared_reduce_view_02_1_rendering.jpg}
&
&
\includegraphics[width=\mywidthx, height=\myheightx, keepaspectratio]{figures/renderings/appendix_diff_spec/horse_merl_green_acrylic_view_02_1_gt.jpg}\\
\hline\hline\\[-0.2cm]
&
 & \cellcolor{cellParamBased}\rpc		%
 & \cellcolor{cellParamBased}\tsc		%
 & \cellcolor{cellParamBased}\fmbrdfc		%
 & \cellcolor{cellParamBased}\disneyc		%
 & \cellcolor{celPurelyNeural}Add Sep		%
 & \cellcolor{celPurelyNeural}Add Shared		%
 & \cellcolor{celPurelyNeural}Add Sep (enh.)		%
 & \cellcolor{celPurelyNeural}Add Shared (enh.)		%
& %
 & \gt
  \end{tabular}
\caption{
Renderings of the diffuse and the specular parts separately for all additive models. Note that for the models with the enhanced additive strategy (\emph{enh.}), the diffuse part is already weighted with $\xi$. Also shown are the combined rendering (\emph{added}) and the ground truth image (\emph{GT}). 
The figure shows non-metallic objects. Most models show a reasonable split into diffuse albedo and specular highlights. However, the figure reveals an issue that we observed occasionally for the vanilla purely neural models based on an additive split: For some materials, like the alumina oxide in this case, there seems to be an ambiguity that allows the model to perform an unreasonable ``color-split''. While the added result yields the correct colors for all of our experiments, and we do not observe a reduced reconstruction quality, this ambiguity might cause problems in specific cases. Note that our enhancement for the additive split (\emph{enh.}) as introduced in 
\iftoggle{arxiv}{\cref{sec:enhancingAddSplit}}{Sec.~4.4}
and in particular the regularizers discussed in 
\iftoggle{arxiv}{\cref{sec:enhancingAddSplit}}{Sec.~4.4} and \cref{sec:supp:regularizers_enhanced} eliminate this issue.
}
\label{fig:supp_diff_spec_synth_1}
\end{figure*}
\begin{figure*}[t]  %
  \centering  %
  \footnotesize
  \newcommand{\mywidthc}{0.02\textwidth}  %
  \newcommand{\mywidthx}{0.10\textwidth}  %
  \newcommand{\mywidthw}{0.008\textwidth}  %
  \newcommand{\myheightx}{0.15\textwidth}  %
  \newcommand{\mywidtht}{0.04\textwidth}  %
  \newcolumntype{C}{ >{\centering\arraybackslash} m{\mywidthc} } %
  \newcolumntype{X}{ >{\centering\arraybackslash} m{\mywidthx} } %
  \newcolumntype{W}{ >{\centering\arraybackslash} m{\mywidthw} } %
  \newcolumntype{T}{ >{\centering\arraybackslash} m{\mywidtht} } %

  \newcommand{\heightcolorbar}{0.10\textwidth}  %
  \newcommand{\xposOne}{-0.95}
  \newcommand{\yposOne}{0.35}
  \newcommand{\xposTwo}{0.35}
  \newcommand{\yposTwo}{0.8}

  \newcommand{\fontsizePSNR}{\ssmall}
  
  \setlength\tabcolsep{0pt} %

  \setlength{\extrarowheight}{1.25pt}
  
  \def\arraystretch{0.8} %
  \begin{tabular}{TTXXXXXXXXWX}

\multirow{2}{*}{\rotatebox{90}{\parbox{3cm}{\centering \merlc \\ hematite}}} 
&
\rotatebox{90}{\centering \emph{diffuse}} 
&
\includegraphics[width=\mywidthx, height=\myheightx, keepaspectratio]{figures/renderings/appendix_diff_spec/ogre_merl_hematite_phong_view_02_1_diffuse.jpg}
&
\includegraphics[width=\mywidthx, height=\myheightx, keepaspectratio]{figures/renderings/appendix_diff_spec/ogre_merl_hematite_micro_view_02_1_diffuse.jpg}
&
\includegraphics[width=\mywidthx, height=\myheightx, keepaspectratio]{figures/renderings/appendix_diff_spec/ogre_merl_hematite_fmbrdf_view_02_1_diffuse.jpg}
&
\includegraphics[width=\mywidthx, height=\myheightx, keepaspectratio]{figures/renderings/appendix_diff_spec/ogre_merl_hematite_disney_view_02_1_diffuse.jpg}
&
\includegraphics[width=\mywidthx, height=\myheightx, keepaspectratio]{figures/renderings/appendix_diff_spec/ogre_merl_hematite_add_sep_view_02_1_diffuse.jpg}
&
\includegraphics[width=\mywidthx, height=\myheightx, keepaspectratio]{figures/renderings/appendix_diff_spec/ogre_merl_hematite_add_shared_view_02_1_diffuse.jpg}
&
\includegraphics[width=\mywidthx, height=\myheightx, keepaspectratio]{figures/renderings/appendix_diff_spec/ogre_merl_hematite_add_sep_reduce_view_02_1_diffuse.jpg}
&
\includegraphics[width=\mywidthx, height=\myheightx, keepaspectratio]{figures/renderings/appendix_diff_spec/ogre_merl_hematite_add_shared_reduce_view_02_1_diffuse.jpg}
&
&
\\
&
\rotatebox{90}{\centering \emph{specular}} 
&
\includegraphics[width=\mywidthx, height=\myheightx, keepaspectratio]{figures/renderings/appendix_diff_spec/ogre_merl_hematite_phong_view_02_1_specular.jpg}
&
\includegraphics[width=\mywidthx, height=\myheightx, keepaspectratio]{figures/renderings/appendix_diff_spec/ogre_merl_hematite_micro_view_02_1_specular.jpg}
&
\includegraphics[width=\mywidthx, height=\myheightx, keepaspectratio]{figures/renderings/appendix_diff_spec/ogre_merl_hematite_fmbrdf_view_02_1_specular.jpg}
&
\includegraphics[width=\mywidthx, height=\myheightx, keepaspectratio]{figures/renderings/appendix_diff_spec/ogre_merl_hematite_disney_view_02_1_specular.jpg}
&
\includegraphics[width=\mywidthx, height=\myheightx, keepaspectratio]{figures/renderings/appendix_diff_spec/ogre_merl_hematite_add_sep_view_02_1_specular.jpg}
&
\includegraphics[width=\mywidthx, height=\myheightx, keepaspectratio]{figures/renderings/appendix_diff_spec/ogre_merl_hematite_add_shared_view_02_1_specular.jpg}
&
\includegraphics[width=\mywidthx, height=\myheightx, keepaspectratio]{figures/renderings/appendix_diff_spec/ogre_merl_hematite_add_sep_reduce_view_02_1_specular.jpg}
&
\includegraphics[width=\mywidthx, height=\myheightx, keepaspectratio]{figures/renderings/appendix_diff_spec/ogre_merl_hematite_add_shared_reduce_view_02_1_specular.jpg}
&
&
\\
&
\rotatebox{90}{\centering \emph{added}} 
&
\includegraphics[width=\mywidthx, height=\myheightx, keepaspectratio]{figures/renderings/appendix_diff_spec/ogre_merl_hematite_phong_view_02_1_rendering.jpg}
&
\includegraphics[width=\mywidthx, height=\myheightx, keepaspectratio]{figures/renderings/appendix_diff_spec/ogre_merl_hematite_micro_view_02_1_rendering.jpg}
&
\includegraphics[width=\mywidthx, height=\myheightx, keepaspectratio]{figures/renderings/appendix_diff_spec/ogre_merl_hematite_fmbrdf_view_02_1_rendering.jpg}
&
\includegraphics[width=\mywidthx, height=\myheightx, keepaspectratio]{figures/renderings/appendix_diff_spec/ogre_merl_hematite_disney_view_02_1_rendering.jpg}
&
\includegraphics[width=\mywidthx, height=\myheightx, keepaspectratio]{figures/renderings/appendix_diff_spec/ogre_merl_hematite_add_sep_view_02_1_rendering.jpg}
&
\includegraphics[width=\mywidthx, height=\myheightx, keepaspectratio]{figures/renderings/appendix_diff_spec/ogre_merl_hematite_add_shared_view_02_1_rendering.jpg}
&
\includegraphics[width=\mywidthx, height=\myheightx, keepaspectratio]{figures/renderings/appendix_diff_spec/ogre_merl_hematite_add_sep_reduce_view_02_1_rendering.jpg}
&
\includegraphics[width=\mywidthx, height=\myheightx, keepaspectratio]{figures/renderings/appendix_diff_spec/ogre_merl_hematite_add_shared_reduce_view_02_1_rendering.jpg}
&
&
\includegraphics[width=\mywidthx, height=\myheightx, keepaspectratio]{figures/renderings/appendix_diff_spec/ogre_merl_hematite_view_02_1_gt.jpg}\\
\hline\hline\\
\multirow{2}{*}{\rotatebox{90}{\parbox{3cm}{\centering \merlc \\ green metallic paint2}}} 
&
\rotatebox{90}{\centering \emph{diffuse}} 
&
\includegraphics[width=\mywidthx, height=\myheightx, keepaspectratio]{figures/renderings/appendix_diff_spec/spot_merl_green_metallic_paint2_phong_view_02_1_diffuse.jpg}
&
\includegraphics[width=\mywidthx, height=\myheightx, keepaspectratio]{figures/renderings/appendix_diff_spec/spot_merl_green_metallic_paint2_micro_view_02_1_diffuse.jpg}
&
\includegraphics[width=\mywidthx, height=\myheightx, keepaspectratio]{figures/renderings/appendix_diff_spec/spot_merl_green_metallic_paint2_fmbrdf_view_02_1_diffuse.jpg}
&
\includegraphics[width=\mywidthx, height=\myheightx, keepaspectratio]{figures/renderings/appendix_diff_spec/spot_merl_green_metallic_paint2_disney_view_02_1_diffuse.jpg}
&
\includegraphics[width=\mywidthx, height=\myheightx, keepaspectratio]{figures/renderings/appendix_diff_spec/spot_merl_green_metallic_paint2_add_sep_view_02_1_diffuse.jpg}
&
\includegraphics[width=\mywidthx, height=\myheightx, keepaspectratio]{figures/renderings/appendix_diff_spec/spot_merl_green_metallic_paint2_add_shared_view_02_1_diffuse.jpg}
&
\includegraphics[width=\mywidthx, height=\myheightx, keepaspectratio]{figures/renderings/appendix_diff_spec/spot_merl_green_metallic_paint2_add_sep_reduce_view_02_1_diffuse.jpg}
&
\includegraphics[width=\mywidthx, height=\myheightx, keepaspectratio]{figures/renderings/appendix_diff_spec/spot_merl_green_metallic_paint2_add_shared_reduce_view_02_1_diffuse.jpg}
&
&
\\
&
\rotatebox{90}{\centering \emph{specular}} 
&
\includegraphics[width=\mywidthx, height=\myheightx, keepaspectratio]{figures/renderings/appendix_diff_spec/spot_merl_green_metallic_paint2_phong_view_02_1_specular.jpg}
&
\includegraphics[width=\mywidthx, height=\myheightx, keepaspectratio]{figures/renderings/appendix_diff_spec/spot_merl_green_metallic_paint2_micro_view_02_1_specular.jpg}
&
\includegraphics[width=\mywidthx, height=\myheightx, keepaspectratio]{figures/renderings/appendix_diff_spec/spot_merl_green_metallic_paint2_fmbrdf_view_02_1_specular.jpg}
&
\includegraphics[width=\mywidthx, height=\myheightx, keepaspectratio]{figures/renderings/appendix_diff_spec/spot_merl_green_metallic_paint2_disney_view_02_1_specular.jpg}
&
\includegraphics[width=\mywidthx, height=\myheightx, keepaspectratio]{figures/renderings/appendix_diff_spec/spot_merl_green_metallic_paint2_add_sep_view_02_1_specular.jpg}
&
\includegraphics[width=\mywidthx, height=\myheightx, keepaspectratio]{figures/renderings/appendix_diff_spec/spot_merl_green_metallic_paint2_add_shared_view_02_1_specular.jpg}
&
\includegraphics[width=\mywidthx, height=\myheightx, keepaspectratio]{figures/renderings/appendix_diff_spec/spot_merl_green_metallic_paint2_add_sep_reduce_view_02_1_specular.jpg}
&
\includegraphics[width=\mywidthx, height=\myheightx, keepaspectratio]{figures/renderings/appendix_diff_spec/spot_merl_green_metallic_paint2_add_shared_reduce_view_02_1_specular.jpg}
&
&
\\
&
\rotatebox{90}{\centering \emph{added}} 
&
\includegraphics[width=\mywidthx, height=\myheightx, keepaspectratio]{figures/renderings/appendix_diff_spec/spot_merl_green_metallic_paint2_phong_view_02_1_rendering.jpg}
&
\includegraphics[width=\mywidthx, height=\myheightx, keepaspectratio]{figures/renderings/appendix_diff_spec/spot_merl_green_metallic_paint2_micro_view_02_1_rendering.jpg}
&
\includegraphics[width=\mywidthx, height=\myheightx, keepaspectratio]{figures/renderings/appendix_diff_spec/spot_merl_green_metallic_paint2_fmbrdf_view_02_1_rendering.jpg}
&
\includegraphics[width=\mywidthx, height=\myheightx, keepaspectratio]{figures/renderings/appendix_diff_spec/spot_merl_green_metallic_paint2_disney_view_02_1_rendering.jpg}
&
\includegraphics[width=\mywidthx, height=\myheightx, keepaspectratio]{figures/renderings/appendix_diff_spec/spot_merl_green_metallic_paint2_add_sep_view_02_1_rendering.jpg}
&
\includegraphics[width=\mywidthx, height=\myheightx, keepaspectratio]{figures/renderings/appendix_diff_spec/spot_merl_green_metallic_paint2_add_shared_view_02_1_rendering.jpg}
&
\includegraphics[width=\mywidthx, height=\myheightx, keepaspectratio]{figures/renderings/appendix_diff_spec/spot_merl_green_metallic_paint2_add_sep_reduce_view_02_1_rendering.jpg}
&
\includegraphics[width=\mywidthx, height=\myheightx, keepaspectratio]{figures/renderings/appendix_diff_spec/spot_merl_green_metallic_paint2_add_shared_reduce_view_02_1_rendering.jpg}
&
&
\includegraphics[width=\mywidthx, height=\myheightx, keepaspectratio]{figures/renderings/appendix_diff_spec/spot_merl_green_metallic_paint2_view_02_1_gt.jpg}\\
\hline\hline\\
\multirow{2}{*}{\rotatebox{90}{\parbox{3cm}{\centering \merlc \\ chrome steel}}} 
&
\rotatebox{90}{\centering \emph{diffuse}} 
&
\includegraphics[width=\mywidthx, height=\myheightx, keepaspectratio]{figures/renderings/appendix_diff_spec/spot_merl_chrome_steel_phong_view_02_1_diffuse.jpg}
&
\includegraphics[width=\mywidthx, height=\myheightx, keepaspectratio]{figures/renderings/appendix_diff_spec/spot_merl_chrome_steel_micro_view_02_1_diffuse.jpg}
&
\includegraphics[width=\mywidthx, height=\myheightx, keepaspectratio]{figures/renderings/appendix_diff_spec/spot_merl_chrome_steel_fmbrdf_view_02_1_diffuse.jpg}
&
\includegraphics[width=\mywidthx, height=\myheightx, keepaspectratio]{figures/renderings/appendix_diff_spec/spot_merl_chrome_steel_disney_view_02_1_diffuse.jpg}
&
\includegraphics[width=\mywidthx, height=\myheightx, keepaspectratio]{figures/renderings/appendix_diff_spec/spot_merl_chrome_steel_add_sep_view_02_1_diffuse.jpg}
&
\includegraphics[width=\mywidthx, height=\myheightx, keepaspectratio]{figures/renderings/appendix_diff_spec/spot_merl_chrome_steel_add_shared_view_02_1_diffuse.jpg}
&
\includegraphics[width=\mywidthx, height=\myheightx, keepaspectratio]{figures/renderings/appendix_diff_spec/spot_merl_chrome_steel_add_sep_reduce_view_02_1_diffuse.jpg}
&
\includegraphics[width=\mywidthx, height=\myheightx, keepaspectratio]{figures/renderings/appendix_diff_spec/spot_merl_chrome_steel_add_shared_reduce_view_02_1_diffuse.jpg}
&
&
\\
&
\rotatebox{90}{\centering \emph{specular}} 
&
\includegraphics[width=\mywidthx, height=\myheightx, keepaspectratio]{figures/renderings/appendix_diff_spec/spot_merl_chrome_steel_phong_view_02_1_specular.jpg}
&
\includegraphics[width=\mywidthx, height=\myheightx, keepaspectratio]{figures/renderings/appendix_diff_spec/spot_merl_chrome_steel_micro_view_02_1_specular.jpg}
&
\includegraphics[width=\mywidthx, height=\myheightx, keepaspectratio]{figures/renderings/appendix_diff_spec/spot_merl_chrome_steel_fmbrdf_view_02_1_specular.jpg}
&
\includegraphics[width=\mywidthx, height=\myheightx, keepaspectratio]{figures/renderings/appendix_diff_spec/spot_merl_chrome_steel_disney_view_02_1_specular.jpg}
&
\includegraphics[width=\mywidthx, height=\myheightx, keepaspectratio]{figures/renderings/appendix_diff_spec/spot_merl_chrome_steel_add_sep_view_02_1_specular.jpg}
&
\includegraphics[width=\mywidthx, height=\myheightx, keepaspectratio]{figures/renderings/appendix_diff_spec/spot_merl_chrome_steel_add_shared_view_02_1_specular.jpg}
&
\includegraphics[width=\mywidthx, height=\myheightx, keepaspectratio]{figures/renderings/appendix_diff_spec/spot_merl_chrome_steel_add_sep_reduce_view_02_1_specular.jpg}
&
\includegraphics[width=\mywidthx, height=\myheightx, keepaspectratio]{figures/renderings/appendix_diff_spec/spot_merl_chrome_steel_add_shared_reduce_view_02_1_specular.jpg}
&
&
\\
&
\rotatebox{90}{\centering \emph{added}} 
&
\includegraphics[width=\mywidthx, height=\myheightx, keepaspectratio]{figures/renderings/appendix_diff_spec/spot_merl_chrome_steel_phong_view_02_1_rendering.jpg}
&
\includegraphics[width=\mywidthx, height=\myheightx, keepaspectratio]{figures/renderings/appendix_diff_spec/spot_merl_chrome_steel_micro_view_02_1_rendering.jpg}
&
\includegraphics[width=\mywidthx, height=\myheightx, keepaspectratio]{figures/renderings/appendix_diff_spec/spot_merl_chrome_steel_fmbrdf_view_02_1_rendering.jpg}
&
\includegraphics[width=\mywidthx, height=\myheightx, keepaspectratio]{figures/renderings/appendix_diff_spec/spot_merl_chrome_steel_disney_view_02_1_rendering.jpg}
&
\includegraphics[width=\mywidthx, height=\myheightx, keepaspectratio]{figures/renderings/appendix_diff_spec/spot_merl_chrome_steel_add_sep_view_02_1_rendering.jpg}
&
\includegraphics[width=\mywidthx, height=\myheightx, keepaspectratio]{figures/renderings/appendix_diff_spec/spot_merl_chrome_steel_add_shared_view_02_1_rendering.jpg}
&
\includegraphics[width=\mywidthx, height=\myheightx, keepaspectratio]{figures/renderings/appendix_diff_spec/spot_merl_chrome_steel_add_sep_reduce_view_02_1_rendering.jpg}
&
\includegraphics[width=\mywidthx, height=\myheightx, keepaspectratio]{figures/renderings/appendix_diff_spec/spot_merl_chrome_steel_add_shared_reduce_view_02_1_rendering.jpg}
&
&
\includegraphics[width=\mywidthx, height=\myheightx, keepaspectratio]{figures/renderings/appendix_diff_spec/spot_merl_chrome_steel_view_02_1_gt.jpg}\\
\hline\hline\\[-0.2cm]
&
 & \cellcolor{cellParamBased}\rpc		%
 & \cellcolor{cellParamBased}\tsc		%
 & \cellcolor{cellParamBased}\fmbrdfc		%
 & \cellcolor{cellParamBased}\disneyc		%
 & \cellcolor{celPurelyNeural}Add Sep		%
 & \cellcolor{celPurelyNeural}Add Shared		%
 & \cellcolor{celPurelyNeural}Add Sep (enh.)		%
 & \cellcolor{celPurelyNeural}Add Shared (enh.)		%
& %
 & \gt
  \end{tabular}
\caption{
Renderings of the diffuse and the specular parts separately for all additive models. Note that for the models with the enhanced additive strategy (\emph{enh.}), the diffuse part is already weighted with $\xi$. Also shown are the combined rendering (\emph{added}) and the ground truth image (\emph{GT}). 
The figure shows metallic objects. This type of material shows almost no subsurface scattering, due to the free electrons \cite{akenine2019realTimeRendering}. All models are able to replicate this behavior, as can be seen clearly by the almost non-existent contribution of the diffuse part.
}
\label{fig:supp_diff_spec_synth_2}
\end{figure*}

In \cref{fig:supp_diff_spec_real,fig:supp_diff_spec_synth_1,fig:supp_diff_spec_synth_2}, we present a quantitative analysis of the diffuse and the specular component of all models that split the BRDF into those two contributions. Overall, the methods mostly show a reasonable split. For the purely neural methods, we notice a tendency to represent a larger fraction of the appearance by the specular part, leading to darker diffuse parts. Our enhancements introduced in \iftoggle{arxiv}{\cref{sec:enhancingAddSplit}}{Sec.~4.4}
and in particular the regularizers discussed in \iftoggle{arxiv}{\cref{sec:enhancingAddSplit}}{Sec.~4.4} and \cref{sec:supp:regularizers_enhanced} seem to improve on the disentanglement of diffuse and specular components. \cref{fig:supp_diff_spec_synth_2} reveals that all models can represent the behavior of metallic objects, where almost no subsurface scattering is present, and indeed predict almost no albedo component.


\subsection{Spatial Variation of the Reconstructed BRDFs}
\label{sec:supp:spat_var_recon_brdf}

\begin{figure*}[t]  %
  \centering  %
  \footnotesize
  \newcommand{\mywidthc}{0.02\textwidth}  %
  \newcommand{\mywidthx}{0.11\textwidth}  %
  \newcommand{\mywidthw}{0.03\textwidth}  %
  \newcommand{\myheightx}{0.14\textwidth}  %
  \newcommand{\mywidtht}{0.035\textwidth}  %
  \newcolumntype{C}{ >{\centering\arraybackslash} m{\mywidthc} } %
  \newcolumntype{X}{ >{\centering\arraybackslash} m{\mywidthx} } %
  \newcolumntype{W}{ >{\centering\arraybackslash} m{\mywidthw} } %
  \newcolumntype{T}{ >{\centering\arraybackslash} m{\mywidtht} } %

  \newcommand{\heightcolorbar}{0.10\textwidth}  %
  \newcommand{\xposOne}{-1.05}
  \newcommand{\yposOne}{0.15}
  \newcommand{\xposTwo}{0.0}
  \newcommand{\yposTwo}{0.6}
  \newcommand{\xposThree}{-0.4}
  \newcommand{\yposThree}{0.7}
  \newcommand{\xposFour}{-0.8}
  \newcommand{\yposFour}{1.05}

  \newcommand{\fontsizePSNR}{\ssmall}
  
  \setlength\tabcolsep{0pt} %

  \setlength{\extrarowheight}{1.25pt}
  
  \def\arraystretch{0.8} %
  \begin{tabular}{TXXXXXXXX}


\rotatebox{90}{\parbox{3cm}{\centering \merlc \\ blue acrylic}} 
 & 
 \includegraphics[width=\mywidthx, height=\myheightx, keepaspectratio]{figures/renderings/appendix_diff_flat/dragon_merl_blue_acrylic_phong_view_02_1_diffuse_flat.jpg}
& 
 \includegraphics[width=\mywidthx, height=\myheightx, keepaspectratio]{figures/renderings/appendix_diff_flat/dragon_merl_blue_acrylic_micro_view_02_1_diffuse_flat.jpg}
& 
 \includegraphics[width=\mywidthx, height=\myheightx, keepaspectratio]{figures/renderings/appendix_diff_flat/dragon_merl_blue_acrylic_fmbrdf_view_02_1_diffuse_flat.jpg}
& 
 \includegraphics[width=\mywidthx, height=\myheightx, keepaspectratio]{figures/renderings/appendix_diff_flat/dragon_merl_blue_acrylic_disney_view_02_1_diffuse_flat.jpg}
& 
 \includegraphics[width=\mywidthx, height=\myheightx, keepaspectratio]{figures/renderings/appendix_diff_flat/dragon_merl_blue_acrylic_add_sep_view_02_1_diffuse_flat.jpg}
& 
 \includegraphics[width=\mywidthx, height=\myheightx, keepaspectratio]{figures/renderings/appendix_diff_flat/dragon_merl_blue_acrylic_add_shared_view_02_1_diffuse_flat.jpg}
& 
 \includegraphics[width=\mywidthx, height=\myheightx, keepaspectratio]{figures/renderings/appendix_diff_flat/dragon_merl_blue_acrylic_add_sep_reduce_view_02_1_diffuse_flat.jpg}
& 
 \includegraphics[width=\mywidthx, height=\myheightx, keepaspectratio]{figures/renderings/appendix_diff_flat/dragon_merl_blue_acrylic_add_shared_reduce_view_02_1_diffuse_flat.jpg}
\\ \hline
\rotatebox{90}{\parbox{3cm}{\centering \merlc \\ ipswich pine 221}} 
 & 
 \includegraphics[width=\mywidthx, height=\myheightx, keepaspectratio]{figures/renderings/appendix_diff_flat/ogre_merl_ipswich_pine_221_phong_view_02_1_diffuse_flat.jpg}
& 
 \includegraphics[width=\mywidthx, height=\myheightx, keepaspectratio]{figures/renderings/appendix_diff_flat/ogre_merl_ipswich_pine_221_micro_view_02_1_diffuse_flat.jpg}
& 
 \includegraphics[width=\mywidthx, height=\myheightx, keepaspectratio]{figures/renderings/appendix_diff_flat/ogre_merl_ipswich_pine_221_fmbrdf_view_02_1_diffuse_flat.jpg}
& 
 \includegraphics[width=\mywidthx, height=\myheightx, keepaspectratio]{figures/renderings/appendix_diff_flat/ogre_merl_ipswich_pine_221_disney_view_02_1_diffuse_flat.jpg}
& 
 \includegraphics[width=\mywidthx, height=\myheightx, keepaspectratio]{figures/renderings/appendix_diff_flat/ogre_merl_ipswich_pine_221_add_sep_view_02_1_diffuse_flat.jpg}
& 
 \includegraphics[width=\mywidthx, height=\myheightx, keepaspectratio]{figures/renderings/appendix_diff_flat/ogre_merl_ipswich_pine_221_add_shared_view_02_1_diffuse_flat.jpg}
& 
 \includegraphics[width=\mywidthx, height=\myheightx, keepaspectratio]{figures/renderings/appendix_diff_flat/ogre_merl_ipswich_pine_221_add_sep_reduce_view_02_1_diffuse_flat.jpg}
& 
 \includegraphics[width=\mywidthx, height=\myheightx, keepaspectratio]{figures/renderings/appendix_diff_flat/ogre_merl_ipswich_pine_221_add_shared_reduce_view_02_1_diffuse_flat.jpg}
\\ \hline
\rotatebox{90}{\parbox{3cm}{\centering \merlc \\ white marble}} 
 & 
 \includegraphics[width=\mywidthx, height=\myheightx, keepaspectratio]{figures/renderings/appendix_diff_flat/happy_merl_white_marble_phong_view_02_1_diffuse_flat.jpg}
& 
 \includegraphics[width=\mywidthx, height=\myheightx, keepaspectratio]{figures/renderings/appendix_diff_flat/happy_merl_white_marble_micro_view_02_1_diffuse_flat.jpg}
& 
 \includegraphics[width=\mywidthx, height=\myheightx, keepaspectratio]{figures/renderings/appendix_diff_flat/happy_merl_white_marble_fmbrdf_view_02_1_diffuse_flat.jpg}
& 
 \includegraphics[width=\mywidthx, height=\myheightx, keepaspectratio]{figures/renderings/appendix_diff_flat/happy_merl_white_marble_disney_view_02_1_diffuse_flat.jpg}
& 
 \includegraphics[width=\mywidthx, height=\myheightx, keepaspectratio]{figures/renderings/appendix_diff_flat/happy_merl_white_marble_add_sep_view_02_1_diffuse_flat.jpg}
& 
 \includegraphics[width=\mywidthx, height=\myheightx, keepaspectratio]{figures/renderings/appendix_diff_flat/happy_merl_white_marble_add_shared_view_02_1_diffuse_flat.jpg}
& 
 \includegraphics[width=\mywidthx, height=\myheightx, keepaspectratio]{figures/renderings/appendix_diff_flat/happy_merl_white_marble_add_sep_reduce_view_02_1_diffuse_flat.jpg}
& 
 \includegraphics[width=\mywidthx, height=\myheightx, keepaspectratio]{figures/renderings/appendix_diff_flat/happy_merl_white_marble_add_shared_reduce_view_02_1_diffuse_flat.jpg}
\\ \hline
\rotatebox{90}{\parbox{3cm}{\centering \merlc \\ maroon plastic}} 
 & 
 \includegraphics[width=\mywidthx, height=\myheightx, keepaspectratio]{figures/renderings/appendix_diff_flat/lucy_merl_maroon_plastic_phong_view_02_1_diffuse_flat.jpg}
& 
 \includegraphics[width=\mywidthx, height=\myheightx, keepaspectratio]{figures/renderings/appendix_diff_flat/lucy_merl_maroon_plastic_micro_view_02_1_diffuse_flat.jpg}
& 
 \includegraphics[width=\mywidthx, height=\myheightx, keepaspectratio]{figures/renderings/appendix_diff_flat/lucy_merl_maroon_plastic_fmbrdf_view_02_1_diffuse_flat.jpg}
& 
 \includegraphics[width=\mywidthx, height=\myheightx, keepaspectratio]{figures/renderings/appendix_diff_flat/lucy_merl_maroon_plastic_disney_view_02_1_diffuse_flat.jpg}
& 
 \includegraphics[width=\mywidthx, height=\myheightx, keepaspectratio]{figures/renderings/appendix_diff_flat/lucy_merl_maroon_plastic_add_sep_view_02_1_diffuse_flat.jpg}
& 
 \includegraphics[width=\mywidthx, height=\myheightx, keepaspectratio]{figures/renderings/appendix_diff_flat/lucy_merl_maroon_plastic_add_shared_view_02_1_diffuse_flat.jpg}
& 
 \includegraphics[width=\mywidthx, height=\myheightx, keepaspectratio]{figures/renderings/appendix_diff_flat/lucy_merl_maroon_plastic_add_sep_reduce_view_02_1_diffuse_flat.jpg}
& 
 \includegraphics[width=\mywidthx, height=\myheightx, keepaspectratio]{figures/renderings/appendix_diff_flat/lucy_merl_maroon_plastic_add_shared_reduce_view_02_1_diffuse_flat.jpg}
\\ \hline
\rotatebox{90}{\parbox{3cm}{\centering \merlc \\ green latex}} 
 & 
 \includegraphics[width=\mywidthx, height=\myheightx, keepaspectratio]{figures/renderings/appendix_diff_flat/spot_merl_green_latex_phong_view_02_1_diffuse_flat.jpg}
& 
 \includegraphics[width=\mywidthx, height=\myheightx, keepaspectratio]{figures/renderings/appendix_diff_flat/spot_merl_green_latex_micro_view_02_1_diffuse_flat.jpg}
& 
 \includegraphics[width=\mywidthx, height=\myheightx, keepaspectratio]{figures/renderings/appendix_diff_flat/spot_merl_green_latex_fmbrdf_view_02_1_diffuse_flat.jpg}
& 
 \includegraphics[width=\mywidthx, height=\myheightx, keepaspectratio]{figures/renderings/appendix_diff_flat/spot_merl_green_latex_disney_view_02_1_diffuse_flat.jpg}
& 
 \includegraphics[width=\mywidthx, height=\myheightx, keepaspectratio]{figures/renderings/appendix_diff_flat/spot_merl_green_latex_add_sep_view_02_1_diffuse_flat.jpg}
& 
 \includegraphics[width=\mywidthx, height=\myheightx, keepaspectratio]{figures/renderings/appendix_diff_flat/spot_merl_green_latex_add_shared_view_02_1_diffuse_flat.jpg}
& 
 \includegraphics[width=\mywidthx, height=\myheightx, keepaspectratio]{figures/renderings/appendix_diff_flat/spot_merl_green_latex_add_sep_reduce_view_02_1_diffuse_flat.jpg}
& 
 \includegraphics[width=\mywidthx, height=\myheightx, keepaspectratio]{figures/renderings/appendix_diff_flat/spot_merl_green_latex_add_shared_reduce_view_02_1_diffuse_flat.jpg}
\\ \hline \\[-0.2cm]
 & \cellcolor{cellParamBased}\rpc		%
 & \cellcolor{cellParamBased}\tsc		%
 & \cellcolor{cellParamBased}\fmbrdfc		%
 & \cellcolor{cellParamBased}\disneyc		%
 & \cellcolor{celPurelyNeural}Add Sep		%
 & \cellcolor{celPurelyNeural}Add Shared		%
 & \cellcolor{celPurelyNeural}Add Sep (enh.)		%
 & \cellcolor{celPurelyNeural}Add Shared (enh.)		%
  \end{tabular}
  \caption{
  Analysis of the spatial variance of the reconstructed BRDFs. Shown are the albedos rendered without the cosine term for five spatially uniform objects of the MERL-based semi-synthetic dataset. We see that overall all models are able to capture the uniformity of the BRDF well and show minimal spatial variation. The results also reveal the ambiguity introduced by an additive splitting strategy, which allows the model to capture the appearance solely by the specular term, leading to almost zero albedo. While this 
  is particularly true for the purely neural models, we also occasionally observe it for other models (\eg green latex for Torrance-Sparrow or Disney for Ipswitch pine).
  }
\label{fig:supp:renderings_diffuse_flat}
\end{figure*}

As described in the main text, the MERL BRDFs are uniform over the respective meshes for the semi-synthetic dataset. To assess how well the models can capture this spatial uniformity, we render the albedo without the cosine term for all models, that employ an additive split. The results for five materials in \cref{fig:supp:renderings_diffuse_flat} reveal very little spatial variation of the albedos, which indicates that all models are able to capture the spatial uniformity quite well.



\subsection{Additional Comparison Results}
\label{sec:supp:additional_experiments_comp}

We report quantitative results for the individual objects of the DiLiGenT-MV dataset in \cref{tab:supp:diligent_quantitative} and qualitative results for both datasets in \cref{fig:supp:renderings_real,fig:supp:renderings_synth_1,fig:supp:renderings_synth_2,fig:supp:renderings_synth_3,fig:supp:renderings_synth_4,fig:supp:renderings_synth_5,fig:supp:renderings_synth_6}.

\paragraph{Real-World Data}
The quantitative evaluation on the individual objects in \cref{tab:supp:diligent_quantitative} confirms that for the DiLiGenT-MV dataset \cite{Li2020DiLiGentMVDataset}, the difference between the approaches based on parametric models and purely neural methods is quite small. We even observe that for individual objects like the bear, some parametric approaches perform slightly better than the purely neural approaches -- in particular, better than approaches with more layers for the directions (\eg Single MLP). This behavior can also be observed qualitatively in \cref{fig:supp:renderings_real}. The reason might be the noise in the real-world data, to which the purely neural methods seem to be slightly more sensitive. The fact that neural approaches with fewer layers for the directions (\eg Additive shared) yield better results supports this claim, in particular in light of the analysis of the number of layers for the directions presented in
\iftoggle{arxiv}{\cref{sec:analysis_brdf_models}}{Sec.~6.2}.

Moreover, \cref{fig:supp:renderings_real} reveals that all methods show errors in similar regions - recesses in particular. We hypothesize un-modelled interreflections as a potential reason. Due to the indicator function in the rendering equation for our scenario
(\iftoggle{arxiv}{\cref{eq:rendering_single_dir_light}}{Eq.~(9)} in the main paper),
shadows in the recesses will be completely black in our renderings. That is, however, a simplification because, in reality, some light reflected off the near surfaces will reach the shaded regions in the recesses. Therefore, larger errors for the image-based metrics in those parts of the mesh are expected.

Finally, \cref{tab:supp:diligent_quantitative} confirms that for the novel additive strategy, which we proposed in
\iftoggle{arxiv}{\cref{sec:enhancingAddSplit}}{Sec.~4.4},
we observe consistent improvements in the extended vanilla additive models.

\paragraph{Semi-Synthetic Data}
\cref{fig:supp:renderings_synth_1,fig:supp:renderings_synth_2,fig:supp:renderings_synth_3,fig:supp:renderings_synth_4,fig:supp:renderings_synth_5,fig:supp:renderings_synth_6} show a systematic advantage of purely neural methods for the challenging materials of the MERL dataset. The error maps reveal that in particular the specular peaks are much better represented, often showing a significant improvement over the parametric methods. While the difference is smaller for more diffuse materials, we still see an advantage of the purely neural methods.

\begin{table*}[t]  %
  \centering  %
  \footnotesize
  
  \setlength\tabcolsep{6pt} %

  \newcolumntype{C}{ >{\centering\arraybackslash} m{0.065\textwidth} } %


\begin{tabular}{l|l||CCCC|CCC|CC}
Datasets & Error metric & \cellcolor{cellParamBased}\rpc		%
 & \cellcolor{cellParamBased}\tsc		%
 & \cellcolor{cellParamBased}\fmbrdf \cite{ichikawa2023fresnel}		%
 & \cellcolor{cellParamBased}\disneyc		%
 & \cellcolor{celPurelyNeural}Single MLP		%
 & \cellcolor{celPurelyNeural}Add Sep		%
 & \cellcolor{celPurelyNeural}Add Shared		%
 & \cellcolor{celPurelyNeural}Add Sep (enh.)		%
 & \cellcolor{celPurelyNeural}Add Shared (enh.)		%
\\ \hline\hline
\multirow{4}{*}{Bear} & \psnrArrow & 43.99 & 44.24 & 44.24 & 44.35 & 44.09 & 43.96 & 44.66 & 44.34 & \textbf{44.74} \\
 & \dssimArrow & 0.476 & 0.464 & 0.464 & 0.454 & 0.489 & 0.477 & 0.445 & 0.463 & \textbf{0.442} \\
 & \lpipsArrow & 0.993 & 0.985 & \textbf{0.973} & 1.022 & 1.124 & 1.115 & 1.030 & 1.062 & 1.022 \\
 & \flipArrow & 2.776 & 2.729 & 2.759 & 2.708 & 2.661 & 2.695 & 2.581 & 2.613 & \textbf{2.550} \\ \hline
\multirow{4}{*}{Buddha} & \psnrArrow & 36.30 & 36.41 & 36.50 & 36.35 & 35.50 & 35.96 & \textbf{36.77} & 36.07 & 36.71 \\
 & \dssimArrow & 1.298 & 1.275 & 1.257 & 1.281 & 1.387 & 1.299 & 1.228 & 1.283 & \textbf{1.224} \\
 & \lpipsArrow & 2.278 & 2.272 & 2.240 & 2.376 & 2.489 & 2.257 & 2.211 & 2.229 & \textbf{2.169} \\
 & \flipArrow & 4.212 & 4.187 & 4.184 & 4.243 & 4.156 & 4.051 & 3.945 & 4.014 & \textbf{3.928} \\ \hline
\multirow{4}{*}{Cow} & \psnrArrow & 46.02 & 46.57 & 43.84 & 46.57 & 46.22 & 46.37 & 47.08 & 46.49 & \textbf{47.12} \\
 & \dssimArrow & 0.395 & 0.373 & 0.422 & 0.372 & 0.391 & 0.381 & 0.358 & 0.374 & \textbf{0.354} \\
 & \lpipsArrow & 1.682 & 1.686 & 1.885 & 1.701 & 1.635 & 1.593 & 1.595 & 1.570 & \textbf{1.568} \\
 & \flipArrow & 2.270 & 2.133 & 3.066 & 2.159 & 1.987 & 1.969 & 1.933 & 1.961 & \textbf{1.919} \\ \hline
\multirow{4}{*}{Pot2} & \psnrArrow & 46.13 & 46.35 & 46.47 & 46.49 & 46.64 & 46.63 & \textbf{46.92} & 46.66 & 46.92 \\
 & \dssimArrow & 0.524 & 0.502 & 0.502 & 0.490 & 0.492 & 0.492 & 0.474 & 0.486 & \textbf{0.472} \\
 & \lpipsArrow & 1.117 & 1.109 & 1.101 & 1.132 & 1.068 & 1.067 & 1.073 & \textbf{1.054} & 1.073 \\
 & \flipArrow & 2.841 & 2.774 & 2.748 & 2.759 & 2.606 & 2.622 & 2.576 & 2.608 & \textbf{2.569} \\ \hline
\multirow{4}{*}{Reading} & \psnrArrow & 35.51 & 35.58 & 36.05 & 35.72 & 35.77 & 35.94 & 36.31 & 35.84 & \textbf{36.38} \\
 & \dssimArrow & 1.184 & 1.173 & 1.114 & 1.155 & 1.185 & 1.139 & 1.066 & 1.116 & \textbf{1.052} \\
 & \lpipsArrow & 2.786 & 2.765 & 2.708 & 2.709 & 2.506 & 2.476 & 2.447 & \textbf{2.403} & 2.466 \\
 & \flipArrow & 3.497 & 3.446 & 3.383 & 3.411 & 3.397 & 3.281 & 3.243 & 3.218 & \textbf{3.172} \\ \hline

\end{tabular}
\caption{
    Quantitative comparison of the BRDF models on all individual objects of the real-world data \cite{Li2020DiLiGentMVDataset}. \psnr, DSSIM and \lpips are computed for the sRGB renderings. All quantities are first averaged over one object and then averaged over all objects in the respective dataset. DSSIM, LPIPS and \FLIP are scaled by 100. We see that for this dataset, the approaches based on parametric models (\mysquare[cellParamBased]) yield results that are comparable to the purely neural approaches (\mysquare[celPurelyNeural]). For individual objects, some parametric approaches even show slightly better results than the purely neural ones, in particular better than approaches with more layers for the directions (\eg Single MLP). Again, a potential reason might be noise in the real-world data to which the purely neural models seem to be a little more sensitive; especially with more layers for the directions. This is supported by the observation that among the purely neural approaches, we see again the tendency that fewer layers for the directions (\eg Additive Separate) is better than more (\eg Single MLP).
    Our enhancement for the additive split (\emph{enh.}) as introduced in 
    \iftoggle{arxiv}{\cref{sec:enhancingAddSplit}}{Sec.~4.4}
    shows consistent improvements of the respective vanilla additive model.
}
\label{tab:supp:diligent_quantitative}
\end{table*}
\begin{figure*}[t]  %
  \centering  %
  \footnotesize
  \newcommand{\mywidthc}{0.02\textwidth}  %
  \newcommand{\mywidthx}{0.11\textwidth}  %
  \newcommand{\mywidthw}{0.04\textwidth}  %
  \newcommand{\myheightx}{0.12\textwidth}  %
  \newcommand{\mywidtht}{0.035\textwidth}  %
  \newcolumntype{C}{ >{\centering\arraybackslash} m{\mywidthc} } %
  \newcolumntype{X}{ >{\centering\arraybackslash} m{\mywidthx} } %
  \newcolumntype{W}{ >{\centering\arraybackslash} m{\mywidthw} } %
  \newcolumntype{T}{ >{\centering\arraybackslash} m{\mywidtht} } %

  \newcommand{\heightcolorbar}{0.10\textwidth}  %
  \newcommand{\xposOne}{-1.05}
  \newcommand{\yposOne}{0.8}
  \newcommand{\xposTwo}{-0.8}
  \newcommand{\yposTwo}{0.8}
  \newcommand{\xposThree}{-1.15}
  \newcommand{\yposThree}{0.6}
  \newcommand{\xposFour}{-0.75}
  \newcommand{\yposFour}{1.0}

  \newcommand{\fontsizePSNR}{\ssmall}
  
  \setlength\tabcolsep{0pt} %

  \setlength{\extrarowheight}{1.25pt}
  
  \def\arraystretch{0.8} %
  \begin{tabular}{TXXXXXXXWX}

\multirow{2}{*}{\rotatebox{90}{\parbox{3cm}{\centering \diligentc \\ bear}}} &
\begin{tikzpicture}
\draw(0,0) node[inner sep=1] {\includegraphics[width=\mywidthx, height=\myheightx, keepaspectratio]{figures/renderings/appendix_qualitative/bearPNG_phong_view_02_1_rendering.jpg}};
\node[align=left] at (\xposOne, \yposOne) {\scriptsize PSNR: \\\scriptsize 41.31};
\end{tikzpicture} &
\begin{tikzpicture}
\draw(0,0) node[inner sep=1] {\includegraphics[width=\mywidthx, height=\myheightx, keepaspectratio]{figures/renderings/appendix_qualitative/bearPNG_micro_view_02_1_rendering.jpg}};
\node[align=left] at (\xposOne, \yposOne) {\scriptsize PSNR: \\\scriptsize 41.96};
\end{tikzpicture} &
\begin{tikzpicture}
\draw(0,0) node[inner sep=1] {\includegraphics[width=\mywidthx, height=\myheightx, keepaspectratio]{figures/renderings/appendix_qualitative/bearPNG_fmbrdf_view_02_1_rendering.jpg}};
\node[align=left] at (\xposOne, \yposOne) {\scriptsize PSNR: \\\scriptsize 41.91};
\end{tikzpicture} &
\begin{tikzpicture}
\draw(0,0) node[inner sep=1] {\includegraphics[width=\mywidthx, height=\myheightx, keepaspectratio]{figures/renderings/appendix_qualitative/bearPNG_disney_view_02_1_rendering.jpg}};
\node[align=left] at (\xposOne, \yposOne) {\scriptsize PSNR: \\\scriptsize 42.44};
\end{tikzpicture} &
\begin{tikzpicture}
\draw(0,0) node[inner sep=1] {\includegraphics[width=\mywidthx, height=\myheightx, keepaspectratio]{figures/renderings/appendix_qualitative/bearPNG_single_MLP_view_02_1_rendering.jpg}};
\node[align=left] at (\xposOne, \yposOne) {\scriptsize PSNR: \\\scriptsize 41.33};
\end{tikzpicture} &
\begin{tikzpicture}
\draw(0,0) node[inner sep=1] {\includegraphics[width=\mywidthx, height=\myheightx, keepaspectratio]{figures/renderings/appendix_qualitative/bearPNG_add_sep_view_02_1_rendering.jpg}};
\node[align=left] at (\xposOne, \yposOne) {\scriptsize PSNR: \\\scriptsize 41.72};
\end{tikzpicture} &
\begin{tikzpicture}
\draw(0,0) node[inner sep=1] {\includegraphics[width=\mywidthx, height=\myheightx, keepaspectratio]{figures/renderings/appendix_qualitative/bearPNG_add_shared_view_02_1_rendering.jpg}};
\node[align=left] at (\xposOne, \yposOne) {\scriptsize PSNR: \\\scriptsize 42.19};
\end{tikzpicture} &
& %
\includegraphics[width=\mywidthx, height=\myheightx, keepaspectratio]{figures/renderings/appendix_qualitative/bearPNG_view_02_1_gt.jpg}\\
 & \includegraphics[width=\mywidthx, height=\myheightx, keepaspectratio]{figures/renderings/appendix_qualitative/bearPNG_phong_view_02_1_flip_error.jpg} &
\includegraphics[width=\mywidthx, height=\myheightx, keepaspectratio]{figures/renderings/appendix_qualitative/bearPNG_micro_view_02_1_flip_error.jpg} &
\includegraphics[width=\mywidthx, height=\myheightx, keepaspectratio]{figures/renderings/appendix_qualitative/bearPNG_fmbrdf_view_02_1_flip_error.jpg} &
\includegraphics[width=\mywidthx, height=\myheightx, keepaspectratio]{figures/renderings/appendix_qualitative/bearPNG_disney_view_02_1_flip_error.jpg} &
\includegraphics[width=\mywidthx, height=\myheightx, keepaspectratio]{figures/renderings/appendix_qualitative/bearPNG_single_MLP_view_02_1_flip_error.jpg} &
\includegraphics[width=\mywidthx, height=\myheightx, keepaspectratio]{figures/renderings/appendix_qualitative/bearPNG_add_sep_view_02_1_flip_error.jpg} &
\includegraphics[width=\mywidthx, height=\myheightx, keepaspectratio]{figures/renderings/appendix_qualitative/bearPNG_add_shared_view_02_1_flip_error.jpg} &
& %
\includegraphics[width=\mywidthx, height=\heightcolorbar, keepaspectratio]{figures/renderings/appendix_qualitative/colorbar_magma.jpg} \\
\hline
\multirow{2}{*}{\rotatebox{90}{\parbox{3cm}{\centering \diligentc \\ buddha}}} &
\begin{tikzpicture}
\draw(0,0) node[inner sep=1] {\includegraphics[width=\mywidthx, height=\myheightx, keepaspectratio]{figures/renderings/appendix_qualitative/buddhaPNG_phong_view_02_1_rendering.jpg}};
\node[align=left] at (\xposTwo, \yposTwo) {\scriptsize PSNR: \\\scriptsize 32.29};
\end{tikzpicture} &
\begin{tikzpicture}
\draw(0,0) node[inner sep=1] {\includegraphics[width=\mywidthx, height=\myheightx, keepaspectratio]{figures/renderings/appendix_qualitative/buddhaPNG_micro_view_02_1_rendering.jpg}};
\node[align=left] at (\xposTwo, \yposTwo) {\scriptsize PSNR: \\\scriptsize 32.47};
\end{tikzpicture} &
\begin{tikzpicture}
\draw(0,0) node[inner sep=1] {\includegraphics[width=\mywidthx, height=\myheightx, keepaspectratio]{figures/renderings/appendix_qualitative/buddhaPNG_fmbrdf_view_02_1_rendering.jpg}};
\node[align=left] at (\xposTwo, \yposTwo) {\scriptsize PSNR: \\\scriptsize 32.58};
\end{tikzpicture} &
\begin{tikzpicture}
\draw(0,0) node[inner sep=1] {\includegraphics[width=\mywidthx, height=\myheightx, keepaspectratio]{figures/renderings/appendix_qualitative/buddhaPNG_disney_view_02_1_rendering.jpg}};
\node[align=left] at (\xposTwo, \yposTwo) {\scriptsize PSNR: \\\scriptsize 32.50};
\end{tikzpicture} &
\begin{tikzpicture}
\draw(0,0) node[inner sep=1] {\includegraphics[width=\mywidthx, height=\myheightx, keepaspectratio]{figures/renderings/appendix_qualitative/buddhaPNG_single_MLP_view_02_1_rendering.jpg}};
\node[align=left] at (\xposTwo, \yposTwo) {\scriptsize PSNR: \\\scriptsize 33.78};
\end{tikzpicture} &
\begin{tikzpicture}
\draw(0,0) node[inner sep=1] {\includegraphics[width=\mywidthx, height=\myheightx, keepaspectratio]{figures/renderings/appendix_qualitative/buddhaPNG_add_sep_view_02_1_rendering.jpg}};
\node[align=left] at (\xposTwo, \yposTwo) {\scriptsize PSNR: \\\scriptsize 33.78};
\end{tikzpicture} &
\begin{tikzpicture}
\draw(0,0) node[inner sep=1] {\includegraphics[width=\mywidthx, height=\myheightx, keepaspectratio]{figures/renderings/appendix_qualitative/buddhaPNG_add_shared_view_02_1_rendering.jpg}};
\node[align=left] at (\xposTwo, \yposTwo) {\scriptsize PSNR: \\\scriptsize 33.69};
\end{tikzpicture} &
& %
\includegraphics[width=\mywidthx, height=\myheightx, keepaspectratio]{figures/renderings/appendix_qualitative/buddhaPNG_view_02_1_gt.jpg}\\
 & \includegraphics[width=\mywidthx, height=\myheightx, keepaspectratio]{figures/renderings/appendix_qualitative/buddhaPNG_phong_view_02_1_flip_error.jpg} &
\includegraphics[width=\mywidthx, height=\myheightx, keepaspectratio]{figures/renderings/appendix_qualitative/buddhaPNG_micro_view_02_1_flip_error.jpg} &
\includegraphics[width=\mywidthx, height=\myheightx, keepaspectratio]{figures/renderings/appendix_qualitative/buddhaPNG_fmbrdf_view_02_1_flip_error.jpg} &
\includegraphics[width=\mywidthx, height=\myheightx, keepaspectratio]{figures/renderings/appendix_qualitative/buddhaPNG_disney_view_02_1_flip_error.jpg} &
\includegraphics[width=\mywidthx, height=\myheightx, keepaspectratio]{figures/renderings/appendix_qualitative/buddhaPNG_single_MLP_view_02_1_flip_error.jpg} &
\includegraphics[width=\mywidthx, height=\myheightx, keepaspectratio]{figures/renderings/appendix_qualitative/buddhaPNG_add_sep_view_02_1_flip_error.jpg} &
\includegraphics[width=\mywidthx, height=\myheightx, keepaspectratio]{figures/renderings/appendix_qualitative/buddhaPNG_add_shared_view_02_1_flip_error.jpg} &
& %
\includegraphics[width=\mywidthx, height=\heightcolorbar, keepaspectratio]{figures/renderings/appendix_qualitative/colorbar_magma.jpg} \\
\hline
\multirow{2}{*}{\rotatebox{90}{\parbox{3cm}{\centering \diligentc \\ pot2}}} &
\begin{tikzpicture}
\draw(0,0) node[inner sep=1] {\includegraphics[width=\mywidthx, height=\myheightx, keepaspectratio]{figures/renderings/appendix_qualitative/pot2PNG_phong_view_02_1_rendering.jpg}};
\node[align=left] at (\xposThree, \yposThree) {\scriptsize PSNR: \\\scriptsize 43.71};
\end{tikzpicture} &
\begin{tikzpicture}
\draw(0,0) node[inner sep=1] {\includegraphics[width=\mywidthx, height=\myheightx, keepaspectratio]{figures/renderings/appendix_qualitative/pot2PNG_micro_view_02_1_rendering.jpg}};
\node[align=left] at (\xposThree, \yposThree) {\scriptsize PSNR: \\\scriptsize 43.98};
\end{tikzpicture} &
\begin{tikzpicture}
\draw(0,0) node[inner sep=1] {\includegraphics[width=\mywidthx, height=\myheightx, keepaspectratio]{figures/renderings/appendix_qualitative/pot2PNG_fmbrdf_view_02_1_rendering.jpg}};
\node[align=left] at (\xposThree, \yposThree) {\scriptsize PSNR: \\\scriptsize 43.89};
\end{tikzpicture} &
\begin{tikzpicture}
\draw(0,0) node[inner sep=1] {\includegraphics[width=\mywidthx, height=\myheightx, keepaspectratio]{figures/renderings/appendix_qualitative/pot2PNG_disney_view_02_1_rendering.jpg}};
\node[align=left] at (\xposThree, \yposThree) {\scriptsize PSNR: \\\scriptsize 44.11};
\end{tikzpicture} &
\begin{tikzpicture}
\draw(0,0) node[inner sep=1] {\includegraphics[width=\mywidthx, height=\myheightx, keepaspectratio]{figures/renderings/appendix_qualitative/pot2PNG_single_MLP_view_02_1_rendering.jpg}};
\node[align=left] at (\xposThree, \yposThree) {\scriptsize PSNR: \\\scriptsize 43.91};
\end{tikzpicture} &
\begin{tikzpicture}
\draw(0,0) node[inner sep=1] {\includegraphics[width=\mywidthx, height=\myheightx, keepaspectratio]{figures/renderings/appendix_qualitative/pot2PNG_add_sep_view_02_1_rendering.jpg}};
\node[align=left] at (\xposThree, \yposThree) {\scriptsize PSNR: \\\scriptsize 43.89};
\end{tikzpicture} &
\begin{tikzpicture}
\draw(0,0) node[inner sep=1] {\includegraphics[width=\mywidthx, height=\myheightx, keepaspectratio]{figures/renderings/appendix_qualitative/pot2PNG_add_shared_view_02_1_rendering.jpg}};
\node[align=left] at (\xposThree, \yposThree) {\scriptsize PSNR: \\\scriptsize 44.36};
\end{tikzpicture} &
& %
\includegraphics[width=\mywidthx, height=\myheightx, keepaspectratio]{figures/renderings/appendix_qualitative/pot2PNG_view_02_1_gt.jpg}\\
 & \includegraphics[width=\mywidthx, height=\myheightx, keepaspectratio]{figures/renderings/appendix_qualitative/pot2PNG_phong_view_02_1_flip_error.jpg} &
\includegraphics[width=\mywidthx, height=\myheightx, keepaspectratio]{figures/renderings/appendix_qualitative/pot2PNG_micro_view_02_1_flip_error.jpg} &
\includegraphics[width=\mywidthx, height=\myheightx, keepaspectratio]{figures/renderings/appendix_qualitative/pot2PNG_fmbrdf_view_02_1_flip_error.jpg} &
\includegraphics[width=\mywidthx, height=\myheightx, keepaspectratio]{figures/renderings/appendix_qualitative/pot2PNG_disney_view_02_1_flip_error.jpg} &
\includegraphics[width=\mywidthx, height=\myheightx, keepaspectratio]{figures/renderings/appendix_qualitative/pot2PNG_single_MLP_view_02_1_flip_error.jpg} &
\includegraphics[width=\mywidthx, height=\myheightx, keepaspectratio]{figures/renderings/appendix_qualitative/pot2PNG_add_sep_view_02_1_flip_error.jpg} &
\includegraphics[width=\mywidthx, height=\myheightx, keepaspectratio]{figures/renderings/appendix_qualitative/pot2PNG_add_shared_view_02_1_flip_error.jpg} &
& %
\includegraphics[width=\mywidthx, height=\heightcolorbar, keepaspectratio]{figures/renderings/appendix_qualitative/colorbar_magma.jpg} \\
\hline
\multirow{2}{*}{\rotatebox{90}{\parbox{3cm}{\centering \diligentc \\ reading}}} &
\begin{tikzpicture}
\draw(0,0) node[inner sep=1] {\includegraphics[width=\mywidthx, height=\myheightx, keepaspectratio]{figures/renderings/appendix_qualitative/readingPNG_phong_view_02_1_rendering.jpg}};
\node[align=left] at (\xposFour, \yposFour) {\scriptsize PSNR: \\\scriptsize 33.35};
\end{tikzpicture} &
\begin{tikzpicture}
\draw(0,0) node[inner sep=1] {\includegraphics[width=\mywidthx, height=\myheightx, keepaspectratio]{figures/renderings/appendix_qualitative/readingPNG_micro_view_02_1_rendering.jpg}};
\node[align=left] at (\xposFour, \yposFour) {\scriptsize PSNR: \\\scriptsize 33.84};
\end{tikzpicture} &
\begin{tikzpicture}
\draw(0,0) node[inner sep=1] {\includegraphics[width=\mywidthx, height=\myheightx, keepaspectratio]{figures/renderings/appendix_qualitative/readingPNG_fmbrdf_view_02_1_rendering.jpg}};
\node[align=left] at (\xposFour, \yposFour) {\scriptsize PSNR: \\\scriptsize 33.71};
\end{tikzpicture} &
\begin{tikzpicture}
\draw(0,0) node[inner sep=1] {\includegraphics[width=\mywidthx, height=\myheightx, keepaspectratio]{figures/renderings/appendix_qualitative/readingPNG_disney_view_02_1_rendering.jpg}};
\node[align=left] at (\xposFour, \yposFour) {\scriptsize PSNR: \\\scriptsize 33.86};
\end{tikzpicture} &
\begin{tikzpicture}
\draw(0,0) node[inner sep=1] {\includegraphics[width=\mywidthx, height=\myheightx, keepaspectratio]{figures/renderings/appendix_qualitative/readingPNG_single_MLP_view_02_1_rendering.jpg}};
\node[align=left] at (\xposFour, \yposFour) {\scriptsize PSNR: \\\scriptsize 34.17};
\end{tikzpicture} &
\begin{tikzpicture}
\draw(0,0) node[inner sep=1] {\includegraphics[width=\mywidthx, height=\myheightx, keepaspectratio]{figures/renderings/appendix_qualitative/readingPNG_add_sep_view_02_1_rendering.jpg}};
\node[align=left] at (\xposFour, \yposFour) {\scriptsize PSNR: \\\scriptsize 34.27};
\end{tikzpicture} &
\begin{tikzpicture}
\draw(0,0) node[inner sep=1] {\includegraphics[width=\mywidthx, height=\myheightx, keepaspectratio]{figures/renderings/appendix_qualitative/readingPNG_add_shared_view_02_1_rendering.jpg}};
\node[align=left] at (\xposFour, \yposFour) {\scriptsize PSNR: \\\scriptsize 34.46};
\end{tikzpicture} &
& %
\includegraphics[width=\mywidthx, height=\myheightx, keepaspectratio]{figures/renderings/appendix_qualitative/readingPNG_view_02_1_gt.jpg}\\
 & \includegraphics[width=\mywidthx, height=\myheightx, keepaspectratio]{figures/renderings/appendix_qualitative/readingPNG_phong_view_02_1_flip_error.jpg} &
\includegraphics[width=\mywidthx, height=\myheightx, keepaspectratio]{figures/renderings/appendix_qualitative/readingPNG_micro_view_02_1_flip_error.jpg} &
\includegraphics[width=\mywidthx, height=\myheightx, keepaspectratio]{figures/renderings/appendix_qualitative/readingPNG_fmbrdf_view_02_1_flip_error.jpg} &
\includegraphics[width=\mywidthx, height=\myheightx, keepaspectratio]{figures/renderings/appendix_qualitative/readingPNG_disney_view_02_1_flip_error.jpg} &
\includegraphics[width=\mywidthx, height=\myheightx, keepaspectratio]{figures/renderings/appendix_qualitative/readingPNG_single_MLP_view_02_1_flip_error.jpg} &
\includegraphics[width=\mywidthx, height=\myheightx, keepaspectratio]{figures/renderings/appendix_qualitative/readingPNG_add_sep_view_02_1_flip_error.jpg} &
\includegraphics[width=\mywidthx, height=\myheightx, keepaspectratio]{figures/renderings/appendix_qualitative/readingPNG_add_shared_view_02_1_flip_error.jpg} &
& %
\includegraphics[width=\mywidthx, height=\heightcolorbar, keepaspectratio]{figures/renderings/appendix_qualitative/colorbar_magma.jpg} \\
\hline\\[-0.2cm]
 & \cellcolor{cellParamBased}\rpc		%
 & \cellcolor{cellParamBased}\tsc		%
 & \cellcolor{cellParamBased}\fmbrdfc		%
 & \cellcolor{cellParamBased}\disneyc		%
 & \cellcolor{celPurelyNeural}Single MLP		%
 & \cellcolor{celPurelyNeural}Add Sep		%
 & \cellcolor{celPurelyNeural}Add Shared		%
& %
 & \gt
  \end{tabular}
  \caption{
  Qualitative evaluation for the four remaining objects of the DiLiGenT-MV dataset \cite{Li2020DiLiGentMVDataset} not presented in the main paper. Shown are renderings in sRGB space with the corresponding PSNR values and the \FLIP error maps for the sRGB renderings. Both, purely neural approaches (\mysquare[celPurelyNeural]) and parametric models (\mysquare[cellParamBased]) show errors in similar regions - recesses in particular - which makes interreflections a likely cause.
  We observe a tendency of overfitting for the purely neural models for the bear object, which is also visible as artifacts in the renderings. Among the purely neural models, approaches with fewer layers for the directions (\eg Additive Separate) are less affected than architectures with more layers for the directions (\eg Single MLP). This matches the observation in
  \iftoggle{arxiv}{\cref{sec:analysis_brdf_models}}{Sec.~6.2}
  that models with fewer layers for the directions are more robust for the potentially more noisy real-world dataset.
  }
\label{fig:supp:renderings_real}
\end{figure*}
\begin{figure*}[t]  %
  \centering  %
  \footnotesize
  \newcommand{\mywidthc}{0.02\textwidth}  %
  \newcommand{\mywidthx}{0.11\textwidth}  %
  \newcommand{\mywidthw}{0.03\textwidth}  %
  \newcommand{\myheightx}{0.14\textwidth}  %
  \newcommand{\mywidtht}{0.035\textwidth}  %
  \newcolumntype{C}{ >{\centering\arraybackslash} m{\mywidthc} } %
  \newcolumntype{X}{ >{\centering\arraybackslash} m{\mywidthx} } %
  \newcolumntype{W}{ >{\centering\arraybackslash} m{\mywidthw} } %
  \newcolumntype{T}{ >{\centering\arraybackslash} m{\mywidtht} } %

  \newcommand{\heightcolorbar}{0.10\textwidth}  %
  \newcommand{\xposOne}{-1.05}
  \newcommand{\yposOne}{0.15}
  \newcommand{\xposTwo}{0.0}
  \newcommand{\yposTwo}{0.6}
  \newcommand{\xposThree}{-0.4}
  \newcommand{\yposThree}{0.7}
  \newcommand{\xposFour}{-0.8}
  \newcommand{\yposFour}{1.05}

  \newcommand{\fontsizePSNR}{\ssmall}
  
  \setlength\tabcolsep{0pt} %

  \setlength{\extrarowheight}{1.25pt}
  
  \def\arraystretch{0.8} %
  \begin{tabular}{TXXXXXXXWX}

\multirow{2}{*}{\rotatebox{90}{\parbox{3cm}{\centering \merlc \\ alumina oxide}}} &
\begin{tikzpicture}
\draw(0,0) node[inner sep=1] {\includegraphics[width=\mywidthx, height=\myheightx, keepaspectratio]{figures/renderings/appendix_qualitative/armadillo_merl_alumina_oxide_phong_view_02_1_rendering.jpg}};
\node[align=left] at (\xposOne, \yposOne) {\scriptsize PSNR: \\\scriptsize 44.05};
\end{tikzpicture} &
\begin{tikzpicture}
\draw(0,0) node[inner sep=1] {\includegraphics[width=\mywidthx, height=\myheightx, keepaspectratio]{figures/renderings/appendix_qualitative/armadillo_merl_alumina_oxide_micro_view_02_1_rendering.jpg}};
\node[align=left] at (\xposOne, \yposOne) {\scriptsize PSNR: \\\scriptsize 43.49};
\end{tikzpicture} &
\begin{tikzpicture}
\draw(0,0) node[inner sep=1] {\includegraphics[width=\mywidthx, height=\myheightx, keepaspectratio]{figures/renderings/appendix_qualitative/armadillo_merl_alumina_oxide_fmbrdf_view_02_1_rendering.jpg}};
\node[align=left] at (\xposOne, \yposOne) {\scriptsize PSNR: \\\scriptsize 49.73};
\end{tikzpicture} &
\begin{tikzpicture}
\draw(0,0) node[inner sep=1] {\includegraphics[width=\mywidthx, height=\myheightx, keepaspectratio]{figures/renderings/appendix_qualitative/armadillo_merl_alumina_oxide_disney_view_02_1_rendering.jpg}};
\node[align=left] at (\xposOne, \yposOne) {\scriptsize PSNR: \\\scriptsize 48.20};
\end{tikzpicture} &
\begin{tikzpicture}
\draw(0,0) node[inner sep=1] {\includegraphics[width=\mywidthx, height=\myheightx, keepaspectratio]{figures/renderings/appendix_qualitative/armadillo_merl_alumina_oxide_single_MLP_view_02_1_rendering.jpg}};
\node[align=left] at (\xposOne, \yposOne) {\scriptsize PSNR: \\\scriptsize 56.29};
\end{tikzpicture} &
\begin{tikzpicture}
\draw(0,0) node[inner sep=1] {\includegraphics[width=\mywidthx, height=\myheightx, keepaspectratio]{figures/renderings/appendix_qualitative/armadillo_merl_alumina_oxide_add_sep_view_02_1_rendering.jpg}};
\node[align=left] at (\xposOne, \yposOne) {\scriptsize PSNR: \\\scriptsize 56.28};
\end{tikzpicture} &
\begin{tikzpicture}
\draw(0,0) node[inner sep=1] {\includegraphics[width=\mywidthx, height=\myheightx, keepaspectratio]{figures/renderings/appendix_qualitative/armadillo_merl_alumina_oxide_add_shared_view_02_1_rendering.jpg}};
\node[align=left] at (\xposOne, \yposOne) {\scriptsize PSNR: \\\scriptsize 54.69};
\end{tikzpicture} &
& %
\includegraphics[width=\mywidthx, height=\myheightx, keepaspectratio]{figures/renderings/appendix_qualitative/armadillo_merl_alumina_oxide_view_02_1_gt.jpg}\\
 & \includegraphics[width=\mywidthx, height=\myheightx, keepaspectratio]{figures/renderings/appendix_qualitative/armadillo_merl_alumina_oxide_phong_view_02_1_flip_error.jpg} &
\includegraphics[width=\mywidthx, height=\myheightx, keepaspectratio]{figures/renderings/appendix_qualitative/armadillo_merl_alumina_oxide_micro_view_02_1_flip_error.jpg} &
\includegraphics[width=\mywidthx, height=\myheightx, keepaspectratio]{figures/renderings/appendix_qualitative/armadillo_merl_alumina_oxide_fmbrdf_view_02_1_flip_error.jpg} &
\includegraphics[width=\mywidthx, height=\myheightx, keepaspectratio]{figures/renderings/appendix_qualitative/armadillo_merl_alumina_oxide_disney_view_02_1_flip_error.jpg} &
\includegraphics[width=\mywidthx, height=\myheightx, keepaspectratio]{figures/renderings/appendix_qualitative/armadillo_merl_alumina_oxide_single_MLP_view_02_1_flip_error.jpg} &
\includegraphics[width=\mywidthx, height=\myheightx, keepaspectratio]{figures/renderings/appendix_qualitative/armadillo_merl_alumina_oxide_add_sep_view_02_1_flip_error.jpg} &
\includegraphics[width=\mywidthx, height=\myheightx, keepaspectratio]{figures/renderings/appendix_qualitative/armadillo_merl_alumina_oxide_add_shared_view_02_1_flip_error.jpg} &
& %
\includegraphics[width=\mywidthx, height=\heightcolorbar, keepaspectratio]{figures/renderings/appendix_qualitative/colorbar_magma.jpg} \\
\hline
\multirow{2}{*}{\rotatebox{90}{\parbox{3cm}{\centering \merlc \\ blue acrylic}}} &
\begin{tikzpicture}
\draw(0,0) node[inner sep=1] {\includegraphics[width=\mywidthx, height=\myheightx, keepaspectratio]{figures/renderings/appendix_qualitative/dragon_merl_blue_acrylic_phong_view_02_1_rendering.jpg}};
\node[align=left] at (\xposTwo, \yposTwo) {\scriptsize PSNR: \\\scriptsize 45.45};
\end{tikzpicture} &
\begin{tikzpicture}
\draw(0,0) node[inner sep=1] {\includegraphics[width=\mywidthx, height=\myheightx, keepaspectratio]{figures/renderings/appendix_qualitative/dragon_merl_blue_acrylic_micro_view_02_1_rendering.jpg}};
\node[align=left] at (\xposTwo, \yposTwo) {\scriptsize PSNR: \\\scriptsize 46.32};
\end{tikzpicture} &
\begin{tikzpicture}
\draw(0,0) node[inner sep=1] {\includegraphics[width=\mywidthx, height=\myheightx, keepaspectratio]{figures/renderings/appendix_qualitative/dragon_merl_blue_acrylic_fmbrdf_view_02_1_rendering.jpg}};
\node[align=left] at (\xposTwo, \yposTwo) {\scriptsize PSNR: \\\scriptsize 50.12};
\end{tikzpicture} &
\begin{tikzpicture}
\draw(0,0) node[inner sep=1] {\includegraphics[width=\mywidthx, height=\myheightx, keepaspectratio]{figures/renderings/appendix_qualitative/dragon_merl_blue_acrylic_disney_view_02_1_rendering.jpg}};
\node[align=left] at (\xposTwo, \yposTwo) {\scriptsize PSNR: \\\scriptsize 43.06};
\end{tikzpicture} &
\begin{tikzpicture}
\draw(0,0) node[inner sep=1] {\includegraphics[width=\mywidthx, height=\myheightx, keepaspectratio]{figures/renderings/appendix_qualitative/dragon_merl_blue_acrylic_single_MLP_view_02_1_rendering.jpg}};
\node[align=left] at (\xposTwo, \yposTwo) {\scriptsize PSNR: \\\scriptsize 53.26};
\end{tikzpicture} &
\begin{tikzpicture}
\draw(0,0) node[inner sep=1] {\includegraphics[width=\mywidthx, height=\myheightx, keepaspectratio]{figures/renderings/appendix_qualitative/dragon_merl_blue_acrylic_add_sep_view_02_1_rendering.jpg}};
\node[align=left] at (\xposTwo, \yposTwo) {\scriptsize PSNR: \\\scriptsize 53.19};
\end{tikzpicture} &
\begin{tikzpicture}
\draw(0,0) node[inner sep=1] {\includegraphics[width=\mywidthx, height=\myheightx, keepaspectratio]{figures/renderings/appendix_qualitative/dragon_merl_blue_acrylic_add_shared_view_02_1_rendering.jpg}};
\node[align=left] at (\xposTwo, \yposTwo) {\scriptsize PSNR: \\\scriptsize 52.69};
\end{tikzpicture} &
& %
\includegraphics[width=\mywidthx, height=\myheightx, keepaspectratio]{figures/renderings/appendix_qualitative/dragon_merl_blue_acrylic_view_02_1_gt.jpg}\\
 & \includegraphics[width=\mywidthx, height=\myheightx, keepaspectratio]{figures/renderings/appendix_qualitative/dragon_merl_blue_acrylic_phong_view_02_1_flip_error.jpg} &
\includegraphics[width=\mywidthx, height=\myheightx, keepaspectratio]{figures/renderings/appendix_qualitative/dragon_merl_blue_acrylic_micro_view_02_1_flip_error.jpg} &
\includegraphics[width=\mywidthx, height=\myheightx, keepaspectratio]{figures/renderings/appendix_qualitative/dragon_merl_blue_acrylic_fmbrdf_view_02_1_flip_error.jpg} &
\includegraphics[width=\mywidthx, height=\myheightx, keepaspectratio]{figures/renderings/appendix_qualitative/dragon_merl_blue_acrylic_disney_view_02_1_flip_error.jpg} &
\includegraphics[width=\mywidthx, height=\myheightx, keepaspectratio]{figures/renderings/appendix_qualitative/dragon_merl_blue_acrylic_single_MLP_view_02_1_flip_error.jpg} &
\includegraphics[width=\mywidthx, height=\myheightx, keepaspectratio]{figures/renderings/appendix_qualitative/dragon_merl_blue_acrylic_add_sep_view_02_1_flip_error.jpg} &
\includegraphics[width=\mywidthx, height=\myheightx, keepaspectratio]{figures/renderings/appendix_qualitative/dragon_merl_blue_acrylic_add_shared_view_02_1_flip_error.jpg} &
& %
\includegraphics[width=\mywidthx, height=\heightcolorbar, keepaspectratio]{figures/renderings/appendix_qualitative/colorbar_magma.jpg} \\
\hline
\multirow{2}{*}{\rotatebox{90}{\parbox{3cm}{\centering \merlc \\ chrome steel}}} &
\begin{tikzpicture}
\draw(0,0) node[inner sep=1] {\includegraphics[width=\mywidthx, height=\myheightx, keepaspectratio]{figures/renderings/appendix_qualitative/spot_merl_chrome_steel_phong_view_02_1_rendering.jpg}};
\node[align=left] at (\xposThree, \yposThree) {\scriptsize PSNR: \\\scriptsize 33.95};
\end{tikzpicture} &
\begin{tikzpicture}
\draw(0,0) node[inner sep=1] {\includegraphics[width=\mywidthx, height=\myheightx, keepaspectratio]{figures/renderings/appendix_qualitative/spot_merl_chrome_steel_micro_view_02_1_rendering.jpg}};
\node[align=left] at (\xposThree, \yposThree) {\scriptsize PSNR: \\\scriptsize 35.06};
\end{tikzpicture} &
\begin{tikzpicture}
\draw(0,0) node[inner sep=1] {\includegraphics[width=\mywidthx, height=\myheightx, keepaspectratio]{figures/renderings/appendix_qualitative/spot_merl_chrome_steel_fmbrdf_view_02_1_rendering.jpg}};
\node[align=left] at (\xposThree, \yposThree) {\scriptsize PSNR: \\\scriptsize 35.13};
\end{tikzpicture} &
\begin{tikzpicture}
\draw(0,0) node[inner sep=1] {\includegraphics[width=\mywidthx, height=\myheightx, keepaspectratio]{figures/renderings/appendix_qualitative/spot_merl_chrome_steel_disney_view_02_1_rendering.jpg}};
\node[align=left] at (\xposThree, \yposThree) {\scriptsize PSNR: \\\scriptsize 35.19};
\end{tikzpicture} &
\begin{tikzpicture}
\draw(0,0) node[inner sep=1] {\includegraphics[width=\mywidthx, height=\myheightx, keepaspectratio]{figures/renderings/appendix_qualitative/spot_merl_chrome_steel_single_MLP_view_02_1_rendering.jpg}};
\node[align=left] at (\xposThree, \yposThree) {\scriptsize PSNR: \\\scriptsize 46.85};
\end{tikzpicture} &
\begin{tikzpicture}
\draw(0,0) node[inner sep=1] {\includegraphics[width=\mywidthx, height=\myheightx, keepaspectratio]{figures/renderings/appendix_qualitative/spot_merl_chrome_steel_add_sep_view_02_1_rendering.jpg}};
\node[align=left] at (\xposThree, \yposThree) {\scriptsize PSNR: \\\scriptsize 46.75};
\end{tikzpicture} &
\begin{tikzpicture}
\draw(0,0) node[inner sep=1] {\includegraphics[width=\mywidthx, height=\myheightx, keepaspectratio]{figures/renderings/appendix_qualitative/spot_merl_chrome_steel_add_shared_view_02_1_rendering.jpg}};
\node[align=left] at (\xposThree, \yposThree) {\scriptsize PSNR: \\\scriptsize 46.54};
\end{tikzpicture} &
& %
\includegraphics[width=\mywidthx, height=\myheightx, keepaspectratio]{figures/renderings/appendix_qualitative/spot_merl_chrome_steel_view_02_1_gt.jpg}\\
 & \includegraphics[width=\mywidthx, height=\myheightx, keepaspectratio]{figures/renderings/appendix_qualitative/spot_merl_chrome_steel_phong_view_02_1_flip_error.jpg} &
\includegraphics[width=\mywidthx, height=\myheightx, keepaspectratio]{figures/renderings/appendix_qualitative/spot_merl_chrome_steel_micro_view_02_1_flip_error.jpg} &
\includegraphics[width=\mywidthx, height=\myheightx, keepaspectratio]{figures/renderings/appendix_qualitative/spot_merl_chrome_steel_fmbrdf_view_02_1_flip_error.jpg} &
\includegraphics[width=\mywidthx, height=\myheightx, keepaspectratio]{figures/renderings/appendix_qualitative/spot_merl_chrome_steel_disney_view_02_1_flip_error.jpg} &
\includegraphics[width=\mywidthx, height=\myheightx, keepaspectratio]{figures/renderings/appendix_qualitative/spot_merl_chrome_steel_single_MLP_view_02_1_flip_error.jpg} &
\includegraphics[width=\mywidthx, height=\myheightx, keepaspectratio]{figures/renderings/appendix_qualitative/spot_merl_chrome_steel_add_sep_view_02_1_flip_error.jpg} &
\includegraphics[width=\mywidthx, height=\myheightx, keepaspectratio]{figures/renderings/appendix_qualitative/spot_merl_chrome_steel_add_shared_view_02_1_flip_error.jpg} &
& %
\includegraphics[width=\mywidthx, height=\heightcolorbar, keepaspectratio]{figures/renderings/appendix_qualitative/colorbar_magma.jpg} \\
\hline
\multirow{2}{*}{\rotatebox{90}{\parbox{3cm}{\centering \merlc \\ color changing paint2}}} &
\begin{tikzpicture}
\draw(0,0) node[inner sep=1] {\includegraphics[width=\mywidthx, height=\myheightx, keepaspectratio]{figures/renderings/appendix_qualitative/nefertiti_merl_color_changing_paint2_phong_view_02_1_rendering.jpg}};
\node[align=left] at (\xposFour, \yposFour) {\scriptsize PSNR: \\\scriptsize 43.83};
\end{tikzpicture} &
\begin{tikzpicture}
\draw(0,0) node[inner sep=1] {\includegraphics[width=\mywidthx, height=\myheightx, keepaspectratio]{figures/renderings/appendix_qualitative/nefertiti_merl_color_changing_paint2_micro_view_02_1_rendering.jpg}};
\node[align=left] at (\xposFour, \yposFour) {\scriptsize PSNR: \\\scriptsize 47.34};
\end{tikzpicture} &
\begin{tikzpicture}
\draw(0,0) node[inner sep=1] {\includegraphics[width=\mywidthx, height=\myheightx, keepaspectratio]{figures/renderings/appendix_qualitative/nefertiti_merl_color_changing_paint2_fmbrdf_view_02_1_rendering.jpg}};
\node[align=left] at (\xposFour, \yposFour) {\scriptsize PSNR: \\\scriptsize 46.04};
\end{tikzpicture} &
\begin{tikzpicture}
\draw(0,0) node[inner sep=1] {\includegraphics[width=\mywidthx, height=\myheightx, keepaspectratio]{figures/renderings/appendix_qualitative/nefertiti_merl_color_changing_paint2_disney_view_02_1_rendering.jpg}};
\node[align=left] at (\xposFour, \yposFour) {\scriptsize PSNR: \\\scriptsize 46.81};
\end{tikzpicture} &
\begin{tikzpicture}
\draw(0,0) node[inner sep=1] {\includegraphics[width=\mywidthx, height=\myheightx, keepaspectratio]{figures/renderings/appendix_qualitative/nefertiti_merl_color_changing_paint2_single_MLP_view_02_1_rendering.jpg}};
\node[align=left] at (\xposFour, \yposFour) {\scriptsize PSNR: \\\scriptsize 47.73};
\end{tikzpicture} &
\begin{tikzpicture}
\draw(0,0) node[inner sep=1] {\includegraphics[width=\mywidthx, height=\myheightx, keepaspectratio]{figures/renderings/appendix_qualitative/nefertiti_merl_color_changing_paint2_add_sep_view_02_1_rendering.jpg}};
\node[align=left] at (\xposFour, \yposFour) {\scriptsize PSNR: \\\scriptsize 47.69};
\end{tikzpicture} &
\begin{tikzpicture}
\draw(0,0) node[inner sep=1] {\includegraphics[width=\mywidthx, height=\myheightx, keepaspectratio]{figures/renderings/appendix_qualitative/nefertiti_merl_color_changing_paint2_add_shared_view_02_1_rendering.jpg}};
\node[align=left] at (\xposFour, \yposFour) {\scriptsize PSNR: \\\scriptsize 47.59};
\end{tikzpicture} &
& %
\includegraphics[width=\mywidthx, height=\myheightx, keepaspectratio]{figures/renderings/appendix_qualitative/nefertiti_merl_color_changing_paint2_view_02_1_gt.jpg}\\
 & \includegraphics[width=\mywidthx, height=\myheightx, keepaspectratio]{figures/renderings/appendix_qualitative/nefertiti_merl_color_changing_paint2_phong_view_02_1_flip_error.jpg} &
\includegraphics[width=\mywidthx, height=\myheightx, keepaspectratio]{figures/renderings/appendix_qualitative/nefertiti_merl_color_changing_paint2_micro_view_02_1_flip_error.jpg} &
\includegraphics[width=\mywidthx, height=\myheightx, keepaspectratio]{figures/renderings/appendix_qualitative/nefertiti_merl_color_changing_paint2_fmbrdf_view_02_1_flip_error.jpg} &
\includegraphics[width=\mywidthx, height=\myheightx, keepaspectratio]{figures/renderings/appendix_qualitative/nefertiti_merl_color_changing_paint2_disney_view_02_1_flip_error.jpg} &
\includegraphics[width=\mywidthx, height=\myheightx, keepaspectratio]{figures/renderings/appendix_qualitative/nefertiti_merl_color_changing_paint2_single_MLP_view_02_1_flip_error.jpg} &
\includegraphics[width=\mywidthx, height=\myheightx, keepaspectratio]{figures/renderings/appendix_qualitative/nefertiti_merl_color_changing_paint2_add_sep_view_02_1_flip_error.jpg} &
\includegraphics[width=\mywidthx, height=\myheightx, keepaspectratio]{figures/renderings/appendix_qualitative/nefertiti_merl_color_changing_paint2_add_shared_view_02_1_flip_error.jpg} &
& %
\includegraphics[width=\mywidthx, height=\heightcolorbar, keepaspectratio]{figures/renderings/appendix_qualitative/colorbar_magma.jpg} \\
\hline\\[-0.2cm]
 & \cellcolor{cellParamBased}\rpc		%
 & \cellcolor{cellParamBased}\tsc		%
 & \cellcolor{cellParamBased}\fmbrdfc		%
 & \cellcolor{cellParamBased}\disneyc		%
 & \cellcolor{celPurelyNeural}Single MLP		%
 & \cellcolor{celPurelyNeural}Add Sep		%
 & \cellcolor{celPurelyNeural}Add Shared		%
& %
 & \gt
  \end{tabular}
  \caption{
  Qualitative evaluation of the reconstruction for four BRDFs from the MERL database \cite{matusik2003MERL} uniformly rendered on common test meshes from \cite{jacobson2020common}. Shown are renderings in sRGB space with the corresponding PSNR values and the \FLIP error maps for the sRGB renderings.
  Purely neural approaches~(\mysquare[celPurelyNeural]) show superior results over the parametric models (\mysquare[cellParamBased]).
  }
\label{fig:supp:renderings_synth_1}
\end{figure*}
\begin{figure*}[t]  %
  \centering  %
  \footnotesize
  \newcommand{\mywidthc}{0.02\textwidth}  %
  \newcommand{\mywidthx}{0.11\textwidth}  %
  \newcommand{\mywidthw}{0.022\textwidth}  %
  \newcommand{\myheightx}{0.13\textwidth}  %
  \newcommand{\mywidtht}{0.035\textwidth}  %
  \newcolumntype{C}{ >{\centering\arraybackslash} m{\mywidthc} } %
  \newcolumntype{X}{ >{\centering\arraybackslash} m{\mywidthx} } %
  \newcolumntype{W}{ >{\centering\arraybackslash} m{\mywidthw} } %
  \newcolumntype{T}{ >{\centering\arraybackslash} m{\mywidtht} } %

  \newcommand{\heightcolorbar}{0.10\textwidth}  %
  \newcommand{\xposOne}{-0.1}
  \newcommand{\yposOne}{0.7}
  \newcommand{\xposTwo}{-0.95}
  \newcommand{\yposTwo}{0.35}
  \newcommand{\xposThree}{-0.1}
  \newcommand{\yposThree}{0.7}
  \newcommand{\xposFour}{-0.3}
  \newcommand{\yposFour}{0.7}

  \newcommand{\fontsizePSNR}{\ssmall}
  
  \setlength\tabcolsep{0pt} %

  \setlength{\extrarowheight}{1.25pt}
  
  \def\arraystretch{0.8} %
  \begin{tabular}{TXXXXXXXWX}

\multirow{2}{*}{\rotatebox{90}{\parbox{3cm}{\centering \merlc \\ color changing paint3}}} &
\begin{tikzpicture}
\draw(0,0) node[inner sep=1] {\includegraphics[width=\mywidthx, height=\myheightx, keepaspectratio]{figures/renderings/appendix_qualitative/horse_merl_color_changing_paint3_phong_view_02_1_rendering.jpg}};
\node[align=left] at (\xposOne, \yposOne) {\scriptsize PSNR: \\\scriptsize 40.80};
\end{tikzpicture} &
\begin{tikzpicture}
\draw(0,0) node[inner sep=1] {\includegraphics[width=\mywidthx, height=\myheightx, keepaspectratio]{figures/renderings/appendix_qualitative/horse_merl_color_changing_paint3_micro_view_02_1_rendering.jpg}};
\node[align=left] at (\xposOne, \yposOne) {\scriptsize PSNR: \\\scriptsize 47.47};
\end{tikzpicture} &
\begin{tikzpicture}
\draw(0,0) node[inner sep=1] {\includegraphics[width=\mywidthx, height=\myheightx, keepaspectratio]{figures/renderings/appendix_qualitative/horse_merl_color_changing_paint3_fmbrdf_view_02_1_rendering.jpg}};
\node[align=left] at (\xposOne, \yposOne) {\scriptsize PSNR: \\\scriptsize 45.78};
\end{tikzpicture} &
\begin{tikzpicture}
\draw(0,0) node[inner sep=1] {\includegraphics[width=\mywidthx, height=\myheightx, keepaspectratio]{figures/renderings/appendix_qualitative/horse_merl_color_changing_paint3_disney_view_02_1_rendering.jpg}};
\node[align=left] at (\xposOne, \yposOne) {\scriptsize PSNR: \\\scriptsize 47.86};
\end{tikzpicture} &
\begin{tikzpicture}
\draw(0,0) node[inner sep=1] {\includegraphics[width=\mywidthx, height=\myheightx, keepaspectratio]{figures/renderings/appendix_qualitative/horse_merl_color_changing_paint3_single_MLP_view_02_1_rendering.jpg}};
\node[align=left] at (\xposOne, \yposOne) {\scriptsize PSNR: \\\scriptsize 48.98};
\end{tikzpicture} &
\begin{tikzpicture}
\draw(0,0) node[inner sep=1] {\includegraphics[width=\mywidthx, height=\myheightx, keepaspectratio]{figures/renderings/appendix_qualitative/horse_merl_color_changing_paint3_add_sep_view_02_1_rendering.jpg}};
\node[align=left] at (\xposOne, \yposOne) {\scriptsize PSNR: \\\scriptsize 49.00};
\end{tikzpicture} &
\begin{tikzpicture}
\draw(0,0) node[inner sep=1] {\includegraphics[width=\mywidthx, height=\myheightx, keepaspectratio]{figures/renderings/appendix_qualitative/horse_merl_color_changing_paint3_add_shared_view_02_1_rendering.jpg}};
\node[align=left] at (\xposOne, \yposOne) {\scriptsize PSNR: \\\scriptsize 48.78};
\end{tikzpicture} &
& %
\includegraphics[width=\mywidthx, height=\myheightx, keepaspectratio]{figures/renderings/appendix_qualitative/horse_merl_color_changing_paint3_view_02_1_gt.jpg}\\
 & \includegraphics[width=\mywidthx, height=\myheightx, keepaspectratio]{figures/renderings/appendix_qualitative/horse_merl_color_changing_paint3_phong_view_02_1_flip_error.jpg} &
\includegraphics[width=\mywidthx, height=\myheightx, keepaspectratio]{figures/renderings/appendix_qualitative/horse_merl_color_changing_paint3_micro_view_02_1_flip_error.jpg} &
\includegraphics[width=\mywidthx, height=\myheightx, keepaspectratio]{figures/renderings/appendix_qualitative/horse_merl_color_changing_paint3_fmbrdf_view_02_1_flip_error.jpg} &
\includegraphics[width=\mywidthx, height=\myheightx, keepaspectratio]{figures/renderings/appendix_qualitative/horse_merl_color_changing_paint3_disney_view_02_1_flip_error.jpg} &
\includegraphics[width=\mywidthx, height=\myheightx, keepaspectratio]{figures/renderings/appendix_qualitative/horse_merl_color_changing_paint3_single_MLP_view_02_1_flip_error.jpg} &
\includegraphics[width=\mywidthx, height=\myheightx, keepaspectratio]{figures/renderings/appendix_qualitative/horse_merl_color_changing_paint3_add_sep_view_02_1_flip_error.jpg} &
\includegraphics[width=\mywidthx, height=\myheightx, keepaspectratio]{figures/renderings/appendix_qualitative/horse_merl_color_changing_paint3_add_shared_view_02_1_flip_error.jpg} &
& %
\includegraphics[width=\mywidthx, height=\heightcolorbar, keepaspectratio]{figures/renderings/appendix_qualitative/colorbar_magma.jpg} \\
\hline
\multirow{2}{*}{\rotatebox{90}{\parbox{3cm}{\centering \merlc \\ gold paint}}} &
\begin{tikzpicture}
\draw(0,0) node[inner sep=1] {\includegraphics[width=\mywidthx, height=\myheightx, keepaspectratio]{figures/renderings/appendix_qualitative/stanfordBunny_merl_gold_paint_phong_view_02_1_rendering.jpg}};
\node[align=left] at (\xposTwo, \yposTwo) {\scriptsize PSNR: \\\scriptsize 44.01};
\end{tikzpicture} &
\begin{tikzpicture}
\draw(0,0) node[inner sep=1] {\includegraphics[width=\mywidthx, height=\myheightx, keepaspectratio]{figures/renderings/appendix_qualitative/stanfordBunny_merl_gold_paint_micro_view_02_1_rendering.jpg}};
\node[align=left] at (\xposTwo, \yposTwo) {\scriptsize PSNR: \\\scriptsize 48.24};
\end{tikzpicture} &
\begin{tikzpicture}
\draw(0,0) node[inner sep=1] {\includegraphics[width=\mywidthx, height=\myheightx, keepaspectratio]{figures/renderings/appendix_qualitative/stanfordBunny_merl_gold_paint_fmbrdf_view_02_1_rendering.jpg}};
\node[align=left] at (\xposTwo, \yposTwo) {\scriptsize PSNR: \\\scriptsize 39.35};
\end{tikzpicture} &
\begin{tikzpicture}
\draw(0,0) node[inner sep=1] {\includegraphics[width=\mywidthx, height=\myheightx, keepaspectratio]{figures/renderings/appendix_qualitative/stanfordBunny_merl_gold_paint_disney_view_02_1_rendering.jpg}};
\node[align=left] at (\xposTwo, \yposTwo) {\scriptsize PSNR: \\\scriptsize 47.86};
\end{tikzpicture} &
\begin{tikzpicture}
\draw(0,0) node[inner sep=1] {\includegraphics[width=\mywidthx, height=\myheightx, keepaspectratio]{figures/renderings/appendix_qualitative/stanfordBunny_merl_gold_paint_single_MLP_view_02_1_rendering.jpg}};
\node[align=left] at (\xposTwo, \yposTwo) {\scriptsize PSNR: \\\scriptsize 54.85};
\end{tikzpicture} &
\begin{tikzpicture}
\draw(0,0) node[inner sep=1] {\includegraphics[width=\mywidthx, height=\myheightx, keepaspectratio]{figures/renderings/appendix_qualitative/stanfordBunny_merl_gold_paint_add_sep_view_02_1_rendering.jpg}};
\node[align=left] at (\xposTwo, \yposTwo) {\scriptsize PSNR: \\\scriptsize 54.76};
\end{tikzpicture} &
\begin{tikzpicture}
\draw(0,0) node[inner sep=1] {\includegraphics[width=\mywidthx, height=\myheightx, keepaspectratio]{figures/renderings/appendix_qualitative/stanfordBunny_merl_gold_paint_add_shared_view_02_1_rendering.jpg}};
\node[align=left] at (\xposTwo, \yposTwo) {\scriptsize PSNR: \\\scriptsize 53.90};
\end{tikzpicture} &
& %
\includegraphics[width=\mywidthx, height=\myheightx, keepaspectratio]{figures/renderings/appendix_qualitative/stanfordBunny_merl_gold_paint_view_02_1_gt.jpg}\\
 & \includegraphics[width=\mywidthx, height=\myheightx, keepaspectratio]{figures/renderings/appendix_qualitative/stanfordBunny_merl_gold_paint_phong_view_02_1_flip_error.jpg} &
\includegraphics[width=\mywidthx, height=\myheightx, keepaspectratio]{figures/renderings/appendix_qualitative/stanfordBunny_merl_gold_paint_micro_view_02_1_flip_error.jpg} &
\includegraphics[width=\mywidthx, height=\myheightx, keepaspectratio]{figures/renderings/appendix_qualitative/stanfordBunny_merl_gold_paint_fmbrdf_view_02_1_flip_error.jpg} &
\includegraphics[width=\mywidthx, height=\myheightx, keepaspectratio]{figures/renderings/appendix_qualitative/stanfordBunny_merl_gold_paint_disney_view_02_1_flip_error.jpg} &
\includegraphics[width=\mywidthx, height=\myheightx, keepaspectratio]{figures/renderings/appendix_qualitative/stanfordBunny_merl_gold_paint_single_MLP_view_02_1_flip_error.jpg} &
\includegraphics[width=\mywidthx, height=\myheightx, keepaspectratio]{figures/renderings/appendix_qualitative/stanfordBunny_merl_gold_paint_add_sep_view_02_1_flip_error.jpg} &
\includegraphics[width=\mywidthx, height=\myheightx, keepaspectratio]{figures/renderings/appendix_qualitative/stanfordBunny_merl_gold_paint_add_shared_view_02_1_flip_error.jpg} &
& %
\includegraphics[width=\mywidthx, height=\heightcolorbar, keepaspectratio]{figures/renderings/appendix_qualitative/colorbar_magma.jpg} \\
\hline
\multirow{2}{*}{\rotatebox{90}{\parbox{3cm}{\centering \merlc \\ green acrylic}}} &
\begin{tikzpicture}
\draw(0,0) node[inner sep=1] {\includegraphics[width=\mywidthx, height=\myheightx, keepaspectratio]{figures/renderings/appendix_qualitative/horse_merl_green_acrylic_phong_view_02_1_rendering.jpg}};
\node[align=left] at (\xposThree, \yposThree) {\scriptsize PSNR: \\\scriptsize 43.61};
\end{tikzpicture} &
\begin{tikzpicture}
\draw(0,0) node[inner sep=1] {\includegraphics[width=\mywidthx, height=\myheightx, keepaspectratio]{figures/renderings/appendix_qualitative/horse_merl_green_acrylic_micro_view_02_1_rendering.jpg}};
\node[align=left] at (\xposThree, \yposThree) {\scriptsize PSNR: \\\scriptsize 45.01};
\end{tikzpicture} &
\begin{tikzpicture}
\draw(0,0) node[inner sep=1] {\includegraphics[width=\mywidthx, height=\myheightx, keepaspectratio]{figures/renderings/appendix_qualitative/horse_merl_green_acrylic_fmbrdf_view_02_1_rendering.jpg}};
\node[align=left] at (\xposThree, \yposThree) {\scriptsize PSNR: \\\scriptsize 45.97};
\end{tikzpicture} &
\begin{tikzpicture}
\draw(0,0) node[inner sep=1] {\includegraphics[width=\mywidthx, height=\myheightx, keepaspectratio]{figures/renderings/appendix_qualitative/horse_merl_green_acrylic_disney_view_02_1_rendering.jpg}};
\node[align=left] at (\xposThree, \yposThree) {\scriptsize PSNR: \\\scriptsize 45.91};
\end{tikzpicture} &
\begin{tikzpicture}
\draw(0,0) node[inner sep=1] {\includegraphics[width=\mywidthx, height=\myheightx, keepaspectratio]{figures/renderings/appendix_qualitative/horse_merl_green_acrylic_single_MLP_view_02_1_rendering.jpg}};
\node[align=left] at (\xposThree, \yposThree) {\scriptsize PSNR: \\\scriptsize 55.91};
\end{tikzpicture} &
\begin{tikzpicture}
\draw(0,0) node[inner sep=1] {\includegraphics[width=\mywidthx, height=\myheightx, keepaspectratio]{figures/renderings/appendix_qualitative/horse_merl_green_acrylic_add_sep_view_02_1_rendering.jpg}};
\node[align=left] at (\xposThree, \yposThree) {\scriptsize PSNR: \\\scriptsize 55.97};
\end{tikzpicture} &
\begin{tikzpicture}
\draw(0,0) node[inner sep=1] {\includegraphics[width=\mywidthx, height=\myheightx, keepaspectratio]{figures/renderings/appendix_qualitative/horse_merl_green_acrylic_add_shared_view_02_1_rendering.jpg}};
\node[align=left] at (\xposThree, \yposThree) {\scriptsize PSNR: \\\scriptsize 53.34};
\end{tikzpicture} &
& %
\includegraphics[width=\mywidthx, height=\myheightx, keepaspectratio]{figures/renderings/appendix_qualitative/horse_merl_green_acrylic_view_02_1_gt.jpg}\\
 & \includegraphics[width=\mywidthx, height=\myheightx, keepaspectratio]{figures/renderings/appendix_qualitative/horse_merl_green_acrylic_phong_view_02_1_flip_error.jpg} &
\includegraphics[width=\mywidthx, height=\myheightx, keepaspectratio]{figures/renderings/appendix_qualitative/horse_merl_green_acrylic_micro_view_02_1_flip_error.jpg} &
\includegraphics[width=\mywidthx, height=\myheightx, keepaspectratio]{figures/renderings/appendix_qualitative/horse_merl_green_acrylic_fmbrdf_view_02_1_flip_error.jpg} &
\includegraphics[width=\mywidthx, height=\myheightx, keepaspectratio]{figures/renderings/appendix_qualitative/horse_merl_green_acrylic_disney_view_02_1_flip_error.jpg} &
\includegraphics[width=\mywidthx, height=\myheightx, keepaspectratio]{figures/renderings/appendix_qualitative/horse_merl_green_acrylic_single_MLP_view_02_1_flip_error.jpg} &
\includegraphics[width=\mywidthx, height=\myheightx, keepaspectratio]{figures/renderings/appendix_qualitative/horse_merl_green_acrylic_add_sep_view_02_1_flip_error.jpg} &
\includegraphics[width=\mywidthx, height=\myheightx, keepaspectratio]{figures/renderings/appendix_qualitative/horse_merl_green_acrylic_add_shared_view_02_1_flip_error.jpg} &
& %
\includegraphics[width=\mywidthx, height=\heightcolorbar, keepaspectratio]{figures/renderings/appendix_qualitative/colorbar_magma.jpg} \\
\hline
\multirow{2}{*}{\rotatebox{90}{\parbox{3cm}{\centering \merlc \\ green latex}}} &
\begin{tikzpicture}
\draw(0,0) node[inner sep=1] {\includegraphics[width=\mywidthx, height=\myheightx, keepaspectratio]{figures/renderings/appendix_qualitative/spot_merl_green_latex_phong_view_02_1_rendering.jpg}};
\node[align=left] at (\xposFour, \yposFour) {\scriptsize PSNR: \\\scriptsize 40.23};
\end{tikzpicture} &
\begin{tikzpicture}
\draw(0,0) node[inner sep=1] {\includegraphics[width=\mywidthx, height=\myheightx, keepaspectratio]{figures/renderings/appendix_qualitative/spot_merl_green_latex_micro_view_02_1_rendering.jpg}};
\node[align=left] at (\xposFour, \yposFour) {\scriptsize PSNR: \\\scriptsize 42.66};
\end{tikzpicture} &
\begin{tikzpicture}
\draw(0,0) node[inner sep=1] {\includegraphics[width=\mywidthx, height=\myheightx, keepaspectratio]{figures/renderings/appendix_qualitative/spot_merl_green_latex_fmbrdf_view_02_1_rendering.jpg}};
\node[align=left] at (\xposFour, \yposFour) {\scriptsize PSNR: \\\scriptsize 41.34};
\end{tikzpicture} &
\begin{tikzpicture}
\draw(0,0) node[inner sep=1] {\includegraphics[width=\mywidthx, height=\myheightx, keepaspectratio]{figures/renderings/appendix_qualitative/spot_merl_green_latex_disney_view_02_1_rendering.jpg}};
\node[align=left] at (\xposFour, \yposFour) {\scriptsize PSNR: \\\scriptsize 43.39};
\end{tikzpicture} &
\begin{tikzpicture}
\draw(0,0) node[inner sep=1] {\includegraphics[width=\mywidthx, height=\myheightx, keepaspectratio]{figures/renderings/appendix_qualitative/spot_merl_green_latex_single_MLP_view_02_1_rendering.jpg}};
\node[align=left] at (\xposFour, \yposFour) {\scriptsize PSNR: \\\scriptsize 55.79};
\end{tikzpicture} &
\begin{tikzpicture}
\draw(0,0) node[inner sep=1] {\includegraphics[width=\mywidthx, height=\myheightx, keepaspectratio]{figures/renderings/appendix_qualitative/spot_merl_green_latex_add_sep_view_02_1_rendering.jpg}};
\node[align=left] at (\xposFour, \yposFour) {\scriptsize PSNR: \\\scriptsize 55.88};
\end{tikzpicture} &
\begin{tikzpicture}
\draw(0,0) node[inner sep=1] {\includegraphics[width=\mywidthx, height=\myheightx, keepaspectratio]{figures/renderings/appendix_qualitative/spot_merl_green_latex_add_shared_view_02_1_rendering.jpg}};
\node[align=left] at (\xposFour, \yposFour) {\scriptsize PSNR: \\\scriptsize 54.52};
\end{tikzpicture} &
& %
\includegraphics[width=\mywidthx, height=\myheightx, keepaspectratio]{figures/renderings/appendix_qualitative/spot_merl_green_latex_view_02_1_gt.jpg}\\
 & \includegraphics[width=\mywidthx, height=\myheightx, keepaspectratio]{figures/renderings/appendix_qualitative/spot_merl_green_latex_phong_view_02_1_flip_error.jpg} &
\includegraphics[width=\mywidthx, height=\myheightx, keepaspectratio]{figures/renderings/appendix_qualitative/spot_merl_green_latex_micro_view_02_1_flip_error.jpg} &
\includegraphics[width=\mywidthx, height=\myheightx, keepaspectratio]{figures/renderings/appendix_qualitative/spot_merl_green_latex_fmbrdf_view_02_1_flip_error.jpg} &
\includegraphics[width=\mywidthx, height=\myheightx, keepaspectratio]{figures/renderings/appendix_qualitative/spot_merl_green_latex_disney_view_02_1_flip_error.jpg} &
\includegraphics[width=\mywidthx, height=\myheightx, keepaspectratio]{figures/renderings/appendix_qualitative/spot_merl_green_latex_single_MLP_view_02_1_flip_error.jpg} &
\includegraphics[width=\mywidthx, height=\myheightx, keepaspectratio]{figures/renderings/appendix_qualitative/spot_merl_green_latex_add_sep_view_02_1_flip_error.jpg} &
\includegraphics[width=\mywidthx, height=\myheightx, keepaspectratio]{figures/renderings/appendix_qualitative/spot_merl_green_latex_add_shared_view_02_1_flip_error.jpg} &
& %
\includegraphics[width=\mywidthx, height=\heightcolorbar, keepaspectratio]{figures/renderings/appendix_qualitative/colorbar_magma.jpg} \\
\hline\\[-0.2cm]
 & \cellcolor{cellParamBased}\rpc		%
 & \cellcolor{cellParamBased}\tsc		%
 & \cellcolor{cellParamBased}\fmbrdfc		%
 & \cellcolor{cellParamBased}\disneyc		%
 & \cellcolor{celPurelyNeural}Single MLP		%
 & \cellcolor{celPurelyNeural}Add Sep		%
 & \cellcolor{celPurelyNeural}Add Shared		%
& %
 & \gt
  \end{tabular}
  \caption{
  Qualitative evaluation of the reconstruction for four BRDFs from the MERL database \cite{matusik2003MERL} uniformly rendered on common test meshes from \cite{jacobson2020common}. Shown are renderings in sRGB space with the corresponding PSNR values and the \FLIP error maps for the sRGB renderings.
  Purely neural approaches~(\mysquare[celPurelyNeural]) show superior results over the parametric models (\mysquare[cellParamBased]).
  }
\label{fig:supp:renderings_synth_2}
\end{figure*}
\begin{figure*}[t]  %
  \centering  %
  \footnotesize
  \newcommand{\mywidthc}{0.02\textwidth}  %
  \newcommand{\mywidthx}{0.11\textwidth}  %
  \newcommand{\mywidthw}{0.015\textwidth}  %
  \newcommand{\myheightx}{0.15\textwidth}  %
  \newcommand{\mywidtht}{0.035\textwidth}  %
  \newcolumntype{C}{ >{\centering\arraybackslash} m{\mywidthc} } %
  \newcolumntype{X}{ >{\centering\arraybackslash} m{\mywidthx} } %
  \newcolumntype{W}{ >{\centering\arraybackslash} m{\mywidthw} } %
  \newcolumntype{T}{ >{\centering\arraybackslash} m{\mywidtht} } %

  \newcommand{\heightcolorbar}{0.10\textwidth}  %
  \newcommand{\xposOne}{-0.3}
  \newcommand{\yposOne}{0.7}
  \newcommand{\xposTwo}{-0.6}
  \newcommand{\yposTwo}{1.1}
  \newcommand{\xposThree}{-0.6}
  \newcommand{\yposThree}{1.1}
  \newcommand{\xposFour}{-0.7}
  \newcommand{\yposFour}{1.1}

  \newcommand{\fontsizePSNR}{\ssmall}
  
  \setlength\tabcolsep{0pt} %

  \setlength{\extrarowheight}{1.25pt}
  
  \def\arraystretch{0.8} %
  \begin{tabular}{TXXXXXXXWX}

\multirow{2}{*}{\rotatebox{90}{\parbox{3cm}{\centering \merlc \\ green metallic paint2}}} &
\begin{tikzpicture}
\draw(0,0) node[inner sep=1] {\includegraphics[width=\mywidthx, height=\myheightx, keepaspectratio]{figures/renderings/appendix_qualitative/spot_merl_green_metallic_paint2_phong_view_02_1_rendering.jpg}};
\node[align=left] at (\xposOne, \yposOne) {\scriptsize PSNR: \\\scriptsize 37.85};
\end{tikzpicture} &
\begin{tikzpicture}
\draw(0,0) node[inner sep=1] {\includegraphics[width=\mywidthx, height=\myheightx, keepaspectratio]{figures/renderings/appendix_qualitative/spot_merl_green_metallic_paint2_micro_view_02_1_rendering.jpg}};
\node[align=left] at (\xposOne, \yposOne) {\scriptsize PSNR: \\\scriptsize 42.39};
\end{tikzpicture} &
\begin{tikzpicture}
\draw(0,0) node[inner sep=1] {\includegraphics[width=\mywidthx, height=\myheightx, keepaspectratio]{figures/renderings/appendix_qualitative/spot_merl_green_metallic_paint2_fmbrdf_view_02_1_rendering.jpg}};
\node[align=left] at (\xposOne, \yposOne) {\scriptsize PSNR: \\\scriptsize 40.33};
\end{tikzpicture} &
\begin{tikzpicture}
\draw(0,0) node[inner sep=1] {\includegraphics[width=\mywidthx, height=\myheightx, keepaspectratio]{figures/renderings/appendix_qualitative/spot_merl_green_metallic_paint2_disney_view_02_1_rendering.jpg}};
\node[align=left] at (\xposOne, \yposOne) {\scriptsize PSNR: \\\scriptsize 44.56};
\end{tikzpicture} &
\begin{tikzpicture}
\draw(0,0) node[inner sep=1] {\includegraphics[width=\mywidthx, height=\myheightx, keepaspectratio]{figures/renderings/appendix_qualitative/spot_merl_green_metallic_paint2_single_MLP_view_02_1_rendering.jpg}};
\node[align=left] at (\xposOne, \yposOne) {\scriptsize PSNR: \\\scriptsize 46.61};
\end{tikzpicture} &
\begin{tikzpicture}
\draw(0,0) node[inner sep=1] {\includegraphics[width=\mywidthx, height=\myheightx, keepaspectratio]{figures/renderings/appendix_qualitative/spot_merl_green_metallic_paint2_add_sep_view_02_1_rendering.jpg}};
\node[align=left] at (\xposOne, \yposOne) {\scriptsize PSNR: \\\scriptsize 46.47};
\end{tikzpicture} &
\begin{tikzpicture}
\draw(0,0) node[inner sep=1] {\includegraphics[width=\mywidthx, height=\myheightx, keepaspectratio]{figures/renderings/appendix_qualitative/spot_merl_green_metallic_paint2_add_shared_view_02_1_rendering.jpg}};
\node[align=left] at (\xposOne, \yposOne) {\scriptsize PSNR: \\\scriptsize 46.44};
\end{tikzpicture} &
& %
\includegraphics[width=\mywidthx, height=\myheightx, keepaspectratio]{figures/renderings/appendix_qualitative/spot_merl_green_metallic_paint2_view_02_1_gt.jpg}\\
 & \includegraphics[width=\mywidthx, height=\myheightx, keepaspectratio]{figures/renderings/appendix_qualitative/spot_merl_green_metallic_paint2_phong_view_02_1_flip_error.jpg} &
\includegraphics[width=\mywidthx, height=\myheightx, keepaspectratio]{figures/renderings/appendix_qualitative/spot_merl_green_metallic_paint2_micro_view_02_1_flip_error.jpg} &
\includegraphics[width=\mywidthx, height=\myheightx, keepaspectratio]{figures/renderings/appendix_qualitative/spot_merl_green_metallic_paint2_fmbrdf_view_02_1_flip_error.jpg} &
\includegraphics[width=\mywidthx, height=\myheightx, keepaspectratio]{figures/renderings/appendix_qualitative/spot_merl_green_metallic_paint2_disney_view_02_1_flip_error.jpg} &
\includegraphics[width=\mywidthx, height=\myheightx, keepaspectratio]{figures/renderings/appendix_qualitative/spot_merl_green_metallic_paint2_single_MLP_view_02_1_flip_error.jpg} &
\includegraphics[width=\mywidthx, height=\myheightx, keepaspectratio]{figures/renderings/appendix_qualitative/spot_merl_green_metallic_paint2_add_sep_view_02_1_flip_error.jpg} &
\includegraphics[width=\mywidthx, height=\myheightx, keepaspectratio]{figures/renderings/appendix_qualitative/spot_merl_green_metallic_paint2_add_shared_view_02_1_flip_error.jpg} &
& %
\includegraphics[width=\mywidthx, height=\heightcolorbar, keepaspectratio]{figures/renderings/appendix_qualitative/colorbar_magma.jpg} \\
\hline
\multirow{2}{*}{\rotatebox{90}{\parbox{3cm}{\centering \merlc \\ hematite}}} &
\begin{tikzpicture}
\draw(0,0) node[inner sep=1] {\includegraphics[width=\mywidthx, height=\myheightx, keepaspectratio]{figures/renderings/appendix_qualitative/ogre_merl_hematite_phong_view_02_1_rendering.jpg}};
\node[align=left] at (\xposTwo, \yposTwo) {\scriptsize PSNR: \\\scriptsize 39.03};
\end{tikzpicture} &
\begin{tikzpicture}
\draw(0,0) node[inner sep=1] {\includegraphics[width=\mywidthx, height=\myheightx, keepaspectratio]{figures/renderings/appendix_qualitative/ogre_merl_hematite_micro_view_02_1_rendering.jpg}};
\node[align=left] at (\xposTwo, \yposTwo) {\scriptsize PSNR: \\\scriptsize 42.87};
\end{tikzpicture} &
\begin{tikzpicture}
\draw(0,0) node[inner sep=1] {\includegraphics[width=\mywidthx, height=\myheightx, keepaspectratio]{figures/renderings/appendix_qualitative/ogre_merl_hematite_fmbrdf_view_02_1_rendering.jpg}};
\node[align=left] at (\xposTwo, \yposTwo) {\scriptsize PSNR: \\\scriptsize 46.50};
\end{tikzpicture} &
\begin{tikzpicture}
\draw(0,0) node[inner sep=1] {\includegraphics[width=\mywidthx, height=\myheightx, keepaspectratio]{figures/renderings/appendix_qualitative/ogre_merl_hematite_disney_view_02_1_rendering.jpg}};
\node[align=left] at (\xposTwo, \yposTwo) {\scriptsize PSNR: \\\scriptsize 46.07};
\end{tikzpicture} &
\begin{tikzpicture}
\draw(0,0) node[inner sep=1] {\includegraphics[width=\mywidthx, height=\myheightx, keepaspectratio]{figures/renderings/appendix_qualitative/ogre_merl_hematite_single_MLP_view_02_1_rendering.jpg}};
\node[align=left] at (\xposTwo, \yposTwo) {\scriptsize PSNR: \\\scriptsize 49.65};
\end{tikzpicture} &
\begin{tikzpicture}
\draw(0,0) node[inner sep=1] {\includegraphics[width=\mywidthx, height=\myheightx, keepaspectratio]{figures/renderings/appendix_qualitative/ogre_merl_hematite_add_sep_view_02_1_rendering.jpg}};
\node[align=left] at (\xposTwo, \yposTwo) {\scriptsize PSNR: \\\scriptsize 49.55};
\end{tikzpicture} &
\begin{tikzpicture}
\draw(0,0) node[inner sep=1] {\includegraphics[width=\mywidthx, height=\myheightx, keepaspectratio]{figures/renderings/appendix_qualitative/ogre_merl_hematite_add_shared_view_02_1_rendering.jpg}};
\node[align=left] at (\xposTwo, \yposTwo) {\scriptsize PSNR: \\\scriptsize 48.48};
\end{tikzpicture} &
& %
\includegraphics[width=\mywidthx, height=\myheightx, keepaspectratio]{figures/renderings/appendix_qualitative/ogre_merl_hematite_view_02_1_gt.jpg}\\
 & \includegraphics[width=\mywidthx, height=\myheightx, keepaspectratio]{figures/renderings/appendix_qualitative/ogre_merl_hematite_phong_view_02_1_flip_error.jpg} &
\includegraphics[width=\mywidthx, height=\myheightx, keepaspectratio]{figures/renderings/appendix_qualitative/ogre_merl_hematite_micro_view_02_1_flip_error.jpg} &
\includegraphics[width=\mywidthx, height=\myheightx, keepaspectratio]{figures/renderings/appendix_qualitative/ogre_merl_hematite_fmbrdf_view_02_1_flip_error.jpg} &
\includegraphics[width=\mywidthx, height=\myheightx, keepaspectratio]{figures/renderings/appendix_qualitative/ogre_merl_hematite_disney_view_02_1_flip_error.jpg} &
\includegraphics[width=\mywidthx, height=\myheightx, keepaspectratio]{figures/renderings/appendix_qualitative/ogre_merl_hematite_single_MLP_view_02_1_flip_error.jpg} &
\includegraphics[width=\mywidthx, height=\myheightx, keepaspectratio]{figures/renderings/appendix_qualitative/ogre_merl_hematite_add_sep_view_02_1_flip_error.jpg} &
\includegraphics[width=\mywidthx, height=\myheightx, keepaspectratio]{figures/renderings/appendix_qualitative/ogre_merl_hematite_add_shared_view_02_1_flip_error.jpg} &
& %
\includegraphics[width=\mywidthx, height=\heightcolorbar, keepaspectratio]{figures/renderings/appendix_qualitative/colorbar_magma.jpg} \\
\hline
\multirow{2}{*}{\rotatebox{90}{\parbox{3cm}{\centering \merlc \\ ipswich pine 221}}} &
\begin{tikzpicture}
\draw(0,0) node[inner sep=1] {\includegraphics[width=\mywidthx, height=\myheightx, keepaspectratio]{figures/renderings/appendix_qualitative/ogre_merl_ipswich_pine_221_phong_view_02_1_rendering.jpg}};
\node[align=left] at (\xposThree, \yposThree) {\scriptsize PSNR: \\\scriptsize 45.49};
\end{tikzpicture} &
\begin{tikzpicture}
\draw(0,0) node[inner sep=1] {\includegraphics[width=\mywidthx, height=\myheightx, keepaspectratio]{figures/renderings/appendix_qualitative/ogre_merl_ipswich_pine_221_micro_view_02_1_rendering.jpg}};
\node[align=left] at (\xposThree, \yposThree) {\scriptsize PSNR: \\\scriptsize 46.03};
\end{tikzpicture} &
\begin{tikzpicture}
\draw(0,0) node[inner sep=1] {\includegraphics[width=\mywidthx, height=\myheightx, keepaspectratio]{figures/renderings/appendix_qualitative/ogre_merl_ipswich_pine_221_fmbrdf_view_02_1_rendering.jpg}};
\node[align=left] at (\xposThree, \yposThree) {\scriptsize PSNR: \\\scriptsize 46.74};
\end{tikzpicture} &
\begin{tikzpicture}
\draw(0,0) node[inner sep=1] {\includegraphics[width=\mywidthx, height=\myheightx, keepaspectratio]{figures/renderings/appendix_qualitative/ogre_merl_ipswich_pine_221_disney_view_02_1_rendering.jpg}};
\node[align=left] at (\xposThree, \yposThree) {\scriptsize PSNR: \\\scriptsize 45.50};
\end{tikzpicture} &
\begin{tikzpicture}
\draw(0,0) node[inner sep=1] {\includegraphics[width=\mywidthx, height=\myheightx, keepaspectratio]{figures/renderings/appendix_qualitative/ogre_merl_ipswich_pine_221_single_MLP_view_02_1_rendering.jpg}};
\node[align=left] at (\xposThree, \yposThree) {\scriptsize PSNR: \\\scriptsize 54.65};
\end{tikzpicture} &
\begin{tikzpicture}
\draw(0,0) node[inner sep=1] {\includegraphics[width=\mywidthx, height=\myheightx, keepaspectratio]{figures/renderings/appendix_qualitative/ogre_merl_ipswich_pine_221_add_sep_view_02_1_rendering.jpg}};
\node[align=left] at (\xposThree, \yposThree) {\scriptsize PSNR: \\\scriptsize 54.52};
\end{tikzpicture} &
\begin{tikzpicture}
\draw(0,0) node[inner sep=1] {\includegraphics[width=\mywidthx, height=\myheightx, keepaspectratio]{figures/renderings/appendix_qualitative/ogre_merl_ipswich_pine_221_add_shared_view_02_1_rendering.jpg}};
\node[align=left] at (\xposThree, \yposThree) {\scriptsize PSNR: \\\scriptsize 55.13};
\end{tikzpicture} &
& %
\includegraphics[width=\mywidthx, height=\myheightx, keepaspectratio]{figures/renderings/appendix_qualitative/ogre_merl_ipswich_pine_221_view_02_1_gt.jpg}\\
 & \includegraphics[width=\mywidthx, height=\myheightx, keepaspectratio]{figures/renderings/appendix_qualitative/ogre_merl_ipswich_pine_221_phong_view_02_1_flip_error.jpg} &
\includegraphics[width=\mywidthx, height=\myheightx, keepaspectratio]{figures/renderings/appendix_qualitative/ogre_merl_ipswich_pine_221_micro_view_02_1_flip_error.jpg} &
\includegraphics[width=\mywidthx, height=\myheightx, keepaspectratio]{figures/renderings/appendix_qualitative/ogre_merl_ipswich_pine_221_fmbrdf_view_02_1_flip_error.jpg} &
\includegraphics[width=\mywidthx, height=\myheightx, keepaspectratio]{figures/renderings/appendix_qualitative/ogre_merl_ipswich_pine_221_disney_view_02_1_flip_error.jpg} &
\includegraphics[width=\mywidthx, height=\myheightx, keepaspectratio]{figures/renderings/appendix_qualitative/ogre_merl_ipswich_pine_221_single_MLP_view_02_1_flip_error.jpg} &
\includegraphics[width=\mywidthx, height=\myheightx, keepaspectratio]{figures/renderings/appendix_qualitative/ogre_merl_ipswich_pine_221_add_sep_view_02_1_flip_error.jpg} &
\includegraphics[width=\mywidthx, height=\myheightx, keepaspectratio]{figures/renderings/appendix_qualitative/ogre_merl_ipswich_pine_221_add_shared_view_02_1_flip_error.jpg} &
& %
\includegraphics[width=\mywidthx, height=\heightcolorbar, keepaspectratio]{figures/renderings/appendix_qualitative/colorbar_magma.jpg} \\
\hline
\multirow{2}{*}{\rotatebox{90}{\parbox{3cm}{\centering \merlc \\ maroon plastic}}} &
\begin{tikzpicture}
\draw(0,0) node[inner sep=1] {\includegraphics[width=\mywidthx, height=\myheightx, keepaspectratio]{figures/renderings/appendix_qualitative/lucy_merl_maroon_plastic_phong_view_02_1_rendering.jpg}};
\node[align=left] at (\xposFour, \yposFour) {\scriptsize PSNR: \\\scriptsize 45.85};
\end{tikzpicture} &
\begin{tikzpicture}
\draw(0,0) node[inner sep=1] {\includegraphics[width=\mywidthx, height=\myheightx, keepaspectratio]{figures/renderings/appendix_qualitative/lucy_merl_maroon_plastic_micro_view_02_1_rendering.jpg}};
\node[align=left] at (\xposFour, \yposFour) {\scriptsize PSNR: \\\scriptsize 47.33};
\end{tikzpicture} &
\begin{tikzpicture}
\draw(0,0) node[inner sep=1] {\includegraphics[width=\mywidthx, height=\myheightx, keepaspectratio]{figures/renderings/appendix_qualitative/lucy_merl_maroon_plastic_fmbrdf_view_02_1_rendering.jpg}};
\node[align=left] at (\xposFour, \yposFour) {\scriptsize PSNR: \\\scriptsize 52.38};
\end{tikzpicture} &
\begin{tikzpicture}
\draw(0,0) node[inner sep=1] {\includegraphics[width=\mywidthx, height=\myheightx, keepaspectratio]{figures/renderings/appendix_qualitative/lucy_merl_maroon_plastic_disney_view_02_1_rendering.jpg}};
\node[align=left] at (\xposFour, \yposFour) {\scriptsize PSNR: \\\scriptsize 52.03};
\end{tikzpicture} &
\begin{tikzpicture}
\draw(0,0) node[inner sep=1] {\includegraphics[width=\mywidthx, height=\myheightx, keepaspectratio]{figures/renderings/appendix_qualitative/lucy_merl_maroon_plastic_single_MLP_view_02_1_rendering.jpg}};
\node[align=left] at (\xposFour, \yposFour) {\scriptsize PSNR: \\\scriptsize 57.37};
\end{tikzpicture} &
\begin{tikzpicture}
\draw(0,0) node[inner sep=1] {\includegraphics[width=\mywidthx, height=\myheightx, keepaspectratio]{figures/renderings/appendix_qualitative/lucy_merl_maroon_plastic_add_sep_view_02_1_rendering.jpg}};
\node[align=left] at (\xposFour, \yposFour) {\scriptsize PSNR: \\\scriptsize 57.41};
\end{tikzpicture} &
\begin{tikzpicture}
\draw(0,0) node[inner sep=1] {\includegraphics[width=\mywidthx, height=\myheightx, keepaspectratio]{figures/renderings/appendix_qualitative/lucy_merl_maroon_plastic_add_shared_view_02_1_rendering.jpg}};
\node[align=left] at (\xposFour, \yposFour) {\scriptsize PSNR: \\\scriptsize 56.27};
\end{tikzpicture} &
& %
\includegraphics[width=\mywidthx, height=\myheightx, keepaspectratio]{figures/renderings/appendix_qualitative/lucy_merl_maroon_plastic_view_02_1_gt.jpg}\\
 & \includegraphics[width=\mywidthx, height=\myheightx, keepaspectratio]{figures/renderings/appendix_qualitative/lucy_merl_maroon_plastic_phong_view_02_1_flip_error.jpg} &
\includegraphics[width=\mywidthx, height=\myheightx, keepaspectratio]{figures/renderings/appendix_qualitative/lucy_merl_maroon_plastic_micro_view_02_1_flip_error.jpg} &
\includegraphics[width=\mywidthx, height=\myheightx, keepaspectratio]{figures/renderings/appendix_qualitative/lucy_merl_maroon_plastic_fmbrdf_view_02_1_flip_error.jpg} &
\includegraphics[width=\mywidthx, height=\myheightx, keepaspectratio]{figures/renderings/appendix_qualitative/lucy_merl_maroon_plastic_disney_view_02_1_flip_error.jpg} &
\includegraphics[width=\mywidthx, height=\myheightx, keepaspectratio]{figures/renderings/appendix_qualitative/lucy_merl_maroon_plastic_single_MLP_view_02_1_flip_error.jpg} &
\includegraphics[width=\mywidthx, height=\myheightx, keepaspectratio]{figures/renderings/appendix_qualitative/lucy_merl_maroon_plastic_add_sep_view_02_1_flip_error.jpg} &
\includegraphics[width=\mywidthx, height=\myheightx, keepaspectratio]{figures/renderings/appendix_qualitative/lucy_merl_maroon_plastic_add_shared_view_02_1_flip_error.jpg} &
& %
\includegraphics[width=\mywidthx, height=\heightcolorbar, keepaspectratio]{figures/renderings/appendix_qualitative/colorbar_magma.jpg} \\
\hline\\[-0.2cm]
 & \cellcolor{cellParamBased}\rpc		%
 & \cellcolor{cellParamBased}\tsc		%
 & \cellcolor{cellParamBased}\fmbrdfc		%
 & \cellcolor{cellParamBased}\disneyc		%
 & \cellcolor{celPurelyNeural}Single MLP		%
 & \cellcolor{celPurelyNeural}Add Sep		%
 & \cellcolor{celPurelyNeural}Add Shared		%
& %
 & \gt
  \end{tabular}
  \caption{
  Qualitative evaluation of the reconstruction for four BRDFs from the MERL database \cite{matusik2003MERL} uniformly rendered on common test meshes from \cite{jacobson2020common}. Shown are renderings in sRGB space with the corresponding PSNR values and the \FLIP error maps for the sRGB renderings.
  Purely neural approaches~(\mysquare[celPurelyNeural]) show superior results over the parametric models (\mysquare[cellParamBased]).
  }
\label{fig:supp:renderings_synth_3}
\end{figure*}
\begin{figure*}[t]  %
  \centering  %
  \footnotesize
  \newcommand{\mywidthc}{0.02\textwidth}  %
  \newcommand{\mywidthx}{0.11\textwidth}  %
  \newcommand{\mywidthw}{0.03\textwidth}  %
  \newcommand{\myheightx}{0.15\textwidth}  %
  \newcommand{\mywidtht}{0.035\textwidth}  %
  \newcolumntype{C}{ >{\centering\arraybackslash} m{\mywidthc} } %
  \newcolumntype{X}{ >{\centering\arraybackslash} m{\mywidthx} } %
  \newcolumntype{W}{ >{\centering\arraybackslash} m{\mywidthw} } %
  \newcolumntype{T}{ >{\centering\arraybackslash} m{\mywidtht} } %

  \newcommand{\heightcolorbar}{0.10\textwidth}  %
  \newcommand{\xposOne}{-0.7}
  \newcommand{\yposOne}{0.8}
  \newcommand{\xposTwo}{0.0}
  \newcommand{\yposTwo}{0.6}
  \newcommand{\xposThree}{-0.1}
  \newcommand{\yposThree}{0.7}
  \newcommand{\xposFour}{-1.05}
  \newcommand{\yposFour}{0.15}

  \newcommand{\fontsizePSNR}{\ssmall}
  
  \setlength\tabcolsep{0pt} %

  \setlength{\extrarowheight}{1.25pt}
  
  \def\arraystretch{0.8} %
  \begin{tabular}{TXXXXXXXWX}

\multirow{2}{*}{\rotatebox{90}{\parbox{3cm}{\centering \merlc \\ neoprene rubber}}} &
\begin{tikzpicture}
\draw(0,0) node[inner sep=1] {\includegraphics[width=\mywidthx, height=\myheightx, keepaspectratio]{figures/renderings/appendix_qualitative/happy_merl_neoprene_rubber_phong_view_02_1_rendering.jpg}};
\node[align=left] at (\xposOne, \yposOne) {\scriptsize PSNR: \\\scriptsize 46.59};
\end{tikzpicture} &
\begin{tikzpicture}
\draw(0,0) node[inner sep=1] {\includegraphics[width=\mywidthx, height=\myheightx, keepaspectratio]{figures/renderings/appendix_qualitative/happy_merl_neoprene_rubber_micro_view_02_1_rendering.jpg}};
\node[align=left] at (\xposOne, \yposOne) {\scriptsize PSNR: \\\scriptsize 47.21};
\end{tikzpicture} &
\begin{tikzpicture}
\draw(0,0) node[inner sep=1] {\includegraphics[width=\mywidthx, height=\myheightx, keepaspectratio]{figures/renderings/appendix_qualitative/happy_merl_neoprene_rubber_fmbrdf_view_02_1_rendering.jpg}};
\node[align=left] at (\xposOne, \yposOne) {\scriptsize PSNR: \\\scriptsize 46.59};
\end{tikzpicture} &
\begin{tikzpicture}
\draw(0,0) node[inner sep=1] {\includegraphics[width=\mywidthx, height=\myheightx, keepaspectratio]{figures/renderings/appendix_qualitative/happy_merl_neoprene_rubber_disney_view_02_1_rendering.jpg}};
\node[align=left] at (\xposOne, \yposOne) {\scriptsize PSNR: \\\scriptsize 45.21};
\end{tikzpicture} &
\begin{tikzpicture}
\draw(0,0) node[inner sep=1] {\includegraphics[width=\mywidthx, height=\myheightx, keepaspectratio]{figures/renderings/appendix_qualitative/happy_merl_neoprene_rubber_single_MLP_view_02_1_rendering.jpg}};
\node[align=left] at (\xposOne, \yposOne) {\scriptsize PSNR: \\\scriptsize 56.62};
\end{tikzpicture} &
\begin{tikzpicture}
\draw(0,0) node[inner sep=1] {\includegraphics[width=\mywidthx, height=\myheightx, keepaspectratio]{figures/renderings/appendix_qualitative/happy_merl_neoprene_rubber_add_sep_view_02_1_rendering.jpg}};
\node[align=left] at (\xposOne, \yposOne) {\scriptsize PSNR: \\\scriptsize 56.67};
\end{tikzpicture} &
\begin{tikzpicture}
\draw(0,0) node[inner sep=1] {\includegraphics[width=\mywidthx, height=\myheightx, keepaspectratio]{figures/renderings/appendix_qualitative/happy_merl_neoprene_rubber_add_shared_view_02_1_rendering.jpg}};
\node[align=left] at (\xposOne, \yposOne) {\scriptsize PSNR: \\\scriptsize 55.95};
\end{tikzpicture} &
& %
\includegraphics[width=\mywidthx, height=\myheightx, keepaspectratio]{figures/renderings/appendix_qualitative/happy_merl_neoprene_rubber_view_02_1_gt.jpg}\\
 & \includegraphics[width=\mywidthx, height=\myheightx, keepaspectratio]{figures/renderings/appendix_qualitative/happy_merl_neoprene_rubber_phong_view_02_1_flip_error.jpg} &
\includegraphics[width=\mywidthx, height=\myheightx, keepaspectratio]{figures/renderings/appendix_qualitative/happy_merl_neoprene_rubber_micro_view_02_1_flip_error.jpg} &
\includegraphics[width=\mywidthx, height=\myheightx, keepaspectratio]{figures/renderings/appendix_qualitative/happy_merl_neoprene_rubber_fmbrdf_view_02_1_flip_error.jpg} &
\includegraphics[width=\mywidthx, height=\myheightx, keepaspectratio]{figures/renderings/appendix_qualitative/happy_merl_neoprene_rubber_disney_view_02_1_flip_error.jpg} &
\includegraphics[width=\mywidthx, height=\myheightx, keepaspectratio]{figures/renderings/appendix_qualitative/happy_merl_neoprene_rubber_single_MLP_view_02_1_flip_error.jpg} &
\includegraphics[width=\mywidthx, height=\myheightx, keepaspectratio]{figures/renderings/appendix_qualitative/happy_merl_neoprene_rubber_add_sep_view_02_1_flip_error.jpg} &
\includegraphics[width=\mywidthx, height=\myheightx, keepaspectratio]{figures/renderings/appendix_qualitative/happy_merl_neoprene_rubber_add_shared_view_02_1_flip_error.jpg} &
& %
\includegraphics[width=\mywidthx, height=\heightcolorbar, keepaspectratio]{figures/renderings/appendix_qualitative/colorbar_magma.jpg} \\
\hline
\multirow{2}{*}{\rotatebox{90}{\parbox{3cm}{\centering \merlc \\ nickel}}} &
\begin{tikzpicture}
\draw(0,0) node[inner sep=1] {\includegraphics[width=\mywidthx, height=\myheightx, keepaspectratio]{figures/renderings/appendix_qualitative/dragon_merl_nickel_phong_view_02_1_rendering.jpg}};
\node[align=left] at (\xposTwo, \yposTwo) {\scriptsize PSNR: \\\scriptsize 40.75};
\end{tikzpicture} &
\begin{tikzpicture}
\draw(0,0) node[inner sep=1] {\includegraphics[width=\mywidthx, height=\myheightx, keepaspectratio]{figures/renderings/appendix_qualitative/dragon_merl_nickel_micro_view_02_1_rendering.jpg}};
\node[align=left] at (\xposTwo, \yposTwo) {\scriptsize PSNR: \\\scriptsize 44.03};
\end{tikzpicture} &
\begin{tikzpicture}
\draw(0,0) node[inner sep=1] {\includegraphics[width=\mywidthx, height=\myheightx, keepaspectratio]{figures/renderings/appendix_qualitative/dragon_merl_nickel_fmbrdf_view_02_1_rendering.jpg}};
\node[align=left] at (\xposTwo, \yposTwo) {\scriptsize PSNR: \\\scriptsize 42.93};
\end{tikzpicture} &
\begin{tikzpicture}
\draw(0,0) node[inner sep=1] {\includegraphics[width=\mywidthx, height=\myheightx, keepaspectratio]{figures/renderings/appendix_qualitative/dragon_merl_nickel_disney_view_02_1_rendering.jpg}};
\node[align=left] at (\xposTwo, \yposTwo) {\scriptsize PSNR: \\\scriptsize 44.22};
\end{tikzpicture} &
\begin{tikzpicture}
\draw(0,0) node[inner sep=1] {\includegraphics[width=\mywidthx, height=\myheightx, keepaspectratio]{figures/renderings/appendix_qualitative/dragon_merl_nickel_single_MLP_view_02_1_rendering.jpg}};
\node[align=left] at (\xposTwo, \yposTwo) {\scriptsize PSNR: \\\scriptsize 48.97};
\end{tikzpicture} &
\begin{tikzpicture}
\draw(0,0) node[inner sep=1] {\includegraphics[width=\mywidthx, height=\myheightx, keepaspectratio]{figures/renderings/appendix_qualitative/dragon_merl_nickel_add_sep_view_02_1_rendering.jpg}};
\node[align=left] at (\xposTwo, \yposTwo) {\scriptsize PSNR: \\\scriptsize 49.01};
\end{tikzpicture} &
\begin{tikzpicture}
\draw(0,0) node[inner sep=1] {\includegraphics[width=\mywidthx, height=\myheightx, keepaspectratio]{figures/renderings/appendix_qualitative/dragon_merl_nickel_add_shared_view_02_1_rendering.jpg}};
\node[align=left] at (\xposTwo, \yposTwo) {\scriptsize PSNR: \\\scriptsize 48.96};
\end{tikzpicture} &
& %
\includegraphics[width=\mywidthx, height=\myheightx, keepaspectratio]{figures/renderings/appendix_qualitative/dragon_merl_nickel_view_02_1_gt.jpg}\\
 & \includegraphics[width=\mywidthx, height=\myheightx, keepaspectratio]{figures/renderings/appendix_qualitative/dragon_merl_nickel_phong_view_02_1_flip_error.jpg} &
\includegraphics[width=\mywidthx, height=\myheightx, keepaspectratio]{figures/renderings/appendix_qualitative/dragon_merl_nickel_micro_view_02_1_flip_error.jpg} &
\includegraphics[width=\mywidthx, height=\myheightx, keepaspectratio]{figures/renderings/appendix_qualitative/dragon_merl_nickel_fmbrdf_view_02_1_flip_error.jpg} &
\includegraphics[width=\mywidthx, height=\myheightx, keepaspectratio]{figures/renderings/appendix_qualitative/dragon_merl_nickel_disney_view_02_1_flip_error.jpg} &
\includegraphics[width=\mywidthx, height=\myheightx, keepaspectratio]{figures/renderings/appendix_qualitative/dragon_merl_nickel_single_MLP_view_02_1_flip_error.jpg} &
\includegraphics[width=\mywidthx, height=\myheightx, keepaspectratio]{figures/renderings/appendix_qualitative/dragon_merl_nickel_add_sep_view_02_1_flip_error.jpg} &
\includegraphics[width=\mywidthx, height=\myheightx, keepaspectratio]{figures/renderings/appendix_qualitative/dragon_merl_nickel_add_shared_view_02_1_flip_error.jpg} &
& %
\includegraphics[width=\mywidthx, height=\heightcolorbar, keepaspectratio]{figures/renderings/appendix_qualitative/colorbar_magma.jpg} \\
\hline
\multirow{2}{*}{\rotatebox{90}{\parbox{3cm}{\centering \merlc \\ pink plastic}}} &
\begin{tikzpicture}
\draw(0,0) node[inner sep=1] {\includegraphics[width=\mywidthx, height=\myheightx, keepaspectratio]{figures/renderings/appendix_qualitative/horse_merl_pink_plastic_phong_view_02_1_rendering.jpg}};
\node[align=left] at (\xposThree, \yposThree) {\scriptsize PSNR: \\\scriptsize 46.96};
\end{tikzpicture} &
\begin{tikzpicture}
\draw(0,0) node[inner sep=1] {\includegraphics[width=\mywidthx, height=\myheightx, keepaspectratio]{figures/renderings/appendix_qualitative/horse_merl_pink_plastic_micro_view_02_1_rendering.jpg}};
\node[align=left] at (\xposThree, \yposThree) {\scriptsize PSNR: \\\scriptsize 49.89};
\end{tikzpicture} &
\begin{tikzpicture}
\draw(0,0) node[inner sep=1] {\includegraphics[width=\mywidthx, height=\myheightx, keepaspectratio]{figures/renderings/appendix_qualitative/horse_merl_pink_plastic_fmbrdf_view_02_1_rendering.jpg}};
\node[align=left] at (\xposThree, \yposThree) {\scriptsize PSNR: \\\scriptsize 48.11};
\end{tikzpicture} &
\begin{tikzpicture}
\draw(0,0) node[inner sep=1] {\includegraphics[width=\mywidthx, height=\myheightx, keepaspectratio]{figures/renderings/appendix_qualitative/horse_merl_pink_plastic_disney_view_02_1_rendering.jpg}};
\node[align=left] at (\xposThree, \yposThree) {\scriptsize PSNR: \\\scriptsize 51.35};
\end{tikzpicture} &
\begin{tikzpicture}
\draw(0,0) node[inner sep=1] {\includegraphics[width=\mywidthx, height=\myheightx, keepaspectratio]{figures/renderings/appendix_qualitative/horse_merl_pink_plastic_single_MLP_view_02_1_rendering.jpg}};
\node[align=left] at (\xposThree, \yposThree) {\scriptsize PSNR: \\\scriptsize 59.50};
\end{tikzpicture} &
\begin{tikzpicture}
\draw(0,0) node[inner sep=1] {\includegraphics[width=\mywidthx, height=\myheightx, keepaspectratio]{figures/renderings/appendix_qualitative/horse_merl_pink_plastic_add_sep_view_02_1_rendering.jpg}};
\node[align=left] at (\xposThree, \yposThree) {\scriptsize PSNR: \\\scriptsize 59.57};
\end{tikzpicture} &
\begin{tikzpicture}
\draw(0,0) node[inner sep=1] {\includegraphics[width=\mywidthx, height=\myheightx, keepaspectratio]{figures/renderings/appendix_qualitative/horse_merl_pink_plastic_add_shared_view_02_1_rendering.jpg}};
\node[align=left] at (\xposThree, \yposThree) {\scriptsize PSNR: \\\scriptsize 58.18};
\end{tikzpicture} &
& %
\includegraphics[width=\mywidthx, height=\myheightx, keepaspectratio]{figures/renderings/appendix_qualitative/horse_merl_pink_plastic_view_02_1_gt.jpg}\\
 & \includegraphics[width=\mywidthx, height=\myheightx, keepaspectratio]{figures/renderings/appendix_qualitative/horse_merl_pink_plastic_phong_view_02_1_flip_error.jpg} &
\includegraphics[width=\mywidthx, height=\myheightx, keepaspectratio]{figures/renderings/appendix_qualitative/horse_merl_pink_plastic_micro_view_02_1_flip_error.jpg} &
\includegraphics[width=\mywidthx, height=\myheightx, keepaspectratio]{figures/renderings/appendix_qualitative/horse_merl_pink_plastic_fmbrdf_view_02_1_flip_error.jpg} &
\includegraphics[width=\mywidthx, height=\myheightx, keepaspectratio]{figures/renderings/appendix_qualitative/horse_merl_pink_plastic_disney_view_02_1_flip_error.jpg} &
\includegraphics[width=\mywidthx, height=\myheightx, keepaspectratio]{figures/renderings/appendix_qualitative/horse_merl_pink_plastic_single_MLP_view_02_1_flip_error.jpg} &
\includegraphics[width=\mywidthx, height=\myheightx, keepaspectratio]{figures/renderings/appendix_qualitative/horse_merl_pink_plastic_add_sep_view_02_1_flip_error.jpg} &
\includegraphics[width=\mywidthx, height=\myheightx, keepaspectratio]{figures/renderings/appendix_qualitative/horse_merl_pink_plastic_add_shared_view_02_1_flip_error.jpg} &
& %
\includegraphics[width=\mywidthx, height=\heightcolorbar, keepaspectratio]{figures/renderings/appendix_qualitative/colorbar_magma.jpg} \\
\hline\\[-0.15cm]
\multirow{2}{*}{\rotatebox{90}{\parbox{3cm}{\centering \merlc \\ polyurethane foam}}} &
\begin{tikzpicture}
\draw(0,0) node[inner sep=1] {\includegraphics[width=\mywidthx, height=\myheightx, keepaspectratio]{figures/renderings/appendix_qualitative/armadillo_merl_polyurethane_foam_phong_view_02_1_rendering.jpg}};
\node[align=left] at (\xposFour, \yposFour) {\scriptsize PSNR: \\\scriptsize 44.41};
\end{tikzpicture} &
\begin{tikzpicture}
\draw(0,0) node[inner sep=1] {\includegraphics[width=\mywidthx, height=\myheightx, keepaspectratio]{figures/renderings/appendix_qualitative/armadillo_merl_polyurethane_foam_micro_view_02_1_rendering.jpg}};
\node[align=left] at (\xposFour, \yposFour) {\scriptsize PSNR: \\\scriptsize 51.35};
\end{tikzpicture} &
\begin{tikzpicture}
\draw(0,0) node[inner sep=1] {\includegraphics[width=\mywidthx, height=\myheightx, keepaspectratio]{figures/renderings/appendix_qualitative/armadillo_merl_polyurethane_foam_fmbrdf_view_02_1_rendering.jpg}};
\node[align=left] at (\xposFour, \yposFour) {\scriptsize PSNR: \\\scriptsize 49.41};
\end{tikzpicture} &
\begin{tikzpicture}
\draw(0,0) node[inner sep=1] {\includegraphics[width=\mywidthx, height=\myheightx, keepaspectratio]{figures/renderings/appendix_qualitative/armadillo_merl_polyurethane_foam_disney_view_02_1_rendering.jpg}};
\node[align=left] at (\xposFour, \yposFour) {\scriptsize PSNR: \\\scriptsize 51.08};
\end{tikzpicture} &
\begin{tikzpicture}
\draw(0,0) node[inner sep=1] {\includegraphics[width=\mywidthx, height=\myheightx, keepaspectratio]{figures/renderings/appendix_qualitative/armadillo_merl_polyurethane_foam_single_MLP_view_02_1_rendering.jpg}};
\node[align=left] at (\xposFour, \yposFour) {\scriptsize PSNR: \\\scriptsize 55.35};
\end{tikzpicture} &
\begin{tikzpicture}
\draw(0,0) node[inner sep=1] {\includegraphics[width=\mywidthx, height=\myheightx, keepaspectratio]{figures/renderings/appendix_qualitative/armadillo_merl_polyurethane_foam_add_sep_view_02_1_rendering.jpg}};
\node[align=left] at (\xposFour, \yposFour) {\scriptsize PSNR: \\\scriptsize 55.20};
\end{tikzpicture} &
\begin{tikzpicture}
\draw(0,0) node[inner sep=1] {\includegraphics[width=\mywidthx, height=\myheightx, keepaspectratio]{figures/renderings/appendix_qualitative/armadillo_merl_polyurethane_foam_add_shared_view_02_1_rendering.jpg}};
\node[align=left] at (\xposFour, \yposFour) {\scriptsize PSNR: \\\scriptsize 54.88};
\end{tikzpicture} &
& %
\includegraphics[width=\mywidthx, height=\myheightx, keepaspectratio]{figures/renderings/appendix_qualitative/armadillo_merl_polyurethane_foam_view_02_1_gt.jpg}\\
 & \includegraphics[width=\mywidthx, height=\myheightx, keepaspectratio]{figures/renderings/appendix_qualitative/armadillo_merl_polyurethane_foam_phong_view_02_1_flip_error.jpg} &
\includegraphics[width=\mywidthx, height=\myheightx, keepaspectratio]{figures/renderings/appendix_qualitative/armadillo_merl_polyurethane_foam_micro_view_02_1_flip_error.jpg} &
\includegraphics[width=\mywidthx, height=\myheightx, keepaspectratio]{figures/renderings/appendix_qualitative/armadillo_merl_polyurethane_foam_fmbrdf_view_02_1_flip_error.jpg} &
\includegraphics[width=\mywidthx, height=\myheightx, keepaspectratio]{figures/renderings/appendix_qualitative/armadillo_merl_polyurethane_foam_disney_view_02_1_flip_error.jpg} &
\includegraphics[width=\mywidthx, height=\myheightx, keepaspectratio]{figures/renderings/appendix_qualitative/armadillo_merl_polyurethane_foam_single_MLP_view_02_1_flip_error.jpg} &
\includegraphics[width=\mywidthx, height=\myheightx, keepaspectratio]{figures/renderings/appendix_qualitative/armadillo_merl_polyurethane_foam_add_sep_view_02_1_flip_error.jpg} &
\includegraphics[width=\mywidthx, height=\myheightx, keepaspectratio]{figures/renderings/appendix_qualitative/armadillo_merl_polyurethane_foam_add_shared_view_02_1_flip_error.jpg} &
& %
\includegraphics[width=\mywidthx, height=\heightcolorbar, keepaspectratio]{figures/renderings/appendix_qualitative/colorbar_magma.jpg} \\
\hline\\[-0.2cm]
 & \cellcolor{cellParamBased}\rpc		%
 & \cellcolor{cellParamBased}\tsc		%
 & \cellcolor{cellParamBased}\fmbrdfc		%
 & \cellcolor{cellParamBased}\disneyc		%
 & \cellcolor{celPurelyNeural}Single MLP		%
 & \cellcolor{celPurelyNeural}Add Sep		%
 & \cellcolor{celPurelyNeural}Add Shared		%
& %
 & \gt
  \end{tabular}
  \caption{
  Qualitative evaluation of the reconstruction for four BRDFs from the MERL database \cite{matusik2003MERL} uniformly rendered on common test meshes from \cite{jacobson2020common}. Shown are renderings in sRGB space with the corresponding PSNR values and the \FLIP error maps for the sRGB renderings.
  Purely neural approaches~(\mysquare[celPurelyNeural]) show superior results over the parametric models (\mysquare[cellParamBased]).
  }
\label{fig:supp:renderings_synth_4}
\end{figure*}
\begin{figure*}[t]  %
  \centering  %
  \footnotesize
  \newcommand{\mywidthc}{0.02\textwidth}  %
  \newcommand{\mywidthx}{0.11\textwidth}  %
  \newcommand{\mywidthw}{0.03\textwidth}  %
  \newcommand{\myheightx}{0.14\textwidth}  %
  \newcommand{\mywidtht}{0.035\textwidth}  %
  \newcolumntype{C}{ >{\centering\arraybackslash} m{\mywidthc} } %
  \newcolumntype{X}{ >{\centering\arraybackslash} m{\mywidthx} } %
  \newcolumntype{W}{ >{\centering\arraybackslash} m{\mywidthw} } %
  \newcolumntype{T}{ >{\centering\arraybackslash} m{\mywidtht} } %

  \newcommand{\heightcolorbar}{0.10\textwidth}  %
  \newcommand{\xposOne}{-0.95}
  \newcommand{\yposOne}{0.35}
  \newcommand{\xposTwo}{-1.05}
  \newcommand{\yposTwo}{0.16}
  \newcommand{\xposThree}{-0.7}
  \newcommand{\yposThree}{1.1}
  \newcommand{\xposFour}{-1.05}
  \newcommand{\yposFour}{0.16}

  \newcommand{\fontsizePSNR}{\ssmall}
  
  \setlength\tabcolsep{0pt} %

  \setlength{\extrarowheight}{1.25pt}
  
  \def\arraystretch{0.8} %
  \begin{tabular}{TXXXXXXXWX}

\multirow{2}{*}{\rotatebox{90}{\parbox{3cm}{\centering \merlc \\ red metallic paint}}} &
\begin{tikzpicture}
\draw(0,0) node[inner sep=1] {\includegraphics[width=\mywidthx, height=\myheightx, keepaspectratio]{figures/renderings/appendix_qualitative/stanfordBunny_merl_red_metallic_paint_phong_view_02_1_rendering.jpg}};
\node[align=left] at (\xposOne, \yposOne) {\scriptsize PSNR: \\\scriptsize 40.33};
\end{tikzpicture} &
\begin{tikzpicture}
\draw(0,0) node[inner sep=1] {\includegraphics[width=\mywidthx, height=\myheightx, keepaspectratio]{figures/renderings/appendix_qualitative/stanfordBunny_merl_red_metallic_paint_micro_view_02_1_rendering.jpg}};
\node[align=left] at (\xposOne, \yposOne) {\scriptsize PSNR: \\\scriptsize 44.40};
\end{tikzpicture} &
\begin{tikzpicture}
\draw(0,0) node[inner sep=1] {\includegraphics[width=\mywidthx, height=\myheightx, keepaspectratio]{figures/renderings/appendix_qualitative/stanfordBunny_merl_red_metallic_paint_fmbrdf_view_02_1_rendering.jpg}};
\node[align=left] at (\xposOne, \yposOne) {\scriptsize PSNR: \\\scriptsize 35.78};
\end{tikzpicture} &
\begin{tikzpicture}
\draw(0,0) node[inner sep=1] {\includegraphics[width=\mywidthx, height=\myheightx, keepaspectratio]{figures/renderings/appendix_qualitative/stanfordBunny_merl_red_metallic_paint_disney_view_02_1_rendering.jpg}};
\node[align=left] at (\xposOne, \yposOne) {\scriptsize PSNR: \\\scriptsize 44.75};
\end{tikzpicture} &
\begin{tikzpicture}
\draw(0,0) node[inner sep=1] {\includegraphics[width=\mywidthx, height=\myheightx, keepaspectratio]{figures/renderings/appendix_qualitative/stanfordBunny_merl_red_metallic_paint_single_MLP_view_02_1_rendering.jpg}};
\node[align=left] at (\xposOne, \yposOne) {\scriptsize PSNR: \\\scriptsize 47.83};
\end{tikzpicture} &
\begin{tikzpicture}
\draw(0,0) node[inner sep=1] {\includegraphics[width=\mywidthx, height=\myheightx, keepaspectratio]{figures/renderings/appendix_qualitative/stanfordBunny_merl_red_metallic_paint_add_sep_view_02_1_rendering.jpg}};
\node[align=left] at (\xposOne, \yposOne) {\scriptsize PSNR: \\\scriptsize 47.78};
\end{tikzpicture} &
\begin{tikzpicture}
\draw(0,0) node[inner sep=1] {\includegraphics[width=\mywidthx, height=\myheightx, keepaspectratio]{figures/renderings/appendix_qualitative/stanfordBunny_merl_red_metallic_paint_add_shared_view_02_1_rendering.jpg}};
\node[align=left] at (\xposOne, \yposOne) {\scriptsize PSNR: \\\scriptsize 47.56};
\end{tikzpicture} &
& %
\includegraphics[width=\mywidthx, height=\myheightx, keepaspectratio]{figures/renderings/appendix_qualitative/stanfordBunny_merl_red_metallic_paint_view_02_1_gt.jpg}\\
 & \includegraphics[width=\mywidthx, height=\myheightx, keepaspectratio]{figures/renderings/appendix_qualitative/stanfordBunny_merl_red_metallic_paint_phong_view_02_1_flip_error.jpg} &
\includegraphics[width=\mywidthx, height=\myheightx, keepaspectratio]{figures/renderings/appendix_qualitative/stanfordBunny_merl_red_metallic_paint_micro_view_02_1_flip_error.jpg} &
\includegraphics[width=\mywidthx, height=\myheightx, keepaspectratio]{figures/renderings/appendix_qualitative/stanfordBunny_merl_red_metallic_paint_fmbrdf_view_02_1_flip_error.jpg} &
\includegraphics[width=\mywidthx, height=\myheightx, keepaspectratio]{figures/renderings/appendix_qualitative/stanfordBunny_merl_red_metallic_paint_disney_view_02_1_flip_error.jpg} &
\includegraphics[width=\mywidthx, height=\myheightx, keepaspectratio]{figures/renderings/appendix_qualitative/stanfordBunny_merl_red_metallic_paint_single_MLP_view_02_1_flip_error.jpg} &
\includegraphics[width=\mywidthx, height=\myheightx, keepaspectratio]{figures/renderings/appendix_qualitative/stanfordBunny_merl_red_metallic_paint_add_sep_view_02_1_flip_error.jpg} &
\includegraphics[width=\mywidthx, height=\myheightx, keepaspectratio]{figures/renderings/appendix_qualitative/stanfordBunny_merl_red_metallic_paint_add_shared_view_02_1_flip_error.jpg} &
& %
\includegraphics[width=\mywidthx, height=\heightcolorbar, keepaspectratio]{figures/renderings/appendix_qualitative/colorbar_magma.jpg} \\
\hline
\multirow{2}{*}{\rotatebox{90}{\parbox{3cm}{\centering \merlc \\ red plastic}}} &
\begin{tikzpicture}
\draw(0,0) node[inner sep=1] {\includegraphics[width=\mywidthx, height=\myheightx, keepaspectratio]{figures/renderings/appendix_qualitative/armadillo_merl_red_plastic_phong_view_02_1_rendering.jpg}};
\node[align=left] at (\xposTwo, \yposTwo) {\scriptsize PSNR: \\\scriptsize 49.17};
\end{tikzpicture} &
\begin{tikzpicture}
\draw(0,0) node[inner sep=1] {\includegraphics[width=\mywidthx, height=\myheightx, keepaspectratio]{figures/renderings/appendix_qualitative/armadillo_merl_red_plastic_micro_view_02_1_rendering.jpg}};
\node[align=left] at (\xposTwo, \yposTwo) {\scriptsize PSNR: \\\scriptsize 49.34};
\end{tikzpicture} &
\begin{tikzpicture}
\draw(0,0) node[inner sep=1] {\includegraphics[width=\mywidthx, height=\myheightx, keepaspectratio]{figures/renderings/appendix_qualitative/armadillo_merl_red_plastic_fmbrdf_view_02_1_rendering.jpg}};
\node[align=left] at (\xposTwo, \yposTwo) {\scriptsize PSNR: \\\scriptsize 49.45};
\end{tikzpicture} &
\begin{tikzpicture}
\draw(0,0) node[inner sep=1] {\includegraphics[width=\mywidthx, height=\myheightx, keepaspectratio]{figures/renderings/appendix_qualitative/armadillo_merl_red_plastic_disney_view_02_1_rendering.jpg}};
\node[align=left] at (\xposTwo, \yposTwo) {\scriptsize PSNR: \\\scriptsize 51.41};
\end{tikzpicture} &
\begin{tikzpicture}
\draw(0,0) node[inner sep=1] {\includegraphics[width=\mywidthx, height=\myheightx, keepaspectratio]{figures/renderings/appendix_qualitative/armadillo_merl_red_plastic_single_MLP_view_02_1_rendering.jpg}};
\node[align=left] at (\xposTwo, \yposTwo) {\scriptsize PSNR: \\\scriptsize 54.15};
\end{tikzpicture} &
\begin{tikzpicture}
\draw(0,0) node[inner sep=1] {\includegraphics[width=\mywidthx, height=\myheightx, keepaspectratio]{figures/renderings/appendix_qualitative/armadillo_merl_red_plastic_add_sep_view_02_1_rendering.jpg}};
\node[align=left] at (\xposTwo, \yposTwo) {\scriptsize PSNR: \\\scriptsize 54.09};
\end{tikzpicture} &
\begin{tikzpicture}
\draw(0,0) node[inner sep=1] {\includegraphics[width=\mywidthx, height=\myheightx, keepaspectratio]{figures/renderings/appendix_qualitative/armadillo_merl_red_plastic_add_shared_view_02_1_rendering.jpg}};
\node[align=left] at (\xposTwo, \yposTwo) {\scriptsize PSNR: \\\scriptsize 53.83};
\end{tikzpicture} &
& %
\includegraphics[width=\mywidthx, height=\myheightx, keepaspectratio]{figures/renderings/appendix_qualitative/armadillo_merl_red_plastic_view_02_1_gt.jpg}\\
 & \includegraphics[width=\mywidthx, height=\myheightx, keepaspectratio]{figures/renderings/appendix_qualitative/armadillo_merl_red_plastic_phong_view_02_1_flip_error.jpg} &
\includegraphics[width=\mywidthx, height=\myheightx, keepaspectratio]{figures/renderings/appendix_qualitative/armadillo_merl_red_plastic_micro_view_02_1_flip_error.jpg} &
\includegraphics[width=\mywidthx, height=\myheightx, keepaspectratio]{figures/renderings/appendix_qualitative/armadillo_merl_red_plastic_fmbrdf_view_02_1_flip_error.jpg} &
\includegraphics[width=\mywidthx, height=\myheightx, keepaspectratio]{figures/renderings/appendix_qualitative/armadillo_merl_red_plastic_disney_view_02_1_flip_error.jpg} &
\includegraphics[width=\mywidthx, height=\myheightx, keepaspectratio]{figures/renderings/appendix_qualitative/armadillo_merl_red_plastic_single_MLP_view_02_1_flip_error.jpg} &
\includegraphics[width=\mywidthx, height=\myheightx, keepaspectratio]{figures/renderings/appendix_qualitative/armadillo_merl_red_plastic_add_sep_view_02_1_flip_error.jpg} &
\includegraphics[width=\mywidthx, height=\myheightx, keepaspectratio]{figures/renderings/appendix_qualitative/armadillo_merl_red_plastic_add_shared_view_02_1_flip_error.jpg} &
& %
\includegraphics[width=\mywidthx, height=\heightcolorbar, keepaspectratio]{figures/renderings/appendix_qualitative/colorbar_magma.jpg} \\
\hline
\multirow{2}{*}{\rotatebox{90}{\parbox{3cm}{\centering \merlc \\ specular maroon phenolic}}} &
\begin{tikzpicture}
\draw(0,0) node[inner sep=1] {\includegraphics[width=\mywidthx, height=\myheightx, keepaspectratio]{figures/renderings/appendix_qualitative/lucy_merl_specular_maroon_phenolic_phong_view_02_1_rendering.jpg}};
\node[align=left] at (\xposThree, \yposThree) {\scriptsize PSNR: \\\scriptsize 46.48};
\end{tikzpicture} &
\begin{tikzpicture}
\draw(0,0) node[inner sep=1] {\includegraphics[width=\mywidthx, height=\myheightx, keepaspectratio]{figures/renderings/appendix_qualitative/lucy_merl_specular_maroon_phenolic_micro_view_02_1_rendering.jpg}};
\node[align=left] at (\xposThree, \yposThree) {\scriptsize PSNR: \\\scriptsize 48.80};
\end{tikzpicture} &
\begin{tikzpicture}
\draw(0,0) node[inner sep=1] {\includegraphics[width=\mywidthx, height=\myheightx, keepaspectratio]{figures/renderings/appendix_qualitative/lucy_merl_specular_maroon_phenolic_fmbrdf_view_02_1_rendering.jpg}};
\node[align=left] at (\xposThree, \yposThree) {\scriptsize PSNR: \\\scriptsize 50.19};
\end{tikzpicture} &
\begin{tikzpicture}
\draw(0,0) node[inner sep=1] {\includegraphics[width=\mywidthx, height=\myheightx, keepaspectratio]{figures/renderings/appendix_qualitative/lucy_merl_specular_maroon_phenolic_disney_view_02_1_rendering.jpg}};
\node[align=left] at (\xposThree, \yposThree) {\scriptsize PSNR: \\\scriptsize 45.41};
\end{tikzpicture} &
\begin{tikzpicture}
\draw(0,0) node[inner sep=1] {\includegraphics[width=\mywidthx, height=\myheightx, keepaspectratio]{figures/renderings/appendix_qualitative/lucy_merl_specular_maroon_phenolic_single_MLP_view_02_1_rendering.jpg}};
\node[align=left] at (\xposThree, \yposThree) {\scriptsize PSNR: \\\scriptsize 55.25};
\end{tikzpicture} &
\begin{tikzpicture}
\draw(0,0) node[inner sep=1] {\includegraphics[width=\mywidthx, height=\myheightx, keepaspectratio]{figures/renderings/appendix_qualitative/lucy_merl_specular_maroon_phenolic_add_sep_view_02_1_rendering.jpg}};
\node[align=left] at (\xposThree, \yposThree) {\scriptsize PSNR: \\\scriptsize 55.31};
\end{tikzpicture} &
\begin{tikzpicture}
\draw(0,0) node[inner sep=1] {\includegraphics[width=\mywidthx, height=\myheightx, keepaspectratio]{figures/renderings/appendix_qualitative/lucy_merl_specular_maroon_phenolic_add_shared_view_02_1_rendering.jpg}};
\node[align=left] at (\xposThree, \yposThree) {\scriptsize PSNR: \\\scriptsize 54.25};
\end{tikzpicture} &
& %
\includegraphics[width=\mywidthx, height=\myheightx, keepaspectratio]{figures/renderings/appendix_qualitative/lucy_merl_specular_maroon_phenolic_view_02_1_gt.jpg}\\
 & \includegraphics[width=\mywidthx, height=\myheightx, keepaspectratio]{figures/renderings/appendix_qualitative/lucy_merl_specular_maroon_phenolic_phong_view_02_1_flip_error.jpg} &
\includegraphics[width=\mywidthx, height=\myheightx, keepaspectratio]{figures/renderings/appendix_qualitative/lucy_merl_specular_maroon_phenolic_micro_view_02_1_flip_error.jpg} &
\includegraphics[width=\mywidthx, height=\myheightx, keepaspectratio]{figures/renderings/appendix_qualitative/lucy_merl_specular_maroon_phenolic_fmbrdf_view_02_1_flip_error.jpg} &
\includegraphics[width=\mywidthx, height=\myheightx, keepaspectratio]{figures/renderings/appendix_qualitative/lucy_merl_specular_maroon_phenolic_disney_view_02_1_flip_error.jpg} &
\includegraphics[width=\mywidthx, height=\myheightx, keepaspectratio]{figures/renderings/appendix_qualitative/lucy_merl_specular_maroon_phenolic_single_MLP_view_02_1_flip_error.jpg} &
\includegraphics[width=\mywidthx, height=\myheightx, keepaspectratio]{figures/renderings/appendix_qualitative/lucy_merl_specular_maroon_phenolic_add_sep_view_02_1_flip_error.jpg} &
\includegraphics[width=\mywidthx, height=\myheightx, keepaspectratio]{figures/renderings/appendix_qualitative/lucy_merl_specular_maroon_phenolic_add_shared_view_02_1_flip_error.jpg} &
& %
\includegraphics[width=\mywidthx, height=\heightcolorbar, keepaspectratio]{figures/renderings/appendix_qualitative/colorbar_magma.jpg} \\
\hline
\multirow{2}{*}{\rotatebox{90}{\parbox{3cm}{\centering \merlc \\ specular white phenolic}}} &
\begin{tikzpicture}
\draw(0,0) node[inner sep=1] {\includegraphics[width=\mywidthx, height=\myheightx, keepaspectratio]{figures/renderings/appendix_qualitative/armadillo_merl_specular_white_phenolic_phong_view_02_1_rendering.jpg}};
\node[align=left] at (\xposFour, \yposFour) {\scriptsize PSNR: \\\scriptsize 44.92};
\end{tikzpicture} &
\begin{tikzpicture}
\draw(0,0) node[inner sep=1] {\includegraphics[width=\mywidthx, height=\myheightx, keepaspectratio]{figures/renderings/appendix_qualitative/armadillo_merl_specular_white_phenolic_micro_view_02_1_rendering.jpg}};
\node[align=left] at (\xposFour, \yposFour) {\scriptsize PSNR: \\\scriptsize 46.02};
\end{tikzpicture} &
\begin{tikzpicture}
\draw(0,0) node[inner sep=1] {\includegraphics[width=\mywidthx, height=\myheightx, keepaspectratio]{figures/renderings/appendix_qualitative/armadillo_merl_specular_white_phenolic_fmbrdf_view_02_1_rendering.jpg}};
\node[align=left] at (\xposFour, \yposFour) {\scriptsize PSNR: \\\scriptsize 49.43};
\end{tikzpicture} &
\begin{tikzpicture}
\draw(0,0) node[inner sep=1] {\includegraphics[width=\mywidthx, height=\myheightx, keepaspectratio]{figures/renderings/appendix_qualitative/armadillo_merl_specular_white_phenolic_disney_view_02_1_rendering.jpg}};
\node[align=left] at (\xposFour, \yposFour) {\scriptsize PSNR: \\\scriptsize 41.20};
\end{tikzpicture} &
\begin{tikzpicture}
\draw(0,0) node[inner sep=1] {\includegraphics[width=\mywidthx, height=\myheightx, keepaspectratio]{figures/renderings/appendix_qualitative/armadillo_merl_specular_white_phenolic_single_MLP_view_02_1_rendering.jpg}};
\node[align=left] at (\xposFour, \yposFour) {\scriptsize PSNR: \\\scriptsize 56.22};
\end{tikzpicture} &
\begin{tikzpicture}
\draw(0,0) node[inner sep=1] {\includegraphics[width=\mywidthx, height=\myheightx, keepaspectratio]{figures/renderings/appendix_qualitative/armadillo_merl_specular_white_phenolic_add_sep_view_02_1_rendering.jpg}};
\node[align=left] at (\xposFour, \yposFour) {\scriptsize PSNR: \\\scriptsize 56.16};
\end{tikzpicture} &
\begin{tikzpicture}
\draw(0,0) node[inner sep=1] {\includegraphics[width=\mywidthx, height=\myheightx, keepaspectratio]{figures/renderings/appendix_qualitative/armadillo_merl_specular_white_phenolic_add_shared_view_02_1_rendering.jpg}};
\node[align=left] at (\xposFour, \yposFour) {\scriptsize PSNR: \\\scriptsize 55.16};
\end{tikzpicture} &
& %
\includegraphics[width=\mywidthx, height=\myheightx, keepaspectratio]{figures/renderings/appendix_qualitative/armadillo_merl_specular_white_phenolic_view_02_1_gt.jpg}\\
 & \includegraphics[width=\mywidthx, height=\myheightx, keepaspectratio]{figures/renderings/appendix_qualitative/armadillo_merl_specular_white_phenolic_phong_view_02_1_flip_error.jpg} &
\includegraphics[width=\mywidthx, height=\myheightx, keepaspectratio]{figures/renderings/appendix_qualitative/armadillo_merl_specular_white_phenolic_micro_view_02_1_flip_error.jpg} &
\includegraphics[width=\mywidthx, height=\myheightx, keepaspectratio]{figures/renderings/appendix_qualitative/armadillo_merl_specular_white_phenolic_fmbrdf_view_02_1_flip_error.jpg} &
\includegraphics[width=\mywidthx, height=\myheightx, keepaspectratio]{figures/renderings/appendix_qualitative/armadillo_merl_specular_white_phenolic_disney_view_02_1_flip_error.jpg} &
\includegraphics[width=\mywidthx, height=\myheightx, keepaspectratio]{figures/renderings/appendix_qualitative/armadillo_merl_specular_white_phenolic_single_MLP_view_02_1_flip_error.jpg} &
\includegraphics[width=\mywidthx, height=\myheightx, keepaspectratio]{figures/renderings/appendix_qualitative/armadillo_merl_specular_white_phenolic_add_sep_view_02_1_flip_error.jpg} &
\includegraphics[width=\mywidthx, height=\myheightx, keepaspectratio]{figures/renderings/appendix_qualitative/armadillo_merl_specular_white_phenolic_add_shared_view_02_1_flip_error.jpg} &
& %
\includegraphics[width=\mywidthx, height=\heightcolorbar, keepaspectratio]{figures/renderings/appendix_qualitative/colorbar_magma.jpg} \\
\hline\\[-0.2cm]
 & \cellcolor{cellParamBased}\rpc		%
 & \cellcolor{cellParamBased}\tsc		%
 & \cellcolor{cellParamBased}\fmbrdfc		%
 & \cellcolor{cellParamBased}\disneyc		%
 & \cellcolor{celPurelyNeural}Single MLP		%
 & \cellcolor{celPurelyNeural}Add Sep		%
 & \cellcolor{celPurelyNeural}Add Shared		%
& %
 & \gt
  \end{tabular}
  \caption{
  Qualitative evaluation of the reconstruction for four BRDFs from the MERL database \cite{matusik2003MERL} uniformly rendered on common test meshes from \cite{jacobson2020common}. Shown are renderings in sRGB space with the corresponding PSNR values and the \FLIP error maps for the sRGB renderings.
  Purely neural approaches~(\mysquare[celPurelyNeural]) show superior results over the parametric models (\mysquare[cellParamBased]).
  }
\label{fig:supp:renderings_synth_5}
\end{figure*}
\begin{figure*}[t]  %
  \centering  %
  \footnotesize
  \newcommand{\mywidthc}{0.02\textwidth}  %
  \newcommand{\mywidthx}{0.11\textwidth}  %
  \newcommand{\mywidthw}{0.017\textwidth}  %
  \newcommand{\myheightx}{0.14\textwidth}  %
  \newcommand{\mywidtht}{0.035\textwidth}  %
  \newcolumntype{C}{ >{\centering\arraybackslash} m{\mywidthc} } %
  \newcolumntype{X}{ >{\centering\arraybackslash} m{\mywidthx} } %
  \newcolumntype{W}{ >{\centering\arraybackslash} m{\mywidthw} } %
  \newcolumntype{T}{ >{\centering\arraybackslash} m{\mywidtht} } %

  \newcommand{\heightcolorbar}{0.10\textwidth}  %
  \newcommand{\xposOne}{-0.7}
  \newcommand{\yposOne}{1.1}
  \newcommand{\xposTwo}{-0.8}
  \newcommand{\yposTwo}{1.05}
  \newcommand{\xposThree}{-0.7}
  \newcommand{\yposThree}{0.8}
  \newcommand{\xposFour}{0.0}
  \newcommand{\yposFour}{0.6}

  \newcommand{\fontsizePSNR}{\ssmall}
  
  \setlength\tabcolsep{0pt} %

  \setlength{\extrarowheight}{1.25pt}
  
  \def\arraystretch{0.8} %
  \begin{tabular}{TXXXXXXXWX}

\multirow{2}{*}{\rotatebox{90}{\parbox{3cm}{\centering \merlc \\ white diffuse bball}}} &
\begin{tikzpicture}
\draw(0,0) node[inner sep=1] {\includegraphics[width=\mywidthx, height=\myheightx, keepaspectratio]{figures/renderings/appendix_qualitative/lucy_merl_white_diffuse_bball_phong_view_02_1_rendering.jpg}};
\node[align=left] at (\xposOne, \yposOne) {\scriptsize PSNR: \\\scriptsize 47.94};
\end{tikzpicture} &
\begin{tikzpicture}
\draw(0,0) node[inner sep=1] {\includegraphics[width=\mywidthx, height=\myheightx, keepaspectratio]{figures/renderings/appendix_qualitative/lucy_merl_white_diffuse_bball_micro_view_02_1_rendering.jpg}};
\node[align=left] at (\xposOne, \yposOne) {\scriptsize PSNR: \\\scriptsize 47.48};
\end{tikzpicture} &
\begin{tikzpicture}
\draw(0,0) node[inner sep=1] {\includegraphics[width=\mywidthx, height=\myheightx, keepaspectratio]{figures/renderings/appendix_qualitative/lucy_merl_white_diffuse_bball_fmbrdf_view_02_1_rendering.jpg}};
\node[align=left] at (\xposOne, \yposOne) {\scriptsize PSNR: \\\scriptsize 52.26};
\end{tikzpicture} &
\begin{tikzpicture}
\draw(0,0) node[inner sep=1] {\includegraphics[width=\mywidthx, height=\myheightx, keepaspectratio]{figures/renderings/appendix_qualitative/lucy_merl_white_diffuse_bball_disney_view_02_1_rendering.jpg}};
\node[align=left] at (\xposOne, \yposOne) {\scriptsize PSNR: \\\scriptsize 48.78};
\end{tikzpicture} &
\begin{tikzpicture}
\draw(0,0) node[inner sep=1] {\includegraphics[width=\mywidthx, height=\myheightx, keepaspectratio]{figures/renderings/appendix_qualitative/lucy_merl_white_diffuse_bball_single_MLP_view_02_1_rendering.jpg}};
\node[align=left] at (\xposOne, \yposOne) {\scriptsize PSNR: \\\scriptsize 59.95};
\end{tikzpicture} &
\begin{tikzpicture}
\draw(0,0) node[inner sep=1] {\includegraphics[width=\mywidthx, height=\myheightx, keepaspectratio]{figures/renderings/appendix_qualitative/lucy_merl_white_diffuse_bball_add_sep_view_02_1_rendering.jpg}};
\node[align=left] at (\xposOne, \yposOne) {\scriptsize PSNR: \\\scriptsize 59.33};
\end{tikzpicture} &
\begin{tikzpicture}
\draw(0,0) node[inner sep=1] {\includegraphics[width=\mywidthx, height=\myheightx, keepaspectratio]{figures/renderings/appendix_qualitative/lucy_merl_white_diffuse_bball_add_shared_view_02_1_rendering.jpg}};
\node[align=left] at (\xposOne, \yposOne) {\scriptsize PSNR: \\\scriptsize 58.60};
\end{tikzpicture} &
& %
\includegraphics[width=\mywidthx, height=\myheightx, keepaspectratio]{figures/renderings/appendix_qualitative/lucy_merl_white_diffuse_bball_view_02_1_gt.jpg}\\
 & \includegraphics[width=\mywidthx, height=\myheightx, keepaspectratio]{figures/renderings/appendix_qualitative/lucy_merl_white_diffuse_bball_phong_view_02_1_flip_error.jpg} &
\includegraphics[width=\mywidthx, height=\myheightx, keepaspectratio]{figures/renderings/appendix_qualitative/lucy_merl_white_diffuse_bball_micro_view_02_1_flip_error.jpg} &
\includegraphics[width=\mywidthx, height=\myheightx, keepaspectratio]{figures/renderings/appendix_qualitative/lucy_merl_white_diffuse_bball_fmbrdf_view_02_1_flip_error.jpg} &
\includegraphics[width=\mywidthx, height=\myheightx, keepaspectratio]{figures/renderings/appendix_qualitative/lucy_merl_white_diffuse_bball_disney_view_02_1_flip_error.jpg} &
\includegraphics[width=\mywidthx, height=\myheightx, keepaspectratio]{figures/renderings/appendix_qualitative/lucy_merl_white_diffuse_bball_single_MLP_view_02_1_flip_error.jpg} &
\includegraphics[width=\mywidthx, height=\myheightx, keepaspectratio]{figures/renderings/appendix_qualitative/lucy_merl_white_diffuse_bball_add_sep_view_02_1_flip_error.jpg} &
\includegraphics[width=\mywidthx, height=\myheightx, keepaspectratio]{figures/renderings/appendix_qualitative/lucy_merl_white_diffuse_bball_add_shared_view_02_1_flip_error.jpg} &
& %
\includegraphics[width=\mywidthx, height=\heightcolorbar, keepaspectratio]{figures/renderings/appendix_qualitative/colorbar_magma.jpg} \\
\hline
\multirow{2}{*}{\rotatebox{90}{\parbox{3cm}{\centering \merlc \\ white fabric2}}} &
\begin{tikzpicture}
\draw(0,0) node[inner sep=1] {\includegraphics[width=\mywidthx, height=\myheightx, keepaspectratio]{figures/renderings/appendix_qualitative/nefertiti_merl_white_fabric2_phong_view_02_1_rendering.jpg}};
\node[align=left] at (\xposTwo, \yposTwo) {\scriptsize PSNR: \\\scriptsize 37.86};
\end{tikzpicture} &
\begin{tikzpicture}
\draw(0,0) node[inner sep=1] {\includegraphics[width=\mywidthx, height=\myheightx, keepaspectratio]{figures/renderings/appendix_qualitative/nefertiti_merl_white_fabric2_micro_view_02_1_rendering.jpg}};
\node[align=left] at (\xposTwo, \yposTwo) {\scriptsize PSNR: \\\scriptsize 47.32};
\end{tikzpicture} &
\begin{tikzpicture}
\draw(0,0) node[inner sep=1] {\includegraphics[width=\mywidthx, height=\myheightx, keepaspectratio]{figures/renderings/appendix_qualitative/nefertiti_merl_white_fabric2_fmbrdf_view_02_1_rendering.jpg}};
\node[align=left] at (\xposTwo, \yposTwo) {\scriptsize PSNR: \\\scriptsize 46.78};
\end{tikzpicture} &
\begin{tikzpicture}
\draw(0,0) node[inner sep=1] {\includegraphics[width=\mywidthx, height=\myheightx, keepaspectratio]{figures/renderings/appendix_qualitative/nefertiti_merl_white_fabric2_disney_view_02_1_rendering.jpg}};
\node[align=left] at (\xposTwo, \yposTwo) {\scriptsize PSNR: \\\scriptsize 46.72};
\end{tikzpicture} &
\begin{tikzpicture}
\draw(0,0) node[inner sep=1] {\includegraphics[width=\mywidthx, height=\myheightx, keepaspectratio]{figures/renderings/appendix_qualitative/nefertiti_merl_white_fabric2_single_MLP_view_02_1_rendering.jpg}};
\node[align=left] at (\xposTwo, \yposTwo) {\scriptsize PSNR: \\\scriptsize 56.62};
\end{tikzpicture} &
\begin{tikzpicture}
\draw(0,0) node[inner sep=1] {\includegraphics[width=\mywidthx, height=\myheightx, keepaspectratio]{figures/renderings/appendix_qualitative/nefertiti_merl_white_fabric2_add_sep_view_02_1_rendering.jpg}};
\node[align=left] at (\xposTwo, \yposTwo) {\scriptsize PSNR: \\\scriptsize 56.57};
\end{tikzpicture} &
\begin{tikzpicture}
\draw(0,0) node[inner sep=1] {\includegraphics[width=\mywidthx, height=\myheightx, keepaspectratio]{figures/renderings/appendix_qualitative/nefertiti_merl_white_fabric2_add_shared_view_02_1_rendering.jpg}};
\node[align=left] at (\xposTwo, \yposTwo) {\scriptsize PSNR: \\\scriptsize 55.99};
\end{tikzpicture} &
& %
\includegraphics[width=\mywidthx, height=\myheightx, keepaspectratio]{figures/renderings/appendix_qualitative/nefertiti_merl_white_fabric2_view_02_1_gt.jpg}\\
 & \includegraphics[width=\mywidthx, height=\myheightx, keepaspectratio]{figures/renderings/appendix_qualitative/nefertiti_merl_white_fabric2_phong_view_02_1_flip_error.jpg} &
\includegraphics[width=\mywidthx, height=\myheightx, keepaspectratio]{figures/renderings/appendix_qualitative/nefertiti_merl_white_fabric2_micro_view_02_1_flip_error.jpg} &
\includegraphics[width=\mywidthx, height=\myheightx, keepaspectratio]{figures/renderings/appendix_qualitative/nefertiti_merl_white_fabric2_fmbrdf_view_02_1_flip_error.jpg} &
\includegraphics[width=\mywidthx, height=\myheightx, keepaspectratio]{figures/renderings/appendix_qualitative/nefertiti_merl_white_fabric2_disney_view_02_1_flip_error.jpg} &
\includegraphics[width=\mywidthx, height=\myheightx, keepaspectratio]{figures/renderings/appendix_qualitative/nefertiti_merl_white_fabric2_single_MLP_view_02_1_flip_error.jpg} &
\includegraphics[width=\mywidthx, height=\myheightx, keepaspectratio]{figures/renderings/appendix_qualitative/nefertiti_merl_white_fabric2_add_sep_view_02_1_flip_error.jpg} &
\includegraphics[width=\mywidthx, height=\myheightx, keepaspectratio]{figures/renderings/appendix_qualitative/nefertiti_merl_white_fabric2_add_shared_view_02_1_flip_error.jpg} &
& %
\includegraphics[width=\mywidthx, height=\heightcolorbar, keepaspectratio]{figures/renderings/appendix_qualitative/colorbar_magma.jpg} \\
\hline
\multirow{2}{*}{\rotatebox{90}{\parbox{3cm}{\centering \merlc \\ white marble}}} &
\begin{tikzpicture}
\draw(0,0) node[inner sep=1] {\includegraphics[width=\mywidthx, height=\myheightx, keepaspectratio]{figures/renderings/appendix_qualitative/happy_merl_white_marble_phong_view_02_1_rendering.jpg}};
\node[align=left] at (\xposThree, \yposThree) {\scriptsize PSNR: \\\scriptsize 42.31};
\end{tikzpicture} &
\begin{tikzpicture}
\draw(0,0) node[inner sep=1] {\includegraphics[width=\mywidthx, height=\myheightx, keepaspectratio]{figures/renderings/appendix_qualitative/happy_merl_white_marble_micro_view_02_1_rendering.jpg}};
\node[align=left] at (\xposThree, \yposThree) {\scriptsize PSNR: \\\scriptsize 42.91};
\end{tikzpicture} &
\begin{tikzpicture}
\draw(0,0) node[inner sep=1] {\includegraphics[width=\mywidthx, height=\myheightx, keepaspectratio]{figures/renderings/appendix_qualitative/happy_merl_white_marble_fmbrdf_view_02_1_rendering.jpg}};
\node[align=left] at (\xposThree, \yposThree) {\scriptsize PSNR: \\\scriptsize 46.11};
\end{tikzpicture} &
\begin{tikzpicture}
\draw(0,0) node[inner sep=1] {\includegraphics[width=\mywidthx, height=\myheightx, keepaspectratio]{figures/renderings/appendix_qualitative/happy_merl_white_marble_disney_view_02_1_rendering.jpg}};
\node[align=left] at (\xposThree, \yposThree) {\scriptsize PSNR: \\\scriptsize 46.51};
\end{tikzpicture} &
\begin{tikzpicture}
\draw(0,0) node[inner sep=1] {\includegraphics[width=\mywidthx, height=\myheightx, keepaspectratio]{figures/renderings/appendix_qualitative/happy_merl_white_marble_single_MLP_view_02_1_rendering.jpg}};
\node[align=left] at (\xposThree, \yposThree) {\scriptsize PSNR: \\\scriptsize 56.18};
\end{tikzpicture} &
\begin{tikzpicture}
\draw(0,0) node[inner sep=1] {\includegraphics[width=\mywidthx, height=\myheightx, keepaspectratio]{figures/renderings/appendix_qualitative/happy_merl_white_marble_add_sep_view_02_1_rendering.jpg}};
\node[align=left] at (\xposThree, \yposThree) {\scriptsize PSNR: \\\scriptsize 56.02};
\end{tikzpicture} &
\begin{tikzpicture}
\draw(0,0) node[inner sep=1] {\includegraphics[width=\mywidthx, height=\myheightx, keepaspectratio]{figures/renderings/appendix_qualitative/happy_merl_white_marble_add_shared_view_02_1_rendering.jpg}};
\node[align=left] at (\xposThree, \yposThree) {\scriptsize PSNR: \\\scriptsize 54.93};
\end{tikzpicture} &
& %
\includegraphics[width=\mywidthx, height=\myheightx, keepaspectratio]{figures/renderings/appendix_qualitative/happy_merl_white_marble_view_02_1_gt.jpg}\\
 & \includegraphics[width=\mywidthx, height=\myheightx, keepaspectratio]{figures/renderings/appendix_qualitative/happy_merl_white_marble_phong_view_02_1_flip_error.jpg} &
\includegraphics[width=\mywidthx, height=\myheightx, keepaspectratio]{figures/renderings/appendix_qualitative/happy_merl_white_marble_micro_view_02_1_flip_error.jpg} &
\includegraphics[width=\mywidthx, height=\myheightx, keepaspectratio]{figures/renderings/appendix_qualitative/happy_merl_white_marble_fmbrdf_view_02_1_flip_error.jpg} &
\includegraphics[width=\mywidthx, height=\myheightx, keepaspectratio]{figures/renderings/appendix_qualitative/happy_merl_white_marble_disney_view_02_1_flip_error.jpg} &
\includegraphics[width=\mywidthx, height=\myheightx, keepaspectratio]{figures/renderings/appendix_qualitative/happy_merl_white_marble_single_MLP_view_02_1_flip_error.jpg} &
\includegraphics[width=\mywidthx, height=\myheightx, keepaspectratio]{figures/renderings/appendix_qualitative/happy_merl_white_marble_add_sep_view_02_1_flip_error.jpg} &
\includegraphics[width=\mywidthx, height=\myheightx, keepaspectratio]{figures/renderings/appendix_qualitative/happy_merl_white_marble_add_shared_view_02_1_flip_error.jpg} &
& %
\includegraphics[width=\mywidthx, height=\heightcolorbar, keepaspectratio]{figures/renderings/appendix_qualitative/colorbar_magma.jpg} \\
\hline
\multirow{2}{*}{\rotatebox{90}{\parbox{3cm}{\centering \merlc \\ yellow matte plastic}}} &
\begin{tikzpicture}
\draw(0,0) node[inner sep=1] {\includegraphics[width=\mywidthx, height=\myheightx, keepaspectratio]{figures/renderings/appendix_qualitative/dragon_merl_yellow_matte_plastic_phong_view_02_1_rendering.jpg}};
\node[align=left] at (\xposFour, \yposFour) {\scriptsize PSNR: \\\scriptsize 45.29};
\end{tikzpicture} &
\begin{tikzpicture}
\draw(0,0) node[inner sep=1] {\includegraphics[width=\mywidthx, height=\myheightx, keepaspectratio]{figures/renderings/appendix_qualitative/dragon_merl_yellow_matte_plastic_micro_view_02_1_rendering.jpg}};
\node[align=left] at (\xposFour, \yposFour) {\scriptsize PSNR: \\\scriptsize 46.47};
\end{tikzpicture} &
\begin{tikzpicture}
\draw(0,0) node[inner sep=1] {\includegraphics[width=\mywidthx, height=\myheightx, keepaspectratio]{figures/renderings/appendix_qualitative/dragon_merl_yellow_matte_plastic_fmbrdf_view_02_1_rendering.jpg}};
\node[align=left] at (\xposFour, \yposFour) {\scriptsize PSNR: \\\scriptsize 51.36};
\end{tikzpicture} &
\begin{tikzpicture}
\draw(0,0) node[inner sep=1] {\includegraphics[width=\mywidthx, height=\myheightx, keepaspectratio]{figures/renderings/appendix_qualitative/dragon_merl_yellow_matte_plastic_disney_view_02_1_rendering.jpg}};
\node[align=left] at (\xposFour, \yposFour) {\scriptsize PSNR: \\\scriptsize 41.82};
\end{tikzpicture} &
\begin{tikzpicture}
\draw(0,0) node[inner sep=1] {\includegraphics[width=\mywidthx, height=\myheightx, keepaspectratio]{figures/renderings/appendix_qualitative/dragon_merl_yellow_matte_plastic_single_MLP_view_02_1_rendering.jpg}};
\node[align=left] at (\xposFour, \yposFour) {\scriptsize PSNR: \\\scriptsize 54.78};
\end{tikzpicture} &
\begin{tikzpicture}
\draw(0,0) node[inner sep=1] {\includegraphics[width=\mywidthx, height=\myheightx, keepaspectratio]{figures/renderings/appendix_qualitative/dragon_merl_yellow_matte_plastic_add_sep_view_02_1_rendering.jpg}};
\node[align=left] at (\xposFour, \yposFour) {\scriptsize PSNR: \\\scriptsize 55.03};
\end{tikzpicture} &
\begin{tikzpicture}
\draw(0,0) node[inner sep=1] {\includegraphics[width=\mywidthx, height=\myheightx, keepaspectratio]{figures/renderings/appendix_qualitative/dragon_merl_yellow_matte_plastic_add_shared_view_02_1_rendering.jpg}};
\node[align=left] at (\xposFour, \yposFour) {\scriptsize PSNR: \\\scriptsize 53.90};
\end{tikzpicture} &
& %
\includegraphics[width=\mywidthx, height=\myheightx, keepaspectratio]{figures/renderings/appendix_qualitative/dragon_merl_yellow_matte_plastic_view_02_1_gt.jpg}\\
 & \includegraphics[width=\mywidthx, height=\myheightx, keepaspectratio]{figures/renderings/appendix_qualitative/dragon_merl_yellow_matte_plastic_phong_view_02_1_flip_error.jpg} &
\includegraphics[width=\mywidthx, height=\myheightx, keepaspectratio]{figures/renderings/appendix_qualitative/dragon_merl_yellow_matte_plastic_micro_view_02_1_flip_error.jpg} &
\includegraphics[width=\mywidthx, height=\myheightx, keepaspectratio]{figures/renderings/appendix_qualitative/dragon_merl_yellow_matte_plastic_fmbrdf_view_02_1_flip_error.jpg} &
\includegraphics[width=\mywidthx, height=\myheightx, keepaspectratio]{figures/renderings/appendix_qualitative/dragon_merl_yellow_matte_plastic_disney_view_02_1_flip_error.jpg} &
\includegraphics[width=\mywidthx, height=\myheightx, keepaspectratio]{figures/renderings/appendix_qualitative/dragon_merl_yellow_matte_plastic_single_MLP_view_02_1_flip_error.jpg} &
\includegraphics[width=\mywidthx, height=\myheightx, keepaspectratio]{figures/renderings/appendix_qualitative/dragon_merl_yellow_matte_plastic_add_sep_view_02_1_flip_error.jpg} &
\includegraphics[width=\mywidthx, height=\myheightx, keepaspectratio]{figures/renderings/appendix_qualitative/dragon_merl_yellow_matte_plastic_add_shared_view_02_1_flip_error.jpg} &
& %
\includegraphics[width=\mywidthx, height=\heightcolorbar, keepaspectratio]{figures/renderings/appendix_qualitative/colorbar_magma.jpg} \\
\hline\\[-0.2cm]
 & \cellcolor{cellParamBased}\rpc		%
 & \cellcolor{cellParamBased}\tsc		%
 & \cellcolor{cellParamBased}\fmbrdfc		%
 & \cellcolor{cellParamBased}\disneyc		%
 & \cellcolor{celPurelyNeural}Single MLP		%
 & \cellcolor{celPurelyNeural}Add Sep		%
 & \cellcolor{celPurelyNeural}Add Shared		%
& %
 & \gt
  \end{tabular}
  \caption{
  Qualitative evaluation of the reconstruction for four BRDFs from the MERL database \cite{matusik2003MERL} uniformly rendered on common test meshes from \cite{jacobson2020common}. Shown are renderings in sRGB space with the corresponding PSNR values and the \FLIP error maps for the sRGB renderings.
  Purely neural approaches~(\mysquare[celPurelyNeural]) show superior results over the parametric models (\mysquare[cellParamBased]).
  }
\label{fig:supp:renderings_synth_6}
\end{figure*}
