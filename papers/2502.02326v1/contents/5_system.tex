\section{Interface Design}
With the constructed data flow and recommended charts, we further design the interface of \system{} to support the recommendation and efficient tracing on the data flow.
We demonstrate the interface design of \system{}.

\begin{figure*}[!htb]
    \centering
    \includegraphics[width=\textwidth]{figures/interface.png}
    \caption{The Interface of \system{}. \system{} contains a chart view showing the recommended charts under each cell (A \& B), which can be opened or collapsed. Users can select a chart of interest for detail tracing across the flow. A flow view is shown on the right, demonstrating the relationships between data tables, with the chart selected and column details (C). A pinned view is situated on the top right of the interface, showing the charts that have been selected. The interface contains a control panel for switching the variables to be traced and downloading, uploading, and re-running the results.}
    \label{fig:interface}
\end{figure*}

To facilitate exploration, we have created an interactive interface that is tightly integrated with Jupyter Notebook. This interface consists of two key components, including chart views (\autoref{fig:interface}-A \& B), a flow view (\autoref{fig:interface}-C), a pinned list (\autoref{fig:interface}-D), and a control panel (\autoref{fig:interface}-E).

\subsection{Chart View}
The chart view (\autoref{fig:interface}-A \& B) is placed under each cell, showing the recommended charts for the tables in the cell. 
The chart view is collapsed by default to avoid information overwhelming (\autoref{fig:interface}-A), and users can click the button (\autoref{fig:interface}-A2) to open the view. 
An example of the opened view can be found in \autoref{fig:interface}-B, where the color of the button is changed (\autoref{fig:interface}-B1).
Note that each cell contains multiple tables, and each table will receive a list of recommendations.
The recommended charts of all tables in a cell are ranked together and listed in the chart view, as mentioned in ~\autoref{sec:ranking}.
Therefore, we provide a filtering panel available for users to accurately specify the data table and columns they are interested in (\autoref{fig:interface}-B2). 
By default, \system{} show the charts of the latest data variable. 
The user can further specify the factors for the recommendation for the chart, such as operation types and fact types (\autoref{fig:interface}-B3). 
Upon applying these filters, the list of charts will be updated, allowing users to explore and choose the charts that align with their preferences and requirements.
Each recommended chart (\autoref{fig:interface}-B4) is accompanied by a list of column details (\autoref{fig:interface}-B5).
Users can click the ``pin'' button (\autoref{fig:interface}-B6) to trace the table changes with the same visual encodings of the pinned chart within the flow view (\autoref{fig:interface}-C). 


\subsection{Flow View}
The flow view (\autoref{fig:interface}-C) provides an overview of the whole EDA flow, with the traced chart acting as the ``sight glasses'' (\textbf{R1}). The flow graph (\autoref{fig:interface}-C3) and traced charts (\autoref{fig:interface}-C1 \& C2) are displayed side by side. Each node represents a data table at a specific line of code, maintaining consistent visual encodings across the flow. For instance, the chart and column details in \autoref{fig:interface}-C2 correspond to the table \code{df\_C3\_L1}, which is the value of the data frame \code{df} at line 1 of cell 3.

Nodes are color-coded to represent their states. Blue nodes (\autoref{fig:interface}-C4) indicate a chart different from the preceding one, and \system{} displays both by default. Light blue nodes (\autoref{fig:interface}-C7) indicate similar charts, which are closed by default but can be opened by clicking. Red nodes (\autoref{fig:interface}-C8) represent untraceable nodes, such as \code{df\_groupby\_C4\_L2}, where a \code{groupby} operation removed the "cylinder" column, making chart rendering impossible.

Link colors encode relationships between nodes and the traced one. Black links indicate a relationship, with the direction showing the order of operations. For example, the link in \autoref{fig:interface}-C5 shows that \code{df\_C3\_L1} was transformed into \code{df\_copy\_C5\_L1} after several operations. Grey links indicate no direct relationship, as in \autoref{fig:interface}-C6, where \code{df\_groupby\_C4\_L2} has no direct transformation link to \code{df\_copy\_C5\_L1}.

\begin{figure*}[!htb]
    \centering
    \includegraphics[width=0.5\textwidth]{figures/re-run.png}
    \caption{The stepped layout after re-running some of the cells. The cells before (A) and after (B) the re-running correspond to the flow on the left column (C) and the right one (D). When \system{} detects a re-running, a new column will be appended on the right, showing the re-running cell and the succeeding ones. A link will connect the nodes of different versions (E) for better understanding.}
    \label{fig:re-run}
\end{figure*}

The flow view utilizes a stepped layout to efficiently trace past tables after re-running cells (\textbf{R4}). As shown in \autoref{fig:re-run}, the analyst revises and re-runs cell four (\autoref{fig:re-run}-A), changing its index to six (\autoref{fig:re-run}-B). Most tools overwrite the original result, losing the previous state. In contrast, \system{} uses a stepped layout to preserve the original state, creating a new column on the right (\autoref{fig:re-run}-D) to display the new results alongside the old ones (\autoref{fig:re-run}-C). The links between nodes indicate their relationships.
This layout offers two advantages (\autoref{fig:re-run}-E). First, side-by-side visualization allows for easy comparison between different code versions. Second, the stepped layout avoids cyclic links within the same column, making the flow clearer than with the overwriting approach.

\subsection{Pinned View and Control Panel}
The pinned charts are listed on the right (\autoref{fig:interface}-D), where users can switch the tables and chart configurations for tracing by clicking on the card in the list.
In this case, users click the first chart for tracing (\autoref{fig:interface}-D1).

The control panel is situated on the top (\autoref{fig:interface}-E).
Users can change the traced node with a toggle list.
Users can also download the exploration log, upload a previous log, and regenerate the charts and flows using the control panel.
