\section{Related Work}
\label{relatedwork}

While research on system identification and optimal control for FOLTI systems remains limited, significant progress has been made in LTI systems. Estimating unknown parameters of linear dynamical systems is a well-established subfield of system identification in control theory \cite{JMLR:v22:19-725, faradonbeh2018finitetimeidentificationunstable, simchowitz2018learningmixingsharpanalysis}, where the task is to estimate parameters from input-output time series generated by the underlying system. With the increasing interest in machine learning and deep learning, many researchers use different deep learning based modeling techniques to solve general nonlinear system identification. \citet{chen1990non} shows a single hidden layer neural network can identify discrete time nonlinear systems and derives new parameter estimation algorithms based on a prediction error formulation. \citet{80202} demonstrates that neural networks can be used effectively for both identification and control of nonlinear dynamical systems. The hierarchical structures of multilayer feedforward neural networks can include dynamic systems features \cite{zancato2021noveldeepneuralnetwork} and bring extra flexibility for probablistic approaches \cite{hendriks2020deepenergybasednarxmodels}. Kernel-based methods also have been studied in system identification. Linear system identification can be seen as an application of learning the impulse response function, which can be approximated by kernels \cite{aronszajn1950theory,dinuzzo2012representer, cho2009kernel}. Deep state-space models \cite{gedon2021deep} like recurrent neural networks \cite{hochreiter1997long, cho2014learningphraserepresentationsusing} and autoencoders \cite{masti2021learning, lusch2018deep} are also gaining in popularity for system identification. For Another critical aspect of optimal control via LQR is the design of control inputs, which are typically formulated as linear combinations of disturbance processes \cite{dean2020sample}.
    
Solving the optimal control problem via LQR involves addressing a potentially large linear system with a block Toeplitz matrix of size $\mathbb{R}^{Tn \times Tn}$. While solutions for Toeplitz and block Toeplitz systems have been extensively studied \cite{trench1986solution, kalouptsidis1984fast, chandrasekaran1998fast}, they still present significant computational challenges. Direct solution methods for Toeplitz systems typically exhibit a computational complexity of $\mathcal{O}(N^2)$, where $N$ represents the degrees of freedom. However, the high memory requirements of these methods often limit their applicability to large-scale problems. Iterative solvers \cite{chan2007introduction, strang1986proposal} are more memory-efficient and can achieve a complexity of $\mathcal{O}(N \log^2(N))$ for a single linear block Toeplitz system. Advanced approaches, such as global and block variants of the generalized minimal residual (GMRES) method \cite{saad1986gmres}, can efficiently address sequences of block Toeplitz systems with multiple right-hand sides \cite{jelich2021efficient}. Machine learning-based methods, such as neural operators \cite{li2021fourierneuraloperatorparametric, gupta2021multiwavelet, gupta2022non, xiao2022coupled}, offer a promising alternative for accelerating the solution of linear systems. Neural operator-assisted Krylov iterations \cite{luo2024neural}, for example, have demonstrated significant computational advantages, achieving up to a $5.5\times$ speedup in computation time and a $16.1\times$ reduction in the number of iterations. These advancements underscore the potential of machine learning techniques to address computational bottlenecks in optimal control for FOLTI systems.