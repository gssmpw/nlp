\section{Background and Related Work}
\subsection{Background}
\textbf{Volumetric Video Streaming.}
Existing volumetric video streaming methods can be categorized based on whether the transmission data constitutes the intermediate results of NN models or not~\cite{gao2022openpointcloud}:
(1) non-DL-based compression, e.g., G-PCC and V-PCC by MPEG~\cite{mpegpcc}, Pleno~\cite{jpeg}
and Draco~\cite{draco}.
(2) DL-based compression, e.g., PCGCv2~\cite{pcgcv2}, SparsePCGC~\cite{spcgc}, etc.

A typical DL-based volumetric video compression workflow is shown in Fig.~\ref{fig:intro}. As the 3D application service provider, the cloud constructs and trains a pair of NN models: an encoder and a decoder through a \textit{private} dataset. Only the well-trained encoder will be distributed to users, while the decoder is secured on the cloud. Upon capturing volumetric video, users compress it into intermediate results using this encoder. These results are then transmitted to the cloud and decoded by the secured decoder, ultimately decompressing to the original volumetric video. 

In this paper, we focus on recovering the volumetric video frame, i.e., point clouds, from the intercepted intermediate results during transmission in DL-based volumetric video compression.
Some researchers aim to enhance non-DL-based approaches for improved efficiency~\cite{akhtar2020point}, reduced distortion~\cite{zhang2023gpccpp}, and privacy protection~\cite{lu2023pagoda} using deep learning approaches. However, these methods are out of our scope because the transmitted data retains the same in non-DL-based approaches, which differs from the intermediate results of NN models in DL-based.

\noindent\textbf{Attacks on Volumetric Video Streaming.}
Attack on streaming is a serious threat that involves unauthorized access and interception of streaming data. Attackers can manipulate or impair transmission data, potentially jeopardizing confidentiality and integrity~\cite{ni2020security}.
Attackers may adopt various tactics, including faking as a colluded or corrupted server to extract users' private information~\cite{ren2023hcnct} or engaging in man-in-the-middle attacks~\cite{vladimirov2022security}. 
Among such attackers, reconstruction attackers target to recover the original input point cloud from the intermediate results. It attracts significant attention in deep learning~\cite{jegorova2022survey}. For instance, in language models, reconstruction attacks aim to reconstruct input text~\cite{brown2022does}, while in computer vision, they are to reconstruct input images~\cite{lu2022preva}. 

Reconstruction attacks usually target the application layer, which is quite different from attackers cracking encryption in the transport layer. For example, a malicious cloud manipulates users into sending their video data, even if the transmission is secured through HTTPS. While encryption methods can safeguard the data during transmission, they do not prevent the exposure of private information contained within the video content. To illustrate this kind of attack, Fig.~\ref{fig:intro} provides an example in red, demonstrating the attacking process of reconstructing point clouds from intercepted intermediate results. 
Before starting an attack, the attacker needs to train a corresponding attacking model capable of reconstructing the intercepted intermediate results. The attack commences by hijacking the intermediate results transmitted from the user to the cloud.
Reconstructed point clouds can be identified with sensitive information like facial features, gender, race, etc.

Various reconstruction attack approaches exist, including Supervised-based~\cite{dosovitskiy2016inverting}, GAN-based~\cite{spgan}, etc. However, the state-of-the-art generative solution, the latent diffusion model~\cite{rombach2022ldm}, has not been used in reconstruction attacks yet. 
Another similar attack is the attribute inference attack (AIA)~\cite{rigaki2023survey}, which extracts attributes of original inputs, e.g., gender, race, etc, from intermediate results. AIA only needs to recover a few points with attribute features.
Unlike AIA, reconstruction attacks demand more from attackers' capability as they need to recover the entire input data accurately.

\subsection{Related Work}
\textbf{Diffusion models (DMs)} have emerged as the most powerful generative models, showcasing remarkable performance in sample quality and good modality coverage~\cite{huang2021variational}, in area of image generation~\cite{anomaly,frido}, audio synthesis~\cite{kong2020diffwave}, etc, compared to other generative schemes, e.g., GANs, VAEs, etc.
DMs contain two phases: the \textit{forward} process to arbitrary noises, e.g., Gaussian noise~\cite{ho2020ddpm}, Poisson noise~\cite{xu2022poisson}, and the \textit{reverse} process to approximate the target distribution by denoising. 
Despite the success across diverse tasks, DMs suffer from sampling delays for thousands of network inference iterations, making them costly in real-world applications. However, it makes them particularly suitable for our attack scenario, where strict time constraints and computation capability are not decisive requirements. 
Generally, the Gaussian distribution in DM may lead to poor sample generation, e.g., the prior hole problem~\cite{sinha2021d2c,vahdat2021score}; however,
\textit{Denoising Diffusion Gamma Model} (DDGM)~\cite{nachmani2021ddgm} replaces it with the Gamma one, which has been proven to exhibit higher model capacity and improved convergence, particularly in high-dimensional data~\cite{chang2023design,croitoru2023diffusion}.

\noindent\textbf{Latent diffusion models (LDMs)} are initially introduced to enhance the efficiency of conventional DMs in high-resolution image generation~\cite{rombach2022ldm}. These models encode high-dimensional input data into a low-dimensional latent space and train DMs within this latent space. Those latents from the high-dimensional point cloud are named \textit{shape latent}~\cite{zeng2022lion}.
Samples generated in the latent space are then decoded back to their original high-dimensional data. 
In reconstruction attacks, the intermediate results happen to be in a low dimension, i.e., \textit{shape latents}, where the encoder of the victim model is the role of encoding high-dimensional to the low-dimensional latent space. All these features naturally contribute to the adoption of a new attacking scheme based on LDMs.