\section{Experimental Setup}
\label{sec:experimental-setup}

We use Cadence Genus v19.15~\cite{genus} and Innovus v20.15~\cite{innovus} in a physical design flow (Verilog to GDSII) using our custom cell library.
The library is functional from 250\,mV, which was chosen as the target voltage for synthesis and physical flow of all the cores.  Each core is given a square floorplan with an area resulting
in around 70\% utilization after routing.  We constrain synthesis and place and route to the smallest clock period resulting in no timing violations, since our evaluation has shown that this results in improved energy efficiency.

% Cores
We evaluate all the cores mentioned in \autoref{sec:cores}.  
Each core is configured with the RV32E profile intended for low-power embedded systems, which reduces the number of registers to 16.
They are configured for the smallest area by disabling as many features as possible, e.g., no hardware support for multiplication.
For the case of SERV, QERV, PicoRV32, and Ibex, a custom latch-based register file is used to further reduce their size.
Modifying the Vex and Rocket designs to use the latch-based register file proved challenging, and we instead used their respective conventional flip-flop-based register files.
Since our main focus is on the impact of the core implementation itself, we have not included a memory in the designs and assume a memory latency of a single cycle.

\begin{table}[t]
  \centering
  \caption{Results from final layout.}
  \label{tab:PRcores}
  \begin{tabular}{l|r|c|r}
    Core     & Clock Period  & Area                       & Transistors \\
    \hline \hline
    SERV     &   478\,ns     & 0.096\,mm\textsuperscript{2} &  25 648 \\
    QERV     &   512\,ns     & 0.109\,mm\textsuperscript{2} &  28 968 \\
    PicoRV32 &   686\,ns     & 0.235\,mm\textsuperscript{2} &  61 858 \\
    Vex-2    & 1 458\,ns     & 0.250\,mm\textsuperscript{2} &  62 294 \\
    Ibex     & 1 450\,ns     & 0.384\,mm\textsuperscript{2} & 103 630 \\
    Vex-5    &   998\,ns     & 0.423\,mm\textsuperscript{2} & 114 532 \\
    Rocket   & 1 434\,ns     & 0.792\,mm\textsuperscript{2} & 209 092 \\
    \hline
    Vex-2 (1.2\,V)  & 16\,ns & 0.090\,mm\textsuperscript{2} &  48 091 \\
    Vex-5 (1.2\,V) &  4\,ns & 0.160\,mm\textsuperscript{2} &  88 402 \\
  \end{tabular}
\end{table}

% Energy optimal supply voltage
We used the SERV and PicoRV32 cores to identify the most energy efficient supply voltage for our cell library by comparing energy usage on a range of voltages from 250\,mV to 600\,mV.  Simulations show that both cores are most energy efficient with a supply voltage close to 300\,mV. 
All cores in this paper are, therefore, evaluated and compared with a supply voltage of 300\,mV. 
This is in line with earlier findings by other researchers~\cite{liu1993trading, wang2006sub}; the energy minimum typically lies below the absolute values of the inherent threshold voltages.

A summary of the subthreshold designs after performing the full physical flow is shown in the top part of \autoref{tab:PRcores}.
% 1.2 V
We also synthesized and evaluated the two Vex configurations using the conventional standard cell library designed for a nominal supply voltage of 1.2\,V.
We restrict the available cells to the same type and number of cells available in our own cell library.
This enables us to compare different architectural design choices and their impact in a conventional cell library and one optimized for subthreshold operation.
The results from the 1.2\,V cores are shown in the lower part of \autoref{tab:PRcores}.  Synopsys PrimeTime~\cite{primetime} %~\cite{primetime}
was used for timing analysis.  

% Benchmarks
We selected eight benchmarks from MachSuite~\cite{machsuite:IISWC2014} as representative workloads for low-power embedded devices.
We used Siemens QuestaSim~\cite{questa}
to simulate each benchmark for each core using the netlist from the final layout.
The resulting waveforms were fed together with the netlist and parasitic information to Synopsys PrimeTime Power~\cite{primetime}
to estimate the average power and energy usage for each benchmark.
The estimated power for a small test case has been compared against an XCelium simulation using the same setup as explained in \autoref{sec:cell_lib_verification}.
This was done for both SERV and PicoRV32.  Energy from PrimeTime and from the XCelium simulations were found to be within 12\% of each other.
