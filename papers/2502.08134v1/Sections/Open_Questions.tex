\section{Open Questions}

\subsection{Balancing Diversity and Relevance in Pairs}
Diversity can be defined as the variation in data samples and pairs used in training while relevance is the level of semantic alignment between the pairs. Ideally, there should be high semantic alignment across pairs of the same category and low alignment across positive and negative categories.

Balancing diversity and relevance is a critical challenge. Both factors are essential: diversity ensures robustness and generalization, while semantic alignment guarantees meaningful and task-relevant representations. Diverse positive pairs may include instances that, while related, are semantically weakly aligned and hence irrelevant, leading to noisy training process and embeddings. Conversely, tightly aligned pairs might be more semantically aligned and relevant but miss significant variations and lack diversity.
Diversity is preferred when the downstream task is unknown, while relevance is prioritized when strong task alignment is needed. 
Future research can focus on dynamic training approaches---
Should early phases focus on introducing diversity to learn general features, followed by more task-aligned features? Or should specialized features be learned first from task-relevant data, and learning generalizable features relegated to later phases?
% with early phases focusing on relevance to learn core features, while later phases introducing diversity for better generalization. 
%Defining positives based on similarity scores rather than binary labels also allows partially aligned pairs to contribute proportionally, ensuring an optimal balance.

\subsection{Dealing with Emerging Modalities in Contrastive Learning}

As new modalities—such as LiDAR, hyperspectral imaging, and haptic feedback, become prominent in various applications, the challenge of integrating these modalities into contrastive learning frameworks emerges \cite{dai2024advancing}. 
Many emerging modalities suffer from the lack of large-scale labeled or even unlabeled datasets. They often exhibit high noise levels or variability due to environmental factors or inherent measurement inaccuracies. Moreover, unlike text or images, pre-trained models for emerging modalities are rare, making initialization more challenging. Future research can focus on scalable and efficient strategies to handle the diversity and complexity of these new data types.

% \clearpage
