\section{Discussion}

% \subsection{Drawback of Generating Positive Pairs from the Single Instance}

%\textit{Single-instance positive pairs}, generated through augmentations like random cropping or color distortion, have inherent limitations in capturing diverse variations such as different viewpoints, object deformations, and semantic similarities across instances within the same class. These methods place the burden of generalization on the data augmentation process, which may not encompass the full range of real-world data variations. Multi-instance positive pair techniques provide diverse alternatives to address these limitations.

\subsection{Trade-offs Between Techniques that Generate Positive Pairs from Multiple Instances}

Embedding-based positive pairs leverage semantic similarity in the embedding space to capture fine-grained variations, enabling models to learn intricate features such as fur color or ear shape when distinguishing dog breeds. However, this approach can be computationally intensive and prone to noisy embeddings, particularly in the early training stages.

Synthetic data generation dynamically creates challenging and diverse positive pairs. However, these models %that use synthetic data 
require careful tuning to prevent degradation from low-quality synthetic samples. A significant challenge is the visual fidelity of synthetic samples, as they might lack the richness and detail of real-world data, resulting in positive pairs that do not fully capture the desired semantic similarity required for a given downstream task. 
% Contextual discrepancies—such as differences in background, lighting, or object relationships, could further widen the domain gap, limiting the relevance of synthetic samples in contrastive learning. 
Another concern is semantic misalignment, where synthetic samples may inadvertently introduce artifacts or distortions that diverge from real-world semantics that can lead to representations that overfit to synthetic peculiarities. Addressing domain gap challenges requires novel strategies like combining synthetic positives with real data, using hybrid training approaches, or incorporating domain adaptation methods that can help bridge the gap.

Supervised pairing leverages label information to create positive pairs. This technique is particularly beneficial when labeled data is available, and class-specific clustering is essential for a downstream task. It is helpful for tasks such as class-specific retrieval, where the goal is to fetch semantically similar items within the same class, or tasks requiring discrimination between subtle intra-class variations. The biggest drawback is that this method assumes the availability of labels, which might not be feasible in every scenario.

Attribute-based pairing leverages contextual attributes and is effective in scenarios where domain-specific context plays a crucial role in learning robust representations. One challenge with this method is that attributes may be unevenly distributed across the dataset, leading to over-representing certain pair types. Another challenge is that models may overfit to specific attribute values instead of learning generalizable representations. %Lastly,

Cross-modal pair generation is particularly useful for multimodal learning, which is becoming increasingly prevalent. However, this method depends on the availability of semantically aligned multimodal pairs. Misaligned pairs can result in less meaningful representations. Another significant challenge is obtaining aligned data for modalities that can be paired for contrastive learning, mainly when one of the modalities is rare or difficult to generate labels for.

\subsection{Trade-offs Between Techniques that Generate Negative Pairs from Multiple Instances}

Hard negative selection is valuable for generating informative embeddings. However, overemphasizing hard negatives can lead to overfitting, where the model learns to differentiate subtle, unimportant variations rather than capturing meaningful representations.  While hard negatives provide valuable gradients, easy negatives ensure stability and prevent overfitting to challenging examples. Hence, creating the right mix of hard and easy negatives in a batch for learning is essential.

Eliminating false negatives helps reduce noise, but accurately defining them is challenging. Over-aggressive removal or reclassification as positives can reduce diversity and make negatives too easy, weakening contrastive learning. Conversely, being too conservative allows false negatives to persist, hindering learning. The key is to balance diversity and difficulty, ensuring negatives remain challenging (hardness) without incorrectly reclassifying true negatives as positives. 

Synthetic negatives offer a scalable approach for generating diverse negative pairs, but they face challenges similar to synthetic positive generation, including semantic misalignment and domain gaps. Additionally, maintaining a large pool of negatives or dynamically synthesizing new ones incurs significant computational costs, making efficiency a key consideration in their implementation. 
%Negative examples are typically defined and selected based on the specific domain and task. For instance, in multimodal contrastive learning, negatives must account for cross-modal discrepancies, such as the semantic gap between text and images. Similarly, in temporal tasks, negatives from non-overlapping time intervals may introduce unwanted biases. Therefore, tailoring negative selection strategies to domain characteristics is essential for optimal results.

