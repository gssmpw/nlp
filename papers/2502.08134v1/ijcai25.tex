%%%% ijcai25.tex

\typeout{IJCAI--25 Instructions for Authors}

% These are the instructions for authors for IJCAI-25.

\documentclass{article}
\pdfpagewidth=8.5in
\pdfpageheight=11in

% The file ijcai25.sty is a copy from ijcai22.sty
% The file ijcai22.sty is NOT the same as previous years'
\usepackage{ijcai25}

% Use the postscript times font!
\usepackage{times}
\usepackage{soul}
\usepackage{url}
\usepackage[hidelinks]{hyperref}
\usepackage[utf8]{inputenc}
\usepackage[small]{caption}
\usepackage{graphicx}
\usepackage{amsmath}
\usepackage{amsthm}
\usepackage{booktabs}
\usepackage{algorithm}
\usepackage{algorithmic}
\usepackage[switch]{lineno}
\usepackage{forest} % For hierarchical trees
\usepackage{xcolor} % For coloring
\usepackage{adjustbox}
% \usepackage{hyperref}
\usepackage{xcolor}

% Define clickable links with color
\newcommand{\citelink}[2]{\textcolor{blue}{\href{#2}{#1}}}
\newcommand{\dc}[1]{{\color{orange}{[Deep] \@#1}}}
% Comment out this line in the camera-ready submission
% \linenumbers

\urlstyle{same}

% the following package is optional:
%\usepackage{latexsym}

% See https://www.overleaf.com/learn/latex/theorems_and_proofs
% for a nice explanation of how to define new theorems, but keep
% in mind that the amsthm package is already included in this
% template and that you must *not* alter the styling.
\newtheorem{example}{Example}
\newtheorem{theorem}{Theorem}

% Following comment is from ijcai97-submit.tex:
% The preparation of these files was supported by Schlumberger Palo Alto
% Research, AT\&T Bell Laboratories, and Morgan Kaufmann Publishers.
% Shirley Jowell, of Morgan Kaufmann Publishers, and Peter F.
% Patel-Schneider, of AT\&T Bell Laboratories collaborated on their
% preparation.

% These instructions can be modified and used in other conferences as long
% as credit to the authors and supporting agencies is retained, this notice
% is not changed, and further modification or reuse is not restricted.
% Neither Shirley Jowell nor Peter F. Patel-Schneider can be listed as
% contacts for providing assistance without their prior permission.

% To use for other conferences, change references to files and the
% conference appropriate and use other authors, contacts, publishers, and
% organizations.
% Also change the deadline and address for returning papers and the length and
% page charge instructions.
% Put where the files are available in the appropriate places.


% PDF Info Is REQUIRED.

% Please leave this \pdfinfo block untouched both for the submission and
% Camera Ready Copy. Do not include Title and Author information in the pdfinfo section
\pdfinfo{
/TemplateVersion (IJCAI.2025.0)
}

\title{A Survey on Data Curation for Visual Contrastive Learning: \\ Why Crafting Effective Positive and Negative Pairs Matters}


% Single author syntax
% \author{
%     Author Name
%     \affiliations
%     Affiliation
%     \emails
%     email@example.com
% }

% Multiple author syntax (remove the single-author syntax above and the \iffalse ... \fi here)
% \iffalse
\author{
Shasvat Desai$^1$
\and
Debasmita Ghose$^2$\And
Deep Chakraborty$^{3}$
\affiliations
$^1$Independent Researcher\\
$^2$Yale University\\
$^3$University of Massachusetts Amherst\\
% $^4$Fourth Affiliation\\
\emails
shasvat.desai@gmail.com, 
debasmita.ghose@yale.edu,
dchakraborty@umass.edu
}
% \fi

\begin{document}

\maketitle

\begin{abstract}
Visual contrastive learning aims to learn representations by contrasting similar (positive) and dissimilar (negative) pairs of data samples. The design of these pairs significantly impacts representation quality, training efficiency, and computational cost. A well-curated set of pairs leads to stronger representations and faster convergence.
As contrastive pre-training sees wider adoption for solving downstream tasks, data curation becomes essential for optimizing its effectiveness. 
In this survey, we attempt to create a taxonomy of existing techniques for positive and negative pair curation in contrastive learning, and describe them in detail.
% This survey introduces the research being done in this direction and aims to create a taxonomy of existing techniques used in positive and negative pair curation for contrastive learning. 
We also examine the trade-offs and open research questions in data curation for contrastive learning.
\end{abstract}

% 
% 
The widespread integration of communication networks and smart devices in modern control systems has increased the vulnerability of industrial systems to online cyber-attacks, e.g., Industroyer, Blackenergy, etc \citep{osti_1505628}.
% Modern control systems have seen a large push to include communication networks and smart devices to increase performance, made possible by improvements in communication device cost and energy consumption. This trend has been coupled with the usage of open-standard communication protocols among industrial control systems, making them vulnerable to online cyber-attacks such as Industroyer, Blackenergy, etc \citep{osti_1505628}. 
To counter this, methods have been developed to improve security by achieving attack detection, mitigation, and monitoring, among others \citep{sandberg2022secure}. This paper focuses on active attack diagnosis to mitigate stealthy attacks. 
%
%\subsection{Literature review}

Active diagnosis techniques rely on the inclusion of additional moduli to control systems
% inclusion within the control system of additional moduli 
to alter the behavior of the system compared to information known by the attacker. 
For instance, the concept of additive watermarking was introduced in \cite{mo2015physical}, where noise signals of known mean and variance are added at the plant and compensated for it at the controller. 
This compensation, however, is not exact, causing some performance degradation. Thus, trade-offs between performance and detectability  are necessary \citep{zhu2023detection}.
% A later work \citep{zhu2023detection} designs the watermark signal by trading performance for detection. Thus, although additive watermarking serves as a good detection scheme, they endure performance losses even in the nominal case. 

In encrypted control \citep{darup2021encrypted}, the sensor data is encrypted, sent to the controller, and then operated on directly. Encrypted input signals are sent back to the plant for decryption. Although encryption is widespread in IT security, in control systems it presents some concerns, such as the introduction of time delays \citep{stabile2024verifiable}, while it may present inherent weaknesses \citep{alisic2023model}.
% they are not preferred as they introduce time delays \citep{stabile2024verifiable} which can cause instability, and some encryption schemes can be very weak  \citep{alisic2023model}. 

In moving target defense \citep{griffioen2020moving}, the plant is augmented with fictitious dynamics, known to the controller. The plant output is transmitted to the controller along with the fictitious states over a network under attack. 
The additional measurements then aide in the detection of attacks. 
This comes at the cost of higher communication bandwidth needs, which increases rapidly with the dimension of the augmented systems.
% Since the dynamics of the fictitious dynamics are exactly known to the controller, the attack is detected easily. However, when the scale of the system increases, the communication bandwidth used by moving the target defense approach increases rapidly. 

Other recently proposed works include two-way coding \citep{fang2019two}, a weak encryuption technique, and dynamic masking \citep{abdalmoaty2023privacy}, which enhances privacy as well as security, have been shown to be effective against zero-dynamics attacks.
% Two-way coding \citep{fang2019two} and dynamic masking \citep{abdalmoaty2023privacy} are other recently proposed approaches. Two-way coding is another form of weak encryption technique whilst dynamic masking proposes an architecture that enhances both privacy and security. These schemes are shown to be effective against zero dynamics attacks but remain to be studied for other classes of attacks. 
% Recent extensions include \citep{mukherjee2021secure,ramos2024privacy}.
% Some other works which are related are \citep{mukherjee2021secure}, an extension of \cite{fang2019two}. The work \citep{ramos2024privacy} is an extension of moving target defense for multi-agent systems. 
Furthermore, filtering techniques for attack detection are proposed by \cite{murguia2020security,hashemi2022codesign,escudero2023safety}, while not focusing on stealthy attacks.
% The works \citep{murguia2020security,hashemi2022codesign,escudero2023safety} develop filtering techniques to guarantee safety, without being focused on stealthy covert attacks.

Multiplicative watermarking (mWM) has been proposed by the authors as a diagnosis technique \citep{ferrari2020switching}. mWM consists of a pair of filters on each communication channel between the plant and its controller; the scheme is affine to weak encryption, whereby ``encoding'' and ``decoding'' are done by changing signals' dynamic characteristics through inverse pairs of filters. This enables original signals to be recovered exactly, and thus does not lead to performance degradation.
% A multiplicative watermark is an affine to a weak encryption technique, through which the signal is ``encoded'' by a filter, changing its dynamic behavior. The use of inverse pairs means that the original signal can be recovered, through ``decoding'' via an inverse filter. As such, differently to techniques based on additive watermarking, no performance is lost due to the injection of noise, and there are no bandwidth limitations.

%\subsection{Contributions}
One of the critical features of multiplicative watermarking is that to detect stealthy attacks, the mWM filter parameters must be switched over time. In this paper, an algorithm to optimally design the mWM parameters after a switching event is presented, enhancing detection performance, without changing the switching time.
% This is done without changing the switching time, which is taken as given.

\textcolor{black}{
To formalize the filter design problem, we suppose the defender is interested in optimal performance against adversaries injecting covert attacks with matched system parameters \citep{smith2015covert}, including the mWM parameters prior to the switch. This scenario represents a worst case where malicious agents can take full control of the system while remaining undetected.
Thus, the attack strategy is explicitly included within the formulation of the closed-loop system, and the mWM filters are chosen by solving an optimization problem minimizing the attack-energy-constrained output-to-output gain (AEC-OOG) \citep{anand2023risk}, a variation of the output-to-output gain proposed in  \cite{teixeira2015strategic}.
}
The main contributions of this paper are:
% We consider an adversary injecting a covert attack with matched system parameters \citep{smith2015covert}, i.e., an attacker with full knowledge of the control system parameters, including those of the mWM filters before the switch. This scenario is taken as a worst case, as it has been shown that this class of attacks can be made stealthy. To quantitatively define a cost, the output-to-output gain (OOG) \citep{teixeira2015strategic} is leveraged,
% a metric introduced to evaluate the impact of an additive attack in a control system. %Specifically, OOG evaluates the worst-case performance loss that an attacker injecting an undetectable attack can obtain. 
% Here, the maximum performance loss caused by a stealthy adversary with limited energy is taken, the attack-energy-constrained OOG (AEC-OOG) \citep{anand2023risk}. The main contributions of this paper are:
\begin{enumerate}
%[label=\alph*.]
\item The problem of optimally designing the switching mWM filters is formulated as an optimization problem, with the AEC-OOG is taken as the objective;%where the AEC-OOG is taken as the impact metric; 
\item The worst-case scenario of a covert attack with exact knowledge of plant and mWM filter parameters is embedded within the design problem;
% The optimization problem is defined to incorporate the worst-case scenario of a covert attack with exact knowledge of plant and mWM filter parameters;
\item The feasibility of the optimization problem is shown to be dependent only on stability conditions; 
\item A solution scheme is proposed to promote randomization of the mWM filter parameters such that an eavesdropping adversary cannot remain stealthy.
\end{enumerate} 

This builds on the results of \cite{ferrari2020switching}, where the focus was on the design of the switching protocols, rather than the parameters themselves.
Compared to previous work \citep{gallo2021design}, this paper introduces an optimization problem which is always feasible (thanks to the use of AEC-OOG in the objective), while also considering a more sophisticated class of covert attacks, where the presence of watermark is known to the adversary. 
Moreover, this paper poses a different objective than \citep{zhang2023hybrid}; indeed, while \citep{zhang2023hybrid} provided a design strategy to ensure certain privacy properties, in this paper we address the problem of optimal parameter design following a switching event.


%\subsection{Organization}
The rest of the paper is organized as follows. 
After formulating the problem in Section~\ref{sec:PF}, we propose our design algorithm in Section~\ref{sec:main}, and analyze its properties. It is then evaluated through a numerical example in Section~\ref{sec:NE}, and concluding remarks are given Section~\ref{sec:Con}.
% We provide the problem background in Section~\ref{sec:PF}. We formulate the design problem in Section~\ref{sec:main}, together with an analysis of its properties. The proposed algorithm is evaluated through a numerical example in Section \ref{sec:NE}. Concluding remarks are offered in Section \ref{sec:Con}.
\section{Mobile Networks Powered by \glspl{LLM}}
\label{sec:LLM_enabled_MNs}
\begin{figure*}[t!]
\centering
\includegraphics[width=.99\textwidth]{Fig1.eps}
    \caption{Possible architectural designs for integrated \gls{LLM} and \gls{MNO} ecosystem.}
    \label{fig:LLM_possible_architectures}
\end{figure*}
The historical data of the \gls{MNO}, archived over years of expertise, constitutes a solid foundation for training the \gls{LLM} using structured and unstructured multi-modal inputs (as illustrated in Fig.~\ref{fig:LLM_possible_architectures}a) such as user intents, network logs, alarm descriptions, trouble tickets, \gls{PCAP} files (e.g. from Wireshark or tcpdump), dashboard screenshots, audio recordings (e.g. from \gls{IVR} systems), video feeds (e.g. from infrastructure surveillance), and \gls{NWDAF} analytics. To this end, a separate collection framework aggregates data from various sources into a centralized repository, and extracts most informative features such as warnings, error codes, timestamps, and user/gNB/session/bearer/\gls{QoS} flow/slice IDs. The extracted features are then converted into unified embeddings that are combined into a common vector space with suitable metadata (e.g. to differentiate data formats). The resulting vector store is used to fine-tune the \gls{LLM} to deeply internalize \gls{MNO}-specific knowledge \cite{Bariah2023understanding}. This allows the \gls{LLM} to learn patterns, sequences, and deviations that correlate with normal or faulty network operations. This is made possible using a timestamp-based cross-referencing to link different entries from several data sources, allowing detailed description and context for each flagged event as well as the resolution workflow for the spotted anomalies.

In live mobile networks, fresh multi-modal data is continuously fed into the \gls{LLM}, either uploaded in batches or streamed in real-time. The \gls{LLM} analyzes this data and identifies potential anomalous behaviors in light of its accumulated learning. In case of new anomalies not covered during the fine-tuning stage, the \gls{LLM} can rely on clustering techniques to group similar patterns and flag outliers as suspected behaviors. The \gls{LLM} is also capable of using \gls{RAG}-enabled external knowledge databases such as \gls{3GPP} documents \cite{Said2024instruct}, \gls{IEEE} standards, \gls{IETF} RFCs and vendors documentation \cite{soman2023observations} to compare the actual network behavior with the expected one to identify misconfigurations and spot unusual trends in protocols and communication flows. Well-crafted prompts, on the other hand, can guide the \gls{LLM} responses to provide focused solutions. Paradigms such as the \gls{CoT} reasoning can be used to break down the \gls{LLM} insights into a series of simplified and actionable sub-tasks. It can be extended by the \gls{ToT} technique to explore different reasoning paths and identify the most optimal solution. The \gls{LLM} can naturally produce stepwise reasoning if datasets used for fine-tuning contain \gls{CoT} and \gls{ToT} examples, or through creative prompting \cite{Zhou2024survey}. In parallel, \gls{NOC} engineers can intervene to confirm, guide or reject the \gls{LLM} findings, if needed, e.g. using its intuitive conversational interface. Through continuous self-learning, the \gls{LLM} will dynamically adapt to evolving network conditions, optimizing its performance over time \cite{Chaparadza2023optimization}.

%For instance, when a network experiences latency issues, the \gls{LLM} seamlessly analyze multi-modal information from diverse origins to identify the root cause, e.g. overloaded \gls{UPF} due to insufficient capacity, and then suggest a solution, e.g. step-by-step instructions including suitable code scripts for the involved \glspl{NF} to autonomously reroute traffic or modify policies. Conventional 5G networks can only alert about anomalies using suitable \gls{NWDAF} analytics that track the violated thresholds and notify the \gls{OAM} center to display the details on complex dashboards.

By incorporating \glspl{LLM} (e.g. as \glspl{NF}) into upcoming 6G networks, expected to be designed with \gls{SbD} principles \cite{Khaloopour2024Resilience}, \glspl{LLM} will naturally inherit the same built-in security safeguards rather than adding them as an afterthought. This design-driven approach focuses on proactive threat management, end-to-end encryption, authentication, network slicing isolation, \gls{AI}-driven threat detection with automated reactions, and stateless designs, fostering a resilient \gls{LLM}.
%The design-driven security in 5G and upcoming 6G networks ensures that security is natively integrated at every layer of the architecture rather than added as an afterthought. This approach focuses on proactive threat management, end-to-end encryption, authentication, network slicing, and \gls{AI}-driven threat detection and automated reactions to counter evolving cyber threats.



\section{Taxonomy of Approaches for Positive and Negative Pair Curation}
\label{sec: Taxonomy}



In contrastive learning, a commonly used loss function is the InfoNCE loss defined below. It pulls similar (positive) pairs together while pushing dissimilar (negative) pairs apart in the embedding space and computes the similarity between an anchor and its positive counterpart, using a softmax over similarity scores. %The InfoNCE loss is defined as follows:


\begin{math}
\hspace{-1em}
\centering
 \mathcal{L} = 
- \frac{1}{N} \sum_{i=1}^{N} \log 
\frac{
\exp\left(\text{sim}\left(\mathbf{z}_i, \mathbf{z}_i^+\right)/\tau\right)
}{
\exp\left(\text{sim}\left(\mathbf{z}_i, \mathbf{z}_i^+\right)/\tau\right) 
+ \sum_{j=1}^{N} 
\exp\left(\text{sim}\left(\mathbf{z}_i, \mathbf{z}_j\right)/\tau\right)}
\end{math}


where \( \mathbf{z}_i \) is the representation of the anchor sample and \( \mathbf{z}_i^+ \) is the representation of the positive sample obtained through augmenting the same instance or using a criterion to select another instance. \( \mathbf{z}_j \) represents all samples in the batch (including negatives). \( \text{sim}(\cdot, \cdot) \) denotes the similarity function (commonly cosine similarity). \( \tau \) is the temperature scaling parameter and \( N \) is the number of samples in the batch.


\subsection{Positive Pair Creation Taxonomy}
The taxonomy of Positive Pair Creation can be categorized into two main groups: \textbf{single-instance positives} and \textbf{multi-instance positives}, as shown in Fig. 1.

%\subsubsection{Single-instance Positives}

% Single-instance positives can be grouped into a single family of technique, as shown in Figure 1. 
Single-instance positive pair creation generates pairs by applying augmentations (e.g., cropping, color changes, geometric transformations) to a single sample \cite{chen2020simple}.  However, this approach limits diversity, as random augmentations fail to capture viewpoint changes, object deformations, or semantically similar instances within the same class. As a result, the model's generalization depends heavily on the augmentation strategy, which may not fully capture the intrinsic variations needed for learning robust embeddings.


% \subsubsection{Multi-instance Positives}

To overcome the limitations of single-instance pairs, multi-instance positive pair curation creates pairs from different data samples rather than augmented views of the same sample, leading to greater diversity \cite{dwibedi2021little}. As shown in Fig. 1, multi-instance curation techniques include: (1) Embedding-based, which selects semantically similar instances in embedding space; (2) Synthetic, which generates positive pairs using generative models; (3) Supervised, which uses human or oracle-labeled data; (4) Attribute-based, which groups samples based on spatial, temporal, or other object-based attributes.  and (5) Cross-modal, which associates samples across different modalities.  By ensuring higher diversity and semantic alignment, multi-instance positive pairs improve representation learning and align embeddings more effectively with downstream tasks.

%comment from here
% \noindent \textbf{Embedding-based similarity} identifies semantically similar samples in the embedding space to form positive pairs. For example, Nearest-Neighbour Contrastive Learning of visual Representations (NNCLR) \cite{dwibedi2021little} retrieves the nearest neighbor of a sample as its positive pair. Similarly, MSF \cite{koohpayegani2021mean} proposes to use the first and $k$ nearest neighbors as the multiple instance positives. All4One \cite{estepa2023all4one} improves MSF by incorporating a centroid contrastive objective to learn contextual information from multiple neighbors using a transformer network. \textbf{Synthetic data generation} employs generative models such as generative adversarial networks (GAN) \cite{wu2023synthetic} or diffusion models \cite{zeng2024contrastive} to create synthetic data points that are semantically similar but distinct from the original, serving as positive pairs. \cite{wu2023synthetic} is jointly trained with the main model to dynamically customize hard samples based on the training state of the main model. \cite{zeng2024contrastive} replaces the features of the intermediate layers in the diffusion model with the semantic features extracted from an anchor image during a random reversed diffusion process. This results in the generation of images possessing similar semantic content to the anchor image but differing in background and context due to the randomness of features in other layers. \textbf{Supervised pairing} utilizes label information to create positive pairs from samples of the same class, as seen in Supervised Contrastive Learning (SupCon) \cite{khosla2020supervised}. Another method \cite{ghose2023tailoring} proposes to create pairs and train the model in an online manner by using human-guided feedback. \cite{wang2022oracle} incorporates human or oracle feedback for a subset of samples to extend the set of positive instance pairs. \textbf{Attribute-based pairing} leverages specific attributes such as spatial location or temporal proximity to form positive pairs. For instance, geographically aligned images captured at different times can be paired \cite{ayush2021geography}. The attributes used to generate optimal views for contrastive representation learning are task-dependent. \textbf{Cross-modal positives} align samples across different modalities (e.g., images and text, audio, speech) that correspond to the same semantic content \cite{radford2021learning}, \cite{wang2022image}, \cite{baevski2020wav2vec}, \cite{li2020unimo}, \cite{morgado2021audio}.
% %comment till here

\subsection{Negative Pair Creation Taxonomy}

In typical contrastive learning approaches, negative pairs are created from samples not used to create the positive pair without considering their semantic content. However, recent work \cite{huynh2022boosting} suggests that uncurated negatives may lead to false negatives, where semantically similar samples are incorrectly treated as negatives. An effective negative sample selection strategy should balance easy and hard negatives while maintaining representativeness. Based on these principles, negative pair curation can be categorized into three main approaches (Fig. 1): (1) Hard Negative Selection, which prioritizes difficult negatives close to the anchor in embedding space; (2) False Negative Elimination, which removes or reclassifies semantically similar false negatives; and (3) Synthetic Negatives, where generative models create diverse, controlled negative samples. There is a subtle trade-off between (1) and (2). Hard negatives improve discrimination but risk overfitting, while false negative elimination reduces noise but may mistakenly remove challenging yet valid negatives, weakening the representations.

% \noindent \textbf{Hard negative selection} involves identifying samples that are particularly challenging for the model to distinguish. \cite{Hardnegativemixing} extends the MoCo v2 framework \cite{chen2020improved} by adding two sets of hard and harder negatives into the queue. The first set is a convex linear combination of pairs of its hardest existing negatives, whereas the second set is created by mixing the negatives with the query. \cite{unremix} introduces UnReMix, a method designed to enhance contrastive learning by selecting hard negative samples based on three key factors: anchor similarity, model uncertainty, and representativeness, ensuring that negative samples are similar to the anchor point, making them challenging for the model to distinguish. \textbf{Removal of false negatives} addresses negative pairs from the same semantic category. \cite{huynh2022boosting} introduces methods to identify these false negatives and propose two strategies to mitigate their impact: elimination and attraction. False Negative Elimination identifies potential false negatives and excludes them from the negative sample set, preventing the model from learning misleading distinctions. In False Negative Attraction, instead of excluding false negatives, this strategy reclassifies them as positives, encouraging the model to learn representations that acknowledge their semantic similarity. \textbf{Synthetic hard negatives} can be created using various techniques, including generative models, feature space interpolation, or rule-based algorithms that modify existing data. \cite{dong2024synthetic} proposes an approach that involves mixing existing negative samples in the feature space to create more challenging negatives, encouraging the model to learn more discriminative representations. It proposes a novel feature-level sampling method to generate more and harder negative samples by mixing them through linear combination and ensuring their reliability by debiasing.


% Next, we dive into details of the most commonly used techniques for crafting effective positive and negative pairs.







\section{Crafting Effective Positive Pairs
}
\label{sec: Positive_Pair}

\begin{figure*}[h!]
    \centering
    \includegraphics[width=0.95\linewidth]{Figures/positive_pairs.pdf}
    \caption{\textbf{Positive Pair Curation Techniques: } Positive pair selection can utilize single-instance and multi-instance techniques. (a) Single-instance curation applies augmentations to a single sample. On the other hand, multi-instance positive pair generation can be further classified into several category of techniques. (b) Embedding-based retrieves the top-K nearest neighbors of the anchor sample's augmentation in the embedding space and pairs them with other augmentations of the anchor. (c) Synthetic pairs generate data conditioned on the input, which is then augmented and paired with the augmented real sample (d) Supervised pairs use external sources (human labels, oracles, or annotations) to fetch another sample from the same category and create positive pairs. (e) Attributed-based: These methods group samples by shared attributes (e.g., golden retrievers paired with golden labrador retrievers based on fur color) and pair their respective augmentations. (f) Cross-modal: This involves creating semantically aligned pairs across multiple modalities. The figure shows image-text and speech-image pairing.}
    \label{fig:positive}
\end{figure*}

\subsection{Single Instance Positives}

This technique creates positive pairs using augmentations of a single sample without explicit curation, as shown in Fig. 2(a).  The negative pairs are also randomly sampled from the dataset and are uncurated. %Data augmentation plays a crucial role by introducing various transformations to the original data, prompting the learning algorithm to identify consistent underlying patterns.

A common contrastive learning technique using this type of data curation is \textbf{SimCLR}\cite{chen2020simple}. SimCLR maximizes agreement between augmented views of the same data point using the InfoNCE loss, relying on large batch sizes to sample enough negatives. \textbf{MoCo} \cite{he2020momentum} addresses a drawback of SimCLR, which requires large negative samples that can be computationally expensive by using a momentum encoder and memory bank to maintain a queue of negatives dynamically. \textbf{SimSiam} \cite{chen2021exploring} eliminates the need for negative examples, using a stop-gradient mechanism to prevent representation collapse in its Siamese architecture. \textbf{BYOL} \cite{grill2020bootstrap} simplifies learning by aligning predictions from an online network with a momentum-maintained target network, achieving strong performance without negatives. \textbf{Barlow Twins} \cite{zbontar2021barlow} focuses on redundancy reduction by aligning embeddings and decorrelating feature dimensions, avoiding collapse naturally without negatives or momentum. \textbf{DINO} \cite{caron2021emerging} combines self-supervised learning with knowledge distillation using a teacher-student framework, producing generalized embeddings. \textbf{VicReg} \cite{bardes2021vicreg} introduces regularization to balance variance, invariance, and decorrelation in embeddings, ensuring quality without negatives or momentum encoders. \textbf{SwAV} \cite{caron2020unsupervised} uses clustering to align augmentations by mapping them to shared cluster assignments without direct contrastive loss. \textbf{CPC} \cite{oord2018representation} leverages contrastive loss in a latent space to predict future data segments, making it particularly effective for time-series tasks. Finally, \textbf{SEED} \cite{fang2021seed} simplifies training by using teacher-student distillation with pseudo-labels, reducing computational complexity. %while maintaining strong performance.





\subsection{Multi Instance Positives}
Multi-instance positive pair curation creates pairs from different samples rather than augmented views of the same sample.

\subsubsection{Embedding-Based Techniques}
Given an input candidate sample, this class of techniques first retrieves the K-nearest neighbors of one of the augmentations using a similarity metric in embedding space, as shown in Figure 2(b). Next, it uses the K-retrieved samples and the other augmentations of the candidate sample as positive pairs. 

Nearest-Neighbour Contrastive Learning of Visual Representations (NNCLR)\cite{dwibedi2021little} samples the nearest neighbors from the dataset in the latent space and treats them as positives. This provides
more semantic and intra-class variations to learn representations that are invariant to different viewpoints, deformations, and variations. The NNCLR framework relies entirely on a single nearest neighbor, limiting its potential. Mean Shift for Self-Supervised Learning (MSF) \cite{koohpayegani2021mean} addresses this limitation by proposing the use of \(k\) nearest neighbors to increase the diversity in the positive pairs. MSF shifts the embedding of each image to be closer to the \textit{mean} of the neighbors of its augmentation.
However, MSF is computationally expensive because the objective function must be computed \(k\) times for each neighbor. To address MSF's computational inefficiency, All4One \cite{estepa2023all4one} contrasts information from multiple neighbors by compiling information from the extracted \(k\) neighbors to create a pair of representations, called \textit{centroids}, which contain contextual information about all the neighbors. 

These techniques can be used when semantic clustering is needed for downstream applications. For instance, if the downstream task involves clustering similar faces, these techniques allow different views of the same person to be closer together, unlike single-instance positive techniques, which treat all other images as negatives.

\subsubsection{Synthetic Data Generation for Positive Pairs}

This class of techniques creates synthetic samples using a generative process conditioned on the candidate input sample. A positive pair is formed by combining the augmented generated sample with the augmentation of the original input sample, which is then processed by the encoder, as illustrated in Fig. 2(c).

Contrastive Learning with Synthetic Positives (CLSP) \cite{zeng2024contrastive} incorporates synthetic positives generated via a diffusion model. By interpolating Gaussian noise with diffusion-based features, CLSP creates images that resemble the anchor while varying the context and background, increasing diversity while preserving semantic meaning. Similarly, \cite{wu2023synthetic} introduces a GAN-based framework that dynamically generates hard positive pairs by jointly optimizing the GAN and contrastive model. However, this simultaneous training introduces instability and quality control challenges. These approaches are particularly useful in data-sparse scenarios, rare modalities, or domains where obtaining real data is challenging, such as cross-modal medical applications (e.g., speech-image pairs).



\subsubsection{Supervised Pairing Techniques}

These techniques use external data sources, such as human preferences, privileged information from an oracle, or an annotated dataset, to derive meaningful metadata and semantic categories and create positive pairs, as shown in Fig. 2(d).


Supervised contrastive learning (SupCon) \cite{khosla2020supervised} leverages ground truth labels to enhance representation learning by incorporating category-level supervision. Instead of defining positive pairs through augmentations of a single instance, SupCon creates positive pairs from multiple samples of the same category as the anchor, ensuring that representations capture category-level semantic similarities rather than just instance-specific features.

Building upon this, \cite{ghose2023tailoring} propose a method to create positive pairs on the fly by passively observing humans provide limited positive examples while working collaboratively with a robot without explicitly marking negatives. This aligns with Positive-Unlabeled (PU) Learning \cite{bekker2020learning}, where only positives are known, and the model infers meaningful distinctions. Contrastive learning then clusters these examples, ensuring representations align with human expectations in a task-adaptive manner.
%Contrastive learning is then applied to cluster human-selected examples, ensuring that representations align with human expectations in a task-adaptive manner.  
Similarly, Oracle-guided Contrastive Clustering (OCC) \cite{wang2022oracle} uses a deep clustering framework designed to create positive pairs for contrastive loss by incorporating oracle feedback into the clustering process, ensuring that learned representations align with user-specific clustering preferences. Instead of relying purely on instance similarity in the embedding space, OCC actively queries an oracle (human or predefined rule) to determine whether two samples should belong to the same cluster.

These techniques are useful in scenarios when labeled data is available, and the goal is to cluster semantically similar items within the same class or the downstream task requires discrimination between subtle intra-class variations. Leveraging labels and semantic information to create pairs enables us to generate embeddings that are better aligned to the downstream application. Intuitively, these techniques should eliminate false negatives (through semantic clustering) and potentially reduce noise in the generated embeddings. 

\subsubsection{Attribute-based Pairing Techniques}

Attribute-based pairing entails selecting positive pairs based on task-specific criteria, as shown in Fig. 2(e). Attributes can be generic, such as \textit{``a golden colored object"} or more specific such as \textit{``dog with golden colored coat"}

For instance, Geography-aware self-supervised learning \cite{ayush2021geography} leverages spatial and temporal attributes to create temporal positive pairs from images of the same geographical location taken at different times. They demonstrate their approach in the Remote sensing domain because it is easy to obtain multiple geo-located images of the exact location over time. Similarly, \cite{pantazis2021focus} leverages the natural variations in sequential images from static cameras, utilizing contextual information such as spatial and temporal relationships to identify high-probability positive pairs—images likely depicting the same visual concept. Yet another approach proposed by \cite{panambur2022self} leveraged domain-specific attributes like geological structures, terrain textures, and spatial and scientific properties to form positive pairs for terrain categorization in Martian terrain.

These techniques are useful when domain-specific contextual attributes are known. However, uneven attribute distribution can lead to over-representation of certain pair types and hinder performance on unseen variations, so careful attribute selection and balancing are essential.

\subsubsection{Cross-modal Positive Pairing Techniques}
Cross-modal contrastive learning aims to learn meaningful representations across different data modalities  to improve performance in various tasks that involve multi-modal data,
as shown in Fig. 2(f). 

\noindent
\textbf{Image-Text Pairing: }
Image-text pairing aims to align visual and textual information to learn effective representations. 
CLIP \cite{radford2021learning}, CLOOB \cite{furst2022cloob}, ALIGN \cite{jia2021scaling} employs contrastive learning to learn shared representations by aligning visual and textual data which facilitate downstream tasks like zero-shot image classification and cross-modal retrieval. BEiT-3 \cite{wang2022image}, VisualBERT \cite{li1908visualbert}, FLAVA \cite{singh2022flava}, LXMERT \cite{tan2019lxmert} are a class of methods that introduce a unified masked data modeling objective. Given a partially masked caption, the objective is to predict the masked words based on the corresponding image. They learn representations that capture the relationships between images and texts by masking parts of the input and training the model to predict the missing information. MAPL \cite{manas2022mapl} and Flamingo \cite{alayrac2022flamingo} keep the pre-trained vision encoder and language model frozen to learn a lightweight mapping between their representation spaces, enabling few-shot learning with minimal parameter updates.


\noindent
\textbf{Audio-Image-Text Pairing: }
AudioCLIP \cite{guzhov2022audioclip}, Wav2CLIP \cite{wu2022wav2clip} learns audio representations by distilling knowledge from the CLIP model to jointly learn a shared representation of audio data alongside image and text modalities. CLAP \cite{wu2023large} trains a dual-encoder model to align audio and text embeddings. %within a shared space.

\begin{figure*}[h!]
    \centering
    \includegraphics[width=0.97\linewidth]{Figures/negative_pairs.pdf}
    \caption{\textbf{Negative Pair Curation Techniques}: 
    % Most positive pair creation techniques rely on uncurated negative data and do not fully utilize the available data's potential. 
    This figure shows three categories of techniques for negative pair curation. (a). Hard Negative Selection prioritizes negatives that are semantically similar to the anchor sample, such as a different cat breed, instead of an unrelated category like an airplane. The negatives are then augmented and fed into the encoder. (b). False Negative Elimination removes or reclassifies negatives that are highly similar to the anchor sample, preventing the model from mistakenly separating highly similar samples. The remaining negatives are then augmented before encoding. Hard negatives improve discrimination but risk overfitting, while false negative elimination reduces noise but may mistakenly remove challenging yet valid negatives, weakening the representations. (c). Synthetic negative pairs are created by feeding the positive and negative samples(dataset) into a generative process and conditioned on the anchor sample to create realistic but distinct negatives. The generated samples then undergo augmentation and are fed with the positive pairs to the downstream encoder.}
    \label{fig:negative}
\end{figure*}

% \subsubsection{Speech-Audio-Text Pairing}

% Wav2Vec \\cite{baevski2020wav2vec} represents a category of models designed for self-supervised representation learning from raw audio waveforms. The wav2vec model is trained by predicting speech units for masked parts of speech audio. CLAPSpeech \cite{ye2023clapspeech} introduces a framework that explicitly learns prosody variations of text tokens under different contexts using cross-modal contrastive pre-training. By aligning text context with corresponding prosody patterns in a joint multi-modal space and employing a multi-scale pre-training pipeline, CLAPSpeech enhances prosody prediction in text-to-speech systems.

\noindent
\textbf{Audio-Visual Pairing: }
Audio-Visual Instance Discrimination (AVID) \cite{morgado2021audio} and \cite{alwassel2020self} emphasize cross-modal discrimination, aiming to align audio and visual features effectively. They use cross-modal clustering, where shared semantic clusters are learned across modalities by mapping audio and video representations into a joint embedding space. Building upon this, \cite{yariv2023audiotoken} adapts pre-trained text-conditioned diffusion models, like Stable Diffusion, by converting audio inputs into text-like embeddings through a learnable adapter. These embeddings serve as prompts for the diffusion model to generate audio-aligned images.

\section{Crafting Effective Negative Pairs}
\label{sec: Negative_Pair}

\subsection{Hard Negative Selection}

Hard negatives are those negative samples that are particularly similar to the anchor (the positive sample) in the embedding space, making them more likely to be misclassified. By incorporating such samples, the model is forced to refine its representation, learning more discriminative features to distinguish between fine-grained differences. These hard negatives are then fed into the encoder along with the positive pair, as shown in Fig. 3(a).

MoCHi, \textit{(M)ixing (o)f (C)ontrastive (H)ard negat(i)ves} \cite{Hardnegativemixing}, creates hard negatives by combining features of existing hard negatives in the embedding space. It identifies existing negatives for a given anchor that are most similar to the candidate positive sample in the embedding space and combines these hard negatives at the feature level to create synthetic negatives that are even closer to the anchor, increasing the difficulty of the contrastive task.

Uncertainty and Representativeness Mixing (UnReMix) \cite{unremix} selects negative samples based on three key properties. Anchor similarity ensures that negative samples closely resemble the anchor but belong to different classes. Model uncertainty prioritizes negative samples with higher prediction uncertainty, focusing the learning process on less confident regions of the data space. Representativeness emphasizes selecting negatives that reflect the overall data distribution rather than outliers. Similarly, \cite{robinson2020contrastive} samples negatives close to the anchor in the embedding space.  These negatives are generated adversarially or synthesized through feature interpolation, promoting fine-grained learning. A balanced mix of hard and easy negatives ensures stability and prevents overfitting during training.

Yet another approach \cite{hu2021adco} follows a min-max optimization framework, where the encoder minimizes the contrastive loss by learning to separate positives from negatives while the negative adversaries maximize the loss by generating challenging and indistinguishable negatives.





\subsection{Removal of False Negatives}
False negatives are samples from different images with the same semantic content, therefore they should hold certain similarity. Contrasting false negatives induces two critical issues in representation learning: 1) discarding semantic information and 2) slow convergence due to the addition of noise in the learning process. For instance, a cat's head in one image might be attracted to its fur (positive pair) but repelled from the similar fur in another image of a cat (negative pair), creating conflicting objectives. Eliminating false negatives involves taking a batch of negative samples and removing those highly similar to positives, as shown in Figure 3(b). The rest of the samples in the batch undergo augmentations and are sent to the encoder along with the positive pairs.

\cite{huynh2022boosting} introduces methods to identify these false negatives and proposes two strategies to mitigate their impact: elimination and attraction. Elimination identifies and excludes potential false negatives from the negative sample set, preventing the model from learning misleading distinctions. In contrast, false negative attraction reclassifies them as positives(makes them true positives), encouraging the model to learn representations that acknowledge their semantic similarity. Similarly, \cite{chen2021incremental} dynamically detects false negatives based on semantic similarity and reclassifies them as positives, thus reducing noise in the learning process. 

\cite{chuang2020debiased} takes a different approach to mitigate the impact of false negatives in contrastive learning by introducing a re-weighted loss function. This loss adjusts the contribution of each negative sample based on its likelihood of being a true negative without requiring label information. The approach improves representation learning by minimizing the influence of false negatives, achieving better performance in self-supervised settings across various domains.
These techniques help ensure the negative pairs are relevant and the generated embeddings are aligned to the downstream task.


\subsection{Synthetic Hard Negatives}
Synthetic negatives can be created using various techniques, including generative models, feature space interpolation, or rule-based algorithms that modify existing data. Once created, their augmentation and positive pairs are sent to the encoder, as shown in Fig. 3(c).

\textit{Synthetic Hard Negative Samples for Contrastive Learning} \cite{dong2024synthetic} involves mixing existing negative samples in the feature space to create more challenging negatives synthetically. To address the issue of false negatives—samples incorrectly labeled as negative but semantically similar to the anchor, this work incorporates a debiasing mechanism, ensuring the model focuses on truly dissimilar negative samples. 
%For a given anchor sample, the method selects the hardest negative samples from the available negative set based on their similarity to the anchor. 
The selected hard negatives are then combined through linear interpolation to create synthetic negative samples that are even closer to the anchor in the feature space. 
%This ensures that the generated negatives are challenging enough to improve the model's discriminative ability.

Similarly, another approach \cite{giakoumoglou2024synco} builds upon the MoCo framework \cite{he2020momentum} to create diverse synthetic hard negatives on the fly with minimal computational overhead. It generates negatives by interpolating between positive and negative samples in the feature space, extrapolating beyond the positive sample in the direction of a negative sample, applying small perturbations to positive samples to generate negatives, and using adversarial methods to craft indistinguishable negatives.
\section{Discussion and Future Work}\label{sec:discussion}
This paper pioneers the novel approach of selective response, showing that withholding responses can be a powerful tool for GenAI systems. By opting not to answer every query as accurately as it can---particularly when new or complex topics emerge---GenAI can encourage user participation on community-driven platforms and thereby generate more high-quality data for future training. This mechanism ultimately enhances GenAI's long-term performance and revenue. From a welfare perspective, our results indicate that such selective engagement can also benefit users, leading to better solutions and increased overall satisfaction. Since this work is the first to address selective response strategies for GenAI, numerous promising directions remain for future research; we highlight some of them below. 

First, from a technical standpoint, all of the results in this paper rely on Assumption~\ref{assumption: data lip}, involving the lipshitz condition of the accuracy function and the sensitivity parameter $\beta$. Future work could seek to relax this assumption. Furthermore, our constrained optimization approach in Subsection~\ref{sec: welfare constrained revenue maximization} could be extended to approximate the optimal (continuous) strategy instead of the optimal discrete strategy.

Second, our stylized model adopts the simplifying---though unrealistic---assumption that only a single GenAI platform exists. Admittedly, this makes it easier to focus on the idea of selective responses, and indeed, this assumption is pivotal in keeping our analysis tractable. Future research could explore scenarios with multiple GenAI platforms and human-centered forums. In such settings, one platform's selective response might redirect users not only to forums but also to competing GenAI platforms, leading to the tragedy of the commons \cite{hardin1968tragedy}: Although all GenAI platforms benefit from fresh data generation, none may choose to respond selectively if it means losing users to competitors. 

Third, we assumed Forum behaves non-strategically. In reality, human-centered platforms often monetize their data by selling it to GenAI platforms, adding a further layer of strategic interaction for GenAI. Moreover, data transfer between the platforms can form the basis for collaboration: GenAI could employ selective response to bolster Forum content creation, and Forum could, in turn, attribute that content to GenAI for subsequent use in retraining.


%Third, we make the (again) simplifying assumption that Forum is non-strategic. However, in practice, human-centered platforms can sell their data to GenAI platforms. This adds additional considerations for GenAI. Furthermore, data transmission between the platforms can also become the basis for collaboration: GenAI can use selective response to ensure enough content is generated in Forum, and Forum could provide the data attributed to this mechanism back to GenAI. 


%Second, this paper makes the simplifying yet unrealistic assumption of the existence of one GenAI platform. Indeed, this simplifies many aspects and allows us to analyze selective responses. Future work could address the data generation process with more than one GenAI platform and possibly several human-centered forums. In such a case, selective response of one GenAI platform can either drive users to forums or to other GenAI platforms; thus, we might face a tragedy of the commons situation~\ref{hardin1968tragedy}, where all GenAI platforms are interested in fresh data generation but none volunteer to selectively respond and lose users. 

%This paper examines the competition between a generative AI platform and human-based platforms, challenging the assumption that always providing answers is optimal. We analyzed the impact of withholding answers on GenAI's revenue and developed an efficient approximately optimal algorithm for this purpose. We further explored how withholding affects users, showing that it can lead to better outcomes compared to always answering. Specifically, we demonstrated that withholding can Pareto-dominate this strategy and derived the necessary and sufficient conditions for that. Finally, we proposed a second approximately optimal algorithm that maximizes GenAI's revenue while ensuring users are better off than when GenAI answers all queries.

%On a more conceptual level, our model assumes that GenAI’s data comes solely from the competing platform (Forum). Future research could explore a scenario where GenAI can purchase additional data from a third party. This extension could provide valuable insights into the interplay between withholding answers and data purchasing, and whether these two strategies can complement each other or must be traded off.
\section{Open Questions}

\subsection{Balancing Diversity and Relevance in Pairs}
Diversity can be defined as the variation in data samples and pairs used in training while relevance is the level of semantic alignment between the pairs. Ideally, there should be high semantic alignment across pairs of the same category and low alignment across positive and negative categories.

Balancing diversity and relevance is a critical challenge. Both factors are essential: diversity ensures robustness and generalization, while semantic alignment guarantees meaningful and task-relevant representations. Diverse positive pairs may include instances that, while related, are semantically weakly aligned and hence irrelevant, leading to noisy training process and embeddings. Conversely, tightly aligned pairs might be more semantically aligned and relevant but miss significant variations and lack diversity.
Diversity is preferred when the downstream task is unknown, while relevance is prioritized when strong task alignment is needed. 
Future research can focus on dynamic training approaches---
Should early phases focus on introducing diversity to learn general features, followed by more task-aligned features? Or should specialized features be learned first from task-relevant data, and learning generalizable features relegated to later phases?
% with early phases focusing on relevance to learn core features, while later phases introducing diversity for better generalization. 
%Defining positives based on similarity scores rather than binary labels also allows partially aligned pairs to contribute proportionally, ensuring an optimal balance.

\subsection{Dealing with Emerging Modalities in Contrastive Learning}

As new modalities—such as LiDAR, hyperspectral imaging, and haptic feedback, become prominent in various applications, the challenge of integrating these modalities into contrastive learning frameworks emerges \cite{dai2024advancing}. 
Many emerging modalities suffer from the lack of large-scale labeled or even unlabeled datasets. They often exhibit high noise levels or variability due to environmental factors or inherent measurement inaccuracies. Moreover, unlike text or images, pre-trained models for emerging modalities are rare, making initialization more challenging. Future research can focus on scalable and efficient strategies to handle the diversity and complexity of these new data types.

% \clearpage


\newpage
%% The file named.bst is a bibliography style file for BibTeX 0.99c
\bibliographystyle{named}
\bibliography{ijcai25}

\end{document}

