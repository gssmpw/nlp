%%%% ijcai25.tex

\typeout{IJCAI--25 Instructions for Authors}

% These are the instructions for authors for IJCAI-25.

\documentclass{article}
\pdfpagewidth=8.5in
\pdfpageheight=11in

% The file ijcai25.sty is a copy from ijcai22.sty
% The file ijcai22.sty is NOT the same as previous years'
\usepackage{ijcai25}

% Use the postscript times font!
\usepackage{times}
\usepackage{soul}
\usepackage{url}
\usepackage[hidelinks]{hyperref}
\usepackage[utf8]{inputenc}
\usepackage[small]{caption}
\usepackage{graphicx}
\usepackage{amsmath}
\usepackage{amsthm}
\usepackage{booktabs}
\usepackage{algorithm}
\usepackage{algorithmic}
\usepackage[switch]{lineno}
\usepackage{forest} % For hierarchical trees
\usepackage{xcolor} % For coloring
\usepackage{adjustbox}
% \usepackage{hyperref}
\usepackage{xcolor}

% Define clickable links with color
\newcommand{\citelink}[2]{\textcolor{blue}{\href{#2}{#1}}}
\newcommand{\dc}[1]{{\color{orange}{[Deep] \@#1}}}
% Comment out this line in the camera-ready submission
% \linenumbers

\urlstyle{same}

% the following package is optional:
%\usepackage{latexsym}

% See https://www.overleaf.com/learn/latex/theorems_and_proofs
% for a nice explanation of how to define new theorems, but keep
% in mind that the amsthm package is already included in this
% template and that you must *not* alter the styling.
\newtheorem{example}{Example}
\newtheorem{theorem}{Theorem}

% Following comment is from ijcai97-submit.tex:
% The preparation of these files was supported by Schlumberger Palo Alto
% Research, AT\&T Bell Laboratories, and Morgan Kaufmann Publishers.
% Shirley Jowell, of Morgan Kaufmann Publishers, and Peter F.
% Patel-Schneider, of AT\&T Bell Laboratories collaborated on their
% preparation.

% These instructions can be modified and used in other conferences as long
% as credit to the authors and supporting agencies is retained, this notice
% is not changed, and further modification or reuse is not restricted.
% Neither Shirley Jowell nor Peter F. Patel-Schneider can be listed as
% contacts for providing assistance without their prior permission.

% To use for other conferences, change references to files and the
% conference appropriate and use other authors, contacts, publishers, and
% organizations.
% Also change the deadline and address for returning papers and the length and
% page charge instructions.
% Put where the files are available in the appropriate places.


% PDF Info Is REQUIRED.

% Please leave this \pdfinfo block untouched both for the submission and
% Camera Ready Copy. Do not include Title and Author information in the pdfinfo section
\pdfinfo{
/TemplateVersion (IJCAI.2025.0)
}

\title{A Survey on Data Curation for Visual Contrastive Learning: \\ Why Crafting Effective Positive and Negative Pairs Matters}


% Single author syntax
% \author{
%     Author Name
%     \affiliations
%     Affiliation
%     \emails
%     email@example.com
% }

% Multiple author syntax (remove the single-author syntax above and the \iffalse ... \fi here)
% \iffalse
\author{
Shasvat Desai$^1$
\and
Debasmita Ghose$^2$\And
Deep Chakraborty$^{3}$
\affiliations
$^1$Independent Researcher\\
$^2$Yale University\\
$^3$University of Massachusetts Amherst\\
% $^4$Fourth Affiliation\\
\emails
shasvat.desai@gmail.com, 
debasmita.ghose@yale.edu,
dchakraborty@umass.edu
}
% \fi

\begin{document}

\maketitle

\begin{abstract}
Visual contrastive learning aims to learn representations by contrasting similar (positive) and dissimilar (negative) pairs of data samples. The design of these pairs significantly impacts representation quality, training efficiency, and computational cost. A well-curated set of pairs leads to stronger representations and faster convergence.
As contrastive pre-training sees wider adoption for solving downstream tasks, data curation becomes essential for optimizing its effectiveness. 
In this survey, we attempt to create a taxonomy of existing techniques for positive and negative pair curation in contrastive learning, and describe them in detail.
% This survey introduces the research being done in this direction and aims to create a taxonomy of existing techniques used in positive and negative pair curation for contrastive learning. 
We also examine the trade-offs and open research questions in data curation for contrastive learning.
\end{abstract}

\section{Introduction}

Large language models (LLMs) have achieved remarkable success in automated math problem solving, particularly through code-generation capabilities integrated with proof assistants~\citep{lean,isabelle,POT,autoformalization,MATH}. Although LLMs excel at generating solution steps and correct answers in algebra and calculus~\citep{math_solving}, their unimodal nature limits performance in plane geometry, where solution depends on both diagram and text~\citep{math_solving}. 

Specialized vision-language models (VLMs) have accordingly been developed for plane geometry problem solving (PGPS)~\citep{geoqa,unigeo,intergps,pgps,GOLD,LANS,geox}. Yet, it remains unclear whether these models genuinely leverage diagrams or rely almost exclusively on textual features. This ambiguity arises because existing PGPS datasets typically embed sufficient geometric details within problem statements, potentially making the vision encoder unnecessary~\citep{GOLD}. \cref{fig:pgps_examples} illustrates example questions from GeoQA and PGPS9K, where solutions can be derived without referencing the diagrams.

\begin{figure}
    \centering
    \begin{subfigure}[t]{.49\linewidth}
        \centering
        \includegraphics[width=\linewidth]{latex/figures/images/geoqa_example.pdf}
        \caption{GeoQA}
        \label{fig:geoqa_example}
    \end{subfigure}
    \begin{subfigure}[t]{.48\linewidth}
        \centering
        \includegraphics[width=\linewidth]{latex/figures/images/pgps_example.pdf}
        \caption{PGPS9K}
        \label{fig:pgps9k_example}
    \end{subfigure}
    \caption{
    Examples of diagram-caption pairs and their solution steps written in formal languages from GeoQA and PGPS9k datasets. In the problem description, the visual geometric premises and numerical variables are highlighted in green and red, respectively. A significant difference in the style of the diagram and formal language can be observable. %, along with the differences in formal languages supported by the corresponding datasets.
    \label{fig:pgps_examples}
    }
\end{figure}



We propose a new benchmark created via a synthetic data engine, which systematically evaluates the ability of VLM vision encoders to recognize geometric premises. Our empirical findings reveal that previously suggested self-supervised learning (SSL) approaches, e.g., vector quantized variataional auto-encoder (VQ-VAE)~\citep{unimath} and masked auto-encoder (MAE)~\citep{scagps,geox}, and widely adopted encoders, e.g., OpenCLIP~\citep{clip} and DinoV2~\citep{dinov2}, struggle to detect geometric features such as perpendicularity and degrees. 

To this end, we propose \geoclip{}, a model pre-trained on a large corpus of synthetic diagram–caption pairs. By varying diagram styles (e.g., color, font size, resolution, line width), \geoclip{} learns robust geometric representations and outperforms prior SSL-based methods on our benchmark. Building on \geoclip{}, we introduce a few-shot domain adaptation technique that efficiently transfers the recognition ability to real-world diagrams. We further combine this domain-adapted GeoCLIP with an LLM, forming a domain-agnostic VLM for solving PGPS tasks in MathVerse~\citep{mathverse}. 
%To accommodate diverse diagram styles and solution formats, we unify the solution program languages across multiple PGPS datasets, ensuring comprehensive evaluation. 

In our experiments on MathVerse~\citep{mathverse}, which encompasses diverse plane geometry tasks and diagram styles, our VLM with a domain-adapted \geoclip{} consistently outperforms both task-specific PGPS models and generalist VLMs. 
% In particular, it achieves higher accuracy on tasks requiring geometric-feature recognition, even when critical numerical measurements are moved from text to diagrams. 
Ablation studies confirm the effectiveness of our domain adaptation strategy, showing improvements in optical character recognition (OCR)-based tasks and robust diagram embeddings across different styles. 
% By unifying the solution program languages of existing datasets and incorporating OCR capability, we enable a single VLM, named \geovlm{}, to handle a broad class of plane geometry problems.

% Contributions
We summarize the contributions as follows:
We propose a novel benchmark for systematically assessing how well vision encoders recognize geometric premises in plane geometry diagrams~(\cref{sec:visual_feature}); We introduce \geoclip{}, a vision encoder capable of accurately detecting visual geometric premises~(\cref{sec:geoclip}), and a few-shot domain adaptation technique that efficiently transfers this capability across different diagram styles (\cref{sec:domain_adaptation});
We show that our VLM, incorporating domain-adapted GeoCLIP, surpasses existing specialized PGPS VLMs and generalist VLMs on the MathVerse benchmark~(\cref{sec:experiments}) and effectively interprets diverse diagram styles~(\cref{sec:abl}).

\iffalse
\begin{itemize}
    \item We propose a novel benchmark for systematically assessing how well vision encoders recognize geometric premises, e.g., perpendicularity and angle measures, in plane geometry diagrams.
	\item We introduce \geoclip{}, a vision encoder capable of accurately detecting visual geometric premises, and a few-shot domain adaptation technique that efficiently transfers this capability across different diagram styles.
	\item We show that our final VLM, incorporating GeoCLIP-DA, effectively interprets diverse diagram styles and achieves state-of-the-art performance on the MathVerse benchmark, surpassing existing specialized PGPS models and generalist VLM models.
\end{itemize}
\fi

\iffalse

Large language models (LLMs) have made significant strides in automated math word problem solving. In particular, their code-generation capabilities combined with proof assistants~\citep{lean,isabelle} help minimize computational errors~\citep{POT}, improve solution precision~\citep{autoformalization}, and offer rigorous feedback and evaluation~\citep{MATH}. Although LLMs excel in generating solution steps and correct answers for algebra and calculus~\citep{math_solving}, their uni-modal nature limits performance in domains like plane geometry, where both diagrams and text are vital.

Plane geometry problem solving (PGPS) tasks typically include diagrams and textual descriptions, requiring solvers to interpret premises from both sources. To facilitate automated solutions for these problems, several studies have introduced formal languages tailored for plane geometry to represent solution steps as a program with training datasets composed of diagrams, textual descriptions, and solution programs~\citep{geoqa,unigeo,intergps,pgps}. Building on these datasets, a number of PGPS specialized vision-language models (VLMs) have been developed so far~\citep{GOLD, LANS, geox}.

Most existing VLMs, however, fail to use diagrams when solving geometry problems. Well-known PGPS datasets such as GeoQA~\citep{geoqa}, UniGeo~\citep{unigeo}, and PGPS9K~\citep{pgps}, can be solved without accessing diagrams, as their problem descriptions often contain all geometric information. \cref{fig:pgps_examples} shows an example from GeoQA and PGPS9K datasets, where one can deduce the solution steps without knowing the diagrams. 
As a result, models trained on these datasets rely almost exclusively on textual information, leaving the vision encoder under-utilized~\citep{GOLD}. 
Consequently, the VLMs trained on these datasets cannot solve the plane geometry problem when necessary geometric properties or relations are excluded from the problem statement.

Some studies seek to enhance the recognition of geometric premises from a diagram by directly predicting the premises from the diagram~\citep{GOLD, intergps} or as an auxiliary task for vision encoders~\citep{geoqa,geoqa-plus}. However, these approaches remain highly domain-specific because the labels for training are difficult to obtain, thus limiting generalization across different domains. While self-supervised learning (SSL) methods that depend exclusively on geometric diagrams, e.g., vector quantized variational auto-encoder (VQ-VAE)~\citep{unimath} and masked auto-encoder (MAE)~\citep{scagps,geox}, have also been explored, the effectiveness of the SSL approaches on recognizing geometric features has not been thoroughly investigated.

We introduce a benchmark constructed with a synthetic data engine to evaluate the effectiveness of SSL approaches in recognizing geometric premises from diagrams. Our empirical results with the proposed benchmark show that the vision encoders trained with SSL methods fail to capture visual \geofeat{}s such as perpendicularity between two lines and angle measure.
Furthermore, we find that the pre-trained vision encoders often used in general-purpose VLMs, e.g., OpenCLIP~\citep{clip} and DinoV2~\citep{dinov2}, fail to recognize geometric premises from diagrams.

To improve the vision encoder for PGPS, we propose \geoclip{}, a model trained with a massive amount of diagram-caption pairs.
Since the amount of diagram-caption pairs in existing benchmarks is often limited, we develop a plane diagram generator that can randomly sample plane geometry problems with the help of existing proof assistant~\citep{alphageometry}.
To make \geoclip{} robust against different styles, we vary the visual properties of diagrams, such as color, font size, resolution, and line width.
We show that \geoclip{} performs better than the other SSL approaches and commonly used vision encoders on the newly proposed benchmark.

Another major challenge in PGPS is developing a domain-agnostic VLM capable of handling multiple PGPS benchmarks. As shown in \cref{fig:pgps_examples}, the main difficulties arise from variations in diagram styles. 
To address the issue, we propose a few-shot domain adaptation technique for \geoclip{} which transfers its visual \geofeat{} perception from the synthetic diagrams to the real-world diagrams efficiently. 

We study the efficacy of the domain adapted \geoclip{} on PGPS when equipped with the language model. To be specific, we compare the VLM with the previous PGPS models on MathVerse~\citep{mathverse}, which is designed to evaluate both the PGPS and visual \geofeat{} perception performance on various domains.
While previous PGPS models are inapplicable to certain types of MathVerse problems, we modify the prediction target and unify the solution program languages of the existing PGPS training data to make our VLM applicable to all types of MathVerse problems.
Results on MathVerse demonstrate that our VLM more effectively integrates diagrammatic information and remains robust under conditions of various diagram styles.

\begin{itemize}
    \item We propose a benchmark to measure the visual \geofeat{} recognition performance of different vision encoders.
    % \item \sh{We introduce geometric CLIP (\geoclip{} and train the VLM equipped with \geoclip{} to predict both solution steps and the numerical measurements of the problem.}
    \item We introduce \geoclip{}, a vision encoder which can accurately recognize visual \geofeat{}s and a few-shot domain adaptation technique which can transfer such ability to different domains efficiently. 
    % \item \sh{We develop our final PGPS model, \geovlm{}, by adapting \geoclip{} to different domains and training with unified languages of solution program data.}
    % We develop a domain-agnostic VLM, namely \geovlm{}, by applying a simple yet effective domain adaptation method to \geoclip{} and training on the refined training data.
    \item We demonstrate our VLM equipped with GeoCLIP-DA effectively interprets diverse diagram styles, achieving superior performance on MathVerse compared to the existing PGPS models.
\end{itemize}

\fi 

\section{Related Work}\label{sec:related_works}
\gls{bp} estimation from \gls{ecg} and \gls{ppg} waveforms has received significant attention due to its potential for continuous, unobtrusive monitoring. Earlier work relied on classical machine learning with handcrafted features, but deep learning methods have since emerged as more robust alternatives. Convolutional or recurrent architectures designed for \gls{ecg}/\gls{ppg} have shown strong performance, including ResUNet with self-attention~\cite{Jamil}, U-Net variants~\cite{Mahmud_2022}, and hybrid \gls{cnn}--\gls{rnn} models~\cite{Paviglianiti2021ACO}. These architectures often outperform traditional feature-engineering approaches, particularly when both \gls{ecg} and \gls{ppg} signals are used~\cite{Paviglianiti2021ACO}.

Nevertheless, many existing methods train solely on \gls{ecg}/\gls{ppg} data, which, while plentiful~\cite{mimiciii,vitaldb,ptb-xl}, often exhibit significant variability in signal quality and patient-specific characteristics. This variability poses challenges for achieving robust generalization across populations. Recent work has explored transfer learning to overcome these issues; for example, Yang \emph{et~al.}~\cite{yang2023cross} studied the transfer of \gls{eeg} knowledge to \gls{ecg} classification tasks, achieving improved performance and reduced training costs. Joshi \emph{et~al.}~\cite{joshi2021deep} also explored the transfer of \gls{eeg} knowledge using a deep knowledge distillation framework to enhance single-lead \gls{ecg}-based sleep staging. However, these studies have largely focused on within-modality or narrow domain adaptations, leaving open the broader question of whether an \gls{eeg}-based foundation model can serve as a versatile starting point for generalized biosignal analysis.

\gls{eeg} has become an attractive candidate for pre-training large models not only because of the availability of large-scale \gls{eeg} repositories~\cite{TUEG} but also due to its rich multi-channel, temporal, and spectral dynamics~\cite{jiang2024large}. While many time-series modalities (for example, voice) also exhibit rich temporal structure, \gls{eeg}, \gls{ecg}, and \gls{ppg} share common physiological origins and similar noise characteristics, which facilitate the transfer of temporal pattern recognition capabilities. In other words, our hypothesis is that the underlying statistical properties and multi-dimensional dynamics in \gls{eeg} make it particularly well-suited for learning robust representations that can be effectively adapted to \gls{ecg}/\gls{ppg} tasks. Our work is the first to validate the feasibility of fine-tuning a transformer-based model initially trained on EEG (CEReBrO~\cite{CEReBrO}) for arterial \gls{bp} estimation using \gls{ecg} and \gls{ppg} data.

Beyond accuracy, real-world deployment of \gls{bp} estimation models calls for efficient inference. Traditional deep networks can be computationally expensive, motivating recent interest in quantization and other compression techniques~\cite{nagel2021whitepaperneuralnetwork}. Few studies have combined large-scale pre-training with post-training quantization for \gls{bp} monitoring. Hence, our method integrates these two aspects: leveraging a potent \gls{eeg}-based foundation model and applying quantization for a compact, high-accuracy cuffless \gls{bp} solution.
\section{Taxonomy}
\label{sec:taxonomy}
In this section, we present our taxonomy for mitigating the multimodal imbalance problem based on the strategies for handling modality imbalance. As shown in Table \ref{tab:mm_algorithms}, we categorize these methods into four groups: Data, Feed-forward, Objective and Optimization. We also summarize the different types of imbalance indicator, which different methods use to evaluate the performance of different modalities.

\begin{table*}[t] 
    \centering
    \setlength{\tabcolsep}{5pt}
    \caption{Multimodal imbalance algorithms. \textbf{Adjustment Strategy} refers to different groups of methods in Section \ref{sec:taxonomy}. \textbf{Imbalance Indicator} denotes the metric used to evaluate modality performance. \textbf{Number of Modalities} indicates the maximum number of modalities included in the experiments of the corresponding paper. \textbf{Dataset Domain} refers to the types of modalities included in the corresponding paper.} 
\vspace{-5pt}
\label{tab:mm_algorithms} 
    \begin{tabular}{l|c|c|c|ccc}
        \toprule
        \multicolumn{1}{c}{}            & \multicolumn{1}{c}{Adjustment}       & \multicolumn{1}{c}{Imbalance}         & \multicolumn{1}{c}{Number of}                & \multicolumn{3}{c}{Dataset Domain} \\
        \multicolumn{1}{l}{Method}   & \multicolumn{1}{c}{Strategy}         & \multicolumn{1}{c}{Indicator}    & \multicolumn{1}{c}{Modalities}         & CV & NLP & Audio \\                 
        \midrule
                                                                
        % Here, we start listing the papers  
        Modality-valuation \cite{Sample_wei}       & Data                                     & Shapley-based Metric                                                     & $2$                                        & \checkmark    &  & \checkmark \\
        MLA \cite{MLA}       & Feed-forward                                     & N/A                                                    & $3$                                          & \checkmark    & \checkmark & \checkmark \\
        OPM \cite{OPM_PAMI}       & Feed-forward                                     & Performance Score                                                    & $3$                                         & \checkmark    & \checkmark & \checkmark \\
        Greedy \cite{Greedy_Wu}       & Feed-forward                                     & Gradient Change                                                     & $2$                                         & \checkmark    &  & \checkmark \\
        AMCo \cite{AMCo}       & Feed-forward                                     & Performance Score                                                     & $3$                                         & \checkmark    & \checkmark & \checkmark \\
        MMCosine \cite{MMCosine_Xu} & Objective                                   & Performance Score                                                       & $2$                                   &     \checkmark  &  & \checkmark \\   
        UMT \cite{UMT_Du}         & Objective                                      & N/A                                                     & $3$                                         & \checkmark    &  & \checkmark \\
        MBSD \cite{MBSD}       & Objective                                     & Performance Score                                                     & $2$                                         & \checkmark    & \checkmark &\\
        CML \cite{CML_Ma}          & Objective                                    & Classification Loss                                                    & $2$                                         & \checkmark    & \checkmark  & \\
        MMPareto \cite{MMPareto_Wei}          & Objective                                    & Performance Score                                                    & $3$                                         & \checkmark    & \checkmark & \checkmark  \\
        GBlending \cite{Gblending_Wang}       & Objective                                     & Classification Loss                                                     & $3$                                          &\checkmark     &   & \checkmark   \\
        LFM \cite{LFM_yang}        & Objective                                      &  N/A                                                              & $3$                                             & \checkmark    & \checkmark  &  \checkmark \\
        OGM \cite{OGM_CVPR}  & Optimazation                                     & Performance Score                                                       & $2$                                          & \checkmark    &  & \checkmark \\
        AGM \cite{AGM_Li}   & Optimazation                                       & Performance Score                                                              & $3$                                         & \checkmark    &  \checkmark & \checkmark  \\
        PMR \cite{PMR_Fan}       & Optimazation                                     & Prototype                                                       & $2$                                           & \checkmark    &  & \checkmark\\
        Relearning \cite{Relearning_wei}       & Optimazation                                     & Clustering                                                     & $3$                                          & \checkmark    & \checkmark & \checkmark \\
        ReconBoost \cite{ReconBoost_Huang}       & Optimazation                                     & Classification Loss                                                    & $3$                                          & \checkmark    & \checkmark & \checkmark \\

        \bottomrule                                
    \end{tabular}
    \vspace{-10pt}
\end{table*}

\subsection{Data}
This part focuses on the method which enhances modality performance through targeted data processing strategies. Wei et al. \cite{Sample_wei} propose a fine-grained evaluation method to facilitate multimodal collaboration. It evaluates modality-specific contributions at the sample level and employs selective resampling techniques to enhance the discriminative capabilities of weak modality modalities.

\subsection{Feed-forward}
These methods alleviate the imbalanced learning across modalities by modifying the forward process during model training and inference. These methods can be categorized into two types based on where modifications are made.

\textbf{Feature Processing.} The first type of methods adjust features during training. Adaptive Mask Co-optimization (AMCo) \cite{AMCo} masks features of dominant modalities based on their performance, while On-the-fly Prediction Modulation (OPM) \cite{OPM_PAMI} drops its feature with dynamical probability in feed-forward stage.

\textbf{Fusion Module.} The second type achieves modality balance by modifying the fusion mechanisms. Multimodal Learning with Alternating Unimodal Adaptation (MLA) \cite{MLA} uses dynamic fusion to integrate different modalities. It also employs an alternating optimization approach to optimize unimodal encoders, minimizing interference between modalities. Greedy \cite{Greedy_Wu} utilizes the MMTM \cite{MMTM_Joze} architecture for intermediate fusion to boost the modality interaction. It also facilitates the learning of weak modality that indicated by conditional learning speed, which is measured by the gradient change ratio.

\subsection{Objective}
Various methods for addressing modality imbalance in multimodal learning focus on modifying objectives. These methods can be categorized into three main directions: 

Firstly, several methods modify the multimodal loss function to mitigate the multimodal imbalance problem. For instance, Multi-Modal Cosine loss (MMCosine) \cite{MMCosine_Xu} proposes a multimodal cosine loss, which effectively increases the learning proportion of weaker modalities by weight constraints and inter-symmetric constraints.
% and inter-modality symmetry constraints.

Secondly, a group of methods leverage modality differences for learning objectives to achieve balanced learning. MBSD \cite{MBSD} constrains the model using the Kullback-Leibler (KL) \cite{KL} divergence of prediction distributions between different modalities to reduce their distance. Calibrating Multimodal Learning (CML) \cite{CML_Ma} uses confidence loss derived from different modalities, which lowers the confidence of the dominant modality. LFM \cite{LFM_yang} bridges heterogeneous data in the feature space through contrastive learning, reducing the distance between different modalities.

Thirdly, several approaches incorporate unimodal loss into the objectives to mitigate the imbalance problem. Uni-Modal Teacher (UMT) \cite{UMT_Du} introduces a unimodal distillation loss, enhancing the learning of unimodal encoders. Gradient-Blending (GBlending) \cite{Gblending_Wang} and MMPareto \cite{MMPareto_Wei} utilize unimodal losses to solve the imbalance problem. GBlending \cite{Gblending_Wang} uses overfitting-to-generalization-ratio (OGR) as an indicator to show which modality is dominant and its corresponding weight, while MMPareto \cite{MMPareto_Wei} borrows ideas from Pareto method \cite{ParetoInMultitask} to guarantee the final gradient is with direction common to all learning objectives to boost the learning of weak modality. 

\subsection{Optimization}
Recent studies have investigated optimization-based approaches to mitigate the multimodal imbalance problem. Both On-the-fly Gradient Modulation (OGM) \cite{OGM_CVPR} and Adaptive Gradient Modulation (AGM) \cite{AGM_Li} aim to balance modality learning by slowing down the gradients of dominant modalities to provide more optimization space for weak modalities. Specifically, OGM \cite{OGM_CVPR} uses performance score as an indicator to achieve this, while AGM \cite{AGM_Li} employs a Shapley value-based method for gradient adjustment. Prototypical Modality Rebalance (PMR) \cite{PMR_Fan} adjusts gradient magnitudes based on category prototypes to accelerate the learning of weak modalities. Diagnosing \& Re-learning (Relearning) \cite{Relearning_wei} uses re-initialization to reduce the dependence on dominant modalities while preventing weak modalities from learning excessive noise. ReconBoost \cite{ReconBoost_Huang} introduces an alternating-boosting optimization way to enhance the unimodal performance, which alleviates the imbalance problem.
\section{Crafting Effective Positive Pairs
}
\label{sec: Positive_Pair}

\begin{figure*}[h!]
    \centering
    \includegraphics[width=0.95\linewidth]{Figures/positive_pairs.pdf}
    \caption{\textbf{Positive Pair Curation Techniques: } Positive pair selection can utilize single-instance and multi-instance techniques. (a) Single-instance curation applies augmentations to a single sample. On the other hand, multi-instance positive pair generation can be further classified into several category of techniques. (b) Embedding-based retrieves the top-K nearest neighbors of the anchor sample's augmentation in the embedding space and pairs them with other augmentations of the anchor. (c) Synthetic pairs generate data conditioned on the input, which is then augmented and paired with the augmented real sample (d) Supervised pairs use external sources (human labels, oracles, or annotations) to fetch another sample from the same category and create positive pairs. (e) Attributed-based: These methods group samples by shared attributes (e.g., golden retrievers paired with golden labrador retrievers based on fur color) and pair their respective augmentations. (f) Cross-modal: This involves creating semantically aligned pairs across multiple modalities. The figure shows image-text and speech-image pairing.}
    \label{fig:positive}
\end{figure*}

\subsection{Single Instance Positives}

This technique creates positive pairs using augmentations of a single sample without explicit curation, as shown in Fig. 2(a).  The negative pairs are also randomly sampled from the dataset and are uncurated. %Data augmentation plays a crucial role by introducing various transformations to the original data, prompting the learning algorithm to identify consistent underlying patterns.

A common contrastive learning technique using this type of data curation is \textbf{SimCLR}\cite{chen2020simple}. SimCLR maximizes agreement between augmented views of the same data point using the InfoNCE loss, relying on large batch sizes to sample enough negatives. \textbf{MoCo} \cite{he2020momentum} addresses a drawback of SimCLR, which requires large negative samples that can be computationally expensive by using a momentum encoder and memory bank to maintain a queue of negatives dynamically. \textbf{SimSiam} \cite{chen2021exploring} eliminates the need for negative examples, using a stop-gradient mechanism to prevent representation collapse in its Siamese architecture. \textbf{BYOL} \cite{grill2020bootstrap} simplifies learning by aligning predictions from an online network with a momentum-maintained target network, achieving strong performance without negatives. \textbf{Barlow Twins} \cite{zbontar2021barlow} focuses on redundancy reduction by aligning embeddings and decorrelating feature dimensions, avoiding collapse naturally without negatives or momentum. \textbf{DINO} \cite{caron2021emerging} combines self-supervised learning with knowledge distillation using a teacher-student framework, producing generalized embeddings. \textbf{VicReg} \cite{bardes2021vicreg} introduces regularization to balance variance, invariance, and decorrelation in embeddings, ensuring quality without negatives or momentum encoders. \textbf{SwAV} \cite{caron2020unsupervised} uses clustering to align augmentations by mapping them to shared cluster assignments without direct contrastive loss. \textbf{CPC} \cite{oord2018representation} leverages contrastive loss in a latent space to predict future data segments, making it particularly effective for time-series tasks. Finally, \textbf{SEED} \cite{fang2021seed} simplifies training by using teacher-student distillation with pseudo-labels, reducing computational complexity. %while maintaining strong performance.





\subsection{Multi Instance Positives}
Multi-instance positive pair curation creates pairs from different samples rather than augmented views of the same sample.

\subsubsection{Embedding-Based Techniques}
Given an input candidate sample, this class of techniques first retrieves the K-nearest neighbors of one of the augmentations using a similarity metric in embedding space, as shown in Figure 2(b). Next, it uses the K-retrieved samples and the other augmentations of the candidate sample as positive pairs. 

Nearest-Neighbour Contrastive Learning of Visual Representations (NNCLR)\cite{dwibedi2021little} samples the nearest neighbors from the dataset in the latent space and treats them as positives. This provides
more semantic and intra-class variations to learn representations that are invariant to different viewpoints, deformations, and variations. The NNCLR framework relies entirely on a single nearest neighbor, limiting its potential. Mean Shift for Self-Supervised Learning (MSF) \cite{koohpayegani2021mean} addresses this limitation by proposing the use of \(k\) nearest neighbors to increase the diversity in the positive pairs. MSF shifts the embedding of each image to be closer to the \textit{mean} of the neighbors of its augmentation.
However, MSF is computationally expensive because the objective function must be computed \(k\) times for each neighbor. To address MSF's computational inefficiency, All4One \cite{estepa2023all4one} contrasts information from multiple neighbors by compiling information from the extracted \(k\) neighbors to create a pair of representations, called \textit{centroids}, which contain contextual information about all the neighbors. 

These techniques can be used when semantic clustering is needed for downstream applications. For instance, if the downstream task involves clustering similar faces, these techniques allow different views of the same person to be closer together, unlike single-instance positive techniques, which treat all other images as negatives.

\subsubsection{Synthetic Data Generation for Positive Pairs}

This class of techniques creates synthetic samples using a generative process conditioned on the candidate input sample. A positive pair is formed by combining the augmented generated sample with the augmentation of the original input sample, which is then processed by the encoder, as illustrated in Fig. 2(c).

Contrastive Learning with Synthetic Positives (CLSP) \cite{zeng2024contrastive} incorporates synthetic positives generated via a diffusion model. By interpolating Gaussian noise with diffusion-based features, CLSP creates images that resemble the anchor while varying the context and background, increasing diversity while preserving semantic meaning. Similarly, \cite{wu2023synthetic} introduces a GAN-based framework that dynamically generates hard positive pairs by jointly optimizing the GAN and contrastive model. However, this simultaneous training introduces instability and quality control challenges. These approaches are particularly useful in data-sparse scenarios, rare modalities, or domains where obtaining real data is challenging, such as cross-modal medical applications (e.g., speech-image pairs).



\subsubsection{Supervised Pairing Techniques}

These techniques use external data sources, such as human preferences, privileged information from an oracle, or an annotated dataset, to derive meaningful metadata and semantic categories and create positive pairs, as shown in Fig. 2(d).


Supervised contrastive learning (SupCon) \cite{khosla2020supervised} leverages ground truth labels to enhance representation learning by incorporating category-level supervision. Instead of defining positive pairs through augmentations of a single instance, SupCon creates positive pairs from multiple samples of the same category as the anchor, ensuring that representations capture category-level semantic similarities rather than just instance-specific features.

Building upon this, \cite{ghose2023tailoring} propose a method to create positive pairs on the fly by passively observing humans provide limited positive examples while working collaboratively with a robot without explicitly marking negatives. This aligns with Positive-Unlabeled (PU) Learning \cite{bekker2020learning}, where only positives are known, and the model infers meaningful distinctions. Contrastive learning then clusters these examples, ensuring representations align with human expectations in a task-adaptive manner.
%Contrastive learning is then applied to cluster human-selected examples, ensuring that representations align with human expectations in a task-adaptive manner.  
Similarly, Oracle-guided Contrastive Clustering (OCC) \cite{wang2022oracle} uses a deep clustering framework designed to create positive pairs for contrastive loss by incorporating oracle feedback into the clustering process, ensuring that learned representations align with user-specific clustering preferences. Instead of relying purely on instance similarity in the embedding space, OCC actively queries an oracle (human or predefined rule) to determine whether two samples should belong to the same cluster.

These techniques are useful in scenarios when labeled data is available, and the goal is to cluster semantically similar items within the same class or the downstream task requires discrimination between subtle intra-class variations. Leveraging labels and semantic information to create pairs enables us to generate embeddings that are better aligned to the downstream application. Intuitively, these techniques should eliminate false negatives (through semantic clustering) and potentially reduce noise in the generated embeddings. 

\subsubsection{Attribute-based Pairing Techniques}

Attribute-based pairing entails selecting positive pairs based on task-specific criteria, as shown in Fig. 2(e). Attributes can be generic, such as \textit{``a golden colored object"} or more specific such as \textit{``dog with golden colored coat"}

For instance, Geography-aware self-supervised learning \cite{ayush2021geography} leverages spatial and temporal attributes to create temporal positive pairs from images of the same geographical location taken at different times. They demonstrate their approach in the Remote sensing domain because it is easy to obtain multiple geo-located images of the exact location over time. Similarly, \cite{pantazis2021focus} leverages the natural variations in sequential images from static cameras, utilizing contextual information such as spatial and temporal relationships to identify high-probability positive pairs—images likely depicting the same visual concept. Yet another approach proposed by \cite{panambur2022self} leveraged domain-specific attributes like geological structures, terrain textures, and spatial and scientific properties to form positive pairs for terrain categorization in Martian terrain.

These techniques are useful when domain-specific contextual attributes are known. However, uneven attribute distribution can lead to over-representation of certain pair types and hinder performance on unseen variations, so careful attribute selection and balancing are essential.

\subsubsection{Cross-modal Positive Pairing Techniques}
Cross-modal contrastive learning aims to learn meaningful representations across different data modalities  to improve performance in various tasks that involve multi-modal data,
as shown in Fig. 2(f). 

\noindent
\textbf{Image-Text Pairing: }
Image-text pairing aims to align visual and textual information to learn effective representations. 
CLIP \cite{radford2021learning}, CLOOB \cite{furst2022cloob}, ALIGN \cite{jia2021scaling} employs contrastive learning to learn shared representations by aligning visual and textual data which facilitate downstream tasks like zero-shot image classification and cross-modal retrieval. BEiT-3 \cite{wang2022image}, VisualBERT \cite{li1908visualbert}, FLAVA \cite{singh2022flava}, LXMERT \cite{tan2019lxmert} are a class of methods that introduce a unified masked data modeling objective. Given a partially masked caption, the objective is to predict the masked words based on the corresponding image. They learn representations that capture the relationships between images and texts by masking parts of the input and training the model to predict the missing information. MAPL \cite{manas2022mapl} and Flamingo \cite{alayrac2022flamingo} keep the pre-trained vision encoder and language model frozen to learn a lightweight mapping between their representation spaces, enabling few-shot learning with minimal parameter updates.


\noindent
\textbf{Audio-Image-Text Pairing: }
AudioCLIP \cite{guzhov2022audioclip}, Wav2CLIP \cite{wu2022wav2clip} learns audio representations by distilling knowledge from the CLIP model to jointly learn a shared representation of audio data alongside image and text modalities. CLAP \cite{wu2023large} trains a dual-encoder model to align audio and text embeddings. %within a shared space.

\begin{figure*}[h!]
    \centering
    \includegraphics[width=0.97\linewidth]{Figures/negative_pairs.pdf}
    \caption{\textbf{Negative Pair Curation Techniques}: 
    % Most positive pair creation techniques rely on uncurated negative data and do not fully utilize the available data's potential. 
    This figure shows three categories of techniques for negative pair curation. (a). Hard Negative Selection prioritizes negatives that are semantically similar to the anchor sample, such as a different cat breed, instead of an unrelated category like an airplane. The negatives are then augmented and fed into the encoder. (b). False Negative Elimination removes or reclassifies negatives that are highly similar to the anchor sample, preventing the model from mistakenly separating highly similar samples. The remaining negatives are then augmented before encoding. Hard negatives improve discrimination but risk overfitting, while false negative elimination reduces noise but may mistakenly remove challenging yet valid negatives, weakening the representations. (c). Synthetic negative pairs are created by feeding the positive and negative samples(dataset) into a generative process and conditioned on the anchor sample to create realistic but distinct negatives. The generated samples then undergo augmentation and are fed with the positive pairs to the downstream encoder.}
    \label{fig:negative}
\end{figure*}

% \subsubsection{Speech-Audio-Text Pairing}

% Wav2Vec \\cite{baevski2020wav2vec} represents a category of models designed for self-supervised representation learning from raw audio waveforms. The wav2vec model is trained by predicting speech units for masked parts of speech audio. CLAPSpeech \cite{ye2023clapspeech} introduces a framework that explicitly learns prosody variations of text tokens under different contexts using cross-modal contrastive pre-training. By aligning text context with corresponding prosody patterns in a joint multi-modal space and employing a multi-scale pre-training pipeline, CLAPSpeech enhances prosody prediction in text-to-speech systems.

\noindent
\textbf{Audio-Visual Pairing: }
Audio-Visual Instance Discrimination (AVID) \cite{morgado2021audio} and \cite{alwassel2020self} emphasize cross-modal discrimination, aiming to align audio and visual features effectively. They use cross-modal clustering, where shared semantic clusters are learned across modalities by mapping audio and video representations into a joint embedding space. Building upon this, \cite{yariv2023audiotoken} adapts pre-trained text-conditioned diffusion models, like Stable Diffusion, by converting audio inputs into text-like embeddings through a learnable adapter. These embeddings serve as prompts for the diffusion model to generate audio-aligned images.

\section{Crafting Effective Negative Pairs}
\label{sec: Negative_Pair}

\subsection{Hard Negative Selection}

Hard negatives are those negative samples that are particularly similar to the anchor (the positive sample) in the embedding space, making them more likely to be misclassified. By incorporating such samples, the model is forced to refine its representation, learning more discriminative features to distinguish between fine-grained differences. These hard negatives are then fed into the encoder along with the positive pair, as shown in Fig. 3(a).

MoCHi, \textit{(M)ixing (o)f (C)ontrastive (H)ard negat(i)ves} \cite{Hardnegativemixing}, creates hard negatives by combining features of existing hard negatives in the embedding space. It identifies existing negatives for a given anchor that are most similar to the candidate positive sample in the embedding space and combines these hard negatives at the feature level to create synthetic negatives that are even closer to the anchor, increasing the difficulty of the contrastive task.

Uncertainty and Representativeness Mixing (UnReMix) \cite{unremix} selects negative samples based on three key properties. Anchor similarity ensures that negative samples closely resemble the anchor but belong to different classes. Model uncertainty prioritizes negative samples with higher prediction uncertainty, focusing the learning process on less confident regions of the data space. Representativeness emphasizes selecting negatives that reflect the overall data distribution rather than outliers. Similarly, \cite{robinson2020contrastive} samples negatives close to the anchor in the embedding space.  These negatives are generated adversarially or synthesized through feature interpolation, promoting fine-grained learning. A balanced mix of hard and easy negatives ensures stability and prevents overfitting during training.

Yet another approach \cite{hu2021adco} follows a min-max optimization framework, where the encoder minimizes the contrastive loss by learning to separate positives from negatives while the negative adversaries maximize the loss by generating challenging and indistinguishable negatives.





\subsection{Removal of False Negatives}
False negatives are samples from different images with the same semantic content, therefore they should hold certain similarity. Contrasting false negatives induces two critical issues in representation learning: 1) discarding semantic information and 2) slow convergence due to the addition of noise in the learning process. For instance, a cat's head in one image might be attracted to its fur (positive pair) but repelled from the similar fur in another image of a cat (negative pair), creating conflicting objectives. Eliminating false negatives involves taking a batch of negative samples and removing those highly similar to positives, as shown in Figure 3(b). The rest of the samples in the batch undergo augmentations and are sent to the encoder along with the positive pairs.

\cite{huynh2022boosting} introduces methods to identify these false negatives and proposes two strategies to mitigate their impact: elimination and attraction. Elimination identifies and excludes potential false negatives from the negative sample set, preventing the model from learning misleading distinctions. In contrast, false negative attraction reclassifies them as positives(makes them true positives), encouraging the model to learn representations that acknowledge their semantic similarity. Similarly, \cite{chen2021incremental} dynamically detects false negatives based on semantic similarity and reclassifies them as positives, thus reducing noise in the learning process. 

\cite{chuang2020debiased} takes a different approach to mitigate the impact of false negatives in contrastive learning by introducing a re-weighted loss function. This loss adjusts the contribution of each negative sample based on its likelihood of being a true negative without requiring label information. The approach improves representation learning by minimizing the influence of false negatives, achieving better performance in self-supervised settings across various domains.
These techniques help ensure the negative pairs are relevant and the generated embeddings are aligned to the downstream task.


\subsection{Synthetic Hard Negatives}
Synthetic negatives can be created using various techniques, including generative models, feature space interpolation, or rule-based algorithms that modify existing data. Once created, their augmentation and positive pairs are sent to the encoder, as shown in Fig. 3(c).

\textit{Synthetic Hard Negative Samples for Contrastive Learning} \cite{dong2024synthetic} involves mixing existing negative samples in the feature space to create more challenging negatives synthetically. To address the issue of false negatives—samples incorrectly labeled as negative but semantically similar to the anchor, this work incorporates a debiasing mechanism, ensuring the model focuses on truly dissimilar negative samples. 
%For a given anchor sample, the method selects the hardest negative samples from the available negative set based on their similarity to the anchor. 
The selected hard negatives are then combined through linear interpolation to create synthetic negative samples that are even closer to the anchor in the feature space. 
%This ensures that the generated negatives are challenging enough to improve the model's discriminative ability.

Similarly, another approach \cite{giakoumoglou2024synco} builds upon the MoCo framework \cite{he2020momentum} to create diverse synthetic hard negatives on the fly with minimal computational overhead. It generates negatives by interpolating between positive and negative samples in the feature space, extrapolating beyond the positive sample in the direction of a negative sample, applying small perturbations to positive samples to generate negatives, and using adversarial methods to craft indistinguishable negatives.
\section{ Task Generalization Beyond i.i.d. Sampling and Parity Functions
}\label{sec:Discussion}
% Discussion: From Theory to Beyond
% \misha{what is beyond?}
% \amir{we mean two things: in the first subsection beyond i.i.d subsampling of parity tasks and in the second subsection beyond parity task}
% \misha{it has to be beyond something, otherwise it is not clear what it is about} \hz{this is suggested by GPT..., maybe can be interpreted as from theory to beyond theory. We can do explicit like Discussion: Beyond i.i.d. task sampling and the Parity Task}
% \misha{ why is "discussion" in the title?}\amir{Because it is a discussion, it is not like separate concrete explnation about why these thing happens or when they happen, they just discuss some interesting scenraios how it relates to our theory.   } \misha{it is not really a discussion -- there is a bunch of experiments}

In this section, we extend our experiments beyond i.i.d. task sampling and parity functions. We show an adversarial example where biased task selection substantially hinders task generalization for sparse parity problem. In addition, we demonstrate that exponential task scaling extends to a non-parity tasks including arithmetic and multi-step language translation.

% In this section, we extend our experiments beyond i.i.d. task sampling and parity functions. On the one hand, we find that biased task selection can significantly degrade task generalization; on the other hand, we show that exponential task scaling generalizes to broader scenarios.
% \misha{we should add a sentence or two giving more detail}


% 1. beyond i.i.d tasks sampling
% 2. beyond parity -> language, arithmetic -> task dependency + implicit bias of transformer (cannot implement this algorithm for arithmatic)



% In this section, we emphasize the challenge of quantifying the level of out-of-distribution (OOD) differences between training tasks and testing tasks, even for a simple parity task. To illustrate this, we present two scenarios where tasks differ between training and testing. For each scenario, we invite the reader to assess, before examining the experimental results, which cases might appear “more” OOD. All scenarios consider \( d = 10 \). \kaiyue{this sentence should be put into 5.1}






% for parity problem




% \begin{table*}[th!]
%     \centering
%     \caption{Generalization Results for Scenarios 1 and 2 for $d=10$.}
%     \begin{tabular}{|c|c|c|c|}
%         \hline
%         \textbf{Scenario} & \textbf{Type/Variation} & \textbf{Coordinates} & \textbf{Generalization accuracy} \\
%         \hline
%         \multirow{3}{*}{Generalization with Missing Pair} & Type 1 & \( c_1 = 4, c_2 = 6 \) & 47.8\%\\ 
%         & Type 2 & \( c_1 = 4, c_2 = 6 \) & 96.1\%\\ 
%         & Type 3 & \( c_1 = 4, c_2 = 6 \) & 99.5\%\\ 
%         \hline
%         \multirow{3}{*}{Generalization with Missing Pair} & Type 1 &  \( c_1 = 8, c_2 = 9 \) & 40.4\%\\ 
%         & Type 2 & \( c_1 = 8, c_2 = 9 \) & 84.6\% \\ 
%         & Type 3 & \( c_1 = 8, c_2 = 9 \) & 99.1\%\\ 
%         \hline
%         \multirow{1}{*}{Generalization with Missing Coordinate} & --- & \( c_1 = 5 \) & 45.6\% \\ 
%         \hline
%     \end{tabular}
%     \label{tab:generalization_results}
% \end{table*}

\subsection{Task Generalization Beyond i.i.d. Task Sampling }\label{sec: Experiment beyond iid sampling}

% \begin{table*}[ht!]
%     \centering
%     \caption{Generalization Results for Scenarios 1 and 2 for $d=10, k=3$.}
%     \begin{tabular}{|c|c|c|}
%         \hline
%         \textbf{Scenario}  & \textbf{Tasks excluded from training} & \textbf{Generalization accuracy} \\
%         \hline
%         \multirow{1}{*}{Generalization with Missing Pair}
%         & $\{4,6\} \subseteq \{s_1, s_2, s_3\}$ & 96.2\%\\ 
%         \hline
%         \multirow{1}{*}{Generalization with Missing Coordinate}
%         & \( s_2 = 5 \) & 45.6\% \\ 
%         \hline
%     \end{tabular}
%     \label{tab:generalization_results}
% \end{table*}




In previous sections, we focused on \textit{i.i.d. settings}, where the set of training tasks $\mathcal{F}_{train}$ were sampled uniformly at random from the entire class $\mathcal{F}$. Here, we explore scenarios that deliberately break this uniformity to examine the effect of task selection on out-of-distribution (OOD) generalization.\\

\textit{How does the selection of training tasks influence a model’s ability to generalize to unseen tasks? Can we predict which setups are more prone to failure?}\\

\noindent To investigate this, we consider two cases parity problems with \( d = 10 \) and \( k = 3 \), where each task is represented by its tuple of secret indices \( (s_1, s_2, s_3) \):

\begin{enumerate}[leftmargin=0.4 cm]
    \item \textbf{Generalization with a Missing Coordinate.} In this setup, we exclude all training tasks where the second coordinate takes the value \( s_2 = 5 \), such as \( (1,5,7) \). At test time, we evaluate whether the model can generalize to unseen tasks where \( s_2 = 5 \) appears.
    \item \textbf{Generalization with Missing Pair.} Here, we remove all training tasks that contain both \( 4 \) \textit{and} \( 6 \) in the tuple \( (s_1, s_2, s_3) \), such as \( (2,4,6) \) and \( (4,5,6) \). At test time, we assess whether the model can generalize to tasks where both \( 4 \) and \( 6 \) appear together.
\end{enumerate}

% \textbf{Before proceeding, consider the following question:} 
\noindent \textbf{If you had to guess.} Which scenario is more challenging for generalization to unseen tasks? We provide the experimental result in Table~\ref{tab:generalization_results}.

 % while the model struggles for one of them while as it generalizes almost perfectly in the other one. 

% in the first scenario, it generalizes almost perfectly in the second. This highlights how exposure to partial task structures can enhance generalization, even when certain combinations are entirely absent from the training set. 

In the first scenario, despite being trained on all tasks except those where \( s_2 = 5 \), which is of size $O(\d^T)$, the model struggles to generalize to these excluded cases, with prediction at chance level. This is intriguing as one may expect model to generalize across position. The failure  suggests that positional diversity plays a crucial role in the task generalization of Transformers. 

In contrast, in the second scenario, though the model has never seen tasks with both \( 4 \) \textit{and} \( 6 \) together, it has encountered individual instances where \( 4 \) appears in the second position (e.g., \( (1,4,5) \)) or where \( 6 \) appears in the third position (e.g., \( (2,3,6) \)). This exposure appears to facilitate generalization to test cases where both \( 4 \) \textit{and} \( 6 \) are present. 



\begin{table*}[t!]
    \centering
    \caption{Generalization Results for Scenarios 1 and 2 for $d=10, k=3$.}
    \resizebox{\textwidth}{!}{  % Scale to full width
        \begin{tabular}{|c|c|c|}
            \hline
            \textbf{Scenario}  & \textbf{Tasks excluded from training} & \textbf{Generalization accuracy} \\
            \hline
            Generalization with Missing Pair & $\{4,6\} \subseteq \{s_1, s_2, s_3\}$ & 96.2\%\\ 
            \hline
            Generalization with Missing Coordinate & \( s_2 = 5 \) & 45.6\% \\ 
            \hline
        \end{tabular}
    }
    \label{tab:generalization_results}
\end{table*}

As a result, when the training tasks are not i.i.d, an adversarial selection such as exclusion of specific positional configurations may lead to failure to unseen task generalization even though the size of $\mathcal{F}_{train}$ is exponentially large. 


% \paragraph{\textbf{Key Takeaways}}
% \begin{itemize}
%     \item Out-of-distribution generalization in the parity problem is highly sensitive to the diversity and positional coverage of training tasks.
%     \item Adversarial exclusion of specific pairs or positional configurations can lead to systematic failures, even when most tasks are observed during training.
% \end{itemize}




%################ previous veriosn down
% \textit{How does the choice of training tasks affect the ability of a model to generalize to unseen tasks? Can we predict which setups are likely to lead to failure?}

% To explore these questions, we crafted specific training and test task splits to investigate what makes one setup appear “more” OOD than another.

% \paragraph{Generalization with Missing Pair.}

% Imagine we have tasks constructed from subsets of \(k=3\) elements out of a larger set of \(d\) coordinates. What happens if certain pairs of coordinates are adversarially excluded during training? For example, suppose \(d=5\) and two specific coordinates, \(c_1 = 1\) and \(c_2 = 2\), are excluded. The remaining tasks are formed from subsets of the other coordinates. How would a model perform when tested on tasks involving the excluded pair \( (c_1, c_2) \)? 

% To probe this, we devised three variations in how training tasks are constructed:
%     \begin{enumerate}
%         \item \textbf{Type 1:} The training set includes all tasks except those containing both \( c_1 = 1 \) and \( c_2 = 2 \). 
%         For this example, the training set includes only $\{(3,4,5)\}$. The test set consists of all tasks containing the rest of tuples.

%         \item \textbf{Type 2:} Similar to Type 1, but the training set additionally includes half of the tasks containing either \( c_1 = 1 \) \textit{or} \( c_2 = 2 \) (but not both). 
%         For the example, the training set includes all tasks from Type 1 and adds tasks like \(\{(1, 3, 4), (2, 3, 5)\}\) (half of those containing \( c_1 = 1 \) or \( c_2 = 2 \)).

%         \item \textbf{Type 3:} Similar to Type 2, but the training set also includes half of the tasks containing both \( c_1 = 1 \) \textit{and} \( c_2 = 2 \). 
%         For the example, the training set includes all tasks from Type 2 and adds, for instance, \(\{(1, 2, 5)\}\) (half of the tasks containing both \( c_1 \) and \( c_2 \)).
%     \end{enumerate}

% By systematically increasing the diversity of training tasks in a controlled way, while ensuring no overlap between training and test configurations, we observe an improvement in OOD generalization. 

% % \textit{However, the question is this improvement similar across all coordinate pairs, or does it depend on the specific choices of \(c_1\) and \(c_2\) in the tasks?} 

% \textbf{Before proceeding, consider the following question:} Is the observed improvement consistent across all coordinate pairs, or does it depend on the specific choices of \(c_1\) and \(c_2\) in the tasks? 

% For instance, consider two cases for \(d = 10, k = 3\): (i) \(c_1 = 4, c_2 = 6\) and (ii) \(c_1 = 8, c_2 = 9\). Would you expect similar OOD generalization behavior for these two cases across the three training setups we discussed?



% \paragraph{Answer to the Question.} for both cases of \( c_1, c_2 \), we observe that generalization fails in Type 1, suggesting that the position of the tasks the model has been trained on significantly impacts its generalization capability. For Type 2, we find that \( c_1 = 4, c_2 = 6 \) performs significantly better than \( c_1 = 8, c_2 = 9 \). 

% Upon examining the tasks where the transformer fails for \( c_1 = 8, c_2 = 9 \), we see that the model only fails at tasks of the form \((*, 8, 9)\) while perfectly generalizing to the rest. This indicates that the model has never encountered the value \( 8 \) in the second position during training, which likely explains its failure to generalize. In contrast, for \( c_1 = 4, c_2 = 6 \), while the model has not seen tasks of the form \((*, 4, 6)\), it has encountered tasks where \( 4 \) appears in the second position, such as \((1, 4, 5)\), and tasks where \( 6 \) appears in the third position, such as \((2, 3, 6)\). This difference may explain why the model generalizes almost perfectly in Type 2 for \( c_1 = 4, c_2 = 6 \), but not for \( c_1 = 8, c_2 = 9 \).



% \paragraph{Generalization with Missing Coordinates.}
% Next, we investigate whether a model can generalize to tasks where a specific coordinate appears in an unseen position during training. For instance, consider \( c_1 = 5 \), and exclude all tasks where \( c_1 \) appears in the second position. Despite being trained on all other tasks, the model fails to generalize to these excluded cases, highlighting the importance of positional diversity in training tasks.



% \paragraph{Key Takeaways.}
% \begin{itemize}
%     \item OOD generalization depends heavily on the diversity and positional coverage of training tasks for the parity problem.
%     \item adversarial exclusion of specific pairs or positional configurations in the parity problem can lead to failure, even when the majority of tasks are observed during training.
% \end{itemize}


%################ previous veriosn up

% \paragraph{Key Takeaways} These findings highlight the complexity of OOD generalization, even in seemingly simple tasks like parity. They also underscore the importance of task design: the diversity of training tasks can significantly influence a model’s ability to generalize to unseen tasks. By better understanding these dynamics, we can design more robust training regimes that foster generalization across a wider range of scenarios.


% #############


% Upon examining the tasks where the transformer fails for \( c_1 = 8, c_2 = 9 \), we see that the model only fails at tasks of the form \((*, 8, 9)\) while perfectly generalizing to the rest. This indicates that the model has never encountered the value \( 8 \) in the second position during training, which likely explains its failure to generalize. In contrast, for \( c_1 = 4, c_2 = 6 \), while the model has not seen tasks of the form \((*, 4, 6)\), it has encountered tasks where \( 4 \) appears in the second position, such as \((1, 4, 5)\), and tasks where \( 6 \) appears in the third position, such as \((2, 3, 6)\). This difference may explain why the model generalizes almost perfectly in Type 2 for \( c_1 = 4, c_2 = 6 \), but not for \( c_1 = 8, c_2 = 9 \).

% we observe a striking pattern: generalization fails entirely in Type 1, regardless of the coordinate pair (\(c_1, c_2\)). However, in Type 2, generalization varies: \(c_1 = 4, c_2 = 6\) achieves 96\% accuracy, while \(c_1 = 8, c_2 = 9\) lags behind at 70\%. Why? Upon closer inspection, the model struggles specifically with tasks like \((*, 8, 9)\), where the combination \(c_1 = 8\) and \(c_2 = 9\) is entirely novel. In contrast, for \(c_1 = 4, c_2 = 6\), the model benefits from having seen tasks where \(4\) appears in the second position or \(6\) in the third. This suggests that positional exposure during training plays a key role in generalization.

% To test whether task structure influences generalization, we consider two variations:
% \begin{enumerate}
%     \item \textbf{Sorted Tuples:} Tasks are always sorted in ascending order.
%     \item \textbf{Unsorted Tuples:} Tasks can appear in any order.
% \end{enumerate}

% If the model struggles with generalizing to the excluded position, does introducing variability through unsorted tuples help mitigate this limitation?

% \paragraph{Discussion of Results}

% In \textbf{Generalization with Missing Pairs}, we observe a striking pattern: generalization fails entirely in Type 1, regardless of the coordinate pair (\(c_1, c_2\)). However, in Type 2, generalization varies: \(c_1 = 4, c_2 = 6\) achieves 96\% accuracy, while \(c_1 = 8, c_2 = 9\) lags behind at 70\%. Why? Upon closer inspection, the model struggles specifically with tasks like \((*, 8, 9)\), where the combination \(c_1 = 8\) and \(c_2 = 9\) is entirely novel. In contrast, for \(c_1 = 4, c_2 = 6\), the model benefits from having seen tasks where \(4\) appears in the second position or \(6\) in the third. This suggests that positional exposure during training plays a key role in generalization.

% In \textbf{Generalization with Missing Coordinates}, the results confirm this hypothesis. When \(c_1 = 5\) is excluded from the second position during training, the model fails to generalize to such tasks in the sorted case. However, allowing unsorted tuples introduces positional diversity, leading to near-perfect generalization. This raises an intriguing question: does the model inherently overfit to positional patterns, and can task variability help break this tendency?




% In this subsection, we show that the selection of training tasks can affect the quality of the unseen task generalization significantly in practice. To illustrate this, we present two scenarios where tasks differ between training and testing. For each scenario, we invite the reader to assess, before examining the experimental results, which cases might appear “more” OOD. 

% % \amir{add examples, }

% \kaiyue{I think the name of scenarios here are not very clear}
% \begin{itemize}
%     \item \textbf{Scenario 1:  Generalization Across Excluded Coordinate Pairs (\( k = 3 \))} \\
%     In this scenario, we select two coordinates \( c_1 \) and \( c_2 \) out of \( d \) and construct three types of training sets. 

%     Suppose \( d = 5 \), \( c_1 = 1 \), and \( c_2 = 2 \). The tuples are all possible subsets of \( \{1, 2, 3, 4, 5\} \) with \( k = 3 \):
%     \[
%     \begin{aligned}
%     \big\{ & (1, 2, 3), (1, 2, 4), (1, 2, 5), (1, 3, 4), (1, 3, 5), \\
%            & (1, 4, 5), (2, 3, 4), (2, 3, 5), (2, 4, 5), (3, 4, 5) \big\}.
%     \end{aligned}
%     \]

%     \begin{enumerate}
%         \item \textbf{Type 1:} The training set includes all tuples except those containing both \( c_1 = 1 \) and \( c_2 = 2 \). 
%         For this example, the training set includes only $\{(3,4,5)\}$ tuple. The test set consists of tuples containing the rest of tuples.

%         \item \textbf{Type 2:} Similar to Type 1, but the training set additionally includes half of the tuples containing either \( c_1 = 1 \) \textit{or} \( c_2 = 2 \) (but not both). 
%         For the example, the training set includes all tuples from Type 1 and adds tuples like \(\{(1, 3, 4), (2, 3, 5)\}\) (half of those containing \( c_1 = 1 \) or \( c_2 = 2 \)).

%         \item \textbf{Type 3:} Similar to Type 2, but the training set also includes half of the tuples containing both \( c_1 = 1 \) \textit{and} \( c_2 = 2 \). 
%         For the example, the training set includes all tuples from Type 2 and adds, for instance, \(\{(1, 2, 5)\}\) (half of the tuples containing both \( c_1 \) and \( c_2 \)).
%     \end{enumerate}

% % \begin{itemize}
% %     \item \textbf{Type 1:} The training set includes tuples \(\{1, 3, 4\}, \{2, 3, 4\}\) (excluding tuples with both \( c_1 \) and \( c_2 \): \(\{1, 2, 3\}, \{1, 2, 4\}\)). The test set contains the excluded tuples.
% %     \item \textbf{Type 2:} The training set includes all tuples in Type 1 plus half of the tuples containing either \( c_1 = 1 \) or \( c_2 = 2 \) (e.g., \(\{1, 2, 3\}\)).
% %     \item \textbf{Type 3:} The training set includes all tuples in Type 2 plus half of the tuples containing both \( c_1 = 1 \) and \( c_2 = 2 \) (e.g., \(\{1, 2, 4\}\)).
% % \end{itemize}
    
%     \item \textbf{Scenario 2: Scenario 2: Generalization Across a Fixed Coordinate (\( k = 3 \))} \\
%     In this scenario, we select one coordinate \( c_1 \) out of \( d \) (\( c_1 = 5 \)). The training set includes all task tuples except those where \( c_1 \) is the second coordinate of the tuple. For this scenario, we examine two variations:
%     \begin{enumerate}
%         \item \textbf{Sorted Tuples:} Task tuples are always sorted (e.g., \( (x_1, x_2, x_3) \) with \( x_1 \leq x_2 \leq x_3 \)).
%         \item \textbf{Unsorted Tuples:} Task tuples can appear in any order.
%     \end{enumerate}
% \end{itemize}




% \paragraph{Discussion of Results.} In the first scenario, for both cases of \( c_1, c_2 \), we observe that generalization fails in Type 1, suggesting that the position of the tasks the model has been trained on significantly impacts its generalization capability. For Type 2, we find that \( c_1 = 4, c_2 = 6 \) performs significantly better than \( c_1 = 8, c_2 = 9 \). 

% Upon examining the tasks where the transformer fails for \( c_1 = 8, c_2 = 9 \), we see that the model only fails at tasks of the form \((*, 8, 9)\) while perfectly generalizing to the rest. This indicates that the model has never encountered the value \( 8 \) in the second position during training, which likely explains its failure to generalize. In contrast, for \( c_1 = 4, c_2 = 6 \), while the model has not seen tasks of the form \((*, 4, 6)\), it has encountered tasks where \( 4 \) appears in the second position, such as \((1, 4, 5)\), and tasks where \( 6 \) appears in the third position, such as \((2, 3, 6)\). This difference may explain why the model generalizes almost perfectly in Type 2 for \( c_1 = 4, c_2 = 6 \), but not for \( c_1 = 8, c_2 = 9 \).

% This position-based explanation appears compelling, so in the second scenario, we focus on a single position to investigate further. Here, we find that the transformer fails to generalize to tasks where \( 5 \) appears in the second position, provided it has never seen any such tasks during training. However, when we allow for more task diversity in the unsorted case, the model achieves near-perfect generalization. 

% This raises an important question: does the transformer have a tendency to overfit to positional patterns, and does introducing more task variability, as in the unsorted case, prevent this overfitting and enable generalization to unseen positional configurations?

% These findings highlight that even in a simple task like parity, it is remarkably challenging to understand and quantify the sources and levels of OOD behavior. This motivates further investigation into the nuances of task design and its impact on model generalization.


\subsection{Task Generalization Beyond Parity Problems}

% \begin{figure}[t!]
%     \centering
%     \includegraphics[width=0.45\textwidth]{Figures/arithmetic_v1.png}
%     \vspace{-0.3cm}
%     \caption{Task generalization for arithmetic task with CoT, it has $\d =2$ and $T = d-1$ as the ambient dimension, hence $D\ln(DT) = 2\ln(2T)$. We show that the empirical scaling closely follows the theoretical scaling.}
%     \label{fig:arithmetic}
% \end{figure}



% \begin{wrapfigure}{r}{0.4\textwidth}  % 'r' for right, 'l' for left
%     \centering
%     \includegraphics[width=0.4\textwidth]{Figures/arithmetic_v1.png}
%     \vspace{-0.3cm}
%     \caption{Task generalization for the arithmetic task with CoT. It has $d =2$ and $T = d-1$ as the ambient dimension, hence $D\ln(DT) = 2\ln(2T)$. We show that the empirical scaling closely follows the theoretical scaling.}
%     \label{fig:arithmetic}
% \end{wrapfigure}

\subsubsection{Arithmetic Task}\label{subsec:arithmetic}











We introduce the family of \textit{Arithmetic} task that, like the sparse parity problem, operates on 
\( d \) binary inputs \( b_1, b_2, \dots, b_d \). The task involves computing a structured arithmetic expression over these inputs using a sequence of addition and multiplication operations.
\newcommand{\op}{\textrm{op}}

Formally, we define the function:
\[
\text{Arithmetic}_{S} \colon \{0,1\}^d \to \{0,1,\dots,d\},
\]
where \( S = (\op_1, \op_2, \dots, \op_{d-1}) \) is a sequence of \( d-1 \) operations, each \( \op_k \) chosen from \( \{+, \times\} \). The function evaluates the expression by applying the operations sequentially from left-to-right order: for example, if \( S = (+, \times, +) \), then the arithmetic function would compute:
\[
\text{Arithmetic}_{S}(b_1, b_2, b_3, b_4) = ((b_1 + b_2) \times b_3) + b_4.
\]
% Thus, the sequence of operations \( S \) defines how the binary inputs are combined to produce an integer output between \( 0 \) and \( d \).
% \[
% \text{Arithmetic}_{S} 
% (b_1,\,b_2,\,\dots,b_d)
% =
% \Bigl(\dots\bigl(\,(b_1 \;\op_1\; b_2)\;\op_2\; b_3\bigr)\,\dots\Bigr) 
% \;\op_{d-1}\; b_d.
% \]
% We now introduce an \emph{Arithmetic} task that, like the sparse parity problem, operates on $d$ binary inputs $b_1, b_2, \dots, b_d$. Specifically, we define an arithmetic function
% \[
% \text{Arithmetic}_{S}\colon \{0,1\}^d \;\to\; \{0,1,\dots,d\},
% \]
% where $S = (i_1, i_2, \dots, i_{d-1})$ is a sequence of $d-1$ operations, each $i_k \in \{+,\,\times\}$. The value of $\text{Arithmetic}_{S}$ is obtained by applying the prescribed addition and multiplication operations in order, namely:
% \[
% \text{Arithmetic}_{S}(b_1,\,b_2,\,\dots,b_d)
% \;=\;
% \Bigl(\dots\bigl(\,(b_1 \;i_1\; b_2)\;i_2\; b_3\bigr)\,\dots\Bigr) 
% \;i_{d-1}\; b_d.
% \]

% This is an example of our framework where $T = d-1$ and $|\Theta_t| = 2$ with total $2^d$ possible tasks. 




By introducing a step-by-step CoT, arithmetic class belongs to $ARC(2, d-1)$: this is because at every step, there is only $\d = |\Theta_t| = 2$ choices (either $+$ or $\times$) while the length is  $T = d-1$, resulting a total number of $2^{d-1}$ tasks. 


\begin{minipage}{0.5\textwidth}  % Left: Text
    Task generalization for the arithmetic task with CoT. It has $d =2$ and $T = d-1$ as the ambient dimension, hence $D\ln(DT) = 2\ln(2T)$. We show that the empirical scaling closely follows the theoretical scaling.
\end{minipage}
\hfill
\begin{minipage}{0.4\textwidth}  % Right: Image
    \centering
    \includegraphics[width=\textwidth]{Figures/arithmetic_v1.png}
    \refstepcounter{figure}  % Manually advances the figure counter
    \label{fig:arithmetic}  % Now this label correctly refers to the figure
\end{minipage}

Notably, when scaling with \( T \), we observe in the figure above that the task scaling closely follow the theoretical $O(D\log(DT))$ dependency. Given that the function class grows exponentially as \( 2^T \), it is truly remarkable that training on only a few hundred tasks enables generalization to an exponentially larger space—on the order of \( 2^{25} > 33 \) Million tasks. This exponential scaling highlights the efficiency of structured learning, where a modest number of training examples can yield vast generalization capability.





% Our theory suggests that only $\Tilde{O}(\ln(T))$ i.i.d training tasks is enough to generalize to the rest of unseen tasks. However, we show in Figure \ref{fig:arithmetic} that transformer is not able to match  that. The transformer out-of distribution generalization behavior is not consistent across different dimensions when we scale the number of training tasks with $\ln(T)$. \hongzhou{implicit bias, optimization, etc}
 






% \subsection{Task generalization Beyond parity problem}

% \subsection{Arithmetic} In this setting, we still use the set-up we introduced in \ref{subsec:parity_exmaple}, the input is still a set of $d$ binary variable, $b_1, b_2,\dots,b_d$ and ${Arithmatic_{S}}:\{0,1\}\rightarrow \{0, 1, \dots, d\}$, where $S = (i_1,i_2,\dots,i_{d-1})$ is a tuple of size $d-1$ where each coordinate is either add($+
% $) or multiplication ($\times$). The function is as following,

% \begin{align*}
%     Arithmatic_{S}(b_1, b_2,\dots,b_d) = (\dots(b1(i1)b2)(i3)b3\dots)(i{d-1})
% \end{align*}
    


\subsubsection{Multi-Step Language Translation Task}

 \begin{figure*}[h!]
    \centering
    \includegraphics[width=0.9\textwidth]{Figures/combined_plot_horiz.png}
    \vspace{-0.2cm}
    \caption{Task generalization for language translation task: $\d$ is the number of languages and $T$ is the length of steps.}
    \vspace{-2mm}
    \label{fig:language}
\end{figure*}
% \vspace{-2mm}

In this task, we study a sequential translation process across multiple languages~\cite{garg2022can}. Given a set of \( D \) languages, we construct a translation chain by randomly sampling a sequence of \( T \) languages \textbf{with replacement}:  \(L_1, L_2, \dots, L_T,\)
where each \( L_t \) is a sampled language. Starting with a word, we iteratively translate it through the sequence:
\vspace{-2mm}
\[
L_1 \to L_2 \to L_3 \to \dots \to L_T.
\]
For example, if the sampled sequence is EN → FR → DE → FR, translating the word "butterfly" follows:
\vspace{-1mm}
\[
\text{butterfly} \to \text{papillon} \to \text{schmetterling} \to \text{papillon}.
\]
This task follows an \textit{AutoRegressive Compositional} structure by itself, specifically \( ARC(D, T-1) \), where at each step, the conditional generation only depends on the target language, making \( D \) as the number of languages and the total number of possible tasks is \( D^{T-1} \). This example illustrates that autoregressive compositional structures naturally arise in real-world languages, even without explicit CoT. 

We examine task scaling along \( D \) (number of languages) and \( T \) (sequence length). As shown in Figure~\ref{fig:language}, empirical  \( D \)-scaling closely follows the theoretical \( O(D \ln D T) \). However, in the \( T \)-scaling case, we observe a linear dependency on \( T \) rather than the logarithmic dependency \(O(\ln T) \). A possible explanation is error accumulation across sequential steps—longer sequences require higher precision in intermediate steps to maintain accuracy. This contrasts with our theoretical analysis, which focuses on asymptotic scaling and does not explicitly account for compounding errors in finite-sample settings.

% We examine task scaling along \( D \) (number of languages) and \( T \) (sequence length). As shown in Figure~\ref{fig:language}, empirical scaling closely follows the theoretical \( O(D \ln D T) \) trend, with slight exceptions at $ T=10 \text{ and } 3$ in Panel B. One possible explanation for this deviation could be error accumulation across sequential steps—longer sequences require each intermediate translation to be approximated with higher precision to maintain test accuracy. This contrasts with our theoretical analysis, which primarily focuses on asymptotic scaling and does not explicitly account for compounding errors in finite-sample settings.

Despite this, the task scaling is still remarkable — training on a few hundred tasks enables generalization to   $4^{10} \approx 10^6$ tasks!






% , this case, we are in a regime where \( D \ll T \), we observe  that the task complexity empirically scales as \( T \log T \) rather than \( D \log T \). 


% the model generalizes to an exponentially larger space of \( 2^T \) unseen tasks. In case $T=25$, this is $2^{25} > 33$ Million tasks. This remarkable exponential generalization demonstrates the power of structured task composition in enabling efficient generalization.


% In the case of parity tasks, introducing CoT effectively decomposes the problem from \( ARC(D^T, 1) \) to \( ARC(D, T) \), significantly improving task generalization.

% Again, in the regime scaling $T$, we again observe a $T\log T$ dependency. Knowing that the function class is scaling as $D^T$, it is remarkable that training on a few hundreds tasks can generalize to $4^{10} \approx 1M$ tasks. 





% We further performed a preliminary investigation on a semi-synthetic word-level translation task to show that (1) task generalization via composition structure is feasible beyond parity and (2) understanding the fine-grained mechanism leading to this generalization is still challenging. 
% \noindent
% \noindent
% \begin{minipage}[t]{\columnwidth}
%     \centering
%     \textbf{\scriptsize In-context examples:}
%     \[
%     \begin{array}{rl}
%         \textbf{Input} & \hspace{1.5em} \textbf{Output} \\
%         \hline
%         \textcolor{blue}{car}   & \hspace{1.5em} \textcolor{red}{voiture \;,\; coche} \\
%         \textcolor{blue}{house} & \hspace{1.5em} \textcolor{red}{maison \;,\; casa} \\
%         \textcolor{blue}{dog}   & \hspace{1.5em} \textcolor{red}{chien \;,\; perro} 
%     \end{array}
%     \]
%     \textbf{\scriptsize Query:}
%     \[
%     \begin{array}{rl}
%         \textbf{Input} & \textbf{Output} \\
%         \hline
%         \textcolor{blue}{cat} & \hspace{1.5em} \textcolor{red}{?} \\
%     \end{array}
%     \]
% \end{minipage}



% \begin{figure}[h!]
%     \centering
%     \includegraphics[width=0.45\textwidth]{Figures/translation_scale_d.png}
%     \vspace{-0.2cm}
%     \caption{Task generalization behavior for word translation task.}
%     \label{fig:arithmetic}
% \end{figure}


\vspace{-1mm}
\section{Conclusions}
% \misha{is it conclusion of the section or of the whole paper?}    
% \amir{The whole paper. It is very short, do we need a separate section?}
% \misha{it should not be a subsection if it is the conclusion the whole thing. We can just remove it , it does not look informative} \hz{let's do it in a whole section, just to conclude and end the paper, even though it is not informative}
%     \kaiyue{Proposal: Talk about the implication of this result on theory development. For example, it calls for more fine-grained theoretical study in this space.  }

% \huaqing{Please feel free to edit it if you have better wording or suggestions.}

% In this work, we propose a theoretical framework to quantitatively investigate task generalization with compositional autoregressive tasks. We show that task to $D^T$ task is theoretically achievable by training on only $O (D\log DT)$ tasks, and empirically observe that transformers trained on parity problem indeed achieves such task generalization. However, for other tasks beyond parity, transformers seem to fail to achieve this bond. This calls for more fine-grained theoretical study the phenomenon of task generalization specific to transformer model. It may also be interesting to study task generalization beyond the setting of in-context learning. 
% \misha{what does this add?} \amir{It does not, i dont have any particular opinion to keep it. @Hongzhou if you want to add here?}\hz{While it may not introduce anything new, we are following a good practice to have a short conclusion. It provides a clear closing statement, reinforces key takeaways, and helps the reader leave with a well-framed understanding of our contributions. }
% In this work, we quantitatively investigate task generalization under autoregressive compositional structure. We demonstrate that task generalization to $D^T$ tasks is theoretically achievable by training on only $\tilde O(D)$ tasks. Empirically, we observe that transformers trained indeed achieve such exponential task generalization on problems such as parity, arithmetic and multi-step language translation. We believe our analysis opens up a new angle to understand the remarkable generalization ability of Transformer in practice. 

% However, for tasks beyond the parity problem, transformers appear to fail to reach this bound. This highlights the need for a more fine-grained theoretical exploration of task generalization, especially for transformer models. Additionally, it may be valuable to investigate task generalization beyond the scope of in-context learning.



In this work, we quantitatively investigated task generalization under the autoregressive compositional structure, demonstrating both theoretically and empirically that exponential task generalization to $D^T$ tasks can be achieved with training on only $\tilde{O}(D)$ tasks. %Our theoretical results establish a fundamental scaling law for task generalization, while our experiments validate these insights across problems such as parity, arithmetic, and multi-step language translation. The remarkable ability of transformers to generalize exponentially highlights the power of structured learning and provides a new perspective on how large language models extend their capabilities beyond seen tasks. 
We recap our key contributions  as follows:
\begin{itemize}
    \item \textbf{Theoretical Framework for Task Generalization.} We introduced the \emph{AutoRegressive Compositional} (ARC) framework to model structured task learning, demonstrating that a model trained on only $\tilde{O}(D)$ tasks can generalize to an exponentially large space of $D^T$ tasks.
    
    \item \textbf{Formal Sample Complexity Bound.} We established a fundamental scaling law that quantifies the number of tasks required for generalization, proving that exponential generalization is theoretically achievable with only a logarithmic increase in training samples.
    
    \item \textbf{Empirical Validation on Parity Functions.} We showed that Transformers struggle with standard in-context learning (ICL) on parity tasks but achieve exponential generalization when Chain-of-Thought (CoT) reasoning is introduced. Our results provide the first empirical demonstration of structured learning enabling efficient generalization in this setting.
    
    \item \textbf{Scaling Laws in Arithmetic and Language Translation.} Extending beyond parity functions, we demonstrated that the same compositional principles hold for arithmetic operations and multi-step language translation, confirming that structured learning significantly reduces the task complexity required for generalization.
    
    \item \textbf{Impact of Training Task Selection.} We analyzed how different task selection strategies affect generalization, showing that adversarially chosen training tasks can hinder generalization, while diverse training distributions promote robust learning across unseen tasks.
\end{itemize}



We introduce a framework for studying the role of compositionality in learning tasks and how this structure can significantly enhance generalization to unseen tasks. Additionally, we provide empirical evidence on learning tasks, such as the parity problem, demonstrating that transformers follow the scaling behavior predicted by our compositionality-based theory. Future research will  explore how these principles extend to real-world applications such as program synthesis, mathematical reasoning, and decision-making tasks. 


By establishing a principled framework for task generalization, our work advances the understanding of how models can learn efficiently beyond supervised training and adapt to new task distributions. We hope these insights will inspire further research into the mechanisms underlying task generalization and compositional generalization.

\section*{Acknowledgements}
We acknowledge support from the National Science Foundation (NSF) and the Simons Foundation for the Collaboration on the Theoretical Foundations of Deep Learning through awards DMS-2031883 and \#814639 as well as the  TILOS institute (NSF CCF-2112665) and the Office of Naval Research (ONR N000142412631). 
This work used the programs (1) XSEDE (Extreme science and engineering discovery environment)  which is supported by NSF grant numbers ACI-1548562, and (2) ACCESS (Advanced cyberinfrastructure coordination ecosystem: services \& support) which is supported by NSF grants numbers \#2138259, \#2138286, \#2138307, \#2137603, and \#2138296. Specifically, we used the resources from SDSC Expanse GPU compute nodes, and NCSA Delta system, via allocations TG-CIS220009. 

\section{Open Questions}

\subsection{Balancing Diversity and Relevance in Pairs}
Diversity can be defined as the variation in data samples and pairs used in training while relevance is the level of semantic alignment between the pairs. Ideally, there should be high semantic alignment across pairs of the same category and low alignment across positive and negative categories.

Balancing diversity and relevance is a critical challenge. Both factors are essential: diversity ensures robustness and generalization, while semantic alignment guarantees meaningful and task-relevant representations. Diverse positive pairs may include instances that, while related, are semantically weakly aligned and hence irrelevant, leading to noisy training process and embeddings. Conversely, tightly aligned pairs might be more semantically aligned and relevant but miss significant variations and lack diversity.
Diversity is preferred when the downstream task is unknown, while relevance is prioritized when strong task alignment is needed. 
Future research can focus on dynamic training approaches---
Should early phases focus on introducing diversity to learn general features, followed by more task-aligned features? Or should specialized features be learned first from task-relevant data, and learning generalizable features relegated to later phases?
% with early phases focusing on relevance to learn core features, while later phases introducing diversity for better generalization. 
%Defining positives based on similarity scores rather than binary labels also allows partially aligned pairs to contribute proportionally, ensuring an optimal balance.

\subsection{Dealing with Emerging Modalities in Contrastive Learning}

As new modalities—such as LiDAR, hyperspectral imaging, and haptic feedback, become prominent in various applications, the challenge of integrating these modalities into contrastive learning frameworks emerges \cite{dai2024advancing}. 
Many emerging modalities suffer from the lack of large-scale labeled or even unlabeled datasets. They often exhibit high noise levels or variability due to environmental factors or inherent measurement inaccuracies. Moreover, unlike text or images, pre-trained models for emerging modalities are rare, making initialization more challenging. Future research can focus on scalable and efficient strategies to handle the diversity and complexity of these new data types.

% \clearpage


\newpage
%% The file named.bst is a bibliography style file for BibTeX 0.99c
\bibliographystyle{named}
\bibliography{ijcai25}

\end{document}

