\section{Related Work}
\label{sec:related}
In the research on data pricing~\cite{ref1}, the unique characteristics of data products—namely, their extremely low replication and distribution costs—significantly reduce the effectiveness of traditional pricing strategies for these products. The core idea to address this issue is to link the price of a product to its value, correlating the product's pricing with the buyer's perceived importance of the information. For instance, different versions can be created~\cite{ref11} to meet the varying needs of buyers. Paid predictive APIs represent a rapidly growing industry and form an essential component of machine learning as a service~\cite{ref50}. Based on this, pricing strategies can be determined by two key factors: first, the importance of the information being sold, and second, the degree to which different buyers value the product. The method of version control has already been extensively researched in the context of data products, such as through relational data set queries~\cite{ref12}. In the process of data pricing and trading, besides the challenges posed by the inherent characteristics of data, it is essential to ensure smooth transactions and maximize the interests of all parties involved. In this regard, Kushal et al.~\cite{ref10} defined the ideal attributes that data products should possess during transactions. Ren et al.~\cite{ref59} summarized the data pricing models in various types of data markets. Zhang et al.~\cite{ref64} conducted a survey on data pricing models within game theory and auctions. Pei et al.~\cite{ref61} were among the first to explore the governance of data assets in collaborative artificial intelligence. Furthermore, Pei~\cite{ref20} proposed a set of standard guidelines that all parties involved in data transactions should follow. These guidelines cover principles such as integrity, fairness, no arbitrage, profit maximization, privacy protection, computational efficiency, and rationality of participants.

With the growing prominence of large models and deep learning, which rely heavily on vast amounts of high-quality datasets, Yang et al.~\cite{ref62} analyzed the impact of data quality on big data analytics from a data science perspective and defined a utility function for data quality. In the context of machine learning marketplaces, recent research and practices have successfully commoditized data in various ways. Data markets sell data either directly or indirectly, yet current pricing mechanisms vary due to different application scenarios~\cite{ref68, ref72, ref70}. Data trading markets can currently be classified according to their pricing mechanisms and the types of data sold.

\textit{Pricing Based on Raw Data:} 
Pricing strategies based on datasets allow data markets to sell datasets and grant buyers access to raw datasets, such as those from Dawex, Twitter, Bloomberg, Iota, and SafeGraph. Traditional approaches price datasets as indivisible units, with inherent attributes, such as quantity, serving as key factors determining the price.

\textit{Query-Based Pricing:} 
Query-based pricing models enable data buyers to purchase datasets of interest. Koutris et al.~\cite{ref27} were the first to propose query-based data pricing, establishing a corresponding transaction framework that sets prices for arbitrary queries while ensuring the no-arbitrage principle. In subsequent work, Koutris et al.~\cite{ref18} redesigned and improved the query market framework, addressing the limitations of simple query pricing. They optimized the complex computations involved in processing large numbers of SQL queries into an integer linear programming problem, enhancing computational efficiency. In addition to view-based pricing, Tang et al.~\cite{ref28} introduced a minimum-source pricing model for tuples. Shen et al.~\cite{ref29} conducted research on personal data attributes and established a personal data pricing platform from the tuple perspective. Niu et al.~\cite{ref66} proposed a context-aware dynamic pricing mechanism with a low-price constraint, which maximizes cumulative revenue by setting reasonable prices for sequential queries and achieves efficient online optimization.

\textit{Model-Based Pricing:} 
Many scenarios today require machine learning models, and many users or companies do not build machine learning models from scratch but rather purchase pre-trained models. Chen et al.~\cite{ref30} proposed a machine learning model marketplace based on the no-arbitrage principle and the revenue-maximization principle, where model owners sell multiple versions of machine learning models to different buyers. Liu et al.~\cite{ref31} introduced an end-to-end model marketplace that accounts for privacy compensation for data sellers and the needs of model buyers. Lin et al.~\cite{ref37} used knowledge graph methods to specifically assess the value of data in models and provided a calculation approach for this evaluation.

\textit{Pricing Based on Data Quality:} 
In data transactions, higher-quality data holds greater value for buyers. Determining the quality of data products is a key focus in data markets. Regarding data quality evaluation, Heckman et al.~\cite{ref32} identified a series of factors for assessing dataset quality and proposed a linear model for pricing based on data quality. Yu et al.~\cite{ref33} explored pricing methods based on multiple quality dimensions of data in a monopoly market and designed a transaction model involving a data market and several data buyers. Ding et al.~~\cite{ref63} developed a fair data pricing evaluation mechanism aimed at meeting the demands of both supply and demand sides. In addressing data quality issues in data markets, they integrated key factors such as accuracy, completeness, consistency, and timeliness into the pricing of data.

\textit{Pricing Based on Privacy Protection:} 
During data transactions, data buyers may infer the privacy of data providers based on the characteristics of the data itself or through model-based inferences. Jiang et al.~\cite{ref34} proposed a market framework for trading private data generators under differential privacy using GANs. In this framework, the trading platform charges query fees to data buyers while compensating data providers for privacy. Li et al.~\cite{ref35} developed an incentive-compatible mechanism from the perspective of data providers to price their data, investigating the privacy issues that may arise when data owners release data with unclear usage purposes, which could be distributed by third-party users. Yu et al.~\cite{ref41} used a matching-based Markov decision process to model multi-round data transactions involving gradually disclosed information, introducing a social welfare-optimized data pricing mechanism to identify the best pricing strategy. Feng et al.~\cite{ref75} proposed a privacy-aware personalized data transaction method based on contract theory, offering a set of optimal contracts with varying levels of privacy protection and data transaction prices for self-interested data owners.

\textit{Pricing Based on Task:} 
When pricing data labels, "golden tasks" can be introduced to incentivize pricing. Shah et al.~\cite{ref36} developed a golden task pricing method, which mixes golden tasks—tasks where the data buyer knows the answers and their purpose—with regular tasks, and then assigns them to workers. Since workers cannot distinguish between golden tasks and regular tasks, this method can be used to assess workers' performance and effectively incentivize them to provide accurate labels.

However, current research on data pricing primarily focuses on raw data or query-based approaches. These pricing methods often consider only specific characteristics of the data itself or the demand-supply dynamics of the buyer-seller relationship, leading to an inability to accurately measure the true value of data. Additionally, many existing data pricing models are based on competitive scenarios, where buyers select a seller from multiple options. However, these models do not address scenarios where data sellers possess monopolistic datasets. Data quality, a key factor influencing the value of data, plays an essential role in data-driven model training, such as in machine learning and large model applications. High-quality data is crucial for improving model accuracy. Currently, data quality evaluation tends to focus on the data itself or the buyer's needs, with some approaches using Shapley values to assess the utility of data in model pricing. However, these methods are often limited and fail to integrate the buyer's utility with data quality characteristics in their evaluations.