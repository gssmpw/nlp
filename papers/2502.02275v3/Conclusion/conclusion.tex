\section{Recommendation \& conclusion}\label{sec:Recommendation}

In this paper, we have studied several sampling strategies on the sphere 
for computing an {approximation} of the Sliced Wasserstein distance.

Regarding theoretical guarantees, this study highlighted the following limitations. {The classical i.i.d. sampling benefits from theoretical guarantees with a convergence rate in $O(1/\sqrt{M})$ and a time complexity linear in the number $M$ of projections. Orthonormal sampling and  L.D.S. such as Halton or Sobol lack convergence rate guarantees on the sphere (these guarantees being only obtained for sequences on hypercubes for L.D.S).} As for deterministic point generation methods (like Riesz), the Sliced Wasserstein integrand also lacks sufficient regularity to guarantee results in dimensions higher than 2.


{While lacking theoretical guarantees in terms of convergence, the experimental study  suggests that Q.M.C methods (L.D.S. or s-Riesz points) provide competitive results in small to intermediate dimension, while having a similar convergence rate to classical random sampling methods in intermediate to higher (for Riesz) dimensions. These results seem to indicate that, while $f$ 
is not regular enough for the convergence guarantees detailed in this paper, there may be some non-proven convergence results 
requiring weaker regularity conditions that would be applicable to $SW$.} 

{Now, considering computation times, as shown by~\autoref{fig:comparisonTimerRateSWIncludedGauss} and~\autoref{tab:summaryConvergenceAndComplexity},  classical i.i.d. sampling remains the slowest method in all our experiments. While orthonormal sampling lacks theoretical guarantees, it seems to be one of the most  efficient methods whatever the dimension, and is particularly competitive in high dimensions, with a very reasonable increase of computation time. L.D.S. methods also remain competitive in pratice for small dimensions.
s-Riesz points, while competitive in terms of convergence rate, have a prohibitive time complexity in $O(M^2)$, which makes them completely unsuitable for a large number of projections.} 

{The experiments also suggest that the S.H.C.V. method is very competitive in intermediate dimensions, while becoming less efficient when $d$ increases. }

{Based on the different experimental results provided in this paper, we make the following recommendations:
\begin{itemize}
\item For small dimensions (less than 3), Q.M.C. methods such as s-Riesz points or L.D.S. mapped onto the sphere can be privileged with respect to uniform sampling,
\item For high dimensions (greater than 20), the orthonormal sampling method emerges as the most suitable choice. {It is also one of the simplest methods to implement, which makes it particularly attractive in practice.}
\item For intermediate dimensions (between 5 and 10), choosing an appropriate method should depend on the experimental requirements. 
Spherical harmonics are an excellent option if computational resources are limited and if the number of $SW$ distances to be computed is low. 
However, it is worth noting that some Q.M.C. strategies, being independent of the 
input measures, have the advantage of allowing the generated points to be 
reused and of {allowing an independent computation in $M$ (except the Riesz points)}. This should be particularly beneficial when a high number of projections is required and a large number of $SW$ distances must be computed. In such cases, we suggest to store the samples to factorize the computing time across experiments.
\end{itemize}
}
