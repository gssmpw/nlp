\section{Related work}
APT detection has long relied on a combination of rule-based approaches, signature matching, and heuristic analysis \cite{che2024systematic}. Classical solutions frequently inspect network traffic, system calls, or event logs to identify suspicious patterns. However, attackers often disguise APT behavior within normal operating system activities, rendering purely signature-based methods insufficient. In the context of stealthy campaigns, conventional detection frameworks often fail to capture low-frequency or polymorphic events leading to data exfiltration. Recent works have introduced machine learning techniques—ranging from supervised classifiers to anomaly detection algorithms—aimed at capturing novel or stealthy behaviors \cite{koufakou_2007},\cite{smets2011},\cite{narita_2008},\cite{BerradaCBMMTW20},\cite{Benabderrahmane21},\cite{DBLP:journals/fgcs/BenabderrahmaneHVCR24}. For example, intrusion detection systems (IDS) increasingly adopt textual, time series or graph-based models that encode dependencies between system events. Despite these advancements, the sophistication of APT tactics continues to outpace purely traditional approaches.\\
In recent years, pre-trained LLMs have demonstrated remarkable capability in capturing semantic and contextual nuances across vast domains. Although they have primarily excelled in natural language processing tasks such as translation, summarization, and question-answering, researchers have begun to apply LLM-based embeddings to cybersecurity problems as well \cite{chen2024survey}. By transforming logs or system traces into textual narratives, LLMs can learn richer representations of process behaviors. These embeddings allow anomaly detection algorithms to isolate subtle deviations indicative of malicious activity.
%
Despite their potential, these methods are still in the early stages, facing challenges such as high computational cost and the need for extensive domain adaptation \cite{bayer2024cysecbert}.\\
Although LLMs have shown promise in detecting anomalies, most studies remain in the proof-of-concept stage, lacking large-scale evaluations on real data or heterogeneous operating systems (OS). In addition, limited attention has been paid to the interpretability of LLM-derived features, which are critical in high-stakes environments, where security analysts need clear rationales for flagged anomalies. To address these gaps, our work proposes a comprehensive pipeline for APT detection that fuses LLM-based embeddings with robust anomaly detection, systematically evaluated across multiple OS datasets and real-world threat scenarios.
%