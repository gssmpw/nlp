\section{Related Works}
Logistics and warehouse management have become crucial components of the supply chain, with businesses demanding more efficient delivery solutions. In recent years, swarms of heterogeneous and homogeneous agents have been employed for this purpose. 

Batinovic et al. \cite{APF_LiDAR} utilized the APF method to address the path planning challenge for aerial robots operating in an unknown environment, focusing on ensuring safe trajectory execution and avoiding intricate obstacles using a LiDAR sensor. In a similar vein, Yu et al. \cite{Yu_2023} proposed an innovative distributed control algorithm that integrates the APF technique within a virtual leader formation scheme, coupled with a switching communication network.
Malopolski et al. \cite{GroundRobot} presented an autonomous mobile robot for transport tasks in warehouses; a drive mechanism was proposed for surface and rail navigation and an elevator for vertical movement. However, the lack of aerial capabilities limits its ability to access hard-to-reach areas and navigate dense environments, reducing adaptability in multi-level spaces.
Zhura et al. \cite{Zhura_2023} investigated the impact of UAV in heterogeneous mapping and the navigation of a quadrupled robot. Sales et al. \cite{Sales_2023} developed a highly-scalable and low-cost multirobot system for inventory management composed of pairs with a micro-UAV and a ground mobile robot. Castro et al. \cite{UAV-UGVs} proposed the strategy to assist the cooperation of a heterogeneous robot team that involves two UGVs and one UAV. The robots operate in a partially known dynamic environment where they exchange information among themselves and perform their task of aerial and ground inspections.

Khan et al. \cite{SwarmPath} proposed a leader-follower approach using an APF path planner and impedance controller for a multi-drone homogeneous system, allowing agents to plan the path and navigate to the target in an unknown yet static environment. However, this approach could not work in dynamic environment scenarios. Additionally, this research lacks the ability to operate over extended periods due to the limited battery life and flight time of the drones.
To overcome the individual limitations of aerial and ground robots, heterogeneous swarm systems have been used. Chen et al. \cite{Chen_2025} target search and navigation and design a heterogeneous robot system consisting of a UAV and a UGV for search and rescue missions in unknown environments. 

The HetSwarm system, inspired by the previous work of SwarmPath \cite{SwarmPath} and SwarmGear \cite{SwarmGear}, introduces a new agile and safe path for heterogeneous systems in dynamic and cluttered environments.