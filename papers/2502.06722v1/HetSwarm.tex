%%%%%%%%%%%%%%%%%%%%%%%%%%%%%%%%%%%%%%%%%%%%%%%%%%%%%%%%%%%%%%%%%%%%%%%%%%%%%%%%
%2345678901234567890123456789012345678901234567890123456789012345678901234567890
%     1         2         3         4         5         6         7         8

\documentclass[letterpaper, 10 pt, conference]{ieeeconf}  % Comment this line out if you need a4paper

%\documentclass[a4paper, 10pt, conference]{ieeeconf}      % Use this line for a4 paper

\IEEEoverridecommandlockouts                              % This command is only needed if 
                                                          % you want to use the \thanks command

\overrideIEEEmargins                                      % Needed to meet printer requirements.

%In case you encounter the following error:
%Error 1010 The PDF file may be corrupt (unable to open PDF file) OR
%Error 1000 An error occurred while parsing a contents stream. Unable to analyze the PDF file.
%This is a known problem with pdfLaTeX conversion filter. The file cannot be opened with acrobat reader
%Please use one of the alternatives below to circumvent this error by uncommenting one or the other
%\pdfobjcompresslevel=0
%\pdfminorversion=4

% See the \addtolength command later in the file to balance the column lengths
% on the last page of the document

% The following packages can be found on http:\\www.ctan.org
\usepackage{graphicx}
\usepackage{multirow}
\usepackage{algorithm}
\usepackage{subcaption}
\usepackage{soul}
\usepackage{comment}
\usepackage[noend]{algpseudocode}
\usepackage{afterpage}
\usepackage{array}
\usepackage{amsmath}
\usepackage{amsmath} 
\usepackage{balance}
% \usepackage{graphics} % for pdf, bitmapped graphics files
%\usepackage{epsfig} % for postscript graphics files
%\usepackage{mathptmx} % assumes new font selection scheme installed
%\usepackage{times} % assumes new font selection scheme installed
%\usepackage{amsmath} % assumes amsmath package installed
%\usepackage{amssymb} % assumes amsmath package installed

\newcolumntype{C}{>{\centering \arraybackslash}m{0.1\textwidth}}

\title{\LARGE \bf
% LinkedUpSwarm / HeteroSwarmLink/ HeteroSwarmPath/HeteroSwarmNav/HeteroNav/: 
HetSwarm: Cooperative Navigation of Heterogeneous Swarm in Dynamic and Dense Environments through Impedance-based Guidance
}

\author{Malaika Zafar\textsuperscript{1}, Roohan Ahmed Khan\textsuperscript{1}, Aleksey Fedoseev\textsuperscript{1}, Kumar Katyayan Jaiswal\textsuperscript{2}, and Dzmitry Tsetserukou\textsuperscript{1}% 
\thanks{\textsuperscript{1}The authors are with the Intelligent Space Robotics Laboratory, Center for Digital Engineering, Skolkovo Institute of Science and Technology, Moscow, Russia. 
\tt \{malaika.zafar, roohan.khan, aleksey.fedoseev, d.tsetserukou\}@skoltech.ru}
\thanks{\textsuperscript{2}The author is with the Multi-Robot Autonomy Laboratory, Center for Digital Engineering, IISER Bhopal, Bhopal, India. 
\tt kumar20@iiserb.ac.in}
}
%
% \thanks{$^{2}$Bernard D. Researcheris with the Department of Electrical Engineering, Wright State University,
% Dayton, OH 45435, USA
% {\tt\small b.d.researcher@ieee.org}}%
% }


\begin{document}

\maketitle
\thispagestyle{empty}
\pagestyle{empty}


%%%%%%%%%%%%%%%%%%%%%%%%%%%%%%%%%%%%%%%%%%%%%%%%%%%%%%%%%%%%%%%%%%%%%%%%%%%%%%%%
\begin{abstract}
With the growing demand for efficient logistics and warehouse management, unmanned aerial vehicles (UAVs) are emerging as a valuable complement to automated guided vehicles (AGVs). UAVs enhance efficiency by navigating dense environments and operating at varying altitudes. However, their limited flight time, battery life, and payload capacity necessitate a supporting ground station. To address these challenges, we propose HetSwarm, a heterogeneous multi-robot system that combines a UAV and a mobile ground robot for collaborative navigation in cluttered and dynamic conditions. Our approach employs an artificial potential field (APF)-based path planner for the UAV, allowing it to dynamically adjust its trajectory in real time. The ground robot follows this path while maintaining connectivity through impedance links, ensuring stable coordination. Additionally, the ground robot establishes temporal impedance links with low-height ground obstacles to avoid local collisions, as these obstacles do not interfere with the UAV's flight. 

Experimental validation of HetSwarm in diverse environmental conditions demonstrated a 90\% success rate across 30 test cases. The ground robot exhibited an average deviation of 45 cm near obstacles, confirming effective collision avoidance. Compared to the Conflict-Based Search (CBS) algorithm, our approach enables agents to navigate within 25 cm of obstacles, whereas CBS maintains a minimum clearance of 73 cm, highlighting our method’s efficiency in utilizing space in real-time. Extensive simulations in the Gym PyBullet environment further validated the robustness of our system for real-world applications, demonstrating its potential for dynamic, real-time task execution in cluttered environments.

% Moreover, HetSwarm exhibited 90\% success rate for 30 different test cases that were performed with varying environmental conditions. Additionally, a small deviation of about 45 cm was observed in the ground robot's path, particularly near the ground obstacles region, indicating that the robot is effectively avoiding them. Compared to the Conflict-Based Search (CBS) algorithm, our approach enables agents to safely yet optimally navigate within 25 cm of obstacles, whereas CBS maintains a minimum distance of 73 cm. The experimental results revealed that the continuous real-time coordination between the drone and the ground robot provides a robust solution to task execution in dynamic environments. Extensive simulations using a Gym PyBullet environment were conducted to validate the approach, considering specific real-world situations in order to demonstrate its functionality in addressing challenges in dynamic environments. 

\end{abstract}

{Keywords: Heterogeneous Robots, Leader-Follower Connectivity, Dynamic Environments, Path Planning, Artificial Potential Fields, Adaptive Systems, Impedance Control, Formation Control}

\begin{figure}[htbp]
\centering
\vspace{0.2cm}
\includegraphics[width=0.95\linewidth]{HeteroImages/1_landscape.png} 
\caption{HetSwarm generates paths in a dense environment where the UAV (black line) navigates towards the goal and the mobile robot (blue dashed line) follows the UAV by maintaining an impedance link connection (red dashed line).}
\vspace{-0.4cm}
\label{exp_setup_1}
\vspace{-0.2cm}

\end{figure}

%%%%%%%%%%%%%%%%%%%%%%%%%%%%%%%%%%%%%%%%%%%%%%%%%%%%%%%%%%%%%%%%%%%%%%%%%%%%%%%%
\section{Introduction}
With the development of automation and artificial intelligence, the logistics and warehouse management sectors have experienced a significant shift toward more efficient and automated solutions capable of decreasing the detection errors during inventory \cite{Lopez_2023}. The tasks of indoor and last-mile delivery commonly relied on AGVs and mobile robots, for example, systems proposed by Paolanti et al. \cite{Paolanti_2017} and Motroni et al. \cite{Motroni_2024}, due to their high payload capacity and localization precision. 

However, UAVs have received wide attention in logistics due to their advantages of fast speed and efficient vertical navigation. For example, an inventory management system incorporating a swarm of mini-drones proposed by Cristiani et al. \cite{Cristiani_2020} with a generic architecture for UAV-based inventory management. The segmentation model for precise position estimation of objects in inventory was proposed by Yoon et al. \cite{Yoon_2023}. 
More recently, heterogeneous systems were proposed to leverage the benefits of UAV mobility and the precise positioning of ground robots. For example, a team of mobile and aerial robots was proposed by Kalinov et al. \cite{Kalinov_2020} for real-time barcode detection and scanning using Convolutional Neural Networks (CNN). The designed approach improved the UAV's localization using scanned barcodes and ground stations as landmarks in a real warehouse with low-light conditions. 

% Previous research on swarm formation control and path planning has been done on homogeneous swarms.

% Researchers proposed a leader-follower approach utilizing APF and an impedance controller for guiding a swarm of homogeneous drone to target positions while avoiding obstacles \cite{SwarmPath}. However, heterogeneous swarms of agents outperform homogeneous ones because of the complementary strengths of different agents. One of the research projects, SwarmGear \cite{SwarmGear}, utilizes a heterogeneous swarm that investigates the environment in a leader-follower formation. The leader drone is able to land on rough terrain, possessing both the functionalities of an aerial and mobile robot. However, in this approach, swarm formation may deviate from its desired path since no controller is present. Therefore, when the leader agent operates as the aerial robot, it may face difficulties in tough weather conditions. 

While heterogeneous may improve both the precision and time of logistics, their navigation suffers from the additional density of the swarm morphology. 
% The ground robot poses limited mobility on the surface, while the aerial robot should maintain connectivity with it for collaborative mapping and emergency landing in case of a low battery. 
This paper introduces a HetSwarm setup that combines a drone and a ground robot in a leader-follower configuration, with the drone acting as the leader while the ground robot follows them through impedance link \cite{Hogan_1984} connectivity. 

In HetSwarm, the drone uses an APF path planner to navigate to target positions while avoiding obstacles in a highly dense environment. Moreover, the ground robot maintains its connection with the drone using impedance linkages. Additionally, the ground robot makes additional impedance links with low-height obstacles that are out of the range of the drone in order to work effectively in an obstacle-dense environment. The custom PID Path Follower was made for the ground robot so that it can follow the intended path with fewer deviations. This collaborative system is also capable of functioning in dynamic environments. 

Furthermore, the heterogeneous setup significantly enhances efficiency for logistics tasks. A drone
has the ability to carry only a limited payload due to its size and battery constraints whereas the ground robot can carry heavier payloads, thereby improving overall performance. Additionally, if the drone’s battery is depleted, it can land on the ground robot for recharging, ensuring uninterrupted operation.

\section{Related Works}

Logistics and warehouse management have become crucial components of the supply chain, with businesses demanding more efficient delivery solutions. In recent years, swarms of heterogeneous and homogeneous agents have been employed for this purpose. 

Batinovic et al. \cite{APF_LiDAR} utilized the APF method to address the path planning challenge for aerial robots operating in an unknown environment, focusing on ensuring safe trajectory execution and avoiding intricate obstacles using a LiDAR sensor. In a similar vein, Yu et al. \cite{Yu_2023} proposed an innovative distributed control algorithm that integrates the APF technique within a virtual leader formation scheme, coupled with a switching communication network.
Malopolski et al. \cite{GroundRobot} presented an autonomous mobile robot for transport tasks in warehouses; a drive mechanism was proposed for surface and rail navigation and an elevator for vertical movement. However, the lack of aerial capabilities limits its ability to access hard-to-reach areas and navigate dense environments, reducing adaptability in multi-level spaces.
Zhura et al. \cite{Zhura_2023} investigated the impact of UAV in heterogeneous mapping and the navigation of a quadrupled robot. Sales et al. \cite{Sales_2023} developed a highly-scalable and low-cost multirobot system for inventory management composed of pairs with a micro-UAV and a ground mobile robot. Castro et al. \cite{UAV-UGVs} proposed the strategy to assist the cooperation of a heterogeneous robot team that involves two UGVs and one UAV. The robots operate in a partially known dynamic environment where they exchange information among themselves and perform their task of aerial and ground inspections.

Khan et al. \cite{SwarmPath} proposed a leader-follower approach using an APF path planner and impedance controller for a multi-drone homogeneous system, allowing agents to plan the path and navigate to the target in an unknown yet static environment. However, this approach could not work in dynamic environment scenarios. Additionally, this research lacks the ability to operate over extended periods due to the limited battery life and flight time of the drones.
To overcome the individual limitations of aerial and ground robots, heterogeneous swarm systems have been used. Chen et al. \cite{Chen_2025} target search and navigation and design a heterogeneous robot system consisting of a UAV and a UGV for search and rescue missions in unknown environments. 

The HetSwarm system, inspired by the previous work of SwarmPath \cite{SwarmPath} and SwarmGear \cite{SwarmGear}, introduces a new agile and safe path for heterogeneous systems in dynamic and cluttered environments. 

\section{Heterogeneous Swarm Technology}

\subsection{System Overview}

\begin{figure*}
      \centering
      \includegraphics[width=0.8\textwidth]{HeteroImages/Arch2.png}
      \caption{Figure shows the overall system architecture and the pipeline of HetSwarm.}
      \label{system arch}
   \end{figure*}

The HetSwarm system in Fig. \ref{system arch} consists of a drone and a ground robot working together in a dynamic environment. The drone generates and continuously updates its path using an APF planner, while the ground robot follows the drone’s path via impedance links. Additionally, these links also help the ground robot to navigate around smaller obstacles since these obstacles are not in the range of a drone. The drone handles navigation while globally avoiding obstacles, while the robot ensures obstacle-free movement while providing a landing platform to the drone in case of reloading or recharging. States of both the agents are controlled by their custom PID controller for accurate path following. This approach enables efficient collaboration in real-time, especially in densely packed dynamic environments. 

\subsection{Artificial Potential Fields for Global Path Generation}
In order for the leader drone to navigate efficiently around the obstacles while setting the path toward the goal, we applied the APF planning algorithm \cite{APF}.The algorithm generates a virtual force that attracts the drone towards the target and a repulsive force that pushes it away from obstacles. An optimized trajectory is generated using the combination of these two forces. The equations for APF planner are as follows \cite{SwarmPath}:
\begin{equation}
F_{\text{total}} = F_{\text{attraction}} + F_{\text{repulsion}} \label{eq:total_force}
\end{equation}
where
\begin{align*}
F_{\text{attraction}}(d_{\text{g}}) &= k_{\text{att}} \cdot d_{\text{g}}, \\
F_{\text{repulsion}}(d_{\text{o}}) &= 
\begin{cases} 
0 & \text{if } d_{\text{o}} > d_{\text{safe}} \\
k_{\text{rep}} \cdot \left( \frac{1}{d_{\text{o}}} - \frac{1}{d_{\text{safe}}} \right) & \text{if } d_{\text{o}} \leq d_{\text{safe}}
\end{cases}
\end{align*}
where $d_{\text{g}}$ and $d_{\text{o}}$ are the distances from drone to goal and to obstacle, respectively, $k_{\text{att}}$ and $k_{\text{rep}}$ are the attraction and repulsion coefficients, respectively.

\subsection{Impedance Controller}
Once the path of the drone was planned, to ensure a smooth connecting mechanism, its connection with the ground robot was established using an impedance controller. These impedance links help the ground robot to stay connected with the leader drone and follow its path toward the target location. 

\subsubsection{Connectivity of Mobile Robot with Drone}
In this configuration, the follower ground robot position is coupled with the leader-drone APF trajectory, which serves as the leading trajectory, through a mass-spring-damper system. Thereby, creating virtual impedance links among the drone and the ground robot. These impedance links provide a smooth connection among the agents. The links are established using a second-order differential equation of mass-spring-damper, which is given as:

\begin{equation}
m\Delta\ddot{x} + d\Delta\dot{x} + k\Delta x = F_{\text{ext}}(t)
\label{eq:dynamic_equation}
\end{equation}
where $\Delta x$ is the difference between the current and desired mobile robot position and $F_{\text{ext}}(t)$ is the virtual external force applied as an input from the leader drone. $m$ is the virtual mass of a link, $d$ is the damping coefficient of the virtual damper, and $k$ is the virtual spring constant. 

\subsubsection{Connectivity of Mobile Robot with Ground Obstacles}
The APF trajectory generated by the leader drone does not account for smaller obstacles. To avoid collisions between the mobile robot and obstacles, we enabled the ground robot to establish additional links with the obstacles while simultaneously disconnecting from the drone, ensuring a collision-free path:

\begin{equation}
\Delta x_{robot,n} = k_{impF} \cdot r_{imp}
\label{eq:deflection_eqn}
\end{equation}
where $r_{imp}$ is the radius of the local deflection region around the obstacle, $k_{impF}$ is the force coefficient adjusted to the ground robot's average velocity, and $n$ is the number of mobile robots. 

\subsection{Dynamic Environment}
HetSwarm introduces a novel approach to path planning and coordination using a heterogeneous robot team with a UAV and a mobile robot. The system adjusts dynamically by continuously updating the drone's APF path based on real-time environmental changes. As obstacles shift, the drone adjusts its trajectory, and the setup allows the mobile robot to adapt to the changes. Additionally, the mobile robot can also dynamically react to obstacles in its path, ensuring it can navigate through unpredictable or cluttered spaces. This adaptability allows HetSwarm to operate efficiently in dynamic environments, ensuring both agents can respond to new challenges as they arise.


\section{Experimental Evaluation}

\subsection{Experimental Setup in GymPybullet Environment}
A custom simulation environment has been developed in the Gym PyBullet environment \ref{exp_gym}, incorporating the dynamics of both a drone and a mobile robot. The environment allows the visualization of multiple obstacles, and the obstacles can be moved dynamically. A custom PID controller has been implemented for the mobile robot that helps it to follow the path by adjusting its velocity and steering angles, while the drone's PID controller ensures stable and controlled drone flight. This setup creates realistic interactions among the drone and mobile robot in dynamic environments, making it suitable for testing heterogeneous robots in real-world-like conditions. The drone utilizes the APF planner to generate and adjust its path in real-time; simultaneously, the mobile robot maintains the connectivity with the drone and ground obstacles using impedance links. 
% This setup promotes realistic interactions between drone, mobile robot, and obstacles, making it ideal for testing heterogeneous robots in real-world scenarios. 
To evaluate the adaptability and robustness of the system, experiments were performed on multiple cases. In a dynamic environment, three cases were performed from sparse to dense scenarios; cases were run thirty times to check the algorithm's robustness and repeatability of performance. Similarly, in static environments, experiments were run under varying conditions to do a detailed analysis.

\begin{figure}[htbp]
\centering
% \vspace{0.2cm}
\includegraphics[width=0.95\linewidth]{HeteroImages/2_landscape.png} 
\caption{Experimental setup in Gym Pybullet environment showing the trajectory of drone and mobile robot under dense environment.}
% \vspace{-0.4cm}
\label{exp_gym}
\vspace{-0.2cm}

\end{figure}

\subsection{Results}
\subsubsection{Results in static environment}
As the leader, the drone generates its path using the APF planner, while the mobile robot follows this path by maintaining impedance links, resulting in a distinct trajectory shown in Fig. \ref{all_cases}. 

\begin{figure*}
\centering
\includegraphics[width=0.7\textwidth]{HeteroImages/all_cases1.jpg}
\caption{Simulation results of two different cases: sparse (a,b) to dense (c,d). Each case is formed with two different initial and target positions. Black dots are the obstacles for both drones and mobile robots, green dots are the ground obstacles deflected only by the mobile robot. Gray and light green-colored circles show the safe deflection regions for both agents.}
\label{all_cases}
\end{figure*}

Furthermore, the mobile robot avoids obstacles encountered along its path, leading to impedance deflections that are also illustrated in the figure. The green circles represent ground-level obstacles, and the graph demonstrates that the drone's trajectory remains straight, passing over these obstacles while the mobile robot effectively navigates around them. Fig. \ref{all_cases}a, Fig. \ref{all_cases}b, and Fig. \ref{all_cases}c represent three cases with the same starting and goal positions of the agent, demonstrating how the algorithm effectively handles both sparse and dense scenarios. Furthermore, by changing the initial and target positions, different path planning scenarios can be tested, as shown in Fig. \ref{all_cases}d.



\subsubsection{Results in dynamic environment}
Real-time simulation was also performed on dynamic environments involving the continuous movement of one ground obstacle and one obstacle that was deflected by both the drone and mobile robot. Fig. \ref{moving obs} shows the tests that were performed in two main cases with sparse (Fig. \ref{moving obs}a) and dense (Fig. \ref{moving obs}b) environments. The change of path can be observed in the region where the movement of the obstacle takes place. The same behavior is observed in a cluttered environment, shown in Fig.\ref{moving obs}c and Fig.\ref{moving obs}d. The results revealed the deficient adaptable behavior of heterogeneous swarm under multiple environmental conditions. 

\begin{figure*}
\centering
\includegraphics[width=0.75\textwidth]{HeteroImages/Moving_Obs_cases1.jpg}
\caption{Simulation examples of dynamic obstacle avoidance in sparse (a,b) and dense (c,d) environments. Dashed purple lines are the obstacle trajectories. Black dots are the obstacles for both robots, green dots are the ground obstacles deflected only by the mobile robot. Gray and light green-colored circles show the safe deflection regions for both agents.}
\label{moving obs}
\end{figure*}

\subsubsection{Error between path planning and ground truth in simulation environment}
Since the planning is conducted in a real-time environment, the drone's position exhibits negligible errors. However, deviations are observed in the mobile robot's impedance path due to its additional responsibility of avoiding ground-level obstacles. These deviations are particularly noticeable in regions where ground obstacles are present, as the mobile robot must adjust its trajectory to navigate around them. This behavior is consistent with the system's design, where the mobile robot prioritizes obstacle avoidance while maintaining connectivity with the drone. Table \ref{Trajectory Deviation} shows the deviations of actual mobile robot positions from the planned positions.

% \begin{table}[h!]
%     \centering
%     \caption{Error Between Path Planning and Actual Performance in Simulation environment}
%     \renewcommand{\arraystretch}{1.2} % Increase the height of rows
%     \begin{tabular}{|c|c|} % 'c' for centered columns, '|' for vertical lines
%         \hline
%         % \multicolumn{2}{|c|}{\textbf{Deviation between actual and planned trajectory}} \\ % Merge 3 columns, center aligned
%         % \hline
%         \textbf{CASE NO.:} & \textbf{mobile robot Deviations (m)}\\
%         \hline
%          I &  0.45\\
%         \hline
%          II & 0.45  \\
%         \hline
%          III & 0.45 \\
%         \hline
%          IV & 0.43 \\
%         \hline
%     \end{tabular}
%     \label{Trajectory Deviation}
% \end{table}

\begin{table}[h!]
    \centering
    \caption{Error between Path Planning and Actual Performance in Simulation Environment}
    \renewcommand{\arraystretch}{1.2} % Increase the height of rows
    \begin{tabular}{|c|c|c|c|c|} % 'c' for centered columns, '|' for vertical lines
        \hline
        \textbf{CASE NO.:} & I & II & III & IV \\
        \hline
        \textbf{Ground Robot Deviations (m)} & 0.45 & 0.45 & 0.45 & 0.43 \\
        \hline
    \end{tabular}
    \label{Trajectory Deviation}
\end{table}

Fig. \ref{all_cases} shows that mobile robot deviations are higher around the ground obstacles; however, they are still maintained in the safe operating region, which shows the reliability of the path planning system and the robot's ability to handle environmental challenges, despite the presence of densely packed obstacles.

\subsubsection{Velocity Distribution along the trajectory}
Fig. \ref{velocity} shows the distribution of velocities along the trajectory path for drone Fig. \ref{velocity}a, and ground robot Fig. \ref{velocity}b. The plot shows how significantly the velocities of each of the agents change over time, providing insights into their acceleration and deceleration corresponding to changing applied forces. Simulation shows that agents move faster with greater velocity near the region of obstacles. This behavior could be indicative of a strategy where the agent adjusts its velocity due to applied repulsive and impedance forces within the obstacle's close proximity to avoid collisions while maintaining a shortest and  time-optimal trajectory.
\begin{figure}[htbp]
\centering
% \vspace{0.2cm}
\includegraphics[width=1\linewidth]{HeteroImages/velocity.png} 
\caption{Visualization of Velocity Variations along the Agents Trajectory}
% \vspace{-0.4cm}
\label{velocity}
\vspace{-0.2cm}

\end{figure}

\subsubsection{Time efficiency in task completion for multiple agents}
To check the robustness and system adaptability under varying conditions, we conducted experiments on two different environments, shown in Fig. \ref{all_cases}c, and Fig. \ref{all_cases}d, with varying starting and goal positions, both in a static and dynamic environment. Table \ref{trajectory} shows how the trajectory lengths and mission time change under varying environmental conditions. 


\begin{table}[h!]
    \centering
    \caption{Swarm Behavior in Varying Conditions}
    \renewcommand{\arraystretch}{1.3}
    \begin{tabular}{| p{0.5cm}|c|c|c|c|} % 'c' for centered columns, '|' for vertical lines
        \hline
        \multicolumn{5}{|c|}{\textbf {Time Efficiency along with length of trajectories of Leader-Drone}} \\ % Merge 5 columns, center aligned
        \hline
        \textbf {Case} & \multicolumn{2}{c|}{\textbf{Static Environment}} & \multicolumn{2}{c|}{\textbf{Dynamic Environment}} \\
        \hline
        & \begin{tabular}[c]{@{}c@{}}Trajectory\\ length (m)\end{tabular} & \begin{tabular}[c]{@{}c@{}}Completion\\ time (s)\end{tabular} & \begin{tabular}[c]{@{}c@{}}Trajectory\\ length (m)\end{tabular} & \begin{tabular}[c]{@{}c@{}}Completion\\ time (s)\end{tabular}  \\
        \hline
        1 & 8.18\ & 25.05\ & 8.02 \ & 25.03\\
        \hline
        2 & 9.11\ & 25.42\ & 9.25\ & 25.87\\
        \hline
    \end{tabular}
    \label{trajectory}
\end{table}

The trajectory length along with completion time may increase or decrease depending on whether the agents encounter the obstacles. These results demonstrate the system's ability to handle real-time tasks and quickly and effectively adapt to environmental changes, making it suitable for real-world applications in logistics and warehouse management.

\subsubsection{Success Rate}
Simulations on the heterogeneous swarm were conducted under multiple environmental conditions. A total of 30 experiments were performed, out of which 27 were successful. Failures occurred when either of the two agents collided with an obstacle in the dynamic environment, particularly when the obstacle's speed exceeded that of the agents, causing it to suddenly block the agent's planned path. Therefore, the success rate, calculated as:
\begin{equation}
\text{Success Rate} = \left( \frac{\text{Successful Experiments}}{\text{Total Experiments}} \right) \times 100
\end{equation}
was computed to be 90\%. % The success rate is significant because it indicates that the system successfully performed its task 80\% of the time in a dynamic environment. 
The success rate shows that the system is relatively robust, with a strong ability to navigate and adapt to different dynamic conditions. 

% \[
% \text{Success Rate} = \left( \frac{16}{20} \right) \times 100 = 80\%
% \]

\subsection{Comparison with Conflict-Based Search (CBS) Algorithm}
The CBS path planning algorithm can operate in both 2D and 3D environments. The drone trajectory generated by APF was compared with the 3D setup of the CBS algorithm, while the mobile robot trajectory was compared with the 2D setup of the CBS algorithm (Table \ref{CBS}). 


\begin{table}[h!]
    \centering
    \caption{Trajectory Length and Minimum Obstacle Distance Comparison with Classic CBS Algorithm}
    \renewcommand{\arraystretch}{1.35}
    \begin{tabular}{|c|c|c|c|c|} % 'c' for centered columns, '|' for vertical lines
        \hline
        \multicolumn{5}{|c|}{\textbf{Trajectory Length}} \\ % Merge 5 columns, center aligned
        \hline
        \textbf{Case} & \multicolumn{2}{c|}{\textbf{2D Environment}} & \multicolumn{2}{c|}{\textbf{3D Environment}} \\
        \hline
        & \textbf{CBS} & \textbf{mobile robot} & \textbf{CBS} & \textbf{Drone} \\
        \hline
        1 & 10.39\ & 9.02\ & 7.5\ & 7.86 \\
        \hline
        2 & 10.39\ & 12.36\ & 7.5
        \ & 8.18 \\
        \hline
        \multicolumn{5}{|c|}{\textbf{Minimum Obstacle Distance}} \\ % Merge 5 columns, center aligned
        \hline
        \textbf{Case} & \multicolumn{2}{c|}{\textbf{2D Environment}} & \multicolumn{2}{c|}{\textbf{3D Environment}} \\
        \hline
        1 & 0.73\ & 0.25\ & 0.75\ & 0.38 \\
        \hline
        2 & 0.72\ & 0.25\ & 0.75\ & 0.37 \\
        \hline
    \end{tabular}
    \label{CBS}
\end{table}

With similar environmental configurations, the CBS trajectory in Case 1, where there are fewer obstacles, shows that agents tend to move around obstacles, resulting in a longer path. In contrast, APF agents navigate within obstacles, leading to more efficient paths. However, in Case 2, where the environment is more densely packed with obstacles, CBS agents ignore the inner regions and move around obstacles to complete the trajectory. On the other hand, APF agents move between obstacles, encountering multiple deflections, which increase their path length. Despite this, APF excels in terms of minimum obstacle distance; agents maintain a very close distance to obstacles while still avoiding collisions. In comparison, CBS agents maintain a  minimum distance of 72 cm, reflecting a more cautious but less flexible path.

% Additionally, the distance between agents and all obstacles was computed, but only the minimum distance for each agent to the obstacle was considered. This is because the minimum distance represents the point at which the agent comes closest to an obstacle, which is crucial for understanding when deflection occurs. Fig. \ref{all_cases}a, and Fig. \ref{all_cases}c show the two cases that have been used. %The CBS algorithm, by comparison, is also able to evaluate these minimal distances and ensure that the paths it generates avoid obstacles effectively.buffers.


 % By identifying the closest proximity, the maximum closeness each agent can reach to an obstacle is determined, which shows how the algorithm avoids failure despite approaching close to obstacles. This approach demonstrates the system's robustness by maintaining safe distances and preventing failure, even under challenging conditions. The minimum distance also highlights how well the system navigates and adapts to the environment while maintaining safety.
\subsection{Conclusion and Future Work}
The research presents a heterogeneous swarm system comprising of a drone and the mobile robot, designed to collaborate in dynamic environments for logistics and warehouse management tasks. This research works on a leader-follower approach where a drone, being a leader, uses an APF path planner to navigate to goal positions while avoiding obstacles. Additionally, the mobile robot follows the drone using impedance links that connect the drone with the mobile robot. Moreover, the mobile robot develops additional impedance links with the ground obstacles to ensure smooth navigation toward the goal. 

The system also provides efficient navigation of the swarm in dynamic and cluttered environments, thereby enhancing the adaptability and performance of the system in real-time scenarios. By combining the two agents, the system proves to have the complementary strengths of both a drone and a mobile robot. Multiple experiments were performed on this system, achieving a success rate of 90\%. Additionally, a small deviation of about 45 cm was observed in the mobile robot's path, particularly near the ground obstacles region, indicating the robot is effectively avoiding them.

 The current approach was later compared with the Conflict-Based Search (CBS) algorithm, which reveals that our approach enables agents to safely yet optimally navigate within 25 cm of obstacles, whereas CBS maintains a minimum distance of 73 cm. 
 
In the future, we plan to expand this system to include multiple drones and mobile robots working together in more complex environments. We also aim to test the system in real-world settings, ensuring its feasibility in practical logistics and warehouse operations. 
\bibliographystyle{IEEEbib}

\balance
\bibliography{HetSwarm}


\end{document}
