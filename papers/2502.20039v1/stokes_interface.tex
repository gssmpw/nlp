%% LyX 2.3.6 created this file.  For more info, see http://www.lyx.org/.
%% Do not edit unless you really know what you are doing.
\documentclass[11pt,twoside]{article}
\usepackage[utf8]{inputenc}
\usepackage[a4paper]{geometry}
\geometry{verbose,tmargin=20mm,bmargin=22mm,lmargin=22mm,rmargin=22mm}
\usepackage{float}
\usepackage{wrapfig}
\usepackage{amsmath}
\usepackage{amsthm}
\usepackage{amssymb}
\usepackage{graphicx}
\usepackage{esint}
\usepackage[unicode=true]
 {hyperref}

\makeatletter

%%%%%%%%%%%%%%%%%%%%%%%%%%%%%% LyX specific LaTeX commands.
%% Because html converters don't know tabularnewline
\providecommand{\tabularnewline}{\\}
%% A simple dot to overcome graphicx limitations
\newcommand{\lyxdot}{.}


\@ifundefined{showcaptionsetup}{}{%
 \PassOptionsToPackage{caption=false}{subfig}}
\usepackage{subfig}
\makeatother

\begin{document}
\title{Higher-order spectral element method for the stationary Stokes interface
problem in two dimensions}
\author{Kishore Kumar Naraparaju\thanks{Associate professor, Department of Mathematics, BITS-Pilani Hyderabad Campus Hyderabad, India, Email: naraparaju@hyderabad.bits-pilani.ac.in}$^{\star}$,
Shivangi Joshi\thanks{Research Scholar, Department of Mathematics, BITS-Pilani Hyderabad Campus, Hyderabad, India, Email: p20200036@hyderabad.bits-pilani.ac.in}$^{\star}$,
and Subhashree Mohapatra\thanks{Assistant Professor, Department of Mathematics,  IIIT Delhi, Delhi, India, Email: subhashree@iiitd.ac.in}\thanks{This work has been supported by National Board for Higher Mathematics,
India.}$^{\dagger}$\\
 }
\date{~}
\maketitle
\begin{abstract}
This article presents a higher-order spectral element method for the
two-dimensional Stokes interface problem involving a piecewise constant
viscosity coefficient. The proposed numerical formulation is based
on least-squares formulation. The mesh is aligned with the interface,
and the interface is completely resolved using blending element functions.
The higher-order spectral element functions are nonconforming, and
the same-order approximation is used for both velocity and pressure
variables. The interface conditions are added to the minimizing functional
in appropriate Sobolev norms. Stability and error estimates are proven.
The proposed method is shown to be exponentially accurate in both
velocity and pressure variables. The theoretical estimates are validated
through various numerical examples.
\end{abstract}
\textbf{Keywords:} Interface, Viscosity coefficient, Spectral element,
Nonconforming, Normal equations, Preconditioner, Exponential accuracy
\\
\\
\textbf{MSC:} 65N35,65F10,35J57 

\section{Introduction}

This paper uses a piecewise constant viscosity coefficient to look at the Stokes interface problem in multi-phase incompressible fluid flows. It applies to situations like emulsions, biological flows, and industrial mixing, where fluids have different densities and viscosities at interfaces. The jump in the viscosity coefficient at the interface results in jumps in the velocity and pressure. This leads to low global regularity and, therefore, makes the Stokes interface problem challenging to solve. These discontinuities, combined with the complex geometry of the interface, add to the difficulty of achieving accurate numerical approximations. Maintaining good accuracy while keeping the computational cost reasonably low is particularly challenging. The numerical approximation of the Stokes interface problem has been widely studied in the literature; however, the regularity of the solution in the individual regions is little known. 

Broadly, we can categorize the numerical methods for the interface problems into two categories: the fitted mesh methods and the unfitted mesh approach. Fitted mesh methods align the mesh with the interface, approximating it using piecewise polynomials. However, re-meshing is a requirement when the interfaces move and increases the computational complexity. Moving interfaces are made easier with unfitted mesh approaches where the mesh has holes and cuts the interface through its elements. These methods are subdivided into enrichment-based methods, which add degrees of freedom near the interface, and modified space methods, which adapt the finite element space to satisfy interface conditions without increasing degrees of freedom. Prominent examples include Nitsche's method, the Immersed Finite Element Method (IFEM), the Extended Finite Element Method (XFEM), and the Cut Finite Element Method (CutFEM). All these methods have advantages and disadvantages; sometimes, accuracy, stability, and computational cost have to be compromised. In all these methods, the finite element spaces are wisely chosen for velocity and pressure variables such that the Ladyzhenskaya--Babu\v{s}ka--Brezzi (LBB) condition is satisfied.

Several other approaches have been explored to solve Stokes interface problems. Finite difference methods, such as those by Li et al.~\cite{LI1} and Dong et al.~\cite{DoZh}, handle discontinuous viscosity along arbitrary interfaces and resolve jumps in the solution and its derivatives. A generalized finite difference method was also introduced by Shao et al.~\cite{shaosongli}. 

In the context of finite volume methods, Sevilla and Duretz~\cite{SeDu} developed a face-centered finite volume method (FCFV) that avoids the Ladyzhenskaya–Babuška–Brezzi (LBB) condition while maintaining desirable properties from hybridizable discontinuous Galerkin methods.

Finite element methods have also been extensively studied. Olshanskii and Reusken~\cite{OL1} analyzed a finite element formulation and its solver, establishing well-posedness and inf-sup results for cases with uniform viscosity jumps. Theoretical insights into the interface conditions, including pressure and velocity jumps, were also derived in \cite{IT1}. Ohmori and Saito~\cite{OH1} proved optimal error bounds for various approximations,  such as MINI element and P1-iso-P2/P1 element approximations, while Song and Gao~\cite{SO1} validated the inf-sup stability of methods for multi-subdomain cases.

A cut finite element method for a Stokes interface problem has been studied in \cite{HA1}. The authors have used Nitsche's formulation to capture interface discontinuities, and to ensure a well-conditioned matrix, stabilization terms are added for both velocity and pressure. The optimal order convergence is proved theoretically and verified numerically. 

Kirchhart et al. \cite{KI1} have addressed Stokes interface problems using $P_{1}$ extended finite element space for pressure
and standard confirming $P_{2}$ finite element space for velocity. A stabilizing term has been added for stability. An optimal preconditioner for the stiffness matrix corresponding to the pair mentioned above is presented. Error estimates are derived. Because of the standard $P_{2}$-FE velocity space, the error bound is not optimal when the normal derivative of the velocity is discontinuous across the interface.
In \cite{LE1}, an unfitted finite element discretisation based on the Taylor-Hood velocity-pressure pair with XFEM (or CutFEM) enhancements has been proposed. Nitsche's formulation
is used to implement interface conditions, and a ghost penalty stabilization is used to ensure inf-sup stability. Further, optimal error bounds are shown.

In 2015, Adjerid et al. \cite{AD} proposed an immersed $Q_{1}-Q_{0}$ discontinuous Galerkin method for Stokes interface problems, in which they considered the coupling of velocity and pressure while constructing the immersed finite element (IFE) spaces. The concept was extended by utilizing immersed $CR-P_{0}$ (where $CR$ stands for Crouzeix--Raviart) and the Rannacher--Turek rotated $Q_{1}-Q_{0}$ elements in \cite{JO1}. Numerical results with optimal convergence rates have
been provided. Chen and Zhang \cite{CH1} have proposed a Taylor-Hood-immersed finite element method for Stokes interface problems. Lower-order immersed finite element spaces $(P_{2}-P_{1})$ are used for approximation.
Numerical examples with different interface conditions and different coefficients demonstrate the optimal convergence of the method. The theoretical study for IFE approaches for Stokes interface problems has been provided in \cite{jiwangli}. The conventional $CR-P_{0}$ finite element space is modified to create IFE spaces. The unisolvence of IFE basis functions and the best approximation abilities of IFE space have been proven. Furthermore, the authors have shown that the IFE approach is stable and produces optimal error estimates. The
authors of \cite{JoKw} came up with a new $P_{1}$-nonconforming-based IFE method for Stokes interface problems that is different from the one used in \cite{AD,JO1} in terms of the modification of the basis functions. They constructed a velocity basis on the interface element,
which is less coupled to the pressure basis compared with \cite{AD} or \cite{JO1}. Another characteristic of their IFE space is that the pressure basis has two degrees of freedom on the interface element to manage the discontinuity of the pressure variable. Unfitted methods have also been studied in \cite{burmandel,huwangchen,nwangchen,WA1,WA2,ramanbhupen}.

Despite these advances, existing methods often rely on lower-order approximations or use different polynomial orders for velocity and pressure to satisfy stability conditions.
 Least-squares methods address this limitation by producing symmetric and positive-definite systems. Spectral methods, known for their high accuracy, have been combined with least-squares formulations for Stokes flows, as discussed in~\cite{BOTH, JN1, PROO2, PR003}. Hessari~\cite{He2} demonstrated spectral convergence for least-squares pseudo-spectral methods, while other works~\cite{HE1, hes3} addressed cases with singular sources and discontinuous viscosities.This problem
has also been studied using various numerical schemes in \cite{AS,CHI,He,johnansonn,Su,YA}.

The nonconforming spectral element method (NSEM) has shown promise for solving steady-state Stokes systems~\cite{kishsubham, S1, S2, S3}.  In
\cite{kishnaga,arbazsubham}, nonconforming spectral element method for elliptic interface problems in two and three dimensions has been studied, respectively. In \cite{kishore kumar}, NSEM for elasticity interface problems has been studied. More details about NSEM can be found in \cite{kishshi}.

Building on such advances, we propose a least-squares-based nonconforming spectral element
method for Stokes interface problems with smooth interfaces. Unlike classical methods, this approach ensures exponential accuracy while maintaining computational efficiency. Key features of the proposed method include the use of same-order spectral element functions for velocity and pressure, incorporation of interface conditions into the minimizing functional via appropriate Sobolev norms, and exponential convergence for both velocity and pressure variables.

The error estimates derived for velocity (in the $H^1$ norm) and pressure (in the $L^2$ norm) demonstrate exponential decay. These results are verified numerically and also highlight the mass conservation property of the scheme through analysis of the error in the continuity equation. This
paper is structured as follows:  Section 2 states the Stokes interface problem, offers regularity
Estimates, and presents the discretisation of the domain. Section 3 formulates the numerical scheme
and establishes the error estimates. Section 4 presents some numerical results in order to test the proposed
method. Finally, Section 5 concludes the paper with insights and future directions.


Here we give some notations and define required function spaces. Let
$\Omega\subset\mathbb{R}^{2},$ be an open bounded set with sufficiently
smooth boundary $\partial\Omega$. $H^{m}(\Omega)$ denotes the Sobolev
space of functions with square integrable derivatives of integer order
less than or equal to $m$ in $\Omega$ equipped with the norm 
\[
\left\Vert u\right\Vert _{H^{m}(\Omega)}^{2}=\sum_{\left|\alpha\right|\leq m}\left\Vert D^{\alpha}u\right\Vert _{L^{2}(\Omega)}^{2}.
\]
Further, let $I=(-1,1).$ Then we define fractional norms $(0<s<1)$
by : 
\begin{eqnarray*}
\left\Vert w\right\Vert _{s,I}^{2}=\|w\|_{0,I}^{2}+\int_{I}\int_{I}\frac{\left|w(\xi)-w(\xi^{\prime})\right|^{2}}{\left|\xi-\xi^{\prime}\right|^{1+2s}}d\xi d\xi^{\prime}.
\end{eqnarray*}
Moreover, 
\begin{align*}
\|w\|_{1+s,I}^{2}=\|w\|_{0,I}^{2}+\left\Vert \:\frac{\partial w}{\partial\xi}\:\right\Vert _{s,I}^{2}+\left\Vert \:\frac{\partial w}{\partial\eta}\:\right\Vert _{s,I}^{2}\;.
\end{align*}
We denote the vectors by bold letters. For example, $\mathbf{u}=(u_{1},u_{2})^{T}$,
$\mathbf{H}^{k}(\Omega)=H^{k}(\Omega)\times H^{k}(\Omega),$ etc.
The norms are given by $||\mathbf{u}||_{k,\Omega}^{2}=||u_{1}||_{k,\Omega}^{2}+||u_{2}||_{k,\Omega}^{2}$
for $\mathbf{u}\in\mathbf{H}^{k}(\Omega),$ $\left\Vert \mathbf{u}\right\Vert _{s,I}^{2}=||u_{1}||_{s,I}^{2}+||u_{2}||_{s,I}^{2},$
etc.

\section{Interface problem and Stability estimate}

In this section, we consider the Stokes interface problem with Dirichlet
condition on the boundary. We state the regularity estimate of the
problem, describe the discretization of the domain and prove the stability
estimate. 

\subsection{Stokes interface problem }

\begin{figure}[H]
\subfloat[Domain with smooth interface $\Gamma_{0}$]{\includegraphics[scale=0.55]{Fig_4.pdf}

}~~~~~~~~~~~~~~~\subfloat[Interface-fitted mesh discretization]{\includegraphics[scale=0.6]{Fig_8}

}

\caption{Domain $\Omega$ and its discretization}

\end{figure}

Let $\Omega$ be a bounded polygon in $\mathbb{R}^{2}$ and $\Omega_{1}\subset\Omega$
be an open domain with a boundary $\Gamma_{0}=\partial\Omega_{1}\subset\Omega.$
Let $\Omega_{2}=\Omega\setminus\overline{\Omega}_{1}$ and $\Omega_{1}\cap\Omega_{2}=\emptyset$
(see Fig. 1.a). Here $\Gamma_{0}$ is the interface between the two
subdomains $\Omega_{1}$ and $\Omega_{2}$ which is assumed to be
very smooth (at least $C^{2}$). Let ${\displaystyle \Gamma={\displaystyle \cup_{i=1}^{4}\Gamma_{i}}=\partial\Omega}$
be the sufficiently smooth boundary of $\Omega.$ $\mathbf{x}=(x,y)$
denotes a point in $\Omega.$

Consider Stokes interface problem of the form 
\begin{align}
\begin{cases}
-\mbox{div}(\nu\nabla\mathbf{u})+\nabla p=\mathbf{f}\quad\!\!\!\mbox{in}\quad\!\!\!\Omega_{1}\cup\Omega_{2},\\
\mbox{div}\,\mathbf{u}=0\quad\!\!\!\mbox{in}\quad\!\!\!\Omega_{1}\cup\Omega_{2},\\
\mathbf{u}=0\quad\!\!\!\mbox{on}\quad\!\!\!\Gamma,
\end{cases}\label{i1}
\end{align}
with interface conditions 
\begin{eqnarray}
[\mathbf{u}]=0,\qquad[(\nu\nabla\mathbf{u}-pI)\mathbf{n}]=\mathbf{g}\,\,\textrm{on}\,\,\Gamma_{0}.\label{eq: 1.b}
\end{eqnarray}
Where the viscosity $\nu$ is given by piecewise constant 
\begin{align*}
\nu=\begin{cases}
\nu_{1}>0\quad\!\!\!\mbox{in}\quad\!\!\!\Omega_{1},\\
\nu_{2}>0\quad\!\!\!\mbox{in}\quad\!\!\!\Omega_{2}.
\end{cases}
\end{align*}
Here $\mathbf{u}=(u_{1},u_{2})^{T},\mathbf{f}=(f_{1},f_{2})^{T}$
and $I$ is the identity matrix. 

Let $\mathcal{L}\mathbf{u}=-\mbox{div}(\nu\nabla\mathbf{u})+\nabla p$
and $\mathcal{D}\mathbf{u}=\mbox{div}\,\mathbf{u}.$ Let $\mathbf{u}_{1}=\mathbf{u}|_{\Omega_{1}}=(u_{1}^{1},u_{2}^{1})^{T},$$\mathbf{u}_{2}=\mathbf{u}|_{\Omega_{2}}=(u_{1}^{2},u_{2}^{2})^{T},p_{1}=p|_{\Omega_{1}}$and
$p_{2}=p|_{\Omega_{2}}.$ $[\mathbf{u}]=\mathbf{u}_{1}-\mathbf{u}_{2}$
denotes the jump across the interface $\Gamma_{0}.$ Let $\mathbf{n}=(n_{1},n_{2})^{T}$
be the outward unit normal vector to $\Gamma_{0}$ pointing from $\Omega_{1}$
to $\Omega_{2},$ then $[(\nu\nabla\mathbf{u}-pI)\mathbf{n}]$ is
defined by $[(\nu\nabla\mathbf{u}-pI)\mathbf{n}]=(\nu_{1}\nabla\mathbf{u}_{1}-p_{1}I)\mathbf{n}-(\nu_{2}\nabla\mathbf{u}_{2}-p_{2}I)\mathbf{n}.$ 

\subsubsection*{Uniqueness and regularity}

Define the broken Sobolev space 
\begin{eqnarray*}
\mathbf{H}^{s}(\Omega_{1}\cup\Omega_{2}) & =\left\{ \mathbf{u}_{i}=\mathbf{u}|_{\Omega_{i}}\in\mathbf{H}^{s}(\Omega_{i}),i=1,2\right\} 
\end{eqnarray*}
with $\left\Vert \mathbf{u}\right\Vert _{s,\Omega_{1}\cup\Omega_{2}}=\left\Vert \mathbf{u}_{1}\right\Vert _{s,\Omega_{1}}+\left\Vert \mathbf{u}_{2}\right\Vert _{s,\Omega_{2}}.$

Assume that the pressure variable satisfies $\int_{\Omega}p\,dx=0.$
Let $\mathbf{f}\in\mathbf{L}^{2}(\Omega)$ and $\mathbf{g}\in\mathbf{H}^{\frac{1}{2}}(\Gamma_{0}),$
then the variational problem of the given Stokes interface problem
has a unique weak solution $(\mathbf{u},p)\in\mathbf{H}_{0}^{1}(\Omega)\times L^{2}(\Omega).$
Further, the solution of the Stokes interface problem (1)-(2) satisfies
the following regularity estimate \cite{shibata,WA1}. \\
\\
\textbf{Theorem 2.1:} Assume that $\mathbf{f}\in\mathbf{L}^{2}(\Omega)$
and $\mathbf{g}\in\mathbf{H}^{\frac{1}{2}}(\Gamma_{0}),$ then $(\mathbf{u},p)$
satisfies the estimate 
\begin{eqnarray*}
\left\Vert \mathbf{u}\right\Vert _{2,\Omega_{1}\cup\Omega_{2}}+\left\Vert p\right\Vert _{1,\Omega_{1}\cup\Omega_{2}}\leq C\left(\left\Vert \mathbf{f}\right\Vert _{0,\Omega}+\left\Vert \mathbf{g}\right\Vert _{\frac{1}{2},\Gamma_{0}}\right),
\end{eqnarray*}
where $C$ is a general constant independent of mesh \cite{ramanbhupen,shibata,WA1}.

Based on the regularity theory of Stokes problem \cite{kishsubham}
and the above regularity estimate, we have
\begin{eqnarray}
\left\Vert \mathbf{u}\right\Vert _{2,\Omega_{1}\cup\Omega_{2}}+\left\Vert p\right\Vert _{1,\Omega_{1}\cup\Omega_{2}}\leq C\left(\left\Vert \mathbf{f}\right\Vert _{0,\Omega}+\left\Vert \mathbf{\mathcal{D}\mathbf{u}}\right\Vert _{1,\Omega}+\left\Vert \left[\mathbf{u}\right]\right\Vert _{\frac{3}{2},\Gamma_{0}}+\left\Vert \mathbf{g}\right\Vert _{\frac{1}{2},\Gamma_{0}}+\left\Vert \mathbf{u}\right\Vert _{\frac{3}{2},\Gamma}\right).\label{eq:-8}
\end{eqnarray}


\subsection{Discretization and spectral element functions}

The subdomains $\Omega_{1}$ and $\Omega_{2}$ are further partitioned
into finite number of quadrilateral elements $\Omega_{1}^{1},\Omega_{1}^{2},...,\Omega_{1}^{m_{1}}$
and $\Omega_{2}^{1},\Omega_{2}^{2},...,\Omega_{2}^{m_{2}}$ such that
the discretization match on the interface (see fig. 1.b). The division
of the domain into quadrilaterals is more general as a triangle can
be further subdivided into three quadrilaterals by joining midpoints
of the sides of the triangle to the center of the triangle. 

Define an analytic map $M_{i}^{l}$ from the master square $S=(-1,1)^{2}$
to $\Omega_{i}^{l}$ by (see \cite{GO})
\begin{eqnarray*}
x & = & X_{i}^{l}(\xi,\eta),\\
y & = & Y_{i}^{l}(\xi,\eta),\,i=1,2.
\end{eqnarray*}
Here and in the rest of this section $l=1,...,m_{1}\text{ for }i=1$
and $l=1,...,m_{2}\text{ for }i=2.$ 

Define the spectral element functions $\left\{ \tilde{\mathbf{u}}_{1}^{l}\right\} _{l},\left\{ \tilde{\mathbf{u}}_{2}^{l}\right\} _{l}$
as the tensor product of polynomials of degree $W$ in each variable
$\xi$ and $\eta$ as 
\begin{eqnarray*}
{\displaystyle \tilde{\mathbf{u}}_{1}^{l}(\xi,\eta)=(\tilde{u}_{1}^{1,l}(\xi,\eta),\tilde{u}_{2}^{1,l}(\xi,\eta))^{T}\,\,\,\,\textrm{where}\,\;\tilde{u}_{1}^{1,l}(\xi,\eta)=\sum_{r=0}^{W}\sum_{s=0}^{W}}g_{r,s}^{1,l}\,\xi^{r}\eta^{s}\,\,\,\textrm{and}\,\,\,{\displaystyle \tilde{u}_{2}^{1,l}(\xi,\eta)=\sum_{r=0}^{W}\sum_{s=0}^{W}}g_{r,s}^{2,l}\,\xi^{r}\eta^{s},\\
{\displaystyle \tilde{\mathbf{u}}_{2}^{l}(\xi,\eta)=(\tilde{u}_{1}^{2,l}(\xi,\eta),\tilde{u}_{2}^{2,l}(\xi,\eta))^{T}\,\,\,\,\textrm{where}\,\;\tilde{u}_{1}^{2,l}(\xi,\eta)=\sum_{r=0}^{W}\sum_{s=0}^{W}}h_{r,s}^{1,l}\,\xi^{r}\eta^{s}\,\,\,\textrm{and}\,\,\,{\displaystyle \tilde{u}_{2}^{2,l}(\xi,\eta)=\sum_{r=0}^{W}\sum_{s=0}^{W}}h_{r,s}^{2,l}\,\xi^{r}\eta^{s}.
\end{eqnarray*}
Then $\left\{ \mathbf{u}_{1}^{l}\right\} _{l},\left\{ \mathbf{u}_{2}^{l}\right\} _{l}$
are given by 
\[
\mathbf{u}_{1}^{l}(x,y)=\tilde{\mathbf{u}}_{1}^{l}\left((M_{1}^{l})^{-1}\right)\,\,\,\,\,\textrm{and}\,\,\,\,\,\,\mathbf{u}_{2}^{l}(x,y)=\tilde{\mathbf{u}}_{2}^{l}\left((M_{2}^{l})^{-1}\right).
\]

Define $\tilde{p}_{i}^{l}(\xi,\eta)={\displaystyle \sum_{r=0}^{W}}{\displaystyle \sum_{s=0}^{W}}b_{r,s}^{i,l}\xi^{r}\eta^{s}$
for $i=1,2$. Then $\left\{ p_{i}^{l}\right\} _{l}$ for $i=1,2$
are given by 
\[
p_{i}^{l}(x,y)=\tilde{p}_{i}^{l}\left((M_{i}^{l})^{-1}\right).
\]
We assume that the spectral element functions are nonconforming. 

\subsection{Stability estimate }

Let 
\begin{align*}
\mathcal{L}(\mathbf{u}_{i}^{l},p_{i}^{l})=-\mbox{div}(\nu_{i}\nabla\mathbf{u}_{i}^{l})+\nabla p_{i}^{l},
\end{align*}
and 
\begin{align*}
\mathcal{D}(\mathbf{u}_{i}^{l})=-\nabla\cdot\mathbf{u}_{i}^{l}\,\, & \textrm{for}\,\,i=1,2.
\end{align*}

Let $J_{i}^{l}(\xi,\eta)$ be the Jacobian of the mapping $M_{i}^{l}(\xi,\eta)$
from $S=(-1,1)^{2}$ to $\Omega_{i}^{l}$ for $i=1,2.$ Now 
\[
\intop_{\Omega_{i}^{l}}\left|\mathcal{L}(\mathbf{u}_{i}^{l},p_{i}^{l})\right|^{2}dxdy=\intop_{S}\left|\mathcal{L}(\tilde{\mathbf{u}}_{i}^{l},\tilde{p}_{i}^{l})\right|^{2}J_{i}^{l}\,d\xi d\eta.
\]
Define $\mathcal{L}_{i}^{l}(\tilde{\mathbf{u}}_{i}^{l},\tilde{p}_{i}^{l})=\mathcal{L}(\tilde{\mathbf{u}}_{i}^{l},\tilde{p}_{i}^{l})\sqrt{J_{i}^{l}}.$
Then 
\[
\intop_{\Omega_{i}^{l}}\left|\mathcal{L}(\mathbf{u}_{i}^{l},p_{i}^{l})\right|^{2}dxdy=\intop_{S}\left|\mathcal{L}_{i}^{l}(\tilde{\mathbf{u}}_{i}^{l},\tilde{p}_{i}^{l})\right|^{2}\,d\xi d\eta.
\]
Similarly, we define $\mathcal{D}_{i}^{l}\tilde{\mathbf{u}}_{i}^{l}=\mathcal{D}\tilde{\mathbf{u}}_{i}^{l}\,\sqrt{J_{i}^{l}}.$ 

Here are some notations that are needed to define the functional for
the stability estimate. 

Let $\gamma_{s}$ be a side common to the two adjacent elements $\Omega_{i}^{m}$
and $\Omega_{i}^{n},i=1,2$. Assume that $\gamma_{s}$ is the image
of $\eta=-1$ under the mapping $M_{i}^{m}$ which maps $S$ to $\Omega_{i}^{m}$
and also the image of $\eta=1$ under the mapping $M_{i}^{n}$ which
maps $S$ to $\Omega_{i}^{n}.$ By chain rule
\begin{eqnarray}
 &  & (\mathbf{u}_{i}^{m})_{x}=\,(\tilde{\mathbf{u}}_{i}^{m})_{\xi}\,\,\xi_{x}+(\tilde{\mathbf{u}}_{i}^{m})_{\eta}\,\,\eta_{x},\,\,\textrm{and}\label{eq:}\\
 &  & (\mathbf{u}_{i}^{m})_{y}=\,(\tilde{\mathbf{u}}_{i}^{m})_{\xi}\,\,\xi_{y}+(\tilde{\mathbf{u}}_{i}^{m})_{\eta}\,\,\eta_{y}.\nonumber 
\end{eqnarray}
Then the jumps along the inter-element boundaries are defined as
\begin{eqnarray*}
 &  & \left\Vert [\mathbf{u}_{i}]\right\Vert _{_{0,\gamma_{s}}}^{2}=\left\Vert \tilde{\mathbf{u}}_{i}^{m}(\xi,-1)-\tilde{\mathbf{u}}_{i}^{n}(\xi,1)\right\Vert _{_{0,I}}^{2},\\
 &  & \left\Vert [(\mathbf{u}_{i})_{x}]\right\Vert _{_{\frac{1}{2},\gamma_{s}}}^{2}=\left\Vert (\mathbf{u}_{i}^{m})_{x}(\xi,-1)-(\mathbf{u}_{i}^{n})_{x}(\xi,1)\right\Vert _{_{\frac{1}{2},I}}^{2},\,\textrm{and}\\
 &  & \left\Vert [(\mathbf{u}_{i})_{y}]\right\Vert _{_{\frac{1}{2},\gamma_{s}}}^{2}=\left\Vert (\mathbf{u}_{i}^{m})_{y}(\xi,-1)-(\mathbf{u}_{i}^{n})_{y}(\xi,1)\right\Vert _{_{\frac{1}{2},I}}^{2}.
\end{eqnarray*}
Here and in what follows, $I$ is an interval $(-1,1).$

As the spectral elements match along the interface, we define the
jump across the interface by taking it (a part of interface) as the
common edge. Consider the elements $\Omega_{1}^{n}$ and $\Omega_{2}^{m}$
which have the common edge $\gamma_{s}\subseteq\Gamma_{0}.$ Let $\gamma_{s}$
be the image of $\xi=1$ under the mapping $M_{1}^{n}$ which maps
$S$ to $\Omega_{1}^{n}$ and also the image of $\xi=-1$ under the
mapping $M_{2}^{m}$ which maps $S$ to $\Omega_{2}^{m}.$ Define 

\begin{eqnarray*}
\left\Vert [\mathbf{u}]\right\Vert _{\frac{3}{2},\gamma_{s}}^{2}=\left\Vert \mathbf{u}_{1}-\mathbf{u}_{2}\right\Vert _{\frac{3}{2},\gamma_{s}}^{2} & = & \left\Vert \tilde{\mathbf{u}}_{1}^{n}(1,\eta)-\tilde{\mathbf{u}}_{2}^{m}(-1,\eta)\right\Vert _{_{0,I}}^{2}+\left\Vert \frac{\partial\tilde{\mathbf{u}}_{1}^{n}}{\partial T}(1,\eta)-\frac{\partial\tilde{\mathbf{u}}_{2}^{m}}{\partial T}(-1,\eta)\right\Vert _{_{\frac{1}{2},I}}^{2},
\end{eqnarray*}
where $\frac{\partial\tilde{\mathbf{u}}_{1}^{n}}{\partial T}\text{ and }\frac{\partial\tilde{\mathbf{u}}_{2}^{m}}{\partial T}$
are the tangential derivatives of $\tilde{\mathbf{u}}_{1}^{n}$ and
$\tilde{\mathbf{u}}_{2}^{m}$ respectively. 

Since $\nu=\nu_{1}$ in $\Omega_{1}$ and $\nu=\nu_{2}$ in $\Omega_{2}$
and as per our notation $\mathbf{u}_{1}^{n}(\xi,\eta)=(u_{1}^{1,n},u_{2}^{1,n})^{T},\mathbf{u}_{2}^{m}=(u_{1}^{2,m},u_{2}^{2,m})^{T},$
we have 
\begin{eqnarray*}
(\nu_{1}\nabla\mathbf{u}_{1}^{n}-p_{1}^{n}I)\mathbf{n} & = & \left((\nu_{1}\frac{\partial u_{1}^{1,n}}{\partial x}-p_{1}^{n})n_{1}+\nu_{1}\frac{\partial u_{1}^{1,n}}{\partial y}n_{2},\,\,\nu_{1}\frac{\partial u_{2}^{1,n}}{\partial x}n_{1}+(\nu_{1}\frac{\partial u_{2}^{1,n}}{\partial y}-p_{1}^{n})n_{2}\right)^{T},\\
(\nu_{2}\nabla\mathbf{u}_{2}^{m}-p_{2}^{m}I)\mathbf{n} & = & \left((\nu_{2}\frac{\partial u_{1}^{2,m}}{\partial x}-p_{2}^{m})n_{1}+\nu_{2}\frac{\partial u_{1}^{2,m}}{\partial y}n_{2},\,\,\nu_{2}\frac{\partial u_{2}^{2,m}}{\partial x}n_{1}+(\nu_{2}\frac{\partial u_{2}^{2,m}}{\partial y}-p_{2}^{m})n_{2}\right)^{T}.
\end{eqnarray*}

Using (4), define

\begin{align*}
\left\Vert [(\nu\nabla\mathbf{u}-pI)\mathbf{n}]\right\Vert _{\frac{1}{2},\gamma_{s}}^{2} & =\left\Vert \left((\nu_{1}\nabla\mathbf{u}_{1}^{n}-p_{1}^{n}I)\mathbf{n}\right)(1,\eta)-\left((\nu_{2}\nabla\mathbf{u}_{2}^{m}-p_{2}^{m}I)\mathbf{n}\right)(-1,\eta)\right\Vert _{_{\frac{1}{2},I}}^{2}.
\end{align*}
Now along the boundary $\Gamma=\cup_{j=1}^{4}\Gamma_{j},$ let $\gamma_{s}\subseteq\Gamma_{j}$(for
some $j$) be the image of $\xi=1$ under the mapping $M_{2}^{m}$
which maps $S$ to $\Omega_{2}^{m}.$ Then 
\begin{eqnarray*}
 &  & \left\Vert \mathbf{u}_{2}\right\Vert _{0,\gamma_{s}}^{2}+\left\Vert \frac{\partial\mathbf{u}_{2}}{\partial T}\right\Vert _{\frac{1}{2},\gamma_{s}}^{2}=\left\Vert \tilde{\mathbf{u}}_{2}^{m}(1,\eta)\right\Vert _{0,I}^{2}+\left\Vert \frac{\partial\tilde{\mathbf{u}}_{2}^{m}}{\partial T}(1,\eta)\right\Vert _{\frac{1}{2},I}^{2}.
\end{eqnarray*}

Let $\Pi^{W}=\left(\left\{ \tilde{\mathbf{u}}_{1}^{k}(\xi,\eta)\right\} _{k},\left\{ \tilde{p}_{1}^{k}(\xi,\eta)\right\} _{k},\left\{ \tilde{\mathbf{u}}_{2}^{_{^{l}}}(\xi,\eta)\right\} _{l},\left\{ \tilde{p}_{2}^{_{^{l}}}(\xi,\eta)\right\} _{l}\right)$
be the space of spectral element functions. Now define the functional
$\mathcal{V}^{^{W}}\left(\left\{ \tilde{\mathbf{u}}_{1}^{k}(\xi,\eta)\right\} _{k},\left\{ \tilde{p}_{1}^{k}(\xi,\eta)\right\} _{k},\left\{ \tilde{\mathbf{u}}_{2}^{_{^{l}}}(\xi,\eta)\right\} _{l},\left\{ \tilde{p}_{2}^{_{^{l}}}(\xi,\eta)\right\} _{l}\right)$
as 
\begin{align*}
 & \mathcal{V}^{^{W}}\left(\left\{ \tilde{\mathbf{u}}_{1}^{k}(\xi,\eta)\right\} _{k},\left\{ \tilde{p}_{1}^{k}(\xi,\eta)\right\} _{k},\left\{ \tilde{\mathbf{u}}_{2}^{_{^{l}}}(\xi,\eta)\right\} _{l},\left\{ \tilde{p}_{2}^{_{^{l}}}(\xi,\eta)\right\} _{l}\right)\\
 & =\sum_{k=1}^{m_{1}}\left\Vert (\mathcal{L}_{1}^{k})(\tilde{\mathbf{u}}_{1}^{k},\tilde{p}_{1}^{k})(\xi,\eta)\right\Vert _{_{0,S}}^{2}+\sum_{l=1}^{m_{2}}\left\Vert (\mathcal{L}_{2}^{l})(\tilde{\mathbf{u}}_{2}^{l},\tilde{p}_{2}^{l})(\xi,\eta)\right\Vert _{_{0,S}}^{2}\\
 & +\sum_{k=1}^{m_{1}}||\mathcal{D}_{1}^{k}\tilde{\mathbf{u}}_{1}^{k}||_{1,S}^{2}+\sum_{l=1}^{m_{2}}||\mathcal{D}_{2}^{l}\tilde{\mathbf{u}}_{2}^{l}||_{1,S}^{2}\\
 & +\sum_{i=1}^{2}\sum_{\gamma_{s}\subseteq\Omega_{i}}\left(\left\Vert [\mathbf{u}_{i}]\right\Vert _{_{0,\gamma_{s}}}^{2}+\left\Vert [(\mathbf{u}_{i})_{x}]\right\Vert _{_{\frac{1}{2},\gamma_{s}}}^{2}+\left\Vert [(\mathbf{u}_{i})_{y}]\right\Vert _{_{\frac{1}{2},\gamma_{s}}}^{2}\right)+\sum_{i=1}^{2}\sum_{\gamma_{s}\subseteq\Omega_{i}}\left\Vert [p_{i}]\right\Vert _{_{\frac{1}{2},\gamma_{s}}}^{2}\\
 & +\sum_{\gamma_{s}\subseteq\Gamma_{0}}\left(\left\Vert [\mathbf{u}]\right\Vert _{_{\frac{3}{2},\gamma_{s}}}^{2}+\left\Vert [(\nu\nabla\mathbf{u}-pI)\mathbf{n}]\right\Vert _{\frac{1}{2},\gamma_{s}}^{2}\right)+\sum_{\gamma_{s}\subseteq\Gamma}\left(\left\Vert \mathbf{u}_{2}\right\Vert _{0,\gamma_{s}}^{2}+\left\Vert \frac{\partial\mathbf{u}_{2}}{\partial T}\right\Vert _{\frac{1}{2},\gamma_{s}}^{2}\right).
\end{align*}
Now we state some important results that are required to prove the
stability estimate.\\
\textbf{}\\
\textbf{Lemma 2.1:} Let $\left\{ \left\{ \tilde{\mathbf{u}}_{1}^{k}(\xi,\eta)\right\} _{k},\left\{ \tilde{\mathbf{u}}_{2}^{_{^{l}}}(\xi,\eta)\right\} _{l}\right\} \in\Pi^{W}$.
Then there exists $\left\{ \left\{ \tilde{\mathbf{v}}_{1}^{k}(\xi,\eta)\right\} _{k},\left\{ \tilde{\mathbf{v}}_{2}^{_{^{l}}}(\xi,\eta)\right\} _{l}\right\} $
(where $\tilde{\mathbf{v}}_{1}^{k}=0\text{ on }\Gamma_{0}$ for all
$k=1,2,..,m_{1}$, $\tilde{\mathbf{v}}_{2}^{_{^{l}}}=0\text{ on }\Gamma_{0}$
for all $l=1,2,..,m_{2}$) such that $\mathbf{v}_{1}^{k}\in\mathbf{H}^{2}(\Omega_{1}^{k}),\mathbf{v}_{2}^{l}\in\mathbf{H}^{2}(\Omega_{2}^{l}),$
$\mathbf{w_{1}=u}_{1}+\mathbf{v}_{1}\in\mathbf{H}^{2}(\Omega_{1}),\mathbf{w_{2}=u}_{2}+\mathbf{v}_{2}\in\mathbf{H}^{2}(\Omega_{2}).$
Moreover the estimate {\small{}
\begin{equation}
\sum_{k=1}^{m_{1}}\left\Vert \tilde{\mathbf{v}}_{1}^{k}(\xi,\eta)\right\Vert _{_{2,S}}^{2}+\sum_{l=1}^{m_{2}}\left\Vert \tilde{\mathbf{v}}_{2}^{_{^{l}}}(\xi,\eta)\right\Vert _{_{2,S}}^{2}\leq c(\ln W)^{2}\left(\sum_{i=1}^{2}\sum_{\gamma_{s}\subseteq\Omega_{i}}\left(\left\Vert [\mathbf{u}_{i}]\right\Vert _{_{0,\gamma_{s}}}^{2}+\left\Vert [(\mathbf{u}_{i})_{x}]\right\Vert _{_{\frac{1}{2},\gamma_{s}}}^{2}+\left\Vert [(\mathbf{u}_{i})_{y}]\right\Vert _{_{\frac{1}{2},\gamma_{s}}}^{2}\right)\right)\label{eq:-9}
\end{equation}
 holds.}\\
\textbf{Proof:} We give a rough sketch of the proof. For the complete
proof, one can look at {\small{}the Lemma 7.1 of \cite{pd1}.}{\small\par}

First we make a correction $\left\{ \left\{ \tilde{\mathbf{r}}_{1}^{k}(\xi,\eta)\right\} _{k},\left\{ \tilde{\mathbf{r}}_{2}^{l}(\xi,\eta)\right\} _{l}\right\} $
at the common vertices of the elements in the subdomains $\Omega_{1}$
and $\Omega_{2}$ such that $\tilde{\mathbf{r}}_{1}^{k}=0$ and $\tilde{\mathbf{r}}_{2}^{l}=0$
on $\Gamma_{0},$ where $\tilde{\mathbf{r}}_{1}^{k}(\xi,\eta)=\left(\tilde{r}_{1}^{1,k}(\xi,\eta),\tilde{r}_{2}^{1,k}(\xi,\eta)\right)^{T}$
and $\tilde{\mathbf{r}}_{2}^{l}(\xi,\eta)=\left(\tilde{r}_{1}^{2,l}(\xi,\eta),\tilde{r}_{2}^{2,l}(\xi,\eta)\right)^{T}.$

Consider an element $\Omega_{1}^{k}$ in $\Omega_{1}.$ As defined
in section 2.2, there is an analytic map $M_{1}^{k}$ from $S$ to
$\Omega_{1}^{k}.$ So $S=(M_{1}^{k})^{-1}(\Omega_{1}^{k}).$ Let $P_{j}$
for $j=1,..,4$ denote the vertices of $S.$ 

We can find a polynomial $\mathbf{\tilde{r}}_{1}^{k}(\xi,\eta)$ such
that for $j=1,..,4$
\begin{eqnarray*}
 &  & (\tilde{\mathbf{u}}_{1}^{k}+\tilde{\mathbf{r}}_{1}^{k})(P_{j})=\bar{\mathbf{u}}(P_{j}),\\
 &  & ((\mathbf{\tilde{u}}_{1}^{k})_{x}+(\mathbf{\tilde{r}}_{1}^{k})_{x})(P_{j})=\bar{\mathbf{u}}_{x}(P_{j}),\,\,\textrm{and}\\
 &  & ((\mathbf{\tilde{u}}_{1}^{k})_{y}+(\mathbf{\tilde{r}}_{1}^{k})_{y})(P_{j})=\bar{\mathbf{u}}_{y}(P_{j}).
\end{eqnarray*}

Here $\bar{z}$ is the average of the values of $z$ at $P_{j}$ over
all the elements that have $P_{j}$ as a vertex. Moreover, $\tilde{\mathbf{r}}_{1}^{k}$
is a polynomial of degree less than or equal to four, and the estimate
\begin{eqnarray*}
 &  & \left\Vert \tilde{\mathbf{r}}_{1}^{k}(\xi,\eta)\right\Vert _{_{2,S}}^{2}\leq C\,(\sum_{l=1}^{4}\left|\mathbf{a}_{l}\right|^{2}+\left|\mathbf{b}_{l}\right|^{2}+\left|\mathbf{c}_{l}\right|^{2}),
\end{eqnarray*}
where $\tilde{\mathbf{r}}_{1}^{k}(P_{l})=\mathbf{a}_{l}=(a_{1,l},a_{2,l})^{T},~\left(\tilde{\mathbf{r}}_{1}^{k}\right)_{x}(P_{l})=\mathbf{b}_{l}=(b_{1,l},b_{2,l})^{T},~\left(\tilde{\mathbf{r}}_{1}^{k}\right)_{y}(P_{l})=\mathbf{c}_{l}=(c_{1,l},c_{2,l})^{T}\,\textrm{for}\,\,l=1,..,4$
and $\left|\mathbf{a}_{l}\right|^{2}=\left|a_{1,l}\right|^{2}+\left|a_{2,l}\right|^{2}.$
Similarly we define $\left|\mathbf{b}_{l}\right|^{2}$ and $\left|\mathbf{c}_{l}\right|^{2}.$
Similar inequalities follow for the elements $\Omega_{2}^{l}$ in
$\Omega_{2}.$ Using the inequality (Theorem 4.79 of \cite{schwab})
\begin{eqnarray*}
\left\Vert q\right\Vert _{_{L^{\infty}(\bar{I})}}^{2}\leq C\,(lnW)\left\Vert q\right\Vert _{_{1/2,I}}^{2},
\end{eqnarray*}
where $C$ is a constant, and $q(y)$ be a polynomial of degree $W$
defined on $I=(-1,1)$, we obtain {\footnotesize{}
\begin{eqnarray*}
\sum_{k=1}^{m_{1}}\left\Vert \tilde{\mathbf{r}}_{1}^{k}(\xi,\eta)\right\Vert _{_{2,S}}^{2}\!\!\!+\sum_{l=1}^{m_{2}}\left\Vert \tilde{\mathbf{r}}_{2}^{l}(\xi,\eta)\right\Vert _{_{2,S}}^{2}\!\!\leq K(\ln W)\left(\sum_{i=1}^{2}\sum_{\gamma_{s}\subseteq\Omega_{i}}\!\!\left(\left\Vert [\mathbf{u}_{i}]\right\Vert _{_{0,\gamma_{s}}}^{2}\!\!+\left\Vert [(\mathbf{u}_{i})_{x}]\right\Vert _{_{\frac{1}{2},\gamma_{s}}}^{2}\!\!+\left\Vert [(\mathbf{u}_{i})_{y}]\right\Vert _{_{\frac{1}{2},\gamma_{s}}}^{2}\right)\right).
\end{eqnarray*}
}{\footnotesize\par}

Let
\begin{eqnarray*}
 &  & \mathbf{\tilde{y}}_{1}^{k}(\xi,\eta)=\mathbf{\tilde{u}}_{1}^{k}(\xi,\eta)+\tilde{\mathbf{r}}_{1}^{k}(\xi,\eta),\,\,\textrm{and \,\,}\mathbf{\tilde{y}}_{2}^{k}(\xi,\eta)=\mathbf{\tilde{u}}_{2}^{k}(\xi,\eta)+\tilde{\mathbf{r}}_{2}^{k}(\xi,\eta).
\end{eqnarray*}

Now we define a correction $\left\{ \left\{ \tilde{\mathbf{s}}_{1}^{k}(\xi,\eta)\right\} _{k},\left\{ \tilde{\mathbf{s}}_{2}^{l}(\xi,\eta)\right\} _{l}\right\} $
on the edges of $S=(M_{i}^{m})^{-1}(\Omega_{i}^{m})$ for all $i$
and $m$ such that $\tilde{\mathbf{s}}_{1}^{k}=0\,\,\,\textrm{and}\,\,\,\tilde{\mathbf{s}}_{2}^{l}=0\,\,\textrm{on}\,\,\Gamma_{0},$
$\mathbf{\tilde{s}}_{1}^{k},\mathbf{\tilde{s}}_{2}^{l}\in H^{2}(S).$
Let $\gamma_{1}$ be a side common to the two adjacent elements $(M_{1}^{m})^{-1}(\Omega_{1}^{m})$
and $(M_{1}^{n})^{-1}(\Omega_{1}^{n}).$

Let
\begin{eqnarray}
 &  & \mathbf{F}_{1}=\left.-\frac{1}{2}(\mathbf{\tilde{y}}_{1}^{m}-\mathbf{\tilde{y}}_{1}^{n})\right|_{\gamma_{1}},\label{2.6}\\
 &  & \mathbf{G}_{1}=\left.-\frac{1}{2}(\mathbf{\tilde{y}}_{1}^{m}-\mathbf{\tilde{y}}_{1}^{n})_{x}\right|_{\gamma_{1}},\,\,\textrm{and}\nonumber \\
 &  & \mathbf{H}_{1}=\left.-\frac{1}{2}(\mathbf{\tilde{y}}_{1}^{m}-\mathbf{\tilde{y}}_{1}^{n})_{y}\right|_{\gamma_{1}}.\nonumber 
\end{eqnarray}
 In the same way we define $\mathbf{F}_{l},\mathbf{G}_{l},\mathbf{H}_{l},\,l=2,3,4$
on all sides of $S.$ Each component of $\mathbf{F}_{l},\,\mathbf{G}_{l},\textrm{and}\,\,\mathbf{H}_{l}$
is a polynomial of degree $W$ that vanish at the end points of $\gamma_{l}$.
Define $\tilde{\mathbf{s}}_{1}^{k}(\xi,\eta)$ on $S$ such that $\left.\tilde{\mathbf{s}}_{1}^{k}\right|_{\gamma_{l}}=\mathbf{F}_{l},$
$\left.(\tilde{\mathbf{s}}_{1}^{k})_{x}\right|_{\gamma_{l}}=\mathbf{G}_{l}$
and $\left.(\tilde{\mathbf{s}}_{1}^{k})_{y}\right|_{\gamma_{l}}=\mathbf{H}_{l}$
for $l=1,..,4.$ Similarly, we define $\tilde{\mathbf{s}}_{2}^{l}(\xi,\eta)$
on $S$ 

Now using Theorem 1.5.2.4 of~ \cite{grisvard} , Theorem 4.82 in
\cite{schwab} and we have {\small{}
\begin{eqnarray*}
\sum_{k=1}^{m_{1}}\left\Vert \tilde{\mathbf{s}}_{1}^{k}(\xi,\eta)\right\Vert _{_{2,S}}^{2}+\sum_{l=1}^{m_{2}}\left\Vert \tilde{\mathbf{s}}_{2}^{l}(\xi,\eta)\right\Vert _{_{2,S}}^{2}\leq K(\ln W)^{2}\left(\sum_{i=1}^{2}\sum_{\gamma_{s}\subseteq\Omega_{i}}\left(\left\Vert [\mathbf{u}_{i}]\right\Vert _{_{0,\gamma_{s}}}^{2}+\left\Vert [(\mathbf{u}_{i})_{x}]\right\Vert _{_{\frac{1}{2},\gamma_{s}}}^{2}+\left\Vert [(\mathbf{u}_{i})_{y}]\right\Vert _{_{\frac{1}{2},\gamma_{s}}}^{2}\right)\right).
\end{eqnarray*}
}{\small\par}
\begin{flushleft}
Define $\mathbf{\tilde{v}}_{1}^{k}(\xi,\eta)=\mathbf{\tilde{y}}_{1}^{k}(\xi,\eta)+\tilde{\mathbf{s}}_{1}^{k}(\xi,\eta)$
and $\mathbf{\tilde{v}}_{2}^{l}(\xi,\eta)=\mathbf{\tilde{y}}_{2}^{l}(\xi,\eta)+\tilde{\mathbf{s}}_{2}^{l}(\xi,\eta)$.
Now, the estimate (5) follows from the above inequalities. \medskip{}
\\
\textbf{Lemma 2.2:} Let $\left\{ \left\{ \tilde{p}_{1}^{k}(\xi,\eta)\right\} _{k},\left\{ \tilde{p}_{2}^{_{^{l}}}(\xi,\eta)\right\} _{l}\right\} \in\Pi^{W}$.
Then there exists $\left\{ \left\{ \tilde{o}_{1}^{k}(\xi,\eta)\right\} _{k},\left\{ \tilde{o}_{2}^{_{^{l}}}(\xi,\eta)\right\} _{l}\right\} $
(where $\tilde{o}_{1}^{k}=0\text{ on }\Gamma_{0}$ for all $k=1,2,..,m_{1}$,
$\tilde{o}_{2}^{_{^{l}}}=0\text{ on }\Gamma_{0}$ for all $l=1,2,..,m_{2}$)
such that $o_{1}^{k}\in H^{1}(\Omega_{1}^{k}),o_{2}^{l}\in H^{1}(\Omega_{2}^{l}),$
$p_{1}^{*}=p_{1}+o_{1}\in H^{1}(\Omega_{1}),\,p_{2}^{*}=p_{2}+o_{2}\in H^{1}(\Omega_{2}).$
Moreover, the estimate 
\begin{align}
\sum_{k=1}^{m_{1}}\left\Vert \tilde{o}_{1}^{k}(\xi,\eta)\right\Vert _{_{1,S}}^{2}+\sum_{l=1}^{m_{2}}\left\Vert \tilde{o}_{2}^{l}(\xi,\eta)\right\Vert _{_{1,S}}^{2}\leq c(lnW)^{2}\left(\sum_{i=1}^{2}\sum_{\gamma_{s}\subseteq\Omega_{i}}\left\Vert [p_{i}]\right\Vert _{_{\frac{1}{2},\gamma_{s}}}^{2}\right)\label{eqn22}
\end{align}
holds. \\
Proof of this is very similar to the Lemma A.3 of \cite{S1}.\medskip{}
\\
\textbf{Lemma 2.3:} Let $\mathbf{w}_{i}=\mathbf{u}_{i}+\mathbf{v}_{i}\in\mathbf{H}^{2}(\Omega_{i})\text{ for }i=1,2.$
Here $\left\{ \left\{ \tilde{\mathbf{u}}_{1}^{k}(\xi,\eta)\right\} _{k},\left\{ \tilde{\mathbf{u}}_{2}^{_{^{l}}}(\xi,\eta)\right\} _{l}\right\} \in\Pi^{W}$
and $\left\{ \left\{ \tilde{\mathbf{v}}_{1}^{k}(\xi,\eta)\right\} _{k},\left\{ \tilde{\mathbf{v}}_{2}^{_{^{l}}}(\xi,\eta)\right\} _{l}\right\} $
is defined in Lemma 2.1. Then the estimate {\small{}
\begin{equation}
\left\Vert \mathbf{w}_{2}\right\Vert _{\frac{3}{2},\Gamma}^{2}\leq c(\ln W)^{2}\left(\sum_{\gamma_{s}\subseteq\Gamma}\left(\left\Vert \mathbf{u}_{2}\right\Vert _{0,\gamma_{s}}^{2}+\left\Vert \frac{\partial\mathbf{u}_{2}}{\partial T}\right\Vert _{\frac{1}{2},\gamma_{s}}^{2}\right)+\sum_{i=1}^{2}\sum_{\gamma_{s}\subseteq\Omega_{i}}\left(\left\Vert [\mathbf{u}_{i}]\right\Vert _{_{0,\gamma_{s}}}^{2}+\left\Vert [(\mathbf{u}_{i})_{x}]\right\Vert _{_{\frac{1}{2},\gamma_{s}}}^{2}+\left\Vert [(\mathbf{u}_{i})_{y}]\right\Vert _{_{\frac{1}{2},\gamma_{s}}}^{2}\right)\right)\label{eq:-10}
\end{equation}
holds. }\\
Proof of this follows from Lemma 7.2 of \cite{pd1}.\medskip{}
\\
Now, we prove the stability estimate. \medskip{}
\\
\textbf{Theorem 2.2:} For $W$ large enough there exists a constant
$c>0$ such that 
\begin{align}
 & {\displaystyle \sum_{k=1}^{m_{1}}}\left\Vert \tilde{\mathbf{u}}_{1}^{k}(\xi,\eta)\right\Vert _{_{2,S}}^{2}+\sum_{l=1}^{m_{2}}\left\Vert \tilde{\mathbf{u}}_{2}^{l}(\xi,\eta)\right\Vert _{_{2,S}}^{2}+\sum_{k=1}^{m_{1}}\left\Vert \tilde{p}_{1}^{k}(\xi,\eta)\right\Vert _{_{1,S}}^{2}+\sum_{l=1}^{m_{2}}\left\Vert \tilde{p}_{2}^{l}(\xi,\eta)\right\Vert _{_{1,S}}^{2}\nonumber \\
 & \leq c(\ln W)^{2}\mathcal{V}^{^{W}}\left(\left\{ \tilde{\mathbf{u}}_{1}^{k}(\xi,\eta)\right\} _{k},\left\{ \tilde{p}_{1}^{k}(\xi,\eta)\right\} _{k},\left\{ \tilde{\mathbf{u}}_{2}^{_{^{l}}}(\xi,\eta)\right\} _{l},\left\{ \tilde{p}_{2}^{_{^{l}}}(\xi,\eta)\right\} _{l}\right).
\end{align}
\textbf{Proof:} By the Lemma 2.1, there exists $\left\{ \left\{ \tilde{\mathbf{v}}_{1}^{k}(\xi,\eta)\right\} _{k},\left\{ \tilde{\mathbf{v}}_{2}^{_{^{l}}}(\xi,\eta)\right\} _{l}\right\} $
(where $\tilde{\mathbf{v}}_{1}^{k}=0\text{ on }\Gamma_{0}$ for all
$k=1,2,..,m_{1}$, $\tilde{\mathbf{v}}_{2}^{_{^{l}}}=0\text{ on }\Gamma_{0}$
for all $l=1,2,..,m_{2}$) such that $\mathbf{w}_{i}=\mathbf{u}_{i}+\mathbf{v}_{i}\in\mathbf{H}^{2}(\Omega_{i})\text{ for }i=1,2.$
Moreover $\mathbf{w}_{1}=\mathbf{u}_{1}$ and $\mathbf{w}_{2}=\mathbf{u}_{2}$
on the interface $\Gamma_{0}.$ Similarly by Lemma 2.2, there exists
$\left\{ \left\{ \tilde{o}_{1}^{k}(\xi,\eta)\right\} _{k},\left\{ \tilde{o}_{2}^{_{^{l}}}(\xi,\eta)\right\} _{l}\right\} $
(where $\tilde{o}_{1}^{k}=0\text{ on }\Gamma_{0}$ for all $k=1,2,..,m_{1}$,
$\tilde{o}_{2}^{_{^{l}}}=0\text{ on }\Gamma_{0}$ for all $l=1,2,..,m_{2}$)
such that for $p_{1}^{*}=p_{1}+o_{1}\in H^{1}(\Omega_{1}),p_{2}^{*}=p_{2}+o_{2}\in H^{1}(\Omega_{2}).$
\par\end{flushleft}

Hence by the regularity result (Eq. (3)) stated in section 2.1 
\begin{eqnarray}
 &  & \left\Vert \mathbf{w}_{1}\right\Vert _{2,\Omega_{1}}^{2}+\left\Vert \mathbf{w}_{2}\right\Vert _{2,\Omega_{2}}^{2}+\left\Vert p_{1}^{*}\right\Vert _{1,\Omega_{1}}^{2}+\left\Vert p_{2}^{*}\right\Vert _{1,\Omega_{2}}^{2}\nonumber \\
 &  & \leq c\left(\left\Vert \mathcal{L}_{1}(\mathbf{w}_{1},p_{1}^{*})\right\Vert _{0,\Omega_{1}}^{2}+\left\Vert \mathcal{L}_{2}(\mathbf{w}_{2},p_{2}^{*})\right\Vert _{0,\Omega_{2}}^{2}+\left\Vert \mathcal{D}_{1}(\mathbf{w}_{1})\right\Vert _{1,\Omega_{1}}^{2}+\left\Vert \mathcal{D}_{2}(\mathbf{w}_{2})\right\Vert _{1,\Omega_{2}}^{2}\right.\label{eq:-4}\\
 &  & \left.+\left\Vert \mathbf{w}_{2}\right\Vert _{\frac{3}{2},\Gamma}^{2}+\left\Vert [\mathbf{u}]\right\Vert _{\frac{3}{2},\Gamma_{0}}^{2}+\left\Vert [(\nu\nabla\mathbf{u}-pI)\mathbf{n}]\right\Vert _{\frac{1}{2},\Gamma_{0}}^{2}\right).\nonumber 
\end{eqnarray}

Now 
\begin{eqnarray*}
 &  & \left\Vert \mathcal{L}_{1}(\mathbf{w}_{1},p_{1}^{*})\right\Vert _{0,\Omega_{1}}^{2}+\left\Vert \mathcal{L}_{2}(\mathbf{w}_{2},p_{2}^{*})\right\Vert _{0,\Omega_{2}}^{2}\\
 &  & \leq C\left(\sum_{k=1}^{m_{1}}\left\Vert (\mathcal{L}_{1}^{k})(\tilde{\mathbf{u}}_{1}^{k},\tilde{p}_{1}^{k})(\xi,\eta)\right\Vert _{_{0,S}}^{2}+\sum_{l=1}^{m_{2}}\left\Vert (\mathcal{L}_{2}^{l})(\tilde{\mathbf{u}}_{2}^{l},\tilde{p}_{2}^{l})(\xi,\eta)\right\Vert _{_{0,S}}^{2}\right)\\
 &  & +c\left(\sum_{k=1}^{m_{1}}\left\Vert \tilde{\mathbf{v}}_{1}^{k}(\xi,\eta)\right\Vert _{_{2,S}}^{2}+\sum_{l=1}^{m_{2}}\left\Vert \tilde{\mathbf{v}}_{2}^{_{^{l}}}(\xi,\eta)\right\Vert _{_{2,S}}^{2}+\sum_{k=1}^{m_{1}}\left\Vert \tilde{o}_{1}^{k}(\xi,\eta)\right\Vert _{_{1,S}}^{2}+\sum_{l=1}^{m_{2}}\left\Vert \tilde{o}_{2}^{l}(\xi,\eta)\right\Vert _{_{1,S}}^{2}\right).
\end{eqnarray*}

Also 
\begin{eqnarray*}
 &  & \left\Vert \mathcal{D}_{1}(\mathbf{w}_{1})\right\Vert _{1,\Omega_{1}}^{2}+\left\Vert \mathcal{D}_{2}(\mathbf{w}_{2})\right\Vert _{1,\Omega_{2}}^{2}\\
 &  & \leq C\left(\sum_{k=1}^{m_{1}}||\mathcal{D}_{1}^{k}\tilde{\mathbf{u}}_{1}^{k}||_{1,S}^{2}+\sum_{l=1}^{m_{2}}||\mathcal{D}_{2}^{l}\tilde{\mathbf{u}}_{2}^{l}||_{1,S}^{2}\right)+c\left(\sum_{k=1}^{m_{1}}\left\Vert \tilde{\mathbf{v}}_{1}^{k}(\xi,\eta)\right\Vert _{_{2,S}}^{2}+\sum_{l=1}^{m_{2}}\left\Vert \tilde{\mathbf{v}}_{2}^{_{^{l}}}(\xi,\eta)\right\Vert _{_{2,S}}^{2}\right).
\end{eqnarray*}

The rest of the proof follows from Lemma 2.1, 2.2, and 2.3. 

\section{Numerical formulation and error estimate}

\subsection{Numerical Formulation}

Here we define least-squares minimizing functional based on the stability
estimate derived in the last section. 

Define $\mathbf{f}_{1}=\mathbf{f}\mid_{\Omega_{1}}$ and $\mathbf{f}_{2}=\mathbf{f}\mid_{\Omega_{2}}.$
Let $J_{1}^{l}(\xi,\eta)$ be the Jacobian of the mapping $M_{1}^{l}(\xi,\eta)$
from $S=(-1,1)^{2}$ to $\Omega_{1}^{l}$ for $l=1,2,...,m_{1}.$
Let $\tilde{\mathbf{f}}_{1}^{l}(\xi,\eta)=\mathbf{f}_{1}(M_{1}^{l}(\xi,\eta))$
and define
\[
\mathbf{F}_{1}^{l}(\xi,\eta)=\tilde{\mathbf{f}}_{1}^{l}(\xi,\eta)\sqrt{J_{1}^{l}(\xi,\eta)},
\]
for $l=1,2,...,m_{1}.$ Similarly one can define $\mathbf{F}_{2}^{k}(\xi,\eta)$
for $k=1,2,...,m_{2}.$

On the interface $\Gamma_{0}$ we have $[\mathbf{u}]=0$ and $[(\nu\nabla\mathbf{u}-pI)\mathbf{n}]=\mathbf{g}.$
Let $\gamma_{s}\subseteq\Gamma_{0}$ be the image of $\xi=1$ under
the mapping $M_{1}^{n}$ which maps $S$ to $\Omega_{1}^{n}$ and
also the image of $\xi=-1$ under the mapping $M_{2}^{m}$ which maps
$S$ to $\Omega_{2}^{m}.$ Let 
\[
\mathbf{l}^{m,n}(\eta)=\mathbf{g}^{m}(-1,\eta)=\mathbf{g}^{n}(1,\eta)\text{ for }-1\leq\eta\leq1.
\]
Define the functional
\begin{eqnarray}
 &  & \mathit{\mathsf{\mathbf{r}}}^{W}\left(\left\{ \tilde{\mathbf{u}}_{1}^{k}(\xi,\eta)\right\} _{k},\left\{ \tilde{p}_{1}^{k}(\xi,\eta)\right\} _{k},\left\{ \tilde{\mathbf{u}}_{2}^{_{^{l}}}(\xi,\eta)\right\} _{l},\left\{ \tilde{p}_{2}^{_{^{l}}}(\xi,\eta)\right\} _{l}\right)\nonumber \\
 &  & =\sum_{k=1}^{m_{1}}\left\Vert (\mathcal{L}_{1}^{k})(\tilde{\mathbf{u}}_{1}^{k},\tilde{p}_{1}^{k})(\xi,\eta)-\mathbf{F}_{1}^{k}(\xi,\eta)\right\Vert _{_{0,S}}^{2}+\sum_{l=1}^{m_{2}}\left\Vert (\mathcal{L}_{2}^{l})(\tilde{\mathbf{u}}_{2}^{l},\tilde{p}_{2}^{l})(\xi,\eta)-\mathbf{F}_{2}^{l}(\xi,\eta)\right\Vert _{_{0,S}}^{2}\nonumber \\
 &  & +\sum_{k=1}^{m_{1}}||\mathcal{D}_{1}^{k}\tilde{\mathbf{u}}_{1}^{k}||_{1,S}^{2}+\sum_{l=1}^{m_{2}}||\mathcal{D}_{2}^{l}\tilde{\mathbf{u}}_{2}^{l}||_{1,S}^{2}\nonumber \\
 &  & +\sum_{i=1}^{2}\sum_{\gamma_{s}\subseteq\Omega_{i}}\left(\left\Vert [\mathbf{u}_{i}]\right\Vert _{_{0,\gamma_{s}}}^{2}+\left\Vert [(\mathbf{u}_{i})_{x}]\right\Vert _{_{\frac{1}{2},\gamma_{s}}}^{2}+\left\Vert [(\mathbf{u}_{i})_{y}]\right\Vert _{_{\frac{1}{2},\gamma_{s}}}^{2}\right)+\sum_{i=1}^{2}\sum_{\gamma_{s}\subseteq\Omega_{i}}\left\Vert [p_{i}]\right\Vert _{_{\frac{1}{2},\gamma_{s}}}^{2}\nonumber \\
 &  & +\sum_{\gamma_{s}\subseteq\Gamma_{0}}\left(\left\Vert [\mathbf{u}]\right\Vert _{_{\frac{3}{2},\gamma_{s}}}^{2}+\left\Vert [(\nu\nabla\mathbf{u}-pI)\mathbf{n}]-\mathbf{l}^{m,n}\right\Vert _{\frac{1}{2},\gamma_{s}}^{2}\right)\label{eq:-11}\\
 &  & +\sum_{\gamma_{s}\subseteq\Gamma}\left(\left\Vert \mathbf{u}_{2}\right\Vert _{0,\gamma_{s}}^{2}+\left\Vert \left(\frac{\partial\mathbf{u}_{2}}{\partial T}\right)\right\Vert _{\frac{1}{2},\gamma_{s}}^{2}\right).\nonumber 
\end{eqnarray}

The approximate solution is chosen as the unique {\footnotesize{}$\left\{ \left\{ \tilde{\mathbf{z}}_{1}^{k}(\xi,\eta)\right\} _{k},\left\{ \tilde{q}_{1}^{k}(\xi,\eta)\right\} _{k},\left\{ \tilde{\mathbf{z}}_{2}^{l}(\xi,\eta)\right\} _{l},\left\{ \tilde{q}_{2}^{l}(\xi,\eta)\right\} _{l}\right\} \in\Pi^{W},$}
which minimizes the functional {\small{}$\mathfrak{\mathcal{\mathcal{\mathfrak{r}}}}^{^{W}}(\left\{ \tilde{\mathbf{u}}_{1}^{k}(\xi,\eta)\right\} _{k},\left\{ \tilde{p}_{1}^{k}(\xi,\eta)\right\} _{k},\left\{ \tilde{\mathbf{u}}_{2}^{_{^{l}}}(\xi,\eta)\right\} _{l},\left\{ \tilde{p}_{2}^{_{^{l}}}(\xi,\eta)\right\} _{l})$}
over all \\
{\small{}$\left\{ \left\{ \tilde{\mathbf{u}}_{1}^{k}(\xi,\eta)\right\} _{k},\left\{ \tilde{p}_{1}^{k}(\xi,\eta)\right\} _{k},\left\{ \tilde{\mathbf{u}}_{2}^{_{^{l}}}(\xi,\eta)\right\} _{l},\left\{ \tilde{p}_{2}^{_{^{l}}}(\xi,\eta)\right\} _{l}\right\} .$}{\small\par}

The minimization problem leads to a system of equations of the form
\begin{equation}
AZ=h.\label{sysofeqs}
\end{equation}
Here $A$ is a symmetric, positive-definite matrix, and the vector
$Z$ is composed of the values of spectral element functions at Gauss-Legendre-Lobatto
(GLL) points. This system is solved using the preconditioned conjugate
gradient method (PCGM). Since each iteration of the PCGM requires
the action of the matrix on a vector, the action of the matrix on
a vector is obtained accurately and efficiently without storing the
matrix $A.$ We use the following quadratic form:

\begin{align}
\mathcal{U}^{W}=\sum_{k=1}^{m_{1}}||\tilde{\mathbf{u}}_{1}^{k}(\xi,\eta)||_{2,S}^{2}+\sum_{k=1}^{m_{1}}||\tilde{p}_{1}^{k}(\xi,\eta)||_{1,S}^{2}+\sum_{l=1}^{m_{2}}||\tilde{\mathbf{u}}_{2}^{l}(\xi,\eta)||_{2,S}^{2}+\sum_{l=1}^{m_{2}}||\tilde{p}_{2}^{l}(\xi,\eta)||_{1,S}^{2}\label{eq3.4d-1}
\end{align}
as a preconditioner which is spectrally equivalent to $\mathcal{V}^{W}.$

\subsection{Error Estimate}

To prove the error estimate, we assume that solution of the interface
problem (1)-(2) is very smooth in the individual subdomains $\Omega_{1}$
and $\Omega_{2}.$ Assume that $\mathbf{u}_{1}\in\mathbf{H}^{k}(\Omega_{1})$
and $\mathbf{u}_{2}\in\mathbf{H}^{k}(\Omega_{2})$ for $k\geq2.$
As defined in section 2.2, $M_{i}^{l}$ is an analytic map from $S=(-1,1)^{2}$
to $\Omega_{i}^{l}.$ Denote $\mathbf{U}_{i}^{k}(\xi,\eta)=\left(U_{1}^{i,k}(\xi,\eta),U_{2}^{i,k}(\xi,\eta)\right)^{T}=\mathbf{u}\left(M_{i}^{k}(\xi,\eta)\right)$
and $P_{i}^{k}(\xi,\eta)=p\left(M_{i}^{k}(\xi,\eta)\right)$ for $(\xi,\eta)\in S,$
where $k=1,2,..,m_{1}$ when $i=1$ and $k=1,2,..,m_{2}$ when $i=2.$
Then $U_{1}^{i,k}(\xi,\eta),U_{2}^{i,k}(\xi,\eta)$ and $P_{i}^{k}$
are analytic on $\bar{S}$ and for some constants $C$ and $d,$ we
have \cite{babuska}
\begin{eqnarray}
\left|D^{\alpha}U_{j}^{i,k}(\xi,\eta)\right|\leq C\,m!d^{m}\label{eq:-1}\\
\left|D^{\alpha}P_{i}^{k}(\xi,\eta)\right|\leq C\,m!d^{m}\label{eq:-2}
\end{eqnarray}
 for $i,j=1,2$ and $|\alpha|=m,m=1,2,....$\\
\\
\textbf{Theorem 3.1:} Let $(\tilde{\mathbf{z}},\tilde{q})$ minimize
$\mathit{\mathsf{\mathbf{r}}}^{W}$. Then for $W$ large enough there
exists constants $C$ and $b$ (being independent of $W$) such that
the estimate 
\begin{align}
\sum_{i=1}^{2}\left(\sum_{k=1}^{L}||\mathbf{\tilde{z}}_{i}^{k}(\hat{\mathbf{x}})-\mathbf{U}_{i}^{k}(\hat{\mathbf{x}})||_{2,S}^{2}+\sum_{k=1}^{L}||\tilde{q}_{i}^{k}(\hat{\mathbf{x}})-P_{i}^{k}(\hat{\mathbf{x}})||_{1,S}^{2}\right)\leq Ce^{-bW}
\end{align}
holds true. Here $\hat{\mathbf{x}}=\left(\xi,\eta\right)$ and $(\tilde{\mathbf{z}},\tilde{q})=\left\{ \left\{ \tilde{\mathbf{z}}_{1}^{k}(\xi,\eta)\right\} _{k},\left\{ \tilde{q}_{1}^{k}(\xi,\eta)\right\} _{k},\left\{ \tilde{\mathbf{z}}_{2}^{_{^{l}}}(\xi,\eta)\right\} _{l},\left\{ \tilde{q}_{2}^{_{^{l}}}(\xi,\eta)\right\} _{l}\right\} .$
Also $L=m_{1}$ if $i=1$ and $L=m_{2}$ if $i=2.$ \\
\textbf{Proof :} From the approximation results in \cite{schwab},
there exists a polynomial $\Phi(\xi,\eta)$ of degree $W$ in each
variable separately such that 
\begin{align}
\|v(\xi,\eta)-\Phi(\xi,\eta)\|_{n,S}^{2}\leq c_{s}W^{4+2n-2s}\|v\|_{s,S}^{2}\label{eq:-3}
\end{align}
for $0\leq n\leq2$ and all $W>s,$ where $c_{s}=ce^{2s}.$

Hence there exist polynomials $\mathbf{\mathbf{\Psi}}_{i}^{k}(\xi,\eta)=\left(\Psi_{1}^{i,k}(\xi,\eta),\Psi_{2}^{i,k}(\xi,\eta)\right)^{T}\,\,\textrm{and}\,\,\Phi_{i}^{k}(\xi,\eta)$,
$i=1,2$ such that (using equations (13),(14) and (16))
\begin{align}
 & \left\Vert \mathbf{\mathbf{U}}_{i}^{k}(\xi,\eta)-\mathbf{\mathbf{\Psi}}_{i}^{k}(\xi,\eta)\right\Vert _{2,S}^{2}\leq c_{s}W^{-2s+8}(Cd^{s}s!)^{2},\quad\!\!\!\mbox{for}\quad\!\!\!i=1,2\nonumber \\
 & \left\Vert P_{i}^{k}(\xi,\eta)-\Phi_{i}^{k}(\xi,\eta)\right\Vert _{1,S}^{2}\leq c_{s}W^{-2s+6}(Cd^{s}s!)^{2}\quad\!\!\!\mbox{for}\quad\!\!\!i=1,2.\label{eq:-5}
\end{align}
Consider the set of functions{\small{} $\Bigg\{\left\{ \mathbf{\mathbf{\Psi}}_{i}^{k}(\xi,\eta)\right\} ,\left\{ \Phi_{i}^{k}(\xi,\eta)\right\} \Bigg\}.$}
Now estimate {\small{}$\mathit{\mathsf{\mathbf{r}}}^{W}\Bigg(\left\{ \mathbf{\mathbf{\Psi}}_{i}^{k}(\xi,\eta)\right\} ,\left\{ \Phi_{i}^{k}(\xi,\eta)\right\} \Bigg).$ }{\small\par}

Using (17), we get 
\begin{eqnarray*}
 & \left\Vert (\mathcal{L}_{i}^{k})(\mathbf{\mathbf{\Psi}}_{i}^{k},\Phi_{i}^{k})(\xi,\eta)-\mathbf{F}_{i}^{k}(\xi,\eta)\right\Vert _{_{0,S}}^{2} & =\left\Vert (\mathcal{L}_{i}^{k})(\mathbf{\mathbf{\Psi}}_{i}^{k},\Phi_{i}^{k})(\xi,\eta)-(\mathcal{L}_{i}^{k})(\mathbf{\mathbf{U}}_{i}^{k},P_{i}^{k})(\xi,\eta)\right\Vert _{_{0,S}}^{2}\\
 &  & \leq C\left(c_{s}W^{-2s+8}(Cd^{s}s!)^{2}+c_{s}W^{-2s+6}(Cd^{s}s!)^{2}\right).
\end{eqnarray*}

for $i=1,2.$ Also 
\begin{eqnarray*}
 & ||\mathcal{D}_{i}^{k}\mathbf{\mathbf{\Psi}}_{i}^{k}||_{1,S}^{2}=||\mathcal{D}_{i}^{k}\mathbf{\mathbf{\Psi}}_{i}^{k}-\mathcal{D}_{i}^{k}\mathbf{U}_{i}^{k}||_{1,S}^{2} & \leq c_{s}W^{-2s+8}(Cd^{s}s!)^{2}.
\end{eqnarray*}

As shown in theorem 3.1 of \cite{tomar}, we can show that
\begin{eqnarray*}
 &  & \sum_{i=1}^{2}\sum_{\gamma_{s}\subseteq\Omega_{i}}\left(\left\Vert [\mathbf{\mathbf{\Psi}}_{i}]\right\Vert _{_{0,\gamma_{s}}}^{2}+\left\Vert [(\mathbf{\mathbf{\Psi}}_{i})_{x}]\right\Vert _{_{\frac{1}{2},\gamma_{s}}}^{2}+\left\Vert [(\mathbf{\mathbf{\Psi}}_{i})_{y}]\right\Vert _{_{\frac{1}{2},\gamma_{s}}}^{2}\right)+\sum_{i=1}^{2}\sum_{\gamma_{s}\subseteq\Omega_{i}}\left\Vert [\Phi_{i}]\right\Vert _{_{\frac{1}{2},\gamma_{s}}}^{2}\\
 &  & \leq C\left(c_{s}W^{-2s+8}(Cd^{s}s!)^{2}+c_{s}W^{-2s+6}(Cd^{s}s!)^{2}\right).
\end{eqnarray*}

Similarly, we can bound the other terms in $\mathit{\mathsf{\mathbf{r}}}^{W}\Bigg(\left\{ \mathbf{\Psi}_{i}^{k}(\xi,\eta)\right\} ,\left\{ \Phi_{i}^{k}(\xi,\eta)\right\} \Bigg)$.
So by combining all the terms, we have 
\begin{eqnarray*}
\mathit{\mathsf{\mathbf{r}}}^{W}\Bigg(\left\{ \mathbf{\Psi}_{i}^{k}(\xi,\eta)\right\} ,\left\{ \Phi_{i}^{k}(\xi,\eta)\right\} \Bigg)\leq C\Bigg(c_{s}W^{-2s+8}(Cd^{s}s!)^{2}+c_{s}W^{-2s+6}(Cd^{s}s!)^{2}\Bigg).
\end{eqnarray*}

Since $(\tilde{\mathbf{z}},\tilde{q})$ is the minimizer of $\mathit{\mathsf{\mathbf{r}}}^{W}\left(\left\{ \tilde{\mathbf{z}}_{1}^{k}(\xi,\eta)\right\} _{k},\left\{ \tilde{q}_{1}^{k}(\xi,\eta)\right\} _{k},\left\{ \tilde{\mathbf{z}}_{2}^{_{^{l}}}(\xi,\eta)\right\} _{l},\left\{ \tilde{q}_{2}^{_{^{l}}}(\xi,\eta)\right\} _{l}\right)$,
we have 
\begin{align*}
 & \mathit{\mathsf{\mathbf{r}}}^{W}\Bigg(\left\{ \tilde{\mathbf{z}}_{1}^{k}(\xi,\eta)\right\} _{k},\left\{ \tilde{q}_{1}^{k}(\xi,\eta)\right\} _{k},\left\{ \tilde{\mathbf{z}}_{2}^{_{^{l}}}(\xi,\eta)\right\} _{l},\left\{ \tilde{q}_{2}^{_{^{l}}}(\xi,\eta)\right\} _{l}\Bigg)\\
 & \leq C\Bigg(c_{s}W^{-2s+8}(Cd^{s}s!)^{2}+c_{s}W^{-2s+6}(Cd^{s}s!)^{2}\Bigg).
\end{align*}

By using Sterling's formula, techniques from theorem 3.1 of \cite{tomar},
and stability estimate Eq. (8) we can see that there exists a constant
$b>0$ such that the estimate 
\begin{align}
\sum_{i=1}^{2}\left(\sum_{k=1}^{L}||\mathbf{\tilde{z}}_{i}^{k}(\hat{\mathbf{x}})-\mathbf{\Psi}_{i}^{k}(\hat{\mathbf{x}})||_{2,S}^{2}+\sum_{k=1}^{L}||\tilde{q}_{i}^{k}(\hat{\mathbf{x}})-\Phi_{i}^{k}(\hat{\mathbf{x}})||_{1,S}^{2}\right)\leq ce^{-bW}.\label{eq:-6}
\end{align}
holds. Also, 
\begin{eqnarray}
\sum_{i=1}^{2}\left(\sum_{k=1}^{L}||\mathbf{U}_{i}^{k}(\hat{\mathbf{x}})-\mathbf{\Psi}_{i}^{k}(\hat{\mathbf{x}})||_{2,S}^{2}+\sum_{k=1}^{L}||P_{i}^{k}(\hat{\mathbf{x}})-\Phi_{i}^{k}(\hat{\mathbf{x}})||_{1,S}^{2}\right)\leq ce^{-bW}.\label{eq:-7}
\end{eqnarray}
Therefore, the required error estimate easily follows from (18) and
(19).

\textbf{\textit{$\!\!\!\!\!\!\!\!\!$Remark:}}\textit{ After obtaining
a nonconforming solution, a set of corrections can be made such that
the velocity variable $\mathbf{z}$ becomes conforming \cite{tomar}.
So $\mathbf{z}\in\mathbf{H}^{1}(\Omega)$ and we have the following
error estimate: }
\begin{align}
||\mathbf{u}-\mathbf{z}||_{1,\Omega}+||p-q||_{0,\Omega}\leq Ce^{-bW}.
\end{align}


\section{Numerical results}

In this section, we present the numerical results. Exponential convergence
of the numerical scheme is verified through various numerical examples.
The numerical examples include the Stokes interface problem on different
types of domains with different interfaces. We have considered the
interface problems with homogeneous as well as non-homogeneous jump
conditions across the interface. Even though we have derived the theoretical
estimates for the interface problem with Dirichlet boundary conditions,
here we also present the numerical results for the interface problem
with mixed boundary conditions. The numerical solution is exponentially
accurate in all the cases.

Higher-order spectral element functions of degree $W$ are used uniformly
in all the elements of the discretization. The nonconforming solution
is obtained using PCGM at GLL points, and the conforming solution
is obtained using a set of corrections. Let $\mathbf{z}$ be the conforming
approximate solution of the velocity $\mathbf{u}$ and $q$ be the
approximate solution of the pressure $p.$ $\|E_{\mathbf{u}}\|_{1}=\frac{\left\Vert \mathbf{u}-\mathbf{z}\right\Vert _{1}}{\left\Vert \mathbf{u}\right\Vert _{1}}$
denotes the relative error in $\mathbf{u}$ in $\mathbf{H}^{1}$ norm,
$\|E_{p}\|_{0}=\frac{\left\Vert p-q\right\Vert _{0}}{\left\Vert p\right\Vert _{0}}$
denotes the relative error in pressure in $L^{2}$ norm, and similarly $\|E_{c}\|_{0}$
denotes the relative error in the continuity equation in $L^{2}$ norm. '\textbf{iters}'
denotes the total number of iterations required to reach the desired
accuracy. To ensure the uniqueness of the pressure variable in the
problems with Dirichlet boundary conditions, pressure is specified
to be zero at one point of the domain. 

\subsubsection*{Example 1: With homogeneous jump conditions on the interface}

Let, $\Omega=[0,1]^{2}$, consider the Stokes interface problem with
a straight line interface $\Gamma_{0}=\{(x,y):y=0.5\}$ and with Dirichlet
boundary condition on the boundary. The interface divides the domain
$\Omega$ into two subdomains $\Omega_{1}=\{(x,y)\in\Omega:0<y<0.5\}\text{ and }\Omega_{2}=\{(x,y)\in\Omega:0.5<y<1\}$.
The data is chosen such that the problem has the following exact solution:
\begin{eqnarray*}
u_{1}(x,y)=\begin{cases}
\frac{1}{\nu_{1}}(y-0.5)x^{2}\text{ in }\Omega_{1},\\
\frac{1}{\nu_{2}}(y-0.5)x^{2}\text{ in }\Omega_{2},
\end{cases}u_{2}(x,y)=\begin{cases}
-\frac{1}{\nu_{1}}x(y-0.5)^{2}\text{ }\text{in }\Omega_{1},\\
-\frac{1}{\nu_{2}}x(y-0.5)^{2}\text{ in }\Omega_{2},
\end{cases}
\end{eqnarray*}
\[
p(x,y)=e^{x}-e^{y}+c.
\]
Here, $c$ is chosen such that pressure will be zero at one point
in the domain. The solution satisfies homogeneous jump conditions
(see (2)) across the interface $y=0.5$.

To obtain the numerical solution, the domain is divided into four
elements such that the discretization matches along the interface
$y=0.5$. The numerical solution for interface problem with different
pairs of viscosity coefficients $\nu_{1}=1,\nu_{2}=0.1,\text{ }\nu_{1}=1,\nu_{2}=0.01$
and $\nu_{1}=1,\nu_{2}=0.001$ is obtained for different values of
$W$. The relative error in $\mathbf{u}$ in $\mathbf{H}^{1}$ norm,
relative error in pressure in $L^{2}$ norm, error in the continuity
equation in $L^{2}$ norm and the total number of iterations ("\textbf{iters}")
required to reach the desired accuracy are tabulated in Table $1$
for different values of $W$. 

\begin{table}[H]
~~~~~~~~~~~~~{\small{}}%
\begin{tabular}{|c|c|c|c|c|c|c|c|c|}
\hline 
\multicolumn{1}{|c}{} & \multicolumn{4}{c|}{{\small{}${\nu_{1}}=1$, $\nu_{2}=0.1$}} & \multicolumn{4}{c|}{{\small{}${\nu_{1}}=1$, $\nu_{2}=0.01$}}\tabularnewline
\hline 
{\small{}$\textbf{W}$} & {\small{}$\rVert\textit{E}_{\textbf{u}}\rVert_{1}$} & {\small{}$\rVert\textit{E}_{\textit{p}}\rVert_{0}$} & {\small{}$\rVert\textit{E}_{\textit{c}}\rVert_{0}$} & \textbf{\small{}iters} & {\small{}$\rVert\textit{E}_{\textbf{u}}\rVert_{1}$} & {\small{}$\rVert\textit{E}_{\textit{p}}\rVert_{0}$} & {\small{}$\rVert\textit{E}_{\textit{c}}\rVert_{0}$} & \textbf{\small{}iters}\tabularnewline
\hline 
{\small{}2} & {\small{}3.50E-02} & {\small{}3.01E-01} & {\small{}1.11E-01} & {\small{}30} & {\small{}2.81E-02} & {\small{}4.19E-01} & {\small{}8.15E-01} & {\small{}46}\tabularnewline
\hline 
{\small{}3} & {\small{}6.07E-03} & {\small{}5.07E-02} & {\small{}1.70E-02} & {\small{}98} & {\small{}4.90E-03} & {\small{}4.82E-02} & {\small{}7.64E-02} & {\small{}151}\tabularnewline
\hline 
{\small{}4} & {\small{}7.65E-04} & {\small{}9.39E-03} & {\small{}2.21E-03} & {\small{}228} & {\small{}6.41E-04} & {\small{}6.16E-03} & {\small{}1.12E-02} & {\small{}290}\tabularnewline
\hline 
{\small{}5} & {\small{}6.68E-05} & {\small{}8.57E-04} & {\small{}2.56E-04} & {\small{}399} & {\small{}4.94E-05} & {\small{}4.86E-04} & {\small{}7.59E-04} & {\small{}648}\tabularnewline
\hline 
{\small{}6} & {\small{}5.80E-06} & {\small{}8.25E-05} & {\small{}2.04E-05} & {\small{}744} & {\small{}3.69E-06} & {\small{}4.96E-05} & {\small{}5.20E-05} & {\small{}1050}\tabularnewline
\hline 
{\small{}7} & {\small{}6.39E-07} & {\small{}1.01E-05} & {\small{}2.07E-06} & {\small{}1273} & {\small{}5.01E-07} & {\small{}4.13E-06} & {\small{}6.21E-06} & {\small{}1474}\tabularnewline
\hline 
{\small{}8} & {\small{}4.57E-08} & {\small{}1.17E-06} & {\small{}1.64E-07} & {\small{}2115} & {\small{}2.24E-08} & {\small{}6.14E-07} & {\small{}5.67E-07} & {\small{}2210}\tabularnewline
\hline 
\end{tabular}{\small\par}

{\small{}~~~~~~~~~~~~~~~~~~~~~~~~~~~~~~~~~~~~~~~~~~~}%
\begin{tabular}{|c|c|c|c|c|}
\hline 
\multicolumn{5}{|c|}{{\small{}${\nu_{1}}=1$, $\nu_{2}=0.001$}}\tabularnewline
\hline 
{\small{}$\textbf{W}$} & {\small{}$\rVert\textit{E}_{\textbf{u}}\rVert_{1}$} & {\small{}$\rVert\textit{E}_{\textit{p}}\rVert_{0}$} & {\small{}$\rVert\textit{E}_{\textit{c}}\rVert_{0}$} & \textbf{\small{}iters}\tabularnewline
\hline 
{\small{}2} & {\small{}2.03E-01} & {\small{}1.13E+01} & {\small{}7.25E+01} & {\small{}8}\tabularnewline
\hline 
{\small{}3} & {\small{}1.39E-02} & {\small{}3.99E+00} & {\small{}3.27E+00} & {\small{}77}\tabularnewline
\hline 
{\small{}4} & {\small{}4.03E-03} & {\small{}5.82E-01} & {\small{}1.35E+00} & {\small{}157}\tabularnewline
\hline 
{\small{}5} & {\small{}1.16E-03} & {\small{}2.48E-02} & {\small{}8.63E-02} & {\small{}413}\tabularnewline
\hline 
{\small{}6} & {\small{}4.23E-04} & {\small{}5.48E-03} & {\small{}2.91E-02} & {\small{}662}\tabularnewline
\hline 
{\small{}7} & {\small{}1.43E-04} & {\small{}7.54E-04} & {\small{}2.92E-03} & {\small{}2603}\tabularnewline
\hline 
{\small{}8} & {\small{}1.55E-05} & {\small{}8.88E-05} & {\small{}2.35E-04} & {\small{}4455}\tabularnewline
\hline 
\end{tabular}{\small\par}

\caption{The relative errors {\small{}$\rVert\textit{E}_{\textbf{u}}\rVert_{1},\rVert\textit{E}_{\textit{p}}\rVert_{0}$}
and {\small{}$\rVert\textit{E}_{\textit{c}}\rVert_{0}$ against $W$ }}

\end{table}

Table 1 shows that the relative errors decay very fast as $W$ increases,
and the number of iterations is increasing as the ratio of the viscosity
coefficients $\frac{\nu_{1}}{\nu_{2}}$ increases. In the cases of
$\frac{\nu_{1}}{\nu_{2}}=10$ and $\frac{\nu_{1}}{\nu_{2}}=100,$
the approximate solution $\mathbf{z}$ is obtained to an accuracy
of $O(10^{-6})$ with a smaller number of iterations (with $W=6$),
but the iteration count is increased to obtain $\mathbf{z}$ to an
accuracy of $O(10^{-8}).$ In the case of $\frac{\nu_{1}}{\nu_{2}}=1000,$
one can see that the approximate solution $\mathbf{z}$ is obtained
to an accuracy of $O(10^{-4})$ with a smaller number of iterations
(with $W=6$). The graph between the log of relative error and the
degree of polynomial $W$ is shown for $\nu_{1}=1,\nu_{2}=0.1,\text{ and }\nu_{1}=1,\nu_{2}=0.01$
for velocity and pressure variables in figures 2.a and 2.b respectively.
The graph is almost linear, which shows the exponential accuracy of
the proposed method. 
\begin{figure}[H]
\begin{centering}
\subfloat[Log of $\|E_{\mathbf{u}}\|_{1}$ vs. $W$ for $\upsilon_{2}=0.1,0.01$]{\begin{centering}
\includegraphics[width=8cm,height=5cm]{P1_velocity}
\par\end{centering}
}\subfloat[Log of $\|E_{p}\|_{0}$ vs. $W$ for $\upsilon_{2}=0.1,0.01$]{\begin{centering}
\includegraphics[width=8cm,height=5cm]{P1_Pressure}
\par\end{centering}
}
\par\end{centering}
\caption{Log of relative errors against $W$}
\end{figure}

We have interchanged the viscosity coefficient data and obtained the
numerical solution. Table 2 shows the relative errors {\small{}$\rVert\textit{E}_{\textbf{u}}\rVert_{1},\rVert\textit{E}_{\textit{p}}\rVert_{0}$
and the error $\rVert\textit{E}_{\textit{c}}\rVert_{0}$ against $W$
for }$\frac{\nu_{2}}{\nu_{1}}=10$ and $\frac{\nu_{2}}{\nu_{1}}=100.$
The results show the exponential convergence of the numerical scheme. 

\begin{table}[H]
~~~%
\begin{tabular}{|c|c|c|c|c|c|c|c|c|}
\hline 
\multicolumn{1}{|c}{} & \multicolumn{4}{c|}{{\small{}${\nu_{1}}=0.1$, $\nu_{2}=1$}} & \multicolumn{4}{c|}{{\small{}${\nu_{1}}=0.01$, $\nu_{2}=1$}}\tabularnewline
\hline 
{\small{}$W$} & {\small{}$\rVert\textit{E}_{\textbf{u}}\rVert_{1}$} & {\small{}$\rVert\textit{E}_{\textit{p}}\rVert_{0}$} & {\small{}$\rVert\textit{E}_{\textit{c}}\rVert_{0}$} & \textbf{\small{}iters} & {\small{}$\rVert\textit{E}_{\textbf{u}}\rVert_{1}$} & {\small{}$\rVert\textit{E}_{\textit{p}}\rVert_{0}$} & {\small{}$\rVert\textit{E}_{\textit{c}}\rVert_{0}$} & \textbf{\small{}iters}\tabularnewline
\hline 
{\small{}2} & {\small{}4.36E-02} & {\small{}5.73E-01} & {\small{}1.57E-01} & {\small{}32} & {\small{}1.88E-01} & {\small{}7.57E-01} & {\small{}7.16E+00} & 9\tabularnewline
\hline 
{\small{}3} & {\small{}7.84E-03} & {\small{}1.55E-01} & {\small{}2.15E-02} & {\small{}93} & {\small{}5.97E-03} & {\small{}4.46E-01} & {\small{}1.19E-01} & 139\tabularnewline
\hline 
{\small{}4} & {\small{}5.39E-04} & {\small{}2.07E-02} & {\small{}1.49E-03} & {\small{}250} & {\small{}1.74E-03} & {\small{}1.57E-01} & {\small{}4.94E-02} & {\small{}220}\tabularnewline
\hline 
{\small{}5} & {\small{}3.80E-05} & {\small{}1.11E-03} & {\small{}1.38E-04} & {\small{}444} & {\small{}4.86E-04} & {\small{}3.13E-03} & {\small{}3.64E-03} & {\small{}452}\tabularnewline
\hline 
{\small{}6} & {\small{}2.92E-06} & {\small{}2.87E-04} & {\small{}9.60E-06} & {\small{}888} & {\small{}4.90E-05} & {\small{}5.72E-04} & {\small{}5.59E-04} & {\small{}784}\tabularnewline
\hline 
{\small{}7} & {\small{}1.47E-07} & {\small{}1.83E-05} & {\small{}5.12E-07} & {\small{}1528} & {\small{}6.77E-07} & {\small{}8.26E-05} & {\small{}8.72E-06} & {\small{}1429}\tabularnewline
\hline 
\end{tabular}

\caption{Errors against $W$ for $\nu_{1}=0.1,0.01$}

\end{table}


\subsubsection*{Example 2: With non-homogeneous jump condition on the interface}

Let $\Omega=[0,1]^{2}.$ Consider the Stokes interface problem with
a straight line interface $\Gamma_{0}=\{(x,y):y=0.5\}$ and with Dirichlet
boundary condition on the boundary. The interface divides the domain
$\Omega$ into two subdomains $\Omega_{1}=\{(x,y)\in\Omega:0<y<0.5\}\text{ and }\Omega_{2}=\{(x,y)\in\Omega:0.5<y<1\}$.
The data is chosen such that the problem has the following exact solution:
\begin{eqnarray*}
u_{1}(x,y)=\begin{cases}
\frac{1}{\nu_{1}}(y-0.5)x^{2}\text{ in }\Omega_{1},\\
\frac{1}{\nu_{2}}(y-0.5)x^{2}\text{ in }\Omega_{2},
\end{cases}u_{2}(x,y)=\begin{cases}
-\frac{1}{\nu_{1}}x(y-0.5)^{2}\text{ in }\Omega_{1},\\
-\frac{1}{\nu_{2}}x(y-0.5)^{2}\text{ in }\Omega_{2},
\end{cases}
\end{eqnarray*}
\[
p(x,y)=\begin{cases}
2xy+x^{2}\text{ in }\Omega_{1},\\
2xy+x^{2}-3\text{ in }\Omega_{2}.
\end{cases}
\]
\vspace{0.2cm}
Here, the solution satisfies non-homogeneous jump condition ($[(\nu\nabla\mathbf{u}-pI)\mathbf{n}]$)
on the interface $y=0.5$. The numerical solution is obtained for
$\nu_{1}=1,\nu_{2}=0.1.$

\begin{wraptable}{o}{0.5\columnwidth}%
{\small{}}%
\begin{tabular}{|c|c|c|c|c|}
\hline 
\multicolumn{5}{|c|}{{\small{}$\nu_{1}=1,\nu_{2}=0.1$}}\tabularnewline
\hline 
{\small{}$\textbf{W}$} & {\small{}$\rVert\textit{E}_{\textbf{u}}\rVert_{1}$} & {\small{}$\rVert\textit{E}_{\textit{p}}\rVert_{0}$} & {\small{}$\rVert\textit{E}_{\textit{c}}\rVert_{0}$} & \textbf{\small{}iters}\tabularnewline
\hline 
{\small{}2} & {\small{}3.46E-02} & {\small{}5.42E-02} & {\small{}1.01E-01} & {\small{}32}\tabularnewline
\hline 
{\small{}4} & {\small{}1.62E-04} & {\small{}6.76E-04} & {\small{}4.77E-04} & {\small{}250}\tabularnewline
\hline 
{\small{}6} & {\small{}1.48E-06} & {\small{}4.91E-06} & {\small{}4.64E-06} & {\small{}847}\tabularnewline
\hline 
{\small{}8} & {\small{}1.06E-08} & {\small{}7.22E-08} & {\small{}3.90E-08} & {\small{}2318}\tabularnewline
\hline 
{\small{}10} & {\small{}7.99E-11} & {\small{}7.53E-10} & {\small{}3.21E-10} & {\small{}4642}\tabularnewline
\hline 
\end{tabular}{\small\par}

\caption{Errors for {\small{}$\nu_{1}=1,\nu_{2}=0.1$} }
\end{wraptable}%
The domain is divided into four elements such that the discretization
matches along the interface $y=0.5$. The numerical solution is obtained
for different values of $W$. The relative errors $\|E_{\mathbf{u}}\|_{1},\|E_{p}\|_{0},$
and the error in the continuity equation $\|E_{c}\|_{0}$ and the
total number of iterations required to reach the desired accuracy
are tabulated in Table $3$ for different values of $W$. Similar
results are expected for $\nu_{1}=1,\nu_{2}=0.01$ with an increase
in the number of iterations. One can see the approximate solution
$\mathbf{z}$ is obtained to an accuracy of $O(10^{-6})$ with a smaller
number of iterations (with $W=6$). The numerical scheme is able to
approximate $\mathbf{u}$ to an accuracy of $O(10^{-11}),$ $p$ to
an accuracy of $O(10^{-10})$ with an increase in the iteration count.
The decay of the error in the continuity equation $\|E_{c}\|_{0}$
shows that the numerical scheme has good mass conservation property. 

\begin{wrapfigure}{o}{0.5\columnwidth}%
$\!\!\!\!\!\!$\includegraphics[scale=0.25]{P2}

\caption{Log of errors $\|E_{\mathbf{u}}\|_{1},\|E_{p}\|_{0}$ vs. $W$ }
\end{wrapfigure}%
The graph between the log of relative errors $\|E_{\mathbf{u}}\|_{1},\|E_{p}\|_{0}$
and the degree of polynomial $W$ is shown for $\nu_{1}=1,\nu_{2}=0.1$
in figure 3. The graph is almost linear, which shows the exponential
accuracy of the method.

\subsubsection*{Example 3: Stokes interface problem on an annular domain}

Consider the stokes interface problem on the annular domain $\Omega=\{(r,\theta):1\le r\le2,0\le\theta\le\frac{\pi}{2}\}$
having curved interface $\Gamma_{0}=\{(r_{0},\theta):r_{0}=1.5,0\le\theta\le\frac{\pi}{2}\}$
that separates $\Omega$ into two subdomains $\Omega_{1}=\{(x,y):1<x^{2}+y^{2}<2.25\}\text{ and }\Omega_{2}=\{(x,y):2.25<x^{2}+y^{2}<4\}$.
We consider Dirichlet boundary condition on the boundary.

\begin{wraptable}{o}{0.5\columnwidth}%
{\small{}}%
\begin{tabular}{|c|c|c|c|c|}
\hline 
\multicolumn{5}{|c|}{{\small{}$\nu_{1}=1,\nu_{2}=0.1$}}\tabularnewline
\hline 
{\small{}$\textbf{W}$} & {\small{}$\rVert\textit{E}_{\textbf{u}}\rVert_{1}$} & {\small{}$\rVert\textit{E}_{\textit{p}}\rVert_{0}$} & {\small{}$\rVert\textit{E}_{\textit{c}}\rVert_{0}$} & \textbf{\small{}iters}\tabularnewline
\hline 
{\small{}2} & {\small{}9.37E-01} & {\small{}9.86E-01} & {\small{}2.35E+00} & {\small{}1}\tabularnewline
\hline 
{\small{}4} & {\small{}2.85E-02} & {\small{}3.16E-01} & {\small{}4.20E-01} & {\small{}120}\tabularnewline
\hline 
{\small{}6} & {\small{}2.72E-04} & {\small{}2.88E-03} & {\small{}1.07E-02} & {\small{}725}\tabularnewline
\hline 
{\small{}8} & {\small{}2.30E-06} & {\small{}4.07E-05} & {\small{}9.54E-05} & {\small{}2644}\tabularnewline
\hline 
\end{tabular}\caption{Errors for $\nu_{1}=1,\nu_{2}=0.1$}
\end{wraptable}%
The data is chosen such that the problem has the following exact solution:
\begin{eqnarray*}
u_{1}(x,y)=\begin{cases}
\frac{-1}{\nu_{1}}\,\ y\,\ \sin(2.25-x^{2}-y^{2})\text{ in }\Omega_{1},\\
\frac{-1}{\nu_{2}}\,\ y\,\ \sin(2.25-x^{2}-y^{2})\text{ in }\Omega_{2},
\end{cases}\\
u_{2}(x,y)=\begin{cases}
\frac{1}{\nu_{1}}\,\ x\,\sin(2.25-x^{2}-y^{2})\text{ in }\Omega_{1},\\
\frac{1}{\nu_{2}}\,\ x\,\sin(2.25-x^{2}-y^{2})\text{ in }\Omega_{2},
\end{cases}
\end{eqnarray*}
\[
p(x,y)=e^{x+y}-e^{2}.
\]
Here, the solution satisfies the homogeneous jump condition on the
interface $\Gamma_{0}$. The annular domain is divided into four elements
such that the discretization matches along the interface $\Gamma_{0}$
(see figure 4.a). The interface is completely resolved using blending
elements \cite{GO}. The numerical solution is obtained for different
values of $W$ for $\nu_{1}=1,\nu_{2}=0.1$. The relative error in
$\mathbf{u}$ in $\mathbf{H}^{1}$ norm, relative error in pressure
in $L^{2}$ norm, error in the continuity equation in $L^{2}$ norm
and the total number of iterations are tabulated in Table $4$ for
different values of $W$. The graph between the log of the relative
error and the degree of polynomial $W$ is shown for $\nu_{1}=1,\nu_{2}=0.1$
in Figure 4.b. The graph is almost linear, which shows the exponential
accuracy of the method. 
\begin{figure}[H]
\subfloat[Discretization of the annular domain]{\includegraphics[width=8cm,height=5cm]{annulardicre}

}~~~~~~\subfloat[Log of the relative errors for $\nu_{1}=1,\nu_{2}=0.1$]{\includegraphics[width=8cm,height=5cm]{Annular_new}

}

\caption{Domain with curved interface and error in the numerical approximation }
\end{figure}


\subsubsection*{Example 4: Stokes interface problem with circular interface}

Consider the stokes interface problem on the domain $\Omega=[-1,1]^{2}$
having interface $\Gamma_{0}=\{(x,y):x^{2}+y^{2}=0.25\}$ that separates
$\Omega$ into two subdomains $\Omega_{1}=\{(x,y):x^{2}+y^{2}<0.25\}\text{ and }\Omega_{2}=\Omega\setminus\overline{\Omega}_{1}$
with Dirichlet boundary condition on the boundary of the domain $\Omega$.
The data is chosen such that the problem has the following exact solution:
\begin{eqnarray*}
u_{1}(x,y)=\begin{cases}
\frac{1}{\nu_{1}}\,\ y(x^{2}+y^{2}-0.25)\text{ in }\Omega_{1},\\
\frac{1}{\nu_{2}}\,\ y(x^{2}+y^{2}-0.25)\text{ in }\Omega_{2},
\end{cases}u_{2}(x,y)=\begin{cases}
-\frac{1}{\nu_{1}}\,\ x(x^{2}+y^{2}-0.25)\text{ in }\Omega_{1},\\
-\frac{1}{\nu_{2}}\,\ x(x^{2}+y^{2}-0.25)\text{ in }\Omega_{2},
\end{cases}
\end{eqnarray*}
\[
p(x,y)=x^{2}-y^{2}.
\]
The solution satisfies the homogeneous jump condition on the interface
$\Gamma_{0}$. The domain is divided into nine elements such that
the discretization matches along the interface $\Gamma_{0}$(very
similar to the discretization shown in Fig. 1.b). The numerical solution
is obtained for different values of $W$. The relative error in $\textbf{\textit{u}}$
in $\mathbf{H}^{1}$ norm, the relative error in pressure in $L^{2}$
norm, and the error in the continuity equation in $L^{2}$ norm are
tabulated in Table $5$ for $\nu_{2}=1,\nu_{1}=0.1\text{ and }\nu_{2}=1,\nu_{1}=0.01$
for different values of $W$. The graph between the log of relative
errors $\|E_{\mathbf{u}}\|_{1},\|E_{p}\|_{0},$ and the degree of
polynomial $W$ is shown for $\nu_{2}=1,\nu_{1}=0.1$ and $\nu_{2}=1,\nu_{1}=0.01$
in Figures 5.a and 5.b respectively. The graphs show the exponential
accuracy of the method. 

\begin{table}[H]
~~~~~~{\small{}}%
\begin{tabular}{|c|c|c|c|c|c|c|c|c|}
\hline 
\multicolumn{1}{|c}{} & \multicolumn{4}{c|}{{\small{}${\nu_{1}}=0.1$, $\nu_{2}=1$}} & \multicolumn{4}{c|}{{\small{}${\nu_{1}}=0.01$, $\nu_{2}=1.0$}}\tabularnewline
\hline 
{\small{}$\textbf{W}$} & {\small{}$\rVert\textit{E}_{\textbf{u}}\rVert_{1}$} & {\small{}$\rVert\textit{E}_{\textit{p}}\rVert_{0}$} & {\small{}$\rVert\textit{E}_{\textit{c}}\rVert_{0}$} & \textbf{\small{}iters} & {\small{}$\rVert\textit{E}_{\textbf{u}}\rVert_{1}$} & {\small{}$\rVert\textit{E}_{\textit{p}}\rVert_{0}$} & {\small{}$\rVert\textit{E}_{\textit{c}}\rVert_{0}$} & \textbf{\small{}iters}\tabularnewline
\hline 
{\small{}2} & {\small{}8.95E-01} & {\small{}9.87E-01} & {\small{}3.69E-01} & {\small{}1} & {\small{}9.92E-01} & {\small{}9.86E-01} & {\small{}3.95E-01} & {\small{}1}\tabularnewline
\hline 
{\small{}3} & {\small{}1.58E-01} & {\small{}4.27E-01} & {\small{}3.39E-01} & {\small{}86} & {\small{}6.65E-01} & {\small{}1.59E-01} & {\small{}2.65E-01} & {\small{}313}\tabularnewline
\hline 
{\small{}4} & {\small{}2.21E-02} & {\small{}9.72E-02} & {\small{}1.06E-01} & {\small{}352} & {\small{}5.62E-02} & {\small{}6.87E-02} & {\small{}1.38E-01} & {\small{}1348}\tabularnewline
\hline 
{\small{}5} & {\small{}2.78E-03} & {\small{}1.62E-02} & {\small{}1.47E-02} & {\small{}727} & {\small{}3.31E-03} & {\small{}8.65E-03} & {\small{}1.77E-02} & {\small{}2360}\tabularnewline
\hline 
{\small{}6} & {\small{}4.13E-04} & {\small{}2.22E-03} & {\small{}2.19E-03} & {\small{}1621} & {\small{}3.89E-04} & {\small{}1.38E-03} & {\small{}2.40E-03} & {\small{}3843}\tabularnewline
\hline 
{\small{}7} & {\small{}5.58E-05} & {\small{}3.20E-04} & {\small{}3.00E-04} & {\small{}2861} & {\small{}5.89E-05} & {\small{}1.92E-04} & {\small{}2.96E-04} & {\small{}5629}\tabularnewline
\hline 
\end{tabular}{\small\par}

\caption{Relative errors {\small{}$\rVert\textit{E}_{\textbf{u}}\rVert_{1}$,$\rVert\textit{E}_{\textit{p}}\rVert_{0}$
and $\rVert\textit{E}_{\textit{c}}\rVert_{0}$} for $\upsilon_{1}=0.1,0.01$}
\end{table}

\begin{figure}[H]
\begin{centering}
\subfloat[Decay of the relative errors in $\mathbf{u}$ and $p$ for $\nu_{1}=0.1$]{\begin{centering}
\includegraphics[width=8.5cm,height=5cm]{Circular_0\lyxdot 1}
\par\end{centering}

}\subfloat[Decay of the relative error in $\mathbf{u}$ and $p$ for $\nu_{1}=0.01$]{\begin{centering}
\includegraphics[width=8cm,height=5cm]{Circular_0\lyxdot 01_5629}
\par\end{centering}

}
\par\end{centering}
\caption{Log of relative errors against $W$}
\end{figure}


\subsubsection*{Example 5: Interface problem with mixed boundary conditions}

Here, we consider the Stokes interface problem on the domain $\Omega=[-1,1]^{2}$
having a circular interface $\Gamma_{0}=\{(x,y):x^{2}+y^{2}=0.25\}$
that separates $\Omega$ into two subdomains $\Omega_{1}=\{(x,y):x^{2}+y^{2}<0.25\}\text{ and }\Omega_{2}=\Omega\setminus\overline{\Omega}_{1}.$
We consider the Neumann type boundary condition $\frac{\partial\mathbf{u}}{\partial\mathbf{n}}-p\mathbf{n}$
on the side $y=-1$ and the Dirichlet boundary condition on all the
other sides of the boundary of the domain. The data is chosen such
that the exact solution is the same as the one considered in Example
$4.$ 

We divided the domain into nine elements such that the discretization
matches along the interface $\Gamma_{0}$. Here we have considered
piecewise constant viscosity coefficients $\nu_{1}=0.1,\nu_{2}=1,\text{ and }\nu_{1}=0.01,\nu_{2}=1$
and obtained the numerical solution for different values of $W$.
Table 6 presents the relative errors in $\mathbf{u},p$ and the error
in the continuity equation for $\nu_{1}=0.1,\nu_{2}=1,\text{ and }\nu_{1}=0.01,\nu_{2}=1.$
The results confirm the exponential convergence in velocity and pressure
variables and the mass conservation property of the numerical scheme. 

\begin{table}[H]
~~~~~~~~{\small{}}%
\begin{tabular}{|c|c|c|c|c|c|c|c|c|}
\hline 
\multicolumn{1}{|c}{} & \multicolumn{4}{c|}{{\small{}${\nu_{1}}=0.1$, $\nu_{2}=1$}} & \multicolumn{4}{c|}{{\small{}${\nu_{1}}=0.01$, $\nu_{2}=1$}}\tabularnewline
\hline 
{\small{}$\textbf{W}$} & {\small{}$\rVert\textit{E}_{\textbf{u}}\rVert_{1}$} & {\small{}$\rVert\textit{E}_{\textit{p}}\rVert_{0}$} & {\small{}$\rVert\textit{E}_{\textit{c}}\rVert_{0}$} & \textbf{\small{}iters} & {\small{}$\rVert\textit{E}_{\textbf{u}}\rVert_{1}$} & {\small{}$\rVert\textit{E}_{\textit{p}}\rVert_{0}$} & {\small{}$\rVert\textit{E}_{\textit{c}}\rVert_{0}$} & \textbf{\small{}iters}\tabularnewline
\hline 
{\small{}2} & {\small{}8.85E-01} & {\small{}9.82E-01} & {\small{}4.87E-01} & {\small{}1} & {\small{}9.93E-01} & {\small{}9.81E-01} & {\small{}5.10E-01} & {\small{}1}\tabularnewline
\hline 
{\small{}3} & {\small{}8.20E-02} & {\small{}2.34E-01} & {\small{}2.78E-01} & {\small{}168} & {\small{}6.81E-01} & {\small{}1.65E-01} & {\small{}2.74E-01} & {\small{}299}\tabularnewline
\hline 
{\small{}4} & {\small{}2.76E-02} & {\small{}1.82E-01} & {\small{}1.10E-01} & {\small{}381} & {\small{}5.64E-02} & {\small{}8.50E-02} & {\small{}1.41E-01} & {\small{}1436}\tabularnewline
\hline 
{\small{}5} & {\small{}4.25E-03} & {\small{}5.23E-02} & {\small{}9.55E-03} & {\small{}1396} & {\small{}2.81E-03} & {\small{}8.25E-03} & {\small{}1.90E-02} & {\small{}2762}\tabularnewline
\hline 
{\small{}6} & {\small{}5.05E-04} & {\small{}5.15E-03} & {\small{}1.85E-03} & {\small{}2378} & {\small{}2.98E-04} & {\small{}1.43E-03} & {\small{}1.84E-03} & {\small{}4650}\tabularnewline
\hline 
{\small{}7} & {\small{}3.58E-05} & {\small{}3.72E-04} & {\small{}1.55E-04} & {\small{}5078} & {\small{}3.40E-05} & {\small{}1.50E-04} & {\small{}3.46E-04} & {\small{}6784}\tabularnewline
\hline 
\end{tabular}{\small\par}

\caption{{\small{}$\rVert\textit{E}_{\textbf{u}}\rVert_{1}$,$\rVert\textit{E}_{\textit{p}}\rVert_{0}$
and $\rVert\textit{E}_{\textit{c}}\rVert_{0}$ for different $W$ }}
\end{table}

\textbf{$\!\!\!\!\!\!\!\!\!\!$}\textbf{\textit{Remark:}}\textit{
In all the above examples, we have considered small viscosity coefficients
in one of the subdomains $\Omega_{1}$ or $\Omega_{2}$ (either $\nu_{1}=0.1/0.01,\nu_{2}=1$
or $\nu_{1}=1,\nu_{2}=0.1/0.01/0.001)$. In all the cases, we have
used the quadratic form given in Eq. (12) as a preconditioner. This
preconditioner is not giving satisfactory results in terms of accuracy
and efficiency when the viscosity coefficients are large (for example,
either $\nu_{1}=100/1000$ or $\nu_{2}=100/1000$). So we have modified
the quadratic form (12) as in these cases and obtained the numerical
results. Here we present the numerical results for the Stokes interface
problem with large piecewise viscocity coefficients with the modified
precondiners. }

\subsubsection*{Numerical results with different preconditioners}

First, we consider the Stokes interface problem stated in Example
2, but with different viscosity coefficients: $\nu_{1}=10,\nu_{2}=1.0$
and $\nu_{1}=100,\nu_{2}=1.0.$ In this case, we have used the following
quadratic form as a preconditioner and obtained the numerical solution
for different values of $W.$
\begin{eqnarray*}
\sum_{k=1}^{m_{1}}\text{\ensuremath{\nu_{1}^{2}}}||\tilde{\mathbf{u}}_{1}^{k}(\xi,\eta)||_{2,S}^{2}+\sum_{k=1}^{m_{1}}||\tilde{p}_{1}^{k}(\xi,\eta)||_{1,S}^{2}+\sum_{l=1}^{m_{2}}\text{\ensuremath{\nu_{2}^{2}}}||\tilde{\mathbf{u}}_{2}^{l}(\xi,\eta)||_{2,S}^{2}+\sum_{l=1}^{m_{2}}||\tilde{p}_{2}^{l}(\xi,\eta)||_{1,S}^{2}.
\end{eqnarray*}

We have used the same discretization as in Example 2 and obtained
the numerical solution using PCGM. Table 7 presents the relative errors
in $\mathbf{u},p,$ and error in the continuity equation for different
values of $W.$ The results show the accuracy of the method, and the
error in the continuity equation shows the mass conservation property
of the numerical method.

\begin{table}[H]
~~~~~~{\small{}}%
\begin{tabular}{|c|c|c|c|c|c|c|c|c|}
\hline 
\multicolumn{1}{|c}{} & \multicolumn{4}{c|}{{\small{}${\nu_{1}}=10$, $\nu_{2}=1$}} & \multicolumn{4}{c|}{{\small{}${\nu_{1}}=100$, $\nu_{2}=1$}}\tabularnewline
\hline 
{\small{}$\textbf{W}$} & {\small{}$\rVert\textit{E}_{\textbf{u}}\rVert_{1}$} & {\small{}$\rVert\textit{E}_{\textit{p}}\rVert_{0}$} & {\small{}$\rVert\textit{E}_{\textit{c}}\rVert_{0}$} & \textbf{\small{}iters} & {\small{}$\rVert\textit{E}_{\textbf{u}}\rVert_{1}$} & {\small{}$\rVert\textit{E}_{\textit{p}}\rVert_{0}$} & {\small{}$\rVert\textit{E}_{\textit{c}}\rVert_{0}$} & \textbf{\small{}iters}\tabularnewline
\hline 
{\small{}2} & {\small{}5.72E-01} & {\small{}6.83E-01} & {\small{}1.37E-01} & {\small{}2} & {\small{}5.19E-01} & {\small{}6.83E-01} & {\small{}1.08E-01} & {\small{}2}\tabularnewline
\hline 
{\small{}3} & {\small{}2.42E-01} & {\small{}3.82E-01} & {\small{}5.56E-02} & 24 & {\small{}1.79E-01} & {\small{}2.80E-01} & {\small{}3.75E-02} & 43\tabularnewline
\hline 
{\small{}4} & {\small{}1.65E-02} & {\small{}7.71E-02} & {\small{}4.60E-03} & {\small{}261} & {\small{}8.17E-03} & {\small{}8.79E-02} & {\small{}3.08E-03} & {\small{}270}\tabularnewline
\hline 
{\small{}5} & {\small{}2.80E-03} & {\small{}1.12E-02} & {\small{}8.86E-04} & {\small{}689} & {\small{}1.85E-03} & {\small{}5.04E-02} & {\small{}5.70E-04} & {\small{}859}\tabularnewline
\hline 
{\small{}6} & {\small{}5.75E-04} & {\small{}1.78E-03} & {\small{}2.52E-04} & {\small{}1548} & {\small{}6.90E-04} & {\small{}2.37E-02} & {\small{}2.02E-04} & {\small{}2174}\tabularnewline
\hline 
\end{tabular}{\small\par}

\caption{Relative errors {\small{}$\rVert\textit{E}_{\textbf{u}}\rVert_{1}$,$\rVert\textit{E}_{\textit{p}}\rVert_{0}$
and $\rVert\textit{E}_{\textit{c}}\rVert_{0}$} for $\upsilon_{1}=10,100$}
\end{table}

Next, we consider the interface problem stated in Example 4, but with
different viscosity coefficients: $\nu_{1}=100,\nu_{2}=1.0$ and $\nu_{1}=1000,\nu_{2}=1.0.$
In this case, we have used the following quadratic form as a preconditioner
and obtained the numerical solution for different values of $W.$
\begin{eqnarray*}
\sum_{k=1}^{m_{1}}\text{\ensuremath{\nu_{1}^{3}}}||\tilde{\mathbf{u}}_{1}^{k}(\xi,\eta)||_{2,S}^{2}+\sum_{k=1}^{m_{1}}||\tilde{p}_{1}^{k}(\xi,\eta)||_{1,S}^{2}+\sum_{l=1}^{m_{2}}\text{\ensuremath{\nu_{2}^{3}}}||\tilde{\mathbf{u}}_{2}^{l}(\xi,\eta)||_{2,S}^{2}+\sum_{l=1}^{m_{2}}||\tilde{p}_{2}^{l}(\xi,\eta)||_{1,S}^{2}.
\end{eqnarray*}

\begin{table}[H]
~~~~~~{\small{}}%
\begin{tabular}{|c|c|c|c|c|c|c|c|c|}
\hline 
\multicolumn{1}{|c}{} & \multicolumn{4}{c|}{{\small{}${\nu_{1}}=100$, $\nu_{2}=1$}} & \multicolumn{4}{c|}{{\small{}${\nu_{1}}=1000$, $\nu_{2}=1$}}\tabularnewline
\hline 
{\small{}$\textbf{W}$} & {\small{}$\rVert\textit{E}_{\textbf{u}}\rVert_{1}$} & {\small{}$\rVert\textit{E}_{\textit{p}}\rVert_{0}$} & {\small{}$\rVert\textit{E}_{\textit{c}}\rVert_{0}$} & \textbf{\small{}iters} & {\small{}$\rVert\textit{E}_{\textbf{u}}\rVert_{1}$} & {\small{}$\rVert\textit{E}_{\textit{p}}\rVert_{0}$} & {\small{}$\rVert\textit{E}_{\textit{c}}\rVert_{0}$} & \textbf{\small{}iters}\tabularnewline
\hline 
{\small{}2} & {\small{}8.71E-01} & {\small{}9.86E-01} & {\small{}3.96E-01} & {\small{}1} & {\small{}8.71E-01} & {\small{}9.86E-01} & {\small{}3.95E-01} & {\small{}1}\tabularnewline
\hline 
{\small{}3} & {\small{}4.92E-02} & {\small{}3.42E-01} & {\small{}1.40E-01} & 30 & {\small{}5.05E-02} & {\small{}2.66E-01} & {\small{}1.18E-01} & 28\tabularnewline
\hline 
{\small{}4} & {\small{}1.39E-02} & {\small{}1.23E-01} & {\small{}4.94E-02} & 170 & {\small{}2.56E-02} & {\small{}7.62E-02} & {\small{}4.75E-02} & {\small{}86}\tabularnewline
\hline 
{\small{}5} & {\small{}4.90E-03} & {\small{}5.66E-02} & {\small{}1.13E-02} & 343 & {\small{}3.98E-03} & {\small{}3.53E-02} & {\small{}1.37E-02} & {\small{}274}\tabularnewline
\hline 
{\small{}6} & {\small{}6.62E-04} & {\small{}1.87E-02} & {\small{}1.48E-03} & {\small{}1086} & {\small{}6.26E-04} & {\small{}1.55E-02} & {\small{}1.61E-03} & {\small{}1586}\tabularnewline
\hline 
\end{tabular}{\small\par}

\caption{Errors for $\nu_{1}=100,1000$}
\end{table}

By discretizing the domain with a circular interface into nine elements,
as in Example 4, we have obtained the relative errors {\small{}$\rVert\textit{E}_{\textbf{u}}\rVert_{1}$,$\rVert\textit{E}_{\textit{p}}\rVert_{0}$
and $\rVert\textit{E}_{\textit{c}}\rVert_{0}$} for different values
of $W.$ Table 8 presents these results for $\nu_{1}=100,\nu_{2}=1.0$
and $\nu_{1}=1000,\nu_{2}=1.0.$ The results show the exponential
accuracy of the numerical method. 

\section{Conclusions}

An exponentially accurate nonconforming spectral element method for
the Stokes interface problem with a smooth interface is presented.
The numerical results confirm the exponential accuracy of the numerical
method. This method has many advantages. The method is nonconforming,
and the interface is completely resolved using blending elements.
The inf-sup condition is not required to choose the approximation
spaces for velocity and pressure variables, so the same order spectral
element functions are used for both velocity and pressure variables.
The numerical formulation always leads to a symmetric, positive-definite
linear system. The method also has good mass conservation property.
The method also works for various boundary conditions different from
the Dirichlet boundary conditions. 

Numerical results for the Stokes interface problems with different
sets of piecewise constant viscosity coefficients and different types
of smooth interfaces are presented. One can see that the iteration
count is high when the ratio of viscosity coefficients is high. Different
types of preconditioners are used in some special cases. So there
is a need to develop an efficient preconditioner that works uniformly
in all cases. The development of an efficient preconditioner is under
investigation. The extension of the method to the three-dimensional
Stokes interface problem is ongoing work. 

\subsubsection*{Declarations }

\textbf{Conflict of interests:} All authors declare no conflict of
interests. \\
\\
\textbf{Data Availability:} The data related to this manuscript will
be available upon request.
\begin{thebibliography}{10}
\bibitem{AD} S. Adjerid, N. Chaabane and T. Lin, An immersed discontinuous
finite element method for Stokes interface problems, Comp. Meth. Appl.
Mech. Engrg., 293, 170-190, 2015.

\bibitem{arbazsubham}Arbaz Khan, Akhlaq Khan, S. Mohapatra and C.
S. Upadhyay, Spectral element method for three dimensional elliptic
problems with smooth interfaces, Comput. Meth. Appl. Mech. Engg.,
315, 522-549, 2017. 

\bibitem{AS} D. Assencio, Virtual Node Algorithms for Stokes Interface
Problems (Doctoral dissertation, UCLA), 2012 \href{https://escholarship.org/uc/item/0m70g8xz}{https://escholarship.org/uc/item/0m70g8xz}.

\bibitem{babuska}I. Babuska and B. Q. Guo, The $h-p$ version of
the finite element on domains with curved boundaries, SIAM J. Num.
Anal., 25, 837-861, 1988. 

\bibitem{BOTH} P. Bolton and R. W. Thatcher, On mass conservation
in least-squares method, J. of Comp. Phy., 203(1), 287-304, 2005.

\bibitem{burmandel} E. Burman, G. Delay and A. Ern, An unfitted hybrid
high order method for the Stokes interface problems, IMA J. of Num.
Anal., 41(4), 2362--2387, 2021.

\bibitem{CH1} Y. Chen and X. Zhang, A $P_{2}-P_{1}$ partially penalized
immersed finite element method for Stokes interface problems, Int.
J. of Num. Anal. and Mod., 18(1), 120-141, 2021.

\bibitem{CHI} E. V. Chizhonkov, Numerical solution to a Stokes interface
problem, Comp. Math. and Math. Phy., 49, 105-16, 2009.

\bibitem{DoZh} H. Dong, Z. Zhao, S. Li, W. Ying and J. Zhang, Second
order convergence of a modified MAC scheme for Stokes interface problems,
J. of Sci. Comp., 96:27, 2023.

\bibitem{pd1}P. K. Dutt, N. K. Kumar and C. S. Upadhyay, Nonconforming
h-p spectral element methods for elliptic problems, Proc. Math. Sci.,
117, 109--145, 2007.

\bibitem{JoKw}G. Jo and D. Kwak, A New Immersed Finite Element Method
for Two-Phase Stokes Problems Having Discontinuous Pressure, Comp.
Meth. in Appl. Math., 2023, \href{https://doi.org/10.1515/cmam-2022-0122}{https://doi.org/10.1515/cmam-2022-0122}.

\bibitem{GO}W. J. Gordan and C. A. Hall, Transfinite element methods:
Blending-function interpolation over arbitrary curved element domains,
Numer. Math., 21(2), 109-129, 1973.

\bibitem{grisvard}P. Grisward, Elliptic problems in nonsmooth domains,
Pitman Advanced publishing program, 1985. 

\bibitem{HA1} P. Hansbo, MG. Larson and S. Zahedi, A cut finite element
method for Stokes interface problem, Appl. Num. Math., Vol. 85, 90-114,
2014.

\bibitem{He}X. He, Fei Song and W. Deng, A stabilized nonconforming
Nitsche's extended finite element method for Stokes interface problems,
Discrete and Continuous Dynamical Systems - B, 27(5), 2849-2871, 2022.

\bibitem{HE1} P. Hessari, First order system least squares method
for the interface problem of the Stokes equations, Comp. Math. with
Appl., 68, 309-324, 2014.

\bibitem{He2}P. Hessari, Least squares spectral method for the two-dimensional
Stokes interface problems, J. of Comp. and Appl. Math., 364:112325,
2020.

\bibitem{hes3}P. Hessari and B. C. Shin, Spectral Legendre and Chebyshev
approximation the Stokes interface problems, J. KSIAM, Vol. 21(3),
109-124, 2017.

\bibitem{IT1} K. Ito and Z. Li, Interface conditions for Stokes equations
with a discontinuous viscosity and surface forces, Appl. Math. Let.,
19, 229-234, 2006.

\bibitem{jiwangli}H. Ji, F. Wang, J. Chen and Z. Li, An immersed
CR-P0 element for Stokes interface problems and the optimal convergence
analysis, Comp. Meth. Appl. Mech. Engg., 399, 11306, 2022.

\bibitem{JN1}B. N. Jiang, On the least-squares method, Comp. Meth.
in Appl. Mech. and Eng., 152, 239-257, 1998.

\bibitem{johnansonn}A. Johansson, M. G. Larson and A. Logg, High
order cut finite element methods for the Stokes problem, Adv. Mod.
and Sim. in Eng. Sci., 2(1), 1-23, 2015.

\bibitem{JO1} D. Jones and X. Zhang, A class of nonconforming immersed
finite element methods for Stokes interface problems, J. of Comp.
and Appl. Math., Vol. 392(15), 113493, 2021.

\bibitem{kishsubham}N. Kishore Kumar and S. Mohapatra, Performance
of nonconforming spectral element method for Stokes problems. Comp.
and Appl. Math., 41(4),156, 2022.

\bibitem{kishnaga}N. Kishore Kumar and G. Naga Raju, Nonconforming
least-squares method for elliptic partial differential equations with
smooth interfaces, J. of Sci. Comp., 53(2), 295-319, 2012.

\bibitem{kishshi}N. Kishore Kumar and Shivangi Joshi, Nonconforming
spectral element method: a friendly introduction in one dimension
and a short review in higher dimensions, Comp. and Appl. Math., 42:139,
1-40, 2023. 

\bibitem{kishore kumar}N. Kishore Kumar, Nonconforming spectral element
method for elasticity interface problems, J. of Appl. Math. and Info.,Vol.
32, 5-6, 761-781, 2014.

\bibitem{KI1} M. Kirchhart, S. Gross and A. Reusken, Analysis of
an XFEM discretization for Stokes interface problems, SIAM J. Sci.
Comp., 38(2), 1019-1043, 2016.

\bibitem{LE1} P. Ledder, C. M. Pfeiler, C. Winstersteiger and C.
Lehnefeld, Higher order unfitted FEM for Stokes interface problems,
Proc. Appl. Math. Mech., 16, 7-10, 2016.

\bibitem{LI1} Z. Li, K. Ito and M. Lai, An augmented approach for
Stokes equations with discontinuous viscosity and singular forces,
Comp. and Flu., 36, 622-635, 2007.

\bibitem{S1} S. Mohapatra, Pravir K. Dutt, B. V. Rathish Kumar and
Marc I. Gerritsma, Non-conforming least squares spectral element method
for Stokes equations on non-smooth domains, J. of Comp. Appl. Math.,
372, 112696, 2020.

\bibitem{S2} S. Mohapatra and Akhlaq Husain, Least squares spectral
element method for three dimensional Stokes equations,\textbf{\textit{
}}Appl. Numer. Math., 102, 31-54, 2016.

\bibitem{S3} S. Mohapatra and S. Ganesan, A non-conforming least
squares spectral element formulation for Oseen equations with applications
to Navier-Stokes equations,\textbf{\textit{ }}Num.\textbf{\textit{
}}Fun. Anal. Optim., 37, 1295-1311, 2016.

\bibitem{OH1} K. Ohmori and N. Saito, On the convergence of finite
element solutions to the interface problem for the Stokes system,
\textit{J. of Comp. Appl. Math.}, 198, 116-128, 2007.

\bibitem{OL1} M. A. Olshanskii and A. Reusken, Analysis of a Stokes
interface problem, Numer. Math., 103, 129-149, 2006.

\bibitem{PROO2} M. Proot and M. I. Gerritsma, Least-squares spectral
elements applied to the Stokes problem, J. of Comp. Phy., 181, 454-477,
2002.

\bibitem{PR003} M. M. J. Proot and M. I. Gerritsma, Mass and momentum
conservation of the least-squares spectral element method for Stokes
problem, \textit{J. of Sci. Comp.}, Vol. 27, 389-401, 2006.

\bibitem{ramanbhupen}Raman Kumar and Bhupen Deka, Analysis and Computation
of a weak Galerkin scheme for solving the 2D/3D stationary Stokes
interface problems with high-order elements, J. of Numer. Math., DOI
10.1515/jnma-2023-0112, 2024. 

\bibitem{schwab}Ch. Schwab, $p$ and $h-p$ Finite element methods,
Clarendon Press, Oxford, 1998.

\bibitem{SeDu} R. Sevilla and T. Duretz, A face-centered finite volume
method for high-contrast Stokes interface problems, Int. J. for Num.
Meth. in Eng., 124(17), 3709-3732, 2023.

\bibitem{shaosongli}M. Shao, L. Song and P. W. Li, A generalized
finite difference method for solving Stokes interface problems, Eng.
Anal. with Boun. Elem., 132, 50-64, 2021. 

\bibitem{shibata}Y. Shibata and S. Shimizu, On a resolvent estimate
of the interface problem for the Stokes system in a bounded domain,
J. of Diff. Eq., 191, 408-444, 2003. 

\bibitem{SO1} L. Song and H. Gao, Analysis of a Stokes interface
problem in multi-subdomains, Appl. Math. Lett., 64, 131-136, 2017.

\bibitem{Su}Y. Sugitani, Numerical analysis of a Stokes interface
problem based on formulation using the characteristic function. Appl.
of Math., 62(5):459-76, 2017.

\bibitem{tomar}S. K. Tomar, h--p spectral element method for elliptic
problems on non-smooth domains using parallel computers, Ph.D. thesis,
IIT Kanpur, India (2001). Reprint available as Tech. rep. No. 1631,
Department of Applied Mathematics, University of Twente, The Netherlands.
\href{http://doc.utwente.nl/65818/}{http://doc.utwente.nl/65818/}

\bibitem{huwangchen}H. Wang, J. Chen, P. Sun and N. Wang, A conforming
enriched finite element method for Stokes interface problems, Comp.
and Math. with Appl., 75, 4256-4271, 2018. 

\bibitem{nwangchen}N. Wang and J. Chen, A nonconforming Nitsche's
extended finite element method for Stokes interface problems, J. of
Sci. Comp., 81, 342-374, 2019.

\bibitem{WA1} Q. Wang and J. Chen, A new unfitted stabilized Nitsche's
finite element method for Stokes interface problems, Comp. and Math.
with Appl., 70(5), 820-834, 2015.

\bibitem{WA2} B. C. Wang and B. C. Khoo, Hybridizable discontinuous
Galerkin method (HDG) for Stokes interface flow, J. Comp. Phys., 247,
262-278, 2013.

\bibitem{YA} L.Yang, H. Peng, Q. Zhai and R. Zhang, The weak Galerkin
finite element method for Stokes interface problems with curved interface,
arXiv preprint arXiv:2211.11926, 2022.
\end{thebibliography}

\end{document}
