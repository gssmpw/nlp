\section{Methodology}
\label{sec:methodology}
\subsection{Diagram Information Parsing from Xlsx}
\begin{figure}
    % 最初の図
    \includegraphics[width=\textwidth]{figures/arch_fig.pdf}
    \caption{The diagram used in this study, a system blueprint drawn in Excel using rectangles, text boxes, straight connectors, and bent connectors.}
    \label{fig:arch}
\end{figure}
\begin{figure}
    \includegraphics[width=\textwidth]{figures/data_trans.pdf}
    \caption{\small
    On the left is an XML excerpt of shape data from an xlsx file, difficult to interpret due to raw numerical values and aliased strings. It also contains excessive, scattered information across multiple files. On the right is a JSON format that parses, transforms, and summarizes meaningful information on connectors and shapes, optimized for diagram understanding and LLM input.
    }
    \label{fig:data_trans}
\end{figure}
As mentioned earlier, this study proposes a workflow that allows LLMs to understand and analyze diagrams and answer related questions by extracting and converting diagram information stored in source files, without treating documents as visual information. The extracted information is appropriately formatted and converted into text.

First, since the information constituting the document is stored in XML format within the source file, we developed a library to parse this XML and extract diagram-related information. In this study, we targeted system design diagrams saved as .xlsx files created using Microsoft Excel (Fig. \ref{fig:arch}). Documents created with Office software are globally prevalent and enabling LLMs to analyze these documents is highly desirable. However, while data loaders such as LangChain Unstructured and Azure Document Intelligence are available, these loaders are primarily limited to loading textual data and do not parse visual elements into text.

Moreover, in Japan, there is a significant reliance on Excel for creating requirement definition documents and system design diagrams. Therefore, there exists a substantial demand to extract not only table data but also diagrams from Excel. System design diagrams consist of shapes, lines, and text that explain the concepts represented by these shapes and lines. These diagrams are a representative form of diagrams where shapes denote entities and lines represent relationships. For these reasons, this study specifically targeted system design diagrams saved in .xlsx files.



To achieve this, we developed a library that parses source files in .xlsx format and extracts diagram-related shape information (\url{https://github.com/galirage/spreadsheet-intelligence}). The data extracted using this library is shown in Fig. \ref{fig:data_trans}. The source files of Office documents, including .xlsx, are essentially XML files compressed into a zip format. By decompressing and reading the XML, it becomes possible to retrieve the elements that constitute the document. The XML contains information about user-placed objects such as text, shapes, and text boxes, along with editable attributes like drawing position, rotation, theme, color, shape formatting, and text formatting.

An example of a system design diagram created in Excel is shown in Fig. \ref{fig:arch}. This diagram consists of objects such as text boxes, rectangles, straight connectors, and bent connectors. We extracted information related to these four types of objects as the components of the diagrams.


\subsection{Parsing and Transforming Diagram Elements for Enhanced Analysis}
Subsequently, the extracted shape information was transformed into a format that the LLM could easily understand. 
Information extracted from XML cannot deliver sufficient analysis accuracy if used as-is. Therefore, we chose to extract only information relevant to the diagram for rectangles, text boxes, straight connectors, and bent connectors.
For shapes and connectors, coordinate information was first extracted. The coordinate information for shapes and connectors, as provided in the XML, does not reflect rotation, flipping, scaling, or other transformations. To address this, a conversion process was applied to incorporate all these factors, ensuring the values accurately represent the appearance of the rendered diagram.
Additionally, for fill, border, and line color information, cases where the color was specified using software-specific color theme names or where the XML data did not correspond directly to the rendered state were also handled through conversion.
On diagrams, relationships are expressed by connecting shapes representing entities with connectors. While the connection information between connectors and shapes is saved if the diagram creator explicitly configures it during editing, diagrams can also represent relationships with disconnected shapes and connectors. In such cases, it is impossible to determine the connections between shapes and connectors directly from the XML. Therefore, instead of relying on the source file's connection information, we opted to decode the direction of each connector's endpoints into four cardinal directions based on the connector's coordinate data. This enables understanding which direction a connector is pointing and identifies the shapes connected to each connector.
Furthermore, Office software shapes can be grouped to represent collections of shapes. However, since the way these groups are created varies by the diagram creator, group information was not utilized as input data in this process.
The parsed information was then organized, as shown in Fig. XX, into JSON format with attributes grouped for each shape (including both text boxes and rectangles) and each connector (including both straight and bent connectors).
