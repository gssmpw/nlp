\section{Conclusion}
In this paper, we propose the idea of constructing a diagram-understanding method that uses an LLM driven by information extracted from source files, such as XML of Excel, rather than relying on visual recognition. Previous, image-driven approaches utilizing VLMs exhibit insufficient visual capabilities, as demonstrated in the hallucination examples in this study, indicating that there are still challenges for real-world applications such as in business contexts. In response, we presented a XML-driven diagram-understanding solution—one that leverages shape data extracted from the original source files used to create the diagrams—and highlighted its practical potential.

However, since our study is limited to preliminary experiments with a specific system design diagram, we have not thoroughly examined statistical rigor or the generalizability of our evaluation data. To validate the effectiveness of this method on a broader scale, further testing on large and diverse datasets will be necessary. Nevertheless, the library we developed to parse the XML source files of Excel documents used in this study is publicly available as open-source software, ensuring reproducibility and applicability. This enables the research community to replicate our method in diverse scenarios, refine it, and further advance its development.

Although there are constraints due to the lack of fully rigorous statistical procedures and insufficient guarantees of large-scale data generalizability, we anticipate that open-sourcing our library will stimulate further collaborative research and industrial applications. Our hope is that this work will spur the practical application of diagram-understanding technologies using language models, boost the accuracy of requirement definitions and system design analysis in business contexts, and ultimately contribute to improved operational efficiency in Japan’s system development landscape.
