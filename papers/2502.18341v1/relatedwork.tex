\section{Related work}
\label{sec:related-work}


\paragraph{Barriers to participation in group discussions for ESL speakers}  

% English-Second-Language (ESL) speakers often encounter significant communication and interaction barriers primarily due to cross-language ability challenges~\cite{higuchi2023impact}. Unlike native speakers, who often have a more intuitive grasp of language nuances, ESL speakers may struggle with constructing sentences that accurately convey their intended meaning \cite{turnbull2015effects}, convert accurate ideas~\cite{hessel2017new}, struggle to start the talk in a second language~\cite{tan2020influence}, and thus with starting and contributing to group conversations~\cite{sampson2024emergence}. This can lead to misunderstandings in interaction and hinder the conversation flow~\cite{lam2006gauging}, as ESL speakers might need extra time to formulate their utterances facing spoken interactions. It can complicate interactions, particularly in informal or spontaneous conversations.

ESL speakers often encounter significant barriers to effective communication and participation in group discussions, stemming from cross-language ability challenges~\cite{higuchi2023impact}. Unlike native speakers, ESL speakers may struggle to construct sentences that accurately convey their intended meaning and ideas~\cite{turnbull2015effects, hessel2017new}. They often require additional time to formulate responses in spoken interactions~\cite{lam2006gauging}, which can be particularly problematic in informal discussions where quick exchanges are expected. Moreover, ESL speakers are less likely to initiate conversations~\cite{tan2020influence}, further limiting their participation. Above barriers disrupt the flow of conversations, increase the risk of misunderstandings, and significantly reduce ESL speakers' engagement in group discussions~\cite{sampson2024emergence,gao2024listenership}.

%Additional factors, such as cultural differences, low self-confidence, social anxiety, and limited motivation to communicate~\citep{jones1999silence, de2009learners, norton2014non, , mak2011exploration}, exacerbate this reluctance. 

Effectively participating in group interactions is essential for ESL speakers~\cite{rao2019enhancing}. Interactive conversations provide ESL speakers with exposure to the practical use of language, allowing them to learn how to naturally communicate~\cite{gao2024listenership,rao2019enhancing}. This learning cannot be fully replaced by passive methods such as reading or listening~\cite{wu2025data,maarof2018effect}. Consequently, ESL speakers face a dilemma: improving their language skills requires active participation, yet their limited proficiency and confidence often undermines their ability to engage effectively. To address this, educational strategies such as structured group discussions and explicit guidance are commonly employed to facilitate meaningful participation~\cite{pino2010english,webster2012teaching}.

% Effectively practising those interaction skills is crucial for ESL speakers to overcome these barriers~\cite{rao2019enhancing}. Interaction exposes ESL speakers to the practical use of language in various contexts, enabling them to learn and internalize how native speakers naturally communicate \cite{gao2024listenership,rao2019enhancing}. Without active engagement, ESL speakers may find it difficult to reach the level of fluency necessary for effective communication~\cite{maarof2018effect}
% , as passive learning methods like reading or listening alone do not provide the same opportunities to apply language skills in real-time situations~\cite{dai2023second,wu2025data}. During second language learning, these strategies need further instructed from language educators, especially under group discussions to be equipped with explicit guidance~\cite{pino2010english,webster2012teaching}.  Therefore, effective moderation among ESL speakers' interaction can prominently improve their ability in second language communication, language learning, and language interactional competence.  

\paragraph{Dialogue quality evaluation}

% Evaluating the quality of conversational dialogue and justifying the evaluation in an explainable way has been a long-standing barrier in dialogue evaluation studies as conversational systems grow increasingly complex and multi-dimensional~\cite{tam2024framework}. However, most dialogue evaluation works focused on syntactical features, like fluency~\cite{ou-etal-2024-dialogbench}, grammar~\cite{lin2023llm}, and delivery accuracy  ~\cite{han2022measuring,chen2023automatic}, which ignores the the nature of dialogue functions--interactions or engagement.  In addition, these methods still lack interpretability and explanations of the evaluation results, rendering them unsuitable for deriving effective insights for conversation refinement or guidance for domain-specific conversations~\cite{de2023evaluation}, which leaves a gap for effectively evaluating the impact of deploying LLMs in domains like language education~\cite{long2024evaluating}. 

% Evaluating conversational dialogue quality while ensuring transparency and explainability remains a significant challenge.
Most existing dialogue evaluation methods focus on assessing the quality of machine-generated dialogues, emphasizing features like fluency~\cite{ou-etal-2024-dialogbench}, grammar~\citep{lin2023llm}, and  accuracy~\citep{han2022measuring,chen2023automatic}, while overlooking nuanced aspects such as interactions. These approaches primarily assess machine responses in isolation and often lack interpretability~\citep{smith2022human}. In contrast, traditional evaluations of human-to-human conversations rely heavily on manual coding and human interpretation~\citep{o2018use,mckenzie2000hope}, which, although detailed, are limited in scalability.

Recent studies have begun utilizing large language models (LLMs) to evaluate human dialogues, such as classroom interactions~\citep{long2024evaluating}. In the context of ESL conversation, \citet{gao2024cnima} proposed a fine-grained automatic evaluation tool that assesses dialogue quality on both micro (e.g., reference word usage) and macro levels (e.g., tone of utterance) for ESL speakers. Building on this foundation, our study adapts their framework to evaluate the quality of ESL dialogues. %under moderator intervention to identify effective moderation strategies.

% Recent works shift the focus from dialogue content accuracy to dialogue conversation natures that emphasis more on dialogue acts, or the nuances of conversational interaction. For example, Takehi et. al~\citet{takehi2023open} evaluates the dialogue quality based on turn-level dialogue acts, which makes the explanation rooted in dialogue turn-level. In addition, Gao et. al ~\citet{gao2024cnima} proposed an automatic evaluation tool to assess the dialogue quality on two levels, spanning from fundamental linguistic features (e.g., reference words, modal verbs, noun \& verb collocations and etc), to dialogue level organization, such as topic management, tone choice appropriateness, conversation opening and conversation closing with . These works serve as rare attempts in evaluating the dialogue quality based on the nature of conversations. Thus, the current study adopts the automatic dialogue quality evaluation tool from Gao et. al~\cite{gao2024cnima} to effectively measure the ESL dialogue quality under moderator actions, and to map the actual evaluation to explainable interpretations is crucial for providing insights in language education. 



\paragraph{Moderation in ESL group discussions}  

% A conversational moderator plays a pivotal role in discussions by (1) minimizing undesirable behaviors, (2) facilitating productive outcomes, and (3) ensuring balanced participation through various conversational interventions ~\citep{wright2009role, grimmelmann2015virtues}.

Previous studies have documented that moderators employ strategies such as linguistic scaffolding~\citep{kayi2013scaffolding, gagne2013cooperative}, providing instructions~\citep{hamzah2016effect}, seeking clarification, and offering acknowledgments~\citep{braham2009acknowledgement} to mitigate linguistic disparities~\citep{jones1999silence}, bridge cultural gaps~\citep{osman2007interaction}, and address knowledge deficiencies~\citep{asterhan2010online, vasodavan2020moderation}. However, these findings rely on manual evaluation methods such as interviews, case studies, and surveys~\citep{osman2007interaction, hew2008attracting, hamzah2016effect, kayi2013scaffolding}, limiting scalability and cross-context applicability.

Large-scale analyses of dialogue transcripts typically use dialogue acts to categorize speakers' intentions~\citep{d1985speech}. While existing moderation dialogue act schema (e.g., \citet{park2012facilitative}) provide a structured approach, they may not fully address the specific needs of ESL moderation. At the same time, defining entirely new dialogue acts in isolation could hinder cross-domain comparability. To address this challenge, we develop a tailored set of dialogue acts by adapting the WHoW moderation analysis framework. This approach ensures that the dialogue acts capture the nuances of ESL moderation while remaining compatible for broader cross-domain comparisons of moderator behavior.


In summary, there exists three major gaps: the absence of an automated method for measuring dialogue quality among ESL speakers in group discussions, the need for dialogue act schema specifically tailored for ESL group discussions moderation analysis, and the lack of quantification of the impact of conversation moderation in ESL settings.