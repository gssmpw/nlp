\begin{algorithm}[t]
\small
    \caption{
    %\Fdecdeg{} function used in \Fonlinepeelwithsampling{}
    Functions used in our algorithm with sampling
    } 
    \label{algo:sample_func}
    
    \DontPrintSemicolon

%The \Fonlinepeelwithsampling{} function is the same as the online peeling function in \cref{algo:onlinepeel}, 
%with replacing the process to atomically decrement the degree of $u$ (line xx) with the following function $\Fdecdeg(u)$.\\
%$u$: the vertex that needs degree decrement.\\
%$B$: the set to buffer vertices in the next frontier. \\
%
%\tcc{Conceptually, the function decrements $u$'s \induceddegree{} $\degreestar[u]$ by 1, and returns the original \induceddegree{} of $u$.
%If $u$ is in the sample mode, we will process the \induceddegree{} approximately, and return -1}.
%\myfunc{$\Fdecdeg(u)$}{
%  \If{$\sampler[u].\mode$} {
%    $\FSampleVertex(u)$\\
%    \Return{-1}\tcp*[f]{Use -1 to indicate that the vertex is in sample mode}
%  } \Else {
%    \Return{$\atomdec(\degreestar[u])$}
%  }
%}

%Global variables from \cref{algo:sampling}: \\
%\quad$\countingbag$: vertices that requires to recount their \induceddegree{s} \\
%\quad$\frontier$: the frontier\\

\let\oldnl\nl% Store \nl in \oldnl
\newcommand{\nonl}{\renewcommand{\nl}{\let\nl\oldnl}}% Remove line number for one line

\SetKwFor{parFor}{parallel\_for}{do}{endfor}
\SetKwFor{parForEach}{parallel\_foreach}{do}{endfor}
\SetKwInOut{Parameters}{Parameters}
\Parameters{
$\reducerate$: when $\degreestar[v]$ decrement to a factor of $\reducerate$, we resample $v$\\
$\mu=4c\ln n$: expected number of hits each sampler, $c>2$
  }
%\nonl For each $v\in V$, maintain a \emph{sampler} structure $\sampler[v]$, defined as follows.\\
  \SetKwProg{mystruc}{struct}{}{}
  \nonl \mystruc(\tcp*[f]{For each $v\in V$, maintain a \emph{sampler} structure $\sampler[v]$}){sampler} {
   \nonl $\vname{\mode{}}$\,: boolean flag indicating whether $v$ is in sample mode\\
   \nonl $\vname{\rate{}}$\,: the sample rate for $v$\\
   \nonl $\vname{\cnt{}}$\,: the number of hits in the sampling process\\
  } 
\smallskip
  \myfunc(\tcp*[f]{peeling process with sampling}){$\FPeel (\frontier,k)$}{ 
    $\nextfrontier\gets \emptyset$; $\countingbag\gets \emptyset$\tcp*[f]{$\countingbag$: vertices to recount their \induceddegree{s}}\\
    \parForEach{$v\in \frontier$} {
        \parForEach{$u\in N(v)$} {
          \If(){$\sampler[u].\mode$}{
            %\FSampleVertex{}$(u)$\label{line:sample_vertex_frame}
            $\delta\gets \atominc(\sampler[u].\cnt)$ with probability $\sampler[u].\rate$ \label{line:sample_vertex_success}\\
            \tcp{If sufficient samples are collected, add $u$ to $\countingbag$}
            \lIf{
                $\delta=\exphits-1$ \label{line:sample_hit_thres}
            }{
                Add $u$ to $\countingbag$  \label{line:sample_disable_and_count} 
            }            
          } 
          \Else {
            $\delta\gets\atomdec(\degreestar[u])$\\
            \lIf{$\delta=k+1$} {Add $u$ to $\nextfrontier$} 
          }            
        }
    }
    \Return $\langle \nextfrontier, \countingbag\rangle$
  }
  
\myfunc{\FInitSampler{$(v, k)$}}{
    \label{algo:init_sample_func}  
    \tcp{If the expected \induceddegree{} of $v$ is still large and far from $k$ 
    even after decrementing to a factor of $\reducerate$, then $v$ can be sampled safely. 
    }
    \If{%\upshape
        ($\degreestar[v] \cdot \reducerate{} > k) \wedge (\degreestar[v]>$ threshold\,$)$  \label{line:sample_cond}
    }{
        $\sampler[v].\mode \gets \true$ \label{line:set_sample_mode}\\
        \tcp{Set the sample rate. This formula is explained in \cref{sec:sampling}.}  
        $\sampler[v].\rate\gets \exphits / ((1 - \reducerate) \cdot \degreestar[v])$ \label{line:set_sample_rate} \\
        $\sampler[v].\cnt\gets 0$
    }
    \lElse{
        $\sampler[v].\mode \gets \false$
    }
}

\myfunc{\FSetSampler{$(v, k, \frontier)$}}{
    \label{algo:set_sample_func}  
    $\degreestar[v]\gets$ the number of \alive{} vertices in $\nei(v)$\\
    \lIf{$\degreestar[v]\le k$}{Add $v$ to $\frontier$}
    $\FInitSampler(v,k)$
}

\myfunc(\tcp*[f]{Explained in \cref{sec:calc_error}}){\FError{$(v, k)$}\label{line:check-resample}}{
    \Return {\upshape $(\degreestar[v]\cdot r>k)\wedge(\sampler[v].\cnt<\sampler[v].\rate\cdot(\degreestar[v] - k)/4)$}
}

%\myfunc{$\FCheckSampleSecurity(v, k,\frontier)$}{
%    \label{algo:check_sample_func}
%     %$\errorp[v] \gets$ \FCalcerror $(v, k, \samplerate)$ \label{line:calc_error}\\
%     $p\gets$ the probability that $v$'s \induceddegree{} is lower than $k$ 
%     when $v$ hits $\sampler[v].\cnt$ samples. The formula is shown in \cref{eqn:xx} \\
%     %\tcp{Calculate the error probability based on rate and \induceddegree{}}\
%    %\tcp{The calculation is explained in \cref{sec:dynamic_checking}}
%
%     \If(\tcp*[f]{$\epsilon$ is upper bound for the error rate}){
%         $p > \epsilon$  \label{line:check_error}
%    }{
%        %\tcp{if with more than $\epsilon$ probability, $v$'s \induceddegree{} reaches $k$}
%        %$\samplemode[v] \gets \false$\\ 
%        $\FCountVertex(v,\frontier)$\label{line:stop_sample_after_check} \\
%    }
%} 
%\vspace{1em}
%\myfunc{$\FSampleVertex(v)$}{
%
%    % $samplers$ holds the sampling counters, and the $\samplerate$ for each vertex.
%    \tcp{Add a sample based on the sample rate}
%    $\delta\gets \atominc(\sampler[v].\cnt)$ with probability $\sampler[v].\rate$\ \label{line:sample_vertex_success}
%
%    \If(\tcp*[f]{The increment that hits $\exphits$ will add $v$ to $\countingbag$}){
%        $\delta=\exphits-1$ \label{line:sample_hit_thres}
%    }{
%        Add $v$ to $\countingbag$  \label{line:sample_disable_and_count}
%    }
%}
%\myfunc{$\FCountVertex(v,\frontier)$} {
%    %Count the real degree of $v$ and update $\degreestar[v]$  \\
%    $\degreestar[v]\gets$ the number of \alive{} vertices in $\nei(v)$\\
%    \FSetSampler{$(v, k)$} \tcp{Reset the sampler for the vertex}
%    \lIf{$\degreestar[v]\le k$}{Add $v$ to $\frontier$}
%}


\end{algorithm}
