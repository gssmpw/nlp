%% For double-blind review submission, w/o CCS and ACM Reference (max submission space)
%\documentclass[sigconf,10pt,review,anonymous]{acmart}
%\settopmatter{printfolios=true,printccs=false,printacmref=false}
%% For double-blind review submission, w/ CCS and ACM Reference

\documentclass[sigconf,screen]{acmart}\settopmatter{printfolios=true,printccs=false,printacmref=false}
%\documentclass[sigconf,nonacm]{acmart}
%\settopmatter{printfolios=true}
%% For single-blind review submission, w/o CCS and ACM Reference (max submission space)
%\documentclass[sigplan,review]{acmart}\settopmatter{printfolios=true,printccs=false,printacmref=false}
%% For single-blind review submission, w/ CCS and ACM Reference
%\documentclass[sigplan,review]{acmart}\settopmatter{printfolios=true}
%% For final camera-ready submission, w/ required CCS and ACM Reference
%\documentclass[sigplan]{acmart}\settopmatter{}

% \def\conference
\def\fullversion

%% Conference information
%% Supplied to authors by publisher for camera-ready submission;
%% use defaults for review submission.
\setcopyright{acmcopyright}
\copyrightyear{2018}
\acmYear{2018}
\acmDOI{XXXXXXX.XXXXXXX}

%% Copyright information
%% Supplied to authors (based on authors' rights management selection;
%% see authors.acm.org) by publisher for camera-ready submission;
%% use 'none' for review submission.
%\setcopyright{none}
%\setcopyright{acmcopyright}
%\setcopyright{acmlicensed}
%\setcopyright{rightsretained}
%\copyrightyear{2018}           %% If different from \acmYear

%% Bibliography style
%\bibliographystyle{ACM-Reference-Format}
%% Citation style
%\citestyle{acmauthoryear}  %% For author/year citations
%\citestyle{acmnumeric}     %% For numeric citations
%\setcitestyle{nosort}      %% With 'acmnumeric', to disable automatic
                            %% sorting of references within a single citation;
                            %% e.g., \cite{Smith99,Carpenter05,Baker12}
                            %% rendered as [14,5,2] rather than [2,5,14].
%\setcitesyle{nocompress}   %% With 'acmnumeric', to disable automatic
                            %% compression of sequential references within a
                            %% single citation;
                            %% e.g., \cite{Baker12,Baker14,Baker16}
                            %% rendered as [2,3,4] rather than [2-4].


%%%%%%%%%%%%%%%%%%%%%%%%%%%%%%%%%%%%%%%%%%%%%%%%%%%%%%%%%%%%%%%%%%%%%%
%% Note: Authors migrating a paper from traditional SIGPLAN
%% proceedings format to PACMPL format must update the
%% '\documentclass' and topmatter commands above; see
%% 'acmart-pacmpl-template.tex'.
%%%%%%%%%%%%%%%%%%%%%%%%%%%%%%%%%%%%%%%%%%%%%%%%%%%%%%%%%%%%%%%%%%%%%%


%% Some recommended packages.
%\usepackage{times}
\usepackage{booktabs}   %% For formal tables:
                        %% http://ctan.org/pkg/booktabs
\usepackage{subcaption} %% For complex figures with subfigures/subcaptions
                        %% http://ctan.org/pkg/subcaption
\usepackage{caption}
\usepackage{colortbl}
%\geometry{lmargin=0.7in,rmargin=0.7in,tmargin=.55in,bmargin=1in}
\usepackage{bigstrut}
\usepackage{microtype}
\usepackage{tabularx}
\usepackage{multirow}
\usepackage{multicol}
\usepackage{rotating}
\usepackage{lipsum}
\usepackage{balance}
\usepackage{mfirstuc}%\captalisewords
\usepackage{tcolorbox}
\usepackage{titlecaps}%\titlecap
\usepackage{listings}
\usepackage{float}
\usepackage[rightcaption]{sidecap}
%\renewcommand{\thefootnote}{\fnsymbol{footnote}}
%\usepackage[compact]{titlesec}
 %\titlespacing*{\section}
%  {0pt}{.7ex plus .5ex minus .2ex}{.3ex plus .1ex}
 %\titlespacing*{\subsection}
 %{0pt}{.5ex plus .5ex minus .2ex}{.3ex plus .1ex}
%\titleformat{\section}{\Large\bfseries}{\thesection}{1em}{}
%\titleformat{\subsection}{\large\bfseries}{\thesubsection}{1em}{}
%\titleformat{\subsubsection}[runin]{\bfseries}{\thesubsubsection}{1em}{}

%\setlength{\textfloatsep}{5pt} % SPACE AROUND FIGURES AND TABLES

%\setlength\textheight{9in}
\usepackage{xcolor}
\usepackage[font=small,labelfont=bf,skip=2pt]{caption}
\usepackage{float}



\settopmatter{printfolios=true,printccs=false,printacmref=false}

%%%%%%%%%%%%%%%%%%%%%%%%%%%%%%%%%%%%%%%%%%%%
%% macro for conf and full versions
%%%%%%%%%%%%%%%%%%%%%%%%%%%%%%%%%%%%%%%%%%%%
% \def\conference
\def\fullversion
\usepackage{etoolbox}
\newcommand{\ifconference}[1]{{{\ifx\fullversion\undefined{#1}\fi}\xspace}}
\newcommand{\iffullversion}[1]{{{\ifx\conference\undefined{#1}\fi}\xspace}}

%%%%%%%%%%%%%%%%%%%%%%%%%%%%%%%%%%%%%%%%%%%%
%% Various Useful Packages
%%%%%%%%%%%%%%%%%%%%%%%%%%%%%%%%%%%%%%%%%%%%
\usepackage{graphicx}  % pictures and figures
\usepackage{lipsum}  % random paragraphs
\newcommand{\hide}[1]{} % hide
\usepackage{xspace}
\usepackage{textcomp}
%\usepackage[T1]{fontenc}
\usepackage{comment} % \begin{comment} ... \end{comment}
\usepackage{verbatim}
%\usepackage{fancyhdr}
\usepackage{dblfloatfix}
\usepackage{afterpage}

%%%%%%%%%%%%%%%%%%%%%%%%%%%%%%%%%%%%%%%%%%%%
%% Without acmart, should include the following:
%%%%%%%%%%%%%%%%%%%%%%%%%%%%%%%%%%%%%%%%%%%%
%\usepackage[margin=1in]{geometry}
%\usepackage[usenames,dvipsnames,svgnames,table,x11names]{xcolor}
%\usepackage[numbers,sort,compress]{natbib}
%\usepackage[colorlinks,
%citecolor=Sepia,
%linkcolor=Blue,
%pagebackref=true
%%pagebackref=true      turn this on and off for citation back references.
%]{hyperref}%         % for online version
%\usepackage{latexsym,amsthm,amsmath,amsfonts,amssymb,stmaryrd,mathtools}

%%%%%%%%%%%%%%%%%%%%%%%%%%%%%%%%%%%%%%%%%%%%
%% Colors
%%%%%%%%%%%%%%%%%%%%%%%%%%%%%%%%%%%%%%%%%%%%
% see color names here: https://www.overleaf.com/learn/latex/Using_colours_in_LaTeX
% default names: red, green, blue, cyan, magenta, yellow, black, gray, white, darkgray, lightgray, brown, lime, olive, orange, pink, purple, teal, violet
% use \color{blue} in an environment to make everything in that color
% use \textcolor{red}{text} to change text color
% use \colorbox{red}{text} to change background color
%% No need to usepackage{xcolor} with acmart
%\usepackage[usenames,dvipsnames,svgnames,table,x11names]{xcolor}
\definecolor{mypink1}{rgb}{0.858, 0.188, 0.478}
\definecolor{mypink2}{RGB}{219, 48, 122}
\definecolor{mypink3}{cmyk}{0, 0.7808, 0.4429, 0.1412}
\definecolor{mygray}{gray}{0.6}
\newcommand{\red}[1]{{\textcolor{red}{#1}}}
\newcommand{\blue}[1]{{\textcolor{blue}{#1}}}
\newcommand{\green}[1]{{\textcolor{green}{#1}}}

%%%%%%%%%%%%%%%%%%%%%%%%%%%%%%%%%%%%%%%%%%%%
%% Reference / Citations
%%%%%%%%%%%%%%%%%%%%%%%%%%%%%%%%%%%%%%%%%%%%
%% No need to usepackage{natbib} with acmart
% Use numbers, sort references, compress 1,2,3 to 1-3
%\usepackage[numbers,sort,compress]{natbib}

%% To change the reference spacing, use this:
% \setlength{\bibsep}{2.0pt}

%%%%%%%%%%%%%%%%%%%%%%%%%%%%%%%%%%%%%%%%%%%%
%% Hyperlink / url
%%%%%%%%%%%%%%%%%%%%%%%%%%%%%%%%%%%%%%%%%%%%
%% No need to usepackage{hyperref} with acmart
%\usepackage[colorlinks,
%citecolor=Sepia,
%linkcolor=Blue,
%pagebackref=true
%%pagebackref=true      turn this on and off for citation back references.
%]{hyperref}%         % for online version
\usepackage{url}

%%%%%%%%%%%%%%%%%%%%%%%%%%%%%%%%%%%%%%%%%%%%
%% fonts
%%%%%%%%%%%%%%%%%%%%%%%%%%%%%%%%%%%%%%%%%%%%
%\usepackage{times}
%\usepackage{txfonts}

%%%%%%%%%%%%%%%%%%%%%%%%%%%%%%%%%%%%%%%%%%%%
%% Formats
%%%%%%%%%%%%%%%%%%%%%%%%%%%%%%%%%%%%%%%%%%%%
\newcommand{\mtext}[1]{{\mbox{{#1}}}} % text in math mode
\newcommand{\mtextit}[1]{{\mbox{\emph{#1}}}} % text in math mode in it
\newcommand{\mtextsc}[1]{{\mbox{\sc{#1}}}} % text in math mode in sc
\newcommand{\smtext}[1]{{\mbox{\scriptsize{#1}}}} % small text in math mode
\newcommand{\smtextit}[1]{{\mbox{\scriptsize\emph{#1}}}} % small text in math mode in it
%\newcommand{\func}[1]{{\mbox{\emph{#1}}}} % function name in match
\newcommand{\algname}[1]{{\textsc{#1}}} % algorithm name format
\newcommand{\defn}[1]{\emph{\textbf{\boldmath #1\unboldmath}}} % definition style
\newcommand{\emp}[1]{\emph{\textbf{#1}}} % highlight
\newcommand{\fname}[1]{\textsf{#1}} % function name format
\newcommand{\vname}[1]{{\mathit{#1}}} % variable name format
\newcommand{\textcode}[1]{{\texttt{#1}}} % code format in text
\newcommand{\mathfunc}[1]{\mathit{#1}}
\newcommand{\nodecircle}[1]{{\textcircled{\footnotesize{#1}}}} % text in a circle

%%%%%%%%%%%%%%%%%%%%%%%%%%%%%%%%%%%%%%%%%%%%
%% Page Margin
%%%%%%%%%%%%%%%%%%%%%%%%%%%%%%%%%%%%%%%%%%%%
%% No need to usepackage{geometry} with acmart
%\usepackage[margin=1in]{geometry}
%% Instead, pass the parameters as below
%\geometry{lmargin=0.75in,rmargin=0.75in,tmargin=1in,bmargin=1in}


%% Other commonly-used commands:
% \setlength\textwidth{7in}
% \setlength\textheight{10in}
% \usepackage[text={7in,10in},centering]{geometry}
% \usepackage[margin=1.5in]{geometry}
% \usepackage[letterpaper,top=1in,bottom=1in,left=1in,right=1in,marginparwidth=1in]{geometry}
% \usepackage[lmargin=0.75in,rmargin=0.75in,tmargin=.9in,bmargin=.9in]{geometry}
% \special{papersize=8.5in,11in}
% \setlength{\pdfpagewidth}{8.5in}
% \setlength{\pdfpageheight}{11in}


%% Previously-used space trick for lipics template:
% \setlength\textwidth{6.1in}
% \setlength\hoffset{-1.4in}
% \setlength\voffset{-0.8in}
% \setlength\textheight{9.2in}
%\setlength\marginparwidth{3pt}
%\setlength\marginparsep{0pt}
% \addtolength{\textwidth}{.08in}

%%%%%%%%%%%%%%%%%%%%%%%%%%%%%%%%%%%%%%%%%%%%
%% Other global spacing settings
%%%%%%%%%%%%%%%%%%%%%%%%%%%%%%%%%%%%%%%%%%%%

%\setlength{\parindent}{0em}  % Changes the indent of paragraph
%\setlength{\parskip}{0em} % Changes the spacing between paragraphs
%\addtolength{\parskip}{0.15ex} % add 0.15ex to parskip
%\renewcommand{\baselinestretch}{.97} % Changes spacing between lines


%%%%%%%%%%%%%%%%%%%%%%%%%%%%%%%%%%%%%%%%%%%%
%% Algorithms
%%%%%%%%%%%%%%%%%%%%%%%%%%%%%%%%%%%%%%%%%%%%
\usepackage[ruled,lined,linesnumbered,noend]{algorithm2e}
\usepackage[noend]{algpseudocode}

\SetAlCapNameFnt{\small}
\SetAlCapFnt{\small}

\makeatletter
% Remove right hand margin in algorithm
\patchcmd{\@algocf@start}% <cmd>
  {-1.5em}% <search>
  {0pt}% <replace>
  {}{}% <success><failure>
\setlength{\algomargin}{.5em}   % left margin

\newcommand{\nosemic}{\renewcommand{\@endalgocfline}{\relax}}% Drop semi-colon ;
\newcommand{\dosemic}{\renewcommand{\@endalgocfline}{\algocf@endline}}% Reinstate semi-colon ;
\newcommand{\popline}{\Indm\dosemic}% Undent
\newcommand{\pushline}{\Indp}% Indent

\SetSideCommentLeft
% use \notations{...} or \notes{...} etc.
\SetKwInput{notations}{Notations}
\SetKwInput{notes}{Notes}
\SetKwInput{maintains}{Maintains}

% use \myfunc(right-aligned comment){function name}{...function content...}
\SetKwProg{myfunc}{Function}{}{}
% use \parForEach as a regular \For
\SetKwFor{parForEach}{ParallelForEach}{do}{endfor}
\SetKwFor{Justrepeat}{Repeat}{}{}

\SetKw{MIN}{min}
\SetKw{MAX}{max}
\SetKw{OR}{or}
\SetKw{AND}{and}

% CommentStyle
\newcommand\mycommfont[1]{\textit{\textcolor{blue}{#1}}}
%\definecolor{commentgreen}{RGB}{0,128,0}
%\newcommand\mycommfont[1]{\textit{\textcolor{commentgreen}{#1}}}
\SetCommentSty{mycommfont}

%%%%%%%%%%%%%%%%%%%%%%%%%%%%%%%%%%%%%%%%%%%%
%% cref (cleveref)
%%%%%%%%%%%%%%%%%%%%%%%%%%%%%%%%%%%%%%%%%%%%
\usepackage{cleveref}
\crefname{section}{Sec.}{Sec.}
\crefname{theorem}{Thm.}{Thm.}
\crefname{lemma}{Lem.}{Lem.}
\crefname{corollary}{Col.}{Col.}
\crefname{table}{Tab.}{Tab.}
\crefname{algorithm}{Alg.}{Alg.}
\crefname{figure}{Fig.}{Fig.}
\crefname{fact}{Fact}{Fact}
\Crefname{table}{Tab.}{Tab.}
\crefname{problem}{Problem}{Problem}


%%%%%%%%%%%%%%%%%%%%%%%%%%%%%%%%%%%%%%%%%%%%
%% Math and Theorem
%%%%%%%%%%%%%%%%%%%%%%%%%%%%%%%%%%%%%%%%%%%%
%% No need to use the packages below with acmart
%\usepackage{latexsym,amsthm,amsmath,amsfonts,amssymb,stmaryrd,mathtools}

\newtheorem{theorem}{Theorem}[section]
\newtheorem{lemma}[theorem]{Lemma}
\newtheorem{corollary}[theorem]{Corollary}
\newtheorem{claim}[theorem]{Claim}
\newtheorem{fact}[theorem]{Fact}
\newtheorem{invariant}[theorem]{Invariant}
\newtheorem{definition}{Definition}
\newtheorem{remark}{Remark}

% \left and \right
\let \originalleft \left
\let\originalright\right
\renewcommand{\left}{\mathopen{}\mathclose\bgroup\originalleft}
\renewcommand{\right}{\aftergroup\egroup\originalright}

\usepackage{scalerel} % stretch a formula

% compact theorem
\newtheoremstyle{exampstyle}
{.5em} % Space above
{1em} % Space below
{\it} % Body font
{.5em} % Indent amount
{\it \bfseries} % Theorem head font
{.} % Punctuation after theorem head
{.5em} % Space after theorem head
{} % Theorem head spec (can be left empty, meaning `normal')
%\theoremstyle{exampstyle} \newtheorem{example}{Example}
%\theoremstyle{exampstyle} \newtheorem{remark}{Remark}
\theoremstyle{exampstyle} \newtheorem{compactdef}{Definition}
\theoremstyle{exampstyle} \newtheorem{compactlem}{Lemma}
\theoremstyle{exampstyle} \newtheorem{compactprob}{Problem}
\theoremstyle{exampstyle} \newtheorem{compactthm}{Theorem}

% compact proof
\makeatletter
\renewenvironment{proof}[1][\proofname]{\par
\vspace{-2\topsep}% remove the space after the theorem
\pushQED{\qed}%
\normalfont
\topsep0pt \partopsep0pt % no space before
\trivlist
\item[\hskip\labelsep
      \itshape
  #1\@addpunct{.}]\ignorespaces
}{%
\popQED\endtrivlist\@endpefalse
%\addvspace{3pt plus 3pt} % some space after
}

%%%%%%%%%%%%%%%%%%%%%%%%%%%%%%%%%%%%%%%%%%%%
%% Math / Notation / Definitions
%%%%%%%%%%%%%%%%%%%%%%%%%%%%%%%%%%%%%%%%%%%%

% a list of symbols: https://math.uoregon.edu/wp-content/uploads/2014/12/compsymb-1qyb3zd.pdf

%    integer/real number Sets:
\newcommand{\R}{\mathbb{R}}
\newcommand{\N}{\mathbb{N}}
\newcommand{\Z}{\mathbb{Z}}
%    Some letters:
\newcommand{\mathd}{\mathcal{D}\xspace}
\newcommand{\mathq}{\mathcal{Q}\xspace}
\newcommand{\mathc}{\mathcal{C}\xspace}

\newcommand{\whp}[1]{\emph{whp}}
\DeclareMathOperator{\E}{\mathbb{E}}
\DeclareMathOperator*{\argmin}{\arg\!\min}
\DeclareMathOperator*{\polylog}{polylog}

\newcommand{\true}{\emph{true}}
\newcommand{\false}{\emph{false}}
\newcommand{\True}{\textsc{True}\xspace}
\newcommand{\False}{\textsc{False}\xspace}
\newcommand{\zero}{\textbf{zero}}
\newcommand{\one}{\textbf{one}}

\usepackage{pifont}
%\newcommand{\cmark}{\ding{51}}%
%\newcommand{\xmark}{\ding{55}}%
%\usepackage{wasysym}
%\usepackage{utfsym}
%\usepackage{bbding}
\newcommand{\cmark}{\ding{51}} % checkmark
\newcommand{\xmark}{\ding{55}} % crossmark

% Binary Forking / other cost models:
\newcommand{\modelop}[1]{\texttt{#1}}
\newcommand{\forkins}{\modelop{fork}}
\newcommand{\insend}{\modelop{end}}
\newcommand{\allocateins}{\modelop{allocate}}
\newcommand{\freeins}{\modelop{free}}
\newcommand{\thread}{thread}
\newcommand{\bfmodel}{binary-forking model}
\newcommand{\MPC}[0]{\ensuremath{\mathsf{MPC}}}
\newcommand{\spanterm}{span}
\newcommand{\paradepth}{span}
\newcommand{\TAS}[0]{$\mf{TestAndSet}$}
\newcommand{\TS}[0]{$\texttt{TS}$}
\newcommand{\insformat}[1]{\texttt{#1}}
\newcommand{\tas}{{\insformat{test\_and\_set}}}
\newcommand{\faa}{{\insformat{fetch\_and\_add}}}
\newcommand{\FAA}{{\insformat{FAA}}}
\newcommand{\faadec}{{\insformat{atomic\_decrement}}}
\newcommand{\faainc}{{\insformat{atomic\_increment}}}
\newcommand{\cas}{{\insformat{compare\_and\_swap}}}
\newcommand{\CAS}{{\insformat{CAS}}}
\newcommand{\atinc}{{\insformat{atomic\_inc}}}
\newcommand{\WriteMin}{\insformat{write\_min}\xspace}
\newcommand{\WriteMax}{\insformat{write_max}\xspace}
\newcommand{\writemax}{\WriteMax}
\newcommand{\writemin}{\WriteMin}




%%%%%%%%%%%%%%%%%%%%%%%%%%%%%%%%%%%%%%%%%%%%
%% List: Enumerate / Itemize
%%%%%%%%%%%%%%%%%%%%%%%%%%%%%%%%%%%%%%%%%%%%
\usepackage[shortlabels]{enumitem}

% Spacing for lists. This can also be set for each list separately by using [...]
\setlist{topsep=0.3em,itemsep=0.2em,parsep=0.1em,leftmargin=*}
% add "wide" to remove in-item indents
%\setlist{topsep=0.3em,itemsep=0.2em,parsep=0.1em,leftmargin=*,wide}
% "nosep" removes all vertical spacing
%\setlist{nosep,wide}

%%%%%%%%%%%%%%%%%%%%%%%%%%%%%%%%%%%%%%%%%%%%
%% Floating: Figure / Table / Algorithm
%%%%%%%%%%%%%%%%%%%%%%%%%%%%%%%%%%%%%%%%%%%%
\usepackage{float}
\usepackage[labelfont=bf,font={small},aboveskip=0em, belowskip=0em]{caption}

%%%%%%%%%% Floating Spacing %%%%%%%%%%%%%
% Can also set space around the caption separately
%\setlength\abovecaptionskip{0em}
%\setlength\belowcaptionskip{0em}
% Space between multiple floatings
\setlength{\floatsep}{0em}
% space below floating (distance to the rest of text)
\setlength{\textfloatsep}{0.5em}
% space above tables/figures (distance from the text above)
% for tables in the middle of the page (i.e., not top or bottom), this number is both the top spacing and bottom spacing
\setlength{\intextsep}{0.5em}
%%%%% These two are for double-column floatings, e.g., figure* and table*
\setlength{\dbltextfloatsep}{1em} % floating to text
\setlength{\dblfloatsep}{0.5em} % between floatings

%%%% \tabcolsep changes the horizontal spacing between columns
%\setlength{\tabcolsep}{10pt} % default = 6
%%%% \arraystretch changes the vertical spacing between rows
%\renewcommand{\arraystretch}{1.1} % default = 1

%%%%%%%%%% Subfigures %%%%%%%%%%%%%
% Use "\begin{subfigure}[b]{0.3\textwidth}" for a subfigure
\usepackage[labelfont=bf,list=true,skip=0em]{subcaption}
\captionsetup[table]{textfont=normalfont,position=bottom}
\captionsetup[figure]{textfont=normalfont,position=bottom}
% Useful environments: \subcaptionbox{caption}{content}
% Useful environments: \subcaptiongroup{ all captions will be labeled as subcaptions }

%%%%%%%%%% Side Captions %%%%%%%%%%%%%
% \begin{SCtable} [⟨relwidth⟩][⟨float⟩] ... \end{SCtable}
% \begin{SCfigure} [⟨relwidth⟩][⟨float⟩] ... \end{SCfigure}
% \begin{SCtable*} [⟨relwidth⟩][⟨float⟩] ... \end{SCtable*}
% \begin{SCfigure*}[⟨relwidth⟩][⟨float⟩] ... \end{SCfigure*}
\usepackage[rightcaption]{sidecap}

%%%%%%%%%% Wrapfigure %%%%%%%%%%%%%
% \begin{wrapfigure}[lineheight]{position}{width}  ... \end{wrapfigure}
\usepackage{wrapfig}

%%%%%%%%% Table settings %%%%%%%%%%%%%%
\usepackage{array}% for extended column definitions
% Multi-line column with fixed length. Use \par for a new line.
\newcolumntype{L}[1]{>{\raggedright\let\newline\\\arraybackslash\hspace{0pt}}m{#1}}
\newcolumntype{C}[1]{>{\centering\let\newline\\\arraybackslash\hspace{0pt}}m{#1}}
\newcolumntype{R}[1]{>{\raggedleft\let\newline\\\arraybackslash\hspace{0pt}}m{#1}}
% bold and center
\newcolumntype{B}{>{\bf}c}
% Rotate text in cells
\usepackage{rotating}

\usepackage{booktabs} % provides toprule, bottomrule, midrule, cmidrule, etc.
\usepackage{multicol,multirow}
\usepackage{longtable} % provides long table
\usepackage{supertabular} % similar to long table, allowing tables to take more than one page
\usepackage{colortbl}
\usepackage{bigstrut}
% minitab alignment to change inside one cell
\newcommand{\minitab}[2]{\multicolumn{1}{#1}{#2}}

%%%%%%%%%%%%%%%%%%%%%%%%%%%%%%%%%%%%%%%%%%%%
%% Section titles
%%%%%%%%%%%%%%%%%%%%%%%%%%%%%%%%%%%%%%%%%%%%
\usepackage{titlesec}
%% Change section and subsection title to normal font size
%\titleformat{\section}{\normalfont\large\bfseries}{\thesection}{1em}{}
\titleformat{\subsection}{\normalfont\large\bfseries}{\thesubsection}{1em}{}

% Change title spacing
\titlespacing*{\section}{0pt}{0.3em}{0.2em} % left margin, space before, space after
\titlespacing*{\subsection}{0pt}{0.3em}{0.2em} % left margin, space before, space after
\titlespacing*{\subsubsection}{0pt}{0.1em}{1em} % left margin, space before, space after (horizontal)
\newcommand{\mysubsubsection}[1]{{#1}.}
\titleformat{\subsubsection}[runin]
{\normalfont\normalsize\bfseries}{\thesubsubsection}{1em}{\mysubsubsection}

\newcommand{\para}[1]{{\bf \emph{#1}}\,}
\newcommand{\myparagraph}[1]{\vspace{.1em}\noindent\emp{#1}\enspace}





%%%%%%%%%%%%%%%%%%%%%%%%%%%%%%%%%%%%%%%%%%%%
%% Framedbox
%%%%%%%%%%%%%%%%%%%%%%%%%%%%%%%%%%%%%%%%%%%%
% Use "\begin{mdframed}[style=mystyle] ... \end{mdframed}"
\usepackage{mdframed}
\definecolor{framelinecolor}{RGB}{68,114,196}
\mdfdefinestyle{mystyle}{linecolor=framelinecolor,innertopmargin=1pt,innerbottommargin=2pt,backgroundcolor=gray!20,skipabove=2pt,skipbelow=0pt}%leftmargin=0,topmargin=
\mdfdefinestyle{densestyle}{linecolor=framelinecolor,innertopmargin=0,innerbottommargin=0,leftmargin=0,rightmargin=0,backgroundcolor=gray!20}
\mdfdefinestyle{compactcode}{linecolor=framelinecolor,innertopmargin=1pt,innerbottommargin=1pt,backgroundcolor=gray!20,skipabove=0pt,skipbelow=0pt,leftmargin=0,rightmargin=0}

% package framed also provides a simple framed box
\usepackage{framed}


%%%%%%%%%%%%%%%%%%%%%%%%%%%%%%%%%%%%%%%%%%%%
%% Listing codes
%%%%%%%%%%%%%%%%%%%%%%%%%%%%%%%%%%%%%%%%%%%%
\usepackage{listings}
\newcommand{\codeskip}{{\vspace{.05in}}}

%% LISTING ENVIRONMENT (lstlisting)
\newdimen\zzsize
\zzsize=8pt
\newdimen\kwsize
\kwsize=8pt

%\newcommand{\basicstyle}{\ttfamily}
\newcommand{\basicstyle}{\fontsize{\zzsize}{1\zzsize}\ttfamily}
%\newcommand{\keywordstyle}{\ttfamily\bf}
\newcommand{\keywordstyle}{\fontsize{\kwsize}{1\kwsize}\ttfamily\bf}

\newdimen\zzlstwidth
%\newlength{\zzlstwidth}
\settowidth{\zzlstwidth}{{\basicstyle~}}
\newcommand{\lcm}{}

\lstset{
%  aboveskip=-0.5 \baselineskip,
%  belowskip=-0.8 \baselineskip,
  xleftmargin=0.5em,
  basewidth=\zzlstwidth,
  basicstyle=\basicstyle,
  columns=fullflexible,
  captionpos=b,
  numbers=left, numberstyle=\small, numbersep=4pt,
  language=C++,
  keywordstyle=\keywordstyle,
  keywords={return,signature,sig,structure,struct,fun,fn,case,type,datatype,let,fn,in,end,functor,alloc,if,then,else,while,with,AND,start,do,parallel,for,parallel_for},
  commentstyle=\rmfamily\slshape,
  morecomment=[l]{\%},
  lineskip={1.5pt},
  columns=fullflexible,
  keepspaces=true,
  mathescape=true,
  escapeinside={@}{@}
}

%%%%%%%%%%%%%%%%%%%%%%%%%%%%%%%%%%%%%%%%%%%%
%% Edits
%%%%%%%%%%%%%%%%%%%%%%%%%%%%%%%%%%%%%%%%%%%%
\newcommand{\newchange}[1]{{\color{red}#1}}
\newcommand{\revision}[1]{{\color{purple}#1}}

%%%%%%%%%%%%%%%%%%%%%%%%%%%%%%%%%%%%%%%%%%%%
%% Other tools
%%%%%%%%%%%%%%%%%%%%%%%%%%%%%%%%%%%%%%%%%%%%
\usepackage{tikz} % draw geometric objects


%%%%%%%%%%%%%%%%%%%%%%%%%%%%%%%%%%%%%%%%%%%%
%% PENALTY
%%%%%%%%%%%%%%%%%%%%%%%%%%%%%%%%%%%%%%%%%%%%

% See their definitions and default values in: https://en.wikibooks.org/wiki/TeX/penalty
\binoppenalty=700
%\brokenpenalty=0 %100
%\clubpenalty=0   %150
\displaywidowpenalty=0   %50
\exhyphenpenalty=50
\floatingpenalty=20000
\hyphenpenalty=50
\interlinepenalty=0
\linepenalty=10
\postdisplaypenalty=0
\predisplaypenalty=0 %10000
\relpenalty=500
%\widowpenalty=0  %150

%%%%%%%%%%%%%%%%%%%%%%%%%%%%%%%%%%%%%%%%%%%%
%% For Submissions, acmart template
%%%%%%%%%%%%%%%%%%%%%%%%%%%%%%%%%%%%%%%%%%%%
\setcopyright{none}
\renewcommand\footnotetextcopyrightpermission[1]{} % This line removes the footnote about the conference and year.
%\def\@titlefont{\huge\sffamily\bfseries} % THIS LINE CHANGES THE FONT OF THE TITLE


%%%%%%%%%%%%%%%%%%%%%%%%%%%%%%%%%%%%%%%%%%%%
%% Display spacing
%%%%%%%%%%%%%%%%%%%%%%%%%%%%%%%%%%%%%%%%%%%%
% Put this after \begin{document}
\setlength\abovedisplayskip{0pt}
\setlength\belowdisplayskip{0pt}
\setlength\abovedisplayshortskip{0pt}
\setlength\belowdisplayshortskip{0pt} 

\def\method{\text MixMin~}
\def\methodnospace{\text MixMin}
\def\genmethod{$\mathbb{R}$\text Min~}
\def\genmethodnospace{ $\mathbb{R}$\text Min}



\begin{document}
% \iffullversion{
% \input{revision-cover-page/cover_page.tex}
% }

% The following includes the CC license icon appropriate for your paper.
% Download the image from www.scomminc.com/pp/acmsig/4ACM-CC-by-88x31.eps
% and place within your figs or figures folder
\makeatletter
\gdef\@copyrightpermission{
  \begin{minipage}{0.2\columnwidth}
   \href{https://creativecommons.org/licenses/by/4.0/}{\includegraphics[width=0.90\textwidth]{figures/cc_by4acm.png}}
  \end{minipage}\hfill
  \begin{minipage}{0.8\columnwidth}
   \href{https://creativecommons.org/licenses/by/4.0/}{This work is licensed under a Creative Commons Attribution International 4.0 License.}
  \end{minipage}
  \vspace{5pt}
}
\makeatother




%%
%% The "title" command has an optional parameter,
%% allowing the author to define a "short title" to be used in page headers.
\title{Parallel $k$-Core Decomposition: Theory and Practice}





%%
%% The "author" command and its associated commands are used to define
%% the authors and their affiliations.
%% Of note is the shared affiliation of the first two authors, and the
%% "authornote" and "authornotemark" commands
%% used to denote shared contribution to the research.
\settopmatter{authorsperrow=4}
\author{Youzhe Liu}
\email{yliu908@ucr.edu}
\affiliation{%
  \institution{UC Riverside}
  \city{}
  \country{}
}
\author{Xiaojun Dong}
\email{xdong038@ucr.edu}
\affiliation{%
  \institution{UC Riverside}
  \city{}
  \country{}
}
\author{Yan Gu}
\email{ygu@cs.ucr.edu}
\affiliation{%
  \institution{UC Riverside}
  \city{}
  \country{}
}
\author{Yihan Sun}
\email{yihans@cs.ucr.edu}
\affiliation{%
  \institution{UC Riverside}
  \city{}
  \country{}
}






%%
%% By default, the full list of authors will be used in the page
%% headers. Often, this list is too long, and will overlap
%% other information printed in the page headers. This command allows
%% the author to define a more concise list
%% of authors' names for this purpose.
\renewcommand{\shortauthors}{Liu et al.}





%%
%% The abstract is a short summary of the work to be presented in the
%% article.
\begin{abstract}  
Test time scaling is currently one of the most active research areas that shows promise after training time scaling has reached its limits.
Deep-thinking (DT) models are a class of recurrent models that can perform easy-to-hard generalization by assigning more compute to harder test samples.
However, due to their inability to determine the complexity of a test sample, DT models have to use a large amount of computation for both easy and hard test samples.
Excessive test time computation is wasteful and can cause the ``overthinking'' problem where more test time computation leads to worse results.
In this paper, we introduce a test time training method for determining the optimal amount of computation needed for each sample during test time.
We also propose Conv-LiGRU, a novel recurrent architecture for efficient and robust visual reasoning. 
Extensive experiments demonstrate that Conv-LiGRU is more stable than DT, effectively mitigates the ``overthinking'' phenomenon, and achieves superior accuracy.
\end{abstract}  





%%
%% The code below is generated by the tool at http://dl.acm.org/ccs.cfm.
%% Please copy and paste the code instead of the example below.
%%
\begin{CCSXML}
<ccs2012>
  <concept>
  <concept_id>10010520.10010553.10010562</concept_id>
  <concept_desc>Computer systems organization~Embedded systems</concept_desc>
  <concept_significance>500</concept_significance>
  </concept>
  <concept>
  <concept_id>10010520.10010575.10010755</concept_id>
  <concept_desc>Computer systems organization~Redundancy</concept_desc>
  <concept_significance>300</concept_significance>
  </concept>
  <concept>
  <concept_id>10010520.10010553.10010554</concept_id>
  <concept_desc>Computer systems organization~Robotics</concept_desc>
  <concept_significance>100</concept_significance>
  </concept>
  <concept>
  <concept_id>10003033.10003083.10003095</concept_id>
  <concept_desc>Networks~Network reliability</concept_desc>
  <concept_significance>100</concept_significance>
  </concept>
</ccs2012>
\end{CCSXML}

\ccsdesc[500]{Computer systems organization~Embedded systems}
\ccsdesc[300]{Computer systems organization~Redundancy}
\ccsdesc{Computer systems organization~Robotics}
\ccsdesc[100]{Networks~Network reliability}

%%
%% Keywords. The author(s) should pick words that accurately describe
%% the work being presented. Separate the keywords with commas.
%\keywords{datasets, neural networks, gaze detection, text tagging}






%\renewcommand\footnotetextcopyrightpermission[1]{} % This line removes the footnote about the conference and year.
\fancyhead{} % This line removes the page headers about the conference and authors.






%%
%% This command processes the author and affiliation and title
%% information and builds the first part of the formatted document.
\maketitle

\section{Introduction}


\begin{figure}[t]
\centering
\includegraphics[width=0.6\columnwidth]{figures/evaluation_desiderata_V5.pdf}
\vspace{-0.5cm}
\caption{\systemName is a platform for conducting realistic evaluations of code LLMs, collecting human preferences of coding models with real users, real tasks, and in realistic environments, aimed at addressing the limitations of existing evaluations.
}
\label{fig:motivation}
\end{figure}

\begin{figure*}[t]
\centering
\includegraphics[width=\textwidth]{figures/system_design_v2.png}
\caption{We introduce \systemName, a VSCode extension to collect human preferences of code directly in a developer's IDE. \systemName enables developers to use code completions from various models. The system comprises a) the interface in the user's IDE which presents paired completions to users (left), b) a sampling strategy that picks model pairs to reduce latency (right, top), and c) a prompting scheme that allows diverse LLMs to perform code completions with high fidelity.
Users can select between the top completion (green box) using \texttt{tab} or the bottom completion (blue box) using \texttt{shift+tab}.}
\label{fig:overview}
\end{figure*}

As model capabilities improve, large language models (LLMs) are increasingly integrated into user environments and workflows.
For example, software developers code with AI in integrated developer environments (IDEs)~\citep{peng2023impact}, doctors rely on notes generated through ambient listening~\citep{oberst2024science}, and lawyers consider case evidence identified by electronic discovery systems~\citep{yang2024beyond}.
Increasing deployment of models in productivity tools demands evaluation that more closely reflects real-world circumstances~\citep{hutchinson2022evaluation, saxon2024benchmarks, kapoor2024ai}.
While newer benchmarks and live platforms incorporate human feedback to capture real-world usage, they almost exclusively focus on evaluating LLMs in chat conversations~\citep{zheng2023judging,dubois2023alpacafarm,chiang2024chatbot, kirk2024the}.
Model evaluation must move beyond chat-based interactions and into specialized user environments.



 

In this work, we focus on evaluating LLM-based coding assistants. 
Despite the popularity of these tools---millions of developers use Github Copilot~\citep{Copilot}---existing
evaluations of the coding capabilities of new models exhibit multiple limitations (Figure~\ref{fig:motivation}, bottom).
Traditional ML benchmarks evaluate LLM capabilities by measuring how well a model can complete static, interview-style coding tasks~\citep{chen2021evaluating,austin2021program,jain2024livecodebench, white2024livebench} and lack \emph{real users}. 
User studies recruit real users to evaluate the effectiveness of LLMs as coding assistants, but are often limited to simple programming tasks as opposed to \emph{real tasks}~\citep{vaithilingam2022expectation,ross2023programmer, mozannar2024realhumaneval}.
Recent efforts to collect human feedback such as Chatbot Arena~\citep{chiang2024chatbot} are still removed from a \emph{realistic environment}, resulting in users and data that deviate from typical software development processes.
We introduce \systemName to address these limitations (Figure~\ref{fig:motivation}, top), and we describe our three main contributions below.


\textbf{We deploy \systemName in-the-wild to collect human preferences on code.} 
\systemName is a Visual Studio Code extension, collecting preferences directly in a developer's IDE within their actual workflow (Figure~\ref{fig:overview}).
\systemName provides developers with code completions, akin to the type of support provided by Github Copilot~\citep{Copilot}. 
Over the past 3 months, \systemName has served over~\completions suggestions from 10 state-of-the-art LLMs, 
gathering \sampleCount~votes from \userCount~users.
To collect user preferences,
\systemName presents a novel interface that shows users paired code completions from two different LLMs, which are determined based on a sampling strategy that aims to 
mitigate latency while preserving coverage across model comparisons.
Additionally, we devise a prompting scheme that allows a diverse set of models to perform code completions with high fidelity.
See Section~\ref{sec:system} and Section~\ref{sec:deployment} for details about system design and deployment respectively.



\textbf{We construct a leaderboard of user preferences and find notable differences from existing static benchmarks and human preference leaderboards.}
In general, we observe that smaller models seem to overperform in static benchmarks compared to our leaderboard, while performance among larger models is mixed (Section~\ref{sec:leaderboard_calculation}).
We attribute these differences to the fact that \systemName is exposed to users and tasks that differ drastically from code evaluations in the past. 
Our data spans 103 programming languages and 24 natural languages as well as a variety of real-world applications and code structures, while static benchmarks tend to focus on a specific programming and natural language and task (e.g. coding competition problems).
Additionally, while all of \systemName interactions contain code contexts and the majority involve infilling tasks, a much smaller fraction of Chatbot Arena's coding tasks contain code context, with infilling tasks appearing even more rarely. 
We analyze our data in depth in Section~\ref{subsec:comparison}.



\textbf{We derive new insights into user preferences of code by analyzing \systemName's diverse and distinct data distribution.}
We compare user preferences across different stratifications of input data (e.g., common versus rare languages) and observe which affect observed preferences most (Section~\ref{sec:analysis}).
For example, while user preferences stay relatively consistent across various programming languages, they differ drastically between different task categories (e.g. frontend/backend versus algorithm design).
We also observe variations in user preference due to different features related to code structure 
(e.g., context length and completion patterns).
We open-source \systemName and release a curated subset of code contexts.
Altogether, our results highlight the necessity of model evaluation in realistic and domain-specific settings.





\section{Preliminaries}
\label{sec:prelim}
\label{sec:term}
We define the key terminologies used, primarily focusing on the hidden states (or activations) during the forward pass. 

\paragraph{Components in an attention layer.} We denote $\Res$ as the residual stream. We denote $\Val$ as Value (states), $\Qry$ as Query (states), and $\Key$ as Key (states) in one attention head. The \attlogit~represents the value before the softmax operation and can be understood as the inner product between  $\Qry$  and  $\Key$. We use \Attn~to denote the attention weights of applying the SoftMax function to \attlogit, and ``attention map'' to describe the visualization of the heat map of the attention weights. When referring to the \attlogit~from ``$\tokenB$'' to  ``$\tokenA$'', we indicate the inner product  $\langle\Qry(\tokenB), \Key(\tokenA)\rangle$, specifically the entry in the ``$\tokenB$'' row and ``$\tokenA$'' column of the attention map.

\paragraph{Logit lens.} We use the method of ``Logit Lens'' to interpret the hidden states and value states \citep{belrose2023eliciting}. We use \logit~to denote pre-SoftMax values of the next-token prediction for LLMs. Denote \readout~as the linear operator after the last layer of transformers that maps the hidden states to the \logit. 
The logit lens is defined as applying the readout matrix to residual or value states in middle layers. Through the logit lens, the transformed hidden states can be interpreted as their direct effect on the logits for next-token prediction. 

\paragraph{Terminologies in two-hop reasoning.} We refer to an input like “\Src$\to$\brga, \brgb$\to$\Ed” as a two-hop reasoning chain, or simply a chain. The source entity $\Src$ serves as the starting point or origin of the reasoning. The end entity $\Ed$ represents the endpoint or destination of the reasoning chain. The bridge entity $\Brg$ connects the source and end entities within the reasoning chain. We distinguish between two occurrences of $\Brg$: the bridge in the first premise is called $\brga$, while the bridge in the second premise that connects to $\Ed$ is called $\brgc$. Additionally, for any premise ``$\tokenA \to \tokenB$'', we define $\tokenA$ as the parent node and $\tokenB$ as the child node. Furthermore, if at the end of the sequence, the query token is ``$\tokenA$'', we define the chain ``$\tokenA \to \tokenB$, $\tokenB \to \tokenC$'' as the Target Chain, while all other chains present in the context are referred to as distraction chains. Figure~\ref{fig:data_illustration} provides an illustration of the terminologies.

\paragraph{Input format.}
Motivated by two-hop reasoning in real contexts, we consider input in the format $\bos, \text{context information}, \query, \answer$. A transformer model is trained to predict the correct $\answer$ given the query $\query$ and the context information. The context compromises of $K=5$ disjoint two-hop chains, each appearing once and containing two premises. Within the same chain, the relative order of two premises is fixed so that \Src$\to$\brga~always precedes \brgb$\to$\Ed. The orders of chains are randomly generated, and chains may interleave with each other. The labels for the entities are re-shuffled for every sequence, choosing from a vocabulary size $V=30$. Given the $\bos$ token, $K=5$ two-hop chains, \query, and the \answer~tokens, the total context length is $N=23$. Figure~\ref{fig:data_illustration} also illustrates the data format. 

\paragraph{Model structure and training.} We pre-train a three-layer transformer with a single head per layer. Unless otherwise specified, the model is trained using Adam for $10,000$ steps, achieving near-optimal prediction accuracy. Details are relegated to Appendix~\ref{app:sec_add_training_detail}.


% \RZ{Do we use source entity, target entity, and mediator entity? Or do we use original token, bridge token, end token?}





% \paragraph{Basic notations.} We use ... We use $\ve_i$ to denote one-hot vectors of which only the $i$-th entry equals one, and all other entries are zero. The dimension of $\ve_i$ are usually omitted and can be inferred from contexts. We use $\indicator\{\cdot\}$ to denote the indicator function.

% Let $V > 0$ be a fixed positive integer, and let $\vocab = [V] \defeq \{1, 2, \ldots, V\}$ be the vocabulary. A token $v \in \vocab$ is an integer in $[V]$ and the input studied in this paper is a sequence of tokens $s_{1:T} \defeq (s_1, s_2, \ldots, s_T) \in \vocab^T$ of length $T$. For any set $\mathcal{S}$, we use $\Delta(\mathcal{S})$ to denote the set of distributions over $\mathcal{S}$.

% % to a sequence of vectors $z_1, z_2, \ldots, z_T \in \real^{\dout}$ of dimension $\dout$ and length $T$.

% Let $\mU = [\vu_1, \vu_2, \ldots, \vu_V]^\transpose \in \real^{V\times d}$ denote the token embedding matrix, where the $i$-th row $\vu_i \in \real^d$ represents the $d$-dimensional embedding of token $i \in [V]$. Similarly, let $\mP = [\vp_1, \vp_2, \ldots, \vp_T]^\transpose \in \real^{T\times d}$ denote the positional embedding matrix, where the $i$-th row $\vp_i \in \real^d$ represents the $d$-dimensional embedding of position $i \in [T]$. Both $\mU$ and $\mP$ can be fixed or learnable.

% After receiving an input sequence of tokens $s_{1:T}$, a transformer will first process it using embedding matrices $\mU$ and $\mP$ to obtain a sequence of vectors $\mH = [\vh_1, \vh_2, \ldots, \vh_T] \in \real^{d\times T}$, where 
% \[
% \vh_i = \mU^\transpose\ve_{s_i} + \mP^\transpose\ve_{i} = \vu_{s_i} + \vp_i.
% \]

% We make the following definitions of basic operations in a transformer.

% \begin{definition}[Basic operations in transformers] 
% \label{defn:operators}
% Define the softmax function $\softmax(\cdot): \real^d \to \real^d$ over a vector $\vv \in \real^d$ as
% \[\softmax(\vv)_i = \frac{\exp(\vv_i)}{\sum_{j=1}^d \exp(\vv_j)} \]
% and define the softmax function $\softmax(\cdot): \real^{m\times n} \to \real^{m \times n}$ over a matrix $\mV \in \real^{m\times n}$ as a column-wise softmax operator. For a squared matrix $\mM \in \real^{m\times m}$, the causal mask operator $\mask(\cdot): \real^{m\times m} \to \real^{m\times m}$  is defined as $\mask(\mM)_{ij} = \mM_{ij}$ if $i \leq j$ and  $\mask(\mM)_{ij} = -\infty$ otherwise. For a vector $\vv \in \real^n$ where $n$ is the number of hidden neurons in a layer, we use $\layernorm(\cdot): \real^n \to \real^n$ to denote the layer normalization operator where
% \[
% \layernorm(\vv)_i = \frac{\vv_i-\mu}{\sigma}, \mu = \frac{1}{n}\sum_{j=1}^n \vv_j, \sigma = \sqrt{\frac{1}{n}\sum_{j=1}^n (\vv_j-\mu)^2}
% \]
% and use $\layernorm(\cdot): \real^{n\times m} \to \real^{n\times m}$ to denote the column-wise layer normalization on a matrix.
% We also use $\nonlin(\cdot)$ to denote element-wise nonlinearity such as $\relu(\cdot)$.
% \end{definition}

% The main components of a transformer are causal self-attention heads and MLP layers, which are defined as follows.

% \begin{definition}[Attentions and MLPs]
% \label{defn:attn_mlp} 
% A single-head causal self-attention $\attn(\mH;\mQ,\mK,\mV,\mO)$ parameterized by $\mQ,\mK,\mV \in \real^{{\dqkv\times \din}}$ and $\mO \in \real^{\dout\times\dqkv}$ maps an input matrix $\mH \in \real^{\din\times T}$ to
% \begin{align*}
% &\attn(\mH;\mQ,\mK,\mV,\mO) \\
% =&\mO\mV\layernorm(\mH)\softmax(\mask(\layernorm(\mH)^\transpose\mK^\transpose\mQ\layernorm(\mH))).
% \end{align*}
% Furthermore, a multi-head attention with $M$ heads parameterized by $\{(\mQ_m,\mK_m,\mV_m,\mO_m) \}_{m=1}^M$ is defined as 
% \begin{align*}
%     &\Attn(\mH; \{(\mQ_m,\mK_m,\mV_m,\mO_m) \}_{m\in[M]}) \\ =& \sum_{m=1}^M \attn(\mH;\mQ_m,\mK_m,\mV_m,\mO_m) \in \real^{\dout \times T}.
% \end{align*}
% An MLP layer $\mlp(\mH;\mW_1,\mW_2)$ parameterized by $\mW_1 \in \real^{\dhidden\times \din}$ and $\mW_2 \in \real^{\dout \times \dhidden}$ maps an input matrix $\mH = [\vh_1, \ldots, \vh_T] \in \real^{\din \times T}$ to
% \begin{align*}
%     &\mlp(\mH;\mW_1,\mW_2) = [\vy_1, \ldots, \vy_T], \\ \text{where } &\vy_i = \mW_2\nonlin(\mW_1\layernorm(\vh_i)), \forall i \in [T].
% \end{align*}

% \end{definition}

% In this paper, we assume $\din=\dout=d$ for all attention heads and MLPs to facilitate residual stream unless otherwise specified. Given \Cref{defn:operators,defn:attn_mlp}, we are now able to define a multi-layer transformer.

% \begin{definition}[Multi-layer transformers]
% \label{defn:transformer}
%     An $L$-layer transformer $\transformer(\cdot): \vocab^T \to \Delta(\vocab)$ parameterized by $\mP$, $\mU$, $\{(\mQ_m^{(l)},\mK_m^{(l)},\mV_m^{(l)},\mO_m^{(l)})\}_{m\in[M],l\in[L]}$,  $\{(\mW_1^{(l)},\mW_2^{(l)})\}_{l\in[L]}$ and $\Wreadout \in \real^{V \times d}$ receives a sequence of tokens $s_{1:T}$ as input and predict the next token by outputting a distribution over the vocabulary. The input is first mapped to embeddings $\mH = [\vh_1, \vh_2, \ldots, \vh_T] \in \real^{d\times T}$ by embedding matrices $\mP, \mU$ where 
%     \[
%     \vh_i = \mU^\transpose\ve_{s_i} + \mP^\transpose\ve_{i}, \forall i \in [T].
%     \]
%     For each layer $l \in [L]$, the output of layer $l$, $\mH^{(l)} \in \real^{d\times T}$, is obtained by 
%     \begin{align*}
%         &\mH^{(l)} =  \mH^{(l-1/2)} + \mlp(\mH^{(l-1/2)};\mW_1^{(l)},\mW_2^{(l)}), \\
%         & \mH^{(l-1/2)} = \mH^{(l-1)} + \\ & \quad \Attn(\mH^{(l-1)}; \{(\mQ_m^{(l)},\mK_m^{(l)},\mV_m^{(l)},\mO_m^{(l)}) \}_{m\in[M]}), 
%     \end{align*}
%     where the input $\mH^{(l-1)}$ is the output of the previous layer $l-1$ for $l > 1$ and the input of the first layer $\mH^{(0)} = \mH$. Finally, the output of the transformer is obtained by 
%     \begin{align*}
%         \transformer(s_{1:T}) = \softmax(\Wreadout\vh_T^{(L)})
%     \end{align*}
%     which is a $V$-dimensional vector after softmax representing a distribution over $\vocab$, and $\vh_T^{(L)}$ is the $T$-th column of the output of the last layer, $\mH^{(L)}$.
% \end{definition}



% For each token $v \in \vocab$, there is a corresponding $d_t$-dimensional token embedding vector $\embed(v) \in \mathbb{R}^{d_t}$. Assume the maximum length of the sequence studied in this paper does not exceed $T$. For each position $t \in [T]$, there is a corresponding positional embedding  







%%%%%%%% ICML 2025 EXAMPLE LATEX SUBMISSION FILE %%%%%%%%%%%%%%%%%
\pdfoutput=1 
\documentclass{article}

% Recommended, but optional, packages for figures and better typesetting:
\usepackage{microtype}
\usepackage{graphicx}
\usepackage{subfigure}
\usepackage{booktabs} % for professional tables

% hyperref makes hyperlinks in the resulting PDF.
% If your build breaks (sometimes temporarily if a hyperlink spans a page)
% please comment out the following usepackage line and replace
% \usepackage{icml2025} with \usepackage[nohyperref]{icml2025} above.
\usepackage{hyperref}


% Attempt to make hyperref and algorithmic work together better:
\newcommand{\theHalgorithm}{\arabic{algorithm}}

% Use the following line for the initial blind version submitted for review:
\usepackage[accepted]{icml2025}

% If accepted, instead use the following line for the camera-ready submission:
%\usepackage[accepted]{icml2025}

% For theorems and such
\usepackage{amsmath}
\usepackage{amssymb}
\usepackage{mathtools}
\usepackage{amsthm}
\usepackage{dsfont}
\usepackage[most]{tcolorbox}

% if you use cleveref..
\usepackage[capitalize,noabbrev]{cleveref}

\definecolor{blue}{HTML}{4073b5} 

%%%%%%%%%%%%%%%%%%%%%%%%%%%%%%%%
% THEOREMS
%%%%%%%%%%%%%%%%%%%%%%%%%%%%%%%%
\theoremstyle{plain}
\newtheorem{theorem}{Theorem}[section]
\newtheorem{proposition}[theorem]{Proposition}
\newtheorem{lemma}[theorem]{Lemma}
\newtheorem{corollary}[theorem]{Corollary}
\theoremstyle{definition}
\newtheorem{definition}[theorem]{Definition}
\newtheorem{assumption}[theorem]{Assumption}
\theoremstyle{remark}
\newtheorem{remark}[theorem]{Remark}

\newcommand{\leo}[2][]{\todo[inline, color=blue!20, #1]{Leo: #2}}
\newcommand{\tom}[2][]{\todo[inline, color=red!20, #1]{Tom: #2}}
\newcommand{\yan}[2][]{\todo[inline, color=orange!20, #1]{Yan: #2}}
\newcommand{\sophie}[2][]{\todo[inline, color=green!20, #1]{Sophie: #2}}
\newcommand{\gauthier}[2][]{\todo[inline, color=purple!20, #1]{Gauthier: #2}}
% ---------- CUSTOM
\usepackage{enumitem}
\usepackage{xcolor}
\usepackage{colortbl}
\definecolor{lightgray}{gray}{0.85}
\definecolor{darkgray}{gray}{0.7}
%\definecolor{customblue}{rgb}{0.8, 0.9, 1}
\definecolor{customblue}{rgb}{0.27, 0.36, 0.55}

% Todonotes is useful during development; simply uncomment the next line
%    and comment out the line below the next line to turn off comments
\usepackage[disable,textsize=tiny]{todonotes}
\usepackage{array}
\usepackage{makecell}
%\usepackage[textsize=tiny]{todonotes}

%%%%% NEW MATH DEFINITIONS %%%%%

\usepackage{amsmath,amsfonts,bm}
\usepackage{derivative}
% Mark sections of captions for referring to divisions of figures
\newcommand{\figleft}{{\em (Left)}}
\newcommand{\figcenter}{{\em (Center)}}
\newcommand{\figright}{{\em (Right)}}
\newcommand{\figtop}{{\em (Top)}}
\newcommand{\figbottom}{{\em (Bottom)}}
\newcommand{\captiona}{{\em (a)}}
\newcommand{\captionb}{{\em (b)}}
\newcommand{\captionc}{{\em (c)}}
\newcommand{\captiond}{{\em (d)}}

% Highlight a newly defined term
\newcommand{\newterm}[1]{{\bf #1}}

% Derivative d 
\newcommand{\deriv}{{\mathrm{d}}}

% Figure reference, lower-case.
\def\figref#1{figure~\ref{#1}}
% Figure reference, capital. For start of sentence
\def\Figref#1{Figure~\ref{#1}}
\def\twofigref#1#2{figures \ref{#1} and \ref{#2}}
\def\quadfigref#1#2#3#4{figures \ref{#1}, \ref{#2}, \ref{#3} and \ref{#4}}
% Section reference, lower-case.
\def\secref#1{section~\ref{#1}}
% Section reference, capital.
\def\Secref#1{Section~\ref{#1}}
% Reference to two sections.
\def\twosecrefs#1#2{sections \ref{#1} and \ref{#2}}
% Reference to three sections.
\def\secrefs#1#2#3{sections \ref{#1}, \ref{#2} and \ref{#3}}
% Reference to an equation, lower-case.
\def\eqref#1{equation~\ref{#1}}
% Reference to an equation, upper case
\def\Eqref#1{Equation~\ref{#1}}
% A raw reference to an equation---avoid using if possible
\def\plaineqref#1{\ref{#1}}
% Reference to a chapter, lower-case.
\def\chapref#1{chapter~\ref{#1}}
% Reference to an equation, upper case.
\def\Chapref#1{Chapter~\ref{#1}}
% Reference to a range of chapters
\def\rangechapref#1#2{chapters\ref{#1}--\ref{#2}}
% Reference to an algorithm, lower-case.
\def\algref#1{algorithm~\ref{#1}}
% Reference to an algorithm, upper case.
\def\Algref#1{Algorithm~\ref{#1}}
\def\twoalgref#1#2{algorithms \ref{#1} and \ref{#2}}
\def\Twoalgref#1#2{Algorithms \ref{#1} and \ref{#2}}
% Reference to a part, lower case
\def\partref#1{part~\ref{#1}}
% Reference to a part, upper case
\def\Partref#1{Part~\ref{#1}}
\def\twopartref#1#2{parts \ref{#1} and \ref{#2}}

\def\ceil#1{\lceil #1 \rceil}
\def\floor#1{\lfloor #1 \rfloor}
\def\1{\bm{1}}
\newcommand{\train}{\mathcal{D}}
\newcommand{\valid}{\mathcal{D_{\mathrm{valid}}}}
\newcommand{\test}{\mathcal{D_{\mathrm{test}}}}

\def\eps{{\epsilon}}


% Random variables
\def\reta{{\textnormal{$\eta$}}}
\def\ra{{\textnormal{a}}}
\def\rb{{\textnormal{b}}}
\def\rc{{\textnormal{c}}}
\def\rd{{\textnormal{d}}}
\def\re{{\textnormal{e}}}
\def\rf{{\textnormal{f}}}
\def\rg{{\textnormal{g}}}
\def\rh{{\textnormal{h}}}
\def\ri{{\textnormal{i}}}
\def\rj{{\textnormal{j}}}
\def\rk{{\textnormal{k}}}
\def\rl{{\textnormal{l}}}
% rm is already a command, just don't name any random variables m
\def\rn{{\textnormal{n}}}
\def\ro{{\textnormal{o}}}
\def\rp{{\textnormal{p}}}
\def\rq{{\textnormal{q}}}
\def\rr{{\textnormal{r}}}
\def\rs{{\textnormal{s}}}
\def\rt{{\textnormal{t}}}
\def\ru{{\textnormal{u}}}
\def\rv{{\textnormal{v}}}
\def\rw{{\textnormal{w}}}
\def\rx{{\textnormal{x}}}
\def\ry{{\textnormal{y}}}
\def\rz{{\textnormal{z}}}

% Random vectors
\def\rvepsilon{{\mathbf{\epsilon}}}
\def\rvphi{{\mathbf{\phi}}}
\def\rvtheta{{\mathbf{\theta}}}
\def\rva{{\mathbf{a}}}
\def\rvb{{\mathbf{b}}}
\def\rvc{{\mathbf{c}}}
\def\rvd{{\mathbf{d}}}
\def\rve{{\mathbf{e}}}
\def\rvf{{\mathbf{f}}}
\def\rvg{{\mathbf{g}}}
\def\rvh{{\mathbf{h}}}
\def\rvu{{\mathbf{i}}}
\def\rvj{{\mathbf{j}}}
\def\rvk{{\mathbf{k}}}
\def\rvl{{\mathbf{l}}}
\def\rvm{{\mathbf{m}}}
\def\rvn{{\mathbf{n}}}
\def\rvo{{\mathbf{o}}}
\def\rvp{{\mathbf{p}}}
\def\rvq{{\mathbf{q}}}
\def\rvr{{\mathbf{r}}}
\def\rvs{{\mathbf{s}}}
\def\rvt{{\mathbf{t}}}
\def\rvu{{\mathbf{u}}}
\def\rvv{{\mathbf{v}}}
\def\rvw{{\mathbf{w}}}
\def\rvx{{\mathbf{x}}}
\def\rvy{{\mathbf{y}}}
\def\rvz{{\mathbf{z}}}

% Elements of random vectors
\def\erva{{\textnormal{a}}}
\def\ervb{{\textnormal{b}}}
\def\ervc{{\textnormal{c}}}
\def\ervd{{\textnormal{d}}}
\def\erve{{\textnormal{e}}}
\def\ervf{{\textnormal{f}}}
\def\ervg{{\textnormal{g}}}
\def\ervh{{\textnormal{h}}}
\def\ervi{{\textnormal{i}}}
\def\ervj{{\textnormal{j}}}
\def\ervk{{\textnormal{k}}}
\def\ervl{{\textnormal{l}}}
\def\ervm{{\textnormal{m}}}
\def\ervn{{\textnormal{n}}}
\def\ervo{{\textnormal{o}}}
\def\ervp{{\textnormal{p}}}
\def\ervq{{\textnormal{q}}}
\def\ervr{{\textnormal{r}}}
\def\ervs{{\textnormal{s}}}
\def\ervt{{\textnormal{t}}}
\def\ervu{{\textnormal{u}}}
\def\ervv{{\textnormal{v}}}
\def\ervw{{\textnormal{w}}}
\def\ervx{{\textnormal{x}}}
\def\ervy{{\textnormal{y}}}
\def\ervz{{\textnormal{z}}}

% Random matrices
\def\rmA{{\mathbf{A}}}
\def\rmB{{\mathbf{B}}}
\def\rmC{{\mathbf{C}}}
\def\rmD{{\mathbf{D}}}
\def\rmE{{\mathbf{E}}}
\def\rmF{{\mathbf{F}}}
\def\rmG{{\mathbf{G}}}
\def\rmH{{\mathbf{H}}}
\def\rmI{{\mathbf{I}}}
\def\rmJ{{\mathbf{J}}}
\def\rmK{{\mathbf{K}}}
\def\rmL{{\mathbf{L}}}
\def\rmM{{\mathbf{M}}}
\def\rmN{{\mathbf{N}}}
\def\rmO{{\mathbf{O}}}
\def\rmP{{\mathbf{P}}}
\def\rmQ{{\mathbf{Q}}}
\def\rmR{{\mathbf{R}}}
\def\rmS{{\mathbf{S}}}
\def\rmT{{\mathbf{T}}}
\def\rmU{{\mathbf{U}}}
\def\rmV{{\mathbf{V}}}
\def\rmW{{\mathbf{W}}}
\def\rmX{{\mathbf{X}}}
\def\rmY{{\mathbf{Y}}}
\def\rmZ{{\mathbf{Z}}}

% Elements of random matrices
\def\ermA{{\textnormal{A}}}
\def\ermB{{\textnormal{B}}}
\def\ermC{{\textnormal{C}}}
\def\ermD{{\textnormal{D}}}
\def\ermE{{\textnormal{E}}}
\def\ermF{{\textnormal{F}}}
\def\ermG{{\textnormal{G}}}
\def\ermH{{\textnormal{H}}}
\def\ermI{{\textnormal{I}}}
\def\ermJ{{\textnormal{J}}}
\def\ermK{{\textnormal{K}}}
\def\ermL{{\textnormal{L}}}
\def\ermM{{\textnormal{M}}}
\def\ermN{{\textnormal{N}}}
\def\ermO{{\textnormal{O}}}
\def\ermP{{\textnormal{P}}}
\def\ermQ{{\textnormal{Q}}}
\def\ermR{{\textnormal{R}}}
\def\ermS{{\textnormal{S}}}
\def\ermT{{\textnormal{T}}}
\def\ermU{{\textnormal{U}}}
\def\ermV{{\textnormal{V}}}
\def\ermW{{\textnormal{W}}}
\def\ermX{{\textnormal{X}}}
\def\ermY{{\textnormal{Y}}}
\def\ermZ{{\textnormal{Z}}}

% Vectors
\def\vzero{{\bm{0}}}
\def\vone{{\bm{1}}}
\def\vmu{{\bm{\mu}}}
\def\vtheta{{\bm{\theta}}}
\def\vphi{{\bm{\phi}}}
\def\va{{\bm{a}}}
\def\vb{{\bm{b}}}
\def\vc{{\bm{c}}}
\def\vd{{\bm{d}}}
\def\ve{{\bm{e}}}
\def\vf{{\bm{f}}}
\def\vg{{\bm{g}}}
\def\vh{{\bm{h}}}
\def\vi{{\bm{i}}}
\def\vj{{\bm{j}}}
\def\vk{{\bm{k}}}
\def\vl{{\bm{l}}}
\def\vm{{\bm{m}}}
\def\vn{{\bm{n}}}
\def\vo{{\bm{o}}}
\def\vp{{\bm{p}}}
\def\vq{{\bm{q}}}
\def\vr{{\bm{r}}}
\def\vs{{\bm{s}}}
\def\vt{{\bm{t}}}
\def\vu{{\bm{u}}}
\def\vv{{\bm{v}}}
\def\vw{{\bm{w}}}
\def\vx{{\bm{x}}}
\def\vy{{\bm{y}}}
\def\vz{{\bm{z}}}

% Elements of vectors
\def\evalpha{{\alpha}}
\def\evbeta{{\beta}}
\def\evepsilon{{\epsilon}}
\def\evlambda{{\lambda}}
\def\evomega{{\omega}}
\def\evmu{{\mu}}
\def\evpsi{{\psi}}
\def\evsigma{{\sigma}}
\def\evtheta{{\theta}}
\def\eva{{a}}
\def\evb{{b}}
\def\evc{{c}}
\def\evd{{d}}
\def\eve{{e}}
\def\evf{{f}}
\def\evg{{g}}
\def\evh{{h}}
\def\evi{{i}}
\def\evj{{j}}
\def\evk{{k}}
\def\evl{{l}}
\def\evm{{m}}
\def\evn{{n}}
\def\evo{{o}}
\def\evp{{p}}
\def\evq{{q}}
\def\evr{{r}}
\def\evs{{s}}
\def\evt{{t}}
\def\evu{{u}}
\def\evv{{v}}
\def\evw{{w}}
\def\evx{{x}}
\def\evy{{y}}
\def\evz{{z}}

% Matrix
\def\mA{{\bm{A}}}
\def\mB{{\bm{B}}}
\def\mC{{\bm{C}}}
\def\mD{{\bm{D}}}
\def\mE{{\bm{E}}}
\def\mF{{\bm{F}}}
\def\mG{{\bm{G}}}
\def\mH{{\bm{H}}}
\def\mI{{\bm{I}}}
\def\mJ{{\bm{J}}}
\def\mK{{\bm{K}}}
\def\mL{{\bm{L}}}
\def\mM{{\bm{M}}}
\def\mN{{\bm{N}}}
\def\mO{{\bm{O}}}
\def\mP{{\bm{P}}}
\def\mQ{{\bm{Q}}}
\def\mR{{\bm{R}}}
\def\mS{{\bm{S}}}
\def\mT{{\bm{T}}}
\def\mU{{\bm{U}}}
\def\mV{{\bm{V}}}
\def\mW{{\bm{W}}}
\def\mX{{\bm{X}}}
\def\mY{{\bm{Y}}}
\def\mZ{{\bm{Z}}}
\def\mBeta{{\bm{\beta}}}
\def\mPhi{{\bm{\Phi}}}
\def\mLambda{{\bm{\Lambda}}}
\def\mSigma{{\bm{\Sigma}}}

% Tensor
\DeclareMathAlphabet{\mathsfit}{\encodingdefault}{\sfdefault}{m}{sl}
\SetMathAlphabet{\mathsfit}{bold}{\encodingdefault}{\sfdefault}{bx}{n}
\newcommand{\tens}[1]{\bm{\mathsfit{#1}}}
\def\tA{{\tens{A}}}
\def\tB{{\tens{B}}}
\def\tC{{\tens{C}}}
\def\tD{{\tens{D}}}
\def\tE{{\tens{E}}}
\def\tF{{\tens{F}}}
\def\tG{{\tens{G}}}
\def\tH{{\tens{H}}}
\def\tI{{\tens{I}}}
\def\tJ{{\tens{J}}}
\def\tK{{\tens{K}}}
\def\tL{{\tens{L}}}
\def\tM{{\tens{M}}}
\def\tN{{\tens{N}}}
\def\tO{{\tens{O}}}
\def\tP{{\tens{P}}}
\def\tQ{{\tens{Q}}}
\def\tR{{\tens{R}}}
\def\tS{{\tens{S}}}
\def\tT{{\tens{T}}}
\def\tU{{\tens{U}}}
\def\tV{{\tens{V}}}
\def\tW{{\tens{W}}}
\def\tX{{\tens{X}}}
\def\tY{{\tens{Y}}}
\def\tZ{{\tens{Z}}}


% Graph
\def\gA{{\mathcal{A}}}
\def\gB{{\mathcal{B}}}
\def\gC{{\mathcal{C}}}
\def\gD{{\mathcal{D}}}
\def\gE{{\mathcal{E}}}
\def\gF{{\mathcal{F}}}
\def\gG{{\mathcal{G}}}
\def\gH{{\mathcal{H}}}
\def\gI{{\mathcal{I}}}
\def\gJ{{\mathcal{J}}}
\def\gK{{\mathcal{K}}}
\def\gL{{\mathcal{L}}}
\def\gM{{\mathcal{M}}}
\def\gN{{\mathcal{N}}}
\def\gO{{\mathcal{O}}}
\def\gP{{\mathcal{P}}}
\def\gQ{{\mathcal{Q}}}
\def\gR{{\mathcal{R}}}
\def\gS{{\mathcal{S}}}
\def\gT{{\mathcal{T}}}
\def\gU{{\mathcal{U}}}
\def\gV{{\mathcal{V}}}
\def\gW{{\mathcal{W}}}
\def\gX{{\mathcal{X}}}
\def\gY{{\mathcal{Y}}}
\def\gZ{{\mathcal{Z}}}

% Sets
\def\sA{{\mathbb{A}}}
\def\sB{{\mathbb{B}}}
\def\sC{{\mathbb{C}}}
\def\sD{{\mathbb{D}}}
% Don't use a set called E, because this would be the same as our symbol
% for expectation.
\def\sF{{\mathbb{F}}}
\def\sG{{\mathbb{G}}}
\def\sH{{\mathbb{H}}}
\def\sI{{\mathbb{I}}}
\def\sJ{{\mathbb{J}}}
\def\sK{{\mathbb{K}}}
\def\sL{{\mathbb{L}}}
\def\sM{{\mathbb{M}}}
\def\sN{{\mathbb{N}}}
\def\sO{{\mathbb{O}}}
\def\sP{{\mathbb{P}}}
\def\sQ{{\mathbb{Q}}}
\def\sR{{\mathbb{R}}}
\def\sS{{\mathbb{S}}}
\def\sT{{\mathbb{T}}}
\def\sU{{\mathbb{U}}}
\def\sV{{\mathbb{V}}}
\def\sW{{\mathbb{W}}}
\def\sX{{\mathbb{X}}}
\def\sY{{\mathbb{Y}}}
\def\sZ{{\mathbb{Z}}}

% Entries of a matrix
\def\emLambda{{\Lambda}}
\def\emA{{A}}
\def\emB{{B}}
\def\emC{{C}}
\def\emD{{D}}
\def\emE{{E}}
\def\emF{{F}}
\def\emG{{G}}
\def\emH{{H}}
\def\emI{{I}}
\def\emJ{{J}}
\def\emK{{K}}
\def\emL{{L}}
\def\emM{{M}}
\def\emN{{N}}
\def\emO{{O}}
\def\emP{{P}}
\def\emQ{{Q}}
\def\emR{{R}}
\def\emS{{S}}
\def\emT{{T}}
\def\emU{{U}}
\def\emV{{V}}
\def\emW{{W}}
\def\emX{{X}}
\def\emY{{Y}}
\def\emZ{{Z}}
\def\emSigma{{\Sigma}}

% entries of a tensor
% Same font as tensor, without \bm wrapper
\newcommand{\etens}[1]{\mathsfit{#1}}
\def\etLambda{{\etens{\Lambda}}}
\def\etA{{\etens{A}}}
\def\etB{{\etens{B}}}
\def\etC{{\etens{C}}}
\def\etD{{\etens{D}}}
\def\etE{{\etens{E}}}
\def\etF{{\etens{F}}}
\def\etG{{\etens{G}}}
\def\etH{{\etens{H}}}
\def\etI{{\etens{I}}}
\def\etJ{{\etens{J}}}
\def\etK{{\etens{K}}}
\def\etL{{\etens{L}}}
\def\etM{{\etens{M}}}
\def\etN{{\etens{N}}}
\def\etO{{\etens{O}}}
\def\etP{{\etens{P}}}
\def\etQ{{\etens{Q}}}
\def\etR{{\etens{R}}}
\def\etS{{\etens{S}}}
\def\etT{{\etens{T}}}
\def\etU{{\etens{U}}}
\def\etV{{\etens{V}}}
\def\etW{{\etens{W}}}
\def\etX{{\etens{X}}}
\def\etY{{\etens{Y}}}
\def\etZ{{\etens{Z}}}

% The true underlying data generating distribution
\newcommand{\pdata}{p_{\rm{data}}}
\newcommand{\ptarget}{p_{\rm{target}}}
\newcommand{\pprior}{p_{\rm{prior}}}
\newcommand{\pbase}{p_{\rm{base}}}
\newcommand{\pref}{p_{\rm{ref}}}

% The empirical distribution defined by the training set
\newcommand{\ptrain}{\hat{p}_{\rm{data}}}
\newcommand{\Ptrain}{\hat{P}_{\rm{data}}}
% The model distribution
\newcommand{\pmodel}{p_{\rm{model}}}
\newcommand{\Pmodel}{P_{\rm{model}}}
\newcommand{\ptildemodel}{\tilde{p}_{\rm{model}}}
% Stochastic autoencoder distributions
\newcommand{\pencode}{p_{\rm{encoder}}}
\newcommand{\pdecode}{p_{\rm{decoder}}}
\newcommand{\precons}{p_{\rm{reconstruct}}}

\newcommand{\laplace}{\mathrm{Laplace}} % Laplace distribution

\newcommand{\E}{\mathbb{E}}
\newcommand{\Ls}{\mathcal{L}}
\newcommand{\R}{\mathbb{R}}
\newcommand{\emp}{\tilde{p}}
\newcommand{\lr}{\alpha}
\newcommand{\reg}{\lambda}
\newcommand{\rect}{\mathrm{rectifier}}
\newcommand{\softmax}{\mathrm{softmax}}
\newcommand{\sigmoid}{\sigma}
\newcommand{\softplus}{\zeta}
\newcommand{\KL}{D_{\mathrm{KL}}}
\newcommand{\Var}{\mathrm{Var}}
\newcommand{\standarderror}{\mathrm{SE}}
\newcommand{\Cov}{\mathrm{Cov}}
% Wolfram Mathworld says $L^2$ is for function spaces and $\ell^2$ is for vectors
% But then they seem to use $L^2$ for vectors throughout the site, and so does
% wikipedia.
\newcommand{\normlzero}{L^0}
\newcommand{\normlone}{L^1}
\newcommand{\normltwo}{L^2}
\newcommand{\normlp}{L^p}
\newcommand{\normmax}{L^\infty}

\newcommand{\parents}{Pa} % See usage in notation.tex. Chosen to match Daphne's book.

\DeclareMathOperator*{\argmax}{arg\,max}
\DeclareMathOperator*{\argmin}{arg\,min}

\DeclareMathOperator{\sign}{sign}
\DeclareMathOperator{\Tr}{Tr}
\let\ab\allowbreak


\newlength{\myboxwidth}
\setlength{\myboxwidth}{7cm}
\newcommand{\myfont}{
    %\large       % Font size
    \ttfamily    % Typewriter (monospace) font
}

% The \icmltitle you define below is probably too long as a header.
% Therefore, a short form for the running title is supplied here:
\icmltitlerunning{Toward Meaningful Progress in Adversarial Alignment for LLMs}

\begin{document}

\twocolumn[
%\icmltitle{\texorpdfstring{Position: We Need to Realign Incentives for Meaningful Progress\\in Adversarial Alignment for LLMs}{Position: We Need to Realign Incentives for Meaningful Progress in Adversarial Alignment for LLMs}}

\icmltitle{\texorpdfstring{Adversarial Alignment for LLMs Requires\\ Simpler, Reproducible, and More Measurable Objectives}{Adversarial Alignment for LLMs Requires Simpler, Reproducible, and More Measurable Objectives}}


% It is OKAY to include author information, even for blind
% submissions: the style file will automatically remove it for you
% unless you've provided the [accepted] option to the icml2025
% package.

% List of affiliations: The first argument should be a (short)
% identifier you will use later to specify author affiliations
% Academic affiliations should list Department, University, City, Region, Country
% Industry affiliations should list Company, City, Region, Country

% You can specify symbols, otherwise they are numbered in order.
% Ideally, you should not use this facility. Affiliations will be numbered
% in order of appearance and this is the preferred way.
\icmlsetsymbol{equal}{*}

% \author{Leo Schwinn \& David Dobre \& Stephan Günnemann \& Gauthier Gidel}
\begin{icmlauthorlist}
\icmlauthor{Leo Schwinn}{tum}
\icmlauthor{Yan Scholten}{tum}
\icmlauthor{Tom Wollschl\"ager}{tum}
\icmlauthor{Sophie Xhonneux}{mila,montreal}
\icmlauthor{Stephen Casper}{mit}
\icmlauthor{Stephan G\"unnemann}{tum}
\icmlauthor{Gauthier Gidel}{mila,montreal,cifar}
\end{icmlauthorlist}

\icmlaffiliation{tum}{Technical University of Munich}
\icmlaffiliation{mila}{Mila}
\icmlaffiliation{montreal}{Universit\'{e} de Montr\'{e}al}
\icmlaffiliation{cifar}{Canada AI CIFAR Chair}
\icmlaffiliation{mit}{MIT}
%\icmlaffiliation{scholar}{ML Alignment \& Theory Scholars}

\icmlcorrespondingauthor{}{adversarial-alignment-position@googlegroups.com}
%\icmlcorrespondingauthor{Firstname2 Lastname2}{first2.last2@www.uk}

% You may provide any keywords that you
% find helpful for describing your paper; these are used to populate
% the "keywords" metadata in the PDF but will not be shown in the document
\icmlkeywords{Machine Learning, ICML}

\vskip 0.3in
]

% this must go after the closing bracket ] following \twocolumn[ ...

% This command actually creates the footnote in the first column
% listing the affiliations and the copyright notice.
% The command takes one argument, which is text to display at the start of the footnote.
% The \icmlEqualContribution command is standard text for equal contribution.
% Remove it (just {}) if you do not need this facility.

%\printAffiliationsAndNotice{}  % leave blank if no need to mention equal contribution
\printAffiliationsAndNotice{} % otherwise use the standard text.


% ========== What does realigned incentives mean? ======

% --> Not trying to solve the full problem at once (e.g., jailbreaking alignment, which includes a full pipeline of steps, vs constructing better optimization algorithms) --> Currently ppl do that because jailbreaking models is taken as the most relevant progress
% Not focussing on direct progress but thorough evaluations, principled ideas, new insights 
% Valuing academic principles, such as reproducibility, comparability over apparent real-world relevance (e.g., performing attacks on the newest proprietary model)
% Adjusting the role that academia plays in this context in combination with industry. --> large scale experiments are predominantly performed by industry/big research labs. Capability advancements are done by industry, Academia played an important role in the past in proof of concepting new ideas / grassroot research 


% Maybe incentives will realign automatically because industry actually wants to solve the problem

% If you reduce complexity you also loose real world comparably



\begin{abstract}
Misaligned research objectives have considerably hindered progress in adversarial robustness research over the past decade.
For instance, an extensive focus on optimizing target metrics, while neglecting rigorous standardized evaluation, has led researchers to pursue ad-hoc heuristic defenses that were seemingly effective. Yet, most of these were exposed as flawed by subsequent evaluations, ultimately contributing little measurable progress to the field.
In this position paper, we illustrate that current research on the robustness of large language models (LLMs) risks repeating past patterns with potentially worsened real-world implications. To address this, we argue that \textbf{realigned objectives are necessary for meaningful progress in adversarial alignment.}
To this end, we build on established cybersecurity taxonomy to formally define differences between past and emerging threat models that apply to LLMs.
Using this framework, we illustrate that progress requires disentangling adversarial alignment into addressable sub-problems and returning to core academic principles, such as measureability, reproducibility, and comparability.
Although the field presents significant challenges, the fresh start on adversarial robustness offers the unique opportunity to build on past experience while avoiding previous mistakes.

    %Building on established cybersecurity taxonomy, we formally define differences between past and emerging threat models that apply to LLMs. Based on this analysis, we derive new research incentives based on
    %Based on this framework we analyze the interplay between academia and industry. We posit that academia can maximize its impact by concentrating on grassroot research pursuing unique ideas and generating insights. 

    
    %Based on this framework, we argue that progress requires redefining the role of academia in an industry-driven field, simplifying problem complexity, and returning to core academic principles, such as reproducibility, comparability, and measureability.
    % alternative
    %Based on this analysis, we discuss how a redefined role for academia and industry, a reduction in problem complexity, clearer definitions, and a renewed focus on foundational academic principles can help to remove current research impediments.
    
    



    %Over the past decade, extensive research has been aimed at enhancing the adversarial robustness of neural networks, yet this problem remains vastly unsolved. We argue that this lack of progress can be largely attributed to misaligned incentives within the research community. %Leo: Merge
    %, where the (1) pursuit of higher evaluation metrics through ad-hoc approaches resulted in flawed evaluations and ultimately yielded little measurable progress. 
    % Merge:
    %Current research on adversarial alignment in large language models (LLMs) risks repeating past patterns, however, with worsened real-world implications. Although it presents significant challenges, the field’s fresh start provides an opportunity to start over and avoid repeating past mistakes. Hence, in this position paper, we argue that realigned incentives are necessary for meaningful progress in adversarial alignment. %based on the history of adversarial robustness research and emerging trends,
% what do we do
%Specifically, we reflect on past issues, identify repeating patterns, explore upcoming changes due to the increased problem complexity, and discuss potential avenues for realigning incentives to foster measurable progress. 
%In this context, we discuss how the role of academia in this field and how its contributions can be facilitated, the need to address overly complex problem definitions, and the challenges posed by reproducibility and comparability in the context of proprietary models and increasing computational complexity.
% forward looking sentence
\end{abstract}

% complexity

% structure into security sepcs (access to model, goal of the attack, ...), past settings, past problems, past solutions, new setting, repeating problems, new problems, potential solutions (incentives criteria: measurable, reprodubible, compararable)
% put role of industry, reproducibility, comparability,  at the end



\section{Introduction}


\begin{figure}[t]
\centering
\includegraphics[width=0.6\columnwidth]{figures/evaluation_desiderata_V5.pdf}
\vspace{-0.5cm}
\caption{\systemName is a platform for conducting realistic evaluations of code LLMs, collecting human preferences of coding models with real users, real tasks, and in realistic environments, aimed at addressing the limitations of existing evaluations.
}
\label{fig:motivation}
\end{figure}

\begin{figure*}[t]
\centering
\includegraphics[width=\textwidth]{figures/system_design_v2.png}
\caption{We introduce \systemName, a VSCode extension to collect human preferences of code directly in a developer's IDE. \systemName enables developers to use code completions from various models. The system comprises a) the interface in the user's IDE which presents paired completions to users (left), b) a sampling strategy that picks model pairs to reduce latency (right, top), and c) a prompting scheme that allows diverse LLMs to perform code completions with high fidelity.
Users can select between the top completion (green box) using \texttt{tab} or the bottom completion (blue box) using \texttt{shift+tab}.}
\label{fig:overview}
\end{figure*}

As model capabilities improve, large language models (LLMs) are increasingly integrated into user environments and workflows.
For example, software developers code with AI in integrated developer environments (IDEs)~\citep{peng2023impact}, doctors rely on notes generated through ambient listening~\citep{oberst2024science}, and lawyers consider case evidence identified by electronic discovery systems~\citep{yang2024beyond}.
Increasing deployment of models in productivity tools demands evaluation that more closely reflects real-world circumstances~\citep{hutchinson2022evaluation, saxon2024benchmarks, kapoor2024ai}.
While newer benchmarks and live platforms incorporate human feedback to capture real-world usage, they almost exclusively focus on evaluating LLMs in chat conversations~\citep{zheng2023judging,dubois2023alpacafarm,chiang2024chatbot, kirk2024the}.
Model evaluation must move beyond chat-based interactions and into specialized user environments.



 

In this work, we focus on evaluating LLM-based coding assistants. 
Despite the popularity of these tools---millions of developers use Github Copilot~\citep{Copilot}---existing
evaluations of the coding capabilities of new models exhibit multiple limitations (Figure~\ref{fig:motivation}, bottom).
Traditional ML benchmarks evaluate LLM capabilities by measuring how well a model can complete static, interview-style coding tasks~\citep{chen2021evaluating,austin2021program,jain2024livecodebench, white2024livebench} and lack \emph{real users}. 
User studies recruit real users to evaluate the effectiveness of LLMs as coding assistants, but are often limited to simple programming tasks as opposed to \emph{real tasks}~\citep{vaithilingam2022expectation,ross2023programmer, mozannar2024realhumaneval}.
Recent efforts to collect human feedback such as Chatbot Arena~\citep{chiang2024chatbot} are still removed from a \emph{realistic environment}, resulting in users and data that deviate from typical software development processes.
We introduce \systemName to address these limitations (Figure~\ref{fig:motivation}, top), and we describe our three main contributions below.


\textbf{We deploy \systemName in-the-wild to collect human preferences on code.} 
\systemName is a Visual Studio Code extension, collecting preferences directly in a developer's IDE within their actual workflow (Figure~\ref{fig:overview}).
\systemName provides developers with code completions, akin to the type of support provided by Github Copilot~\citep{Copilot}. 
Over the past 3 months, \systemName has served over~\completions suggestions from 10 state-of-the-art LLMs, 
gathering \sampleCount~votes from \userCount~users.
To collect user preferences,
\systemName presents a novel interface that shows users paired code completions from two different LLMs, which are determined based on a sampling strategy that aims to 
mitigate latency while preserving coverage across model comparisons.
Additionally, we devise a prompting scheme that allows a diverse set of models to perform code completions with high fidelity.
See Section~\ref{sec:system} and Section~\ref{sec:deployment} for details about system design and deployment respectively.



\textbf{We construct a leaderboard of user preferences and find notable differences from existing static benchmarks and human preference leaderboards.}
In general, we observe that smaller models seem to overperform in static benchmarks compared to our leaderboard, while performance among larger models is mixed (Section~\ref{sec:leaderboard_calculation}).
We attribute these differences to the fact that \systemName is exposed to users and tasks that differ drastically from code evaluations in the past. 
Our data spans 103 programming languages and 24 natural languages as well as a variety of real-world applications and code structures, while static benchmarks tend to focus on a specific programming and natural language and task (e.g. coding competition problems).
Additionally, while all of \systemName interactions contain code contexts and the majority involve infilling tasks, a much smaller fraction of Chatbot Arena's coding tasks contain code context, with infilling tasks appearing even more rarely. 
We analyze our data in depth in Section~\ref{subsec:comparison}.



\textbf{We derive new insights into user preferences of code by analyzing \systemName's diverse and distinct data distribution.}
We compare user preferences across different stratifications of input data (e.g., common versus rare languages) and observe which affect observed preferences most (Section~\ref{sec:analysis}).
For example, while user preferences stay relatively consistent across various programming languages, they differ drastically between different task categories (e.g. frontend/backend versus algorithm design).
We also observe variations in user preference due to different features related to code structure 
(e.g., context length and completion patterns).
We open-source \systemName and release a curated subset of code contexts.
Altogether, our results highlight the necessity of model evaluation in realistic and domain-specific settings.





% \section{Taxonomy}

% As illustrated by Fig. \ref{}, the typical process of vision models based time series analysis has five components: (1) normalization/scaling; (2) time series to image transformation; (3) image modeling; (4) image to time series recovery; and (5) task processing. In the rest of this paper, we will discuss the typical methods for each of these components. The detailed taxonomy of the methods are summarized in Table \ref{tab.taxonomy}.

%Typical step: normalization/scaling, transformation, vision modeling, task-specific head, inverse transformation (for tasks that output time series, e.g., forecasting, generation, imputation, anomaly detection). Normalization is to fit the arbitrary range of time series values to RGB representation.

\begin{figure*}[!t]
\centering
\includegraphics[width=1.0\textwidth]{fig/fig_3.pdf}
% \vspace{-1em}
\caption{An illustration of different methods for imaging time series with a sample (length=336) from the \textit{Electricity} benchmark dataset \protect\cite{nie2023time}. (a)(c)(d)(e)(f) %are univariate methods.
visualize the same variate. (b) visualizes all 321 variates. Filterbank is omitted due to its %high
similarity to STFT.}\label{fig.tsimage}
\vspace{-0.2cm}
\end{figure*}

\begin{table*}[t]
\centering
\scriptsize
\setlength{\tabcolsep}{2.7pt}{
% \begin{tabular}{llllllllllll}
\begin{tabular}{llcccccccccl}
\toprule[1pt]
\multirow{2}{*}{Method} & \multirow{2}{*}{TS-Type} & \multirow{2}{*}{Imaging} & \multicolumn{5}{c}{Imaged Time Series Modeling} & \multirow{2}{*}{TS-Recover} & \multirow{2}{*}{Task} & \multirow{2}{*}{Domain} & \multirow{2}{*}{Code}\\ \cmidrule{4-8}
 & & & Multi-modal & Model & Pre-trained & Fine-tune & Prompt & & & & \\ \midrule
\cite{silva2013time} & UTS & RP & \xmark & \texttt{K-NN} & \xmark & \xmark & \xmark & \xmark & Classification & General & \xmark\\
\cite{wang2015encoding} & UTS & GAF & \xmark & \texttt{CNN} & \xmark & \cmark$^{\flat}$ & \xmark & \cmark & Classification & General & \xmark\\
\cite{wang2015imaging} & UTS & GAF & \xmark & \texttt{CNN} & \xmark & \cmark$^{\flat}$ & \xmark & \cmark & Multiple & General & \xmark\\
% \multirow{2}{*}{\cite{wang2015imaging}} & \multirow{2}{*}{UTS} & \multirow{2}{*}{GAF} & \multirow{2}{*}{\xmark} & \multirow{2}{*}{\texttt{CNN}} & \multirow{2}{*}{\xmark} & \multirow{2}{*}{\cmark$^{\flat}$} & \multirow{2}{*}{\xmark} & \multirow{2}{*}{\cmark} & Classification & \multirow{2}{*}{General} & \multirow{2}{*}{\xmark}\\
% & & & & & & & & & \& Imputation & & \\
\cite{ma2017learning} & MTS & Heatmap & \xmark & \texttt{CNN} & \xmark & \cmark$^{\flat}$ & \xmark & \cmark & Forecasting & Traffic & \xmark\\
\cite{hatami2018classification} & UTS & RP & \xmark & \texttt{CNN} & \xmark & \cmark$^{\flat}$ & \xmark & \xmark & Classification & General & \xmark\\
\cite{yazdanbakhsh2019multivariate} & MTS & Heatmap & \xmark & \texttt{CNN} & \xmark & \cmark$^{\flat}$ & \xmark & \xmark & Classification & General & \cmark\textsuperscript{\href{https://github.com/SonbolYb/multivariate_timeseries_dilated_conv}{[1]}}\\
MSCRED \cite{zhang2019deep} & MTS & Other ($\S$\ref{sec.othermethod}) & \xmark & \texttt{ConvLSTM} & \xmark & \cmark$^{\flat}$ & \xmark & \xmark & Anomaly & General & \cmark\textsuperscript{\href{https://github.com/7fantasysz/MSCRED}{[2]}}\\
\cite{li2020forecasting} & UTS & RP & \xmark & \texttt{CNN} & \cmark & \cmark & \xmark & \xmark & Forecasting & General & \cmark\textsuperscript{\href{https://github.com/lixixibj/forecasting-with-time-series-imaging}{[3]}}\\
\cite{cohen2020trading} & UTS & LinePlot & \xmark & \texttt{Ensemble} & \xmark & \cmark$^{\flat}$ & \xmark & \xmark & Classification & Finance & \xmark\\
% \cite{du2020image} & UTS & Spectrogram & \xmark & \texttt{CNN} & \xmark & \cmark$^{\flat}$ & \xmark & \xmark & Classification & Finance & \xmark\\
\cite{barra2020deep} & UTS & GAF & \xmark & \texttt{CNN} & \xmark & \cmark$^{\flat}$ & \xmark & \xmark & Classification & Finance & \xmark\\
% \cite{barra2020deep} & UTS & GAF & \xmark & \texttt{VGG-16} & \xmark & \cmark$^{\flat}$ & \xmark & \xmark & Classification & Finance & \xmark\\
% \cite{cao2021image} & UTS & RP & \xmark & \texttt{CNN} & \xmark & \cmark$^{\flat}$ & \xmark & \xmark & Classification & General & \xmark\\
VisualAE \cite{sood2021visual} & UTS & LinePlot & \xmark & \texttt{CNN} & \xmark & \cmark$^{\flat}$ & \xmark & \cmark & Forecasting & Finance & \xmark\\
% VisualAE \cite{sood2021visual} & UTS & LinePlot & \xmark & \texttt{CNN} & \xmark & \cmark$^{\flat}$ & \xmark & \xmark & Img-Generation & Finance & \xmark\\
\cite{zeng2021deep} & MTS & Heatmap & \xmark & \texttt{CNN,LSTM} & \xmark & \cmark$^{\flat}$ & \xmark & \cmark & Forecasting & Finance & \xmark\\
% \cite{zeng2021deep} & MTS & Heatmap & \xmark & \texttt{SRVP} & \xmark & \cmark$^{\flat}$ & \xmark & \cmark & Forecasting & Finance & \xmark\\
AST \cite{gong2021ast} & UTS & Spectrogram & \xmark & \texttt{DeiT} & \cmark & \cmark & \xmark & \xmark & Classification & Audio & \cmark\textsuperscript{\href{https://github.com/YuanGongND/ast}{[4]}}\\
TTS-GAN \cite{li2022tts} & MTS & Heatmap & \xmark & \texttt{ViT} & \xmark & \cmark$^{\flat}$ & \xmark & \cmark & Ts-Generation & Health & \cmark\textsuperscript{\href{https://github.com/imics-lab/tts-gan}{[5]}}\\
SSAST \cite{gong2022ssast} & UTS & Spectrogram & \xmark & \texttt{ViT} & \cmark$^{\natural}$ & \cmark & \xmark & \xmark & Classification & Audio & \cmark\textsuperscript{\href{https://github.com/YuanGongND/ssast}{[6]}}\\
MAE-AST \cite{baade2022mae} & UTS & Spectrogram & \xmark & \texttt{MAE} & \cmark$^{\natural}$ & \cmark & \xmark & \xmark & Classification & Audio & \cmark\textsuperscript{\href{https://github.com/AlanBaade/MAE-AST-Public}{[7]}}\\
AST-SED \cite{li2023ast} & UTS & Spectrogram & \xmark & \texttt{SSAST,GRU} & \cmark & \cmark & \xmark & \xmark & EventDetection & Audio & \xmark\\
\cite{jin2023classification} & UTS & %Multiple
LinePlot & \xmark & \texttt{CNN} & \cmark & \cmark & \xmark & \xmark & Classification & Physics & \xmark\\
ForCNN \cite{semenoglou2023image} & UTS & LinePlot & \xmark & \texttt{CNN} & \xmark & \cmark$^{\flat}$ & \xmark & \xmark & Forecasting & General & \xmark\\
Vit-num-spec \cite{zeng2023pixels} & UTS & Spectrogram & \xmark & \texttt{ViT} & \xmark & \cmark$^{\flat}$ & \xmark & \xmark & Forecasting & Finance & \xmark\\
% \cite{wimmer2023leveraging} & MTS & LinePlot & \xmark & \texttt{CLIP,LSTM} & \cmark & \cmark & \xmark & \xmark & Classification & Finance & \xmark\\
ViTST \cite{li2023time} & MTS & LinePlot & \xmark & \texttt{Swin} & \cmark & \cmark & \xmark & \xmark & Classification & General & \cmark\textsuperscript{\href{https://github.com/Leezekun/ViTST}{[8]}}\\
MV-DTSA \cite{yang2023your} & UTS\textsuperscript{*} & LinePlot & \xmark & \texttt{CNN} & \xmark & \cmark$^{\flat}$ & \xmark & \cmark & Forecasting & General & \cmark\textsuperscript{\href{https://github.com/IkeYang/machine-vision-assisted-deep-time-series-analysis-MV-DTSA-}{[9]}}\\
TimesNet \cite{wu2023timesnet} & MTS & Heatmap & \xmark & \texttt{CNN} & \xmark & \cmark$^{\flat}$ & \xmark & \cmark & Multiple & General & \cmark\textsuperscript{\href{https://github.com/thuml/TimesNet}{[10]}}\\
ITF-TAD \cite{namura2024training} & UTS & Spectrogram & \xmark & \texttt{CNN} & \cmark & \xmark & \xmark & \xmark & Anomaly & General & \xmark\\
\cite{kaewrakmuk2024multi} & UTS & GAF & \xmark & \texttt{CNN} & \cmark & \cmark & \xmark & \xmark & Classification & Sensing & \xmark\\
HCR-AdaAD \cite{lin2024hierarchical} & MTS & RP & \xmark & \texttt{CNN,GNN} & \xmark & \cmark$^{\flat}$ & \xmark & \xmark & Anomaly & General & \xmark\\
FIRTS \cite{costa2024fusion} & UTS & Other ($\S$\ref{sec.othermethod}) & \xmark & \texttt{CNN} & \xmark & \cmark$^{\flat}$ & \xmark & \xmark & Classification & General & \cmark\textsuperscript{\href{https://sites.google.com/view/firts-paper}{[11]}}\\
% \multirow{2}{*}{FIRTS \cite{costa2024fusion}} & \multirow{2}{*}{UTS} & Spectrogram & \multirow{2}{*}{\xmark} & \multirow{2}{*}{\texttt{CNN}} & \multirow{2}{*}{\xmark} & \multirow{2}{*}{\cmark$^{\flat}$} & \multirow{2}{*}{\xmark} & \multirow{2}{*}{\xmark} & \multirow{2}{*}{Classification} & \multirow{2}{*}{General} & \multirow{2}{*}{\cmark\textsuperscript{\href{https://sites.google.com/view/firts-paper}{[2]}}}\\
%  & & \& GAF,RP,MTF & & & & & & & & & \\
% \cite{homenda2024time} & UTS\textsuperscript{*} & Multiple & \xmark & \texttt{CNN} & \xmark & \cmark$^{\flat}$ & \xmark & \xmark & Classification & General & \xmark\\
CAFO \cite{kim2024cafo} & MTS & RP & \xmark & \texttt{CNN,ViT} & \xmark & \cmark$^{\flat}$ & \xmark & \xmark & Explanation & General & \cmark\textsuperscript{\href{https://github.com/eai-lab/CAFO}{[12]}}\\
% \multirow{2}{*}{CAFO \cite{kim2024cafo}} & \multirow{2}{*}{MTS} & \multirow{2}{*}{RP} & \multirow{2}{*}{\xmark} & \texttt{ShuffleNet,ResNet} & \multirow{2}{*}{\cmark} & \multirow{2}{*}{\cmark} & \multirow{2}{*}{\xmark} & \multirow{2}{*}{\xmark} & Classification & \multirow{2}{*}{General} & \multirow{2}{*}{\cmark}\\
%  & & & & \texttt{MLP-Mixer,ViT} & & & & & \& Explanation & & \\
ViTime \cite{yang2024vitime} & UTS\textsuperscript{*} & LinePlot & \xmark & \texttt{ViT} & \cmark$^{\natural}$ & \cmark & \xmark & \cmark & Forecasting & General & \cmark\textsuperscript{\href{https://github.com/IkeYang/ViTime}{[13]}}\\
ImagenTime \cite{naiman2024utilizing} & MTS & Other ($\S$\ref{sec.othermethod}) & \xmark & %\texttt{Diffusion}
\texttt{CNN} & \xmark & \cmark$^{\flat}$ & \xmark & \cmark & Ts-Generation & General & \cmark\textsuperscript{\href{https://github.com/azencot-group/ImagenTime}{[14]}}\\
TimEHR \cite{karami2024timehr} & MTS & Heatmap & \xmark & \texttt{CNN} & \xmark & \cmark$^{\flat}$ & \xmark & \cmark & Ts-Generation & Health & \cmark\textsuperscript{\href{https://github.com/esl-epfl/TimEHR}{[15]}}\\
VisionTS \cite{chen2024visionts} & UTS\textsuperscript{*} & Heatmap & \xmark & \texttt{MAE} & \cmark & \cmark & \xmark & \cmark & Forecasting & General & \cmark\textsuperscript{\href{https://github.com/Keytoyze/VisionTS}{[16]}}\\ \midrule
InsightMiner \cite{zhang2023insight} & UTS & LinePlot & \cmark & \texttt{LLaVA} & \cmark & \cmark & \cmark & \xmark & Txt-Generation & General & \xmark\\
\cite{wimmer2023leveraging} & MTS & LinePlot & \cmark & \texttt{CLIP,LSTM} & \cmark & \cmark & \xmark & \xmark & Classification & Finance & \xmark\\
% \cite{dixit2024vision} & UTS & Spectrogram & \cmark & \texttt{GPT4o,Gemini} & \cmark & \xmark & \cmark & \xmark & Classification & Audio & \xmark\\
\multirow{2}{*}{\cite{dixit2024vision}} & \multirow{2}{*}{UTS} & \multirow{2}{*}{Spectrogram} & \multirow{2}{*}{\cmark} & \texttt{GPT4o,Gemini} & \multirow{2}{*}{\cmark} & \multirow{2}{*}{\xmark} & \multirow{2}{*}{\cmark} & \multirow{2}{*}{\xmark} & \multirow{2}{*}{Classification} & \multirow{2}{*}{Audio} & \multirow{2}{*}{\xmark}\\
 & & & & \& \texttt{Claude3} & & & & & & & \\
\cite{daswani2024plots} & MTS & LinePlot & \cmark & \texttt{GPT4o,Gemini} & \cmark & \xmark & \cmark & \xmark & Multiple & General & \xmark\\
TAMA \cite{zhuang2024see} & UTS & LinePlot & \cmark & \texttt{GPT4o} & \cmark & \xmark & \cmark & \xmark & Anomaly & General & \xmark\\
\cite{prithyani2024feasibility} & MTS & LinePlot & \cmark & \texttt{LLaVA} & \cmark & \cmark & \cmark & \xmark & Classification & General & \cmark\textsuperscript{\href{https://github.com/vinayp17/VLM_TSC}{[17]}}\\
\bottomrule[1pt]
\end{tabular}}
\vspace{-0.25cm}
\caption{Taxonomy of vision models on time series. The top panel includes single-modal models. The bottom panel includes multi-modal models. {\bf TS-Type} denotes type of time series. {\bf TS-Recover} denotes %whether time series recovery ($\S$\ref{sec.processing}) has been performed.
recovering time series from predicted images ($\S$\ref{sec.processing}). \textsuperscript{*}: %the model has been %applied on MTSs by %processing %modeling the individual UTSs of each MTS.
the method has been used to model the individual UTSs of an MTS. $^{\natural}$: a new pre-trained model was proposed in the work. $^{\flat}$: %without using a pre-trained model, fine-tune means training from scratch.
when pre-trained models were unused, ``Fine-tune'' refers to train a task-specific model from scratch. %In the
{\bf Model} column: \texttt{CNN} could be regular CNN, ResNet, VGG-Net, %U-Net,
{\em etc.}}\label{tab.taxonomy}
% The code only include verified official code from the authors.
\vspace{-0.3cm}
\end{table*}

\begin{table*}[t]
\centering
\small
\setlength{\tabcolsep}{2.9pt}{
\begin{tabular}{l|l|l|l}\hline
% \toprule[1pt]
\rowcolor{gray!20}
{\bf Method} & {\bf TS-Type} & {\bf Advantages} & {\bf Limitations}\\ \hline
Line Plot ($\S$\ref{sec.lineplot}) & UTS, MTS & matches human perception of time series & limited to MTSs with a small number of variates\\ \hline
Heatmap ($\S$\ref{sec.heatmap}) & UTS, MTS & straightforward for both UTSs and MTSs & the order of variates may affect their correlation learning\\ \hline
Spectrogram ($\S$\ref{sec.spectrogram}) & UTS & encodes the time-frequency space & limited to UTSs; needs a proper choice of window/wavelet\\ \hline
GAF ($\S$\ref{sec.gaf}) & UTS & encodes the temporal correlations in a UTS & limited to UTSs; $O(T^{2})$ time and space complexity\\ \hline% for long time series\\ \hline
% RP ($\S$\ref{sec.rp}) & UTS & flexibility in image size by tuning $m$ and $\tau$ & limited to UTSs; the pattern has a threshold-dependency\\ \hline
RP ($\S$\ref{sec.rp}) & UTS & flexibility in image size by tuning $m$ and $\tau$ & limited to UTSs; information loss after thresholding\\ \hline
% \bottomrule[1pt]
\end{tabular}}
\vspace{-0.2cm}
\caption{Summary of the five primary methods for transforming time series to images. {\bf TS-Type} denotes type of time series.}\label{tab.tsimage}
\vspace{-0.2cm}
\end{table*}

\section{Time Series To Image Transformation}\label{sec.tsimage}

% This section summarizes 5 major methods for imaging time series ($\S$\ref{sec.lineplot}-$\S$\ref{sec.rp}). We also discuss some other methods ($\S$\ref{sec.othermethod}) and how to model MTS with these methods ($\S$\ref{sec.modelmts}).
This section summarizes the methods for imaging time series ($\S$\ref{sec.lineplot}-$\S$\ref{sec.othermethod}) and their extensions to encode MTSs ($\S$\ref{sec.modelmts}).

% This section summarizes 5 major methods for transforming time series to images, including Line Plot, Heatmap, Spetrogram, GAF and RP, and several minor methods. We discuss their pros and cons and how to deal with MTS.

% This section discusses the advantages and limitations of different methods for time series to image transformation (invertible, efficiency, information preservation, MTS, long-range time series, parametric, etc.).

%\subsection{Methods}

\vspace{-0.08cm}

\subsection{Line Plot}\label{sec.lineplot}

Line Plot is a straightforward way for visualizing UTSs for human analysis ({\em e.g.}, stocks, power consumption, {\em etc.}). As illustrated by Fig. \ref{fig.tsimage}(a), the simplest approach is to draw a 2D image with x-axis representing %the time horizon
time steps and y-axis representing %the magnitude of the normalized time series.
time-wise values, %A line is used to connect all values of the series over time.
with a line connecting all values of the series over time. This image can be %represented by either three-channel pixels or single-channel pixels
either three-channel ({\em i.e.}, RGB) or single-channel as the colors may not %provide additional information
be informative %\cite{cohen2020trading,sood2021visual,jin2023classification,zhang2023insight,zhuang2024see}.
\cite{cohen2020trading,sood2021visual,jin2023classification,zhang2023insight}. ForCNN \cite{semenoglou2023image} even uses a single 8-bit integer to represent each pixel for black-white images. So far, there is no consensus on whether other graphical components, such as legend, grids and tick labels, could provide extra benefits in any task. For example, ViTST \cite{li2023time} finds these components are superfluous in a classification task, while TAMA \cite{zhuang2024see} finds grid-like auxiliary lines help enhance anomaly detection.

In addition to the regular Line Plot, MV-DTSA \cite{yang2023your} and ViTime \cite{yang2024vitime} divide an image into $h\times L$ grids, %where $h$ is the number of rows and $L$ is the number of columns,
and %introduced
define a function to map each time step of a UTS to a grid, producing a grid-like Line Plot. Also, we include methods that use Scatter Plot \cite{daswani2024plots,prithyani2024feasibility} in this category because %the only difference between a Scatter Plot and a Line Plot is whether the time-wise values are connected by lines.
a Scatter Plot resembles a Line Plot but doesn't connect %time-wise values
data points with a line. By comparing them, \cite{prithyani2024feasibility} finds a Line Plot could induce better time series classification.

For MTSs, we defer the discussion on Line Plot to $\S$\ref{sec.modelmts}.

% For MTS, some methods use the channel-independence assumption proposed in \cite{nie2023time} and represent each variate in MTS with an individual Line Plot \cite{yang2023your,yang2024vitime}. ViTST \cite{li2023time} also uses an individual Line Plot per variate, but colors different lines and assembles all plots to form a bigger image. The method in \cite{wimmer2023leveraging} plots %the time series of
% all variates in a single Line Plot and distinguish them by %use different
% types of lines ({\em e.g.}, solid, dashed, dotted, {\em etc.}). %to distinguish them.
% However, these methods only work for a small number of variates. For example, in \cite{wimmer2023leveraging}, there are only 4 variates in its financial MTSs.

%\cite{li2023time} space-costly because of blank pixels. scatter plot.

%Invertible with a numeric prediction head \cite{sood2021visual}. It fits tasks such as forecasting, imputation, etc.

\vspace{-0.08cm}

\subsection{Heatmap}\label{sec.heatmap}

As shown in Fig. \ref{fig.tsimage}(b), Heatmap is a 2D visualization of the magnitude of the values in a matrix using color. %The variation of color represents the intensity of each value. %Therefore,
It has been used to %directly
represent the matrix of an MTS, {\em i.e.}, $\mat{X} \in \mathbb{R}^{d\times T}$, as a one-channel $d\times T$ image \cite{li2022tts,yazdanbakhsh2019multivariate}. Similarly, TimEHR \cite{karami2024timehr} represents an {\em irregular} MTS, where the intervals between time steps are uneven, as a $d\times H$ Heatmap image by grouping the uneven time steps into $H$ even time bins. In \cite{zeng2021deep}, a different method is used for visualizing a 9-variate financial %time series.
MTS. It reshapes the 9 variates at each time step to a $3\times 3$ Heatmap image, and uses the sequence of images to forecast future %image
frames, achieving %time series
%MTS
time series forecasting. In contrast, VisionTS \cite{chen2024visionts} uses Heatmap to visualize UTSs. %instead.
Similar to TimesNet \cite{wu2023timesnet}, it first segments a length-$T$ UTS into $\lfloor T/P\rfloor$ length-$P$ subsequences, where $P$ is a parameter representing a periodicity of the UTS. Then the subsequences are stacked into a $P\times \lfloor T/P\rfloor$ matrix, %and duplicated 3 times to produce a 3-channel
with 3 duplicated channels, to produce a grayscale image %which serves as an
input to %a vision foundation model.
an LVM. To encode MTSs, VisionTS adopts the channel independence assumption \cite{nie2023time} and individually models each variate in an MTS.

\vspace{0.2cm}

\noindent{\bf Remark.} Heatmap can be used to visualize matrices of various forms. It is also used for matrices generated by the subsequent methods ({\em e.g.}, Spectrogram, GAF, RP) in this section. In this paper, the name Heatmap refers specifically to images that use color to visualize the (normalized) values in UTS $\mat{x}$ or MTS $\mat{X}$ without performing other transformations.

%\cite{chen2024visionts,karami2024timehr} bin version of TSH \cite{karami2024timehr}, DE and STFT \cite{naiman2024utilizing} (DE can be used for constructing RP), rearrange variates for video version of TSH \cite{zeng2021deep}.

%\vspace{0.2cm}

\subsection{Spectrogram}\label{sec.spectrogram}

A {\em spectrogram} is a visual representation of the spectrum of frequencies of a signal as it varies with time, which are extensively used for analyzing audio signals \cite{gong2021ast}. Since audio signals are a type of UTS, spectrogram can be considered as a method for imaging a UTS. As shown in Fig. \ref{fig.tsimage}(c), a common format is a 2D heatmap image with x-axis representing time steps and y-axis representing frequency, {\em a.k.a.} a time-frequency space. %The color at each point
Each pixel in the image represents the (logarithmic) amplitude of a specific frequency at a specific time point. Typical methods for %transforming a UTS to
producing a spectrogram include {\bf Short-Time Fourier Transform (STFT)} \cite{griffin1984signal}, {\bf Wavelet Transform} \cite{daubechies1990wavelet}, and {\bf Filterbank} \cite{vetterli1992wavelets}.

\vspace{0.2cm}

\noindent{\bf STFT.} %Discrete Fourier transform (DFT) can be used to represent a UTS signal %$\mat{x}=[x_{1}, ..., x_{T}]$
%$\mat{x}\in\mathbb{R}^{1\times T}$ as a sum of sinusoidal components. The output of the transform is a function of frequency $f(w)$, describing the intensity of each constituent frequency $w$ of the entire UTS. 
Discrete Fourier transform (DFT) can be used to describe the intensity $f(w)$ of each constituent frequency $w$ of a UTS signal $\mat{x}\in\mathbb{R}^{1\times T}$. However, $f(w)$ has no time dependency. It cannot provide dynamic information such as when a specific frequency appear in the UTS. STFT addresses this deficiency by sliding a window function $g(t)$ over the time steps in %the UTS,
$\mat{x}$, and computing the DFT within each window by
\begin{equation}\label{eq.stft}
\small
\begin{aligned}
f(w,\tau) = \sum_{t=1}^{T}x_{t}g(t - \tau)e^{-iwt}
\end{aligned}
\end{equation}
where $w$ is frequency, $\tau$ is the position of the window, $f(w,\tau)$ describes the intensity of frequency $w$ at time step $\tau$.

%With a proper selection of the
By selecting a proper window function $g(\cdot)$ ({\em e.g.}, Gaussian/Hamming/Bartlett window), %({\em e.g.}, Gaussian window, Hamming window, Bartlett window), %{\em etc.}),
a 2D spectrogram ({\em e.g.}, Fig. \ref{fig.tsimage}(c)) can be drawn via a heatmap on the squared values $|f(w,\tau)|^{2}$, with $w$ as the y-axis, and $\tau$ as the x-axis. For example, \cite{dixit2024vision} uses STFT based spectrogram as an input to LMMs %\hh{do you mean LVMs? check}
for time series classification.

%Fourier transform is a powerful data analysis tool that represents any complex signal as a sum of sines and cosines and transforms the signal from the time domain to the frequency domain. However, Fourier transform can only show which frequencies are present in the signal, but not when these frequencies appear. The STFT divides original signal into several parts using a sliding window to fix this problem. STFT involves a sliding window for extracting frequency components within the window.

\vspace{0.2cm}

\noindent{\bf Wavelet Transform.} %Like Fourier transform, %\hh{this paragraph needs a citation}
Continuous Wavelet Transform (CWT) uses the inner product to measure the similarity between a signal function $x(t)$ and an analyzing function. %In STFT (Eq.~\eqref{eq.stft}), the analyzing function is a windowed exponential $g(t - \tau)e^{-iwt}$.
%In CWT,
The analyzing function is a {\em wavelet} $\psi(t)$, where the typical choices include Morse wavelet, Morlet wavelet, %Daubechies wavelet, %Beylkin wavelet, 
{\em etc.} %The
CWT compares $x(t)$ to the shifted and scaled ({\em i.e.}, stretched or shrunk) versions of the wavelet, and output a CWT coefficient by
\begin{equation}\label{eq.cwt}
\small
\begin{aligned}
c(s,\tau) = \int_{-\infty}^{\infty}x(t)\frac{1}{s}\psi^{*}(\frac{t - \tau}{s})dt
\end{aligned}
\end{equation}
where $*$ denotes complex conjugate, $\tau$ is the time step to shift, and $s$ represents the scale. In practice, a discretized version of CWT in Eq.~\eqref{eq.cwt} is implemented for UTS $[x_{1}, ..., x_{T}]$.

It is noteworthy that the scale $s$ controls the frequency encoded in a wavelet -- a larger $s$ leads to a stretched wavelet with a lower frequency, and vice versa. As such, by varying $s$ and $\tau$, a 2D spectrogram ({\em e.g.}, Fig. \ref{fig.tsimage}(d)) can be drawn %, often with a heatmap
on $|c(s,\tau)|$, where $s$ is the y-axis and $\tau$ is the x-axis. Compared to STFT, which uses a fixed window size, Wavelet Transform allows variable wavelet sizes -- a larger size %region
for more precise low frequency information. 
%Usually, $s$ and $\tau$ vary dependently -- a larger $s$ leads to a stretched wavelet that shifts slowly, {\em i.e.}, a smaller $\tau$. This property %of CWT
%yields a spectrogram that balances the resolutions of frequency %$s$
%and time, %$\tau$,
%which is an advantage over the fixed time resolution in STFT.
% Thus, both of the methods in %\cite{du2020image}
% \cite{namura2024training} and \cite{zeng2023pixels} choose CWT (with Morlet wavelet) to generate the spectrogram.
Thus, the methods in \cite{du2020image,namura2024training,zeng2023pixels} choose CWT (with Morlet wavelet) to generate the spectrogram.

%A wavelet is a wave-like oscillation that has zero mean and is localized in both time and frequency space.

\vspace{0.2cm}

\noindent{\bf Filterbank.} This method %is relevant to
resembles STFT and is often used in processing audio signals. Given an audio signal, it firstly goes through a {\em pre-emphasis filter} to boost high frequencies, which helps improve the clarity of the signal. Then, STFT is applied on the signal. %with a sliding window $g(t)$ of size $k$ that shifts in a fixed stride $\tau$. %where the adjacent windows may overlap in $k$ time length.
%Finally, filterbank features are computed by applying multiple ``triangle-shaped'' filters spaced on the Mel-scale to the STFT output $f(w, \tau)$. %where Mel-scale is a method to make the filters more discriminative on lower frequencies, %than higher frequencies,
%imitating the non-linear human ear perception of sound.
Finally, multiple ``triangle-shaped'' filters spaced on a Mel-scale are applied to the STFT power spectrum $|f(w, \tau)|^{2}$ to extract frequency bands. The outcome filterbank features $\hat{f}(w, \tau)$ can be used to yield a spectrogram with $w$ as the y-axis, and $\tau$ as the x-axis.

%Filterbank was introduced in AST \cite{gong2021ast} with %$k$=25ms
Filterbank was adopted in AST \cite{gong2021ast} with 
a 25ms Hamming window that shifts every 10ms for classifying audio signals using Vision Transformer (ViT). It then becomes widely used in the follow-up works such as SSAST \cite{gong2022ssast}, MAE-AST \cite{baade2022mae}, and AST-SED \cite{li2023ast}, as summarized in Table \ref{tab.taxonomy}.



%Use MLP to predict TS directly \cite{zeng2023pixels}.

%\vspace{0.2cm}

% \vspace{0.2cm}

\subsection{Gramian Angular Field (GAF)}\label{sec.gaf}

GAF was introduced for classifying UTSs using CNNs %using %image based CNNs
by \cite{wang2015encoding}. It was then extended %with an extension
to an imputation task in \cite{wang2015imaging}. Similarly, \cite{barra2020deep} applied GAF for financial time series forecasting.

Given a UTS $\mat{x}\in\mathbb{R}^{1\times T}$, %$[x_{1}, ..., x_{T}]$,
the first step %before GAF
is to rescale each $x_{t}$ to a value $\tilde{x}_{t}$ %in the interval of
within $[0, 1]$ (or $[-1, 1]$). %by min-max normalization.
This range enables mapping $\tilde{x}_{t}$ to polar coordinates by $\phi_{t}=\text{arccos}(\tilde{x}_{i})$, with a radius $r=t/N$ encoding the time stamp, where $N$ is a constant factor to regularize the span of the polar coordinates. %system. Then,
Two types of GAF, Gramian Sum Angular Field (GASF) and Gramian Difference Angular Field (GADF) are defined as
\begin{equation}\label{eq.gaf}
\small
\begin{aligned}
&\text{GASF:}~~\text{cos}(\phi_{t} + \phi_{t'})=x_{t}x_{t'} - \sqrt{1 - x_{t}^{2}}\sqrt{1 - x_{t'}^{2}}\\
&\text{GADF:}~~\text{sin}(\phi_{t} - \phi_{t'})=x_{t'}\sqrt{1 - x_{t}^{2}} - x_{t}\sqrt{1 - x_{t'}^{2}}
\end{aligned}
\end{equation}
which exploits the pairwise temporal correlations in the UTS. Thus, the outcome is a $T\times T$ matrix $\mat{G}$ with $\mat{G}_{t,t'}$ specified by either type in Eq.~\eqref{eq.gaf}. A GAF image is a heatmap on $\mat{G}$ with both axes representing time, as illustrated by Fig. \ref{fig.tsimage}(e).

% Invertible.

% \vspace{0.2cm}

\subsection{Recurrence Plot (RP)}\label{sec.rp}

%RP \cite{eckmann1987recurrence} is a method to encode a UTS into an image that aims to capture the periodic patterns in the UTS by using its reconstructed {\em phase space}. The phase space of a UTS $[x_{1}, ..., x_{T}]$ can be reconstructed by {\em time delay embedding}, which is a set of new vectors $\mat{v}_{1}$, ..., $\mat{v}_{l}$ with

RP \cite{eckmann1987recurrence} encodes a UTS into an image that captures its periodic patterns by using its reconstructed {\em phase space}. The phase space of %a UTS %$[x_{1}, ..., x_{T}]$
$\mat{x}\in\mathbb{R}^{1\times T}$ can be reconstructed by {\em time delay embedding} -- a set of new vectors $\mat{v}_{1}$, ..., $\mat{v}_{l}$ with
\begin{equation}\label{eq.de}
\small
\begin{aligned}
\mat{v}_{t}=[x_{t}, x_{t+\tau}, x_{t+2\tau}, ..., x_{t+(m-1)\tau}]\in\mathbb{R}^{m\tau},~~~1\le t \le l
\end{aligned}
\end{equation}
where $\tau$ is the time delay, $m$ is the dimension of the phase space, both %of which
are hyperparameters. Hence, $l=T-(m-1)\tau$. With vectors $\mat{v}_{1}$, ..., $\mat{v}_{l}$, an RP image %is constructed by measuring
measures their pairwise distances, results in an $l\times l$ image whose element
\begin{equation}\label{eq.rp}
\small
\begin{aligned}
\text{RP}_{i,j}=\Theta(\varepsilon - \|\mat{v}_{i} - \mat{v}_{j}\|),~~~1\le i,j\le l
\end{aligned}
\end{equation}
where $\Theta(\cdot)$ is the Heaviside step function, $\varepsilon$ is a threshold, and $\|\cdot\|$ is a norm function such as $\ell_{2}$ norm. Eq.~\eqref{eq.rp} %states RP produces a heatmap image on a binary matrix with $\text{RP}_{i,j}=1$ if $\mat{v}_{i}$ and $\mat{v}_{j}$ are sufficiently similar.
generates a binary matrix with $\text{RP}_{i,j}=1$ if $\mat{v}_{i}$ and $\mat{v}_{j}$ are sufficiently similar, producing a black-white image ({\em e.g.}, Fig. \ref{fig.tsimage}(f)).% ({\em e.g.}, a periodic pattern).

An advantage of RP is its flexibility in image size by tuning $m$ and $\tau$. Thus it has been used for time series classification %\cite{cao2021image},
\cite{silva2013time,hatami2018classification}, forecasting \cite{li2020forecasting}, anomaly detection \cite{lin2024hierarchical} and %feature-wise
explanation \cite{kim2024cafo}. Moreover, the method in \cite{hatami2018classification}, and similarly in HCR-AdaAD \cite{lin2024hierarchical}, omit the thresholding in Eq.~\eqref{eq.rp} and uses $\|\mat{v}_{i} - \mat{v}_{j}\|$ to produce continuously valued images %in a classification task
to avoid information loss.


% \vspace{0.2cm}

\subsection{Other Methods}\label{sec.othermethod}

%There are some less commonly used methods. For example, in
Additionally, %there are some peripheral methods. %In addition to GAF,
\cite{wang2015encoding} introduces Markov Transition Field (MTF) for imaging a UTS. %$\mat{x}\in\mathbb{R}^{1\times T}$. 
%MTF first assigns each $x_{t}$ to one of $Q$ quantile bins, then builds a $Q\times Q$ Markov transition matrix $\mat{M}$ {\em s.t.} $\mat{M}_{i,j}$ represents the frequency %with which
%of the case when a point $x_{t}$ in the $i$-th bin is followed by a point $x_{t'}$ in the $j$-th bin, {\em i.e.}, $t=t'+1$. Matrix $\mat{M}$ serves as the input of a heatmap image.
MTF is a matrix $\mat{M}\in\mathbb{R}^{Q\times Q}$ encoding the transition probabilities over time segments, where $Q$ is the number of segments. %Moreover,
ImagenTime \cite{naiman2024utilizing} stacks the delay embeddings $\mat{v}_{1}$, ..., $\mat{v}_{l}$ in Eq.~\eqref{eq.de} to an $l\times m\tau$ matrix for visualizing UTSs. %It also uses a variant of STFT.
% The method in \cite{homenda2024time} introduces five different 2D images by counting, rearranging, replicating the values in a UTS. 
MSCRED \cite{zhang2019deep} uses heatmaps on the $d\times d$ correlation matrices of MTSs with $d$ variates for anomaly detection. 
Furthermore, some methods use a mixture of imaging methods by stacking different transformations. \cite{wang2015imaging} stacks GASF, GADF, MTF to a 3-channel image. %Similarly,
FIRTS \cite{costa2024fusion} builds a 3-channel image by stacking GASF, MTF and RP. %the GASF, MTF, RP representations of each UTS.
%\cite{jin2023classification} combines Line Plot with Constant-Q Transform (CQT) \cite{brown1991calculation}, a method related to wavelet transform ($\S$\ref{sec.spectrogram}), to generate 2-channel images.
The mixture methods encode a UTS with multiple views and were found more robust than single-view images in these works for %time series
classification tasks.

\subsection{How to Model MTS}\label{sec.modelmts}

In the above methods, Heatmap ($\S$\ref{sec.heatmap}) can be %directly
used to visualize the %2D
variate-time matrices, $\mat{X}$, of MTSs ({\em e.g.}, Fig. \ref{fig.structure}(b)), where correlated variates %are better to
should be spatially close to each other. Line Plot ($\S$\ref{sec.lineplot}) can be used to visualize MTSs by plotting all variates in the same image \cite{wimmer2023leveraging,daswani2024plots} or combining all univariate images to compose a bigger %1-channel
image \cite {li2023time}, but these methods only work for a small number of variates. Spectrogram ($\S$\ref{sec.spectrogram}), GAF ($\S$\ref{sec.gaf}), and RP ($\S$\ref{sec.rp}) were designed specifically for UTSs. For these methods and Line Plot, which are not straightforward %for MTS transformation,
in imaging MTSs, the general approaches %to use them %for MTS
include using channel independence assumption to model each variate individually \cite{nie2023time}, %like VisionTS \cite{chen2024visionts},
or stacking the images of $d$ variates to form a $d$-channel image %as did by
\cite{naiman2024utilizing,kim2024cafo}. %\cite{prithyani2024feasibility,naiman2024utilizing,kim2024cafo}.
However, the latter does not fit some vision models pre-trained on RGB images which requires 3-channel inputs (more discussions are deferred to $\S$\ref{sec.processing}).

\vspace{0.2cm}

\noindent{\bf Remark.} As a summary, Table \ref{tab.tsimage} recaps the salient advantages and limitations of the five primary imaging methods that are introduced in this section.

% \hh{can we have a table (e.g., rows are different imaging methods and columns are a few desirable propoerties) or a short paragraph to discuss/summarize/compare the strenths and weakness of different imaging methods for ts? This might bring some structure/comprehension to this section (as opposed to, e.g., some reviewer might complain that what we do here is a laundry list)}

\section{Imaged Time Series Modeling}\label{sec.model}

With image representations, time series analysis can be readily performed with vision models. This section discusses such solutions from %traditional vision models %($\S$\ref{sec.cnns})
%to the recent large vision models %($\S$\ref{sec.lvms})
%and large multimodal models.% ($\S$\ref{sec.lmms}).
the traditional models to the SOTA models.

\begin{figure*}[!t]
\centering
\includegraphics[width=0.9\textwidth]{fig/fig_2.pdf}
% \vspace{-1em}
\caption{An illustration of different modeling strategies on imaged time series in (a)(b)(c) and task-specific heads in (d).}\label{fig.models}
\vspace{-0.2cm}
\end{figure*}

\subsection{Conventional Vision Models}\label{sec.cnns}

%Similar to
Following traditional %methods on
image classification, \cite{silva2013time} applies a K-NN classifier on the RPs of time series, \cite{cohen2020trading} applies an ensemble of fundamental classifiers such as %linear regression, SVM, Ada Boost, {\em etc.}
SVM and AdaBoost on the Line Plots %images
for time series classification. As an image encoder, %a typical encoder, %of images,
CNNs have been %extensively
widely used for learning image representations. %\cite{he2016deep}.
Different from using 1D CNNs on sequences %UTS or MTS
\cite{bai2018empirical}, %regular
2D or 3D CNNs can be applied on imaged time series as shown in Fig. \ref{fig.models}(a). %to learn time series representations by encoding their image transformations.
For example, %standard
regular CNNs have been used on Spectrograms \cite{du2020image}, tiled CNNs have been used on GAF images \cite{wang2015encoding,wang2015imaging}, dilated CNNs have been used on Heatmap images \cite{yazdanbakhsh2019multivariate}. More frequently, ResNet \cite{he2016deep}, Inception-v1 \cite{szegedy2015going}, and VGG-Net \cite{simonyan2014very} have been used on Line Plots \cite{jin2023classification,semenoglou2023image}, Heatmap images \cite{zeng2021deep}, RP images \cite{li2020forecasting,kim2024cafo}, GAF images \cite{barra2020deep,kaewrakmuk2024multi}, 
% Heatmaps \cite{zeng2021deep}, RPs \cite{li2020forecasting,kim2024cafo}, GAFs \cite{barra2020deep,kaewrakmuk2024multi},
and even a mixture of GAF, MTF and RP images \cite{costa2024fusion}. In particular, for time series generation tasks, %a diffusion model with U-Nets \cite{naiman2024utilizing} and GAN frameworks of CNNs \cite{li2022tts,karami2024timehr} have also been explored.%investigated.
GAN frameworks of CNNs \cite{li2022tts,karami2024timehr} and a diffusion model with U-Nets \cite{naiman2024utilizing} have also been explored.

Due to their small to medium sizes, these models are often trained from scratch using task-specific training data. %per task using the task's training set. %of time series images.
Meanwhile, fine-tuning {\em pre-trained vision models}  %such as those pre-trained on ImageNet, %\cite{deng2009imagenet}, 
have already been found promising in cross-modality knowledge transfer for time series anomaly detection \cite{namura2024training}, forecasting \cite{li2020forecasting} and classification \cite{jin2023classification}.

% \cite{li2020forecasting} uses ImageNet pretrained CNNs.

\subsection{Large Vision Models (LVMs)}\label{sec.lvms}

Vision Transformer (ViT) \cite{dosovitskiy2021image} has %given birth to
inspired the development of %some
modern LVMs %large vision models (LVMs)
such as %DeiT \cite{touvron2021training}, 
Swin \cite{liu2021swin}, BEiT \cite{bao2022beit}, and MAE \cite{he2022masked}. %Given an input image, ViT splits it
As Fig. \ref{fig.models}(b) shows, ViT splits an %input
image into {\em patches} of fixed size, then embeds each patch and augments it with a positional embedding. The %resulting
vectors of patches are processed by a Transformer %encoder
as if they were token embeddings. Compared to CNNs, ViTs are less data-efficient, but have higher capacity. %Consequently,
Thus, %the
{\em pre-trained} ViTs have been explored for modeling %the images of time series.
imaged time series. For example, AST \cite{gong2021ast} fine-tunes DeiT \cite{touvron2021training} on the filterbank spetrogram of audios %signals
for classification tasks and finds %using
ImageNet-pretrained DeiT is remarkably effective in knowledge transfer. The fine-tuning paradigm has also been %similarly
adopted in \cite{zeng2023pixels,li2023time} but with different pre-trained models %initializations
such as Swin by \cite{li2023time}. 
VisionTS \cite{chen2024visionts} %explains
attributes %the superiority of LVMs
LVMs' superiority over LLMs in knowledge transfer %over LLMs %as an outcome of
to the small gap between the pre-trained images and imaged time series. %the patterns learned from the large-scale pre-trained images and the patterns in the images of time series.
It %also
finds that with one-epoch fine-tuning, MAE becomes the SOTA time series forecasters on %many
some benchmark datasets.

Similar to %build
time series foundation models %\cite{das2024decoder,goswami2024moment,ansari2024chronos,shi2024time}, %such as TimesFM \cite{das2024decoder}, MOMENT \cite{goswami2024moment}, Chronos \cite{ansari2024chronos} and Time-MoE \cite{shi2024time},
such as TimesFM \cite{das2024decoder}, %and MOMENT \cite{goswami2024moment}, 
there are some initial efforts in pre-training ViT architectures with imaged time series. Following AST, SSAST \cite{gong2022ssast} introduced a %joint discriminative and generative
%masked spectrogram patch prediction self-supervised learning framework
masked spectrogram patch prediction framework for pre-training ViT on a large dataset -- AudioSet-2M. Then it becomes a backbone of some follow-up works such as AST-SED \cite{li2023ast} for sound event detection. %To be effective for UTSs,
For UTSs, ViTime \cite{yang2024vitime} generates a large set of Line Plots of synthetic UTSs for pre-training ViT, which was found superior over TimesFM in zero-shot forecasting tasks on benchmark datasets.

\subsection{Large Multimodal Models (LMMs)}\label{sec.lmms}

%As Large Multimodal Models (LMMs)
As LMMs %are getting
get growing attentions, some %of the
notable LMMs, such as LLaVA \cite{liu2023visual}, Gemini \cite{team2023gemini}, GPT-4o \cite{achiam2023gpt} and Claude-3 \cite{anthropic2024claude}, have been explored to consolidate the power of LLMs %on time series
and LVMs in time series analysis. 
Since LMMs support multimodal input via prompts, methods in this thread typically prompt LMMs with the textual and imaged representations of time series, %textual representation of time series and their %image transformations, transformed images,
%then instruct LMMs
and instructions on what tasks to perform ({\em e.g.}, Fig. \ref{fig.models}(c)).

InsightMiner \cite{zhang2023insight} is a pioneer work that uses the LLaVA architecture to generate %textual descriptions about
texts describing the trend of each input UTS. It extracts the trend of a UTS by Seasonal-Trend decomposition, encodes the Line Plot of the trend, and concatenates the embedding of the Line Plot with the embeddings of a textual instruction, which includes a sequence of numbers representing the UTS, {\em e.g.}, ``[1.1, 1.7, ..., 0.3]''. The concatenated embeddings are taken by a language model for generating trend descriptions. %It also fine-tunes a few layers with the generated texts to align LLaVA checkpoints with time series domain.
Similarly, \cite{prithyani2024feasibility} adopts the LLaVA architecture, but for MTS classification. An MTS is encoded by %a sequence of
the visual %token
embeddings of the stacked Line Plots of all variates. %meanwhile
%The method also stacks
%The time series of all variate are also stacked in a prompt % of all variates in a prompt
The matrix of the MTS is also verbalized in a prompt 
as the textual modality. %By manipulating token embeddings,
By integrating token embeddings, both %of these %works propose to
methods fine-tune some layers of the LMMs with some synthetic data.

Moreover, zero-shot and in-context learning performance of several commercial LMMs have been evaluated for audio classification \cite{dixit2024vision}, anomaly detection \cite{zhuang2024see}, and some synthetic tasks \cite{daswani2024plots}, where the image %({\em e.g.}, spectrograms, Line Plots)
and textual representations of a query %UTS or MTS
time series are integrated into a prompt. For in-context learning, these methods inject the images of a few example time series and their labels ({\em e.g.}, classes) %({\em e.g.}, classes, normal status)
into an instruction to prompt LMMs for assisting the prediction of the query time series.

\subsection{Task-Specific Heads}\label{sec.task}

%With the image embedding of a time series, the next step is to produce its prediction.
For classification tasks, most of the methods in Table \ref{tab.taxonomy} adopt a fully connected (FC) layer or multilayer perceptron (MLP) to transform an embedding into a probability distribution over all classes. For forecasting tasks, there are two approaches: (1) using a $d_{e}\times W$ MLP/FC layer to directly predict (from the $d_{e}$-dimensional embedding) the time series values in a future time window of size $W$ \cite{li2020forecasting,semenoglou2023image}; (2) predicting the pixel values that represent the future part of the time series and then recovering the time series from the predicted image \cite{yang2023your,chen2024visionts,yang2024vitime} ($\S$\ref{sec.processing} discusses the recovery methods). Imputation and generation tasks resemble forecasting %in the sense of predicting
as they also predict time series values. Thus approach (2) has been used for imputation \cite{wang2015imaging} and generation \cite{naiman2024utilizing,karami2024timehr}. %LMMs have been used for classification, text generation, and anomaly detection. For these tasks,
When using LMMs for classification, text generation, and anomaly detection, most of the methods prompt LMMs to produce the desired outputs in textual answers, circumventing task-specific heads \cite{zhang2023insight,dixit2024vision,zhuang2024see}.

%Forecasting: MLP, FC to predict numerical values using embeddings. Imputation of images (TSH). Classification: MLP, FC using embeddings.

\section{Pre-Processing and Post-Processing}\label{sec.processing}

To be successful in using vision models, some subtle design desiderata %to be considered
include {\bf time series normalization}, {\bf image alignment} and {\bf time series recovery}.

\vspace{0.2cm}

\noindent{\bf Time Series Normalization.} Vision models are usually trained on %images after Gaussian normalization (GN).
standardized images. To be aligned, the images introduced in $\S$\ref{sec.tsimage} should be normalized with a controlled mean and standard deviation, as did by \cite{gong2021ast} on spectrograms. In particular, as Heatmap is built on raw time series values, the commonly used Instance Normalization (IN) \cite{kim2022reversible} can be applied on the time series as suggested by VisionTS \cite{chen2024visionts} since IN share similar merits as Standardization. %although min-max normalization was used by \cite{karami2024timehr,zeng2021deep}.
Using Line Plot requires a proper range of y-axis. In addition to rescaling time series %by min-max or GN
\cite{zhuang2024see}, ViTST \cite{li2023time} introduced several methods to remove extreme values from the plot. GAF requires min-max normalization on its input, as it transforms time series values withtin $[0, 1]$ to polar coordinates ({\em i.e.}, arccos). In contrast, input to RP is usually normalization-free as an $\ell_{2}$ norm is involved in Eq.~\eqref{eq.rp} before thresholding.%for a comparison with a threshold.

\vspace{0.2cm}

\noindent{\bf Image Alignment.} When using pre-trained models, it is imperative to fit the image size to the input requirement of the models. This is especially true for Transformer based models as they use a fixed number of positional embeddings to encode the spacial information of image patches. For 3-channel RGB images such as Line Plot, it is straightforward to meet a pre-defined size by adjusting the resolution when producing the image. For images built upon matrices such as Heatmap, Spectrogram, GAF, RP, the number of channels and matrix size need adjustment. For the channels, one method is to duplicate a matrix to 3 channels \cite{chen2024visionts}, another way is to average the weights of the 3-channel patch embedding layer into a 1-channel layer \cite{gong2021ast}. For the image size, bilinear interpolation is a common method to resize input images \cite{chen2024visionts}. Alternatively, AST \cite{gong2021ast} %use cut and bilinear interpolation on
resizes the positional embeddings instead of the images to fit the model to a desired input size. However, the interpolation in these methods may either alter the time series or the spacial information in positional embeddings.

% single-channel (UTS), RGB channel (UTS), duplicate channels (UTS), multi-channel (MTS).

%Bilinear interpolation.

%Correlated variates are better to be spatially close to each other.

%\subsection{Pre-training}

\vspace{0.2cm}

\noindent{\bf Time Series Recovery.} As stated in $\S$\ref{sec.task}, tasks such as forecasting, imputation and generation requires predicting time series values. For models that predict pixel values of images, post-processing involves recovering time series from the predicted images. Recovery from Line Plots is tricky, it requires locating pixels that %correspond to
represent time series and mapping them back to the original values. This can be done by manipulating a grid-like Line Plot as introduced in \cite{yang2023your,yang2024vitime}, which has a recovery function. In contrast, recovery from Heatmap is straightforward as it directly stores the predicted time series values \cite{zeng2021deep,chen2024visionts}. Spectrogram is underexplored in these tasks and it remains open on how to recover time series from it. The existing work \cite{zeng2023pixels} uses Spectrogram for forecasting only with an MLP head that directly predicts time series. %predicts time series values.
GAF supports accurate recovery by an inverse mapping from polar coordinates to normalized time series \cite{wang2015imaging}. However, RP lost time series information during thresholding (Eq.~\ref{eq.rp}), thus may not fit recovery-demanded tasks without using an {\em ad-hoc} prediction head.


% Line Plot was regarded as matrices with rows and columns for mapping in \cite{sood2021visual}.


%\section{Tasks and Time Series Recovery}

%\subsection{Task-Specific Head}

% \subsection{Time Series Recovery}



%%%%%%%% ICML 2025 EXAMPLE LATEX SUBMISSION FILE %%%%%%%%%%%%%%%%%

\documentclass{article}

% Recommended, but optional, packages for figures and better typesetting:
\usepackage{microtype}
\usepackage{graphicx}
\usepackage{subfigure}
\usepackage{multirow}
\usepackage{booktabs} % for professional tables

% hyperref makes hyperlinks in the resulting PDF.
% If your build breaks (sometimes temporarily if a hyperlink spans a page)
% please comment out the following usepackage line and replace
% \usepackage{icml2025} with \usepackage[nohyperref]{icml2025} above.
\usepackage{hyperref}


% Attempt to make hyperref and algorithmic work together better:
\newcommand{\theHalgorithm}{\arabic{algorithm}}

% Use the following line for the initial blind version submitted for review:
%\usepackage{icml2025}

% If accepted, instead use the following line for the camera-ready submission:
 \usepackage[accepted]{icml2025}

% For theorems and such
\usepackage{amsmath}
\usepackage{amssymb}
\usepackage{mathtools}
\usepackage{amsthm}

% if you use cleveref..
\usepackage[capitalize,noabbrev]{cleveref}

%%%%%%%%%%%%%%%%%%%%%%%%%%%%%%%%
% THEOREMS
%%%%%%%%%%%%%%%%%%%%%%%%%%%%%%%%
\theoremstyle{plain}
\newtheorem{theorem}{Theorem}[section]
\newtheorem{proposition}[theorem]{Proposition}
\newtheorem{lemma}[theorem]{Lemma}
\newtheorem{corollary}[theorem]{Corollary}
\theoremstyle{definition}
\newtheorem{definition}[theorem]{Definition}
\newtheorem{assumption}[theorem]{Assumption}
\theoremstyle{remark}
\newtheorem{remark}[theorem]{Remark}

% Todonotes is useful during development; simply uncomment the next line
%    and comment out the line below the next line to turn off comments
%\usepackage[disable,textsize=tiny]{todonotes}
\usepackage[textsize=tiny]{todonotes}


% The \icmltitle you define below is probably too long as a header.
% Therefore, a short form for the running title is supplied here:
%\icmltitlerunning{Submission and Formatting Instructions for ICML 2025}
\icmltitlerunning{}

\begin{document}

\twocolumn[
\icmltitle{Transformer-Enhanced Variational Autoencoder for \\Crystal Structure Prediction}

% It is OKAY to include author information, even for blind
% submissions: the style file will automatically remove it for you
% unless you've provided the [accepted] option to the icml2025
% package.

% List of affiliations: The first argument should be a (short)
% identifier you will use later to specify author affiliations
% Academic affiliations should list Department, University, City, Region, Country
% Industry affiliations should list Company, City, Region, Country

% You can specify symbols, otherwise they are numbered in order.
% Ideally, you should not use this facility. Affiliations will be numbered
% in order of appearance and this is the preferred way.
\icmlsetsymbol{equal}{*}

\begin{icmlauthorlist}
\icmlauthor{Ziyi Chen}{yyy,sch}
\icmlauthor{Yang Yuan}{yyy,sch}
\icmlauthor{Siming Zheng}{comp1}
\icmlauthor{Jialong Guo}{yyy}
\icmlauthor{Sihan Liang}{yyy,sch}
\icmlauthor{Yangang Wang}{yyy,sch}
\icmlauthor{Zongguo Wang}{yyy,sch}
%\icmlauthor{}{sch}
%\icmlauthor{}{sch}
%\icmlauthor{}{sch}
\end{icmlauthorlist}

\icmlaffiliation{yyy}{Computer Network Information Center, Chinese Academy of Sciences, Beijing, China}
\icmlaffiliation{comp1}{Siming Zheng is with vivo Mobile Communication Co., Ltd, Hangzhou, China}

\icmlaffiliation{sch}{University of Chinese Academy of Sciences, Beijing, China}

\icmlcorrespondingauthor{Zongguo Wang}{wangzg@cnic.cn}

% You may provide any keywords that you
% find helpful for describing your paper; these are used to populate
% the "keywords" metadata in the PDF but will not be shown in the document
\icmlkeywords{Machine Learning, ICML}

\vskip 0.3in
]

% this must go after the closing bracket ] following \twocolumn[ ...

% This command actually creates the footnote in the first column
% listing the affiliations and the copyright notice.
% The command takes one argument, which is text to display at the start of the footnote.
% The \icmlEqualContribution command is standard text for equal contribution.
% Remove it (just {}) if you do not need this facility.

%\printAffiliationsAndNotice{}  % leave blank if no need to mention equal contribution
\printAffiliationsAndNotice{\icmlEqualContribution} % otherwise use the standard text.

\begin{abstract}
Crystal structure forms the foundation for understanding the physical and chemical properties of materials. Generative models have emerged as a new paradigm in crystal structure prediction(CSP), however, accurately capturing key characteristics of crystal structures, such as periodicity and symmetry, remains a significant challenge. In this paper, we propose a Transformer-Enhanced Variational Autoencoder for Crystal Structure Prediction (TransVAE-CSP), who learns the characteristic distribution space of stable materials, enabling both the reconstruction and generation of crystal structures. TransVAE-CSP integrates adaptive distance expansion with irreducible representation to effectively capture the periodicity and symmetry of crystal structures, and the encoder is a transformer network based on an equivariant dot product attention mechanism. Experimental results on the carbon\_24, perov\_5, and mp\_20 datasets demonstrate that TransVAE-CSP outperforms existing methods in structure reconstruction and generation tasks under various modeling metrics, offering a powerful tool for crystal structure design and optimization.

\end{abstract}

\section{Introduction}
\label{introduction}

Crystal structure prediction (CSP) methods play a crucial role in the discovery of novel materials and remain a persistent challenge in condensed matter physics and materials chemistry. With the advancement of first-principles calculations, such as Density Functional Theory (DFT)\cite{hohenberg1964inhomogeneous} and structure search algorithms, including simulated annealing \cite{kirkpatrick1983optimization} and genetic algorithms\cite{wu2013adaptive}, the precision and efficiency of CSP have improved significantly. These methods typically involve generating candidate crystal structures and performing DFT calculations to screen them. Notable examples include USPEX\cite{lyakhov2013new} and AIRSS\cite{pickard2011ab}. However, these approaches are computationally intensive and time-consuming. Traditional computationally driven CSP methods often lack effective learning priors and are generally limited to structural searches within a single system. Furthermore, due to the inherent randomness of the search space, they require substantial computational resources and are susceptible to getting trapped in local optima.

With the rapid advancement of deep learning technologies, CSP has progressed into a new phase of data-driven. Neural network models now substitute DFT calculations to predict material properties\cite{merchant2023scaling}, significantly enhancing the efficiency of structure searches. Generative models, including variational autoencoder\cite{kingma2013auto}, adversarial generative networks\cite{goodfellow2014generative}, and diffusion models\cite{ho2020denoising} with prior knowledge, are increasingly used in crystal structure generation. These models are capable of learning the feature space distribution of sample crystal structures and directly sampling from this space to generate crystal structures that adhere to the learned distribution. This approach significantly improves search efficiency compared to traditional CSP methods.

\begin{figure*}[ht]
\vskip 0.2in
\begin{center}
\centerline{\includegraphics[width=(\columnwidth*2)]{figures/summary.jpg}}
\caption{TransVAE-CSP, based on the Variational Autoencoder (VAE) paradigm, excels in both crystal structure reconstruction and ab initio generation tasks. Experimental results show exceptional performance across evaluation metrics. Compared to previous work\cite{xie2021crystal}, we have introduced innovations and conducted verifications in crystal structure representation and the encoder network.}
\label{introduce_fig}
\end{center}
\vskip -0.2in
\end{figure*}

CSP methods based on generative models have progressively evolved from utilizing composition information to directly constructing complex crystal structures. Early works, such as MatGAN\cite{dan2020generative} and Binded-VAE\cite{oubari:hal-04464893}, rely exclusively on crystal structure composition information for training, capturing implicit rules of chemical composition. However, these models serve primarily as reference models for structure generation and have a limited capacity to produce diverse structures. More advanced methods, including Composition-Conditioned Crystal GAN\cite{kim2020generative} and CCDCGAN\cite{long2021constrained}, have made significant progress in feature representation, but they still constrain the crystal structure space and are limited to specific compound groups. In contrast, works for complex systems, such as CDVAE\cite{xie2021crystal}, DiffCSP\cite{jiao2023crystal}, UniMat\cite{yang2024scalable}, and MatterGen\cite{zeni2025generative}, have achieved crystal structure generation without being restricted by representation space, aligning with the primary research goals in materials discovery. Therefore, optimizing existing model frameworks and exploring new approaches to continuously enhance both the efficiency and quality of structure generation represent key directions for advancing the application of artificial intelligence in materials science.

The development of CSP methods highlights that the accuracy of algorithms and models for crystal structure characterization is a critical factor influencing the stability and precision of generated structures. This accuracy remains a significant challenge in generative models for CSP. In particular, capturing the periodicity, symmetry, and other fundamental characteristics of crystal structures continues to be a major obstacle in CSP methodologies. To address these challenges, we propose a Transformer-Enhanced Variational Autoencoder (TransVAE-CSP) for crystal structure prediction, as shown in Fig. \ref{introduce_fig}. This model facilitates the accurate reconstruction and generation of crystal structures. The primary contributions of this work are as follows:
\begin{itemize}
    \item We developed an adaptive distance expansion method to characterize crystal structures and introduced an innovative hybrid function method to optimize a single distance expansion function. Our verification showed that this hybrid strategy can yield optimal results under different crystal sample environments;
    \item We proposed a Transformer network based on an equivariant dot product attention mechanism as encoder for Variational Autoencoder (VAE) model, enhancing its ability to learn the characteristics of crystal samples;
    \item Our work was evaluated using the carbon\_24, perov\_5 and mp\_20 datasets. The results demonstrated significant advantages in structure reconstruction and generation tasks under various modeling metrics. This research offers a powerful tool for crystal structure design and optimization.
\end{itemize}

\section{Related Work}
In this section, we will introduce the primary challenges encountered by CSP methods (Section \ref{profor}), commonly used crystal structure characterization techniques (Section \ref{cryrep}), encoding network models (Section \ref{equnet}) and crystal generation models (Section \ref{crygenmodel}).

\subsection{Problem Formulation}
\label{profor}
A crystal structure consists of atoms interacting with each other, where each atom is characterized by its element type and coordinates. The goal of learning its representation is to train an encoder that accurately maps the crystal structure into a representation within a sampling space. To preserve the properties of the crystal structure after transformations, the structure must maintain translation, rotation, inversion, and permutation invariance\cite{keating1966effect, lin2022general}. Among these, translation, rotation, and inversion operations form E(3) symmetry, while translation and rotation operations form SE(3) symmetry. However, current models often struggle to effectively capture key features of crystal structures like symmetry and periodicity, particularly global symmetry, which can result in generated crystal structures failing to retain correct physical properties. Effectively incorporating these symmetries into material structure representation and neural networks has emerged as a significant challenge in crystal structure prediction. This issue also represents a classic problem in structural encoding.

\subsection{Crystal Representation}
\label{cryrep}

The crystal structure representation methods form the foundation for studying the relationship between crystal properties and structures. Previous representation techniques, such as those based on simple matrices\cite{oubari:hal-04464893, dan2020generative, kim2020generative, long2021constrained}, are inadequate for representing complex systems. The advent of graph representation methods has significantly improved the ability of models to learn crystal properties\cite{xie2018crystal, schutt2018schnet, chen2019graph, chen2022universal, yuan2024tripartite}. Notably, the incorporation of information regarding three-body interactions has further improved the predictive power of these models.

\subsection{Equivariant Network}
\label{equnet}
To facilitate the representation of crystal structure symmetries in generative network models, e3nn \cite{mario_geiger_2022_6459381}, based on previous work\cite{Thomas2018TensorFN, weiler20183d}, is specifically designed to handle the symmetry of E(3). E3nn simplifies the process of constructing and training neural networks with these characteristics.

Equivariant networks\cite{kondor2018clebsch,unke2021se, tholke2022equivariant, brandstetter2022geometric} utilize geometric functions constructed from spherical harmonics and irreducible features to implement 3D rotation and translation equivariance, as proposed in Tensor Field Networks (TFN)\cite{Thomas2018TensorFN}. The SE(3) Transformer\cite{fuchs2020se} employs equivariant dot product (DP) attention\cite{vaswani2017attention} with linear messages. Equiformer \cite{liao2023equiformer} integrates MLP attention with non-linear messages and various types of support vectors, enhancing the model's expressiveness.

\subsection{Crystal Generative Model}
\label{crygenmodel}
The generative model paradigm has been extensively utilized to generate material structures, including VAEs\cite{oubari:hal-04464893, noh2019inverse, hoffmann2019data, Ren2020InverseDO, court20203, xie2021crystal}, GANs\cite{nouira2018crystalgan, dan2020generative, kim2020generative, long2021constrained}, and diffusion models\cite{jiao2023crystal, yang2024scalable, zeni2025generative}. Most of these studies focus on binary compounds\cite{noh2019inverse, long2021constrained}, ternary compounds\cite{nouira2018crystalgan, kim2020generative}, or simpler materials within the cubic system\cite{hoffmann2019data, court20203}. Xie et al.\cite{xie2021crystal} incorporated the diffusion concept into the decoder component of the VAE, addressing the challenge of predicting material coordinates, while also retaining the crystal reconstruction capabilities of the VAE. The diffusion model\cite{jiao2023crystal, yang2024scalable, zeni2025generative} facilitates applications in more complex systems, such as the Materials Project dataset\cite{jain2013commentary}, and is capable of performing conditional generation tasks.

\section{Methods}
In this section, we will provide a comprehensive introduction to the TransVAE-CSP that we developed. As illustrated in Fig. \ref{overview_fig}, the model consists of two primary components: training and generation. It uses well-established and highly accurate CDVAE. Building upon this work, we have implemented a new encoder network to train a high-quality learning model.

During the training process, TransVAE-CSP focuses primarily on optimization of the crystal structure representation and the encoder neural network. In Sections \ref{ade} and \ref{equdpan}, we will elucidate the design concepts and work principles of our proposed adaptive distance expansion method for crystal structure representation, as well as the Transformer network encoder that utilizes the equivariant dot product attention mechanism. We provide a detailed explanation of how to capture the periodicity and symmetry characteristics of crystal structures, thereby enabling effective encoding of the crystal structure. Finally, Section \ref{ooa} will present a comprehensive overview of the module design and workflow of TransVAE-CSP.

\begin{figure*}[ht]
\vskip 0.2in
\begin{center}
\centerline{\includegraphics[width=(\columnwidth*2)]{figures/overview.jpg}}
\caption{Overview of TransVAE-CSP. Training VAE model: Given $M=(A,N,X,L)$, it captures feature information through the \textbf{Embedding} layer and the equivariant attention encoder to obtain the latent space variable $z$, which is utilized as a condition to guide the output of both the \textbf{Predictor} and the denoising \textbf{Decoder}. The loss function comprises three parts:$\mathcal{L}_{Pred}, \mathcal{L}_{Dec}, \mathcal{L}_{KL}$. Generation: The variable $z$ is sampled from a multi-dimensional standard normal distribution $\mathcal{N}\sim(0,I)$ and ab initio generation is achieved by the \textbf{Predictor} and the denoising \textbf{Decoder} to produce a crystal structure that aligns with the feature space of the training samples. Note: Our work optimizes the algorithm based on CDVAE, thereby the framework refers to Xie et al.\cite{xie2021crystal}.}
\label{overview_fig}
\end{center}
\vskip -0.2in
\end{figure*}

\subsection{Adaptive Distance Expansion}
\label{ade}
The distance between atoms in a crystal structure directly influences its stability and physical properties. Employing mathematical methods or functions to describe the relationships between atomic distances is a common approach in the analysis of crystal structures within the field of machine learning. This includes techniques such as distance matrices, Fourier series expansions, and radial basis functions. In materials design, particularly in high-throughput screening and inverse material design, predicting material performance requires an efficient computational model. The radial basis function is particularly effective at capturing the complexities of nonlinear relationships, allowing it to flexibly manage intricate structural interactions. Furthermore, it can be utilized to establish a mapping between material structure and physical properties, facilitating material screening and optimization\cite{gasteiger_dimenetpp_2020, gasteiger2021gemnet, choudhary2021atomistic, chen2022universal, liao2023equiformer}.

The distance expansion adopts the Radial Basis Function (RBF) method\cite{buhmann2000radial}, which is expressed as $\phi(x)=\varphi(\|x-c\|)$. $x$ is the input vector, $c$ is the center point, $\|x-c\|$ represents the Euclidean distance from the input $x$ to the center point $c$. $\varphi$ is a function, usually selected as some smooth function, such as Gaussian function\cite{fornberg2011stable}, Bessel function\cite{javaran2011first}, etc.

\textbf{Gaussian RBF} Its function value decreases rapidly as the distance between the input $x$ and the center point $c$ increases. The Gaussian RBF exhibits characteristics of locality, smoothness, and nonlinear mapping. It transforms distances into a high-dimensional feature space, allowing the original low-dimensional input data to be mapped to a higher-dimensional feature space, thereby facilitating the embedding of distance features. For detailed information on Gaussian RBF, please refer to Appendix \ref{appendix:gaussian_rbf}

\textbf{Bessel RBF} Because Bessel RBF exhibits distinct oscillation and periodic characteristics compared to other radial basis functions, such as the Gaussian RBF, so it offer unique advantages in specific periodic data processing tasks. The function is typically constructed using Bessel function.For detailed information on Gaussian RBF, please refer to Appendix \ref{appendix:bessel_rbf}

\textbf{Hybrid RBF} To effectively capture the intricate relationships within crystal structures and enhance the model's capacity to accommodate these complexities, we propose a hybrid radial basis function method that integrates the Gaussian radial basis function with the Bessel radial basis function. This approach employs a function fusion algorithm based on weighted concatenation. To address the issue of differing orders of magnitude in the eigenvalues of the two functions, we implement a weight scaling algorithm. The formula for the hybrid radial basis function is expressed as: $H(x) = \Phi(x)\bigoplus k\cdot\Psi(x)$, where $k$ denotes the weight, $\Phi$ represents the Bessel RBF, $\bigoplus$signifies the concatenation operation, and $\Psi$ denotes the Gaussian RBF.

\textbf{Overall}, when training the model, we adaptively employ various radial basis functions for distance expansion across different data sets by adjusting the coefficients. We then compare the training results and select the radial basis function that performs best for each specific data set.

\begin{figure}[htbp]
  \centering
  \includegraphics[width=(\columnwidth*4/5)]{figures/equi_attention_2.jpg}
  \caption{The technique of equivariant dot product attention network. This is the core network of Transformer. “DTP” stands for depth-wise tensor product, $\bigoplus$ denotes addition, $\bigotimes$ denotes multiplication.$\sum$ denotes scatter operation.}
  \label{fig:equnet}
\end{figure}
\subsection{Equivariant Dot Product Attention Network}
\label{equdpan}
The encoder utilizes an adapted SE(3)-Transformer network \cite{fuchs2020se}, with the core structure being the equivariant dot product attention network \cite{liao2023equiformer}, as shown in Fig. \ref{fig:equnet}. Node \(i\) serves as the center, and the purpose of this network is to analyze the neighboring environmental information of the central node to obtain its higher-order features. \(X_i\) outputs the query \(q_i\) through a linear layer \cite{liao2023equiformer}. The features of \(X_i\) and its neighbor node \(X_j\) are combined after passing through linear layers to obtain the initial message:$x_{ij} = \text{Linear}_{\text{dist}}(x_{i}) + \text{Linear}_{\text{src}}(x_{j})$. Then, \(x_{ij}\) and its distance expansion \(\vec{r}_{ij}\) are processed via the DTP network \citep{howard2017mobilenets}:
\begin{equation}
\label{dtp}
f_{ij} = \text{Linear}\left( 
  x_{ij} \otimes_{\omega\left(\left\Vert \vec{r}_{ij} \right\Vert\right)}^{\text{DTP}} 
  \Bigl[ \text{SH}(\vec{r}_{ij}),\, \text{RBF}_{ij} \Bigr] 
\right)
\end{equation}
SH is the spherical harmonic embedding, \(\omega(\| \vec{r}_{ij} \|)\) is the weight of the Tensor Product. \(F_{ij}\) represents the fused features of the target node and the source node, and is used to derive the attention weights and messages. We split \(f_{ij}\) into two incompatible features, key \(k_{ij}\) and value \(v_{ij}\). Then, we perform a scaled dot product between \(q_i\) and \(k_{ij}\) \cite{NIPS2017_3f5ee243} to obtain the attention weights. Finally, the weights and messages are multiplied, and the higher-order feature messages of all target nodes are obtained through Reshape and aggregation functions, as well as linear layers:
\begin{equation}
y_i = \text{Linear}\left( \sum_{\text{scatter}} a_{ij} \otimes v_{ij} \right)
\end{equation}
The scatter operation performs a summation aggregation on the messages calculated for the same target atom and all its neighbors according to the index, in order to obtain the environmental feature message for each atom.



\subsection{Overall of Architecture}
\label{ooa}
Figure \ref{overview_fig} shows the framework of TransVAE-CSP. Building on the work of CDVAE \cite{xie2021crystal}, our research focuses on optimizing crystal structure representation and the encoder network.

\textbf{Representation} Any crystal structure is an infinite repetition of a unit cell in 3D space. A unit cell can be represented as follows, where \( A \) represents the type of atoms, \( N \) represents the number of atoms, \( X \) represents the atomic coordinates, and \( L \) represents the lattice parameters.If the matrix \( N \) contains elements in \( \mathbb{R} \), let \( A = (a_{0}, a_{1}, \ldots, a_{n-1}) \in \mathbb{E}^{N} \) represent the set of all elements in the matrix. Let \( X = (x_{0}, x_{1}, \ldots, x_{m-1}) \in \mathbb{R}^{N \times 3} \) and \( L = (a, b, c, d, e) \in \mathbb{R}^{6} \).

\textbf{Embedding} This module is adapted from\cite{liao2023equiformer}, as shown in Fig. \ref{fig:embedding}, and consists of Atom Embedding and Edge Degree Embedding. Atom Embedding uses a linear layer to one-hot encode the atom type. Edge Degree Embedding first expands the RBF features (marked in red) of the constant vector, interaction information, and atom pair distance. It then encodes the local geometric message through two linear layers and an intermediate DTP layer, followed by aggregation to encode the message. The form of the DTP layer is the same as in formula \ref{dtp}. Finally, the aggregated features are scaled by dividing by the square root of the average degree in the training set (the calculation method is provided in Appendix \ref{appendix:gvfd}) to ensure that the standard deviation of the aggregated features is close to 1. After adding the Atom Embedding and Edge Degree Embedding together, the feature embedding of the final node is obtained.
\begin{figure}[htbp]
  \centering
  \includegraphics[width=\columnwidth*4/5]{figures/embedding.jpg}
  \caption{Embedding block. Compared to pervious work\cite{liao2023equiformer}, our adaptation work is highlighted in red, and \textbf{RBF} refers to the feature vector obtained after the radial basis function is expanded based on the distance.}
  \label{fig:embedding}
\end{figure}

\textbf{Encoder} The encoder employs a transformer network based on the equivariant dot product attention mechanism (refer to Section \ref{equdpan}). The final output is generated through aggregation operations to derive the intermediate variables of the VAE, which are subsequently used as inputs to the mean and linear layers.

\textbf{Decoder} The decoder consists of two parts: multiple independent MLP networks that predict the atomic type components \( A_c \), the number of atoms \( N \), and the lattice parameters \( L \). The prediction of atomic coordinates and types is based on the diffusion concept, incorporating random noise and employing the GemNet-T network\cite{gasteiger2021gemnet} to predict the atomic type and coordinate noise of the input structure.

\textbf{Training and Generation} In the training phase, the encoder generates the intermediate variable \( z \), which is used as the condition for the denoising encoder and as input for the predictor. The loss function comprises the aggregation loss from the predictor, the loss from the denoising network, and the KL divergence loss (see Appendix \ref{appendix:loss_fuc} for details). In the generation phase, the intermediate variable \( Z \) is sampled from the standard normal distribution, and the structural components \( A_c \), the number of atoms \( N \), and the lattice parameters \( L \) are generated through the predictor. The coordinate information and atomic types are randomly generated as the initial structure. Given \( z \) as the condition, annealed Langevin dynamics\cite{song2019generative} is performed through the denoising decoder to continuously update the atomic types and coordinates.

\section{Results}

We evaluated the effectiveness of the fusion model in various tasks related to generating crystal structures and used the test standards from Xie et al. \cite{xie2021crystal} to compare the results. We assessed our work in \ref{cryrecon} crystal structure reconstruction and \ref{crystrgen} crystal structure generation. Additionally, we demonstrate in \ref{comprbf} that the radial basis function exhibits varying adaptability across different datasets. The datasets employed in the experiment consist of three high-quality datasets of differing complexities. Perov-5 \cite{castelli2012new, castelli2012computational} focuses on perovskite materials and contains 18,928 perovskite materials with similar structures but different elemental compositions, including 56 elements, with each structure containing 5 atoms per unit cell. Carbon-24 \cite{pickard2020airss} focuses on carbon-based materials and contains 10,153 carbon materials, which consist solely of carbon elements, with each unit cell containing between 6 and 24 atoms. MP-20 selects 45,231 stable inorganic materials from Materials Project \cite{jain2013commentary}, which includes most experimentally generated materials with up to 20 atoms per unit cell. We applied a 60-20-20 split in accordance with Xie et al. \cite{xie2021crystal}.

\begin{table*}[t]
\caption{Reconstruction Performance of Different Models.}
\label{tab:recon_perform}
\vskip 0.15in
\begin{center}
\begin{small}
\begin{sc}
\begin{tabular}{lcccccc}
        \toprule
        \textbf{Model} & \multicolumn{2}{c}{\textbf{Perov\_5}} & \multicolumn{2}{c}{\textbf{Carbon\_24}} & \multicolumn{2}{c}{\textbf{MP\_20}} \\
        & \textbf{Match Rate} & \textbf{RMSE} & \textbf{Match Rate} & \textbf{RMSE} & \textbf{Match Rate} & \textbf{RMSE} \\
        \midrule
        FTCP & \textbf{99.34} & 0.0259 & 62.28 & 0.2563 & 69.89 & 0.1593 \\
        Cond-DFC-VAE & 51.65 & 0.0217 & - &  - & - & - \\
        CDVAE & 97.52 & 0.0156 & 55.22 & 0.1251 & 45.43 & \textbf{0.0356} \\
        TransVAE-CSP & 98.19 & \textbf{0.0092} & \textbf{80.75} & \textbf{0.0938} & \textbf{71.14} & 0.0377 \\
    \bottomrule
    \end{tabular}
\end{sc}
\end{small}
\end{center}
\vskip -0.1in
\end{table*}

\begin{table*}[t]
\caption{Real Performance of TransVAE-CSP reconstruction structures. }
\label{recon_compar}
\vskip 0in
\begin{center}
\begin{small}
\begin{sc}
\begin{tabular}{ccccccc}
\toprule
 & \multicolumn{2}{c}{Perov\_5} & \multicolumn{2}{c}{Carbon\_24} & \multicolumn{2}{c}{MP\_20} \\
\midrule
Ground Truth & \includegraphics[width=2cm]{figures/perov_1_t.jpg} & \includegraphics[width=2cm]{figures/perov_2_t.jpg} & \includegraphics[width=2cm]{figures/carbon_1_t.jpg} & \includegraphics[width=2cm]{figures/carbon_2_t.jpg} & \includegraphics[width=2cm]{figures/mp_1_t.jpg} & \includegraphics[width=2cm]{figures/mp_2_t.jpg} \\
Prediction   & \includegraphics[width=2cm]{figures/perov_1_p.jpg} & \includegraphics[width=2cm]{figures/perov_2_p.jpg} & \includegraphics[width=2cm]{figures/carbon_1_p.jpg} & \includegraphics[width=2cm]{figures/carbon_2_p.jpg} & \includegraphics[width=2cm]{figures/mp_1_p.jpg} & \includegraphics[width=2cm]{figures/mp_2_p.jpg}\\

\bottomrule
\end{tabular}
\end{sc}
\end{small}
\end{center}
\vskip -0.1in
\end{table*}

\subsection{Crystal Structure Reconstruction}
\label{cryrecon}
\subsubsection{Baseline}
In the crystal reconstruction task, we compared our model with three baselines from \cite{xie2021crystal}: FTCP \cite{Ren2022}, Cond-DFC-VAE \cite{court20203}, and CDVAE \cite{xie2021crystal}. CDVAE encodes the atoms, atomic pair distances, and angles of three atoms in the crystal structure. It predicts the composition, lattice, and number of atoms from the latent variable space \( z \), initializes a crystal structure with random atomic types and coordinates, uses the Langevin dynamics method to denoise the atomic types and coordinates, and finally generates a new structure.

\subsubsection{Metrics}
We follow the common practice \cite{xie2021crystal,jiao2023crystal} of predicting crystal structures by testing the latent intermediate variables corresponding to all structures in the test set. We use the \texttt{StructureMatcher} class from \texttt{Pymatgen}, setting the thresholds to \texttt{stol}=0.5, \texttt{angle\_tol}=10, and \texttt{ltol}=0.3 to measure whether the real structure matches the reconstructed structure, thus calculating the match rate to evaluate the reconstruction performance. The RMSE is calculated as the average over the successfully matched reconstructed structures and real structures, normalized by \(\sqrt[3]{V/N}\), where \(V\) is the volume of the structure's unit cell and \(N\) is the number of atoms in the unit cell.


\begin{table*}[th]
\caption{Generation Performace}
\label{generation_performace}
\vskip 0.15in
\begin{center}
\begin{small}
\begin{sc}
\begin{tabular}{ccccccccc}
\toprule
 \multirow{2}*{\textbf{Dataset}} & \multirow{2}*{\textbf{Model}} & \multicolumn{2}{c}{\textbf{Validity}(\%) $\uparrow$} & \multicolumn{2}{c}{\textbf{Coverage}(\%) $\uparrow$} & \multicolumn{3}{c}{\textbf{Property statistics} $\downarrow$} \\
 ~ & ~ & Stru. & Comp. & Cov-R & Cov-P & $d_{density}$ & $d_E$ & $d_{elem}$ \\
\midrule
\multirow{7}*{perov\_5} & FTCP & 0.24 & 54.24 & 0.0 & 0.0 & 10.27& 156 & 0.6297 \\
& Cond-DFC-VAE & 73.6&	82.95&73.92&10.13&	2.268&	4.111	&0.8373 \\
& G-SchNet& 99.92&	98.79&	0.18&	0.23&	1.625&	4.746&	0.0368 \\
&P-G-SchNet &79.63&	99.13&	0.37&	0.25&	0.2755&	1.388&	0.4552\\
&CDVAE &100	&98.59	&99.45&	\textbf{98.46}&	0.1258&	0.0264&	0.0628 \\
& DiffCSP&100&	\textbf{98.85}&	\textbf{99.74}&	98.27&	0.111&	 \textbf{0.0263}&	\textbf{0.0128} \\
& TransVAE-CSP &\textbf{100}	&98.82&	99.52&	98.33&	\textbf{0.1082}&	0.0311&	0.0992 \\
\hline
\multirow{6}*{carbon\_24} & FTCP &0.08&	-&	0.0&	0.0&	5.206&	19.05&	- \\
& G-SchNet&99.94&	-&	0.0&	0.0&	0.9427&	1.32&	- \\
&P-G-SchNet &48.39&	-&	0.0&	0.0&	1.533&	134.7&	- \\
&CDVAE & \textbf{100} &	- &	99.8&	83.08&	0.1407&	0.285&	- \\
& DiffCSP&\textbf{100} &	-	& 99.9 &	\textbf{97.27}&	\textbf{0.0805}&	\textbf{0.082}&	- \\
& TransVAE-CSP &99.9& - &\textbf{100}	&78.62	&0.1636	&1.744	&- \\
\hline
\multirow{6}*{mp\_20} & FTCP &1.55 &	48.37&	4.72&	0.09	&23.71	&160.9&	0.7363 \\
& G-SchNet&99.65&	75.96&	38.33&	99.57&	3.034&	42.09&	0.6411 \\
&P-G-SchNet &77.51&	76.4&	41.93&	99.74&	4.04&	2.448&	0.6234 \\
&CDVAE & 100&	86.7&	99.15&	99.49&	0.6875&	0.2778&	1.432\\
& DiffCSP&100&	83.25&	\textbf{99.71}&	\textbf{99.76}&	0.3502&	0.1247&	\textbf{0.3398} \\
& TransVAE-CSP & \textbf{100}&	\textbf{87.97}&	99.67&	99.23&	\textbf{0.1694}&\textbf{0.1182}&	0.7348 \\
\bottomrule
\end{tabular}
\end{sc}
\end{small}
\end{center}
\vskip -0.1in
\end{table*}
\subsubsection{Results}
The indicators of the reconstructed structure are presented in Table \ref{tab:recon_perform}. Unlike previous studies, we adaptively employ different radial basis functions (RBFs) for distance expansion across various data sets by adjusting coefficients during model training. We compare the training outcomes and automatically select the radial basis function that demonstrates the best performance for each data set. The optimal results are achieved using Gaussian, Hybrid, and Bessel RBFs for MP\_20, Carbon\_24, and Perov\_5, respectively. This selection is influenced by the structural complexity of the data set, the long-range interactions between atoms, and the diversity of samples, as well as the adaptability of RBFs within the atomic pair distance distribution space of different data sets. In terms of indicator performance, our model exhibits a slightly lower Match Rate than FTCP for Perov\_5 and a marginally higher RMSE than CDVAE for MP\_20. Other indicators surpass the baseline models to varying extents. For detailed information on model training, please refer to Appendix \ref{appendix:train_para}.


The comparison between the reconstructed structure and the real structure is shown in Table \ref{recon_compar}. The observation direction of each pair of structures remains the same. Considering that our model maintains the invariance of 3D space under translation, rotation, and periodicity, the visualization results of atoms at corresponding symmetric points in the reconstructed and real structures may differ due to their different origin positions. However, by applying lattice symmetry operations (such as translation), the alignment of atomic positions within the unit cell can be achieved.







\subsection{Crystal Structure Generation}
\label{crystrgen}
\subsubsection{Baseline}
For structure generation, we adopt ab initio crystal generation. We compare our model with six baselines \cite{jiao2023crystal}: FTCP \cite{Ren2022}, Cond-DFC-VAE \cite{court20203}, G-SchNet, P-G-SchNet \cite{gebauer2019symmetry}, CDVAE \cite{xie2021crystal}, and DiffCSP \cite{jiao2023crystal}. DiffCSP requires the input of specified atom types and the number of atoms to complete the generation task. For consistency, we sample the number of atoms from the pre-computed distribution in the training set \cite{hoogeboom2022equivariant}, allowing DiffCSP to achieve the same ab initio crystal generation as the other works.

\subsubsection{Metrics}
Similar to \cite{xie2021crystal}, we employ three metrics to evaluate generation performance: Validity (including Structure Validity and Composition Validity), Coverage (comprising Coverage Recall and Coverage Precision), and Property Statistics. These metrics assess the validity of the predicted crystal, the similarity between the test set and the generated samples, and the distribution distance of property calculations related to density, formation energy, and the number of elements. For detailed information on the metrics, please refer to Appendix \ref{appendix:metric}.

\subsubsection{Results}
The results of the indicators are shown in Table \ref{generation_performace}. Our model performs well across all these indicators. On the perov\_5 dataset, although our model does not perform optimally in terms of coverage, COV-R and COV-P are in the middle range of the corresponding indicator values of the two optimal baseline models, and our model achieves the best performance in terms of density distribution distance. On the carbon\_24 dataset, our model achieves 100\% in the COV-R indicator, but COV-P does not perform as well. On the mp\_20 dataset, the composition validity, density distribution distance, and energy distribution distance indicators exceed those of all baselines, and the other indicators also achieve excellent performance.

The results show that our work performs best on the mp\_20 dataset, which may be related to the authenticity of the structures in the mp\_20 dataset. In the perov\_5 and carbon\_24 datasets, most structures are thermodynamically unstable, which is also a factor affecting the model's performance.

\subsection{Comparison of RBF Strategies}
\label{comprbf}
We explored the impact of radial basis function expansion methods for encoding interatomic distances in crystal structure characterization on the model. Here, we conducted comparative tests on radial basis functions based on Gaussian functions, Bessel functions, and hybrid functions of the two on the mp\_20, carbon\_24, and perov\_5 datasets. The test indicator is the convergence of the loss value over 200 iterations on the training set and the validation set during training. The results are shown in Appendix \ref{appendix:rbf}. On the mp\_20, carbon\_24, and perov\_5 datasets, the model optimization effects of the Gaussian, hybrid, and Bessel radial basis function distance expansion methods are the best.


\section{Conclusion and Outlook}
\label{conclusion}
In this work, we propose a crystal structure generation method based on the Transformer-Enhanced Variational Autoencoder for Crystal Structure Prediction (TransVAE-CSP). Through experiments, we demonstrate that the equivariant dot-product attention Transformer network outperforms traditional Graph Convolutional Networks (GCNs) in understanding the chemical and physical properties of crystal structures. We innovatively introduce an adaptive distance expansion method, and the experimental results further validate that this approach achieves the desired objectives, providing a new optimization perspective for crystal structure representation. Moreover, compared to diffusion models, the Variational Autoencoder (VAE) shows unique advantages in structure reconstruction and distribution consistency verification, highlighting its potential in crystal structure generation.

Future work will optimize the VAE-based crystal structure generation model to match the diffusion models, while enhancing the framework for efficient prediction of structures with specified elemental compositions. This advancement will drive crystal structure prediction technologies and enable innovative design of high-performance materials.

%restatable

\section*{Impact Statement}
This paper presents work whose goal is to advance the field of 
Machine Learning. There are many potential societal consequences 
of our work, none which we feel must be specifically highlighted here.


% In the unusual situation where you want a paper to appear in the
% references without citing it in the main text, use \nocite
\nocite{langley00}

\bibliography{main-reference}
\bibliographystyle{icml2025}


%%%%%%%%%%%%%%%%%%%%%%%%%%%%%%%%%%%%%%%%%%%%%%%%%%%%%%%%%%%%%%%%%%%%%%%%%%%%%%%
%%%%%%%%%%%%%%%%%%%%%%%%%%%%%%%%%%%%%%%%%%%%%%%%%%%%%%%%%%%%%%%%%%%%%%%%%%%%%%%
% APPENDIX
%%%%%%%%%%%%%%%%%%%%%%%%%%%%%%%%%%%%%%%%%%%%%%%%%%%%%%%%%%%%%%%%%%%%%%%%%%%%%%%
%%%%%%%%%%%%%%%%%%%%%%%%%%%%%%%%%%%%%%%%%%%%%%%%%%%%%%%%%%%%%%%%%%%%%%%%%%%%%%%
\newpage
\appendix
% \onecolumn
\section{Radial Basis Function (RBF)}
\subsection{Gaussian RBF}
\label{appendix:gaussian_rbf}
The gaussian RBF formula is followed:
\begin{equation}
    \label{gaussian_base_equation}
    \phi_i(x)=exp(-\frac{\|x-c_i\|^2}{2\sigma_i^2})
\end{equation}
where $c_i$ is the center point, $\sigma_i$ is the breadth of basis functions. The Gaussian RBF expression for mapping distances into N-dimensional feature spaces is represented by Formula \ref{gaussian_expand_equation}, which requires initializing $N$ parameters $ci$ and $\sigma_i$.


\begin{equation}
    \label{gaussian_expand_equation}
    \Phi(x)=[\phi_1(x), \phi_1(x), \phi_2(x), ..., \phi_N(x)]
\end{equation}
\subsection{Bessel RBF}
\label{appendix:bessel_rbf}
Bessel RBF is typically constructed using Bessel function. The formula is followed as: $\phi(x)=J_v(\|x-c\|)$, where J(r) represents the v-th order Bessel function, where v is typically a non-negative real number. The variable r is defined as the Euclidean distance between the input vector x and the center point c, expressed as r = ‖x - c‖. The distance is transformed into a high-dimensional feature space using the following formulation.
\begin{equation}
\begin{split}
    \Phi(x)=[
    & J_v(\|x-c_1\|)\cos{(\omega_1\|x-c_1\|)}, \\
    & J_v(\|x-c_2\|)\cos{(\omega_2\|x-c_2\|)}, \\
    & ..., \\
    & J_v(\|x-c_n\|)\cos{(\omega_n\|x-c_n\|)}]
\end{split}
\end{equation}
Among the formula, $\omega$ denotes the frequency parameter, which is utilized to regulate the oscillation frequency of the function, while $\cos$ represents the cosine function.

\section{Experiment Details}
\subsection{Global Variables for Datasets}
\label{appendix:gvfd}
The encoder network must provide several statistics from the sample dataset: the maximum number of nodes, the average number of nodes, and the average coordination number. These statistics are essential for performing more accurate scaling operations across different sample spaces within the network. The maximum number of nodes refers to the highest number of atoms present in the unit cell across all crystal structures in the statistical dataset. The average number of nodes is determined by calculating the total number of nodes for all structures in the sample dataset, expressed mathematically as $\frac{1}{N}\sum_{i=1}^N NUM_{node}$, where $NUM_{node}$ denotes the number node of a structure in a unit cell, $N$ denotes the total number of all structures. The coordination number of each atom is computed using the \texttt{CrystalNN} class from the \texttt{Pymatgen} library, which is a tool designed for analyzing atomic neighbors in crystalline materials. The method `get\_cn()` is a crucial function of the \texttt{CrystalNN} class, utilized to calculate the coordination number (CN, also called \textbf{degree}) of a specific atom. The coordination number represents the number of directly connected neighboring atoms surrounding a given atom and is typically used to describe the local environment of that atom within the crystal structure. The formula for calculating the average coordination number is
\begin{equation}
    AVG_{degree}=\frac{1}{N}\sum_{i=1}^N (\frac{1}{M}\sum_{j=1}^M CN_j),
\end{equation}
where $N$ denotes the total number of all structures, $M$ denotes the total number of node in a structure, $CN_j$ denotes the coordination number of a node in a structure.
\subsection{Training Parameters}
\label{appendix:train_para}
The hyperparameters of model training are shown in Table \ref{tab:hyperparameter}.
\begin{table}[h]
\caption{Hyper Parameters of the mdoels.}
\label{tab:hyperparameter}
\vskip 0.15in
\begin{center}
\begin{small}
\begin{sc}
\begin{tabular}{cccc}
        \toprule
        Parameters & Perov\_5 & Carbon\_24 &MP\_20 \\
        \midrule
        Epoch & 3500 &4000 & 1500 \\
        RBF & Bessel & Hybrid & Gaussian \\
        Cutoff & 6 & 6 & 10 \\
        Max Neighbors & 20  & 20  & 50 \\
        Latent Dims & 64& 64  & 64 \\
        lr & 0.0001 & 0.0001 & 0.0001 \\
    \bottomrule
    \end{tabular}
\end{sc}
\end{small}
\end{center}
\vskip -0.1in
\end{table}
\subsection{Loss Function}
\label{appendix:loss_fuc}
The loss function is analogous to that used in CDVAE and can be expressed as follows:
\begin{equation}
\begin{split}
    \mathcal{L}
    & =\mathcal{L}_{Pred}+\mathcal{L}_{Dec}+\mathcal{L}_{KL} \\
    & =\lambda_{A_c}\mathcal{L}_{A_c}+\lambda_{N}\mathcal{L}_{N}+\lambda_{L}\mathcal{L}_{L}+\\
& \space\space\space\space\space    \lambda_{A}\mathcal{L}_{A}+\lambda_{X}\mathcal{L}_{X}+\beta\mathcal{L}_{KL},
\end{split}
\end{equation}
For all three datasets, we use $\lambda_{A_c}=1$, $\lambda_{L}= 10$, $\lambda_{N}=1$, $\lambda_{X}=10$, $\lambda_{A} = 1$. For Perov\_5, MP\_20, we use $\beta=0.01$, and for Carbon\_24, we use $\beta=0.03$.
\subsection{RBF Comparison Results}
\label{appendix:rbf}
On each of the three datasets—Perov\_5, Carbon\_24, and MP\_20—three models with identical parameters are trained, differing only in the distance expansion function. The control variables include the Gaussian RBF, Bessel RBF, and Hybrid RBF. The convergence curves of the training loss are analyzed, and the results are presented in Fig.\ref{rbf_comp}.

\begin{figure}[ht]
\vskip 0.1in
\begin{center}
\centerline{\includegraphics[width=\columnwidth]{figures/rbf_comp.jpg}}
\caption{The Comparison of different RBF on Perov\_5, Carbon\_24, MP\_20. Figure a) illustrates the curve generated by training on Perov\_5. It is evident that the model utilizing the Bessel RBF exhibits the most effective training results. Similarly, b) Carbon\_24 - Hybrid RBF, c) MP\_20 - Gaussian RBF.}
\label{rbf_comp}
\end{center}
\vskip -0.1in
\end{figure}

\section{Metrics}
\label{appendix:metric}
\subsection{Validity}
Validity indicators encompass structural validity and component validity. The criterion for structural validity stipulates that as long as the shortest distance between any pair of atoms in the crystal structure exceeds 0.5 \AA, the structure is considered valid. The criterion for component validity requires that the total charge calculated by SMACT\cite{davies2019smact} is neutral, indicating that the combination is valid.
\subsection{Coverage}
Coverage metrics encompass recall and precision. For each true crystal, determine the minimum structural distance $D_{struct}$ and composition distance $D_{comp}$ to all generated crystals. Similarly, for each generated crystal, calculate the minimum structural distance and composition distance to all true crystals. Recall is defined as the proportion of minimum structural and composition distances that fall below specified thresholds - $T_{struct}$ and $T_{comp}$, reflecting the ability of the generated crystals to accurately represent the true crystals in terms of structure and composition. Precision, on the other hand, is the proportion of minimum structural and composition distances that are below the same thresholds, indicating the accuracy of the generated crystals in relation to the true crystals' structure and composition. The detailed settings of thresholds are seen in Table \ref{tab:thresholds}.
\begin{table}[h]
\caption{Coverage Thresholds of various datasets.}
\label{tab:thresholds}
\vskip 0.15in
\begin{center}
\begin{small}
\begin{sc}
\begin{tabular}{ccc}
        \toprule
        Datasets & Structure &  Composition\\
        \midrule
        Perov\_5 & 0.2 & 4.0 \\
        Carbon\_24 & 0.2 & 4.0 \\
        MP\_20 & 0.4 & 10.0 \\
    \bottomrule
    \end{tabular}
\end{sc}
\end{small}
\end{center}
\vskip -0.1in
\end{table}

\subsection{Property statistics}
Property statistics calculate the Earth Mover's Distance (EMD) between the generated material and the ground truth material property distributions, which include density ($\rho$, unit \texttt{$g/cm^{3}$}), energy predicted by an independent Graph Neural Network (GNN) ($E$, unit \texttt{$eV/atom$})\cite{xie2021crystal}, and the number of unique elements.

%%%%%%%%%%%%%%%%%%%%%%%%%%%%%%%%%%%%%%%%%%%%%%%%%%%%%%%%%%%%%%%%%%%%%%%%%%%%%%%
%%%%%%%%%%%%%%%%%%%%%%%%%%%%%%%%%%%%%%%%%%%%%%%%%%%%%%%%%%%%%%%%%%%%%%%%%%%%%%%


\end{document}


% This document was modified from the file originally made available by
% Pat Langley and Andrea Danyluk for ICML-2K. This version was created
% by Iain Murray in 2018, and modified by Alexandre Bouchard in
% 2019 and 2021 and by Csaba Szepesvari, Gang Niu and Sivan Sabato in 2022.
% Modified again in 2023 and 2024 by Sivan Sabato and Jonathan Scarlett.
% Previous contributors include Dan Roy, Lise Getoor and Tobias
% Scheffer, which was slightly modified from the 2010 version by
% Thorsten Joachims & Johannes Fuernkranz, slightly modified from the
% 2009 version by Kiri Wagstaff and Sam Roweis's 2008 version, which is
% slightly modified from Prasad Tadepalli's 2007 version which is a
% lightly changed version of the previous year's version by Andrew
% Moore, which was in turn edited from those of Kristian Kersting and
% Codrina Lauth. Alex Smola contributed to the algorithmic style files.


% \section*{Acknowledgements}
%\clearpage
\section*{Impact Statement}
Adversarial attacks on LLMs can have considerable consequences in real-world applications. Still, as machine-learning robustness has been an unsolved research problem for the last decade, we believe that the best way to approach this problem is through culminating awareness. Currently, it seems unlikely that the robustness issue can be completely resolved through technical means. Thus, making people aware of the harmful use cases and limitations of these models appears to be necessary to avoid irresponsible deployment of such models for critical applications and to reduce the harm malicious actors can cause. Moreover, this work does not introduce new technical innovations. Instead, it addresses impediments in robustness research and explores potential solutions to accelerate research progress.
\clearpage

\bibliography{references}
\bibliographystyle{icml2025}

% APPENDIX
\newpage
\appendix
\onecolumn

\setlength{\myboxwidth}{13cm}
%\subsection{Lloyd-Max Algorithm}
\label{subsec:Lloyd-Max}
For a given quantization bitwidth $B$ and an operand $\bm{X}$, the Lloyd-Max algorithm finds $2^B$ quantization levels $\{\hat{x}_i\}_{i=1}^{2^B}$ such that quantizing $\bm{X}$ by rounding each scalar in $\bm{X}$ to the nearest quantization level minimizes the quantization MSE. 

The algorithm starts with an initial guess of quantization levels and then iteratively computes quantization thresholds $\{\tau_i\}_{i=1}^{2^B-1}$ and updates quantization levels $\{\hat{x}_i\}_{i=1}^{2^B}$. Specifically, at iteration $n$, thresholds are set to the midpoints of the previous iteration's levels:
\begin{align*}
    \tau_i^{(n)}=\frac{\hat{x}_i^{(n-1)}+\hat{x}_{i+1}^{(n-1)}}2 \text{ for } i=1\ldots 2^B-1
\end{align*}
Subsequently, the quantization levels are re-computed as conditional means of the data regions defined by the new thresholds:
\begin{align*}
    \hat{x}_i^{(n)}=\mathbb{E}\left[ \bm{X} \big| \bm{X}\in [\tau_{i-1}^{(n)},\tau_i^{(n)}] \right] \text{ for } i=1\ldots 2^B
\end{align*}
where to satisfy boundary conditions we have $\tau_0=-\infty$ and $\tau_{2^B}=\infty$. The algorithm iterates the above steps until convergence.

Figure \ref{fig:lm_quant} compares the quantization levels of a $7$-bit floating point (E3M3) quantizer (left) to a $7$-bit Lloyd-Max quantizer (right) when quantizing a layer of weights from the GPT3-126M model at a per-tensor granularity. As shown, the Lloyd-Max quantizer achieves substantially lower quantization MSE. Further, Table \ref{tab:FP7_vs_LM7} shows the superior perplexity achieved by Lloyd-Max quantizers for bitwidths of $7$, $6$ and $5$. The difference between the quantizers is clear at 5 bits, where per-tensor FP quantization incurs a drastic and unacceptable increase in perplexity, while Lloyd-Max quantization incurs a much smaller increase. Nevertheless, we note that even the optimal Lloyd-Max quantizer incurs a notable ($\sim 1.5$) increase in perplexity due to the coarse granularity of quantization. 

\begin{figure}[h]
  \centering
  \includegraphics[width=0.7\linewidth]{sections/figures/LM7_FP7.pdf}
  \caption{\small Quantization levels and the corresponding quantization MSE of Floating Point (left) vs Lloyd-Max (right) Quantizers for a layer of weights in the GPT3-126M model.}
  \label{fig:lm_quant}
\end{figure}

\begin{table}[h]\scriptsize
\begin{center}
\caption{\label{tab:FP7_vs_LM7} \small Comparing perplexity (lower is better) achieved by floating point quantizers and Lloyd-Max quantizers on a GPT3-126M model for the Wikitext-103 dataset.}
\begin{tabular}{c|cc|c}
\hline
 \multirow{2}{*}{\textbf{Bitwidth}} & \multicolumn{2}{|c|}{\textbf{Floating-Point Quantizer}} & \textbf{Lloyd-Max Quantizer} \\
 & Best Format & Wikitext-103 Perplexity & Wikitext-103 Perplexity \\
\hline
7 & E3M3 & 18.32 & 18.27 \\
6 & E3M2 & 19.07 & 18.51 \\
5 & E4M0 & 43.89 & 19.71 \\
\hline
\end{tabular}
\end{center}
\end{table}

\subsection{Proof of Local Optimality of LO-BCQ}
\label{subsec:lobcq_opt_proof}
For a given block $\bm{b}_j$, the quantization MSE during LO-BCQ can be empirically evaluated as $\frac{1}{L_b}\lVert \bm{b}_j- \bm{\hat{b}}_j\rVert^2_2$ where $\bm{\hat{b}}_j$ is computed from equation (\ref{eq:clustered_quantization_definition}) as $C_{f(\bm{b}_j)}(\bm{b}_j)$. Further, for a given block cluster $\mathcal{B}_i$, we compute the quantization MSE as $\frac{1}{|\mathcal{B}_{i}|}\sum_{\bm{b} \in \mathcal{B}_{i}} \frac{1}{L_b}\lVert \bm{b}- C_i^{(n)}(\bm{b})\rVert^2_2$. Therefore, at the end of iteration $n$, we evaluate the overall quantization MSE $J^{(n)}$ for a given operand $\bm{X}$ composed of $N_c$ block clusters as:
\begin{align*}
    \label{eq:mse_iter_n}
    J^{(n)} = \frac{1}{N_c} \sum_{i=1}^{N_c} \frac{1}{|\mathcal{B}_{i}^{(n)}|}\sum_{\bm{v} \in \mathcal{B}_{i}^{(n)}} \frac{1}{L_b}\lVert \bm{b}- B_i^{(n)}(\bm{b})\rVert^2_2
\end{align*}

At the end of iteration $n$, the codebooks are updated from $\mathcal{C}^{(n-1)}$ to $\mathcal{C}^{(n)}$. However, the mapping of a given vector $\bm{b}_j$ to quantizers $\mathcal{C}^{(n)}$ remains as  $f^{(n)}(\bm{b}_j)$. At the next iteration, during the vector clustering step, $f^{(n+1)}(\bm{b}_j)$ finds new mapping of $\bm{b}_j$ to updated codebooks $\mathcal{C}^{(n)}$ such that the quantization MSE over the candidate codebooks is minimized. Therefore, we obtain the following result for $\bm{b}_j$:
\begin{align*}
\frac{1}{L_b}\lVert \bm{b}_j - C_{f^{(n+1)}(\bm{b}_j)}^{(n)}(\bm{b}_j)\rVert^2_2 \le \frac{1}{L_b}\lVert \bm{b}_j - C_{f^{(n)}(\bm{b}_j)}^{(n)}(\bm{b}_j)\rVert^2_2
\end{align*}

That is, quantizing $\bm{b}_j$ at the end of the block clustering step of iteration $n+1$ results in lower quantization MSE compared to quantizing at the end of iteration $n$. Since this is true for all $\bm{b} \in \bm{X}$, we assert the following:
\begin{equation}
\begin{split}
\label{eq:mse_ineq_1}
    \tilde{J}^{(n+1)} &= \frac{1}{N_c} \sum_{i=1}^{N_c} \frac{1}{|\mathcal{B}_{i}^{(n+1)}|}\sum_{\bm{b} \in \mathcal{B}_{i}^{(n+1)}} \frac{1}{L_b}\lVert \bm{b} - C_i^{(n)}(b)\rVert^2_2 \le J^{(n)}
\end{split}
\end{equation}
where $\tilde{J}^{(n+1)}$ is the the quantization MSE after the vector clustering step at iteration $n+1$.

Next, during the codebook update step (\ref{eq:quantizers_update}) at iteration $n+1$, the per-cluster codebooks $\mathcal{C}^{(n)}$ are updated to $\mathcal{C}^{(n+1)}$ by invoking the Lloyd-Max algorithm \citep{Lloyd}. We know that for any given value distribution, the Lloyd-Max algorithm minimizes the quantization MSE. Therefore, for a given vector cluster $\mathcal{B}_i$ we obtain the following result:

\begin{equation}
    \frac{1}{|\mathcal{B}_{i}^{(n+1)}|}\sum_{\bm{b} \in \mathcal{B}_{i}^{(n+1)}} \frac{1}{L_b}\lVert \bm{b}- C_i^{(n+1)}(\bm{b})\rVert^2_2 \le \frac{1}{|\mathcal{B}_{i}^{(n+1)}|}\sum_{\bm{b} \in \mathcal{B}_{i}^{(n+1)}} \frac{1}{L_b}\lVert \bm{b}- C_i^{(n)}(\bm{b})\rVert^2_2
\end{equation}

The above equation states that quantizing the given block cluster $\mathcal{B}_i$ after updating the associated codebook from $C_i^{(n)}$ to $C_i^{(n+1)}$ results in lower quantization MSE. Since this is true for all the block clusters, we derive the following result: 
\begin{equation}
\begin{split}
\label{eq:mse_ineq_2}
     J^{(n+1)} &= \frac{1}{N_c} \sum_{i=1}^{N_c} \frac{1}{|\mathcal{B}_{i}^{(n+1)}|}\sum_{\bm{b} \in \mathcal{B}_{i}^{(n+1)}} \frac{1}{L_b}\lVert \bm{b}- C_i^{(n+1)}(\bm{b})\rVert^2_2  \le \tilde{J}^{(n+1)}   
\end{split}
\end{equation}

Following (\ref{eq:mse_ineq_1}) and (\ref{eq:mse_ineq_2}), we find that the quantization MSE is non-increasing for each iteration, that is, $J^{(1)} \ge J^{(2)} \ge J^{(3)} \ge \ldots \ge J^{(M)}$ where $M$ is the maximum number of iterations. 
%Therefore, we can say that if the algorithm converges, then it must be that it has converged to a local minimum. 
\hfill $\blacksquare$


\begin{figure}
    \begin{center}
    \includegraphics[width=0.5\textwidth]{sections//figures/mse_vs_iter.pdf}
    \end{center}
    \caption{\small NMSE vs iterations during LO-BCQ compared to other block quantization proposals}
    \label{fig:nmse_vs_iter}
\end{figure}

Figure \ref{fig:nmse_vs_iter} shows the empirical convergence of LO-BCQ across several block lengths and number of codebooks. Also, the MSE achieved by LO-BCQ is compared to baselines such as MXFP and VSQ. As shown, LO-BCQ converges to a lower MSE than the baselines. Further, we achieve better convergence for larger number of codebooks ($N_c$) and for a smaller block length ($L_b$), both of which increase the bitwidth of BCQ (see Eq \ref{eq:bitwidth_bcq}).


\subsection{Additional Accuracy Results}
%Table \ref{tab:lobcq_config} lists the various LOBCQ configurations and their corresponding bitwidths.
\begin{table}
\setlength{\tabcolsep}{4.75pt}
\begin{center}
\caption{\label{tab:lobcq_config} Various LO-BCQ configurations and their bitwidths.}
\begin{tabular}{|c||c|c|c|c||c|c||c|} 
\hline
 & \multicolumn{4}{|c||}{$L_b=8$} & \multicolumn{2}{|c||}{$L_b=4$} & $L_b=2$ \\
 \hline
 \backslashbox{$L_A$\kern-1em}{\kern-1em$N_c$} & 2 & 4 & 8 & 16 & 2 & 4 & 2 \\
 \hline
 64 & 4.25 & 4.375 & 4.5 & 4.625 & 4.375 & 4.625 & 4.625\\
 \hline
 32 & 4.375 & 4.5 & 4.625& 4.75 & 4.5 & 4.75 & 4.75 \\
 \hline
 16 & 4.625 & 4.75& 4.875 & 5 & 4.75 & 5 & 5 \\
 \hline
\end{tabular}
\end{center}
\end{table}

%\subsection{Perplexity achieved by various LO-BCQ configurations on Wikitext-103 dataset}

\begin{table} \centering
\begin{tabular}{|c||c|c|c|c||c|c||c|} 
\hline
 $L_b \rightarrow$& \multicolumn{4}{c||}{8} & \multicolumn{2}{c||}{4} & 2\\
 \hline
 \backslashbox{$L_A$\kern-1em}{\kern-1em$N_c$} & 2 & 4 & 8 & 16 & 2 & 4 & 2  \\
 %$N_c \rightarrow$ & 2 & 4 & 8 & 16 & 2 & 4 & 2 \\
 \hline
 \hline
 \multicolumn{8}{c}{GPT3-1.3B (FP32 PPL = 9.98)} \\ 
 \hline
 \hline
 64 & 10.40 & 10.23 & 10.17 & 10.15 &  10.28 & 10.18 & 10.19 \\
 \hline
 32 & 10.25 & 10.20 & 10.15 & 10.12 &  10.23 & 10.17 & 10.17 \\
 \hline
 16 & 10.22 & 10.16 & 10.10 & 10.09 &  10.21 & 10.14 & 10.16 \\
 \hline
  \hline
 \multicolumn{8}{c}{GPT3-8B (FP32 PPL = 7.38)} \\ 
 \hline
 \hline
 64 & 7.61 & 7.52 & 7.48 &  7.47 &  7.55 &  7.49 & 7.50 \\
 \hline
 32 & 7.52 & 7.50 & 7.46 &  7.45 &  7.52 &  7.48 & 7.48  \\
 \hline
 16 & 7.51 & 7.48 & 7.44 &  7.44 &  7.51 &  7.49 & 7.47  \\
 \hline
\end{tabular}
\caption{\label{tab:ppl_gpt3_abalation} Wikitext-103 perplexity across GPT3-1.3B and 8B models.}
\end{table}

\begin{table} \centering
\begin{tabular}{|c||c|c|c|c||} 
\hline
 $L_b \rightarrow$& \multicolumn{4}{c||}{8}\\
 \hline
 \backslashbox{$L_A$\kern-1em}{\kern-1em$N_c$} & 2 & 4 & 8 & 16 \\
 %$N_c \rightarrow$ & 2 & 4 & 8 & 16 & 2 & 4 & 2 \\
 \hline
 \hline
 \multicolumn{5}{|c|}{Llama2-7B (FP32 PPL = 5.06)} \\ 
 \hline
 \hline
 64 & 5.31 & 5.26 & 5.19 & 5.18  \\
 \hline
 32 & 5.23 & 5.25 & 5.18 & 5.15  \\
 \hline
 16 & 5.23 & 5.19 & 5.16 & 5.14  \\
 \hline
 \multicolumn{5}{|c|}{Nemotron4-15B (FP32 PPL = 5.87)} \\ 
 \hline
 \hline
 64  & 6.3 & 6.20 & 6.13 & 6.08  \\
 \hline
 32  & 6.24 & 6.12 & 6.07 & 6.03  \\
 \hline
 16  & 6.12 & 6.14 & 6.04 & 6.02  \\
 \hline
 \multicolumn{5}{|c|}{Nemotron4-340B (FP32 PPL = 3.48)} \\ 
 \hline
 \hline
 64 & 3.67 & 3.62 & 3.60 & 3.59 \\
 \hline
 32 & 3.63 & 3.61 & 3.59 & 3.56 \\
 \hline
 16 & 3.61 & 3.58 & 3.57 & 3.55 \\
 \hline
\end{tabular}
\caption{\label{tab:ppl_llama7B_nemo15B} Wikitext-103 perplexity compared to FP32 baseline in Llama2-7B and Nemotron4-15B, 340B models}
\end{table}

%\subsection{Perplexity achieved by various LO-BCQ configurations on MMLU dataset}


\begin{table} \centering
\begin{tabular}{|c||c|c|c|c||c|c|c|c|} 
\hline
 $L_b \rightarrow$& \multicolumn{4}{c||}{8} & \multicolumn{4}{c||}{8}\\
 \hline
 \backslashbox{$L_A$\kern-1em}{\kern-1em$N_c$} & 2 & 4 & 8 & 16 & 2 & 4 & 8 & 16  \\
 %$N_c \rightarrow$ & 2 & 4 & 8 & 16 & 2 & 4 & 2 \\
 \hline
 \hline
 \multicolumn{5}{|c|}{Llama2-7B (FP32 Accuracy = 45.8\%)} & \multicolumn{4}{|c|}{Llama2-70B (FP32 Accuracy = 69.12\%)} \\ 
 \hline
 \hline
 64 & 43.9 & 43.4 & 43.9 & 44.9 & 68.07 & 68.27 & 68.17 & 68.75 \\
 \hline
 32 & 44.5 & 43.8 & 44.9 & 44.5 & 68.37 & 68.51 & 68.35 & 68.27  \\
 \hline
 16 & 43.9 & 42.7 & 44.9 & 45 & 68.12 & 68.77 & 68.31 & 68.59  \\
 \hline
 \hline
 \multicolumn{5}{|c|}{GPT3-22B (FP32 Accuracy = 38.75\%)} & \multicolumn{4}{|c|}{Nemotron4-15B (FP32 Accuracy = 64.3\%)} \\ 
 \hline
 \hline
 64 & 36.71 & 38.85 & 38.13 & 38.92 & 63.17 & 62.36 & 63.72 & 64.09 \\
 \hline
 32 & 37.95 & 38.69 & 39.45 & 38.34 & 64.05 & 62.30 & 63.8 & 64.33  \\
 \hline
 16 & 38.88 & 38.80 & 38.31 & 38.92 & 63.22 & 63.51 & 63.93 & 64.43  \\
 \hline
\end{tabular}
\caption{\label{tab:mmlu_abalation} Accuracy on MMLU dataset across GPT3-22B, Llama2-7B, 70B and Nemotron4-15B models.}
\end{table}


%\subsection{Perplexity achieved by various LO-BCQ configurations on LM evaluation harness}

\begin{table} \centering
\begin{tabular}{|c||c|c|c|c||c|c|c|c|} 
\hline
 $L_b \rightarrow$& \multicolumn{4}{c||}{8} & \multicolumn{4}{c||}{8}\\
 \hline
 \backslashbox{$L_A$\kern-1em}{\kern-1em$N_c$} & 2 & 4 & 8 & 16 & 2 & 4 & 8 & 16  \\
 %$N_c \rightarrow$ & 2 & 4 & 8 & 16 & 2 & 4 & 2 \\
 \hline
 \hline
 \multicolumn{5}{|c|}{Race (FP32 Accuracy = 37.51\%)} & \multicolumn{4}{|c|}{Boolq (FP32 Accuracy = 64.62\%)} \\ 
 \hline
 \hline
 64 & 36.94 & 37.13 & 36.27 & 37.13 & 63.73 & 62.26 & 63.49 & 63.36 \\
 \hline
 32 & 37.03 & 36.36 & 36.08 & 37.03 & 62.54 & 63.51 & 63.49 & 63.55  \\
 \hline
 16 & 37.03 & 37.03 & 36.46 & 37.03 & 61.1 & 63.79 & 63.58 & 63.33  \\
 \hline
 \hline
 \multicolumn{5}{|c|}{Winogrande (FP32 Accuracy = 58.01\%)} & \multicolumn{4}{|c|}{Piqa (FP32 Accuracy = 74.21\%)} \\ 
 \hline
 \hline
 64 & 58.17 & 57.22 & 57.85 & 58.33 & 73.01 & 73.07 & 73.07 & 72.80 \\
 \hline
 32 & 59.12 & 58.09 & 57.85 & 58.41 & 73.01 & 73.94 & 72.74 & 73.18  \\
 \hline
 16 & 57.93 & 58.88 & 57.93 & 58.56 & 73.94 & 72.80 & 73.01 & 73.94  \\
 \hline
\end{tabular}
\caption{\label{tab:mmlu_abalation} Accuracy on LM evaluation harness tasks on GPT3-1.3B model.}
\end{table}

\begin{table} \centering
\begin{tabular}{|c||c|c|c|c||c|c|c|c|} 
\hline
 $L_b \rightarrow$& \multicolumn{4}{c||}{8} & \multicolumn{4}{c||}{8}\\
 \hline
 \backslashbox{$L_A$\kern-1em}{\kern-1em$N_c$} & 2 & 4 & 8 & 16 & 2 & 4 & 8 & 16  \\
 %$N_c \rightarrow$ & 2 & 4 & 8 & 16 & 2 & 4 & 2 \\
 \hline
 \hline
 \multicolumn{5}{|c|}{Race (FP32 Accuracy = 41.34\%)} & \multicolumn{4}{|c|}{Boolq (FP32 Accuracy = 68.32\%)} \\ 
 \hline
 \hline
 64 & 40.48 & 40.10 & 39.43 & 39.90 & 69.20 & 68.41 & 69.45 & 68.56 \\
 \hline
 32 & 39.52 & 39.52 & 40.77 & 39.62 & 68.32 & 67.43 & 68.17 & 69.30  \\
 \hline
 16 & 39.81 & 39.71 & 39.90 & 40.38 & 68.10 & 66.33 & 69.51 & 69.42  \\
 \hline
 \hline
 \multicolumn{5}{|c|}{Winogrande (FP32 Accuracy = 67.88\%)} & \multicolumn{4}{|c|}{Piqa (FP32 Accuracy = 78.78\%)} \\ 
 \hline
 \hline
 64 & 66.85 & 66.61 & 67.72 & 67.88 & 77.31 & 77.42 & 77.75 & 77.64 \\
 \hline
 32 & 67.25 & 67.72 & 67.72 & 67.00 & 77.31 & 77.04 & 77.80 & 77.37  \\
 \hline
 16 & 68.11 & 68.90 & 67.88 & 67.48 & 77.37 & 78.13 & 78.13 & 77.69  \\
 \hline
\end{tabular}
\caption{\label{tab:mmlu_abalation} Accuracy on LM evaluation harness tasks on GPT3-8B model.}
\end{table}

\begin{table} \centering
\begin{tabular}{|c||c|c|c|c||c|c|c|c|} 
\hline
 $L_b \rightarrow$& \multicolumn{4}{c||}{8} & \multicolumn{4}{c||}{8}\\
 \hline
 \backslashbox{$L_A$\kern-1em}{\kern-1em$N_c$} & 2 & 4 & 8 & 16 & 2 & 4 & 8 & 16  \\
 %$N_c \rightarrow$ & 2 & 4 & 8 & 16 & 2 & 4 & 2 \\
 \hline
 \hline
 \multicolumn{5}{|c|}{Race (FP32 Accuracy = 40.67\%)} & \multicolumn{4}{|c|}{Boolq (FP32 Accuracy = 76.54\%)} \\ 
 \hline
 \hline
 64 & 40.48 & 40.10 & 39.43 & 39.90 & 75.41 & 75.11 & 77.09 & 75.66 \\
 \hline
 32 & 39.52 & 39.52 & 40.77 & 39.62 & 76.02 & 76.02 & 75.96 & 75.35  \\
 \hline
 16 & 39.81 & 39.71 & 39.90 & 40.38 & 75.05 & 73.82 & 75.72 & 76.09  \\
 \hline
 \hline
 \multicolumn{5}{|c|}{Winogrande (FP32 Accuracy = 70.64\%)} & \multicolumn{4}{|c|}{Piqa (FP32 Accuracy = 79.16\%)} \\ 
 \hline
 \hline
 64 & 69.14 & 70.17 & 70.17 & 70.56 & 78.24 & 79.00 & 78.62 & 78.73 \\
 \hline
 32 & 70.96 & 69.69 & 71.27 & 69.30 & 78.56 & 79.49 & 79.16 & 78.89  \\
 \hline
 16 & 71.03 & 69.53 & 69.69 & 70.40 & 78.13 & 79.16 & 79.00 & 79.00  \\
 \hline
\end{tabular}
\caption{\label{tab:mmlu_abalation} Accuracy on LM evaluation harness tasks on GPT3-22B model.}
\end{table}

\begin{table} \centering
\begin{tabular}{|c||c|c|c|c||c|c|c|c|} 
\hline
 $L_b \rightarrow$& \multicolumn{4}{c||}{8} & \multicolumn{4}{c||}{8}\\
 \hline
 \backslashbox{$L_A$\kern-1em}{\kern-1em$N_c$} & 2 & 4 & 8 & 16 & 2 & 4 & 8 & 16  \\
 %$N_c \rightarrow$ & 2 & 4 & 8 & 16 & 2 & 4 & 2 \\
 \hline
 \hline
 \multicolumn{5}{|c|}{Race (FP32 Accuracy = 44.4\%)} & \multicolumn{4}{|c|}{Boolq (FP32 Accuracy = 79.29\%)} \\ 
 \hline
 \hline
 64 & 42.49 & 42.51 & 42.58 & 43.45 & 77.58 & 77.37 & 77.43 & 78.1 \\
 \hline
 32 & 43.35 & 42.49 & 43.64 & 43.73 & 77.86 & 75.32 & 77.28 & 77.86  \\
 \hline
 16 & 44.21 & 44.21 & 43.64 & 42.97 & 78.65 & 77 & 76.94 & 77.98  \\
 \hline
 \hline
 \multicolumn{5}{|c|}{Winogrande (FP32 Accuracy = 69.38\%)} & \multicolumn{4}{|c|}{Piqa (FP32 Accuracy = 78.07\%)} \\ 
 \hline
 \hline
 64 & 68.9 & 68.43 & 69.77 & 68.19 & 77.09 & 76.82 & 77.09 & 77.86 \\
 \hline
 32 & 69.38 & 68.51 & 68.82 & 68.90 & 78.07 & 76.71 & 78.07 & 77.86  \\
 \hline
 16 & 69.53 & 67.09 & 69.38 & 68.90 & 77.37 & 77.8 & 77.91 & 77.69  \\
 \hline
\end{tabular}
\caption{\label{tab:mmlu_abalation} Accuracy on LM evaluation harness tasks on Llama2-7B model.}
\end{table}

\begin{table} \centering
\begin{tabular}{|c||c|c|c|c||c|c|c|c|} 
\hline
 $L_b \rightarrow$& \multicolumn{4}{c||}{8} & \multicolumn{4}{c||}{8}\\
 \hline
 \backslashbox{$L_A$\kern-1em}{\kern-1em$N_c$} & 2 & 4 & 8 & 16 & 2 & 4 & 8 & 16  \\
 %$N_c \rightarrow$ & 2 & 4 & 8 & 16 & 2 & 4 & 2 \\
 \hline
 \hline
 \multicolumn{5}{|c|}{Race (FP32 Accuracy = 48.8\%)} & \multicolumn{4}{|c|}{Boolq (FP32 Accuracy = 85.23\%)} \\ 
 \hline
 \hline
 64 & 49.00 & 49.00 & 49.28 & 48.71 & 82.82 & 84.28 & 84.03 & 84.25 \\
 \hline
 32 & 49.57 & 48.52 & 48.33 & 49.28 & 83.85 & 84.46 & 84.31 & 84.93  \\
 \hline
 16 & 49.85 & 49.09 & 49.28 & 48.99 & 85.11 & 84.46 & 84.61 & 83.94  \\
 \hline
 \hline
 \multicolumn{5}{|c|}{Winogrande (FP32 Accuracy = 79.95\%)} & \multicolumn{4}{|c|}{Piqa (FP32 Accuracy = 81.56\%)} \\ 
 \hline
 \hline
 64 & 78.77 & 78.45 & 78.37 & 79.16 & 81.45 & 80.69 & 81.45 & 81.5 \\
 \hline
 32 & 78.45 & 79.01 & 78.69 & 80.66 & 81.56 & 80.58 & 81.18 & 81.34  \\
 \hline
 16 & 79.95 & 79.56 & 79.79 & 79.72 & 81.28 & 81.66 & 81.28 & 80.96  \\
 \hline
\end{tabular}
\caption{\label{tab:mmlu_abalation} Accuracy on LM evaluation harness tasks on Llama2-70B model.}
\end{table}

%\section{MSE Studies}
%\textcolor{red}{TODO}


\subsection{Number Formats and Quantization Method}
\label{subsec:numFormats_quantMethod}
\subsubsection{Integer Format}
An $n$-bit signed integer (INT) is typically represented with a 2s-complement format \citep{yao2022zeroquant,xiao2023smoothquant,dai2021vsq}, where the most significant bit denotes the sign.

\subsubsection{Floating Point Format}
An $n$-bit signed floating point (FP) number $x$ comprises of a 1-bit sign ($x_{\mathrm{sign}}$), $B_m$-bit mantissa ($x_{\mathrm{mant}}$) and $B_e$-bit exponent ($x_{\mathrm{exp}}$) such that $B_m+B_e=n-1$. The associated constant exponent bias ($E_{\mathrm{bias}}$) is computed as $(2^{{B_e}-1}-1)$. We denote this format as $E_{B_e}M_{B_m}$.  

\subsubsection{Quantization Scheme}
\label{subsec:quant_method}
A quantization scheme dictates how a given unquantized tensor is converted to its quantized representation. We consider FP formats for the purpose of illustration. Given an unquantized tensor $\bm{X}$ and an FP format $E_{B_e}M_{B_m}$, we first, we compute the quantization scale factor $s_X$ that maps the maximum absolute value of $\bm{X}$ to the maximum quantization level of the $E_{B_e}M_{B_m}$ format as follows:
\begin{align}
\label{eq:sf}
    s_X = \frac{\mathrm{max}(|\bm{X}|)}{\mathrm{max}(E_{B_e}M_{B_m})}
\end{align}
In the above equation, $|\cdot|$ denotes the absolute value function.

Next, we scale $\bm{X}$ by $s_X$ and quantize it to $\hat{\bm{X}}$ by rounding it to the nearest quantization level of $E_{B_e}M_{B_m}$ as:

\begin{align}
\label{eq:tensor_quant}
    \hat{\bm{X}} = \text{round-to-nearest}\left(\frac{\bm{X}}{s_X}, E_{B_e}M_{B_m}\right)
\end{align}

We perform dynamic max-scaled quantization \citep{wu2020integer}, where the scale factor $s$ for activations is dynamically computed during runtime.

\subsection{Vector Scaled Quantization}
\begin{wrapfigure}{r}{0.35\linewidth}
  \centering
  \includegraphics[width=\linewidth]{sections/figures/vsquant.jpg}
  \caption{\small Vectorwise decomposition for per-vector scaled quantization (VSQ \citep{dai2021vsq}).}
  \label{fig:vsquant}
\end{wrapfigure}
During VSQ \citep{dai2021vsq}, the operand tensors are decomposed into 1D vectors in a hardware friendly manner as shown in Figure \ref{fig:vsquant}. Since the decomposed tensors are used as operands in matrix multiplications during inference, it is beneficial to perform this decomposition along the reduction dimension of the multiplication. The vectorwise quantization is performed similar to tensorwise quantization described in Equations \ref{eq:sf} and \ref{eq:tensor_quant}, where a scale factor $s_v$ is required for each vector $\bm{v}$ that maps the maximum absolute value of that vector to the maximum quantization level. While smaller vector lengths can lead to larger accuracy gains, the associated memory and computational overheads due to the per-vector scale factors increases. To alleviate these overheads, VSQ \citep{dai2021vsq} proposed a second level quantization of the per-vector scale factors to unsigned integers, while MX \citep{rouhani2023shared} quantizes them to integer powers of 2 (denoted as $2^{INT}$).

\subsubsection{MX Format}
The MX format proposed in \citep{rouhani2023microscaling} introduces the concept of sub-block shifting. For every two scalar elements of $b$-bits each, there is a shared exponent bit. The value of this exponent bit is determined through an empirical analysis that targets minimizing quantization MSE. We note that the FP format $E_{1}M_{b}$ is strictly better than MX from an accuracy perspective since it allocates a dedicated exponent bit to each scalar as opposed to sharing it across two scalars. Therefore, we conservatively bound the accuracy of a $b+2$-bit signed MX format with that of a $E_{1}M_{b}$ format in our comparisons. For instance, we use E1M2 format as a proxy for MX4.

\begin{figure}
    \centering
    \includegraphics[width=1\linewidth]{sections//figures/BlockFormats.pdf}
    \caption{\small Comparing LO-BCQ to MX format.}
    \label{fig:block_formats}
\end{figure}

Figure \ref{fig:block_formats} compares our $4$-bit LO-BCQ block format to MX \citep{rouhani2023microscaling}. As shown, both LO-BCQ and MX decompose a given operand tensor into block arrays and each block array into blocks. Similar to MX, we find that per-block quantization ($L_b < L_A$) leads to better accuracy due to increased flexibility. While MX achieves this through per-block $1$-bit micro-scales, we associate a dedicated codebook to each block through a per-block codebook selector. Further, MX quantizes the per-block array scale-factor to E8M0 format without per-tensor scaling. In contrast during LO-BCQ, we find that per-tensor scaling combined with quantization of per-block array scale-factor to E4M3 format results in superior inference accuracy across models. 


\end{document}
\subsection{Parallel Peeling Process in Existing Work}\label{sec:peeling}
Under the proposed framework, multiple strategies for solving the $\FPeel$ function on~\cref{line:process_bucket} in parallel can be applied. 
We first review the two most relevant strategies in the literature. 

\myparagraph{The Offline Strategy in \Julienne{}.} 
\Julienne{}~\cite{dhulipala2017} employs a batch-synchronous strategy for the $\FPeel{}$ function (\cref{algo:peel_offline}). 
In each \subround, it gathers all vertices that require degree changes in a list $L$, 
which concatenates the neighbor lists for all vertices in $\frontier$. 
Each appearance of a vertex $u$ in $L$ means to decrement the \induceddegree{} $\degreestar[u]$ by 1. 
Hence, a \mf{Histogram} algorithm is used to count the number of appearances for each vertex in the list,
which can be performed by a parallel semisort~\cite{gu2015top,dong2023high} with $O(n)$ work with high probability. 
%Then a semisort algorithm is used to compute the sum of the decremented degrees for each neighbor, which sorts the keys in $O(n)$ work with high probability~\cite{gu2015top}.
%Based on these keys, the coreness values of the neighbors are updated accordingly. 
The \induceddegree{} for each vertex in $L$ will be decremented in a batch accordingly, and the next frontier can be computed as
all vertices that have degree drop to $k$ or lower by a parallel pack. 
We refer to this approach as an \defn{offline} approach. 
Since each vertex and edge is processed exactly once in the peeling process, 
the total work for this step is proportional to the total neighborhood size of the vertices in the frontier.
Based on \cref{thm:work}, the \kcore implementation in \Julienne{} has $O(n+m)$ work.

\begin{algorithm}[t]
  \small
    \caption{Offline peeling process}
    \label{algo:peel_offline}
    % \KwIn{Frontier vertices' union neighbor set $L$,  $k$}
    \SetKwFor{parFor}{parallel\_for}{do}{endfor}   
    \SetKwProg{myproc}{procedure}{}{}
    \myfunc{\FPeel{$(\frontier, k)$}}{
    % Apply parallel \fname{histogram} on $edgesRemoved$ \
    %\tcp{Union of neighbors of vertices in $\frontier$}\
    % $L \gets \bigcup_{v \in \frontier} \nei(v), v \in \frontier$ \\ 
    $L \gets $ \text{the list of vertices $u$, s.t. $(u,v)\in E, v\in \frontier$; duplicates are kept.} \\
    $H \gets \mf{Histogram}(L)$ \label{func:histogram}\tcp*[f]{Count the frequency for each $u$ in $L$}\\

    %\tcp{Update the degree of vertices in $H$ using the histogram result}
    \parFor(\tcp*[f]{for each $u$ with frequency $f_u$}) {$(\vname{u}, f_u) \in H$ }{

      \customIf{$\degreestar[u] > k$}{
          $\degreestar[u] \gets \degreestar[u] - f_u$\label{line:offline:decdeg}
        }
      }
    $\nextfrontier\gets \{u\in L, \degreestar[u]$ dropped to $k$ or lower by \cref{line:offline:decdeg}  $\}$ \\
    \Return{$\nextfrontier$}
    }
    \end{algorithm}


%An additional optimization in \Julienne{} is to maintain multiple instead of just one frontier.
%We review this part in \cref{sec:bucketing-existing}.
The actual implementation of \Julienne{} is more complicated than our framework. 
Our new algorithm with offline peeling is much simpler than both their theoretical algorithm in the paper and their implementation. 
Our result indicates that achieving work-efficiency does not require the complicated bucketing structure. 
We note that adding a bucketing structure can likely improve the performance in practice, which we discuss in \cref{sec:bucketing-existing}. 
%For example, in every $b$ rounds, \Julienne{} constructs the frontier for the next $b$ rounds by a bucketing structure. 
%We review this part in \cref{sec:bucketing-existing}.
%Here our result indicates that achieving work-efficiency does not require the complicated bucketing structure. 


While \Julienne{} achieves efficient work, the parallelism of \Julienne{} can be bottlenecked the scheduling overhead caused by the offline algorithm.
Note that each \subround{} requires to distribute work to all threads and synchronize them at the end. 
On sparse graphs, this scheduling overhead can dominate the actual computation cost, 
making \Julienne{} slower than a sequential algorithm in certain cases. 
In \cref{sec:vgc} we discuss more details and provide our solution to overcome this challenge. 
%In our experiments, there can be tens of thousands of subrounds on many real-world graphs, causing a large scheduling overhead to synchronize threads between rounds.
%insufficient number of vertices processed simultaneously, thus limited parallelism.
%The large number of subrounds leads to a large scheduling overhead to synchronize threads between rounds.
%As shown in \cref{fig:localsearch}, a $\sqrt{n}\times \sqrt{n}$ grid has all vertices with coreness values of 2. 
%On this graph, $\sqrt{n}$ subrounds are needed, making \Julienne{} slower than the sequential algorithm.
%This motivate us to consider \defn{online} approach that can avoid such subrounds.
%In this case, the large scheduling overhead makes the algorithm slower than a sequential implementation in our experiments (see \cref{xx}).
%For many other graphs that requires many subrounds, \Julienne{} also shows worse performance than other implementations. 

\myparagraph{The Online Strategy in \ParK and \PKC.} 
Both \Park{}~\cite{dasari2014park} and \pkc{}~\cite{kabir2017parallel} use an asynchronous peeling algorithm (\cref{algo:peel_online}), which
removes vertices in the frontier in parallel, 
and directly decrements the \induceddegree{s} of their neighbors using the \atomdec{} operation. 
We refer to this approach as an \defn{online} approach, 
since when we process a vertex $v$ in the frontier, the \induceddegree{s} $\degreestar[\cdot]$ of its neighbors are updated immediately. 
%When decrementing $\degreestar$ of a vertex $u$, if the new $\degreestar{}$ value hits $k$, 
When $\degreestar[u]$ is decremented from $k+1$ to $k$, 
$u$ will be put in the next frontier by atomically appending it to an array $\nextfrontier$. 
Assuming constant work for each \atomdec{} operation and appending each element to $\nextfrontier$, 
each peeling \subround has cost proportional to the total neighborhood size of the vertices in the frontier. 

Unfortunately, neither of the two algorithms maintains the \alive{} set during the algorithm, 
and simply use the original vertex set $V$ to generate each frontier. Therefore, the two algorithms both have $O(m+\maxcoreness n)$ work. 
However, introducing the \alive{} set in the algorithm will directly lead to work-efficiency. 


The computation in the online approach is simpler than the offline approach, and does not require strong synchronization between subrounds.   
Certain optimizations can be used to alleviate the scheduling overhead (e.g., \PKC{}~\cite{kabir2017parallel} and our new solution in \cref{sec:vgc}). 
The major performance bottleneck of \park{} and \pkc{} comes from the potential contention caused by the atomic operations. 
For a high-degree vertex, many of its neighbors can decrement its \induceddegree{} concurrently. 
As we can observe from the results in~\cref{table:fulltable}, \park{} and \pkc{} both suffer from poor performance on power-law graphs such as social and web graphs. 
Even a small fraction of high-degree vertices may result in high contention that degrade the performance. 
In addition, the data structure $\nextfrontier$ to generate the next frontier may also suffer from contention when many vertices are appended concurrently. 
%The same volume of atomic operations causes additional heavy contention for generating the next frontier $B$, which harms parallelism.
In \cref{sec:sampling}, we propose our solution to overcome this challenge.

\begin{algorithm}[t]
  \small
    \caption{Online peeling process}
    \label{algo:peel_online}
    %\KwIn{Vertices removed from  $\frontier$}
    \SetKwFor{parForEach}{parallel\_foreach}{do}{endfor}

    \SetKwProg{myfunc}{Function}{}{}
%    \myfunc{\mf{Peel}{$(V', k)$}}{
%        $todo \gets todo - |\vname{V'}|$\
%
%        \parForEach{$v \in \vname{V'}$}{
%            $\degreestar[v] \gets \vname{deg}[v]$\\
%            $\vname{next\_id}$ $\gets 0$\\
%            \parForEach{$u \in N(v)$}{
%                \If{$\vname{deg}[u] > k$}{
%                    \tcp{decrease degree of $u$ atomically}
%
%                    $\vname{deg}[u]. \faadec (1) $\ \label{line:ParK_Decrement}
%                    
%                    \tcp{add by atomically increase index}
%
%                    \If{$\vname{deg}[u] = \text{level}$}{
%                        % $\vname{next}[next\_id] \gets u$\\ \label{line:ParK_B}
%                        % $\vname{next\_id}. \faainc(1)$\
%                        Add $u$ to $next\_V'$ \
%                    }
%                    }
%                }
%            }
%        }

  \myfunc{$\FPeel (\frontier,k)$\label{line:peel}}{ 
    Initialize $\nextfrontier\gets \emptyset$ \tcp*[f]{Buffers the next frontier}\\
    \parForEach{$v\in \frontier$} {
        \parForEach{$u\in N(v)$} {
          $\delta\gets\atomdec(\degreestar[u])$ \label{line:atomic-dec-degree}\tcp*[f]{decrement atomically}\\
          \tcp{The last decrement adds $u$ to the next frontier}
          \customIf{$\delta=k+1$} { Add $u$ to $\nextfrontier$\label{line:update_frontier}} 
        }
    }
    \Return $\nextfrontier$\tcp*[f]{Returns the next frontier}
  }
\end{algorithm}


\revise{
    \myparagraph{Span Analysis.} 
    To analyze the span of the algorithm, \Julienne{}~\cite{dhulipala2017} defined a parameter $\rho$ to represent the number of peeling subrounds, 
    referred to as the \defn{peeling complexity}, and showed that the span of their algorithm is $\tilde{O}(\rho)$ \whp{}.
    For the framework in \cref{algo:framework} with both online and offline peeling algorithms, 
    the same $\tilde{O}(\rho)$ span bound holds (\whp{} for the offline version) following the same analysis in \Julienne{}.         
    However, note that each of the $\rho$ subrounds requires a parallel-for loop, indicating a global synchronization among threads. Therefore, the \emph{burdened span} (defined in \cref{sec:prelim}) is $\tilde{O}(\rho\omega)$. 
    Given $\omega$ as a large constant, when $\rho$ is also large, the parallelism in the algorithm can be limited. 
    In \cref{sec:vgc}, we discuss our new techniques that improve the burdened span, thus parallelism. 
}
%\youzhe{define high probability earlier here?} 


































%\myparagraph{Work-efficient peeling Strategies.}
%Under the proposed framework, multiple strategies for solving the $\FPeel$ function at ~\cref{line:process_bucket} in parallel can be applied.
%\ParK~\cite{dasari2014park} and \PKC~\cite{kabir2017parallel} directly use asynchronous parallelism to solve the frontier processing.
%While removing the vertices in the frontier in parallel, 
%the two algorithms directly decrease the degrees of the neighbors of the vertices in the frontier using \faadec{} operations,
%which is a unit-cost operation in the work-span model.
%The implementations keep another array to collect the vertices 
%that are decremented to the coreness value of the current frontier.
%The vertices are also arranged using atomic operations \faainc.
%\hide{
%    \PKC improves \ParK by introducing a local buffer to reduce the synchronization overhead.
%    Instead of using another array to store the vertices that are decremented to the coreness value of the current frontier,
%    \PKC uses a buffer with size $n/p$ for each thread to store the vertices.
%    In this way, those vertices that are decremented to the coreness value of the current frontier are stored in the buffer,
%    as long as the buffer is not full.
%    The buffer does not improve the work-efficiency of the algorithm.
%    The work of the two implementations can be affected by the model definition of the atomic operations.
%    If the atomic operations are not unit-cost, the total work of the peeling and updating cost will be $O(m + n)$.
%}
%
%\begin{algorithm}[t]
  \small
    \caption{Online peeling process}
    \label{algo:peel_online}
    %\KwIn{Vertices removed from  $\frontier$}
    \SetKwFor{parForEach}{parallel\_foreach}{do}{endfor}

    \SetKwProg{myfunc}{Function}{}{}
%    \myfunc{\mf{Peel}{$(V', k)$}}{
%        $todo \gets todo - |\vname{V'}|$\
%
%        \parForEach{$v \in \vname{V'}$}{
%            $\degreestar[v] \gets \vname{deg}[v]$\\
%            $\vname{next\_id}$ $\gets 0$\\
%            \parForEach{$u \in N(v)$}{
%                \If{$\vname{deg}[u] > k$}{
%                    \tcp{decrease degree of $u$ atomically}
%
%                    $\vname{deg}[u]. \faadec (1) $\ \label{line:ParK_Decrement}
%                    
%                    \tcp{add by atomically increase index}
%
%                    \If{$\vname{deg}[u] = \text{level}$}{
%                        % $\vname{next}[next\_id] \gets u$\\ \label{line:ParK_B}
%                        % $\vname{next\_id}. \faainc(1)$\
%                        Add $u$ to $next\_V'$ \
%                    }
%                    }
%                }
%            }
%        }

  \myfunc{$\FPeel (\frontier,k)$\label{line:peel}}{ 
    Initialize $\nextfrontier\gets \emptyset$ \tcp*[f]{Buffers the next frontier}\\
    \parForEach{$v\in \frontier$} {
        \parForEach{$u\in N(v)$} {
          $\delta\gets\atomdec(\degreestar[u])$ \label{line:atomic-dec-degree}\tcp*[f]{decrement atomically}\\
          \tcp{The last decrement adds $u$ to the next frontier}
          \customIf{$\delta=k+1$} { Add $u$ to $\nextfrontier$\label{line:update_frontier}} 
        }
    }
    \Return $\nextfrontier$\tcp*[f]{Returns the next frontier}
  }
\end{algorithm}

%
%The proposed algorithm in \GBBS employs a batch-synchronize strategy for frontier processing.
%Initially, it gathers all neighbors with degree changes after removing the vertices in the frontier. 
%Then a semisort algorithm is used to compute the sum of the decremented degrees for each neighbor, which sorts the keys in $O(n)$ work with high probability~\cite{gu2015top}.
%Based on these keys, the coreness values of the neighbors are updated accordingly. 
%Since each edge is processed exactly once during the frontier processing step, the total work for this step is $O(m)$.
%In practice, the semisort algorithm is slow because of too much data movement,
%and they use a parallel block-based histogram implementation to avoid shuffle,
%which is a $groupby$ operation in the database system.
%
%\begin{algorithm}[t]
  \small
    \caption{Offline peeling process}
    \label{algo:peel_offline}
    % \KwIn{Frontier vertices' union neighbor set $L$,  $k$}
    \SetKwFor{parFor}{parallel\_for}{do}{endfor}   
    \SetKwProg{myproc}{procedure}{}{}
    \myfunc{\FPeel{$(\frontier, k)$}}{
    % Apply parallel \fname{histogram} on $edgesRemoved$ \
    %\tcp{Union of neighbors of vertices in $\frontier$}\
    % $L \gets \bigcup_{v \in \frontier} \nei(v), v \in \frontier$ \\ 
    $L \gets $ \text{the list of vertices $u$, s.t. $(u,v)\in E, v\in \frontier$; duplicates are kept.} \\
    $H \gets \mf{Histogram}(L)$ \label{func:histogram}\tcp*[f]{Count the frequency for each $u$ in $L$}\\

    %\tcp{Update the degree of vertices in $H$ using the histogram result}
    \parFor(\tcp*[f]{for each $u$ with frequency $f_u$}) {$(\vname{u}, f_u) \in H$ }{

      \customIf{$\degreestar[u] > k$}{
          $\degreestar[u] \gets \degreestar[u] - f_u$\label{line:offline:decdeg}
        }
      }
    $\nextfrontier\gets \{u\in L, \degreestar[u]$ dropped to $k$ or lower by \cref{line:offline:decdeg}  $\}$ \\
    \Return{$\nextfrontier$}
    }
    \end{algorithm}

%
%However, although the two approaches are work-efficient if embedded in our framework,
%the practical parallelism of the two algorithms is poor for different types of graphs.
%Specifically, the atomic operations in \ParK and \PKC may cause high-volume contention in the frontier processing step,
%which leads to a huge overhead of the peeling process on dense graphs.
%The synchronous implementation in \GBBS may can not be optimized for those graphs have a very large maximum coreness value.
%% \GBBS uses a fixed number of buckets to store the vertices in the frontier,
%% which causes a large number of frontier generating rounds as shown in ~\cref{line:generate_frontier} in ~\cref{algo:framework}.
%Specifically, \GBBS utilizes batch-update algorithm to generate and peel the frontier in each round,
%which may cause a large number of frontier generating rounds that can not be reduced because it is not an online processing workflow.
%Meanwhile, the structures and degree distribution of the graphs are various,
%which may cause the parallelism of the algorithm worse in the \FPeel step.
%
%
%
%To solve the problems of the two types of algorithms mentioned above, 
%we propose multiple design strategies to improve the performance of the algorithm framework.
%Because online processing workflow provides more flexibility for optimizing the parallelism of the algorithm,
%we design the algorithm to process the frontier in an online manner.
%
%\youzhe{mention the hashbag here? otherwise no difference from ParK}
%\youzhe{how to mention or begin our design here?}
%When we remove the vertices in the frontier,
%atomic operations \faadec{} are used to decrease the degrees of the neighbors.
%Instead of using another array to store the vertices that are decremented to the coreness value of the current frontier,
%we directly insert the neighbors into the hashbag.
%The decremented degree of the neighbors is bounded by the coreness value $k$ of the current round.
%Whenever the degree of a neighbor is decremented to $k$,
%it is inserted into the hashbag directly.
%Then the hashbag will maintain all the vertices that should be processed in the next round.
%Because the insertion operations into the hashbag are work-efficient, 
%the total work of the framework is work-efficient based on the proof above.
%Having the theoretically work-efficient frontier processing step,
%we further design several strategies to improve the performance of the framework,
%both in theoretical analysis and in practice.
%In following sections, we will introduce the design of the algorithm in detail. 


\hide{
implementation, an additional optimization is to extract the frontier and update the \alive{} set every 16 rounds. 
More precisely, every 16 rounds, the algorithm processes the \alive{} set, and extract the vertices with $\coreness$ in range $[k,k+16)$ to 16 buckets. 
This approach is a constant optimization to reduce the cost of processing the frontier and \alive{} set on \cref{line:extract,line:pack}. 
In \cref{sec:bucketing} we will introduce our proposed solution to improve this approach. 

Computing the histogram of all neighbors of the frontier also requires a few rounds of global synchronization. 
While treated as a constant in theoretical analysis, in practice this may cause large scheduling overhead, 
which can be $10^2$ to $10^4$ CPU cycles~\cite{cpuops}. \yihan{may need more careful justification}. 
Therefore, when the number of subrounds is large, the scheduling overhead may overweigh the benefit of parallelism. 

}



\begin{algorithm}[h!]
\caption{Gait-Net-augmented Sequential CMPC}
\label{alg:gaitMPC}
\begin{algorithmic}[1]
\Require $\mathbf q, \: \dot{\mathbf q}, \: \mathbf q^\text{cmd}, \: \dot{\mathbf q}^\text{cmd}$
\State \textbf{intialize} $\bm x_0 = f_\text{j2m}(\mathbf q, \: \dot{\mathbf q}), \: \bm u^0 =\bm u_\text{IG}, \: dt^0 = 0.05$ 
\State $\{ \mathbf q^\text{ref},\:\dot{\mathbf q}^\text{ref},\:\bm p_f^\text{ref}\} = f_\text{ref} \big(\mathbf q, \: \dot{\mathbf q}, \: \mathbf q^\text{cmd}, \: \dot{\mathbf q}^\text{cmd} \big)$
\State $\bm x^\text{ref} = f_\text{j2m}(\mathbf q^\text{ref},\:\dot{\mathbf q}^\text{ref},\:\bm p_f^\text{ref})$
\State $ j = 0$ 
\While{$j \leq j_\text{max} \:\text{and}\: \bm \eta \leq \delta \bm u  $} 
\State $\delta \bm u^{j} = \texttt{cmpc}(\bm x^\text{ref},\:\bm p_f^\text{ref},\:\bm p_c^\text{ref},\: \bm x_0,\: dt^j, \: \bm u^j)$
\State $\bm u^{j+1} = \bm u^j + \delta \bm u^j$ 
\State $dt^{j+1} = \Pi_\text{GN}(\mathbf q, \: \dot{\mathbf q},\: \bm p_f^{j})$
\State $\{ \bm x^\text{ref},\:\bm p_f^\text{ref}\}= f_\text{IK}(\bm p_f^{j},\:\bm p_c^{j},\: dt^{j+1})$
\State $j=j+1$
\EndWhile \\
\Return $\bm u^{j+1} $
\end{algorithmic}
\end{algorithm}
%\begin{figure*}[ht]
    \centering
    \includegraphics[width=\textwidth, trim=79 280 93 123, clip]{figures/framework_img.pdf}
    \caption{The pipeline of the \ENDow{} framework 
    %where each component is specified in a given configuration. 
    which yields a downstream task score and a WER score of the transcript set input to the task. The pipeline is executed for several severeties of noising and types of cleaning techniques. %Acoustic noising is applied at $k$ intensities, providing $k+1$ audio versions (including the non-noised version), eventually producing $k+2$ transcript versions (including the source transcript). Applying transcript cleaning reveals the effect of \textit{types} of noise. 
    Resulting scores are plotted on a graph for the analyses, as in, e.g., \autoref{fig_cleaning_graphs}.}
    %The pipeline is executed on $k+1$ intensities of acoustic noising (including the non-noised version), producing $k+2$ scores for the downstream task (including execution on the source transcripts). This process eventually describes the effect of the \textit{intensity} of transcript noise on the downstream task. The process is repeated for $m$ cleaning techniques ($m+1$ when including no cleaning), to analyze the benefit of a cleaning approach and the effect of the \textit{types} of transcript noise.}
    \label{fig_framework}
\end{figure*}
\section{Hierarchical Bucketing Structure}\label{sec:bucketing}

\cref{thm:work} shows that explicitly checking the active vertices in $\alivevertices$ (\cref{line:filter_out_alive}) in each round is asymptotically optimal.
However, this simple solution can be slow in practice. 
%For instance, on graph \SD with a large maximum coreness value $\maxcoreness$ (around $10^4$), 
%this step (\cref{line:filter_out_alive}) takes 87\% running time.
To optimize the performance, some existing designs, 
such as \Julienne{}, choose to use a bucketing structure (defined in \cref{sec:prelim}) to maintain the active set $\alivevertices$. %such that we can easily extract all vertices with \induceddegree{} $k$ in each round. 
%Some existing solutions, such as \Julienne{}, use a bucketing structure (defined in \cref{sec:prelim}) to optimize the performan, 
%which maintain the active set $\alivevertices$, such that we can easily extract all vertices with \induceddegree{} $k$ in each round. 
A bucketing structure usually maintains a (partial) mapping from each value $d$ to all vertices in $\alivevertices$ with (induced) degree $d$. 
In this section, we introduce our new design for the bucketing structure.  

We proposed the \emph{hierarchical bucketing structure} (\HBS), and show
how it improves our \kcore{} algorithm. 
We note that this data structure may also be of independent interest, 
since it provides the interface a special parallel priority queue with integer keys, which is useful in many applications~\cite{li2013parallel, shi2021parallel, shi2023theoretically,dhulipala2017,gbbs2021}.

%\HBS is used to maintain the active set $\alivevertices$ in \cref{algo:framework}.
%We will show how \HBS can significantly improve the performance on this part. 
%Hence, to improve the performance, the idea of ``bucketing structure'' is proposed to avoid scanning over the active set each round.
%We will first overview the approach in Julienne/GBBS and its issue, and then proposed our solution.

\subsection{Interface and Related Work}\label{sec:bucketing-existing}

%Among our baselines, ParK and PKC use the na\"ive approach, 
%which results in poor performance on social and web graphs with large $\maxcoreness$ (see \cref{table:fulltable,fig:overall}).
%Such inefficiency is due to its memory access pattern. 
%To compute the new active set, we check all existing vertices in the current set $\alivevertices$ by an indirect access through $\degreestar[v]$ for $v\in \alivevertices$, causing the \emph{random} access.
%For a vertex $v$, we access the $\degreestar[v]$ up to $\degree(v)$ times in all rounds, which can be much slower than the other part of the algorithm with $O(n+m)$ work---the process of scanning the edge list in the $\FPeel()$ function.
%In $\FPeel()$, accessing all $v$'s neighbors only requires $\degree(v)$ \emph{serial} access to the edge list in the CSR format.

As mentioned, while refining the active set in each round does not increase the asymptotic cost, it can affect performance due to extra computations and memory accesses. 
To reduce the overhead, \Julienne{} implements a bucketing structure, 
which maintains a collection of elements (vertices), each of which has a unique identifier (vertex ID) and an integer key (induced degree). 
The structure supports the following three functions (we note that there are more functions in the interface of a bucket structure~\cite{dhulipala2017}; For simplicity, we only list the functions used in \kcore):

\begin{itemize}
    \item \textbf{\FBuildBuckets{$R$, $A$}}: initialize the bucketing structure with key range 0 to $R$, and insert each element $a\in A$ into it.
    \item \textbf{\FGetNextBucket~~$\mapsto \frontier$}: return all elements with the smallest key in the bucketing structure.
    \item \textbf{\FUpdate{$a$}}: update $a$ in the structure with its new key.
\end{itemize}

\Julienne{} maintains a bucketing structure with $b$ buckets. Every $b$ rounds, it generates all frontiers for the next $b$ rounds by $\FBuildBuckets{b,\alivevertices}$,
which extract vertices with induced degree $k+i$ to bucket $i$ ($k$ is the current peeling round). 
%whenever it recomputes $\frontier$ from $\alivevertices$.  
%Each frontier is maintained as a \emph{bucket}. The $i$-th bucket contains the current vertices in $\alivevertices$ with degree $i$. 
%In particular, when finishing round $k$ where $k$ is a multiply of 16, they refine the current $\alivevertices$ by moving all vertices with degree $k+i$ into bucket $i$, for $1\le i\le 16$.
Vertices with \induceddegree{s} more than $k+16$ remains in $\alivevertices$, which they call a \emph{overflow bucket}. 
%In their implementation, they use $b=16$.
Using an efficient algorithm, the next $b$ frontiers can be generated by one pass of memory access to $\alivevertices$, leading to better performance. 
This strategy reduces the number of accesses to $\alivevertices$ by a factor of $b$. 
A vertex $v$ will be accessed by $\FBuildBuckets{}$ until it is extracted from $\alivevertices$, which means $O(\deg(v)/b)$ times of accesses.
However, when the induced degree of $v$ is decremented, 
we also need to move $v$ to the new bucket by $\FUpdate{v}$.
%Since Julienne deploys a fully offline setting, it checks all the incident vertices in a peeling step, computes the \FHistogram (frequency of appearances) of all incident vertices, decrements their induced degrees, moves them to the new buckets, and sometimes resizes the arrays of the buckets when they overflow.
%When a vertex needs to go to another of the $b$ frontiers, Julienne moves this vertex to the array of the frontier, and reallocated a larger array if needed.
In the worst case, a vertex may be moved $b-1$ times across the buckets.
%Hence, it is important to choose the parameter $b$ carefully. A large~$b$ allows for fewer times of accessing the active set $\alivevertices$, can result in excessive movements among the frontiers. 
%On the other hand, a small~$b$ reduces need to move vertices among buckets, but also leads to frequent refinement of the active set~$\alivevertices$.
%In general, a vertex $v$ with degree $\deg(v)$ will be processed in two conditions. 
%Second, moving $v$
%Second, after $v$ is extracted from $\alivevertices$, 
%it will be put in a bucket but may move to buckets with smaller ids later. In the worst case, $v$ my be moved $O(b)$ times. 
Therefore, the total cost to process $v$ in the bucketing structure is $O(\degree(v)/b+b)$, adding up to $O(m/b+nb)$ for all vertices. 
This function has its minimum value at $b=O(\sqrt{\dbar})$, where $\dbar=\Theta(m/n)$ is the average degree. 
Our framework can be viewed as a special case where $b=1$, which only extracts the next frontier (bucket). 

%If we refer to \emph{processing} a vertex by either checking it in~$\alivevertices$ or moving it among the buckets, then an vertex $v$ is processed $O(\degree(v)/b+b)$ times. 

In practice, \Julienne{} uses $b=16$, which is a reasonable trade-off.
Indeed, in our experiments, using $b=16$ provides better performance than $b=1$ on most graphs with a large $\dbar$. 
However, we still observe some challenges with this solution. 
First, on most graphs with a small $\dbar$, using 16 buckets may cause 20--70\% overhead than using a single bucket.
Second, on graphs with very large $\dbar$, the cost of $O(\sqrt{m/n})$ per vertex may still be significant. 
Next, we propose our new design to overcome these issues. 
%However, picking a single static value is not ideal.
%A larger $b$ is preferred on graphs with large $\maxcoreness$, but it will cause significant overhead for graphs with small coreness.
%Meanwhile, the scan step for larger number of buckets is still a bottleneck for the performance (e.g., 30\% of the running time on graph \SD{}).
%Hence, it remains to be an open problem on the design of the bucketing structure that performs well on all graphs.

% \begin{figure}[htbp!]
  \centering
  \includegraphics[width=\columnwidth]{figures/eg-fixed_buckets.pdf}
  \caption{\textbf{An illustration of bucketing structure.}}\label{fig:eg-bucketing}
\end{figure}

\begin{figure}[t]
  \centering
  \includegraphics[width=.65\columnwidth]{figures/hier_buckets.pdf}
  \caption{\textbf{The execution of the hierarchical bucketing structure for the first 10 rounds of execution.}  The number in each box indicates the key (\induceddegree) range of the associate bucket. \revise{
    The first row shows that a vertex with degree $d$ is initially inserted to bucket $\lceil\log_2 (d+1)\rceil$.
    %The vertices with degree 0 and 1 are inserted into the first two buckets with ID 0 and 1, respectively.
    %The vertices with degree 2 and 3 are inserted into the third bucket with ID 2, 
    %and the vertices with degree 4 to 7 are inserted into the fourth bucket with ID 3, and so on.
    When $k=0$ or $1$, vertices with degree 0 or 1 are directly extracted from buckets 0 and 1. 
    We then redistribute vertices in bucket 2 (with degrees 2 or 3) to buckets 0 and 1 (shown in the second row), 
    such that they can be directly identified when $k=2$ or 3. 
    Similarly, after that, we redistributed vertices in bucket 3 (with degrees 4 to 7) to the first three buckets. 
    Vertices with degree 4 and 5 are moved to bucket 0 and 1, respectively, and vertices with degree 6 and 7 are moved to bucket 2, so on so forth. 
    %The remaining movements are similar, and the vertices are moved to the buckets with smaller IDs.
}
}
  \label{fig:hier_bucketing}
\end{figure} 


\subsection{Hierarchical Bucketing Structure (\HBS)}
We first propose the \emph{hierarchical bucketing structure} (\HBS) to improve the $O(\sqrt{m/n})$ cost per vertex. 
%For simplicity, in this section we use \emph{degree} to refer to the \emph{\induceddegree{}} of a vertex at a certain point of the algorithm. 
Let $d_{\max}$ be the maximum degree in the graph. 
%An \HBS consists of $O(\log d_{\max})$ buckets, each bucket storing vertices within a specific range of (induced) degrees, 
%implemented by a parallel hash bag (introduced in \cref{sec:prelim}). 
An \HBS maintains $1+\lceil\log_2 (d_{\max}+1)\rceil$ buckets. 
%, each implemented by a parallel hash bag (introduced in \cref{sec:prelim})
Each bucket stores vertices within a range of (induced) degrees of 1, 1, 2, 4, ..., growing exponentially. 
%Each bucket is implemented by a parallel hash bag (introduced in \cref{sec:prelim}). 
%each maintains vertices with degree ranges of 1, 1, 2, 4, ... , in a prefix doubling manner.
%By using \HBS, a vertex $v$ is processed $O(\log \degree(v))$ times, which is always better than $O(\degree(v)/b+b)$ times (note that $\degree(v)/b+b\ge 2\sqrt{\degree(v)}$).
\cref{fig:hier_bucketing} illustrates an \HBS and the (induced) degree range that each bucket maintains in the first ten rounds of execution.
When the induced degree of a vertex $v$ drops across a boundary, $v$ is moved to the appropriate new bucket. 
%The invariant in this design is that, a vertex can only be moved to a bucket with a smaller ID (left side in \cref{fig:hier_bucketing}).
%Consequently, a vertex~$v$ will be processed by $O(\log \degree(v))$ times in the entire algorithm.

To use an \HBS in \cref{algo:framework}, we will call \FBuildBuckets{$d_{\max}$, $V$} at the beginning of the algorithm, use \FGetNextBucket{} to generate the frontier at \cref{line:generate_frontier}, and call \FUpdate{$v$} when we decrement $v$'s induced degree. 
During the execution, we keep a counter~$\dmin$ indicating the current minimum key in a \HBS{}, which is also the current round $k$.
Finding the bucket ID of an element is based on the most significant differing bit between its key and $k$.
%Specifically, \FUpdate{$v$} provides the destination bucket\_id for a vertex $v$ with its degree $d'$ (key value) for insertion.

Next, we show how these functions are implemented efficiently on \HBS{}.
We will maintain each of the $O(\log d_{\max})$ buckets by a parallel hash bag (see \cref{sec:prelim}) that supports $O(1)$ expected cost for insertion and $O(t)$ work to extract all $t$ elements from the bag.
\FBuildBuckets{} simply inserts all elements to the corresponding bucket in parallel, based on the most significant bits of their key.
%We keep a buffered bucket (also maintained by a parallel hash bag) to hold the elements updated with key $k$, which avoids concurrent edits to the current bucket.

When \FGetNextBucket{} is called, %we first check the first bucket.
%If so, the \HBS will call \bagpack{}() to extract the keys in it.
%If it is empty, we increment the counter $k$, and check its associated list 
we find the first non-empty bucket with id $j$.
If it is one of the first two lists, we directly call \bagpack{} to return all keys in it.
Otherwise, %it will be the first non-empty one (say the $j$-th list) among the $O(\log d_{\max})$ buckets.
if $j>2$, we call \bagpack{} to get all elements in this bucket, and redistribute them to the first $j-1$ buckets, 
shown as the arrows in \cref{fig:hier_bucketing}.
%Finding the associated list in all of these operations can be implemented using some bitwise operations.

Finally, when an \FUpdate{$a$} is called, we check $a$'s new bucket ID and insert it to the corresponding bucket.  
%When the new ID is different from $a$'s previous bucket ID, we first check if $a$'s key is $k$.
%If so, we insert it into the buffer bucket; otherwise, we insert it to the bucket.
We do not delete $a$ from the original bucket since a hash bag does not support deletion.
In this case, an element may have copies in multiple buckets. Hence, when we call \bagpack{}, 
we also filter out those elements if their keys does not match the bucket ID. 
We note that this does not increase the asymptotic work of \kcore{}---since a vertex $v$ may be have at most $O(\log d(v))$ copies, the total cost here is at most $\sum \log d(v)=O(m)$. 
%Note that since a hash bag does not support deletion (see \cref{sec:prelim}), an element can have multiple copies in all buckets.
%Hence, we an \bagpack{}() is called for any bucket, we filter out those elements if their keys have been updated and should have been deleted.
%This step can be implemented using a \FFilter{} with linear work and polylogarithmic span, and the cost is asymptotically bounded by \bagpack{}.\yan{check}



In our implementation, instead of setting degree ranges for each bucket as 1, 1, 2, 4, 8, ... , 
we set the first eight buckets as single-key bucket. 
In other words, the first 8 buckets each maintain vertices with degree $\dmin,\dmin+1, \ldots, k+7$, 
and the next buckets correspond to the ranges of $[\dmin+8,\dmin+15], [\dmin+16, \dmin+31]$, and so on.
This optimization avoids frequently redistributing vertices in small buckets. 

\myparagraph{Cost Analysis.}
%The analysis of the \HBS is quite simple, relying on the fact that an element can only be reinserted a bucket with a smaller ID. 
%To analyze the cost of \HBS{}, note that an element can only be reinserted to a bucket with a smaller ID. 
%Such reinsertion can occur during \FUpdate{} or \FGetNextBucket{} (when redistributing $v$ to a new bucket). %, and in both cases an element is reinserted into a list that is in the front of the current list.
The cost of \HBS{} on a vertex $v$ includes inserting $v$ (in \FUpdate{}) or redistributing $v$ (in \FGetNextBucket{}) to a new bucket. 
In \FBuildBuckets{}, $v$ is placed in the \revise{$\lceil\log_2 (\degree(v)+1))\rceil$-th} bucket, and will only move to buckets with smaller IDs. 
During \FUpdate{} or \FGetNextBucket{}, $v$ can be packed from each bucket for redistribution at most once, and be inserted to any bucket at most once. 
Hence, the total cost to access $v$ in the bucketing structure is $O(\log \degree(v))$. 
Recall that using a fixed number of $b$ buckets leads to $O(\degree(v)/b+b)$ cost for vertex $v$, 
where $\degree(v)/b+b\ge 2\sqrt{\degree(v)}$.
Therefore, our new design provides a better solution. 
%Then, it can only be inserted in the buckets with smaller IDs, limiting the maximum number of reinsertions for $v$ to $O(\log \degree(v))$.
%The cost $O(1)$ expected for an insertion (each bucket is maintained by a parallel hash bag).
%In \FGetNextBucket{} we pack all elements in a bucket before outputting them or reinserting them.
%Since each bucket in packed only once, the cost is proportional to the current bucket size.
%Hence, we can amortize this cost to the insertions to this bucket, which is $O(1)$ per element.
%Putting both pieces together, the processing cost for vertex $v$ is $O(\log \degree(v))$ expected, which is an improvement of the na\"ive solution in \cref{algo:framework} with $O(\degree(v))$ work, when $\degree(v)=\omega(1)$.
%We show the speedup of the \HBS in practice in \cref{fig:bucketing-exp}.


\subsection{Our Final Design}

We now propose our final design of \HBS{}.
The cost to handle vertex $v$ in \HBS{} is $O(\log \degree(v))$, 
compared to $O(\sqrt{m/n})$ using a fixed number of buckets, or $O(\degree(v))$ if no bucketing structure is used.
We first note that when the average degree is constant, 
using a bucketing structure offers no benefit, 
as the overhead may introduce a large hidden constant in the complexity.
Therefore, we only use our \HBS{} when the average degree is larger than a constant $\theta$, which is set to 16 in our code.
Note that even if the average degree of the original graph is lower than $\theta$, 
as more low-degree vertices are peeled, the average degree of the graph may become larger. 
Ideally, when the average degree surpasses $\theta$, we should switch to \HBS{}. 
In our code, we simply switch to \HBS{} when a $\theta$-core is reached, 
which is guaranteed to have an average degree of at least $\theta$. 

%To implement the \HBS more efficiently, we introduce some implementation optimizations that do not change the asymptotic cost.
%The first optimization is that, instead of keeping all lists in strict prefix doubling manner (with sizes of 1, 1, 2, 4, 8, ...), we keep the first eight buckets for a single key, and then use ranges of 8, 16, 32, ..., for the next buckets.
%This will avoid frequent bucket unpacking in \FGetNextBucket{}, leading to some small performance gain.
%
%Another observation is that, our \HBS with $O(\log \degree(v))$ processing time is better the straightforward approach in \cref{algo:framework} when $\degree(v)=\omega(1)$.
%Otherwise, it will incur a small overhead due to the additional components in the \HBS.
%Hence, we set up a cut-off threshold~$\theta$, and activate the \HBS only when the average degree in the graph is larger than~$\theta$.
%We set $\theta=8$ in all of our experiments, but it can be a tunable parameter for different graphs.


\hide{
\myparagraph{Hybrid Bucketing Structure with Heuristic Threshold}
In practice, we still observed an overhead of hierarchical bucketing structure or
a fixed large number of buckets for large-diameter graphs, which have a small $\maxcoreness$ value in general.
such as road networks and some synthetic graphs.
In fact, it is theoretically most efficient to compute the $k$-core decomposition in a single bucket for large-diameter graphs under our framework
to avoid heavy movements between the buckets of the vertices,
since most of the vertices will be reduced to a smaller degree value in the end of the computation.
To reduce the effect of the two bucketing structure,
it is natural to combine the two bucketing structure in a hybrid way to hide the overhead of the hierarchical bucketing structure.
For any type of graphs, we first compute a small range in $(0, \Delta)$ of degree increment in a single bucket.
If the degree increases over $\Delta$, we switch the buckets into the hierarchical bucketing strategy without any effect to the computing process.
In experimental results shown in ~\ref{para:exp_bucketing}, there is no significant overhead for the hybrid bucketing structure for dense social networks and web networks,
but can accelerate the algorithm on road networks.
}






\hide{
We show the bitwise operation for \FUpdate{$\cdot$} function in \cref{algo:HBS_update}.
Essentially, we use the bitwise operation ``leading zero count" to find the first bit that is different between $d$ and 0, and use this bit to decide the returned key value.
Thus the function is $O(1)$ work.
For the \FGetNextBucket function, we simply add the bucket\_id by 1 if $\frontier$ is empty, and otherwise return the same bucket\_id.

\begin{algorithm}[h]
    \small
    \caption{\FGetBucket $()$ in \HBS}
    \label{algo:getbucket}
    \KwIn{identifier $a$ (vertex) with its key $\degreestar[a]$}
    \KwOut{$a$'s bucket index}
    \SetKwInOut{Maintains}{Maintains}
    \Maintains{
        $base\_k$: the key value of the first bucket\\  
        $\degreestar[a]$: the \induceddegree of vertex $a$\\
    }
    
    \SetKwProg{myproc}{procedure}{}{}
    \myproc{\FGetBucket{$(a)$}}{
        \tcc{The function returns the bucket index of the indentifier $a$, 
        and is used in the \HBS algorithm to determine the bucket index of a vertex.}
        % \tcp{Check if degree is within the valid range}
        % \If{$d$ is not in the HBS range}{
        %     \textbf{return}
        % }
        \tcp{Check if the identifier is in the first 2 buckets}
        \If{$\degreestar[a] < base\_k + 2$}{
            \textbf{return} $\degreestar[a] - base\_k$ \\
        }
        \Else{
            \tcc{Calculate the index for \HBS based on bit difference, $\oplus $ is the bitwise XOR operation, $\mf{leading\_zero\_count}$ is the number of leading zeros}
            \textbf{return} $\gets 64 - \mf{leading\_zero\_count}(\degreestar[a]$ $ \oplus $ $base\_k)$ 

        }
    }
\end{algorithm}

    The \HBS supports the workflow of $k$-core decomposition in an continuous manner.
    First, \textbf{BuildBuckets($b_1$, $b_2$)} initializes the buckets with the number of frontier-bucket $b_1$ and range-bucket $b_2$.
    Then, all the vertices $V$ are mapped to the buckets based on their degree values using \textbf{MakeBuckets($V, deg[\cdot]$)}.
    In each round, the vertices in the frontier are processed in the frontier buckets.
    \textbf{GetNextBucket()} returns the next available bucket with the vertices in it,
    which are peeled in the next round.
    When the frontier buckets are empty (\textbf{GetNextBucket()} returns a pair with $bucket\_id id_{next}$  of range-buckets),
    the vertices in the next available range buckets are moved forward to the buckets using \textbf{UpdateBuckets(id\_next)}.
    Once \textbf{GetNextBucket()} returns \textsc{NULL},
    \textbf{MakeBuckets($V', deg[\cdot]$)} is called again to map the remaining vertices that have not been processed yet to the hierarchical buckets for the next round.


\myparagraph{Analysis of the Hierarchical Bucketing Structure}
The work-efficiency of our algorithm does not rely on the number and structures of buckets.
However, in practical scenarios, we observe that performance varies with the number of buckets used for different graphs.
Every vertex $v \in V$ will be removed from the frontier exactly once.
For a vertex $v$ with degree $\degree(v)$, assume that $v$ has a final \induceddegree{} value of $k'$.
Thus, the total number of insertions into hashbags are not fixed for different bucketing numbers.
From the start of the insertion into the buckets, the insertions of a vertex $v$ are a series of footsteps of insertions until $v$ is finally peeled from the frontier. 
For example, assume a vertex $v_e$ is inserted into the bucket $b_3$ with range $4-7$ in the first round, and finally peeled in bucket $b_1$ with degree value $1$,
then $v_e$ will lead 3 insertions in the hashbags, i.e., $b_3$, $b_2$, and $b_1$.
Then there are two possible workload before $v$ is finally removed from the frontier for different bucketing strategies:

\begin{itemize}
    \item When using a hierarchical bucketing structure,
    vertices are inserted into the $\log n$ buckets,
    resulting in an insertion overhead work of $O(\sum_{i \in V} \log d_i)$.
    \item Using a single bucket entails batch assignment of vertices for $k_{max}$ rounds,
    The dominant factor of the work is the assignment of vertices to the buckets.
    where each vertex $v \in V$ is assigned to a bucket exactly once.
    The dominant factor in this case is the work of assignment that is $O(\sum_{i \in V} d_i)$.

\end{itemize}

The overhead of hashbag operations is balanced by the structure of the buckets.
The $k$-core decomposition algorithm in \Julienne~\cite{dhulipala2017} implements the only multiple-bucket algorithm.
However, with a fixed number of buckets specified in advance, their relative performance is suboptimal for graphs with varying structures.
To address this and enhance adaptability to different graphs,
we design a dynamic bucketing strategy that maps vertices into a single bucket or hierarchical buckets based on the number of peeling rounds, $k$.
As $k$ increases,
maintaining a fixed number of buckets causes the overhead of hashbag operations to increase.
Our implementation dynamically adjusts the bucketing structure based on the computational process,
balancing the two workload by reducing the two overhead
without manually setting a fixed number of buckets for different graphs in advance.


In Figure~\ref{fig:hier_bucketing}, we illustrate the first three bucketing mappings with $k$ increasing.

Although the hybrid bucketing structure is a trade-off between the two bucketing strategies for different types of graphs,
there is adversarial cases can be built to incur overhead.
For example, a graph with a $\maxcoreness$ of $k_m = \beta + 1$ will always force our algorithm to establish the hierarchical bucketing structure,
where $\beta$ is the threshold for the triggering of the hierarchical bucketing structure.
In this case, unnecessary overhead will be incurred for the hierarchical bucketing structure,
because the algorithm will finish in another round with a single bucket structure that is more efficient.
As we can see in experimental results in \cref{fig:overall} and \cref{table:fulltable}, graph \GLsfull is a case that can show the overhead with a threshold $\beta = 8$.
as it has a large diameter and a small maximum $\maxcoreness$ value.

Although adversarial cases like above can be built,
the hybrid bucketing structure is still a good trade-off for most graphs in practice,
and it can be easily adjusted by changing the threshold $\beta$ for the triggering of the hierarchical bucketing structure.
}


\hide{
Our proposed algorithm framework is based on the peeling process,
which removes the vertices in the frontier in parallel.
At the same time, the degrees of the neighbors of the vertices in the frontier are decremented,
which may cause the neighbors to be added to the frontier for the next round or to be processed in the next few rounds.
Thus, it is worth considering the design of the data structure to maintain the vertices to save work for the synchronization in each round.

Bucketing is a common technique used in parallel algorithms to reduce the synchronization overhead of the parallel operations.
Because the neighbors of the vertices in the frontier will be decremented to a new degree value,
which may be the coreness value or close to the coreness value of the current round,
the vertices can be arranged into buckets based on their degree values.
In this way, we can save the synchronization work for check all the remaining vertices in each round.
For example, the figure in \cref{fig:eg-bucketing} shows the bucketing structure for the vertices in the frontier.
While processing the vertices in the frontier with coreness value $k = 3$,
multiple neighbors are moved between the buckets based on their decreased degree values,
or added to the buckets from the remaining bucket that contains the vertices with coreness value not in the current range of the buckets.
The buckets are processed with $k$ increasing, and will be synchronized once the buckets are all empty.

The implementation in \GBBS utilizes a multi-bucket data structure to map vertices to buckets based on their degree,
which is a straightforward way to reduce the synchronization overhead of the batch insertion operations (scan) for the buckets for each $k$-core round.
The advantage of the multi-bucket data structure is that it can reduce the synchronization overhead of the batch
insertion operations for the buckets for each $k$-core round.
Although the number of buckets can be determined by any fixed number,
it is a trade-off between the synchronization overhead and the space usage, as well as the frontier generating step mentioned in ~\ref{sec:framework}.
In comparison, previous algorithms take a single-bucket approach to store the vertices, i.e.,
there is only the frontier containing vertices that have the same coreness value $k$ in each round.
However, some graphs such as web networks and social networks have a very large maximum coreness value,
which may lead to a dominant synchronization overhead.

\subsection{Hierarchical Bucketing Structure}

To reduce the hashbag operations while arranging vertices into the buckets,
we design a hierarchical bucketing structure to balance the performance of the hashbag operations
and the space usage for the vertices movement while maintaining the work-efficiency of the algorithm.
For a graph with $n$ vertices,
we need to assign a space of $O(n)$ for each hashbag to store the vertices.
As we described in the previous subsection,
for dense graphs with larger numbers of peeling rounds,
opening more hashbags will increase the overhead of the hashbag operations,
as well as the space usage.
In contrast, for large-diameter graphs with fewer peeling rounds,
opening fewer hashbags will reduce the overhead of the hashbag operations.
To solve the unpredictable performance influenced by the factor of the number of peeling rounds,
we propose a hierarchical bucketing structure to organize the vertices.

Instead of assigning each bucket a single coreness value,
we assign a range of coreness values to each bucket,
which is increased exponentially.
Therefore, the number of buckets to maintain all the vertices is $\log n$,
which is space-efficient,
and the overhead of hashbag operations is reduced without setting a fixed number of buckets in advance.
The vertices are assigned to the buckets of the corresponding range based on their coreness values,
which can be calculated using bitwise operations.



\subsection{Bucketing Strategy}

}



\hide{

For instance, at the beginning of the algorithm, vertices are categorized into degrees of 0, 1, 2--3, 4--7, 8--15, ... .
In the first two rounds, we have $\frontier_0$ and $\frontier_1$ from the first two sets.
After that, we regenerate $\frontier_2$ and $\frontier_3$ from the third set with degrees 2--3.
On the fourth peeling round, vertices with degree 4--7 will be split into the first three sets, and the same process repeats.
We illustrate this process in \cref{fig:hier_bucketing}.


and the $i$-th set maintains vertices

The \HBS does not maintain the vertices with the same coreness value in a single bucket. Instead, it maintains the vertices with a range of coreness values in a single bucket.
Figure \ref{fig:hier_bucketing} shows an example of the hierarchical bucketing structure concept.
The coreness range of the vertices in each bucket is increased exponentially,
i.e., the first two buckets store the vertices with single coreness values (frontier buckets),
and from the third bucket, the coreness range is increased by  a factor of 2.
In the example of Figure \ref{fig:hier_bucketing}, the first bucket stores the vertices with coreness value 0, the second bucket stores the vertices with coreness value 1,
and the third bucket stores the vertices with coreness value 2 and 3, the range of the coreness value is increased by a factor of 2 for the next buckets.
\todo{}
In practice, we set an upper bound $F$ for the coreness range of all the buckets,
and the vertices with coreness value larger than $F$ will be maintained in a container for remaining vertices.


    \item \textbf{MakeBuckets($\mathbb{I}: identifiers$, \textsf{D[$\cdot$]} $\mapsto$ \textsf{bucket\_id})}: For an identifiers $i$ (vertex) in the remaining identifiers $\mathbb{I}$
        maintained outside of the buckets, map the identifiers (vertices) $i$ to the buckets with their corresponding range based on the $bucket\_id$ $D[i]$.
    \item \textbf{Insert($\textsf{D[i]} \mapsto \textsf{bucket\_id}$)}\\
        Insert a vertex to the bucket with the corresponding bucket id based on $d[i]$.

} 

%%%%%%%%%%%%%%%%%%%%%%%%%%%%%%%%%%%%%%%%%%%%%%%%%%%%%%%%%%%%%%%%%%%%%%%%%%%%%%%%%%%%%%%%%%%%%%%%%%%%%%

%%%%%%%%%%%%%%%%%%%%%%%%%%%%%%%%%%%%%%%%%%%%%%
\begin{table*}[t]
\setlength{\tabcolsep}{3pt}
\centering
\renewcommand{\arraystretch}{1.1}
\tabcolsep=0.2cm
\begin{adjustbox}{max width=\textwidth}  % Set the maximum width to text width
\begin{tabular}{c| cccc ||  c| cc cc}
\toprule
General & \multicolumn{3}{c}{Preference} & Accuracy & Supervised & \multicolumn{3}{c}{Preference} & Accuracy \\ 
LLMs & PrefHit & PrefRecall & Reward & BLEU & Alignment & PrefHit & PrefRecall & Reward & BLEU \\ 
\midrule
GPT-J & 0.2572 & 0.6268 & 0.2410 & 0.0923 & Llama2-7B & 0.2029 & 0.803 & 0.0933 & 0.0947 \\
Pythia-2.8B & 0.3370 & 0.6449 & 0.1716 & 0.1355 & SFT & 0.2428 & 0.8125 & 0.1738 & 0.1364 \\
Qwen2-7B & 0.2790 & 0.8179 & 0.1593 & 0.2530 & Slic & 0.2464 & 0.6171 & 0.1700 & 0.1400 \\
Qwen2-57B & 0.3086 & 0.6481 & 0.6854 & 0.2568 & RRHF & 0.3297 & 0.8234 & 0.2263 & 0.1504 \\
Qwen2-72B & 0.3212 & 0.5555 & 0.6901 & 0.2286 & DPO-BT & 0.2500 & 0.8125 & 0.1728 & 0.1363 \\ 
StarCoder2-15B & 0.2464 & 0.6292 & 0.2962 & 0.1159 & DPO-PT & 0.2572 & 0.8067 & 0.1700 & 0.1348 \\
ChatGLM4-9B & 0.2246 & 0.6099 & 0.1686 & 0.1529 & PRO & 0.3025 & 0.6605 & 0.1802 & 0.1197 \\ 
Llama3-8B & 0.2826 & 0.6425 & 0.2458 & 0.1723 & \textbf{\shortname}* & \textbf{0.3659} & \textbf{0.8279} & \textbf{0.2301} & \textbf{0.1412} \\ 
\bottomrule
\end{tabular}
\end{adjustbox}
\caption{Main results on the StaCoCoQA. The left shows the performance of general LLMs, while the right presents the performance of the fine-tuned LLaMA2-7B across various strong benchmarks for preference alignment. Our method SeAdpra is highlighted in \textbf{bold}.}
\label{main}
\vspace{-0.2cm}
\end{table*}
%%%%%%%%%%%%%%%%%%%%%%%%%%%%%%%%%%%%%%%%%%%%%%%%%%%%%%%%%%%%%%%%%%%%%%%%%%%%%%%%%%%%%%%%%%%%%%%%%%%%
\begin{table}[h]
\centering
\renewcommand{\arraystretch}{1.02}
% \tabcolsep=0.1cm
\begin{adjustbox}{width=0.48\textwidth} % Adjust table width
\begin{tabularx}{0.495\textwidth}{p{1.2cm} p{0.7cm} p{0.95cm}p{0.95cm}p{0.7cm}p{0.7cm}}
     \toprule
    \multirow{2}{*}{\small \textbf{Dataset}} & \multirow{2}{*}{\small Model} & \multicolumn{2}{c}{\small Preference} & \multicolumn{2}{c}{\small Acc } \\ 
    & & \small \textit{PrefHit} & \small \textit{PrefRec} & \small \textit{Reward} & \small \textit{Rouge} \\ 
    \midrule
    \multirow{2}{*}{\small \textbf{Academia}}   & \small PRO & 33.78 & 59.56 & 69.94 & 9.84 \\ 
                                & \small \textbf{Ours} & 36.44 & 60.89 & 70.17 & 10.69 \\ 
    \midrule
    \multirow{2}{*}{\small \textbf{Chemistry}}  & \small PRO & 36.31 & 63.39 & 69.15 & 11.16 \\ 
                                & \small \textbf{Ours} & 38.69 & 64.68 & 69.31 & 12.27 \\ 
    \midrule
    \multirow{2}{*}{\small \textbf{Cooking}}    & \small PRO & 35.29 & 58.32 & 69.87 & 12.13 \\ 
                                & \small \textbf{Ours} & 38.50 & 60.01 & 69.93 & 13.73 \\ 
    \midrule
    \multirow{2}{*}{\small \textbf{Math}}       & \small PRO & 30.00 & 56.50 & 69.06 & 13.50 \\ 
                                & \small \textbf{Ours} & 32.00 & 58.54 & 69.21 & 14.45 \\ 
    \midrule
    \multirow{2}{*}{\small \textbf{Music}}      & \small PRO & 34.33 & 60.22 & 70.29 & 13.05 \\ 
                                & \small \textbf{Ours} & 37.00 & 60.61 & 70.84 & 13.82 \\ 
    \midrule
    \multirow{2}{*}{\small \textbf{Politics}}   & \small PRO & 41.77 & 66.10 & 69.52 & 9.31 \\ 
                                & \small \textbf{Ours} & 42.19 & 66.03 & 69.74 & 9.38 \\ 
    \midrule
    \multirow{2}{*}{\small \textbf{Code}} & \small PRO & 26.00 & 51.13 & 69.17 & 12.44 \\ 
                                & \small \textbf{Ours} & 27.00 & 51.77 & 69.46 & 13.33 \\ 
    \midrule
    \multirow{2}{*}{\small \textbf{Security}}   & \small PRO & 23.62 & 49.23 & 70.13 & 10.63 \\ 
                                & \small \textbf{Ours} & 25.20 & 49.24 & 70.92 & 10.98 \\ 
    \midrule
    \multirow{2}{*}{\small \textbf{Mean}}       & \small PRO & 32.64 & 58.05 & 69.64 & 11.51 \\ 
                                & \small \textbf{Ours} & \textbf{34.25} & \textbf{58.98} & \textbf{69.88} & \textbf{12.33} \\ 
    \bottomrule
\end{tabularx}
\end{adjustbox}
\caption{Main results (\%) on eight publicly available and popular CoQA datasets, comparing the strong list-wise benchmark PRO and \textbf{ours with bold}.}
\label{public}
\end{table}



%%%%%%%%%%%%%%%%%%%%%%%%%%%%%%%%%%%%%%%%%%%%%%%%%%%%%
\begin{table}[h]
\centering
\renewcommand{\arraystretch}{1.02}
\begin{tabularx}{0.48\textwidth}{p{1.45cm} p{0.56cm} p{0.6cm} p{0.6cm} p{0.50cm} p{0.45cm} X}
\toprule
\multirow{2}{*}{Method} & \multicolumn{3}{c}{Preference \((\uparrow)\)} & \multicolumn{3}{c}{Accuracy \((\uparrow)\)} \\ \cmidrule{2-4} \cmidrule{5-7}
& \small PrefHit & \small PrefRec & \small Reward & \small CoSim & \small BLEU & \small Rouge \\ \midrule
\small{SeAdpra} & \textbf{34.8} & \textbf{82.5} & \textbf{22.3} & \textbf{69.1} & \textbf{17.4} & \textbf{21.8} \\ 
\small{-w/o PerAl} & \underline{30.4} & 83.0 & 18.7 & 68.8 & \underline{12.6} & 21.0 \\
\small{-w/o PerCo} & 32.6 & 82.3 & \underline{24.2} & 69.3 & 16.4 & 21.0 \\
\small{-w/o \(\Delta_{Se}\)} & 31.2 & 82.8 & 18.6 & 68.3 & \underline{12.4} & 20.9 \\
\small{-w/o \(\Delta_{Po}\)} & \underline{29.4} & 82.2 & 22.1 & 69.0 & 16.6 & 21.4 \\
\small{\(PerCo_{Se}\)} & 30.9 & 83.5 & 15.6 & 67.6 & \underline{9.9} & 19.6 \\
\small{\(PerCo_{Po}\)} & \underline{30.3} & 82.7 & 20.5 & 68.9 & 14.4 & 20.1 \\ 
\bottomrule
\end{tabularx}
\caption{Ablation Results (\%). \(PerCo_{Se}\) or \(PerCo_{Po}\) only employs Single-APDF in Perceptual Comparison, replacing \(\Delta_{M}\) with \(\Delta_{Se}\) or \(\Delta_{Po}\). The bold represents the overall effect. The underlining highlights the most significant metric for each component's impact.}
\label{ablation}
% \vspace{-0.2cm}
\end{table}

\subsection{Dataset}

% These CoQA datasets contain questions and answers from the Stack Overflow data dump\footnote{https://archive.org/details/stackexchange}, intended for training preference models. 

Due to the additional challenges that programming QA presents for LLMs and the lack of high-quality, authentic multi-answer code preference datasets, we turned to StackExchange \footnote{https://archive.org/details/stackexchange}, a platform with forums that are accompanied by rich question-answering metadata. Based on this, we constructed a large-scale programming QA dataset in real-time (as of May 2024), called StaCoCoQA. It contains over 60,738 programming directories, as shown in Table~\ref{tab:stacocoqa_tags}, and 9,978,474 entries, with partial data statistics displayed in Figure~\ref{fig:dataset}. The data format of StaCoCoQA is presented in Table~\ref{fig::stacocoqa}.

The initial dataset \(D_I\) contains 24,101,803 entries, and is processed by the following steps:
(1) Select entries with "Questioner-picked answer" pairs to represent the preferences of the questioners, resulting in 12,260,106 entries in the \(D_Q\).
(2) Select data where the question includes at least one code block to focus on specific-domain programming QA, resulting in 9,978,474 entries in the dataset \(D_C\).
(3) All HTML tags were cleaned using BeautifulSoup \footnote{https://beautiful-soup-4.readthedocs.io/en/latest/} to ensure that the model is not affected by overly complex and meaningless content.
(4) Control the quality of the dataset by considering factors such as the time the question was posted, the size of the response pool, the difference between the highest and lowest votes within a pool, the votes for each response, the token-level length of the question and the answers, which yields varying sizes: 3K, 8K, 18K, 29K, and 64K. 
The controlled creation time variable and the data details after each processing step are shown in Table~\ref{tab:statistics}.

To further validate the effectiveness of SeAdpra, we also select eight popular topic CoQA datasets\footnote{https://huggingface.co/datasets/HuggingFaceH4/stack-exchange-preferences}, which have been filtered to meet specific criteria for preference models \cite{askell2021general}. Their detailed data information is provided in Table~\ref{domain}.
% Examples of some control variables are shown in Table~\ref{tab:statistics}.
% \noindent\textbf{Baselines}. 
% Following the DPO \cite{rafailov2024direct}, we evaluated several existing approaches aligned with human preference, including GPT-J \cite{gpt-j} and Pythia-2.8B \cite{biderman2023pythia}.  
% Next, we assessed StarCoder2 \cite{lozhkov2024starcoder}, which has demonstrated strong performance in code generation, alongside several general-purpose LLMs: Qwen2 \cite{qwen2}, ChatGLM4 \cite{wang2023cogvlm, glm2024chatglm} and LLaMA serials \cite{touvron2023llama,llama3modelcard}.
% Finally, we fine-tuned LLaMA2-7B on the StaCoCoQA and compared its performance with other strong baselines for supervised learning in preference alignment, including SFT, RRHF \cite{yuan2024rrhf}, Silc \cite{zhao2023slic}, DPO, and PRO \cite{song2024preference}.
%%%%%%%%%%%%%%%%%%%%%%%%%%%%%%%%%%%%%%%%%%%%%%%%%%%%%%%%%%%%%%%%%%%%%%%%%%%%%%%%%%%%%%%%%%%%%%%%%%%%%%%%%%%%%%%%%%%%%%%%%%%%%%%%%%

% For preference evaluation, traditional win-rate assessments are costly and not scalable. For instance, when an existing model \(M_A\) is evaluated against comparison methods \((M_B, M_C, M_D)\) in terms of win rates, upgrading model \(M_A\) would necessitate a reevaluation of its win rates against other models. Furthermore, if a new comparison method \(M_E\) is introduced, the win rates of model \(M_A\) against \(M_E\) would also need to be reassessed. Whether AI or humans are employed as evaluation mediators, binary preference between preferred and non-preferred choices or to score the inference results of the modified model, the costs of this process are substantial. 
% Therefore, from an economic perspective, we propose a novel list preference evaluation method. We utilize manually ranking results as the gold standard for assessing human preferences, to calculate the Hit and Recall, referred to as PrefHit and PrefRecall, respectively. Regardless of whether improving model \(M_A\) or expanding comparison method \(M_E\), only the calculation of PrefHit and PrefRecall for the modified model is required, eliminating the need for human evaluation. 
% We also employ a professional reward model\footnote{https://huggingface.co/OpenAssistant/reward-model-deberta-v3-large}
% for evaluation, denoted as the Reward metric.

% \subsection{Baseline} 
% Following the DPO \cite{rafailov2024direct}, we evaluated several existing approaches aligned with human preference, including GPT-J \cite{gpt-j} and Pythia-2.8B \cite{biderman2023pythia}.  
% Next, we assessed StarCoder2 \cite{lozhkov2024starcoder}, which has demonstrated strong performance in code generation, alongside several general-purpose LLMs: Qwen2 \cite{qwen2}, ChatGLM4 \cite{wang2023cogvlm, glm2024chatglm} and LLaMA serials \cite{touvron2023llama,llama3modelcard}.
% Finally, we fine-tuned LLaMA2-7B on the StaCoCoQA and compared its performance with other strong baselines for supervised learning in preference alignment, including SFT, RRHF \cite{yuan2024rrhf}, Silc \cite{zhao2023slic}, DPO, and PRO \cite{song2024preference}.
\subsection{Evaluation Metrics}
\label{sec: metric}
For preference evaluation, we design PrefHit and PrefRecall, adhering to the "CSTC" criterion outlined in Appendix \ref{sec::cstc}, which overcome the limitations of existing evaluation methods, as detailed in Appendix \ref{metric::mot}.
In addition, we demonstrate the effectiveness of thees new evaluation from two main aspects: 1) consistency with traditional metrics, and 2) applicability in different application scenarios in Appendix \ref{metric::ana}.
Following the previous \cite{song2024preference}, we also employ a professional reward.
% Following the previous \cite{song2024preference}, we also employ a professional reward model\footnote{https://huggingface.co/OpenAssistant/reward-model-deberta-v3-large} \cite{song2024preference}, denoted as the Reward.

For accuracy evaluation, we alternately employ BLEU \cite{papineni2002bleu}, RougeL \cite{lin2004rouge}, and CoSim. Similar to codebertscore \cite{zhou2023codebertscore}, CoSim not only focuses on the semantics of the code but also considers structural matching.
Additionally, the implementation details of SeAdpra are described in detail in the Appendix \ref{sec::imp}.
\subsection{Main Results}
We compared the performance of \shortname with general LLMs and strong preference alignment benchmarks on the StaCoCoQA dataset, as shown in Table~\ref{main}. Additionally, we compared SeAdpra with the strongly supervised alignment model PRO \cite{song2024preference} on eight publicly available CoQA datasets, as presented in Table~\ref{public} and Figure~\ref{fig::public}.

\textbf{Larger Model Parameters, Higher Preference.}
Firstly, the Qwen2 series has adopted DPO \cite{rafailov2024direct} in post-training, resulting in a significant enhancement in Reward.
In a horizontal comparison, the performance of Qwen2-7B and LLaMA2-7B in terms of PrefHit is comparable.
Gradually increasing the parameter size of Qwen2 \cite{qwen2} and LLaMA leads to higher PrefHit and Reward.
Additionally, general LLMs continue to demonstrate strong capabilities of programming understanding and generation preference datasets, contributing to high BLEU scores.
These findings indicate that increasing parameter size can significantly improve alignment.

\textbf{List-wise Ranking Outperforms Pair-wise Comparison.}
Intuitively, list-wise DPO-PT surpasses pair-wise DPO-{BT} on PrefHit. Other list-wise methods, such as RRHF, PRO, and our \shortname, also undoubtedly surpass the pair-wise Slic.

\textbf{Both Parameter Size and Alignment Strategies are Effective.}
Compared to other models, Pythia-2.8B achieved impressive results with significantly fewer parameters .
Effective alignment strategies can balance the performance differences brought by parameter size. For example, LLaMA2-7B with PRO achieves results close to Qwen2-57B in PrefHit. Moreover, LLaMA2-7B combined with our method SeAdpra has already far exceeded the PrefHit of Qwen2-57B.

\textbf{Rather not Higher Reward, Higher PrefHit.}
It is evident that Reward and PrefHit are not always positively correlated, indicating that models do not always accurately learn human preferences and cannot fully replace real human evaluation. Therefore, relying solely on a single public reward model is not sufficiently comprehensive when assessing preference alignment.

% In conclusion, during ensuring precise alignment, SeAdpra will focuse on PrefHit@1, even though the trade-off between PrefHit and PrefRecall is a common issue and increasing recall may sometimes lead to a decrease in hit rate. The positive correlation between Reward and BLEU, indicates that improving the quality of the generated text typically enhances the Reward. 
% Most importantly, evaluating preferences solely based on reward is clearly insufficient, as a high reward does not necessarily correspond to a high PrefHit or PrefRecall.
%%%%%%%%%%%%%%%%%%%%%%%%%%%%%%%%%%%%%%%%%%%
%%%%%%%%%%%%
\begin{figure}
  \centering
  \begin{subfigure}{0.49\linewidth}
    \includegraphics[width=\linewidth]{latex/pic/hit.png}
    \caption{The PrefHit}
    \label{scale:hit}
  \end{subfigure}
  \begin{subfigure}{0.49\linewidth}
    \includegraphics[width=\linewidth]{latex/pic/Recall.png}
    \caption{The PrefRecall}
    \label{scale:recall}
  \end{subfigure}
  \medskip
  \begin{subfigure}{0.48\linewidth}
    \includegraphics[width=\linewidth]{latex/pic/reward.png}
    \caption{The Reward}
    \label{scale:reward}
  \end{subfigure}
  \begin{subfigure}{0.48\linewidth}
    \includegraphics[width=\linewidth]{latex/pic/bleu.png}
    \caption{The BLEU}
    \label{scale:bleu}
  \end{subfigure}
  \caption{The performance with Confidence Interval (CI) of our SeAdpra and PRO at different data scales.}
  \label{fig:scale}
  % \vspace{-0.2cm}
\end{figure}
%%%%%%%%%%%%%%%%%%%%%%%%%%%%%%%%%%%%%%%%%%%%%%%%%%%%%%%%%%%%%%%%%%%%%%%%%%%%%%%%%%%%%%%%%%%%%%%%%%%%%%%%%%%%%%%%

\subsection{Ablation Study}

In this section, we discuss the effectiveness of each component of SeAdpra and its impact on various metrics. The results are presented in Table \ref{ablation}.

\textbf{Perceptual Comparison} aims to prevent the model from relying solely on linguistic probability ordering while neglecting the significance of APDF. Removing this Reward will significantly increase the margin, but PrefHit will decrease, which may hinder the model's ability to compare and learn the preference differences between responses.

\textbf{Perceptual Alignment} seeks to align with the optimal responses; removing it will lead to a significant decrease in PrefHit, while the Reward and accuracy metrics like CoSim will significantly increase, as it tends to favor preference over accuracy.

\textbf{Semantic Perceptual Distance} plays a crucial role in maintaining semantic accuracy in alignment learning. Removing it leads to a significant decrease in BLEU and Rouge. Since sacrificing accuracy recalls more possibilities, PrefHit decreases while PrefRecall increases. Moreover, eliminating both Semantic Perceptual Distance and Perceptual Alignment in \(PerCo_{Po}\) further increases PrefRecall, while the other metrics decline again, consistent with previous observations.


\textbf{Popularity Perceptual Distance} is most closely associated with PrefHit. Eliminating it causes PrefHit to drop to its lowest value, indicating that the popularity attribute is an extremely important factor in code communities.

% In summary, each module has a varying impact on preference and accuracy, but all outperform their respective foundation models and other baselines, as shown in Table \ref{main}, proving their effectiveness.


\subsection{Analysis and Discussion}

\textbf{SeAdpra adept at high-quality data rather than large-scale data.}
In StaCoCoQA, we tested PRO and SeAdpra across different data scales, and the results are shown in Figure~\ref{fig:scale}.
Since we rely on the popularity and clarity of questions and answers to filter data, a larger data scale often results in more pronounced deterioration in data quality. In Figure~\ref{scale:hit}, SeAdpra is highly sensitive to data quality in PrefHit, whereas PRO demonstrates improved performance with larger-scale data. Their performance on Prefrecall is consistent. In the native reward model of PRO, as depicted in Figure~\ref{scale:reward}, the reward fluctuations are minimal, while SeAdpra shows remarkable improvement.

\textbf{SeAdpra is relatively insensitive to ranking length.} 
We assessed SeAdpra's performance on different ranking lengths, as shown in Figure 6a. Unlike PRO, which varied with increasing ranking length, SeAdpra shows no significant differences across different lengths. There is a slight increase in performance on PrefHit and PrefRecall. Additionally, SeAdpra performs better at odd lengths compared to even lengths, which is an interesting phenomenon warranting further investigation.


\textbf{Balance Preference and Accuracy.} 
We analyzed the effect of control weights for Perceptual Comparisons in the optimization objective on preference and accuracy, with the findings presented in Figure~\ref{para:weight}.
When \( \alpha \) is greater than 0.05, the trends in PrefHit and BLEU are consistent, indicating that preference and accuracy can be optimized in tandem. However, when \( \alpha \) is 0.01, PrefHit is highest, but BLEU drops sharply.
Additionally, as \( \alpha \) changes, the variations in PrefHit and Reward, which are related to preference, are consistent with each other, reflecting their unified relationship in the optimization. Similarly, the variations in Recall and BLEU, which are related to accuracy, are also consistent, indicating a strong correlation between generation quality and comprehensiveness. 

%%%%%%%%%%%%%%%%%%%%%%%%%%%%%%%%%%%%%%%%%%%%%%%%%%%%%%%%%%%%%%%%%%%%%%%%%%%%%%%%%
\begin{figure}
  \centering
  \begin{subfigure}{0.475\linewidth}
    \includegraphics[width=\linewidth]{latex/pic/Rank1.png}
    \caption{Ranking length}
    \label{para:rank}
  \end{subfigure}
  \begin{subfigure}{0.475\linewidth}
    \includegraphics[width=\linewidth]{latex/pic/weights1.png}
    \caption{The \(\alpha\) in \(Loss\)}
    \label{para:weight}
  \end{subfigure}
  \caption{Parameters Analysis. Results of experiments on different ranking lengths and the weight \(\alpha\) in \(Loss\).}
  \label{fig:para}
  % \vspace{-0.2cm}
\end{figure}
%%%%%%%%%%%%%%%%%%%%%%%%%%%%%%%%%%%%%%%%%%%%
\begin{figure*}
  \centering
  \begin{subfigure}{0.305\linewidth}
    \includegraphics[width=\linewidth]{latex/pic/se2.pdf}
    \caption{The \(\Delta_{Se}\)}
    \label{visual:se}
  \end{subfigure}
  \begin{subfigure}{0.305\linewidth}
    \includegraphics[width=\linewidth]{latex/pic/po2.pdf}
    \caption{The \(\Delta_{Po}\)}
    \label{visual:po}
  \end{subfigure}
  \begin{subfigure}{0.305\linewidth}
    \includegraphics[width=\linewidth]{latex/pic/sv2.pdf}
    \caption{The \(\Delta_{M}\)}
    \label{visual:sv}
  \end{subfigure}
  \caption{The Visualization of Attribute-Perceptual Distance Factors (APDF) matrix of five responses. The blue represents the response with the highest APDF, and SeAdpra aligns with the fifth response corresponding to the maximum Multi-APDF in (c). The green represents the second response that is next best to the red one.}
  \label{visual}
  % \vspace{-0.2cm}
\end{figure*}
%%%%%%%%%%%%%%%%%%%%%%%%%%%%%%%%%%%%%%%%%
\textbf{Single-APDF Matrix Cannot Predict the Optimal Response.} We randomly selected a pair with a golden label and visualized its specific iteration in Figure~\ref{visual}.
It can be observed that the optimal response in a Single-APDF matrix is not necessarily the same as that in the Multi-APDF matrix.
Specifically, the optimal response in the Semantic Perceptual Factor matrix \(\Delta_{Se}\) is the fifth response in Figure~\ref{visual:se}, while in the Popularity Perceptual Factor matrix \(\Delta_{Po}\) (Figure~\ref{visual:po}), it is the third response. Ultimately, in the Multiple Perceptual Distance Factor matrix \(\Delta_{M}\), the third response is slightly inferior to the fifth response (0.037 vs. 0.038) in Figure~\ref{visual:sv}, and this result aligns with the golden label.
More key findings regarding the ADPF are described in Figure \ref{fig::hot1} and Figure \ref{fig::hot2}.
%\putsec{related}{Related Work}

\noindent \textbf{Efficient Radiance Field Rendering.}
%
The introduction of Neural Radiance Fields (NeRF)~\cite{mil:sri20} has
generated significant interest in efficient 3D scene representation and
rendering for radiance fields.
%
Over the past years, there has been a large amount of research aimed at
accelerating NeRFs through algorithmic or software
optimizations~\cite{mul:eva22,fri:yu22,che:fun23,sun:sun22}, and the
development of hardware
accelerators~\cite{lee:cho23,li:li23,son:wen23,mub:kan23,fen:liu24}.
%
The state-of-the-art method, 3D Gaussian splatting~\cite{ker:kop23}, has
further fueled interest in accelerating radiance field
rendering~\cite{rad:ste24,lee:lee24,nie:stu24,lee:rho24,ham:mel24} as it
employs rasterization primitives that can be rendered much faster than NeRFs.
%
However, previous research focused on software graphics rendering on
programmable cores or building dedicated hardware accelerators. In contrast,
\name{} investigates the potential of efficient radiance field rendering while
utilizing fixed-function units in graphics hardware.
%
To our knowledge, this is the first work that assesses the performance
implications of rendering Gaussian-based radiance fields on the hardware
graphics pipeline with software and hardware optimizations.

%%%%%%%%%%%%%%%%%%%%%%%%%%%%%%%%%%%%%%%%%%%%%%%%%%%%%%%%%%%%%%%%%%%%%%%%%%
\myparagraph{Enhancing Graphics Rendering Hardware.}
%
The performance advantage of executing graphics rendering on either
programmable shader cores or fixed-function units varies depending on the
rendering methods and hardware designs.
%
Previous studies have explored the performance implication of graphics hardware
design by developing simulation infrastructures for graphics
workloads~\cite{bar:gon06,gub:aam19,tin:sax23,arn:par13}.
%
Additionally, several studies have aimed to improve the performance of
special-purpose hardware such as ray tracing units in graphics
hardware~\cite{cho:now23,liu:cha21} and proposed hardware accelerators for
graphics applications~\cite{lu:hua17,ram:gri09}.
%
In contrast to these works, which primarily evaluate traditional graphics
workloads, our work focuses on improving the performance of volume rendering
workloads, such as Gaussian splatting, which require blending a huge number of
fragments per pixel.

%%%%%%%%%%%%%%%%%%%%%%%%%%%%%%%%%%%%%%%%%%%%%%%%%%%%%%%%%%%%%%%%%%%%%%%%%%
%
In the context of multi-sample anti-aliasing, prior work proposed reducing the
amount of redundant shading by merging fragments from adjacent triangles in a
mesh at the quad granularity~\cite{fat:bou10}.
%
While both our work and quad-fragment merging (QFM)~\cite{fat:bou10} aim to
reduce operations by merging quads, our proposed technique differs from QFM in
many aspects.
%
Our method aims to blend \emph{overlapping primitives} along the depth
direction and applies to quads from any primitive. In contrast, QFM merges quad
fragments from small (e.g., pixel-sized) triangles that \emph{share} an edge
(i.e., \emph{connected}, \emph{non-overlapping} triangles).
%
As such, QFM is not applicable to the scenes consisting of a number of
unconnected transparent triangles, such as those in 3D Gaussian splatting.
%
In addition, our method computes the \emph{exact} color for each pixel by
offloading blending operations from ROPs to shader units, whereas QFM
\emph{approximates} pixel colors by using the color from one triangle when
multiple triangles are merged into a single quad.


\section{Related Work}
The $k$-core decomposition problem has been extensively studied since the introduction by Seidman~\cite{seidman1983network}.
The first sequential algorithm was proposed by Matula and Beck~\cite{matula1983smallest},
using a bucket sort to arrange vertices by degree and iteratively deleting vertices with degree $k$ (peeling) until all are removed.
The algorithm runs in $O(m + n)$.
Batagelj and Zaversnik (\BZ)~\cite{batagelj2003m} provided a sequential implementation with the same time complexity. 

The \kcore{} problem is also carefully studied in the shared-memory parallel setting~\cite{dasari2014park,kabir2017parallel,dhulipala2017,montresor2011distributed,khaouid2015k,li2021k}, as well as included in many parallel graph libraries such as GraphX~\cite{gonzalez2014graphx}, Powergraph~\cite{gonzalez2012powergraph}, Ligra~\cite{shun2013ligra} and Julienne~\cite{dhulipala2017} (later integrated into the GBBS library~\cite{gbbs2021}).
In this paper, we compared to the three state-of-the-part solutions, including \ParK~\cite{dasari2014park}, \PKC~\cite{kabir2017parallel}, and \Julienne{}~\cite{dhulipala2017}.
Among them, \ParK and \PKC use the online peeling process, and \Julienne is offline.
More details about them were overviewed in \cref{sec:peeling}.

Given the wide applicability, \kcore is also extensively studied in other settings, such as on GPUs~\cite{mehrafsa2020vectorising, ahmad2023accelerating, zhao2024pico, li2021k, tripathy2018scalable, zhao2024speedcore, zhang2017accelerating, wang2016gunrock}, the external memory (disk) setting~\cite{cheng2011efficient, wen2018efficient}, and the low-memory setting~\cite{khaouid2015k}.
The techniques proposed in these papers mostly focused on the specific challenges in each setting, and have small overlaps with the new techniques introduced in this paper.

We are also aware of variants of $k$-core decomposition.
A direct extension is to maintain \kcore with vertex and edge updates.
Research has been done on this topic, for both the dynamic (online) setting~\cite{aksu2014distributed, aridhi2016distributed, gabert2022batch,liu2022parallel} and the streaming (offline) setting~\cite{sariyuce2013streaming, esfandiari2018parallel, sariyuce2016incremental}.
Another direction is computing approximate $k$-core decomposition, both in sequential settings~\cite{king2022computing} 
and parallel settings~\cite{esfandiari2018parallel,liu2022parallel, liu2024parallel, dhulipala2022differential}.
On directed graphs, s similar problem is referred to as $D$-core decomposition.
Algorithms for it have also been studied recently~\cite{luo2024efficient, liao2022distributed, giatsidis2013d}.
The $k$-core decomposition also relates to many other problems, 
such as dense subgraph discovery~\cite{luo2023scalable} and hierarchical core decomposition~\cite{chu2022hierarchical}.
How to apply our new techniques in these related problems can be an interesting future work.

\hide{are two state-of-the-art parallel algorithms using the online peeling process. 
They both use $O(n \maxcoreness + m)$ work. 
We introduced both of them in \cref{sec:peeling}. 
\PKC{} introduces optimizations, such as packing the remaining vertices after 98\% have been peeled, 
and using a local buffer to avoid subrounds.
While \PKC{} performs well on sparse graphs, 
both \Park{} and \PKC{} suffer from poor performance on dense graphs due to work-inefficiency and contention in updating induced degrees.

Dhulipala et al.~\cite{dhulipala2017} proposed a parallel \kcore{} algorithm in the \Julienne{} paper. 
In their paper, they use a bucketing structure that that maps each value $d$ to all vertices with induced degree $d$.
They proved that this algorithm has $O(m+n)$ work. In their implementation, they used a simplified bucketing structure that only 
maintains up to $b$ buckets. Our analysis proves the work-efficiency of their implementation, 
and an even simpler version without bucketing structure is also work-efficient. 
% Their algorithm is base on the offline peeling algorithm, 
\Julienne{} is based on the offline peeling approach, 
as we introduced in \cref{sec:peeling},
which is fully synchronized and race-free. 
It performs well on dense graphs but has unsatisfactory performance on sparse graphs, 
where synchronization overhead can be comparable to the computation cost.



We select the state-of-the-art exact \kcore{} algorithms mentioned above as our baselines for comparison,
given their performance and more general applicability to various graph types. 
Besides, \emph{MPM} algorithm solves \kcore{} problem in distributed settings ~\cite{montresor2011distributed}.
There are also studies focusing on optimizing \kcore{} decomposition in
low-memory settings~\cite{khaouid2015k} with implementation on GraphChi~\cite{KBG12},
as well as external memory settings~\cite{cheng2011efficient, wen2018efficient}.
Li et al.~\cite{li2021k} uses GraphBLAS to
adapt linear algebraic language to the \kcore{} problem on sparse graphs.

There are also recent studies based on GPU settings~\cite{mehrafsa2020vectorising, ahmad2023accelerating, zhao2024pico, li2021k, tripathy2018scalable, zhao2024speedcore, zhang2017accelerating, wang2016gunrock},
Many of them focus on adapting the peeling framework with different optimizations on GPU settings~\cite{mehrafsa2020vectorising,ahmad2023accelerating, zhao2024pico, tripathy2018scalable, zhao2024speedcore, wang2016gunrock}.
Specifically, \emph{PeelBlock}~\cite{ahmad2023accelerating} follows the \PKC{} design with buffer optimizations for GPUs.
\emph{SpeedCore} ~\cite{zhao2024speedcore} improves the performance of \emph{PeelBlock} by further optimizations.

Several shared-memory parallel and distributed computing frameworks support $k$-core decomposition as part of graph analytics,
including GraphX~\cite{gonzalez2014graphx}, Powergraph~\cite{gonzalez2012powergraph}, Ligra~\cite{shun2013ligra} and Julienne~\cite{dhulipala2017} (later integrated into the GBBS library~\cite{gbbs2021}).
While these frameworks provide primitives for parallel or distributed graph processing, 
they do not specifically optimize for the \kcore{} problem.
} 

\section{CONCLUSION}
\label{sec:concl}
The rapid rise of AI-generated media challenges information authenticity and societal trust, necessitating robust detection mechanisms. This survey examines the evolution of AI-generated media detection, focusing on the shift from Non-MLLM-based domain-specific detectors to MLLM-based general-purpose approaches. We compare these methods across authenticity, explainability, and localization tasks from both single-modal and multi-modal perspectives. Additionally, we review datasets, methodologies, and evaluation metrics, identifying key limitations and research challenges.
Beyond technical concerns, MLLM-based detection raises ethical and security issues. As GenAI sees broader deployment, regulatory frameworks vary significantly across jurisdictions, complicating governance. By summarizing these regulations, we provide insights for researchers navigating legal and ethical challenges.
While many challenges remain, We hope this survey sparks further discussion, informs future research, and contributes to a more secure and trustworthy AI ecosystem.
% \smallskip
% \myparagraph{Acknowledgments} We thank the reviewers for their comments.
% The work by Moshe Tennenholtz was supported by funding from the
% European Research Council (ERC) under the European Union's Horizon
% 2020 research and innovation programme (grant agreement 740435).

\bibliographystyle{ACM-Reference-Format}
\balance
\bibliography{bib/strings, bib/main}

\iffullversion{
  \clearpage
  \appendix
\subsection{Lloyd-Max Algorithm}
\label{subsec:Lloyd-Max}
For a given quantization bitwidth $B$ and an operand $\bm{X}$, the Lloyd-Max algorithm finds $2^B$ quantization levels $\{\hat{x}_i\}_{i=1}^{2^B}$ such that quantizing $\bm{X}$ by rounding each scalar in $\bm{X}$ to the nearest quantization level minimizes the quantization MSE. 

The algorithm starts with an initial guess of quantization levels and then iteratively computes quantization thresholds $\{\tau_i\}_{i=1}^{2^B-1}$ and updates quantization levels $\{\hat{x}_i\}_{i=1}^{2^B}$. Specifically, at iteration $n$, thresholds are set to the midpoints of the previous iteration's levels:
\begin{align*}
    \tau_i^{(n)}=\frac{\hat{x}_i^{(n-1)}+\hat{x}_{i+1}^{(n-1)}}2 \text{ for } i=1\ldots 2^B-1
\end{align*}
Subsequently, the quantization levels are re-computed as conditional means of the data regions defined by the new thresholds:
\begin{align*}
    \hat{x}_i^{(n)}=\mathbb{E}\left[ \bm{X} \big| \bm{X}\in [\tau_{i-1}^{(n)},\tau_i^{(n)}] \right] \text{ for } i=1\ldots 2^B
\end{align*}
where to satisfy boundary conditions we have $\tau_0=-\infty$ and $\tau_{2^B}=\infty$. The algorithm iterates the above steps until convergence.

Figure \ref{fig:lm_quant} compares the quantization levels of a $7$-bit floating point (E3M3) quantizer (left) to a $7$-bit Lloyd-Max quantizer (right) when quantizing a layer of weights from the GPT3-126M model at a per-tensor granularity. As shown, the Lloyd-Max quantizer achieves substantially lower quantization MSE. Further, Table \ref{tab:FP7_vs_LM7} shows the superior perplexity achieved by Lloyd-Max quantizers for bitwidths of $7$, $6$ and $5$. The difference between the quantizers is clear at 5 bits, where per-tensor FP quantization incurs a drastic and unacceptable increase in perplexity, while Lloyd-Max quantization incurs a much smaller increase. Nevertheless, we note that even the optimal Lloyd-Max quantizer incurs a notable ($\sim 1.5$) increase in perplexity due to the coarse granularity of quantization. 

\begin{figure}[h]
  \centering
  \includegraphics[width=0.7\linewidth]{sections/figures/LM7_FP7.pdf}
  \caption{\small Quantization levels and the corresponding quantization MSE of Floating Point (left) vs Lloyd-Max (right) Quantizers for a layer of weights in the GPT3-126M model.}
  \label{fig:lm_quant}
\end{figure}

\begin{table}[h]\scriptsize
\begin{center}
\caption{\label{tab:FP7_vs_LM7} \small Comparing perplexity (lower is better) achieved by floating point quantizers and Lloyd-Max quantizers on a GPT3-126M model for the Wikitext-103 dataset.}
\begin{tabular}{c|cc|c}
\hline
 \multirow{2}{*}{\textbf{Bitwidth}} & \multicolumn{2}{|c|}{\textbf{Floating-Point Quantizer}} & \textbf{Lloyd-Max Quantizer} \\
 & Best Format & Wikitext-103 Perplexity & Wikitext-103 Perplexity \\
\hline
7 & E3M3 & 18.32 & 18.27 \\
6 & E3M2 & 19.07 & 18.51 \\
5 & E4M0 & 43.89 & 19.71 \\
\hline
\end{tabular}
\end{center}
\end{table}

\subsection{Proof of Local Optimality of LO-BCQ}
\label{subsec:lobcq_opt_proof}
For a given block $\bm{b}_j$, the quantization MSE during LO-BCQ can be empirically evaluated as $\frac{1}{L_b}\lVert \bm{b}_j- \bm{\hat{b}}_j\rVert^2_2$ where $\bm{\hat{b}}_j$ is computed from equation (\ref{eq:clustered_quantization_definition}) as $C_{f(\bm{b}_j)}(\bm{b}_j)$. Further, for a given block cluster $\mathcal{B}_i$, we compute the quantization MSE as $\frac{1}{|\mathcal{B}_{i}|}\sum_{\bm{b} \in \mathcal{B}_{i}} \frac{1}{L_b}\lVert \bm{b}- C_i^{(n)}(\bm{b})\rVert^2_2$. Therefore, at the end of iteration $n$, we evaluate the overall quantization MSE $J^{(n)}$ for a given operand $\bm{X}$ composed of $N_c$ block clusters as:
\begin{align*}
    \label{eq:mse_iter_n}
    J^{(n)} = \frac{1}{N_c} \sum_{i=1}^{N_c} \frac{1}{|\mathcal{B}_{i}^{(n)}|}\sum_{\bm{v} \in \mathcal{B}_{i}^{(n)}} \frac{1}{L_b}\lVert \bm{b}- B_i^{(n)}(\bm{b})\rVert^2_2
\end{align*}

At the end of iteration $n$, the codebooks are updated from $\mathcal{C}^{(n-1)}$ to $\mathcal{C}^{(n)}$. However, the mapping of a given vector $\bm{b}_j$ to quantizers $\mathcal{C}^{(n)}$ remains as  $f^{(n)}(\bm{b}_j)$. At the next iteration, during the vector clustering step, $f^{(n+1)}(\bm{b}_j)$ finds new mapping of $\bm{b}_j$ to updated codebooks $\mathcal{C}^{(n)}$ such that the quantization MSE over the candidate codebooks is minimized. Therefore, we obtain the following result for $\bm{b}_j$:
\begin{align*}
\frac{1}{L_b}\lVert \bm{b}_j - C_{f^{(n+1)}(\bm{b}_j)}^{(n)}(\bm{b}_j)\rVert^2_2 \le \frac{1}{L_b}\lVert \bm{b}_j - C_{f^{(n)}(\bm{b}_j)}^{(n)}(\bm{b}_j)\rVert^2_2
\end{align*}

That is, quantizing $\bm{b}_j$ at the end of the block clustering step of iteration $n+1$ results in lower quantization MSE compared to quantizing at the end of iteration $n$. Since this is true for all $\bm{b} \in \bm{X}$, we assert the following:
\begin{equation}
\begin{split}
\label{eq:mse_ineq_1}
    \tilde{J}^{(n+1)} &= \frac{1}{N_c} \sum_{i=1}^{N_c} \frac{1}{|\mathcal{B}_{i}^{(n+1)}|}\sum_{\bm{b} \in \mathcal{B}_{i}^{(n+1)}} \frac{1}{L_b}\lVert \bm{b} - C_i^{(n)}(b)\rVert^2_2 \le J^{(n)}
\end{split}
\end{equation}
where $\tilde{J}^{(n+1)}$ is the the quantization MSE after the vector clustering step at iteration $n+1$.

Next, during the codebook update step (\ref{eq:quantizers_update}) at iteration $n+1$, the per-cluster codebooks $\mathcal{C}^{(n)}$ are updated to $\mathcal{C}^{(n+1)}$ by invoking the Lloyd-Max algorithm \citep{Lloyd}. We know that for any given value distribution, the Lloyd-Max algorithm minimizes the quantization MSE. Therefore, for a given vector cluster $\mathcal{B}_i$ we obtain the following result:

\begin{equation}
    \frac{1}{|\mathcal{B}_{i}^{(n+1)}|}\sum_{\bm{b} \in \mathcal{B}_{i}^{(n+1)}} \frac{1}{L_b}\lVert \bm{b}- C_i^{(n+1)}(\bm{b})\rVert^2_2 \le \frac{1}{|\mathcal{B}_{i}^{(n+1)}|}\sum_{\bm{b} \in \mathcal{B}_{i}^{(n+1)}} \frac{1}{L_b}\lVert \bm{b}- C_i^{(n)}(\bm{b})\rVert^2_2
\end{equation}

The above equation states that quantizing the given block cluster $\mathcal{B}_i$ after updating the associated codebook from $C_i^{(n)}$ to $C_i^{(n+1)}$ results in lower quantization MSE. Since this is true for all the block clusters, we derive the following result: 
\begin{equation}
\begin{split}
\label{eq:mse_ineq_2}
     J^{(n+1)} &= \frac{1}{N_c} \sum_{i=1}^{N_c} \frac{1}{|\mathcal{B}_{i}^{(n+1)}|}\sum_{\bm{b} \in \mathcal{B}_{i}^{(n+1)}} \frac{1}{L_b}\lVert \bm{b}- C_i^{(n+1)}(\bm{b})\rVert^2_2  \le \tilde{J}^{(n+1)}   
\end{split}
\end{equation}

Following (\ref{eq:mse_ineq_1}) and (\ref{eq:mse_ineq_2}), we find that the quantization MSE is non-increasing for each iteration, that is, $J^{(1)} \ge J^{(2)} \ge J^{(3)} \ge \ldots \ge J^{(M)}$ where $M$ is the maximum number of iterations. 
%Therefore, we can say that if the algorithm converges, then it must be that it has converged to a local minimum. 
\hfill $\blacksquare$


\begin{figure}
    \begin{center}
    \includegraphics[width=0.5\textwidth]{sections//figures/mse_vs_iter.pdf}
    \end{center}
    \caption{\small NMSE vs iterations during LO-BCQ compared to other block quantization proposals}
    \label{fig:nmse_vs_iter}
\end{figure}

Figure \ref{fig:nmse_vs_iter} shows the empirical convergence of LO-BCQ across several block lengths and number of codebooks. Also, the MSE achieved by LO-BCQ is compared to baselines such as MXFP and VSQ. As shown, LO-BCQ converges to a lower MSE than the baselines. Further, we achieve better convergence for larger number of codebooks ($N_c$) and for a smaller block length ($L_b$), both of which increase the bitwidth of BCQ (see Eq \ref{eq:bitwidth_bcq}).


\subsection{Additional Accuracy Results}
%Table \ref{tab:lobcq_config} lists the various LOBCQ configurations and their corresponding bitwidths.
\begin{table}
\setlength{\tabcolsep}{4.75pt}
\begin{center}
\caption{\label{tab:lobcq_config} Various LO-BCQ configurations and their bitwidths.}
\begin{tabular}{|c||c|c|c|c||c|c||c|} 
\hline
 & \multicolumn{4}{|c||}{$L_b=8$} & \multicolumn{2}{|c||}{$L_b=4$} & $L_b=2$ \\
 \hline
 \backslashbox{$L_A$\kern-1em}{\kern-1em$N_c$} & 2 & 4 & 8 & 16 & 2 & 4 & 2 \\
 \hline
 64 & 4.25 & 4.375 & 4.5 & 4.625 & 4.375 & 4.625 & 4.625\\
 \hline
 32 & 4.375 & 4.5 & 4.625& 4.75 & 4.5 & 4.75 & 4.75 \\
 \hline
 16 & 4.625 & 4.75& 4.875 & 5 & 4.75 & 5 & 5 \\
 \hline
\end{tabular}
\end{center}
\end{table}

%\subsection{Perplexity achieved by various LO-BCQ configurations on Wikitext-103 dataset}

\begin{table} \centering
\begin{tabular}{|c||c|c|c|c||c|c||c|} 
\hline
 $L_b \rightarrow$& \multicolumn{4}{c||}{8} & \multicolumn{2}{c||}{4} & 2\\
 \hline
 \backslashbox{$L_A$\kern-1em}{\kern-1em$N_c$} & 2 & 4 & 8 & 16 & 2 & 4 & 2  \\
 %$N_c \rightarrow$ & 2 & 4 & 8 & 16 & 2 & 4 & 2 \\
 \hline
 \hline
 \multicolumn{8}{c}{GPT3-1.3B (FP32 PPL = 9.98)} \\ 
 \hline
 \hline
 64 & 10.40 & 10.23 & 10.17 & 10.15 &  10.28 & 10.18 & 10.19 \\
 \hline
 32 & 10.25 & 10.20 & 10.15 & 10.12 &  10.23 & 10.17 & 10.17 \\
 \hline
 16 & 10.22 & 10.16 & 10.10 & 10.09 &  10.21 & 10.14 & 10.16 \\
 \hline
  \hline
 \multicolumn{8}{c}{GPT3-8B (FP32 PPL = 7.38)} \\ 
 \hline
 \hline
 64 & 7.61 & 7.52 & 7.48 &  7.47 &  7.55 &  7.49 & 7.50 \\
 \hline
 32 & 7.52 & 7.50 & 7.46 &  7.45 &  7.52 &  7.48 & 7.48  \\
 \hline
 16 & 7.51 & 7.48 & 7.44 &  7.44 &  7.51 &  7.49 & 7.47  \\
 \hline
\end{tabular}
\caption{\label{tab:ppl_gpt3_abalation} Wikitext-103 perplexity across GPT3-1.3B and 8B models.}
\end{table}

\begin{table} \centering
\begin{tabular}{|c||c|c|c|c||} 
\hline
 $L_b \rightarrow$& \multicolumn{4}{c||}{8}\\
 \hline
 \backslashbox{$L_A$\kern-1em}{\kern-1em$N_c$} & 2 & 4 & 8 & 16 \\
 %$N_c \rightarrow$ & 2 & 4 & 8 & 16 & 2 & 4 & 2 \\
 \hline
 \hline
 \multicolumn{5}{|c|}{Llama2-7B (FP32 PPL = 5.06)} \\ 
 \hline
 \hline
 64 & 5.31 & 5.26 & 5.19 & 5.18  \\
 \hline
 32 & 5.23 & 5.25 & 5.18 & 5.15  \\
 \hline
 16 & 5.23 & 5.19 & 5.16 & 5.14  \\
 \hline
 \multicolumn{5}{|c|}{Nemotron4-15B (FP32 PPL = 5.87)} \\ 
 \hline
 \hline
 64  & 6.3 & 6.20 & 6.13 & 6.08  \\
 \hline
 32  & 6.24 & 6.12 & 6.07 & 6.03  \\
 \hline
 16  & 6.12 & 6.14 & 6.04 & 6.02  \\
 \hline
 \multicolumn{5}{|c|}{Nemotron4-340B (FP32 PPL = 3.48)} \\ 
 \hline
 \hline
 64 & 3.67 & 3.62 & 3.60 & 3.59 \\
 \hline
 32 & 3.63 & 3.61 & 3.59 & 3.56 \\
 \hline
 16 & 3.61 & 3.58 & 3.57 & 3.55 \\
 \hline
\end{tabular}
\caption{\label{tab:ppl_llama7B_nemo15B} Wikitext-103 perplexity compared to FP32 baseline in Llama2-7B and Nemotron4-15B, 340B models}
\end{table}

%\subsection{Perplexity achieved by various LO-BCQ configurations on MMLU dataset}


\begin{table} \centering
\begin{tabular}{|c||c|c|c|c||c|c|c|c|} 
\hline
 $L_b \rightarrow$& \multicolumn{4}{c||}{8} & \multicolumn{4}{c||}{8}\\
 \hline
 \backslashbox{$L_A$\kern-1em}{\kern-1em$N_c$} & 2 & 4 & 8 & 16 & 2 & 4 & 8 & 16  \\
 %$N_c \rightarrow$ & 2 & 4 & 8 & 16 & 2 & 4 & 2 \\
 \hline
 \hline
 \multicolumn{5}{|c|}{Llama2-7B (FP32 Accuracy = 45.8\%)} & \multicolumn{4}{|c|}{Llama2-70B (FP32 Accuracy = 69.12\%)} \\ 
 \hline
 \hline
 64 & 43.9 & 43.4 & 43.9 & 44.9 & 68.07 & 68.27 & 68.17 & 68.75 \\
 \hline
 32 & 44.5 & 43.8 & 44.9 & 44.5 & 68.37 & 68.51 & 68.35 & 68.27  \\
 \hline
 16 & 43.9 & 42.7 & 44.9 & 45 & 68.12 & 68.77 & 68.31 & 68.59  \\
 \hline
 \hline
 \multicolumn{5}{|c|}{GPT3-22B (FP32 Accuracy = 38.75\%)} & \multicolumn{4}{|c|}{Nemotron4-15B (FP32 Accuracy = 64.3\%)} \\ 
 \hline
 \hline
 64 & 36.71 & 38.85 & 38.13 & 38.92 & 63.17 & 62.36 & 63.72 & 64.09 \\
 \hline
 32 & 37.95 & 38.69 & 39.45 & 38.34 & 64.05 & 62.30 & 63.8 & 64.33  \\
 \hline
 16 & 38.88 & 38.80 & 38.31 & 38.92 & 63.22 & 63.51 & 63.93 & 64.43  \\
 \hline
\end{tabular}
\caption{\label{tab:mmlu_abalation} Accuracy on MMLU dataset across GPT3-22B, Llama2-7B, 70B and Nemotron4-15B models.}
\end{table}


%\subsection{Perplexity achieved by various LO-BCQ configurations on LM evaluation harness}

\begin{table} \centering
\begin{tabular}{|c||c|c|c|c||c|c|c|c|} 
\hline
 $L_b \rightarrow$& \multicolumn{4}{c||}{8} & \multicolumn{4}{c||}{8}\\
 \hline
 \backslashbox{$L_A$\kern-1em}{\kern-1em$N_c$} & 2 & 4 & 8 & 16 & 2 & 4 & 8 & 16  \\
 %$N_c \rightarrow$ & 2 & 4 & 8 & 16 & 2 & 4 & 2 \\
 \hline
 \hline
 \multicolumn{5}{|c|}{Race (FP32 Accuracy = 37.51\%)} & \multicolumn{4}{|c|}{Boolq (FP32 Accuracy = 64.62\%)} \\ 
 \hline
 \hline
 64 & 36.94 & 37.13 & 36.27 & 37.13 & 63.73 & 62.26 & 63.49 & 63.36 \\
 \hline
 32 & 37.03 & 36.36 & 36.08 & 37.03 & 62.54 & 63.51 & 63.49 & 63.55  \\
 \hline
 16 & 37.03 & 37.03 & 36.46 & 37.03 & 61.1 & 63.79 & 63.58 & 63.33  \\
 \hline
 \hline
 \multicolumn{5}{|c|}{Winogrande (FP32 Accuracy = 58.01\%)} & \multicolumn{4}{|c|}{Piqa (FP32 Accuracy = 74.21\%)} \\ 
 \hline
 \hline
 64 & 58.17 & 57.22 & 57.85 & 58.33 & 73.01 & 73.07 & 73.07 & 72.80 \\
 \hline
 32 & 59.12 & 58.09 & 57.85 & 58.41 & 73.01 & 73.94 & 72.74 & 73.18  \\
 \hline
 16 & 57.93 & 58.88 & 57.93 & 58.56 & 73.94 & 72.80 & 73.01 & 73.94  \\
 \hline
\end{tabular}
\caption{\label{tab:mmlu_abalation} Accuracy on LM evaluation harness tasks on GPT3-1.3B model.}
\end{table}

\begin{table} \centering
\begin{tabular}{|c||c|c|c|c||c|c|c|c|} 
\hline
 $L_b \rightarrow$& \multicolumn{4}{c||}{8} & \multicolumn{4}{c||}{8}\\
 \hline
 \backslashbox{$L_A$\kern-1em}{\kern-1em$N_c$} & 2 & 4 & 8 & 16 & 2 & 4 & 8 & 16  \\
 %$N_c \rightarrow$ & 2 & 4 & 8 & 16 & 2 & 4 & 2 \\
 \hline
 \hline
 \multicolumn{5}{|c|}{Race (FP32 Accuracy = 41.34\%)} & \multicolumn{4}{|c|}{Boolq (FP32 Accuracy = 68.32\%)} \\ 
 \hline
 \hline
 64 & 40.48 & 40.10 & 39.43 & 39.90 & 69.20 & 68.41 & 69.45 & 68.56 \\
 \hline
 32 & 39.52 & 39.52 & 40.77 & 39.62 & 68.32 & 67.43 & 68.17 & 69.30  \\
 \hline
 16 & 39.81 & 39.71 & 39.90 & 40.38 & 68.10 & 66.33 & 69.51 & 69.42  \\
 \hline
 \hline
 \multicolumn{5}{|c|}{Winogrande (FP32 Accuracy = 67.88\%)} & \multicolumn{4}{|c|}{Piqa (FP32 Accuracy = 78.78\%)} \\ 
 \hline
 \hline
 64 & 66.85 & 66.61 & 67.72 & 67.88 & 77.31 & 77.42 & 77.75 & 77.64 \\
 \hline
 32 & 67.25 & 67.72 & 67.72 & 67.00 & 77.31 & 77.04 & 77.80 & 77.37  \\
 \hline
 16 & 68.11 & 68.90 & 67.88 & 67.48 & 77.37 & 78.13 & 78.13 & 77.69  \\
 \hline
\end{tabular}
\caption{\label{tab:mmlu_abalation} Accuracy on LM evaluation harness tasks on GPT3-8B model.}
\end{table}

\begin{table} \centering
\begin{tabular}{|c||c|c|c|c||c|c|c|c|} 
\hline
 $L_b \rightarrow$& \multicolumn{4}{c||}{8} & \multicolumn{4}{c||}{8}\\
 \hline
 \backslashbox{$L_A$\kern-1em}{\kern-1em$N_c$} & 2 & 4 & 8 & 16 & 2 & 4 & 8 & 16  \\
 %$N_c \rightarrow$ & 2 & 4 & 8 & 16 & 2 & 4 & 2 \\
 \hline
 \hline
 \multicolumn{5}{|c|}{Race (FP32 Accuracy = 40.67\%)} & \multicolumn{4}{|c|}{Boolq (FP32 Accuracy = 76.54\%)} \\ 
 \hline
 \hline
 64 & 40.48 & 40.10 & 39.43 & 39.90 & 75.41 & 75.11 & 77.09 & 75.66 \\
 \hline
 32 & 39.52 & 39.52 & 40.77 & 39.62 & 76.02 & 76.02 & 75.96 & 75.35  \\
 \hline
 16 & 39.81 & 39.71 & 39.90 & 40.38 & 75.05 & 73.82 & 75.72 & 76.09  \\
 \hline
 \hline
 \multicolumn{5}{|c|}{Winogrande (FP32 Accuracy = 70.64\%)} & \multicolumn{4}{|c|}{Piqa (FP32 Accuracy = 79.16\%)} \\ 
 \hline
 \hline
 64 & 69.14 & 70.17 & 70.17 & 70.56 & 78.24 & 79.00 & 78.62 & 78.73 \\
 \hline
 32 & 70.96 & 69.69 & 71.27 & 69.30 & 78.56 & 79.49 & 79.16 & 78.89  \\
 \hline
 16 & 71.03 & 69.53 & 69.69 & 70.40 & 78.13 & 79.16 & 79.00 & 79.00  \\
 \hline
\end{tabular}
\caption{\label{tab:mmlu_abalation} Accuracy on LM evaluation harness tasks on GPT3-22B model.}
\end{table}

\begin{table} \centering
\begin{tabular}{|c||c|c|c|c||c|c|c|c|} 
\hline
 $L_b \rightarrow$& \multicolumn{4}{c||}{8} & \multicolumn{4}{c||}{8}\\
 \hline
 \backslashbox{$L_A$\kern-1em}{\kern-1em$N_c$} & 2 & 4 & 8 & 16 & 2 & 4 & 8 & 16  \\
 %$N_c \rightarrow$ & 2 & 4 & 8 & 16 & 2 & 4 & 2 \\
 \hline
 \hline
 \multicolumn{5}{|c|}{Race (FP32 Accuracy = 44.4\%)} & \multicolumn{4}{|c|}{Boolq (FP32 Accuracy = 79.29\%)} \\ 
 \hline
 \hline
 64 & 42.49 & 42.51 & 42.58 & 43.45 & 77.58 & 77.37 & 77.43 & 78.1 \\
 \hline
 32 & 43.35 & 42.49 & 43.64 & 43.73 & 77.86 & 75.32 & 77.28 & 77.86  \\
 \hline
 16 & 44.21 & 44.21 & 43.64 & 42.97 & 78.65 & 77 & 76.94 & 77.98  \\
 \hline
 \hline
 \multicolumn{5}{|c|}{Winogrande (FP32 Accuracy = 69.38\%)} & \multicolumn{4}{|c|}{Piqa (FP32 Accuracy = 78.07\%)} \\ 
 \hline
 \hline
 64 & 68.9 & 68.43 & 69.77 & 68.19 & 77.09 & 76.82 & 77.09 & 77.86 \\
 \hline
 32 & 69.38 & 68.51 & 68.82 & 68.90 & 78.07 & 76.71 & 78.07 & 77.86  \\
 \hline
 16 & 69.53 & 67.09 & 69.38 & 68.90 & 77.37 & 77.8 & 77.91 & 77.69  \\
 \hline
\end{tabular}
\caption{\label{tab:mmlu_abalation} Accuracy on LM evaluation harness tasks on Llama2-7B model.}
\end{table}

\begin{table} \centering
\begin{tabular}{|c||c|c|c|c||c|c|c|c|} 
\hline
 $L_b \rightarrow$& \multicolumn{4}{c||}{8} & \multicolumn{4}{c||}{8}\\
 \hline
 \backslashbox{$L_A$\kern-1em}{\kern-1em$N_c$} & 2 & 4 & 8 & 16 & 2 & 4 & 8 & 16  \\
 %$N_c \rightarrow$ & 2 & 4 & 8 & 16 & 2 & 4 & 2 \\
 \hline
 \hline
 \multicolumn{5}{|c|}{Race (FP32 Accuracy = 48.8\%)} & \multicolumn{4}{|c|}{Boolq (FP32 Accuracy = 85.23\%)} \\ 
 \hline
 \hline
 64 & 49.00 & 49.00 & 49.28 & 48.71 & 82.82 & 84.28 & 84.03 & 84.25 \\
 \hline
 32 & 49.57 & 48.52 & 48.33 & 49.28 & 83.85 & 84.46 & 84.31 & 84.93  \\
 \hline
 16 & 49.85 & 49.09 & 49.28 & 48.99 & 85.11 & 84.46 & 84.61 & 83.94  \\
 \hline
 \hline
 \multicolumn{5}{|c|}{Winogrande (FP32 Accuracy = 79.95\%)} & \multicolumn{4}{|c|}{Piqa (FP32 Accuracy = 81.56\%)} \\ 
 \hline
 \hline
 64 & 78.77 & 78.45 & 78.37 & 79.16 & 81.45 & 80.69 & 81.45 & 81.5 \\
 \hline
 32 & 78.45 & 79.01 & 78.69 & 80.66 & 81.56 & 80.58 & 81.18 & 81.34  \\
 \hline
 16 & 79.95 & 79.56 & 79.79 & 79.72 & 81.28 & 81.66 & 81.28 & 80.96  \\
 \hline
\end{tabular}
\caption{\label{tab:mmlu_abalation} Accuracy on LM evaluation harness tasks on Llama2-70B model.}
\end{table}

%\section{MSE Studies}
%\textcolor{red}{TODO}


\subsection{Number Formats and Quantization Method}
\label{subsec:numFormats_quantMethod}
\subsubsection{Integer Format}
An $n$-bit signed integer (INT) is typically represented with a 2s-complement format \citep{yao2022zeroquant,xiao2023smoothquant,dai2021vsq}, where the most significant bit denotes the sign.

\subsubsection{Floating Point Format}
An $n$-bit signed floating point (FP) number $x$ comprises of a 1-bit sign ($x_{\mathrm{sign}}$), $B_m$-bit mantissa ($x_{\mathrm{mant}}$) and $B_e$-bit exponent ($x_{\mathrm{exp}}$) such that $B_m+B_e=n-1$. The associated constant exponent bias ($E_{\mathrm{bias}}$) is computed as $(2^{{B_e}-1}-1)$. We denote this format as $E_{B_e}M_{B_m}$.  

\subsubsection{Quantization Scheme}
\label{subsec:quant_method}
A quantization scheme dictates how a given unquantized tensor is converted to its quantized representation. We consider FP formats for the purpose of illustration. Given an unquantized tensor $\bm{X}$ and an FP format $E_{B_e}M_{B_m}$, we first, we compute the quantization scale factor $s_X$ that maps the maximum absolute value of $\bm{X}$ to the maximum quantization level of the $E_{B_e}M_{B_m}$ format as follows:
\begin{align}
\label{eq:sf}
    s_X = \frac{\mathrm{max}(|\bm{X}|)}{\mathrm{max}(E_{B_e}M_{B_m})}
\end{align}
In the above equation, $|\cdot|$ denotes the absolute value function.

Next, we scale $\bm{X}$ by $s_X$ and quantize it to $\hat{\bm{X}}$ by rounding it to the nearest quantization level of $E_{B_e}M_{B_m}$ as:

\begin{align}
\label{eq:tensor_quant}
    \hat{\bm{X}} = \text{round-to-nearest}\left(\frac{\bm{X}}{s_X}, E_{B_e}M_{B_m}\right)
\end{align}

We perform dynamic max-scaled quantization \citep{wu2020integer}, where the scale factor $s$ for activations is dynamically computed during runtime.

\subsection{Vector Scaled Quantization}
\begin{wrapfigure}{r}{0.35\linewidth}
  \centering
  \includegraphics[width=\linewidth]{sections/figures/vsquant.jpg}
  \caption{\small Vectorwise decomposition for per-vector scaled quantization (VSQ \citep{dai2021vsq}).}
  \label{fig:vsquant}
\end{wrapfigure}
During VSQ \citep{dai2021vsq}, the operand tensors are decomposed into 1D vectors in a hardware friendly manner as shown in Figure \ref{fig:vsquant}. Since the decomposed tensors are used as operands in matrix multiplications during inference, it is beneficial to perform this decomposition along the reduction dimension of the multiplication. The vectorwise quantization is performed similar to tensorwise quantization described in Equations \ref{eq:sf} and \ref{eq:tensor_quant}, where a scale factor $s_v$ is required for each vector $\bm{v}$ that maps the maximum absolute value of that vector to the maximum quantization level. While smaller vector lengths can lead to larger accuracy gains, the associated memory and computational overheads due to the per-vector scale factors increases. To alleviate these overheads, VSQ \citep{dai2021vsq} proposed a second level quantization of the per-vector scale factors to unsigned integers, while MX \citep{rouhani2023shared} quantizes them to integer powers of 2 (denoted as $2^{INT}$).

\subsubsection{MX Format}
The MX format proposed in \citep{rouhani2023microscaling} introduces the concept of sub-block shifting. For every two scalar elements of $b$-bits each, there is a shared exponent bit. The value of this exponent bit is determined through an empirical analysis that targets minimizing quantization MSE. We note that the FP format $E_{1}M_{b}$ is strictly better than MX from an accuracy perspective since it allocates a dedicated exponent bit to each scalar as opposed to sharing it across two scalars. Therefore, we conservatively bound the accuracy of a $b+2$-bit signed MX format with that of a $E_{1}M_{b}$ format in our comparisons. For instance, we use E1M2 format as a proxy for MX4.

\begin{figure}
    \centering
    \includegraphics[width=1\linewidth]{sections//figures/BlockFormats.pdf}
    \caption{\small Comparing LO-BCQ to MX format.}
    \label{fig:block_formats}
\end{figure}

Figure \ref{fig:block_formats} compares our $4$-bit LO-BCQ block format to MX \citep{rouhani2023microscaling}. As shown, both LO-BCQ and MX decompose a given operand tensor into block arrays and each block array into blocks. Similar to MX, we find that per-block quantization ($L_b < L_A$) leads to better accuracy due to increased flexibility. While MX achieves this through per-block $1$-bit micro-scales, we associate a dedicated codebook to each block through a per-block codebook selector. Further, MX quantizes the per-block array scale-factor to E8M0 format without per-tensor scaling. In contrast during LO-BCQ, we find that per-tensor scaling combined with quantization of per-block array scale-factor to E4M3 format results in superior inference accuracy across models. 

}
\end{document}
\endinput
%%
%% End of file `main.tex'.
