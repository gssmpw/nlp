\clearpage
% \section*{Appendix}
\label{sec:appendix}


%\section*{B. Pseudo-code for the \FUpdate{} Function}
%\begin{algorithm}[h]
    \small
    \caption{\FGetBucket $()$ in \HBS}
    \label{algo:getbucket}
    \KwIn{identifier $a$ (vertex) with its key $\degreestar[a]$}
    \KwOut{$a$'s bucket index}
    \SetKwInOut{Maintains}{Maintains}
    \Maintains{
        $base\_k$: the key value of the first bucket\\  
        $\degreestar[a]$: the \induceddegree of vertex $a$\\
    }
    
    \SetKwProg{myproc}{procedure}{}{}
    \myproc{\FGetBucket{$(a)$}}{
        \tcc{The function returns the bucket index of the indentifier $a$, 
        and is used in the \HBS algorithm to determine the bucket index of a vertex.}
        % \tcp{Check if degree is within the valid range}
        % \If{$d$ is not in the HBS range}{
        %     \textbf{return}
        % }
        \tcp{Check if the identifier is in the first 2 buckets}
        \If{$\degreestar[a] < base\_k + 2$}{
            \textbf{return} $\degreestar[a] - base\_k$ \\
        }
        \Else{
            \tcc{Calculate the index for \HBS based on bit difference, $\oplus $ is the bitwise XOR operation, $\mf{leading\_zero\_count}$ is the number of leading zeros}
            \textbf{return} $\gets 64 - \mf{leading\_zero\_count}(\degreestar[a]$ $ \oplus $ $base\_k)$ 

        }
    }
\end{algorithm} 

\revise{
\section*{A.~~Speedup with and without Sampling}

\begin{figure}[!t]
  \centering
  \includegraphics[width=\columnwidth]{figures/charts/sampling.pdf}
  \caption{\textbf{Running time comparison w/ and w/o sampling.}
  The numbers at the title of each subfigure are the running time speedup of the sampling method over the non-sampling method.
  }\label{fig:exp-sampling}
\end{figure}

To make the effect of sampling clearer, here we show a separate figure to illustrate the improvement from sampling on the eight graphs that trigger sampling. 
We compare the running time of our algorithm with and without sampling. The results are shown in \cref{fig:exp-sampling}. 
The only graph whose performance is not improved by sampling is \HCNS{},
because sampling does not provide contention reduction for the graph,
while it adds overhead to the algorithm.
}

\section*{B. Maximum $k$-Core Subgraph}

As a fundamental component of many dense subgraph discovery algorithms, $k$-core decomposition has applications across a variety of domains. 
One frequently encountered task is identifying the maximum $k'$-core subgraph within a given graph, where all vertices have a degree of at least $k'$. Similar to the $k$-core decomposition process, this requires iterative peeling rounds to extract the $k'$-core subgraph. 
Our framework can be easily adapted to address this problem by modifying the peeling condition. Furthermore, local search techniques and sampling-based optimizations can be integrated to enhance performance across different graph types.

To the best of our knowledge, Galois~\cite{Nguyen2014} implemented the state-of-the-art algorithm for this subgraph finding problem.
We adapt our algorithm framework to the maximum $k-$core problem this problem and compared it with Galois.
In practice, subgraph finding is usually applied to dense social networks or web graphs,
which is critical for community detection and anomaly detection~\cite{zhang2017finding, kitsak2010identification,boguna2004models, giatsidis2011evaluating}.
We tested our implementation on two representative social networks, com-orkut~\cite{yang2015defining} and twitter~\cite{kwak2010twitter},
with $k$ value range from 16 to 2048.
Our algorithm outperforms Galois on both graphs, with a speedup of up to $3.5\times$.
The detailed results are shown in \cref{fig:subk}.
\revise{
\section*{C. Burdened Span Analysis}
As mentioned in the main paper, to better understand how VGC improves the burdened span and performance, we test the burdened span for our algorithm (with and without VGC), and compare it with \GBBS{}. 
In \cref{fig:app:burdened_span}, we present the speedup of our algorithm (with and without VGC) over \Julienne{} on burdened span. This is the same figure shown in our main paper. 
Here we further put a comparison on the running time of our algorithm (with and without VGC) over \Julienne{} in \cref{fig:app:burdened_span_speedup}. 
%We compare the running speedup of our algorithm with and without VGC compared to \GBBS~\cite{dhulipala2017, gbbs2021} on all graphs in \cref{fig:app:burdened_span_speedup},
%addressing the running time improvement related to burdened span speedup.
Comparing the trends, in most of the cases, the graphs that achieves the most significant speedup in burdened span also benefit the most from the running time. This indicates that a major performance gain of our algorithm over \Julienne{} is to reduce the burdened span using VGC. 


\begin{figure}[t]
  \centering
  \includegraphics[width=\columnwidth]{figures/charts/subk.pdf}
  \caption{\textbf{Comparison of our adapted subgraph finding algorithm and Galois. 
  The $x$-axis is the $k$ value for subgraph finding.
  The running time at $y$-axis is the time in seconds.}
  The running time at the title is the $k$-core decomposition time of our algorithm in seconds.}\label{fig:subk}
\end{figure}


\begin{figure}[!t]
  \centering
  \includegraphics[width=\columnwidth]{figures/heatmap.pdf}
  \caption{\label{fig:app:heatmap}\textbf{
    Heatmap of our 8 versions of algorithms (with and without VGC, sampling, and HBS). }
    The data is normalized to the minimum running time for each graph.
    The color gradient represents the percentile distribution of the running time data for the corresponding graph.
  }
\end{figure}


\section*{D. Evaluating All Combinations of the Three Techniques}

The results in \cref{table:combinations} shows the relative running time of eight combinations of the three proposed techniques (VGC, sampling, and HBS) on all graphs. The data is normalized to the minimum running time for each graph. We also give the raw data in \cref{table:combinations}. 
The heatmap in \cref{fig:app:heatmap} visualizes this data, 
with colors ranging from green to dark red representing relative running times on a 1 to 40 scale based on percentiles.
HBS, VGC, and sampling each contribute to performance improvements, though their impacts vary across graph types.
Among social graphs such as \TW and \CW, sampling plays a more critical role in improving the performance.
For sparse graphs such as \GL and \COS, VGC is the key technology to reduce the running time.
HBS provides consistent performance gains across all graphs and is particularly effective on sparse graphs. 
Interestingly, for each of graph, especially the adversarial cases, we observed that usually the good performance is provided by a specific combination of the techniques. For example, \CBC{} relies on both VGC and HBS to achieve a satisfactory performance; \SD{} requires both VGC and sampling; \HPL{} can achieve close-to-the-best performance just by sampling; and some graphs, such as \EH{}, cannot achieve the best performance without any of the three techniques. This indicates that the combinations of the three techniques in our solution is essential to guarantee the overall good performance and to handle various adversarial input instances. 

\clearpage

\begin{figure*}[b]
  \centering
    \includegraphics[width=\textwidth]{figures/charts/span_ratio.pdf}
  \caption{\textbf{Burdened span~\cite{he2010cilkview} speedup of our algorithm (with and without VGC) over \GBBS~\cite{dhulipala2017, gbbs2021} (green dotted line, always$=1$) on all graphs. Higher is better.}
  The bars are truncated at 18 for better visualization. The text on the bars are actual burdened span speedup.
  When no HBS is used, we uses 16 buckets as in the \Julienne{} implementation. 
  % ``N/A'' indicates that the running time is greater than 1000 seconds, or the algorithm runs out of memory.
  \label{fig:app:burdened_span}}
  \includegraphics[width=\textwidth]{figures/charts/span_speedup.pdf}
  \caption{\textbf{Running time speedup of our algorithm (with and without VGC) over \GBBS~\cite{dhulipala2017, gbbs2021} (green dotted line, always$=1$) on all graphs. Higher is better.}
  The bars are truncated at 4 for better visualization. The text on the bars are actual running time speedup.
  When no HBS is used, we uses 16 buckets as in the \Julienne{} implementation. 
  % ``N/A'' indicates that the running time is greater than 1000 seconds, or the algorithm runs out of memory.
  \label{fig:app:burdened_span_speedup}}
\end{figure*} 

% Table generated by Excel2LaTeX from sheet 'Graph_Information'
\begin{table*}[htbp] 
      \centering
      \small
      \vspace{-1em}

    \begin{tabular}{ccrrrr|rrrrrrrr|l}
          & \multicolumn{5}{c|}{\textit{\textbf{Graph Statistics}}} & \multicolumn{1}{c}{\multirow{2}[1]{*}{Plain}} & \multicolumn{1}{c}{\multirow{2}[1]{*}{VGC}} & \multicolumn{1}{c}{\multirow{2}[1]{*}{Sample}} & \multicolumn{1}{c}{\multirow{2}[1]{*}{HBS}} & \multicolumn{1}{c}{VGC+} & \multicolumn{1}{c}{VGC+} & \multicolumn{1}{c}{Sample+} & \multicolumn{1}{c|}{\multirow{2}[1]{*}{All}} & \multicolumn{1}{c}{\multirow{2}[1]{*}{\textit{\textbf{Notes}}}} \\
          & Name  & \multicolumn{1}{c}{$n$} & \multicolumn{1}{c}{$m$} & \multicolumn{1}{c}{$\maxcoreness$} & \multicolumn{1}{c|}{$\rho$} &       &       &       &       & \multicolumn{1}{c}{Sample} & \multicolumn{1}{c}{HBS} & \multicolumn{1}{c}{HBS} &       &  \\
\cmidrule{2-15}    \multicolumn{1}{c}{\multirow{5}[2]{*}{\begin{sideways}Social\end{sideways}}} & LJ    & 4.85M & 85.7M & 372   & 3,480 & .275  & .220  & .276  & .272  & .265  & \textbf{.200} & .265  & .203  & soc-LiveJournal1~\cite{backstrom2006group} \\
          & OK    & 3.07M & 234M  & 253   & 5,667 & .528  & .540  & .488  & .487  & .474  & .510  & .474  & .526  & com-orkut~\cite{yang2015defining} \\
          & WB    & 58.7M & 523M  & 193   & 2,910 & .934  & .831  & .902  & .937  & .946  & .913  & .946  & .935  & soc-sinaweibo~\cite{nr} \\
          & TW    & 41.7M & 2.41B & 2,488 & 14,964 & 7.15  & 7.09  & 2.71  & 6.77  & 2.74  & 6.73  & 2.74  & 2.72  & Twitter~\cite{kwak2010twitter} \\
          & FS    & 65.6M & 3.61B & 304   & 10,034 & 3.85  & 3.90  & 3.59  & 3.86  & 3.67  & 3.70  & 3.67  & 3.67  & Friendster~\cite{yang2015defining} \\
\cmidrule{2-15}    \multirow{5}[2]{*}{\begin{sideways}Web\end{sideways}} & EH    & 11.3M & 522M  & 9,877 & 7,393 & 1.25  & 1.07  & 1.04  & 1.23  & .996  & 1.00  & .996  & .795  & eu-host~\cite{webgraph} \\
          & SD    & 89.3M & 3.88B & 10,507 & 19,063 & 5.03  & 5.07  & 5.70  & 4.96  & 4.37  & 4.97  & 4.37  & 4.39  & sd-arc~\cite{webgraph} \\
          & CW    & 978M  & 74.7B & 4,244 & 106,819 & 171   & 166   & 36.1  & 165   & 38.3  & 157   & 38.3  & 28.6  & ClueWeb~\cite{webgraph} \\
          & HL14  & 1.72B & 124B  & 4,160 & 58,737 & 123   & 103   & 78.0  & 118   & 65.0  & 103   & 65.0  & 54.7  & Hyperlink14~\cite{webgraph} \\
          & HL12  & 3.56B & 226B  & 10,565 & 130,737 & 166   & 148   & 143   & 157   & 138   & 130   & 138   & 108.4 & Hyperlink12~\cite{webgraph} \\
\cmidrule{2-15}    \multirow{4}[2]{*}{\begin{sideways}Road\end{sideways}} & AF    & 33.5M & 88.9M & 3     & 189   & .372  & .219  & .366  & .294  & .288  & .154  & .288  & .155  & OSM Africa~\cite{roadgraph} \\
          & NA    & 87.0M & 220M  & 4     & 286   & .946  & .605  & .931  & .751  & .739  & .437  & .739  & .432  & OSM North America~\cite{roadgraph} \\
          & AS    & 95.7M & 244M  & 4     & 343   & 1.02  & .674  & 1.01  & .818  & .816  & .471  & .816  & .480  & OSM Asia~\cite{roadgraph} \\
          & EU    & 131M  & 333M  & 4     & 513   & 1.39  & .948  & 1.40  & 1.11  & 1.10  & .666  & 1.10  & .679  & OSM Europe~\cite{roadgraph} \\
\cmidrule{2-15}    \multirow{5}[2]{*}{\begin{sideways}\KNN\end{sideways}} & CH5   & 4.21M & 29.7M & 5     & 7     & .058  & .033  & .059  & .045  & .046  & .021  & .046  & .021  & Chem~\cite{fonollosa2015reservoir,wang2021geograph}, $k=5$ \\
          & GL2   & 24.9M & 65.3M & 2     & 12    & .223  & .133  & .224  & .187  & .187  & .106  & .187  & .109  & GeoLife~\cite{geolife,wang2021geograph}, $k=2$ \\
          & GL5   & 24.9M & 157M  & 5     & 42    & .306  & .168  & .299  & .253  & .246  & .120  & .246  & .125  & GeoLife~\cite{geolife,wang2021geograph}, $k=5$ \\
          & GL10  & 24.9M & 310M  & 10    & 16    & .380  & .206  & .370  & .320  & .319  & .154  & .319  & .162  & GeoLife~\cite{geolife,wang2021geograph}, $k=10$ \\
          & COS5  & 321M  & 1.96B & 2     & 23    & 4.33  & 2.58  & 4.38  & 3.71  & 3.68  & 2.04  & 3.68  & 2.04  & Cosmo50~\cite{cosmo50,wang2021geograph}, $k=5$ \\
\cmidrule{2-15}    \multirow{6}[2]{*}{\begin{sideways}Others\end{sideways}} & TRCE  & 16.0M & 48.0M & 2     & 1,839 & .638  & .095  & .628  & .521  & .545  & .067  & .545  & .066  & Huge traces~\cite{nr} \\
          & BBL   & 21.2M & 63.6M & 2     & 1,915 & .712  & .129  & .699  & .616  & .605  & .082  & .605  & .077  & Huge bubbles~\cite{nr} \\
          & GRID  & 100M  & 400M  & 2     & 50,499 & 11.0  & .718  & 11.0  & 8.86  & 8.91  & .284  & 8.91  & .282  & Huge grids \\
          & CUBE  & 1.00B & 6.0B  & 3     & 2,895 & 13.2  & 7.98  & 13.0  & 9.57  & 9.38  & 4.11  & 9.38  & 4.01  & Huge cubic \\
          & HCNS  & 0.1M  & 5.0B  & 50,000 & 50,000 & 6.96  & 5.98  & 31.1  & 1.56  & 1.94  & 1.51  & 1.94  & 2.01  & Huge Coreness \\
          & HPL   & 100M  & 1.20B & 3,980 & 6,297 & 2.58  & 2.50  & 1.89  & 2.52  & 1.75  & 2.52  & 1.75  & 1.77  & Huge scale-free \\
\cmidrule{2-15}    \end{tabular}%

  \caption{\textbf{Graph information and running time (in seconds) of the combinations of our three techniques (sample, VGC, and HBS).} 
  \textnormal{
    $n= |V|$, $m= |E|$.
    $\maxcoreness=$ maximum coreness value. 
    $\rho$ = the peeling complexity~\cite{dhulipala2017}, i.e., the number of subrounds using offline peeling. 
    }
      \label{table:combinations}
      }
    
    \end{table*}%
 
} 