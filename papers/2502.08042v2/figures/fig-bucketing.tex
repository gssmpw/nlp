\begin{figure}[t]
  \centering
  \includegraphics[width=.65\columnwidth]{figures/hier_buckets.pdf}
  \caption{\textbf{The execution of the hierarchical bucketing structure for the first 10 rounds of execution.}  The number in each box indicates the key (\induceddegree) range of the associate bucket. \revise{
    The first row shows that a vertex with degree $d$ is initially inserted to bucket $\lceil\log_2 (d+1)\rceil$.
    %The vertices with degree 0 and 1 are inserted into the first two buckets with ID 0 and 1, respectively.
    %The vertices with degree 2 and 3 are inserted into the third bucket with ID 2, 
    %and the vertices with degree 4 to 7 are inserted into the fourth bucket with ID 3, and so on.
    When $k=0$ or $1$, vertices with degree 0 or 1 are directly extracted from buckets 0 and 1. 
    We then redistribute vertices in bucket 2 (with degrees 2 or 3) to buckets 0 and 1 (shown in the second row), 
    such that they can be directly identified when $k=2$ or 3. 
    Similarly, after that, we redistributed vertices in bucket 3 (with degrees 4 to 7) to the first three buckets. 
    Vertices with degree 4 and 5 are moved to bucket 0 and 1, respectively, and vertices with degree 6 and 7 are moved to bucket 2, so on so forth. 
    %The remaining movements are similar, and the vertices are moved to the buckets with smaller IDs.
}
}
  \label{fig:hier_bucketing}
\end{figure} 