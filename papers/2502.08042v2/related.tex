\section{Related Work}
The $k$-core decomposition problem has been extensively studied in various settings since it was first introduced by Seidman~\cite{seidman1983network}.
The first sequential algorithm for exact $k$-core decomposition was proposed by Matula and Beck~\cite{matula1983smallest},
using a bucket sort to arrange vertices by degree and iteratively deleting vertices with degree $k$ (peeling) until all are removed.
The algorithm runs in $O(m + n)$.
Batagelj and Zaversnik (\BZ) provided a sequential implementation that also runs within the same time complexity~\cite{batagelj2003m}. 

Several shared-memory parallel and distributed computing frameworks support $k$-core decomposition as part of graph analytics,
including GraphX~\cite{gonzalez2014graphx}, Powergraph~\cite{gonzalez2012powergraph}, Ligra~\cite{shun2013ligra} and Julienne~\cite{dhulipala2017} (later integrated into the GBBS library~\cite{gbbs2021}).
While these frameworks provide primitives for parallel or distributed graph processing, 
they do not specifically optimize for the \kcore{} problem.

There also exist studies specifically focusing on parallel \kcore{} algorithm. 
\ParK~\cite{dasari2014park} and \PKC~\cite{kabir2017parallel} are two state-of-the-art parallel algorithms using the online peeling process. 
They both use $O(n \maxcoreness + m)$ work. 
We introduced both of them in \cref{sec:peeling}. 
\PKC{} introduces optimizations, such as packing the remaining vertices after 98\% have been peeled, 
and using a local buffer to avoid subrounds.
While \PKC{} performs well on sparse graphs, 
both \Park{} and \PKC{} suffer from poor performance on dense graphs due to work-inefficiency and contention in updating induced degrees.

Dhulipala et al.~\cite{dhulipala2017} proposed a parallel \kcore{} algorithm in the \Julienne{} paper. 
In their paper, they use a bucketing structure that that maps each value $d$ to all vertices with induced degree $d$.
They proved that this algorithm has $O(m+n)$ work. In their implementation, they used a simplified bucketing structure that only 
maintains up to $b$ buckets. Our analysis proves the work-efficiency of their implementation, 
and an even simpler version without bucketing structure is also work-efficient. 
% Their algorithm is base on the offline peeling algorithm, 
\Julienne{} is based on the offline peeling approach, 
as we introduced in \cref{sec:peeling},
which is fully synchronized and race-free. 
It performs well on dense graphs but has unsatisfactory performance on sparse graphs, 
where synchronization overhead can be comparable to the computation cost.

\kcore{} is closely related to many other graph processing problems,
such as $D$-core decomposition, which extends $k$-core to directed graphs.
Several studies have explored $D$-core decomposition in both sequential and parallel/distributed settings~\cite{luo2024efficient, liao2022distributed, giatsidis2013d}.
There are also numerous approaches for approximate $k$-core decomposition, both in sequential settings~\cite{king2022computing} 
and parallel settings~\cite{esfandiari2018parallel,liu2022parallel, liu2024parallel, dhulipala2022differential}.
Additionally, streaming algorithms have been proposed for incremental $k$-core decomposition~\cite{sariyuce2013streaming, esfandiari2018parallel, sariyuce2016incremental}.















% For example, $D$-core decomposition extends $k$-core decomposition to directed graphs.
% There are some work on $D$-core decomposition both in sequential and parallel or distributed settings~\cite{luo2024efficient, liao2022distributed, giatsidis2013d}. 
%Specifically, the $D$-core decomposition problem can be reduced to the $k$-core decomposition problem when applied to undirected graphs.
%Similar to \PKC~\cite{kabir2017parallel}, their approach utilizes the same peeling process and buffer optimizations. 
%However, their algorithm is optimized for directed graphs, leading to suboptimal performance on undirected graphs.
% Thus, in this paper, we consider the exact $k$-core decomposition problem setting and related work. 










% Their algorithm initially arranges vertices by degree using a bucket sort process, 
% then iteratively deletes vertices whose degree matches the current level $k$ (peeling),
% until all vertices are removed.

%As real-world graphs have grown to billion-node scales, improving the efficiency of $k$-core decomposition algorithms has become increasingly important.
%They provide a set of primitives for parallel or distributed graph processing and solving graph problems in a more general context. 
% Most of them are graph processing frameworks which include \kcore{} as a problem in the library, and do not specifically optimize the \kcore{} problem. 
%\kcore decomposition can be solved using these frameworks, but they are not optimized for the $k$-core decomposition problem specifically.

%While the BZ algorithm is a sequential algorithm for $k$-core decomposition with a time complexity of $O(m + n)$,
%While the best sequential time complexity for \kcore{} is $O(m+n)$, 
%most existing parallel algorithms following the peeling process require $O(n \maxcoreness + m)$ work. 


%\PKC reduces the number of \subround by introducing a buffer optimization,
%but still requires $O(n \maxcoreness + m)$ work.
% \PKC{} further employs several optimizations, including packing the remaining vertices after $98\%$ of the total vertices are peeled, 
% and a local buffer that avoids subrounds in each round.  %does not significantly reduce the overall work.
% In our experiments, \PKC{} performs well on sparse graphs due to the optimizations. 
% However, on dense graphs, both \Park{} and \PKC{} may suffer from poor performance due to work-inefficiency 
% and contention in updating induced degrees concurrently.

% which is fully synchronized and is race-free. 
% In our experiments, \Julienne{} performs well on dense graphs due to race-freedom. 
% However, on sparse graphs where synchronize overhead can be comparable to the computation cost, 
% they may have unsatisfactory performance. 

% using a bucketing structure and the offline peeling framework , and Julienne~\cite{dhulipala2017}

% \begin{algorithm}[h]
    \caption{ParK $k$-core Decomposition Algorithm}
    \KwIn{Graph $G = (V, E)$}
    \KwOut{Core number of each vertex}
    \SetKwFor{parForEach}{parallel\_for\_each}{do}{endfor}
    
    
    \SetKwProg{myfunc}{Function}{}{}
    \myfunc{\FMain{$(G)$}}{
        $\vname{deg}[u]\gets \text{degree}$(u)\;

        $\vname{curr} \gets \emptyset$\

        $\vname{next} \gets \emptyset$\

        $\vname{coreness} \gets \emptyset$ \tcp{output array}

        $todo \gets n$\

        $level \gets 1$\

        \While{$todo > 0$}{
            \tcp{Add all nodes of $\vname{deg}[v] = \text{level}$ to $\vname{curr}$}
            \parForEach{$v \in V$}{
                \If{$\vname{deg}[v] = \text{level}$}{
                    $\vname{curr} \gets \vname{curr} \cup \{v\}$\
                }
            }
            \While{$|\vname{curr}| > 0$}{
                $todo \gets todo - |\vname{curr}|$\

                \parForEach{$v \in \vname{curr}$}{
                    $\vname{coreness}[v] \gets \vname{deg}[v]$\

                    \parForEach{$u \in \text{Neighbors}(v)$}{
                        \If{$\vname{deg}[u] > \vname{deg}[v]$}{
                            $\vname{deg}[u] \gets \vname{deg}[u] - 1$\

                            \tcp{decrease degree of $u$ atomically}
                            \If{$\vname{deg}[u] = \text{level}$}{
                                $\vname{next} \gets \vname{next} \cup \{u\}$\

                                \tcp{add sublevel node by atomically increase index}
                            }
                        }
                    }
                }
                $\vname{cur} \gets \vname{next}$\

                $\vname{next} \gets \emptyset$\
            }
            $level \gets level + 1$\
        }
        \Return{$\vname{coreness}$}\
    }
\end{algorithm}

% Luo et al.~\cite{luo2024efficient} presented a parallel algorithm for $D$-core decomposition, 
