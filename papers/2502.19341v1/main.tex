\documentclass[lettersize,journal]{IEEEtran}
%\documentclass[conference,a4paper,10pt]{IEEEtran}
\usepackage[latin1]{inputenc}
\usepackage{times,amsmath}
\usepackage{amssymb}
\usepackage{pstool}
%\usepackage{subfigure}
\usepackage{multirow,multicol}
\usepackage{enumerate}
\usepackage{graphicx}
\usepackage{MnSymbol}
\usepackage{stfloats} 
\usepackage[table]{xcolor}
\usepackage[square, comma, sort&compress, numbers]{natbib}
%\usepackage[bookmarks=false]{hyperref}
%\usepackage{nohyperref}
%\usepackage{algorithm,algorithmic}
\usepackage{epstopdf}
\epstopdfDeclareGraphicsRule{.tif}{png}{.png}{convert #1 \OutputFile}
\AppendGraphicsExtensions{.tif}
\usepackage{hyperref}
\usepackage{algorithm}
\usepackage{algpseudocode}
\usepackage{subfig}
\graphicspath{ {Figures/} }
\usepackage[export]{adjustbox}


\begin{document}


%\title{Openness of MCS tables by 3GPP enables new zero-cost 6G sensing paradigm}
\title{Unveiling Wireless Users' Locations via Modulation Classification-based Passive Attack}
%\title{Unveiling Wireless Users' Locations: Leveraging Modulation Classification in Passive Attacks}
%\title{Modulation classification as an enabler for malicious digital twin in wireless 6G/WiFi7 networks}
%\title{Modulation classification enables malicious digital twin in wireless 6G/WiFi7 networks}
%\title{Modulation classification attack could help Eve localize a UE}
%\title{Modulation Classification Empowered Passive Attack: A Pathway to Wireless User Localization}
%\title{Modulation classification based digital twin radio map of wireless 6G/WiFi7 networks: blessing or curse? }


\author{Ali Hanif, \IEEEmembership{Student Member, IEEE}, Abdulrahman Katranji, Nour Kouzayha, \IEEEmembership{Member, IEEE}, \\ 
Muhammad Mahboob Ur Rahman, \IEEEmembership{Senior Member, IEEE}, and Tareq Y. Al-Naffouri, \IEEEmembership{Fellow, IEEE}%
\thanks{The authors are with the Department of Electrical and Computer Engineering, CEMSE Division, King Abdullah University of Science and Technology  (KAUST), Saudi Arabia.}
\vspace{-0.5cm}}

\maketitle


\begin{abstract} 
%Smooth communication is possible owing to the broadcast nature of wireless signals without exposing specific physical locations of users. %However, this characteristic creates a major challenge in precisely identifying the location of targeted user devices for legitimate reasons.

%This article seeks to present an innovative approach to sniff the location of line-of-sight (LOS) user equipment through a modulation classification-based passive attack. Exploiting the open accessibility of MCS tables, an adversarial node constructs a fingerprint map of the wireless communication system based on modulation type. 

The broadcast nature of the wireless medium and openness of wireless standards, e.g., 3GPP releases 16-20, invite adversaries to launch various active and passive attacks on cellular and other wireless networks. This work identifies one such loose end of wireless standards and presents a novel passive attack method enabling an eavesdropper (Eve) to localize a line of sight wireless user (Bob) who is communicating with a base station or WiFi access point (Alice). The proposed attack involves two phases. In the first phase, Eve performs modulation classification by intercepting the downlink channel between Alice and Bob. This enables Eve to utilize the publicly available modulation and coding scheme (MCS) tables to do pesudo-ranging, i.e., the Eve determines the ring within which Bob is located, which drastically reduces the search space. In the second phase, Eve sniffs the uplink channel, and employs multiple strategies to further refine Bob's location within the ring. Towards the end, we present our thoughts on how this attack can be extended to non-line-of-sight scenarios, and how this attack could act as a scaffolding to construct a malicious digital twin map. 
\end{abstract}

\begin{keywords}
Modulation and coding scheme, modulation classification, physical layer security, passive attack, eavesdropping, ranging, and localization.
\end{keywords}

\section{Introduction}

The open nature of wireless medium invites adversaries to launch an ever-increasing spectrum of active and passive attacks, i.e., it is always possible to intercept, jam, and manipulate the ongoing communication between the legitimate nodes of a cellular/WiFi network through a low-cost, off-the-shelf spectrum analyzer or a software-defined radio~\cite{chorti2022context}. Such attacks, when occur, lead to a wide range of problems, e.g., false data injection, loss of data integrity, breach of data confidentiality, service outage, etc., and therefore, pose a great threat to 6G communication systems in particular~\cite{mucchi2021physical}, and all kinds of wireless communication networks at large~\cite{aman2023security}. Recently, there have been some reactive attempts by key stakeholders from the industry to design additional countermeasures to thwart a subset of adversarial attacks on cellular networks after they had actually occurred. For example, the 3rd generation partnership project (3GPP) has published a technical report (TR 33.809, Release 16) that provides a list of 5G security enhancements against false base stations (also known as rogue or fake base stations) that impersonate legitimate network elements in order to intercept, manipulate, or degrade communications~\cite{3gppTR33.809}. Nevertheless, in the era of generative artificial intelligence (AI), when there is a considerable mind shift to make the individual components of future 6G networks more transparent and AI-native through novel concepts such as open radio access network (RAN) systems~\cite{polese2023understanding}, software-defined networking (SDN), it is high time to reassess the security profile of future 6G systems.

%Some popular breaches include the following: . According to a careful estimate, the annual cost burden of the inherent security flaws in wireless networks is xxx usd. 

Moreover, the openness of various wireless standards may allow attackers to identify additional vulnerabilities in cellular/WiFi networks and capitalize on them. In fact, this work discovers one such loose end and, thereafter, presents a novel passive attack from the ethical hacking perspective. Specifically, this work focuses on modulation and coding scheme (MCS) tables that are routinely published by the 3GPP under releases 1x (\cite{3gppTS38.213}, see TS 38.213/38.214 specification by 3GPP) and by IEEE under releases 802.11ax~\cite{9442429}. Traditionally, MCS tables are utilized by wireless networks to realize adaptive modulation and coding (AMC), which is a mandatory operation that aims to make the best use of the fading wireless channel~\cite{4224371}. In order to help the base station (BS) utilize the MCS table, the user equipment (UE) measures the channel quality on the downlink and computes the channel quality index (CQI), which is then mapped to an MCS index. Accordingly, a modulation scheme and a coding rate are picked by the BS for the downlink communication during the next slot. In short, MCS-based AMC is an attempt to realize intelligent communication. Nevertheless, the fact that MCS tables are public information could expose the cellular/WiFi networks to attackers, as explained in the rest of this paper.

This work studies an innovative scenario whereby a malicious sensing node (Eve) sits close to a BS and intercepts the ongoing communication between the BS and a number of UEs, on both downlink and uplink. Essentially, this is the same as the old-school concept of cognitive radio~\cite{mitola1999cognitive}, except that Eve now aims to do location sensing via the MCS tables, instead of traditional spectrum sensing.  

%As will be explained later, this allows Eve to obtain location estimates of various users in the cellular/WiFi network. 




%Sensing and localization are two essential ingredients of 6G cellular networks. Traditionally, spectrum sensing has been done under the umbrella term of cognitive radio. More recently, under integrated sensing and communication framework, multiple enablers have been considered, e.g., joint radar communication, gas sensing in THz band, etc. 

%This work harnesses the openness of MCS tables by 3GPP to enable a new zero-cost 6G sensing paradigm. We dub it as a zero-cost sensing method because one just needs a spectrum analyzer/SDR in order to get a number of analytics. Though this work focuses on localization, other kinds of sensing could be done as well, e.g., UE aggregate behavior analysis, UE mobility/behavior analysis, analytics amid pandemic, insights for service providers, for community, digital twin radio map, etc. 

%This work could be thought as a contribution to the following domains: ranging and localization, modulation classification, security, ethical hacking, malicious cognitive radio, malicious sensing under ISAC, localization of non-cooperative target, digital twin. This work could positively and negatively impact many location based services. It could have an impact on many usecase scenarios, e.g., V2X, UAV/HAPS/LEO based, 6G, military comm. Can this be regulated? will regulation help? Do we have any suggestions for 3GPP standards committee? MCS is an attempt to make intelligent and realistic communication but it has a downside as well.   



% In wireless and cellular networks, there are situations where it becomes imperative for authorities to locate user devices, for example, to track criminals, terrorists, fraudsters \cite{7961983}, identifying illegal devices, and for network design, management, and optimization. 
%\subsection{Physical Layer Security and Attacks}
%The broadcast nature of radio frequency (RF) wireless networks makes them prone to different malicious attacks. Information security is crucial for protecting communication between legitimate parties and relies on preserving four key properties: confidentiality (ensuring information remains secret), authenticity (verifying the identity of exchanging parties), integrity (ensuring information is unaltered), and availability (maintaining system access for authorized users). To achieve these goals, two main strategies are employed: traditional cryptography, which encrypts data to prevent unauthorized access, and physical layer security (PLS), which leverages the randomness in the physical attributes of communication channels to secure information during transmission. The randomness of the physical layer arises from two main sources: wireless channel (multi-paths and noise) and hardware (manufacturing impurities) \cite{aman2023security}. 

%Malicious attacks that threaten the core security properties---confidentiality, authenticity, integrity, and availability--- can be broadly categorized as active or passive \cite{stallings2006cryptography}. Active attacks involve actively transmitting signals to gain unauthorized access or disrupt communication between legitimate parties. Active attacks include Interruption, which disrupts system availability; Modification, which alters transmitted data compromising integrity; and Impersonation, which involves pretending to be legitimate to gain unauthorized access thereby compromising authentication. Passive attacks contrast with active attacks by involving eavesdropping on communications without transmitting signals, enabling attackers to silently listen to communication between legitimate nodes. Passive attacks include Eavesdropping, where communication is intercepted and decoded without alteration thereby compromising confidentiality, and Traffic Analysis, which infers sensitive information from communication patterns. Examples of malicious attacks include spoofing, replay, sybil, wormhole, jamming,...etc. Traditionally, the passive attack is studied at the physical layer from the perspective of information-theoretic bounds where secrecy rate and secrecy outage probability are considered as performance metrics. It is worth mentioning that the purpose of passive attacks is usually not localization. This work aims to leverage modulation classification in a passive attack to effectively narrow down the search space for efficient localization of a wireless user.


% Due to the random and time-varying nature of the wireless channel, the quality of the channel varies between poor and excellent. When channel quality is good then a high data rate can be achieved through varying modulation order (also known as adaptive modulation). 
%\subsection{Modulation Classification}
%Modulation classification is an important task whereby an intelligent receiver tries to infer the modulation scheme in use by a transmitter, by processing the received signal only. The need for modulation classification arises because modern WiFi transmitters utilize an adaptive modulation and coding scheme whereby the modulation order and type change in reaction to the channel conditions. Standard modulation and coding scheme (MCS) tables for different standardized systems let the system decide about the coding rate and modulation order based on channel quality indicator. In literature, plenty of techniques have been proposed for blind modulation classification, e.g., classical statistics-based methods \cite{octaviapaper}, and more recent machine/deep learning-based methods \cite{peng2021survey}. 

%In this project, we consider an AWGN channel whereby the noisy received signal at the intelligent receiver is: $r=x+n$. Here, $x$ is the transmitted symbol, and $n\sim N(0,\Sigma)$ is the additive white Gaussian noise with zero-mean and variance $\Sigma$. Assuming that the transmitter utilizes different modulation schemes during different time slots, we model the received signal $r$ as a Gaussian mixture model (GMM). We then utilize the classical expectation-maximization (EM) algorithm to estimate the parameters of the GMM mixture densities. Eventually, we will make use of the symmetric Kullback-Leibler divergence (KLD) to discriminate between the various candidate modulation schemes.


% \subsection{Related Work}

% \subsubsection{Modulation Classification}
%Addressing the automatic modulation classification (AMC) challenge involves employing three distinct methodologies: likelihood-based AMC, a feature-based approach, and the utilization of deep neural networks. Likelihood-based AMC relies on the assumption of an ideal signal distribution \cite{Maroto_likelihood} and a time-invariant channel. This condition may not be universally applicable to all wireless communication systems. On the other hand, feature-based AMC necessitates expert feature engineering to extract relevant characteristics from each modulation class, thereby restricting its capacity to generalize across diverse modulation schemes \cite{feature-based}.

%With the rapid increase of data in communication systems and enhanced computational power, deep neural networks (DNNs) swiftly became a key solution for the AMC challenge. In recent years, diverse architectures ---ranging from convolutional neural network (CNN)-based models \cite{O'shea_CNN} to recurrent neural network-based structures \cite{AMC_RNN} and hybrid neural networks \cite{AMC_Hybrid}--- have been investigated to assess their efficacy as automatic modulation classifiers. Unlike feature-based methods, the DNN approach can automatically extract hierarchical features of each modulation type. This enhances its ability to generalize results to other modulation types while maintaining a simpler implementation. The advancements made by DNNs prompted scientists to explore various DNN architectures in order to determine which one could achieve the optimal likelihood-based AMC \cite{Optimal_DNNs}. Wei et al. \cite{Weipaper} conducted a comprehensive analysis of diverse DNNs, assessing their performance in terms of size and computational complexity. 
  % The models were fed with in-phase and quadrature components of received signals, along with higher-order cumulants. The study revealed that CNNs , exemplified by LeNets, exhibit superior accuracy and prediction time compared to recurrent neural networks of similar size, particularly in evaluations on edge devices. Despite CNNs proving effective as lightweight models for AMC, their reliance solely on spatial information renders their results suboptimal for similar modulation types. Addressing this limitation, an alternative CNN variant was proposed in \cite{Qi_ResNet}, aiming to enhance results for similar modulation types by leveraging multi-domain features.
  % In the work by Lin et al. \cite{Lin_Attention_CNN}, an attention mechanism is employed to extract channel, frequency, and time features from spectrogram images prior to their input into a CNN. While this method yielded improved results compared to hybrid models like stacked convolutional long short-term memory deep neural networks, it comes with the trade-offs of slower inference and training times, along with the need for preprocessing the raw signal. 

%\subsection{User Localization}
%The localization problem seeks to find the geographical coordinates of a user node, crucial for applications in communications, logistics, and military operations. Active methods of localization in wireless networks include trilateration (using signal propagation time to estimate range) and triangulation (via angle estimation). Active localization methods require line-of-sight, and necessitate the transmission of high bandwidth reference signals from multiple access points, thereby resulting in high overhead and stringent synchronization requirements \cite{8226757}. Numerous methodologies address the passive localization challenge in wireless communication systems, with the deep learning approach prominently relying on the utilization of radio maps \cite{8226757}. 
  % Radio maps represent a fingerprinting technique wherein the receiver systematically gathers an extensive signals database alongside their associated coordinates offline. This database is subsequently utilized for real-time localization by comparing incoming signals with existing records, identifying the most closely resembling signal for accurate position estimation. Fingerprinting approach doesn't suffer from the drawbacks associated with conventional active localization techniques. In the study by Mohamed et al. \cite{Mohamed_features}, a comprehensive set of twelve features, encompassing both signal and channel characteristics, is employed for user localization. The results demonstrate the effectiveness of a CNN model, showcasing high precision in localizing users within a broad communication system.  To reduce the feature set, Elawaad et al. \cite{Elawaad_regression} proposed a regression-based localization method relying solely on RSSI. %In an attempt to optimize storage requirements, a preprocessing technique is introduced in \cite{Nguyen_preprocessing}, where three anchor nodes receive signals that are subsequently transformed into a single RGB-colored image. The localization task is then performed using a CNN, contributing to a more efficient user localization strategy.

  % The main drawback of fingerprinting method is the extensive data collection effort required for building a database with all the measurements and associated locations \cite{morais2024localization}. %An illustration of this complexity can be found in \cite{RSSI_Fingerprint}, where the radio map comprises RSSI data from multiple receivers along with their corresponding coordinates.
  % This necessitates precise measurements across the entire area of  interest for effective training of a DNN. A digital twin radio frequency map approach is proposed in  \cite{morais2024localization} to populate massive fingerprinting databases with less human effort by utilization realistic digital replicas.   %Addressing this challenge, Li et al. \cite{Li_localization_MFS} proposed an innovative approach by incorporating magnetic field strength (MFS) and RSSI measurements recorded through a mobile smartphone. This integration aims to enhance the reliability of the radio map, providing a more comprehensive representation of the communication environment. In \cite{Yapar_AIMAP}, a deep neural network, known as RadioUNet, is employed to solve propagation equations to generate a radio map. This method leverages a DNN to infer the complex propagation patterns, offering an efficient solution for localization by utilizing the information encoded in the generated map. 


{\bf Contributions:} 
This work introduces a novel passive attack that allows a malicious eavesdropper to obtain the approximate range and location estimates of one or more line-of-sight (LOS) users associated with a wireless cellular base station or a WiFi access point. 
The main contributions of this work are as follows:
 \begin{enumerate}
    \item {\it Pseudo ranging:} Eve sniffs the broadcast transmission of the base station intended for licensed user(s) during the downlink phase, in order to perform modulation classification on the intercepted signal(s). Thereafter, Eve utilizes the publicly available MCS tables in order to reverse-map the detected modulation scheme to an MCS index/CQI, to obtain upper and lower bounds on the range (i.e., the distance between the user(s) and the base station). This significantly narrows down the search space for a given user--from a full cell coverage region down to a ring.
    \item {\it UE Localization:} Eve sniffs the transmission of the user during the uplink phase, while traversing through the ring (i.e., the narrowed search space) where a given user is. The sniffing at many grid points during the circular motion of Eve inside the ring allows Eve to record signal-to-noise ratio (SNR) at each grid point and declare its location where the SNR is maximum as the initial location estimate of Bob.
    %The sniffing at many grid points during the circular motion of Eve inside the ring allows Eve to utilize signal-to-noise ratio (SNR) gradient strategy to localize Bob within an even smaller region where a change in gradient of SNR is observed. 
    Eve further does a coarse ranging followed by another circular motion in order to obtain a precise location estimate of the user.
    %Eve utilizes a range of strategies which successively shrink the search space
 \end{enumerate}
Since this work introduces a novel passive attack, the first of its kind, it makes a fundamental contribution to the field of physical layer security~\cite{mitev2023physical}. 
Further, this work stands out compared to previous works on non-cooperative source localization due to the fact that it is based on passive sensing only and doesn't require the deployment of any anchor nodes with known locations~\cite{8734111}. At the same time, this work, being a malicious sensing method, also makes a contribution to integrated sensing and communication (ISAC) methods of future 6G systems~\cite{kaushik2024toward}.  
%Fig. \ref{scenario} depicts some application scenarios which could witness and get affected by the proposed passive attack. 
%Last but not the least, we note that the proposed method is applicable for all non-cooperative localization scenarios in general. 

\begin{figure*}
    \centering
    \includegraphics[width=0.99\textwidth,keepaspectratio]{1.pdf}
    \caption{Potential application scenarios of the proposed modulation classification-based passive attack.}
    \label{scenario}
\end{figure*}

\section{System Model} \label{sys_model}
We consider a system model comprising a base station/access point (Alice), one UE node (Bob), and one adversary node (Eve), which has the malicious intentions to localize Bob. A ray tracing channel model with additive white Gaussian noise (AWGN) is used to model all three communication channels, i.e., the pair-wise channels between Alice, Bob and Eve. For simplicity, we assume that all the nodes are equipped with single antennas. We consider a line-of-sight (LOS) scenario in order to assess the performance of the proposed modulation classification-based attack in estimating the location of the UE. Alice initiates the downlink communication by receiving the channel quality indicator from Bob and mapping it to a modulation and coding scheme using a standard MCS table shown in Table~\ref{MCStable}. Multiple frequency bands ($5$~GHz, $28$~GHz, and $100$~GHz) and various transmit powers of Alice are considered to validate the proposed approach under different scenarios. Finally, the scenarios involving two UE nodes, and multiple antennas at the adversary node Eve are also discussed. 
%in Sections~\ref{multibob} and~\ref{mimoEve}, respectively.
%depicted in Table~\ref{tab:results_phase1}. 

We note that the proposed infrastructure-less passive attack can be employed to localize various types of users as shown in Fig.~\ref{scenario}, including cellular users, users in vehicle-to-everything (V2X) networks, users communicating with non-terrestrial nodes, e.g., high altitude platform systems (HAPS), and military unmanned aerial vehicles (UAV)/drones connected to airborne warning and control systems (AWACS).

\begin{table}[h!]
    \centering
    \caption{MCS table for IEEE 802.11ac wireless local area network (WLAN). SNR stands for signal-to-noise ratio, VHT stands for very high throughput. IEEE 802.11ac offers 20 MHz bandwidth, and two frame intervals of 800 ns and 400 ns.  }
    \begin{tabular}{|c|c|c|c|c|c|}
    \hline
         \multirow[c]3{*}[0in]{\shortstack{\textbf{VHT}\\\textbf{MCS}}} & \multirow[c]{3}{*}[0in]{\textbf{Modulation}}  & \multirow[c]{3}{*}[0in]{\textbf{Coding}}   & \multicolumn{3}{|c|}{\textbf{20 MHz}}\\ \cline{4-6}
         
         &&&  \multicolumn{2}{|c|}{\textbf{Data Rate}}& \multirow[c]{2}{*}[0in]{\shortstack{\textbf{Min.}\\ \textbf{SNR}}} \\ \cline{4-5}
       
       &&& \textbf{800ns}   & \textbf{400ns}&\\\hline
      
         $0$ &  BPSK & $1/2$  & $6.5$  & $7.2$ & $2$  \\\hline
         $1$ &  QPSK & $1/2$  & $13$  & $14.4$ & $5$  \\\hline
         $2$ &  QPSK & $3/4$  & $19.5$  & $21.7$ & $9$  \\\hline
         $3$ &  16-QAM & $1/2$  & $26$  & $28.9$ & $11$  \\\hline
         $4$ &  16-QAM & $3/4$  & $39$  & $43.3$ & $15$  \\\hline
         $5$ &  64-QAM & $2/3$  & $52$  & $57.8$ & $18$  \\\hline
         $6$ &  64-QAM & $3/4$  & $58.5$  & $65$ & $20$  \\\hline
         7 &  64-QAM & $5/6$  & $65$  & $72.2$ & $25$ \\\hline
        %\multicolumn2{l}{$^{\mathrm{*}}$} Very High Throughput.\\ 
    \end{tabular}
    \label{MCStable}
\end{table}

\textbf{Assumptions}: We assume that the location of the base station Alice, the MCS tables, the frequency of operation, and the transmit powers of Alice and Bob constitute public information, and thus, are known to Eve.

%For simplicity, we assume channel reciprocity, i.e., the same channel characteristics are applicable during the downlink and uplink transmissions.

\section{Modulation Classification-based Passive Attack}
In this section, we describe the details of the proposed modulation classification-based passive attack. The attack consists of two distinct phases. 1) During downlink phase, the eavesdropper, Eve, intercepts the broadcast message from Alice to Bob, and does modulation classification on it in order to identify the modulation type of the transmitted signal. This allows Eve to utilize the MCS table to identify a small geographical region (basically, a ring) within the cell where presumably Bob is present. 2) During the uplink phase, Eve traverses through the geographical region narrowed down during the downlink phase. Eve continues its passive attack by intercepting the uplink signal. Eventually, Bob's location is obtained by employing diverse strategies tailored to Eve's capabilities. 

% \begingroup
% \renewcommand{\arraystretch}{1.5}
% \begin{table}
%     \centering
%     \caption{Simulation parameters of the different scenarios.}
%     \begin{tabular}{|c|c|c|c|}
%     \hline
%          \textbf{Scenario} & \shortstack{\textbf{Frequency}\\ \textbf{(GHz)}}  & \shortstack{\\\textbf{Alice's Transmit}\\ \textbf{Power (mW)}} & \shortstack{\\\textbf{Bob's Transmit}\\ \textbf{Power (mW)}} \\ \hline
%          a. &5 & \shortstack{200, 300, 400} & 0.5 \\ \hline
%          b. & 28 & \shortstack{200, 300, 400} & 0.5 \\ \hline
%          c. & 100 & \shortstack{200, 300, 400} & 0.5 \\ \hline
%     \end{tabular}
%     \label{tab:my_label}
% \end{table}
% \endgroup

% \begingroup
% %\setlength{\tabcolsep}{12pt} % Default value: 6pt
% \renewcommand{\arraystretch}{1.5}
% \begin{table}[h]
%     \centering
%     \caption{Simulation parameters of the different scenarios.}
%     \begin{tabular}{|c|c|c|c|}
%     \hline
         % \textbf{Scenario} & \shortstack{\textbf{Frequency}\\ \textbf{(GHz)}}  & \shortstack{\\\textbf{Alice's Transmit}\\ \textbf{Power (mW)}} & \shortstack{\\\textbf{Bob's Transmit}\\ \textbf{Power (mW)}} \\ \hline
      
         % a. & \multirow[c]{3}{*}[0in]{5} & 200 & \multirow[c]{3}{*}[0in]{0.5} \\\cline{1-1}\cline{3-3}
        
         % b. & & 300 & \\\cline{1-1}\cline{3-3}
      
         % c. &  & 400 & \\\hline
      
         % d. & \multirow[c]{3}{*}[0in]{28} & 200 & \multirow[c]{3}{*}[0in]{0.5} \\\cline{1-1}\cline{3-3}
        
         % e. & & 300 &  \\\cline{1-1}\cline{3-3}
         % f. &  & 400 & \\\hline
         % g. & \multirow[c]{3}{*}[0in]{100} & 200 & \multirow[c]{3}{*}[0in]{0.5} \\\cline{1-1}\cline{3-3}
         % h. & & 300 & \\\cline{1-1}\cline{3-3}
         % i. & & 400 & \\\hline
         
%     \end{tabular}
%     \label{tab:phase1}
% \end{table}
% \endgroup

\subsection{Downlink Phase: Pseudo Ranging through Modulation Classification}

%During downlink transmission phase, Eve captures the broadcast signal from Alice to Bob and performs modulation classification to identify the modulation type and infer from it a region where Bob is most likely to be located. 
We first present sufficient details about the modulation classification method we have implemented, followed by the details of how Eve could map the detected modulation scheme to a ring, a process we call pseudo-ranging.

During the downlink phase, Alice sends a modulated signal to Bob using either phase-shift keying (PSK) or quadrature amplitude modulation (QAM). Specifically, Alice utilizes one of the specific modulation types from $4$ different modulation schemes, namely, BPSK, QPSK, 16-QAM, and 64-QAM (see Table~\ref{MCStable}). Eve sniffs the signal broadcast by Alice and carries out modulation classification. We note that modulation classification has been traditionally done through statistical methods which: i) first differentiate between the PSK and QAM using the fact that the PSK scheme leads to a constant-envelop signal, while the QAM scheme leads to a variable-envelop signal; ii) identify the order of modulation $M$ \cite{octaviapaper}. However, more recently, there is an increased interest in doing modulation classification through a range of deep learning methods \cite{peng2021survey}. 

In this work, inline with recent research trends \cite{peng2021survey}, we implement a two-dimensional convolutional neural network (CNN) as the modulation classifier at Eve. The input to the neural network is the in-phase and quadrature components of the sniffed signal. The CNN takes a decision after processing $128$ samples of the received signal. The CNN-based classifier architecture consists of four convolutional layers, three dropout layers, one average pooling layer, and two dense layers, along with a total of $2.13$ million parameters.

To train the CNN-based modulation classifier, an offline training dataset is constructed. By moving Eve away from Alice in different directions, sufficient samples for each modulation type at varying SNR values are obtained. The final dataset contains $240,000$ samples for the four modulation types: BPSK, QPSK, 16-QAM, and 64-QAM over a range of SNR values from $2$~dB to $28$~dB. This is in accordance with the MCS Table in Table~\ref{MCStable}. The training dataset is balanced to mitigate bias by ensuring an equal number of samples ($60,000$) for each of the four classes. The training dataset is further split into training and validation datasets with an $80:20$ ratio. After training the model offline, the CNN-based modulation classifier at Eve is tested using an unbiased testing dataset of $10,000$ samples. The different scenarios considered, along with the classification results, are presented in Section~\ref{mod_results}.

\begin{figure}
    \centering
    \includegraphics[width=0.9\linewidth]{2.pdf}
    \caption{Possible rings of Bob based on the modulation type for $100$~GHz frequency and $400$~mW transmit power of Alice, within the coverage region of a single cell.}
    \label{fig:rings}
\end{figure}


Detection of a modulation classification through a CNN allows Eve to infer a small geographical region based on the possible signal-to-noise-ratio (SNR) values that are associated with the classified modulation type in accordance with the MCS table given in Table~\ref{MCStable}. For example, when BPSK modulation scheme is detected, it implies that the operational SNR range of the downlink Alice-Bob channel is 2-5 dB (see Table~\ref{MCStable}). Eve plugs this SNR range into the Friis equation which returns upper and lower bounds on the distance between Alice and Bob. This corresponds to two concentric circles which together form a ring. Thus, the inferred regions of Bob are ring-shaped around Alice due to the LOS assumption and are depicted in Fig.~\ref{fig:rings} for a transmit frequency of $100$~GHz and transmit power of $400$~mW by Alice. We call this mapping process pseudo-ranging because it enables Eve to obtain loose upper and lower bounds on the distance between Alice and Bob. Pseudo-ranging greatly narrows down the search space for Eve--from from the full cell region down to a ring. 



% \textcolor{red}{We also need to discuss the existing public datasets containing I/Q data, that are used for modulation classification purposes.}
% \begin{figure}[t!]
 %   \centering
  %  \includegraphics[width=0.9\linewidth]{Figures/search space narrow down.pdf}
   % \caption{Progressively narrowing down the search space to identify the user's location.}
    %\label{fig:narrow}
%\end{figure}

%\subsubsection{Strategy II: Single Mobile Eve Node  with Improved Performance}
%This strategy is an improved version of the first strategy and produces the best results. It improves upon the previous strategy by incorporating an additional step after the completion of strategy I. 
\begin{algorithm}[t!]
\caption{\textbf{:} UE localization during uplink phase}
\label{Algo2}
\textbf{Input:} BS/Alice's location, lower ($r_a$) and upper ($r_b$) limits of the location ring of UE, propagation model\\
\textbf{Initialize: } Step sizes $\Delta\theta_1$ and $\Delta\theta_2$, best and worst SNRs
\begin{algorithmic}[1]
\State Eve's initial location = $\frac{r_b-r_a}{2}$
\State $N\gets 360^{\circ}/\Delta\theta_1$
\State $i\gets 1$
\For{$i \leq N$}
\State {Compute  $SNR$ at Eve}
\If{$SNR >$  best $SNR$}
    \State {possible Bob's location $\gets$ current Eve's}
    \Statex {\hspace{1cm}location}
     \State {best $ SNR \gets$ current $SNR$}
\ElsIf{$SNR <$ best $SNR$}
    \State {\textbf{continue}}
\EndIf
\State $i=i+1$
\EndFor
\Statex \textbf{Output: } Initial estimate $(x_{B_c},y_{B_c})$ of Bob's location 

\State {Eve's location $\gets$ $(x_{B_c},y_{B_c})$}
\State {Compute  $SNR$ at Eve}
\State {Compute distance $d_{BE}$ between Bob \& Eve using Friis equation based pathloss model}
\State {Eve moves to a distance$d_{BE}$ from its current location}
\State $N\gets 360^{\circ}/\Delta\theta_2$
\While {location error $\geq$ tolerance}
\State $i\gets 1$
\For{$i \leq N$}
\State {Compute  $SNR$ at Eve}
\If{$SNR >$  best $SNR$}
    \State {Possible Bob's location $\gets$ current Eve's}
    \Statex {\hspace{1.5cm}location}
     \State {best $ SNR \gets$ current $SNR$}
\ElsIf{$SNR <$ best $SNR$}
    \State {\textbf{continue}}
    
\EndIf
\State $i=i+1$
\EndFor
\EndWhile
\end{algorithmic}
\textbf{Output: } Predicted location of Bob
\end{algorithm}

\subsection{Uplink Phase: UE Localization} \label{uplink_strategy}
%Upon receiving the downlink signal, Bob initiates uplink communication by transmitting an uplink signal to Alice. 
After identifying the modulation scheme during the downlink phase, Eve moves to the ring where Bob is likely to be located and chooses a midpoint radius between the upper ($r_b$) and lower ($r_a$) limits of Bob's location ring, i.e., $\frac{r_b-r_a}{2}$. 
This can be easily realized by an eavesdropper UAV flying towards Bob's ring. Since Bob sends a signal to Alice at a significantly lower transmit power compared to the downlink transmission, this makes it more challenging to intercept the signal if Eve is far away from Bob. Thus, Eve wants to be in close proximity of Bob which helps it sniff a relatively high-quality copy of Bob's signal during the uplink phase. The sniffing of uplink channel helps Eve further narrow down the search space as follows. Inside the ring, Eve moves along a circular path of the midpoint radius. At each step, Eve sniffs the uplink signal of Bob, and computes the received SNR. After completion of the circular path, Eve declares its location where it receives the highest SNR as initial location $(x_{B_c},y_{B_c})$ of Bob.  %The algorithm for this strategy is given as Algorithm \ref{Algo1}.


%Hence, Eve starts moving towards Bob to overcome the poor uplink signal quality. 
%and sniffing the uplink signals. 


% \begin{algorithm}
% \caption{\textbf{:} Algorithm for Strategy I}
% \label{Algo1}
% \textbf{Input:} Alice's location, lower (a) and upper (b) limits of the ring, propagation model\\
% \textbf{Initialize: } Step size $\Delta\theta$, best and worst SNRs
% \begin{algorithmic}[1]
% \State Eve's initial location = $\frac{b-a}{2}$
% \State $N\gets 360/\Delta\theta$
% \State $i\gets 1$
% \For{$i \leq N$}
% \State {Compute  $SNR$ at Eve}
% \If{$SNR >$  best $SNR$}
%     \State {possible Bob's location $\gets$ current Eve's}
%     \Statex {\hspace{1cm}location}
%      \State {best $ SNR \gets$ current $SNR$}
% \ElsIf{$SNR <$ best $SNR$}
%     \State {\textbf{continue}}
% \EndIf
% \State $i=i+1$
% \EndFor
% \end{algorithmic}
% \textbf{Output: } Predicted Bob's location
% \end{algorithm}

% \begin{figure}
%     \centering
%     \includegraphics[width=0.9\linewidth]{Figures/strat1.png}
%     \caption{Histogram of the first strategy with single mobile Eve.}
%     \label{fig:strat1}
% \end{figure}

Further, as the SNR of Bob's signal is available at Eve, it again utilizes the Friis equation-based pathloss model to do coarse ranging, i.e., the measurement of approximate distance $d_{BE}$ between Eve and Bob. This allows Eve to move again in a smaller circle of origin $(x_{B_c},y_{B_c})$ and radius $d_{BE}$. This way, Eve declares the location where it receives the highest SNR as Bob's location. This step can be repeated multiple times until a desired tolerance is met. This strategy considerably improves localization accuracy at the cost of additional resources required for distance $d_{BE}$ estimation. The detailed algorithm for the uplink phase is given in Algorithm~\ref{Algo2}. The entire process consists of refining/shrinking the search space, initially through modulation classification and subsequently utilizing the described strategy. 
%is summarized in Fig.~\ref{fig:narrow}.



% \begin{figure}
%     \centering
%     \includegraphics[width=0.9\linewidth]{Figures/improved-strat1.png}
%     \caption{Histogram of strategy II with improved single mobile Eve node.}
%     \label{fig:strat2}
% \end{figure}

%\subsubsection{Strategy III: Multiple Stationary Eve Nodes}
%We distribute multiple stationary Eves in each ring in a circular pattern with a radius equal to the midpoint between the lower and upper limits of Bob's location ring. Each Eve receives the signal, computes the SNR of the signal, and returns it to a central unit. This central unit calculates the weighted average location of two Eves that received the highest SNR from Bob and reports it as the location of Bob. We evaluated various numbers of Eaves in this strategy and found that eight distributed Eaves yield the best results with minimal resources. Increasing the number of Eaves beyond eight does not improve the results.

% which are presented in Section \ref{sim_results}. The algorithm is provided as Algorithm \ref{Algo3}.

% \begin{algorithm}
% \caption{\textbf{:} Algorithm for Strategy III}
% \label{Algo3}
% \textbf{Input:} Alice's location, lower (a) and upper (b) limits of the ring, propagation model\\
% \textbf{Initialize: } Eight Eves' locations on the ring with radius $\frac{b-a}{2}$ 
% \begin{algorithmic}[1]
% \State {Compute  $SNR$ at eight Eves' locations}
% \State {Compare all $SNRs$ and select two Eves with highest $SNRs$}
% \State {Possible Bob's location $\gets$ weighted average location of two Eves with highest $SNRs$}
% \end{algorithmic}
% \textbf{Output: } Predicted Bob's location
% \end{algorithm}

% \begin{figure}
%     \centering
%     \includegraphics[width=0.9\linewidth]{Figures/strat3.PNG}
%     \caption{Histogram of strategy III with multiple stationary Eve nodes.}
%     \label{fig:strat3}
% \end{figure}

\begingroup
\setlength{\tabcolsep}{12pt} % Default value: 6pt
\renewcommand{\arraystretch}{1.5}
\begin{table*}[h]
    \centering
    \caption{Modulation classification accuracy for the different scenarios during downlink phase.}
    \begin{tabular}{|c|c|c|c|c|c|c|}
    \hline
         \multirow[c]3{*}[0in]{\textbf{Scenario}} & \multirow[c]{3}{*}[0in]{\textbf{Frequency (GHz)}}  & \multirow[c]{3}{*}[0in]{\textbf{Alice's Transmit Power (mW)}}   & \multicolumn{4}{|c|}{\textbf{Classification Accuracy (\%)}}\\ \cline{4-7}
      
       &&&  \multicolumn{3}{|c|}{\textbf{Testing Dataset}} & \multirow[c]{2}{*}[0in]{\textbf{Validation Dataset}} \\ \cline{4-6}
       
       &&&\textbf{Near Eve}   & \textbf{Mid Eve} & \textbf{Far Eve}&\\\hline
    
         a. & \multirow[c]{3}{*}[0in]{5} & $200$   & $87.52$ & $76.42$  & $50.38$ & $80.62$ \\ \cline{1-1}\cline{3-7}
        
         b. & & $300$   & $93.3$ & $76.92$ & $48.75$ & $83.31$ \\\cline{1-1}\cline{3-7}
      
         c. &  & $400$   & $93.87$ & $71.93$ & $50.01$ & $84.06$ \\\hline \hline
      
         d. & \multirow[c]{3}{*}[0in]{28} & $200$   & $91.38$ & $75.09$ & $40.43$ & $82.67$ \\ \cline{1-1}\cline{3-7}
        
         e. & & $300$ & $87.58$ & $74.24$ & $51.52$ & $81.79$ \\\cline{1-1}\cline{3-7}
         f. &  & $400$  & $85.23$ & $71.15$ & $52.81$ & $80.26$ \\ \hline\hline
         g. & \multirow[c]{3}{*}[0in]{100} & $200$   & $84.74$ & $66.74$  & $41.98$ & $81.24$ \\ \cline{1-1}\cline{3-7}
         h. & & $300$ & $95.07$ & $74.64$ & $49.9$ & $84.19$\\\cline{1-1}\cline{3-7}
         k. & & $400$  & $81.84$ & $74.58$ & $47.36$ & $82.52$ \\ \hline
         
    \end{tabular}
    \label{tab:results_phase1}
\end{table*}
\endgroup

\section{Performance Evaluation} \label{sim_results}
We evaluate the impact of the proposed passive attack across three different frequencies, in the microwave band (the sub-$6$ GHz band) and millimeter-wave band. We also consider three different transmit powers of Alice. This results in nine distinct scenarios depicted in Table~\ref{tab:results_phase1}. In the downlink phase, modulation classification results are obtained by testing the trained CNN on the corresponding testing dataset for each scenario. The CNN implementation and training are performed in JupyterLab, while the training and testing datasets are generated using MATLAB. For the uplink phase, the scenarios in Table~\ref{tab:results_phase1} are simulated using MATLAB Site Viewer, with the system model defined in Section~\ref{sys_model}. From the downlink phase, Eve is provided with the upper and lower limits of Bob's ring. Localization performance is assessed by repeating the localization task $1000$ times with Bob positioned arbitrarily each time. The following subsections provide key results and discussions for the different phases of the proposed approach.

\subsection{Downlink Modulation Classification Results} \label{mod_results}
The 2D CNN is trained for $10$ epochs to identify the modulation type out of four possible schemes: BPSK, QPSK, 16QAM, and 64QAM. The model comprises approximately $2.13$ million parameters and requires around $2$ minutes per epoch for training on a personal computer equipped with an Intel(R) Xeon(R) Gold 6230R CPU, operating at a clock frequency of $2.1$~GHz and with $256$~GB of RAM. The testing dataset that consists of intercepted downlink signals is created using three different locations for Eve to reflect realistic scenarios where Eve could be very near to or far from Alice/BS. The three Eve locations are at near, mid, and far distances from Alice depending upon the transmit power of Alice.

Table~\ref{tab:results_phase1} shows the classification accuracies of the CNN on the validation and testing datasets for the different considered frequency and transmit power values. The validation accuracy exceeds $80\%$ for all the simulated scenarios. For the testing dataset, the accuracy of the model decreases significantly as Eve moves away from Alice, thereby reducing the SNR of the received signal at Eve. Moreover, with the increase in frequency, the total coverage area is reduced significantly owing to higher path loss at higher frequencies. 
Table~\ref{tab:results_phase1} illustrates that {\it Eve must stay close to Alice/BS in order to do high-quality pseudo-ranging while it intercepts the downlink signal}. This is because a near location of Eve results in accurate modulation classification, which in turn helps identify the correct location ring for Bob, and vice versa.

%{\color{red} What is the impact of the frequency and the transmit power on the accuracy results? Add more explanations}. {\color{blue} They have no significant impact. As the network is trained to classify the modulation type so it shouldn't be affected by the change in frequency. The transmit power is not impacting the accuracy because with the increase in power, the coverage area increases and the distribution of SNR remains the same across the coverage area. We have taken an unbiased dataset for training(this I will mention in training of CNN subsection).However, for a fixed transmit power, I have explained in the text that as Eve moves away from Alice, the accuracy decreases due to decrease in SNR.}

% \begin{figure}
%     \centering
%     \includegraphics[width=0.9\linewidth]{Figures/MCS.png}
%     \caption{MCS table of 802.11n and 802.11ac.}
%     \label{fig: Confusion matrix of 2DCNN}
% \end{figure}

% \begin{figure*}[h]
%   \centering
%   \subfloat[Freq: 5 GHz, Eve's distance: Near]{\includegraphics[width=0.32\textwidth,height=.2\textwidth,keepaspectratio]{Figures/Confusion_Matrix_near_5GHz_400mW.png}}
%   \subfloat[Freq: 5 GHz, Eve's distance: Mid]{\includegraphics[width=0.32\textwidth,height=.2\textwidth,keepaspectratio]{Figures/Confusion_Matrix_Mid_5GHz_400mW.png}}
%   \subfloat[Freq: 100 GHz, Eve's distance: Near]{\includegraphics[width=0.32\textwidth,height=.2\textwidth,keepaspectratio]{Figures/Confusion_Matrix_near_100GHz_400mW.png}}\\
%  \subfloat[Freq: 28 GHz, Eve's distance: Near]{\includegraphics[width=0.32\textwidth,height=.2\textwidth,keepaspectratio]{Figures/Confusion_Matrix_near_28GHz_400mW.png}}
%  \subfloat[Freq: 28 GHz, Eve's distance: Mid]{\includegraphics[width=0.32\textwidth,height=.2\textwidth,keepaspectratio]{Figures/Confusion_Matrix_Mid_28GHz_400mW.png}}
%  \subfloat[Freq: 100 GHz, Eve's distance: Near]{\includegraphics[width=0.32\textwidth,height=.2\textwidth,keepaspectratio]{Figures/Confusion_Matrix_Mid_100GHz_400mW.png}}
%   \caption{Confusion matrices of CNN on testing data for three different distances of Eve, three different frequencies, and Tx power of 400 mW.}
%   \label{fig: Confusion matrix of the 2DCNN model }
% \end{figure*}


% \begin{figure}
%     \centering
%     \includegraphics[width=0.9\linewidth]{Figures/location of eave vs accuracy.png}
%     \caption{The accuracy of the model as Eve moves away from Alice.}
%     \label{fig:Location vs Accuracy}
% \end{figure}

\subsection{Uplink Malicious Localization Results} \label{uplink_results}
To evaluate the performance of the proposed malicious localization attack during the uplink phase, we conduct Monte-Carlo simulations with $1000$ iterations. In each iteration, Eve utilizes the upper and lower limits of Bob's ring obtained via modulation classification and follows the strategy proposed in Algorithm~\ref{Algo2} in order to localize Bob, who is randomly positioned within the ring. The localization performance is measured by calculating the average distance error between Eve's predicted locations and Bob's actual locations over the $1000$ runs. The results of the Monte-Carlo simulations are summarized in Fig.~\ref{fig:strat2result}. It can be seen that the distance error increases when the transmit power of Alice increases from $200$~mW to $400$~mW. This is mainly due to the fact that wireless signals can propagate to further distances with higher transmit powers, hence expanding the coverage region. Thus, the ring in which Bob potentially exists gets bigger, thereby expanding the search space for Eve and increasing; as a result, the localization error. 
Moreover, the distance error decreases with the increase in frequency from microwave ($5$~GHz) to mmWave ($28$~GHz and $100$~GHz) frequencies. The location ring reduces at mmWave frequencies due to higher attenuation experienced at high frequencies. Thus, higher frequencies are more prone to such threats. 
%and further measures are required to secure the location of users.

\begin{figure}
  \centering
  % \subfloat[]
  % {\adjustbox{valign=b}
  % {\begin{tabular}{|c|c|c|c|}
  %   \hline
  %        \multirow[c]2{*}[0in]{\textbf{Scenario}} & \multirow[c]2{*}[0in]{\shortstack{\textbf{Frequency}\\\textbf{(GHz)}}}  & \multirow[c]{2}{*}[0in]{\shortstack{\textbf{Alice's Transmit}\\\textbf{Power (mW)}}}   & \multirow[c]2{*}[0in]{\shortstack{\textbf{Average Distance}\\\textbf{Error (m)}}}\\
  %       &&& \\\hline
  %        a. & \multirow[c]{3}{*}[0in]{5} & 200 &79.1   \\ \cline{1-1}\cline{3-4}
        
  %        b. & & 300 & 94.5   \\\cline{1-1}\cline{3-4}
      
  %        c. &  & 400 &108.2  \\\hline
      
  %        d. & \multirow[c]{3}{*}[0in]{28} & 200 &14.5  \\ \cline{1-1}\cline{3-4}
        
  %        e. & & 300 &17.4   \\\cline{1-1}\cline{3-4}
  %        f. &  & 400 &20  \\ \hline
  %        g. & \multirow[c]{3}{*}[0in]{100} & 200 &3.85 \\ \cline{1-1}\cline{3-4}
  %        h. & & 300 &4.75  \\\cline{1-1}\cline{3-4}
  %        k. & & 400 &5.6  \\ \hline
         
  %   \end{tabular}}}
  \includegraphics[width=0.49\textwidth,height=0.9\textwidth,keepaspectratio]{3.png} %\hspace{0.5em}

  \caption{Average distance error for different scenarios during the Uplink phase.}
  \label{fig:strat2result}
\end{figure}

% \begin{figure}[h]
%   \centering
% \includegraphics[width=0.99\linewidth]{Figures/3D_strat2.png}
%   \caption{Distance error during the Uplink phase.}
%   \label{fig:strat2result}
% \end{figure}

\begin{figure*}[h]
  \centering
  \subfloat[Error for Bob1 (two Bobs case).]{\includegraphics[width=0.33\textwidth,height=0.8\textwidth,keepaspectratio]{4a.png}\label{bob1}}
  \subfloat[Error for Bob2 (two Bobs case).]{\includegraphics[width=0.33\textwidth,height=0.8\textwidth,keepaspectratio]{4b.png}\label{bob2}}
  \subfloat[Distance error for scenario (d) when Eve employs a 10-element ULA.]{\includegraphics[width=0.33\textwidth,height=0.8\textwidth,keepaspectratio]{4c.png}\label{fig: ULA}}
    \caption{Histograms of distance error for the case of two Bobs (scenario (d)) and with Eve having multiple antennas. %{\color{red}Fig 4a and Fig 4b are almost the same. Do we need both? Can we keep only one plot representing the distance error for one Bob (Can be Bob1 or Bob2) as their performances are similar?}.{\color{blue} The purpose here is to show that our approach is generalizable to multiple users. So, showing the error for both Bobs is crucial as per my opinion. However, it can be discussed and finalized.}
    }
\end{figure*}

%Next, we study the effectiveness of the proposed approach for the cases when there are multiple users and when Eve has multiple antennas. 

\subsection{Localization of Multiple UEs} \label{multibob}
To evaluate the generalization performance of the proposed attack, we consider a multi-user scenario where Eve attempts to localize multiple UEs/Bobs. This setup reflects a more realistic environment, such as a cellular or WiFi network, where several users connect to the same base station or access point. In the uplink phase, Eve would require increased receiver bandwidth proportional to the number of Bobs being monitored and localized. Additionally, this scenario demands greater resources, including a larger number of Eves, longer time requirements, and increased processing power. For example, if two Bobs are located in different rings, either a minimum of two Eves (one in each ring) is needed, or a single Eve could localize them sequentially, though this would be more time-intensive.

In the conducted simulation, we assume that Eve aims to localize two users, referred to as Bob1 and Bob2, who are communicating with Alice/BS using different orthogonal frequency-division multiple access (OFDMA) sub-carriers. Since the sub-carrier spacing is relatively small in 5G systems (on the order of kHz), the possible location rings for Bob1 and Bob2, determined during the modulation classification step in the downlink phase, remain unchanged. In the uplink phase, where both Bobs are to be localized within their respective rings, we employ Algorithm~\ref{Algo2}, similar to the single-user case. The key difference is that Eve now intercept two UE channels to take two SNR measurements at each step instead of one.

To evaluate the effectiveness of the proposed attack to localize two UEs, we chose scenario (d) from Table~\ref{tab:results_phase1}, i.e., we set the center frequency to $28$~GHz and transmit power of Alice to $200$~mW. 
We ran Monte Carlo simulations, repeating the experiment $1000$ times. In each iteration, Eve attempted to localize two Bobs, each randomly positioned within one of the several rings. The average localization errors are $0.66$~m for Bob1 and $0.67$~m for Bob2, which are comparable to the results obtained for a single Bob for the same scenario (see Fig.~\ref{bob1} and Fig~\ref{bob2}). Thus, the proposed passive attack could localize multiple UEs without any performance degradation, but at the expense of more compute resources at Eve.

%In the conducted simulation, we assume that Eve wants to localize two users, denoted as Bob1 and Bob2, communicating with Alice over different orthogonal frequency-division multiple access (OFDMA) sub-carriers. {\color{blue}As the sub-carrier spacing is relatively small (on the order of kHz), the possible location rings of Bob1 and Bob2, determined by the modulation classification step during the downlink phase, remain unchanged. In the uplink phase of localizing two Bobs within the ring, Algorithm~\ref{Algo2} is employed as in the case of a single Bob. The key difference is that Eve now makes two SNR measurements at each step rather than a single measurement.} We run Monte-Carlo simulations for scenario (d) in Table~\ref{tab:results_phase1}, repeating the experiment $1000$ times. {\color{blue}During each iteration, Eve attempts to localize two Bobs, each positioned randomly within a ring. The average distance errors are $0.66$~m for Bob1 and $0.67$~m for Bob2, respectively, which are similar to the results achieved for a single Bob in the comparable scenario.} The results of our approach for the case of localizing two Bobs are shown in Fig.~\ref{bob1} and~\ref{bob2}. Hence, the proposed passive attack method is extendable to the case of multiple users without any performance degradation.

\subsection{Eve with Multiple Antennas} \label{mimoEve}
%Massive-input-massive-output (MIMO) systems are envisioned to increase the perceived data rate per user and enhance the system capacity of wireless networks. MIMO systems are also shown to enhance localization accuracy by improving spatial resolution, Angle of Arrival (AoA) estimation, and exploiting multipath signals, leading to more accurate and reliable position detection. Eve can benefit from the potential of MIMO systems to enhance its capability in revealing Bob's location information. In this section, 

We now evaluate the situation whereby Eve employs a uniform linear array (ULA) and performs phased array processing in order to determine the direction of arrival (DoA) of Bob's signal during uplink phase. 
%We note that the accuracy of determining Bob's direction is proportional to the number of antennas at Eve's side. 
When equipped with a ULA, Eve first obtains an initial estimate of location of Bob using Algorithm~\ref{Algo2}. Then, Eve utilizes phased array processing method in order to obtain a DoA estimate $\theta$ from Bob's signal. This DoA estimate $\theta$ coupled with the distance estimate $R$ is utilized by Eve to move by a distance $R$ directly towards Bob's direction $\theta$ in a single-shot.

To evaluate the efficacy of the proposed attack by an Eve with multiple antennas, we chose scenario (d) from Table~\ref{tab:results_phase1}, i.e., we set the center frequency to $28$~GHz and transmit power of Alice to $200$~mW. Eve was equipped with a $10$-element ULA, with element spacing set to half the wavelength. 
We note that though a number of algorithms such as beam scan, multiple signal classification (MUSIC), minimum variance distortionless response (aka CAPON), and amplitude and phase estimation (APES) exist, we chose the root-MUSIC algorithm to estimate the DoA, due to its ability to resolve closely spaced signals and perform well at low SNRs. 

We conducted Monte-Carlo simulations, with each experiment repeated $1000$ times. 
In each experiment, Eve obtained the upper and lower bounds on Bob's location ring, through modulation classification during the downlink phase. Eve then applied Algorithm~\ref{Algo2} to compute Bob's intermediate position within the ring. This was followed by the estimation of DoA $\theta$ and distance $R$ to accurately localize Bob at location $Re^{j\theta}$. As shown in Fig.~\ref{fig: ULA}, the localization error is significantly reduced compared to the single-antenna case under similar conditions. Specifically, we achieve a much lower average localization error of $0.27$~m compared to $0.66$~m for the single-antenna case in scenario (d), as illustrated in Fig.~\ref{fig:strat2result}. This demonstrates that employing multiple antennas at the adversarial node Eve enhances the performance of the proposed attack, improving its ability to extract location information.


%In terms of percentiles, 90\% of the predicted locations are within 22m, 80\% within 9m, and 70\% lie within 6m of the actual location of Bob. 

% \subsection{Strategy III: singular stationary eve node}
% In this strategy, we examine the use of modulation classification similar to the procedure done in down-link simulation. However, in this scenario, the strategy is highly constrained by the transmission power of Bob which is usually low. This means that if Eve does not start close enough to Bob, it will not be able to localize Bob and will need to revert to another strategy. This is further shown in Figure 7 as the regions identified by Eve via the modulation classification on the up-link signal are shown. Notice that for most of the communication area, Eve cannot receive a good signal to perform the modulation classification. 
% \begin{figure}
%     \centering
%     \includegraphics[width=0.5\linewidth]{Figures/Bob Rings.png}
%     \caption{Modulation classification rings in the uplink communication}
%     \label{fig:enter-label}
% \end{figure}
% \subsection{Strategy IV: intelligent multiple stationary eve nodes}
% Another approach is to combine the third strategy with the second strategy. That is, we utilize multiple stationary Eve nodes distributed in a circular pattern around the ring each performing modulation classification and reporting the modulation back to a central computing unit as shown in Figure 8. The central computing unit then determines the regions reported by each node and takes the intersection of all rings as the neighborhood of Bob as shown in Figure 9. Although this method does not pinpoint Bob, it provides a good estimation of Bob's neighborhood and can be narrowed down by the use of more stationary Eves. It also has an advantage over the first strategy as it consumes significantly less time with fewer stationary nodes. The area can be reduced down to 0.05-0.2 of the identified ring in the downlink phase.
% \begin{figure}
%     \centering
%     \includegraphics[width=0.5\linewidth]{Figures/Strat IV.png}
%     \caption{The distribution and rings of the Eve nodes in the uplink phase}
%     \label{fig:enter-label}
% \end{figure}

\section{Discussion}

%In this section, we outline future directions to enhance the performance of the proposed modulation-classification-based attack as follows.
% \subsection{Communication System Model}
% To accurately simulate NLOS communication scenarios, we employ Matlab to replicate the urban landscape of Hong Kong, including its buildings. This simulation allows us to finely tune various parameters, such as building materials, surface reflectivity, transmitter location, transmit power, frequency, and receiver placement. An example SNR heatmap at Hong Kong city is depicted in Fig. \ref{fig:heatmap}.

% In choosing the propagation model, we opt for a RayTracing channel, providing the flexibility to specify the maximum number of reflections for consideration. This choice reduces computational complexity while acknowledging a potential compromise on the absolute accuracy of the real scenario. Notably, the simulation can become more realistic as we aggregate multiple propagation models, conveniently accessible through Matlab, allowing for the exploration of complex channel scenarios. For instance, a RayTracing channel can be configured to enable the receiver to identify signals reflected up to a defined number of times, adding a layer of complexity to the simulation. Furthermore, we can enhance the model by introducing environmental factors like rain and atmospheric gases, contributing to a more nuanced and realistic simulation.

% For a detailed exploration of the simulation options provided by Matlab, please refer to the following site: https://www.mathworks.com/help/antenna/ref/txsite.raytrace.html

\subsection{Modulation Classification-based Attack for NLOS Users}
%In addition to LOS scenarios, an interesting future direction is to investigate the ability of the proposed modulation classification-based attack to localize users in non-line-of-sight (NLOS) conditions. 

For the NLOS scenario, the mapping of a modulation scheme to a nicely shaped geographical region, i.e., a ring, does not hold anymore. Nevertheless, in NLOS scenarios, the localization search space can again be significantly reduced through fingerprinting method as follows. Initially, Eve masquerades as a legitimate UE and traverses through the entire cell region. At each step, Eve captures the signal from the BS/Alice, in order to link the modulation scheme being used with its current location, forming a fingerprint map. This way, the resulting radio map consists of $M$ modulation schemes that are mapped to $R$ regions, each containing multiple discontinuous areas. Once the radio map is established, Eve illicitly senses the downlink signal, employing a CNN for modulation classification. The identified modulation scheme is then linked to the specific region $R_i$ belonging to modulation class $M_i$, significantly reducing the search space from $R$ regions to region $R_i$. 

Further localization strategies, in line with the strategies discussed before, could be employed based on Eve's capabilities. For instance, Eve can apply an {SNR gradient localization strategy}; that is, while monitoring Bob's uplink signal, Eve advances one step at a time (with the step size determined through experimentation) toward the center of each area in the region $R_i$. Eve then computes the gradient of the SNR and selects the path with the highest positive gradient. Another approach is to deploy {multiple stationary adversarial nodes}, each monitoring Bob's uplink signal and reporting the received SNR to a fusion center. Eve (the fusion center) then declares the area within region $R_i$ with the highest SNR values as the predicted location of Bob. Further, Eve may also employ {multiple mobile adversarial nodes} following a trajectory similar to the Grey-Wolf optimizer technique.
%; i.e., Eve utilizes the {received signal strength indicator (RSSI)} from the uplink signal to localize Bob. 
Finally, Eve can benefit from {multiple antennas} to estimate the DoA when there is a LOS component. Implementing such techniques and others allows for the non-cooperative localization of NLOS users and is worth the investigation to boost the performance of the proposed attack. 

%\begin{enumerate}
%    \item SNR-Gradient Strategy: While monitoring the uplink signal, Eve advances one step (with the step size determined through experimentation) toward the center of each area in the region $R_i$. Eve then computes the gradient of the SNR and selects the path with the highest positive gradient.
%    \item Distributed Voting Nodes: Deploying multiple distributed stationary nodes, each monitoring the uplink signal, Eve gathers votes based on received SNR. The area within $R_i$ with the highest votes becomes the predicted location of Bob.
%    \item Mobile Nodes with Received Signal Strength Indicator (RSSI): Employing multiple mobile nodes following a trajectory similar to the grey wolf optimizer technique, Eve utilizes RSSI from the uplink signal to localize Bob. 
%    \item DoA Estimation with Multiple Antennas: multiple antennas can be used for DoA estimation when there is a dominant LOS component.
%\end{enumerate}
%These diverse strategies showcase the adaptability of the proposed method in NLOS scenarios, offering flexibility based on the capabilities of the Eve node.
% \begin{figure}
%     \centering
%     \includegraphics[width=0.95\linewidth]{3Reflection_1Diffraction_SNR_Heatmap.png}
%     \caption{SNR heatmap at Hong Kong city with max number of reflections set to 3 and diffractions set to 1.}
%     \label{fig:heatmap}
% \end{figure}

\subsection{Passive sensing-based Synthesis of Digital Twin Map}

%Introducing a novel approach, Morais et al. \cite{morais2024localization} generated digital twin (DT) radiofrequency (RF) maps to build extensive fingerprinting databases for localization. It creates digital replicas that produce RF map fingerprints by leveraging precise 3D mapping and realistic propagation simulations. This method replaces the labor-intensive process of real-world data collection with creating accurate digital models, resulting in larger and more effective localization databases. The reduced need for human effort in data collection enables researchers to enrich databases with comprehensive information, enhancing localization accuracy.
%{\color{red} The description is not clear here. You should describe a future extension to your work. You are only describing the work in~\cite{morais2024localization}. Re-write this paragraph and link it with your idea.}
%{\color{blue}By leveraging our proposed methodology and integrating it with a tracking mechanism such as Kalman filtering and precise 3D mapping, we can develop a digital twin of the environment. This digital replica enables real-time monitoring of user movements. From Alice's perspective, it facilitates more efficient radio resource allocation, allowing for the dynamic allocation of resources to user-dense areas.}


It is possible to create a digital twin of the environment by mere passive sensing only, i.e., by combining our proposed localization method with a tracking mechanism like Bayesian filtering (e.g., Kalman filtering) along with detailed 3D mapping. This digital replica enables real-time tracking of user movements, and therefore, has both positive and negative implications in a number of scenarios. For example, the mobile operators could utilize the digital twin to enhance the efficiency of radio resource management by enabling the dynamic allocation of resources to areas with higher user density. Further, the mobile operators could also utilize the digital twin to realize additional revenue streams by providing insights to third party businesses about the mobility patterns of people in a region. On the other hand, governments could utilize the digital twin approach for monitoring various kinds of aggregate level behaviors of people in a neighborhood. Need not to say that the adversaries could also synthesize and maintain a digital twin in a fully passive and covert manner, and utilize it in a number of unforeseen malicious ways to cause various kinds of security threats, disrupt the social fabric, inflict financial losses, and more.  

%{\color{red}\subsection{Sensing??}}

\section{Conclusion} \label{sec:conclusion}
%This paper presents a novel two-phase passive attack for localizing LOS wireless users. The proposed passive attack utilizes information from MCS tables through modulation classification to narrow down the search space during the downlink phase. Different strategies are then proposed to localize Bob during uplink phase and the best performing strategy is discussed and analyzed in detail. The scenarios involving multiple users and multiple antennas at Eve are also analyzed. Finally, suggestions are provided to extend the proposed approach to NLOS scenarios and employ a digital twin map approach to achieve good localization accuracy in NLOS scenarios.

This paper argues that the openness of wireless cellular (3GPP Releases 16-20) and WLAN (IEEE 802.11ax) standards could make them vulnerable to various kinds of malicious attacks by adversaries. Specifically, this work demonstrates how an eavesdropper can passively localize a user in a cellular or WiFi network through a two-phase attack. By intercepting downlink communication and utilizing modulation classification, the eavesdropper can estimate the user's location within a ring. In the second phase, more precise localization is achieved by analyzing the user's uplink data. Our simulations validate the effectiveness of this attack in single-user, multi-user, and multiple-antenna scenarios. While this attack currently applies to line-of-sight conditions, future research could extend it to non-line-of-sight scenarios, posing broader security risks to wireless networks.

% \textcolor{red}{The Eve could also benefit from open google maps to learn about the urban environment/buildings in order to have better guess about the location or proximity of UE. }

% \textcolor{red}{
% TODO: maybe, we pitch our work on modulation classification based attack as a digital twin problem where we do malicious sensing of whatever signals are out there in the air in order to create a blueprint of the environment (though with malicious intent). so, modulation classification based proximity detection is one privacy breach due to malicious sensing, and it is quite possible to unearth other such vulnerabilities.
% }



%%%%%%%%%%%%%%%%%%%%%%%%%%%%%%%%%%%%%%%%%%%%%%%%%%%%%%%%

\footnotesize{
\bibliographystyle{IEEEtran}
\bibliography{references}
}
%\newpage


\vfill\break

\end{document}
