\section{Related Work}
% \subsubsection{Modulation Classification}
%Addressing the automatic modulation classification (AMC) challenge involves employing three distinct methodologies: likelihood-based AMC, a feature-based approach, and the utilization of deep neural networks. Likelihood-based AMC relies on the assumption of an ideal signal distribution \cite{Maroto_likelihood} and a time-invariant channel. This condition may not be universally applicable to all wireless communication systems. On the other hand, feature-based AMC necessitates expert feature engineering to extract relevant characteristics from each modulation class, thereby restricting its capacity to generalize across diverse modulation schemes \cite{feature-based}.

%With the rapid increase of data in communication systems and enhanced computational power, deep neural networks (DNNs) swiftly became a key solution for the AMC challenge. In recent years, diverse architectures ---ranging from convolutional neural network (CNN)-based models \cite{O'shea_CNN} to recurrent neural network-based structures \cite{AMC_RNN} and hybrid neural networks \cite{AMC_Hybrid}--- have been investigated to assess their efficacy as automatic modulation classifiers. Unlike feature-based methods, the DNN approach can automatically extract hierarchical features of each modulation type. This enhances its ability to generalize results to other modulation types while maintaining a simpler implementation. The advancements made by DNNs prompted scientists to explore various DNN architectures in order to determine which one could achieve the optimal likelihood-based AMC \cite{Optimal_DNNs}. Wei et al. \cite{Weipaper} conducted a comprehensive analysis of diverse DNNs, assessing their performance in terms of size and computational complexity. 
  % The models were fed with in-phase and quadrature components of received signals, along with higher-order cumulants. The study revealed that CNNs , exemplified by LeNets, exhibit superior accuracy and prediction time compared to recurrent neural networks of similar size, particularly in evaluations on edge devices. Despite CNNs proving effective as lightweight models for AMC, their reliance solely on spatial information renders their results suboptimal for similar modulation types. Addressing this limitation, an alternative CNN variant was proposed in \cite{Qi_ResNet}, aiming to enhance results for similar modulation types by leveraging multi-domain features.
  % In the work by Lin et al. \cite{Lin_Attention_CNN}, an attention mechanism is employed to extract channel, frequency, and time features from spectrogram images prior to their input into a CNN. While this method yielded improved results compared to hybrid models like stacked convolutional long short-term memory deep neural networks, it comes with the trade-offs of slower inference and training times, along with the need for preprocessing the raw signal. 

%