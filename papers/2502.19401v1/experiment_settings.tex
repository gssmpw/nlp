\subsection{Experiments Setup}

The framework proposed in this paper is applied to a power line inspection problem \cite{wenkai2017, park2019}, using the LineDrone robot \cite{miralles2018, hamelin2019}. This UAV was developed by Hydro-Québec and DroneVolt to conduct in-contact non-destructive inspections on energized high voltage power lines. It has a GPS with a dual antenna for localization and orientation, a vertically mounted LiDAR for obstacle detection, and a Jetson Xavier NX to handle the computational load. The algorithm is implemented in C++11 and deployed using the Robot Operating System (ROS). We use different libraries, including the Open Motion Planning Library (OMPL) \cite{ioan2012} for the initial trajectory conditions, Octomap \cite{hornung2013} for obstacle detection, the Flexible Collision Library (FCL) \cite{pan2012} for collision detection on the initial trajectory, and Pagmo2 \cite{Biscani2020} for multi-objective optimization. Simulations were performed with a 2.3GHz i7 CPU running an 8-core processor and 16 GB RAM. The simulated drone mirrored the perception and localization capabilities of the LineDrone.
