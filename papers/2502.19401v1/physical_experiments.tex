\subsection{Physical Experiments}
\label{sec:physical_experiment}

The feasibility of our approach was validated through real-world flight tests with the LineDrone in a scaffold structure mimicking a power line inspection mission. Fig. \ref{fig:flight_test} shows the results, where the UAV followed two trajectories: one optimized for energy efficiency, passing between structures to reach its goal faster, and another optimized for safety, with a larger average distance to obstacles. The computation times on the LineDrone's computer were slightly over 4 seconds, using the recommended hyperparameter values. The results demonstrate the algorithm’s ability to plan behaviors tailored to specific risk scenarios, prioritizing either energy savings or obstacle clearance.

\begin{figure}[]
    \begin{minipage}{\linewidth}
        \centering
        \captionsetup[subfloat]{textfont=small}
        \subfloat[]{
            \includegraphics[width=0.49\linewidth, height=1.25in]{mashup_image_with_anotations.png}
            \label{fig:drone_outside_mashup_pic}
        }
        \subfloat[]{
            \includegraphics[width=0.49\linewidth, height=1.25in]{solution_set_flight_test.png}
            \label{fig:solution_set_echafaudage}
        } 
        \captionsetup{width=\linewidth}
        \caption{Flight test with the LineDrone on a power line model. (a) Path reconstructed from a video. (b) Planned Pareto front. Safety trajectory has coefficients $k_{s}=0.5$, $k_{t}=0.1$, and $k_{e}=0.4$. Energy trajectory has coefficients $k_{s}=0.05$, $k_{t}=0.05$, and $k_{e}=0.9$. For simplicity, no non-insertion zone was defined, allowing the optimizer to plan trajectories going over the infrastructure.}
        \label{fig:flight_test}
    \end{minipage}
\end{figure}

To construct and validate our power consumption model, we conducted 42 real-world flight tests, where the UAV flew in 42 different directions at a speed of 2 m/s for 14 seconds to capture steady-state power consumption. We used data from the six directional axes (14\% of the dataset) to construct the model, reserving the remaining 86\% for validation. Fig. \ref{fig:bland_altman_plot_more_points} illustrates model performance with a Bland-Altman plot, where the six data points used for model construction are marked with vertical lines and excluded from evaluation. The blue shaded area corresponds to pitch and roll maneuvers with descent, while the green shaded area represents pitch and roll maneuvers involving ascent. The results show good agreement between our model and the dataset, with the model's validation dataset yielding a mean error of 11W $\pm$ 118W corresponding to 0.97\% of the full power range (1137.6W). We also applied the model to real-world flight trajectories, as shown in Fig. \ref{fig:empirical_power_consumption_vs_model_estimation_for_safety_traj} and Fig. \ref{fig:empirical_power_consumption_vs_model_estimation_for_energy_traj}. The estimations returned by our model closely follow the real power consumption trends observed during flights. For the safest trajectory, the model's absolute mean error was 160.9W (14\% of the full power range), and for the energy-efficient trajectory, the error was 74.7W (6.6\% of the same range). Transient effects did not induce noticeable power peaks, confirming our initial hypothesis. These results show that our model is sufficiently accurate to estimate power consumption, enabling the optimizer to generate more energy-efficient trajectories.

\begin{figure*}[]
    \centering
    \subfloat[]{
        \includegraphics[width=2.15in]{bland_altman_plot_more_points_v2.png}
        \label{fig:bland_altman_plot_more_points}
    }
    \hspace{0.025em}%
    \subfloat[]{
        \includegraphics[width=2.15in]{empirical_power_consumption_vs_model_estimation_for_safety_traj.png}
        \label{fig:empirical_power_consumption_vs_model_estimation_for_safety_traj}
    }
    \hspace{0.025em}%
    \subfloat[]{
        \includegraphics[width=2.15in]{empirical_power_consumption_vs_model_estimation_for_energy_traj.png}
        \label{fig:empirical_power_consumption_vs_model_estimation_for_energy_traj}
    }
    \captionsetup{width=\linewidth}
    \caption{Empirical power consumption data of a flight test with the LineDrone in windy conditions. (a) Bland-Altman graph of the empirical data VS the power model consumption's best fitting model (b)-(c) Model estimation VS real power consumption for the "safety" and "energy" trajectories in Fig. \ref{fig:flight_test}}
    \label{fig:power_consumption_model_visualization}
    \vspace*{-\baselineskip}
\end{figure*}


