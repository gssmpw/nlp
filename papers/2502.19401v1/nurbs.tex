\subsection{Non-Uniform Rational B-Splines (NURBS)}
\label{sec:nurbs}

A NURBS curve of order $p + 1$ is defined by Piegl et al. \cite{piegl1995} as

\begin{equation}
\label{nurbs_equation}
C(u) = \frac{\sum_{i=0}^{n} N_{i, p}(u)w_{i}P_{i}}{\sum_{i=0}^{n} N_{i, p}(u)w_{i}}
\end{equation}

where
\begin{itemize}
    \item $n$ is the number of control points,
    \item $p$ is the degree of the basis function $N_{i, p}$,
    \item $P_{i} = [x_{i} \ \ y_{i} \ \ z_{i}]^{T}$ is the $i^{th}$ control point (assuming a 3D curve), and
    \item $w_{i}$ is its weight.
\end{itemize}

The basis functions $N_{i, p}$ are defined with respect to the parameter $u$ and a fixed knot vector

\begin{equation}
\label{knot_vector_equation}
    U = [u_{0} \ \ ... \ \ u_{m}]^{T}
\end{equation}
containing $m + 1$ knots, whereas $m = n + p$. The De-Boor-Cox formulas allow to calculate the basis function in recursion \cite{piegl1995, deboor1972, deboor1978}.