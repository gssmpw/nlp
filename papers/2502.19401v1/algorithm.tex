\subsection{Algorithm}
\label{sec:risk_aware_path_planning_solver}


Our adaptive risk-aware path planning algorithm has two steps. First, we use a multi-objective solver that returns a Pareto front of feasible trajectories. Then, we filter these trajectories with our voting algorithm to select the most suitable one based on real-time mission risks.

\subsubsection{Solver}
\label{sec:solver}

Among many multi-objective optimizers, we chose the NSGA-II genetic algorithm for its efficiency with continuous problems \cite{deb2002}. It uses polynomial mutation and simulated binary crossover to explore the search space. In its default implementation, NSGA-II initializes a population of $\mu$ individuals using a normal distribution within the bounded search space. However, due to hard constraints in our problem definition, at least one feasible trajectory must be in the initial population. We generate this trajectory using a rapidly exploring random tree algorithm named RRT-Rope \cite{petit2021}, where equidistant path nodes serve as control points for the first decision vector as outlined in Eq. \ref{eq:design_vector}. A component-wise Gaussian filter $\mathcal{N}(z_{i}, \sigma^{2}{G})$ is applied to all decision variables except control point weights. We empirically set $\sigma^{2}{G} = 15m$ for position variables and $\sigma^{2}{G} = \frac{v{max}}{2} m/s$ for velocity norms. All weights are set to $w_{i} = 1$. After optimizing the cost functions, the process yields a set of feasible solutions on a Pareto frontier.

\begin{figure}
    \smallskip
    \smallskip
    \centering
    \includegraphics[width=0.95\linewidth]{voting_algorithm.png}
    \caption{Voting algorithm diagram}
    \label{fig:voting_algorithm_diagram}
\end{figure}

\subsubsection{Voting}
\label{sec:voting}
We implemented a voting algorithm to select the best trajectory based on real-time risks (Fig. \ref{fig:voting_algorithm_diagram}). Trajectories from the Pareto set are ranked according to the objectives, and using risk coefficients, the algorithm selects the most suitable solution. We leverage the plurality of optimized solutions to choose the one that best fits the current mission status. The voting behavior is given by:

\begin{equation}
    \label{eq:voting_equation}
    R_{i_{Final}} = k_{T} R_{i_{T}} + k_{S} R_{i_{S}} + k_{E} R_{i_{E}}
\end{equation}
where $R_{i_{Final}}$ is the final ranking of a solution and $R_{i_{T}}$, $R_{i_{S}}$, and $R_{i_{E}}$ are the rankings of the solution's objectives compared to others in the Pareto set. The ranks coefficients  $k_{T}$, $k_{S}$, and $k_{E}$ are defined by our 4 mission risks: (a) wind risk $\mathcal{WR}$, (b) communication risk $\mathcal{CR}$, (c) localization risk $\mathcal{LR}$, and (d) battery risk $\mathcal{BR}$. These risks are estimated in real-time during the mission, based on factors like wind speed, motor saturation, data transmission speed, GPS satellite availability, dilution of precision, battery level, and return-to-home distance. As mentioned in \cite{petit2024}, these coefficients can be adapted for other applications.


\begin{flalign}
    \label{eq:coefficient_adjustment_functions}
    \begin{aligned}
        k_{S} &= \gamma k_{S, 0} (1 + (\frac{1}{2}\mathcal{WR} + \frac{1}{4}\mathcal{CR} + \frac{1}{4}\mathcal{LR} - \mathcal{BR})),&& \\
        k_{T} &= \gamma k_{T, 0} (1 - (\frac{1}{2}\mathcal{WR} + \frac{1}{4}\mathcal{CR} + \frac{1}{4}\mathcal{LR} - \mathcal{BR})),&& \\
        k_{E} &= \gamma k_{E, 0} (1 + (\frac{1}{2}\mathcal{WR} + \frac{1}{2}\mathcal{BR})),&& \\
        1 &= k_{S} + k_{T} + k_{E}&&
    \end{aligned}
\end{flalign}

Here, $\gamma$ serves as a scaling factor for normalization. Adjusting the risks in real-time to modify an objective's influence addresses a key challenge in multi-objective optimization highlighted in \cite{hayes2022}.


\subsubsection{Hyperparameters}
\label{sec:hyperparameters}

In table \ref{table:parameters_table} we give an overview of the most important hyperparameters of our path planning method. Note that the NSGA-II algorithm's crossover and mutation parameters were chosen according to the recommendation of the Pagmo library's NSGA-II implementation \cite{Biscani2020}.

\begin{table}[h]
    \caption{Hyperparameters for every step of our Adaptive Risk-Aware path planning method}
    \label{table:parameters_table}

    \resizebox{\columnwidth}{!}
    {
        \renewcommand{\arraystretch}{1.5}
        \begin{tabular}{c|c|c|c|}
             \cline{2-4}
             & \textbf{Parameter} & \textbf{Symbol} & \textbf{Value} \\ \hline
             \multicolumn{1}{|c|}{General}
             & \begin{tabular}[c]{@{}l@{}}
                UAV mass ($kg$) \\ UAV speed ($m/s$) \\ UAV acceleration ($m/s^{2}$) \\ NURBS curve degree \\ SDF influence radius \\ Convex hull influence radius
            \end{tabular}
            & \begin{tabular}[c]{@{}c@{}}
               $m_{uav}$ \\ $v_{max}$ \\ $a_{max}$ \\ $p$ \\ $[ r_{sdf_{min}}, r_{sdf_{max}} ]$ \\ $r_{ch_{max}}$
            \end{tabular}
            & \begin{tabular}[c]{@{}c@{}}
               $\mathbb{R}_{> 0}$ \\ $\mathbb{R}$ \\ $\mathbb{R}$ \\ $\{ x \in \mathbb{N} \mid 1 < x \leq 5 \}$ \\ $\mathbb{N}_{> 0}$ \\ $\mathbb{N}_{> 0}$
            \end{tabular} \\ \hline
            
             \multicolumn{4}{c}{} \\ \hline

            \multicolumn{1}{|c|}{\textbf{Algorithm step}} & \textbf{Parameter} & \textbf{Symbol} & \textbf{Value} \\ \hline
            \multicolumn{1}{|c|}{Initialization} & RRT-Rope node distance & $\delta_{rope}$ & $\mathbb{R}_{> 0}$ \\ \hline
            \multicolumn{1}{|c|}{NSGA-II}
            & \begin{tabular}[c]{@{}l@{}}
                Number of generations\\ Population size\\ Number of NURBS points
            \end{tabular}
            & \begin{tabular}[c]{@{}c@{}}
                $N_{gen}$\\ $N_{pop}$\\ $N_{nurbs}$
            \end{tabular}
            & \begin{tabular}[c]{@{}c@{}}
                $\mathbb{N}_{> 0}$\\ $\mathbb{N}_{> 0}$\\ $\mathbb{N}_{> 0}$
            \end{tabular} \\ \hline
            \multicolumn{1}{|c|}{Voting algorithm}
            & \begin{tabular}[c]{@{}l@{}}
                Time coefficient\\ Safety coefficient\\ Energy coefficient
            \end{tabular}
            & \begin{tabular}[c]{@{}l@{}}
                $k_{t}$\\ $k_{s}$\\ $k_{e}$
            \end{tabular}
            & \begin{tabular}[c]{@{}l@{}}
                $\{ x \in \mathbb{R} \mid 0 \leq x \leq 1 \}$\\ $\{ x \in \mathbb{R} \mid 0 \leq x \leq 1 \}$\\ $\{ x \in \mathbb{R} \mid 0 \leq x \leq 1 \}$
            \end{tabular} \\ \hline
        \end{tabular}
    }
\end{table}





