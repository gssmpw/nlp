\subsection{Problem Definition and Representation}
\label{sec:risk_aware_path_planning_problem_definition}

We generalize the infrastructure inspection path planning scenario as a 4D problem, which we define as $\mathbb{D}^{4} = \{ (x, y, z, v) \mid x \in [x_{min}, x_{max}], y \in [y_{min}, y_{max}], z \in [z_{min}, z_{max}], v \in [-v_{max}, v_{max}] \}$ with start and goal positions $\{(x_{a}, y_{a}, z_{a}, v_{a}), (x_{b}, y_{b}, z_{b}, v_{b})\} \in \mathbb{D}^{4}$. The goal of this MOPP problem is to find the NURBS curve $C = \{C(u): u \in [a, b]\}$ that minimizes the set of objectives $\mathcal{F}$ described below.


We choose to represent trajectories using NURBS \cite{piegl1995}, described in Section \ref{sec:nurbs}, to benefit from their inherent properties for multidimensional multi-objective optimization. These include \textit{strong convex hull} properties, \textit{local approximation} capability, and \textit{infinite differentiability} apart from knot multiplicity \cite{piegl1995}. The first two properties are helpful during optimization to constrain the curve inside a bounded area and to escape local minima by varying the curve locally. The last property ensures the smoothness of the curve according to the knot multiplicity of the curve, which is beneficial for UAVs that have strict actuator limitations, such as low acceleration. The design vector is given by:
\begin{equation}
    \label{eq:design_vector}
    \begin{aligned}
        z ={} & [w_{0} \ \ x_{1} \ \ y_{1} \ \ z_{1} \ \left \| \vec{v}_{1} \right \| \ w_{1}\\
              &\ \ \ \ \ \ \ \ \ \ \ \ \ \ \ ...\\
              &x_{n-1} \ \ y_{n-1} \ \ z_{n-1} \ \left \| \vec{v}_{n-1} \right \| \ w_{n-1} \ \ w_{n}]^{T}
    \end{aligned}
\end{equation}


We introduce the 4$^{th}$ dimension to the parametric curve $C$, representing the velocity norm at each control point, enabling velocity profile modulation along trajectories as risks evolve. The start and goal positions, along with their speeds, are fixed and excluded from the decision vector $z$. The problem is constrained by two hard limits: acceleration ($a_{max}$ in Table \ref{table:parameters_table}) to meet actuator limitations and collision avoidance to ensure feasibility.


Costs are generalized into three functions to encompass multiple scenarios: time, safety, and energy consumption, which are described in the following subsections.

%%%% TIME COST %%%%
\hfill
\subsubsection{Time cost}
\label{sec:time_cost}
The time cost is computed as:

\begin{equation}
    \label{eq:time_cost}
    \mathcal{F}_{Time} = \sum_{i=0}^{Q-1} \frac{d_{i}}{ \left \| \vec{v}_{i + 1} \right \| }
\end{equation}
where $Q$ is the number of sample points and $d_{i}$ is the distance on the path segment. The $i + 1$ velocity is used, assuming inspection drones are slow and can quickly reach low speeds. Under normal conditions, with no significant risks during an inspection mission, the algorithm minimizes this cost function, prioritizing speeding up the inspection process over safety or energy consumption.

%%%% SAFETY COST %%%%
\hfill
\subsubsection{Safety cost}
\label{sec:safety_cost}
The safety cost consists of two terms:

\begin{equation}
    \label{eq:safety_cost}
    \begin{aligned}
        \mathcal{F}_{Safety}={} & k_{a} (\frac{\sum_{i=0}^{Q} \mathcal{F}_{sdf_{i}}}{Q} + max(\mathcal{F}_{sdf})) \\ & + k_{b} (\frac{\sum_{i=0}^{Q} \mathcal{F}_{ch_{i}}}{Q} + max(\mathcal{F}_{ch}))
    \end{aligned}
\end{equation}
where $\mathcal{F}_{sdf_{i}}$ is a collision cost to ensure a safe UAV-obstacle distance and $\mathcal{F}{ch_{i}}$ denotes a non-insertion cost for obstacles forming convex hulls to keep trajectories outside critical zones. In this letter, convex hulls, modeled as oriented bounding boxes (OBB), are dynamically adjusted around cables and pylons using semantic information during power line inspections. The coefficients $k_{a}$ and $k_{b}$ in Eq. \ref{eq:safety_cost} are normalized since $\mathcal{F}{sdf_{i}}$ and $\mathcal{F}{ch_{i}}$ return normalized values. Combining mean and maximum costs ensures overall trajectory safety while maximizing the minimum obstacle distance. To support this, we use a signed distance field (SDF) \cite{jones2006} alongside our path planning algorithm to maintain obstacle information within the free space. $\mathcal{F}{sdf_{i}}$ is defined using the closest obstacle distance in the SDF:

\begin{equation}
    \label{eq:sdf_cost}
    \mathcal{F}_{sdf_{i}} = 
    \begin{cases} 
        0 & d_{obs_{i}} \geq r_{sdf_{max}} \\
        \frac{\lambda}{d_{obs_{i}}} - 1 & r_{sdf_{min}} < d_{obs_{i}} < r_{sdf_{max}} \\
        1 & d_{obs_{i}} \leq r_{sdf_{min}}
    \end{cases}
\end{equation}
where $\lambda = \frac{r_{sdf_{min}} r_{sdf_{max}}}{r_{sdf_{max}} - r_{sdf_{min}}}$ is a scaling factor and $d_{obs_{i}}$ is the distance between the UAV and the nearest obstacle. Both $r_{sdf_{min}}$ and $r_{sdf_{max}}$ are listed in Table \ref{table:parameters_table}. This cost function reflects the increasing risks as the UAV gets closer to obstacles, which rapidly increases near them. The non-insertion cost function $\mathcal{F}_{ch_{i}}$ is formulated as follows:

\begin{equation}
    \label{eq:convex_hull_cost}
    \mathcal{F}_{ch_{i}} = \sum_{i=0}^{N}
    \begin{cases}
        0 & d_{ch_{i}} \geq r_{ch_{max}} \\
        1 - \frac{d_{ch_{i}}}{r_{ch_{max}}} & 0 < d_{ch_{i}} < r_{ch_{max}} \\
        1 & d_{ch_{i}} \leq 0
    \end{cases}
\end{equation}
where $d_{ch_{i}}$ is the distance to the nearest convex hull and $r_{ch_{max}}$ is the maximum influence radius of convex hulls. Under high wind, communication, or localization risks, the algorithm prioritizes safety by selecting trajectories that minimize this cost function over time or energy efficiency.

%%%% ENERGY COST %%%%
\hfill
\subsubsection{Energy cost}
\label{sec:energy_cost}
To evaluate the energy consumption of a planned trajectory, our approach integrates physics-based principles with experimental data. \cite{guoku2022} proposed a model for quadrotor UAVs with BLDC motors, showing that energy consumption varies with the flight state. According to this model, it can be hypothesized that for a given flight speed, horizontal movements require more power than hovering, with vertical movements further affecting it. The pitch/roll power relationship with the vertical axis forms a quadric surface \cite{hilbert1999}, described by a general equation:


\begin{equation}
    \label{eq:quadric_surface}
    \begin{aligned}
        {} & ax^{2}+by^{2}+cz^{2}+dxy+eyz+fxz\\
        & \ \ \ +gx+hy+kz+L = 0 
    \end{aligned}
\end{equation}

Because of a quadrotor's symmetry and to simplify the model, coupling terms are eliminated. The term $L$ acts as a geometric translation factor and is fixed so the system doesn't return the trivial solution. With these simplifications, power data from minimally the six flight directions ($\pm Z, \pm Roll, \pm Pitch$) creates a solvable system of equations. To locate a point on this surface during optimization, we project along a unit vector $\hat{v} = (v_{x}, v_{y}, v_{z})$, parameterized by $t$: $x = t v_{x}, y = t v_{y}, z = t v_{z}$. Substituting into Eq. \ref{eq:quadric_surface} simplifies to:

\begin{equation}
    \label{eq:simplified_quadric_equation}
    At^{2} + Bt + L = 0
\end{equation}
where $A = av_{x}^{2} + bv_{y}^{2} + cv_{z}^{2}$, $B = gv_{x} + hv_{y} + kv_{z}$, and $L = 1$. The roots of this equation, solved using the quadratic formula, determine the points on the quadric surface along the unit vector, providing steady-state power consumption $P(\hat{v})$ at these coordinates. The energy consumption cost function is given by:

\begin{equation}
    \label{eq:energy_cost}
    \mathcal{F}_{Energy} = \sum_{i=1}^{Q} P(\hat{v}_{i}) \Delta t
\end{equation}

Since inspection drones are slow and quickly reach cruising speed, we assume steady-state flight energy consumption remains constant, with transient states contributing minimally. Unlike the time cost function, the energy cost function includes a directional factor influencing the optimizer. In low battery conditions, the algorithm prioritizes energy and time over safety.