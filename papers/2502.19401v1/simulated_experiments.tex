\subsection{Simulated experiments}
\label{sec:simulated_experiments}


\begin{figure}[]
    \centering
    \captionsetup[subfloat]{textfont=small}
    \subfloat[Env. 1 Risks variation]{
        \includegraphics[width=0.48\linewidth]{risk_variation_2_iter_ksl_small_V3.png}
        \label{fig:risk_variation_1_iter_ksl_small_V5}
    }
    \subfloat[Env. 1 Gazebo visualization]{
        \includegraphics[width=0.48\linewidth]{risk_variation_small_gazebo_V2.png}
        \label{fig:risk_variation_small_gazebo_V1}
    }\\
    \subfloat[Env. 2 Risks variation]{
        \includegraphics[width=0.48\linewidth]{risk_variation_2_iter_CL_V1.png}
        \label{fig:risk_variation_1_iter_CL_V2}
    }
    \subfloat[Env. 2 Gazebo visualization]{
        \includegraphics[width=0.48\linewidth]{risk_variation_CL_gazebo_V2.png}
        \label{fig:risk_variation_CL_gazebo_V1}
    }
    \captionsetup{width=\linewidth}
    \caption{Path comparison and Gazebo visualization: (a)-(c) Path comparison for different risk sets, with an RGB model indicating the risk influencing trajectory selection: red for low battery, green for localization, and blue for high wind risk. Obstacles are shown using an Octomap. (b)-(d) Gazebo simulations of a construction site and power lines model.}
    \label{fig:path_comparison}
\end{figure}

The approach was tested in 7 Gazebo environments: five reconstructed high voltage power lines environment, simulating power line inspections, and two industrial construction site scenarios with variations in the drone’s position. This paper highlights two of these environments, each presenting unique challenges, such as a wide-open search space or multiple trade-offs between safety, time efficiency, and energy efficiency.


\begin{figure*}[!t]
    \centering
    \captionsetup[subfloat]{textfont=small}
    \hfill
    \subfloat[Time cost]{
        \includegraphics[width=2.0in, height=1.6in]{heatmap_time.png}
        \label{fig:heatmap_time}
    }
    \hfill
    \subfloat[Safety cost]{
        \includegraphics[width=2.0in, height=1.6in]{heatmap_safety.png}
        \label{fig:heatmap_security}
    }
    \hfill
    \subfloat[Energy cost]{
        \includegraphics[width=2.0in, height=1.6in]{heatmap_energy.png}
        \label{fig:heatmap_energy}
    }
    \hfill
    \captionsetup{width=\linewidth}
    \caption{Sensitivity study of the voting algorithm. The hyperparameters are set as follows: $v_{max}=2.0$, $a_{max}=2.2$, $p=3$, $r_{sdf_{min}}=1$, $r_{sdf{max}}=5$, $r_{ch_{max}}=2$, $\delta_{rope}=5.0$, $N_{gen}=1000$, $N_{pop}=40$, $N_{nurbs}=50$. Every coefficients set is evaluated over 3 iterations.}
    \label{fig:sensibility_study}
    \vspace*{-\baselineskip}
\end{figure*}

The trajectories, shown in Fig. \ref{fig:risk_variation_1_iter_ksl_small_V5}, illustrate distinct behaviors: (a) red trajectories are faster but offer less obstacle clearance when battery risk is high, (b) green and cyan trajectories maintain greater distances from obstacles when wind and/or GPS risks are high, (c), blue trajectories save energy while maintaining minimum distance with obstacles, (d) intermediate behaviors arise from varying risk combinations.
 

The execution time of our algorithm depends on several hyperparameters. For our application, we found that $\delta_{rope} = 5.0$, $N_{gen} = 1000$, $N_{pop} = 40$, and $N_{nurbs} = 50$ with a NURBS curve degree of 3 consistently provide satisfactory results within 2 seconds on the simulation setup. The introduction of a 4$^{th}$ dimension to the NURBS makes $\delta_{rope}$ the most influential hyperparameter, as it controls the number of control points in the decision vector, affecting the size of the search space.


To benchmark our algorithm's performance, we converted the multi-objective problem into a single objective one for each criterion. The algorithm was run ($\delta_{rope} = 2.0$, $N_{gen} = 2500$, $N_{pop} = 200$, $N_{nurbs} = 50$) 1000 times in each environment for each objective in every environment. We selected the minimum value for each criterion demonstrating specific behaviors in each environment: the fastest, the most energy-efficient, and the safest.


Fig. \ref{fig:sensibility_study} illustrates a sensitivity analysis of our voting algorithm conducted in Env. 1 (Fig. \ref{fig:risk_variation_small_gazebo_V1}). We use the three cost functions to assess the algorithm's response to varying mission risks. The coefficients are normalized and summed to one, ensuring only their relative values influence the objective rankings. This normalization allows representation on a 2D plane, with the $k_{E}$ axis at 45 degrees. Outliers above $3\sigma$ are marked in red, and the maximum value is indicated. The results show that the algorithm consistently produces a Pareto frontier across a broad range of objective values. Increasing a coefficient reduces its associated cost: higher safety coefficients yield safer trajectories, while higher time and energy coefficients lead to faster, more energy-efficient ones. The safety objective shows more distinct minima, while energy and time costs display smoother transitions, indicating greater sensitivity to weight adjustments for these objectives. This supports dynamic trade-offs as risks change.  Our algorithm achieves optimal safety-focused solutions and produces fast and energy-efficient solutions that deviate only 7.8\% and 7.9\%, respectively, from the benchmarks. These results demonstrate the algorithm's robustness and capacity to yield near-optimal solutions even with moderate hyperparameter adjustments.

