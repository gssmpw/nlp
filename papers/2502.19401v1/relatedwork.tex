\section{Related Work}
\label{sec:related_work}


\paragraph{Multi-Objective Optimization and Risk Adaptivity} MOPP methods are widely used for global trajectory generation in UAV applications, addressing various objectives \cite{hayes2022}. A common approach to handling multiple objectives combines cost functions into a single scalar value. This typically involves weighing the objectives to obtain desirable behavior. In our previous work, we applied this method, integrating safety, energy, and time \cite{petit2024}. Similarly, the authors in \cite{weiss2006} combined path length with a risk-utility function based on a risk map. Zhou et al. \cite{zhou2021} introduced a gradient-based method that incorporates smoothness and feasibility functions to local graph search-based collision risks. While this approach simplifies the inherently multi-objective problems, the authors in \cite{hayes2022} highlight several issues with this workflow. One key concern is that preferences between objectives may evolve over time and missions, making the transformation of an inherently multi-objective problem into a single cost function a semi-blind, manual process. Another approach to handling multiple objectives is proposed by the authors in \cite{korpan2023} where they use A* independently on every objective to mimic specific behaviors and then use a voting algorithm to choose between them. While this method produces paths at the extreme edge of the Pareto front, it fails to address the key issue of evolving preferences between objectives over time, and does not allow for compromises that could better align with a specific situation.

\paragraph{Risk Assessment} Conducting a thorough risk assessment is essential to establish trust in UAV autonomous operations. Some studies introduce safety objectives to prevent unsafe trajectories during MOO processes. Ahmed et al. \cite{ahmed2011, ahmed2013} assess trajectory safety using the robot's range of vision. Many approaches maximize safety by minimizing collision risks with static or dynamic obstacles using clearance-based objectives \cite{khan2021}. Authors in \cite{weiss2006} use collision risk maps, while \cite{rubiohervas2018} combines Gaussian processes and Evolutionary algorithms (EA) to map risks like signal strength and wind speed for single or multiple UAVs. However, these methods address fixed mission statements. As we highlighted in this section, preferences between objectives can shift dynamically during and between missions. Factors like worsening weather or GPS reception may prioritize safety, while battery constraints could favor faster or more energy-efficient paths. In our recent work, we introduced the concepts of damage cost in the event of a crash and non-insertion cost, which quantifies the degree of intrusion into important obstacle structures \cite{petit2024}, providing situational awareness and a margin of error to respond effectively to dangerous situations.

\paragraph{Trajectory Representation} While not tied to a specific mission statement, our two concepts of damage and non-insertion costs are constrained to a 2D plane with discretely constructed paths, similar to \cite{song2019, primatesta2019}, leading to resolution loss and limited path optimality. To overcome this, \cite{zhou2021} proposes B-spline parametrization to represent trajectories continuously, using knot vectors based on actuator limitations. However, this method lacks flexibility in modulating trajectories. For instance, control points are positioned at fixed intervals, constraining the optimizer's ability to discover more optimal solutions. Another method, described in \cite{cheatham2005}, proposes an n-dimensional NURBS approach that adds orientation components to XYZ control points. While this enables smoothing additional components of an Euclidean path, it relies on a look-ahead process to ensure compliance with actuator limitations. However, post-optimization adjustments undermine the optimization purpose, as they may deviate from the originally optimized solution.

\paragraph{Energy Efficiency} Energy efficiency is crucial for UAV missions, especially under battery constraints. Several methods exist to model energy consumption \cite{zhang2021}. For example, \cite{zeng2017} defines a detailed energy model for fixed-wing UAVs based on speeds and accelerations along trajectories. Other methods, like \cite{hasini2018}, empirically derive energy consumption by evaluating a drone under various scenarios to create situational equations. While validated through theoretical energy consumption, this approach depends on best-fitting curves rather than physics-based principles, making it scenario-dependent. To our knowledge, no method formulates a generalized data-driven energy model using physics-based principles.


In this paper, we introduce a novel framework for an online adaptive energy-efficient 3D multi-objective path planning algorithm designed to generate a Pareto set of trajectories that accommodate the risk variation during a mission. Our contributions are as follows: (a) we introduce a novel voting algorithm for choosing the best trajectory according to evolving mission risks, (b) we push further the notion of non-insertion cost introduced in our recent work \cite{petit2024}, (c) we introduce a 4$^{th}$ dimension in a continuous representation tool to modulate and optimize speed while complying with actuator constraints, and (d) we present a novel semi-empirical energy consumption model that incorporates different flight situations.