\documentclass[letterpaper, 10 pt, conference]{ieeeconf}
\IEEEoverridecommandlockouts
\overrideIEEEmargins  

\usepackage{amsmath,amsfonts}
\usepackage{algorithmic}
\usepackage{algorithm}
\usepackage{array}
\usepackage[caption=false,font=normalsize,labelfont=sf,textfont=sf]{subfig}
\usepackage{textcomp}
\usepackage{stfloats}
\usepackage{url}
\usepackage{verbatim}
\usepackage{graphicx}
\usepackage{cite}
\usepackage{lipsum}
\usepackage{xcolor}
\usepackage{soul}
\usepackage{doi}
\usepackage{caption}
\usepackage{multirow}

\usepackage{hyperref} 
\hypersetup{
    colorlinks,
    citecolor=black,
    filecolor=black,
    linkcolor=black,
    urlcolor=black
}
\usepackage[all]{hypcap}

\usepackage{notoccite}
\hyphenation{op-tical net-works semi-conduc-tor IEEE-Xplore}
% updated with editorial comments 8/9/2021



\begin{document}

\title{ARENA: Adaptive Risk-aware and Energy-efficient NAvigation for Multi-Objective 3D Infrastructure Inspection with a UAV}

\author{David-Alexandre Poissant$^{1-2}$, Alexis Lussier Desbiens$^{1}$, François Ferland$^{2}$, and Louis Petit$^{1-2}$% <-this % stops a space
\thanks{This research was funded by Alliance grant number 2601-2600-703 and CRIAQ between Université de Sherbrooke, Hydro-Québec and DRONE VOLT®.}% <-this % stops a space
\thanks{$^{1}$The authors are with the Createk Design Lab, University of Sherbrooke, Sherbrooke, J1K 2R1, Canada, {\tt\small [david-alexandre.poissant, louis.petit, alexis.lussier.desbiens]@usherbrooke.ca} }
\thanks{$^{2}$The authors are with the Intelligent Interactive Integrated Interdisciplinary Robot Lab (IntRoLab), University of Sherbrooke, Sherbrooke, J1K 2R1, Canada, {\tt\small françois.ferland@usherbrooke.ca} }}


% The paper headers
%\markboth{Journal of \LaTeX\ Class Files,~Vol.~14, No.~8, August~2021}%
%{Shell \MakeLowercase{\textit{et al.}}: A Sample Article Using IEEEtran.cls for IEEE Journals}

%\IEEEpubid{0000--0000/00\$00.00~\copyright~2024 IEEE}
% Remember, if you use this you must call \IEEEpubidadjcol in the second
% column for its text to clear the IEEEpubid mark.

\maketitle

\begin{abstract}
Autonomous robotic inspection missions require balancing multiple conflicting objectives while navigating near costly obstacles. Current multi-objective path planning (MOPP) methods struggle to adapt to evolving risks like localization errors, weather, battery state, and communication issues. This letter presents an Adaptive Risk-aware and Energy-efficient NAvigation (ARENA) MOPP approach for UAVs in complex 3D environments. Our method enables online trajectory adaptation by optimizing safety, time, and energy using 4D NURBS representation and a genetic-based algorithm to generate the Pareto front. A novel risk-aware voting algorithm ensures adaptivity. Simulations and real-world tests demonstrate the planner's ability to produce diverse, optimized trajectories covering 95\% or more of the range defined by single-objective benchmarks and its ability to estimate power consumption with a mean error representing 14\% of the full power range. The ARENA framework enhances UAV autonomy and reliability in critical, evolving 3D missions.
\end{abstract}

\begin{keywords}
Motion and Path Planning, Autonomous Vehicle Navigation, Aerial Systems: Applications, Optimization and Optimal Control, Robust/Adaptive Control
\end{keywords}

\section{Introduction}
\label{sec:introduction}
Recommendation systems are a standard component of most online platforms, providing personalized suggestions for products, movies, articles, and more.
In addition to generic recommendation, these platforms often present the option for the user to search for items, either via natural language or structured queries.
While collaborative filtering methods like matrix factorization have proven successful in addressing data sparsity for unconditional generation, they often fall short when attempting to combine them with more complicated queries. 
This is not unexpected, as vector embeddings, while effectively capturing linear relationships, are ill-equipped to handle the complex set-theoretic relationships. Even advanced neural network-based approaches, which are designed to capture intricate relationships, have been shown to struggle with set-theoretic compositionally that underlie many real-world preferences. 

% Consider the common scenario where a user desires a movie that is both a "comedy" and "action," but not a "romance."
% This demonstrates a need for a recommendation model capable of handling set operations such as conjunction and negation.

% Recommending items according some logical constraints of their attributes is a key problem in many modern applications, such as e-commerce and video/music streaming platforms. These facets are invoked by simple user queries, which typically correspond to categories, tags, or attributes of the items. While some user queries are straightforward, like "comedy movies," more often they are complex, such as "comedy but not romantic comedies." 

Let us consider an example where a user named Bob wants to watch a comedy which is not a romantic comedy.
Assuming we have a prior watch history for users, standard collaborative filtering techniques (e.g. low-rank matrix factorization) would yield a learned score function $\score(m, \Bob)$ for each movie $m$.
% , however this does not incorporate Bob's search request.
If we also have movie-attribute annotations, we could form the set of comedies $C$ and set of romance movies $R$ and simply filter to those movies in $C \setminus R$, however this assumes that the movie-attribute annotations are complete, which is rarely the case.

A standard approach in a setting with sparse data is to learn a low-rank approximation for the {attribute $\times$ movie} matrix $\mathbf A$, yielding a dense matrix $\hat {\mathbf A}$. We can then form sets of movies based on this dense matrix using an (attribute-specific) threshold, \eg $\hat C \defeq \{m \mid \hat A_{\comedy, m} > \tau_\comedy\}$ and $\hat R \defeq \{m \mid \hat A_{\romance, m} > \tau_\romance\}$, and then rank movies $m \in \hat C \setminus \hat R$ according to $\score(m, \Bob)$. While this approach does allow for performing the sort of queries we are after, it suffers from three fundamental issues:

% \begin{figure}[h!]
%   \centering
%   \subfloat[Standard matrix completion assumes you are given partial information about the user $\times$ movie matrix $\mathbf U$, and potentially incomplete information about the attribute $\times$ movie matrix $\mathbf A$, and asks you to recover any unobserved entries. The task of set-theoretic matrix completion extends this to being able to predict the entries of arbitrary set-theoretic combinations of these rows.]{
%     \includegraphics[width=0.45\textwidth]{pictures/set-theoretic matrix completion.jpg}
%     \label{figure: set-theoretic matrix completion}
%   }
%   \hfill
%   \subfloat[Box embeddings represent the movies, users, and attributes as "boxes" (Cartesian products of intervals) in $\mathbb R^n$. The score for a specific movie in relation to a given query is determined by the proportion of the movie box's volume that falls within the corresponding region. During training, this membership score for a movie, w.r.t the $U$ and $A$ are optimized, creating a set-geometric representation of the matrix.]{
%     \includegraphics[width=0.45\textwidth]{pictures/box depiction.jpg}
%     \label{figure: box depiction}
%   }
%   \caption{Set-theoretic matrix completion for movies, users, and attributes, illustrating how box embeddings, trained in a set-theoretic manner, address this task.}
% \end{figure}
\begin{figure}[]
    \centering
    \includegraphics[width=0.8\columnwidth]{pictures/set-theoretic_matrix_completion.jpg}
    \caption{Standard matrix completion assumes you are given partial information about the user $\times$ movie matrix $\mathbf U$, and potentially incomplete information about the attribute $\times$ movie matrix $\mathbf A$.}
    \label{fig:set_theoretic_mc}
\end{figure}

\begin{figure}[]
    \centering
    \includegraphics[width=0.8\columnwidth]{pictures/box_depiction.jpg}
    \caption{Box embeddings represent the movies, users, and attributes as "boxes" (Cartesian products of intervals) in $\mathbb R^n$.}
    \label{fig:box_depiction}
\end{figure}


% \begin{figure}[h]
%     \centering
%     \includegraphics[width=0.8\textwidth]{ICLR 2025 Template/pictures/set-theoretic matrix completion.png} % Adjust width as necessary
%     \caption{The task of set-theoretic matrix completion depicted in the setting where users and attributes form the rows, and movies are the columns. Set-theoretic matrix completion is concerned with not simply filling in additional entries of the user $\times$ Movie matrix $\mathbf U$ or the attribute $\times$ movie matrix $\mathbf A$, but also being able to predict the entries of arbitrary set-theoretic combinations of these rows.}
%     \label{fig:side_caption_image}
% \end{figure}

\begin{enumerate}
    \item Limited user-attribute interaction:
    % separately classifying attributes and then ranking for each user does not take into account user-attribute interactions.
    Since the attribute classification is done independently from the user, any latent relationships between the user and attribute cannot be taken into account.
    \item Error compounding: Errors in the completion of attribute sets accumulate as the number of sets involved in the query increase.
    \item Mismatched inductive-bias: Our queries can be viewed as set-theoretic combinations of the rows, not linear combinations. As such, using a low-rank approximation of the matrix may be misaligned with the eventual use.
\end{enumerate}


% The recommender system has access to the ground truth of the set of movies Bob would like to watch (\textbf{Bob}), the set of comedy movies (\textbf{comedy}), and the set of romantic movies (\textbf{romance}). In this ideal scenario, the system would trivially return \textbf{Bob} $\cap$ \textbf{comedy} $\setminus$ \textbf{romance}. However, in practice, we can only construct these sets from item tags and user history, which are often incomplete and noisy. Consequently, the set operation might yield an inaccurate or empty set of items. This problem is exacerbated as the queries become more complex. (forward reference to experiment sections).

% A standard approach to mitigating the incompleteness issue is to learn representations of \Bob, \romance, and \comedy. One of the traditional yet most effective methods (cite) is to learn a low-rank approximation of the observed matrix $O$ which is the concatenation of the {User $\times$ Movie} interaction matrix $U$ and the {Tags $\times$ Movie} attribute matrix $A$. The learned representations can provide smooth score functions over all possible items for all users and attributes. In our example, we would be able to calculate $\score(\Bob, m)$, $\score(\comedy, m)$, and $\score(\romance, m)$ for all movies $m$ by calculating dot products between the vector representations for the each movie and the vector representations for \Bob, \comedy, and \romance.

% While these scores generalize to the incomplete part of the observed matrix $O$, they do not naturally allow us to compute set-theoretic queries. For example, consider how one might use these representations to address Bob's query from before. 

% This is not optimal for several reasons: the selection of the threshold is an ad-hoc process, and the prediction error for thresholding will snowball rapidly as query complexity increases (see Section ref). A better approach would be to devise a smooth score function for the entire query \textbf{Bob} $\cap$ \textbf{comedy} $\setminus$ \textbf{romance}. A common method to achieve this is by multiplying the scores corresponding to each query, e.g., $s(bob \cap comedy \cap \neg romance, m) = s(bob, m) \times s(comedy, m) \times (1 - s(romance, m))$. However, this approach ignores the interdependence between attributes and users, again resulting in suboptimal behavior for the recommender.\\

In this paper, we formulate the problem of attribute-specific recommendation as matrix completion where rows are not necessarily \emph{linear combinations} of each other but, rather, are \emph{set-theoretic combinations} of each other. More precisely, given some user $\times$ movie interaction matrix $\mathbf U$ and attribute $\times$ movie matrix $\mathbf A$, the queries we are considering are set-theoretic combinations of these rows (see \Cref{fig:set_theoretic_mc}). For example, the ground-truth data for comedies which are not romance movies which Bob likes would be the vector $x \in \{0,1\}^{|M|}$, where $x_m = 1$ if and only if $\mathbf U_{\Bob, m} = 1$ and $\mathbf A_{\comedy, m} = 1$ and $\mathbf A_{\romance, m} = 0$. Note that this is not a linear combination of the previous rows, and so while the inductive bias of low-rank factorization has proven immensely effective for collaborative filtering we should not expect it to be directly applicable in this setting.


% if the observed matrix $O$ is the concatenation of $[U; A]$, the query answering task essentially involves predicting the entries of the rows of the joint matrix $O_{q} = [U; A; U \cap A; U \cap \neg A; U \cap A \cap A; U \cap A \cap \neg A; \cdots]$. 

% Note that, the low-rank approximated vector model is capable of capturing linear dependencies between similar user or attribute rows or between movie columns. This inductive bias proves to be immensely effective for collaborative filtering. However, in our case the relationship amongst the rows of the $O_{q}$ is non-linear and strictly set-theoretic in nature, e.g., the row of \textbf{Bob} $\cap$ \textbf{comedy} is strictly an intersection between the individual rows of \textbf{Bob} and \textbf{comedy}. \\

Instead, we propose to learn representations for the users and attributes that are consistent with specific set-theoretic axioms. These representations must also be compactly parameterizable in a lower-dimensional space, differentiable with respect to some appropriate score function, and allow for efficient computation of various set operations.
% . Additionally, we need to define a measure (similar to vector dot products) to train these representations.
Box Embeddings \citep{hard_box, gumbel_box}, which are axis-parallel $n$-dimensional hyperrectangles, meet these criteria (see \Cref{fig:box_depiction}).
The volume of a box is easily calculated as the product of its side-lengths. Furthermore, box embeddings are closed under intersection (\ie the intersection of two boxes is another box). Inclusion-exclusion thus allows us to calculate the volume of arbitrary set-theoretic combinations of boxes.
% The simple axis-parallel geometry allows for the calculation of intersections of multiple boxes.
% The embedding space is closed under intersection (the intersection of two or more boxes is also a box) and the volume of a box is easily calculated as the product of its side lengths. Via inclusion-exclusion, this allows us to efficiently calculate arbitrary set-theoretic combinations of boxes.
% This ease of parameterization, along with straightforward volume and intersection calculations, makes box embeddings an excellent candidate for our purpose.


The contributions of our paper are as follows -
\begin{enumerate}
    \item We model the problem of attribute-specific query recommendation as "set-theoretic matrix completion", where attributes and users are treated as sets of items. We discuss the challenges faced by existing machine-learning approaches for this problem setup.
    \item We demonstrate the inconsistency of existing vector embedding models for this task. Additionally, we establish box embeddings as a suitable embedding method for addressing such set-theoretic problems.\mb{We don't do this, so we either need to or we need to weaken this claim.}
    \item We conduct an extensive empirical study comparing various vector and box embedding models for the task of set-theoretic query recommendation.
\end{enumerate}

Box embeddings, with their geometric set operations, significantly outperform all vector-based methods. We also evaluate score multiplication and threshold-based prediction for both vector and box embedding models, and find that performing set operations directly on the box embeddings performs best, solidifying our claim that the inductive bias of box embeddings provides the necessary generalization capabilities to address set-theoretic queries.
\section{Related Work}

RLAIF is a popular LLM post-training method \citep{bai2022constitutional}.
Its idea to generate feedback by a model itself is called Pseudo-Labeling (PL) and well studied in semi-supervised learning \citep{scudder1965probability,lee2013pseudo}. Below, we will review PL and LLM self-training methods related to ours. For a thorough review of each topic, please refer to \citet{yang2023survey} and \citet{xiao2025foundations}.

\paragraph{Pseudo-Labeling:}
PL trains a student model so that its output is close to the output of a teacher model on unlabeled data \citep{scudder1965probability,lee2013pseudo}.
Various methods for constructing a teacher model has been proposed.
For example, $\Pi$-model \citep{bachman2014learning,sajjadi2016regularization,laine2017temporal} and virtual adversarial training \citep{miyato2019virtual} perturbs input and/or neural networks of student models.
Some methods use weighted average of previous student models' predictions or weights as a teacher model \citep{laine2017temporal,tarvainen2017mean}, and
other methods even try to optimize a teacher model by viewing PL as an optimization problem \citep{yi2019probabilistic,wang2020repetitive,pham2021meta}.
Our work is complimentary to this line of works since what we modify is the risk estimator rather than teacher models.

PL is known to suffer from erroneous labels generated by a model-in-training \citep{haase2021iterative,rizve2021in}, and this observation well aligns with recent reports on RLAIF that erroneous self-feedback is a major source of failure \citep{Madaan2023-fn}.
A straightforward approach is to filter potentially incorrect labels based on confidence \citep{sohn2020fixmatch,haase2021iterative,rizve2021in} or the amplitude of loss as in self-paced learning \citep{bengio2009curriculum,kumar2010self,jiang2015self}.
Another method is refining pseudo-labels in a way similar to label propagation \citep{zhu2002learning} but with similarity scores computed using neural networks \citep{kutt2024contrastive}.
Our work tackles the issue of erroneous labels by using a risk estimate robust to erroneous self-feedback based on UU learning \citep{Lu2019-sd,Lu2020-dx}. Even though our approach requires only a minimal change, we observed a significant performance boost.

\paragraph{LLM's Self-Refinement:}
To enhance the reasoning capabilities of LLMs, early efforts primarily explored various prompt engineering techniques \citep{Brown2020-bw,Wei2022-kt,Wang2023-pf,Yao2023-ey}. Even with refined prompting strategies, an LLM’s initial outputs can remain limited in some scenarios. Recent studies have proposed iteratively improved answer strategies called self-refinement approaches~\citep{Madaan2023-fn,Kim2023-zz,Chen2024-zm}, and our work falls within this lineage.

Self-refinement involves generating responses through agents in distinct roles that iteratively provide feedback \citep{Shinn2023-no, Zhu2023-wn}. For example, multi-agent debate frameworks \citep{Estornell2024-gh} use an answering agent and an evaluation agent to collaboratively improve answer quality \citep{Chen2024-ua, Du2024-pi, Smit2024-cn}. However, these methods usually assume that the LLM has enough internal knowledge to generate effective feedback and revisions; when it doesn’t, performance gains can be minimal or even negative \citep{Huang2024-co,Li2024-fo,Kamoi2024-lc,Kamoi2024-od}—sometimes degrading performance \citep{Huang2024-co}. Our approach minimizes this reliance: internal knowledge is used only initially for classification, with subsequent improvements relying on extracting features directly from the data via UU learning.

Other work addresses knowledge gaps by retrieving external information or using external tools \citep{Huang2022-mv, Wang2023-fz, Shi2024-gx, Wu2024-cb}, but setting up such systems can be costly. In contrast, our method requires only a small amount of labeled data for initialization without assuming the availability of external knowledge or tools.

\chapter{Deep Learning Fundamentals} % 
\chaptermark{Deep Learning Fundamentals}
%\thispagestyle{empty}
\label{sec:fundamentals}
Tom Mitchell's ML preface \cite{Mitchell1997} defines ML as:
\quotes{A computer program is said to learn from experience $E$ with respect to some class of tasks $T$ and performance measure $P$, if its performance at tasks in $T$, as measured by $P$, improves with experience $E$}.
The experience is primarily described by the amount and quality of data used for the learning process. 
According to different interpretations
of the experience, it is possible to divide the learning approach into supervised, unsupervised and reinforcement learning. 

In the following, we recall different learning strategies and statistical tools we use throughout the thesis, keeping in mind that, in communication theory, signals are studied as stochastic processes. 

\section{Learning theory concepts}
\sectionmark{Learning theory concepts}
\label{sec:distances}
This section introduces key learning concepts that are essential for understanding subsequent chapters.
\subsection{Supervised learning}
Let $(\mathbf{x}_i,\mathbf{y}_i)\sim p_{XY}(\mathbf{x},\mathbf{y})$, $i=1,\dots, N$, be samples collected into a training set $\mathcal{D}$ belonging to the joint probability density function  (PDF) $p_{XY}(\mathbf{x},\mathbf{y})$. Probabilistic supervised learning predicts $\mathbf{y}$ from $\mathbf{x}$ by estimating $p_{Y|X}(\mathbf{y}|\mathbf{x})$ under a \textit{discriminative model} or by estimating the joint distribution $p_{XY}(\mathbf{x},\mathbf{y})$ under a \textit{generative model}. 

A \textit{regression} problem comprises a continuous output $\mathbf{y}$, meanwhile a discrete target is associated to a \textit{classification} problem.
The standard way to formulate the learning process is to define a cost function $C$, namely a performance measure that evaluates the quality of the prediction $\mathbf{\hat{y}}$. In most applications, we can rely only on the observed dataset $\mathcal{D}$ and derive an empirical sample distribution since we do not have knowledge of the true joint distribution $p_{XY}(\mathbf{x},\mathbf{y})$. In particular, the training objective minimizes
\begin{equation}
\label{eq:general_cost}
C(\mathbf{\hat{y}}) = \mathbb{E}_{(\mathbf{x},\mathbf{y}) \sim \mathcal{D}}[\delta(\mathbf{y},\mathbf{\hat{y}})]
\end{equation}
where $\delta$ is a measure of distance between the desired target $\mathbf{y}$ and the prediction $\mathbf{\hat{y}}$.

\subsection{Unsupervised learning}
Let $\mathbf{x}_i\sim p_X(\mathbf{x})$, $i=1,\dots, N$, be samples collected into a training set $\mathcal{D}$ belonging to the PDF $p_X(\mathbf{x})$.
Unsupervised learning aims at finding useful properties of the structure of a dataset $\mathcal{D}$, ideally inferring the true unknown distribution $p_X(\mathbf{x})$.

Several different tasks are solved using unsupervised learning, for instance: clustering, which divides the data into cluster of similar samples; feature extraction, which transforms data in a different latent space easier to handle and interpret; density estimation and generation/synthesis of new samples. The latter objective consists of learning, from data in $\mathcal{D}$, the distribution $p_X(\mathbf{x})$ and producing new unseen samples from it.

Unsupervised learning tasks require the introduction of an hidden variable $\mathbf{z}_i$ for each sample $\mathbf{x}_i$, leading to the selection of different models under a probabilistic approach. In the \textit{discriminative models}, the latent code $\mathbf{z}_i$ is extracted from $\mathbf{x}_i$ by defining a probabilistic mapping $p_{Z|X}(\mathbf{z|x};\theta)$ parameterized by $\theta$. \textit{Autoencoders} encode $\mathbf{x}_i$ into a latent variable $\mathbf{z}_i$ so that recovering $\mathbf{x}_i$ from $\mathbf{z}_i$ is possible through a decoder. The encoder models the posterior distribution $p_{Z|X}(\mathbf{z|x};\theta)$, while the decoder models the likelihood $p_{X|Z}(\mathbf{x|z};\theta)$.
Lastly, in the \textit{generative models}, an hidden variable $\mathbf{z}_i$ generates the observation $\mathbf{x}_i$. After a specification of a parameterized family $p_Z(\mathbf{z}|\theta)$, the distribution of the observation can be rewritten as $p_X(\mathbf{x|\theta})=\sum_z p_Z(\mathbf{z|\theta})p_{X|Z}(\mathbf{x|z;\theta})$.
Fig. \ref{img:taxonomy} schematically
summarizes the discussed learning models.

\begin{figure}
\centering
	\includegraphics[width=\textwidth]{images/fundamentals/LearningModels.png}
	\caption{Taxonomy of learning models. Deterministic models extract either fixed relationships between input and output (supervised) or patterns of the input (unsupervised). Probabilistic discriminative models use the input to predict either the output (supervised) or the hidden variable causing the input (unsupervised). Probabilistic generative models learn the statistical relationship between either input and output (unsupervised) or input and the hidden variable (supervised). Autoencoders model how to encode the input into the hidden variable, as well as how to decode from the hidden variable to the input.}
	\label{img:taxonomy}
\end{figure}

\subsection{Reinforcement learning}
Reinforcement learning (RL) addresses the problem of an \textit{agent} learning to act in a dynamic \textit{environment} by finding the best sequence of actions that maximizes a \textit{reward} function. The basic idea is that the agent explores the interactive environment. According to the observation experience it gets, it changes his actions in order to receive higher rewards. 

Basic RL can be modeled as a Markov decision process (MDP). Let $S_t$ be the observation (or state) provided to the agent at time $t$. The agent reacts by selecting an action $A_t$ to obtain from the environment the updated reward $R_{t+1}$, the discount $\gamma_{t+1}$, and the next state $S_{t+1}$.
In particular, the agent-environment interaction is formalized by a tuple $\langle \mathcal{S}, \mathcal{A}, T,r,\gamma\rangle$ where $\mathcal{S}$ is a finite set of states, $\mathcal{A}$ is a finite set of actions, $T(s,a,s')=P[S_{t+1}=s'|S_t = s, A_t = a]$ is the transition probability from state $s$ to state $s'$ under the action $a$, $r(s,a) = \mathbb{E}[R_{t+1}|S_t = s, A_t = a]$ is the reward function, and $\gamma \in [0,1]$ is a discount factor.
To find out which actions are good, the agent builds a \textit{policy}, i.e, a map $\pi : \mathcal{S} \times \mathcal{A} \to [0,1]$ that defines the probability of taking an action $a$ when the state is $s$. If we denote with $G_t = \sum_{k=0}^{\infty}{\biggl(\prod_{i=1}^{k}{\gamma_{t+i}}\biggr) R_{t+k+1}}$ the discount return, then the goal of the agent is to maximize the expected discount return, i.e., \textit{value} $q^{\pi}(s,a) = \mathbb{E}_{\pi}[G_t|S_t = s, A_t = a]$, by finding a good policy $\pi(s,a)$. 

RL algorithms can be categorized as \cite{SuttonRL,ArulkumaranDRL}: a) policy based methods, when the agent, given the observation as input, optimizes the policy $\pi$ without using a value function $q$; b) value based methods, when the agent, given the observation and the action as inputs, learns a value function $q$; c) actor critic methods, where a \textit{critic} measures how good the action taken is (value-based), and an \textit{actor} controls the behaviour of the agent (policy-based).
Despite several applications of reinforcement learning for physical layer communications (see Ch. 9 of \cite{Eldar2022}), the thesis mostly focuses on the first two learning approaches.

\subsection{Maximum likelihood estimation}
Maximum likelihood (MaxL) estimation is a statistical method commonly used in DL for estimating the parameters of a PDF $p_{\text{model}}(\mathbf{x};\theta)$ that best explains the observed data $p_{X}(\mathbf{x})$, assuming the parameters are fixed but unknown.
Most of generative models work with the MaxL principle \cite{Goodfellow2016}; given a probability distribution parameterized by $\theta$, the estimator for $p_X(\mathbf{x})$ is defined as
\begin{equation}
\label{eq:MaxL}
    \theta_{\text{MaxL}} = \argmax_{\theta}p_{\text{model}}(\mathbf{x};\theta), \; \mathbf{x}\sim p_{X}(\mathbf{x}).
\end{equation}
Given $N$ data points $\mathbf{x}_i\sim p_X(\mathbf{x})$, $i=1,\dots, N$, we seek to maximize
\begin{equation}
    \theta_{\text{MaxL}} = \argmax_{\theta}\prod_{i=1}^{N}{p_{\text{model}}(\mathbf{x}_i;\theta)},
\end{equation}
which can be conveniently rewritten as
\begin{equation}
\label{eq:LogL}
    \theta_{\text{MaxL}} = \argmax_{\theta}\sum_{i=1}^{N}{\log p_{\text{model}}(\mathbf{x}_i;\theta)},
\end{equation}
or alternatively
\begin{equation}
\label{eq:CE_MaxL}
    \theta_{\text{MaxL}} = \argmin_{\theta} \mathbb{E}_{\mathbf{x}\sim p_X(\mathbf{x})}\bigl[-\log p_{\text{model}}(\mathbf{x};\theta)\bigr].
\end{equation}
The last expression is equivalent to a cross-entropy minimization over the parameter $\theta$. Notice that \eqref{eq:CE_MaxL} often appears in terms of Kullback-Leibler (KL) divergence
\begin{equation}
\theta_{\text{MaxL}} = \argmin_{\theta} D_{\text{KL}}(p_{X}(\mathbf{x})||p_{\text{model}}(\mathbf{x};\theta)) = \argmin_{\theta} \mathbb{E}_{\mathbf{x}\sim p_X(\mathbf{x})}\biggl[\log \biggl(\frac{p_{X}(\mathbf{x})}{p_{\text{model}}(\mathbf{x};\theta)}\biggr)\biggr].
\label{MaxLKL}
\end{equation}
Practically, we are interested in a parameterization of $p_{\text{model}}$ which is expressive enough to fully capture data patterns and which allows iterative optimization. Artificial neural networks (NNs) represent a viable solution as explain in Sec. \ref{sec:tips-tricks}.

Similarly, MaxL can be applied to estimate the conditional probability $p_{Y|X}(\mathbf{y|x})$
\begin{equation}
\label{eq:CE_CMaxL}
    \theta_{\text{MaxL}} = \argmin_{\theta} \mathbb{E}_{(\mathbf{x},\mathbf{y})\sim p_{Y|X}(\mathbf{y|x})}\bigl[-\log p_{\text{model}}(\mathbf{y|x};\theta)\bigr].
\end{equation}
In communications, the estimation of the transition probability or likelihood $p_{Y|X}$ is extremely relevant as it is used in the maximum a-posteriori (MAP) decoding strategy. Indeed, we are typically interested in estimating the transmitted input $\mathbf{x}$ given the observation of the output $\mathbf{y}$. Formally, we need to solve
\begin{equation}
    \mathbf{\hat{x}} = \argmax_{\mathbf{x}} p_{X|Y}(\mathbf{x|y}),
\end{equation}
which using Bayes' rule and the logarithmic trick reads as
\begin{equation}
    \mathbf{\hat{x}} = \argmin_{\mathbf{x}} -\log p_{Y|X}(\mathbf{y|x}) - \log p_{X}(\mathbf{x}).
\end{equation}
A famous example is described by the scalar additive white Gaussian noise (AWGN) channel of variance $\sigma_N^2$
\begin{equation}
    p_{Y|X}(y|x)= \frac{1}{\sqrt{2\pi}\sigma_N}\exp\biggl[{-\frac{1}{2}\biggl(\frac{y-x}{\sigma_N}\biggr)^2\biggr]}
\end{equation}
and uniform discrete input source with alphabet dimension $M$, for which it is known that the minimal Euclidean distance represents the optimal decoding criterion \cite{Proakis2001}
\begin{equation}
    \hat{x} = \argmin_{x_i \in \{x_1, \dots, x_{M}\}} |y-x_i|^2.
\end{equation}
It is thus prerogative of any decoding algorithm to estimate the channel model. We leave further details to Ch.~\ref{sec:medium} and Ch.~\ref{sec:decoder}.

\subsection{Statistical distances}
Statistical distances quantify the difference or similarity between two probability distributions. 
They are widely used in various fields such as statistics, machine learning, data analysis, and information theory. Different statistical distances capture different aspects of the distributions and are suitable for different tasks. Arguably the most common statistical distance is the KL divergence.

Let $P$ and $Q$ be absolutely continuous measures w.r.t. $\diff x $ and assume they possess
densities $p$ and $q$, then the KL-divergence is defined as
\begin{equation}
D_{\text{KL}}(P||Q) = \int_{\mathcal{X}}{p(x)\log\biggl(\frac{p(x)}{q(x)}\biggr)\diff x},
\end{equation}
where $\mathcal{X}$ is a compact domain.
The KL divergence measures how much the distribution $P$ differs from a reference distribution $Q$ \cite{KL1951}. The divergence equals zero if and only if $P=Q$ as measures, it is not symmetric and it is generally not upper bounded.
A fundamental particular case of KL divergence is represented by the mutual information (MI) between two random variables $X$ and $Y$. It quantifies the statistical dependence between $X$ and $Y$ by measuring the amount of information obtained about one variable via the observation of the other. The MI is symmetric and it is defined as
\begin{equation}
\label{eq:fundamentals_MI}
    I(X;Y) = \mathbb{E}_{(\mathbf{x},\mathbf{y})\sim p_{XY}(\mathbf{x},\mathbf{y})}\biggl[\log \frac{p_{XY}(\mathbf{x},\mathbf{y})}{p_{X}(\mathbf{x})p_{Y}(\mathbf{y})}\biggr],
\end{equation}
which rewrites in terms of KL divergence as
\begin{equation}
    I(X;Y) = D_{\text{KL}}(p_{XY}(\mathbf{x},\mathbf{y})||p_{X}(\mathbf{x})p_{Y}(\mathbf{y})).
\end{equation}

The KL divergence can be extended to a more general class of divergences referred to as $f$-divergence, where $f$ is a convex lower semicontinuous function $f:\mathbb{R}_+ \to \mathbb{R}$ satisfying $f(1)=0$. The $f$-divergence between $P$ and $Q$ is defined as
\begin{equation}
\label{eq:f-divergence}
D_f(P||Q) = \int_{\mathcal{X}}{q(\mathbf{x})f\biggl(\frac{p(\mathbf{x})}{q(\mathbf{x})}\biggr)\diff \mathbf{x}}.
\end{equation}
It is immediate to notice that the KL divergence directly follows from the $f$-divergence using the generator function $f(u)=u\log u$. Other popular statistical measures based on \eqref{eq:f-divergence} are:
\begin{itemize}
    \item the Jensen-Shannon (JS) divergence, defined in terms of KL divergence as
\begin{equation}
    D_{\text{JS}}(P||Q) = \frac{1}{2}D_{\text{KL}}\biggl(P\biggl|\biggl|\frac{P+Q}{2}\biggr)+\frac{1}{2}D_{\text{KL}}\biggl(Q\biggl|\biggl|\frac{P+Q}{2}\biggr);
\end{equation}
\item the squared Hellinger distance (HD), defined as
\begin{equation}
\label{eq:HD-distance}
H^2(P,Q) :=  D_{\text{HD}}(P||Q) = \frac{1}{2} \int_{\mathcal{X}}{\biggl(\sqrt{p(\mathbf{x})}-\sqrt{q(\mathbf{x})}\biggr)^2\diff \mathbf{x}},
\end{equation}
with $0\leq H(P,Q) \leq 1$, and generator $f(u) = \frac{1}{2}(\sqrt{u}-1)^2$;
\item the total variation (TV) distance, which can be rewritten in terms of $f$-divergence as
\begin{equation}
\label{eq:TV-distance}
V(P,Q) =  D_{\text{TV}}(P||Q) = \frac{1}{2} \int_{\mathcal{X}}{|p(\mathbf{x})-q(\mathbf{x})|\diff \mathbf{x}},
\end{equation}
when $f(u)=\frac{1}{2}|u-1|$. 
\end{itemize}
It is also known that $V(P,Q)\leq \sqrt{1-\exp(-D_{\text{KL}}(P||Q))}$ and $H^2(P,Q)\leq V(P,Q) \leq \sqrt{2}H(P,Q)$.
The challenge that unites statistical distances consists in their explicit calculation. One of the objective of Ch.~\ref{sec:mi_estimators} is to shed some light on how to exploit NNs to estimate such distances.

\section{Machine learning tools}
\label{sec:tips-tricks}
In this section, we briefly introduce ML and DL tools utilized to demonstrate the main results of the thesis. In particular, we focus our attention to NNs since they can handle diverse type of data, including numerical data, text, images, and sequences. NNs are designed and trained for specific tasks by adjusting their architecture, activation functions, and other parameters, making them versatile for solving various problems.

\subsection{Neural networks}
\label{subsec:NN}
Neural networks are among the most popular tools in the ML community as they are known being universal function approximators \cite{Hornik1989}, they can be implemented in parallel on concurrent architectures and most importantly, they can be trained by backpropagation \cite{Rumelhart1985}.

A feedforward NN with $L$ layers maps a given input $\mathbf{x}_0 \in \mathbb{R}^{D_0}$ to an output $\mathbf{x}_L \in \mathbb{R}^{D_L}$ by implementing a function $F(\mathbf{x}_0;\mathbf{\theta})$ where $\mathbf{\theta}$ represents the parameters of the NN. To do so, the input is processed through $L$ iterative steps
\begin{equation}
\mathbf{x}_l = f_l(\mathbf{x}_{l-1};\theta_l), \; l=1,\dots, L
\end{equation}
where $f_l(\mathbf{x}_{l-1};\theta_l)$ maps the input of the $l$-th layer to its output. The most used layer is the fully-connected one, whose mapping is expressed as
\begin{equation}
f_l(\mathbf{x}_{l-1};\theta_l) = \sigma(\mathbf{W}_l\cdot \mathbf{x}_{l-1}+\mathbf{b}_l)
\end{equation}
where $\sigma(\cdot)$ is the activation function while $\mathbf{W}_l$ and $\mathbf{b}_l$ are the parameters, weights and the biases, respectively.
According to the specific application, several different types of layers and activation functions can be defined. Fig. \ref{img:fundamentals_NN} shows a general fully-connected architecture.

\begin{figure}
%strip if you want 2 columns
\centering
\includegraphics[scale = 0.33]{images/fundamentals/NN.pdf}
\caption{Architecture of a fully connected neural network with two hidden layers.}
\label{img:fundamentals_NN}
\end{figure}

Defined a metric $\delta$ and a cost function $C$, the easiest and most classical algorithm to find the feasible set of parameters $\mathbf{\theta}$ is the gradient descent method which iteratively updates $\mathbf{\theta}$ as $\mathbf{\theta}_t = \mathbf{\theta}_{t-1}-\eta\nabla C(\mathbf{\theta}_{t-1})$ where $\eta$ is the learning rate. Its popular variants are stochastic gradient descent (SGD) and adaptive learning rates (Adam) \cite{AdamKingma}.
Common choices for the cost function are the mean squared error and categorical cross-entropy, for which $\delta$ in \eqref{eq:general_cost} takes the form $\delta(\mathbf{y},\mathbf{\hat{y}}) = ||\mathbf{y}-\mathbf{\hat{y}}||^2$ and $\delta(\mathbf{y},\mathbf{\hat{y}}) = -\mathbf{y}\cdot \log{\mathbf{\hat{y}}}$, respectively.

\subsection{Convolutional neural networks}
Convolutional neural networks (CNNs) (Fig. \ref{img:fundamentals_CNN}) are able to capture the spatial and temporal dependencies of data, and for this reason they find application in image and document recognition \cite{LeCun1998}, medical image analysis \cite{Li2014}, natural language processing \cite{Collobert2008}, and more in general pattern recognition. CNNs are multi layer perceptrons (MLP)  with a regularization approach since they consist of multiple convolutional layers to ensure the translation invariance characteristics. In particular, given an input data matrix $\mathbf{I}_i$, the feature map $\mathbf{F}_j$ is obtained as
\begin{equation}
\mathbf{F}_j = \sigma\biggl(\sum_{i=1}^{C}{\mathbf{I}_i \ast \mathbf{K}_{i,j}+\mathbf{B}_{j}}\biggr)
\end{equation}
namely, through the superposition of $C$ layers, e.g., $C=3$ for RGB images, each comprising a convolution between the input matrix $\mathbf{I}_{i}$ and a kernel matrix $\mathbf{K}_{i,j}$, plus an additive bias term $\mathbf{B}_j$, and a final application of non-linear activation function $\sigma(\cdot)$, typically, a \textit{sigmoid}, \textit{tanh}, or \textit{ReLU}. Each set of kernel matrices represents a filter that extracts local features. To control the problem of overfitting, the dimension of data and features to be extracted is reduced by pooling layers. Finally, fully-connected layers are used to extract semantic information from features.

\begin{figure}
\centering
\includegraphics[scale = 0.40]{images/fundamentals/CNN.pdf}
\caption{Structure of a convolutional neural network with one convolutional layer.}
\label{img:fundamentals_CNN}
\end{figure}


\subsection{Autoencoders}
An autoencoder (AE) is a particular type of NN consisting of an encoding block which tries to learn a latent representation $\mathbf{z}$, typically in a lower-dimensional space, of the input variable $\mathbf{x}$ and a decoding block which reconstructs $\mathbf{x}$ at the output using the information inside the code $\mathbf{z}$ (Fig. \ref{img:fundamentals_autoencoder-architecture}).

A classical learning formulation for \textit{deterministic} AEs requires to solve the following optimization problem
\begin{equation}
\label{eq:AutoCost}
\theta_{\text{opt}} = \argmin_{\theta} \delta(\mathbf{x},G(F(\mathbf{x};\theta_1);\theta_2)),
\end{equation}
where $\theta = (\theta_1,\theta_2)$ are the parameters of the NN, $\delta$ is a measure of distance, while $F$ and $G$ stand for the encoder and decoder function, respectively.
When $F$ and $G$ are linear functions, \eqref{eq:AutoCost} collapse to the principal component analysis (PCA) algorithm. Given a $N \times D$ matrix $\mathbf{x}$ where each row is a sample $\mathbf{x}_i \in \mathbb{R}^D$, PCA represents the encoder as $F(\mathbf{x};\theta) = \mathbf{W}^T\mathbf{x}$ and the decoder as $G(\mathbf{z};\theta) = \mathbf{W}\mathbf{z}$ where $\mathbf{W}$ is the unknown parameter, a $D\times M$ matrix where $M$ is the dimension of the latent space. If $\delta$ is the classic quadratic loss function, then \eqref{eq:AutoCost} can be rewritten as
\begin{equation}
\label{eq:PCA}
\mathbf{W}_{\text{opt}} = \argmin_{\mathbf{W}} \sum_{i=1}^{N}{||\mathbf{x}_i-\mathbf{WW}^T \mathbf{x}_i||_2^2}.
\end{equation}
Let $\Sigma$ be the sample covariance matrix of $\mathbf{x}$, then $\mathbf{W}$ is given by the $M$ principal eigenvectors of $\Sigma$. The resulting transformation $\mathbf{z}=\mathbf{W}^T\mathbf{x}$ is a matrix whose row elements are mutually uncorrelated. PCA is broadly used as a dimensionality reduction and feature extractor tool.

\begin{figure}[t]
\centering
	\includegraphics[scale=0.3]{images/fundamentals/autoencoder_NN.pdf}
	\caption{Architecture of an AE.}
	\label{img:fundamentals_autoencoder-architecture}
\end{figure}

Correlation is an indicator of linear dependence, but in most of cases we are interested in representations where features have a different form of dependence and the ability to identify and extract non-linear dependencies is the main reason for adopting AEs. 
An interesting type of AE for communications purposes is the denoising autoencoder (DAE) \cite{Vincent2008}. The idea is to train a network in order to minimize the following denoising criterion:
\begin{equation}
\label{DAE}
\mathcal{L}_{\text{DAE}} = \mathbb{E}_{\mathbf{x}\sim \mathcal{D}}[\delta(\mathbf{x},G(F(\mathbf{\tilde{x}})))],
\end{equation}
where $\mathbf{\tilde{x}}$ is a stochastic corruption of $\mathbf{x}$. When we train a DAE using the quadratic loss and a corruption Gaussian noise $\mathbf{\tilde{x}} = \mathbf{x}+\epsilon$ with $\epsilon \sim \mathcal{N}(0,\sigma I)$, the work in \cite{Alain2014} proved that the AE recovers properties of the training density $p_X(\mathbf{x})$.
This is somehow remarkable in a communication framework because it asserts that corrupting the transmitted signal with some form of noise can be beneficial in the reconstruction's phase. 

They also showed that the DAE with a small corruption of variance $\sigma^2$ is similar to a contractive autoencoder (CAE) with penalty coefficient $\lambda=\sigma^2$. The CAE \cite{Rifai2011} is a particular form of regularized AE which is trained in order to minimize the following reconstruction criterion:
\begin{equation}
\label{eq:CAE}
\mathcal{L}_{\text{CAE}} = \mathbb{E}_{\mathbf{x}\sim \mathcal{D}}\biggl[\delta(\mathbf{x},G(F(\mathbf{x})))+\lambda \biggl| \frac{\partial F(\mathbf{x})}{\partial \mathbf{x}} \biggr|_F^2 \biggr],
\end{equation}
where $|\mathbf{A}|_F$ is the Frobenius norm. The idea behind CAE is that the regularization term attempts to make $F(\cdot)$ or $G(F(\cdot))$ as simple as possible, but at the same time the reconstruction error must be small.

Despite the fact that typically AEs find a low-dimensional representation of the input vector $\mathbf{x}$ at some intermediate level, when redundancy is desired or a design property, sparse AEs learn an hidden representation of the data in a way such data $\mathbf{z}$ is a $k$-sparse vector. Sparsity is mathematically described in terms of pseudo-norm $l_0$ which, due to its non-differentiability and non-convexity, results intractable. For this reason sparsity is often relaxed and described in terms of norm $l_1$. In this way, sparse AEs are trained in order to minimize the following reconstruction criterion:
\begin{equation}
\label{eq:SAE}
\mathcal{L}_{\text{SAE}} = \mathbb{E}_{\mathbf{x}\sim \mathcal{D}}[\delta(\mathbf{x},G(F(\mathbf{x})))+\lambda | F(\mathbf{x})|_1].
\end{equation}
AEs can be used inside a \textit{probabilistic} framework that aims at modeling the underlying distributions. In this context, variational autoencoders (VAEs) have been introduced in \cite{Kingma2013} as generative probabilistic models based on variational inference.

\subsection{Generative models}
\sectionmark{Generative models}
\label{sec:generative_networks}
Variational autoencoders \cite{Kingma2013} are a particular class of generative models based on variational inference. Let $\mathbf{z}$ be the latent variable of the observed value $\mathbf{x}$ for a parameter $\theta$, then $p_{\theta}(\mathbf{z|x})$ represents the intractable true posterior which can be approximated by a tractable one, $q_{\phi}(\mathbf{z|x})$, for a parameter $\phi$. A probabilistic encoder produces $q_{\phi}(\mathbf{z|x})$ while a probabilistic decoder produces $p_{\theta}(\mathbf{x|z})$. The idea is to maximize a variational lower bound $\mathcal{L}$, often referred to as evidence lower bound (ELBO), on the marginal log-likelihood (evidence)
\begin{equation}
\log p_{\theta}(\mathbf{x}_i) = D_{\text{KL}}(q_{\phi}(\mathbf{z|x}_i)||p_{\theta}(\mathbf{z|x}_i))+\mathcal{L}(\theta,\phi;\mathbf{x}_i)
\label{eq:VALower1}
\end{equation}
where
\begin{equation}
\mathcal{L}(\theta,\phi;\mathbf{x}_i) =  -D_{\text{KL}}(q_{\phi}(\mathbf{z|x}_i)||p_{\theta}(\mathbf{z}))  +\mathbb{E}_{q_{\phi}(\mathbf{z|x}_i)}\biggl[\log p_{\theta}(\mathbf{x}_i|\mathbf{z})\biggr].
\label{eq:VALower2}
\end{equation}
Rather than outputting the code $\mathbf{z}$, the encoder outputs parameters describing a distribution for each dimension of the latent space. In the case where the prior is assumed to be Gaussian, $\mathbf{z}$ will consist of mean and variance. Tuning in the latent space and processing the new latent samples through the decoder is a way to generate new data. 
An hierarchical variational autoencoder (HVAE) is an extended version of a VAE, encompassing multiple hierarchies of latent variables. Therein, latent variables are considered to be generated from higher-level, more abstract latent variables themselves, forming a hierarchical structure. In a Markovian HVAE (MHVAE) with $T$ hierarchical latents, the generative process follows a Markov chain, thus it can be interpreted as stacking VAEs on top of each other (see Fig. \ref{img:fundamentals_MHVAE}). The joint and posterior distributions rewrite as
\begin{align}
    \label{eq:MHVAE}
    p_{\theta}(\mathbf{x},\mathbf{z}_{1:T}) & = p_{Z_T}(\mathbf{z}_T)p_{\theta}(\mathbf{x|z}_{1})\prod_{t=2}^{T}{p_{\theta}(\mathbf{z}_{t-1}|\mathbf{z}_{t})} \\ 
    q_{\phi}(\mathbf{z}_{1:T}|\mathbf{x}) & = q_{\phi}(\mathbf{z}_1|\mathbf{x})\prod_{t=2}^{T}{q_{\phi}(\mathbf{z}_{t}|\mathbf{z}_{t-1})}.
\end{align}

\begin{figure}
\centering
	\includegraphics[scale=0.4]{images/fundamentals/MHVAE.pdf}
	\caption{Graph of a Markovian hierarchical variational autoencoder with $T$ hierarchical latents. Each latent $\mathbf{z}_t$ is generated only from the previous latent $\mathbf{z}_{t+1}$.}
	\label{img:fundamentals_MHVAE}
\end{figure}

Probabilistic diffusion models \cite{Ho2020} have been gaining increasing attention and importance in the field of generative modeling. They can be thought as a particular case of a MHVAE when:
\begin{itemize}
    \item the dimension of the latent $\mathbf{z}_t$, $\forall t=1,\dots,T$, equals the dimension of the data $\mathbf{x}$;
    \item the encoder transition is a simple Gaussian model $q(\mathbf{z}_{t}|\mathbf{z}_{t-1}) = \mathcal{N}(\sqrt{\alpha_t}\mathbf{z}_{t-1}, (1-\alpha_t)\mathbf{I})$, where $\alpha_t$ is a learnable coefficient; 
    \item the distribution of the latent $p_{Z_T}(\mathbf{z}_{T})$ at the timestep $T$ is a multivariate normal distribution. 
\end{itemize}
Under such restrictions, it is possible to show that the task of diffusion models consists of learning how to predict the original ground truth sample from an arbitrarily noisy version of it \cite{CalvinLuo2022}.

Flow-based generative models map the data via a non-linear deterministic transformation into a latent space of independent variables where the probability density results tractable \cite{Dinh2014}. The framework lies behind the change of variable rule
\begin{equation}
p_{X}(\mathbf{x}) = p_{Z}(g^{-1}(\mathbf{x}))\cdot \biggl| \det \frac{\partial g^{-1}(\mathbf{x})}{\partial \mathbf{x}} \biggr|
\label{NICE}
\end{equation}
when both the determinant of the Jacobian and $g^{-1}$ are easy to compute; in that case it is straightforward to directly sample from $p_{X}(\mathbf{x})$ since $\mathbf{x} = g(\mathbf{z})$.
NICE, Real NVP and GLOW \cite{Kingma2018} belong to flow-based generative models which are able to provide a good latent-variable inference and excellent log-likelihood evaluation.

The Gaussian Process Latent Variable Model (GP-LVM) \cite{GPLVM2005} aims at inferring both the latent code $\mathbf{z}$ and the mapping function $\mathbf{f}$ that lead to the dataset $\mathbf{x}$. The prior distribution over $\mathbf{z}$ is set as Gaussian, while $\mathbf{f}(\cdot)$ is described as a Gaussian Process (GP) $\mathbf{f} \sim \mathcal{GP}(\mathbf{0},\mathbf{K})$ where $K(\cdot,\cdot)$ is the covariance function, commonly referred to as squared exponential kernel.
This approach ensures a smooth mapping from the latent to the sample space while providing a closed form expression to approximate the true posterior distribution $p_{Z|X}(\mathbf{z}|\mathbf{x})$ and to find a variational lower bound for a robust training procedure \cite{GPLVM2010}.

The methods presented so far have been mostly and successfully applied to images and they all share the ability to generate new samples in parallel. 
The synthesis of fully visible belief networks (FVBNs) \cite{Frey1995,Frey1998} and autoregressive models \cite{Gregor2014} is difficult to parallelize, therefore, due to their sequential nature, they are relatively slow. Indeed, their core idea is to factorize the joint probability distribution of $D$ dimensional inputs $\mathbf{x}$ into products of one-dimensional conditional distributions:
\begin{equation}
 p_{\text{model}}(\mathbf{x}) = \prod_{j=1}^{D}{p_{\text{model}}(x_{j}|x_{1},\dots, x_{j-1}}).
\label{eq:ChainRule}
\end{equation}
Generation is done by generating one dimension at a time leading to good quality of the samples (like WaveNet for human speech \cite{Oord2016}) since they directly optimize the likelihood. 
However, the recent success of the \textit{transformer} architecture \cite{Vaswani2017} shows that parallelization is possible, and together with a self-attention mechanism to effectively model long-range dependencies, transformers have demonstrated remarkable performance in generating human-like text. The pre-training and fine-tuning paradigm used in transformer-based generative models, often referred to as generative pre-trained transformer (GPT), enables them to be highly adaptable to a variety of tasks. 

Boltzmann machines \cite{Hinton06afast} and generative stochastic network \cite{AlainB2015} rely on estimating the transition operator of a Markov Chain $p(x_{j}|x_{j-1})$ but are now less used since Markov chains fail to scale in high dimensional spaces.

One of the most successful data generation techniques is represented by generative adversarial networks (GANs) \cite{Goodfellow2014}.
Considering the importance of GANs to this thesis, we have dedicated a separate section to delve into their functioning and principles.

\section{Generative adversarial networks}
\label{sec:gans}
The main idea behind GANs is to train a pair of networks in competition with each other: a generator network $G$ that captures the observed data distribution $p_{X}(\mathbf{x})$ and a discriminator network $D$ that distinguishes if a sample is an original coming from real data rather than a fake coming from data generated by the generator $G$ (see Fig. \ref{img:fundamentals_GANs}). 
The training procedure for $G$ is to maximize the probability of $D$ making a mistake. GANs can be thought as a minimax two-player game which will end when a Nash equilibrium point is reached. 

In detail, the generator takes as input a noise vector $\mathbf{z}$ with distribution $p_{\text{noise}}(\mathbf{z})$ and maps it to the data space via a non-linear transformation $\hat{\mathbf{x}} = G(\mathbf{z})$. 
The vanilla value function $V(G,D)$ used to train GANs reads as follows
\begin{equation}
V(G,D) = \mathbb{E}_{\mathbf{x} \sim p_{X}(\mathbf{x})}[\log D(\mathbf{x})] + \mathbb{E}_{\mathbf{z} \sim p_{\text{noise}}(\mathbf{z})}[\log(1-D(G(\mathbf{z})))],
\label{eq:fundamentals_GANs}
\end{equation}
where the objective for the generator is to minimize $V$ while the discriminator maximizes $V$.
It can be proved that such optimization strategy, i.e.,
\begin{equation}
\label{eq:min_max_GAN}
    G^* = \argmin_G \max_D V(G,D),
\end{equation}
enables the generator to implicitly learn the true data distribution $p_{X}(\mathbf{x})$. Indeed, if $p_{\hat{X}}(\mathbf{x})$ is the distribution of the output $\hat{\mathbf{x}}$, it is true that $p_{\hat{X}} = p_{X}$ at the equilibrium.

\begin{figure}
\centering
\includegraphics[scale = 0.4]{images/fundamentals/GAN.pdf}
\caption{GAN framework in which generator and discriminator are learned during the training process.}
\label{img:fundamentals_GANs}
\end{figure}

Several architectures such as the deep convolutional generative adversarial network (DCGAN) \cite{Radford2016} and the StyleGAN \cite{Karras2019} have been proposed in the literature to stabilize the learning process as it is well-known that GANs suffer from problems such as mode collapse and training instability. The latter architecture is amongst the most recent evolution of GANs as it produces state-of-the-art results in synthesizing high-resolution images. The authors achieve improved performance by modifying the generator's architecture (a mapping network for the latent representation and a synthesis one with different resolution levels) for a better understanding of the output. 
However, it is also true that minor efforts have been carried out in generating structured non-images data with GANs.  In Sec.~\ref{sec:medium}, we attempted to study the GAN-based synthesis of time-domain signals.

An alternative method for enhancing GANs involves gaining a deeper comprehension of their training process. Indeed, GANs represent a unique paradigm in ML, where the traditional notion of a cost function is replaced by a value function and a game-theoretic framework. This distinctive characteristic sets GANs apart from many other optimization problems in the field. An extremely useful interpretation of GANs is offered in \cite{Nowozin2016}, where it was shown that the discriminator in GANs acts as an auxiliary NN responsible for the estimation of the variational $f$-divergence, essentially the $\max$ operator in \eqref{eq:min_max_GAN}. The generator, instead, is a neural sampler trained to minimize such divergence, thus, any $f$-divergence can be used for training generative neural samplers, allowing for flexible divergence choices to better match the desired distribution. 

The upcoming chapters will revisit and build upon the latest concept along with the explanations provided thus far.
\section{Adaptive Risk-Aware Path Planning}
\label{sec:risk_aware_path_planning}

We begin this section by defining the multi-objective path planning problem. We present the decision vector, the continuous path representation method, and the objectives in Section \ref{sec:risk_aware_path_planning_problem_definition}. Section \ref{sec:risk_aware_path_planning_solver} shows our adaptive risk-aware multi-objective path planning method, including our multi-objective solver and how we initialize it, the voting algorithm, the real-time adjustment of its coefficients, and the hyperparameters of the entire path planning method.

\chapter{Problem Definition}{\label{ch:problem_definition}}


% \epigraph{\textit{In theory there is no difference between theory and practice—in practice there is.}}{Yogi Berra}
This chapter defines the problem of score following. We begin by briefly discussing why score following is a challenging problem, especially when contrasted with score alignment. Next, we examine the techniques used in manual score following (done by human musicians) and assess which are viable in automatic approaches to score following. Finally, we outline a general three-step approach to score following, which is used in the next chapter to discuss the existing literature, as well as to design our implementation.

\section{Defining Score Following}
We will adopt the general definition proposed by Arshia Cont, a prominent figure affiliated with IRCAM,\footnote{\href{https://www.ircam.fr/}{https://www.ircam.fr/}} who offers a concise definition of score following \cite{cont_arshia}:

\begin{definition}
   Score following serves as a real-time mapping interface from Audio abstractions towards Music symbols and from performer(s) live performance to the score in question.
\end{definition}


\subsection{Difficulties for Score Following}
% TODO Change technical abstract
% TODO new title?

Score following is a non-trivial task because every live performance of the same piece, even by the same performer, may vary significantly. As detailed in \hyperref[subsection:perception_of_music]{subsection \ref*{subsection:perception_of_music}}, the score at best partially determines what occurs in a performance of a work, not to mention performer error and acoustic phenomena external to the performance. The space of all possible events that can be considered performances of, say, Beethoven's \textit{Moonlight Sonata} is incredibly diverse, yet a robust score follower must be able abstract from irrelevant differences and focus on the common features derived from the score.

%At a first glance, the problem of score following seems almost contrived. Since we are presumably equipped with a fully specified score, one might believe we could use a linear mapping between performance times and score locations. However, as detailed in \hyperref[subsection:perception_of_music]{subsection \ref*{subsection:perception_of_music}}, there are complex relationships between the underlying physical properties and score, how this influences the performer, their interpretation and, ultimately, music perception. Effectively, a performance of a piece is much more than what is presented on paper, which is why score following is not a trivial task. 
% \hyperref[section:challenges]{section \ref*{section:challenges}} outlines specific challenges.




\subsection{Score Following Vs Score Alignment}{\label{subsection:score_following_v_alignment}}
Score following is not to be confused with score alignment. Score alignment assumes access to the entire performance recording, relaxing the real-time constraint of score following. Therefore, score alignment poses a lesser challenge, since computation time is not critical, and inference can be performed over all the available data. By contrast, for score following, efficiency is highly important, and estimations of score position must be constantly updated upon receiving new data.\\

% TODO: citation for many score alignment solutions
The greater flexibility of score alignment means it has been heavily researched, yielding many robust solutions. Although score following has also long been researched, and several commercial products have been developed (see \hyperref[ch:score_following_literature_review]{chapter \ref*{ch:score_following_literature_review}}), none have proved reliable enough to receive broad uptake among musicians.

\section{Manual and Automatic Score Following}{\label{section:score_following_approaches}
\subsection{Manual Score Following}{\label{subsection:manual_score_following}}
During manual score following, human musicians can effectively utilise all four perceptual features of music: not just pitch and duration, but also loudness and timbre. This is because, in addition to the score, human musicians also have knowledge of the conventions of music performance, subjective memories of loudness and timbre, emotional responses corresponding to performance directions, and in some cases even visual cues. Despite—or because of—this wealth of information, musicians do not consider all these features simultaneously. Rather, they focus their attention on prominent score features that they intuit will help locate themselves within the piece. For instance, rather than trying to track every pitch present, a musician may focus on overall melodic contour (i.e., rising and falling patterns in note pitch), certain noticeable lines (e.g. melody or bass lines), or prominent \gls{harmonic progression}s. Similarly, musicians are likely to use an innate sense of rhythm to count beats of \gls{bars}s to help identify \gls{phrasing}, abstracting from the rhythmic detail of single notes. All this requires knowledge and understanding of music and music theory.

% During score following, a human musician will listen for pitch and \gls{rhythm}, which most clearly show the influence of the score (see \hyperref[subsection:score_influences]{subsection \ref*{subsection:score_influences}}), as well as notable performance directions. However, musicians do not focus on all these features simultaneously. Rather, they focus their attention on prominent features that they intuit will help locate themselves within the piece. For instance, rather than trying to track every pitch present, a musician may focus on overall melodic contour (i.e., rising and falling patterns in note pitch), certain noticeable lines (e.g. melody or bass lines), or prominent \gls{harmonic progression}s. Similarly, musicians are likely to use an innate sense of rhythm to count beats of \gls{bars}s to help identify \gls{phrasing}, abstracting from the rhythmic detail of single notes. Identifying prominent features requires musical understanding and/or music theory knowledge.

\subsection{Automatic Score Following}
Automatic score followers face three distinct challenges compared to manual score followers. First, we have established that automatic score followers are largely limited to using only pitch and duration (see \hyperref[subsection:score_influences]{subsection \ref*{subsection:score_influences}}), since the perceptual features of loudness and \gls{timbre} do not have quantitative representations in the score and are therefore more difficult for a computer to use. Second, while skilled manual score followers use musical understanding to pay attention to only contextually relevant features, this high-level understanding is impossible to imperatively encode.\footnote{Currently, only deep learning techniques offer methods for learning these intuitions. E.g., see \cite{10.3389/fcomp.2021.718340}.} Finally, while human score followers directly perceive pitch and rhythm, computers do not, and must be programmed to infer these features from raw audio. \\

To counter some of the variation caused by irrelevant underlying physical properties (e.g. \gls{timbre}), as well as the possibility of background noise, we design our model for solo piano and assume the audio data contains no significant sounds other than the piano. 

\subsection{General Approach}{\label{subsection:general_approach}}
 Given the lesser resources available to automatic score followers, score followers typically need to follow a three-step approach: \textbf{feature extraction, similarity calculation}, and \textbf{alignment} \cite{chen_2018_an}. Feature extraction involves inferring pitch and rhythmic components from audio. This step mimics basic human auditory perception. Next, similarity calculation looks for correspondences between these features and particular locations in the score. This step requires the ability to read the score's notes and match them to pitches and rhythms. Finally, alignment makes a single best estimate of score position from a sequence of these correspondences. This requires a functional understanding of how single locations in a score `fit together' in context. Different techniques used at each step depends on various implementations, as presented in the next chapter. 


\section{The general case: Proof of \texorpdfstring{\Cref{thm:main-decomp}}{Theorem 1.6}}\label{sec:algo}

First, we show that data structure of \Cref{l:max_min_query} can be used to compute distances witnessed by shortest paths that pass through a constant-size separator.

\begin{lemma}\label{l:single_adhesion}
Fix a constant $k \in \mathbb{N}$. There exists an algorithm which as the input receives an edge-weighted graph $G$ on $n$ vertices and $m$ edges together with a partition of its vertices into three sets $A, B, C$ such that $|B| \leq k$ and there are no edges between $A$ and $C$, and as the output computes $\max_{c \in C} \dist(a, c)$ for every $a \in A$. The running time is $\Oh(m \log n + n \log^{k - 1} n)$.
\end{lemma}

\begin{proof}
Let $B = \{b_1, \ldots, b_k\}$. For any $a \in A, c \in C$, we have $\dist(a, c) = \min_{i \in [k]} \dist(a, b_i) + \dist(c, b_i)$. First, we run Dijkstra's algorithm from every vertex in $B$ to find $\dist(v, b_i)$ for every $v \in V(G)$ and $i \in [k]$. Next, we use \Cref{l:max_min_query} to construct a data structure $\mathbb{D}$ for the point set $\{(\dist(c, b_1), \dots, \dist(c, b_k))\colon c\in C\}\subseteq \mathbb{R}^k$. Now, the value $\max_{c \in C} \dist(a, c)$ for any given $a$ is equal to the answer of $\mathbb{D}$ to the query with argument $(\dist(a, b_1), \dots, \dist(a, b_k))$.
\end{proof}

After computing the distances over a constant-size separator, we will use the following observation to simplify one of the sides of the separation.

\begin{lemma}\label{l:inserting_paths}
Let $G$ be a edge-weighted connected graph and let $A, B, C$ be a partition of its vertices such that there are no edges between $A$ and $C$. For every pair of vertices $u, v \in B$, let $P_{u, v}$ be any shortest path from $u$ to $v$ with all internal vertices in $C$ (assuming such a path exists).

Let $G'$ denote a graph obtained from $G[A \cup B]$ by adding an edge from $u$ to $v$ of weight equal to the length of $P_{u, v}$, for all $u, v \in B$ for which $P_{u, v}$ exists. Then,  $$\dist_G(s, t) = \dist_{G'}(s, t)\qquad\textrm{for all }s,t\in A\cup B.$$
\end{lemma}
\begin{proof}
Let $G''$ be the graph obtained by adding new edges of $G'$ to $G$.
Fix any $s, t \in A \cup B$ and let $P$ denote the shortest path from $s$ to $t$ in $G''$ which minimizes the number of vertices from $C$ visited. Naturally, the weight of $P$ is equal $\dist_G(s, t)$. Assume that such path visits at least one vertex of $C$. Then, the path $P$ is of the form $s \xrightarrow{P_1} x \xrightarrow{P_2} y \xrightarrow{P_3} t$, where $x, y \in B$ and all the internal vertices of $P_2$ are in $C$. By the construction of $G'$, $P_2$ can be replaced with a direct edge from $x$ to $y$ of the same weight. We obtain a same weight path with a smaller number of vertices of $C$ visited, which is a contradiction. Therefore, $P$ is entirely contained in $A \cup B$, hence it exists in $G'$. This shows that $\dist_G(s, t) = \dist_{G'}(s, t)$.
\end{proof}


The next lemma encapsulates the main algorithmic content of the proof of \Cref{thm:main-decomp}. The algorithm will split the tree decomposition provided on input into smaller parts for which the eccentricities are easier to calculate. We use the following lemma to handle a single such part.
\begin{lemma}\label{l:star}
Fix constants $k, g \in \mathbb{N}, 0 < \delta < \frac{1}{54}$. Assume we are given $n \in \mathbb{N}$, an edge-weighted graph $G$ on at most $n$ vertices with a weight function $w \colon E(G) \to \mathbb{N}$, a vertex subset $A$ and a collection of non-empty vertex subsets $V_0, V_1, \dots, V_\ell$ satisfying the following conditions:
\begin{itemize}[nosep]
	\item The sum of weights of all the edges in $G$ is bounded by $\Oh(n)$.
	\item $V(G) \setminus A = V_0 \cup V_1 \cup \dots \cup V_\ell$.
	\item $|A| \leq k$.
	\item For every $i \in [\ell]$, $G[V_i \setminus V_0]$ is connected, $N_G(V_i \setminus V_0) = V_i \cap V_0$, $|V_i| = \Oh(n^\delta)$, and $|V_0 \cap V_i| \leq 4$.
	\item For all $i, j \in [\ell], i \neq j$, $V_i \setminus V_0$ and $V_j \setminus V_0$ are disjoint and non-adjacent in $G$.
	\item Every edge $uv \in E(G)$ with $u, v \not\in A$ is contained in $G[V_i]$ for some $i\in \{0,1,\ldots,\ell\}$.
	\item The graph obtained by taking $G[V_0]$ and adding a clique on $V_0 \cap V_i$ for every $i \in [\ell]$ has Euler genus bounded by $g$.
\end{itemize}
Then, we can compute the eccentricity of every vertex of $G$ in time $\Oh \left( n^{1 + \frac{150 + 54 \delta}{151}} \log^k n \right)$.
\end{lemma}

\begin{proof}
Fix $\delta' = \frac{1 + 97 \delta}{151}$; we have $\delta' - \delta = \frac{1 - 54\delta}{151} > 0$.
Let $E_i$ denote the set of edges with one endpoint in $V_i$ and the other endpoint in $V_i \setminus V_0$. For $i \in [\ell]$, we shall say that $V_i$ is {\em{heavy}} if the sum of weights of $E_i$ is larger than $n^{\delta'}$. Since the sets $E_i$ are pairwise disjoint and the total sum of weights of all the edges is bounded by $\Oh(n)$, the number of heavy subsets is bounded by $\Oh(n^{1 - \delta'})$. Without loss of generality, we may assume that $V_{\ell' + 1}, \dots, V_\ell$ are heavy and $V_1, \dots, V_{\ell'}$ are not, for some $\ell'\in \{0,\ldots,\ell\}$.


For any source vertex $s$, we can calculate distances from $s$ to every vertex of $G$  using breadth first search in time $\Oh(\sum_{e \in E(G)} w(e)) = \Oh(n)$.
In particular, for every $\ell' < i \leq \ell$, we can compute the distances from every vertex of $V_i$ to every vertex of $G$ in total time $\Oh(n^{2 - \delta' + \delta})$, because $$|V_{\ell'+1}\cup \ldots\cup V_{\ell}|\leq n^{1-\delta'}\cdot \Oh(n^\delta)=\Oh(n^{1-\delta'+
\delta}).$$
Additionally, we calculate distances $\dist_G(a, v)$ for every $a \in A, v \in V(G)$ in time $O(n)$.

For every $i \in [\ell]$ and $u,v \in V_0 \cap V_i$, there exists a shortest path $P_{i,u,v}$ from $u$ to $v$ with all internal vertices belonging to $V_i - V_0$ due to the assumption that $G[V_i - V_0]$ is connected and $N_G(V_i - V_0) = V_i \cap V_0$. Therefore, the distance from $u$ to $v$ is bounded by the sum of weights of edges in $E_i$. In particular, for $i \in [\ell']$, $\dist_G(u, v) \leq n^{\delta'}$.

We define $\widetilde{G}$ to be the graph obtained by taking $G[A \cup V_0 \cup \dots \cup V_{\ell'}]$ and applying the following operation for every $i \in \{\ell' + 1, \dots, \ell\}$:
for each pair of vertices $u, v \in A \cup (V_0 \cap V_i)$, add an edge in $\widetilde{G}$ between $u$ and $v$ with weight equal to the total weight of $P_{i,u,v}$. For a fixed $i, u$, we can find $P_{i, u, v}$ for all $v$ using breadth first search in time $\Oh(n)$. Taking a sum over all $i, u$, we get that $\tilde{G}$ can be computed in total time $\Oh(n^{2 - \delta'})$.


\begin{claim}\label{cl:wG}
The sum of the edge weights in $\widetilde{G}$ is $\Oh(n)$. Moreover, for all $u, v \in V(\widetilde{G})$, we have $\dist_{\widetilde{G}}(u, v) = \dist_{G}(u, v)$.
\end{claim}

\begin{proof}
Consider $i \in \{\ell' + 1, \dots, \ell\}$ and any $u, v \in A \cup (V_0 \cap V_i)$ for which we added an edge. Its weight is bounded by the sum of weights of edges in $E_i$. Therefore, the total weight of all edges added is at most
$$
\sum_{i \in \{\ell' + 1, \dots, \ell\}} \left( |A \cup (V_0 \cap V_i)|^2 \sum_{e \in E_i} w(e) \right) \leq (4 + k)^2 \sum_{e \in E(G)} w(e) = \Oh(n).
$$
This proves the first part of the claim.

For the second part of the claim, consider any $i \in \{\ell' + 1, \dots, \ell \}$ and observe that by our assumptions, $A \cup (V_0 \cap V_i)$ separates $(V_0 \cup \dots \cup V_{\ell'} \cup V_{i + 1} \cup \dots \cup V_\ell) \setminus V_i$ from $V_i \setminus V_0$. Hence it suffices to repeatedly apply \Cref{l:inserting_paths}.
\end{proof}

For every $u \in V(\widetilde{G})$, we have $\ecc_G(u) = \max(\ecc_{\widetilde{G}}(v), \max_{v \in V(G) \setminus V(\widetilde{G})} \dist_G(u, v))$. Note, that we already know all the distances $\dist_G(u, v)$ for $v \in V(G) \setminus V(\widetilde{G})$. Similarly, we can already compute $\ecc_G(u)$ for every $u \in V(G) \setminus V(\widetilde{G})$. Therefore, it remains to compute $\ecc_{\widetilde{G}}(v)$ for each $v \in V(\widetilde{G})$. Our goal is to show that this can be done efficiently using \Cref{l:main_ecc}.

Now, let $G'$ be the graph obtained from $\tilde{G}$ by replacing every edge $e$ non-indicent to $A$ with $w(e)\geq 2$ with a path of length $w(e)$ consisting of unit-weight edges. This operation again preserves the distances. Since the sum of edge weights in $\tilde{G}$ is of $\Oh(n)$, the total number of vertices in $G'$ is of $\Oh(n)$. For $0 \leq i \leq \ell'$, we write $V'_i$ to denote the set $V_i$ together with all the vertices added as a part of a path between two endpoints in $V_i$.
As $V_i$ is not heavy for each $i\in [\ell']$, we have
$$
|V'_i \setminus V'_0| \leq |V_i| + \sum_{e \in E_i} w(e) = \Oh(n^{\delta'})\qquad \textrm{for all }i\in [\ell'].
$$

Let $G_0$ denote the graph $G'[V'_0]$ and let $G_0^*$ denote the graph $G'- A$ with $V'_i - V'_0$ contracted to a single vertex $v_i^*$, for each $i \in [\ell']$; note that, all edges of $G_0$ and $G_0^*$ have unit weight.

\begin{claim}
	The graph $G_0^*$ is does not contain $K_{t}$ as a minor, where $t = \Oh(\sqrt{g})$.
\end{claim}

\begin{proof}
Let $\bar{G}_0$ denote the graph obtained by taking $G_0$ and adding a clique on $V_0 \cap V_i$ for every $i \in [\ell']$.
By lemma assumptions and the fact that subdividing edges does not increase the Euler genus, $\bar{G}_0$ has Euler genus at most $g$. In particular, $\bar{G}_0$ is $K_{t'}$-minor-free for some $t' = \Oh(\sqrt{g})$, because the Euler genus of $K_{t'}$ is $\Omega({t'}^2)$.

Similarly, let $\bar{G}_0^*$ be the graph obtained by taking $G_0^*$ and adding a clique on each $V_0 \cap V_i$.
Note, that $\bar{G}_0^* - \{v_1^*, \dots, v_{\ell'}^*\}$ is precisely $\bar{G}_0$. Let $t = \max(t', 6)$.
Recall that a minor model of a clique $K_t$ consists of $t$ pairwise vertex-disjoint connected subgraphs, called
branch sets, such that there is at least one edge between each pair of the branch sets.
Consider a minor model $\varphi$ of $K_{t}$ in $\bar{G}^*_0$.
Note that $\varphi$ cannot contain any singleton branch set of the form $\{v^*_i\}$, for the degree of $v^*_i$ in $\bar{G}^*_0$ is at most $4 < t - 1$. Furthermore, since $N_{\bar{G}^*_0}(v^*_i) = V_0 \cap V_i$, any branch set containing $v^*_i$ and at least one other vertex contains some $u \in V_0 \cap V_i$, and $N_{\bar{G}^*_0}(v^*_i)\subseteq N_{\bar{G}^*_0}(u)$, hence removing $v^*_i$ from this branch set preserves the model. Therefore, we can assume without loss of generality that all branch sets of $\varphi$ are disjoint from $\{v^*_1, \dots, v^*_{\ell'}\}$, hence $\varphi$ is a minor model of $K_{t}$ in $\bar{G}_0$. This is a contradiction, as $t \geq t'$ and $\bar{G}_0$ is $K_{t'}$-minor-free. Therefore, $\bar{G}_0^*$ is $K_t$-minor-free, hence $G_0^*$ also.
\end{proof}

Let $\rho' = \frac{2 - 108 \delta}{151} > 0$. The graph $G^*_0$ is a unit-weight graph and is $K_{t}$-minor-free.
Hence, by applying \Cref{t:r_division} to $G^*_0$ (with $\varepsilon = \rho'/2$)
we obtain an $n^{\rho'}$-division $\mathcal{R}_0$ in time $\Oh(n^{1 + \rho'})$.
We extend it to $G' - A$ by mapping every contracted vertex $v^*_i$ to $N_{G' - A}[V'_i - V'_0] = (V'_i - V'_0) \cup (V_0 \cap V_i)$. Formally, we put $V''_i \coloneqq N_{G' - A}[V'_i - V'_0]$ and 
$$
\mathcal{R} \coloneqq \left\{ (R_0 \cap V'_0) \cup \bigcup_{i \colon v^*_i \in R_0} V''_i \colon R_0 \in \mathcal{R}_0 \right\}.
$$

Now, we argue that $\mathcal{R}$ is a reasonable division of $G' - A$. Clearly, all sets $R \in \mathcal{R}$ are connected in $G' - A$. Pick any $R \in \mathcal{R}$ and let $R_0$ be its corresponding set in $\mathcal{R}_0$.
Every vertex $v^*_i$ is mapped to a set of size $\Oh(n^{\delta'})$, therefore
$$|R| \leq |R_0| \cdot \Oh(n^{\delta'}) = \Oh(n^{\rho' + \delta'}).$$

By our construction, for every $i \in [\ell']$, $R$ is either disjoint from $V'_i - V'_0$ or contains whole $N_{G' - A}[V'_i - V'_0]$. This means that no vertex belonging to any $V'_i - V'_0$ can be in $\partial R$, hence $\partial R \subseteq V'_0$.

Pick any $u \in \partial R \cap R_0$. Assume that $u \not\in \partial R_0$. Then every vertex of $N_{G_0^*}(u)$ must be in $R_0$, hence $N_{G - A'}(u) \subseteq R$, which is a contradiction. This means that $\partial R \cap R_0 \subseteq \partial R_0$.

Pick any $u \in \partial R - R_0$. Then, $u \in V_0 \cap V_i$ for some $i \in [\ell']$ such that $v_i^* \in R_0$. Moreover, $v_i^* \in \partial R_0$ and is adjacent to $u$ in $G_0^*$. The number of such $u$ is bounded by $4 |\partial R_0 \cap \{ v_1^*, \dots, v_{\ell'}^* \}|$.

Putting two cases together, we obtain:
$$
\sum_{R \in \mathcal{R}} |\partial R| = \sum_{R \in \mathcal{R}} \left(|\partial R \cap R_0| + |\partial R - R_0|\right) \leq \sum_{R_0 \in \mathcal{R}_0} \left(|\partial R_0| + 4 |\partial R_0 \cap \{ v_1^*, \dots, v_{\ell'}^* \}|\right) = \Oh(n^{1 - \frac{1}{2}\rho'}).
$$

It remains to show the following claim.

\begin{claim}
Pick any $R \in \mathcal{R}, s_R \in R$. The number of different distance profiles on $R$ relative to $s_R$ in $G' - A$ is of $\Oh(n^{48\rho' + 54\delta'})$.
\end{claim}
\begin{proof}
We look at every vertex $v \in V(G') \setminus A$ and consider three cases: $v \in R$, $v \in V'_0$, and $v \in V'_i \setminus (V'_0 \cup R)$ for some $i \in [\ell']$. By our construction, $R \cap V'_0$ is non-empty, hence w.l.o.g. we can assume that $s_R \in V'_0$ as whether two vertices have the same profile on $R$ is independent of the choice of the pivot vertex.

In the first case, there are at most $|R| = \Oh(n^{\rho' + \delta'})$ such vertices, hence they realise at most that many profiles.

In the second case, we want to observe that profile of any vertex $v \in V'_0$ on $R$ depends only on its profile on $R \cap V'_0$ (relative to $s_R$). Pick any $t \in R - V'_0$. Then $t \in V'_i - V'_0$ for some $i \in [\ell']$, $V_i \cap V_0 \subseteq R \cap V'_0$, and every path from $v$ to $t$ intersects $V_i \cap V_0$. In particular, distances from $v$ to vertices of $V_i \cap V_0$ determine its distance to $t$, which proves the observation.

Let $\tilde{G}_0$ denote the graph obtained by taking $G'[V'_0]$ and for every $i \in [\ell'], u, v \in V_0 \cap V_i$ adding a disjoint path from $u$ to $v$ of length $\dist(u, v)$. Let $P_i$ denote the vertex set of paths added between $V_0 \cap V_i$. For every $t \in V'_0$ we have $\dist_{G' - A}(v, t) = \dist_{\tilde{G}_0}(v, t)$, so it suffices to bound the number of profiles on $R \cap V'_0$ in $\tilde{G}_0$. By our assumptions, $\tilde{G}_0$ has Euler genus bounded by $g$ and all $P_i$ are of size $\Oh(n^{\delta'})$.

Let $R_0$ be the set of $\mathcal{R}_0$ corresponding to $R$. Let $\tilde{R}_0$ denote the set $(R \cap V'_0) \cup \bigcup_{i : v^*_i \in R_0} P_i$. Such set is connected in $\tilde{G}_0$. Moreover, similarly to $R$, its size is $\Oh(n^{\rho' + \delta'})$. Applying \Cref{thm:distprofiles}, we get that the number of distance profiles on $\tilde{R}_0$ in $\tilde{G}_0$ is $\Oh(n^{12(\rho' + \delta')})$, which also bounds the number of profiles on $R$ in $G' - A$ realised by $V'_0$.

For the third case, assume $v \in V'_i \setminus (V'_0 \cup R)$ for some $i\in [\ell']$. Every path from $v$ to any vertex of $R$ in $G' - A$ intersects $V_i \cap V_0$. Let $v_1, \dots v_p$ be the vertices of $V_i \cap V_0$, where $p \leq 4$. The profile of $v$ on $R$ is then determined by the following:
\begin{itemize}[nosep]
 \item[(a)] the profile of each $v_j$ on $R$,
 \item[(b)] $\dist_{G' - A}(v, v_j) - \dist_{G' - A}(v, v_1)$ for each $2 \leq j \leq p$, and
 \item[(c)] $\dist_{G' - A}(s_R, v_j) - \dist_{G' - A}(s_R, v_1)$ for each $2 \leq j \leq p$ where $s_R$ is some pivot vertex of $R$.
\end{itemize}
By the previous case, the number of distance profiles of each $v_j$ is $\Oh(n^{12(\rho' + \delta')})$. The distances between $v$ and $v_j$ are bounded by $|V'_i|$, hence each quantity described in (b) can take $\Oh(n^{\delta'})$ different possible values. Similarly, since $v_1$ and $v_j$ are connected via $V'_i$, $|\dist_{G' - A}(s_R, v_j) - \dist_{G' - A}(s_R, v_1)| \leq \Oh(n^{\delta'})$. The number of different possible profiles of such $v$ is therefore bounded by $\Oh(n^{48(\rho' + \delta') + 6\delta'}) = \Oh(n^{48\rho' + 54\delta'})$. This finishes the proof of the claim.
\end{proof}

Now we can apply \Cref{l:main_ecc} to graph $G'$ with apex set $A$, $X = V(\widetilde{G})$, and the following constants: $$\rho = \rho' + \delta',\qquad \gamma = 1 - \frac{1}{2}\rho',\quad \textrm{and}\quad \alpha = 48\rho' + 54 \delta'.$$ This allows us to calculate all $V(\widetilde{G})$-eccentricities in $G'$ in time
$$
\Oh \left( \left(
	n^{ 2 - \frac{1}{2} \rho' } +
	n^{ 1 + 48\rho' + 54 \delta' }
\right) \log^k n \right) =
\Oh \left( n^{1 + \frac{150 + 54 \delta}{151}} \log^k n \right).
$$
Since for each $v\in V(\widetilde{G})$ we have $\ecc_{\widetilde{G}}(v) = \max_{u \in V(\widetilde{G})} \dist_{\widetilde{G}}(v, u) = \max_{u \in V(\widetilde{G})} \dist_{G'}(v, u)$, this means that we have successfully computed all the eccentricities in $\widetilde{G}$; and as we argued, this is enough to compute all the eccentricities in $G$ as well.

Finally, the total running time of the algorithm is
$$
\Oh \left( n^{1 + \frac{150 + 54 \delta}{151}} \log^k n + n^{2 - \delta' + \delta} \right) =
\Oh \left( n^{1 + \frac{150 + 54 \delta}{151}} \log^k n \right).
$$\qedhere
\end{proof}


\begin{lemma}\label{l:star2}
Fix constants $k, g \in \mathbb{N}, 0 < \delta < \frac{1}{54}$. Assume we are given $n \in \mathbb{N}$, an edge-weighted graph $G$ on at most $n$ vertices with a weight function $w \colon E(G) \to \mathbb{N}$, a vertex subset $A$ and a collection of non-empty vertex subsets $V_0, V_1, \dots, V_\ell$ satisfying the same conditions as in \Cref{l:star} with the following differences:
\begin{itemize}
	\item we don't require $G[V_i - V_0]$ to be connected and $V_i - V_0$ to be adjacent to whole $V_i \cap V_0$;
	\item instead of $|V_0 \cap V_i| \leq 4$, we require $|V_0 \cap V_i| \leq k$.
\end{itemize}
Then, we can compute the eccentricity of every vertex of $G$ in time $\Oh \left( n^{1 + \frac{150 + 54 \delta}{151}} \log^{k + 5g} n \right)$.
\end{lemma}

\begin{proof}
We will reduce our input to one which will satisfy the conditions of \Cref{l:star}. We start by addressing the adhesions $V_0 \cap V_i$ containing too many vertices.

Let $G_0$ denote the graph $G[V_0]$ with cliques placed at $V_0 \cap V_i$ for every $i \in [\ell]$.
For every $i \in [\ell]$ we repeat the following procedure: while $|V_0 \cap V_i| > 4$,
remove arbitrary $5$ vertices from $V_0 \cap V_i$. Since $|V_0 \cap V_i| \leq k$ for each $i\in [\ell]$,
this procedure can be implemented in total time $\Oh(n)$. As a result, at the end we have $|V_0 \cap V_i| \leq 4$ for all $i \in [\ell]$. Let $M$ be the set of all the removed vertices. By our assumptions, $G_0$ has Euler genus bounded by $g$, hence it cannot contain $g + 1$ pairwise disjoint copies of $K_5$
(as the Euler genus of a graph is the sum of the Euler genera of its 2-connected components~\cite{StahlB77} and $K_5$ is not planar). Each removed quintiple of vertices induces a $K_5$ in $G_0$, hence we have $|M| \leq 5g$. We set $A' = A \cup M$ and may thus assume that $V_i$ is disjoint from $A'$ for all $0 \leq i \leq \ell$.

Now, fix $i \in [\ell]$. Let $C^i_1, \dots, C^i_{r_i}$ denote the connected components of $V_i - V_0$ in $G - A'$. We define $W^i_j := N_{G - A'}[C^i_j]$ for every $j \in [r_i]$. Clearly, all $W^i_j$ induce a connected subgraph of $G$ and satisfy $N_{G - A'}(W^i_j - V_0) = W^i_j \cap V_0$. We put $V'_0 := V_0$ and enumerate
$$
\{V'_1, V'_2, \dots V'_{\ell'}\} := \{ W^i_j \colon i \in [\ell], j \in [r_i] \}.
$$
It is easy to verify that the sets $A'$ and $V'_0, V'_1, \dots, V'_{\ell'}$ satisfy the conditions of \Cref{l:star}. We apply said lemma to calculate the eccentricity of every vertex of $G$ in the desired time.
\end{proof}



The next statement is a reformulation of \Cref{thm:main-decomp}.

\begin{theorem}
Fix constants $k, g \in \mathbb{N}$. Assume we are given a graph $G$ on $n$ vertices together with its tree decomposition $(T, \beta)$ and a set of private apices $A_t \subseteq \beta(t)$ for each node $t\in V(T)$ such that the following conditions hold:
\begin{itemize}[nosep]
 \item For every node $t \in V(T)$, we have $|A_t| \leq k$.
 \item For every edge $st \in E(T)$,  we have $|\beta(v) \cap \beta(u)|\leq k$.
 \item For every node $t \in V(T)$, graph obtained by taking $G[\beta(t)] - A_t$ and turning  $(\beta(t) \cap \beta(s))\setminus A_t$ into a clique for every edge $st \in E(T)$ has Euler genus bounded by $g$.
\end{itemize}
Then, we can compute the eccentricity of every vertex of $G$ in time $\Oh \left( n^{1 + \frac{355}{356}} \log^{k + 5g} n \right)$.
\end{theorem}

\begin{proof}
We may assume that $|V(T)|\leq n$, for every tree decomposition with no two bags comparable by inclusion has this property; and adjacent comparable bags can be merged by contracting the edge between them.

For a node $t\in V(T)$, by the {\em{weight}} of $t$ we mean the size of the corresponding bag, that is, $|\beta(t)|$. For any subset of nodes $S \subseteq V(T)$, we define $\beta(S) \coloneqq \bigcup_{t \in S} \beta(t)$ By the {\em{weight}} of $S$, we mean the total weight of the elements of $S$, that is, $\sum_{t\in S} |\beta(t)|$. 

\begin{claim}\label{cl:weight-T}
The weight of $V(T)$ is of $\Oh(n)$.
\end{claim}

\begin{proof}
The sets $\beta'(t) := \beta(t) - \bigcup_{s \in N_T(t)} \beta(s)$ are pairwise disjoint. We have
$$
\sum_{t \in V(T)} |\beta(t)| =
\sum_{t \in V(T)} |\beta'(t)| + 2 \cdot \sum_{st \in E(T)} |\beta(s) \cap \beta(t)| \leq
|V(T)| + 2k|E(T)| = \Oh(n).
$$
\end{proof}

Since every bag induces a graph of bounded Euler genus, the number of edges contained in a bag is linear in its size. In particular, this implies that the total number of edges of $G$ is also bounded by $\Oh(n)$.

We set $$\delta \coloneqq \frac{1}{356}\qquad\textrm{and}\qquad \Delta \coloneqq \frac{355}{356}.$$ Root the tree $T$ in an arbitrarily chosen node; this naturally imposes an ancestor-descendant relation in $T$ (for convenience, every node is considered its own ancestor and descendant).

We start by partitioning $T$ into connected subtrees using the following procedure.
We proceed bottom-up over $T$, processing nodes in any order so that a node is processed after all its strict descendants have been processed. Along the way, we mark some nodes and split the edges of $T$ into heavy and light. Let $t \in V(T)$ be the currently processed non-root node of $T$ and let $e \in E(T)$ be the edge connecting $t$ with its parent. If the total weight of all the unmarked nodes that are descendants of $t$ is at least $n^\delta$ (recall that this includes $t$ itself as well), then we declare $e$ heavy and mark all the descendants of $t$ that were unmarked so far. Otherwise, the edge $e$ is declared light and the procedure proceeds to further nodes of $T$.

Observe that
removing all heavy edges splits $T$ into connected subtrees, say $T'_1, \cdots T'_m$. All of the subtrees, except for possibly the subtree containing the root node, are of weight at least $n^\delta$. In particular, the number of subtrees $m$, and therefore the number of heavy edges, is  bounded by $\Oh(n^{1 - \delta})$. Moreover, in every subtree $T'_i$, removing the node closest to the root splits $T'_i$ into smaller components, each of weight less than $n^\delta$.

Fix a heavy edge $e$ and let $T^e_1$ and $T^e_2$ be the two subtrees into which $T$ splits after removing~$e$. Let $X^e_i = \beta(T^e_i)$ for $i \in \{1, 2\}$. Put $A_e = X^e_1 \setminus X^e_2$, $C_e = X^e_2 \setminus X^e_1$, and $B_e = X^e_1 \cap X^e_2$. By the properties of tree decompositions, such choice of $A_e, B_e, C_e$ satisfies the conditions of \Cref{l:single_adhesion}, hence in time $\Oh(n \log^{k - 1} n)$ we can compute $\max_{v \in X^e_2} \dist_G(u,v)$ for every $u \in X^e_1$, and $\max_{u \in X^e_1} \dist_G(u,v)$ for every $v \in X^e_2$. Computing this for every heavy edge $e$ takes total time $\Oh(n^{2 - \delta} \log^{k - 1} n)$.

Fix any subtree $T'=T'_j$. Let $e_1 = t^{e_1}_1t^{e_1}_2, e_2 = t^{e_2}_1 t^{e_2}_2, \dots, e_\ell = t^{e_\ell}_1 t^{e_\ell}_2$ denote the heavy edges incident to $T'$, where $t^{e_i}_1 \in V(T')$ and $V(T') \subseteq V(T_1^{e_i})$ for every $i \in [\ell]$.
For a vertex $v \in \beta(T')$, let
$$d_0(v) = \max_{u \in \beta(T')} \dist_G(v, u)\qquad\textrm{and}\qquad d_i(v) = \max_{u \in X_2^{e_i}}\dist_G(v,u),\quad\textrm{for } i \in [\ell].$$ We have $\ecc(v) = \max \{ d_i(v)\colon i\in \{0,1,\ldots,\ell\}\}$.The values of $d_i(v)$ are already calculated for all $i\in [\ell]$, hence it remains to compute $d_0(v)$.

For every $i \in [\ell]$ and every pair of vertices $u, v \in \beta(t^{e_i}_1) \cap \beta(t^{e_i}_2)$ we find a shortest path between $u$ and $v$ with all internal vertices inside $X^{e_i}_2$ (or determine that it doesn't exist). For a fixed $u, v$ this can be done in time $\Oh(n)$. Since in total we perform this step at most $2k^2$ times per heavy edge, it takes $\Oh(n^{2 - \delta})$ time in total. Let $P_{i, u, v}$ denote such path, assuming it exists.

Let $G'$ denote the graph obtained from $G[\beta(T')]$ by taking every $i, u, v$ for which $P_{i, u, v}$ exists and adding an edge between $u$ and $v$ of weight equal to the total weight of $P_{i, u, v}$.
The weight of every edge inserted in $\beta(t^{e_i}_1) \cap \beta(t^{e_i}_2)$ is bounded by $|X^{e_i}_2|+1$. The total weight of all edges inserted is therefore at most
$$
\sum_{i \in [\ell]} |\beta(t^{e_i}_1) \cap \beta(t^{e_i}_2)|^2 \cdot (|X^{e_i}_2|+1) \leq
k^2 \sum_{i \in [\ell]} (|X^{e_i}_2|+1) = \Oh(n),
$$
where the last equality follows from the fact that all the trees $T^{e_i}_2$ are pairwise disjoint.
By \Cref{l:inserting_paths}, we have $\dist_{G'}(u, v) = \dist_G(u, v)$ for each $u, v \in \beta(T')$. Hence, computing $d_0(v)$ for every $v \in \beta(T')$ is equivalent to computing the eccentricity of every vertex in $G'$.

If the size of $\beta(T')$ is smaller than $n^\Delta$, we compute the eccentricities naively in time $\Oh(|\beta(T')|^2)$, 
noting that $G'$ has $\Oh(|\beta(T')|)$ edges (thanks to Claim~\ref{cl:weight-T} and bounded genus assumption 
of the last bullet of the theorem statement). Otherwise, we argue that we can use the algorithm in \Cref{l:star} as follows.

Let $t$ be the node of $T'$ closest to the root. Let $s_1, \dots, s_p$ be the children of $t$ in $T$ and let $T''_i$ denote the connected component of $T' - \{t\}$ containing $s_i$. Set $V_0 = \beta(t)$ and $V_i = \beta(T''_i)$ for $i \in [p]$.

It is now easy to verify that $G'$ and sets $A, \{V_i\colon 0\leq i\leq p\}$ selected this way satisfy the assumptions of \Cref{l:star2}. This allows us to use it to compute the eccentricities in $G'$ in time
$$
\Oh \left( n^{1 + \frac{150 + 54\delta}{151}} \log^{k + 5g} n \right) =
\Oh \left( n^{1 + \frac{354}{356}} \log^{k + 5g} n \right).
$$
As we argued, from these eccentricities, we may easily compute all the eccentricities in $G$.

Now, let us analyse the total running time of the whole algorithm. We invoke \Cref{l:star} $\Oh(n^{1 - \Delta})$ times, since we apply it only to subtrees $T'_i$ of size at least $n^\Delta$. The total running time of those applications is hence
$$
\Oh \left( n^{2 + \frac{354}{356} - \Delta} \log^{k + 5g} n \right) =
\Oh \left( n^{1 + \frac{355}{356}} \log^{k + 5g} n \right).
$$
We compute the eccentricities naively for subtrees smaller than $n^\Delta$, hence the total running time of this computation is
$$
\sum_{i \in [m] \colon |\beta(T'_i)| \leq n^\Delta} |\beta(T'_i)|^2 \leq
n^\Delta \cdot \sum_{i \in m} |\beta(T'_i)| = \Oh(n^{1 + \Delta})=\Oh\left(n^{1+\frac{355}{356}}\right).
$$
The rest of computation can be done in $\Oh(n^{2 - \delta} \log^k n)$. Therefore, the whole algorithm runs in time $\Oh \left( n^{1 + \frac{355}{356}} \log^{k + 5g} n \right)$.
\end{proof}




\section{Results}
\label{sec:results}
Following \nksr, we evaluate our method using metrics including the standard Chamfer-$L_1$ Distance~(CD-$L_1 \times 10^{-2}$, $\downarrow$) and F-score~($\uparrow$) with a threshold~($\delta{=}0.010$). 
We also report additional metrics proposed in \nksr~including Chamfer-$L_1$ Distance by Completeness (Comp.~$\times 10^{-2}$, $\downarrow$) and Accuracy (Acc.~$\times 10^{-2}$, $\downarrow$) in the \texttt{Supplementary Material}. 
We evaluate our method on multiple datasets, under two settings including in-domain evaluation for accuracy estimation -- training set and test set are from same dataset, and cross-domain evaluation for generalization ability estimation where training set and test set are from different datasets. 
Additionally, for cross-domain evaluation we use the following datasets prepared by the leading voxel-based baseline, \nksr, and one additional dataset from RangeUDF~\cite{wang2022rangeudf}:

\begin{itemize}
    \item \synthetic{}  is a synthetic dataset created from ShapeNet objects~\cite{chang2015shapenet}. Each scene contains 2-3 objects. 
    Following prior works~\cite{wang2022rangeudf,chibane2020ndf}, we re-scale the synthetic rooms to roughly match real-world scale.
    There are 3750 scenes as training set and \ws{995 scenes} as the test set. 
    \item \scannet{} is a real-world indoor scene dataset. We use the setting from previous work~\cite{wang2022rangeudf, tang2021SACon, peng2020convoccnet, boulch2022poco} where we train on 1201 rooms and test on 312 rooms. 
    \item \carla is a large-scale outdoor driving scene prepared by NKSR~\cite{huang2023neural} using the CARLA simulator~\cite{dosovitskiy2017carla}. 
    \ws{Following NSKR~\cite{huang2023neural}, we test on two subsets including the 'Original' subset (10 random drives simulated on 3 towns) and the 'Novel' subset (3 drives from an additional town only for testing).}
    To avoid exploding GPU memory during training, we follow NKSR~\cite{huang2023neural} to divide a large scene into patches. The resultant training set has {3757} patches. 
    \item \scenenn{}  is a real-world indoor dataset prepared by RangeUDF~\cite{wang2022rangeudf} which we used for cross-domain evaluation. We only use its test set which consists of 20 scenes.
\end{itemize}



\begin{table*}
\centering
\resizebox{\linewidth}{!}{
\setlength{\tabcolsep}{3pt}
\begin{tabular}{LccccccccccccC}
\toprule
Methods & & \multicolumn{3}{c}{\ws{{\bf \synthetic}}}  &  \multicolumn{3}{c}{{\bf \scannet}} & \multicolumn{3}{c}{\ws{{\bf \carla(Original)}}} & \multicolumn{3}{c}{\ws{{\bf \carla(Novel)}}} \\ 
 \cmidrule(lr){3-5} \cmidrule(lr){6-8} \cmidrule(lr){9-11} \cmidrule(lr){12-14} 
&Primitive& CD ($10^{-2}$) $\downarrow$ & F-Score  $\uparrow$ & Latency (s) $\downarrow$  & CD ($10^{-2}$) $\downarrow$ & F-Score  $\uparrow$ & Latency (s) $\downarrow$  & CD (cm) $\downarrow$ & F-Score  $\uparrow$ & Latency (s) $\downarrow$ & CD (cm) $\downarrow$ & F-Score  $\uparrow$ & Latency (s) $\downarrow$ \\        
\midrule
SA-CONet~\cite{tang2021SACon} & Voxels & {0.496} & {93.60} & - & - & - & - & - & - & - & - & - & -\\
ConvOcc~\cite{peng2020convoccnet} & Voxels & {0.420} & {96.40} & - & - & - & - & - & - & - & - & - & -\\
NDF~\cite{chibane2020ndf} & Voxels & {0.408} & {95.20} & - & 0.385  & 96.40  & -  & - & - & - & - & - & -\\
RangeUDF~\cite{wang2022rangeudf} & Voxels & {0.348} & {97.80} & {-} & 0.286 & 98.80 & - & - & - & - & - & - & -\\
\ws{TSDF-Fusion~\cite{zeng20163dmatch}} & -  & - & - & - & - & - & - & 8.1 & 80.2 & - & 7.6 & 80.7 & - \\
\ws{POCO~\cite{boulch2022poco}} & - & - & - & - & - & - & - & 7.0 & 90.1 & - & 12.0 & 92.4 & - \\
\ws{SPSR~\cite{kazhdan2013screened}} & - & - & - & - & - & - & - & 13.3 & 86.5 & - & 11.3 & 88.3 & - \\
\nksr & Voxels &  \underline{0.346} &  \underline{97.41} & \underline{0.40} & \underline{0.246} & \underline{99.51} & \underline{1.54} &  \underline{3.9} &  \underline{93.9} &  \underline{2.0} &  \underline{2.9} &  \underline{96.0} &  \underline{1.8} \\
\nksr (more data) & Voxels & - & - & - & - & - & - & {3.6} & {94.0} & {2.0} & {3.0} & {96.0} & {1.8}\\
Ours~(Minkowski)~\cite{choy20194d} \scriptsize{(w/ KNN)} & Voxels & - & \todo{} & \todo{} & 0.254 & 99.41 & 0.46 & 3.4 & 97.2 &1.9 & 2.7 & 98.1 & 2.0 \\
Ours~(Minkowski)~\cite{choy20194d} & Voxels & - & \todo{} & \todo{} & 0.301 & 98.48 & 0.31 & 3.8 & 96.2 & 1.5 & 3.0 & 97.4 & 1.5\\
\rowcolor{1st} Ours \scriptsize{(w/ KNN)} & Points &{0.321} & {98.34} & {0.13} & {0.243} & {99.61} & {0.48} &{3.2} & {97.5} & {3.2} &{2.6} & {98.3} & {3.4}\\
\rowcolor{1st}Ours & Points & {0.360} & {96.32} & 0.14 & 0.257 & 99.33 & 0.49 & {3.3} & {97.4} & 1.7 & {2.7} & {98.2} & 1.7 \\

\bottomrule
\end{tabular}
}
\caption{\textbf{In-domain evaluation} -- We show that our method achieves the best accuracy (CD and F-score) with significantly improved time efficiency~(inference latency).
Note we retrain \nksr (numbers are underlined) for fairer comparison, \ws{as the training data for \nksr is different from ours -- i.e., they reported some models trained on a ``mix'' of datasets, which is impossible to reproduce.
}
}
\label{tab:indomain}
\end{table*}


\paragraph{Evaluation pipeline}
To evaluate our method, we first extract the mesh with Dual Marching Cubes~\cite{schaefer2004dual} on the predicted SDF, and then compute the CD and F-score between 100k points sampled on the mesh, and 100k points sampled from the ground-truth dense point cloud.
We use the same approach as \nksr to prepare the input point clouds for training and evaluation from the ground-truth dense point clouds through downsampling.
Specifically, for indoor datasets (i.e., \synthetic, 
\scannet and \scenenn), we uniformly sample 10K points sampled from the ground truth dense point cloud. 
For outdoor driving scenes~(i.e., \carla), we follow the evaluation pipeline from \nksr.
We sample sparse input point clouds with a sparse 32-beam LiDAR with a ray distance noise of 0-5 cm and pose noise of $0-3^\circ$, and obtain the ground truth from a noise-free dense 256-beam LiDAR.

\begin{figure*}
\centering
\includegraphics[width=\linewidth]{visualizations/test_set_results.pdf}
\caption{
{\textbf{Qualitative results on \carla and \synthetic}} -- our method achieves high quality surface reconstructions which preserve more details than \nksr~which loses information due to quantization for large and non-uniformly sampled datasets like Carla.
}
\label{fig:qual_results_carla_syn}
\end{figure*}
 
\begin{figure*}
\centering
\vspace{-1em}
\includegraphics[width=.95\linewidth]{visualizations/scannet_results_0.pdf}
\caption{
Qualitative results on \scannet: We compare our method with prior SOTA~\cite{huang2023neural} and Ours~(Minkowski)~\cite{choy20194d} that is more comparable as it only differs from ours in the backbone. Our method achieves reconstruction of similar quality to the SOTA. It also \textit{significantly} outperforms Ours~(Minkowski), highlighting the importance of point-based methods. 
% \TODO{callouts too small? almost no zoom? why?}
}
\vspace{-1em}
\label{fig:scannet_results}
\end{figure*}
  

\paragraph{Implementation details}
We base our feature backbone on PointTransformerV3~\cite{wu2024point} with 4-levels.
The PointNet-style network is a 2-layered residual connection MLP, with hidden dimension of $32$ and output feature dimension of $32$.    
The grid size used in neighborhood function is $0.01$ meters.
Following \nksr, we use the similar coefficients for loss terms -- i.e., $\lambda_{\text{SDF}}$ is $300$ and $\lambda_{\text{mask}}$ is $150$.
However, we empirically set $\lambda_{\text{Eikonal}}$ to $10$~(\nksr does not need this regularizer thanks to its specialized surface solver).
We train our model with a batch size of $4$ on either a single \texttt{NVIDIA RTX A6000 ADA} or an \texttt{NVIDIA L40S}, and a learning rate of $10^{-3}$.
We adopt the Adam optimizer with default parameters.
We set the maximum number of epochs to 200 and employ a cosine learning rate decay starting from epoch 120.


\begin{table*}
\centering
\resizebox{\linewidth}{!}{
\setlength{\tabcolsep}{2pt}
\begin{tabular}{LccccccccccC}
\toprule
Methods & & \multicolumn{3}{c}{{\bf \synthetic $\rightarrow$ \scannet}}  &  \multicolumn{3}{c}{{{\bf \scannet $\rightarrow$ \synthetic}}} & \multicolumn{3}{c}{{{\bf \scannet $\rightarrow$ \scenenn}}} \\ 
 \cmidrule(lr){3-5} \cmidrule(lr){6-8} \cmidrule(lr){9-11}
&Primitive& CD ($10^{-2}$) $\downarrow$ & F-Score  $\uparrow$ & {Latency (s) $\downarrow$ } & CD ($10^{-2}$) $\downarrow$ & F-Score  $\uparrow$ & {Latency (s) $\downarrow$ } & CD ($10^{-2}$) $\downarrow$ & F-Score  $\uparrow$ & {Latency (s) }$\downarrow$ \\       
\midrule
SA-CONet~\cite{tang2021SACon} & Voxels & 0.845 & 77.80 & - & - & - & - & - & - & - \\
ConvOcc~\cite{peng2020convoccnet} & Voxels & 0.776 & 83.30  & - & - & - & - & - & - & - \\
NDF~\cite{chibane2020ndf} & Voxels & 0.452 & 96.00 & - & {0.568} & {88.10} & - & 0.425 & 94.80 & - \\
RangeUDF~\cite{wang2022rangeudf} & Voxels & {0.303} & {98.60} & {-} & 0.481& 91.50 & - & 0.324 & 97.80 & - \\
\nksr & Voxels & {0.329} & {97.37} & {2.02} & {0.351} & {97.41} & {0.46} & {0.268} & {99.18} & {1.95} \\
\rowcolor{1st} Ours (w/ KNN) & Points & {0.284} & {98.65} & {0.54} & {0.327} &{98.37} & {0.13} & {0.277} & {99.00} & {0.50} \\
\bottomrule
\end{tabular}
}
\caption{\textbf{Cross-domain evaluation} -- we achieve the best generalization ability in two cases with much better time efficiency. In the other case where we generalize from \scannet to \scenenn, we achieve accuracy on par with the SOTA baseline~\cite{huang2023neural} with less than a half of their latency.  
}
\vspace{-1.4em}
\label{tab:across_domain}
\end{table*}


\paragraph{Reconstruction latency}
For both our models and NKSR, we record the reconstruction latency for all indoor scenes on a single \texttt{NVIDIA RTX 3090}, and for large outdoor scenes on a single \texttt{NVIDIA L40s} given that more GPU memory is required.
We omit data loading time, and only record the average forward pass time. 

\subsection{In-domain evaluation}
We compare against \nksr~(the current state-of-the-art), RangeUDF~\cite{wang2022rangeudf},  SPSR~\cite{kazhdan2013screened}, NDF~\cite{chibane2020ndf}, ConvOcc~\cite{peng2020convoccnet} and SA-CONet~\cite{tang2021SACon}.     
We further include a baseline that replaces our backbone with MinkowskiNet~\cite{choy20194d} (i.e., Ours~(Minkowski)) to show the degraded performance due to the information loss caused by voxelization.

\paragraph{Quantitative results -- \Cref{tab:indomain}}
Across indoor and outdoor datasets, our method outperforms baselines in terms of accuracy and time efficiency. Especially in outdoor datasets, our method achieves the best surface reconstruction with the smallest latency -- nearly \textit{half} of the second best's latency.
In indoor datasets, which have relatively uniform sampling patterns, we achieve accuracy on par with the previous state-of-the-art, but with significantly improved time efficiency.
Note that we achieve this advantage even with KNN because, in smaller indoor point clouds, the highly engineered KNN implementation has similar time efficiency to that of our neighborhood function.
We further detail our analysis on this matter in the \texttt{Supplementary Material}. 
We also note that our approximate neighborhood function is still effective, as it outperforms the directly comparable baseline MinkowskiNet~\cite{choy20194d}, which shares the same structure except for the backbone and neighborhood function.


\paragraph{Qualitative results -- \Cref{fig:qual_results_carla_syn,fig:scannet_results}}
We show that our method tends to reconstruct surfaces of the best quality among the compared methods.
Especially, on the non-uniform large scale \carla, our method tends to preserve more details than the previous state-of-the-art~\cite{huang2023neural}, which voxelizes the point cloud.   

\subsection{Cross-domain evaluation -- \Cref{tab:across_domain}}
We further test the generalization ability of our method with a cross-domain evaluation.
We evaluate models trained with dataset A on other a different dataset B; we denote this as~A $\rightarrow$ B. 
As shown in \Cref{tab:across_domain}, there are three cases in total.
In two cases (i.e., \synthetic $\rightarrow$ \scannet and \scannet $\rightarrow$ \synthetic), our method achieves the best accuracy with the best time efficiency. 
In another case (\scannet $\rightarrow$ \scenenn), we achieve accuracy on par with SOTA~\cite{huang2023neural} with a much better time efficiency, i.e., less than a half of the latency required by the SOTA~\cite{huang2023neural}.

\subsection{Ablation studies}
Our ablations are executed on \scannet, as it is a real-world dataset, and is equipped with precise ground truth surface meshes.

\begin{table}
\centering
\resizebox{.9\columnwidth}{!}{
\begin{tabular}{LccccccC}
\toprule
{\bf Neighbor Num.} & {CD (10\textsuperscript{-2})} $\downarrow$ & {F-score} $\uparrow$ & Latency (s) $\downarrow$ \\ \midrule
 2 & 0.246 & 99.56 & 109 \\
 4 & 0.244 & 99.59 & 127 \\
 \rowcolor{1st} 
8 & {0.243} & 99.61 & 151 \\
16 & 0.256 & 99.28 & 187 \\
\bottomrule
\end{tabular}
}
\caption{{\bf The impact of neighborhood size} -- larger neighborhoods lead to increased computational cost, and we find that 8 neighbors gives the best balance of cost and quality.}
\label{tab:numpts_neighbor}
\vspace{-1em}
\end{table}

\paragraph{Impact of neighborhood size -- \Cref{tab:numpts_neighbor}}
We analyze the impact of neighborhood size on performance. Larger neighborhood size leads to increased computation overhead. 
We show that the 8-nearest neighboring points gives the best trade-off between accuracy and time efficiency.
Considering a large number (e.g., 16) of neighboring points degrades performance as the the aggregation module has limited capacity to predict the precise SDF from a large local point cloud.

\begin{table}
\centering
\resizebox{.95\columnwidth}{!}{
\begin{tabular}{@{}lcccccc@{}}
\toprule
\makecell{\bf Num. of hidden\\\bf layers in $\aggregation$} & CD (10\textsuperscript{-2}) $\downarrow$ & F-score $\uparrow$ & Latency (s) $\downarrow$ \\ \midrule
 2 & 0.257 & 99.33 & 152 \\
 4 & 0.256 & 99.32 & 166 \\
\bottomrule
\end{tabular}
}
\caption{{\bf Impact of capacity of $\aggregation$} -- we find that increasing the number of layers in $\aggregation$ beyond 2 decreases time efficiency without substantially improving the reconstruction quality.}
\label{tab:agg_capacity}
\vspace{-1em}
\end{table}

\paragraph{Impact of capacity of $\aggregation$ -- \Cref{tab:agg_capacity}} 
We report how the capacity of the aggregation module $\aggregation$ (i.e., different number of hidden layers) impacts the performance.
We observe that aggregation modules of higher capacity give better performance but degraded time efficiency. However, as shown in~\Cref{tab:agg_capacity}, a very large capacity (4 layers) for $\aggregation$ does not help.
We show that we we use 2 layers to have a good trade-off between accuracy and time efficiency. 
We supplement~\Cref{tab:agg_capacity} with an analysis across even more levels in the \texttt{Supplementary Material}.

\begin{table}
\centering
\resizebox{.9\columnwidth}{!}{
\begin{tabular}{@{}lcccc@{}}
\toprule
\textbf{Num. of scales} &KNN & Minkowski & Z-order & Hilbert  \\ \midrule
0 & 1.00 & 0.17 & 0.44  & \cellcolor{1st}0.46  \\
1 & 1.00 & 0.29 & 0.48  & \cellcolor{1st}0.50  \\
2 & 1.00 & 0.38 & 0.49  & \cellcolor{1st}0.52  \\
3 & 1.00 & 0.44 & 0.49  & \cellcolor{1st}0.53  \\ %
\bottomrule
\end{tabular}
}
\caption{\textbf{Recall rate of our Hilbert-curve based $\neighbor$} -- we find that the Hilbert curve consistently outperforms both the Z-order curve~\cite{morton1966computer} and the one-ring neighborhood from Minkowski relative to the exact k-nearest neighbors.
}
\vspace{-1em}
\label{tab:locality_neighbor}
\end{table}

\paragraph{Analysis of neighbors retrieved by~$\neighbor$ -- \Cref{tab:locality_neighbor}}
\at{We now investigate the quality of the point neighborhoods retrieved by various possible implementations for $\neighbor$.
In particular, we are interested to experimentally study whether our serialization indeed preserves locality.
To quantify this, we treat the neighborhood retrieved with KNN as the ground-truth.}
We report the recall rate of a local neighborhood by comparing it with this ground truth~(we ignore the precision rate because we remove false positives with a distance threshold).
We also report the recall rate of the one-ring neighborhood retrieved in Minkowski~\cite{choy20194d}.
We show that the recall rate of our Hilbert $\neighbor$ is the best across variants, and across all scales.

\begin{table}[t]
\centering
\resizebox{\columnwidth}{!}{
\begin{tabular}{L rr rR}
\toprule
Methods & \multicolumn{2}{c}{Uniform} & \multicolumn{2}{c}{Non-Uniform}   \\ 
\cmidrule(r){1-1}
\cmidrule(lr){2-3}
\cmidrule(l){4-5}
\nksr & 0.246 & 480s & 0.273 & 668s  \\
Ours~(Minkowski)~\cite{choy20194d}  & 0.301 & 97s & 0.349 & 94s \\
Ours~(Minkowski)~\cite{choy20194d} {(w/ KNN)} & 0.254 & 145s & 0.294 & 155s \\
\rowcolor{1st} Ours~(w/ serialization) & {0.257} & {152s} & {0.296} & {145s} \\
\rowcolor{1st} Ours~(w/ KNN) & \textbf{0.243} & \textbf{151s} & \textbf{0.273} & \textbf{142s}  \\
\bottomrule
\end{tabular}
}
\caption{
\textbf{The impact of sampling} -- we evaluate uniform vs non-uniform sampling on ScanNet. We find that our method achieves the best accuracy (in terms of CD ($10^{-2}$)) and good time efficiency compared to \nksr~for both sampling types.
}
\vspace{-1em}
\label{tab:nonuniform_scannet}
\end{table}

\paragraph{The impact of sampling pattern --~\Cref{tab:nonuniform_scannet}} 
We report the impact of sampling pattern on performance by evaluating models on ScanNet point clouds that are uniformly or non-uniformly sampled. 
{To non-uniformly sample the ScanNet point clouds, we first partitioned the scene into eight blocks and randomly sampled a different number of points from each block. The number of samples followed an arithmetic sequence with a common difference of 200. Finally, we padded the last block to ensure that the total number of points remained 10K.}
 
We show that our method achieves better robustness to non-uniform sampling than the baselines, highlighting the importance of avoiding quantization of the point cloud for high quality surface reconstruction. 


\section{Conclusion}

%In this paper, w
We propose a new PEFT method called DiffoRA, which enables efficient and adaptive LLM fine-tuning based on LoRA. 
Instead of adjusting every interior rank, 
%of the decomposition matrices 
%of all modules, 
we argue that adopting LoRA module-wisely is sufficient. 
To achieve this, we construct a DAM to select the modules that are most suitable and essential to fine-tune. We theoretically analyze how the DAM impacts the convergence rate and generalization capability.
%of the pre-trained model. 
Furthermore, we adopt continuous relaxation and discretization to establish DAM.
%for each task. 
To alleviate the issue of discretization discrepancy, we utilize the weight-sharing strategy for optimization. 
%We fully implement our method and t
The experimental results demonstrate that our DiffoRA works consistently better than the baselines across all benchmarks. 

\section*{Acknowledgments}
The authors would like to thank Hydro-Québec and DRONE VOLT® for supplying the LineDrone. The authors
also thank Sophie Stratford and Marc-Antoine Leclerc for their help during outdoor flights.

\bibliographystyle{IEEEtran}
\bibliography{main}

\end{document}


\typeout{get arXiv to do 4 passes: Label(s) may have changed. Rerun}