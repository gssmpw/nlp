\documentclass[letterpaper, 10 pt, conference]{ieeeconf}
\IEEEoverridecommandlockouts
\overrideIEEEmargins  

\usepackage{amsmath,amsfonts}
\usepackage{algorithmic}
\usepackage{algorithm}
\usepackage{array}
\usepackage[caption=false,font=normalsize,labelfont=sf,textfont=sf]{subfig}
\usepackage{textcomp}
\usepackage{stfloats}
\usepackage{url}
\usepackage{verbatim}
\usepackage{graphicx}
\usepackage{cite}
\usepackage{lipsum}
\usepackage{xcolor}
\usepackage{soul}
\usepackage{doi}
\usepackage{caption}
\usepackage{multirow}

\usepackage{hyperref} 
\hypersetup{
    colorlinks,
    citecolor=black,
    filecolor=black,
    linkcolor=black,
    urlcolor=black
}
\usepackage[all]{hypcap}

\usepackage{notoccite}
\hyphenation{op-tical net-works semi-conduc-tor IEEE-Xplore}
% updated with editorial comments 8/9/2021



\begin{document}

\title{ARENA: Adaptive Risk-aware and Energy-efficient NAvigation for Multi-Objective 3D Infrastructure Inspection with a UAV}

\author{David-Alexandre Poissant$^{1-2}$, Alexis Lussier Desbiens$^{1}$, François Ferland$^{2}$, and Louis Petit$^{1-2}$% <-this % stops a space
\thanks{This research was funded by Alliance grant number 2601-2600-703 and CRIAQ between Université de Sherbrooke, Hydro-Québec and DRONE VOLT®.}% <-this % stops a space
\thanks{$^{1}$The authors are with the Createk Design Lab, University of Sherbrooke, Sherbrooke, J1K 2R1, Canada, {\tt\small [david-alexandre.poissant, louis.petit, alexis.lussier.desbiens]@usherbrooke.ca} }
\thanks{$^{2}$The authors are with the Intelligent Interactive Integrated Interdisciplinary Robot Lab (IntRoLab), University of Sherbrooke, Sherbrooke, J1K 2R1, Canada, {\tt\small françois.ferland@usherbrooke.ca} }}


% The paper headers
%\markboth{Journal of \LaTeX\ Class Files,~Vol.~14, No.~8, August~2021}%
%{Shell \MakeLowercase{\textit{et al.}}: A Sample Article Using IEEEtran.cls for IEEE Journals}

%\IEEEpubid{0000--0000/00\$00.00~\copyright~2024 IEEE}
% Remember, if you use this you must call \IEEEpubidadjcol in the second
% column for its text to clear the IEEEpubid mark.

\maketitle

\begin{abstract}
Autonomous robotic inspection missions require balancing multiple conflicting objectives while navigating near costly obstacles. Current multi-objective path planning (MOPP) methods struggle to adapt to evolving risks like localization errors, weather, battery state, and communication issues. This letter presents an Adaptive Risk-aware and Energy-efficient NAvigation (ARENA) MOPP approach for UAVs in complex 3D environments. Our method enables online trajectory adaptation by optimizing safety, time, and energy using 4D NURBS representation and a genetic-based algorithm to generate the Pareto front. A novel risk-aware voting algorithm ensures adaptivity. Simulations and real-world tests demonstrate the planner's ability to produce diverse, optimized trajectories covering 95\% or more of the range defined by single-objective benchmarks and its ability to estimate power consumption with a mean error representing 14\% of the full power range. The ARENA framework enhances UAV autonomy and reliability in critical, evolving 3D missions.
\end{abstract}

\begin{keywords}
Motion and Path Planning, Autonomous Vehicle Navigation, Aerial Systems: Applications, Optimization and Optimal Control, Robust/Adaptive Control
\end{keywords}

\section{Introduction}
\label{sec:introduction}
Recommendation systems are a standard component of most online platforms, providing personalized suggestions for products, movies, articles, and more.
In addition to generic recommendation, these platforms often present the option for the user to search for items, either via natural language or structured queries.
While collaborative filtering methods like matrix factorization have proven successful in addressing data sparsity for unconditional generation, they often fall short when attempting to combine them with more complicated queries. 
This is not unexpected, as vector embeddings, while effectively capturing linear relationships, are ill-equipped to handle the complex set-theoretic relationships. Even advanced neural network-based approaches, which are designed to capture intricate relationships, have been shown to struggle with set-theoretic compositionally that underlie many real-world preferences. 

% Consider the common scenario where a user desires a movie that is both a "comedy" and "action," but not a "romance."
% This demonstrates a need for a recommendation model capable of handling set operations such as conjunction and negation.

% Recommending items according some logical constraints of their attributes is a key problem in many modern applications, such as e-commerce and video/music streaming platforms. These facets are invoked by simple user queries, which typically correspond to categories, tags, or attributes of the items. While some user queries are straightforward, like "comedy movies," more often they are complex, such as "comedy but not romantic comedies." 

Let us consider an example where a user named Bob wants to watch a comedy which is not a romantic comedy.
Assuming we have a prior watch history for users, standard collaborative filtering techniques (e.g. low-rank matrix factorization) would yield a learned score function $\score(m, \Bob)$ for each movie $m$.
% , however this does not incorporate Bob's search request.
If we also have movie-attribute annotations, we could form the set of comedies $C$ and set of romance movies $R$ and simply filter to those movies in $C \setminus R$, however this assumes that the movie-attribute annotations are complete, which is rarely the case.

A standard approach in a setting with sparse data is to learn a low-rank approximation for the {attribute $\times$ movie} matrix $\mathbf A$, yielding a dense matrix $\hat {\mathbf A}$. We can then form sets of movies based on this dense matrix using an (attribute-specific) threshold, \eg $\hat C \defeq \{m \mid \hat A_{\comedy, m} > \tau_\comedy\}$ and $\hat R \defeq \{m \mid \hat A_{\romance, m} > \tau_\romance\}$, and then rank movies $m \in \hat C \setminus \hat R$ according to $\score(m, \Bob)$. While this approach does allow for performing the sort of queries we are after, it suffers from three fundamental issues:

% \begin{figure}[h!]
%   \centering
%   \subfloat[Standard matrix completion assumes you are given partial information about the user $\times$ movie matrix $\mathbf U$, and potentially incomplete information about the attribute $\times$ movie matrix $\mathbf A$, and asks you to recover any unobserved entries. The task of set-theoretic matrix completion extends this to being able to predict the entries of arbitrary set-theoretic combinations of these rows.]{
%     \includegraphics[width=0.45\textwidth]{pictures/set-theoretic matrix completion.jpg}
%     \label{figure: set-theoretic matrix completion}
%   }
%   \hfill
%   \subfloat[Box embeddings represent the movies, users, and attributes as "boxes" (Cartesian products of intervals) in $\mathbb R^n$. The score for a specific movie in relation to a given query is determined by the proportion of the movie box's volume that falls within the corresponding region. During training, this membership score for a movie, w.r.t the $U$ and $A$ are optimized, creating a set-geometric representation of the matrix.]{
%     \includegraphics[width=0.45\textwidth]{pictures/box depiction.jpg}
%     \label{figure: box depiction}
%   }
%   \caption{Set-theoretic matrix completion for movies, users, and attributes, illustrating how box embeddings, trained in a set-theoretic manner, address this task.}
% \end{figure}
\begin{figure}[]
    \centering
    \includegraphics[width=0.8\columnwidth]{pictures/set-theoretic_matrix_completion.jpg}
    \caption{Standard matrix completion assumes you are given partial information about the user $\times$ movie matrix $\mathbf U$, and potentially incomplete information about the attribute $\times$ movie matrix $\mathbf A$.}
    \label{fig:set_theoretic_mc}
\end{figure}

\begin{figure}[]
    \centering
    \includegraphics[width=0.8\columnwidth]{pictures/box_depiction.jpg}
    \caption{Box embeddings represent the movies, users, and attributes as "boxes" (Cartesian products of intervals) in $\mathbb R^n$.}
    \label{fig:box_depiction}
\end{figure}


% \begin{figure}[h]
%     \centering
%     \includegraphics[width=0.8\textwidth]{ICLR 2025 Template/pictures/set-theoretic matrix completion.png} % Adjust width as necessary
%     \caption{The task of set-theoretic matrix completion depicted in the setting where users and attributes form the rows, and movies are the columns. Set-theoretic matrix completion is concerned with not simply filling in additional entries of the user $\times$ Movie matrix $\mathbf U$ or the attribute $\times$ movie matrix $\mathbf A$, but also being able to predict the entries of arbitrary set-theoretic combinations of these rows.}
%     \label{fig:side_caption_image}
% \end{figure}

\begin{enumerate}
    \item Limited user-attribute interaction:
    % separately classifying attributes and then ranking for each user does not take into account user-attribute interactions.
    Since the attribute classification is done independently from the user, any latent relationships between the user and attribute cannot be taken into account.
    \item Error compounding: Errors in the completion of attribute sets accumulate as the number of sets involved in the query increase.
    \item Mismatched inductive-bias: Our queries can be viewed as set-theoretic combinations of the rows, not linear combinations. As such, using a low-rank approximation of the matrix may be misaligned with the eventual use.
\end{enumerate}


% The recommender system has access to the ground truth of the set of movies Bob would like to watch (\textbf{Bob}), the set of comedy movies (\textbf{comedy}), and the set of romantic movies (\textbf{romance}). In this ideal scenario, the system would trivially return \textbf{Bob} $\cap$ \textbf{comedy} $\setminus$ \textbf{romance}. However, in practice, we can only construct these sets from item tags and user history, which are often incomplete and noisy. Consequently, the set operation might yield an inaccurate or empty set of items. This problem is exacerbated as the queries become more complex. (forward reference to experiment sections).

% A standard approach to mitigating the incompleteness issue is to learn representations of \Bob, \romance, and \comedy. One of the traditional yet most effective methods (cite) is to learn a low-rank approximation of the observed matrix $O$ which is the concatenation of the {User $\times$ Movie} interaction matrix $U$ and the {Tags $\times$ Movie} attribute matrix $A$. The learned representations can provide smooth score functions over all possible items for all users and attributes. In our example, we would be able to calculate $\score(\Bob, m)$, $\score(\comedy, m)$, and $\score(\romance, m)$ for all movies $m$ by calculating dot products between the vector representations for the each movie and the vector representations for \Bob, \comedy, and \romance.

% While these scores generalize to the incomplete part of the observed matrix $O$, they do not naturally allow us to compute set-theoretic queries. For example, consider how one might use these representations to address Bob's query from before. 

% This is not optimal for several reasons: the selection of the threshold is an ad-hoc process, and the prediction error for thresholding will snowball rapidly as query complexity increases (see Section ref). A better approach would be to devise a smooth score function for the entire query \textbf{Bob} $\cap$ \textbf{comedy} $\setminus$ \textbf{romance}. A common method to achieve this is by multiplying the scores corresponding to each query, e.g., $s(bob \cap comedy \cap \neg romance, m) = s(bob, m) \times s(comedy, m) \times (1 - s(romance, m))$. However, this approach ignores the interdependence between attributes and users, again resulting in suboptimal behavior for the recommender.\\

In this paper, we formulate the problem of attribute-specific recommendation as matrix completion where rows are not necessarily \emph{linear combinations} of each other but, rather, are \emph{set-theoretic combinations} of each other. More precisely, given some user $\times$ movie interaction matrix $\mathbf U$ and attribute $\times$ movie matrix $\mathbf A$, the queries we are considering are set-theoretic combinations of these rows (see \Cref{fig:set_theoretic_mc}). For example, the ground-truth data for comedies which are not romance movies which Bob likes would be the vector $x \in \{0,1\}^{|M|}$, where $x_m = 1$ if and only if $\mathbf U_{\Bob, m} = 1$ and $\mathbf A_{\comedy, m} = 1$ and $\mathbf A_{\romance, m} = 0$. Note that this is not a linear combination of the previous rows, and so while the inductive bias of low-rank factorization has proven immensely effective for collaborative filtering we should not expect it to be directly applicable in this setting.


% if the observed matrix $O$ is the concatenation of $[U; A]$, the query answering task essentially involves predicting the entries of the rows of the joint matrix $O_{q} = [U; A; U \cap A; U \cap \neg A; U \cap A \cap A; U \cap A \cap \neg A; \cdots]$. 

% Note that, the low-rank approximated vector model is capable of capturing linear dependencies between similar user or attribute rows or between movie columns. This inductive bias proves to be immensely effective for collaborative filtering. However, in our case the relationship amongst the rows of the $O_{q}$ is non-linear and strictly set-theoretic in nature, e.g., the row of \textbf{Bob} $\cap$ \textbf{comedy} is strictly an intersection between the individual rows of \textbf{Bob} and \textbf{comedy}. \\

Instead, we propose to learn representations for the users and attributes that are consistent with specific set-theoretic axioms. These representations must also be compactly parameterizable in a lower-dimensional space, differentiable with respect to some appropriate score function, and allow for efficient computation of various set operations.
% . Additionally, we need to define a measure (similar to vector dot products) to train these representations.
Box Embeddings \citep{hard_box, gumbel_box}, which are axis-parallel $n$-dimensional hyperrectangles, meet these criteria (see \Cref{fig:box_depiction}).
The volume of a box is easily calculated as the product of its side-lengths. Furthermore, box embeddings are closed under intersection (\ie the intersection of two boxes is another box). Inclusion-exclusion thus allows us to calculate the volume of arbitrary set-theoretic combinations of boxes.
% The simple axis-parallel geometry allows for the calculation of intersections of multiple boxes.
% The embedding space is closed under intersection (the intersection of two or more boxes is also a box) and the volume of a box is easily calculated as the product of its side lengths. Via inclusion-exclusion, this allows us to efficiently calculate arbitrary set-theoretic combinations of boxes.
% This ease of parameterization, along with straightforward volume and intersection calculations, makes box embeddings an excellent candidate for our purpose.


The contributions of our paper are as follows -
\begin{enumerate}
    \item We model the problem of attribute-specific query recommendation as "set-theoretic matrix completion", where attributes and users are treated as sets of items. We discuss the challenges faced by existing machine-learning approaches for this problem setup.
    \item We demonstrate the inconsistency of existing vector embedding models for this task. Additionally, we establish box embeddings as a suitable embedding method for addressing such set-theoretic problems.\mb{We don't do this, so we either need to or we need to weaken this claim.}
    \item We conduct an extensive empirical study comparing various vector and box embedding models for the task of set-theoretic query recommendation.
\end{enumerate}

Box embeddings, with their geometric set operations, significantly outperform all vector-based methods. We also evaluate score multiplication and threshold-based prediction for both vector and box embedding models, and find that performing set operations directly on the box embeddings performs best, solidifying our claim that the inductive bias of box embeddings provides the necessary generalization capabilities to address set-theoretic queries.
\section{Related Work}

RLAIF is a popular LLM post-training method \citep{bai2022constitutional}.
Its idea to generate feedback by a model itself is called Pseudo-Labeling (PL) and well studied in semi-supervised learning \citep{scudder1965probability,lee2013pseudo}. Below, we will review PL and LLM self-training methods related to ours. For a thorough review of each topic, please refer to \citet{yang2023survey} and \citet{xiao2025foundations}.

\paragraph{Pseudo-Labeling:}
PL trains a student model so that its output is close to the output of a teacher model on unlabeled data \citep{scudder1965probability,lee2013pseudo}.
Various methods for constructing a teacher model has been proposed.
For example, $\Pi$-model \citep{bachman2014learning,sajjadi2016regularization,laine2017temporal} and virtual adversarial training \citep{miyato2019virtual} perturbs input and/or neural networks of student models.
Some methods use weighted average of previous student models' predictions or weights as a teacher model \citep{laine2017temporal,tarvainen2017mean}, and
other methods even try to optimize a teacher model by viewing PL as an optimization problem \citep{yi2019probabilistic,wang2020repetitive,pham2021meta}.
Our work is complimentary to this line of works since what we modify is the risk estimator rather than teacher models.

PL is known to suffer from erroneous labels generated by a model-in-training \citep{haase2021iterative,rizve2021in}, and this observation well aligns with recent reports on RLAIF that erroneous self-feedback is a major source of failure \citep{Madaan2023-fn}.
A straightforward approach is to filter potentially incorrect labels based on confidence \citep{sohn2020fixmatch,haase2021iterative,rizve2021in} or the amplitude of loss as in self-paced learning \citep{bengio2009curriculum,kumar2010self,jiang2015self}.
Another method is refining pseudo-labels in a way similar to label propagation \citep{zhu2002learning} but with similarity scores computed using neural networks \citep{kutt2024contrastive}.
Our work tackles the issue of erroneous labels by using a risk estimate robust to erroneous self-feedback based on UU learning \citep{Lu2019-sd,Lu2020-dx}. Even though our approach requires only a minimal change, we observed a significant performance boost.

\paragraph{LLM's Self-Refinement:}
To enhance the reasoning capabilities of LLMs, early efforts primarily explored various prompt engineering techniques \citep{Brown2020-bw,Wei2022-kt,Wang2023-pf,Yao2023-ey}. Even with refined prompting strategies, an LLM’s initial outputs can remain limited in some scenarios. Recent studies have proposed iteratively improved answer strategies called self-refinement approaches~\citep{Madaan2023-fn,Kim2023-zz,Chen2024-zm}, and our work falls within this lineage.

Self-refinement involves generating responses through agents in distinct roles that iteratively provide feedback \citep{Shinn2023-no, Zhu2023-wn}. For example, multi-agent debate frameworks \citep{Estornell2024-gh} use an answering agent and an evaluation agent to collaboratively improve answer quality \citep{Chen2024-ua, Du2024-pi, Smit2024-cn}. However, these methods usually assume that the LLM has enough internal knowledge to generate effective feedback and revisions; when it doesn’t, performance gains can be minimal or even negative \citep{Huang2024-co,Li2024-fo,Kamoi2024-lc,Kamoi2024-od}—sometimes degrading performance \citep{Huang2024-co}. Our approach minimizes this reliance: internal knowledge is used only initially for classification, with subsequent improvements relying on extracting features directly from the data via UU learning.

Other work addresses knowledge gaps by retrieving external information or using external tools \citep{Huang2022-mv, Wang2023-fz, Shi2024-gx, Wu2024-cb}, but setting up such systems can be costly. In contrast, our method requires only a small amount of labeled data for initialization without assuming the availability of external knowledge or tools.

\section{Fundamentals}
\label{sec:fundamentals}

This section covers the essential concepts required to understand the algorithm described in Section \ref{sec:risk_aware_path_planning}. We begin by outlining the definition of a MOO process and conclude with an introduction to the trajectory representation tool utilized in our approach.



\subsection{Multi-objective Optimization}

The goal in MOO is to find a \textit{D}-dimensional solution vector 
$z = [z_{1} \ ... \ z_{D}]^{T}$ 
that minimize or maximize a set $\mathcal{F}$ of $E$ objective functions

\begin{equation}
\label{MOO_objective_equation}
    f_{e}(z), \quad e=1, ..., E
\end{equation}
that can be subject to $F$ inequality constraints

\begin{equation}
\label{MOO_inequality_constraint_equation}
    g_{f}(z)\geq 0, \quad f=1, ..., F
\end{equation}
as well to G equality constraints

\begin{equation}
\label{MOO_equality_constraint_equation}
    h_{g}(z)= 0, \quad g=1, ..., G
\end{equation}

Additionally, each element of $z$ can be limited by a lower and upper bound

\begin{equation}
\label{MOO_variable_bounds_equation}
    z_{d}^{(L)} \leq z_{d} \leq z_{d}^{(U)}, \quad d=1, ..., D
\end{equation}

The search space $S$ contains all feasible solutions to the optimization problem. A solution is possible if it satisfies all constraints and bounds. Feasible solutions can be ranked by dominance, as no single solution may optimize all objectives simultaneously. A solution $z_{1}$ is said to dominate another solution $z_{2}$ $(z_{1} \preceq z_{2})$ if $z_{1}$ is equal or better than $z_{2}$ for all objectives and strictly better in at least one. Each non-dominated solution forms the Pareto set.
\subsection{Non-Uniform Rational B-Splines (NURBS)}
\label{sec:nurbs}

A NURBS curve of order $p + 1$ is defined by Piegl et al. \cite{piegl1995} as

\begin{equation}
\label{nurbs_equation}
C(u) = \frac{\sum_{i=0}^{n} N_{i, p}(u)w_{i}P_{i}}{\sum_{i=0}^{n} N_{i, p}(u)w_{i}}
\end{equation}

where
\begin{itemize}
    \item $n$ is the number of control points,
    \item $p$ is the degree of the basis function $N_{i, p}$,
    \item $P_{i} = [x_{i} \ \ y_{i} \ \ z_{i}]^{T}$ is the $i^{th}$ control point (assuming a 3D curve), and
    \item $w_{i}$ is its weight.
\end{itemize}

The basis functions $N_{i, p}$ are defined with respect to the parameter $u$ and a fixed knot vector

\begin{equation}
\label{knot_vector_equation}
    U = [u_{0} \ \ ... \ \ u_{m}]^{T}
\end{equation}
containing $m + 1$ knots, whereas $m = n + p$. The De-Boor-Cox formulas allow to calculate the basis function in recursion \cite{piegl1995, deboor1972, deboor1978}.
\section{Adaptive Risk-Aware Path Planning}
\label{sec:risk_aware_path_planning}

We begin this section by defining the multi-objective path planning problem. We present the decision vector, the continuous path representation method, and the objectives in Section \ref{sec:risk_aware_path_planning_problem_definition}. Section \ref{sec:risk_aware_path_planning_solver} shows our adaptive risk-aware multi-objective path planning method, including our multi-objective solver and how we initialize it, the voting algorithm, the real-time adjustment of its coefficients, and the hyperparameters of the entire path planning method.

\subsection{Problem Definition and Representation}
\label{sec:risk_aware_path_planning_problem_definition}

We generalize the infrastructure inspection path planning scenario as a 4D problem, which we define as $\mathbb{D}^{4} = \{ (x, y, z, v) \mid x \in [x_{min}, x_{max}], y \in [y_{min}, y_{max}], z \in [z_{min}, z_{max}], v \in [-v_{max}, v_{max}] \}$ with start and goal positions $\{(x_{a}, y_{a}, z_{a}, v_{a}), (x_{b}, y_{b}, z_{b}, v_{b})\} \in \mathbb{D}^{4}$. The goal of this MOPP problem is to find the NURBS curve $C = \{C(u): u \in [a, b]\}$ that minimizes the set of objectives $\mathcal{F}$ described below.


We choose to represent trajectories using NURBS \cite{piegl1995}, described in Section \ref{sec:nurbs}, to benefit from their inherent properties for multidimensional multi-objective optimization. These include \textit{strong convex hull} properties, \textit{local approximation} capability, and \textit{infinite differentiability} apart from knot multiplicity \cite{piegl1995}. The first two properties are helpful during optimization to constrain the curve inside a bounded area and to escape local minima by varying the curve locally. The last property ensures the smoothness of the curve according to the knot multiplicity of the curve, which is beneficial for UAVs that have strict actuator limitations, such as low acceleration. The design vector is given by:
\begin{equation}
    \label{eq:design_vector}
    \begin{aligned}
        z ={} & [w_{0} \ \ x_{1} \ \ y_{1} \ \ z_{1} \ \left \| \vec{v}_{1} \right \| \ w_{1}\\
              &\ \ \ \ \ \ \ \ \ \ \ \ \ \ \ ...\\
              &x_{n-1} \ \ y_{n-1} \ \ z_{n-1} \ \left \| \vec{v}_{n-1} \right \| \ w_{n-1} \ \ w_{n}]^{T}
    \end{aligned}
\end{equation}


We introduce the 4$^{th}$ dimension to the parametric curve $C$, representing the velocity norm at each control point, enabling velocity profile modulation along trajectories as risks evolve. The start and goal positions, along with their speeds, are fixed and excluded from the decision vector $z$. The problem is constrained by two hard limits: acceleration ($a_{max}$ in Table \ref{table:parameters_table}) to meet actuator limitations and collision avoidance to ensure feasibility.


Costs are generalized into three functions to encompass multiple scenarios: time, safety, and energy consumption, which are described in the following subsections.

%%%% TIME COST %%%%
\hfill
\subsubsection{Time cost}
\label{sec:time_cost}
The time cost is computed as:

\begin{equation}
    \label{eq:time_cost}
    \mathcal{F}_{Time} = \sum_{i=0}^{Q-1} \frac{d_{i}}{ \left \| \vec{v}_{i + 1} \right \| }
\end{equation}
where $Q$ is the number of sample points and $d_{i}$ is the distance on the path segment. The $i + 1$ velocity is used, assuming inspection drones are slow and can quickly reach low speeds. Under normal conditions, with no significant risks during an inspection mission, the algorithm minimizes this cost function, prioritizing speeding up the inspection process over safety or energy consumption.

%%%% SAFETY COST %%%%
\hfill
\subsubsection{Safety cost}
\label{sec:safety_cost}
The safety cost consists of two terms:

\begin{equation}
    \label{eq:safety_cost}
    \begin{aligned}
        \mathcal{F}_{Safety}={} & k_{a} (\frac{\sum_{i=0}^{Q} \mathcal{F}_{sdf_{i}}}{Q} + max(\mathcal{F}_{sdf})) \\ & + k_{b} (\frac{\sum_{i=0}^{Q} \mathcal{F}_{ch_{i}}}{Q} + max(\mathcal{F}_{ch}))
    \end{aligned}
\end{equation}
where $\mathcal{F}_{sdf_{i}}$ is a collision cost to ensure a safe UAV-obstacle distance and $\mathcal{F}{ch_{i}}$ denotes a non-insertion cost for obstacles forming convex hulls to keep trajectories outside critical zones. In this letter, convex hulls, modeled as oriented bounding boxes (OBB), are dynamically adjusted around cables and pylons using semantic information during power line inspections. The coefficients $k_{a}$ and $k_{b}$ in Eq. \ref{eq:safety_cost} are normalized since $\mathcal{F}{sdf_{i}}$ and $\mathcal{F}{ch_{i}}$ return normalized values. Combining mean and maximum costs ensures overall trajectory safety while maximizing the minimum obstacle distance. To support this, we use a signed distance field (SDF) \cite{jones2006} alongside our path planning algorithm to maintain obstacle information within the free space. $\mathcal{F}{sdf_{i}}$ is defined using the closest obstacle distance in the SDF:

\begin{equation}
    \label{eq:sdf_cost}
    \mathcal{F}_{sdf_{i}} = 
    \begin{cases} 
        0 & d_{obs_{i}} \geq r_{sdf_{max}} \\
        \frac{\lambda}{d_{obs_{i}}} - 1 & r_{sdf_{min}} < d_{obs_{i}} < r_{sdf_{max}} \\
        1 & d_{obs_{i}} \leq r_{sdf_{min}}
    \end{cases}
\end{equation}
where $\lambda = \frac{r_{sdf_{min}} r_{sdf_{max}}}{r_{sdf_{max}} - r_{sdf_{min}}}$ is a scaling factor and $d_{obs_{i}}$ is the distance between the UAV and the nearest obstacle. Both $r_{sdf_{min}}$ and $r_{sdf_{max}}$ are listed in Table \ref{table:parameters_table}. This cost function reflects the increasing risks as the UAV gets closer to obstacles, which rapidly increases near them. The non-insertion cost function $\mathcal{F}_{ch_{i}}$ is formulated as follows:

\begin{equation}
    \label{eq:convex_hull_cost}
    \mathcal{F}_{ch_{i}} = \sum_{i=0}^{N}
    \begin{cases}
        0 & d_{ch_{i}} \geq r_{ch_{max}} \\
        1 - \frac{d_{ch_{i}}}{r_{ch_{max}}} & 0 < d_{ch_{i}} < r_{ch_{max}} \\
        1 & d_{ch_{i}} \leq 0
    \end{cases}
\end{equation}
where $d_{ch_{i}}$ is the distance to the nearest convex hull and $r_{ch_{max}}$ is the maximum influence radius of convex hulls. Under high wind, communication, or localization risks, the algorithm prioritizes safety by selecting trajectories that minimize this cost function over time or energy efficiency.

%%%% ENERGY COST %%%%
\hfill
\subsubsection{Energy cost}
\label{sec:energy_cost}
To evaluate the energy consumption of a planned trajectory, our approach integrates physics-based principles with experimental data. \cite{guoku2022} proposed a model for quadrotor UAVs with BLDC motors, showing that energy consumption varies with the flight state. According to this model, it can be hypothesized that for a given flight speed, horizontal movements require more power than hovering, with vertical movements further affecting it. The pitch/roll power relationship with the vertical axis forms a quadric surface \cite{hilbert1999}, described by a general equation:


\begin{equation}
    \label{eq:quadric_surface}
    \begin{aligned}
        {} & ax^{2}+by^{2}+cz^{2}+dxy+eyz+fxz\\
        & \ \ \ +gx+hy+kz+L = 0 
    \end{aligned}
\end{equation}

Because of a quadrotor's symmetry and to simplify the model, coupling terms are eliminated. The term $L$ acts as a geometric translation factor and is fixed so the system doesn't return the trivial solution. With these simplifications, power data from minimally the six flight directions ($\pm Z, \pm Roll, \pm Pitch$) creates a solvable system of equations. To locate a point on this surface during optimization, we project along a unit vector $\hat{v} = (v_{x}, v_{y}, v_{z})$, parameterized by $t$: $x = t v_{x}, y = t v_{y}, z = t v_{z}$. Substituting into Eq. \ref{eq:quadric_surface} simplifies to:

\begin{equation}
    \label{eq:simplified_quadric_equation}
    At^{2} + Bt + L = 0
\end{equation}
where $A = av_{x}^{2} + bv_{y}^{2} + cv_{z}^{2}$, $B = gv_{x} + hv_{y} + kv_{z}$, and $L = 1$. The roots of this equation, solved using the quadratic formula, determine the points on the quadric surface along the unit vector, providing steady-state power consumption $P(\hat{v})$ at these coordinates. The energy consumption cost function is given by:

\begin{equation}
    \label{eq:energy_cost}
    \mathcal{F}_{Energy} = \sum_{i=1}^{Q} P(\hat{v}_{i}) \Delta t
\end{equation}

Since inspection drones are slow and quickly reach cruising speed, we assume steady-state flight energy consumption remains constant, with transient states contributing minimally. Unlike the time cost function, the energy cost function includes a directional factor influencing the optimizer. In low battery conditions, the algorithm prioritizes energy and time over safety.
\begin{algorithm}[ht!]
\caption{\textit{NovelSelect}}
\label{alg:novelselect}
\begin{algorithmic}[1]
\State \textbf{Input:} Data pool $\mathcal{X}^{all}$, data budget $n$
\State Initialize an empty dataset, $\mathcal{X} \gets \emptyset$
\While{$|\mathcal{X}| < n$}
    \State $x^{new} \gets \arg\max_{x \in \mathcal{X}^{all}} v(x)$
    \State $\mathcal{X} \gets \mathcal{X} \cup \{x^{new}\}$
    \State $\mathcal{X}^{all} \gets \mathcal{X}^{all} \setminus \{x^{new}\}$
\EndWhile
\State \textbf{return} $\mathcal{X}$
\end{algorithmic}
\end{algorithm}




\section{Experimental Results}
In this section, we present the main results in~\secref{sec:main}, followed by ablation studies on key design choices in~\secref{sec:ablation}.

\begin{table*}[t]
\renewcommand\arraystretch{1.05}
\centering
\setlength{\tabcolsep}{2.5mm}{}
\begin{tabular}{l|l|c|cc|cc}
type & model     & \#params      & FID$\downarrow$ & IS$\uparrow$ & Precision$\uparrow$ & Recall$\uparrow$ \\
\shline
GAN& BigGAN~\cite{biggan} & 112M & 6.95  & 224.5       & 0.89 & 0.38     \\
GAN& GigaGAN~\cite{gigagan}  & 569M      & 3.45  & 225.5       & 0.84 & 0.61\\  
GAN& StyleGan-XL~\cite{stylegan-xl} & 166M & 2.30  & 265.1       & 0.78 & 0.53  \\
\hline
Diffusion& ADM~\cite{adm}    & 554M      & 10.94 & 101.0        & 0.69 & 0.63\\
Diffusion& LDM-4-G~\cite{ldm}   & 400M  & 3.60  & 247.7       & -  & -     \\
Diffusion & Simple-Diffusion~\cite{diff1} & 2B & 2.44 & 256.3 & - & - \\
Diffusion& DiT-XL/2~\cite{dit} & 675M     & 2.27  & 278.2       & 0.83 & 0.57     \\
Diffusion&L-DiT-3B~\cite{dit-github}  & 3.0B    & 2.10  & 304.4       & 0.82 & 0.60    \\
Diffusion&DiMR-G/2R~\cite{liu2024alleviating} &1.1B& 1.63& 292.5& 0.79 &0.63 \\
Diffusion & MDTv2-XL/2~\cite{gao2023mdtv2} & 676M & 1.58 & 314.7 & 0.79 & 0.65\\
Diffusion & CausalFusion-H$^\dag$~\cite{deng2024causal} & 1B & 1.57 & - & - & - \\
\hline
Flow-Matching & SiT-XL/2~\cite{sit} & 675M & 2.06 & 277.5 & 0.83 & 0.59 \\
Flow-Matching&REPA~\cite{yu2024representation} &675M& 1.80 & 284.0 &0.81 &0.61\\    
Flow-Matching&REPA$^\dag$~\cite{yu2024representation}& 675M& 1.42&  305.7& 0.80& 0.65 \\
\hline
Mask.& MaskGIT~\cite{maskgit}  & 227M   & 6.18  & 182.1        & 0.80 & 0.51 \\
Mask. & TiTok-S-128~\cite{yu2024image} & 287M & 1.97 & 281.8 & - & - \\
Mask. & MAGVIT-v2~\cite{yu2024language} & 307M & 1.78 & 319.4 & - & - \\ 
Mask. & MaskBit~\cite{weber2024maskbit} & 305M & 1.52 & 328.6 & - & - \\
\hline
AR& VQVAE-2~\cite{vqvae2} & 13.5B    & 31.11           & $\sim$45     & 0.36           & 0.57          \\
AR& VQGAN~\cite{vqgan}& 227M  & 18.65 & 80.4         & 0.78 & 0.26   \\
AR& VQGAN~\cite{vqgan}   & 1.4B     & 15.78 & 74.3   & -  & -     \\
AR&RQTran.~\cite{rq}     & 3.8B    & 7.55  & 134.0  & -  & -    \\
AR& ViTVQ~\cite{vit-vqgan} & 1.7B  & 4.17  & 175.1  & -  & -    \\
AR & DART-AR~\cite{gu2025dart} & 812M & 3.98 & 256.8 & - & - \\
AR & MonoFormer~\cite{zhao2024monoformer} & 1.1B & 2.57 & 272.6 & 0.84 & 0.56\\
AR & Open-MAGVIT2-XL~\cite{luo2024open} & 1.5B & 2.33 & 271.8 & 0.84 & 0.54\\
AR&LlamaGen-3B~\cite{llamagen}  &3.1B& 2.18& 263.3 &0.81& 0.58\\
AR & FlowAR-H~\cite{flowar} & 1.9B & 1.65 & 296.5 & 0.83 & 0.60\\
AR & RAR-XXL~\cite{yu2024randomized} & 1.5B & 1.48 & 326.0 & 0.80 & 0.63 \\
\hline
MAR & MAR-B~\cite{mar} & 208M & 2.31 &281.7 &0.82 &0.57 \\
MAR & MAR-L~\cite{mar} &479M& 1.78 &296.0& 0.81& 0.60 \\
MAR & MAR-H~\cite{mar} & 943M&1.55& 303.7& 0.81 &0.62 \\
\hline
VAR&VAR-$d16$~\cite{var}   & 310M  & 3.30& 274.4& 0.84& 0.51    \\
VAR&VAR-$d20$~\cite{var}   &600M & 2.57& 302.6& 0.83& 0.56     \\
VAR&VAR-$d30$~\cite{var}   & 2.0B      & 1.97  & 323.1 & 0.82 & 0.59      \\
\hline
\modelname& \modelname-B    &172M   &1.72&280.4&0.82&0.59 \\
\modelname& \modelname-L   & 608M   & 1.28& 292.5&0.82&0.62\\
\modelname& \modelname-H    & 1.1B    & 1.24 &301.6&0.83&0.64\\
\end{tabular}
\caption{
\textbf{Generation Results on ImageNet-256.}
Metrics include Fréchet Inception Distance (FID), Inception Score (IS), Precision, and Recall. $^\dag$ denotes the use of guidance interval sampling~\cite{guidance}. The proposed \modelname-H achieves a state-of-the-art 1.24 FID on the ImageNet-256 benchmark without relying on vision foundation models (\eg, DINOv2~\cite{dinov2}) or guidance interval sampling~\cite{guidance}, as used in REPA~\cite{yu2024representation}.
}\label{tab:256}
\end{table*}

\subsection{Main Results}
\label{sec:main}
We conduct experiments on ImageNet~\cite{deng2009imagenet} at 256$\times$256 and 512$\times$512 resolutions. Following prior works~\cite{dit,mar}, we evaluate model performance using FID~\cite{fid}, Inception Score (IS)~\cite{is}, Precision, and Recall. \modelname is trained with the same hyper-parameters as~\cite{mar,dit} (\eg, 800 training epochs), with model sizes ranging from 172M to 1.1B parameters. See Appendix~\secref{sec:sup_hyper} for hyper-parameter details.





\begin{table}[t]
    \centering
    \begin{tabular}{c|c|c|c}
      model    &  \#params & FID$\downarrow$ & IS$\uparrow$ \\
      \shline
      VQGAN~\cite{vqgan}&227M &26.52& 66.8\\
      BigGAN~\cite{biggan}& 158M&8.43 &177.9\\
      MaskGiT~\cite{maskgit}& 227M&7.32& 156.0\\
      DiT-XL/2~\cite{dit} &675M &3.04& 240.8 \\
     DiMR-XL/3R~\cite{liu2024alleviating}& 525M&2.89 &289.8 \\
     VAR-d36~\cite{var}  & 2.3B& 2.63 & 303.2\\
     REPA$^\ddagger$~\cite{yu2024representation}&675M &2.08& 274.6 \\
     \hline
     \modelname-L & 608M&1.70& 281.5 \\
    \end{tabular}
    \caption{
    \textbf{Generation Results on ImageNet-512.} $^\ddagger$ denotes the use of DINOv2~\cite{dinov2}.
    }
    \label{tab:512}
\end{table}

\noindent\textbf{ImageNet-256.}
In~\tabref{tab:256}, we compare \modelname with previous state-of-the-art generative models.
Out best variant, \modelname-H, achieves a new state-of-the-art-performance of 1.24 FID, outperforming the GAN-based StyleGAN-XL~\cite{stylegan-xl} by 1.06 FID, masked-prediction-based MaskBit~\cite{maskgit} by 0.28 FID, AR-based RAR~\cite{yu2024randomized} by 0.24 FID, VAR~\cite{var} by 0.73 FID, MAR~\cite{mar} by 0.31 FID, and flow-matching-based REPA~\cite{yu2024representation} by 0.18 FID.
Notably, \modelname does not rely on vision foundation models~\cite{dinov2} or guidance interval sampling~\cite{guidance}, both of which were used in REPA~\cite{yu2024representation}, the previous best-performing model.
Additionally, our lightweight \modelname-B (172M), surpasses DiT-XL (675M)~\cite{dit} by 0.55 FID while achieving an inference speed of 9.8 images per second—20$\times$ faster than DiT-XL (0.5 images per second). Detailed speed comparison can be found in Appendix \ref{sec:speed}.



\noindent\textbf{ImageNet-512.}
In~\tabref{tab:512}, we report the performance of \modelname on ImageNet-512.
Similarly, \modelname-L sets a new state-of-the-art FID of 1.70, outperforming the diffusion based DiT-XL/2~\cite{dit} and DiMR-XL/3R~\cite{liu2024alleviating} by a large margin of 1.34 and 1.19 FID, respectively.
Additionally, \modelname-L also surpasses the previous best autoregressive model VAR-d36~\cite{var} and flow-matching-based REPA~\cite{yu2024representation} by 0.93 and 0.38 FID, respectively.




\noindent\textbf{Qualitative Results.}
\figref{fig:qualitative} presents samples generated by \modelname (trained on ImageNet) at 512$\times$512 and 256$\times$256 resolutions. These results highlight \modelname's ability to produce high-fidelity images with exceptional visual quality.

\begin{figure*}
    \centering
    \vspace{-6pt}
    \includegraphics[width=1\linewidth]{figures/qualitative.pdf}
    \caption{\textbf{Generated Samples.} \modelname generates high-quality images at resolutions of 512$\times$512 (1st row) and 256$\times$256 (2nd and 3rd row).
    }
    \label{fig:qualitative}
\end{figure*}

\subsection{Ablation Studies}
\label{sec:ablation}
In this section, we conduct ablation studies using \modelname-B, trained for 400 epochs to efficiently iterate on model design.

\noindent\textbf{Prediction Entity X.}
The proposed \modelname extends next-token prediction to next-X prediction. In~\tabref{tab:X}, we evaluate different designs for the prediction entity X, including an individual patch token, a cell (a group of surrounding tokens), a subsample (a non-local grouping), a scale (coarse-to-fine resolution), and an entire image.

Among these variants, cell-based \modelname achieves the best performance, with an FID of 2.48, outperforming the token-based \modelname by 1.03 FID and surpassing the second best design (scale-based \modelname) by 0.42 FID. Furthermore, even when using standard prediction entities such as tokens, subsamples, images, or scales, \modelname consistently outperforms existing methods while requiring significantly fewer parameters. These results highlight the efficiency and effectiveness of \modelname across diverse prediction entities.






\begin{table}[]
    \centering
    \scalebox{0.92}{
    \begin{tabular}{c|c|c|c|c}
        model & \makecell[c]{prediction\\entity} & \#params & FID$\downarrow$ & IS$\uparrow$\\
        \shline
        LlamaGen-L~\cite{llamagen} & \multirow{2}{*}{token} & 343M & 3.80 &248.3\\
        \modelname-B& & 172M&3.51&251.4\\
        \hline
        PAR-L~\cite{par} & \multirow{2}{*}{subsample}& 343M & 3.76 & 218.9\\
        \modelname-B&  &172M& 3.58&231.5\\
        \hline
        DiT-L/2~\cite{dit}& \multirow{2}{*}{image}& 458M&5.02&167.2 \\
         \modelname-B& & 172M&3.13&253.4 \\
        \hline
        VAR-$d16$~\cite{var} & \multirow{2}{*}{scale} & 310M&3.30 &274.4\\
        \modelname-B& &172M&2.90&262.8\\
        \hline
        \baseline{\modelname-B}& \baseline{cell} & \baseline{172M}&\baseline{2.48}&\baseline{269.2} \\
    \end{tabular}
    }
    \caption{\textbf{Ablation on Prediction Entity X.} Using cells as the prediction entity outperforms alternatives such as tokens or entire images. Additionally, under the same prediction entity, \modelname surpasses previous methods, demonstrating its effectiveness across different prediction granularities. }%
    \label{tab:X}
\end{table}

\noindent\textbf{Cell Size.}
A prediction entity cell is formed by grouping spatially adjacent $k\times k$ tokens, where a larger cell size incorporates more tokens and thus captures a broader context within a single prediction step.
For a $256\times256$ input image, the encoded continuous latent representation has a spatial resolution of $16\times16$. Given this, the image can be partitioned into an $m\times m$ grid, where each cell consists of $k\times k$ neighboring tokens. As shown in~\tabref{tab:cell}, we evaluate different cell sizes with $k \in \{1,2,4,8,16\}$, where $k=1$ represents a single token and $k=16$ corresponds to the entire image as a single entity. We observe that performance improves as $k$ increases, peaking at an FID of 2.48 when using cell size $8\times8$ (\ie, $k=8$). Beyond this, performance declines, reaching an FID of 3.13 when the entire image is treated as a single entity.
These results suggest that using cells rather than the entire image as the prediction unit allows the model to condition on previously generated context, improving confidence in predictions while maintaining both rich semantics and local details.





\begin{table}[t]
    \centering
    \scalebox{0.98}{
    \begin{tabular}{c|c|c|c}
    cell size ($k\times k$ tokens) & $m\times m$ grid & FID$\downarrow$ & IS$\uparrow$ \\
       \shline
       $1\times1$ & $16\times16$ &3.51&251.4 \\
       $2\times2$ & $8\times8$ & 3.04& 253.5\\
       $4\times4$ & $4\times4$ & 2.61&258.2 \\
       \baseline{$8\times8$} & \baseline{$2\times2$} & \baseline{2.48} & \baseline{269.2}\\
       $16\times16$ & $1\times1$ & 3.13&253.4  \\
    \end{tabular}
    }
    \caption{\textbf{Ablation on the cell size.}
    In this study, a $16\times16$ continuous latent representation is partitioned into an $m\times m$ grid, where each cell consits of $k\times k$ neighboring tokens.
    A cell size of $8\times8$ achieves the best performance, striking an optimal balance between local structure and global context.
    }
    \label{tab:cell}
\end{table}



\begin{table}[t]
    \centering
    \scalebox{0.95}{
    \begin{tabular}{c|c|c|c}
      previous cell & noise time step &  FID$\downarrow$ & IS$\uparrow$ \\
       \shline
       clean & $t_i=0, \forall i<n$& 3.45& 243.5\\
       increasing noise & $t_1<t_2<\cdots<t_{n-1}$& 2.95&258.8 \\
       decreasing noise & $t_1>t_2>\cdots>t_{n-1}$&2.78 &262.1 \\
      \baseline{random noise}  & \baseline{no constraint} &\baseline{2.48} & \baseline{269.2}\\
    \end{tabular}
    }
    \caption{
    \textbf{Ablation on Noisy Context Learning.}
    This study examines the impact of noise time steps ($t_1, \cdots, t_{n-1} \subset [0, 1]$) in previous entities ($t=0$ represents pure Gaussian noise).
    Conditioning on all clean entities (the ``clean'' variant) results in suboptimal performance.
    Imposing an order on noise time steps, either ``increasing noise'' or ``decreasing noise'', also leads to inferior results. The best performance is achieved with the "random noise" setting, where no constraints are imposed on noise time steps.
    }
    \label{tab:ncl}
\end{table}


\noindent\textbf{Noisy Context Learning.}
During training, \modelname employs Noisy Context Learning (NCL), predicting $X_n$ by conditioning on all previous noisy entities, unlike Teacher Forcing.
The noise intensity of previous entities is contorlled by noise time steps $\{t_1, \dots, t_{n-1}\} \subset [0, 1]$, where $t=0$ corresponds to pure Gaussian noise.
We analyze the impact of NCL in~\tabref{tab:ncl}.
When conditioning on all clean entities (\ie, the ``clean'' variant, where $t_i=0, \forall i<n$), which is equivalent to vanilla AR (\ie, Teacher Forcing), the suboptimal performance is obtained.
We also evaluate two constrained noise schedules: the ``increasing noise'' variant, where noise time steps increase over AR steps ($t_1<t_2< \cdots < t_{n-1}$), and the `` decreasing noise'' variant, where noise time steps decrease ($t_1>t_2> \cdots > t_{n-1}$).
While both settings improve over the ``clean'' variant, they remain inferior to our final ``random noise'' setting, where no constraints are imposed on noise time steps, leading to the best performance.




        

\section{Conclusion}
We introduce a novel approach, \algo, to reduce human feedback requirements in preference-based reinforcement learning by leveraging vision-language models. While VLMs encode rich world knowledge, their direct application as reward models is hindered by alignment issues and noisy predictions. To address this, we develop a synergistic framework where limited human feedback is used to adapt VLMs, improving their reliability in preference labeling. Further, we incorporate a selective sampling strategy to mitigate noise and prioritize informative human annotations.

Our experiments demonstrate that this method significantly improves feedback efficiency, achieving comparable or superior task performance with up to 50\% fewer human annotations. Moreover, we show that an adapted VLM can generalize across similar tasks, further reducing the need for new human feedback by 75\%. These results highlight the potential of integrating VLMs into preference-based RL, offering a scalable solution to reducing human supervision while maintaining high task success rates. 

\section*{Impact Statement}
This work advances embodied AI by significantly reducing the human feedback required for training agents. This reduction is particularly valuable in robotic applications where obtaining human demonstrations and feedback is challenging or impractical, such as assistive robotic arms for individuals with mobility impairments. By minimizing the feedback requirements, our approach enables users to more efficiently customize and teach new skills to robotic agents based on their specific needs and preferences. The broader impact of this work extends to healthcare, assistive technology, and human-robot interaction. One possible risk is that the bias from human feedback can propagate to the VLM and subsequently to the policy. This can be mitigated by personalization of agents in case of household application or standardization of feedback for industrial applications. 

\section*{Acknowledgments}
The authors would like to thank Hydro-Québec and DRONE VOLT® for supplying the LineDrone. The authors
also thank Sophie Stratford and Marc-Antoine Leclerc for their help during outdoor flights.

\bibliographystyle{IEEEtran}
\bibliography{main}

\end{document}


\typeout{get arXiv to do 4 passes: Label(s) may have changed. Rerun}