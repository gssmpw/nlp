\subsection{Multi-objective Optimization}

The goal in MOO is to find a \textit{D}-dimensional solution vector 
$z = [z_{1} \ ... \ z_{D}]^{T}$ 
that minimize or maximize a set $\mathcal{F}$ of $E$ objective functions

\begin{equation}
\label{MOO_objective_equation}
    f_{e}(z), \quad e=1, ..., E
\end{equation}
that can be subject to $F$ inequality constraints

\begin{equation}
\label{MOO_inequality_constraint_equation}
    g_{f}(z)\geq 0, \quad f=1, ..., F
\end{equation}
as well to G equality constraints

\begin{equation}
\label{MOO_equality_constraint_equation}
    h_{g}(z)= 0, \quad g=1, ..., G
\end{equation}

Additionally, each element of $z$ can be limited by a lower and upper bound

\begin{equation}
\label{MOO_variable_bounds_equation}
    z_{d}^{(L)} \leq z_{d} \leq z_{d}^{(U)}, \quad d=1, ..., D
\end{equation}

The search space $S$ contains all feasible solutions to the optimization problem. A solution is possible if it satisfies all constraints and bounds. Feasible solutions can be ranked by dominance, as no single solution may optimize all objectives simultaneously. A solution $z_{1}$ is said to dominate another solution $z_{2}$ $(z_{1} \preceq z_{2})$ if $z_{1}$ is equal or better than $z_{2}$ for all objectives and strictly better in at least one. Each non-dominated solution forms the Pareto set.