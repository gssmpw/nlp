\section{Introduction}
Uncrewed aerial vehicles (UAVs) are becoming crucial tools in various scenarios where human involvement can become too risky or incur high costs, such as search and rescue \cite{khan2021}, surveillance \cite{shakhatreh2019}, and inspection \cite{leclerc2023, petit2022}. Achieving autonomy in these scenarios heavily relies on the path planning module to generate safe and feasible trajectories. Numerous approaches have been proposed to find the shortest or safest path in a cluttered environment. For intricate tasks, such as infrastructure inspection, ensuring safe maneuvers involves considering multiple factors, prompting increased interest in multi-objective optimization (MOO) methods.

\begin{figure}[!t]
    \centering
    \includegraphics[width=2.69in]{optimization_visualisation3.png}
    \caption{Representation of every non-dominated trajectory at the end of ARENA's process. An RGB model depicts the various objectives' influence on a trajectory.}
    \label{fig_1}
\end{figure}


In many cases, optimizing multiple objectives for path planning results in various viable paths, representing the optimal Pareto front, the primary goal of MOO. To our knowledge, most Multi-Objective Path Planning (MOPP) methods include static terms like safety, time, smoothness, or others to find the optimal front \cite{hayes2022}. Some research estimates risks using offline data, such as satellite imagery and building positions \cite{causa2023}, but lacks dynamic adaptation to mission conditions. Our recent work introduced the Multi-Objective and Adaptive Risk-aware (MOAR) path planner \cite{petit2024}, enabling real-time trajectory modulation by considering evolving risks. However, it relies on exhaustive graph searches in discretized 2D environments, limiting real-time applicability in large 3D spaces. In this letter, we propose an extended version of the MOAR path planner, broadening its use to 3D spaces and continuous representations. The proposed framework, named Adaptive Risk-aware and Energy-efficient Navigation (ARENA) MOPP, leverages a 4D Non-uniform rational B-spline (NURBS) \cite{piegl1995} representation and the non-dominated sorting genetic algorithm (NSGA-II) \cite{deb2002} to compute feasible trajectories and modulate the velocity profile along them. It optimizes safety, time, and a novel energy objective while adapting to evolving risks like localization precision, wind, battery state, and communication. We show the reliability and efficiency of our framework in numerous simulations and real-world power line inspection scenarios. Although developed primarily for power line inspections, our method can be easily applied to different path planning problems in complex 3D environments, including most infrastructure inspection missions.


Section \ref{sec:related_work} reviews the state of the art on MOPP. Section \ref{sec:fundamentals} explains the fundamentals of MOO and NURBS to understand better our approach presented in Section \ref{sec:risk_aware_path_planning}. Section \ref{sec:results} shows the results of the sensitivity analysis and risk adaptability. We conclude this letter in Section \ref{sec:conclusion} by suggesting potential future works.



