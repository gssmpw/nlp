\section{Conclusion}
\label{sec:conclusion}


In this paper, we investigate the MOPP problem of inspection missions with a UAV. We propose a novel approach, combining risk-awareness through a voting algorithm and a genetic optimizer, a 4D continuous trajectory representation tool, an energy consumption model, and coefficient adjustment to dynamically adapt to changing risk factors and transition smoothly between different planning behaviors. Our contributions to the field of MOPP are as follows: (a) we introduce a novel voting algorithm for choosing the best trajectory according to evolving mission risks, (b) we push further the notion of insertion cost introduced in our recent work [6], (c) we introduced a 4$^{th}$ dimension in a continuous representation tool to modulate and optimize speed while complying with actuator constraints, and (d) we present a novel semi-empirical energy consumption model that incorporates different flight situations.

The results highlight the effectiveness of our adaptive MOPP framework in generating diverse and near-optimal trajectories for inspection scenarios. Through simulated and real-world experiments, our algorithm proved its ability to adapt to varying risks, balancing safety, energy efficiency, and execution time spanning a broad range of evaluation metrics. Specifically, it covered 63\% and 80\% of the range defined by our behavioral benchmarks for path duration and average obstacle distance, respectively. Moreover, our energy consumption model successfully estimated real-world flight test trajectories with a 14\% error at its worst. However, our method has limitations. Its computing time may be too high for applications requiring frequent replanning. Additionally, the energy consumption model assumes a steady-state regime and does not account for transient dynamics, limiting its applicability to UAVs where transient effects are significant.

In future studies, we aim to extend our energy consumption model to accommodate transient dynamics, thereby extending its use to more dynamic UAVs. We also plan to explore deep learning techniques to accelerate inference. Lastly, we intend to generalize our framework for other robotic platforms, such as ground and nautical systems, to demonstrate its versatility across a broad range of applications.