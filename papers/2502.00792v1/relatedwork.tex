\section{Related Work}
\subsection{LLM-based Agent Systems}
LLMs are making significant strides towards achieving Artificial General Intelligence (AGI) by enhancing the capabilities of intelligent agents. These models improve autonomy, responsiveness, and social interaction skills, enabling agents to handle complex tasks such as natural language processing, knowledge integration, information retention, logical reasoning, and strategic planning. Recent developments in intelligent agent frameworks, such as AutoGPT~\cite{yang2023auto} and Metagpt~\cite{hong2023metagpt}, have advanced multi-agent collaboration by incorporating standardized operating procedures (SOPs). These frameworks facilitate research by streamlining agent system integration~\cite{huang2023agentcoder,gur2023real,xu2024eduagent,li2024agent,zhang2024finagent}. For instance, EduAgent~\cite{xu2024eduagent} integrates cognitive science principles to guide LLMs, enhancing their ability to model and understand diverse learning behaviors and outcomes. Additionally, Agent Hospital~\cite{li2024agent} utilizes a large-scale language model to simulate hospital environments, enabling medical agents to adapt and improve their treatment strategies through interactive learning.  

\subsection{Bidding Optimization in RTB}
RTB has been a critical focus in online advertising~\cite{wang2015real}, aiming to maximize the value of ad placements within a given budget. Traditional methods employ static parameters~\cite{perlich2012bid,yu2017online,yu2020low} to optimize revenue, often using historical bid data to set bidding parameters. ~\cite{yu2020low} use linear programming to address these optimization problems. Such methods usually fall short in dynamic bidding environments. Researchers have increasingly framed RTB as a sequential decision problem to overcome these limitations, applying RL techniques to enhance automated bidding strategies~\cite{cai2017real,zhao2018deep,wu2018budget,he2021unified}. DRLB~\cite{wu2018budget} approaches budget-constrained bidding as a Markov decision process, offering a model-free RL framework for optimization. USCB~\cite{he2021unified} introduces an RL method that dynamically adjusts parameters for optimal performance, improving convergence rates through recursive optimization. Despite these advancements, RL faces challenges such as training complexity and interpretability, indicating a need for more robust machine learning models in RTB strategies.