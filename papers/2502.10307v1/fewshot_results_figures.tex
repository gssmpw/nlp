\begin{figure}[h!]
    \centering
    \includegraphics[width=150 pt]{figs/finetune_nowcast.png}
    \caption{
    We compare the nowcasting performance of SPIRIT and Gao \emph{et al}. using nMAP error. The solid lines represent the average performance across different fine-tuning training sizes, measured in weeks of data. The shaded regions indicate the x\% confidence interval, reflecting variability across multiple experimental settings, including training on one dataset and testing on another, as well as selecting fine-tuning data from different starting points throughout the year.
    % The figure shows the performance of SPIRIT and Gao et al. The line graph indicates average performance with varying the fine-tuning training size (in weeks of data). Further, the colored area indicates x\% confidence interval for the different experiments that we performed by changing the transfer learning settings (i.e., training on one dataset and testing on the other + picking fine-tuning data with different starting times spread across the year)
    }
    \label{fig:finetune_nowcast}
\end{figure}

\begin{figure*}[h!]
    \centering
    \includegraphics[width=\textwidth]{figs/fewshot_finetune7.drawio.png}
    \caption{
 We compare the forecasting performance of SPIRIT and Gao \emph{et al}. using nMAP error across different forecast intervals. Subfigures (a), (b), (c), and (d) correspond to 1-hour, 2-hour, 3-hour, and 4-hour forecasting, respectively. The solid lines represent the average performance for each forecast interval, with varying fine-tuning training sizes measured in weeks of data. The shaded regions denote the 95\% confidence interval, illustrating the variability across multiple experimental settings, including training on one dataset and testing on another, as well as selecting weeks of contiguous fine-tuning data from different starting points throughout the year. SPIRIT exhibits consistently low variance compared to the baseline, particularly in settings with severely limited data, demonstrating its ability to maintain stability. In contrast, the baseline shows high variance, indicating uncertainty in its predictions.
 }
    \label{fig:finetune_forecast}
\end{figure*}
