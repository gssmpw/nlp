\section{Dataset Details}
\label{sec:appendix_dataset_details}
\begin{table*}
  \caption{A Comparative Overview of the TSI880, ASI16, and SKIPP'D Datasets: Key Attributes Including Geographical Location, Data Provided, Image Resolution, Collection Frequency, and Annual Sample Size}
  \label{tab:dataset_comparison}
  \centering
  \renewcommand{\arraystretch}{1.2}
  \begin{tabularx}{\textwidth}{X X X X}
    \hline
    \textbf{Attribute} & \textbf{TSI880 Dataset} & \textbf{ASI16 Dataset} & \textbf{SKIPP'D Dataset} \\ 
    \hline
    Location & Golden, Colorado, USA & Golden, Colorado, USA & Stanford, California, USA \\ 
    Data Type & Sky images \& Irradiance data & Sky images \& Irradiance data & Sky images \& PV power output \\ 
    Data Frequency & 10-minutes & 10-minutes & 1-minute \\ 
    Image Resolution & 288x352 & 1536x1536 & 64x64 \\ 
    Camera Model & Aero-Laser TSI-880 & EKO ASI-16 & Hikvision DS-2CD6362F-IV \\ 
    Number of Samples / Year & 24,948 & 25,107 & 121,125 \\ 
    \hline
  \end{tabularx}
\end{table*}


\subsection{Overview of Datasets}
\label{subsec:appendix_overview_of_datasets}
\textbf{TSI880 Dataset:}
The TSI880 dataset is collected from the NREL Solar Radiation Research Laboratory in Golden, Colorado. The camera captures an image every 10 minutes from 7:50 to 16:40 daily, providing raw sky images along with corresponding global horizontal irradiance values. Additionally, the dataset includes auxiliary information such as air temperature, relative humidity, azimuth angle, and zenith angle.

\textbf{ASI16 Dataset:}
The ASI16 dataset is also sourced from the Solar Radiation Research Laboratory in Golden, Colorado, but it differs in that the camera setup captures images at a higher resolution. Similar to the TSI880 dataset, it provides global horizontal irradiance values and auxiliary data including azimuth angle, zenith angle, air temperature, relative humidity, and average wind speed.

\textbf{SKIPP'D Dataset:}
The SKIPP'D dataset consists of raw sky images and photovoltaic (PV) power output data collected from Stanford University, California, USA. Images are captured every minute with a resolution of 64×64 pixels, emphasizing finer temporal granularity at the expense of lower image resolution.

% datasets images grid
\begin{figure}[t]
    \centering
\includegraphics[width=240pt]{figs/datasets_diagram_wo_text.png}
    
    \caption{Examples of sunny, partly cloudy, and overcast conditions, captured by different sky cameras, are shown from left to right, across the three datasets: TSI, ASI, and SKIPP'D, displayed from top to bottom.}
    \label{fig:dataset_grid}
\end{figure}

\subsection{Temporal Consistency in Forecasting}
\label{subsec:appendix_temporal_consistency_in_forecasting}
Valid samples for forecasting are formed such that all the data points from time steps \( 1 \) to \( T \), and their corresponding forecast intervals \( T + \tau_1, T + \tau_2, \dots, T + \tau_H \), fall within the same day. This is an essential requirement because the predictions for future intervals rely on the assumption that both historical and forecast data belong to the same day. Using data from the current day to predict values for the following day is not a valid forecasting approach, as the discontinuity between days renders such predictions unreliable. Any samples that violate this condition are considered invalid and are excluded from training or evaluation.


\section{Clear Sky Global Horizontal Irradiance}
\label{sec:appendix_clear_sky_global_horizontal_irradiance}
Clear Sky Global Horizontal Irradiance (GHI) is the solar irradiance received on a horizontal surface under cloud-free conditions. Most of the time, it serves as an upper bound for the actual GHI at a given location and time.

Clear Sky GHI plays a key role in solar forecasting by serving as a baseline for estimating how much clouds reduce solar irradiance. By comparing actual irradiance with Clear Sky GHI, we can get an estimate of the impact of cloud cover, which helps in enhancing short-term predictions, and improving the accuracy of forecasting models.

Given the latitude and longitude of a location, the clear sky values can be estimated for any timestamp. This becomes very useful in solar forecasting, as this value would give a reference of how much the prediction needs to be.

Clear Sky GHI is computed using mathematical models incorporating solar position, atmospheric transmittance, and radiative transfer principles. A common approach is the Ineichen-Perez model \cite{clearsky1}:

\begin{equation}
    GHI_{\text{clear}} = I_0 \cdot \tau \cdot \cos(\theta_z)
\end{equation}

where \( I_0 \) is the extraterrestrial irradiance (W/m²), \( \tau \) is the atmospheric transmittance factor, \( \theta_z \) is the solar zenith angle.


\subsection{Extraterrestrial Irradiance (\( I_0 \))}
\label{subsec:appendix_extraterrestrial_irradiance}
Extraterrestrial irradiance (\( I_0 \)) is the solar irradiance just outside Earth's atmosphere, slightly varying due to Earth's elliptical orbit around the Sun. It is given by:

\begin{equation} 
    I_0 = S_c \cdot \left(1 + 0.033 \cos\left(\frac{2\pi n}{365}\right) \right)
\end{equation}

where \( S_c = 1367 \) W/m² (solar constant), \( n \) is the day of the year (1 for January 1, 365 for December 31).

\subsection{Atmospheric Transmittance (\( \tau \))}
\label{subsec:appendix_atmospheric_transmittance}
The atmospheric transmittance \( \tau \) accounts for the attenuation of solar radiation by the atmosphere. It is often estimated using empirical models, such as the Ineichen-Perez model \cite{clearsky1}:

\begin{equation}
    \tau = a \cdot e^{-b \cdot m}
\end{equation}

where \( a, b \) are empirical coefficients dependent on location and aerosol content, \( m \) is the air mass, given by \cite{ineichen1}:

\begin{equation}
    m = \frac{1}{\cos(\theta_z) + 0.15 (93.885 - \theta_z)^{-1.253}}
\end{equation}
where \( \theta_z \) is the solar zenith angle.

\section{Fine-tuning Detailed Results}
\label{sec:appendix_fine-tuning_detailed_results}
\subsection{Nowcasting}
\label{subsec:appendix_nowcasting}
To understand the impact of fine-tuning duration and the training size, we conducted a series of experiments by varying the amount of training data used for fine-tuning, by using subsets of the data consisting of 1, 2, 3, and 4 weeks.

Our results show that even with only one week of training data at a new location, the fine-tuned model performs remarkably well. Furthermore, in all experimental configurations, our model significantly outperforms the baseline.

Detailed results for these experiments are presented in Tables 6-9.

\begin{table}[h]
  \caption{
  Nowcasting Performance with 1 week training
  }
  \label{tab:oneweek_nowcast}
  \centering
  % \setlength{\tabcolsep}{20pt} 
  \renewcommand{\arraystretch}{1.2}
  \begin{tabular}{@{}c c c c@{}}
    \hline
    \textbf{Trained on} & \textbf{Finetuned on} & \textbf{SPIRIT} & \textbf{Gao et al.~\cite{wacv2022}} \\
    \hline
    \multirow{3}{*}{TSI} & ASI & \textbf{20.23}  & 52.01 \\
                          & SKIPP'D & \textbf{29.89}  & 63.82 \\
    \cline{1-4}
    \multirow{3}{*}{ASI} & TSI & \textbf{14.99}  & 27.98 \\
                          & SKIPP'D & \textbf{27.51}  & 40.92 \\
    \hline
  \end{tabular}
\end{table}

\begin{table}[h]
  \caption{
  Nowcasting Performance with 2 weeks training
  }
  \label{tab:twoweek_nowcast}
  \centering
  % \setlength{\tabcolsep}{20pt} 
  \renewcommand{\arraystretch}{1.2}
  \begin{tabular}{@{}c c c c@{}}
    \hline
    \textbf{Trained on} & \textbf{Finetuned on} & \textbf{SPIRIT} & \textbf{Gao et al.~\cite{wacv2022}} \\
    \hline
    \multirow{2}{*}{TSI} & ASI & \textbf{18.96}  & 51.45 \\
                          & SKIPP'D & \textbf{29.07}  & 62.91 \\
    \cline{1-4}
    \multirow{2}{*}{ASI} & TSI & \textbf{14.91}  & 27.71 \\
                          & SKIPP'D & \textbf{26.41}  & 40.25 \\
    \hline
  \end{tabular}
\end{table}

\begin{table}[h]
  \caption{
  Nowcasting Performance with 3 weeks training
  }
  \label{tab:threeweek_nowcast}
  \centering
  % \setlength{\tabcolsep}{20pt} 
  \renewcommand{\arraystretch}{1.2}
  \begin{tabular}{@{}c c c c@{}}
    \hline
    \textbf{Trained on} & \textbf{Finetuned on} & \textbf{SPIRIT} & \textbf{Gao et al.~\cite{wacv2022}} \\
    \hline
    \multirow{2}{*}{TSI} & ASI & \textbf{16.52}  & 50.38 \\
                          & SKIPP'D & \textbf{27.42}  & 62.05 \\
    \cline{1-4}
    \multirow{2}{*}{ASI} & TSI & \textbf{14.59}  & 27.53 \\
                          & SKIPP'D & \textbf{25.68}  & 39.89 \\
    \hline
  \end{tabular}
\end{table}

\begin{table}[h]
  \caption{
  Nowcasting Performance with 4 weeks training
  }
  \label{tab:fourweek_nowcast}
  \centering
  % \setlength{\tabcolsep}{20pt} 
  \renewcommand{\arraystretch}{1.2}
  \begin{tabular}{@{}c c c c@{}}
    \hline
    \textbf{Trained on} & \textbf{Finetuned on} & \textbf{SPIRIT} & \textbf{Gao et al.~\cite{wacv2022}} \\
    \hline
    \multirow{2}{*}{TSI} & ASI & \textbf{15.63}  & 50.01 \\
                          & SKIPP'D & \textbf{26.51}  & 61.17 \\
    \cline{1-4}
    \multirow{2}{*}{ASI} & TSI & \textbf{14.12}  & 27.28 \\
                          & SKIPP'D & \textbf{24.32}  & 39.43 \\
    \hline
  \end{tabular}
\end{table}


\subsection{Forecasting}
\label{subsec:appendix_forecasting}
\begin{table}[h]
  \caption{
  Forecasting Performance with 2 weeks of training.
  }
  \label{tab:twoweek_forecast}
  \centering
  \setlength{\tabcolsep}{2pt}
  \renewcommand{\arraystretch}{1.2} 
  \begin{tabular}{c c c c c}
    \hline
    \textbf{Interval} & \textbf{Trained on} & \textbf{Tested on} & \textbf{SPIRIT} & \textbf{Gao et al.~\cite{wacv2022}} \\
    \hline
    \multirow{4}{*}{1hr} & \multirow{2}{*}{TSI} & ASI & \textbf{31.15} & 33.86 \\
                          & & SKIPP'D & \textbf{32.35} & 38.24 \\
                          \cline{2-5}
                          & \multirow{2}{*}{ASI} & TSI & \textbf{24.47} & 36.18 \\
                          & & SKIPP'D & \textbf{27.00} & 30.48 \\
    \hline
    \multirow{4}{*}{2hr} & \multirow{2}{*}{TSI} & ASI & \textbf{32.70} & 36.44 \\
                          & & SKIPP'D & \textbf{29.41} & 39.06 \\
                          \cline{2-5}
                          & \multirow{2}{*}{ASI} & TSI & \textbf{25.93} & 36.71 \\
                          & & SKIPP'D & \textbf{25.96} & 33.55 \\
    \hline
    \multirow{4}{*}{3hr} & \multirow{2}{*}{TSI} & ASI & \textbf{34.41} & 38.24 \\
                          & & SKIPP'D & \textbf{31.53} & 39.84 \\
                          \cline{2-5}
                          & \multirow{2}{*}{ASI} & TSI & \textbf{30.45} & 41.46 \\
                          & & SKIPP'D & \textbf{30.03} & 39.76 \\
    \hline
    \multirow{4}{*}{4hr} & \multirow{2}{*}{TSI} & ASI & \textbf{38.19} & 43.76 \\
                          & & SKIPP'D & \textbf{36.83} & 41.76 \\
                          \cline{2-5}
                          & \multirow{2}{*}{ASI} & TSI & \textbf{36.44} & 45.89 \\
                          & & SKIPP'D & \textbf{36.84} & 44.16 \\
    \hline
  \end{tabular}
\end{table}

\begin{table}[h]
  \caption{
  Forecasting Performance with 4 weeks of training.
  }
  \label{tab:fourweek_forecast}
  \centering
  \setlength{\tabcolsep}{2pt}
  \renewcommand{\arraystretch}{1.2} 
  \begin{tabular}{c c c c c}
    \hline
    \textbf{Interval} & \textbf{Trained on} & \textbf{Tested on} & \textbf{SPIRIT} & \textbf{Gao et al.~\cite{wacv2022}} \\
    \hline
    \multirow{4}{*}{1hr} & \multirow{2}{*}{TSI} & ASI & \textbf{22.17} & 29.03 \\
                          & & SKIPP'D & \textbf{32.44} & 39.82 \\
                          \cline{2-5}
                          & \multirow{2}{*}{ASI} & TSI & \textbf{27.65} & 35.77 \\
                          & & SKIPP'D & \textbf{26.54} & 30.70 \\
    \hline
    \multirow{4}{*}{2hr} & \multirow{2}{*}{TSI} & ASI & \textbf{25.13} & 32.69 \\
                          & & SKIPP'D & \textbf{29.56} & 40.21 \\
                          \cline{2-5}
                          & \multirow{2}{*}{ASI} & TSI & \textbf{31.06} & 36.62 \\
                          & & SKIPP'D & \textbf{25.53} & 33.63 \\
    \hline
    \multirow{4}{*}{3hr} & \multirow{2}{*}{TSI} & ASI & \textbf{30.12} & 38.64 \\
                          & & SKIPP'D & \textbf{31.79} & 40.18 \\
                          \cline{2-5}
                          & \multirow{2}{*}{ASI} & TSI & \textbf{34.47} & 38.76 \\
                          & & SKIPP'D & \textbf{29.73} & 39.70 \\
    \hline
    \multirow{4}{*}{4hr} & \multirow{2}{*}{TSI} & ASI & \textbf{36.14} & 41.92 \\
                          & & SKIPP'D & \textbf{37.24} & 41.31 \\
                          \cline{2-5}
                          & \multirow{2}{*}{ASI} & TSI & \textbf{39.72} & 40.02 \\
                          & & SKIPP'D & \textbf{36.67} & 44.16 \\
    \hline
  \end{tabular}
\end{table}

\begin{table}[h]
  \caption{
  Forecasting Performance with 8 weeks of training.
  }
  \label{tab:eightweek_forecast}
  \centering
  \setlength{\tabcolsep}{2pt}
  \renewcommand{\arraystretch}{1.2} 
  \begin{tabular}{c c c c c}
    \hline
    \textbf{Interval} & \textbf{Trained on} & \textbf{Tested on} & \textbf{SPIRIT} & \textbf{Gao et al.~\cite{wacv2022}} \\
    \hline
    \multirow{4}{*}{1hr} & \multirow{2}{*}{TSI} & ASI & \textbf{22.62} & 32.45 \\
                          & & SKIPP'D & \textbf{33.56} & 36.94 \\
                          \cline{2-5}
                          & \multirow{2}{*}{ASI} & TSI & \textbf{26.38} & 35.70 \\
                          & & SKIPP'D & \textbf{26.61} & 31.25 \\
    \hline
    \multirow{4}{*}{2hr} & \multirow{2}{*}{TSI} & ASI & \textbf{25.15} & 33.58 \\
                          & & SKIPP'D & \textbf{30.65} & 38.06 \\
                          \cline{2-5}
                          & \multirow{2}{*}{ASI} & TSI & \textbf{26.68} & 35.26 \\
                          & & SKIPP'D & \textbf{25.30} & 33.95 \\
    \hline
    \multirow{4}{*}{3hr} & \multirow{2}{*}{TSI} & ASI & \textbf{28.66} & 35.57 \\
                          & & SKIPP'D & \textbf{32.64} & 39.29 \\
                          \cline{2-5}
                          & \multirow{2}{*}{ASI} & TSI & \textbf{29.81} & 36.44 \\
                          & & SKIPP'D & \textbf{29.25} & 39.85 \\
    \hline
    \multirow{4}{*}{4hr} & \multirow{2}{*}{TSI} & ASI & \textbf{34.76} & 39.41 \\
                          & & SKIPP'D & \textbf{37.80} & 41.63 \\
                          \cline{2-5}
                          & \multirow{2}{*}{ASI} & TSI & \textbf{34.97} & 38.23 \\
                          & & SKIPP'D & \textbf{36.23} & 44.25 \\
    \hline
  \end{tabular}
\end{table}

\begin{table}[h]
  \caption{
  Forecasting Performance with 12 weeks of training.
  }
  \label{tab:twelveweek_forecast}
  \centering
  \setlength{\tabcolsep}{2pt}
  \renewcommand{\arraystretch}{1.2} 
  \begin{tabular}{c c c c c}
    \hline
    \textbf{Interval} & \textbf{Trained on} & \textbf{Tested on} & \textbf{SPIRIT} & \textbf{Gao et al.~\cite{wacv2022}} \\
    \hline
    \multirow{4}{*}{1hr} & \multirow{2}{*}{TSI} & ASI & \textbf{22.03} & 34.63 \\
                          & & SKIPP'D & \textbf{33.76} & 37.35 \\
                          \cline{2-5}
                          & \multirow{2}{*}{ASI} & TSI & \textbf{24.87} & 35.24 \\
                          & & SKIPP'D & \textbf{28.28} & 31.20 \\
    \hline
    \multirow{4}{*}{2hr} & \multirow{2}{*}{TSI} & ASI & \textbf{24.95} & 35.81 \\
                          & & SKIPP'D & \textbf{30.38} & 38.16 \\
                          \cline{2-5}
                          & \multirow{2}{*}{ASI} & TSI & \textbf{27.42} & 35.31 \\
                          & & SKIPP'D & \textbf{26.17} & 35.01 \\
    \hline
    \multirow{4}{*}{3hr} & \multirow{2}{*}{TSI} & ASI & \textbf{29.86} & 38.02 \\
                          & & SKIPP'D & \textbf{31.80} & 39.12 \\
                          \cline{2-5}
                          & \multirow{2}{*}{ASI} & TSI & \textbf{30.04} & 36.61 \\
                          & & SKIPP'D & \textbf{29.61} & 41.13 \\
    \hline
    \multirow{4}{*}{4hr} & \multirow{2}{*}{TSI} & ASI & \textbf{34.37} & 41.27 \\
                          & & SKIPP'D & \textbf{36.60} & 41.34 \\
                          \cline{2-5}
                          & \multirow{2}{*}{ASI} & TSI & \textbf{35.71} & 38.67 \\
                          & & SKIPP'D & \textbf{36.16} & 45.28 \\
    \hline
  \end{tabular}
\end{table}

\begin{table}[h]
  \caption{
  Forecasting Performance with 16 weeks of training.
  }
  \label{tab:sixteenweek_forecast}
  \centering
  \setlength{\tabcolsep}{2pt}
  \renewcommand{\arraystretch}{1.2} 
  \begin{tabular}{c c c c c}
    \hline
    \textbf{Interval} & \textbf{Trained on} & \textbf{Tested on} & \textbf{SPIRIT} & \textbf{Gao et al.~\cite{wacv2022}} \\
    \hline
    \multirow{4}{*}{1hr} & \multirow{2}{*}{TSI} & ASI & \textbf{22.76} & 28.97 \\
                          & & SKIPP'D & \textbf{33.12} & 36.93 \\
                          \cline{2-5}
                          & \multirow{2}{*}{ASI} & TSI & \textbf{23.33} & 35.07 \\
                          & & SKIPP'D & \textbf{25.74} & 31.01 \\
    \hline
    \multirow{4}{*}{2hr} & \multirow{2}{*}{TSI} & ASI & \textbf{25.30} & 31.55 \\
                          & & SKIPP'D & \textbf{30.75} & 38.18 \\
                          \cline{2-5}
                          & \multirow{2}{*}{ASI} & TSI & \textbf{27.48} & 36.57 \\
                          & & SKIPP'D & \textbf{25.83} & 32.22 \\
    \hline
    \multirow{4}{*}{3hr} & \multirow{2}{*}{TSI} & ASI & \textbf{28.86} & 36.28 \\
                          & & SKIPP'D & \textbf{33.10} & 39.83 \\
                          \cline{2-5}
                          & \multirow{2}{*}{ASI} & TSI & \textbf{31.92} & 39.46 \\
                          & & SKIPP'D & \textbf{31.04} & 37.69 \\
    \hline
    \multirow{4}{*}{4hr} & \multirow{2}{*}{TSI} & ASI & \textbf{33.99} & 41.36 \\
                          & & SKIPP'D & \textbf{38.20} & 42.66 \\
                          \cline{2-5}
                          & \multirow{2}{*}{ASI} & TSI & \textbf{37.50} & 42.14 \\
                          & & SKIPP'D & \textbf{38.25} & 42.46 \\
    \hline
  \end{tabular}
\end{table}

We conducted a series of experiments to assess the impact of training data size on model performance during fine-tuning. We utilized training splits of 2, 4, 8, 12, and 16 weeks of data at the new site. For each training duration, we performed experiments with different random splits of the corresponding number of weeks and reported the results accordingly.

The results are presented in Tables~\ref{tab:twoweek_forecast}, \ref{tab:fourweek_forecast}, \ref{tab:eightweek_forecast}, \ref{tab:twelveweek_forecast}, and \ref{tab:sixteenweek_forecast}. Figure~\ref{fig:finetune_forecast} was constructed by systematically aggregating the results from our fine-tuning experiments, encapsulating the performance trends observed across different training durations. By leveraging visualization techniques, the figure provides a holistic representation of how the model adapts as more site-specific data becomes available. It effectively summarizes variations in performance across different random splits of training data and across different sets of source and target datasets. 

We employed 95\% confidence intervals for all experiments, spanning diverse transfer learning settings and random sampling of the fine-tuning data. To rigorously compare our method with the baseline across different weekly intervals, we applied a paired t-test at a significance level of 0.001 (i.e., less than a 0.1\% chance of incorrectly rejecting the null hypothesis). In every instance, the observed p-values fell below this threshold, demonstrating that SPIRIT achieves statistically significant performance improvements over the baseline.