%<Rhetorical question - why existing approaches have not taken off.   -Cite why current approaches are not widely deployed. 

%Write complete storyline with experiments - compare assertions with the results. Few shot learning>

The proliferation of solar energy is paramount for electrification and the global energy transition to meet the Net Zero commitments of the United Nations~\cite{sadhukhan2022net}. As the world moves toward renewable sources, solar energy is notable for its accessibility and potential to significantly reduce carbon emissions~\cite{sen2008solar}. Expanding the solar energy infrastructure is crucial to mitigate the effects of climate change~\cite{bashir2021enabling} and meet the energy demands arising from sectors such as data centers~\cite{agarwal2021redesigning}, transportation~\cite{lee2016shared}, and buildings~\cite{iyengar2017cloud}.

Unlike conventional power sources such as thermal and nuclear, solar energy has inherent shortcomings. Its intermittency, due to the daily and seasonal variations in sunlight, poses significant challenges for energy grid stability \cite{grid_management}. One notable issue arising from the higher penetration of solar power is the ``duck curve''~\cite{iyengar2016analyzing}, where the mismatch between solar energy production and peak energy demand leads to significant challenges in grid management. Although storage capacity is increasing, electricity grids typically operate as a \textit{just-in-time} system where energy supply and demand must be balanced~\cite{joskow2012creating}. To ensure grid efficiency, renewable operators must pay a deviation penalty to discourage unplanned energy contributions, thereby maintaining a balanced and predictable energy supply~\cite{yang2020penalty}. Thus, accurate short-term solar predictions are crucial for the efficient operation of the energy grid~\cite{iyengar2014solarcast}.


%Solar forecasting inherently translates to 

%Moreover, grid and transmission operators enforce deviation penalties on solar farm operators~\cite{} as a mechanism to dissuade the solar farm operators to variability in supply --- thereby helping to maintain grid stability and reliability. 

%This is where nowcasting and short-term forecasting play a vital role. Nowcasting refers to the prediction of weather phenomena in the very near future, typically up to a few hours ahead, while short-term forecasting extends this prediction window to a few days. These forecasting techniques are essential for managing the variability and intermittency of solar energy, providing operators with the necessary information to make informed decisions.

%Traditional solar forecasting approaches primarily focus on day-ahead forecasting, relying on weather forecasts and satellite data. While these methods are useful for longer-term predictions, they often lack the granularity required for near-term forecasts. The use of sky cameras, which capture real-time images of the sky with a fish-eye lens, offers a promising alternative for more accurate short-term solar energy forecasting.

%Unfortunetely, existing approaches for short-term forecasting require extensive, site-specific training data~\cite{hammond2024,wacv2022}. With overall solar PV fleet expected to increase from 1 TW in 2022 to 10 TW by 2030~\cite{isa2023}, 9 in 10 solar farms operational by 2030 has not has not yet commissioned. The lack of sufficient historical data makes it difficult to train models from scratch, highlighting the need for innovative solutions that can leverage limited data effectively.

Existing approaches for short-term forecasting use Sky Cameras --- i.e., a fish-eye lens camera positioned to look directly towards the zenith --- which require extensive site-specific data to train models~\cite{hammond2024,wacv2022}. These models have demonstrated the ability to develop high-accuracy models, albeit using training data spanning multiple years. With the overall solar PV fleet expected to increase from 1 TW in 2022 to 10 TW by 2030~\cite{isa2023}, 90\% of the solar farms worldwide will have negligible data to train custom models from scratch. Thus, lack of sufficient site-specific solar data underscores the need for approaches that do not compromise model performance.

With the advent of vision foundation models, we have seen improvement in accuracy of various Computer Vision tasks --- such as feature extraction, object detection, etc. --- using zero-shot and few-shot approaches (i.e., with limited or no custom training data)~\cite{google_vit, object_detection_foundation_models, vits_gt_cnns1}. In addition, physics-inspired feature engineering has significantly improved model performance by incorporating domain-specific knowledge, leading to more accurate and interpretable predictions in real-world problems~\cite{ompusunggu2021physics,erdmann2020physics}. In this work, our hypothesis is as follows: \textit{Can we leverage state-of-the-art vision foundation models and physics-inspired features, along with transfer learning strategies, to reduce the dependence on site-specific sky camera imagery data?}


To address these challenges, we introduce SPIRIT, a novel approach to solar irradiance forecasting with an inductive bias toward enhanced generalizability. In designing, implementing, and evaluating our approach, we make the following contributions:

(1) We develop a novel system that leverages foundation models and physics-informed features, eliminating the need for site-specific model training while enabling effective adaptation across diverse transfer learning scenarios. The flexibility of our framework ensures seamless integration of future advancements in vision models without requiring significant architectural modifications.

(2) Motivated by real-world deployment constraints, we demonstrate that SPIRIT can rapidly scale to new solar plant locations without prior sky camera data, significantly accelerating integration into operational workflows.