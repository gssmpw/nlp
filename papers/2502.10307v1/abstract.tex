% Accurate solar irradiance forecasting is essential for efficient energy grid management and optimizing renewable energy utilization in the face of growing adoption of solar energy worldwide. Conventional deep learning models require several years of extensive location-specific datasets, making adaptation to new solar sites both resource-intensive and time-consuming due to the need for prolonged data collection. In this study, we propose a novel approach leveraging foundation models for solar irradiance forecasting, enabling rapid and efficient deployment in new locations. Our approach outperforms state-of-the-art models in zero-shot transfer learning by about 70\% without requiring any local data. When provided with as little as one week of local data in the new location, our model further enhances its performance, significantly reducing the setup time compared to conventional methods. Even when larger datasets are available, our method achieves performance on par with the state of the art, demonstrating its robustness. These findings represent a pivotal step toward scalable, accessible, and adaptable solar forecasting solutions, advancing the integration of renewable energy into global power systems.



% Training conventional deep learning models for solar irradiance forecasting require several years of  location-specific data, making adaptation to new solar sites both resource-intensive and time-consuming due to the need for prolonged data collection. Accurate forecasting is essential to predict the intermittency of renewables by helping with increased proliferation needed for meeting UN net-zero emissions goal.

% efficient energy grid management and optimizing renewable energy utilization in the face of growing adoption of solar energy worldwide, essential for energy trade management and increased proliferation of renewables to aid energy transition.


% The UN's primary energy transition goal is to achieve a "net-zero emissions" energy system by mid-century, aiming to drastically reduce greenhouse gas emissions by transitioning away from fossil fuels and towards renewable energy sources, while ensuring equitable access to clean energy for all nations, particularly focusing on energy efficiency across all sectors.


% Traditional forecasting models rely on several years of historical site-specific irradiance data, which is often unavailable for newer solar plants. This problem will only get compounded with the increasing penetration of solar energy, crucial for achieving the United Nations' net-zero goals. Accurate forecasting of solar irradiance is essential for mitigating these challenges.
% \hline
% Traditional forecasting models rely on several years of historical site-specific irradiance data, which is unavailable for newer solar plants. As renewable energy is highly intermittent, building accurate solar irradiance forecasting systems is essential for efficient grid management and enabling ongoing proliferation of solar energy crucial to achieve United Nations' net zero goals. In this work, we propose a novel approach leveraging foundation models for solar irradiance forecasting, making it applicable for newer solar installations. Our approach outperforms state-of-the-art models in zero-shot transfer learning by about 70\% without requiring any local data. When provided with as little as one week of local data in the new location, our model further enhances its performance, significantly reducing the setup time compared to conventional methods. Even when larger datasets are available, our method achieves performance on par with the state of the art, demonstrating its robustness. These findings represent a pivotal step toward scalable, accessible, and adaptable solar forecasting solutions, advancing the integration of renewable energy into global power systems.


Traditional solar forecasting models are based on several years of site-specific historical irradiance data, often spanning five or more years, which are unavailable for newer photovoltaic farms. As renewable energy is highly intermittent, building accurate solar irradiance forecasting systems is essential for efficient grid management and enabling the ongoing proliferation of solar energy, which is crucial to achieve the United Nations' net zero goals. In this work, we propose SPIRIT, a novel approach leveraging foundation models for solar irradiance forecasting, making it applicable to newer solar installations. Our approach outperforms state-of-the-art models in zero-shot transfer learning by about 70\%, enabling effective performance at new locations without relying on any historical data.  Further improvements in performance are achieved through fine-tuning, as more location-specific data becomes available. These findings are supported by statistical significance, further validating our approach. SPIRIT represents a pivotal step towards rapid, scalable, and adaptable solar forecasting solutions, advancing the integration of renewable energy into global power systems.

% When provided with as little as one week of local data in the new location, our system further enhances its performance by up to 50\%, significantly reducing the setup time compared to conventional methods. 

% Traditional solar forecasting models are based on several years of site-specific historical irradiance data, often spanning five or more years, which are unavailable for newer photovoltaic farms. As renewable energy is highly intermittent, building accurate solar irradiance forecasting systems is essential for efficient grid management and enabling the ongoing proliferation of solar energy, which is crucial to achieving the United Nations' net zero goals. In this work, we propose a novel approach leveraging foundation models for solar irradiance forecasting, making it applicable to newer solar installations. Our approach outperforms state-of-the-art models in zero-shot transfer learning by about 70\% without requiring local data. When provided with as little as one week of local data in the new location, our model further enhances its performance by up to 50\%, significantly reducing the setup time compared to conventional methods. These findings represent a pivotal step toward rapid, scalable, and adaptable solar forecasting solutions, advancing the integration of renewable energy into global power systems.