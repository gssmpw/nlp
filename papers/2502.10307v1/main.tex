%%
%% This is file `sample-sigconf.tex',
%% generated with the docstrip utility.
%%
%% The original source files were:
%%
%% samples.dtx  (with options: `all,proceedings,bibtex,sigconf')
%% 
%% IMPORTANT NOTICE:
%% 
%% For the copyright see the source file.
%% 
%% Any modified versions of this file must be renamed
%% with new filenames distinct from sample-sigconf.tex.
%% 
%% For distribution of the original source see the terms
%% for copying and modification in the file samples.dtx.
%% 
%% This generated file may be distributed as long as the
%% original source files, as listed above, are part of the
%% same distribution. (The sources need not necessarily be
%% in the same archive or directory.)
%%
%%
%% Commands for TeXCount
%TC:macro \cite [option:text,text]
%TC:macro \citep [option:text,text]
%TC:macro \citet [option:text,text]
%TC:envir table 0 1
%TC:envir table* 0 1
%TC:envir tabular [ignore] word
%TC:envir displaymath 0 word
%TC:envir math 0 word 
%TC:envir comment 0 0
%%
%% The first command in your LaTeX source must be the \documentclass
%% command.
%%
%% For submission and review of your manuscript please change the
%% command to \documentclass[manuscript, screen, review]{acmart}.
%%
%% When submitting camera ready or to TAPS, please change the command
%% to \documentclass[sigconf]{acmart} or whichever template is required
%% for your publication.
%%
%%
\documentclass[sigconf]{acmart}
% \documentclass[sigconf, screen, review]{acmart}
% \documentclass[sigconf,review]{acmart}

%%
%% \BibTeX command to typeset BibTeX logo in the docs
\AtBeginDocument{%
  \providecommand\BibTeX{{%
    Bib\TeX}}}
\let\Bbbk\relax    
\usepackage{graphicx}  
\usepackage{pgfplots}
\pgfplotsset{compat=1.17}
\usepackage{multirow}
\usepackage{float} % Add this in your preamble
\usepackage{amsmath}
\usepackage{amssymb}    % For additional mathematical symbols
\usepackage{mathtools}  % Enhanced math tools, including better \argmin
\usepackage{bm}         % For bold math symbols if needed
\usepackage{booktabs}
% Optional but potentially useful:
\usepackage{mathrsfs}   % For script letters
\usepackage{physics}    
\usepackage{tabularx}
\usepackage{xcolor}
\usepackage{placeins}

\definecolor{mydarkgreen}{rgb}{0.0, 0.5, 0.0}
\definecolor{mydarkred}{rgb}{0.5, 0.0, 0.0}

%% Rights management information.  This information is sent to you
%% when you complete the rights form.  These commands have SAMPLE
%% values in them; it is your responsibility as an author to replace
%% the commands and values with those provided to you when you
%% complete the rights form.
\setcopyright{acmlicensed}
\copyrightyear{2025}
\acmYear{2025}
\acmDOI{XXXXXXX.XXXXXXX}
%% These commands are for a PROCEEDINGS abstract or paper.
\acmConference[KDD'25]{Make sure to enter the correct
  conference title from your rights confirmation email}
%%
%%  Uncomment \acmBooktitle if the title of the proceedings is different
%%  from ``Proceedings of ...''!
%%
%%\acmBooktitle{Woodstock '18: ACM Symposium on Neural Gaze Detection,
%%  June 03--05, 2018, Woodstock, NY}
\acmISBN{978-1-4503-XXXX-X/18/06}


%%
%% Submission ID.
%% Use this when submitting an article to a sponsored event. You'll
%% receive a unique submission ID from the organizers
%% of the event, and this ID should be used as the parameter to this command.
%%\acmSubmissionID{123-A56-BU3}

%%
%% For managing citations, it is recommended to use bibliography
%% files in BibTeX format.
%%
%% You can then either use BibTeX with the ACM-Reference-Format style,
%% or BibLaTeX with the acmnumeric or acmauthoryear sytles, that include
%% support for advanced citation of software artefact from the
%% biblatex-software package, also separately available on CTAN.
%%
%% Look at the sample-*-biblatex.tex files for templates showcasing
%% the biblatex styles.
%%

%%
%% The majority of ACM publications use numbered citations and
%% references.  The command \citestyle{authoryear} switches to the
%% "author year" style.
%%
%% If you are preparing content for an event
%% sponsored by ACM SIGGRAPH, you must use the "author year" style of
%% citations and references.
%% Uncommenting
%% the next command will enable that style.
%%\citestyle{acmauthoryear}


%%
%% end of the preamble, start of the body of the document source.
\begin{document}

%%
%% The "title" command has an optional parameter,
%% allowing the author to define a "short title" to be used in page headers.
\title{SPIRIT: Short-term Prediction of solar IRradIance for zero-shot Transfer learning using Foundation Models}


%%
%% The "author" command and its associated commands are used to define
%% the authors and their affiliations.
%% Of note is the shared affiliation of the first two authors, and the
%% "authornote" and "authornotemark" commands
%% used to denote shared contribution to the research.

\author{Aditya Mishra}
\authornote{Equal contribution.}
\affiliation{%
  \institution{International Institute of Information Technology, Hyderabad}
  % \city{Hyderabad}
  \country{India}
  }
\email{aditya.mishra@students.iiit.ac.in}

% \orcid{1234-5678-9012}

\author{T Ravindra}
\authornotemark[1] 
\affiliation{%
  \institution{International Institute of Information Technology, Hyderabad}
   \country{India}
  % \city{Hyderabad}
  % \country{India}
  }
\email{t.ravindra@students.iiit.ac.in}

\author{Srinivasan Iyengar}
\affiliation{
  \institution{Microsoft Corporation}
   \country{India}
}
\email{sriyengar@microsoft.com}

\author{Shivkumar Kalyanaraman}
\affiliation{%
 \institution{Microsoft Corporation}
 % \city{Doimukh}
 % \state{Arunachal Pradesh}
 \country{India}
 }
 \email{shkalya@microsoft.com}

\author{Ponnurangam Kumaraguru}
\affiliation{%
  \institution{International Institute of Information Technology, Hyderabad}
  \country{India}
  }
  \email{pk.guru@iiit.ac.in}

% \author{Charles Palmer}
% \affiliation{%
%   \institution{Palmer Research Laboratories}
%   \city{San Antonio}
%   \state{Texas}
%   \country{USA}}
% \email{cpalmer@prl.com}

% \author{John Smith}
% \affiliation{%
%   \institution{The Th{\o}rv{\"a}ld Group}
%   \city{Hekla}
%   \country{Iceland}}
% \email{jsmith@affiliation.org}

% \author{Julius P. Kumquat}
% \affiliation{%
%   \institution{The Kumquat Consortium}
%   \city{New York}
%   \country{USA}}
% \email{jpkumquat@consortium.net}

%%
%% By default, the full list of authors will be used in the page
%% headers. Often, this list is too long, and will overlap
%% other information printed in the page headers. This command allows
%% the author to define a more concise list
%% of authors' names for this purpose.
% \renewcommand{\shortauthors}{Trovato et al.}

%%
%% The abstract is a short summary of the work to be presented in the
%% article.
\begin{abstract}
\begin{abstract}  
Test time scaling is currently one of the most active research areas that shows promise after training time scaling has reached its limits.
Deep-thinking (DT) models are a class of recurrent models that can perform easy-to-hard generalization by assigning more compute to harder test samples.
However, due to their inability to determine the complexity of a test sample, DT models have to use a large amount of computation for both easy and hard test samples.
Excessive test time computation is wasteful and can cause the ``overthinking'' problem where more test time computation leads to worse results.
In this paper, we introduce a test time training method for determining the optimal amount of computation needed for each sample during test time.
We also propose Conv-LiGRU, a novel recurrent architecture for efficient and robust visual reasoning. 
Extensive experiments demonstrate that Conv-LiGRU is more stable than DT, effectively mitigates the ``overthinking'' phenomenon, and achieves superior accuracy.
\end{abstract}  
\end{abstract}

%%
%% The code below is generated by the tool at http://dl.acm.org/ccs.cfm.
%% Please copy and paste the code instead of the example below.
%%

\begin{CCSXML}
<ccs2012>
 <concept>
  <concept_id>10010147.10010341.10010349.10010350</concept_id>
  <concept_desc>Computing methodologies~Machine learning</concept_desc>
  <concept_significance>500</concept_significance>
 </concept>
 <concept>
  <concept_id>10010405.10010481.10010487</concept_id>
  <concept_desc>Applied computing~Forecasting</concept_desc>
  <concept_significance>300</concept_significance>
 </concept>
 <concept>
  <concept_id>10010583.10010717.10010721</concept_id>
  <concept_desc>Hardware~Renewable energy</concept_desc>
  <concept_significance>300</concept_significance>
 </concept>
 <concept>
  <concept_id>10010147.10010257.10010293.10010307</concept_id>
  <concept_desc>Computing methodologies~Transfer learning</concept_desc>
  <concept_significance>100</concept_significance>
 </concept>
 <concept>
  <concept_id>10010147.10010257.10010321</concept_id>
  <concept_desc>Computing methodologies~Foundation models</concept_desc>
  <concept_significance>200</concept_significance>
 </concept>
</ccs2012>
\end{CCSXML}

\ccsdesc[500]{Computing methodologies~Machine learning}
\ccsdesc[300]{Applied computing~Forecasting}
\ccsdesc[300]{Hardware~Renewable energy}
\ccsdesc[200]{Computing methodologies~Foundation models}
\ccsdesc[100]{Computing methodologies~Transfer learning}

%%
%% Keywords. The author(s) should pick words that accurately describe
%% the work being presented. Separate the keywords with commas.
\keywords{Solar Forecasting, Renewable Energy, Foundation Models, Transfer Learning, Zero-shot Learning, Fine-tuning, Deep Learning}


\received{10 February 2025}
\maketitle

% crown jewel figure
\begin{figure*}[h!]
    % \flushleft
    \centering
    \includegraphics[width=\textwidth*10/10]{figs/crown_jewel_v6.pdf}
    \caption{
    % Illustration of our method: A foundational vision encoder (top-left) extracts embeddings from a sky camera image from any location or setup. Physics-inspired features are mathematically derived from auxiliary values and then concatenated with them to form a unified auxiliary vector. The image embedding vector is then merged with this auxiliary vector (top-middle) to construct a comprehensive representation vector. For nowcasting (right), a regressor predicts Global Horizontal Irradiance (GHI) from this feature vector. For forecasting (bottom), a time-series model processes a sequence of past feature vectors to create a past context embedding which is concatenated with the future covariate vector, created from future values known in the present, to create the final latent representation which is then mapped to GHI values (bottom-right) corresponding to the future intervals using a regressor.
    Illustration of our system: A vision encoder (top-left) extracts embeddings from a sky camera image sampled from a diverse set spanning multiple locations and setups. Physics-inspired features are derived and integrated with auxiliary values, then merged with the image embedding (top-middle) into a unified representation. For nowcasting (right), a regressor predicts Global Horizontal Irradiance from this feature vector. For forecasting (bottom), a time-series model processes past feature vectors to create a context embedding, which is concatenated with a future covariate vector—constructed from known future values—to form the final latent representation. A regressor then maps this representation to future GHI values (bottom-right).
}  
    \label{fig:crown_jewel}
\end{figure*}



\section{Introduction}
\section{Introduction}
\label{sec:introduction}
The business processes of organizations are experiencing ever-increasing complexity due to the large amount of data, high number of users, and high-tech devices involved \cite{martin2021pmopportunitieschallenges, beerepoot2023biggestbpmproblems}. This complexity may cause business processes to deviate from normal control flow due to unforeseen and disruptive anomalies \cite{adams2023proceddsriftdetection}. These control-flow anomalies manifest as unknown, skipped, and wrongly-ordered activities in the traces of event logs monitored from the execution of business processes \cite{ko2023adsystematicreview}. For the sake of clarity, let us consider an illustrative example of such anomalies. Figure \ref{FP_ANOMALIES} shows a so-called event log footprint, which captures the control flow relations of four activities of a hypothetical event log. In particular, this footprint captures the control-flow relations between activities \texttt{a}, \texttt{b}, \texttt{c} and \texttt{d}. These are the causal ($\rightarrow$) relation, concurrent ($\parallel$) relation, and other ($\#$) relations such as exclusivity or non-local dependency \cite{aalst2022pmhandbook}. In addition, on the right are six traces, of which five exhibit skipped, wrongly-ordered and unknown control-flow anomalies. For example, $\langle$\texttt{a b d}$\rangle$ has a skipped activity, which is \texttt{c}. Because of this skipped activity, the control-flow relation \texttt{b}$\,\#\,$\texttt{d} is violated, since \texttt{d} directly follows \texttt{b} in the anomalous trace.
\begin{figure}[!t]
\centering
\includegraphics[width=0.9\columnwidth]{images/FP_ANOMALIES.png}
\caption{An example event log footprint with six traces, of which five exhibit control-flow anomalies.}
\label{FP_ANOMALIES}
\end{figure}

\subsection{Control-flow anomaly detection}
Control-flow anomaly detection techniques aim to characterize the normal control flow from event logs and verify whether these deviations occur in new event logs \cite{ko2023adsystematicreview}. To develop control-flow anomaly detection techniques, \revision{process mining} has seen widespread adoption owing to process discovery and \revision{conformance checking}. On the one hand, process discovery is a set of algorithms that encode control-flow relations as a set of model elements and constraints according to a given modeling formalism \cite{aalst2022pmhandbook}; hereafter, we refer to the Petri net, a widespread modeling formalism. On the other hand, \revision{conformance checking} is an explainable set of algorithms that allows linking any deviations with the reference Petri net and providing the fitness measure, namely a measure of how much the Petri net fits the new event log \cite{aalst2022pmhandbook}. Many control-flow anomaly detection techniques based on \revision{conformance checking} (hereafter, \revision{conformance checking}-based techniques) use the fitness measure to determine whether an event log is anomalous \cite{bezerra2009pmad, bezerra2013adlogspais, myers2018icsadpm, pecchia2020applicationfailuresanalysispm}. 

The scientific literature also includes many \revision{conformance checking}-independent techniques for control-flow anomaly detection that combine specific types of trace encodings with machine/deep learning \cite{ko2023adsystematicreview, tavares2023pmtraceencoding}. Whereas these techniques are very effective, their explainability is challenging due to both the type of trace encoding employed and the machine/deep learning model used \cite{rawal2022trustworthyaiadvances,li2023explainablead}. Hence, in the following, we focus on the shortcomings of \revision{conformance checking}-based techniques to investigate whether it is possible to support the development of competitive control-flow anomaly detection techniques while maintaining the explainable nature of \revision{conformance checking}.
\begin{figure}[!t]
\centering
\includegraphics[width=\columnwidth]{images/HIGH_LEVEL_VIEW.png}
\caption{A high-level view of the proposed framework for combining \revision{process mining}-based feature extraction with dimensionality reduction for control-flow anomaly detection.}
\label{HIGH_LEVEL_VIEW}
\end{figure}

\subsection{Shortcomings of \revision{conformance checking}-based techniques}
Unfortunately, the detection effectiveness of \revision{conformance checking}-based techniques is affected by noisy data and low-quality Petri nets, which may be due to human errors in the modeling process or representational bias of process discovery algorithms \cite{bezerra2013adlogspais, pecchia2020applicationfailuresanalysispm, aalst2016pm}. Specifically, on the one hand, noisy data may introduce infrequent and deceptive control-flow relations that may result in inconsistent fitness measures, whereas, on the other hand, checking event logs against a low-quality Petri net could lead to an unreliable distribution of fitness measures. Nonetheless, such Petri nets can still be used as references to obtain insightful information for \revision{process mining}-based feature extraction, supporting the development of competitive and explainable \revision{conformance checking}-based techniques for control-flow anomaly detection despite the problems above. For example, a few works outline that token-based \revision{conformance checking} can be used for \revision{process mining}-based feature extraction to build tabular data and develop effective \revision{conformance checking}-based techniques for control-flow anomaly detection \cite{singh2022lapmsh, debenedictis2023dtadiiot}. However, to the best of our knowledge, the scientific literature lacks a structured proposal for \revision{process mining}-based feature extraction using the state-of-the-art \revision{conformance checking} variant, namely alignment-based \revision{conformance checking}.

\subsection{Contributions}
We propose a novel \revision{process mining}-based feature extraction approach with alignment-based \revision{conformance checking}. This variant aligns the deviating control flow with a reference Petri net; the resulting alignment can be inspected to extract additional statistics such as the number of times a given activity caused mismatches \cite{aalst2022pmhandbook}. We integrate this approach into a flexible and explainable framework for developing techniques for control-flow anomaly detection. The framework combines \revision{process mining}-based feature extraction and dimensionality reduction to handle high-dimensional feature sets, achieve detection effectiveness, and support explainability. Notably, in addition to our proposed \revision{process mining}-based feature extraction approach, the framework allows employing other approaches, enabling a fair comparison of multiple \revision{conformance checking}-based and \revision{conformance checking}-independent techniques for control-flow anomaly detection. Figure \ref{HIGH_LEVEL_VIEW} shows a high-level view of the framework. Business processes are monitored, and event logs obtained from the database of information systems. Subsequently, \revision{process mining}-based feature extraction is applied to these event logs and tabular data input to dimensionality reduction to identify control-flow anomalies. We apply several \revision{conformance checking}-based and \revision{conformance checking}-independent framework techniques to publicly available datasets, simulated data of a case study from railways, and real-world data of a case study from healthcare. We show that the framework techniques implementing our approach outperform the baseline \revision{conformance checking}-based techniques while maintaining the explainable nature of \revision{conformance checking}.

In summary, the contributions of this paper are as follows.
\begin{itemize}
    \item{
        A novel \revision{process mining}-based feature extraction approach to support the development of competitive and explainable \revision{conformance checking}-based techniques for control-flow anomaly detection.
    }
    \item{
        A flexible and explainable framework for developing techniques for control-flow anomaly detection using \revision{process mining}-based feature extraction and dimensionality reduction.
    }
    \item{
        Application to synthetic and real-world datasets of several \revision{conformance checking}-based and \revision{conformance checking}-independent framework techniques, evaluating their detection effectiveness and explainability.
    }
\end{itemize}

The rest of the paper is organized as follows.
\begin{itemize}
    \item Section \ref{sec:related_work} reviews the existing techniques for control-flow anomaly detection, categorizing them into \revision{conformance checking}-based and \revision{conformance checking}-independent techniques.
    \item Section \ref{sec:abccfe} provides the preliminaries of \revision{process mining} to establish the notation used throughout the paper, and delves into the details of the proposed \revision{process mining}-based feature extraction approach with alignment-based \revision{conformance checking}.
    \item Section \ref{sec:framework} describes the framework for developing \revision{conformance checking}-based and \revision{conformance checking}-independent techniques for control-flow anomaly detection that combine \revision{process mining}-based feature extraction and dimensionality reduction.
    \item Section \ref{sec:evaluation} presents the experiments conducted with multiple framework and baseline techniques using data from publicly available datasets and case studies.
    \item Section \ref{sec:conclusions} draws the conclusions and presents future work.
\end{itemize}


\section{Related Work}
\section{RELATED WORK}
\label{sec:relatedwork}
In this section, we describe the previous works related to our proposal, which are divided into two parts. In Section~\ref{sec:relatedwork_exoplanet}, we present a review of approaches based on machine learning techniques for the detection of planetary transit signals. Section~\ref{sec:relatedwork_attention} provides an account of the approaches based on attention mechanisms applied in Astronomy.\par

\subsection{Exoplanet detection}
\label{sec:relatedwork_exoplanet}
Machine learning methods have achieved great performance for the automatic selection of exoplanet transit signals. One of the earliest applications of machine learning is a model named Autovetter \citep{MCcauliff}, which is a random forest (RF) model based on characteristics derived from Kepler pipeline statistics to classify exoplanet and false positive signals. Then, other studies emerged that also used supervised learning. \cite{mislis2016sidra} also used a RF, but unlike the work by \citet{MCcauliff}, they used simulated light curves and a box least square \citep[BLS;][]{kovacs2002box}-based periodogram to search for transiting exoplanets. \citet{thompson2015machine} proposed a k-nearest neighbors model for Kepler data to determine if a given signal has similarity to known transits. Unsupervised learning techniques were also applied, such as self-organizing maps (SOM), proposed \citet{armstrong2016transit}; which implements an architecture to segment similar light curves. In the same way, \citet{armstrong2018automatic} developed a combination of supervised and unsupervised learning, including RF and SOM models. In general, these approaches require a previous phase of feature engineering for each light curve. \par

%DL is a modern data-driven technology that automatically extracts characteristics, and that has been successful in classification problems from a variety of application domains. The architecture relies on several layers of NNs of simple interconnected units and uses layers to build increasingly complex and useful features by means of linear and non-linear transformation. This family of models is capable of generating increasingly high-level representations \citep{lecun2015deep}.

The application of DL for exoplanetary signal detection has evolved rapidly in recent years and has become very popular in planetary science.  \citet{pearson2018} and \citet{zucker2018shallow} developed CNN-based algorithms that learn from synthetic data to search for exoplanets. Perhaps one of the most successful applications of the DL models in transit detection was that of \citet{Shallue_2018}; who, in collaboration with Google, proposed a CNN named AstroNet that recognizes exoplanet signals in real data from Kepler. AstroNet uses the training set of labelled TCEs from the Autovetter planet candidate catalog of Q1–Q17 data release 24 (DR24) of the Kepler mission \citep{catanzarite2015autovetter}. AstroNet analyses the data in two views: a ``global view'', and ``local view'' \citep{Shallue_2018}. \par


% The global view shows the characteristics of the light curve over an orbital period, and a local view shows the moment at occurring the transit in detail

%different = space-based

Based on AstroNet, researchers have modified the original AstroNet model to rank candidates from different surveys, specifically for Kepler and TESS missions. \citet{ansdell2018scientific} developed a CNN trained on Kepler data, and included for the first time the information on the centroids, showing that the model improves performance considerably. Then, \citet{osborn2020rapid} and \citet{yu2019identifying} also included the centroids information, but in addition, \citet{osborn2020rapid} included information of the stellar and transit parameters. Finally, \citet{rao2021nigraha} proposed a pipeline that includes a new ``half-phase'' view of the transit signal. This half-phase view represents a transit view with a different time and phase. The purpose of this view is to recover any possible secondary eclipse (the object hiding behind the disk of the primary star).


%last pipeline applies a procedure after the prediction of the model to obtain new candidates, this process is carried out through a series of steps that include the evaluation with Discovery and Validation of Exoplanets (DAVE) \citet{kostov2019discovery} that was adapted for the TESS telescope.\par
%



\subsection{Attention mechanisms in astronomy}
\label{sec:relatedwork_attention}
Despite the remarkable success of attention mechanisms in sequential data, few papers have exploited their advantages in astronomy. In particular, there are no models based on attention mechanisms for detecting planets. Below we present a summary of the main applications of this modeling approach to astronomy, based on two points of view; performance and interpretability of the model.\par
%Attention mechanisms have not yet been explored in all sub-areas of astronomy. However, recent works show a successful application of the mechanism.
%performance

The application of attention mechanisms has shown improvements in the performance of some regression and classification tasks compared to previous approaches. One of the first implementations of the attention mechanism was to find gravitational lenses proposed by \citet{thuruthipilly2021finding}. They designed 21 self-attention-based encoder models, where each model was trained separately with 18,000 simulated images, demonstrating that the model based on the Transformer has a better performance and uses fewer trainable parameters compared to CNN. A novel application was proposed by \citet{lin2021galaxy} for the morphological classification of galaxies, who used an architecture derived from the Transformer, named Vision Transformer (VIT) \citep{dosovitskiy2020image}. \citet{lin2021galaxy} demonstrated competitive results compared to CNNs. Another application with successful results was proposed by \citet{zerveas2021transformer}; which first proposed a transformer-based framework for learning unsupervised representations of multivariate time series. Their methodology takes advantage of unlabeled data to train an encoder and extract dense vector representations of time series. Subsequently, they evaluate the model for regression and classification tasks, demonstrating better performance than other state-of-the-art supervised methods, even with data sets with limited samples.

%interpretation
Regarding the interpretability of the model, a recent contribution that analyses the attention maps was presented by \citet{bowles20212}, which explored the use of group-equivariant self-attention for radio astronomy classification. Compared to other approaches, this model analysed the attention maps of the predictions and showed that the mechanism extracts the brightest spots and jets of the radio source more clearly. This indicates that attention maps for prediction interpretation could help experts see patterns that the human eye often misses. \par

In the field of variable stars, \citet{allam2021paying} employed the mechanism for classifying multivariate time series in variable stars. And additionally, \citet{allam2021paying} showed that the activation weights are accommodated according to the variation in brightness of the star, achieving a more interpretable model. And finally, related to the TESS telescope, \citet{morvan2022don} proposed a model that removes the noise from the light curves through the distribution of attention weights. \citet{morvan2022don} showed that the use of the attention mechanism is excellent for removing noise and outliers in time series datasets compared with other approaches. In addition, the use of attention maps allowed them to show the representations learned from the model. \par

Recent attention mechanism approaches in astronomy demonstrate comparable results with earlier approaches, such as CNNs. At the same time, they offer interpretability of their results, which allows a post-prediction analysis. \par





\section{SPIRIT Design}
\section{\projecttitle Library}
\label{sec:abstraction}
\label{sec:recipe-implementation}
%\dimitra{system level design details here and the low-level API!}




\subsection{\projecttitle{} Architecture Overview} \label{subsec:overview}
\myparagraph{Distributed systems architecture} 
A distributed data store system uses a tiered architecture with a distributed data store layer for routing requests, a replication layer (provided by \projecttitle{}) for consistent data replication, and a data layer (Key-Value stores) for storage.  \projecttitle{} provides a trusted computing base using trusted execution environments (TEEs) to protect consensus fault tolerance protocols, including secure initialization of replicas and a direct I/O layer for efficient, secure communication.


Figure~\ref{fig:overview} shows the overview of a distributed data store system that builds on top of the \projecttitle{} system.  Distributed data stores implement a tiered architecture consisting of a \emph{distributed data store layer}, \emph{replication layer}, and  \emph{data layer}. In our case, the replication and data layers are provided by \projecttitle{}. The distributed data store layer maintains a routing table that matches the keyspace with the owners' nodes. This layer is responsible for forwarding client requests to the appropriate coordinator nodes (e.g., leader of the replication protocol) for execution. The \projecttitle{} replication layer is responsible for consistently replicating the data by executing the implemented protocol. After the protocol execution, \projecttitle{} nodes store the data in their Key-Value stores (KVs), the data layer, and they reply to the client~\cite{redis, rocksdb, leveldb, memcached2004}.

\myparagraph{\projecttitle{} architecture} \projecttitle{} design is based on a distributed setting of TEEs that implement a (distributed) trusted computing base (TCB) and shield the execution of unmodified CFT protocols against Byzantine failures. \projecttitle{}'s TCB contains the CFT protocol's code along with some metadata specific to the protocol. 

The code and TEEs of all replicas are attested before instantiating the protocol to ensure that the TEE hardware and the residing code are genuine. All authenticated replicas receive secrets (e.g., signing or encryption keys) and configuration data securely at initialization. 

Further, \projecttitle{} builds a \emph{direct I/O layer} comprised of a networking library for low-latency communication between nodes ($\S$~\ref{subsec:networkin}). The library bypasses the kernel stack for performance and shields the communication to guarantee non-equivocation and transferable authentication against Byzantine actors in the network. \projecttitle{} guarantees both properties by layering the non-equivocation and authentication layers on top of the direct I/O layer. In addition,  to strengthen \projecttitle{}'s security properties and eliminate syscalls, we map the network library software stack to the TEE's address space.

Lastly, \projecttitle{} builds the  \emph{data layer} on top of local KV store instances. Our design of the KV store increases the trust in individual nodes, allowing for local reads ($\S$~\ref{subsec:KV}). Our KV store achieves two goals: first, we guarantee trust to individual replicas to serve reads locally, and second, we limit the TCB size, optimizing the enclave memory usage. As shown in Figure~\ref{fig:overview}, \projecttitle{} keeps bulk data (values) in the host memory and stores only minimal data (keys + metadata) in the TEE area. The metadata, e.g., hash of the value, timestamps, etc., are kept along with keys in the TEE for integrity verification. 



Our work shows how to leverage modern hardware to build efficient, robust, and easily adaptable distributed protocols by meeting the aforementioned transformation requirements.
%(see Q1---Q3 below). Motivated by the recently launched cloud-hosted blockchain systems, we also argue that confidential BFT protocols are required to satisfy modern applications' needs for confidentiality (see Q4 below).
To achieve our goal, we need to address the following technical questions discussed in~$\S$~\ref{sec:motivation}.
Next, we present the implementation details or our work focusing on four core components of \projecttitle{} ($\S$~\ref{sec:recipe_impl_apis}). Table~\ref{tab:api} summarizes the \projecttitle{}'s API for each component. 

\begin{figure*}[t]
    \begin{center}
        \includegraphics[width=1\textwidth]{figs/recipe_full_system.pdf}
        %\vspace{-2pt}
        \caption{\projecttitle{}'s system architecture.}
       % \vspace{-1pt}
        \label{fig:overview}
    \end{center}
\end{figure*}


\section{Bellman Error Centering}

Centering operator $\mathcal{C}$ for a variable $x(s)$ is defined as follows:
\begin{equation}
\mathcal{C}x(s)\dot{=} x(s)-\mathbb{E}[x(s)]=x(s)-\sum_s{d_{s}x(s)},
\end{equation} 
where $d_s$ is the probability of $s$.
In vector form,
\begin{equation}
\begin{split}
\mathcal{C}\bm{x} &= \bm{x}-\mathbb{E}[x]\bm{1}\\
&=\bm{x}-\bm{x}^{\top}\bm{d}\bm{1},
\end{split}
\end{equation} 
where $\bm{1}$ is an all-ones vector.
For any vector $\bm{x}$ and $\bm{y}$ with a same distribution $\bm{d}$,
we have
\begin{equation}
\begin{split}
\mathcal{C}(\bm{x}+\bm{y})&=(\bm{x}+\bm{y})-(\bm{x}+\bm{y})^{\top}\bm{d}\bm{1}\\
&=\bm{x}-\bm{x}^{\top}\bm{d}\bm{1}+\bm{y}-\bm{y}^{\top}\bm{d}\bm{1}\\
&=\mathcal{C}\bm{x}+\mathcal{C}\bm{y}.
\end{split}
\end{equation}
\subsection{Revisit Reward Centering}


The update (\ref{src3}) is an unbiased estimate of the average reward
with  appropriate learning rate $\beta_t$ conditions.
\begin{equation}
\bar{r}_{t}\approx \lim_{n\rightarrow\infty}\frac{1}{n}\sum_{t=1}^n\mathbb{E}_{\pi}[r_t].
\end{equation}
That is 
\begin{equation}
r_t-\bar{r}_{t}\approx r_t-\lim_{n\rightarrow\infty}\frac{1}{n}\sum_{t=1}^n\mathbb{E}_{\pi}[r_t]= \mathcal{C}r_t.
\end{equation}
Then, the simple reward centering can be rewrited as:
\begin{equation}
V_{t+1}(s_t)=V_{t}(s_t)+\alpha_t [\mathcal{C}r_{t+1}+\gamma V_{t}(s_{t+1})-V_t(s_t)].
\end{equation}
Therefore, the simple reward centering is, in a strict sense, reward centering.

By definition of $\bar{\delta}_t=\delta_t-\bar{r}_{t}$,
let rewrite the update rule of the value-based reward centering as follows:
\begin{equation}
V_{t+1}(s_t)=V_{t}(s_t)+\alpha_t \rho_t (\delta_t-\bar{r}_{t}),
\end{equation}
where $\bar{r}_{t}$ is updated as:
\begin{equation}
\bar{r}_{t+1}=\bar{r}_{t}+\beta_t \rho_t(\delta_t-\bar{r}_{t}).
\label{vrc3}
\end{equation}
The update (\ref{vrc3}) is an unbiased estimate of the TD error
with  appropriate learning rate $\beta_t$ conditions.
\begin{equation}
\bar{r}_{t}\approx \mathbb{E}_{\pi}[\delta_t].
\end{equation}
That is 
\begin{equation}
\delta_t-\bar{r}_{t}\approx \mathcal{C}\delta_t.
\end{equation}
Then, the value-based reward centering can be rewrited as:
\begin{equation}
V_{t+1}(s_t)=V_{t}(s_t)+\alpha_t \rho_t \mathcal{C}\delta_t.
\label{tdcentering}
\end{equation}
Therefore, the value-based reward centering is no more,
 in a strict sense, reward centering.
It is, in a strict sense, \textbf{Bellman error centering}.

It is worth noting that this understanding is crucial, 
as designing new algorithms requires leveraging this concept.


\subsection{On the Fixpoint Solution}

The update rule (\ref{tdcentering}) is a stochastic approximation
of the following update:
\begin{equation}
\begin{split}
V_{t+1}&=V_{t}+\alpha_t [\bm{\mathcal{T}}^{\pi}\bm{V}-\bm{V}-\mathbb{E}[\delta]\bm{1}]\\
&=V_{t}+\alpha_t [\bm{\mathcal{T}}^{\pi}\bm{V}-\bm{V}-(\bm{\mathcal{T}}^{\pi}\bm{V}-\bm{V})^{\top}\bm{d}_{\pi}\bm{1}]\\
&=V_{t}+\alpha_t [\mathcal{C}(\bm{\mathcal{T}}^{\pi}\bm{V}-\bm{V})].
\end{split}
\label{tdcenteringVector}
\end{equation}
If update rule (\ref{tdcenteringVector}) converges, it is expected that
$\mathcal{C}(\mathcal{T}^{\pi}V-V)=\bm{0}$.
That is 
\begin{equation}
    \begin{split}
    \mathcal{C}\bm{V} &= \mathcal{C}\bm{\mathcal{T}}^{\pi}\bm{V} \\
    &= \mathcal{C}(\bm{R}^{\pi} + \gamma \mathbb{P}^{\pi} \bm{V}) \\
    &= \mathcal{C}\bm{R}^{\pi} + \gamma \mathcal{C}\mathbb{P}^{\pi} \bm{V} \\
    &= \mathcal{C}\bm{R}^{\pi} + \gamma (\mathbb{P}^{\pi} \bm{V} - (\mathbb{P}^{\pi} \bm{V})^{\top} \bm{d_{\pi}} \bm{1}) \\
    &= \mathcal{C}\bm{R}^{\pi} + \gamma (\mathbb{P}^{\pi} \bm{V} - \bm{V}^{\top} (\mathbb{P}^{\pi})^{\top} \bm{d_{\pi}} \bm{1}) \\  % 修正双重上标
    &= \mathcal{C}\bm{R}^{\pi} + \gamma (\mathbb{P}^{\pi} \bm{V} - \bm{V}^{\top} \bm{d_{\pi}} \bm{1}) \\
    &= \mathcal{C}\bm{R}^{\pi} + \gamma (\mathbb{P}^{\pi} \bm{V} - \bm{V}^{\top} \bm{d_{\pi}} \mathbb{P}^{\pi} \bm{1}) \\
    &= \mathcal{C}\bm{R}^{\pi} + \gamma (\mathbb{P}^{\pi} \bm{V} - \mathbb{P}^{\pi} \bm{V}^{\top} \bm{d_{\pi}} \bm{1}) \\
    &= \mathcal{C}\bm{R}^{\pi} + \gamma \mathbb{P}^{\pi} (\bm{V} - \bm{V}^{\top} \bm{d_{\pi}} \bm{1}) \\
    &= \mathcal{C}\bm{R}^{\pi} + \gamma \mathbb{P}^{\pi} \mathcal{C}\bm{V} \\
    &\dot{=} \bm{\mathcal{T}}_c^{\pi} \mathcal{C}\bm{V},
    \end{split}
    \label{centeredfixpoint}
    \end{equation}
where we defined $\bm{\mathcal{T}}_c^{\pi}$ as a centered Bellman operator.
We call equation (\ref{centeredfixpoint}) as centered Bellman equation.
And it is \textbf{centered fixpoint}.

For linear value function approximation, let define
\begin{equation}
\mathcal{C}\bm{V}_{\bm{\theta}}=\bm{\Pi}\bm{\mathcal{T}}_c^{\pi}\mathcal{C}\bm{V}_{\bm{\theta}}.
\label{centeredTDfixpoint}
\end{equation}
We call equation (\ref{centeredTDfixpoint}) as \textbf{centered TD fixpoint}.

\subsection{On-policy and Off-policy Centered TD Algorithms
with Linear Value Function Approximation}
Given the above centered TD fixpoint,
 mean squared centered Bellman error (MSCBE), is proposed as follows:
\begin{align*}
    \label{argminMSBEC}
 &\arg \min_{{\bm{\theta}}}\text{MSCBE}({\bm{\theta}}) \\
 &= \arg \min_{{\bm{\theta}}} \|\bm{\mathcal{T}}_c^{\pi}\mathcal{C}\bm{V}_{\bm{{\bm{\theta}}}}-\mathcal{C}\bm{V}_{\bm{{\bm{\theta}}}}\|_{\bm{D}}^2\notag\\
 &=\arg \min_{{\bm{\theta}}} \|\bm{\mathcal{T}}^{\pi}\bm{V}_{\bm{{\bm{\theta}}}} - \bm{V}_{\bm{{\bm{\theta}}}}-(\bm{\mathcal{T}}^{\pi}\bm{V}_{\bm{{\bm{\theta}}}} - \bm{V}_{\bm{{\bm{\theta}}}})^{\top}\bm{d}\bm{1}\|_{\bm{D}}^2\notag\\
 &=\arg \min_{{\bm{\theta}},\omega} \| \bm{\mathcal{T}}^{\pi}\bm{V}_{\bm{{\bm{\theta}}}} - \bm{V}_{\bm{{\bm{\theta}}}}-\omega\bm{1} \|_{\bm{D}}^2\notag,
\end{align*}
where $\omega$ is is used to estimate the expected value of the Bellman error.
% where $\omega$ is used to estimate $\mathbb{E}[\delta]$, $\omega \doteq \mathbb{E}[\mathbb{E}[\delta_t|S_t]]=\mathbb{E}[\delta]$ and $\delta_t$ is the TD error as follows:
% \begin{equation}
% \delta_t = r_{t+1}+\gamma
% {\bm{\theta}}_t^{\top}\bm{{\bm{\phi}}}_{t+1}-{\bm{\theta}}_t^{\top}\bm{{\bm{\phi}}}_t.
% \label{delta}
% \end{equation}
% $\mathbb{E}[\delta_t|S_t]$ is the Bellman error, and $\mathbb{E}[\mathbb{E}[\delta_t|S_t]]$ represents the expected value of the Bellman error.
% If $X$ is a random variable and $\mathbb{E}[X]$ is its expected value, then $X-\mathbb{E}[X]$ represents the centered form of $X$. 
% Therefore, we refer to $\mathbb{E}[\delta_t|S_t]-\mathbb{E}[\mathbb{E}[\delta_t|S_t]]$ as Bellman error centering and 
% $\mathbb{E}[(\mathbb{E}[\delta_t|S_t]-\mathbb{E}[\mathbb{E}[\delta_t|S_t]])^2]$ represents the the mean squared centered Bellman error, namely MSCBE.
% The meaning of (\ref{argminMSBEC}) is to minimize the mean squared centered Bellman error.
%The derivation of CTD is as follows.

First, the parameter  $\omega$ is derived directly based on
stochastic gradient descent:
\begin{equation}
\omega_{t+1}= \omega_{t}+\beta_t(\delta_t-\omega_t).
\label{omega}
\end{equation}

Then, based on stochastic semi-gradient descent, the update of 
the parameter ${\bm{\theta}}$ is as follows:
\begin{equation}
{\bm{\theta}}_{t+1}=
{\bm{\theta}}_{t}+\alpha_t(\delta_t-\omega_t)\bm{{\bm{\phi}}}_t.
\label{theta}
\end{equation}

We call (\ref{omega}) and (\ref{theta}) the on-policy centered
TD (CTD) algorithm. The convergence analysis with be given in
the following section.

In off-policy learning, we can simply multiply by the importance sampling
 $\rho$.
\begin{equation}
    \omega_{t+1}=\omega_{t}+\beta_t\rho_t(\delta_t-\omega_t),
    \label{omegawithrho}
\end{equation}
\begin{equation}
    {\bm{\theta}}_{t+1}=
    {\bm{\theta}}_{t}+\alpha_t\rho_t(\delta_t-\omega_t)\bm{{\bm{\phi}}}_t.
    \label{thetawithrho}
\end{equation}

We call (\ref{omegawithrho}) and (\ref{thetawithrho}) the off-policy centered
TD (CTD) algorithm.

% By substituting $\delta_t$ into Equations (\ref{omegawithrho}) and (\ref{thetawithrho}), 
% we can see that Equations (\ref{thetawithrho}) and (\ref{omegawithrho}) are formally identical 
% to the linear expressions of Equations (\ref{rewardcentering1}) and (\ref{rewardcentering2}), respectively. However, the meanings 
% of the corresponding parameters are entirely different.
% ${\bm{\theta}}_t$ is for approximating the discounted value function.
% $\bar{r_t}$ is an estimate of the average reward, while $\omega_t$ 
% is an estimate of the expected value of the Bellman error.
% $\bar{\delta_t}$ is the TD error for value-based reward centering, 
% whereas $\delta_t$ is the traditional TD error.

% This study posits that the CTD is equivalent to value-based reward 
% centering. However, CTD can be unified under a single framework 
% through an objective function, MSCBE, which also lays the 
% foundation for proving the algorithm's convergence. 
% Section 4 demonstrates that the CTD algorithm guarantees 
% convergence in the on-policy setting.

\subsection{Off-policy Centered TDC Algorithm with Linear Value Function Approximation}
The convergence of the  off-policy centered TD algorithm
may not be guaranteed.

To deal with this problem, we propose another new objective function, 
called mean squared projected centered Bellman error (MSPCBE), 
and derive Centered TDC algorithm (CTDC).

% We first establish some relationships between
%  the vector-matrix quantities and the relevant statistical expectation terms:
% \begin{align*}
%     &\mathbb{E}[(\delta({\bm{\theta}})-\mathbb{E}[\delta({\bm{\theta}})]){\bm{\phi}}] \\
%     &= \sum_s \mu(s) {\bm{\phi}}(s) \big( R(s) + \gamma \sum_{s'} P_{ss'} V_{\bm{\theta}}(s') - V_{\bm{\theta}}(s)  \\
%     &\quad \quad-\sum_s \mu(s)(R(s) + \gamma \sum_{s'} P_{ss'} V_{\bm{\theta}}(s') - V_{\bm{\theta}}(s))\big)\\
%     &= \bm{\Phi}^\top \mathbf{D} (\bm{TV}_{\bm{{\bm{\theta}}}} - \bm{V}_{\bm{{\bm{\theta}}}}-\omega\bm{1}),
% \end{align*}
% where $\omega$ is the expected value of the Bellman error and $\bm{1}$ is all-ones vector.

The specific expression of the objective function 
MSPCBE is as follows:
\begin{align}
    \label{MSPBECwithomega}
    &\arg \min_{{\bm{\theta}}}\text{MSPCBE}({\bm{\theta}})\notag\\ 
    % &= \arg \min_{{\bm{\theta}}}\big(\mathbb{E}[(\delta({\bm{\theta}}) - \mathbb{E}[\delta({\bm{\theta}})]) \bm{{\bm{\phi}}}]^\top \notag\\
    % &\quad \quad \quad\mathbb{E}[\bm{{\bm{\phi}}} \bm{{\bm{\phi}}}^\top]^{-1} \mathbb{E}[(\delta({\bm{\theta}}) - \mathbb{E}[\delta({\bm{\theta}})]) \bm{{\bm{\phi}}}]\big) \notag\\
    % &=\arg \min_{{\bm{\theta}},\omega}\mathbb{E}[(\delta({\bm{\theta}})-\omega) \bm{\bm{{\bm{\phi}}}}]^{\top} \mathbb{E}[\bm{\bm{{\bm{\phi}}}} \bm{\bm{{\bm{\phi}}}}^{\top}]^{-1}\mathbb{E}[(\delta({\bm{\theta}}) -\omega)\bm{\bm{{\bm{\phi}}}}]\\
    % &= \big(\bm{\Phi}^\top \mathbf{D} (\bm{TV}_{\bm{{\bm{\theta}}}} - \bm{V}_{\bm{{\bm{\theta}}}}-\omega\bm{1})\big)^\top (\bm{\Phi}^\top \mathbf{D} \bm{\Phi})^{-1} \notag\\
    % & \quad \quad \quad \bm{\Phi}^\top \mathbf{D} (\bm{TV}_{\bm{{\bm{\theta}}}} - \bm{V}_{\bm{{\bm{\theta}}}}-\omega\bm{1}) \notag\\
    % &= (\bm{TV}_{\bm{{\bm{\theta}}}} - \bm{V}_{\bm{{\bm{\theta}}}}-\omega\bm{1})^\top \mathbf{D} \bm{\Phi} (\bm{\Phi}^\top \mathbf{D} \bm{\Phi})^{-1} \notag\\
    % &\quad \quad \quad \bm{\Phi}^\top \mathbf{D} (\bm{TV}_{\bm{{\bm{\theta}}}} - \bm{V}_{\bm{{\bm{\theta}}}}-\omega\bm{1})\notag\\
    % &= (\bm{TV}_{\bm{{\bm{\theta}}}} - \bm{V}_{\bm{{\bm{\theta}}}}-\omega\bm{1})^\top {\bm{\Pi}}^\top \mathbf{D} {\bm{\Pi}} (\bm{TV}_{\bm{{\bm{\theta}}}} - \bm{V}_{\bm{{\bm{\theta}}}}-\omega\bm{1}) \notag\\
    &= \arg \min_{{\bm{\theta}}} \|\bm{\Pi}\bm{\mathcal{T}}_c^{\pi}\mathcal{C}\bm{V}_{\bm{{\bm{\theta}}}}-\mathcal{C}\bm{V}_{\bm{{\bm{\theta}}}}\|_{\bm{D}}^2\notag\\
    &= \arg \min_{{\bm{\theta}}} \|\bm{\Pi}(\bm{\mathcal{T}}_c^{\pi}\mathcal{C}\bm{V}_{\bm{{\bm{\theta}}}}-\mathcal{C}\bm{V}_{\bm{{\bm{\theta}}}})\|_{\bm{D}}^2\notag\\
    &= \arg \min_{{\bm{\theta}},\omega}\| {\bm{\Pi}} (\bm{\mathcal{T}}^{\pi}\bm{V}_{\bm{{\bm{\theta}}}} - \bm{V}_{\bm{{\bm{\theta}}}}-\omega\bm{1}) \|_{\bm{D}}^2\notag.
\end{align}
In the process of computing the gradient of the MSPCBE with respect to ${\bm{\theta}}$, 
$\omega$ is treated as a constant.
So, the derivation process of CTDC is the same 
as for the TDC algorithm \cite{sutton2009fast}, the only difference is that the original $\delta$ is replaced by $\delta-\omega$.
Therefore, the updated formulas of the centered TDC  algorithm are as follows:
\begin{equation}
 \bm{{\bm{\theta}}}_{k+1}=\bm{{\bm{\theta}}}_{k}+\alpha_{k}[(\delta_{k}- \omega_k) \bm{\bm{{\bm{\phi}}}}_k\\
 - \gamma\bm{\bm{{\bm{\phi}}}}_{k+1}(\bm{\bm{{\bm{\phi}}}}^{\top}_k \bm{u}_{k})],
\label{thetavmtdc}
\end{equation}
\begin{equation}
 \bm{u}_{k+1}= \bm{u}_{k}+\zeta_{k}[\delta_{k}-\omega_k - \bm{\bm{{\bm{\phi}}}}^{\top}_k \bm{u}_{k}]\bm{\bm{{\bm{\phi}}}}_k,
\label{uvmtdc}
\end{equation}
and
\begin{equation}
 \omega_{k+1}= \omega_{k}+\beta_k (\delta_k- \omega_k).
 \label{omegavmtdc}
\end{equation}
This algorithm is derived to work 
with a given set of sub-samples—in the form of 
triples $(S_k, R_k, S'_k)$ that match transitions 
from both the behavior and target policies. 

% \subsection{Variance Minimization ETD Learning: VMETD}
% Based on the off-policy TD algorithm, a scalar, $F$,  
% is introduced to obtain the ETD algorithm, 
% which ensures convergence under off-policy 
% conditions. This paper further introduces a scalar, 
% $\omega$, based on the ETD algorithm to obtain VMETD.
% VMETD by the following update:
% \begin{equation}
% \label{fvmetd}
%  F_t \leftarrow \gamma \rho_{t-1}F_{t-1}+1,
% \end{equation}
% \begin{equation}
%  \label{thetavmetd}
%  {{\bm{\theta}}}_{t+1}\leftarrow {{\bm{\theta}}}_t+\alpha_t (F_t \rho_t\delta_t - \omega_{t}){\bm{{\bm{\phi}}}}_t,
% \end{equation}
% \begin{equation}
%  \label{omegavmetd}
%  \omega_{t+1} \leftarrow \omega_t+\beta_t(F_t  \rho_t \delta_t - \omega_t),
% \end{equation}
% where $\rho_t =\frac{\pi(A_t | S_t)}{\mu(A_t | S_t)}$ and $\omega$ is used to estimate $\mathbb{E}[F \rho\delta]$, i.e., $\omega \doteq \mathbb{E}[F \rho\delta]$.

% (\ref{thetavmetd}) can be rewritten as
% \begin{equation*}
%  \begin{array}{ccl}
%  {{\bm{\theta}}}_{t+1}&\leftarrow& {{\bm{\theta}}}_t+\alpha_t (F_t \rho_t\delta_t - \omega_t){\bm{{\bm{\phi}}}}_t -\alpha_t \omega_{t+1}{\bm{{\bm{\phi}}}}_t\\
%   &=&{{\bm{\theta}}}_{t}+\alpha_t(F_t\rho_t\delta_t-\mathbb{E}_{\mu}[F_t\rho_t\delta_t|{{\bm{\theta}}}_t]){\bm{{\bm{\phi}}}}_t\\
%  &=&{{\bm{\theta}}}_t+\alpha_t F_t \rho_t (r_{t+1}+\gamma {{\bm{\theta}}}_t^{\top}{\bm{{\bm{\phi}}}}_{t+1}-{{\bm{\theta}}}_t^{\top}{\bm{{\bm{\phi}}}}_t){\bm{{\bm{\phi}}}}_t\\
%  & & \hspace{2em} -\alpha_t \mathbb{E}_{\mu}[F_t \rho_t \delta_t]{\bm{{\bm{\phi}}}}_t\\
%  &=& {{\bm{\theta}}}_t+\alpha_t \{\underbrace{(F_t\rho_tr_{t+1}-\mathbb{E}_{\mu}[F_t\rho_t r_{t+1}]){\bm{{\bm{\phi}}}}_t}_{{b}_{\text{VMETD},t}}\\
%  &&\hspace{-7em}- \underbrace{(F_t\rho_t{\bm{{\bm{\phi}}}}_t({\bm{{\bm{\phi}}}}_t-\gamma{\bm{{\bm{\phi}}}}_{t+1})^{\top}-{\bm{{\bm{\phi}}}}_t\mathbb{E}_{\mu}[F_t\rho_t ({\bm{{\bm{\phi}}}}_t-\gamma{\bm{{\bm{\phi}}}}_{t+1})]^{\top})}_{\textbf{A}_{\text{VMETD},t}}{{\bm{\theta}}}_t\}.
%  \end{array}
% \end{equation*}
% Therefore, 
% \begin{equation*}
%  \begin{array}{ccl}
%   &&\textbf{A}_{\text{VMETD}}\\
%   &=&\lim_{t \rightarrow \infty} \mathbb{E}[\textbf{A}_{\text{VMETD},t}]\\
%   &=& \lim_{t \rightarrow \infty} \mathbb{E}_{\mu}[F_t \rho_t {\bm{{\bm{\phi}}}}_t ({\bm{{\bm{\phi}}}}_t - \gamma {\bm{{\bm{\phi}}}}_{t+1})^{\top}]\\  
%   &&\hspace{1em}- \lim_{t\rightarrow \infty} \mathbb{E}_{\mu}[  {\bm{{\bm{\phi}}}}_t]\mathbb{E}_{\mu}[F_t \rho_t ({\bm{{\bm{\phi}}}}_t - \gamma {\bm{{\bm{\phi}}}}_{t+1})]^{\top}\\
%   &=& \lim_{t \rightarrow \infty} \mathbb{E}_{\mu}[{\bm{{\bm{\phi}}}}_tF_t \rho_t ({\bm{{\bm{\phi}}}}_t - \gamma {\bm{{\bm{\phi}}}}_{t+1})^{\top}]\\   
%   &&\hspace{1em}-\lim_{t \rightarrow \infty} \mathbb{E}_{\mu}[ {\bm{{\bm{\phi}}}}_t]\lim_{t \rightarrow \infty}\mathbb{E}_{\mu}[F_t \rho_t ({\bm{{\bm{\phi}}}}_t - \gamma {\bm{{\bm{\phi}}}}_{t+1})]^{\top}\\
%   && \hspace{-2em}=\sum_{s} d_{\mu}(s)\lim_{t \rightarrow \infty}\mathbb{E}_{\mu}[F_t|S_t = s]\mathbb{E}_{\mu}[\rho_t\bm{{\bm{\phi}}}_t(\bm{{\bm{\phi}}}_t - \gamma \bm{{\bm{\phi}}}_{t+1})^{\top}|S_t= s]\\   
%   &&\hspace{1em}-\sum_{s} d_{\mu}(s)\bm{{\bm{\phi}}}(s)\sum_{s} d_{\mu}(s)\lim_{t \rightarrow \infty}\mathbb{E}_{\mu}[F_t|S_t = s]\\
%   &&\hspace{7em}\mathbb{E}_{\mu}[\rho_t(\bm{{\bm{\phi}}}_t - \gamma \bm{{\bm{\phi}}}_{t+1})^{\top}|S_t = s]\\
%   &=& \sum_{s} f(s)\mathbb{E}_{\pi}[\bm{{\bm{\phi}}}_t(\bm{{\bm{\phi}}}_t- \gamma \bm{{\bm{\phi}}}_{t+1})^{\top}|S_t = s]\\   
%   &&\hspace{1em}-\sum_{s} d_{\mu}(s)\bm{{\bm{\phi}}}(s)\sum_{s} f(s)\mathbb{E}_{\pi}[(\bm{{\bm{\phi}}}_t- \gamma \bm{{\bm{\phi}}}_{t+1})^{\top}|S_t = s]\\
%   &=&\sum_{s} f(s) \bm{\bm{{\bm{\phi}}}}(s)(\bm{\bm{{\bm{\phi}}}}(s) - \gamma \sum_{s'}[\textbf{P}_{\pi}]_{ss'}\bm{\bm{{\bm{\phi}}}}(s'))^{\top}  \\
%   &&-\sum_{s} d_{\mu}(s) {\bm{{\bm{\phi}}}}(s) * \sum_{s} f(s)({\bm{{\bm{\phi}}}}(s) - \gamma \sum_{s'}[\textbf{P}_{\pi}]_{ss'}{\bm{{\bm{\phi}}}}(s'))^{\top}\\
%   &=&{\bm{\bm{\Phi}}}^{\top} \textbf{F} (\textbf{I} - \gamma \textbf{P}_{\pi}) \bm{\bm{\Phi}} - {\bm{\bm{\Phi}}}^{\top} {d}_{\mu} {f}^{\top} (\textbf{I} - \gamma \textbf{P}_{\pi}) \bm{\bm{\Phi}}  \\
%   &=&{\bm{\bm{\Phi}}}^{\top} (\textbf{F} - {d}_{\mu} {f}^{\top}) (\textbf{I} - \gamma \textbf{P}_{\pi}){\bm{\bm{\Phi}}} \\
%   &=&{\bm{\bm{\Phi}}}^{\top} (\textbf{F} (\textbf{I} - \gamma \textbf{P}_{\pi})-{d}_{\mu} {f}^{\top} (\textbf{I} - \gamma \textbf{P}_{\pi})){\bm{\bm{\Phi}}} \\
%   &=&{\bm{\bm{\Phi}}}^{\top} (\textbf{F} (\textbf{I} - \gamma \textbf{P}_{\pi})-{d}_{\mu} {d}_{\mu}^{\top} ){\bm{\bm{\Phi}}},
%  \end{array}
% \end{equation*}
% \begin{equation*}
%  \begin{array}{ccl}
%   &&{b}_{\text{VMETD}}\\
%   &=&\lim_{t \rightarrow \infty} \mathbb{E}[{b}_{\text{VMETD},t}]\\
%   &=& \lim_{t \rightarrow \infty} \mathbb{E}_{\mu}[F_t\rho_tR_{t+1}{\bm{{\bm{\phi}}}}_t]\\
%   &&\hspace{2em} - \lim_{t\rightarrow \infty} \mathbb{E}_{\mu}[{\bm{{\bm{\phi}}}}_t]\mathbb{E}_{\mu}[F_t\rho_kR_{k+1}]\\  
%   &=& \lim_{t \rightarrow \infty} \mathbb{E}_{\mu}[{\bm{{\bm{\phi}}}}_tF_t\rho_tr_{t+1}]\\
%   &&\hspace{2em} - \lim_{t\rightarrow \infty} \mathbb{E}_{\mu}[  {\bm{{\bm{\phi}}}}_t]\mathbb{E}_{\mu}[{\bm{{\bm{\phi}}}}_t]\mathbb{E}_{\mu}[F_t\rho_tr_{t+1}]\\ 
%   &=& \lim_{t \rightarrow \infty} \mathbb{E}_{\mu}[{\bm{{\bm{\phi}}}}_tF_t\rho_tr_{t+1}]\\
%   &&\hspace{2em} - \lim_{t \rightarrow \infty} \mathbb{E}_{\mu}[ {\bm{{\bm{\phi}}}}_t]\lim_{t \rightarrow \infty}\mathbb{E}_{\mu}[F_t\rho_tr_{t+1}]\\  
%   &=&\sum_{s} f(s) {\bm{{\bm{\phi}}}}(s)r_{\pi} - \sum_{s} d_{\mu}(s) {\bm{{\bm{\phi}}}}(s) * \sum_{s} f(s)r_{\pi}  \\
%   &=&\bm{\bm{\bm{\Phi}}}^{\top}(\textbf{F}-{d}_{\mu} {f}^{\top}){r}_{\pi}.
%  \end{array}
% \end{equation*}



\if 0
This section describes four core components of \projecttitle{}. Table~\ref{tab:api} summarizes the \projecttitle{}'s API for each component.%  (a) networking library, (b) KV store, (c) secure runtime, and (d) attestation and configuration management.


%(\projecttitle{}-lib): \emph{(i)} the shielded networking library which leverages direct I/O while also preventing Byzantine behaviors in the untrusted network infrastructure, \emph{(ii)} the KV store which guarantees trust to local reads and, \emph{(iii)} the attestation and secrets distribution service which ensures that only trusted nodes know the configuration, keys, etc.


%\pramod{fix missing citations and a lot of typos and grammar errors.}


\subsection{\projecttitle Implementation and APIs}
\label{sec:recipe_impl_apis}
\myparagraph{\projecttitle{} networking}
\label{subsec:networkin}
%\myparagraph{Networking overview}
\projecttitle{} adopts the Remote Procedure Call (RPC) paradigm~\cite{286500} over a generic network library with various transportation layers (Infiniband, RoCE, and DPDK), which is also favorable in the context of TEEs where traditional kernel-based networking is impractical~\cite{kuvaiskii2017sgxbounds}. %Below, we explain how the networking layer is initialized in \projecttitle{}, the requests workflow and the core implementation details.



\begin{table}[t]
%\small
\fontsize{7}{9}\selectfont 

\begin{center}
\begin{tabular}{ |c|c| }
 \hline
 \bf{Attestation API} &  \\ \hline
 \multirow{1}{*}{\texttt{attest(measurement)}} & Attests the node based on  a measurement.  \\  \hline \hline
 \bf{Initialization API} &  \\ \hline
 \texttt{create\_rpc(app\_ctx)} & Initializes an RPCobj. \\
  \texttt{init\_store()} & Initializes the KV store. \\
  \texttt{reg\_hdlr(\&func)} & Registers request handlers. \\ \hline \hline
 \bf{Network API} &  \\ \hline
 \texttt{send(\&msg\_buf)} & Prepares a req for transmission. \\
 %\hline
% \texttt{multicast(\&msg\_buf, nodes)} & Prepares a request for multicast. \\
 \multirow{1}{*}{\texttt{respond(\&msg\_buf)}} & Prepares a resp for transmission. \\
 \texttt{poll()} & Polls for incoming messages. \\\hline \hline
% \texttt{aggregates\_multicast()} &  \\ 
 \bf{Security API} &  \\ \hline
 \texttt{verify\_msg(\&msg\_buf)} & Verifies the authenticity/integrity and cnt of a msg. \\
 \texttt{shield\_msg(\&msg\_buf)} & Generates a shielded msg. \\ \hline \hline
% \texttt{aggregates\_multicast()} &  \\ 
 \bf{KV Store API} &  \\ \hline
 \texttt{write(key, value)} & Writes a KV to the store. \\
 \hline
 \multirow{2}{*}{\texttt{get(key, \&v$_{TEE}$)}} & Reads the value into \texttt{v$_{TEE}$} \\ & and verifies integrity. \\ \hline %\hline
 \if 0
 \bf{Trusted Leases API} &  \\ \hline
 \texttt{init\_lease(node\_id, thread\_id)} & Requests a lease from the grander.\\ \hline
 \texttt{renew\_lease(\&lease)} & Updates a lease.\\ \hline
 \texttt{grand\_or\_update\_lease(node\_id, thread\_id)} & Grands a lease.\\ \hline
 \texttt{exec\_with\_lease(\&lease, \&func, \&args\_list)} & Executes the func within the lease ownership.\\ [1ex] \hline
 \fi
\end{tabular}
\end{center}
%\vspace{-10pt}
\caption{\projecttitle{} library APIs.} \label{tab:api}
\vspace{-6pt}
\end{table}





\if 0

Developer effort – initialization. The developer must spec-
ify the number and the nature of the logical message flows
they require. In RDMA parlance each flow corresponds to
one queue pair (QP), i.e., a send and a receive queue. For
instance, consider Hermes where a write requires two broad-
cast rounds: invalidations (invs) and validations (vals). Each
worker in each node sets up three QPs: 1) to send and re-
ceive invs, 2) to send and receive acks (for the invs) and 3) to
send and receive vals. Splitting the communication in mes-
sage flows is the responsibility of the developer. To create
the QP for each message flow, the developer simply calls a
Odyssey function, passing details about the nature of the QP.

\fi





% \myparagraph{Initialization}
% Prior to application's execution, developers need to initialize the networking layer by specifying the number of concurrent available connections, the types of the available requests and by registering the appropriate (custom) request handlers. In \projecttitle{} terms, a communication endpoint corresponds to a per-thread RPC object (RPCobj) with private send/receive queues. All RPCobjs are registered to the same physical port (configurable). Initially, \projecttitle{} creates a handle to the NIC which is passed to all RPCobjs. Developers need to define the types of the RPC requests, each of which might be served by a different request handler. Request handlers are functions written by developers that are registered with the handle prior to the creation of the communication endpoints. Lastly, before executing the application's code, the connections between RPCobjs need to be correctly established.

\if 0
\projecttitle{} offers a \texttt{create\_rpc()} function that creates Remote Procedure Call (RPC) objects (rpc) bound to the NIC. Specifically this function takes the application context as an argument, i.e., node's NIC specification and port, remote IP and port, creates a communication endpoint and continuously tries to establish connection with the remote side. The function returns after the connection establishment. An rpc offers bidirectional communication between the two sides. Additionally, we need to register the request handler functions to the rpcs, i.e., pass a pointer function a the construction of the endpoint which states what will happen when a request of a specific type is received. The developer might to overwrite/implement the \texttt{init\_store()} function which will keep an application's state and metadata in the trusted enclave. By default \projecttitle{} comes with a thread-safe and lock-free hybrid skiplist based on~\cite{avocado, folly}. While implementing our use cases in $\$$~\ref{sec:eval}, we used two  \projecttitle{} skiplists for metadata and data accordingly.  %Lastly, we need to register the request handler functions to the \texttt{rpc}s, i.e., pass a pointer function a the construction of the endpoint which states what will happen when a request of a specific type is received.
\fi

\if 0
Developer effort – send and receive. For each QP, Odys-
sey maintains a send-FIFO and a receive-FIFO. Sending re-
quires that the developer first inserts messages in the send-
FIFO via an Odyssey insert function; later they can call a send
function to trigger the sending of all inserted messages. To re-
ceive messages, the developer need only call an Odyssey func-
tion that polls the receive-FIFO. Notably, the developer can
specify and register handlers to be called when calling any
one of the Odyssey functions. Therefore, the Odyssey polling
function will deliver the incoming messages, if any, to the
developer-specified handler.
\fi





%\myparagraph{send/receive operations}
We offer asynchronous network operations following the RPC paradigm. For each RPCobj, \projecttitle{} keeps a transmission (TX) and reception (RX) queue, organized as ring buffers. Developers enqueue requests and responses to requests via \projecttitle{}'s specific functions which place the message in the RPCobj's TX queue. Later, they can call a polling function that flushes the messages in the TX and drains the RX queues of an RPCobj. The function will trigger the sending of all queued messages and process all received requests and responses. Reception of a request triggers the execution of the request handler for that specific type. Reception of a response to a request triggers a cleanup function that releases all resources allocated for the request, e.g., message buffers and rate limiters (for congestion). %The cleanup functions can be overwritten by the developers for extra functionalities.

\if 0
\projecttitle{} offers high performance RPCs by extending eRPC~\cite{erpc} and DPDK~\cite{dpdk} in the context of TEEs. eRPC is .. Specifically, we place the message buffers outside the trusted enclave to both overcome the limited enclave memory and enable DMA operations~\footnote{DMA mappings are prohibited in the trusted area of a TEE as this violates their security properties~\cite{intel-sgx}}. We design a \texttt{send()} operation is used to submit a message for transmission. The message buffer is allocated by our library in \texttt{Hugepage} memory area and is later copied to the transmission queue (TX). Further, we provide a multicast() operation which creates identical copies of a message for all the recipient group.    Upon a reception of a request, the program control passes to the registered request handler where the function \texttt{respond()} can submit a response or \texttt{ACK} to that request. Lastly, the function \texttt{poll()} needs to be called regularly to fetch and process and send the incoming responses or requests and send the queued responses and requests respectively. 
\fi




%\myparagraph{Non-equivocation and authentication layers} %\manos{changed the paragraph titles to match the title and the first sentence below}
\label{non-equivocation-design}
%extends the properties of conventional CFT protocols to tolerate Byzantine settings by 
\projecttitle{}'s networking library embodies a non-equivocation and an authentication layer through two TEE-assisted primitives, the shield\_request() and verify\_request().%, shown in Algorithm~\ref{algo:primitives}.% (the \texttt{Attest()} primitive is used for initialization and is discussed in~$\S$~\ref{subsec:attestation}). We now explain the mechanisms, correctness arguments are presented in $\S$~\ref{sec:theory}.



\noindent{\underline{Non-equivocation layer}}: \projecttitle{} prevents replay attacks in the network with sequence numbers for the exchanged messages. Each replica maintains local sequence tuples of the form (view, cq, cnt$_{cq}$) where view is the current view number, cq is the communication endpoint(s) between two nodes, and cnt$_{cq}$ is the current trusted counter value in that view for the latest request sent over the cq. The sender assigns to messages a unique tuple of the form (view, cq, cnt$_{cq}$) and then increments cnt$_{cq}$ to guarantee monotonicity. % and rollback/forking attacks resilience.  %the request with the correct tuple and increments the $seq$.
Replicas execute the implemented CFT protocol for verified valid requests. Replicas can verify the freshness of a message by examining its cnt$_{cq}$ (verify\_request() primitive). The primitive verifies that the message's id (as part of the metadata) is consistent with the receiver's local counter rcnt$_{cq}$ (rcnt$_{cq}$ is the last seen valid message counter for received messages in cq). \projecttitle{}'s replicas are willing to accept ``future'' valid messages as these might come out of order, i.e., messages whose cnt$_{cq}$ is $>$ (rcnt$_{cq}$+1). These messages are processed and committed according to the CFT protocol. %~\footnote{The attestation that takes place at the TEE setup ($\S$~\ref{subsec:attestation}) ensures that only trusted nodes are capable of generating valid messages.}

\noindent{\underline{Authentication layer}}: For the authentication, we use cryptographic primitives (e.g., MAC and encryption functions when \projecttitle{} aims for confidentiality) to verify the integrity and the authenticity of the messages. Each message $m$ sent from a node $n_i$ to a node $n_j$ over a communication channel cq is accompanied by metadata (e.g., cnt$_{cq}$, view, sender and receiver nodes id) and the calculated message authentication code (MAC) $h_{cq}_{\sigma}_q$. The MAC is calculated over the payload and the metadata, then follows the message $m$. The sender node calls into the shield\_request(req, cq) and generates such a trusted message for the request req. %The trusted message is of the form [(req, (ReplicaView, cq, cnt\_{cq})), ${h_{cq}_{\sigma_{cq}})}$$>$.% containing the encrypted metadata and hash of the $req$ and the $req$ payload. %This function will marshal the current value of its trusted monotonic counter, the current view number, and the (cryptographic) hash of the message into one string. 

%\myparagraph{Non-equivocation} \projecttitle{} limits the equivocation of Byzantine (malicious) faults in the networking infrastructure using TEEs. Specifically, we guarantee non-equivocation via trusted counter assignment and verification. Each replica maintains a local sequence tuple $(v, cq, seq)$ where $v$ is the current view number, $cq$ is the communication (pair) channel between two nodes and $seq$ is the
%current trusted counter value in that view for the latest committed request sent in that communication channel. Each request is assigned a unique tuple $(v, cq, seq)$ which is maintained by the TEE of each replica to guarantee monotonic increments and rollback/forking attacks resilience. The coordinator node of a request assigns the request with the correct tuple and increments the $seq$.  Once a replica receives a request, they only accept it after its verification. The accepted requests pass through the underlying CFT protocol. Replicas verify the received requests using the \texttt{VerifyCounter}$(<req, (v, cq, seq)>, {h_{cq}})$ function. Specifically, the replica verifies the freshness of the message/request by examining its counter id. The message passes through the non-equivocation layer to verify that the counter associated with the received request (as part of the message metadata) is consistent with its local counter. Replicas in \projecttitle{} are willing to accept ``future'' valid messages as these might come out of order, i.e., messages whose seq number is $> (seq'_{cq}+1)$ ($seq'_{cq}$ is the last seen request number from that communication channel). Such messages are valid so \projecttitle{} accepts them. However, they are processed and committed when the underlying CFT protocol allows that.

%\dimitra{fix this}
%\myparagraph{Integrity verification} In \projecttitle{} we leverage basic primitives in modern cryptography such as hash functions to check and verify the integrity of the data the might reside in the untrusted areas including, the host memory and the network infrastructure. Each message $m$ sent from $n_i$ to $n_j$ over a communication channel $cq$ is accompanied by its calculated hash $h_{cq}$ that allows the recipient $n_j$ to verify that the message payload is genuine. A node that drives the client's request (coordinator) before sending the request to replicas need to call \texttt{ShieldRequest}$(req, cq)$ to generate an integrity-protected message for that request. This function will marshal the current value of its trusted monotonic counter, the current view number, and the (cryptographic) hash of the message into one string. The output is a bytestream of the form $<req, (\texttt{(ReplicaView}, cq, cnt_{cq})>, {h_{cq})}$ containing the metadata, the request payload and the computed hash of metadata and payload.

\if 0
\myparagraph{API} We offer a create\_rpc() function that creates a bound-to-the-NIC RPCobj. The function takes as an argument the application context, i.e., NIC specification and port, remote IP and port, creates a communication endpoint and establishes connection with the remote side. The function returns after the connection establishment. RPCobjs offer bidirectional communication between the two sides. Prior to the creation of RPCobj, developers need to specify and register the request types and handlers using the reg\_hdlr() which takes as an argument a reference to the preferred handler function. %The developer might to overwrite/implement the \texttt{init\_store()} function which will keep an application's state and metadata in the trusted enclave. By default \projecttitle{} comes with a thread-safe and lock-free hybrid skiplist based on~\cite{avocado, folly}. While implementing our use cases in $\$$~\ref{sec:eval}, we used two  \projecttitle{} skiplists for metadata and data accordingly.  %Lastly, we need to register the request handler functions to the \texttt{rpc}s, i.e., pass a pointer function a the construction of the endpoint which states what will happen when a request of a specific type is received.

For exchanging network messages, we designed a send() function which takes as arguments the session (connection) identifier, the message buffer to be sent, the request type and the cleanup function. This function submits a message for transmission. Upon a reception of a request, the program control passes to the registered request handler where the function respond() can submit a response or ACK to that request. Lastly, the function poll() needs to be called regularly to fetch or transmit the network messages in the TX and RX queues.




\fi 
\begin{comment}
~\footnote{DMA mappings are prohibited in the trusted area of a TEE as this violates TEE's security properties~\cite{intel-sgx, avocado, treaty}}
\end{comment}

\if 0
\myparagraph{Implementation details}
We designed \projecttitle{}'s high performance RPCs by extending eRPC~\cite{erpc} in the context of TEEs. We place the message buffers outside the enclave to overcome the limited enclave memory and enable DMA operations~\cite{intel-sgx, avocado, treaty}. The message buffers are allocated in Hugepage area and are later copied or mapped to the TX/RX queues. The networking buffers residing outside the TEE follow the trusted message format we discussed in \ref{subsec:overview}. As such, while outside the trusted area, their integrity (or confidentiality) can be verified upon reception.

We also adopted a rate limiter which can be configured to limit read and/or write requests. The use of a rate limiter was found to be useful in protocols which vastly saturated the available host memory (we found out that the R-ABD protocol was quickly exhausting all available memory in the system while the tail-node in the R-CR protocol also ran out-of-memory for read-heavy workloads). Lastly, we implement on top of \projecttitle{}'s network library a batching technique which queues the message buffers and merges them into a bigger buffer before transmission. The batching factor is configurable and has been proven extremely efficient for small messages (e.g., \SI{256}{\byte}).

\fi 


\myparagraph{Secure runtime} We build our codebase in C++ using \scone{} to access the TEE hardware. \scone{} exposes a modified libc library and combines user-level threading and asynchronous syscalls~\cite{flexsc} to reduce the cost of syscall execution. While we limit the number of syscalls, leveraging  \scone{}'s exit-less approach allows us to optimize the initialization phase that vastly allocates host memory for the network stack and the KV store. To enable NIC's DMA operations and memory mappings to the hugepages (for message buffers and TX/RX queues) ($\S$~\ref{subsec:networkin}), we overwrite the \texttt{mmap()} syscall of \scone{} to bypass its shield layer and allow the allocation of (untrusted) host memory. 

%For the cryptographic primitives, we build on OpenSSL~\cite{openssl}. Lastly, we build on a lease mechanism~\cite{t-lease} in \scone{} for auxiliary operations, e.g., failures detection and leader's election.

\if 0
\begin{algorithm}
\SetAlgoLined
%\fontsize
\small
%\fontsize{9}{10}\selectfont 

%\texttt{$seq'_{cq}$ the last committed sequence for that cq} \\

$\triangleright$ cnt$_{cq}$: the latest sent message id from  cq\\$\triangleright$ rcnt$_{cq}$: the last committed message id from cq


%\vspace{0.1cm}

\textbf{function} shield\_request(req, cq) \{ \\
\Indp
cnt$_{cq}$ $\leftarrow$ cnt$_{cq}$+1; t$\leftarrow$ (view, cq, cnt$_{cq}$);\\
$[$$h_{\sigma_{cq}}$, (req,t)$]$  $\leftarrow$ singed\_hash(req, t);\\
\textbf{return} $[$$h_{\sigma_{cq}}$, (req,t)$]$;\\
\Indm
\} \\



%\vspace{0.1cm}
\textbf{function} verify\_request($h_{\sigma_{cq}}$, req, (view, cq, cnt$_{cq}$)) \{ \\
\Indp
    \textbf{if} verify\_signature($h_{\sigma_{cq}}$, req, (view, cq, cnt$_{cq}$)) == True \textbf{then}\\
    \Indp
        \textbf{if} view == current\_view \textbf{then}\\
        \Indp
            \textbf{if} cnt$_{cq}$ <= rcnt$_{cq}$ \textbf{then}\\
            \Indp
                \textbf{return} [False, req, (view, cq, cnt$_{cq}$)]; \\
            \Indm
            \textbf{if} cnt$_{cq}$ == rcnt$_{cq}$+1) \textbf{then} rcnt$_{cq}$ $\leftarrow$ rcnt$_{cq}$+1;
            buffer\_locally(req, (view, cq, cnt$_{cq}$));\\
                \textbf{return} [True, req, (view, cq, cnt$_{cq}$)]; \\
            
        \Indm
    \Indm
    \textbf{return} [False, req, (view, cq, cnt$_{cq}$)]; \\

\Indm
\} \\
%\vspace{0.1cm}
\vspace{-1pt}
\caption{\projecttitle{}'s authentication primitives.}
\vspace{-3pt}
\label{algo:primitives}
\end{algorithm}

\fi


\myparagraph{\projecttitle{} key-value store}
\label{subsec:KV}
\projecttitle{} provides a lock-free, high-performant KV store based on a skip-list. We partition the keys from the values' space by placing the keys along with metadata (and a pointer to the value in host memory) inside the TEE's memory area, the {\em enclave}, and storing the values in the host memory. %Our partitioned KVs reduce the number of calculations for integrity checks, compared to prior work~\cite{shieldstore}, which implements (per-bucket) Merkle trees and re-calculates the root on each update. Importantly, separating the (keys + metadata) and the values between the enclave and untrusted unlimited memory decreases the Enclave Page Cache (EPC) pressure~\cite{speicher-fast}. %Our lock-free data structure supports concurrent operations and it is well-suited
%for increased parallelism.
\projecttitle{}'s KV store design resolves Byzantine errors since the metadata (and the code that accesses them) reside in the enclave. That said, \projecttitle{} allows for local reads as nodes can verify the integrity of the stored values.

\if 0
\myparagraph{Implementation and API} The developer might want to overwrite/implement the init\_store() function which will keep an application's state and metadata in the trusted enclave. \projecttitle{} implements its hybrid skiplist based on folly library~\cite{folly}. The write() function updates the KV while the get() function copies the value of the given key in the protected area. The function also verifies the value's integrity. We implement an allocator for host memory that is given as an initialization parameter to the KV store.

\fi 
%\myparagraph{Data properties}
 %Our partitioned scheme {\em seamlessly} strengthens the system's security properties further and can offer confidentiality by encrypting the values outside the TEE. \projecttitle{}-transformed protocols that further offer confidentiality outperform the BFT systems ($\S$~\ref{sec:eval}).


%that offers parallel write and read operations, however the developer might wish to overwrite those functions. Both operations calculate the hash of the given data which is placed in the enclave memory. The hash is used to verify the integrity of the data stored in the host memory (if any).




\myparagraph{Attestation and secrets distribution}~\label{subsec:attestation}
Remote attestation is the building block to verify the authenticity of a TEE, i.e., the code and the TEE state are the expected~\cite{Parno2010}. As such, \projecttitle{} provides attest(), generate\_quote() and remote\_attestation() primitives  that allow replicas to prove their trustworthiness to other replicas or clients. The attestation takes place before the control passes to the protocol's code. Only successfully attested nodes get access to secrets (e.g., signing or encryption keys, etc.) and configurations. 

%Essentially, \projecttitle{} needs to \emph{(1)} offer low-latency attestation of the joiner nodes (for fast recovery) and \emph{(2)} securely distribute the secrets and configuration data. \projecttitle{}'s attestation shields against Sybil attacks~\cite{sybilAttack}.

\if 0
\begin{algorithm}
\SetAlgoLined
%\fontsize
\small
%\fontsize{9}{10}\selectfont 

\textbf{function} remote\_attestation() \{ \\
 \Indp
 nonce $\leftarrow$ generate\_nonce();\\
 \textbf{send}(nonce, k$_{pub}$); \textbf{DHKE}(); quote$_{\sigma_{k_{pub}}}$ $\leftarrow$ \textbf{recv}();\\
 \textbf{if} verify\_signature(quote$_{\sigma_{k_{pub}}}$) == True \textbf{then}\\
    \Indp
        $\mu_{TEE}$ $\leftarrow$ decrypt(quote$_{\sigma_{k_{pub}}}$, k$_{priv}$);\\
        \textbf{if} (verify\_quote$(\mu_{TEE})$ == True) send\_secrets();\\
    \Indm
 \Indm
\} \\


%\vspace{0.1cm}
\textbf{function} attest() \{ \\
\Indp
    $\mu$ $\leftarrow$ gen\_enclave\_report(); \textbf{return} $\mu$;\\ 
\Indm
\} \\

%\vspace{0.1cm}
\textbf{function} generate\_quote($\mu$, k$_{pub}$) \{ \\
\Indp
    key$_{hw}$ $\leftarrow$ EGETKEY();\\
    quote $\leftarrow$ sign($\mu$, key$_{hw}$); 
    quote$_{\sigma_{k_{pub}}}$ $\leftarrow$ sign(quote, k$_{pub}$);\\
    \textbf{return } quote$_{\sigma_{k_{pub}}}$;\\
\Indm
\} \\
\caption{\projecttitle{}'s attestation primitive.}
\label{algo:attestation}
\vspace{-3pt}
\end{algorithm}
\fi 

%\vspace{-8pt}



%\myparagraph{Attestation design} 

%The application and the challenger first establish communication and, then, the application asks the challenger to provision secrets. 
The attestation process is initialized by the \emph{challenger}, a remote process that can verify the authenticity of a specific TEE. The challenger executes the remote\_attestation() function to send an attestation request to the application---usually in the form a nonce (a random number). The challenger and the application, then, pass through a Diffie-Hellman key exchange process~\cite{10.1145/359460.359473}. The application generates an ephemeral public key which is used by the challenger later to provision any secrets.

%When the TEE receives the nonce, it calls the attest() and generates a \emph{measurement} ($\mu$) of its state and loaded code. Following this, the TEE calls into the generate\_quote$(\mu, k_{pub})$ to sign $\mu$ (quote) with the $key_{hw}$ which is fetched from the TEE's h/w. The TEE signs and encrypts the quote quote$_{\sigma_{k_{pub}}}$ over the challenger's public key $k_{pub}$ which is, then, sent back to the challenger. Upon successful verifications of the quote$_{\sigma_{k_{pub}}}$, the challenger shares secrets and configurations.

%To offer low-latency attestations within the same datacenter that \projecttitle{} runs, we build a Configuration and Attestation service (CAS). The Protocol Designer (PD) deploys the CAS inside a TEE and attests it through the hardware vendor's attestation service---e.g., Intel Attestation Service (IAS~\cite{ias}). Once the CAS is attested, it is trusted and the PB can upload secrets and configurations. 

%The challenger asserts upon a failed verification and denies to share any secret or configuration data. Otherwise, it distributes all necessary shared secrets.

\if 0 
\myparagraph{Implementation details} 
%\projecttitle{} builds on top of a Configuration and Attestation Service (CAS) that is shipped with \scone{} and ultimately relies on hardware-based TEEs for secure secrets and configuration management. 
A trusted entity, i.e., the developer, must deploy the  Configuration and Attestation Service  CAS inside a TEE in a node that can also be part of the membership. Afterward, they need to attest the CAS through the TEE's attestation service---in our case, Intel Attestation Service (IAS~\cite{ias}). Once the CAS is attested, it can replace the TEE's attestation service. The trusted entity needs to spawn further Local Attestation Services (LASes) that perform local (intra-platform) attestation of processes and offer low-latency attestation of the new nodes. Before the CFT protocol commences, the CAS must also attest all LASes. 

When a \projecttitle{} process is loaded in the TEE (before the control passes to the application code), the LAS instructs a local attestation. The attestee generates a quote of the enclave (measurement). The LAS forwards the enclave's measurement to the CAS, which replies with a success or failure message indicating the authenticity of the process. After a successful attestation, the CAS stores the node's IP and provides the trusted process with secrets and configurations.

\myparagraph{API}
\projecttitle{} provides an attestation API to developers. Particularly, we provide the attest function that takes as arguments the IP of a trusted third-party service, CAS's IP for \projecttitle{}, and a generated enclave measurement of the code. Then, this service verifies that both the enclave signer and measurement are in the expected state and replies accordingly.
\fi 
%\dimitra{fix this}
%\myparagraph{Attestation and secrets management} Attestation is the process of demonstrating that the correct software is securely running within a TEE-enclave on an enabled platform. Secrets (e.g., certificates, encryption keys, etc.) and configurations (membership IPs and information) should only be provided to a replica after its successful attestation. Once an enclave is initialized and before the control is relinquished to the application's code, the attestation process is launched to verify the integrity and authenticity of the included code and data. Essentially, \projecttitle{} needs to (1.) offer low-latency attestation of the joiner nodes and (2.) securely provide the trusted enclave applications with secrets and configuration data (usually over the network).

%\projecttitle{} leverages TEEs to implement \texttt{RemoteAttestation}$()$, \texttt{Attest}$()$ and \texttt{GenerateQuote}$()$ primitives that allow a third party to attest an enclave. Next we describe the workflow of the attestation and the properties of those primitives.

%The attestation process is initialized by the challenger (or verifier)---a remote process (typically provided by the hardware vendor) that can prove the authenticity of a specific enclave. The application and the challenger first establish communication and, then, the application asks the challenger to provision secrets. The challenger executes the \texttt{RemoteAttestation}$()$ function where it first replies with an attestation request to the application---usually in the form a nonce (a random number
%generated for this one occasion). The challenger and the application also pass through a Diffie-Hellman key exchange (DHKE) process. The application generates an ephemeral public key which is used by the challenger later for provisioning secrets to the enclave.

%After receiving the nonce, the enclave calls into the \texttt{Attest}$()$ and generates a \emph{measurement} or \emph{report} that includes information about the enclave. The report needs to be sent to the challenger for verification. The enclave calls into \texttt{GenerateQuote$(\mu, k_{public})$} which sings the measurement $\mu$ over $key_{hw}$, where $key_{hw}$ is fetched from the TEE's hardware. Additionally, the function encrypts the quote over challenger's public key $k_{public}$. The encrypted quote can be sent and verified by the challenger with the corresponding verification key.

%The challenger asserts upon a failed verification and denies to share any secret or configuration data. Otherwise, it distributes all necessary shared secrets.

\if 0

\subsection{\projecttitle{} API}

\if 0
\subsection{\projecttitle{} client API}
\projecttitle{} exposes a simple \texttt{PUT}/\texttt{GET} API to clients. Both functions take as arguments the coordinator node's id, the view and the leader identifier (if any) that are known to the client. The API assigns a unique monotonically increased id to every request executed by the client. That is to help CFT distinguish already executed requests.
\fi

%\subsection{\projecttitle{} system-level API}
Table~\ref{tab:api} presents the core library that \projecttitle{} exposes to the developers. We implement our \projecttitle{} on top of \scone~\cite{arnautov2016scone} and \textsc{Palaemon}~\cite{palaemon} that use Intel SGX~\cite{intel-sgx} as the TEE and we extend eRPC~\cite{erpc} on top of DPDK~\cite{dpdk} for fast networking. Next we discuss the use-cases and the implementation details for each core function of \projecttitle{}.

\myparagraph{Attestation API} 

\myparagraph{Initialization API} Developers need to initialize the protocol by creating the communication endpoints between replicas. \projecttitle{} offers a \texttt{create\_rpc()} function that creates Remote Procedure Call (RPC) objects (rpc) bound to the NIC. Specifically this function takes the application context as an argument, i.e., node's NIC specification and port, remote IP and port, creates a communication endpoint and continuously tries to establish connection with the remote side. The function returns after the connection establishment. An rpc offers bidirectional communication between the two sides. Additionally, we need to register the request handler functions to the rpcs, i.e., pass a pointer function a the construction of the endpoint which states what will happen when a request of a specific type is received. The developer might to overwrite/implement the \texttt{init\_store()} function which will keep an application's state and metadata in the trusted enclave. By default \projecttitle{} comes with a thread-safe and lock-free hybrid skiplist based on~\cite{avocado, folly}. While implementing our use cases in $\$$~\ref{sec:eval}, we used two  \projecttitle{} skiplists for metadata and data accordingly.  %Lastly, we need to register the request handler functions to the \texttt{rpc}s, i.e., pass a pointer function a the construction of the endpoint which states what will happen when a request of a specific type is received.

\myparagraph{Network API} \projecttitle{} offers high performance RPCs by extending eRPC~\cite{erpc} in the context of TEEs. Specifically, we place the message buffers outside the trusted enclave to both overcome the limited enclave memory and enable DMA operations~\footnote{DMA mappings are prohibited in the trusted area of a TEE as this violates their security properties~\cite{intel-sgx}}. We design a \texttt{send()} operation is used to submit a message for transmission. The message buffer is allocated by our library in \texttt{Hugepage} memory area and is later copied to the transmission queue (TX). Further, we provide a \texttt{multicast()} operation which creates identical copies of a message for all the recipient group.    Upon a reception of a request, the program control passes to the registered request handler where the function \texttt{respond()} can submit a response or \texttt{ACK} to that request. Lastly, the function \texttt{poll()} needs to be called regularly to fetch and process and send the incoming responses or requests and send the queued responses and requests respectively. 

\myparagraph{KV Store API} 

\dimitra{
\myparagraph{Trusted Leases API} \projecttitle{} exposes an API for leases to guarantee linearizable local reads when the CFT protocol allows that. A thread on a node initializes a lease (\texttt{init\_lease()}) and afterwards can exec a function within the lease's ownership (\texttt{exec\_with\_lease()}). In case the lease has expired, the function is not executed. The protocol updates the expiration date of a current lease with the \texttt{renew\_lease()}. The lease granter node/service updates its lease table with the \texttt{grand\_or\_update\_lease()} function.
}
\fi


\fi



%\section{\projecttitle Library}
%\label{sec:abstraction}
%\label{sec:recipe-implementation}
%\dimitra{system level design details here and the low-level API!}


%This section describes four core components of \projecttitle{}. .%  (a) networking library, (b) KV store, (c) secure runtime, and (d) attestation and configuration management.


%(\projecttitle{}-lib): \emph{(i)} the shielded networking library which leverages direct I/O while also preventing Byzantine behaviors in the untrusted network infrastructure, \emph{(ii)} the KV store which guarantees trust to local reads and, \emph{(iii)} the attestation and secrets distribution service which ensures that only trusted nodes know the configuration, keys, etc.


%\pramod{fix missing citations and a lot of typos and grammar errors.}

\subsection{\projecttitle Implementation and APIs}
\label{sec:recipe_impl_apis}
\label{subsec:networkin}
\label{subsec:KV}
\label{subsec:attestation}
\label{non-equivocation-design}

Table~\ref{tab:api} summarizes the \projecttitle{}'s API for each system component.

\myparagraph{\projecttitle{} networking} \projecttitle{} adopts the Remote Procedure Call (RPC) paradigm~\cite{286500} over a generic network library with various transportation layers (Infiniband, RoCE, and DPDK), which is also favorable in the context of TEEs where traditional kernel-based networking is impractical~\cite{kuvaiskii2017sgxbounds}. %Below, we explain how the networking layer is initialized in \projecttitle{}, the requests workflow and the core implementation details.



\begin{table}[t]
\small
%\fontsize{7}{10}\selectfont 

\begin{center}
\begin{tabular}{ |c|c| }
 \hline
 \bf{Attestation API} &  \\ \hline
 \multirow{1}{*}{\texttt{attest(measurement)}} & Attests the node based on  a measurement.  \\  \hline \hline
 \bf{Initialization API} &  \\ \hline
 \texttt{create\_rpc(app\_ctx)} & Initializes an RPCobj. \\
  \texttt{init\_store()} & Initializes the KV store. \\
  \texttt{reg\_hdlr(\&func)} & Registers request handlers. \\ \hline \hline
 \bf{Network API} &  \\ \hline
 \texttt{send(\&msg\_buf)} & Prepares a req for transmission. \\
 %\hline
% \texttt{multicast(\&msg\_buf, nodes)} & Prepares a request for multicast. \\
 \multirow{1}{*}{\texttt{respond(\&msg\_buf)}} & Prepares a resp for transmission. \\
 \texttt{poll()} & Polls for incoming messages. \\\hline \hline
% \texttt{aggregates\_multicast()} &  \\ 
 \bf{Security API} &  \\ \hline
 \texttt{verify\_msg(\&msg\_buf)} & Verifies the authenticity/integrity and cnt of a msg. \\
 \texttt{shield\_msg(\&msg\_buf)} & Generates a shielded msg. \\ \hline \hline
% \texttt{aggregates\_multicast()} &  \\ 
 \bf{KV Store API} &  \\ \hline
 \texttt{write(key, value)} & Writes a KV to the store. \\
 \hline
 \multirow{2}{*}{\texttt{get(key, \&v$_{TEE}$)}} & Reads the value into \texttt{v$_{TEE}$} \\ & and verifies integrity. \\ \hline %\hline
 \if 0
 \bf{Trusted Leases API} &  \\ \hline
 \texttt{init\_lease(node\_id, thread\_id)} & Requests a lease from the grander.\\ \hline
 \texttt{renew\_lease(\&lease)} & Updates a lease.\\ \hline
 \texttt{grand\_or\_update\_lease(node\_id, thread\_id)} & Grands a lease.\\ \hline
 \texttt{exec\_with\_lease(\&lease, \&func, \&args\_list)} & Executes the func within the lease ownership.\\ [1ex] \hline
 \fi
\end{tabular}
\end{center}
%\vspace{-10pt}
\caption{\projecttitle{} library APIs.} \label{tab:api}
\vspace{-6pt}
\end{table}





\if 0

Developer effort – initialization. The developer must spec-
ify the number and the nature of the logical message flows
they require. In RDMA parlance each flow corresponds to
one queue pair (QP), i.e., a send and a receive queue. For
instance, consider Hermes where a write requires two broad-
cast rounds: invalidations (invs) and validations (vals). Each
worker in each node sets up three QPs: 1) to send and re-
ceive invs, 2) to send and receive acks (for the invs) and 3) to
send and receive vals. Splitting the communication in mes-
sage flows is the responsibility of the developer. To create
the QP for each message flow, the developer simply calls a
Odyssey function, passing details about the nature of the QP.

\fi





\noindent\underline{Initialization.} Prior to the application's execution, developers need to initialize the networking layer by specifying the number of concurrent available connections, the types of the available requests, and by registering the appropriate (custom) request handlers. In \projecttitle{} terms, a communication endpoint corresponds to a per-thread RPC object (RPCobj) with private send/receive queues. All RPCobjs are registered to the same physical port (configurable). Initially, \projecttitle{} creates a handle to the NIC, which is passed to all RPCobjs. Developers need to define the types of RPC requests, each of which might be served by a different request handler. Request handlers are functions written by developers that are registered with the handle prior to the creation of the communication endpoints. Lastly, before executing the application's code, the connections between RPCobjs need to be correctly established.

\if 0
\projecttitle{} offers a \texttt{create\_rpc()} function that creates Remote Procedure Call (RPC) objects (rpc) bound to the NIC. Specifically, this function takes the application context as an argument, i.e., node's NIC specification and port, remote IP and port, create a communication endpoint and continuously tries to establish a connection with the remote side. The function returns after the connection establishment. An RPC offers bidirectional communication between the two sides. Additionally, we need to register the request handler functions to the rpcs, i.e., pass a pointer function to the construction of the endpoint, which states what will happen when a request of a specific type is received. The developer might overwrite/implement the \texttt{init\_store()} function, which will keep an application's state and metadata in the trusted enclave. By default \projecttitle{} comes with a thread-safe and lock-free hybrid skiplist based on~\cite{avocado, folly}. While implementing our use cases in $\$$~\ref{sec:eval}, we used two  \projecttitle{} skiplists for metadata and data accordingly.  %Lastly, we need to register the request handler functions to the \texttt{rpc}s, i.e., pass a pointer function a the construction of the endpoint which states what will happen when a request of a specific type is received.
\fi

\if 0
Developer effort – send and receive. For each QP, Odys-
sey maintains a send-FIFO and a receive-FIFO. Sending re-
quires that the developer first inserts messages in the send-
FIFO via an Odyssey insert function; later they can call a send
function to trigger the sending of all inserted messages. To re-
ceive messages, the developer need only call an Odyssey func-
tion that polls the receive-FIFO. Notably, the developer can
specify and register handlers to be called when calling any
one of the Odyssey functions. Therefore, the Odyssey polling
function will deliver the incoming messages, if any, to the
developer-specified handler.
\fi





\noindent\underline{send/receive operations.} We offer asynchronous network operations following the RPC paradigm. For each RPCobj, \projecttitle{} keeps a transmission (TX) and reception (RX) queue, organized as ring buffers. Developers enqueue requests and responses to requests via \projecttitle{}'s specific functions, which place the message in the RPCobj's TX queue. Later, they can call a polling function that flushes the messages in the TX and drains the RX queues of an RPCobj. The function will trigger the sending of all queued messages and process all received requests and responses. Reception of a request triggers the execution of the request handler for that specific type. Reception of a response to a request triggers a cleanup function that releases all resources allocated for the request, e.g., message buffers and rate limiters (for congestion). %The cleanup functions can be overwritten by the developers for extra functionalities.

\if 0
\projecttitle{} offers high performance RPCs by extending eRPC~\cite{erpc} and DPDK~\cite{dpdk} in the context of TEEs. eRPC is .. Specifically, we place the message buffers outside the trusted enclave to both overcome the limited enclave memory and enable DMA operations~\footnote{DMA mappings are prohibited in the trusted area of a TEE as this violates their security properties~\cite{intel-sgx}}. We design a \texttt{send()} operation is used to submit a message for transmission. The message buffer is allocated by our library in \texttt{Hugepage} memory area and is later copied to the transmission queue (TX). Further, we provide a multicast() operation which creates identical copies of a message for all the recipient group.    Upon a reception of a request, the program control passes to the registered request handler where the function \texttt{respond()} can submit a response or \texttt{ACK} to that request. Lastly, the function \texttt{poll()} needs to be called regularly to fetch and process and send the incoming responses or requests and send the queued responses and requests respectively. 
\fi



 %The trusted message is of the form [(req, (ReplicaView, cq, cnt\_{cq})), ${h_{cq}_{\sigma_{cq}})}$$>$.% containing the encrypted metadata and hash of the $req$ and the $req$ payload. %This function will marshal the current value of its trusted monotonic counter, the current view number, and the (cryptographic) hash of the message into one string. 

%\myparagraph{Non-equivocation} \projecttitle{} limits the equivocation of Byzantine (malicious) faults in the networking infrastructure using TEEs. Specifically, we guarantee non-equivocation via trusted counter assignment and verification. Each replica maintains a local sequence tuple $(v, cq, seq)$ where $v$ is the current view number, $cq$ is the communication (pair) channel between two nodes and $seq$ is the
%current trusted counter value in that view for the latest committed request sent in that communication channel. Each request is assigned a unique tuple $(v, cq, seq)$ which is maintained by the TEE of each replica to guarantee monotonic increments and rollback/forking attacks resilience. The coordinator node of a request assigns the request with the correct tuple and increments the $seq$.  Once a replica receives a request, they only accept it after its verification. The accepted requests pass through the underlying CFT protocol. Replicas verify the received requests using the \texttt{VerifyCounter}$(<req, (v, cq, seq)>, {h_{cq}})$ function. Specifically, the replica verifies the freshness of the message/request by examining its counter id. The message passes through the non-equivocation layer to verify that the counter associated with the received request (as part of the message metadata) is consistent with its local counter. Replicas in \projecttitle{} are willing to accept ``future'' valid messages as these might come out of order, i.e., messages whose seq number is $> (seq'_{cq}+1)$ ($seq'_{cq}$ is the last seen request number from that communication channel). Such messages are valid so \projecttitle{} accepts them. However, they are processed and committed when the underlying CFT protocol allows that.

%\dimitra{fix this}
%\myparagraph{Integrity verification} In \projecttitle{} we leverage basic primitives in modern cryptography such as hash functions to check and verify the integrity of the data the might reside in the untrusted areas including, the host memory and the network infrastructure. Each message $m$ sent from $n_i$ to $n_j$ over a communication channel $cq$ is accompanied by its calculated hash $h_{cq}$ that allows the recipient $n_j$ to verify that the message payload is genuine. A node that drives the client's request (coordinator) before sending the request to replicas need to call \texttt{ShieldRequest}$(req, cq)$ to generate an integrity-protected message for that request. This function will marshal the current value of its trusted monotonic counter, the current view number, and the (cryptographic) hash of the message into one string. The output is a bytestream of the form $<req, (\texttt{(ReplicaView}, cq, cnt_{cq})>, {h_{cq})}$ containing the metadata, the request payload and the computed hash of metadata and payload.

\noindent\underline{API.} We offer a create\_rpc() function that creates a bound-to-the-NIC RPCobj. The function takes the application context, i.e., NIC specification and port, remote IP and port, as an argument, creates a communication endpoint, and establishes a connection with the remote side. The function returns after the connection establishment. RPCobjs offer bidirectional communication between the two sides. Prior to the creation of RPCobj, developers need to specify and register the request types and handlers using the reg\_hdlr() which takes as an argument a reference to the preferred handler function. %The developer might to overwrite/implement the \texttt{init\_store()} function which will keep an application's state and metadata in the trusted enclave. By default \projecttitle{} comes with a thread-safe and lock-free hybrid skiplist based on~\cite{avocado, folly}. While implementing our use cases in $\$$~\ref{sec:eval}, we used two  \projecttitle{} skiplists for metadata and data accordingly.  %Lastly, we need to register the request handler functions to the \texttt{rpc}s, i.e., pass a pointer function a the construction of the endpoint which states what will happen when a request of a specific type is received.

For exchanging network messages, we designed a send() function that takes the session (connection) identifier, the message buffer to be sent, the request type, and the cleanup function as arguments. This function submits a message for transmission. Upon reception of a request, the program control passes to the registered request handler, where the function respond() can submit a response or ACK to that request. Lastly, the function poll() needs to be called regularly to fetch or transmit the network messages in the TX and RX queues.





\begin{comment}
~\footnote{DMA mappings are prohibited in the trusted area of a TEE as this violates TEE's security properties~\cite{intel-sgx, avocado, treaty}}
\end{comment}

\if 0
\myparagraph{Implementation details}
We designed \projecttitle{}'s high-performance RPCs by extending eRPC~\cite{erpc} in the context of TEEs. We place the message buffers outside the enclave to overcome the limited enclave memory and enable DMA operations~\cite{intel-sgx, avocado, treaty}. The message buffers are allocated in Hugepage area and are later copied or mapped to the TX/RX queues. The networking buffers residing outside the TEE follow the trusted message format we discussed in \ref{subsec:overview}. As such, while outside the trusted area, their integrity (or confidentiality) can be verified upon reception.

We also adopted a rate limiter which can be configured to limit read and/or write requests. The use of a rate limiter was found to be useful in protocols which vastly saturated the available host memory (we found out that the R-ABD protocol was quickly exhausting all available memory in the system while the tail-node in the R-CR protocol also ran out-of-memory for read-heavy workloads). Lastly, we implement on top of \projecttitle{}'s network library a batching technique which queues the message buffers and merges them into a bigger buffer before transmission. The batching factor is configurable and has been proven extremely efficient for small messages (e.g., \SI{256}{\byte}).

\fi 


\myparagraph{Secure runtime} We build our codebase in C++ using \scone{} to access the TEE hardware. \scone{} exposes a modified libc library and combines user-level threading and asynchronous syscalls~\cite{flexsc} to reduce the cost of syscall execution. While we limit the number of syscalls, leveraging  \scone{}'s exit-less approach allows us to optimize the initialization phase that vastly allocates host memory for the network stack and the KV store. To enable NIC's DMA operations and memory mappings to the hugepages (for message buffers and TX/RX queues) ($\S$~\ref{subsec:networkin}), we overwrite the \texttt{mmap()} syscall of \scone{} to bypass its shield layer and allow the allocation of (untrusted) host memory. 

For the cryptographic primitives, we build on OpenSSL~\cite{openssl}. Lastly, we build on a lease mechanism~\cite{t-lease} in \scone{} for auxiliary operations, e.g., failures detection and leader's election.




\myparagraph{\projecttitle{} key-value store} \projecttitle{} provides a lock-free, high-performant KV store based on a skip-list. We partition the keys from the values' space by placing the keys along with metadata (and a pointer to the value in host memory) inside the TEE's memory area, the {\em enclave}, and storing the values in the host memory. Our partitioned KVs reduces the number of calculations for integrity checks, compared to prior work~\cite{shieldstore} which implements (per-bucket) merkle trees and re-calculates the root on each update. Importantly, separating the (keys + metadata) and the values between the enclave and untrusted unlimited memory decreases the Enclave Page Cache (EPC) pressure~\cite{speicher-fast}. %Our lock-free data structure supports concurrent operations and it is well-suited
%for increased parallelism.

The developer might want to overwrite/implement the init\_store() function, which will keep an application's state and metadata in the trusted enclave. \projecttitle{} implements its hybrid skiplist based on folly library~\cite{folly}. The write() function updates the KV, while the get() function copies the value of the given key in the protected area. The function also verifies the value's integrity. We implement an allocator for host memory that is given as an initialization parameter to the KV store.

\if 0
\myparagraph{Implementation and API} 

\fi 
%\myparagraph{Data properties}
\projecttitle{}'s KV store design resolves Byzantine errors since the metadata (and the code that accesses them) reside in the enclave. That said, \projecttitle{} allows for local reads as nodes can verify the integrity of the stored values. Our partitioned scheme {\em seamlessly} strengthens the system's security properties further and can offer confidentiality by encrypting the values outside the TEE. \projecttitle{}-transformed protocols that further offer confidentiality outperform the BFT systems ($
\S$~\ref{sec:eval}).


%that offers parallel write and read operations, however the developer might wish to overwrite those functions. Both operations calculate the hash of the given data which is placed in the enclave memory. The hash is used to verify the integrity of the data stored in the host memory (if any).





%\vspace{-8pt}
%The application and the challenger first establish communication, and then the application asks the challenger to provide secrets.


\myparagraph{Attestation process} The attestation process is initialized by the \emph{challenger}, a remote process that can verify the authenticity of a specific TEE. The challenger executes the remote\_attestation() function to send an attestation request to the application---usually in the form of a nonce (a random number). The challenger and the application then pass through a Diffie-Hellman key exchange process~\cite{10.1145/359460.359473}. The application generates an ephemeral public key which is used by the challenger later to provision any secrets.

When the TEE receives the nonce, it calls the attest() and generates a \emph{measurement} ($\mu$) of its state and loaded code. Following this, the TEE calls into the generate\_quote$(\mu, k_{pub})$ to sign $\mu$ (quote) with the $key_{hw}$ which is fetched from the TEE's h/w. The TEE signs and encrypts the quote quote$_{\sigma_{k_{pub}}}$ over the challenger's public key $k_{pub}$, which is then sent back to the challenger. Upon successful verifications of the quote$_{\sigma_{k_{pub}}}$, the challenger shares secrets and configurations.

To offer low-latency attestations within the same datacenter that \projecttitle{} runs, we build a Configuration and Attestation service (CAS). The Protocol Designer (PD) deploys the CAS inside a TEE and attests it through the hardware vendor's attestation service---e.g., Intel Attestation Service (IAS~\cite{ias}). Once the CAS is attested, it is trusted, and the PB can upload secrets and configurations. 

The challenger asserts upon a failed verification and denies sharing any secret or configuration data. Otherwise, it distributes all necessary shared secrets.

\if 0 
\myparagraph{Implementation details} 
%\projecttitle{} builds on top of a Configuration and Attestation Service (CAS) that is shipped with \scone{} and ultimately relies on hardware-based TEEs for secure secrets and configuration management. 
A trusted entity, i.e., the developer, must deploy the  Configuration and Attestation Service  CAS inside a TEE in a node that can also be part of the membership. Afterward, they need to attest the CAS through the TEE's attestation service---in our case, Intel Attestation Service (IAS~\cite{ias}). Once the CAS is attested, it can replace the TEE's attestation service. The trusted entity needs to spawn further Local Attestation Services (LASes) that perform local (intra-platform) attestation of processes and offer low-latency attestation of the new nodes. Before the CFT protocol commences, the CAS must also attest all LASes. 

When a \projecttitle{} process is loaded in the TEE (before the control passes to the application code), the LAS instructs a local attestation. The attestee generates a quote of the enclave (measurement). The LAS forwards the enclave's measurement to the CAS, which replies with a success or failure message indicating the authenticity of the process. After a successful attestation, the CAS stores the node's IP and provides the trusted process with secrets and configurations.

\myparagraph{API}
\projecttitle{} provides an attestation API to developers. Particularly, we provide the attest function that takes as arguments the IP of a trusted third-party service, CAS's IP for \projecttitle{}, and a generated enclave measurement of the code. Then, this service verifies that both the enclave signer and measurement are in the expected state and replies accordingly.
\fi 
%\dimitra{fix this}
%\myparagraph{Attestation and secrets management} Attestation is the process of demonstrating that the correct software is securely running within a TEE-enclave on an enabled platform. Secrets (e.g., certificates, encryption keys, etc.) and configurations (membership IPs and information) should only be provided to a replica after its successful attestation. Once an enclave is initialized and before the control is relinquished to the application's code, the attestation process is launched to verify the integrity and authenticity of the included code and data. Essentially, \projecttitle{} needs to (1.) offer low-latency attestation of the joiner nodes and (2.) securely provide the trusted enclave applications with secrets and configuration data (usually over the network).

%\projecttitle{} leverages TEEs to implement \texttt{RemoteAttestation}$()$, \texttt{Attest}$()$ and \texttt{GenerateQuote}$()$ primitives that allow a third party to attest an enclave. Next we describe the workflow of the attestation and the properties of those primitives.

%The attestation process is initialized by the challenger (or verifier)---a remote process (typically provided by the hardware vendor) that can prove the authenticity of a specific enclave. The application and the challenger first establish communication and, then, the application asks the challenger to provision secrets. The challenger executes the \texttt{RemoteAttestation}$()$ function where it first replies with an attestation request to the application---usually in the form a nonce (a random number
%generated for this one occasion). The challenger and the application also pass through a Diffie-Hellman key exchange (DHKE) process. The application generates an ephemeral public key which is used by the challenger later for provisioning secrets to the enclave.

%After receiving the nonce, the enclave calls into the \texttt{Attest}$()$ and generates a \emph{measurement} or \emph{report} that includes information about the enclave. The report needs to be sent to the challenger for verification. The enclave calls into \texttt{GenerateQuote$(\mu, k_{public})$} which sings the measurement $\mu$ over $key_{hw}$, where $key_{hw}$ is fetched from the TEE's hardware. Additionally, the function encrypts the quote over challenger's public key $k_{public}$. The encrypted quote can be sent and verified by the challenger with the corresponding verification key.

%The challenger asserts upon a failed verification and denies to share any secret or configuration data. Otherwise, it distributes all necessary shared secrets.

\if 0

\subsection{\projecttitle{} API}

\if 0
\subsection{\projecttitle{} client API}
\projecttitle{} exposes a simple \texttt{PUT}/\texttt{GET} API to clients. Both functions take as arguments the coordinator node's id, the view and the leader identifier (if any) that are known to the client. The API assigns a unique monotonically increased id to every request executed by the client. That is to help CFT distinguish already executed requests.
\fi

%\subsection{\projecttitle{} system-level API}
Table~\ref{tab:api} presents the core library that \projecttitle{} exposes to the developers. We implement our \projecttitle{} on top of \scone~\cite{arnautov2016scone} and \textsc{Palaemon}~\cite{palaemon} that use Intel SGX~\cite{intel-sgx} as the TEE and we extend eRPC~\cite{erpc} on top of DPDK~\cite{dpdk} for fast networking. Next we discuss the use-cases and the implementation details for each core function of \projecttitle{}.

\myparagraph{Attestation API} 

\myparagraph{Initialization API} Developers need to initialize the protocol by creating the communication endpoints between replicas. \projecttitle{} offers a \texttt{create\_rpc()} function that creates Remote Procedure Call (RPC) objects (rpc) bound to the NIC. Specifically this function takes the application context as an argument, i.e., node's NIC specification and port, remote IP and port, creates a communication endpoint and continuously tries to establish connection with the remote side. The function returns after the connection establishment. An rpc offers bidirectional communication between the two sides. Additionally, we need to register the request handler functions to the rpcs, i.e., pass a pointer function a the construction of the endpoint which states what will happen when a request of a specific type is received. The developer might to overwrite/implement the \texttt{init\_store()} function which will keep an application's state and metadata in the trusted enclave. By default \projecttitle{} comes with a thread-safe and lock-free hybrid skiplist based on~\cite{avocado, folly}. While implementing our use cases in $\$$~\ref{sec:eval}, we used two  \projecttitle{} skiplists for metadata and data accordingly.  %Lastly, we need to register the request handler functions to the \texttt{rpc}s, i.e., pass a pointer function a the construction of the endpoint which states what will happen when a request of a specific type is received.

\myparagraph{Network API} \projecttitle{} offers high performance RPCs by extending eRPC~\cite{erpc} in the context of TEEs. Specifically, we place the message buffers outside the trusted enclave to both overcome the limited enclave memory and enable DMA operations~\footnote{DMA mappings are prohibited in the trusted area of a TEE as this violates their security properties~\cite{intel-sgx}}. We design a \texttt{send()} operation is used to submit a message for transmission. The message buffer is allocated by our library in \texttt{Hugepage} memory area and is later copied to the transmission queue (TX). Further, we provide a \texttt{multicast()} operation which creates identical copies of a message for all the recipient group.    Upon a reception of a request, the program control passes to the registered request handler where the function \texttt{respond()} can submit a response or \texttt{ACK} to that request. Lastly, the function \texttt{poll()} needs to be called regularly to fetch and process and send the incoming responses or requests and send the queued responses and requests respectively. 

\myparagraph{KV Store API} 

\dimitra{
\myparagraph{Trusted Leases API} \projecttitle{} exposes an API for leases to guarantee linearizable local reads when the CFT protocol allows that. A thread on a node initializes a lease (\texttt{init\_lease()}) and afterwards can exec a function within the lease's ownership (\texttt{exec\_with\_lease()}). In case the lease has expired, the function is not executed. The protocol updates the expiration date of a current lease with the \texttt{renew\_lease()}. The lease granter node/service updates its lease table with the \texttt{grand\_or\_update\_lease()} function.
}
\fi




\section{SPIRIT Implementation}
\chapter{Implementation}{\label{ch:implementation}}
In this chapter, we present the implementation of the final product. We start by discussing how the four steps introduced in \hyperref[ch:high_level_approach]{chapter \ref*{ch:high_level_approach}} are integrated. We then outline the main system components of our score follower, presenting each as an independent, self-contained module. We then combine this into an overall system architecture and finally introduce the open-source score renderer used to display the score and evaluate the score follower.       

% \section{Aims and Requirements}
% The overall aim of the score follower was to 


\section{Score Follower Framework Details}
Our score follower conforms to the high-level framework presented in \hyperref[section:score_follower_framework]{section \ref*{section:score_follower_framework}}. In step 1, two score features are extracted from a MIDI file (see \hyperref[subsection:midi]{subsection \ref*{subsection:midi}}), namely MIDI note numbers\footnote{\href{https://inspiredacoustics.com/en/MIDI_note_numbers_and_center_frequencies}{https://inspiredacoustics.com/en/MIDI\_note\_numbers\_and\_center\_frequencies}} (corresponding to pitch) and note onsets (corresponding to duration). In step 2, the audio is streamed (whether from a file or into a microphone) and audioframes that exceed some predefined energy threshold are extracted. Here, audioframes are groups of contiguous audio samples, whose length can be specified by the argument \verb|frame_length|, usually between 800 and 2000 samples. The period between consecutive audioframes can also be defined by the argument \verb|hop_length|, typically between 2000 and 5000 audio samples. In step 3, score following is performed via a `Windowed' Viterbi algorithm (see  \hyperref[subsection:adjusting_viterbi]{subsection \ref*{subsection:adjusting_viterbi}}) which uses the Gaussian Process (GP) log marginal likelihoods (LMLs) for emission probabilities (see \hyperref[section:state_duration_model]{section \ref*{section:state_duration_model}}) and a state duration model for transition probabilities (see \hyperref[section:state_duration_model]{section \ref*{section:state_duration_model}}). Finally, in step 4 we render our results using an adapted version of the open source user interface, \textit{Flippy Qualitative Testbench}.

\section{Following Modes}
Two modes are available to the user: Pre-recorded Mode and Live Mode. The former requires a pre-recorded $\verb|.wav|$ file, whereas the latter takes an input stream of audio via the device's microphone. Note that both modes are still forms of score following, as opposed to score alignment, since in each mode we receive audioframes at the sampling rate, not all at once.\\

Live Mode offers a practical example of a score follower, displaying a score and position marker which a musician can read off while playing. However, this mode is not suitable for evaluation because the input and results cannot be easily replicated. Even ignoring repeatability, Live Mode is not suitable for one-off testing since a musician using this application may be influenced by the movement of the marker. For instance, the performer may speed up if the score follower `gets ahead' or slow down if the position marker lags or `gets lost'. To avoid this, we use Pre-recorded Mode when evaluating the performance of our score follower. Furthermore, Pre-recorded Mode offers the advantage of testing away from the music room, providing the opportunity to evaluate a variety of recordings available online. 

\section{System Architecture}
Our guiding principle for development was to build modular code in order to create a streamlined system where each component performs a specific task independently. This structure facilitates easy testing and debugging. \hyperref[fig:black_box]{Figure \ref*{fig:black_box}} presents a high-level architecture diagram, where each black box abstracts a key component of the score follower. When operating in Pre-recorded Mode, there is the option to stream the recording during run-time, which outputs to the device's speakers (as indicated by the dashed lines).

\begin{figure}[H]
    \centering
    \includegraphics[width=1\textwidth]{figs/Part_4_Implementation_And_Results/black_box.png}
    \caption{Abstracted system architecture diagram displaying inputs in grey, the 4 main components of the score follower in black and the outputs in green.}
    \label{fig:black_box}
\end{figure}

\subsection{Score Preprocessor}
The architecture for the Score Preprocesor is given in \hyperref[fig:score_preprocessor]{Figure \ref*{fig:score_preprocessor}}. First, MIDI note number and note onset times are extracted from each MIDI event. Simultaneous notes can be gathered into states and returned as a time-sorted list of lists called \verb|score|, where each element of the outer list is a list of simultaneous note onsets. Similarly, a list of note durations calculated as the time difference between consecutive states is returned as \verb|times_to_next|. Finally, all covariance matrices are precalculated and stored in a dictionary, where the key of the dictionary is determined by the notes present. This is because the distribution of notes and chords in a score is not random: notes tend to belong to a home \gls{key} and melodies tend to be repeated or related (similar to subject fields in speech processing). Therefore, states tend to be reused often, allowing us to achieve amortised time and space savings (by avoiding repeated calculation of the same covariance matrices). 

\begin{figure}[H]
    \centering
    \includegraphics[width=1\textwidth]{figs/Part_3_Implementation/Stage_2_Alignment/score_preprocessor.png}
    \caption{System architecture diagram representing the Score Preprocessor with inputs in grey, processes in blue and objects in yellow.}
    \label{fig:score_preprocessor}
\end{figure}


\subsection{Audio Preprocessor}
The architecture for the Audio Preprocessor is illustrated in \hyperref[fig:audio_preprocessor]{Figure \ref*{fig:audio_preprocessor}}. In Pre-recorded Mode, the Slicer receives a $\verb|.wav|$ file and returns audioframes separated by the \verb|hop_length|. These audioframes are periodically added to a multiprocessing queue, \verb|AudioFramesQueue|, to simulate real-time score following. In Live Mode, we use the python module \verb|sounddevice| to receive a stream of audio, using a periodic callback function to place audioframes on \verb|AudioFramesQueue|. 

\begin{figure}[H]
    \centering
    \includegraphics[width=1\textwidth]{figs/Part_4_Implementation_And_Results/audio_preprocessor.png}
    \caption{System architecture diagram representing the Audio Preprocessor with inputs in grey, processes in blue and objects in yellow.}
    \label{fig:audio_preprocessor}
\end{figure}

\subsection{Follower and Backend}
The joint Follower and Backend architecture diagram is shown in \hyperref[fig:follwer_and_backend]{Figure \ref*{fig:follwer_and_backend}}. The Viterbi Follower (detailed in \hyperref[subsection:adjusting_viterbi]{section \ref*{subsection:adjusting_viterbi}}) calculates the most probable state in the score, given audioframes continually taken from \verb|AudioFramesQueue|. These states are placed on another multiprocessing queue, the \verb|FollowerOutputQueue|, for the Backend to process and send. This prevents any bottle-necking occurring at the Follower stage. The Backend first sets up a UDP connection and then reads off values from \verb|FollowerOutputQueue|, sending them via UDP packets to the score renderer.

\begin{figure}[H]
    \centering
    \includegraphics[width=1\textwidth]{figs/Part_4_Implementation_And_Results/follower_and_backend.png}
    \caption{System architecture diagram representing the Follower and Backend processes with processes in blue, objects in yellow and outputs in green.}
    \label{fig:follwer_and_backend}
\end{figure}

\subsection{Player}
In Pre-recorded Mode, the Player sets up a new process and begins streaming the recording once the Follower process begins. This provides a baseline for testing purposes, as a trained musician can observe the score position marker and judge how well it matches the music. 

\subsection{Overall System Architecture}
The overall system architecture is presented in \hyperref[fig:overall_system_architecture]{Figure \ref*{fig:overall_system_architecture}}. Since the Follower runs a real-time, time sensitive process, parallelism is employed to reduce the total system latency. We use two \verb|multiprocessing| queues\footnote{\href{https://docs.python.org/3/library/multiprocessing.html}{https://docs.python.org/3/library/multiprocessing.html}} to avoid bottle-necking, which allows us to run 4 concurrent processes (Audio Preprocessor, Follower, Backend, and Audio Player). Hence, this architecture allows the components to run independently of one another to avoid blocking. Furthermore, this allows the system to take advantage of the multiple cores and high computational power offered by most modern machines.  

\begin{figure}[H]
    \centering
    \includegraphics[width=1\textwidth]{figs/Part_4_Implementation_And_Results/overall_score_follower_2.png}
    \caption{System architecture diagram representing the overall score follower running in Pre-recorded mode, with inputs in grey, processes in blue, objects in yellow and outputs in green.}
    \label{fig:overall_system_architecture}
\end{figure}


\section{Rendering Results}{\label{section:renderer}}
To visualise the results of our score follower, we adapted an open source tool for testing different score followers.\footnote{\href{https://github.com/flippy-fyp/flippy-qualitative-testbench/blob/main/README.md}{https://github.com/flippy-fyp/flippy-qualitative-testbench/blob/main/README.md}} \hyperref[fig:flippy_example]{Figure \ref*{fig:flippy_example}} shows the user interface of the score position renderer, where the green bar indicates score position. 

\begin{figure}[H]
    \centering
    \includegraphics{figs/Part_4_Implementation_And_Results/example_renderer.png}
    \caption{Screenshot of the score renderer user interface which displays a score (here we show a keyboard arrangement of \textit{O Haupt voll Blut und Wunden} by Bach). The green marker represents the score follower position.}
    \label{fig:flippy_example}
\end{figure}





\section{Evaluation Methodology}
\section{Research Methodology}~\label{sec:Methodology}

In this section, we discuss the process of conducting our systematic review, e.g., our search strategy for data extraction of relevant studies, based on the guidelines of Kitchenham et al.~\cite{kitchenham2022segress} to conduct SLRs and Petersen et al.~\cite{PETERSEN20151} to conduct systematic mapping studies (SMSs) in Software Engineering. In this systematic review, we divide our work into a four-stage procedure, including planning, conducting, building a taxonomy, and reporting the review, illustrated in Fig.~\ref{fig:search}. The four stages are as follows: (1) the \emph{planning} stage involved identifying research questions (RQs) and specifying the detailed research plan for the study; (2) the \emph{conducting} stage involved analyzing and synthesizing the existing primary studies to answer the research questions; (3) the \emph{taxonomy} stage was introduced to optimize the data extraction results and consolidate a taxonomy schema for REDAST methodology; (4) the \emph{reporting} stage involved the reviewing, concluding and reporting the final result of our study.

\begin{figure}[!t]
    \centering
    \includegraphics[width=1\linewidth]{fig/methodology/searching-process.drawio.pdf}
    \caption{Systematic Literature Review Process}
    \label{fig:search}
\end{figure}

\subsection{Research Questions}
In this study, we developed five research questions (RQs) to identify the input and output, analyze technologies, evaluate metrics, identify challenges, and identify potential opportunities. 

\textbf{RQ1. What are the input configurations, formats, and notations used in the requirements in requirements-driven
automated software testing?} In requirements-driven testing, the input is some form of requirements specification -- which can vary significantly. RQ1 maps the input for REDAST and reports on the comparison among different formats for requirements specification.

\textbf{RQ2. What are the frameworks, tools, processing methods, and transformation techniques used in requirements-driven automated software testing studies?} RQ2 explores the technical solutions from requirements to generated artifacts, e.g., rule-based transformation applying natural language processing (NLP) pipelines and deep learning (DL) techniques, where we additionally discuss the potential intermediate representation and additional input for the transformation process.

\textbf{RQ3. What are the test formats and coverage criteria used in the requirements-driven automated software
testing process?} RQ3 focuses on identifying the formulation of generated artifacts (i.e., the final output). We map the adopted test formats and analyze their characteristics in the REDAST process.

\textbf{RQ4. How do existing studies evaluate the generated test artifacts in the requirements-driven automated software testing process?} RQ4 identifies the evaluation datasets, metrics, and case study methodologies in the selected papers. This aims to understand how researchers assess the effectiveness, accuracy, and practical applicability of the generated test artifacts.

\textbf{RQ5. What are the limitations and challenges of existing requirements-driven automated software testing methods in the current era?} RQ5 addresses the limitations and challenges of existing studies while exploring future directions in the current era of technology development. %It particularly highlights the potential benefits of advanced LLMs and examines their capacity to meet the high expectations placed on these cutting-edge language modeling technologies. %\textcolor{blue}{CA: Do we really need to focus on LLMs? TBD.} \textcolor{orange}{FW: About LLMs, I removed the direct emphase in RQ5 but kept the discussion in RQ5 and the solution section. I think that would be more appropriate.}

\subsection{Searching Strategy}

The overview of the search process is exhibited in Fig. \ref{fig:papers}, which includes all the details of our search steps.
\begin{table}[!ht]
\caption{List of Search Terms}
\label{table:search_term}
\begin{tabularx}{\textwidth}{lX}
\hline
\textbf{Terms Group} & \textbf{Terms} \\ \hline
Test Group & test* \\
Requirement Group & requirement* OR use case* OR user stor* OR specification* \\
Software Group & software* OR system* \\
Method Group & generat* OR deriv* OR map* OR creat* OR extract* OR design* OR priorit* OR construct* OR transform* \\ \hline
\end{tabularx}
\end{table}

\begin{figure}
    \centering
    \includegraphics[width=1\linewidth]{fig/methodology/search-papers.drawio.pdf}
    \caption{Study Search Process}
    \label{fig:papers}
\end{figure}

\subsubsection{Search String Formulation}
Our research questions (RQs) guided the identification of the main search terms. We designed our search string with generic keywords to avoid missing out on any related papers, where four groups of search terms are included, namely ``test group'', ``requirement group'', ``software group'', and ``method group''. In order to capture all the expressions of the search terms, we use wildcards to match the appendix of the word, e.g., ``test*'' can capture ``testing'', ``tests'' and so on. The search terms are listed in Table~\ref{table:search_term}, decided after iterative discussion and refinement among all the authors. As a result, we finally formed the search string as follows:


\hangindent=1.5em
 \textbf{ON ABSTRACT} ((``test*'') \textbf{AND} (``requirement*'' \textbf{OR} ``use case*'' \textbf{OR} ``user stor*'' \textbf{OR} ``specifications'') \textbf{AND} (``software*'' \textbf{OR} ``system*'') \textbf{AND} (``generat*'' \textbf{OR} ``deriv*'' \textbf{OR} ``map*'' \textbf{OR} ``creat*'' \textbf{OR} ``extract*'' \textbf{OR} ``design*'' \textbf{OR} ``priorit*'' \textbf{OR} ``construct*'' \textbf{OR} ``transform*''))

The search process was conducted in September 2024, and therefore, the search results reflect studies available up to that date. We conducted the search process on six online databases: IEEE Xplore, ACM Digital Library, Wiley, Scopus, Web of Science, and Science Direct. However, some databases were incompatible with our default search string in the following situations: (1) unsupported for searching within abstract, such as Scopus, and (2) limited search terms, such as ScienceDirect. Here, for (1) situation, we searched within the title, keyword, and abstract, and for (2) situation, we separately executed the search and removed the duplicate papers in the merging process. 

\subsubsection{Automated Searching and Duplicate Removal}
We used advanced search to execute our search string within our selected databases, following our designed selection criteria in Table \ref{table:selection}. The first search returned 27,333 papers. Specifically for the duplicate removal, we used a Python script to remove (1) overlapped search results among multiple databases and (2) conference or workshop papers, also found with the same title and authors in the other journals. After duplicate removal, we obtained 21,652 papers for further filtering.

\begin{table*}[]
\caption{Selection Criteria}
\label{table:selection}
\begin{tabularx}{\textwidth}{lX}
\hline
\textbf{Criterion ID} & \textbf{Criterion Description} \\ \hline
S01          & Papers written in English. \\
S02-1        & Papers in the subjects of "Computer Science" or "Software Engineering". \\
S02-2        & Papers published on software testing-related issues. \\
S03          & Papers published from 1991 to the present. \\ 
S04          & Papers with accessible full text. \\ \hline
\end{tabularx}
\end{table*}

\begin{table*}[]
\small
\caption{Inclusion and Exclusion Criteria}
\label{table:criteria}
\begin{tabularx}{\textwidth}{lX}
\hline
\textbf{ID}  & \textbf{Description} \\ \hline
\multicolumn{2}{l}{\textbf{Inclusion Criteria}} \\ \hline
I01 & Papers about requirements-driven automated system testing or acceptance testing generation, or studies that generate system-testing-related artifacts. \\
I02 & Peer-reviewed studies that have been used in academia with references from literature. \\ \hline
\multicolumn{2}{l}{\textbf{Exclusion Criteria}} \\ \hline
E01 & Studies that only support automated code generation, but not test-artifact generation. \\
E02 & Studies that do not use requirements-related information as an input. \\
E03 & Papers with fewer than 5 pages (1-4 pages). \\
E04 & Non-primary studies (secondary or tertiary studies). \\
E05 & Vision papers and grey literature (unpublished work), books (chapters), posters, discussions, opinions, keynotes, magazine articles, experience, and comparison papers. \\ \hline
\end{tabularx}
\end{table*}

\subsubsection{Filtering Process}

In this step, we filtered a total of 21,652 papers using the inclusion and exclusion criteria outlined in Table \ref{table:criteria}. This process was primarily carried out by the first and second authors. Our criteria are structured at different levels, facilitating a multi-step filtering process. This approach involves applying various criteria in three distinct phases. We employed a cross-verification method involving (1) the first and second authors and (2) the other authors. Initially, the filtering was conducted separately by the first and second authors. After cross-verifying their results, the results were then reviewed and discussed further by the other authors for final decision-making. We widely adopted this verification strategy within the filtering stages. During the filtering process, we managed our paper list using a BibTeX file and categorized the papers with color-coding through BibTeX management software\footnote{\url{https://bibdesk.sourceforge.io/}}, i.e., “red” for irrelevant papers, “yellow” for potentially relevant papers, and “blue” for relevant papers. This color-coding system facilitated the organization and review of papers according to their relevance.

The screening process is shown below,
\begin{itemize}
    \item \textbf{1st-round Filtering} was based on the title and abstract, using the criteria I01 and E01. At this stage, the number of papers was reduced from 21,652 to 9,071.
    \item \textbf{2nd-round Filtering}. We attempted to include requirements-related papers based on E02 on the title and abstract level, which resulted from 9,071 to 4,071 papers. We excluded all the papers that did not focus on requirements-related information as an input or only mentioned the term ``requirements'' but did not refer to the requirements specification.
    \item \textbf{3rd-round Filtering}. We selectively reviewed the content of papers identified as potentially relevant to requirements-driven automated test generation. This process resulted in 162 papers for further analysis.
\end{itemize}
Note that, especially for third-round filtering, we aimed to include as many relevant papers as possible, even borderline cases, according to our criteria. The results were then discussed iteratively among all the authors to reach a consensus.

\subsubsection{Snowballing}

Snowballing is necessary for identifying papers that may have been missed during the automated search. Following the guidelines by Wohlin~\cite{wohlin2014guidelines}, we conducted both forward and backward snowballing. As a result, we identified 24 additional papers through this process.

\subsubsection{Data Extraction}

Based on the formulated research questions (RQs), we designed 38 data extraction questions\footnote{\url{https://drive.google.com/file/d/1yjy-59Juu9L3WHaOPu-XQo-j-HHGTbx_/view?usp=sharing}} and created a Google Form to collect the required information from the relevant papers. The questions included 30 short-answer questions, six checkbox questions, and two selection questions. The data extraction was organized into five sections: (1) basic information: fundamental details such as title, author, venue, etc.; (2) open information: insights on motivation, limitations, challenges, etc.; (3) requirements: requirements format, notation, and related aspects; (4) methodology: details, including immediate representation and technique support; (5) test-related information: test format(s), coverage, and related elements. Similar to the filtering process, the first and second authors conducted the data extraction and then forwarded the results to the other authors to initiate the review meeting.

\subsubsection{Quality Assessment}

During the data extraction process, we encountered papers with insufficient information. To address this, we conducted a quality assessment in parallel to ensure the relevance of the papers to our objectives. This approach, also adopted in previous secondary studies~\cite{shamsujjoha2021developing, naveed2024model}, involved designing a set of assessment questions based on guidelines by Kitchenham et al.~\cite{kitchenham2022segress}. The quality assessment questions in our study are shown below:
\begin{itemize}
    \item \textbf{QA1}. Does this study clearly state \emph{how} requirements drive automated test generation?
    \item \textbf{QA2}. Does this study clearly state the \emph{aim} of REDAST?
    \item \textbf{QA3}. Does this study enable \emph{automation} in test generation?
    \item \textbf{QA4}. Does this study demonstrate the usability of the method from the perspective of methodology explanation, discussion, case examples, and experiments?
\end{itemize}
QA4 originates from an open perspective in the review process, where we focused on evaluation, discussion, and explanation. Our review also examined the study’s overall structure, including the methodology description, case studies, experiments, and analyses. The detailed results of the quality assessment are provided in the Appendix. Following this assessment, the final data extraction was based on 156 papers.

% \begin{table}[]
% \begin{tabular}{ll}
% \hline
% QA ID & QA Questions                                             \\ \hline
% Q01   & Does this study clearly state its aims?                  \\
% Q02   & Does this study clearly describe its methodology?        \\
% Q03   & Does this study involve automated test generation?       \\
% Q04   & Does this study include a promising evaluation?          \\
% Q05   & Does this study demonstrate the usability of the method? \\ \hline
% \end{tabular}%
% \caption{Questions for Quality Assessment}
% \label{table:qa}
% \end{table}

% automated quality assessment

% \textcolor{blue}{CA: Our search strategy focused on identifying requirements types first. We covered several sources, e.g., ~\cite{Pohl:11,wagner2019status} to identify different formats and notations of specifying requirements. However, this came out to be a long list, e.g., free-form NL requirements, semi-formal UML models, free-from textual use case models, UML class diagrams, UML activity diagrams, and so on. In this paper, we attempted to primarily focus on requirements-related aspects and not design-level information. Hence, we generalised our search string to include generic keywords, e.g., requirement*, use case*, and user stor*. We did so to avoid missing out on any papers, bringing too restrictive in our search strategy, and not creating a too-generic search string with all the aforementioned formats to avoid getting results beyond our review's scope.}


%% Use \subsection commands to start a subsection.



%\subsection{Study Selection}

% In this step, we further looked into the content of searched papers using our search strategy and applied our inclusion and exclusion criteria. Our filtering strategy aimed to pinpoint studies focused on requirements-driven system-level testing. Recognizing the presence of irrelevant papers in our search results, we established detailed selection criteria for preliminary inclusion and exclusion, as shown in Table \ref{table: criteria}. Specifically, we further developed the taxonomy schema to exclude two types of studies that did not meet the requirements for system-level testing: (1) studies supporting specification-driven test generation, such as UML-driven test generation, rather than requirements-driven testing, and (2) studies focusing on code-based test generation, such as requirement-driven code generation for unit testing.





\section{Results}

\begin{table*}[t]
\centering
\fontsize{11pt}{11pt}\selectfont
\begin{tabular}{lllllllllllll}
\toprule
\multicolumn{1}{c}{\textbf{task}} & \multicolumn{2}{c}{\textbf{Mir}} & \multicolumn{2}{c}{\textbf{Lai}} & \multicolumn{2}{c}{\textbf{Ziegen.}} & \multicolumn{2}{c}{\textbf{Cao}} & \multicolumn{2}{c}{\textbf{Alva-Man.}} & \multicolumn{1}{c}{\textbf{avg.}} & \textbf{\begin{tabular}[c]{@{}l@{}}avg.\\ rank\end{tabular}} \\
\multicolumn{1}{c}{\textbf{metrics}} & \multicolumn{1}{c}{\textbf{cor.}} & \multicolumn{1}{c}{\textbf{p-v.}} & \multicolumn{1}{c}{\textbf{cor.}} & \multicolumn{1}{c}{\textbf{p-v.}} & \multicolumn{1}{c}{\textbf{cor.}} & \multicolumn{1}{c}{\textbf{p-v.}} & \multicolumn{1}{c}{\textbf{cor.}} & \multicolumn{1}{c}{\textbf{p-v.}} & \multicolumn{1}{c}{\textbf{cor.}} & \multicolumn{1}{c}{\textbf{p-v.}} &  &  \\ \midrule
\textbf{S-Bleu} & 0.50 & 0.0 & 0.47 & 0.0 & 0.59 & 0.0 & 0.58 & 0.0 & 0.68 & 0.0 & 0.57 & 5.8 \\
\textbf{R-Bleu} & -- & -- & 0.27 & 0.0 & 0.30 & 0.0 & -- & -- & -- & -- & - &  \\
\textbf{S-Meteor} & 0.49 & 0.0 & 0.48 & 0.0 & 0.61 & 0.0 & 0.57 & 0.0 & 0.64 & 0.0 & 0.56 & 6.1 \\
\textbf{R-Meteor} & -- & -- & 0.34 & 0.0 & 0.26 & 0.0 & -- & -- & -- & -- & - &  \\
\textbf{S-Bertscore} & \textbf{0.53} & 0.0 & {\ul 0.80} & 0.0 & \textbf{0.70} & 0.0 & {\ul 0.66} & 0.0 & {\ul0.78} & 0.0 & \textbf{0.69} & \textbf{1.7} \\
\textbf{R-Bertscore} & -- & -- & 0.51 & 0.0 & 0.38 & 0.0 & -- & -- & -- & -- & - &  \\
\textbf{S-Bleurt} & {\ul 0.52} & 0.0 & {\ul 0.80} & 0.0 & 0.60 & 0.0 & \textbf{0.70} & 0.0 & \textbf{0.80} & 0.0 & {\ul 0.68} & {\ul 2.3} \\
\textbf{R-Bleurt} & -- & -- & 0.59 & 0.0 & -0.05 & 0.13 & -- & -- & -- & -- & - &  \\
\textbf{S-Cosine} & 0.51 & 0.0 & 0.69 & 0.0 & {\ul 0.62} & 0.0 & 0.61 & 0.0 & 0.65 & 0.0 & 0.62 & 4.4 \\
\textbf{R-Cosine} & -- & -- & 0.40 & 0.0 & 0.29 & 0.0 & -- & -- & -- & -- & - & \\ \midrule
\textbf{QuestEval} & 0.23 & 0.0 & 0.25 & 0.0 & 0.49 & 0.0 & 0.47 & 0.0 & 0.62 & 0.0 & 0.41 & 9.0 \\
\textbf{LLaMa3} & 0.36 & 0.0 & \textbf{0.84} & 0.0 & {\ul{0.62}} & 0.0 & 0.61 & 0.0 &  0.76 & 0.0 & 0.64 & 3.6 \\
\textbf{our (3b)} & 0.49 & 0.0 & 0.73 & 0.0 & 0.54 & 0.0 & 0.53 & 0.0 & 0.7 & 0.0 & 0.60 & 5.8 \\
\textbf{our (8b)} & 0.48 & 0.0 & 0.73 & 0.0 & 0.52 & 0.0 & 0.53 & 0.0 & 0.7 & 0.0 & 0.59 & 6.3 \\  \bottomrule
\end{tabular}
\caption{Pearson correlation on human evaluation on system output. `R-': reference-based. `S-': source-based.}
\label{tab:sys}
\end{table*}



\begin{table}%[]
\centering
\fontsize{11pt}{11pt}\selectfont
\begin{tabular}{llllll}
\toprule
\multicolumn{1}{c}{\textbf{task}} & \multicolumn{1}{c}{\textbf{Lai}} & \multicolumn{1}{c}{\textbf{Zei.}} & \multicolumn{1}{c}{\textbf{Scia.}} & \textbf{} & \textbf{} \\ 
\multicolumn{1}{c}{\textbf{metrics}} & \multicolumn{1}{c}{\textbf{cor.}} & \multicolumn{1}{c}{\textbf{cor.}} & \multicolumn{1}{c}{\textbf{cor.}} & \textbf{avg.} & \textbf{\begin{tabular}[c]{@{}l@{}}avg.\\ rank\end{tabular}} \\ \midrule
\textbf{S-Bleu} & 0.40 & 0.40 & 0.19* & 0.33 & 7.67 \\
\textbf{S-Meteor} & 0.41 & 0.42 & 0.16* & 0.33 & 7.33 \\
\textbf{S-BertS.} & {\ul0.58} & 0.47 & 0.31 & 0.45 & 3.67 \\
\textbf{S-Bleurt} & 0.45 & {\ul 0.54} & {\ul 0.37} & 0.45 & {\ul 3.33} \\
\textbf{S-Cosine} & 0.56 & 0.52 & 0.3 & {\ul 0.46} & {\ul 3.33} \\ \midrule
\textbf{QuestE.} & 0.27 & 0.35 & 0.06* & 0.23 & 9.00 \\
\textbf{LlaMA3} & \textbf{0.6} & \textbf{0.67} & \textbf{0.51} & \textbf{0.59} & \textbf{1.0} \\
\textbf{Our (3b)} & 0.51 & 0.49 & 0.23* & 0.39 & 4.83 \\
\textbf{Our (8b)} & 0.52 & 0.49 & 0.22* & 0.43 & 4.83 \\ \bottomrule
\end{tabular}
\caption{Pearson correlation on human ratings on reference output. *not significant; we cannot reject the null hypothesis of zero correlation}
\label{tab:ref}
\end{table}


\begin{table*}%[]
\centering
\fontsize{11pt}{11pt}\selectfont
\begin{tabular}{lllllllll}
\toprule
\textbf{task} & \multicolumn{1}{c}{\textbf{ALL}} & \multicolumn{1}{c}{\textbf{sentiment}} & \multicolumn{1}{c}{\textbf{detoxify}} & \multicolumn{1}{c}{\textbf{catchy}} & \multicolumn{1}{c}{\textbf{polite}} & \multicolumn{1}{c}{\textbf{persuasive}} & \multicolumn{1}{c}{\textbf{formal}} & \textbf{\begin{tabular}[c]{@{}l@{}}avg. \\ rank\end{tabular}} \\
\textbf{metrics} & \multicolumn{1}{c}{\textbf{cor.}} & \multicolumn{1}{c}{\textbf{cor.}} & \multicolumn{1}{c}{\textbf{cor.}} & \multicolumn{1}{c}{\textbf{cor.}} & \multicolumn{1}{c}{\textbf{cor.}} & \multicolumn{1}{c}{\textbf{cor.}} & \multicolumn{1}{c}{\textbf{cor.}} &  \\ \midrule
\textbf{S-Bleu} & -0.17 & -0.82 & -0.45 & -0.12* & -0.1* & -0.05 & -0.21 & 8.42 \\
\textbf{R-Bleu} & - & -0.5 & -0.45 &  &  &  &  &  \\
\textbf{S-Meteor} & -0.07* & -0.55 & -0.4 & -0.01* & 0.1* & -0.16 & -0.04* & 7.67 \\
\textbf{R-Meteor} & - & -0.17* & -0.39 & - & - & - & - & - \\
\textbf{S-BertScore} & 0.11 & -0.38 & -0.07* & -0.17* & 0.28 & 0.12 & 0.25 & 6.0 \\
\textbf{R-BertScore} & - & -0.02* & -0.21* & - & - & - & - & - \\
\textbf{S-Bleurt} & 0.29 & 0.05* & 0.45 & 0.06* & 0.29 & 0.23 & 0.46 & 4.2 \\
\textbf{R-Bleurt} & - &  0.21 & 0.38 & - & - & - & - & - \\
\textbf{S-Cosine} & 0.01* & -0.5 & -0.13* & -0.19* & 0.05* & -0.05* & 0.15* & 7.42 \\
\textbf{R-Cosine} & - & -0.11* & -0.16* & - & - & - & - & - \\ \midrule
\textbf{QuestEval} & 0.21 & {\ul{0.29}} & 0.23 & 0.37 & 0.19* & 0.35 & 0.14* & 4.67 \\
\textbf{LlaMA3} & \textbf{0.82} & \textbf{0.80} & \textbf{0.72} & \textbf{0.84} & \textbf{0.84} & \textbf{0.90} & \textbf{0.88} & \textbf{1.00} \\
\textbf{Our (3b)} & 0.47 & -0.11* & 0.37 & 0.61 & 0.53 & 0.54 & 0.66 & 3.5 \\
\textbf{Our (8b)} & {\ul{0.57}} & 0.09* & {\ul 0.49} & {\ul 0.72} & {\ul 0.64} & {\ul 0.62} & {\ul 0.67} & {\ul 2.17} \\ \bottomrule
\end{tabular}
\caption{Pearson correlation on human ratings on our constructed test set. 'R-': reference-based. 'S-': source-based. *not significant; we cannot reject the null hypothesis of zero correlation}
\label{tab:con}
\end{table*}

\section{Results}
We benchmark the different metrics on the different datasets using correlation to human judgement. For content preservation, we show results split on data with system output, reference output and our constructed test set: we show that the data source for evaluation leads to different conclusions on the metrics. In addition, we examine whether the metrics can rank style transfer systems similar to humans. On style strength, we likewise show correlations between human judgment and zero-shot evaluation approaches. When applicable, we summarize results by reporting the average correlation. And the average ranking of the metric per dataset (by ranking which metric obtains the highest correlation to human judgement per dataset). 

\subsection{Content preservation}
\paragraph{How do data sources affect the conclusion on best metric?}
The conclusions about the metrics' performance change radically depending on whether we use system output data, reference output, or our constructed test set. Ideally, a good metric correlates highly with humans on any data source. Ideally, for meta-evaluation, a metric should correlate consistently across all data sources, but the following shows that the correlations indicate different things, and the conclusion on the best metric should be drawn carefully.

Looking at the metrics correlations with humans on the data source with system output (Table~\ref{tab:sys}), we see a relatively high correlation for many of the metrics on many tasks. The overall best metrics are S-BertScore and S-BLEURT (avg+avg rank). We see no notable difference in our method of using the 3B or 8B model as the backbone.

Examining the average correlations based on data with reference output (Table~\ref{tab:ref}), now the zero-shoot prompting with LlaMA3 70B is the best-performing approach ($0.59$ avg). Tied for second place are source-based cosine embedding ($0.46$ avg), BLEURT ($0.45$ avg) and BertScore ($0.45$ avg). Our method follows on a 5. place: here, the 8b version (($0.43$ avg)) shows a bit stronger results than 3b ($0.39$ avg). The fact that the conclusions change, whether looking at reference or system output, confirms the observations made by \citet{scialom-etal-2021-questeval} on simplicity transfer.   

Now consider the results on our test set (Table~\ref{tab:con}): Several metrics show low or no correlation; we even see a significantly negative correlation for some metrics on ALL (BLEU) and for specific subparts of our test set for BLEU, Meteor, BertScore, Cosine. On the other end, LlaMA3 70B is again performing best, showing strong results ($0.82$ in ALL). The runner-up is now our 8B method, with a gap to the 3B version ($0.57$ vs $0.47$ in ALL). Note our method still shows zero correlation for the sentiment task. After, ranks BLEURT ($0.29$), QuestEval ($0.21$), BertScore ($0.11$), Cosine ($0.01$).  

On our test set, we find that some metrics that correlate relatively well on the other datasets, now exhibit low correlation. Hence, with our test set, we can now support the logical reasoning with data evidence: Evaluation of content preservation for style transfer needs to take the style shift into account. This conclusion could not be drawn using the existing data sources: We hypothesise that for the data with system-based output, successful output happens to be very similar to the source sentence and vice versa, and reference-based output might not contain server mistakes as they are gold references. Thus, none of the existing data sources tests the limits of the metrics.  


\paragraph{How do reference-based metrics compare to source-based ones?} Reference-based metrics show a lower correlation than the source-based counterpart for all metrics on both datasets with ratings on references (Table~\ref{tab:sys}). As discussed previously, reference-based metrics for style transfer have the drawback that many different good solutions on a rewrite might exist and not only one similar to a reference.


\paragraph{How well can the metrics rank the performance of style transfer methods?}
We compare the metrics' ability to judge the best style transfer methods w.r.t. the human annotations: Several of the data sources contain samples from different style transfer systems. In order to use metrics to assess the quality of the style transfer system, metrics should correctly find the best-performing system. Hence, we evaluate whether the metrics for content preservation provide the same system ranking as human evaluators. We take the mean of the score for every output on each system and the mean of the human annotations; we compare the systems using the Kendall's Tau correlation. 

We find only the evaluation using the dataset Mir, Lai, and Ziegen to result in significant correlations, probably because of sparsity in a number of system tests (App.~\ref{app:dataset}). Our method (8b) is the only metric providing a perfect ranking of the style transfer system on the Lai data, and Llama3 70B the only one on the Ziegen data. Results in App.~\ref{app:results}. 


\subsection{Style strength results}
%Evaluating style strengths is a challenging task. 
Llama3 70B shows better overall results than our method. However, our method scores higher than Llama3 70B on 2 out of 6 datasets, but it also exhibits zero correlation on one task (Table~\ref{tab:styleresults}).%More work i s needed on evaluating style strengths. 
 
\begin{table}%[]
\fontsize{11pt}{11pt}\selectfont
\begin{tabular}{lccc}
\toprule
\multicolumn{1}{c}{\textbf{}} & \textbf{LlaMA3} & \textbf{Our (3b)} & \textbf{Our (8b)} \\ \midrule
\textbf{Mir} & 0.46 & 0.54 & \textbf{0.57} \\
\textbf{Lai} & \textbf{0.57} & 0.18 & 0.19 \\
\textbf{Ziegen.} & 0.25 & 0.27 & \textbf{0.32} \\
\textbf{Alva-M.} & \textbf{0.59} & 0.03* & 0.02* \\
\textbf{Scialom} & \textbf{0.62} & 0.45 & 0.44 \\
\textbf{\begin{tabular}[c]{@{}l@{}}Our Test\end{tabular}} & \textbf{0.63} & 0.46 & 0.48 \\ \bottomrule
\end{tabular}
\caption{Style strength: Pearson correlation to human ratings. *not significant; we cannot reject the null hypothesis of zero corelation}
\label{tab:styleresults}
\end{table}

\subsection{Ablation}
We conduct several runs of the methods using LLMs with variations in instructions/prompts (App.~\ref{app:method}). We observe that the lower the correlation on a task, the higher the variation between the different runs. For our method, we only observe low variance between the runs.
None of the variations leads to different conclusions of the meta-evaluation. Results in App.~\ref{app:results}.



\section{Ablation Studies}
\label{sec:ablation_studies}
% \subsection{Impact of System Components}
% To better understand the contributions of different input components in our model, we conduct an ablation study where we evaluate the importance of image embeddings, auxiliary features, and physics-inspired features. The goal of this study is to quantify the effect of each component on the overall performance of our nowcasting and forecasting frameworks. 

% We perform this experiment by training models on the TSI 2020 dataset and testing them on the TSI 2021 in five different configurations: (i) using only image embeddings from the vision encoder, (ii) combining image embeddings with the auxiliary data available in the dataset, (iii) combining image embeddings with the physics-inspired features, (iv) using only auxiliary and physics-inspired features without image embeddings, and (v) using all three components together. 

% As shown in Table~\ref{tab:ablation_components}, the results highlight that image embeddings are essential for achieving high predictive accuracy. The model trained with only auxiliary and physics-inspired features performs significantly worse, reinforcing the importance of visual information for solar irradiance forecasting. Furthermore, incorporating auxiliary or physics-inspired features alongside image embeddings improves performance, as observed in the models trained with either auxiliary data or physics-based inputs in addition to embeddings. This demonstrates that the auxiliary values and physics-based features individually contribute to our architecture, each providing valuable information that enhances predictive accuracy. However, the best performance is achieved when all three components are used together, demonstrating that leveraging all available data sources yields the most robust and reliable predictions. Notably, our engineered physics-inspired features play a crucial role in capturing atmospheric patterns such as the actual position of the sun more accurately. Our final architecture effectively combines all available information, leveraging both available and engineered features to achieve the best performance.


\subsection{Investigating Different Vision Encoders}
We examine the impact of different vision models on SPIRIT's performance, also highlighting the versatility of our system across different foundation models. We evaluate the CNN-based ResNet-152 \cite{resnet}, the vision transformer-based DINOv2 Giant \cite{dinov2}, and our implementation using Google ViT-Huge \cite{google_vit}. Results summarized in Table \ref{tab:encoders_ablation}, demonstrate that the ViT-based models consistently outperform the ResNet-152 CNN model, which can be attributed to the superior capability of ViT architectures in capturing global image features \cite{vits_gt_cnns1}.
\begin{table}[h]
  \caption{We explore the impact of using different vision encoders on the overall model performance for nowcasting and forecasting, with training on TSI 2020 and testing on TSI 2021, measured by nMAP error.}
  \label{tab:encoders_ablation}
  \centering
  \begin{tabular}{cccccc}
    \toprule
    Model & \multicolumn{1}{c}{Nowcast} & \multicolumn{4}{c}{Forecast} \\
    \cmidrule(lr){3-6}
    & &  +1hr & +2hr & +3hr & +4hr \\
    \midrule
    ResNet-152 & 10.50 & 24.56 & 27.82 & 31.23 & 35.85 \\
    DINOv2 Giant & 9.74 & 21.22 & 23.56 & 27.93 & 33.13 \\
    Google ViT-Huge & \textbf{9.32} & \textbf{19.96} & \textbf{22.64} & \textbf{26.30} & \textbf{31.58} \\
    \bottomrule
  \end{tabular}
\end{table}



\subsection{Foundation Model Size}
Table \ref{tab:encoder_sizes_ablation} presents an analysis of how the size of the foundation model influences the performance of our nowcasting and forecasting architectures. Although increasing model size has traditionally been linked to performance gains, we observe that beyond a certain threshold, further scaling yields diminishing returns. This suggests that larger models do not always lead to better performance. In fact, models with 304M and 86M parameters outperform their larger counterparts with 632M parameters in forecasting and nowcasting, respectively. This aligns with recent work, which highlights that adjusting model size based on a computational budget, rather than blindly increasing model size, can lead to more efficient architectures with reduced inference costs \cite{getting_vit_in_shape}.

\begin{table}[h!]
  \caption{We evaluate the impact of varying size of the Google ViT vision encoder on the overall performance of the model for both nowcasting and forecasting tasks, with training on TSI 2020 and testing performed on TSI 2021.}
  \label{tab:encoder_sizes_ablation}
  \begin{tabular}{cccccc cccc}
    \toprule
    Model Parameters & \multicolumn{1}{c}{Nowcast} & \multicolumn{4}{c}{Forecast} \\
     \cmidrule(lr){3-6}
    & & +1hr & +2hr & +3hr & +4hr \\
    \midrule
    86M & \textbf{9.14} & 21.92 & 24.07 & 28.73 & 34.50 \\
    304M & 9.45 & \textbf{19.58} & \textbf{21.95} & \textbf{25.54} & \textbf{30.60} \\
    632M & 9.32 & 19.96 & 22.64 & 26.30 & 31.58 \\
    \bottomrule
  \end{tabular}
\end{table}





% 
% \begin{table}[H]
%   \caption{Impact of Different Vision Encoders on Performance: This study examines the effect of employing different vision encoders on the overall performance of the model for both nowcasting and forecasting tasks, with training conducted on TSI 2020 and testing on TSI 2021.}
%   \label{tab:encoders_ablation}
%   \begin{tabular}{ccccccc cccc}
%     \toprule
%     Model & Size & \multicolumn{1}{c}{Nowcast} & \multicolumn{4}{c}{Forecast} \\
%      \cmidrule(lr){4-7}
%     & & +1hr & +2hr & +3hr & +4hr \\
%     \midrule
%     ResNet-152 & 115M & 10.50 & 27.50 & 33.20 & 38.56 & 44.47 \\
%     DINOv2 Giant & 1.14B & 9.74 & \textbf{24.52} & \textbf{29.98} & \textbf{35.63} & 43.13 \\
%     Google ViT-Huge & 632M & \textbf{9.32} & 24.77 & 30.86 & 36.83 & \textbf{42.54} \\
%     Apple AIMv2 & 2.7B & - & - & - & - & - \\
%     \bottomrule
%   \end{tabular}
% \end{table}

\begin{table}[H]
  \caption{Impact of Different Vision Encoders on Performance: We examine the effect of employing different vision encoders on the overall performance of the model for both nowcasting and forecasting tasks, with training conducted on TSI 2020 and testing on TSI 2021.}
  \label{tab:encoders_ablation}
  \centering
  \begin{tabular}{cccccc}
    \toprule
    Model & \multicolumn{1}{c}{Nowcast} & \multicolumn{4}{c}{Forecast} \\
    \cmidrule(lr){2-6}
    & +1hr & +2hr & +3hr & +4hr \\
    \midrule
    ResNet-152 & 10.50 & 27.50 & 33.20 & 38.56 & 44.47 \\
    DINOv2 Giant & 9.74 & \textbf{24.52} & \textbf{29.98} & \textbf{35.63} & 43.13 \\
    Google ViT-Huge & \textbf{9.32} & 24.77 & 30.86 & 36.83 & \textbf{42.54} \\
    Apple AIMv2 & - & - & - & - & - \\
    \bottomrule
  \end{tabular}
\end{table}



\begin{table}[H]
  \caption{Impact of Foundation Model Size on Performance: We explore the impact of varying size of the AIMv2 vision encoder on the overall performance of the model for both nowcasting and forecasting tasks, with training on TSI 2020 and testing performed on TSI 2021.}
  \label{tab:encoder_sizes_ablation}
  \begin{tabular}{cccccc cccc}
    \toprule
    Model Parameters & \multicolumn{1}{c}{Nowcast} & \multicolumn{4}{c}{Forecast} \\
     \cmidrule(lr){3-6}
    & & +1hr & +2hr & +3hr & +4hr \\
    \midrule
    300M & - & - & - & - & - \\
    600M & - & - & - & - & - \\
    1.2B & - & - & - & - & - \\
    2.7B & - & - & - & - & - \\
    \bottomrule
  \end{tabular}
\end{table}


\section{Conclusion}

This work addresses a critical challenge in solar irradiance forecasting: adapting models to new geographic locations with no prior data. By utilizing transfer learning and pre-trained models, SPIRIT generalizes well to new locations, reducing the reliance on large, location-specific datasets. As more site-specific data becomes available post-deployment, the system can be effectively fine-tuned, improving prediction accuracy and supporting better energy yield estimates and operational planning. Additionally, SPIRIT’s modular design allows for the seamless integration of any emerging vision models, ensuring that the framework remains up-to-date with the latest advancements. This scalable solution for solar irradiance forecasting can accelerate the deployment of solar farms—particularly in remote and emerging markets. SPIRIT supports the transition to renewable energy by enhancing the reliability, cost-effectiveness, and accessibility of solar energy generation.





% Our system offers significant practical benefits for solar panel deployment. In new installations, where historical data is unavailable, our approach generalizes well. As additional site-specific data becomes available, the system can be fine-tuned to further enhance prediction accuracy, thereby improving energy yield estimates and operational planning. This inherent adaptability reduces reliance on extensive datasets, making the technology especially valuable in emerging markets and remote regions. Ultimately, our system facilitates more efficient, reliable, and cost-effective solar energy generation, supporting the broader adoption of renewable energy solutions.

% This work addresses a critical challenge in the domain of solar irradiance forecasting — creating models that can adapt seamlessly to new geographic locations and camera setups, even with little to no prior data. By leveraging foundation models and integrating physics-inspired feature engineering, we developed a comprehensive system that achieves state-of-the-art generalizability, outperforming traditional approaches requiring years of site-specific data. Our contributions offer a scalable and efficient solution that enables faster deployment of solar farms in new locations, facilitating a smoother and more accessible transition to renewable energy.

\section{Future Work and Limitations}
One key limitation is that the datasets used for evaluation are all from North America, largely due to the limited availability of publicly accessible datasets from other regions. Specifically, the solar movement patterns and dynamics change in the Southern Hemisphere and need to be studied. To improve the generalizability of our system, future work will incorporate data from other continents. Additionally, while our model performs well, the use of foundation models introduces real-time inference costs and computational overheads. Future efforts will focus on reducing computational efficiency, enabling deployment on resource-constrained edge devices without sacrificing accuracy.









\bibliographystyle{ACM-Reference-Format}
\bibliography{main}

\appendix
\subsection{Lloyd-Max Algorithm}
\label{subsec:Lloyd-Max}
For a given quantization bitwidth $B$ and an operand $\bm{X}$, the Lloyd-Max algorithm finds $2^B$ quantization levels $\{\hat{x}_i\}_{i=1}^{2^B}$ such that quantizing $\bm{X}$ by rounding each scalar in $\bm{X}$ to the nearest quantization level minimizes the quantization MSE. 

The algorithm starts with an initial guess of quantization levels and then iteratively computes quantization thresholds $\{\tau_i\}_{i=1}^{2^B-1}$ and updates quantization levels $\{\hat{x}_i\}_{i=1}^{2^B}$. Specifically, at iteration $n$, thresholds are set to the midpoints of the previous iteration's levels:
\begin{align*}
    \tau_i^{(n)}=\frac{\hat{x}_i^{(n-1)}+\hat{x}_{i+1}^{(n-1)}}2 \text{ for } i=1\ldots 2^B-1
\end{align*}
Subsequently, the quantization levels are re-computed as conditional means of the data regions defined by the new thresholds:
\begin{align*}
    \hat{x}_i^{(n)}=\mathbb{E}\left[ \bm{X} \big| \bm{X}\in [\tau_{i-1}^{(n)},\tau_i^{(n)}] \right] \text{ for } i=1\ldots 2^B
\end{align*}
where to satisfy boundary conditions we have $\tau_0=-\infty$ and $\tau_{2^B}=\infty$. The algorithm iterates the above steps until convergence.

Figure \ref{fig:lm_quant} compares the quantization levels of a $7$-bit floating point (E3M3) quantizer (left) to a $7$-bit Lloyd-Max quantizer (right) when quantizing a layer of weights from the GPT3-126M model at a per-tensor granularity. As shown, the Lloyd-Max quantizer achieves substantially lower quantization MSE. Further, Table \ref{tab:FP7_vs_LM7} shows the superior perplexity achieved by Lloyd-Max quantizers for bitwidths of $7$, $6$ and $5$. The difference between the quantizers is clear at 5 bits, where per-tensor FP quantization incurs a drastic and unacceptable increase in perplexity, while Lloyd-Max quantization incurs a much smaller increase. Nevertheless, we note that even the optimal Lloyd-Max quantizer incurs a notable ($\sim 1.5$) increase in perplexity due to the coarse granularity of quantization. 

\begin{figure}[h]
  \centering
  \includegraphics[width=0.7\linewidth]{sections/figures/LM7_FP7.pdf}
  \caption{\small Quantization levels and the corresponding quantization MSE of Floating Point (left) vs Lloyd-Max (right) Quantizers for a layer of weights in the GPT3-126M model.}
  \label{fig:lm_quant}
\end{figure}

\begin{table}[h]\scriptsize
\begin{center}
\caption{\label{tab:FP7_vs_LM7} \small Comparing perplexity (lower is better) achieved by floating point quantizers and Lloyd-Max quantizers on a GPT3-126M model for the Wikitext-103 dataset.}
\begin{tabular}{c|cc|c}
\hline
 \multirow{2}{*}{\textbf{Bitwidth}} & \multicolumn{2}{|c|}{\textbf{Floating-Point Quantizer}} & \textbf{Lloyd-Max Quantizer} \\
 & Best Format & Wikitext-103 Perplexity & Wikitext-103 Perplexity \\
\hline
7 & E3M3 & 18.32 & 18.27 \\
6 & E3M2 & 19.07 & 18.51 \\
5 & E4M0 & 43.89 & 19.71 \\
\hline
\end{tabular}
\end{center}
\end{table}

\subsection{Proof of Local Optimality of LO-BCQ}
\label{subsec:lobcq_opt_proof}
For a given block $\bm{b}_j$, the quantization MSE during LO-BCQ can be empirically evaluated as $\frac{1}{L_b}\lVert \bm{b}_j- \bm{\hat{b}}_j\rVert^2_2$ where $\bm{\hat{b}}_j$ is computed from equation (\ref{eq:clustered_quantization_definition}) as $C_{f(\bm{b}_j)}(\bm{b}_j)$. Further, for a given block cluster $\mathcal{B}_i$, we compute the quantization MSE as $\frac{1}{|\mathcal{B}_{i}|}\sum_{\bm{b} \in \mathcal{B}_{i}} \frac{1}{L_b}\lVert \bm{b}- C_i^{(n)}(\bm{b})\rVert^2_2$. Therefore, at the end of iteration $n$, we evaluate the overall quantization MSE $J^{(n)}$ for a given operand $\bm{X}$ composed of $N_c$ block clusters as:
\begin{align*}
    \label{eq:mse_iter_n}
    J^{(n)} = \frac{1}{N_c} \sum_{i=1}^{N_c} \frac{1}{|\mathcal{B}_{i}^{(n)}|}\sum_{\bm{v} \in \mathcal{B}_{i}^{(n)}} \frac{1}{L_b}\lVert \bm{b}- B_i^{(n)}(\bm{b})\rVert^2_2
\end{align*}

At the end of iteration $n$, the codebooks are updated from $\mathcal{C}^{(n-1)}$ to $\mathcal{C}^{(n)}$. However, the mapping of a given vector $\bm{b}_j$ to quantizers $\mathcal{C}^{(n)}$ remains as  $f^{(n)}(\bm{b}_j)$. At the next iteration, during the vector clustering step, $f^{(n+1)}(\bm{b}_j)$ finds new mapping of $\bm{b}_j$ to updated codebooks $\mathcal{C}^{(n)}$ such that the quantization MSE over the candidate codebooks is minimized. Therefore, we obtain the following result for $\bm{b}_j$:
\begin{align*}
\frac{1}{L_b}\lVert \bm{b}_j - C_{f^{(n+1)}(\bm{b}_j)}^{(n)}(\bm{b}_j)\rVert^2_2 \le \frac{1}{L_b}\lVert \bm{b}_j - C_{f^{(n)}(\bm{b}_j)}^{(n)}(\bm{b}_j)\rVert^2_2
\end{align*}

That is, quantizing $\bm{b}_j$ at the end of the block clustering step of iteration $n+1$ results in lower quantization MSE compared to quantizing at the end of iteration $n$. Since this is true for all $\bm{b} \in \bm{X}$, we assert the following:
\begin{equation}
\begin{split}
\label{eq:mse_ineq_1}
    \tilde{J}^{(n+1)} &= \frac{1}{N_c} \sum_{i=1}^{N_c} \frac{1}{|\mathcal{B}_{i}^{(n+1)}|}\sum_{\bm{b} \in \mathcal{B}_{i}^{(n+1)}} \frac{1}{L_b}\lVert \bm{b} - C_i^{(n)}(b)\rVert^2_2 \le J^{(n)}
\end{split}
\end{equation}
where $\tilde{J}^{(n+1)}$ is the the quantization MSE after the vector clustering step at iteration $n+1$.

Next, during the codebook update step (\ref{eq:quantizers_update}) at iteration $n+1$, the per-cluster codebooks $\mathcal{C}^{(n)}$ are updated to $\mathcal{C}^{(n+1)}$ by invoking the Lloyd-Max algorithm \citep{Lloyd}. We know that for any given value distribution, the Lloyd-Max algorithm minimizes the quantization MSE. Therefore, for a given vector cluster $\mathcal{B}_i$ we obtain the following result:

\begin{equation}
    \frac{1}{|\mathcal{B}_{i}^{(n+1)}|}\sum_{\bm{b} \in \mathcal{B}_{i}^{(n+1)}} \frac{1}{L_b}\lVert \bm{b}- C_i^{(n+1)}(\bm{b})\rVert^2_2 \le \frac{1}{|\mathcal{B}_{i}^{(n+1)}|}\sum_{\bm{b} \in \mathcal{B}_{i}^{(n+1)}} \frac{1}{L_b}\lVert \bm{b}- C_i^{(n)}(\bm{b})\rVert^2_2
\end{equation}

The above equation states that quantizing the given block cluster $\mathcal{B}_i$ after updating the associated codebook from $C_i^{(n)}$ to $C_i^{(n+1)}$ results in lower quantization MSE. Since this is true for all the block clusters, we derive the following result: 
\begin{equation}
\begin{split}
\label{eq:mse_ineq_2}
     J^{(n+1)} &= \frac{1}{N_c} \sum_{i=1}^{N_c} \frac{1}{|\mathcal{B}_{i}^{(n+1)}|}\sum_{\bm{b} \in \mathcal{B}_{i}^{(n+1)}} \frac{1}{L_b}\lVert \bm{b}- C_i^{(n+1)}(\bm{b})\rVert^2_2  \le \tilde{J}^{(n+1)}   
\end{split}
\end{equation}

Following (\ref{eq:mse_ineq_1}) and (\ref{eq:mse_ineq_2}), we find that the quantization MSE is non-increasing for each iteration, that is, $J^{(1)} \ge J^{(2)} \ge J^{(3)} \ge \ldots \ge J^{(M)}$ where $M$ is the maximum number of iterations. 
%Therefore, we can say that if the algorithm converges, then it must be that it has converged to a local minimum. 
\hfill $\blacksquare$


\begin{figure}
    \begin{center}
    \includegraphics[width=0.5\textwidth]{sections//figures/mse_vs_iter.pdf}
    \end{center}
    \caption{\small NMSE vs iterations during LO-BCQ compared to other block quantization proposals}
    \label{fig:nmse_vs_iter}
\end{figure}

Figure \ref{fig:nmse_vs_iter} shows the empirical convergence of LO-BCQ across several block lengths and number of codebooks. Also, the MSE achieved by LO-BCQ is compared to baselines such as MXFP and VSQ. As shown, LO-BCQ converges to a lower MSE than the baselines. Further, we achieve better convergence for larger number of codebooks ($N_c$) and for a smaller block length ($L_b$), both of which increase the bitwidth of BCQ (see Eq \ref{eq:bitwidth_bcq}).


\subsection{Additional Accuracy Results}
%Table \ref{tab:lobcq_config} lists the various LOBCQ configurations and their corresponding bitwidths.
\begin{table}
\setlength{\tabcolsep}{4.75pt}
\begin{center}
\caption{\label{tab:lobcq_config} Various LO-BCQ configurations and their bitwidths.}
\begin{tabular}{|c||c|c|c|c||c|c||c|} 
\hline
 & \multicolumn{4}{|c||}{$L_b=8$} & \multicolumn{2}{|c||}{$L_b=4$} & $L_b=2$ \\
 \hline
 \backslashbox{$L_A$\kern-1em}{\kern-1em$N_c$} & 2 & 4 & 8 & 16 & 2 & 4 & 2 \\
 \hline
 64 & 4.25 & 4.375 & 4.5 & 4.625 & 4.375 & 4.625 & 4.625\\
 \hline
 32 & 4.375 & 4.5 & 4.625& 4.75 & 4.5 & 4.75 & 4.75 \\
 \hline
 16 & 4.625 & 4.75& 4.875 & 5 & 4.75 & 5 & 5 \\
 \hline
\end{tabular}
\end{center}
\end{table}

%\subsection{Perplexity achieved by various LO-BCQ configurations on Wikitext-103 dataset}

\begin{table} \centering
\begin{tabular}{|c||c|c|c|c||c|c||c|} 
\hline
 $L_b \rightarrow$& \multicolumn{4}{c||}{8} & \multicolumn{2}{c||}{4} & 2\\
 \hline
 \backslashbox{$L_A$\kern-1em}{\kern-1em$N_c$} & 2 & 4 & 8 & 16 & 2 & 4 & 2  \\
 %$N_c \rightarrow$ & 2 & 4 & 8 & 16 & 2 & 4 & 2 \\
 \hline
 \hline
 \multicolumn{8}{c}{GPT3-1.3B (FP32 PPL = 9.98)} \\ 
 \hline
 \hline
 64 & 10.40 & 10.23 & 10.17 & 10.15 &  10.28 & 10.18 & 10.19 \\
 \hline
 32 & 10.25 & 10.20 & 10.15 & 10.12 &  10.23 & 10.17 & 10.17 \\
 \hline
 16 & 10.22 & 10.16 & 10.10 & 10.09 &  10.21 & 10.14 & 10.16 \\
 \hline
  \hline
 \multicolumn{8}{c}{GPT3-8B (FP32 PPL = 7.38)} \\ 
 \hline
 \hline
 64 & 7.61 & 7.52 & 7.48 &  7.47 &  7.55 &  7.49 & 7.50 \\
 \hline
 32 & 7.52 & 7.50 & 7.46 &  7.45 &  7.52 &  7.48 & 7.48  \\
 \hline
 16 & 7.51 & 7.48 & 7.44 &  7.44 &  7.51 &  7.49 & 7.47  \\
 \hline
\end{tabular}
\caption{\label{tab:ppl_gpt3_abalation} Wikitext-103 perplexity across GPT3-1.3B and 8B models.}
\end{table}

\begin{table} \centering
\begin{tabular}{|c||c|c|c|c||} 
\hline
 $L_b \rightarrow$& \multicolumn{4}{c||}{8}\\
 \hline
 \backslashbox{$L_A$\kern-1em}{\kern-1em$N_c$} & 2 & 4 & 8 & 16 \\
 %$N_c \rightarrow$ & 2 & 4 & 8 & 16 & 2 & 4 & 2 \\
 \hline
 \hline
 \multicolumn{5}{|c|}{Llama2-7B (FP32 PPL = 5.06)} \\ 
 \hline
 \hline
 64 & 5.31 & 5.26 & 5.19 & 5.18  \\
 \hline
 32 & 5.23 & 5.25 & 5.18 & 5.15  \\
 \hline
 16 & 5.23 & 5.19 & 5.16 & 5.14  \\
 \hline
 \multicolumn{5}{|c|}{Nemotron4-15B (FP32 PPL = 5.87)} \\ 
 \hline
 \hline
 64  & 6.3 & 6.20 & 6.13 & 6.08  \\
 \hline
 32  & 6.24 & 6.12 & 6.07 & 6.03  \\
 \hline
 16  & 6.12 & 6.14 & 6.04 & 6.02  \\
 \hline
 \multicolumn{5}{|c|}{Nemotron4-340B (FP32 PPL = 3.48)} \\ 
 \hline
 \hline
 64 & 3.67 & 3.62 & 3.60 & 3.59 \\
 \hline
 32 & 3.63 & 3.61 & 3.59 & 3.56 \\
 \hline
 16 & 3.61 & 3.58 & 3.57 & 3.55 \\
 \hline
\end{tabular}
\caption{\label{tab:ppl_llama7B_nemo15B} Wikitext-103 perplexity compared to FP32 baseline in Llama2-7B and Nemotron4-15B, 340B models}
\end{table}

%\subsection{Perplexity achieved by various LO-BCQ configurations on MMLU dataset}


\begin{table} \centering
\begin{tabular}{|c||c|c|c|c||c|c|c|c|} 
\hline
 $L_b \rightarrow$& \multicolumn{4}{c||}{8} & \multicolumn{4}{c||}{8}\\
 \hline
 \backslashbox{$L_A$\kern-1em}{\kern-1em$N_c$} & 2 & 4 & 8 & 16 & 2 & 4 & 8 & 16  \\
 %$N_c \rightarrow$ & 2 & 4 & 8 & 16 & 2 & 4 & 2 \\
 \hline
 \hline
 \multicolumn{5}{|c|}{Llama2-7B (FP32 Accuracy = 45.8\%)} & \multicolumn{4}{|c|}{Llama2-70B (FP32 Accuracy = 69.12\%)} \\ 
 \hline
 \hline
 64 & 43.9 & 43.4 & 43.9 & 44.9 & 68.07 & 68.27 & 68.17 & 68.75 \\
 \hline
 32 & 44.5 & 43.8 & 44.9 & 44.5 & 68.37 & 68.51 & 68.35 & 68.27  \\
 \hline
 16 & 43.9 & 42.7 & 44.9 & 45 & 68.12 & 68.77 & 68.31 & 68.59  \\
 \hline
 \hline
 \multicolumn{5}{|c|}{GPT3-22B (FP32 Accuracy = 38.75\%)} & \multicolumn{4}{|c|}{Nemotron4-15B (FP32 Accuracy = 64.3\%)} \\ 
 \hline
 \hline
 64 & 36.71 & 38.85 & 38.13 & 38.92 & 63.17 & 62.36 & 63.72 & 64.09 \\
 \hline
 32 & 37.95 & 38.69 & 39.45 & 38.34 & 64.05 & 62.30 & 63.8 & 64.33  \\
 \hline
 16 & 38.88 & 38.80 & 38.31 & 38.92 & 63.22 & 63.51 & 63.93 & 64.43  \\
 \hline
\end{tabular}
\caption{\label{tab:mmlu_abalation} Accuracy on MMLU dataset across GPT3-22B, Llama2-7B, 70B and Nemotron4-15B models.}
\end{table}


%\subsection{Perplexity achieved by various LO-BCQ configurations on LM evaluation harness}

\begin{table} \centering
\begin{tabular}{|c||c|c|c|c||c|c|c|c|} 
\hline
 $L_b \rightarrow$& \multicolumn{4}{c||}{8} & \multicolumn{4}{c||}{8}\\
 \hline
 \backslashbox{$L_A$\kern-1em}{\kern-1em$N_c$} & 2 & 4 & 8 & 16 & 2 & 4 & 8 & 16  \\
 %$N_c \rightarrow$ & 2 & 4 & 8 & 16 & 2 & 4 & 2 \\
 \hline
 \hline
 \multicolumn{5}{|c|}{Race (FP32 Accuracy = 37.51\%)} & \multicolumn{4}{|c|}{Boolq (FP32 Accuracy = 64.62\%)} \\ 
 \hline
 \hline
 64 & 36.94 & 37.13 & 36.27 & 37.13 & 63.73 & 62.26 & 63.49 & 63.36 \\
 \hline
 32 & 37.03 & 36.36 & 36.08 & 37.03 & 62.54 & 63.51 & 63.49 & 63.55  \\
 \hline
 16 & 37.03 & 37.03 & 36.46 & 37.03 & 61.1 & 63.79 & 63.58 & 63.33  \\
 \hline
 \hline
 \multicolumn{5}{|c|}{Winogrande (FP32 Accuracy = 58.01\%)} & \multicolumn{4}{|c|}{Piqa (FP32 Accuracy = 74.21\%)} \\ 
 \hline
 \hline
 64 & 58.17 & 57.22 & 57.85 & 58.33 & 73.01 & 73.07 & 73.07 & 72.80 \\
 \hline
 32 & 59.12 & 58.09 & 57.85 & 58.41 & 73.01 & 73.94 & 72.74 & 73.18  \\
 \hline
 16 & 57.93 & 58.88 & 57.93 & 58.56 & 73.94 & 72.80 & 73.01 & 73.94  \\
 \hline
\end{tabular}
\caption{\label{tab:mmlu_abalation} Accuracy on LM evaluation harness tasks on GPT3-1.3B model.}
\end{table}

\begin{table} \centering
\begin{tabular}{|c||c|c|c|c||c|c|c|c|} 
\hline
 $L_b \rightarrow$& \multicolumn{4}{c||}{8} & \multicolumn{4}{c||}{8}\\
 \hline
 \backslashbox{$L_A$\kern-1em}{\kern-1em$N_c$} & 2 & 4 & 8 & 16 & 2 & 4 & 8 & 16  \\
 %$N_c \rightarrow$ & 2 & 4 & 8 & 16 & 2 & 4 & 2 \\
 \hline
 \hline
 \multicolumn{5}{|c|}{Race (FP32 Accuracy = 41.34\%)} & \multicolumn{4}{|c|}{Boolq (FP32 Accuracy = 68.32\%)} \\ 
 \hline
 \hline
 64 & 40.48 & 40.10 & 39.43 & 39.90 & 69.20 & 68.41 & 69.45 & 68.56 \\
 \hline
 32 & 39.52 & 39.52 & 40.77 & 39.62 & 68.32 & 67.43 & 68.17 & 69.30  \\
 \hline
 16 & 39.81 & 39.71 & 39.90 & 40.38 & 68.10 & 66.33 & 69.51 & 69.42  \\
 \hline
 \hline
 \multicolumn{5}{|c|}{Winogrande (FP32 Accuracy = 67.88\%)} & \multicolumn{4}{|c|}{Piqa (FP32 Accuracy = 78.78\%)} \\ 
 \hline
 \hline
 64 & 66.85 & 66.61 & 67.72 & 67.88 & 77.31 & 77.42 & 77.75 & 77.64 \\
 \hline
 32 & 67.25 & 67.72 & 67.72 & 67.00 & 77.31 & 77.04 & 77.80 & 77.37  \\
 \hline
 16 & 68.11 & 68.90 & 67.88 & 67.48 & 77.37 & 78.13 & 78.13 & 77.69  \\
 \hline
\end{tabular}
\caption{\label{tab:mmlu_abalation} Accuracy on LM evaluation harness tasks on GPT3-8B model.}
\end{table}

\begin{table} \centering
\begin{tabular}{|c||c|c|c|c||c|c|c|c|} 
\hline
 $L_b \rightarrow$& \multicolumn{4}{c||}{8} & \multicolumn{4}{c||}{8}\\
 \hline
 \backslashbox{$L_A$\kern-1em}{\kern-1em$N_c$} & 2 & 4 & 8 & 16 & 2 & 4 & 8 & 16  \\
 %$N_c \rightarrow$ & 2 & 4 & 8 & 16 & 2 & 4 & 2 \\
 \hline
 \hline
 \multicolumn{5}{|c|}{Race (FP32 Accuracy = 40.67\%)} & \multicolumn{4}{|c|}{Boolq (FP32 Accuracy = 76.54\%)} \\ 
 \hline
 \hline
 64 & 40.48 & 40.10 & 39.43 & 39.90 & 75.41 & 75.11 & 77.09 & 75.66 \\
 \hline
 32 & 39.52 & 39.52 & 40.77 & 39.62 & 76.02 & 76.02 & 75.96 & 75.35  \\
 \hline
 16 & 39.81 & 39.71 & 39.90 & 40.38 & 75.05 & 73.82 & 75.72 & 76.09  \\
 \hline
 \hline
 \multicolumn{5}{|c|}{Winogrande (FP32 Accuracy = 70.64\%)} & \multicolumn{4}{|c|}{Piqa (FP32 Accuracy = 79.16\%)} \\ 
 \hline
 \hline
 64 & 69.14 & 70.17 & 70.17 & 70.56 & 78.24 & 79.00 & 78.62 & 78.73 \\
 \hline
 32 & 70.96 & 69.69 & 71.27 & 69.30 & 78.56 & 79.49 & 79.16 & 78.89  \\
 \hline
 16 & 71.03 & 69.53 & 69.69 & 70.40 & 78.13 & 79.16 & 79.00 & 79.00  \\
 \hline
\end{tabular}
\caption{\label{tab:mmlu_abalation} Accuracy on LM evaluation harness tasks on GPT3-22B model.}
\end{table}

\begin{table} \centering
\begin{tabular}{|c||c|c|c|c||c|c|c|c|} 
\hline
 $L_b \rightarrow$& \multicolumn{4}{c||}{8} & \multicolumn{4}{c||}{8}\\
 \hline
 \backslashbox{$L_A$\kern-1em}{\kern-1em$N_c$} & 2 & 4 & 8 & 16 & 2 & 4 & 8 & 16  \\
 %$N_c \rightarrow$ & 2 & 4 & 8 & 16 & 2 & 4 & 2 \\
 \hline
 \hline
 \multicolumn{5}{|c|}{Race (FP32 Accuracy = 44.4\%)} & \multicolumn{4}{|c|}{Boolq (FP32 Accuracy = 79.29\%)} \\ 
 \hline
 \hline
 64 & 42.49 & 42.51 & 42.58 & 43.45 & 77.58 & 77.37 & 77.43 & 78.1 \\
 \hline
 32 & 43.35 & 42.49 & 43.64 & 43.73 & 77.86 & 75.32 & 77.28 & 77.86  \\
 \hline
 16 & 44.21 & 44.21 & 43.64 & 42.97 & 78.65 & 77 & 76.94 & 77.98  \\
 \hline
 \hline
 \multicolumn{5}{|c|}{Winogrande (FP32 Accuracy = 69.38\%)} & \multicolumn{4}{|c|}{Piqa (FP32 Accuracy = 78.07\%)} \\ 
 \hline
 \hline
 64 & 68.9 & 68.43 & 69.77 & 68.19 & 77.09 & 76.82 & 77.09 & 77.86 \\
 \hline
 32 & 69.38 & 68.51 & 68.82 & 68.90 & 78.07 & 76.71 & 78.07 & 77.86  \\
 \hline
 16 & 69.53 & 67.09 & 69.38 & 68.90 & 77.37 & 77.8 & 77.91 & 77.69  \\
 \hline
\end{tabular}
\caption{\label{tab:mmlu_abalation} Accuracy on LM evaluation harness tasks on Llama2-7B model.}
\end{table}

\begin{table} \centering
\begin{tabular}{|c||c|c|c|c||c|c|c|c|} 
\hline
 $L_b \rightarrow$& \multicolumn{4}{c||}{8} & \multicolumn{4}{c||}{8}\\
 \hline
 \backslashbox{$L_A$\kern-1em}{\kern-1em$N_c$} & 2 & 4 & 8 & 16 & 2 & 4 & 8 & 16  \\
 %$N_c \rightarrow$ & 2 & 4 & 8 & 16 & 2 & 4 & 2 \\
 \hline
 \hline
 \multicolumn{5}{|c|}{Race (FP32 Accuracy = 48.8\%)} & \multicolumn{4}{|c|}{Boolq (FP32 Accuracy = 85.23\%)} \\ 
 \hline
 \hline
 64 & 49.00 & 49.00 & 49.28 & 48.71 & 82.82 & 84.28 & 84.03 & 84.25 \\
 \hline
 32 & 49.57 & 48.52 & 48.33 & 49.28 & 83.85 & 84.46 & 84.31 & 84.93  \\
 \hline
 16 & 49.85 & 49.09 & 49.28 & 48.99 & 85.11 & 84.46 & 84.61 & 83.94  \\
 \hline
 \hline
 \multicolumn{5}{|c|}{Winogrande (FP32 Accuracy = 79.95\%)} & \multicolumn{4}{|c|}{Piqa (FP32 Accuracy = 81.56\%)} \\ 
 \hline
 \hline
 64 & 78.77 & 78.45 & 78.37 & 79.16 & 81.45 & 80.69 & 81.45 & 81.5 \\
 \hline
 32 & 78.45 & 79.01 & 78.69 & 80.66 & 81.56 & 80.58 & 81.18 & 81.34  \\
 \hline
 16 & 79.95 & 79.56 & 79.79 & 79.72 & 81.28 & 81.66 & 81.28 & 80.96  \\
 \hline
\end{tabular}
\caption{\label{tab:mmlu_abalation} Accuracy on LM evaluation harness tasks on Llama2-70B model.}
\end{table}

%\section{MSE Studies}
%\textcolor{red}{TODO}


\subsection{Number Formats and Quantization Method}
\label{subsec:numFormats_quantMethod}
\subsubsection{Integer Format}
An $n$-bit signed integer (INT) is typically represented with a 2s-complement format \citep{yao2022zeroquant,xiao2023smoothquant,dai2021vsq}, where the most significant bit denotes the sign.

\subsubsection{Floating Point Format}
An $n$-bit signed floating point (FP) number $x$ comprises of a 1-bit sign ($x_{\mathrm{sign}}$), $B_m$-bit mantissa ($x_{\mathrm{mant}}$) and $B_e$-bit exponent ($x_{\mathrm{exp}}$) such that $B_m+B_e=n-1$. The associated constant exponent bias ($E_{\mathrm{bias}}$) is computed as $(2^{{B_e}-1}-1)$. We denote this format as $E_{B_e}M_{B_m}$.  

\subsubsection{Quantization Scheme}
\label{subsec:quant_method}
A quantization scheme dictates how a given unquantized tensor is converted to its quantized representation. We consider FP formats for the purpose of illustration. Given an unquantized tensor $\bm{X}$ and an FP format $E_{B_e}M_{B_m}$, we first, we compute the quantization scale factor $s_X$ that maps the maximum absolute value of $\bm{X}$ to the maximum quantization level of the $E_{B_e}M_{B_m}$ format as follows:
\begin{align}
\label{eq:sf}
    s_X = \frac{\mathrm{max}(|\bm{X}|)}{\mathrm{max}(E_{B_e}M_{B_m})}
\end{align}
In the above equation, $|\cdot|$ denotes the absolute value function.

Next, we scale $\bm{X}$ by $s_X$ and quantize it to $\hat{\bm{X}}$ by rounding it to the nearest quantization level of $E_{B_e}M_{B_m}$ as:

\begin{align}
\label{eq:tensor_quant}
    \hat{\bm{X}} = \text{round-to-nearest}\left(\frac{\bm{X}}{s_X}, E_{B_e}M_{B_m}\right)
\end{align}

We perform dynamic max-scaled quantization \citep{wu2020integer}, where the scale factor $s$ for activations is dynamically computed during runtime.

\subsection{Vector Scaled Quantization}
\begin{wrapfigure}{r}{0.35\linewidth}
  \centering
  \includegraphics[width=\linewidth]{sections/figures/vsquant.jpg}
  \caption{\small Vectorwise decomposition for per-vector scaled quantization (VSQ \citep{dai2021vsq}).}
  \label{fig:vsquant}
\end{wrapfigure}
During VSQ \citep{dai2021vsq}, the operand tensors are decomposed into 1D vectors in a hardware friendly manner as shown in Figure \ref{fig:vsquant}. Since the decomposed tensors are used as operands in matrix multiplications during inference, it is beneficial to perform this decomposition along the reduction dimension of the multiplication. The vectorwise quantization is performed similar to tensorwise quantization described in Equations \ref{eq:sf} and \ref{eq:tensor_quant}, where a scale factor $s_v$ is required for each vector $\bm{v}$ that maps the maximum absolute value of that vector to the maximum quantization level. While smaller vector lengths can lead to larger accuracy gains, the associated memory and computational overheads due to the per-vector scale factors increases. To alleviate these overheads, VSQ \citep{dai2021vsq} proposed a second level quantization of the per-vector scale factors to unsigned integers, while MX \citep{rouhani2023shared} quantizes them to integer powers of 2 (denoted as $2^{INT}$).

\subsubsection{MX Format}
The MX format proposed in \citep{rouhani2023microscaling} introduces the concept of sub-block shifting. For every two scalar elements of $b$-bits each, there is a shared exponent bit. The value of this exponent bit is determined through an empirical analysis that targets minimizing quantization MSE. We note that the FP format $E_{1}M_{b}$ is strictly better than MX from an accuracy perspective since it allocates a dedicated exponent bit to each scalar as opposed to sharing it across two scalars. Therefore, we conservatively bound the accuracy of a $b+2$-bit signed MX format with that of a $E_{1}M_{b}$ format in our comparisons. For instance, we use E1M2 format as a proxy for MX4.

\begin{figure}
    \centering
    \includegraphics[width=1\linewidth]{sections//figures/BlockFormats.pdf}
    \caption{\small Comparing LO-BCQ to MX format.}
    \label{fig:block_formats}
\end{figure}

Figure \ref{fig:block_formats} compares our $4$-bit LO-BCQ block format to MX \citep{rouhani2023microscaling}. As shown, both LO-BCQ and MX decompose a given operand tensor into block arrays and each block array into blocks. Similar to MX, we find that per-block quantization ($L_b < L_A$) leads to better accuracy due to increased flexibility. While MX achieves this through per-block $1$-bit micro-scales, we associate a dedicated codebook to each block through a per-block codebook selector. Further, MX quantizes the per-block array scale-factor to E8M0 format without per-tensor scaling. In contrast during LO-BCQ, we find that per-tensor scaling combined with quantization of per-block array scale-factor to E4M3 format results in superior inference accuracy across models. 



\end{document}
\endinput
%%
%% End of file `sample-sigconf.tex'.
