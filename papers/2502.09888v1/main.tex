%%
%% This is file `sample-sigconf.tex',
%% generated with the docstrip utility.
%%
%% The original source files were:
%%
%% samples.dtx  (with options: `all,proceedings,bibtex,sigconf')
%% 
%% IMPORTANT NOTICE:
%% 
%% For the copyright see the source file.
%% 
%% Any modified versions of this file must be renamed
%% with new filenames distinct from sample-sigconf.tex.
%% 
%% For distribution of the original source see the terms
%% for copying and modification in the file samples.dtx.
%% 
%% This generated file may be distributed as long as the
%% original source files, as listed above, are part of the
%% same distribution. (The sources need not necessarily be
%% in the same archive or directory.)
%%
%%
%% Commands for TeXCount
%TC:macro \cite [option:text,text]
%TC:macro \citep [option:text,text]
%TC:macro \citet [option:text,text]
%TC:envir table 0 1
%TC:envir table* 0 1
%TC:envir tabular [ignore] word
%TC:envir displaymath 0 word
%TC:envir math 0 word
%TC:envir comment 0 0
%%
%%
%% The first command in your LaTeX source must be the \documentclass
%% command.
%%
%% For submission and review of your manuscript please change the
%% command to \documentclass[manuscript, screen, review]{acmart}.
%%
%% When submitting camera ready or to TAPS, please change the command
%% to \documentclass[sigconf]{acmart} or whichever template is required
%% for your publication.
%%
%%
\documentclass[sigconf]{acmart}
\usepackage{enumitem}
\setlist[itemize]{leftmargin=*, itemsep=0pt, topsep=0pt}
\usepackage{graphicx}
\graphicspath{ {./graph/} }
\usepackage{svg}
\usepackage{multirow}
\usepackage{amsmath}
\usepackage{subfigure}
\usepackage{booktabs} % 用于更美观的表格线条
\usepackage{tabularx} % 用于设置表格宽度
\usepackage{xcolor} % 用于文本颜色
\usepackage{makecell} % 用于多行表头
% \usepackage{soul} % 加载 soul 宏包
% \usepackage{ulem} % 加载 ulem 宏包
\usepackage{hyperref}
\usepackage{microtype}

% \usepackage{geometry}
% \geometry{a4paper, margin=1in, columnsep=10pt}
% \usepackage{setspace}
% \setlength{\parskip}{0.5em}
% \usepackage{microtype}
\usepackage{balance}

\newcommand{\MSP}{MSP }
\newcommand{\ATL}{ATL }
\newcommand{\BFM}{BFM }


%%
%% \BibTeX command to typeset BibTeX logo in the docs
\AtBeginDocument{%
  \providecommand\BibTeX{{%
    Bib\TeX}}}

%% Rights management information.  This information is sent to you
%% when you complete the rights form.  These commands have SAMPLE
%% values in them; it is your responsibility as an author to replace
%% the commands and values with those provided to you when you
%% complete the rights form.
\setcopyright{acmlicensed}
\copyrightyear{2018}
\acmYear{2018}
\acmDOI{XXXXXXX.XXXXXXX}

%% These commands are for a PROCEEDINGS abstract or paper.
\acmConference[Conference acronym 'XX]{Make sure to enter the correct
  conference title from your rights confirmation emai}{June 03--05,
  218}{Woodstock, NY}
%%
%%  Uncomment \acmBooktitle if the title of the proceedings is different
%%  from ``Proceedings of ...''!
%%
%%\acmBooktitle{Woodstock '18: ACM Symposium on Neural Gaze Detection,
%%  June 03--05, 2018, Woodstock, NY}
\acmISBN{978-1-4503-XXXX-X/18/06}


%%
%% Submission ID.
%% Use this when submitting an article to a sponsored event. You'll
%% receive a unique submission ID from the organizers
%% of the event, and this ID should be used as the parameter to this command.
%%\acmSubmissionID{123-A56-BU3}

%%
%% For managing citations, it is recommended to use bibliography
%% files in BibTeX format.
%%
%% You can then either use BibTeX with the ACM-Reference-Format style,
%% or BibLaTeX with the acmnumeric or acmauthoryear sytles, that include
%% support for advanced citation of software artefact from the
%% biblatex-software package, also separately available on CTAN.
%%
%% Look at the sample-*-biblatex.tex files for templates showcasing
%% the biblatex styles.
%%

%%
%% The majority of ACM publications use numbered citations and
%% references.  The command \citestyle{authoryear} switches to the
%% "author year" style.
%%
%% If you are preparing content for an event
%% sponsored by ACM SIGGRAPH, you must use the "author year" style of
%% citations and references.
%% Uncommenting
%% the next command will enable that style.
%%\citestyle{acmauthoryear}


%%
%% end of the preamble, start of the body of the document source.
\begin{document}

%%
%% The "title" command has an optional parameter,
%% allowing the author to define a "short title" to be used in page headers.
\title{An Efficient Large Recommendation Model: Towards a Resource-Optimal Scaling Law}
\renewcommand{\thefootnote}{\fnsymbol{footnote}} % 设置脚注符号为非数字形式
\author{Songpei Xu, Shijia Wang\footnotemark, Da Guo, Xianwen Guo, Qiang Xiao\footnotemark, Fangjian Li, Chuanjiang Luo}
\email{{xusongpei,wangshijia1,guoda,guoxianwen,hzxiaoqiang,hzlifangjian,luochuanjiang03}@corp.netease.com}
\affiliation{
  \institution{NetEase Cloud Music}
  \city{Hangzhou}
  \country{China}
}
% \author{Songpei Xu}
% \email{xusongpei@corp.netease.com}
% \affiliation{%
%   \institution{NetEase Cloud Music}
%   \city{Hangzhou}
%   \country{China}
% }

% \author{Shijia Wang}
% % \authornote{Both authors contributed equally to this research.}
% \authornote{Corresponding author.}
% \email{wangshijia1@corp.netease.com}
% % \authornotemark[2]
% \affiliation{%
%   \institution{NetEase Cloud Music}
%   \city{Hangzhou}
%   \country{China}
% }

% \author{Da Guo}
% % \authornote{Both authors contributed equally to this research.}
% \email{guoda@corp.netease.com}
% % \authornotemark[2]
% \affiliation{%
%   \institution{NetEase Cloud Music}
%   \city{Hangzhou}
%   \country{China}
% }

% \author{Xianwen Guo}
% % \authornote{Both authors contributed equally to this research.}
% \email{guoxianwen@corp.netease.com}
% % \authornotemark[2]
% \affiliation{%
%   \institution{NetEase Cloud Music}
%   \city{Hangzhou}
%   \country{China}
% }

% \author{Qiang Xiao}
% \authornote{Corresponding author.}
% \email{hzxiaoqiang@corp.netease.com}
% % \authornotemark[2]
% \affiliation{%
%   \institution{NetEase Cloud Music}
%   \city{Hangzhou}
%   \country{China}
% }

% \author{Fangjian Li}
% \email{hzlifangjian@corp.netease.com}
% % \authornotemark[2]
% \affiliation{%
%   \institution{NetEase Cloud Music}
%   \city{Hangzhou}
%   \country{China}
% }

% \author{Chuanjiang Luo}
% \email{luochuanjiang03@corp.netease.com}
% \affiliation{%
%   \institution{NetEase Cloud Music}
%   \city{Hangzhou}
%   \country{China}
% }
%%
%% The "author" command and its associated commands are used to define
%% the authors and their affiliations.
%% Of note is the shared affiliation of the first two authors, and the
%% "authornote" and "authornotemark" commands
%% used to denote shared contribution to the research.

% \author{Ben Trovato}
% \authornote{Both authors contributed equally to this research.}
% \email{trovato@corporation.com}
% \orcid{1234-5678-9012}
% \author{G.K.M. Tobin}
% \authornotemark[1]
% \email{webmaster@marysville-ohio.com}
% \affiliation{%
%   \institution{Institute for Clarity in Documentation}
%   \city{Dublin}
%   \state{Ohio}
%   \country{USA}
% }

% \author{Lars Th{\o}rv{\"a}ld}
% \affiliation{%
%   \institution{The Th{\o}rv{\"a}ld Group}
%   \city{Hekla}
%   \country{Iceland}}
% \email{larst@affiliation.org}

% \author{Valerie B\'eranger}
% \affiliation{%
%   \institution{Inria Paris-Rocquencourt}
%   \city{Rocquencourt}
%   \country{France}
% }

% \author{Aparna Patel}
% \affiliation{%
%  \institution{Rajiv Gandhi University}
%  \city{Doimukh}
%  \state{Arunachal Pradesh}
%  \country{India}}

% \author{Huifen Chan}
% \affiliation{%
%   \institution{Tsinghua University}
%   \city{Haidian Qu}
%   \state{Beijing Shi}
%   \country{China}}

% \author{Charles Palmer}
% \affiliation{%
%   \institution{Palmer Research Laboratories}
%   \city{San Antonio}
%   \state{Texas}
%   \country{USA}}
% \email{cpalmer@prl.com}

% \author{John Smith}
% \affiliation{%
%   \institution{The Th{\o}rv{\"a}ld Group}
%   \city{Hekla}
%   \country{Iceland}}
% \email{jsmith@affiliation.org}

% \author{Julius P. Kumquat}
% \affiliation{%
%   \institution{The Kumquat Consortium}
%   \city{New York}
%   \country{USA}}
% \email{jpkumquat@consortium.net}

%%
%% By default, the full list of authors will be used in the page
%% headers. Often, this list is too long, and will overlap
%% other information printed in the page headers. This command allows
%% the author to define a more concise list
%% of authors' names for this purpose.
\renewcommand{\shortauthors}{Trovato et al.}

%%
%% The abstract is a short summary of the work to be presented in the
%% article.
\begin{abstract}
The pursuit of scaling up recommendation models confronts intrinsic tensions between expanding model capacity and preserving computational tractability. While prior studies have explored scaling laws for recommendation systems, their resource-intensive paradigms---often requiring tens of thousands of A100 GPU hours---remain impractical for most industrial applications. This work addresses a critical gap: achieving sustainable model scaling under strict computational budgets. We propose Climber, a resource-efficient recommendation framework comprising two synergistic components: the ASTRO model architecture for algorithmic innovation and the TURBO acceleration framework for engineering optimization. ASTRO (Adaptive Scalable Transformer for RecOmmendation) adopts two core innovations: (1) multi-scale sequence partitioning that reduces attention complexity from $O(n^2d)$ to $O(n^2d/N_b)$ via hierarchical blocks, enabling more efficient scaling with sequence length; (2) dynamic temperature modulation that adaptively adjusts attention scores for multimodal distributions arising from inherent multi-scenario and multi-behavior interactions. Complemented by TURBO (Two-stage Unified Ranking with Batched Output), a co-designed acceleration framework integrating gradient-aware feature compression and memory-efficient Key-Value caching, Climber achieves 5.15$\times$ throughput gains without performance degradation.

Comprehensive offline experiments on multiple datasets validate that Climber exhibits a more ideal scaling curve. To our knowledge, this is the first publicly documented framework where controlled model scaling drives continuous online metric growth (12.19\% overall lift) without prohibitive resource costs. Climber has been successfully deployed on Netease Cloud Music, one of China's largest music streaming platforms, serving tens of millions of users daily. These advancements establish a new paradigm for industrial recommendation systems, demonstrating that coordinated algorithmic-engineering innovation---not just brute-force scaling---unlocks sustainable performance growth under real-world computational constraints.
\end{abstract}

%%
%% The code below is generated by the tool at http://dl.acm.org/ccs.cfm.
%% Please copy and paste the code instead of the example below.
%%
\ccsdesc[500]{Information systems~Retrieval models and ranking}

%%
%% Keywords. The author(s) should pick words that accurately describe
%% the work being presented. Separate the keywords with commas.
\keywords{Recommendation system, Transformer, Scaling law}
%% A "teaser" image appears between the author and affiliation
%% information and the body of the document, and typically spans the
%% page.
% \begin{teaserfigure}
%   \includegraphics[width=\textwidth]{sampleteaser}
%   \caption{Seattle Mariners at Spring Training, 2010.}
%   \Description{Enjoying the baseball game from the third-base
%   seats. Ichiro Suzuki preparing to bat.}
%   \label{fig:teaser}
% \end{teaserfigure}

% \received{20 February 2007}
% \received[revised]{12 March 2009}
% \received[accepted]{5 June 2009}

%%
%% This command processes the author and affiliation and title
%% information and builds the first part of the formatted document.
\maketitle

\footnotetext[1]{Corresponding Author}
\footnotetext[2]{Corresponding Author}
% \footnotetext[3]{The baseline model that has been implemented for online deployment.}


\section{Introduction}
Scaling laws, initially explored in language models \cite{kaplan2020scaling,hoffmann2022training}, establish predictable relationships between model performance and key factors such as model size, training data volume, and computational budget. For instance, Kaplan et al. \cite{kaplan2020scaling} demonstrated that transformer-based language models follow power-law improvements in perplexity as model parameters and token counts increase. Similar trends have been observed in vision \cite{cherti2023reproducible} and multimodal models \cite{bai2023qwen}, where scaling model dimensions and data diversity directly correlate with the performance of downstream tasks. 

Recent research has shown that scaling laws also apply to recommendation systems, providing valuable insights into model design and resource allocation\cite{ardalani2022understanding,guo2024scaling}. The HSTU model \cite{zhai2024actions} employs hierarchical self-attention mechanisms to model long-term user behavior sequences, achieving better performance than traditional Transformers. Similarly, the MARM model \cite{lv2024marm} introduces memory augmentation techniques to reduce computational complexity, enabling multi-layer sequence modeling with minimal inference costs.However, these approaches demand prohibitive computational resources (e.g., thousands of GPU hours or terabytes of memory), rendering them impractical for real-world deployment. Furthermore, the interplay between key scaling factors---sequence length, model depth, and heterogeneous user behaviors---remains underexplored, leading to suboptimal resource allocation and diminished returns on scaling efforts.


% Recently, scalable recommendation models have been developed and have demonstrated promising online performance. However, these models still exhibit certain limitations. In the WuKong model, scalability is primarily determined by feature engineering and interaction, which is not generalizable to other scenarios. In the HSTU research, the model is based on Transformer sequence modeling. When comparing DLRM with HSTU in ranking set, DLRM shows superior performance with fewer computational complexity (FLOPs). This phenomenon is also observed in our system, as illustrated in Figure \ref{fig:intro}(a). Furthermore, both HSTU and MARM have achieved significant online results. However, they also consume substantial resources, such as GPU and storage resources.

% Inspired by the above work, we effectively implement the Transformer model within Netease’s large-scale music recommendation system to replace our traditional DLRM. However, there are still several specific challenges in recommendation systems:

\renewcommand{\thefootnote}{\fnsymbol{footnote}} % 设置脚注符号为非数字形式
% Inspired by DeepSeek series work \cite{liu2024deepseek1,liu2024deepseek2,bi2024deepseek3}, which greatly improve the efficiency of buliding LLMs and reduce the cost of computional resource, we aim to answer the question: how to scaling up recommendation model with significant lower cost?  
Inspired by the DeepSeek series \cite{liu2024deepseek1,liu2024deepseek2,bi2024deepseek3}, which has significantly enhanced the efficiency of large language model (LLM) development and reduced computational resource costs, we aim to address the following question: \textbf{\textit{how can we scale up the recommendation model at a substantially lower cost?}} To gain insights, we conducted industrial-scale analysis on two mainstream models, the Deep Learning Recommendation Model (DLRM)\cite{mudigere2022software} and the Transformer model.
In Figure \ref{fig:intro}(a), we present the scaling curves for both DLRM\footnotemark and Transformer models, and an oracle curve is included to represent an ideal scaling curve, characterized by a higher starting point and a larger slope. In Figure \ref{fig:intro}(b), we present an AUC curves derived from simulations of various combinations of sequence lengths and layer number, and introduce the concept of "performance interval", which represents the range of AUC variation for the model under the equivalent FLOPs.
\footnotetext{The baseline model that has been implemented for online deployment.}
% In Figure \ref{fig:intro}(b), it is shown that the Transformer model exhibits different AUC values under the same FLOPs, highlighting the performance and computational resource losses caused by unreasonable resource allocation in the model.
However, our findings reveal that some issues still remain in recommendation systems: 
\begin{itemize}
\item \textbf{Transformer's degraded performance under FLOPS constraints}: As shown in Figure \ref{fig:intro}(a), the FLOPs corresponding to the crossover point are $10^{8.2}$. Using this FLOPs value as the boundary, the performance comparison between DLRM and Transformer shows different trends. Transformer model outperforms traditional architectures such as DLRM when FLOPS exceed $10^{8.2}$ magnitude. However, when FLOPS less than $10^{8.2}$, Transformer model shows worse performance compared to DLRM. This highlights the pursuit for models with the oracle curve shown in Figure \ref{fig:intro}(a), which enables the model to achieve better performance even when the FLOPs are limited.
\item \textbf{The impact of factor combinations on model performance under equivalent FLOPs}: In recommender systems, factors such as sequence length and layer number significantly influence FLOPs, and different combinations of  these factors can lead to varying model performance. For instance, there is a obvious performance interval(nearly 1\%) under different combinations at $10^9$ FLOPs, as shown in Figure \ref{fig:intro}(b). Current researches lack comprehensive analysis of how factor combinations impact recommendation model performance, which hinders efficient scaling for models. 
% Similarly, FLOPs values of $10^{7.4}$ and $10^{8.1}$ achieve the same AUC, yet their computational complexity or inference efficiency varies by a factor of $10^{0.7}$.
\item \textbf{Huge resource consumption}: As depicted in Figure \ref{fig:intro}(a), to outperform the DLRM, Transformers require FLOPs on the order of $10^{9}$ to $10^{10}$. However, the FLOPs of the currently deployed DLRM models are only at the order of $10^7$, indicating a 100$\times$ increase of FLOPS to achieve comparable performance, which requires tens of thousands of GPU hours in practice. These resource demands are already beyond the affordability of most industrial applications. Consequently, achieving resource-optimal scaling remains a significant challenge for industrial recommender systems.
\end{itemize}

\begin{figure}[t]
  \centering
  \includegraphics[width=\linewidth]{intro_v4.pdf}
  \caption{(a) Scalability: DLRM vs. Transformer. The left part of crossover point indicates that Transformer underperforms DLRM when FLOPs are limited. The right part of crossover point highlights the increasing computational demands as Transformer performance improves. 
  (b) Toy Example: Simulated AUC curve with performance interval. This figure depicts an AUC curve derived from simulations of various combinations of sequence length and layer number.
  % For instance, at $10^9$
  % FLOPs, the lower bound of this interval corresponds to 4X sequence length and Y layers, yielding an AUC of 0.827. Conversely, the upper bound corresponds to X sequence length and 4Y layers, yielding an AUC of 0.834. The lower and upper bounds of the performance interval represent the worst and best model performance under given FLOPs, respectively.
  }
  \label{fig:intro}
  \Description{}
\end{figure}

% Inspired by the above work, we effectively implement the Transformer model within Netease’s large-scale music recommendation system to replace our traditional DLRM. However, there are still several major challenges: 1) low efficiency: We find that the performance of the model improves as the number of computations (flops) increases. These factors that affect the flops of the recommender system typically include the sequence length and the number of layers, etc. Besides, different combinations of length and number of layers affect the model's performance under the same flops. Currently, there is a lack of research on the effect of different combinations of parameters on the model, which results in a lot of work that cannot be scaled up quickly. 2) high cost: Increasing the sequence length and the number of layers will improve the model performance in recommender systems. However, in industrial systems, the model cannot be infinitely enlarged due to the requirement for fast and efficient online requests. It is struggling to balance model performance and training and inference efficiency in the case of limited resources. 3) low extensibility: The existing scalable models in recommendation systems are commonly designed for specific scenarios. If a specific model is designed for each scenario, the cost will also increase many times over. Therefore, a scalable model that can be used in multiple scenarios is needed. 

% In this paper, we propose a new sequence recommendation model, Adaptive Self-attention based TRansformer blOck (ASTRO). ASTRO adjusts the distribution of attention across different sequences and scenarios by adaptively learning the temperature coefficient in softmax. It then uses multiple transformers to perform parallel operations on multiple sequences. This approach mitigates problems such as high computational complexity due to sequence length and varying task distributions within recommendation systems, thereby facilitating rapid scale-up under all traffic conditions. Besides, we implement a two-stage Unified Ranking with Batched Output (TURBO) framework. As an unified sorting framework, TURBO compresses traditional (1 user, 1 item) samples into a sample format that is closer to the actual online requests (1 user, N item), and accelerates forward propagation during training and inference through Encoder level KV Cache. TURBO achieves significant efficiency improvements in training and inference with the same computational resources. Finally, we explore the scaling laws of multiple factor combinations in the block perspective, which can be used to allocate resources more rationally and guide us to identify the key factors to quickly scale the model.

Driven by the above insights, we introduce Climber, a novel framework that rethinks the scaling paradigm for recommendation systems. At its core, Climber integrates two complementary innovations: Adaptive Scalable Transformer for RecOmmendation (ASTRO) model architecture and Two-stage Unified Ranking with Batched Output (TURBO) acceleration framework. ASTRO redefines how recommendation systems handle heterogeneous user behaviors by introducing multi-scale sequence partitioning, which decomposes user behavior sequences into smaller, fine-grained subsequences. This approach not only reduces computational complexity, but also enables more precise modeling of user interests across diverse scenarios. In addition, ASTRO incorporates dynamic temperature modulation, a mechanism that adaptively adjusts attention scores to account for the varying importance of different behaviors and scenarios. On the engineering side, TURBO introduces a unified ranking framework that optimizes both training and inference efficiency. As a unified ranking framework, TURBO transforms traditional "Single User, Single Item" samples into a format that aligns with actual online requests, namely "Single User, Multiple Items". TURBO accelerates forward propagation during training and inference with encoder-level KV Cache, achieving significant efficiency improvements.
Finally, we investigate the scalability of our proposed model and the impact of factor combinations on AUC under equivalent FLOPs. These insights enable more rational resource allocation and guide the identification of key factors for rapid model scaling.

Our contributions are mainly categorized as follows:
\begin{itemize}
% \item We give details of both offline and online scaling laws from a new perspective, and are the first to explore the parameter combinations with the highest ROI under the same FLOPs. This allows us to quickly scale up the model performance at the lowest cost. The online metric is increased by 11.45\%, achieving the largest improvement in recent one year in our music recommender system.
\item We present the first industrial-scale study of scaling laws in recommendation systems from the computational resources-constrained aspects, explicitly quantifying the impact of factor combinations under equivalent FLOPs. This analysis reveals that balanced scaling—alternating sequence and depth expansions—yields both offline and online metric gains.
\item We propose a novel Transformer variant, ASTRO, which resolves the scaling dilemmas in recommendation systems through multi-scale partitioning and adaptive temperature modulation. To our knowledge, the proposed method pioneers sustainable scaling—delivering +12.19\% online metric growth, which is the highest annual improvement in our production system.
\item A unified framework bridging training-inference gaps via dynamic feature compression and block-parallel KV caching. Deployed on Netease Cloud Music, TURBO sustains 5.15$\times$ throughput gains, enabling 100$\times$ model scaling with moderate increase in computational resources.
\end{itemize}

\section{RELATED WORK}
Wukong \cite{zhang2024wukong} explored parameter scaling in retrieval models but relied on strong assumptions about feature engineering. HSTU \cite{zhai2024actions} reformulated recommendations as sequential transduction tasks, achieving trillion-parameter models with linear computational scaling via hierarchical attention and stochastic sequence sampling. However, HSTU’s focus on generative modeling left gaps in bridging traditional feature-based DLRMs. Concurrently, MARM \cite{lv2024marm} proposed caching intermediate attention results to reduce inference complexity from ${O(n^2d)}$ to $O(nd)$, empirically validating cache size as a new scaling dimension. While effective, MARM’s caching strategy assumes static user patterns, overlooking real-time behavior shifts.

Techniques to mitigate computational costs have been widely adopted. In NLP, KV caching \cite{liu2024scissorhands,dong2024get} avoids redundant attention computations during autoregressive inference. MARM adapted this idea to recommendations by storing historical attention outputs, enabling multi-layer target-attention with minimal FLOPs overhead. Similarly, HSTU introduced Stochastic Length to sparsify long sequences algorithmically, reducing training costs by 80\% without quality degradation. For advertisement retrieval, Wang et al. \cite{wang2024scaling} designed $R/R^*$, an eCPM-aware offline metric, to estimate online revenue scaling laws at low experimental costs. These works collectively highlight the importance of tailoring efficiency strategies to recommendation-specific constraints, such as high-cardinality features and millisecond-level latency requirements.

\section{Method}

% In this section, we focus on introducing the Climber framework. We discuss the challenges associated with model scaling. These challenges include the demand for computational resources and the mismatch between heterogeneous user behaviors and Transformer architecture. To address these issues, we introduce ASTRO model architecture and a co-designed TURBO acceleration framework.  
To achieve efficient scaling, we first propose the ASTRO model, which reduces computational complexity through multi-scale sequence partitioning and adapts Transformer architecture to recommender systems by integrating multi-scenario and multi-behavior characteristics. Besides, we introduce a co-design acceleration framework, TURBO, which incorporates dynamic compression and encoder-level KV cache to enhance the efficiency of training and inference without performance degradation.

% However, there still exists issues encountered when scaling up recommendation model:

% Complexity: it is known that the time complexity $O(n^2 * d)$ of self attention mechanism is much higher than $O(n * d)$ of target attention mechanism. Especially in the case of longer sequence, self attention mechanism can cause a large increasing in FLOPs, resulting in offline/online evident computation time increment. It means that when the sequence length increases by $m$ times, the time complexity will increase by $m^2$ times, which cause the degradation in training and inference efficiency.

% Extensibility: As an important feature, user history behavior is a continuous list of behaviors generated by users in the past period of time. Many recommendation systems typically choose a fixed time interval for behavior selection, such as 30, 90, 180, 360 days. However, there are many items in the sequence that users are almost unaware of (i.e. browsing time less than 5 seconds in video/music scene), and these items provide very little effective information. As a result, the benefits of extending the sequence in this way are in fact very small.

% Conflict: The user behavior type of each platform is diverse, and on our music platform, the important behaviors include clicking, playing-end, collecting and etc. Some important behaviors are actually relatively sparse, which leads to  conflicts in attention allocation between a few important behaviors and the majority of unimportant behaviors. Taking collection behavior as an example, according to the above way of organizing one sequence in a limited time window, the collection behavior accounts for less than 5\% in the whole sequence, so its attention score is expected to be $E(1/20)$. Important behavior will be diluted in a single sequential organization, which is not conducive to important tasks prediction.

% Under these specialized issues in the recommendation domain, it is difficult to simply elongate the sequence and increase the number of model layers to enable the model to have the ability to scale up quickly even with a transformer structure. The framework described in the following section is able to overcome these problems to achieve the purpose of our fast scale-up.

% \begin{figure}[t]
%   \centering
%   \includegraphics[width=\linewidth]{kv_cache.pdf}
%   \caption{Turbo}
%   \Description{}
% \end{figure}



\subsection{Scaling Dilemma in Recommender Systems}
% \subsubsection{Computational Complexity}
In Recommendation Systems(RS), model scaling mainly involves two approaches: feature scaling and model capacity scaling. Feature scaling, particularly the extension of user behavior sequences, has been a focal point, with extensive research demonstrating its effectiveness in enhancing model performance \cite{pi2020search,chen2021end,chang2023twin,si2024twin}. Meanwhile, the integration of Transformer architectures has markedly enhanced the modeling capabilities of recommendation systems\cite{zivic2024scaling,liu2024kuaiformer,sun2019bert4rec,chen2019behavior}. However, combining these two approaches for synchronous scaling results in a quadratic increase in computational cost. This increase not only raises the resource requirements for training and inference but also puts more stress on real-time efficiency. Therefore, achieving a balance between performance improvement and computational efficiency has become an urgent challenge in recommendation systems.

In NLP, words form continuous sequences with explicit syntactic and semantic relationships. In contrast, user behavior sequences in RS are inherently fragmented, spanning multiple scenarios (e.g., browsing, purchasing) and heterogeneous behaviors (e.g., clicks, likes)\cite{zhang2024scaling,hou2022towards,li2023text,hou2023learning}. When fed into Transformer models, these fragmented sequences often lead to disordered attention distributions, as the model struggles to prioritize relevant behaviors within sparse and irregular patterns. This inefficiency limits the Transformer's scalability, particularly under computational constraints, where specialized models like DLRMs often outperform due to their tailored handling of sparse data.
% }

\begin{figure}[htbp]
  \centering
  % \includegraphics[width=6in]{model_v2.pdf}
  \includegraphics[width=\linewidth]{model_v6.pdf}
  \caption{
  ASTRO Model Architecture.
  % 1) Multi-Scale Sequence Partitioning is adopted to generate multi-scale sequences from user lifecycle sequence. 2) Adaptive Transformer Layer learns temperature coefficient in Softmax activation to adjust attention score distribution. 3) Bit-wise Gating Fusion module is designed to fuse the representations of different subsequences extracted by different blocks.
  }
  \Description{}
  \label{fig:model_arch}
\end{figure}

\subsection{ASTRO Model Architecture}
\subsubsection{Overall}  
% To explore the fast and effective scalability of recommendation systems, we propose the ASTRO model, which integrates the multi-behavior and multi-scenario characteristics of recommendation systems into the Transformer architecture. Unlike NLP tasks that typically handle single continuous sequences, recommendation systems involve non-continuous and diverse sequences. Our design philosophy follows the principle of "1 to N, N to 1," and our model comprises three modules: positive behavior extraction, adaptive Transformer layer, and block fusion module. Specifically, positive behavior extraction generates multiple subsequences from user lifecycle behaviors (1 to N). The adaptive Transformer layer then discriminatively learns the similarity between candidate items and these multiple subsequences. Finally, block fusion module aggregates the representations from subsequences to produce a unified output (N to 1).

% A more detailed workflow of our model architecture is presented in Figure \ref{fig:comp_sketch}. We transform sequence to different blocks, and these subsequences can represent different types of behavior, as well as long-term and short-term sequences of a certain behavior. We use a corresponding transformer block for each subsequence to extract users' interests. Besides, we treat some important behaviors as a separate block, and extent its time period to the whole lifetime. Then an adaptive temperature in activation is used to adjust the attention distribution of different kinds of subsequences in different scenarios. For example, if some behaviors are related to the user's short-term interest, a lower temperature coefficient may be adopted, while some behaviors are related to the user's long-term interest, a higher temperature coefficient may be adopted. Different scenarios will also have different temperature coefficients in the training process. With user interest representation extracted from user subsequences, a fusion module is proposed to realize cross attention between different subsequences by self attention and bit-wise gating mechanism. 

% Figure \ref{fig:comp_sketch} illustrates the detailed workflow of our model architecture. In positive behavior extraction, we divide the user behavior sequences into multiple subsequences, where each subsequence represents different types of sequence (multi-scenario/multi-behavior/short and long-term). For each subsequence, we apply a corresponding Transformer block to extract user interests. Additionally, we extend important subsequences' time span to the user's entire lifecycle. In adaptive transformer layer, we use an adaptive temperature coefficient in the activation function to adjust the attention distribution across different subsequences in various scenarios. For example, behaviors related to short-term interests learn a lower temperature coefficient, while those related to long-term interests learn a higher temperature coefficient. Different scenarios are assigned different temperature coefficients during training. In block fusion module, we integrate the extracted user interest representations from these subsequences. This module enables interest fusion between different subsequences through self-attention and a bit-wise gating mechanism.

To address the dual challenges of computational complexity and fragmented sequence structures in recommendation systems, we propose ASTRO model. This model integrates the multi-behavior and multi-scenario characteristics of recommendation systems into the Transformer architecture, while considering the resource demands associated with scaling. 
% Unlike NLP tasks that typically handle single continuous sequences, recommendation systems involve non-continuous and diverse sequences. 
% Our design philosophy follows the principle of "one to many, many to one,"
Our model comprises three modules: Multi-scale Sequence Partitioning (MSP), Adaptive Transformer Layer (ATL), and Bit-wise Gating Fusion (BGF).
Specifically, \MSP generates multi-scale sequences from user lifecycle sequence. These multi-scale sequences represent different types of subsequences (multi-scenario/multi-behavior/short and long-term behaviors). For each subsequence, we apply a corresponding transformer block composed of stacked ATLs to extract user interest. Additionally, we extend the time span of important subsequences to cover the user's entire lifecycle.
\ATL uses an adaptive temperature coefficient in the activation function to adjust the attention distribution across different subsequences in various scenarios. 
% For example, behaviors related to short-term interests learn a lower temperature coefficient, while those related to long-term interests learn a higher temperature coefficient. Different scenarios are also assigned different temperature coefficients during training.
Finally, BGF aggregates the representations from adaptive Transformer blocks corresponding to each subsequence to produce a unified output, which enables interest fusion between multi-scale sequences through ATL and a bit-wise gating mechanism.
% This design not only enhances the model's ability to handle diverse user behaviors across multiple scenarios but also balances performance improvement with computational efficiency.
Figure \ref{fig:model_arch} illustrates the detailed workflow of ASTRO model.


% Through the above programs, various problems in preliminaries are basically solved:

% Complextity: When the sequence length is increased by $m$ times, the time complexity can be reduced from the $m^2$ increase to the $m$ increase. Significant acceleration can be obtained during training and inference stage.

% Extensibility: It has been demonstrated that the temporal limitations imposed on critical behaviors can be circumvented. For behaviors that are of paramount importance and infrequent, the full lifecycle behavior is employed, yielding a more pristine sequence than the equivalent sequence of length $10^3$. This approach is compatible with the block framework, thereby enhancing the efficacy of training and inference processes.

% Conflict: This issue is addressed by extending the critical sparse behaviors to the user's entire lifecycle, which contributes significantly to the available information augment. Additionally, the block approach alleviates the attention conflicts between different subsequences.

\subsubsection{\textbf{Multi-Scale Sequence Partitioning}}
The approach of \MSP involves reorganizing user sequences based on different strategies, which can be represented by the following formula:
\begin{gather}
S = \{x_1,x_2,...,x_{n_{s}}\} \label{eqn:S} \\
S_k = \text{MSP}(S, a_k) = \{x^{a_k}_{1},x^{a_k}_{2},...,x^{a_k}_{n_k}\} \label{eqn:Sk}
\end{gather}
where $S$ represents the user lifecycle sequence,  $x_i$ denotes $i$-th item ID from the entire item set $\textbf{X}$. $S_k$ represents the $k$-th subsequence extracted from the user lifecycle sequence $S$ based on the extraction strategy $a_k$, and $x^{a_k}_{j}$ indicates the $j$-th item in the subsequence $S_k$. The $n_s$ and $n_k$ denote the length of $S$ and $S_k$, respectively. We assume that there are a total of $N_b$ extraction strategies and $\sum_{k=1}^{N_b} n_k=n$. In our practical application, $n \ll n_s$. This is because the extraction strategy typically retains only the user's positive behaviors. Consequently, the computational complexity under a single Transformer can be reduced from $O(n_s^2d)$ to $O(n^2d)$. We further improve the training process by employing the corresponding Transformer block for each subsequence $S_k$, thereby achieving a time complexity of $O(n_k^2d)$. We design each extraction strategy to extract subsequences of equal length, such that $n_k=n/N_b$. In the case of fully serial operations, this results in a time complexity of $O(n^2d/N_b)$. Therefore, even when $N_b=2$, we can still achieve substantial training acceleration.
In summary, \MSP reduces the computational complexity from $O(n^2d)$ to $O(n^2d/N_b)$, and transform user lifecycle sequence into multi-scale sequences. These enhancements improve the efficiency and scalability of the recommendation system.

% This section introduces positive behavior extraction using a multi-behavior segmentation approach, motivated by two key challenges. First, the time complexity $O(n^2d)$ of Transformer significantly increases computational burdens for training and inference as sequence length increases. Second, integrating multiple behaviors into a single sequence can dilute the impact of critical yet sparse behaviors. To address these challenges, we organize sequences based on distinct behaviors. Specifically, for frequent behaviors (e.g., playend), we use data from the past 3 months; for extremely sparse behaviors (e.g., share and comment), we use data from the past 1 year; and for user asset behaviors (e.g., collect), which are sparse but crucial for capturing user interests, we consider the user entire lifecycle. It should be noted that the sequence length is constrained. For entire lifecycle behavior sequences, the collect sequence length is set to $10^3$, covering over 95\% of users' all collected items in our application scenario. Importantly, subsequence extraction is flexible: if a frequent behavior sequence is excessively long, it is divided temporally to balance model performance and efficiency. Thus, positive behavior extraction can be defined as follows:
% \begin{equation}
% \scriptstyle (i_1,i_2,...,i_{n_i}),(j_1,j_2,...,j_{n_j}),(k_1,k_2,...k_{n_k}) = PSE(x_{1}, x_{2},..., x_{n})
% \end{equation}
% we transform Equation 4 to a new multi-subsequence function as:
% \begin{equation}
% \scriptstyle f((a_{i}^{'},i_1,i_2,...,i_{n_i}),(a_{j}^{'},j_1,j_2,...,j_{n_j}),(a_{k}^{'},k_1,k_2,...k_{n_k}),x_{c})
% \end{equation}
% where we transform sequence to different blocks, and $i, j, k$ represent different subsequences divided according to extraction strategy $a^{'}$, respectively.

% In this approach, subsequence division is predominantly predicated on the type of behavior. There are two main factors that inspired this part. Firstly, as the sequence lengthens, the Transformer's $n^2$ time complexity imposes a substantial burden on the training and inference processes due to its high computational demands. Secondly, when a multi-behavior sequence is integrated into a single sequence, the impact of other behaviors on critical behaviors is exacerbated due to their inherent sparsity. To address these challenges, we propose a modification to the sequence organization, wherein multiple sequences are arranged based on distinct behaviors, facilitating more efficient scalability. The definition of organizational behaviors is as follows: for relatively frequent behaviors, such as play-end behaviors, we select $N$ months of behavioral data; for extremely sparse behaviors, such as user sharing and commenting, the time is stretched to 1 year; for user asset sequences, such as user collection, which is sparse but expresses the user's interest in the core, the time is expanded to the whole life cycle. It should be noted that the number of sequences is limited. For the full life cycle behavior sequences, the order of magnitude in the model training is set to $10^3$, which can cover more than 95\% of the users' collection sequences in the application scenario under study. It is noteworthy that there is no rigid prerequisite for subsequence splitting. In instances where a frequent behavior sequence is exceedingly extensive, it will be divided according to time, thereby achieving a balance between model performance and efficiency. Thus, the sequence organization called positive signal extraction can be defined as:

\subsubsection{\textbf{Adaptive Transformer Layer}}
The Softmax activation function, which normalizes the attention scores, plays a pivotal role in the Transformer architecture. Specifically, the attention mechanism is normalized by $\frac{QK^T}{\sqrt{d_{k}}}$ and subsequently multiplied by $V$. The division by $\sqrt{d_{k}}$ ensures that the distribution of the attention matrix aligns with that of $Q$ and $K$ \cite{hinton2015distilling}. 
However, we generate multi-scale sequences based on different extraction strategies from user lifecycle sequence, and apply corresponding Transformer block to each subsequence. A single scaling factor $\sqrt{d_{k}}$ is insufficient to accommodate the diverse requirements of all Transformer blocks corresponding to multi-scale sequences \cite{he2018determining}.
To further refine the attention distribution within each Transformer block, we introduce an adaptive temperature coefficient for each layer of each block.
We refer to this change as Adaptive Transformer Layer (ATL) , which can be mathematically expressed as follows:
% \begin{equation}
% \begin{gathered}
% Q,K,V =f_{QKV}(S_k) \\
% A(S_k) = \text{Softmax}(QK^{T}/f_T(a_k, r)) \\
% Y(S_k) = f_{FFN}(A(S_k)V)
% \end{gathered}
% \end{equation}
\begin{equation}
\begin{gathered}
Q,K,V =f_{QKV}(X(S_k)), \\
R(S_k) = QK^{T} + f_{b}^{p,t}(a_k, r),  \\ 
A(S_k) = \text{Softmax}(R(S_k)/f_{tc}(a_k, r)), \\
Y(S_k) = f_{FFN}(A(S_k)V).
\end{gathered}
\label{eqn:ATL}
\end{equation}
Compared to conventional Transformer layer, we introduce an adaptive temperature coefficient and adjust relative attention bias from a recommendation perspective. Here, $X(S_k)\in \mathbb{R}^{s \times d}$, $R(S_k)\in \mathbb{R}^{h \times s \times s} $, $A(S_k) \in \mathbb{R}^{h \times s \times s}$ and $Y(S_k) \in \mathbb{R}^{s \times d}$ represent layer input, raw attention matrix, normalized attention matrix, and layer output, respectively. $s$, $h$, $d$ represent sequence length, head number and feature dimension, respectively.
$f_{QKV}(X(S_k))$ is used to derive query, key, and value matrices from input $X(S_k)$.
$f_{b}^{p,t}(a_k, r)$ denotes relative attention bias \cite{zhai2024actions, raffel2020exploring} that incorporates positional ($p$) and temporal ($t$) information.
$f_{tc}(a_k, r)$ denotes a function that derives the temperature coefficient.
It is important to note that both extraction strategy $a_k$ and recommendation scenario $r$ influence the relative attention bias and the temperature coefficient.
$f_{FFN}$ processes the attention-weighted value matrix through a feedforward neural network (FFN) to produce the final output of the layer. This approach is inspired by the multi-scenario and multi-behavior characteristics of recommendation systems. Specifically, the adaptive temperature coefficient allows for more flexible attention weighting, addressing the limitations of the fixed scaling factor $\sqrt{d_{k}}$ in capturing the inherence of diverse behaviors and scenarios.
  
% The detailed distribution of attention scores will be elaborated in the experimental section.
% From multi-scenario perspective, our model is applied to all application scenarios, which are divided into two main categories: list recommendation and stream recommendation. List recommendation scenarios generate a static list of items for users, while stream recommendation scenarios continuously deliver items based on user real-time interactions. Therefore, stream recommendation benefit from lower temperature coefficients to emphasize recent behaviors, and list recommendation is more suitable for higher temperature coefficients. From multi-behavior perspective, playend sequence is more influenced by recent items and thus is suitable for lower temperature coefficient, while collect sequence reflecting long-term preferences, is suitable for higher temperature coefficient.

% The Softmax activation function used to normalize the attention score plays a pivotal role in the Transformer's architecture. The Softmax is normalized by $\frac{QK}{\sqrt{d_{k}}}$ and subsequently multiplied by V, where dividing by $d_{k}$ is to ensure that the distribution of the attention matrix is consistent with $Q$ and $K$. An adaptive temperature coefficient has been incorporated to model the personalization within each block. This approach draws inspiration from the multi-scene and multi-task problem inherent to recommender systems. First, the model is applied to all the recommendation scenarios of cloud music, which includes two core scenarios, "dailysong" and "userfm". The recommendation form of "dailysong" is list recommendation, while "userfm" is a real-time stream recommendation. Therefore, "userfm" may be more suitable for lower temperature coefficients to strengthen the expression of recent behaviors, while "dailysong" is suitable for smoother temperature coefficients. For different tasks, behaviors such as sharing and commenting are more susceptible to the influence of recent behavior, while collecting behavior is more susceptible to the influence of long-term behavior. Consequently, these two tasks are suitable for lower and higher temperature coefficients, respectively. This point will be discussed in detail in the experimental section. We define it as an adaptive transformer, which can be expressed using this formula:

\begin{figure*}[htbp]
  \centering
  \subfigure[]{\includegraphics[scale=0.4]{CompSketch2.pdf}}
  \hspace{5mm}
  \subfigure[]{\includegraphics[scale=0.3]{kv_cache_v1.pdf}}
  \caption{Two key features of TURBO: (a) Sketch of compression algorithm on training dataset; (b) Sketch of encoder-level KV cache with block-wise parallelism}
  \Description{}
  \label{fig:comp_sketch}
\end{figure*}

\begin{figure}[htbp]
  \centering
  % \includegraphics[width=\linewidth]{CompRatioAndFLOPs.pdf}
  \includegraphics[width=\linewidth]{Compression2.pdf}
  \caption{Compression Ratio and Training FLOPs based on our dataset}
  \Description{}
  \label{fig:comp_ratio_flops}
\end{figure}

\subsubsection{\textbf{Bit-wise Gating Fusion}}
However, when user lifecycle sequence is separated into multi-scale sequences, there is a lack of interaction between them \cite{xiao2020deep,li2019multi}. Therefore, we propose a bit-wise gating fusion module that integrates information across the $N_b$ subsequences corresponding to our $N_b$ blocks. Specifically, each block generates an output vector $E(S_k)$ through each adaptive Transformer block, and these $N_b$ output vectors are concatenated as $E(S) \in \mathbb{R}^{N_b \times d}$. The concatenated vectors are then processed through a new ATL, followed by a sigmoid activation function to achieve bit-level gating. Finally, the vectors pass through a subsequent network to produce the final output score. The bit-wise gating fusion module can be formally expressed as:
\begin{equation}
\begin{gathered}
    E(S) = \{E(S_1),E(S_2),...,E(S_{N_b})\}, \\
    G(S) = \text{ATL}(E(S)), \\
    Y(S) = G(S) \odot \sigma\big(f_{gate}(G(S))\big).
\end{gathered}
\end{equation}
where $\sigma$ is sigmoid activation and $f_{gate}$ represents a squeeze-and-excitation module \cite{hu2018squeeze}, which ensure the identity between the input and output dimensions to dynamically adjusts the contribution of each element in $G(S) \in \mathbb{R}^{N_b \times d}$.
$f_{gate}(G(S)) \in \mathbb{R}^{N_b \times d}$ and $Y(S) \in \mathbb{R}^{N_b \times d}$ represent the bit-wise attention matrix and the output of fusion module, respectively.
$\text{ATL}$ (Adaptive Transformer Layer in Equation \ref{eqn:ATL}) is used to calculate the similarity between different blocks to interact the different interest from subsequences. In contrast, the ATL in fusion function doesn't incorporate relative attention bias, and the temperature coefficient is solely determined by the recommendation scenario.
The attention operation of ATL on $N_b$ blocks can be regarded as field-wise interactions \cite{hu2018squeeze,huang2019fibinet}, which facilitate information exchange at the feature level. Our method enhances interest fusion by adding bit-wise interactions. This allows the model to capture precise relationships among sequences. As a result, the model's ability to understand multi-scale sequences is significantly improved.
It is worth mentioning that although this attention mechanism has a complexity of $O(N_b^2d)$, the number of extraction strategy $N_b$ is much smaller than the dimension $d$ of the feature vector in the fusion stage. Thus the computational complexity of bit-wise gating fusion module is relatively low.

% The proposed recommendation model exhibits remarkable merits that significantly enhance its performance and applicability. 1) the model complexity achieves a substantial reduction from $O(n^{2}d)$ to $O(n^{2}d/m)$, where $m$ denotes the number of subsequence block. This enhancement enables a palpable acceleration in both training and inference processes even in the case of $m=2$, thereby rendering the model more efficient and scalable for handling large-scale datasets.
% 2) our model breaks through temporal constraints for certain crucial behaviors. For highly sparse yet significant behaviors, we utilize users' entire life-cycle behaviors. Compared to sequences of the same length $10^{3}$, our approach yields purer sequences. Moreover, the compatibility with the block framework not only ensures seamless integration but also further boosts the efficiency of training and inference, making the model more adaptable to diverse recommendation tasks and datasets.
% 3) By extending important sparse behaviors to the entire life-cycle behaviors, the model gains a significant enrichment of information. Additionally, the block-wise approach alleviates unfair allocation of attention scores between different subsequences, which will be discussed in the next section. Overall, these advantages collectively contribute to the robustness and superiority of our recommendation model, setting it apart as a highly promising solution in the realm of recommendation systems.

\subsection{TURBO Acceleration Framework}
A typical ranking model for recommendation systems involves elaborate features from one user and one item as model inputs. Probabilities or scores for specific tasks such as  is-click or not are calculated as model outputs through a well-designed neural network. Such an arrangement for model inputs and outputs can be summarized as "Single User, Single Item (SUSI)". It's instinctive to expand the first dimension of input tensor to batch size during training stage to achieve "Multiple Users, Multiple Items (MUMI)".

When it comes to the online serving system, "Single User, Multiple Items (SUMI)" is required to get scores for multiple candidates in a single request from one single user. Simply stacking user features to the number of candidates as "MUMI" seems like a common approach. However, "MUMI" might not be affordable for online serving due to the extra time costs caused by repeated computation of user feature. This redundant computation is almost unacceptable especially for transformer-dominated architectures with user features composed by long sequences. Zhai, et al.\cite{zhai2024actions} proposed the M-FALCON algorithm to achieve "SUMI"-style batched inference. Online throughput performance can be boosted with the aid of kv-cache \cite{pope2023efficiently} and micro-batch parallelism. 

Further, we propose TURBO, formulating these prefill-encode two stages into one forward pass. With re-arrangement of training dataset via dynamic compression algorithm explained below, TURBO serves as a unified framework compatible for both offline training and online inference.

\subsubsection{Dynamic Compression on Training Data}
To accommodate the training phase to the "SUMI" setting in TURBO, a fixed number of labels and features from the item side related to the same user need to be aggregated to the form of label list and feature list. The list can be padded to a constant size $N_c$ across all users, and we call $N_c$ as the degree of aggregation. 
It should be noted that user behavior might slowly accumulate over a period of time. If user behavior sequence grows up within an aggregated batch, the longest as well as the latest sequence should be kept to avoid loss of information. What's more, an extra dependence list need to be stored for each row of training sample as shown in Figure \ref{fig:comp_sketch}. With the help of this auxiliary list, we can build customized attention mask matrix for each row to prevent violation of causality. 

The value of $N_c$ should be determined carefully for a given training dataset. Either too small or too large value of $N_c$ would be harmful to data compression and computation efficiency. Sample distribution $g(n)$ of a given dataset can be defined as the number of unique user who have the total number of record $n$. For instance, $g(2)=1000$ means that there are 1000 unique users each has 2 row of records in a given dataset. Let $N$ denote total number of record of training dataset, $N_p$ denote total number of rows to be padded, $D_{user}$ and $D_{item}$ denote the average number of bytes in each row to store user features (including user behavior sequence) and item features respectively. The compression ratio $\alpha$ can be estimated by the following formulation \ref{eqn:comp_ratio}. The term "$\operatorname{sizeof}(int)$" shows the part of storage for dependence list composed by integer element.

\begin{gather}
p(n) = \left\lceil \frac{n}{N_c} \right\rceil \cdot N_c - n \\
N_p = \sum_n g(n)\cdot p(n) \\
\alpha = \frac{(N + N_p)D_{user}/N_c + (N + N_p)(D_{item} + \operatorname{sizeof}(int))}{N(D_{user} + D_{item})} \label{eqn:comp_ratio}
\end{gather}


\subsubsection{Encoder-level KV-cache with Block-wise Parallelism}
Since each block under ASTRO architecture is independent, calculations between blocks can be parallelized. We propose to fuse all transformer blocks with the same number of layers into a whole module. The first dimension in input tensors, weight and bias parameter is extended, which represents the block number in multi-head attention module and FFN module. Block-wise parallelism is actually achieved through scheduling of larger matrix operations at CUDA level. 

To achieve lower computation costs, we followed the idea in M-FALCON to conduct a two-stage forward pass. The first stage is to get key/value cache vector for items in user behavior sequence. For the last layer in transformer module, results of kv-cache can be returned immediately without subsequent calculating like attention scores and FFN module. We named this trick "last layer short-circuit" to save a certain amount of computation.
The second stage is to get scores for target items in a batched way with kv-cache obtained in the first stage. Computation cost in the second stage is proportional to the size of micro-batch, which is $N_c$ during model training phase. The total training FLOPs $F_{total}$ for one epoch can be estimated as summation of the two stages:
\begin{align}
F_{total} = \frac{N + N_p}{N_c} \cdot \left( F_{stage1} + F_{stage2}(N_c) \right) \label{eqn:train_flops}
\end{align}

We plot the total training FLOPs for one epoch of our dataset against $N_c$ in Figure \ref{fig:comp_ratio_flops}, along with the compression ratio estimated through Equation \ref{eqn:comp_ratio}. It can be shown that the optimal $N_c$ to achieve maximum storage savings and the one to achieve minimum training cost are very close. Thus, we determined to use $N_c=5$ as the base setting for the rest of the discussion.

\section{Experiments}
% In this chapter, we will first compare the performance of our model structure with other state-of-the-art methods on different datasets; Then, the attention score will be used as an intermediary to discuss how the model structure modification points are related to specific problems in the recommendation system; Finally, we presented the scaling laws curves for sequence length, and model layers, as well as how we determined the iterative path for online scaling up based on the curves. Finally, we obtained online metrics for scaling up in multiple scenarios and corresponding performance improvements.
In this section, we detail the offline and online experiments conducted on real industrial data to evaluate our proposed method, addressing the following four research questions:
\begin{itemize}
\item RQ1: How does ASTRO model perform in offline evaluation compared to state-of-the-art (SOTA) models?
\item RQ2: How does ASTRO model demonstrate superior scalability compared to DLRM and Transformer?
\item RQ3: How can we allocate resources to scale up our model by considering the impact of different factor combinations on AUC under equivalent FLOPs? 
\item RQ4: How does Climber framework perform in industrial recommendation systems?
\end{itemize}

% \begin{table}[t]
% \centering
% \caption{Dataset Statics}
% \label{tab:Dataset}
% \begin{tabular}{@{}lcccccc@{}}
% \toprule
% Dataset & \#User & \#Item & \#Interaction \\
% \midrule
% Industrial &  &  &  \\
% Spotify &  &  &  \\
% 30Music &  &  & \\
% Amazon-Book & &  & \\
% \bottomrule
% \end{tabular}
% \end{table}

% \subsection{The results of model structure on different datasets}
\subsection{Experimental Setting}
\subsubsection{Dataset}
To validate the effectiveness of our method in recommendation systems, we construct a dataset using real user behavior sequence as the primary feature. The dataset is used to predicts different user actions on the candidate items based on historical interactions. Important user behaviors include full-play, like, share and comment. 
In order to protect data privacy, we only present statistical data for our recommendation scenarios.
% Our dataset comprises \textcolor{red}{tens of} millions daily active users, \textcolor{red}{several} million tracks available for recommendation each day, and \textcolor{red}{>100} million samples collected daily. 
% The average user behaviors per day are approximately \textcolor{red}{>60}. Due to the time required for users to consume one track, the number of generated interactions is smaller compared to other short-video platforms. The average sequence length of user full-play and like interactions over a month is \textcolor{red}{>30} and \textcolor{red}{>2}, respectively. 
Additionally, we evaluated our model on three recommendation datasets, \textbf{Spotify}\cite{brost2019music}, \textbf{30Music}\cite{turrin201530music} and \textbf{Amazon-Book}\cite{mcauley2015image}. 
A more detailed analysis of these datasets after preprocessing is presented in Appendix \hyperref[appendix:B.1]{B.1}.
% In our dataset, we exclude statistical features but retain only some categorical features of users and items. 

% \renewcommand{\thefootnote}{\fnsymbol{footnote}} % 设置脚注符号为非数字形式
\begin{table*}[htbp]
\centering
\caption{Evaluation of methods on public/industrial datasets}
\begin{tabular}{c|cc|cc|cc|cc}
% \begin{tabularx}{\columnwidth}{@{}ccccccccc@{}}
\hline
\multicolumn{1}{c|}{} & \multicolumn{2}{c|}{Spotify} & \multicolumn{2}{c|}{Amazon-Book} & \multicolumn{2}{c|}{30Music} & \multicolumn{2}{c}{Industrial} \\
 & AUC & LogLoss & AUC & LogLoss & AUC & LogLoss & AUC & LogLoss \\ \hline
DLRM(Baseline) & 0.7606 & 0.5761 & 0.7842 & 0.5541 & 0.8927 & 0.2012 & 0.8216 & 0.7067 \\ \hline
DIN & 0.7557 & 0.5803 & 0.7796 & 0.5580 & 0.8861 & 0.2044 & 0.8158 & 0.7109 \\
TWIN & 0.7589 & 0.5772 & 0.7831 & 0.5563 & 0.8903 & 0.2019 & 0.8203 & 0.7078 \\ 
Transformer & 0.7621 & 0.5735 & 0.7836 & 0.5557 & 0.8930 & 0.2007 & 0.8214 & 0.7074 \\ 
HSTU & 0.7626 & 0.5722 & 0.7869 & 0.5520 & 0.8938 & 0.1982 & 0.8217 & 0.7053 \\ \hline
ASTRO (-ATL,-BGF) & 0.7635 & 0.5710 & 0.7873 & 0.5518 & 0.8944 & 0.1978 & 0.8221 & 0.7045 \\
ASTRO (-BGF) & 0.7655 & 0.5694 & 0.7881 & 0.5510 & 0.8950 & 0.1972 & 0.8225 & 0.7034 \\ 
\underline{ASTRO} & \underline{0.7663} & \underline{0.5687} & \underline{0.7887} & \underline{0.5501} & \underline{0.8986} & \underline{0.1957} & \underline{0.8230} & \underline{0.7029}
\\
\textbf{ASTRO-large} & \textbf{0.7702} & \textbf{0.5666} & \textbf{0.7914} & \textbf{0.5472} & \textbf{0.9035} & \textbf{0.1916} & \textbf{0.8398} & \textbf{0.6911} \\ \hline
\textbf{\%Improve} & \textbf{+1.26\%} & \textbf{-1.64\%} & \textbf{+0.91\%} & \textbf{-1.24\%} & \textbf{+1.20\%} & \textbf{-4.77\%} & \textbf{+2.21\%} & \textbf{-2.20\%} \\ \hline
\end{tabular}
\label{tab:ASTRO}
\end{table*}
% \footnotetext{The baseline model that has been implemented for online deployment.}

\subsubsection{Compared methods}
% In the selection of comparison methods, in addition to DLRM as the comparison model on our front line, we also introduced two transformer based models, BST and HSTU, as the comparison models. In order to compare the model structure, we introduced block, activation, and fusion for experimentation. In order to ensure the fairness of the experiment, a unified setting was used for both the structure and features

% In our comparative study, we select DLRM as the primary benchmark model. In addition, we incorporate two Transformer-based models: Transformer and HSTU. For a systematic evaluation of the model architectures, we focused on three key aspects: postive behavior extraction, adaptive transformer layer, and block fusion module. Throughout the experiments, we ensured fairness and consistency by adopting a unified configuration for both the structural components and feature representations across all models.

In our comparative study, we select DLRM as benchmark model. DLRM is a model that leverages lifelong sequences and complex feature interactions, which has been deployed in our online systems. In addition to DLRM, we incorporate several SOTA models to provide a comprehensive evaluation. These models include: DIN\cite{zhou2018deep}, TWIN\cite{chang2023twin}, Transformer\cite{vaswani2017attention}, and HSTU\cite{zhai2024actions}. To systematically evaluate the model architectures, we focus on three key components: multi-scale sequence partitioning, adaptive Transformer layer and bit-wise gating fusion. We utilize the ASTRO-large model (6$\times$ layer number, and 4$\times$ sequence length in Industrial setting) to validate the model's scalability. 
Throughout the experiments, we ensure fairness and consistency by adopting a unified configuration for both training hyper-parameters and feature representations across all models. More details about compared methods and experimental setting are presented in Appendix \hyperref[appendix:B.2]{B.2}.
% \footnotetext[1]{http://jmcauley.ucsd.edu/data/amazon/}

\subsection{Overall Performance (RQ1)}
% \subsubsection{Results on Public Datasets}
\subsubsection{Performance Comparison}
% In this section, we will demonstrate the performance improvement of the model after reshaping, with relevant indicators shown in Table 1. From the table, it can be seen that no additional structural changes are required, and significant performance improvements can be achieved by simply disassembling the sequence; Our dataset contains data for all recommended scenarios, and it can be seen that with the introduction of adaptive temperature coefficients for each scenario, offline AUC also shows an increase; The fusion module facilitates the fusion of sequence information between different types, which is helpful for multi task prediction

As shown in Table \ref{tab:ASTRO}, the ASTRO model achieve the best performance in four recommendation datasets. Notably, the application scenarios of \textbf{Spotify} and \textbf{30Music} belong to the same domain as our industrial dataset, which is music recommendation systems. In contrast, \textbf{Amazon-Book} differs significantly from our scenario. However, our model still achieves favorable results on this dataset, indicating its potential adaptability to diverse applications.
Next, we focus on comparing the AUC improvements of ASTRO relative to other methods. 1) as our primary online model, DLRM outperforms DIN and TWIN because DLRM includes a wide range of feature interaction structures beyond lifelong sequence and attention mechanisms. 
2) The Transformer achieves +0.134\% AUC improvement over TWIN. This is because Transformer computes the similarity between all historical items and target item in a single stage.
% 3) The Transformer exhibits an AUC decrease of 0.0002(-0.024\%) compared to DLRM. This phenomenon is also observed in other datasets, suggesting a general trend across different recommendation scenarios. The result is primarily attributed to DLRM's focus on feature interaction, which allows it to achieve better performance before the overall model complexity reaches a level that would be required for the Transformer to match or exceed its performance.
3) HSTU implements several enhancements to the Transformer architecture, achieving an AUC improvement of +0.036\% on our dataset compared to Transformer. However, these enhancements also result in increased computational complexity.
4) ASTRO reduces computational complexity by sequence partitioning and adjusts attention distribution across multi-scenario and multi-behavior through adaptive temperature coefficients. This results in a +0.170\% improvement over DLRM. Furthermore, ASTRO-large achieves a +2.21\% AUC improvement by scaling up the model, achieving the largest offline gain in the past year.

\subsubsection{Abalation Study}
% First, we compare the Transformer and DLRM. The Transformer exhibits an AUC decrease of 0.0025 relative to DLRM. This phenomenon is also observed in other datasets, suggesting a general trend across different recommendation scenarios. The result is primarily attributed to DLRM's focus on feature interaction, which allows it to achieve better performance before the overall model complexity reaches a level that would be required for the Transformer to match or exceed its performance.
% Second, HSTU implements several enhancements to the Transformer architecture, achieving an AUC improvement of 0.0033 on our dataset compared Transformer. However, these enhancements also result in increased computational complexity.

% The subsequent analysis focuses on the impact of our model on the Industrial dataset. 
% Compared to Transformer, ASTRO (-ATL, -Fusion) achieves a positive AUC gain of 0.0040 by transformer users' lifecycle sequence into multiple subsequence blocks through positive behavior extraction. Besides, ASTRO (-Fusion) further enhances the Transformer by incorporating an adaptive temperature coefficient, resulting in an additional AUC improvement of 0.0009. Both enhancements directly impact the attention scores, which will be analyzed in detail in the following section.
% Following the incorporation of the fusion module, an improvement of 0.0005 in AUC is observed. This module is designed to integrate user interests represented by different subsequences, thereby emphasizing the importance of interest fusion in recommendation systems.
% In summary, our model demonstrates strong performance and adaptability based on the offline evaluations across various datasets.

To systematically evaluate the contributions of each component in our ASTRO model, we conduct a comprehensive set of experiments across multiple datasets. For illustrative purposes, we focus on the detailed ablation study conducted on the Industrial dataset and select Transformer for comparison with ASTRO series. By incorporating MSP, ASTRO (-ATL, -BGF) transforms user lifecycle sequence into multi-scale subsequence blocks. This enhancement results in a positive AUC gain of +0.085\% compared to the Transformer model. ASTRO (-BGF) further improves the model by introducing an adaptive temperature coefficient. This component adjusts the attention distribution dynamically, leading to an AUC improvement of +0.134\%. Finally, the incorporation of BGF brings about an AUC improvement of +0.195\%. This module integrates user interests represented by different subsequences, emphasizing the importance of interest fusion in recommendation systems. In summary, our model demonstrates strong performance and adaptability based on the offline evaluations across various datasets.

% \subsection{Discussion on the Influence of Model Structure}
% \subsection{Attention Distribution Analysis (RQ2)}
% Sequence modeling in recommendation systems is essential for identifying items in a user's historical sequence that are similar to the target item, with this similarity quantified through attention scores. Before the adoption of attention mechanisms, sequence processing relied on average pooling, which assigned uniform attention scores to all items. The attention mechanism in DIN has been shown to enhance recommendation performance by assigning distinct attention scores to different items. The Transformer-based model emphasizes attention mechanisms. Thus this section explores the impact of positive behavior extraction and adaptive temperature coefficients on the attention distribution. The attention score will be introduced for two distinct behaviors: playend and collect.

% \subsubsection{The impact of positive behavior extraction}
% Positive behavior extraction not only reduces computational complexity but also directly impacts attention distribution. Specifically, the proportion of "playend" behaviors is higher than that of "collect" behaviors. To address this imbalance, we distinguish between these two behaviors and extend the time window for "collect" behaviors, thereby making the number of "collect" and "playend" behaviors approximately equal.
% We conduct comparative experiments between the Transformer and ASTRO. In the Transformer experiment, sequences contain equal proportions of "collect" and "play" behaviors for computation. In contrast, the ASTRO model separates these behaviors and trains them in a block-wise manner, with attention distributions illustrated in Figure \ref{fig:deploy}.
% In the Transformer structure, even with a similar number of "collect" and "playend" behaviors, the attention ratio for "playend" behaviors reaches 71\%. This is primarily because "playend" samples have a higher proportion and naturally occupy more gradients. Consequently, "playend" behaviors receive higher attention scores due to the nature of the task, resulting in less attention to "collect" behaviors and poorer performance on "collect" tasks.
% In the ASTRO model, each behavior is assigned to a separate block with its own attention distribution. Through positive behavior extraction, we can not only increase the number of sparse yet important behaviors but also improve their attention distribution, thereby enhancing model effectiveness. Performance comparisons between the Transformer and ASTRO (-ATL, -FUSION) across different datasets are shown in Table 1, which indicates that significant improvement can be achieved solely through positive behavior extraction.

% \begin{figure}[t]
%   \centering
%   \includegraphics[width=\linewidth]{attention.pdf}
%   \caption{Attention Distribution of Different Models on Different Behaviors}
%   \Description{}
% \end{figure}

% \subsubsection{The influence of adaptive temperature on attention distribution}
% The motivation for this point is that after breaking down the block, a point can be found that the high scores of the end behavior are concentrated in the recent part in Figure \ref{fig:comp_sketch}, and the polarization phenomenon of scoring will be more severe; However, there is no obvious concentration phenomenon in the collect behavior, and the scoring is relatively more uniform. Therefore, considering the setting of adaptive temperature coefficients for different scenarios and tasks, The temperature coefficients corresponding to (fm,end), (fm,red), (daily, end), and (daily, red) are 1, 2, 3, and 4, respectively. It can be seen that in the same scenario, the end play action has a lower coefficient than the collect action, because the end play action tends to be more inclined towards the user's short-term interests, while the red heart, as a representative of the user's long-term interests, has a higher temperature coefficient. In the same task, real-time recommended scenes will also have lower temperature coefficients in different scenarios. In temperature adaptation, there are only two independent variables: scene and task. In order to have a more direct impact, the most direct influence method is selected.

\begin{figure*}[htbp]
  \centering
  \includegraphics[width=\linewidth]{scale_v1.pdf}
  \caption{Scaling Curve. (a) Scalability: DLRM vs Transformer vs ASTRO in Industrial Dataset. (b) Model performance of scaling layer number with constant sequence length. (c) Model performance of scaling sequence length with constant layer number.}
  \Description{}
  \label{fig:scale}
\end{figure*}

% \subsection{Efficient scaling law for recommendation systems}
% \subsubsection{Exploring online guidance from scaling curves}
\subsection{Scalability (RQ2)}
% Before discussing the scaling curve, we first define FLOPs as $C \propto s * l$, where $s$ represents the length of the sequence and $l$ represents the number of layers in the model, respectively. As described in Figure \ref{fig:comp_sketch}, with the increase of FLOPs, the AUC in the test dataset follows a certain power-law improvement. As the comparison method, the Transformer model is equivalent to ASTRO model without fusion module and $T=1$ and $blockNum=1$. Under the same FLOPs, the Astro model has better improvement than the Transformer model. It is indicated that splitting blocks, adaptive temperature and fusion module can help the model achieve efficient scale up almost without increasing complexity. Besides, We further analyzed the relationship between the Astro’s performance with the sequence length and the layer number. Although we found that increasing the sequence length and layer number can improve the model performance, simply controlling one factor can't improve the performance without limitations. The next step is to discuss how to reasonably control model variables to scale up our model.

Before discussing model scalability, we formally define FLOPs:
\begin{equation}
C \propto s * l 
\end{equation}
where $s$ represents the sequence length and $l$ represents the layer number in the model. 
% Our analysis reveals that when $s$ is small, $s$ dominates the relationship between $C$ and $s$ compared to $s^2$, hence only $s$ is retained.
In large-scale scenarios, the quadratic computational complexity of attention mechanisms accounts for only a minor proportion of the model's overall FLOPs even with longer sequence\cite{casson2023transformerflops,kaplan2020scaling,hoffmann2022training}. Thus in our computational analysis, we focus solely on the linear part of sequence length $s$ within  $C \propto s * l$ relationship.
The scaling curves of DLRM, Transformer, and ASTRO are shown in Figure \ref{fig:scale}(a). Although Transformer can achieve better performance than DLRM when FLOPs exceed $10^9$, its efficiency is notably lower than DLRM between $10^7$ and $10^8$. Compared to the Transformer, ASTRO exhibits a more ideal scaling curve due to its higher starting point and larger slope. When FLOPs are below $10^{7.5}$, ASTRO's performance remains weaker than DLRM, but the crossover point shifts to the left, enabling the ASTRO model to achieve performance transformation more efficiently than Transformer.
In this experiment, the two main factors affecting FLOPs are layer number and sequence length. For the ASTRO, we illustrate the relationship between model performance and the layer number in Figure \ref{fig:scale}(b) and sequence length in Figure \ref{fig:scale}(c), respectively. When the sequence length is fixed, model performance improves in a manner similar to a power-law with the increase in the layer number; when the layer number is fixed, there is a similar improvement in model performance as the sequence length increases.
Thus, our proposed ASTRO model exhibits scaling curves in terms of FLOPs, sequence length, and layer number, and has a more efficient scaling curve compared to Transformer.

\subsection{Efficient Allocation (RQ3)}
% Results are shown in Table 3. We observe that under the same FLOPs, the combination of layer number and sequence length can lead to significant changes on offline test AUC. Firstly, due to the linear relationship between flops and the layer number and the sequence length, the product of both variables is also basically the same with the same FLOPs. When the flops are on the order of $10^9$, the model with 8 layers and a length of 400 achieves the best performance; When flops are on the order of $10^8$, the model with 4 layers and 400 lengths achieves the best performance. We can find that the expansion of a single factor may limit the development of the model, so when it comes to expanding the model, it's best to consider multiple factors simultaneously. An interesting phenomenon is that, based on the model with a length of 400 and 4 layers, doubling the number of layers yields greater results, which leads to the observation that stacking layers can achieve greater benefits before the model flops reaches a higher level. Another reason is that block design itself can support important sparse behaviors with longer lifetimes. Therefore, in our scenario, blocks can amplify the sequence benefits, and our engineering side can ensure that both the sequence length and layers can be effectively increased.

From Figure \ref{fig:scale}(b, c), it is evident that increasing the sequence length and the layer number can improve the model's AUC. However, the priority of these two parameters has not been extensively discussed. Table \ref{tab:performance_comparison} presents the model's AUC for the same FLOPs with different layer number and sequence length. It is clear that under the same FLOPs, the combination of layer number and sequence length can lead to significant changes in the offline testing AUC.
According to $C \propto s*l$, the product of layer number and sequence length remains constant for the same FLOPs. When the FLOPs is $4.11 \times 10^8$, the model achieves the best performance of 0.8301 AUC on a ($400s\times4l$) model; When the FLOPs is $1.01 \times 10^9$, the model achieves the best performance of 0.8335 AUC on a ($400s\times8l$) model.
We observe that expanding a single factor may limit the model's development. Therefore, it is best to consider both layer number and sequence length equally when scaling up the model. For example, with a ($400s\times4l$) model, if we need to increase FLOPs by 4 times, we can choose between ($1600s\times4l$), ($800s\times8l$), and ($400s\times16l$). From Table \ref{tab:performance_comparison}, it can be found that the best choice is ($800s\times8l$), which jointly expands both factors to increase the model's AUC from 0.8301 to 0.8382.
This conclusion also guides how to allocate resources online. In our practical recommendation systems, usually only one factor is chosen for each iteration. Therefore, when scaling up online, we alternate between increasing sequence length and layer number.

\begin{table}[htbp]
\centering
\caption{
% Performance Comparison of Sequence Length and Layer Number under Equivalent FLOPs
Performance Comparison under Equivalent FLOPs
}
\label{tab:performance_comparison}
\begin{tabular}{@{}lcccccc@{}}
\toprule
FLOPs & Sequence Length & Layer Number & AUC \\
\midrule
\multirow{4}{*}{$4.11 \times 10^{8}$} & 1600 & 1 & 0.8212   \\
& 800 & 2 & 0.8280   \\
& \textbf{400} & \textbf{4} & \textbf{0.8301}  \\
& 200 & 8 & 0.8297   \\
\midrule
\multirow{4}{*}{$1.01 \times 10^{9}$} & 1600 & 2 & 0.8286  \\
 & 800 & 4 & 0.8323   \\
 & \textbf{400} & \textbf{8} & \textbf{0.8335}  \\
 & 200 & 16 & 0.8321 \\
 \midrule
\multirow{3}{*}{$2.55 \times 10^{9}$} & 1600 & 4 & 0.8365  \\
 & \textbf{800} & \textbf{8} & \textbf{0.8382}   \\
 & 400 & 16 & 0.8367  \\
\bottomrule
\end{tabular}
\end{table}

\begin{table}[htbp]
  \caption{Online Metric Improvement Compared with DLRM}
  \label{tab:online_metric}
  \begin{tabularx}{\columnwidth}{@{}ccccc@{}}
    \toprule
    Method & FLOPs & \makecell{Sequence \\ Length} & \makecell{Layer \\ Number} & \makecell{Online Metric} \\ 
    \midrule
    DLRM & $3.45\times 10^7 (6\times)$ & - & - & $+0\%$\\
    \midrule
    \multirow{8}{*}{Climber} & $5.82\times 10^6 (1\times)$ & $100$ & $1$ & $-4.95\%$\\
     & $1.84\times 10^7 (3\times)$ & $400$ & $1$ & $-1.31\%$\\
     & $3.46\times 10^7 (6\times)$ & $800$ & $1$ & $-1.22\%$\\
     & $4.11\times 10^8 (71\times)$ & $400$ & $4$ & $+3.65\%$\\
     & $1.01\times 10^9 (174\times)$ & $800$ & $4$ & $+4.29\%$\\
     & $2.79\times 10^9 (479\times)$ & $1600$ & $4$ & $+7.78\%$\\
     & $2.31\times 10^9 (397\times)$ & $800$ & $8$ & $+10.68\%$\\
     & $3.61\times 10^9 (620\times)$ & $800$ & $12$ & \textbf{$+12.19\%$}\\
  \bottomrule
  \end{tabularx}
\end{table}


% \subsection{Scaling up online recommendation models}
\subsection{Online A/B Test (RQ4)}
% As shown in Table 3, the results of deploying ASTRO model online using TURBO framework are presented. Here, Metric indicates the relative improvement of core online metirc, and Efficiency reflects the relative improvement of actual online inference efficiency. Consistent with the conclusion that sequence length and layer number are equally important, our model demonstrates online scaling curves for Metric and FLOPs by only adjusting sequence length and layer number.
% Firstly, ASTRO model with $5.82\times10^7$ FLOPs exhibits a negative metric improvement but achieves a 400\% improvement in efficiency due to TURBO framework. When the FLOPs value of the ASTRO model match that of DLRM, there is only a slight negative metric, indicating that ASTRO model's efficiency is lower with fewer FLOPs compared with DLRM, but there is still a significant improvement in inference efficiency. Furthermore, when the FLOPs value of ASTRO model is $2.79\times10^9$, the online Metric improvement reaches 7.78\%, accompanied by a +16\% increase in inference efficiency. Finally, when the FLOPs value is $3.61\times10^9$, an online metric improvement of +12.19\% is achieved. Despite a 100x increase in FLOPs compared to DLRM, the negative inference efficiency improvement is considered acceptable in practical online applications. At present, this model has been applied to serve all user traffic in our actual music recommendation system.
% It is noteworthy that there has been no increase in the CPU/GPU/Memory resources throughout this process.
% To the best of our knowledge, ASTRO is the first recommendation model to display both offline and online scaling curves while maintaining resource balance. Moreover, it achieves +12.19\% metric improvement, representing the largest improvement in the past year.

As shown in Table \ref{tab:online_metric}, the results of deploying Climber framework are presented. 
% Here, \textcolor{blue}{online metric} indicates the relative improvement of core online metirc. 
Consistent with the conclusion that sequence length and layer number are equally important, our model demonstrates online scaling curves for metric and FLOPs by only adjusting sequence length and layer number.
Firstly, ASTRO model with $5.82\times10^6$ FLOPs exhibits a negative metric improvement. When the FLOPs value of the ASTRO (6$\times$) model match that of DLRM (6$\times$), there is only a slight negative metric, indicating that ASTRO model's efficiency is lower with fewer FLOPs compared with DLRM. Furthermore, when the FLOPs value of ASTRO (479$\times$) model is $2.79\times10^9$, the online metric improvement reaches +7.78\%. Finally, when the FLOPs value of ASTRO (620$\times$) model is $3.61\times10^9$, an online metric improvement of +12.19\% is achieved. Despite a 100$\times$ increase in FLOPs compared to DLRM, the change in online inference efficiency is regarded as acceptable in practical online applications due to TURBO acceleration framework. Our TURBO framework achieves a 5.15$\times$ acceleration during training and a 2.92$\times$ acceleration in online inference with the same FLOPs. This significantly boosts the efficiency of both training and inference stage. Currently, this model has been applied to serve all user traffic in our actual music recommendation system.
It is noteworthy that there has been only a moderate increase in the CPU/GPU/Memory resources throughout this process.
To the best of our knowledge, ASTRO is the first recommendation model to display both offline and online scaling curves while maintaining resource balance. Moreover, it achieves +12.19\% metric improvement, representing the largest improvement in the past year.

\section{Conclusion}
The Climber method is designed to scale up recommendation models under resource constraints, which consists of two main components: the ASTRO model architecture and the TURBO acceleration framework. 
The ASTRO model integrates the multi-scenario and multi-behavior characteristics of recommendation systems into the Transformer architecture through multi-scale sequence partitioning, adaptive Transformer layer, and bit-wise gating fusion, while reducing the model's time complexity. 
This integration enables the model to exhibit superior scalability in offline evaluation compared to both DLRM and Transformer models.
Furthermore, we introduce the TURBO acceleration framework, which employs dynamic compression and an encoder-level KV cache. This framework allows the deployment of models that are 100$\times$ more complex without increasing prohibitive computing resources. 
The Climber method demonstrates a scaling curve online and achieves a 12.19\% improvement in online metric.
Additionally, experimental findings have elucidated the impact of sequence length and layer number on model performance under equivalent FLOPs, emphasizing the significance of both factors in scaling up model. This insight contributes to the rational allocation of resources.
Our findings demonstrate the superiority of the Climber method in efficiently implementing scaling laws while maintaining resource balance. This capability empowers recommendation systems to overcome resource constraints and explore larger models.

% We propose the ASTRO model, which integrates the unique multi-scenario and multi-behavior characteristics of recommendation systems into the Transformer architecture through multi-scale sequence partitioning, adaptive Transformer layer, and bit-wise gating fusion. This integration enables the model to exhibit superior scalability in offline settings compared to both DLRM and Transformer models.
% Furthermore, we introduce the TURBO framework, which employs dynamic compression and an encoder-level KV cache. This framework allows the deployment of models that are 100x more complex without increasing computing resources. The ASTRO model demonstrates a scaling curve online and achieves a 12.19\% improvement in online metric.
% Additionally, our experimental results elucidate the impact of sequence length and layer number on model performance under equivalent FLOPs, underscoring the significance of both factors in scaling the model. This insight facilitates the rational allocation of resources.
% Our findings demonstrate the superiority of the ASTRO model in efficiently implementing scaling laws while maintaining constant resources. This capability empowers recommendation systems to overcome resource constraints and explore larger models.








%%
%% The acknowledgments section is defined using the "acks" environment
%% (and NOT an unnumbered section). This ensures the proper
%% identification of the section in the article metadata, and the
%% consistent spelling of the heading.

%%
%% The next two lines define the bibliography style to be used, and
%% the bibliography file.
% 指定参考文献样式
\bibliographystyle{plain}

% 引入.bbl文件
\documentclass{MITstyle}

%\usepackage[table]{xcolor}
\usepackage{chngcntr}
\usepackage{hyperref}
\usepackage{microtype}

\title{A Lightweight and Extensible Cell Segmentation and Classification Model for Whole Slide Images}

\author{Nikita Shvetsov~$^{1, }$\footnote{Correspondence e-mail: nikita.shvetsov@uit.no}, Thomas K. Kilvaer~$^{2, 3}$, Masoud Tafavvoghi~$^{4}$, Anders Sildnes~$^{1}$, \\ Kajsa Møllersen~$^{4}$, Lill-Tove Rasmussen Busund~$^{5, 6}$, Lars Ailo Bongo~$^{1}$ \\
%
\vspace{1em} % Space between authors and afilliations
%
\normalfont{\small $^{1}$Department of Computer Science, UiT The Arctic University of Norway}\\
\normalfont{\small $^{2}$Department of Oncology, University Hospital of North Norway}\\
\normalfont{\small $^{3}$Department of Clinical Medicine, UiT The Arctic University of Norway}\\
\normalfont{\small $^{4}$Department of Community Medicine, UiT The Arctic University of Norway}\\
\normalfont{\small $^{5}$Department of Medical Biology, UiT The Arctic University of Norway} \\
\normalfont{\small $^{6}$Department of Clinical Pathology, University Hospital of North Norway} %\vspace{2em}
}

\begin{document}
\maketitle

\section*{Abstract}

% \begin{abstract}
% Developing clinically useful cell-level analysis tools in digital pathology remains challenging due to limitations in dataset granularity, inconsistent annotations, computational demands of advanced models, and difficulties in integrating new technologies into clinical workflows. To address these challenges, we propose a multi-faceted solution that enhances data quality, model performance, and usability to create a lightweight and extensible cell segmentation and classification model.

% First, we update data labels by employing a cross-relabeling process that refines the labels of two existing datasets, PanNuke and MoNuSAC, to create a new unified dataset with enhanced granularity, encompassing seven distinct cell types. Second, we leverage the H-Optimus foundation model as a fixed encoder to improve feature representation for simultaneous cell segmentation and classification tasks. Third, to address the computational demands of foundation models, we employ knowledge distillation to reduce model size and complexity while maintaining comparable performance. Finally, to facilitate integration into clinical workflows, we integrate the distilled model into the QuPath software, a widely used open-source platform in digital pathology.

% Our results demonstrate improvements in cell segmentation and classification performance using the H‑Optimus-based model compared to a CNN-based model. Specifically, the average $R^2$ improved from 0.575 to 0.871, and the average $PQ$ score improved from 0.450 to 0.492, indicating better alignment with actual cell counts and enhanced segmentation and classification quality. Furthermore, the distilled student model maintains performance comparable to the larger foundation model while reducing the parameter count by a factor of 48.
% Overall, by reducing computational complexity and integrating it into existing workflows, the proposed approach may significantly impact diagnostic processes, reduce the workload of pathologists, and contribute to improved patient outcomes. Though our approach shows potential enhancements in efficiency and usability of cell segmentation and classification models in digital pathology, extensive validation is needed to deploy these models in clinical practice.
% \end{abstract}

%%% shortened abstract
\begin{abstract}
Developing clinically useful cell-level analysis tools in digital pathology remains challenging due to limitations in dataset granularity, inconsistent annotations, high computational demands, and difficulties integrating new technologies into workflows. To address these issues, we propose a solution that enhances data quality, model performance, and usability by creating a lightweight, extensible cell segmentation and classification model. 

First, we update data labels through cross-relabeling to refine annotations of PanNuke and MoNuSAC, producing a unified dataset with seven distinct cell types. Second, we leverage the H-Optimus foundation model as a fixed encoder to improve feature representation for simultaneous segmentation and classification tasks. Third, to address foundation models' computational demands, we distill knowledge to reduce model size and complexity while maintaining comparable performance. Finally, we integrate the distilled model into QuPath, a widely used open-source digital pathology platform. 

Results demonstrate improved segmentation and classification performance using the H-Optimus-based model compared to a CNN-based model. Specifically, average $R^2$ improved from 0.575 to 0.871, and average $PQ$ score improved from 0.450 to 0.492, indicating better alignment with actual cell counts and enhanced segmentation quality. The distilled model maintains comparable performance while reducing parameter count by a factor of 48. By reducing computational complexity and integrating into workflows, this approach may significantly impact diagnostics, reduce pathologist workload, and improve outcomes. Although the method shows promise, extensive validation is necessary prior to clinical deployment.
\end{abstract}
\clearpage

\section{Introduction}
In digital pathology, accurate segmentation and classification of cells are crucial for many diagnostic, prognostic, and predictive analyses \cite{Jaber_Beziaeva_etal._2019,Lin_Pan_etal._2022,Park_Ock_etal._2022,Shen_Choi_etal._2024}. Nowadays, developments in computational pathology offer multiple solutions \cite{H._Qu_P._Wu_etal._2020,Javed_Mahmood_etal._2020} to utilize cell-level datasets to train machine learning models that solve these problems. The quality and specificity of training datasets are critical for robust and accurate models. Adhering to the principle of "garbage in, garbage out", it is essential to ensure that these datasets are extensively and accurately labeled with distinct classes that reflect the diverse biological characteristics of different cell types. Unfortunately, the number of open-source datasets comprising such high-quality annotations is limited. Existing cell segmentation datasets \cite{Gamper_Koohbanani_etal._2019,Graham_Vu_etal._2019,Verma_Kumar_etal._2021} may offer extensive annotations for certain cell types while providing more general labels for others. For example, in PanNuke, which is one of the largest open-source datasets comprising labeled cells, various types of morphologically and functionally different inflammatory cells like macrophages and lymphocytes are clustered in a broad "inflammatory" class. Consequently, these classes are frequently omitted from analyses or aggregated into broader meta-classes \cite{Gamper_Koohbanani_etal._2020} and likely interfere with other cell classes included in the dataset. This and similar inconsistencies in annotation granularity limit the ability of machine learning models to learn the comprehensive and nuanced features necessary for accurate cell segmentation and classification. To address these challenges, methods for refining and standardizing dataset annotations are essential to enhance the quality of training data.

A complementary approach to mitigate the absence of high-quality training data is the use of foundation models. Foundation models as encoders are defined as large-scale, versatile networks pre-trained on vast, diverse datasets using self-supervised learning, contrasting with convolutional neural network (CNN) pre-trained encoders that rely on supervised learning with labeled data. In practice, foundation models leverage enormous amounts of weakly or unlabeled data from millions of whole slide images (WSIs) and employ self-attention mechanisms to capture long-range dependencies and global context \cite{Chen_Ding_etal._2024,Saillard_Jenatton_etal._2024,Vorontsov_Bozkurt_etal._2024,Xu_Usuyama_etal._2024}. As a consequence, foundation models are able to produce transferable feature representations across different cell types and tissue environments. The feature representations can be leveraged by decoder networks to produce segmentation masks and pixel-level classifications. Because foundation models have comprehensive feature representations, they can be effectively fine-tuned using much smaller amounts of cell-level data compared to the large datasets needed to train models from scratch. Furthermore, foundation models incorporate adversarial training elements or contrastive learning \cite{Chen_Ding_etal._2024,Xu_Usuyama_etal._2024}, enhancing their resilience and adaptability by exposing them to challenging and varied scenarios during training. This may result in more generalizable models, often making them well-suited for diverse and complex tasks in digital pathology.

Despite the inherent advantages of foundation models, their deployment for practical use faces its own obstacles. In particular, they require substantial computational power, financial investments and rigorous testing to ensure reliability and efficacy for a given task \cite{Akkus_Dangott_etal._2022,Dragomir_Cocuz_etal._2022,Go_2022,Jafri_Farooqui_etal._2024}. Moreover, while foundation models enhance feature representation and performance, they depend on the quality of available annotations for decoder fine-tuning and, like any other model, cannot resolve existing inconsistencies or ambiguities in data labels. Therefore, there remains a critical need for solutions that address both data quality and practical deployment considerations.
Further, integrating new technologies into existing clinical workflows often encounters resistance, as it necessitates adjustments to established diagnostic processes. So, there is a need to develop solutions that could be integrated into current practices, minimizing the burden on medical professionals to adopt new tools \cite{King_Williams_etal._2023}.

Existing solutions \cite{Goldsborough_Philps_etal._2024,Hörst_Rempe_etal._2024}, while addressing some aspects of these challenges, fall short in providing a comprehensive approach. To address the data quality and clinical deployment issues, we propose a multi-faceted solution that encompasses data refinement, model optimization, and integration with existing pathology tools (\hyperref[fig:fig1]{Figure 1}). The outcome is a lightweight cell segmentation and classification model that can be integrated into digital pathology workflows for practical clinical use.

\begin{figure}[h!]
    \centering
    \includegraphics[width=\textwidth, height=0.82\textheight, keepaspectratio]{images/Figure_1.pdf}
    \caption{Overview of the proposed solution, including 1) Data refinement using cross-relabeling, 2) Teacher model development and fine tuning, 3) Student model optimization with knowledge distillation and 4) Student model and QuPath integration}
    \label{fig:fig1}
\end{figure}
\clearpage

Our approach begins with preparing the data for the fine-tuning and training of the machine learning models. We create a refined dataset, acquired via cross-relabeling two cell-level datasets, enhancing annotation specificity and consistency of the labeled data. Subsequently, we create a cell segmentation and classification model based on the foundation model. We leverage the foundation model as a fixed encoder and fine-tune a decoder using the refined dataset to improve generalization across diverse tissue- and cell types.
To ensure that the model remains lightweight and deployable in a possibly resource-constrained environment, we employ knowledge distillation to approximate the functionality of the foundation model. Finally, to facilitate the practical application of our model in digital pathology workflows, we integrate it with the QuPath \cite{Bankhead_Loughrey_etal._2017} application. Each methodological component contributes to the overarching goal of enhancing model performance, generalizability, and usability in clinical settings.

The primary contributions of this paper are:
\begin{enumerate}
    \item \textit{Data labels refinement through cross-relabeling:}
    
    We propose a new method for refining labels of cell-level datasets through cross-relabeling. This method employs classification models to re-label broad and ambiguous instances, resulting in a more diverse dataset. Our evaluation demonstrates that these classification models achieve high accuracy on test subsets, indicating the reliability of the method for label refinement.

    \item \textit{Enhanced model performance via foundation models:}
    
    We employ a foundation model as a feature extractor for the cell segmentation and classification task. In comparison with training a CNN model from scratch, the foundation model backbone only needs fine-tuning, which significantly reduces training time, computational resources and data requirements. We show that using a foundation model encoder leads to better performance in cell segmentation and classification networks than using a CNN-based encoder. This improvement may enable the model to generalize more effectively across various tissue types and imaging methods.
    
    \item \textit{Model optimization through knowledge distillation:}
    
    We show that a smaller student model trained using knowledge distillation on the refined dataset obtained via our cross-relabeling approach from a foundation model achieves comparable performance in cell segmentation and quantification tasks. As a result, this model is more suitable for deployment in environments without high-performance computing resources.
    
    \item \textit{Integration with QuPath:}
    
    We integrate the distilled cell segmentation and classification model into QuPath, a widely used open-source digital pathology platform, to accelerate clinical adaptation by enabling pathologists to more easily incorporate advanced computational tools into their existing workflows.
\end{enumerate}

Through these methodological steps, we aim to bridge the gap between advanced machine learning techniques and practical clinical applications, making accurate and efficient digital pathology accessible in a broader range of healthcare settings.

\section{Refining Existing Datasets Using Cross-Relabeling}
To address the limitations of sparse and ambiguous labeling of cell-level datasets, we propose a generalizable cross-relabeling strategy that can be applied to any dataset containing broadly categorized or imprecisely labeled cell types. This approach involves training and subsequently leveraging classification models to refine broad categories into more specific or biologically relevant classes.
When applied to cell-level data, the methodology includes extracting individual cell images from the dataset patches, preprocessing these images to standardize the size and accommodate partial cells, and then training deep learning classifiers capable of distinguishing between the finer cell subtypes within the coarser categories. 
To illustrate our approach, we focus on the PanNuke \cite{Gamper_Koohbanani_etal._2020, Gamper_Koohbanani_etal._2019} and MoNuSAC \cite{Verma_Kumar_etal._2021} datasets that we have used to train models for cell quantification in our previous works \cite{Shvetsov_Grønnesby_etal._2022,Shvetsov_Sildnes_etal._2024}. We find that for better cell differentiation we have to introduce more granular labels. PanNuke includes a broad classification of "inflammatory" cells, encompassing lymphocytes, macrophages, and neutrophils. Each cell type differs significantly in structure, function, and clinical relevance. Conversely, MoNuSAC uses the label "epithelial" for a class that comprises both benign epithelial cells and malignant neoplastic cells. This practice makes it challenging to differentiate between benign and malignant epithelial cells in the dataset, which is a critical distinction when identifying tumor areas within tissue samples. To address these issues, we implement a cross-relabeling strategy as shown in \hyperref[fig:fig2]{Figure 2}. The key components are two classification models: one is trained on singular cell images from PanNuke data to classify the epithelial meta-class into epithelial and neoplastic classes. The other is trained on MoNuSAC to refine the inflammatory class into lymphocytes, neutrophils, and macrophages.

\begin{figure}[h!]
    \centering
    \includegraphics[width=\textwidth]{images/Figure_2.pdf}
    \caption{Refined dataset generation via cross relabeling}
    \label{fig:fig2}
\end{figure}

The refining approach consists of three consecutive steps. The first is the preprocessing step, in which we extract individual cells from both datasets (\hyperref[fig:fig3]{Figure 3}). The specifics of PanNuke and MoNuSAC patch preparation before cell preprocessing are provided in \hyperref[chap:S1]{Appendix S1}.

\begin{figure}[h!]
    \centering
    \includegraphics[width=\textwidth]{images/Figure_3.pdf}
    \caption{Cell instances preprocessing including (1) cell map extraction, (2) bounding box delineation, (3) adjusting cell boxes and (4) cropping and resizing of cell images}
    \label{fig:fig3}
\end{figure}

During preprocessing, we extract cell type maps from the ground truth label mask and calculate bounding boxes around each cell instance. To accommodate partial cells at patch borders, a common issue in cropped patch images, we employ mirror padding and extend the field of view of the cell label by 15 pixels to capture adjacent cells. We then crop and resize the identified regions to $64 \times 64$ pixels using bicubic interpolation.

The preprocessed PanNuke dataset comprises 68,031 neoplastic and 23,207 epithelial cell images, while MoNuSAC comprises  33,104 lymphocytes, 1,252 neutrophils, and 1,695 macrophages, which we subsequently use in training cell classification models and classifying the cell image data \hyperref[fig:S2]{Appendix Figure S2 (1)}. 

The next step is to train two distinct ResNet50-based classifiers tailored to address the specific labeling challenges inherent in each dataset. We use ResNet50 for classification models due to its proven effectiveness for image classification tasks in histopathology \cite{pan2022reviewmachinelearningapproaches}, and its compatibility with small images. For the PanNuke dataset, we design the classifier, trained on MoNuSAC data, to disaggregate the heterogeneous "inflammatory" cell category into distinct subtypes: lymphocytes, macrophages, and neutrophils. Similarly, for the MoNuSAC dataset, the classifier is trained on PanNuke data and distinguishes between benign and malignant epithelial cells within the overarching "epithelial" label. By applying these targeted classifiers to their respective datasets, we assign more specific labels to individual cell instances, thus enabling us to create a unified dataset.
To ensure a balanced representation of classes, we train both models on datasets that had been equalized to match the size of the least represented class. Thus, we obtain datasets comprising 23,207 samples per class for PanNuke and 1,252 samples per class for MoNuSAC data. Next, we partition both of them into training (70\%), validation (20\%), and testing (10\%) subsets. To mitigate the risk of overfitting, we use a single dropout layer with a rate of p=0.5 in both models and data augmentation using randomized color perturbations, rotation, and horizontal and vertical flipping. We employ AdamW optimizer and the cross-entropy loss function for the training criterion.

To evaluate the two trained models, we measure the classification accuracy on the respective test subsets. The accuracies on the test subset for both classifiers are presented in \hyperref[tab:1]{Table 1}. The PanNuke model achieves an average accuracy of 93.57\%, with higher accuracy for neoplastic cells (96.06\%) compared to epithelial cells (86.26\%). The confusion matrix in Figure A3.1 shows that the model predominantly distinguishes accurately between epithelial and neoplastic tissues, with a substantial number of correct classifications and relatively few misclassifications. The MoNuSAC model demonstrates an average accuracy of 98.92\%, excelling in classifying lymphocytes (99.67\%) and macrophages (94.12\%), with lower performance for neutrophils (85.71\%). The confusion matrix in Figure A3.2 shows that the model identifies lymphocytes and performs reasonably well with macrophages and neutrophils.

\begin{table}[h!]
\renewcommand{\arraystretch}{1.5}
  \centering
  \caption{Cell classification results for PanNuke and MoNuSAC trained models (CI 95\%).}
  \label{tab:1}
  \begin{tabular}{|l|c|c|}
   \hline
   %\rowcolor{gray!30}
    Accuracy               & PanNuke model              & MoNuSAC model              \\
    \hline
    Average      & 0.936 (0.931--0.941)         & 0.989 (0.986--0.993)        \\
    \hline
    Neoplastic   & 0.961 (0.956--0.965)         & -                          \\
    \hline
    Epithelial   & 0.863 (0.849--0.877)         & -                          \\
    \hline
    Lymphocytes  & -                          & 0.997 (0.995--0.999)        \\
    \hline
    Neutrophils  & -                          & 0.857 (0.796--0.918)        \\
    \hline
    Macrophages  & -                          & 0.941 (0.906--0.976)        \\
    \hline
  \end{tabular}
\end{table}

Finally, during the last step, we use the model trained on PanNuke data for epithelial cells in MoNuSAC and the model trained on MoNuSAC for the inflammatory cells class in PanNuke. Specifically, we use classifier models to relabel epithelial cells in MoNuSAC and inflammatory cells in PanNuke data. Then we combine cells with refined labels and the rest of the cells in both datasets to create a refined dataset (\hyperref[fig:S2]{Appendix Figure S2 (2)}). The process of relabeling cells and visualizing them on a patch is shown in \hyperref[fig:fig4]{Figure 4}. The cell counts in the refined dataset are provided in \hyperref[tab:S4]{Appendix Table S4}.

\begin{figure}[h!]
    \centering
    \includegraphics[width=\textwidth, height=0.42\textheight, keepaspectratio]{images/Figure_4.pdf}
    \caption{Cell relabeling procedure for epithelial and inflammatory cell classes}
    \label{fig:fig4}
\end{figure}

%\hfill

Relabeling and combining datasets have been explored in a prior study \cite{Parulekar_Kanwat_etal._2023}, where consecutive fine-tuning on multiple datasets was employed to account for hierarchical class label structures. While the method presented in \cite{Parulekar_Kanwat_etal._2023} is intuitive, it often lacks consistency and requires multiple fine-tuning runs, which can be cumbersome and time-consuming. 
In contrast, cross-relabeling simplifies this process by using specialized classification models tailored to each dataset's specific labeling challenges. This approach provides better transparency and produces a unified dataset encompassing seven distinct cell types across multiple tissue samples, enhancing data diversity for further model training or fine-tuning.

Despite these improvements, cross-relabeling does not entirely resolve issues related to poor labeling quality or the amount of labeled data. Specifically, our results show lower accuracies persist for underrepresented classes, such as macrophages, which may stem from a limited sample availability and intrinsic challenges in distinguishing these cells based solely on H\&E staining. Furthermore, while our method enhances label specificity, it relies on the initial quality of the broad labels; thus, any fundamental inaccuracies in the original annotations can propagate through the relabeling process. Addressing the overall problem of limited data labels may require integrating additional data sources or utilizing complementary immunohistochemical staining methods.
Although the reported performance metrics are obtained from evaluations on the native test sets of each dataset, it is important to note that the primary application of these classifiers is to perform cross-relabeling, where a model trained on one dataset (e.g., PanNuke) is applied to another (e.g., MoNuSAC) and vice versa. We acknowledge that a more systematic evaluation of cross-dataset generalization is needed and could be performed in future work.

Overall, the refined dataset produced by our approach can enhance the supervised training or fine-tuning of cell segmentation and classification models, especially those that utilize pre-trained foundation models to improve feature extraction robustness. In addition, these models can detect nuanced classes that enable researchers to conduct more detailed analyses of biological processes in computational pathology.

\section{Foundation models for robust cell segmentation and classification}

Accurate cell segmentation and classification in digital pathology are hindered by limited labeled data and the fact that conventional CNNs are unable to capture global contextual information due to their local receptive field constraints \cite{Gheflati_Rivaz_2022,Yang_Marcus_etal.}. Traditional approaches in cell quantification have predominantly relied on CNN encoders, such as ResNet50, given their proven effectiveness in semantic segmentation tasks \cite{Deshmane_2023,Graham_Vu_etal._2019,Mukasheva_Koishiyeva_etal._2024,Stringer_Wang_etal._2021}. However, approaches that include fine-tuning of pretrained CNNs, data augmentation, and stain normalization to partially increase data variability and address staining differences often fail to achieve the necessary generalization and robustness across diverse tissue types and staining conditions \cite{G._Wang_W._Li_etal._2018,Gao_Bagci_etal._2018,Karim_El_Khoury_Martin_Fockedey_etal._2021}.

To overcome these challenges, we leverage an encoder-decoder network that uses a foundation model as the encoder and a CNN upsampling decoder (\hyperref[fig:fig5]{Figure 5}) for simultaneous cell segmentation and classification in 2D patches extracted from WSIs. Foundation models with transformer-based architectures are viable alternatives to CNN-based encoders \cite{Shamshad_Khan_etal._2023,Sourget_2023}. They enable the creation of more advanced architectures that can decode or transform learned features more effectively \cite{Chen_Duan_etal._2023,Cheng_Misra_etal._2022,Xie_Wang_etal._2021}.

\begin{figure}[h!]
    \centering
    \includegraphics[width=\textwidth]{images/Figure_5.pdf}
    \caption{UNETR-like model with foundational model as backbone}
    \label{fig:fig5}
\end{figure}

By utilizing a transformer-based encoder, we incorporate global contextual information into the feature extraction process, which is a key advantage of such architectures \cite{Chen_Lu_etal._2021}. This foundation model integration facilitates accurate pixel-wise segmentation and classification without the need for extensive encoder training, thereby potentially improving generalization across varied cellular structures and tissue types.
In our implementation, we employ a modified UNETR \cite{Hatamizadeh_Tang_etal._2021} architecture that combines a vision transformer (ViT) \cite{Dosovitskiy_Beyer_etal._2021} encoder with a CNN-based decoder. The encoder utilizes the pretrained H-Optimus foundation model, which contains 1.1 billion parameters and is trained on over 500,000 H\&E stained WSIs \cite{Saillard_Jenatton_etal._2024}. We extract outputs from four evenly spaced transformer blocks $Z_i$, where $i \in [1, 14, 26, 38]$, to serve as residual connections for the CNN decoder. We select these blocks based on our observation that features from non-adjacent levels of the encoder lead to better overall performance on the test subset.

The CNN decoder upsamples the feature representations, acquired from the transformer blocks, to generate an intermediate vector that is handled by two task-specific layers that generate cell segmentation and classification masks. The first task-specific layer is the ‘Cellpose head’,  which is used to delineate cell instances. The layer generates horizontal and vertical gradient maps to form vector fields that are refined through gradient tracking in a post-processing step using the Cellpose algorithm \cite{Stringer_Wang_etal._2021}, known for its efficacy in cell segmentation tasks and generalizability across multiple domains \cite{Pachitariu_Stringer_2022,Stringer_Pachitariu_2024}. The second task-specific layer is the "Cell type head", which assigns labels to individual pixels. In the post-processing step, we determine the output classification label of each segmented cell instance by majority voting over the labeled pixels that comprise the cell in the segmentation map.

To evaluate model performance and measure the impact of adding a foundation model as backbone, we compare it to a ResNet50-based model. ResNet50 is a widely used solution for encoders in segmentation architectures in the medical domain \cite{Deshmane_2023,Graham_Vu_etal._2019,Mukasheva_Koishiyeva_etal._2024,Stringer_Wang_etal._2021}. For the H-Optimus-based model, we utilize frozen weights for the encoder and only fine-tune the decoder to take advantage of the extensive pre-training of the foundation model. For the ResNet50-based model we start with ImageNet \cite{Deng_Dong_etal.} weights and train both encoder and decoder parts. Hyperparameters for the training step are set to be identical, where possible, for comparable evaluation. 
For this evaluation, we deliberately use the PanNuke dataset to provide a standardized and controlled comparison between the H‑Optimus and ResNet50-based models (\hyperref[fig:S2]{Appendix Figure S2 (3)}). Specifically, we use two of the default PanNuke dataset splits (66\%) for training and validation, and reserve the third split (33\%) for testing.

To address the challenge of cell class imbalance in the PanNuke dataset, which is a common characteristic in most cell-level H\&E patch datasets, both models’ training processes employ a weighted loss function comprising cross-entropy and focal loss \cite{Lin_Goyal_etal._2018}. The focal loss component is adjusted with coefficients derived from each cell class' instance frequency, emphasizing learning from underrepresented classes and enhancing the model's sensitivity to rare but significant cellular patterns. The cross-entropy loss is augmented with spectral decoupling regularization \cite{Pezeshki_Kaba_etal._2021,Pohjonen_Stürenberg_etal._2022} and spatially varying label smoothing \cite{Islam_Glocker_2021}, which potentially stabilizes training and improves generalization in case of complex tissue morphologies. For optimization, we employ the \textit{AdamW} \cite{Loshchilov_Hutter_2019} to counter unbalanced class scenarios, with cosine annealing learning rate scheduler.

We utilize the scikit-learn library \cite{Van_der_Walt_Schönberger_etal._2014} and HoVer-Net \cite{Graham_Vu_etal._2019} implementations of $R^2$ (the coefficient of determination) and $PQ$ (panoptic quality) to evaluate our experiments. Complete mathematical formulations and detailed explanations of these metrics are provided in \hyperref[chap:S5]{Appendix S5}. To compute confidence intervals, we use nonparametric bootstrapping, where after calculating the metric on the full sample, we generated 1000 bootstrap replicates by resampling with replacement and then determined the 95\% confidence intervals as the 2.5th and 97.5th percentiles of the resulting empirical distribution.

%\hfill

The model comparisons are summarized in \hyperref[tab:2]{Table 2}. The H‑Optimus-based model achieves higher $R^2$ across all cell classes compared to the ResNet50-based model, which means that its predictions are more closely aligned with the PanNuke cell counts, indicating a stronger correlation with the observed data. Notably, the improvement of $R^2_{dead}$ may be an indicator of better global contextual representations provided by the foundation model backbone. In terms of segmentation and classification quality combined, measured by the PQ score, the H‑Optimus-based model demonstrates notable improvements across most cell classes. Overall, the average $R^2$ improved from 0.575 to 0.871, while the average $PQ$ score improved from 0.450 to 0.492, demonstrating better performance of the H-Optimus-based model.

\begin{table}[h!]
\renewcommand{\arraystretch}{1.5}
  \centering
  \caption{Cell quantification metrics for baseline and proposed models (CI 95\%).}
  \label{tab:2}
  \begin{tabular}{|l|c|c|}
    \hline
    %\rowcolor{gray!30}
    Metric             & Resnet50-based            & H-optimus-based              \\
    \hline
    $R^2_{neoplastic}$    & 0.681 (0.576--0.769)       & \textbf{0.941 (0.917--0.960)} \\
    \hline
    $R^2_{inflammatory}$  & 0.863 (0.778--0.903)       & \textbf{0.949 (0.918--0.966)} \\
    \hline
    $R^2_{connective}$    & 0.600 (0.488--0.698)       & 0.609 (0.436--0.772)          \\
    \hline
    $R^2_{dead}$          & 0.097 (-11.389--0.669)     & 0.925 (0.404--0.982)          \\
    \hline
    $R^2_{epithelial}$    & 0.635 (0.490--0.747)       & \textbf{0.930 (0.886--0.964)} \\
    \hline
    $PQ_{neoplastic}$       & 0.517 (0.499--0.535)       & \textbf{0.589 (0.575--0.604)} \\
    \hline
    $PQ_{inflammatory}$     & 0.455 (0.429--0.482)       & \textbf{0.528 (0.507--0.549)} \\
    \hline
    $PQ_{connective}$       & 0.416 (0.400--0.431)       & \textbf{0.451 (0.436--0.465)} \\
    \hline
    $PQ_{dead}$             & 0.374 (0.342--0.408)       & 0.292 (0.209--0.365)          \\
    \hline
    $PQ_{epithelial}$       & 0.488 (0.460--0.519)       & \textbf{0.599 (0.579--0.618)} \\
    \hline
  \end{tabular}
\end{table}

Our results  show that integrating the H‑Optimus foundation model within the UNETR architecture enhances the model's ability to segment and classify cells across diverse tissues from PanNuke data. The pretrained transformer encoder provides robust feature representations, resulting in higher average $R^2$ and $PQ$ scores compared to the CNN-based model. This leads to more reliable cell quantification and more accurate downstream analysis. Additionally, the streamlined fine-tuning process reduces computational overhead and training time, making the model more adaptable for new data.

Despite these advancements, the foundation model-based approach does not fully resolve all challenges related to cell segmentation and classification. We observe lower metric scores for underrepresented classes in the training data. Furthermore, foundation models typically encompass billions of parameters, resulting in substantial computational and memory requirements. It therefore poses challenges for deployment in resource-constrained environments, limiting their practical applicability in certain clinical settings.

\section{Model optimization via Knowledge Distillation}

To address the limitations posed by the extensive size of foundation models, we implement knowledge distillation — a model compression technique that leverages the teacher-student paradigm \cite{Hinton_Vinyals_etal._2015}. By training a smaller, more efficient student model to replicate the output of a larger, pre-trained teacher model, we retain performance while significantly reducing the model's complexity and resource requirements (\hyperref[fig:fig6]{Figure 6}).

\begin{figure}[h!]
    \centering
    \includegraphics[width=\textwidth, height=0.45\textheight, keepaspectratio]{images/Figure_6.pdf}
    \caption{Knowledge distillation framework for training a student model using a pre-trained teacher}
    \label{fig:fig6}
\end{figure}

We employ knowledge distillation to compress the H‑Optimus-based teacher model into a more efficient student model. The teacher model is the modified UNETR architecture with the H‑Optimus foundation model described in the previous chapter. The student model is based on a UNet architecture augmented with residual connections and incorporates a smaller ViT encoder with 9 million parameters \cite{Steiner_Kolesnikov_etal._2022,Wightman_2019}. 

First, we fine-tune the teacher model using the refined dataset from the cross-relabeling procedure (Section 2). Initially we train the decoder of the teacher model while keeping the encoder weights frozen. We split the refined dataset into train (70\%), validation (20\%) and test (10\%) subsets (\hyperref[fig:S2]{Appendix Figure S2 (4)}). During fine-tuning, we use the train and validation subsets, while leaving the test subset for model evaluation. We set the training procedure and model hyperparameters to be identical to those that were used to demonstrate the utility of foundation models for the simultaneous cell segmentation and classification task.

Next, we perform knowledge distillation from teacher to student using the refined dataset used to fine-tune the teacher model. The student model is trained to replicate the teacher model's outputs. We utilize a specialized loss function that aligns the student's predicted probability distribution with the teacher's, incorporating the teacher's class probability distribution derived from the output. Following the methodology of Hinton et al. \cite{Hinton_Vinyals_etal._2015}, we experiment with various hyperparameter settings for the temperature ($T$) and the balancing coefficients ($\alpha$ and $\beta$) in the loss function. We vary $T$ from 1 to 20 and adjust $\alpha$ and $\beta$ to balance the distillation and student losses. Through iterative tuning and evaluation, we identify that setting $T=14$, $\alpha=0.3$, and $\beta=0.7$ yields a configuration that converges and closely approximates the teacher model's performance during training.

Finally, we assess the performance of both models using the $R^2$ and $PQ$ (defined in \hyperref[chap:S5]{Appendix S5}) on the test set of the refined dataset (\hyperref[tab:3]{Table 3}). We observe that the 95\% confidence intervals overlap for most cell types, so we cannot claim statistically significant performance differences between the teacher and student models. One exception appears in the neoplastic class. The teacher model produces an $R^2$ of 0.919, while the student model shows an $R^2$ of 0.852. In addition, the student model achieves higher $PQ$ values for the neoplastic and connective classes, though the confidence intervals show overlap.

\begin{table}[h!]
\renewcommand{\arraystretch}{1.5}
  \centering
  \caption{Cell quantification metrics for teacher and distilled student models (CI 95\%).}
  \label{tab:3}
  \begin{tabular}{|l|c|c|}
    \hline
    %\rowcolor{gray!30}
    Metric & Teacher & Student \\
    \hline
    $R^2_{neoplastic}$    & \textbf{0.919} (0.898--0.939) & 0.852 (0.800--0.891) \\
    \hline
    $R^2_{lymphocyte}$    & 0.969 (0.956--0.977)         & 0.969 (0.956--0.978) \\
    \hline
    $R^2_{connective}$    & 0.694 (0.548--0.809)         & 0.618 (0.469--0.741) \\
    \hline
    $R^2_{dead}$          & 0.755 (0.400--0.908)         & 0.424 (0.100--0.731) \\
    \hline
    $R^2_{epithelial}$    & 0.922 (0.870--0.958)         & 0.843 (0.738--0.917) \\
    \hline
    $R^2_{macrophage}$    & 0.384 (-0.369--0.724)        & 0.704 (0.352--0.859) \\
    \hline
    $R^2_{neutrofil}$     & 0.854 (0.578--0.929)         & 0.833 (0.502--0.925) \\
    \hline
    $PQ_{neoplastic}$       & 0.581 (0.569--0.593)         & 0.601 (0.588--0.613) \\
    \hline
    $PQ_{lymphocyte}$       & 0.536 (0.520--0.553)         & 0.563 (0.544--0.579) \\
    \hline
    $PQ_{connective}$       & 0.436 (0.421--0.451)         & 0.457 (0.441--0.474) \\
    \hline
    $PQ_{dead}$             & 0.272 (0.235--0.315)         & 0.279 (0.201--0.369) \\
    \hline
    $PQ_{epithelial}$       & 0.522 (0.500--0.545)         & 0.530 (0.506--0.555) \\
    \hline
    $PQ_{macrophage}$       & 0.524 (0.459--0.588)         & 0.474 (0.405--0.543) \\
    \hline
    $PQ_{neutrofil}$        & 0.541 (0.490--0.592)         & 0.565 (0.522--0.607) \\
    \hline
  \end{tabular}
\end{table}


We further decompose the $PQ$ metric into its $SQ$ and $DQ$ components (\hyperref[tab:S6]{Appendix Table S6}). Both models produce nearly identical $SQ$ values, which indicates that they predict instance boundaries with similar precision. Although the student model shows some improvement in $DQ$ scores for certain classes, the confidence intervals overlap and do not confirm a statistically significant difference.

We observe that the student and teacher models yield comparable detection performance despite the student model using a much smaller and simpler architecture. A model with fewer parameters reduces the risk of overfitting when training data are scarce relative to the model’s complexity \cite{Farias_Ludermir_etal._2022}. The knowledge distillation process also encourages the student model to focus on the most generalizable detection features learned from the teacher. These factors enable the student model to achieve similar detection performance across different cell types.

Additionally, considering the model sizes reported in \hyperref[tab:4]{Table 4}, the distilled model achieves a significant reduction compared to the teacher model, with a 48-fold decrease in parameter count and a 5.5-fold reduction in on-disk size. In inference mode, the teacher model requires 16 GB of VRAM for a batch size of 32, while the distilled model only needs 3 GB of VRAM for the same batch size. These reductions make the distilled model significantly more practical for fine-tuning and deployment in resource-constrained environments.

\begin{table}[h!]
\renewcommand{\arraystretch}{1.5}
  \centering
  \caption{Parameter counts and size of teacher and distilled model}
  \label{tab:4}
  \adjustbox{max width=\textwidth}{%
  \begin{tabular}{|l|c|c|c|}
    \hline
    %\rowcolor{gray!30}
    Metric & H-optimus-based (Teacher) & mobileViT-based (Student) & Magnitude of difference \\
    \hline
    Parameters count       & 1,158,917,906   & \textbf{24,093,393}   & \textbf{48x}  \\
    \hline
    Estimated Total Size (MB) & 87,912       & \textbf{15,935}    & \textbf{5.5x} \\
    \hline
  \end{tabular}%
}
\end{table}

%\hfill

With recent advancements in complex network architectures and the use of pretrained encoders to achieve state-of-the-art performance \cite{Baumann_Dislich_etal._2024,Hörst_Rempe_etal._2024} in cell segmentation and classification tasks, model size, computational complexity, and processing times have increased. This limits the scalability and accessibility of these models. As we demonstrate, this may be mitigated using knowledge distillation. Studies in the field of natural language processing have demonstrated the efficacy of knowledge distillation in retaining the capabilities of the teacher model while achieving significant reductions in size and complexity \cite{Huangpu_Gao_2024,Sun_Yu_etal.}. 

We demonstrate the feasibility of knowledge distillation in digital pathology, specifically for cell segmentation and classification tasks. Moreover, we achieve this performance while also significantly reducing the parameter count. In addressing the challenge of knowledge transfer, we found that distillation from a transformer-based model to a smaller transformer is more straightforward than attempting to map transformer features to CNN blocks. In our experiments, using a CNN-based network as a student results in worse cell quantification performance due to the structural constraints of CNN feature space dimensions. 

Although our primary approach relies on a transformer-based student model that performs well, it can be further optimized to incorporate advantages from CNN architectures. For example, employing alternative techniques such as using ViT adapters \cite{Chen_Duan_etal._2023} or $1 \times 1$ convolutions to adjust feature map sizes may be beneficial for harnessing CNN advantages like enhanced local feature extraction. Moreover, if additional performance improvements are desired, the process can be further enhanced by applying supplementary knowledge distillation techniques, such as self-distillation \cite{Zhang_Song_etal._2019} or online distillation \cite{Houyon_Cioppa_etal._2023}.

Despite these promising results, further validation on independent datasets is necessary to fully understand the model's limitations. Underrepresented classes may pose challenges when addressing complex cases. Pathologists need to validate these models to adopt them in clinical settings. While the distilled models are smaller and more deployable, a technological gap persists because pathologists traditionally rely on established methods for inspecting WSIs and diagnosing diseases. Addressing the complexities involved in deploying models for inference and supporting pathologists in adopting new tools is essential for integrating these models into clinical workflows.

\section{Model integration with QuPath}
Digital pathology tools with graphical user interfaces are essential for visualizing and analyzing WSIs. To make our student model useful in clinical pathology workflows, it needs to be integrated into a tool that enables inspecting regions, creating annotations, and providing quantitative analyses of biomarkers. Therefore, we integrate the trained student model from the previous chapter into the QuPath open‑source platform \cite{Bankhead_Loughrey_etal._2017}. QuPath provides the required annotation, visualization, and analysis tools to interpret complex histological data, including workflows for cell segmentation, classification, and quantification (\hyperref[fig:fig7]{Figure 7}). 

\begin{figure}[h!]
    \centering
    \includegraphics[width=\textwidth]{images/Figure_7.pdf}
    \caption{Visualization of model-generated cell quantification annotations (left) and the corresponding unannotated slide (right) in QuPath}
    \label{fig:fig7}
\end{figure}

To identify the regions in a WSI critical for prognosticating tumor development, such as specific tumor areas or border regions without overlapping healthy tissue, the pathologist uses QuPath to outline these regions. Then, the pathologist initiates a cell segmentation and classification script through the QuPath interface for the selected regions. The resulting annotations and quantified cell information are then directly overlaid onto the WSI in the QuPath interface. Additional design and implementation details are in \hyperref[chap:S7]{Appendix S7}. 

Two common approaches for integrating deep learning models into QuPath are Java‑based native QuPath extensions \cite{Goldsborough_Philps_etal._2024} and the execution of RESTful API requests to a model server coupled with handling the response via an extension, as demonstrated in the application of cell segmentation models applied to immunofluorescence images \cite{Sugawara_2023}. While the community is actively working on these integration strategies, there is currently no universal solution that fully addresses all integration and performance requirements.

Extensions may offer better integration with QuPath, allowing slightly improved performance and more widespread usage of the built-in QuPath models, but they lack the flexibility to customize models and modify their behavior. For example, the newest version of QuPath includes models such as StarDist \cite{Weigert_Schmidt} and InstanSeg \cite{Goldsborough_Philps_etal._2024} that can perform cell segmentation. Both models pose limitations when applied to simultaneous cell segmentation and classification. StarDist performs well only on convex, round shapes by design, whereas some neoplastic, inflammatory, and connective cells exhibit complex and non-convex shapes. InstanSeg provides only semantic segmentation without assigning classes to the segmented cells.

%\hfill

In contrast, our approach offers an alternative integration strategy. It utilizes the paquo library to directly interact with QuPath’s internal application programming interface from within Python. This enables data exchange and processing without the need for intermediate conversion steps and provides greater control over model customization, retraining, and the incorporation of custom processing steps.

The integration of our custom model with QuPath underscores its potential to significantly enhance the diagnostic process by reducing the time burden on pathologists and enabling them to focus on more complex interpretative tasks using familiar software. Leveraging a tool that is already well-established among pathologists increases the likelihood of its adoption into daily clinical workflows. The quantitative data generated through the automated workflow is critical for both clinical decision-making and research, facilitating more accurate biomarker analysis, enabling robust statistical evaluations, and supporting hypothesis generation and testing. Additionally, by streamlining cell segmentation and classification, the tool enhances the scalability and reproducibility of pathological assessments, ultimately contributing to improved diagnostic accuracy and patient outcomes.

\section{Conclusion and future work}

In this study, we address critical challenges in digital pathology and tackle the usability and deployment issues of the developed models in standard computing environments without the need for high-performance computing systems. Our multi-faceted approach encompasses data refinement through cross-relabeling, leveraging foundation models for robust cell segmentation and classification, optimizing model performance via knowledge distillation, and integrating the optimized model into the QuPath software for practical application. This approach is used to construct a capable, versatile, and adjustable model for cell segmentation and classification, with enhanced performance and usability.

\begin{sloppypar}
While our approach shows potential in the field of computational pathology, certain limitations persist. 
For example, our implementation currently exhibits lower performance in detecting macrophages. 
This serves as an instance of the broader challenge of accurately identifying complex cell types. In order to address this issue, extending our approach to incorporate additional data sources, exploring alternative modeling approaches, and integrating other imaging modalities such as immunohistochemical staining may help improve detection accuracy. Moreover, although the distilled model reduces computational demands, integrating advanced deep learning models into clinical practice requires addressing technological gaps and potential resistance to adopting new tools within established diagnostic processes.
\end{sloppypar}

Future work could focus on several key areas to refine the proposed approach and facilitate its adoption in clinical environments. Enhancing the cell-relabeling process with additional datasets \cite{Graham_Jahanifar_etal._2021} could improve the representation of underrepresented cell types and enhance overall model performance. Also, incorporating additional data sources, such as multi-modal imaging or complementary staining methods, may address limitations related to cell type differentiation and class imbalance. Exploring other foundation models \cite{Vorontsov_Bozkurt_etal._2024,Zimmermann_Vorontsov_etal._2024} or introducing additional modalities \cite{Ding_Wagner_etal._2024,Vaidya_Zhang_etal._2025} may provide alternative architectures better suited to specific tasks or offer improved efficiency. Implementing more complex knowledge distillation techniques \cite{Houyon_Cioppa_etal._2023,Zhang_Song_etal._2019} could further optimize the model's performance and adaptability. Additionally, deeper integration with QuPath or other digital pathology software could provide pathologists more control over cell quantification analysis directly within the QuPath interface, thereby increasing accessibility and usability. Such enhancements would not only refine model performance but also ensure greater adaptability and scalability within various clinical environments. Finally, extensive validation of the model by pathologists and benchmarking against independent datasets are essential steps toward establishing the model's reliability and fostering confidence in its clinical utility.

\section*{Acknowledgments} 
This work was funded in part by the Research Council of Norway grant no. 309439 SFI Visual Intelligence, and the North Norwegian Health Authority grant no. HNF1521-20.

\bibliographystyle{IEEEtran}
\begin{sloppypar}
\begin{thebibliography}{99}

\bibitem{chaplot2020neural} Chaplot, Devendra Singh, et al. "Neural topological slam for visual navigation." Proceedings of the IEEE/CVF conference on computer vision and pattern recognition. 2020.

\bibitem{maksymets2021thda} Maksymets, Oleksandr, et al. "Thda: Treasure hunt data augmentation for semantic navigation." Proceedings of the IEEE/CVF International Conference on Computer Vision. 2021.

\bibitem{mezghan2022memory} Mezghan, Lina, et al. "Memory-augmented reinforcement learning for image-goal navigation." 2022 IEEE/RSJ International Conference on Intelligent Robots and Systems (IROS). IEEE, 2022.

\bibitem{al2022zero} Al-Halah, Ziad, Santhosh Kumar Ramakrishnan, and Kristen Grauman. "Zero experience required: Plug \& play modular transfer learning for semantic visual navigation." Proceedings of the IEEE/CVF Conference on Computer Vision and Pattern Recognition. 2022.

\bibitem{ye2021auxiliary} Ye, Joel, et al. "Auxiliary tasks and exploration enable objectgoal navigation." Proceedings of the IEEE/CVF international conference on computer vision. 2021.

\bibitem{chaplot2020object} Chaplot, Devendra Singh, et al. "Object goal navigation using goal-oriented semantic exploration." Advances in Neural Information Processing Systems 33 (2020)

\bibitem{ramakrishnan2022poni} Ramakrishnan, Santhosh Kumar, et al. "Poni: Potential functions for objectgoal navigation with interaction-free learning." Proceedings of the IEEE/CVF Conference on Computer Vision and Pattern Recognition. 2022.

\bibitem{ramrakhya2022habitat} Ramrakhya, Ram, et al. "Habitat-web: Learning embodied object-search strategies from human demonstrations at scale." Proceedings of the IEEE/CVF Conference on Computer Vision and Pattern Recognition. 2022.

\bibitem{mousavian2019visual} Mousavian, Arsalan, et al. "Visual representations for semantic target driven navigation." 2019 International Conference on Robotics and Automation (ICRA). IEEE, 2019.

\bibitem{dhariwal2021diffusion} Dhariwal, Prafulla, and Alexander Nichol. "Diffusion models beat gans on image synthesis." Advances in neural information processing systems 34 (2021)

\bibitem{ho2022classifier} Ho, Jonathan, and Tim Salimans. "Classifier-free diffusion guidance." arXiv preprint arXiv:2207.12598 (2022).

\bibitem{nichol2021glide} Nichol, Alex, et al. "Glide: Towards photorealistic image generation and editing with text-guided diffusion models." arXiv preprint arXiv:2112.10741 (2021)

\bibitem{brooks2023instructpix2pix} Brooks, Tim, Aleksander Holynski, and Alexei A. Efros. "Instructpix2pix: Learning to follow image editing instructions." Proceedings of the IEEE/CVF Conference on Computer Vision and Pattern Recognition. 2023.

\bibitem{fu2023guiding} Fu, Tsu-Jui, et al. "Guiding instruction-based image editing via multimodal large language models." arXiv preprint arXiv:2309.17102 (2023).

\bibitem{geng2024instructdiffusion} Geng, Zigang, et al. "Instructdiffusion: A generalist modeling interface for vision tasks." Proceedings of the IEEE/CVF Conference on Computer Vision and Pattern Recognition. 2024.

\bibitem{zhou2024minedreamer} Zhou, Enshen, et al. "Minedreamer: Learning to follow instructions via chain-of-imagination for simulated-world control." arXiv preprint arXiv:2403.12037 (2024).

\bibitem{zhou2023esc} Zhou, Kaiwen, et al. "Esc: Exploration with soft commonsense constraints for zero-shot object navigation." International Conference on Machine Learning. PMLR, 2023.

\bibitem{yu2023l3mvn} Yu, Bangguo, Hamidreza Kasaei, and Ming Cao. "L3mvn: Leveraging large language models for visual target navigation." 2023 IEEE/RSJ International Conference on Intelligent Robots and Systems (IROS). IEEE, 2023.

\bibitem{gadre2023cows} Gadre, Samir Yitzhak, et al. "Cows on pasture: Baselines and benchmarks for language-driven zero-shot object navigation." Proceedings of the IEEE/CVF Conference on Computer Vision and Pattern Recognition. 2023.

\bibitem{shah2023navigation} Shah, Dhruv, et al. "Navigation with large language models: Semantic guesswork as a heuristic for planning." Conference on Robot Learning. PMLR, 2023.

\bibitem{cai2024bridging} Cai, Wenzhe, et al. "Bridging zero-shot object navigation and foundation models through pixel-guided navigation skill." 2024 IEEE International Conference on Robotics and Automation (ICRA). IEEE, 2024.

\bibitem{yu2023co} Yu, Bangguo, Hamidreza Kasaei, and Ming Cao. "Co-NavGPT: Multi-robot cooperative visual semantic navigation using large language models." arXiv preprint arXiv:2310.07937 (2023).

\bibitem{wu2024voronav} Wu, Pengying, et al. "Voronav: Voronoi-based zero-shot object navigation with large language model." arXiv preprint arXiv:2401.02695 (2024).

\bibitem{qin2023mp5} Qin, Yiran, et al. "Mp5: A multi-modal open-ended embodied system in minecraft via active perception." arXiv preprint arXiv:2312.07472 (2023).

\bibitem{du2024learning} Du, Yilun, et al. "Learning universal policies via text-guided video generation." Advances in Neural Information Processing Systems 36 (2024).

\bibitem{ajay2024compositional} Ajay, Anurag, et al. "Compositional foundation models for hierarchical planning." Advances in Neural Information Processing Systems 36 (2024).

\bibitem{liang2024skilldiffuser} Liang, Zhixuan, et al. "Skilldiffuser: Interpretable hierarchical planning via skill abstractions in diffusion-based task execution." Proceedings of the IEEE/CVF Conference on Computer Vision and Pattern Recognition. 2024.

\bibitem{heusel2017gans} Heusel, Martin, et al. "Gans trained by a two time-scale update rule converge to a local nash equilibrium." Advances in neural information processing systems 30 (2017).

\bibitem{zhang2018unreasonable} Zhang, Richard, et al. "The unreasonable effectiveness of deep features as a perceptual metric." Proceedings of the IEEE conference on computer vision and pattern recognition. 2018.

\bibitem{brown2020language} Brown, Tom B. "Language models are few-shot learners." arXiv preprint arXiv:2005.14165 (2020).

\bibitem{podell2023sdxl} Podell, Dustin, et al. "Sdxl: Improving latent diffusion models for high-resolution image synthesis." arXiv preprint arXiv:2307.01952 (2023).

\bibitem{brohan2022rt} Brohan, Anthony, et al. "Rt-1: Robotics transformer for real-world control at scale." arXiv preprint arXiv:2212.06817 (2022).

\bibitem{brohan2023rt} Brohan, Anthony, et al. "Rt-2: Vision-language-action models transfer web knowledge to robotic control." arXiv preprint arXiv:2307.15818 (2023).

\bibitem{li2024manipllm} Li, Xiaoqi, et al. "Manipllm: Embodied multimodal large language model for object-centric robotic manipulation." Proceedings of the IEEE/CVF Conference on Computer Vision and Pattern Recognition. 2024.

\bibitem{shah2023vint} Shah, Dhruv, et al. "ViNT: A foundation model for visual navigation." arXiv preprint arXiv:2306.14846 (2023).

\bibitem{liu2024visual} Liu, Haotian, et al. "Visual instruction tuning." Advances in neural information processing systems 36 (2024).

\bibitem{hu2021lora} Hu, Edward J., et al. "Lora: Low-rank adaptation of large language models." arXiv preprint arXiv:2106.09685 (2021).

\bibitem{qin2023supfusion} Qin, Yiran, et al. "SupFusion: Supervised LiDAR-camera fusion for 3D object detection." Proceedings of the IEEE/CVF International Conference on Computer Vision. 2023.

\bibitem{qin2024worldsimbench} Qin, Yiran, et al. "Worldsimbench: Towards video generation models as world simulators." arXiv preprint arXiv:2410.18072 (2024).

\bibitem{yu2025gamefactory} Yu, Jiwen, et al. "GameFactory: Creating New Games with Generative Interactive Videos." arXiv preprint arXiv:2501.08325 (2025).

\bibitem{zhou2024code} Zhou, Enshen, et al. "Code-as-Monitor: Constraint-aware Visual Programming for Reactive and Proactive Robotic Failure Detection." arXiv preprint arXiv:2412.04455 (2024).

\bibitem{zhang2024ad} Zhang, Zaibin, et al. "AD-H: Autonomous Driving with Hierarchical Agents." arXiv preprint arXiv:2406.03474 (2024).

\bibitem{wang2024toward} Wang, Chaoqun, et al. "Toward Accurate Camera-based 3D Object Detection via Cascade Depth Estimation and Calibration." arXiv preprint arXiv:2402.04883 (2024).

\bibitem{huang2024story3d} Huang, Yuzhou, et al. "Story3d-agent: Exploring 3d storytelling visualization with large language models." arXiv preprint arXiv:2408.11801 (2024).

\bibitem{savinov2018semi} Savinov, Nikolay, Alexey Dosovitskiy, and Vladlen Koltun. "Semi-parametric topological memory for navigation." arXiv preprint arXiv:1803.00653 (2018).

\bibitem{majumdar2022zson} Majumdar, Arjun, et al. "Zson: Zero-shot object-goal navigation using multimodal goal embeddings." Advances in Neural Information Processing Systems 35 (2022): 32340-32352.

\bibitem{yadav2023offline} Yadav, Karmesh, et al. "Offline visual representation learning for embodied navigation." Workshop on Reincarnating Reinforcement Learning at ICLR 2023. 2023.

\bibitem{yadav2023ovrl} Yadav, Karmesh, et al. "Ovrl-v2: A simple state-of-art baseline for imagenav and objectnav." arXiv preprint arXiv:2303.07798 (2023).

\bibitem{sun2024fgprompt} Sun, Xinyu, et al. "FGPrompt: fine-grained goal prompting for image-goal navigation." Advances in Neural Information Processing Systems 36 (2024).

\bibitem{zhu2017target} Zhu, Yuke, et al. "Target-driven visual navigation in indoor scenes using deep reinforcement learning." 2017 IEEE international conference on robotics and automation (ICRA). IEEE, 2017.

\bibitem{koh2024generating} Koh, Jing Yu, Daniel Fried, and Russ R. Salakhutdinov. "Generating images with multimodal language models." Advances in Neural Information Processing Systems 36 (2024).

\bibitem{krantz2022instance} Krantz, Jacob, et al. "Instance-specific image goal navigation: Training embodied agents to find object instances." arXiv preprint arXiv:2211.15876 (2022).

\bibitem{schulman2017proximal} Schulman, John, et al. "Proximal policy optimization algorithms." arXiv preprint arXiv:1707.06347 (2017).

\bibitem{anderson2018evaluation} Anderson, Peter, et al. "On evaluation of embodied navigation agents." arXiv preprint arXiv:1807.06757 (2018).

\bibitem{lin2024navcot} Lin, Bingqian, et al. "NavCoT: Boosting LLM-Based Vision-and-Language Navigation via Learning Disentangled Reasoning." arXiv preprint arXiv:2403.07376 (2024).

\bibitem{NavGPT} Zhou, Gengze, Yicong Hong, and Qi Wu. "Navgpt: Explicit reasoning in vision-and-language navigation with large language models." Proceedings of the AAAI Conference on Artificial Intelligence.

\bibitem{hahn2021no} Hahn, Meera, et al. "No rl, no simulation: Learning to navigate without navigating." Advances in Neural Information Processing Systems 34 (2021): 26661-26673.

\bibitem{li2025t2isafety} Li, Lijun, et al. "T2ISafety: Benchmark for Assessing Fairness, Toxicity, and Privacy in Image Generation." arXiv preprint arXiv:2501.12612 (2025).

\bibitem{an2024agfsync} An, Jingkun, et al. "AGFSync: Leveraging AI-Generated Feedback for Preference Optimization in Text-to-Image Generation." arXiv preprint arXiv:2403.13352 (2024).


\end{thebibliography}
\end{sloppypar}

\clearpage
\beginsupplement
\section*{Appendix}
\renewcommand{\thesubsection}{S\arabic{subsection}}

\subsection{\label{chap:S1}PanNuke and MoNuSAC preprocessing}
The PanNuke dataset comprises a set of 7,901 RGB patches, each with dimensions of $256 \times 256$ pixels, which we set as the standard patch size for our analysis. In contrast, the MoNuSAC dataset encompasses 294 images of heterogeneous dimensions. To standardize the MoNuSAC images with our experiments, we implement a standardization protocol. Specifically, for images exceeding the dimensions of $256 \times 256$ pixels, we segment them into equal-sized patches and apply mirror padding to the remaining portions to avoid information loss at the peripherals. Patches with dimensions less than $128 \times 128$ pixels are excluded from the dataset due to the insufficient resolution to capture relevant cellular details. For patches where either dimension falls between 128 and 256 pixels, we employ upsampling to achieve the standard patch size. As a result, we obtain a total of 2,823 RGB patches derived from the MoNuSAC dataset for subsequent analysis. For additional details on the MoNuSAC data preparation process, refer to the source code \cite{Shvetsov_2025a}.
\clearpage

\subsection{\label{chap:S2}Data usage for the methodology}

\counterwithin{figure}{subsection}
\renewcommand{\thefigure}{S\arabic{subsection}}

\begin{figure}[h!]
    \centering
    \includegraphics[width=\textwidth, height=0.85\textheight, keepaspectratio]{images/A2.pdf}
    \caption{Overview of the methodology for cross-labeling, dataset refinement, and model comparison. (1) Cross-relabeling - training and testing cell classification models, (2) Cross-relabeling - using cell classification models to create refined dataset, (3) Fine-tuning and training models for comparison, (4) Student knowledge distillation with refined dataset}
    \label{fig:S2}
\end{figure}
\clearpage

\subsection{\label{chap:S3}Confusion matrices for classification models}
\counterwithin{figure}{subsection}
\renewcommand{\thefigure}{S\arabic{subsection}.\arabic{figure}}

\begin{figure}[h!]
    \centering
    \includegraphics[width=\textwidth, height=0.4\textheight, keepaspectratio]{images/A3_1.pdf}
    \caption{Confusion matrix for PanNuke trained model}
    \label{fig:S3.1}
\end{figure}

\begin{figure}[h!]
    \centering
    \includegraphics[width=\textwidth, height=0.4\textheight, keepaspectratio]{images/A3_2.pdf}
    \caption{Confusion matrix for MoNuSAC trained model}
    \label{fig:S3.2}
\end{figure}

\clearpage

\subsection{\label{chap:S4}Datasets cell counts}

\counterwithin{table}{subsection}
\renewcommand{\thetable}{S\arabic{subsection}}

\begin{table}[h!]
\renewcommand{\arraystretch}{2.0}
\centering
\caption{\label{tab:S4}Cell counts for PanNuke, MoNuSAC and refined datasets. Numbers in parentheses indicate preprocessed cell counts for cell classifier models training and testing.}
%\adjustbox{max width=\textwidth}{%
\begin{tabular}{|l|c|c|c|}
\hline
%\rowcolor{gray!30}
Cell type & PanNuke & MoNuSAC & Refined \\
\hline
Neoplastic & 77,403 (68,031) & - & 105,451 \\
\hline
Epithelial & 26,572 (23,207) & - & 29,926 \\
\hline
Epithelial (benign and malignant) & - & 31,402 & - \\
\hline
Inflammatory & 32,276 & - & - \\
\hline
Lymphocytes & - & 37,045 (33,104) & 65,275 \\
\hline
Neutrophils & - & 1,355 (1,252) & 3,833 \\
\hline
Macrophage & - & 1,842 (1,695) & 3,410 \\
\hline
Dead & 2,908 & - & 2,908 \\
\hline
Connective & 50,585 & - & 50,585 \\
\hline
\end{tabular}
%
%}
\end{table}



\clearpage

\subsection{\label{chap:S5}Definition of validation metrics}
\counterwithin{equation}{subsection}
\renewcommand{\theequation}{\arabic{equation}}

\subsubsection{\label{chap:S5.1}R\textsuperscript{2}}
The coefficient of determination, denoted as $R^2$, is a statistical measure that represents the proportion of variance in the dependent variable that is predictable from the independent variables. In the context of cell quantification in pathology, $R^2$ is used to assess how well the predicted quantities of different cell types in a patch align with the actual quantities observed in the ground truth data, with higher values representing more accurate quantification. $R^2$ is defined as
\begin{equation*}
R^2 = 1 - \frac{\sum_{i=1}^n (y_i - \hat{y}_i)^2}{\sum_{i=1}^n (y_i - \bar{y})^2},
\end{equation*}
where $y_i$ represents the actual number of cells of a specific type in the $i$-th image, $\hat{y}_i$ represents the predicted number of cells of that type in the $i$-th image, $\bar{y}$ is the mean of the actual numbers across all images, and $n$ is the total number of images in the dataset.

The $R^2$ metric has a range of $(-\infty, 1]$. An $R^2$ of 1 indicates perfect prediction, where all predicted values exactly match the actual values. An $R^2$ of 0 suggests that the model explains none of the variability of the response data around its mean. If $R^2$ is negative, it indicates that the model performs worse than a model that simply predicts the mean of the actual values for all observations.

\subsubsection{\label{chap:S5.2}PQ}
Panoptic Quality ($PQ$) is a comprehensive metric used to evaluate the performance of segmentation models in tasks that require both instance segmentation and classification. $PQ$ provides a single score that encapsulates both the detection accuracy (i.e., how many objects were correctly identified) and the segmentation quality (i.e., how accurately the objects' boundaries were delineated). This metric is particularly useful in multiclass scenarios where each pixel is classified into distinct categories, such as different cell types in pathology images.

$PQ$ is calculated as the product of two terms: Detection Quality ($DQ$) and Segmentation Quality ($SQ$). It can be expressed as
\begin{equation*}
PQ = DQ \cdot SQ,
\end{equation*}
where
\begin{equation*}
DQ = \frac{TP}{TP + 0.5\, FP + 0.5\, FN},
\end{equation*}
\begin{equation*}
SQ = \frac{\sum_{(p, g) \in \mathcal{M}} IoU(p, g)}{TP}.
\end{equation*}
In these formulas, $TP$ denotes the number of correctly matched instances between ground truth and prediction, $FP$ denotes the predicted instances that have no corresponding ground truth, $FN$ denotes the ground truth instances that were not detected, $IoU(p, g)$ is the Intersection over Union for a pair of matched instances $p$ (prediction) and $g$ (ground truth), and $\mathcal{M}$ is the set of matched pairs.

The $PQ$ metric is calculated for each class and is averaged across classes to provide a global performance measure.

The $PQ$ score has a range of $[0, 1.0]$, where a higher score indicates better performance in both detecting and segmenting the instances correctly. A $PQ$ of 1 signifies perfect identification and segmentation of all instances, whereas a $PQ$ of 0 indicates that no instances were correctly identified and segmented.

\clearpage

\subsection{\label{chap:S6}Segmentation and Detection quality metrics for teacher and student models}

\begin{table}[h!]
\renewcommand{\arraystretch}{2.0}
\centering
\caption{Segmentation and detection quality for student and teacher models (CI 95\%)}
\label{tab:S6}
%\adjustbox{max width=\textwidth}{%
\begin{tabular}{|l|c|c|}
\hline
%\rowcolor{gray!30}
Metric & Teacher & Student \\
\hline
$SQ_{neoplastic}$ & 0.819 (0.815--0.823) & 0.824 (0.819--0.828) \\
\hline
$SQ_{lymphocyte}$ & 0.795 (0.788--0.802) & 0.790 (0.783--0.796) \\
\hline
$SQ_{connective}$ & 0.770 (0.762--0.776) & 0.780 (0.772--0.786) \\
\hline
$SQ_{dead}$ & 0.659 (0.623--0.688) & 0.657 (0.624--0.695) \\
\hline
$SQ_{epithelial}$ & 0.780 (0.770--0.790) & 0.788 (0.779--0.797) \\
\hline
$SQ_{macrophage}$ & 0.788 (0.760--0.810) & 0.757 (0.730--0.783) \\
\hline
$SQ_{neutrofil}$ & 0.782 (0.761--0.801) & 0.775 (0.759--0.792) \\
\hline
$DQ_{neoplastic}$ & 0.706 (0.692--0.719) & 0.727 (0.712--0.741) \\
\hline
$DQ_{lymphocyte}$ & 0.675 (0.656--0.698) & 0.713 (0.691--0.734) \\
\hline
$DQ_{connective}$ & 0.566 (0.546--0.584) & 0.583 (0.565--0.602) \\
\hline
$DQ_{dead}$ & 0.410 (0.361--0.465) & 0.435 (0.306--0.561) \\
\hline
$DQ_{epithelial}$ & 0.668 (0.639--0.694) & 0.673 (0.644--0.702) \\
\hline
$DQ_{macrophage}$ & 0.657 (0.583--0.727) & 0.615 (0.531--0.703) \\
\hline
$DQ_{neutrofil}$ & 0.691 (0.625--0.753) & 0.729 (0.679--0.778) \\
\hline
\end{tabular}
%
%}
\end{table}

\clearpage

\subsection{\label{chap:S7}QuPath integration method}
We adopt an integration strategy leveraging the paquo \cite{Bayer_AG} library, a Python package that enables direct interaction with QuPath’s internal API, thereby facilitating seamless data exchange without intermediate conversion steps. The data processing pipeline (\hyperref[fig:S7]{Appendix Figure S7}) begins with the acquisition of WSIs and their associated annotations from QuPath, which are represented as Shapely \cite{Gillies_Wel_etal._2024} polygons. Utilizing paquo, we directly read, create, and modify these annotations and detections within a QuPath project in the Python environment. Images are then cropped using these polygons and processed by cell segmentation and classification models employing standard vision processing toolkits such as OpenCV, pyvips, and PyTorch. Additionally, QuPath employs Groovy scripts to initiate a Python process that starts the entire pipeline from QuPath graphical interface: fetching polygons, extracting images from them, and running deep learning model inference on the cropped images. 
The results are returned to QuPath, leveraging paquo's Python bindings to manipulate QuPath data while minimizing the computational overhead typically associated with cross-environment communication.

\counterwithin{figure}{subsection}
\renewcommand{\thefigure}{S\arabic{subsection}}

\begin{figure}[h!]
    \centering
    \includegraphics[width=\textwidth]{images/A7.pdf}
    \caption{QuPath integration workflow using Python environment}
    \label{fig:S7}
\end{figure}

Compared to traditional workflows that involve exporting annotations as GeoJSON, classifying them in Python, and reimporting them into QuPath, our approach offers several advantages. We eliminate the need to switch between programming languages, providing a cohesive and streamlined development process entirely within QuPath software and removing the necessity to use other tools. Meanwhile, we avoid storing annotations as intermediate JSON files unless required for external use or archiving. By conducting the entire inference and post-processing workflow within the Python environment, we leverage the power and flexibility of Python libraries for image processing and machine learning. This approach also enables adjustments to any set of labels and models, thereby improving its applicability.

%\hfill

The distilled model and QuPath integration code are packaged into a Docker container, enabling streamlined execution with the Docker engine. Detailed integration code and deployment instructions can be found in the GitHub repository \cite{Shvetsov_2025b}.

Despite these benefits, we acknowledge that the paquo library is a proof‑of‑concept project in its early development stage and has not been tested across all versions of QuPath.

\clearpage

\subsection{\label{chap:S8}Data and code availability statement}
All datasets, models, and code used in this study are publicly available and can be obtained from the repositories listed below. 
The PanNuke \cite{Gamper_Koohbanani_etal._2019} and MoNuSAC \cite{Verma_Kumar_etal._2021} datasets are publicly accessible, and download information along with detailed descriptions can be found in their respective articles. Preprocessing scripts for PanNuke and MoNuSAC data, as well as individual cell extraction scripts, are available on GitHub \cite{Shvetsov_2025a}. The H-Optimus foundation model used in our experiments can be downloaded from the HuggingFace repository \cite{hoptimus2024}, and model information is available on GitHub \cite{Saillard_Jenatton_etal._2024}. In addition, the integration code for QuPath and the distilled model packaged in a Docker container are provided in the repository \cite{Shvetsov_2025b}, and paquo Python library is available from the authors GitHub repository \cite{Bayer_AG}.
\clearpage

\end{document}


\appendix
\section{More Details about Method and Deployment}

\subsection{Rank Task Paradigm Statement}
In recommendation systems, Click-Through Rate (CTR) prediction is a key task that estimates the probability of a user clicking on a recommended item. This task is crucial for enhancing user engagement and optimizing system performance. CTR prediction can be formulated as a supervised binary classification problem. Given a user $u$ and a candidate item $c$, the goal is to predict the probability that the user will click on the item. This can be mathematically represented as:
\begin{equation}
\hat{y}_{u,c} = \sigma(f(X_u, X_c))
\end{equation}
where $y_{u,c}$ is a binary label indicating whether the user clicked on the item (1) or not (0), $\hat{y}_{u,c}$ is the prediction value of $y_{u,c}$. $X_u$ and $X_c$ are the feature vectors representing the user and item, respectively. $f(\cdot)$ is a model that estimates the probability of a click and $\sigma(\cdot)$ is the sigmoid function. The model is trained by minimizing the negative log-likelihood:
\begin{equation}
\mathcal{L} = -\frac{1}{N_s} \sum_{u,c} \left[ y_{u,c} \log(\hat{y}_{u,c}) + (1 - y_{u,c}) \log(1 - \hat{y}_{u,c}) \right]
\end{equation}
where $N_s$ is the total number of samples.

Traditional DLRM predicts users' next actions on candidate items based on their historical characteristics and elaborately engineered item features. This approach is typically represented as: 
\begin{equation}
f(X_u, X_c) = f(u_{fea},c_{fea})
\end{equation}
where $u_{fea}$ and $c_{fea}$ respectively denote the user features and item features generated through feature engineering. However, due to limitations in scalability and the complexity of feature engineering, we propose replacing handcrafted features with user behavior sequences. Specifically, we adopt a Transformer-based sequence model to scale up our recommendation system.

% With the sequence modeling, our task is to recommend the next song that the user is interested in based on the user's historical behavior sequence. It is commonly known that we utilize the user's historical $n$ songs to predict the user's $(n+1)_{th}$ song. It can be formulated as $p(a_{n+1}|(x_{1}, a_{1}), (x_{2}, a_{2}), ..., (x_{n}, a_{n}), x_{n+1})$, in which $x_{i}$  denotes a music ID from entire music set X, and $x_1$-$x_n$ denotes items on which the user has already taken action, while $x_{n+1}$ is from the result of recall and prerank, indicating the items on which the user may take action next. $a_{i}$ represents the behavior corresponding to the item $x_{i}$. But the method proposed in this paper will divide the user sequence into sub sequences with different periods or behaviors, and the specific prediction paradigm will be mentioned later.

\begin{figure*}[t]
  \centering
  % \includegraphics[width=6in]{model_v2.pdf}
  % \includegraphics[width=\linewidth]{overview_v1.pdf}
  % \includegraphics[scale=0.35]{overview_v1.pdf}
  \includegraphics[width=\linewidth]{overview_v1.pdf}
  \caption{An Overview of our efficient large recommendation model deployment}
  \Description{}
  \label{fig:deploy}
\end{figure*}

With sequence modeling, our task is to predict the user's behaviors on candidate items based on their historical behavior sequences. Specifically, we utilize the user's historical $n$ items and their corresponding behaviors to predict the potential behavior on the candidate item $c$, which can be formulated as:
\begin{equation}
f(X_u, X_c) = f(\{x_{1}, x_{2}, ..., x_{n}\}, x_{c})
\end{equation}
where $x_i$ denotes a item ID from the entire music set X. Here, $x_1$ to $x_n$ represent items on which the user has already taken action, while $x_c$ is a candidate item obtained through recall and prerank stage.

ASTRO model employs multi-scale sequence partitioning to extract multi-scale subsequences with extraction strategy from user lifecycle behaviors. By integrating Equation \ref{eqn:S} and \ref{eqn:Sk}, the task paradigm associated with this sequence structuring under the ASTRO model is characterized as follows:
\begin{equation}
f(X_u, X_c) = f(\{a_k,x^{a_k}_{1},x^{a_k}_{2},...,x^{a_k}_{n_k}\}_{k=1}^{N_b},x_{c})
\end{equation}
where $N_b$ represents the number of extraction strategy $a_k$, and $n_k$ represents the length of subsequence extracted by $a_k$.

To ensure unified training and inference stage and align training data more closely with real online requests, TURBO organizes samples according to "SUMI" style. Therefore, we predict multiple candidate items for one user simultaneously in the training stage, which can be expressed as:
\begin{equation}
f(X_u, \{X_{c,j}\}_{j=1}^{N_c}) = f(\{a_k,x^{a_k}_{1},x^{a_k}_{2},...,x^{a_k}_{n_k}\}_{k=1}^{N_b},\{x_{c,j}\}_{j=1}^{N_c})
\end{equation}
where $x_{c,j}$ represents the $j$-th candidate item, and $N_c$ represents the number of total candidate items.

% However, the method proposed in this paper divides the user sequence into subsequences based on different periods or behaviors. The specific prediction paradigm will be detailed later.


The proposed recommendation model offers several notable advantages that significantly enhance its performance and applicability:
(1) The model achieves a substantial reduction in computational complexity from $O(n^{2}d)$ to $O(n^{2}d/N_b)$, where $N_b$ denotes the number of subsequence blocks. This improvement enables a marked acceleration in both training and inference processes, even when $N_b=2$. (2) Our model overcomes temporal limitations for certain crucial behaviors. For highly sparse yet significant behaviors, we leverage users' entire lifecycle behaviors. Compared to the previous method that relies solely on data within a single time window, our method enables the incorporation of a broader range of items that more effectively represent user interests. (3) By extending important sparse behaviors to the entire lifecycle, the model gains a significant enrichment of information. Additionally, the block-wise approach alleviates the unfair allocation of attention scores between different subsequences, which will be further discussed below.
Overall, these advantages contribute to the robustness and superiority of our recommendation model, thereby reinforcing its potential as a valuable approach in recommendation systems.

\subsection{Deployment}
As depicted in Figure \ref{fig:deploy}, the implementation of the comprehensive recommendation system integrates online services, offline training, and model inference procedures. The system's performance is augmented by the TURBO acceleration framework and the ASTRO model.
At the online service stage, the system initially conducts recall and prerank operations to identify the candidate item set. Subsequently, it utilizes the ASTRO model in conjunction with the TURBO framework to process these candidates, eventually generating a prediction list.
In the offline training phase, user behavior logs are recorded and stored in a database. These logs are then dynamically compressed and transferred to an offline storage system. The log data are used to create a list of project labels, offering crucial data support for model training. During the training process, the system computes the loss and optimizes the model parameters via the backpropagation mechanism.
The actual sorting process is divided into two stages. In the first stage, the model analyzes the user behavior sequence to generate a KV-cache vector, thus swiftly capturing user characteristics. In the second stage, the model capitalizes on these extracted features to score the candidate items and generate the final recommendation list. The collaborative design of the TURBO framework and the ASTRO model boosts the efficiency of the recommendation system and guarantees the accuracy of the recommendation results, thereby furnishing users with more personalized and satisfactory recommendation services.

\begin{table}[t]
\centering
\caption{Dataset Statistics}
\label{tab:Dataset}
\begin{tabular}{@{}lcccccc@{}}
\toprule
Dataset & \#User & \#Item & \#Interaction \\
\midrule
% Industrial & 7.0M  & 6.0M  & 21.8M  \\
% Industrial & 44.9M  & 6.0M  & 930M  \\
Industrial & >40M  & >6M  & >1B  \\
Spotify & 0.16M & 3.7M & 1.2M  \\
30Music & 0.02M & 4.5M & 16M \\
Amazon-Book & 0.54M & 0.37M  & 1.09M \\
\bottomrule
\end{tabular}
\end{table}

\begin{figure}[t]
  \centering
  \includegraphics[width=\linewidth]{attention_v1.pdf}
  \caption{Attention Distribution of Different Models on Different Behaviors}
  \Description{}
  \label{fig:attn}
\end{figure}

\section{More Details about Experimental Setting}
\subsection{Dataset}
\label{appendix:B.1}
Table \ref{tab:Dataset} presents the number of users, items, and interactions for the four processed datasets. To protect data privacy, we have applied special processing to the Industrial dataset. As a result, the data volume shown in the table is lower than the actual amount. However, it is clear that the scale of our industrial dataset still significantly exceeds that of the other datasets. This substantial data volume provides a robust foundation for conducting scaling experiments.

\subsection{Compared Methods}
\label{appendix:B.2}
This study assesses the performance of various models, including DLRM, DIN, TWIN, Transformer, HSTU, and ASTRO variants. The subsequent section will introduce the detailed experimental setting in Industrial Dataset associated with each of these models.
\begin{itemize}
\item DLRM: The DLRM is selected as the baseline model after extensive online iterations, which has demonstrated remarkable effectiveness in both offline and online metrics. Its feature architecture incorporates statistical features and user behavior sequences. In terms of modeling methodology, we primarily employ a two-stage TWIN approach to extract user sequence representations. These representations are subsequently combined with statistical features and processed through MLP layers for feature interaction. The sequence length is set to 2000.
\item DIN: DIN is a widely-used model that captures user interests through a target attention mechanism, focusing on the interaction between user historical behaviors and the target item. In terms of the experimental setup, we do not employ statistical features and only utilize behavior sequences with 1000 length.
\item TWIN: A model that aligns the computation methods of GSU and ESU to enhance consistency and accuracy in modeling long-term user behavior. In the experimental setup, we also do not employ statistical features. The input sequence length for GSU is set to 2000, while for ESU, it is set to 1000.
\item Transformer: A foundational model leveraging the Transformer architecture for sequence modeling. We employ a one-stage Transformer encoder to directly process behavior sequences with a length of 2000. 
\item HSTU: The feature-level sequence undergoes temporal reorganization by chronological ordering of item-action pairs. This restructured sequence is then processed through the HSTU model at the structural level to predict target item-specific user action. The sequence length is set to 2000.
\item ASTRO(-ATL,-BGF): In the previous method, the sequence length of 2000 is confined to a fixed time window and user behavior is not filtered. After introducing multi-scale sequence partitioning, the time period of the behavior sequence is extended to the entire user lifecycle (> 10 years), and extraction strategies are established based on the business logic. Ultimately, this process retains a sequence length of 200. 
\item ASTRO(-BGF): Based on multi-scale sequence partitioning, all traditional Transformer layers are replaced with Adaptive Transformer Layers. The attention distribution of each block is adaptively adjusted using the recommendation scenario and extraction strategy. 
\item ASTRO: This model integrates multi-scale sequence partitioning, adaptive transformer layer, and bit-wise gating fusion. Meanwhile, the layer number in the model is set to 2, which is consistent with the layer setting in the previous method.
\item ASTRO-large: Compared to the ASTRO model, we have increased 4x sequence length and 6x layer number. As a result, the final sequence length is 800, and the layer number is 12. 
\end{itemize}

\section{Attention Distribution Analysis}
Sequence modeling in recommendation systems is essential for identifying items in a user's historical sequence that are similar to the target item, with this similarity quantified through attention scores. Before the adoption of attention mechanisms, sequence processing relied on average pooling, which assigned uniform attention scores to all items. The attention mechanism in DIN has been shown to enhance recommendation performance by assigning distinct attention scores to different items. The Transformer-based model emphasizes attention mechanisms. Thus this section explores the impact of multi-scale sequence partitioning on the attention distribution. The attention score will be introduced for two distinct behaviors: full-play and like.

% \subsection{Impact of Multi-Scale Sequence Partitioning}
Multi-scale sequence partitioning not only reduces computational complexity but also directly impacts attention distribution. Specifically, the proportion of "full-play" behaviors is higher than that of "like" behaviors. To address this imbalance, we distinguish between these two behaviors and extend the time window for "like" behaviors, thereby making the number of "like" and "full-play" behaviors approximately equal.
We conduct comparative experiments between the Transformer and ASTRO. In the Transformer experiment, sequences contain equal proportions of "like" and "full-play" behaviors for computation. In contrast, the ASTRO model separates these behaviors and trains them in a block-wise manner, with attention distributions illustrated in Figure \ref{fig:attn}.
In the Transformer structure, even with a similar number of "like" and "full-play" behaviors, the attention ratio for "full-play" behaviors reaches 71\%. This is primarily because "full-play" samples have a higher proportion and naturally occupy more gradients. Consequently, "full-play" behaviors receive higher attention scores due to the nature of the task, resulting in less attention to "like" behaviors and poorer performance on "like" tasks.
In the ASTRO model, each behavior is assigned to a separate block with its own attention distribution. Through multi-scale sequence partitioning, we can not only increase the number of sparse yet important behaviors but also improve their attention distribution, thereby enhancing model effectiveness. Performance comparisons between the Transformer and ASTRO (-ATL, -BGF) across different datasets are shown in Table \ref{tab:ASTRO}, which indicates that significant improvement can be achieved solely through multi-scale partitioning.

% Figure \ref{fig:attn} shows the attention score distribution when the red heart and completion are in the same sequence. It can be seen that the overall proportion of completion is higher than that of the red heart, because the ratio of completion to red heart in the task is approximately 71: 29, The completion task naturally occupies more gradients due to its high proportion, which leads to a tendency to have higher attention scores for the completion behavior during attention operations due to the task itself, resulting in less attention paid to the red heart behavior and poor performance in the red heart task. And when blocks are introduced, there will be their own distributions. From a specific case perspective, within a single sequence, the item corresponding to the max score of the finished broadcast is 50\%, and the item corresponding to the max score of the red heart is 50\%; When split into multiple sequences, the item corresponding to the max score of the finished broadcast is 50\%, and the item corresponding to the max score of the red heart is xx; After disassembling the sequence, more attention will be paid to the sequence corresponding to a sparser task, thus achieving better prediction results on that task
% \subsection{The influence of adaptive temperature on attention distribution}
% The motivation for this point is that after breaking down the block, a point can be found that the high scores of the end behavior are concentrated in the recent part in Figure \ref{fig:comp_sketch}, and the polarization phenomenon of scoring will be more severe; However, there is no obvious concentration phenomenon in the collect behavior, and the scoring is relatively more uniform. Therefore, considering the setting of adaptive temperature coefficients for different scenarios and tasks, The temperature coefficients corresponding to (fm,end), (fm,red), (daily, end), and (daily, red) are 1, 2, 3, and 4, respectively. It can be seen that in the same scenario, the end play action has a lower coefficient than the collect action, because the end play action tends to be more inclined towards the user's short-term interests, while the red heart, as a representative of the user's long-term interests, has a higher temperature coefficient. In the same task, real-time recommended scenes will also have lower temperature coefficients in different scenarios. In temperature adaptation, there are only two independent variables: scene and task. In order to have a more direct impact, the most direct influence method is selected.


%%
%% If your work has an appendix, this is the place to put it.

\end{document}
% \endinput
%%
%% End of file `sample-sigconf.tex'.
