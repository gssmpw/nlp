Similar to other sign languages, ASL is also a visual-based natural language, expressed by using both manual and non-manual markers~\cite{Stokoe1961SignLS}.  A common misconception is that substituting each written English word with a corresponding ASL sign would be enough as a translation~\cite{aslgrammar}. However, this approach does not produce true ASL~\cite{hanson2012computers}, as ASL has its own grammar and lexicon, distinct from English~\cite{lucas2001sociolinguistics,valli2000linguistics}. Moreover, there is no one-to-one mapping between English words and ASL signs, which makes direct substitution less appropriate~\cite{neidle2007signstream}. 

% \subsubsection{ASL linguistics}
% ASL consists of various linguistic aspects, including phonology, morphology, syntax, semantics, and pragmatics, each contributing to the language’s unique characteristics~\cite{valli2000linguistics}. ASL phonology involves the study of the smallest units, or parameters, such as handshape, movement, location, palm orientation, and non-manual signals like facial expressions~\cite{sandler2006sign}. ASL morphology involves the formation of signs from smaller units, demonstrating both derivational and inflectional processes~\cite{aronoff2005paradox}. The syntax of ASL often follows a topic-comment order rather than the subject-verb-object structure typical of English~\cite{liddell2021american,aslgrammar}, and relies heavily on spatial grammar, where the location and movement of signs correspond to syntactic roles~\cite{neidle2000syntax}. ASL semantics and pragmatics involve the use of signs in context, with meaning often modified through facial expressions, body language, and the specific use of signing space~\cite{wilbur2013phonological}.

\subsubsection{ASL Written Representation}
ASL-LEX~\cite{caselli2017asl} has been used as a gloss reference for annotation of ASL in several works (\eg~\cite{desai2024asl,joze2018ms,ma2018signfi,bragg2021asl}). However, ASL-LEX glosses often lack representation of non-manual markers, such as facial expressions and body movement, which can limit the naturalness and understandability of generated signs when used in SLG~\cite{huenerfauth_evaluation_2008}. To address this, ASL linguists have developed conventions to capture non-manual markers in addition to manual behaviors~\cite{neidle2001signstream,neidle2007signstream,neidle2002signstream}. These include behaviors such as head position and movements, eye gaze and aperture, eyebrow position and movements, and body movements. 
% In this work, we adopt these conventions for our annotations.
% \rotem{only leave this part if we end up using the meta-data}

% \begin{table*}[t]
\caption{Existing ASL datasets. SL stands for sign language. ``-'' represents relevant information was not provided. ``Unknown'' represents relevant information was not found. \label{tab:asl_datasets}}
\renewcommand{\arraystretch}{1.1}
   \resizebox{\textwidth}{!}{
\begin{tabular}{L{0.15\textwidth}|R{0.06\textwidth}|R{0.06\textwidth}|R{0.06\textwidth}| C{0.1\textwidth}| L{0.12\textwidth}|L{0.45\textwidth}|C{0.15\textwidth}}
% {l|c|c|c|c|c}
\toprule\hline
\multicolumn{1}{c|}{{\textbf{\makecell[c]{Dataset}}}} & \multicolumn{1}{c|}{{\textbf{\makecell[c]{Vocab.}}}}  &
\multicolumn{1}{c|}{{\textbf{\makecell[c]{Hours}}}} &  \multicolumn{1}{c|}{{\textbf{\makecell[c]{Signers}}}} & \multicolumn{1}{c|}{{\textbf{\makecell[c]{Resolutions \\ (pixels)}}}} & \multicolumn{1}{c|}{{\textbf{\makecell[c]{Modalities}}}} &\multicolumn{1}{c|}{{\textbf{\makecell[c]{Gloss Labeling Standard}}}} & \multicolumn{1}{c}{{\textbf{\makecell[c]{Annotation \\ Tools}}}}  \\ \hline
% \multicolumn{6}{c}{\textbf{Utterance-level American Sign Language}} \\
RWTH-BOSTON-50~\cite{zahedi2005combination} & 50 & >9 & 3& 195 $\times$ 165 & Video, word & - & - \\\hline
Purdue RVL-SLLL~\cite{martinez2002purdue} & 104 & 14  &14 & 640 $\times$ 480 & Video, Gloss & Glosses include manual English-based labels, and non-manual behaviors such as handshapes and motions for two hands.  & Human Annotator \\\hline
RWTH-BOSTON-400~\cite{dreuw2008benchmark} & 483 & - & 5& 648 $\times$ 484 & Video, Gloss, Utterance & Glosses include manual English-based labels and non-manual behaviors, both anatomical (\eg raised eyebrows) and functional (\eg wh-questions). Glosses do not include handshape annotations. & SignStream$^@$2~\cite{neidle2001signstream} \\\hline
MS-ASL~\cite{joze2018ms} & 1K & 24 & 222  & 224 $\times$ 224 & Video, Pose, Word & Glosses were generated by referencing ASL Tutorial books~\cite{zinza2006master,caselli2017asl}. & Human Annotator \\\hline
DSP~\cite{neidle2022asl}& >1.7K & - & 15 & - & Video, Gloss, Utterance, Word & Glosses include manual English-based gloss labels, sign type, start and end handshapes (both hands), grammatical markers (\eg questions, negation, topic/focus, conditional, relative clauses), and anatomical behaviors (\eg head nods/shakes, eye aperture, gaze). & SignStream$^@$3~\cite{neidle2017user} \\\hline
NCSLGR~\cite{neidle2012new}& 1.8K & 5.3   & 4 & - & Video, Gloss, Utterance & Glosses include manual English-based labels and non-manual behaviors, both anatomical (\eg raised eyebrows) and functional (\eg wh-questions). Glosses do not include handshape annotations. & SignStream$^@$2~\cite{neidle2001signstream}\\\hline
ASLLRP~\cite{neidle2022asl} & >2.7K & 3.6   & 4 & - & Video, Gloss, Utterance, Word & Glosses include manual English-based gloss labels, sign type, start and end handshapes (both hands), grammatical markers (\eg questions, negation, topic/focus, conditional, relative clauses), and anatomical behaviors (\eg head nods/shakes, eye aperture, gaze). & SignStream$^@$3~\cite{neidle2017user}\\\hline
ASL Citizen~\cite{desai2024asl}& >2.7K & 30.5 & 52 & - & Video, Gloss, Pose, Word & Glosses include manual English-based labels by referencing a lexical database of ASL (\ie ASL-LEX~\cite{caselli2017asl}).  & Unknown \\\hline
Signing Savvy~\cite{signingsavvy} & >13K & - & - & -  &Video, Gloss, Utterance, Word & Glosses include manual English-based labels. & Unknown \\\hline
How2Sign~\cite{duarte_how2sign_2021}& 16K & 80  &  11 & 1280 $\times$ 720 & Video, Pose, Gloss, Utterance, Speech & Glosses include English-based labels, but do not include information such as hand-shape, hand movement/orientation, and facial expressions, such as raised eyebrows in yes/no questions. & ELAN~\cite{crasborn2008enhanced} \\\hline
% ASLG-PC12~\cite{othman2012english}& 24,002,570 Utterances in total, more than 800 million words in English. & Gloss, Utterance &  & - (from books but not human) & & Signs are in small caps. Lexicalized finger-spelled words use ``\#'' before small caps. Full finger-spelling uses dashes between small caps (\eg A-C-R-A-F). Non-manual signals and eye-gaze are shown above the glosses. & Not specify \\\hline
OpenASL~\cite{shi2022open} & 33K & 288 &  220 & - & Video, Utterance & - & - \\\hline
\bottomrule

\end{tabular}}
\end{table*}

\subsubsection{ASL Datasets} 
\label{subsubsec:asl_datasets}
Sign language datasets often pose a bottleneck for SLG research~\cite{bragg_sign_2019}. Reviewing ASL datasets reveals substantial variation in vocabulary size, recording duration, number of signers, image resolution, modalities, gloss annotation conventions, and annotation tools~\cite{zahedi2005combination,dreuw2008benchmark,martinez2002purdue,joze2018ms,neidle2012new,neidle2022asl,desai2024asl,signingsavvy,duarte_how2sign_2021,shi2022open,uthus2023youtubeasl} (Table \ref{tab:asl_datasets}). For instance, OpenASL~\cite{shi2022open} and YouTube-ASL~\cite{uthus2023youtubeasl} stand out with their extensive vocabularies of approximately 33,000 and 60,000 signs, respectively, offering a broad lexical base. However, these datasets provide only videos and English captions, without their corresponding written representations. 

RWTH-BOSTON-50~\cite{zahedi2005combination} and Purdue RVL-SLLL~\cite{martinez2002purdue} are among the earliest publicly available ASL datasets. Despite their pioneering role, their relatively small vocabularies, lack of detailed gloss annotations, non-expert human annotators, and variable image quality limit their utility for more advanced ASL research and applications. MS-ASL~\cite{joze2018ms} and ASL Citizen~\cite{desai2024asl} provide word-level isolated ASL signs from a wide range of signers, serving as valuable resources for sign language recognition research. However, for tasks such as generating ASL signs from English sentences, word-level datasets lack crucial contextual information, such as sentence structure, non-manual markers, and signer consistency.

Datasets like NCSLGR~\cite{neidle2012new}, ASLLRP\cite{neidleboston}, and DSP~\cite{neidle2022asl}, resulting from collaborations among multiple universities, as well as the How2Sign~\cite{duarte_how2sign_2021} dataset collected with higher resolution cameras, offer more comprehensive data. These datasets include English sentences with corresponding written representations, detailed annotation conventions (\eg \cite{neidle2001signstream,neidle2007signstream}), and videos featuring both continuous and citation-form signs. These advancements have allowed some of these datasets, such as NCSLGR and How2Sign datasets, to be used as benchmarks for ASL processing research (\eg~\cite{zhu_neural_2023,moryossef_data_2021,baltatzis2024neural}). While these datasets address some of the critical gaps in earlier resources, issues such as their relatively small sizes (\eg \cite{neidle2012new,neidle2022asl}), inconsistent annotation conventions across datasets, and limited accessibility of the DSP and How2Sign gloss datasets make some tasks of ASL processing both promising and challenging. 