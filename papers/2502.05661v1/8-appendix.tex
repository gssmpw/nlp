\appendix

\section{A Review of American Sign Language and Publicly Available Datasets}\label{appendix:ASL}

Similar to other sign languages, ASL is also a visual-based natural language, expressed by using both manual and non-manual markers~\cite{Stokoe1961SignLS}.  A common misconception is that substituting each written English word with a corresponding ASL sign would be enough as a translation~\cite{aslgrammar}. However, this approach does not produce true ASL~\cite{hanson2012computers}, as ASL has its own grammar and lexicon, distinct from English~\cite{lucas2001sociolinguistics,valli2000linguistics}. Moreover, there is no one-to-one mapping between English words and ASL signs, which makes direct substitution less appropriate~\cite{neidle2007signstream}. 

% \subsubsection{ASL linguistics}
% ASL consists of various linguistic aspects, including phonology, morphology, syntax, semantics, and pragmatics, each contributing to the language’s unique characteristics~\cite{valli2000linguistics}. ASL phonology involves the study of the smallest units, or parameters, such as handshape, movement, location, palm orientation, and non-manual signals like facial expressions~\cite{sandler2006sign}. ASL morphology involves the formation of signs from smaller units, demonstrating both derivational and inflectional processes~\cite{aronoff2005paradox}. The syntax of ASL often follows a topic-comment order rather than the subject-verb-object structure typical of English~\cite{liddell2021american,aslgrammar}, and relies heavily on spatial grammar, where the location and movement of signs correspond to syntactic roles~\cite{neidle2000syntax}. ASL semantics and pragmatics involve the use of signs in context, with meaning often modified through facial expressions, body language, and the specific use of signing space~\cite{wilbur2013phonological}.

\subsubsection{ASL Written Representation}
ASL-LEX~\cite{caselli2017asl} has been used as a gloss reference for annotation of ASL in several works (\eg~\cite{desai2024asl,joze2018ms,ma2018signfi,bragg2021asl}). However, ASL-LEX glosses often lack representation of non-manual markers, such as facial expressions and body movement, which can limit the naturalness and understandability of generated signs when used in SLG~\cite{huenerfauth_evaluation_2008}. To address this, ASL linguists have developed conventions to capture non-manual markers in addition to manual behaviors~\cite{neidle2001signstream,neidle2007signstream,neidle2002signstream}. These include behaviors such as head position and movements, eye gaze and aperture, eyebrow position and movements, and body movements. 
% In this work, we adopt these conventions for our annotations.
% \rotem{only leave this part if we end up using the meta-data}

% \begin{table*}[t]
\caption{Existing ASL datasets. SL stands for sign language. ``-'' represents relevant information was not provided. ``Unknown'' represents relevant information was not found. \label{tab:asl_datasets}}
\renewcommand{\arraystretch}{1.1}
   \resizebox{\textwidth}{!}{
\begin{tabular}{L{0.15\textwidth}|R{0.06\textwidth}|R{0.06\textwidth}|R{0.06\textwidth}| C{0.1\textwidth}| L{0.12\textwidth}|L{0.45\textwidth}|C{0.15\textwidth}}
% {l|c|c|c|c|c}
\toprule\hline
\multicolumn{1}{c|}{{\textbf{\makecell[c]{Dataset}}}} & \multicolumn{1}{c|}{{\textbf{\makecell[c]{Vocab.}}}}  &
\multicolumn{1}{c|}{{\textbf{\makecell[c]{Hours}}}} &  \multicolumn{1}{c|}{{\textbf{\makecell[c]{Signers}}}} & \multicolumn{1}{c|}{{\textbf{\makecell[c]{Resolutions \\ (pixels)}}}} & \multicolumn{1}{c|}{{\textbf{\makecell[c]{Modalities}}}} &\multicolumn{1}{c|}{{\textbf{\makecell[c]{Gloss Labeling Standard}}}} & \multicolumn{1}{c}{{\textbf{\makecell[c]{Annotation \\ Tools}}}}  \\ \hline
% \multicolumn{6}{c}{\textbf{Utterance-level American Sign Language}} \\
RWTH-BOSTON-50~\cite{zahedi2005combination} & 50 & >9 & 3& 195 $\times$ 165 & Video, word & - & - \\\hline
Purdue RVL-SLLL~\cite{martinez2002purdue} & 104 & 14  &14 & 640 $\times$ 480 & Video, Gloss & Glosses include manual English-based labels, and non-manual behaviors such as handshapes and motions for two hands.  & Human Annotator \\\hline
RWTH-BOSTON-400~\cite{dreuw2008benchmark} & 483 & - & 5& 648 $\times$ 484 & Video, Gloss, Utterance & Glosses include manual English-based labels and non-manual behaviors, both anatomical (\eg raised eyebrows) and functional (\eg wh-questions). Glosses do not include handshape annotations. & SignStream$^@$2~\cite{neidle2001signstream} \\\hline
MS-ASL~\cite{joze2018ms} & 1K & 24 & 222  & 224 $\times$ 224 & Video, Pose, Word & Glosses were generated by referencing ASL Tutorial books~\cite{zinza2006master,caselli2017asl}. & Human Annotator \\\hline
DSP~\cite{neidle2022asl}& >1.7K & - & 15 & - & Video, Gloss, Utterance, Word & Glosses include manual English-based gloss labels, sign type, start and end handshapes (both hands), grammatical markers (\eg questions, negation, topic/focus, conditional, relative clauses), and anatomical behaviors (\eg head nods/shakes, eye aperture, gaze). & SignStream$^@$3~\cite{neidle2017user} \\\hline
NCSLGR~\cite{neidle2012new}& 1.8K & 5.3   & 4 & - & Video, Gloss, Utterance & Glosses include manual English-based labels and non-manual behaviors, both anatomical (\eg raised eyebrows) and functional (\eg wh-questions). Glosses do not include handshape annotations. & SignStream$^@$2~\cite{neidle2001signstream}\\\hline
ASLLRP~\cite{neidle2022asl} & >2.7K & 3.6   & 4 & - & Video, Gloss, Utterance, Word & Glosses include manual English-based gloss labels, sign type, start and end handshapes (both hands), grammatical markers (\eg questions, negation, topic/focus, conditional, relative clauses), and anatomical behaviors (\eg head nods/shakes, eye aperture, gaze). & SignStream$^@$3~\cite{neidle2017user}\\\hline
ASL Citizen~\cite{desai2024asl}& >2.7K & 30.5 & 52 & - & Video, Gloss, Pose, Word & Glosses include manual English-based labels by referencing a lexical database of ASL (\ie ASL-LEX~\cite{caselli2017asl}).  & Unknown \\\hline
Signing Savvy~\cite{signingsavvy} & >13K & - & - & -  &Video, Gloss, Utterance, Word & Glosses include manual English-based labels. & Unknown \\\hline
How2Sign~\cite{duarte_how2sign_2021}& 16K & 80  &  11 & 1280 $\times$ 720 & Video, Pose, Gloss, Utterance, Speech & Glosses include English-based labels, but do not include information such as hand-shape, hand movement/orientation, and facial expressions, such as raised eyebrows in yes/no questions. & ELAN~\cite{crasborn2008enhanced} \\\hline
% ASLG-PC12~\cite{othman2012english}& 24,002,570 Utterances in total, more than 800 million words in English. & Gloss, Utterance &  & - (from books but not human) & & Signs are in small caps. Lexicalized finger-spelled words use ``\#'' before small caps. Full finger-spelling uses dashes between small caps (\eg A-C-R-A-F). Non-manual signals and eye-gaze are shown above the glosses. & Not specify \\\hline
OpenASL~\cite{shi2022open} & 33K & 288 &  220 & - & Video, Utterance & - & - \\\hline
\bottomrule

\end{tabular}}
\end{table*}

\subsubsection{ASL Datasets} 
\label{subsubsec:asl_datasets}
Sign language datasets often pose a bottleneck for SLG research~\cite{bragg_sign_2019}. Reviewing ASL datasets reveals substantial variation in vocabulary size, recording duration, number of signers, image resolution, modalities, gloss annotation conventions, and annotation tools~\cite{zahedi2005combination,dreuw2008benchmark,martinez2002purdue,joze2018ms,neidle2012new,neidle2022asl,desai2024asl,signingsavvy,duarte_how2sign_2021,shi2022open,uthus2023youtubeasl} (Table \ref{tab:asl_datasets}). For instance, OpenASL~\cite{shi2022open} and YouTube-ASL~\cite{uthus2023youtubeasl} stand out with their extensive vocabularies of approximately 33,000 and 60,000 signs, respectively, offering a broad lexical base. However, these datasets provide only videos and English captions, without their corresponding written representations. 

RWTH-BOSTON-50~\cite{zahedi2005combination} and Purdue RVL-SLLL~\cite{martinez2002purdue} are among the earliest publicly available ASL datasets. Despite their pioneering role, their relatively small vocabularies, lack of detailed gloss annotations, non-expert human annotators, and variable image quality limit their utility for more advanced ASL research and applications. MS-ASL~\cite{joze2018ms} and ASL Citizen~\cite{desai2024asl} provide word-level isolated ASL signs from a wide range of signers, serving as valuable resources for sign language recognition research. However, for tasks such as generating ASL signs from English sentences, word-level datasets lack crucial contextual information, such as sentence structure, non-manual markers, and signer consistency.

Datasets like NCSLGR~\cite{neidle2012new}, ASLLRP\cite{neidleboston}, and DSP~\cite{neidle2022asl}, resulting from collaborations among multiple universities, as well as the How2Sign~\cite{duarte_how2sign_2021} dataset collected with higher resolution cameras, offer more comprehensive data. These datasets include English sentences with corresponding written representations, detailed annotation conventions (\eg \cite{neidle2001signstream,neidle2007signstream}), and videos featuring both continuous and citation-form signs. These advancements have allowed some of these datasets, such as NCSLGR and How2Sign datasets, to be used as benchmarks for ASL processing research (\eg~\cite{zhu_neural_2023,moryossef_data_2021,baltatzis2024neural}). While these datasets address some of the critical gaps in earlier resources, issues such as their relatively small sizes (\eg \cite{neidle2012new,neidle2022asl}), inconsistent annotation conventions across datasets, and limited accessibility of the DSP and How2Sign gloss datasets make some tasks of ASL processing both promising and challenging. 

\section{Module 1: English Text-to-ASL Gloss}
\subsection{Data Preprocessing}\label{appendix:data_prep}
\paragraph{Step 1: Data Extraction} We obtained the ASLLRP dataset from the project web interface\footnote{DAI 2: \url{https://dai.cs.rutgers.edu/dai/s/cart}, login required.}. The dataset includes ASL sentence-level signed videos and XML files\footnote{These XML files are generated from the SignStream annotation tool. More details about these files can be found here: \url{https://dai.cs.rutgers.edu/downloads/XML-Export-format.pdf}.} containing corresponding English translations and annotations. For the translation task in Module 1, we focused on extracting manual information from the textual annotations to capture the primary meaning of the English translations. Specifically, we extracted existing English sentences from the XML files and systematically spliced English-based annotations, including vocabulary and compound symbols, fingerspelling, name signs, classifiers, locative words, and gestures, in chronological order. In total, we extracted 2,119 English sentences with corresponding English-based glosses. Additionally, we trimmed the signing videos based on the XML data so that each English sentence corresponds to a specific sign language video (utterance) for our subsequent tasks.

\paragraph{Step 2: Data Cleaning} Following a similar approach to prior work~\cite{amin_sign_2021}, we removed gloss annotations that did not alter the overall meaning of the sentences when omitted, such as repetition (annotated as a single or multiple ``+'' signs), number of signing hands (annotated as ``(1h)'' and ``(2h)''), and signs indicator that both hands move in an alternating manner (annotated as ``alt.''). To reduce translation errors, we standardized all fingerspelling-related glosses from fs-XXX to fs-X-X-X (\eg from ``fs-JOHN'' to ``fs-J-O-H-N'') and unified annotations for spatial locations (\eg ``i:GIVE:j'' and ``i:GIVE:k'' were standardized to ``i:GIVE:j''). While classifiers play a crucial role in ASL, we excluded them from this work because they typically appear only once or very few times in the datasets, so there was insufficient data for effective model prompting. After data cleaning, we retained 1,843 English sentences with corresponding English-based glosses for the remaining experiments. 

\paragraph{Step 3: Text-to-Gloss Dictionary Generation} To improve consistency in sign representations across different sentences and datasets, we constructed a text-to-gloss dictionary using the ASLLRP Sign Bank\footnote{\url{https://dai.cs.rutgers.edu/dai/s/signbank}}, which contains isolated signs along with their corresponding English-based glosses and translations. We then systematically unified the glosses based on step 2 to ensure consistency between the dictionary and the gloss annotations for the sentences. During the dictionary generation, we observed that some words may have variants of glosses depending on the context (\eg ``ask, inquire, query, question'' can be annotated as ``ASK'', ``ASK:i'', or ``i:ASK:j'', depending on whether the previous and following words are signed in a neutral location). Therefore, our dictionary employs a one-to-multiple mapping, accommodating the variability in gloss annotations. In total, the dictionary contains 3,915 text-to-gloss pairs. Notably, we identified 43 words that do not have corresponding glosses (\ie out-of-vocab words). For these words, which lack corresponding videos, fingerspelling is used as an alternative. 

\paragraph{Step 4: Ground True Correction} During the process of extracting ground truth from XML files to determine whether a sentence is a yes/no question, wh- question, conditional statement, and/or contains negation, we discovered that the ground truth labels were based on the signing rather than the English text, leading to some misalignments between the English text and the linguistic labels. For example, ``I guarantee that the parents will be mad if the children dye their hair orange'' was originally labeled as a negation statement, because the signing of it contains negation, although the English sentence does not. To address these issues, four of our researchers iteratively re-labeled and discussed the test set sentence categories, refining the labels to better reflect the text content. These revised labels were then used as the ground truth, allowing us to calculate precision and recall for each sentence type predictions and to identify patterns in the model's errors.

\subsection{ASL Grammar Guidelines for LLM Prompt}\label{asl_grammar_rules}
% \begin{minted}[
%     fontsize=\small,
%     linenos=true,          % Add line numbers
%     breaklines=true,       % Enable line wrapping
%     breakanywhere=true,    % Allow breaks at any point
%     frame=single,           % Add a border around the text
%     framesep=5pt,     
%     breaksymbolleft={}     % Remove the return icon on line breaks
% ]{text}


\lstset{
  % frame=top,frame=bottom,
% frame=,  
  % basicstyle=\small,    % the size of the fonts that are used for the code
  basicstyle=\small\normalfont\sffamily,    % the size of the fonts that are used for the code
  % stepnumber=1,                           % the step between two line-numbers. If it is 1 each line will be numbered
  numbersep=5pt,                         % how far the line-numbers are from the code
  tabsize=2,                              % tab size in blank spaces
  % extendedchars=true,                     %
  breaklines=true,                        % sets automatic line breaking
  % captionpos=t,                           % sets the caption-position to top
  % mathescape=true,
  % stringstyle=\color{white}\ttfamily, % Farbe der String
  stringstyle=\ttfamily,
  showspaces=false,
  showtabs=false,  
  % flexiblecolumns=true,
  % columns=[c]fixed
  linewidth=\linewidth,
  % breakanywhere=true,    % Allow breaks at any point
  % xleftmargin=17pt,
  % framexleftmargin=17pt,
  % framexrightmargin=17pt,
  % framexbottommargin=5pt,
  % framextopmargin=5pt,
  % showstringspaces=false
 }

\begin{lstlisting}[language={},breakindent=10pt]
American Sign Language (ASL) commonly uses a type of sentence structure called topicalization. Topicalization is when the topic of a sentence is placed at the beginning of the sentence. For instance, in English, the topicalized form of the sentence, "I see my friend" would be "My friend, I see them". This is often referred to in ASL as topic/comment structure. Any description of the topic, such as including adjectives, would also come before the comment. The sentence "I see a big orange cat" would be signed as follows: CAT ORANGE BIG IX-1p SEE.

As a very visual language, ASL often requires signers to visualize a sentence and arrange their signs accordingly. Sentences that involve cause-and-effect statements, real-time sequencing, or general-to-specific details follow a specific pattern. Cause-and-effect sentences in ASL tend to place the cause before the effect in the sentence. For example, in the statement "I feel calm when I go to the park", the cause of "go to the park" would be expressed before the effect of "I feel calm". The sentence would be signed as: PARK GO-TO FEEL CALM ME.

Some sentences involve real-time sequencing, where events must be arranged in chronological
order according to how they happened in real time. For instance, the sentence "I'm worried
because my brother didn't call me after he left" would be rearranged as: POSS-1p BROTHER LEAVE CALL-BY-PHONE-1p NOT CONCERN IX-1p.

In sentences where a signer is setting a scene, the signer should move from general to specific
details. For example, in the statement "I am excited after moving to my new house in Virginia",
the signer would begin with the biggest detail ("Virginia") and work their way down to the
smallest detail ("I"). The sentence would be signed as: VIRGINIA HOUSE NEW MOVE FINISH EXCITED IX-1p.

Verbs are not conjugated based on tense in ASL, so every verb is in its base form. This means that "ate", "eats", "eating", and "eaten" are all expressed by the sign EAT. The tense is established separately by including a time indicator in the sentence. Time signs are usually placed at the beginning of the sentence, before the topic, which tells the watcher when the rest of the sentence takes place. Signers can also express tense using a sign that relates the progress of the activity, like in the image above, which uses the FINISH sign to indicate that the action is in the past and translates to "I saw".

Basic sentence structure in ASL follows the pattern of Time + Topic + Comment. The word order can change depending on the needs of the signer, but this is the most common format.
    - Time = Any necessary time indicators (establishes tense)
    - Topic = The main focus of the sentence (a noun)
    - Comment = What is being said about the topic (includes the verb)
For example,in English, one might say, "I went to the library yesterday." In ASL, the sentence might be structured like this: 
    - Time = YESTERDAY
    - Topic = LIBRARY
    - Comment = IX-1p GO-TO
As is the case with English sentence structure, sign choice and order often vary based on
context. The example above is shown in Object-Subject-Verb (OSV) order, in which the object
(the library) is the topic. However, the sentence can also be arranged in Subject-Verb-Object
(SVO) order, in which "I" is the topic and "GO-TO LIBRARY" becomes the comment: YESTERDAY IX-1p GO-TO LIBRARY

Both sentences are grammatically correct, and different factors can influence which structure the signer chooses, such as how familiar the watcher is with the library, and therefore what level of emphasis is needed.

When a question is asked in ASL, the WHO, WHAT, WHEN, WHERE, WHY, WHICH, or HOW sign is located at the end of the sentence, or if emphasis is needed, both the beginning and the end. This word order reflects topic/comment structure. For example, in English, one might ask, "What is your name?" In ASL, the sentence would be structured in this way: YOUR NAME WHAT

Additionally, while English often employs different forms of the verb "to be" in sentences, this
verb is not used in ASL and should not be included in signed conversations. 

When using negating signs in a sentence, such as NOT or NONE, the negative sign typically follows the word it is negating. For example, "I don't have any pets" would be signed as: PET HAVE NOT.
\end{lstlisting}
% \end{minted}

\subsection{Experiments on English Text-to-ASL Gloss}\label{appendix:llm_experiments}
\subsubsection{Model Selection} We experimented with various versions of GPT and tested multiple configurations to identify the optimal model. As shown in Table \ref{tab:llm_experiment_results}, GPT-4o-2024-05-13 (our adopted model) outperformed other GPT-4 variants under identical settings. Additionally, we fine-tuned two versions of GPT models capable of fine-tuning, but their performance was lower than that of few-shot prompting with the adopted model. However, fine-tuning GPT-4 models with larger datasets could hold promise, and exploring this option when the feature becomes more widely available may yield further improvements.

\subsubsection{Prompting Examples} For Module 1, we varied the prompts for the ``SYSTEM'' in different setups for the English Text-to-ASL Gloss task (depicted on the left side of Figure \ref{fig:module_1}), while maintaining consistency in the ``ASSISTANT'' and ``USER'' prompts. No additional prompt engineering was performed for generating linguistic information (task on the right side of Figure \ref{fig:module_1}). A summary of these setups is provided in Table \ref{tab:prompt_engineering}.


\subsection{Additional Experiments on English-to-Gloss Translation}\label{appendix:additional_text-to-gloss}
To enhance our translation capabilities, we implemented Retrieval Augmented Generation (RAG)~\cite{lewis2020retrieval} with anonymized embeddings. First, as a pre-process, we anonymized all train sentences by converting name references into pronouns. Next, we embedded the anonymized sentences using an OpenAI embeddings model. Finally, at inference, for each test sentence, we embedded it as well and look for the $N$ most similar examples to this sentence based on the cosine similarity between the embedding of the test example, and the embeddings of the anonymized train examples. This way, the model is presented with the most accurate and relevant examples. As Table~\ref{tab:text-to-gloss_RAG_eval_results_appx} shows, when using RAG the results are better than using all of the train examples. Moreover, using fewer examples and anonymized embeddings yields better results in most cases. The reason for using anonymization, is that names are given high weight in the embedding, which leads to less relevant examples in some cases. For examples, the 3 most similar sentence for the sentence "Which college did Mary go to?" before anonymization, are: "Which college does Mary go to?", "What did Mary's name used to be?", "Mary used to live in Boston.", While after anonymization they are: "Which college does Mary go to?", "Which high school did you go to?", "Where did you go to high school?", which are more relevant and similar examples.


\subsection{Summary of Existing Results}\label{appendix:existing_text-to-gloss_results}

Unlike German datasets such as RWTH-PHOENIX-Weather 2014T~\cite{camgoz_neural_2018} and the public DGS corpus~\cite{hanke2020extending}, which are widely used and frequently reported in the literature~\cite{chen2022two,yin2023gloss,li2020tspnet,saunders_progressive_2020}, there is comparatively less work utilizing ASL datasets. We summarize the existing translation results for ASL in Table~\ref{tab:text-to-gloss_sota}.




\section{Survey}\label{appendix:survey}
\begin{figure*}
\centerline{\includegraphics[width=0.7\linewidth]{images/survey/challenges.pdf}}
\caption{Participant responses to the question \emph{What are the biggest challenges with your current captioning or transcription device/technology? (select all that apply)?}}
\label{fig: survey-challenges}
\end{figure*}

\begin{figure*}
\centerline{\includegraphics[width=0.9\linewidth]{images/survey/scenarios_labels.pdf}}
\caption{Survey results of how often participants encountered challenging scenarios with today's transcription technology. The number of participants and percentage is shown for each choice.}
\label{fig: survey-scenarios}
\end{figure*}

\section{Challenges with mobile captioning: Large-Scale survey with 263 frequent users} \label{surveys_foundational}
As we were interested in exploring the potential for more advanced mobile speech perception, we conducted a brief large-scale survey to learn about the challenges of using captioning for speech understanding in in-person meetings and conversations. % and opportunities for more advanced mobile speech perception. 

\subsection{Participants}
We used Google Surveys \cite{GoogleSurveys} to deploy a survey to the general population in the US of all ages and genders (\emph{“Android users of the Google Opinion Rewards app”}), screening for individuals that use technology to understand speech in meetings and conversations, and are frequent users of captioning technology. Our goal was to recruit deaf or hard-of-hearing participants, as we believed that they would have the most relevant experience and insights around mobile captioning technology and interfaces. To mitigate spam, the survey system analyzes question response times. By considering the distribution of response times across questions, it adapts to different question types and response patterns, rejecting sessions with unusually fast responses.

We acknowledge that our survey's focus on frequent users of captioning technology and the growing user base for mobile captioning limits its relevance for other populations, such as individuals who identify with Deaf culture and might be less likely to rely on ASR technology \cite{deaf_community_questionnaires}. Unfortunately, we cannot quantify the representation in the survey since restrictions from our institution do not allow us to collect participant hearing levels or use of sign language. 

Of the 1502 respondents that met our criteria, we focus on the 263 participants (18\%), who reported that they used captioning technology to understand people (not TV/video) multiple times per week or more frequently, and for 2 hours or more on the days that they used it. For these 263 participants, the platform reported that 49.8\% were women, 48.7\% men, and 1.5\% unknown, across all age ranges (27\%: 18--24, 33\%: 25--34, 15\%: 35--44, 12\%: 45--54, 6\%: 55--64, 7\%: 65+). 

The participants were prompted to select challenges among the choices from the list in Figure~\ref{fig: survey-challenges}. The choices were synthesized by user feedback from our previous experience with mobile captioning and informed by previous work in mobile captioning~\cite{localization_glasses, wearable_subtitle}

\subsection{Survey results: The use of captioning to understand people in conversations}
64\% of the participants reported daily use of captions in meetings or conversations to understand people, whereas 36\% used it multiple times per week. Half of the participants (49\%) use technology to understand people face-to-face for 2-3 hours on the day of use. Almost a quarter (23\%) of participants use captions for 4-5 hours on the days of use and another quarter (28\%) for 6 or more hours. Real-time captions, such as CART (69\%), and the \emph{Android Live Transcribe and Sound Notifications} app (55\%) were the top two technologies that were used daily to understand people.  %See Figure \ref{fig: survey-usage}.

% \begin{figure}
% \centerline{\includesvg[width=0.75\columnwidth]{images/survey/usage.svg}}
% \caption{How often do you use captions/transcriptions in meetings/conversations to understand people? (e.g., CART, Live Transcribe - not including closed captions for TV/video)}.
% \label{fig: survey-usage}
% \end{figure}

The top two issues with current transcription technology, as reported by our participants, were background noise (60\%) and the combining of text from different speakers (46\%), without the ability to separate them. Participants selected all that apply from the choices shown in Figure ~\ref{fig: survey-challenges}.

Finally, we asked participants about scenarios that are known to be challenging with today's transcription technology but have the potential to be addressed with more advanced microphone arrays and speech perception algorithms. Scenarios of interest included conversations where ignoring music, noise, or adjacent speech would be critical. We were also interested in group conversations and situations where separating speech from two people is critical. 68-70\% of participants experienced these scenarios multiple times per week or more frequently, whereas only 11-12\% rarely or never experienced them, as shown in Figure \ref{fig: survey-scenarios}.

% 38-48\% of participants experienced these scenarios daily, and 25-29\% experienced them multiple times per week. 

\subsection{Discussion}

Our large-scale survey enabled us to identify essential challenges with current transcription technology for frequent users of captioning and shows that 68-70\% of the participants are frequently in situations that today's technology cannot adequately support. The findings suggest that more advanced speech technology for suppressing noise from adjacent speakers, music, or noise could help address those issues and that speech separation technology can potentially improve group conversations through more readable transcripts. 


\onecolumn
\begin{table*}[t]
\caption{Existing ASL datasets. SL stands for sign language. ``-'' represents relevant information was not provided. ``Unknown'' represents relevant information was not found. \label{tab:asl_datasets}}
\renewcommand{\arraystretch}{1.1}
   \resizebox{\textwidth}{!}{
\begin{tabular}{L{0.15\textwidth}|R{0.06\textwidth}|R{0.06\textwidth}|R{0.06\textwidth}| C{0.1\textwidth}| L{0.12\textwidth}|L{0.45\textwidth}|C{0.15\textwidth}}
% {l|c|c|c|c|c}
\toprule\hline
\multicolumn{1}{c|}{{\textbf{\makecell[c]{Dataset}}}} & \multicolumn{1}{c|}{{\textbf{\makecell[c]{Vocab.}}}}  &
\multicolumn{1}{c|}{{\textbf{\makecell[c]{Hours}}}} &  \multicolumn{1}{c|}{{\textbf{\makecell[c]{Signers}}}} & \multicolumn{1}{c|}{{\textbf{\makecell[c]{Resolutions \\ (pixels)}}}} & \multicolumn{1}{c|}{{\textbf{\makecell[c]{Modalities}}}} &\multicolumn{1}{c|}{{\textbf{\makecell[c]{Gloss Labeling Standard}}}} & \multicolumn{1}{c}{{\textbf{\makecell[c]{Annotation \\ Tools}}}}  \\ \hline
% \multicolumn{6}{c}{\textbf{Utterance-level American Sign Language}} \\
RWTH-BOSTON-50~\cite{zahedi2005combination} & 50 & >9 & 3& 195 $\times$ 165 & Video, word & - & - \\\hline
Purdue RVL-SLLL~\cite{martinez2002purdue} & 104 & 14  &14 & 640 $\times$ 480 & Video, Gloss & Glosses include manual English-based labels, and non-manual behaviors such as handshapes and motions for two hands.  & Human Annotator \\\hline
RWTH-BOSTON-400~\cite{dreuw2008benchmark} & 483 & - & 5& 648 $\times$ 484 & Video, Gloss, Utterance & Glosses include manual English-based labels and non-manual behaviors, both anatomical (\eg raised eyebrows) and functional (\eg wh-questions). Glosses do not include handshape annotations. & SignStream$^@$2~\cite{neidle2001signstream} \\\hline
MS-ASL~\cite{joze2018ms} & 1K & 24 & 222  & 224 $\times$ 224 & Video, Pose, Word & Glosses were generated by referencing ASL Tutorial books~\cite{zinza2006master,caselli2017asl}. & Human Annotator \\\hline
DSP~\cite{neidle2022asl}& >1.7K & - & 15 & - & Video, Gloss, Utterance, Word & Glosses include manual English-based gloss labels, sign type, start and end handshapes (both hands), grammatical markers (\eg questions, negation, topic/focus, conditional, relative clauses), and anatomical behaviors (\eg head nods/shakes, eye aperture, gaze). & SignStream$^@$3~\cite{neidle2017user} \\\hline
NCSLGR~\cite{neidle2012new}& 1.8K & 5.3   & 4 & - & Video, Gloss, Utterance & Glosses include manual English-based labels and non-manual behaviors, both anatomical (\eg raised eyebrows) and functional (\eg wh-questions). Glosses do not include handshape annotations. & SignStream$^@$2~\cite{neidle2001signstream}\\\hline
ASLLRP~\cite{neidle2022asl} & >2.7K & 3.6   & 4 & - & Video, Gloss, Utterance, Word & Glosses include manual English-based gloss labels, sign type, start and end handshapes (both hands), grammatical markers (\eg questions, negation, topic/focus, conditional, relative clauses), and anatomical behaviors (\eg head nods/shakes, eye aperture, gaze). & SignStream$^@$3~\cite{neidle2017user}\\\hline
ASL Citizen~\cite{desai2024asl}& >2.7K & 30.5 & 52 & - & Video, Gloss, Pose, Word & Glosses include manual English-based labels by referencing a lexical database of ASL (\ie ASL-LEX~\cite{caselli2017asl}).  & Unknown \\\hline
Signing Savvy~\cite{signingsavvy} & >13K & - & - & -  &Video, Gloss, Utterance, Word & Glosses include manual English-based labels. & Unknown \\\hline
How2Sign~\cite{duarte_how2sign_2021}& 16K & 80  &  11 & 1280 $\times$ 720 & Video, Pose, Gloss, Utterance, Speech & Glosses include English-based labels, but do not include information such as hand-shape, hand movement/orientation, and facial expressions, such as raised eyebrows in yes/no questions. & ELAN~\cite{crasborn2008enhanced} \\\hline
% ASLG-PC12~\cite{othman2012english}& 24,002,570 Utterances in total, more than 800 million words in English. & Gloss, Utterance &  & - (from books but not human) & & Signs are in small caps. Lexicalized finger-spelled words use ``\#'' before small caps. Full finger-spelling uses dashes between small caps (\eg A-C-R-A-F). Non-manual signals and eye-gaze are shown above the glosses. & Not specify \\\hline
OpenASL~\cite{shi2022open} & 33K & 288 &  220 & - & Video, Utterance & - & - \\\hline
\bottomrule

\end{tabular}}
\end{table*}
\section{Gloss Annotation Conventions}
\small
\begin{longtable}{p{0.15\textwidth}|p{0.14\textwidth}|p{0.22\textwidth}|p{0.4\textwidth}}
    \toprule\hline
     \multicolumn{1}{c|}{\textbf{Category}} &  \multicolumn{1}{c|}{\textbf{Gloss}} &  \multicolumn{1}{c|}{\textbf{Example}} &  \multicolumn{1}{c}{\textbf{Explanation}} \\
    \hline
    \multirow{4}{*}{English-based glosses} & \multirow{2}{*}{-} & OH-I-SEE & \multirow{2}{6cm}{Used to separate words if the English translation of a single sign requires more than one.} \\\cline{3-3}
    &  & THANK-YOU &  \\\cline{2-4}
    & \multirow{2}{*}{/} & BOLD/TOUGH & \multirow{2}{6cm}{Used when one sign has two different English equivalents.} \\\cline{3-3}
    &  & THANK-YOU &  \\\hline
    \multirow{2}{*}{Fingerspelling} & fs- & fs-J-O-H-N & Fingerspelled word. \\\cline{2-4}
    & $\#$ & $\#$EARLY & Fingerspelled loan sign. \\\hline
    Name Signs & ns- & ns-PARIS & Used for names of places (\eg Paris). \\\hline
    \multirow{2}{*}{Compounds} & \multirow{2}{*}{+} & \multirow{2}{*}{MOTHER+FATHER} & A type of sign formation where two or more signs are joined to create a new sign with a distinct meaning (\eg ``parent''). \\\hline
    Phonological issues & QMwg & FRIEND FINISH DRIVE QMwg & Question marking sign (with wiggling) \\\hline
    \multirow{7}{2.5cm}{Subject and object verb agreement } & \multirow{3}{*}{i:GLOSS:j} & \multirow{2}{*}{i:GIVE:j} & ``i'' and ``j'' designate unique spatial locations associated with the subject and object referents. \\\cline{3-4}
    &  & 1p:GIVE:2p & ``(I) give (you)...''  \\\cline{2-4}
    & Noun & fs-J-O-H-N i:GIVE:j & John is signed in a neutral location. \\\cline{2-4}
    & \multirow{3}{*}{Noun:i} & \multirow{3}{*}{fs-J-O-H-N:i i:GIVE:j} & John is signed in the location associated with the referent (the same location with which the verb displays manual subject-verb agreement). \\\hline
    \multirow{6}{2.5cm}{Agreement marking on adjectives, nouns, pronouns, determiners, possessives, and emphatic reflexives} & \multirow{6}{*}{\makecell[l]{Pronoun\\ IX-[person]:i \\ \\ Determiner\\ IX-3p:i \\ }} & IX-1p & 1st person pronoun \\\cline{3-4}
    &  & POSS-1p & 1st person possessive marker \\\cline{3-4}
    &  & SELF-1p & 1st person emphatic reflexive marker (as in ``I did it myself'') \\\cline{3-4}
    &  & IX-2p & Pronoun referring to addressee. \\\cline{3-4}
    &  & POSS-2p & Possessive marker referring to addressee. \\\cline{3-4}
    &  & SELF-2p & Emphatic reflexive marker referring to addressee. \\\cline{3-4}
    & \multirow{6}{2.5cm}{\makecell[l]{Possessive\\ POSS-[person]:i \\ \\ Emphatic reflexive\\ SELF-[person]:i}} & \multirow{2}{6cm}{IX-3p:i} & Pronoun or determiner referring to singular third person referent associated with location ``i''. \\\cline{3-4}
    &  & \multirow{2}{6cm}{POSS-3p:i} & Possessive marker referring to singular third person referent associated with location ``i''. \\\cline{3-4}
    &  & \multirow{2}{6cm}{SELF:i} & Emphatic reflexive marker referring to a singular third person referent associated with location ``i''. \\\cline{2-4}
    & \multirow{2}{*}{-} & \multirow{2}{6cm}{THUMB-IX-3p:i} & Pronoun referring to singular third person referent associated with location ``i'' articulated with the thumb. \\\hline
    \multirow{6}{2.5cm}{Adverbials of location and direction} & \multirow{6}{2cm}{\makecell[l]{Adverbial \\ IX-loc:[location] \\ IX-dir:[direction]}} & {IX-loc:i} & Adverbial produced with index finger pointing to location ``i''. \\\cline{3-4}
    &  & IX-loc"under table" & \multirow{5}{*}{Adverbial with location described.} \\\cline{3-3}
    &  & IX-dir"around the corner to the right" &  \\\cline{3-3}
    &  & IX-loc"far" &  \\\cline{3-3}
    &  & THUMB-IX-loc"behind" &  \\\hline
    \multirow{13}{2.5cm}{Singular vs. plural}& \multirow{7}{2cm}{\makecell[l]{IX-[person]-[num]:i/j }} & IX-3p-pl-2:x/y & \multirow{2}{6cm}{Third person pronoun referring to the 2 (or 3) referents: x, y (or z).} \\\cline{3-3}
    &  & IX-3p-pl-3:x/y/z &  \\\cline{3-4}
    &  & \multirow{2}{2.9cm}{IX-1p-pl-2:x} &  First person pronoun referring to singer plus the referent associated with the location ``i''.\\\cline{3-4}
    &  & \multirow{2}{2.9cm}{IX-2p-pl-2:x} & Second person pronoun referring to addressee plus the two referents associated with locations ``x'' and ``y''. \\\cline{2-4}
    & \multirow{6}{2.2cm}{-3p-pl-arc} & IX-3p-pl-arc & \multirow{3}{6cm}{Pronoun (or possessive or emphatic reflexive) referring to singular third person referent associated with location ``i'' articulated with the thumb.} \\\cline{3-3}
    &  & POSS-3p-pl-arc &  \\\cline{3-3}
    &  & SELF-3p-pl-arc &  \\\cline{3-4}
    &  & \multirow{1}{2.9cm}{1p:GIVE-3p-arc} & \makecell[l]{``I give (it) to them.'' \\ Subject agreement is 1st person. Object agreement (the end \\ point of the sign) is plural (an arc).} \\\cline{2-4}
    & -loc-arc & IX-loc-arc & Adberbial (``there'') using an arc to designate locations. \\\hline
    \multirow{2}{2.5cm}{Reduplicative aspect marking}& Gloss-aspect & STUDY-continuative & Aspectual inflections are indicated following the gloss. \\\cline{2-4}
    & Gloss-aspect(:i) & GIFT-distributive:i & ``(they) each gave (one person)...'' \\\hline
    \multirow{2}{2.5cm}{Reciprocal inflection} & \multirow{2}{2cm}{GLOSS-recip} & \multirow{2}{6cm}{LOOK-AT-recip:i,j} & The referents associated with locations ``i'' and ``j'' look at each other. \\\hline
    \bottomrule
\end{longtable}\label{tab:gloss_convention}


\begin{table*}[htb!]
\footnotesize
\centering
\caption{Experimental results for different setups of English text-to-ASL gloss translation. Note: For ``Fine-tuning,'' the model was not constrained to the word-to-gloss dictionary vocabulary, unlike in few-shot prompting.
\Description[]}\label{tab:llm_experiment_results}
\renewcommand{\arraystretch}{1.1}
 \resizebox{\textwidth}{!}{
\begin{tabular}
{C{0.15\textwidth}|C{0.15\textwidth}|C{0.1\textwidth}|C{0.2\textwidth}|C{0.15\textwidth}}
\toprule\hline
\multicolumn{1}{c|}{{\textbf{\makecell[c]{Model}}}} &\multicolumn{1}{c|}{{\textbf{\makecell[c]{Training Method}}}} &\multicolumn{1}{c|}{{\textbf{\makecell[c]{Limited Vocab}}}} &\multicolumn{1}{c|}{{\textbf{\makecell[c]{Number of Examples}}}} & \multicolumn{1}{c}{{\textbf{\makecell[c]{BLEU-4 \bm{$\uparrow$}}}}}  \\\hline 
GPT-2 & Fine-tuning & - & 1474 (80\% of the entire dataset) & <0.000 \\\hline
\multirow{2}{*}{GPT-3.5-turbo-0125} & Few-shot prompting & No & 100 & 0.102\\\cline{2-5}
 & Fine-tuning & - & 1474 (80\% of the entire dataset) & 0.161 \\\hline
\multirow{2}{*}{GPT-4-turbo-2024-04-09} & \multirow{2}{*}{Few-shot prompting} & \multirow{2}{*}{No} & 100 & 0.115 \\\cline{4-5}
&  &  & 300 & 0.145 \\\hline
\multirow{3}{*}{GPT-4-0125-preview}& \multirow{3}{*}{Few-shot prompting} & \multirow{2}{*}{No} & 100 & 0.117 \\\cline{4-5}
&  &  & 300 & 0.143 \\\cline{3-5}
&  & Yes & 300 & 0.176 \\\hline
\multirow{3}{*}{\makecell[c]{GPT-4o-2024-05-13 \\ (Our adopted model)}} & \multirow{3}{*}{Few-shot prompting} & \multirow{2}{*}{No} & 100 & 0.133 \\\cline{4-5}
&  &  & 300 & 0.173 \\\cline{3-5}
&  & \textbf{Yes} & \textbf{300} & \textbf{0.226} \\\hline
 \bottomrule
\end{tabular}}
\end{table*}


\begin{table*}[htb!]
\footnotesize
\centering
\caption{Prompts for different setups.
\Description[]}\label{tab:prompt_engineering}
\renewcommand{\arraystretch}{1.1}
 \resizebox{\textwidth}{!}{
\begin{tabular}
{C{0.1\textwidth}|C{0.1\textwidth}|L{0.6\textwidth}}
\toprule\hline
\multicolumn{1}{c|}{{\textbf{\makecell[c]{Limited Vocab}}}} &\multicolumn{1}{c|}{{\textbf{\makecell[c]{Grammar Rules}}}} &\multicolumn{1}{c}{{\textbf{\makecell[c]{Prompts}}}}\\\hline 
 \multirow{3}{*}{No} & No & You are an ASL translator. Your task is to translate an English sentence to an ASL gloss format. \\\cline{2-3}
  & Yes & You are an ASL translator. Your task is to translate an English sentence to an ASL gloss format. First, familiarize yourself with the following ASL grammar rules: \textcolor{RoyalBlue}{\textsf{GRAMMER$\_$RULES}}. \\\hline
 \multirow{5}{*}{Yes} & No & You are an ASL translator. Your task is to translate an English sentence into ASL gloss format. First, familiarize yourself the following vocabulary dictionary: \textcolor{RoyalBlue}{\textsf{TEXT$\_$TO$\_$GLOSS$\_$DICTIONARY}}.  \\\cline{2-3}
  & Yes & You are an ASL translator. Your task is to translate an English sentence into ASL gloss format. First, familiarize yourself with the following ASL grammar rules: \textcolor{RoyalBlue}{\textsf{GRAMMER$\_$RULES}}. Also, review the following vocabulary dictionary: \textcolor{RoyalBlue}{\textsf{TEXT$\_$TO$\_$GLOSS$\_$DICTIONARY}}.  \\\hline
 \bottomrule
\end{tabular}}
\end{table*}

\begin{table*}[htb!]
\caption{Evaluation results of translating English text into glosses (Task on the left side in Module 1) using RAG. \bm{$^*$}All BLEU-4 and SacreBLEU scores are identical. \bm{$\uparrow$} indicates that higher values represent better performance, while \bm{$\downarrow$} indicates that lower values represent better performance. Best results in \textbf{bold}. The presented results are $mean\pm std$ across 10 repetitions of test set evaluation. Note: If ``Anonymized Embeddings'' is set to ``No'', RAG was performed using embeddings of the original data, else, it was performed using embeddings of the anonymized data.
\Description[This table presents evaluation results of translating English text to English-based glosses. The first row contains ten headers, including number of examples, anonymized embeddings, bleu-1, bleu-2, bleu-3, bleu-4, rouge-l, meteor, chrf, and ter. Regarding the columns, the first column shows number of used examples, where the first row is for using 1474 (all examples -not using RAG) and the other rows are for using 200, 100, and 50 examples. The second column, anonymized embeddings, indicates whether anonymized embeddings are used (yes or no). The remaining columns display various evaluation metrics with the symbol uparrow indicating higher values are better, and downarrow indicating lower values are better. Key findings are: presenting the model with less examples that are more relevant yields better results than using all examples. Moreover, in most cases using less examples (n=50) gives better results than using more examples (e.g. 100, 200). However, for some metrics (ROUGE, TER) using more examples gives better results. The best scores are bolded across all metrics, indicating the optimal settings for this translation model.]}\label{tab:text-to-gloss_RAG_eval_results_appx} 

\renewcommand{\arraystretch}{1.1}
 \resizebox{\textwidth}{!}{
\begin{tabular}{c | c | c c c c c c c c }
\toprule\hline
\multicolumn{1}{c|}{{\textbf{\makecell[c]{Number of \\ Examples}}}} &\multicolumn{1}{c|}{{\textbf{\makecell[c]{Anonymized \\ Embeddings}}}} & \multicolumn{1}{c}{{\textbf{\makecell[c]{BLEU-1 \bm{$\uparrow$}}}}} &\multicolumn{1}{c}{{\textbf{\makecell[c]{BLEU-2 \bm{$\uparrow$}}}}} & \multicolumn{1}{c}{{\textbf{\makecell[c]{BLEU-3 \bm{$\uparrow$}}}}}
& \multicolumn{1}{c}{{\textbf{\makecell[c]{BLEU-4\bm{$^*$} \bm{$\uparrow$}}}}} 
& \multicolumn{1}{c}{{\textbf{\makecell[c]{ROUGE-L \bm{$\uparrow$}}}}} 
& \multicolumn{1}{c}{{\textbf{\makecell[c]{METEOR \bm{$\uparrow$}}}}} 
& \multicolumn{1}{c}{{\textbf{\makecell[c]{CHrF \bm{$\uparrow$}}}}} & \multicolumn{1}{c}{{\textbf{\makecell[c]{TER \bm{$\downarrow$}}}}} \\\hline 
% 1,474 (All - no RAG)& - & $0.516\pm0.002$ & $0.392\pm0.002$ & $0.307\pm0.001$ & $0.244\pm0.001$ & $0.644\pm0.001$ & $0.534\pm0.001$ & $0.524\pm0.002$ & $0.538\pm0.002$ \\ \hline
 \multirow{2}{*}{\makecell[c]{200}} & No & $0.562\pm0.003$ & $0.433\pm0.003$ & $0.345\pm0.003$ & $0.278\pm0.003$ & \bm{$0.669\pm0.001$} & $0.56\pm0.001$ & $0.557\pm0.002$ & \bm{$0.52\pm0.003$}\\
 & Yes & \bm{$0.569\pm0.003$} & \bm{$0.437\pm0.004$} & $0.347\pm0.004$ & $0.278\pm0.003$ & $0.668\pm0.002$ & $0.559\pm0.006$ & \bm{$0.559\pm0.001$} & \bm{$0.52\pm0.002$} \\ \hline
 \multirow{2}{*}{\makecell[c]{100}} & No & $0.563\pm0.003$ & $0.433\pm0.003$ & $0.345\pm0.002$ & $0.278\pm0.003$ & $0.663\pm0.003$ & $0.556\pm0.003$ & $0.557\pm0.001$ & $0.522\pm0.004$\\
 & Yes & $0.567\pm0.003$ & \bm{$0.437\pm0.003$} & $0.345\pm0.002$ & \bm{$0.279\pm0.003$} & $0.666\pm0.002$ & $0.562\pm0.002$ & \bm{$0.559\pm0.002$} & $0.523\pm0.002$\\ \hline
 \multirow{2}{*}{\makecell[c]{50}} & No & $0.563\pm0.003$ & $0.432\pm0.003$ & $0.342\pm0.004$ & $0.275\pm0.005$ &  $0.663\pm0.002$ & $0.557\pm0.003$ & $0.554\pm0.003$ & $0.525\pm0.006$ \\
 & Yes & \bm{$0.569\pm0.003$} & \bm{$0.437\pm0.003$} & \bm{$0.348\pm0.003$} & \bm{$0.279\pm0.003$} & $0.667\pm0.002$ & \bm{$0.564\pm0.002$} & $0.558\pm0.002$ & $0.523\pm0.002$ \\ \hline %\cline{3-8}
 \bottomrule
\end{tabular}
}
\end{table*}


% 	     N=10 (a)	    N=50	   N=50 (a)	   N=100	  N=100 (a)	   N=200	 N=200 (a)	All (No RAG)
% BLEU-1	0.567±0.004	0.563±0.003	0.569±0.003	0.563±0.003	0.567±0.003	0.562±0.003	0.569±0.003	0.516±0.002
% BLEU-2	0.434±0.005	0.432±0.003	0.437±0.003	0.433±0.003	0.437±0.003	0.433±0.003	0.437±0.004	0.392±0.002
% BLEU-3	0.345±0.005	0.342±0.004	0.348±0.003	0.345±0.002	0.348±0.004	0.345±0.003	0.347±0.004	0.307±0.001
% BLEU-4	0.276±0.004	0.275±0.005	0.279±0.003	0.278±0.003	0.279±0.003	0.278±0.003	0.278±0.003	0.244±0.001
% ROUGE-L	0.663±0.007	0.663±0.002	0.667±0.002	0.663±0.003	0.666±0.002	0.669±0.001	0.668±0.002	0.644±0.001
% METEOR	0.558±0.007	0.557±0.003	0.564±0.002	0.556±0.003	0.562±0.002	0.56±0.001	0.559±0.006	0.534±0.001
% chrF	0.555±0.005	0.554±0.003	0.558±0.002	0.557±0.001	0.559±0.002	0.557±0.002	0.559±0.001	0.524±0.002
% TER	    0.53±0.011	0.525±0.006	0.523±0.002	0.522±0.004	0.523±0.002	0.52±0.003	0.52±0.002	0.538±0.002
% no RAG limited vocab

% {"BLEU1_mean": 0.5159437806050521, "BLEU1_std": 0.003454672572734358, "BLEU2_mean": 0.38897016118052236, "BLEU2_std": 0.0028731244740485724, "BLEU3_mean": 0.30530892012246635, "BLEU3_std": 0.0025373167689406537, "BLEU4_mean": 0.24224145504637956, "BLEU4_std": 0.002550512811054208, "ROUGE-L_mean": 0.6474144359271453, "ROUGE-L_std": 0.0021641753310813203, "METEOR_mean": 0.5282678162580199, "METEOR_std": 0.0014865716129251368, "chrF_mean": 0.5248349525557509, "chrF_std": 0.0018704591429950929, "TER_mean": 0.5358369098712447, "TER_std": 0.001581784290951081}

% RAG not anonymized n=100
	
% BLEU1_mean	0.5455374812911825
% BLEU1_std	0.021259150530312045
% BLEU2_mean	0.41598934338640714
% BLEU2_std	0.019446169531874013
% BLEU3_mean	0.3296644900350907
% BLEU3_std	0.017469579545021832
% BLEU4_mean	0.2644840841895223
% BLEU4_std	0.015922169361690355
% ROUGE-L_mean	0.6591913965881979
% ROUGE-L_std	0.008720951711266689
% METEOR_mean	0.547730117803161
% METEOR_std	0.014436450520054931
% chrF_mean	0.5453822654767951
% chrF_std	0.014687713184110024
% TER_mean	0.5291010968049595
% TER_std	0.005350069834582146

% RAG anonymized n=100

% BLEU1_mean	0.5720393921577391
% BLEU1_std	0.00274117658610426
% BLEU2_mean	0.44054559856394065
% BLEU2_std	0.002577262554280309
% BLEU3_mean	0.35171660482465006
% BLEU3_std	0.0025317104604301987
% BLEU4_mean	0.28517372714271527
% BLEU4_std	0.002508105049942758
% ROUGE-L_mean	0.6679973048001264
% ROUGE-L_std	0.001586072330475446
% METEOR_mean	0.5637764829568743
% METEOR_std	0.002436934889019184
% chrF_mean	0.562076679578778
% chrF_std	0.0015521067803102827
% TER_mean	0.5126251788268956
% TER_std	0.0011539532049286017

% RAG anonymized n=50

% BLEU1 mean 0.5706730298296665, std 0.0011395024907270722
% BLEU2 mean 0.43994092422892345, std 0.0012661254966549753
% BLEU3 mean 0.3512897656913279, std 0.0015775761215516147
% BLEU4 mean 0.283507833656901, std 0.0019545845771352837
% ROUGE-L mean 0.6647126531157136, std 0.001713237054159925
% METEOR mean 0.5665425039542302, std 0.0016266401435384072
% chrF mean 0.5553272316766698, std 0.0012587247293627947
% TER mean 0.5271816881258942, std 0.0024151563685456553

% RAG not anonymized n=50

% BLEU1_mean0.5397737934228959
% BLEU1_std0.024017866156022007
% BLEU2_mean0.41089801678747245
% BLEU2_std0.02207836952282585
% BLEU3_mean0.3247356541689541
% BLEU3_std0.01956381851690021
% BLEU4_mean0.2594108569315144
% BLEU4_std0.01732293687764026
% ROUGE-L_mean0.6573590939942247
% ROUGE-L_std0.010103468997267644
% METEOR_mean0.5452883105095815
% METEOR_std0.017077434675839744
% chrF_mean0.541194397856029
% chrF_std0.01642813968856003
% TER_mean0.5318669527896995
% TER_std0.0043273152809346896

% Example:
% Test English sentence: "If John stays home it probably means he's sick." 

% Original most-similar sentences:

% ["If John isn't here, does anyone know where he went?", 
% 'If Mary gets home late, John will probably be upset.', 
% 'Maybe John will live here.', 
% 'If John eats spinach, he will get sick.', 
% "John isn't here. Does anyone know where he is?", 
% 'Maybe John will move here.', 
% 'If it\'s raining, I"m staying home.', 
% 'I hope John arrives on time next week.', 
% "If it's raining, I stay home.", 
% 'If John eats spinach, he will get nauseous and sick.', 
% 'Everyone knows that John left to drive to Maine.', 
% 'John works nights.', 
% "Jack lied. He's not sick.", 
% 'Who is sick?', 
% "If it's raining tomorrow, I'll stay home.", 
% 'What did John decide yesterday?']

% Anonymized most-similar sentences:

% ['If it\'s raining, I"m staying home.', 
% "If it's raining, I stay home.", 
%  'Who is sick?', 
%  "If it's raining tomorrow, I'll stay home.", 
%  'I stay home when it rains.', 
%  'Mom is sick.', 
%  'Is mom sick?', 
%  'If Mother is sick, I have to take care of her.', 
%  'When a bad storm comes to Boston, I will stay home.', 
%  "If it rains tomorrow, that means that when I arrive at my uncle's house, I will stay inside.", 
%  'Is Mother sick?', 
%  "Mother isn't sick today.", 
%  "He/she will go to school if it's not cancelled.", 
%  "he lied. He's not sick.", 
%  'If she gets home late, he will probably be upset.', 
%  "If it rains tomorrow when I arrive at my uncle's house, I will stay inside."]

% De-Anonymized most-similar sentences:

% ['If it\'s raining, I"m staying home.', 
% "If it's raining, I stay home.", 
% 'Who is sick?', 
% "If it's raining tomorrow, I'll stay home.", 
% 'I stay home when it rains.', 
% 'Mom is sick.', 
% 'Is mom sick?', 
% 'If Mother is sick, I have to take care of her.', 
% 'When a bad storm comes to Boston, I will stay home.', 
% "If it rains tomorrow, that means that when I arrive at my uncle's house, I will stay inside.", 
% 'Is Mother sick?', 
% "Mother isn't sick today.", 
% "He/she will go to school if it's not cancelled.", 
% "Jack lied. He's not sick.", 
% 'If Mary gets home late, John will probably be upset.', 
% "If it rains tomorrow when I arrive at my uncle's house, I will stay inside."] 


% Results for de-anonymized examples only:

% 	RAG 200 examples	RAG de-anonymized (n=200)	RAG 50 examples	RAG de-anonymized (n=50)	RAG 100 examples	RAG de-anonymized (n=100)	RAG 150 examples	RAG de-anonymized (n=150)	RAG de-anonymized (n=10)
% bleu1	0.562+-0.003	0.569+-0.003	0.563+-0.003	0.569+-0.003	0.563+-0.003	0.567+-0.003	0.56+-0.002	0.567+-0.003	0.567+-0.004
% bleu2	0.433+-0.003	0.437+-0.004	0.432+-0.003	0.437+-0.003	0.433+-0.003	0.437+-0.003	0.432+-0.002	0.434+-0.002	0.434+-0.005
% bleu3	0.345+-0.003	0.347+-0.004	0.342+-0.004	0.348+-0.003	0.345+-0.002	0.348+-0.004	0.345+-0.002	0.344+-0.002	0.345+-0.005
% bleu4	0.278+-0.003	0.278+-0.003	0.275+-0.005	0.279+-0.003	0.278+-0.003	0.279+-0.003	0.279+-0.002	0.276+-0.002	0.276+-0.004
% rougeL	0.669+-0.001	0.668+-0.002	0.663+-0.002	0.667+-0.002	0.663+-0.003	0.666+-0.002	0.666+-0.001	0.666+-0.001	0.663+-0.007
% meteor	0.56+-0.001	0.559+-0.006	0.557+-0.003	0.564+-0.002	0.556+-0.003	0.562+-0.002	0.558+-0.003	0.554+-0.002	0.558+-0.007
% chrf	0.557+-0.002	0.559+-0.001	0.554+-0.003	0.558+-0.002	0.557+-0.001	0.559+-0.002	0.557+-0.001	0.559+-0.001	0.555+-0.005
% ter	0.52+-0.003	0.52+-0.002	0.525+-0.006	0.523+-0.002	0.522+-0.004	0.523+-0.002	0.518+-0.003	0.52+-0.001	0.53+-0.011


% More anonymization similarity examples:

% Test sentence: 'Which college did Mary go to?'

% Original:

% ['Which college does Mary go to?', 
% "What did Mary's name used to be?", 
% 'Mary used to live in Boston.', 
% 'Who loves Mary?', 
% 'Mary had a baby girl.', 
% 'John got Mary and drove her to work.', 
% 'Mary is a vegetarian.', 
% 'Mary has many different friends.', 
% 'Mary arrived late.', 
% 'John took Mary and drove her to work every day.', 
% 'Mary used to teach but now she works online.', 
% 'Mary loves many men.', 
% "Mary's not here, she just left.", 
% "Mary isn't a vegetarian.", 
% 'What does Mary own a lot of?', 
% 'Mary owns many different cars.', 
% 'John used to take Mary and drive her to work every day.', 
% 'Mary likes red shoes.', 
% 'Mary knows many different languages.']

% De-anonymized:

% ['Which college does Mary go to?', 
% 'Which high school did you go to?', 
% 'Where did you go to high school?', 
% 'Where did Sue move to?', 
% '"Mary is a vegetarian isn\'t she?" "No, she had a bbq last month."', 
% 'Where did Sue move?', 
% 'Did Mother go to the store?', 
% "What did Mary's name used to be?", 
% "She had to miss her school's scheduled SAT exam due to illness, so she had to take the SAT at a hearing high school nearby.", 
% 'Where are you going?', 
% 'Did Bill move to New York?', 
% "He grew up, he entered college, and he met a hearing CODA woman. He started to like her, and he was around his early to mid 20's at that time.", 
% 'I decided I wanted to transfer  to the Rochester Institute.', 
% 'I am going to school.', 
% 'Mary used to live in Boston.', 
% 'Did you graduate high school?', 
% 'The woman who married Prince Charles`s son is famous now.', 
% 'Now I am studying to go to school.', "Who is Liz's famous sister?", 
% 'What did mom clean?']


% Test sentence: 'Mary arrived in Chicago last week.'

% original:

% ['Mary arrived late.', 
% 'Mary used to live in Boston.', 
% "Mary's not here, she just left.", 
% 'Mary knows it rained here yesterday.', 
% 'Mary bought a car yesterday.', 
% 'Mary and John will leave tomorrow.', 
% 'Mary had a baby girl.', 
% 'Mary decided to buy a new car.', 
% 'Mary knows many different languages.', 
% 'John got Mary and drove her to work.', 
% 'Mary has many different friends.', 
% 'Who loves Mary?', 
% 'John took Mary and drove her to work every day.', 
% 'Mary has a different bike.', 
% 'Which college does Mary go to?', 
% 'Mary owns many different cars.', 
% 'Mary used to teach but now she works online.', 
% 'When did Mary buy the car?', 
% "Since Mary's car broke, John has been taking her to work.", 
% 'Mary is a vegetarian.']

% De-anonymized:

% ['Sue and Ben decided to move here to Boston. They arrived last year.', 
% 'My father moved here to Chicago.', 
% 'Did Bill move to New York?', 
% 'Where did Sue move?', 
% 'My friend told me she/he was born in Chicago.', 
% 'Where did Sue move to?', 
% 'Last week John arrived late.', 
% 'My best friend visited Boston yesterday.', 
% 'Where was the man born? He was born in Chicago.', 
% 'The man and the woman arrived here together.', 
% 'Mother is now ill.', 
% 'Mary arrived late.', 
% 'Did Mother go to the store?', 
% 'Mary used to live in Boston.', 
% 'Now the two of us know who arrived yesterday.', 
% 'Mother has to clean the kitchen soon because a group of friends are coming over for the weekend.', 
% 'Mary bought a car yesterday.', 
% 'Is Bill planning on moving to New York?', 
% 'When did Father arrive home?']

\begin{table}[htb!]
    \small
    \centering
    \caption{Existing English Text-to-ASL Gloss translation results reported in the literature.}
    \begin{tabular}{l|l|r}
    \toprule\hline
        \makecell[c]{\textbf{References}} & \makecell[c]{\textbf{Dataset}} & \textbf{BLEU-4}\\\hline
         Inan \etal~\cite{inan2024generating} &  Self-Collected ASL Dataset &0.002\\
        Zhu \etal~\cite{zhu_neural_2023} & NCSLGR & 0.124 \\
        Moryossef \etal~\cite{moryossef_data_2021} & NCSLGR & 0.191\\\hline
         \bottomrule
    \end{tabular}
    \label{tab:text-to-gloss_sota}
\end{table}


