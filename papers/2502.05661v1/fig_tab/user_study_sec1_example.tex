% \begin{figure}[t]
%     \centering
%     \includegraphics[width=0.6\linewidth]{figures/user_study_sec1_blur.jpg}
%     \caption{An Example of the User Study Interface (Section 1). The generated human signer's face is blurred in this figure to address privacy concerns; however, participants were able to view the unblurred versions during the survey.
%     \Description[]}
%     \label{fig:user_study_sec1}
% \end{figure}

\begin{figure*}[t]
    \centering
    \subfloat[A video presented to participants.]{ \includegraphics[width=0.352\textwidth]{figures/sec1_1_updated.png}
    \label{fig:sec1_1}}
    % \hspace{0.02\textwidth}
    \subfloat[Follow-up questions regarding the video.]{\includegraphics[width=0.63\textwidth]{figures/sec1_2_updated.png}
    \label{fig:sec1_2}}
    \caption{An example screen from Section 1 of the survey. Videos generated using the How2Sign dataset were presented to participants, followed by a series of evaluation questions. The signer’s face is blurred here to preserve privacy for publication. However, participants viewed an unblurred version during the survey. 
    \Description[This figure contains an example screen from section 1 of our survey. It includes two main sub-figures. The left sub-figure is the video presented to participants, it contains a video frame with a blurred face of a generated human-realistic AI signer. The video was generated using data from the How2Sign dataset. The right sub-figure is the follow-up questions related to the video shown to participants. The questions ask participants to evaluate the video based on: how easy it is to understand (rated from very hard to very easy); the visual quality of the signing (rated from very poor to excellent); and the naturalness of motion (rated from very hard to very easy). Each question includes a scale and an optional text box for participants to suggest improvements if they rated the video negatively in any aspect.]}
    \label{fig:user_study_sec1}
\end{figure*}

