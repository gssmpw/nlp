\begin{figure*}[t]
    \centering
    \includegraphics[width=1\linewidth]{figures/module_1_w_example_updated.png}
    \caption{An example from Module 1 showcases two tasks: on the left, translating an English sentence into its corresponding ASL gloss, and on the right, predicting the linguistic features of the same English sentence. \textcolor{RoyalBlue}{\textsf{TEXT$\_$TO$\_$GLOSS$\_$DICTIONARY}} represents the examples provided to the LLM for each shot. \textcolor{RoyalBlue}{\textsf{A$\_$BATCH$\_$OF$\_$EXAMPLES}} refers to the examples we provide to the LLM each shot. \textcolor{RoyalBlue}{\textsf{ENGLISH$\_$SENTENCE}} indicates the user-provided input, which, in this example, ``Did the kids play at the park?''
    \Description[This figure uses an example to illustrate a flowchart describing the steps in our Module 1. The process starts with an input sentence at the top: ``Did the kids play at the park?'' Then the Module 1: English Text to ASL Representations is depicted, where two sub-process occur. On the left side, the task is to let the large language model to translate the given English sentence to the corresponding glosses. Specifically, we prompt the system to act as an ASL translator, by providing the following prompt: You are an ASL translator. Your task is to translate an English sentence in o ASL gloss format. Begin by familiarizing yourself with the following vocabulary dictionary: TEXT underscore TO underscore GLOSS underscore DICTIONARY. Then we provide the model with in-context example, by providing the following prompt: Here are some examples of English sentences with their corresponding ASL glosses: A underscore BATCH underscore OF underscore EXAMPLES. Lastly, we ask the model to translate the given English sentence into ASL glosses, while restricting the translation to our word-to-gloss dictionary as its vocabulary. The prompt we provide to the model as the user is: Translate the following English sentence to ASL gloss: ENGLISH underscore SENTENCE. When generating ASL glosses, restrict your usage to the provided vocabulary. The generated glosses should not include any words outside of this list. Provide only the ASL punctuation at the end of the ASL glosses. On the right side, the task is to predict the linguistic features regarding the non-manual markers of the given English sentence. Specifically, we prompt the system with the following information: Your task it to let me know whether the following sentence: 1) is a yes/no question, 2) is a wh- question, 3) contains conditions, and 4) contains negation. The user instructs this process to label each feature as 1 (yes) or 0 (no). The user prompt is: Here is the sentence: ENGLISH underscore SENTENCE. If yes, output 1. If no, output 0. User comma to separate multiple labels. For example, fi the sentence is a yes/no question and a wh- question, and none of other, the output is 1,1,0,0 respectively. Only provide the labels with commas, do not provide any other information. At the bottom of this flowchart, the output section shows the results of translating the input sentence into ``KID PLAY-continuative fs-P-A-R-K-QMwg.'' The sentence type labels are outputted as 1,0,0,0, indicating the sentence is a yes/no question.]}
    \label{fig:module_1}
\end{figure*}