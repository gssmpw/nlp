\begin{figure*}[t]
    \centering
    % \subfloat[Motion and Visual Quality.]{ \includegraphics[width=0.48\textwidth,valign=c]{figures/UserStudy_results_imageQuality_new.png}
    \subfloat[Motion and Visual Quality.]{ \includegraphics[width=0.48\textwidth]{figures/UserStudy_results_imageQuality_new.png}
    \label{fig:visual_motion_quality}}
    % \subfloat[Translation Quality.]{\includegraphics[width=0.48\textwidth,valign=c]{figures/UserStudy_results_translation_new.png}
    \subfloat[Translation Quality.]{\includegraphics[width=0.48\textwidth]{figures/UserStudy_results_translation_new.png}
    \label{fig:translation_quality}}
    \caption{
% \todo{Update results and caption.} 
    Descriptive statistics summarize participants' ratings of motion, visual, and translation quality across model types.  Each bar represents the percentage of videos rated within a given response. The right side of each chart (blue) indicates a positive (or neutral) result and the left side (red) indicates a negative result. All models except \textit{Raw video} were trained on the How2Sign dataset to generate signed videos, using English sentences from the ASLLRP dataset as input. In contrast, \textit{Raw video} refers to unprocessed, human-signed videos directly sourced from the ASLLRP dataset.
    \Description[This figure presents the results from two sections of our user study evaluating video quality. It contains two sub-figures. Each sub-figure displays the percentage of responses, with the right side (blue) representing better ratings and the left side (red) representing worse rating. The left sub-figure focused on visual and motion quality and the right sub-figure focused on translation quality. In the left sub-figure, it contains three questions: how easy is it to understand this video? Rate the visual quality of the signing in this video. Rate the naturalness of the motion. In the right sub-figure, it also contains three questions: is the meaning of the video the same as the English text? What is the quality of the translation? Take into account ASL grammar and style. How accurately does the facial expression match the English text?]}
    \label{fig:userstudyresults}
\end{figure*}


