\begin{figure}[t]
    \centering
    \subfloat[Understanding ratings per sentence type.]{ \includegraphics[width=0.48\textwidth,valign=c]{figures/per_sentence_translation_ratings.png}
    \label{fig:per_sentence}}
    \subfloat[Semantic Translation Assessment.]{\includegraphics[width=0.35\textwidth,valign=c]{figures/translation_ratings.png}
    \label{fig:translation_ratings}}
    \caption{
% \todo{Update results and caption.} 
    Violin plots show estimated marginal means of translation ratings for each of the four video conditions. Translations of raw videos were rated significantly more accurate than all AI video conditions. Indicated on the figure, translations of AI (Full) and AI (w/o Expression) models were significantly more accurate than ratings of AI w/ annotation videos. Error bars show 95 percent confidence intervals. ** denotes p < .01.
    % Signed videos were generated using the following approaches: \textit{AI w/ Annotation} (leveraging human-annotated English-based glosses (manual markers) from the ASLLRP dataset, along with our manually annotated linguistic information (non-manual markers), as input for Modules 2 and 3 of our prototype system), \textit{Video Retargeting} (using skeletal poses from the ASLLRP dataset as inputs for Module 3), and \textit{AI w/ LLM} (our prototype system). All these approaches trained on the How2Sign dataset to generate the final signed output. In contrast, \textit{Raw Video} samples were directly taken from the ASLLRP dataset without additional processing or training.
    % \Description[This figure presents the results from two sections of our user study evaluating video quality. It contains two sub-figures. Each sub-figure displays the percentage of responses, with the right side (blue) representing better ratings and the left side (red) representing worse rating. The left sub-figure focused on visual and motion quality and the right sub-figure focused on translation quality. In the left sub-figure, it contains three questions: how easy is it to understand this video? Rate the visual quality of the signing in this video. Rate the naturalness of the motion. In the right sub-figure, it also contains three questions: is the meaning of the video the same as the English text? What is the quality of the translation? Take into account ASL grammar and style. How accurately does the facial expression match the English text?]}
    \label{fig:userstudyresults}
    }
\end{figure}

