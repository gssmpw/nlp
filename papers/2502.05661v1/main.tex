
% \documentclass[sigconf,authordraft]{acmart}
% \documentclass[manuscript,review]{acmart} % Use this for internal review
\documentclass[sigconf]{acmart} % For final publication
% \documentclass[manuscript,review,anonymous]{acmart} % Use this for submission

%%
%% \BibTeX command to typeset BibTeX logo in the docs
% \AtBeginDocument{%
%   \providecommand\BibTeX{{%
%     Bib\TeX}}}

% \newcommand{\tabincell}[2]{\begin{tabular}{@{}#1@{}}#2\end{tabular}}
\newcommand{\rowstyle}[1]{\gdef\currentrowstyle{#1}%
	#1\ignorespaces
}

\newcommand{\className}[1]{\textbf{\sf #1}}
\newcommand{\functionName}[1]{\textbf{\sf #1}}
\newcommand{\objectName}[1]{\textbf{\sf #1}}
\newcommand{\annotation}[1]{\textbf{#1}}
\newcommand{\todo}[1]{\textcolor{blue}{\textbf{[[TODO: #1]]}}}
\newcommand{\change}[1]{\textcolor{blue}{#1}}
\newcommand{\fetch}[1]{\textbf{\em #1}}
\newcommand{\phead}[1]{\vspace{1mm} \noindent {\bf #1}}
\newcommand{\wei}[1]{\textcolor{blue}{{\it [Wei says: #1]}}}
\newcommand{\peter}[1]{\textcolor{red}{{\it [Peter says: #1]}}}
\newcommand{\zhenhao}[1]{\textcolor{dkblue}{{\it [Zhenhao says: #1]}}}
\newcommand{\feng}[1]{\textcolor{magenta}{{\it [Feng says: #1]}}}
\newcommand{\jinqiu}[1]{\textcolor{red}{{\it [Jinqiu says: #1]}}}
\newcommand{\shouvick}[1]{\textcolor{violet(ryb)}{{\it [Shouvick says: #1]}}}
\newcommand{\pattern}[1]{\emph{#1}}
%\newcommand{\tool}{{{DectGUILag}}\xspace}
\newcommand{\tool}{{{GUIWatcher}}\xspace}


\newcommand{\guo}[1]{\textcolor{yellow}{{\it [Linqiang says: #1]}}}

\newcommand{\rqbox}[1]{\begin{tcolorbox}[left=4pt,right=4pt,top=4pt,bottom=4pt,colback=gray!5,colframe=gray!40!black,before skip=2pt,after skip=2pt]#1\end{tcolorbox}}



  \usepackage{bm}
  \usepackage{subcaption}
  \usepackage{caption}
  \usepackage{hyperref}
  \usepackage{verbatim}
  \usepackage{xcolor}
  \usepackage{listings}
  % \usepackage[poorman]{cleveref}
  % \usepackage[nameinlink]{cleveref} # original
  \usepackage{multirow}
  \usepackage{makecell}
  % \usepackage{minted}
  % \usepackage{adjustbox}
  % \usepackage{colortbl}
  \usepackage{tabularx}
  \usepackage{comment}
  \usepackage{longtable}
  \newcommand{\rev}[1]{\textcolor{black}{#1}}
  \definecolor{RoyalBlue}{rgb}{0.48, 0.41, 0.93}  

  \newcommand{\eg}{\textit{e.g.,\ }}
  \newcommand{\ie}{\textit{i.e.,\ }}
  \newcommand{\etal}{\textit{et al.\ }}
  \newcommand{\etals}{\textit{et al's\ }}

  \newcolumntype{L}[1]{>{\raggedright\let\newline\\\arraybackslash\hspace{0pt}}m{#1}}
  \newcolumntype{C}[1]{>{\centering\let\newline\\\arraybackslash\hspace{0pt}}m{#1}}
  \newcolumntype{R}[1]{>{\raggedleft\let\newline\\\arraybackslash\hspace{0pt}}m{#1}}

  
  \newcommand\addauthornote[1]{%
    \if@ACM@anonymous\else
      \g@addto@macro\addresses{\@addauthornotemark{#1}}%
    \fi}
  

%% Rights management information.  This information is sent to you
%% when you complete the rights form.  These commands have SAMPLE
%% values in them; it is your responsibility as an author to replace
%% the commands and values with those provided to you when you
%% complete the rights form.
% \copyrightyear{2025}
% \acmYear{2025}
% \setcopyright{cc}
% % \setcctype[4.0]{by-nc-nd}
% \setcctype{by-nc-sa}
% \acmConference[CHI '25]{CHI Conference on Human Factors in Computing Systems}{April 26-May 1, 2025}{Yokohama, Japan}
% \acmBooktitle{CHI Conference on Human Factors in Computing Systems (CHI '25), April 26-May 1, 2025, Yokohama, Japan}\acmDOI{10.1145/3706598.3713855}
% \acmISBN{979-8-4007-1394-1/25/04}


\copyrightyear{2025}
\acmYear{2025}
\setcopyright{cc}
% \setcctype[4.0]{by}
\setcctype{by-nc-nd}
\acmConference[CHI '25]{CHI Conference on Human Factors in Computing Systems}{April 26-May 1, 2025}{Yokohama, Japan}
\acmBooktitle{CHI Conference on Human Factors in Computing Systems (CHI '25), April 26-May 1, 2025, Yokohama, Japan}\acmDOI{10.1145/3706598.3713855}
\acmISBN{979-8-4007-1394-1/25/04}



%%
%% Submission ID.
%% Use this when submitting an article to a sponsored event. You'll
%% receive a unique submission ID from the organizers
%% of the event, and this ID should be used as the parameter to this command.
%%\acmSubmissionID{123-A56-BU3}

%%
%% For managing citations, it is recommended to use bibliography
%% files in BibTeX format.
%%
%% You can then either use BibTeX with the ACM-Reference-Format style,
%% or BibLaTeX with the acmnumeric or acmauthoryear sytles, that include
%% support for advanced citation of software artefact from the
%% biblatex-software package, also separately available on CTAN.
%%
%% Look at the sample-*-biblatex.tex files for templates showcasing
%% the biblatex styles.
%%

%%
%% The majority of ACM publications use numbered citations and
%% references.  The command \citestyle{authoryear} switches to the
%% "author year" style.
%%
%% If you are preparing content for an event
%% sponsored by ACM SIGGRAPH, you must use the "author year" style of
%% citations and references.
%% Uncommenting
%% the next command will enable that style.
%%\citestyle{acmauthoryear}


%%
%% end of the preamble, start of the body of the document source.
\begin{document}

%%
%% The "title" command has an optional parameter,
%% allowing the author to define a "short title" to be used in page headers.
\title{Towards AI-driven Sign Language Generation\\ with Non-manual Markers}
% \title{Leveraging LLMs to Support Sign Language Video Generation}

%%
%% The "author" command and its associated commands are used to define
%% the authors and their affiliations.
%% Of note is the shared affiliation of the first two authors, and the
%% "authornote" and "authornotemark" commands
%% used to denote shared contribution to the research.

% \author{}
% \affiliation{%
%   \institution{}
%   \city{}
%   \country{}}
% \email{}





\author{Han Zhang}

\orcid{0000-0002-1377-1168}
\authornote{Work done entirely at Apple.}
\affiliation{%
  \institution{University of Washington, USA}
  \country{}
}
\email{micohan@cs.washington.edu}
% \additionalaffiliation{
    % \institution{Work done entirely at Apple.}
     % }

\author{Rotem Shalev-Arkushin}
\orcid{0009-0009-8376-0171}
\authornotemark[1]
\affiliation{%
  \institution{Tel-Aviv University, Israel}\country{}
}
\email{rotems7@mail.tau.ac.il}

\author{Vasileios Baltatzis}
\orcid{0000-0001-7748-4152}
\affiliation{%
  \institution{Apple, USA} 
  \country{}
}
\email{vbaltatzis@apple.com}

\author{Connor Gillis}
\orcid{0009-0008-3383-2950}
\affiliation{%
  \institution{Apple, USA} \country{}
}
\email{connorgillis@apple.com}

\author{Gierad Laput}
\orcid{0009-0003-6856-2544}
\affiliation{%
  \institution{Apple, USA} \country{}
}
\email{gierad@apple.com}

\author{Raja Kushalnagar}
\orcid{0000-0002-0493-413X}
\authornotemark[1]
\affiliation{%
  \institution{Gallaudet University, USA} \country{}
}
\email{raja.kushalnagar@gallaudet.edu}
% \footnote{Work was done entirely with Apple}

\author{Lorna Quandt}
\orcid{0000-0002-0032-1918}
\authornotemark[1]
\affiliation{%
  \institution{Gallaudet University, USA} \country{}
}
\email{lorna.quandt@gallaudet.edu}


\author{Leah Findlater}
\orcid{0000-0002-5619-4452}
\affiliation{%
  \institution{Apple, USA} \country{}
}
\email{lfindlater@apple.com}

\author{Abdelkareem Bedri}
\orcid{0009-0000-6927-901X}
\affiliation{%
  \institution{Apple, USA} \country{}
}
\email{bedri@apple.com}


\author{Colin Lea}
\orcid{0000-0001-7068-3351}
\affiliation{%
  \institution{Apple, USA} \country{}
}
\email{colin\_lea@apple.com}


%%
%% By default, the full list of authors will be used in the page
%% headers. Often, this list is too long, and will overlap
%% other information printed in the page headers. This command allows
%% the author to define a more concise list
%% of authors' names for this purpose.
\renewcommand{\shortauthors}{Zhang \etal}
\renewcommand{\shorttitle}{Towards AI-driven SLG with Non-manual Markers}

%%
%% The abstract is a short summary of the work to be presented in the
%% article.
\begin{abstract}
Large language model (LLM)-based agents have shown promise in tackling complex tasks by interacting dynamically with the environment. 
Existing work primarily focuses on behavior cloning from expert demonstrations and preference learning through exploratory trajectory sampling. However, these methods often struggle in long-horizon tasks, where suboptimal actions accumulate step by step, causing agents to deviate from correct task trajectories.
To address this, we highlight the importance of \textit{timely calibration} and the need to automatically construct calibration trajectories for training agents. We propose \textbf{S}tep-Level \textbf{T}raj\textbf{e}ctory \textbf{Ca}libration (\textbf{\model}), a novel framework for LLM agent learning. 
Specifically, \model identifies suboptimal actions through a step-level reward comparison during exploration. It constructs calibrated trajectories using LLM-driven reflection, enabling agents to learn from improved decision-making processes. These calibrated trajectories, together with successful trajectory data, are utilized for reinforced training.
Extensive experiments demonstrate that \model significantly outperforms existing methods. Further analysis highlights that step-level calibration enables agents to complete tasks with greater robustness. 
Our code and data are available at \url{https://github.com/WangHanLinHenry/STeCa}.
\end{abstract}

%%
%% The code below is generated by the tool at http://dl.acm.org/ccs.cfm.
%% Please copy and paste the code instead of the example below.
%%
\begin{CCSXML}
<ccs2012>
<concept>
<concept_id>10003120.10003138.10003142</concept_id>
<concept_desc>Human-centered computing~Ubiquitous and mobile computing design and evaluation methods</concept_desc>
<concept_significance>500</concept_significance>
</concept>
<concept>
<concept_id>10003120.10011738.10011776</concept_id>
<concept_desc>Human-centered computing~Accessibility systems and tools</concept_desc>
<concept_significance>500</concept_significance>
</concept>
</ccs2012>
\end{CCSXML}

\ccsdesc[500]{Human-centered computing~Ubiquitous and mobile computing design and evaluation methods}
\ccsdesc[500]{Human-centered computing~Accessibility systems and tools}

%%
%% Keywords. The author(s) should pick words that accurately describe
%% the work being presented. Separate the keywords with commas.
\keywords{Sign language generation, assistive technology, accessibility, human-centered design, DHH community}
%% A "teaser" image appears between the author and affiliation
%% information and the body of the document, and typically spans the
%% page.
% \begin{teaserfigure}
%   \includegraphics[width=\textwidth]{sampleteaser}
%   \caption{}
%   \Description{}
%   \label{fig:teaser}
% \end{teaserfigure}
\begin{teaserfigure}
    \centering
    \includegraphics[width=\textwidth]{figures/teaser_feb7.png}
    \caption{Our prototype translates English text into a photorealistic ASL video which includes both manual and non-manual information. It starts with an English text input (top), and translates it into ASL representations capturing both manual elements (\eg hand movements) and non-manual information (\eg facial expressions). From those, it produces a skeletal pose sequence, and finally converts it into a photorealistic ASL video. In this example, raised eyebrows signal a yes/no question. Without this non-manual marker, the same sentence would be interpreted as a statement.
    \Description[This figure illustrates the process of our prototype system in translating an English sentence  into an American Sign Language video. The top of the figure shows the example English text (``Do you commute to work by bike?''). Below it, the glosses and non-manual markers for each word in the sentence are displayed. The glosses are: IX-2p, TEND, COMMUTE, WORK, and BICYCLE, with a notation indicating a yes/no question. The next row contains a series of five diagrams representing five skeletal poses. Each diagram shows a specific handshape and location corresponding to the glosses and non-manual markers, color-coded for clarity. The bottom row contains images of a person performing the ASL signs, matching the glosses and non-manual markers above. In these images, the person is shown against a green background, demonstrating each sign with corresponding hand positions and facial expressions. Zoomed-in sections highlight the facial expressions associated with the yes/no question. Note that, faces that are not in the zoomed-in sections are blurred to preserve anonymity.]}
    \label{fig:system_overview}
\end{teaserfigure}

%%
%% This command processes the author and affiliation and title
%% information and builds the first part of the formatted document.

% \begin{teaserfigure}
    \centering
    \includegraphics[width=1\linewidth]{figures/hero_fig.png}
    \caption{Our system translates English text into a photorealistic ASL video with non-manual information. It starts with an English text input (first row), generates ASL tokens capturing both manual and non-manual details (second row), produces a skeletal pose sequence (third row), and finally creates the photorealistic ASL video (fourth row). \han{@all, please check the stylization.}}
    % \han{should we have a longer sentence? Ideally has 6-7 video frames. Still working on stylization...} \rotem{thinking about it again, instead of putting it here, I would have 1-2 examples here- only text->gloss->ours with expressions (or through pose), then have a no expression vs expression fig later on in the paper..}}
    \label{fig:system_overview}
\end{teaserfigure}
\maketitle
%!TEX root = gcn.tex
\section{Introduction}
Graphs, representing structural data and topology, are widely used across various domains, such as social networks and merchandising transactions.
Graph convolutional networks (GCN)~\cite{iclr/KipfW17} have significantly enhanced model training on these interconnected nodes.
However, these graphs often contain sensitive information that should not be leaked to untrusted parties.
For example, companies may analyze sensitive demographic and behavioral data about users for applications ranging from targeted advertising to personalized medicine.
Given the data-centric nature and analytical power of GCN training, addressing these privacy concerns is imperative.

Secure multi-party computation (MPC)~\cite{crypto/ChaumDG87,crypto/ChenC06,eurocrypt/CiampiRSW22} is a critical tool for privacy-preserving machine learning, enabling mutually distrustful parties to collaboratively train models with privacy protection over inputs and (intermediate) computations.
While research advances (\eg,~\cite{ccs/RatheeRKCGRS20,uss/NgC21,sp21/TanKTW,uss/WatsonWP22,icml/Keller022,ccs/ABY318,folkerts2023redsec}) support secure training on convolutional neural networks (CNNs) efficiently, private GCN training with MPC over graphs remains challenging.

Graph convolutional layers in GCNs involve multiplications with a (normalized) adjacency matrix containing $\numedge$ non-zero values in a $\numnode \times \numnode$ matrix for a graph with $\numnode$ nodes and $\numedge$ edges.
The graphs are typically sparse but large.
One could use the standard Beaver-triple-based protocol to securely perform these sparse matrix multiplications by treating graph convolution as ordinary dense matrix multiplication.
However, this approach incurs $O(\numnode^2)$ communication and memory costs due to computations on irrelevant nodes.
%
Integrating existing cryptographic advances, the initial effort of SecGNN~\cite{tsc/WangZJ23,nips/RanXLWQW23} requires heavy communication or computational overhead.
Recently, CoGNN~\cite{ccs/ZouLSLXX24} optimizes the overhead in terms of  horizontal data partitioning, proposing a semi-honest secure framework.
Research for secure GCN over vertical data  remains nascent.

Current MPC studies, for GCN or not, have primarily targeted settings where participants own different data samples, \ie, horizontally partitioned data~\cite{ccs/ZouLSLXX24}.
MPC specialized for scenarios where parties hold different types of features~\cite{tkde/LiuKZPHYOZY24,icml/CastigliaZ0KBP23,nips/Wang0ZLWL23} is rare.
This paper studies $2$-party secure GCN training for these vertical partition cases, where one party holds private graph topology (\eg, edges) while the other owns private node features.
For instance, LinkedIn holds private social relationships between users, while banks own users' private bank statements.
Such real-world graph structures underpin the relevance of our focus.
To our knowledge, no prior work tackles secure GCN training in this context, which is crucial for cross-silo collaboration.


To realize secure GCN over vertically split data, we tailor MPC protocols for sparse graph convolution, which fundamentally involves sparse (adjacency) matrix multiplication.
Recent studies have begun exploring MPC protocols for sparse matrix multiplication (SMM).
ROOM~\cite{ccs/SchoppmannG0P19}, a seminal work on SMM, requires foreknowledge of sparsity types: whether the input matrices are row-sparse or column-sparse.
Unfortunately, GCN typically trains on graphs with arbitrary sparsity, where nodes have varying degrees and no specific sparsity constraints.
Moreover, the adjacency matrix in GCN often contains a self-loop operation represented by adding the identity matrix, which is neither row- nor column-sparse.
Araki~\etal~\cite{ccs/Araki0OPRT21} avoid this limitation in their scalable, secure graph analysis work, yet it does not cover vertical partition.

% and related primitives
To bridge this gap, we propose a secure sparse matrix multiplication protocol, \osmm, achieving \emph{accurate, efficient, and secure GCN training over vertical data} for the first time.

\subsection{New Techniques for Sparse Matrices}
The cost of evaluating a GCN layer is dominated by SMM in the form of $\adjmat\feamat$, where $\adjmat$ is a sparse adjacency matrix of a (directed) graph $\graph$ and $\feamat$ is a dense matrix of node features.
For unrelated nodes, which often constitute a substantial portion, the element-wise products $0\cdot x$ are always zero.
Our efficient MPC design 
avoids unnecessary secure computation over unrelated nodes by focusing on computing non-zero results while concealing the sparse topology.
We achieve this~by:
1) decomposing the sparse matrix $\adjmat$ into a product of matrices (\S\ref{sec::sgc}), including permutation and binary diagonal matrices, that can \emph{faithfully} represent the original graph topology;
2) devising specialized protocols (\S\ref{sec::smm_protocol}) for efficiently multiplying the structured matrices while hiding sparsity topology.


 
\subsubsection{Sparse Matrix Decomposition}
We decompose adjacency matrix $\adjmat$ of $\graph$ into two bipartite graphs: one represented by sparse matrix $\adjout$, linking the out-degree nodes to edges, the other 
by sparse matrix $\adjin$,
linking edges to in-degree nodes.

%\ie, we decompose $\adjmat$ into $\adjout \adjin$, where $\adjout$ and $\adjin$ are sparse matrices representing these connections.
%linking out-degree nodes to edges and edges to in-degree nodes of $\graph$, respectively.

We then permute the columns of $\adjout$ and the rows of $\adjin$ so that the permuted matrices $\adjout'$ and $\adjin'$ have non-zero positions with \emph{monotonically non-decreasing} row and column indices.
A permutation $\sigma$ is used to preserve the edge topology, leading to an initial decomposition of $\adjmat = \adjout'\sigma \adjin'$.
This is further refined into a sequence of \emph{linear transformations}, 
which can be efficiently computed by our MPC protocols for 
\emph{oblivious permutation}
%($\Pi_{\ssp}$) 
and \emph{oblivious selection-multiplication}.
% ($\Pi_\SM$)
\iffalse
Our approach leverages bipartite graph representation and the monotonicity of non-zero positions to decompose a general sparse matrix into linear transformations, enhancing the efficiency of our MPC protocols.
\fi
Our decomposition approach is not limited to GCNs but also general~SMM 
by 
%simply 
treating them 
as adjacency matrices.
%of a graph.
%Since any sparse matrix can be viewed 

%allowing the same technique to be applied.

 
\subsubsection{New Protocols for Linear Transformations}
\emph{Oblivious permutation} (OP) is a two-party protocol taking a private permutation $\sigma$ and a private vector $\xvec$ from the two parties, respectively, and generating a secret share $\l\sigma \xvec\r$ between them.
Our OP protocol employs correlated randomnesses generated in an input-independent offline phase to mask $\sigma$ and $\xvec$ for secure computations on intermediate results, requiring only $1$ round in the online phase (\cf, $\ge 2$ in previous works~\cite{ccs/AsharovHIKNPTT22, ccs/Araki0OPRT21}).

Another crucial two-party protocol in our work is \emph{oblivious selection-multiplication} (OSM).
It takes a private bit~$s$ from a party and secret share $\l x\r$ of an arithmetic number~$x$ owned by the two parties as input and generates secret share $\l sx\r$.
%between them.
%Like our OP protocol, o
Our $1$-round OSM protocol also uses pre-computed randomnesses to mask $s$ and $x$.
%for secure computations.
Compared to the Beaver-triple-based~\cite{crypto/Beaver91a} and oblivious-transfer (OT)-based approaches~\cite{pkc/Tzeng02}, our protocol saves ${\sim}50\%$ of online communication while having the same offline communication and round complexities.

By decomposing the sparse matrix into linear transformations and applying our specialized protocols, our \osmm protocol
%($\prosmm$) 
reduces the complexity of evaluating $\numnode \times \numnode$ sparse matrices with $\numedge$ non-zero values from $O(\numnode^2)$ to $O(\numedge)$.

%(\S\ref{sec::secgcn})
\subsection{\cgnn: Secure GCN made Efficient}
Supported by our new sparsity techniques, we build \cgnn, 
a two-party computation (2PC) framework for GCN inference and training over vertical
%ly split
data.
Our contributions include:

1) We are the first to explore sparsity over vertically split, secret-shared data in MPC, enabling decompositions of sparse matrices with arbitrary sparsity and isolating computations that can be performed in plaintext without sacrificing privacy.

2) We propose two efficient $2$PC primitives for OP and OSM, both optimally single-round.
Combined with our sparse matrix decomposition approach, our \osmm protocol ($\prosmm$) achieves constant-round communication costs of $O(\numedge)$, reducing memory requirements and avoiding out-of-memory errors for large matrices.
In practice, it saves $99\%+$ communication
%(Table~\ref{table:comm_smm}) 
and reduces ${\sim}72\%$ memory usage over large $(5000\times5000)$ matrices compared with using Beaver triples.
%(Table~\ref{table:mem_smm_sparse}) ${\sim}16\%$-

3) We build an end-to-end secure GCN framework for inference and training over vertically split data, maintaining accuracy on par with plaintext computations.
We will open-source our evaluation code for research and deployment.

To evaluate the performance of $\cgnn$, we conducted extensive experiments over three standard graph datasets (Cora~\cite{aim/SenNBGGE08}, Citeseer~\cite{dl/GilesBL98}, and Pubmed~\cite{ijcnlp/DernoncourtL17}),
reporting communication, memory usage, accuracy, and running time under varying network conditions, along with an ablation study with or without \osmm.
Below, we highlight our key achievements.

\textit{Communication (\S\ref{sec::comm_compare_gcn}).}
$\cgnn$ saves communication by $50$-$80\%$.
(\cf,~CoGNN~\cite{ccs/KotiKPG24}, OblivGNN~\cite{uss/XuL0AYY24}).

\textit{Memory usage (\S\ref{sec::smmmemory}).}
\cgnn alleviates out-of-memory problems of using %the standard 
Beaver-triples~\cite{crypto/Beaver91a} for large datasets.

\textit{Accuracy (\S\ref{sec::acc_compare_gcn}).}
$\cgnn$ achieves inference and training accuracy comparable to plaintext counterparts.
%training accuracy $\{76\%$, $65.1\%$, $75.2\%\}$ comparable to $\{75.7\%$, $65.4\%$, $74.5\%\}$ in plaintext.

{\textit{Computational efficiency (\S\ref{sec::time_net}).}} 
%If the network is worse in bandwidth and better in latency, $\cgnn$ shows more benefits.
$\cgnn$ is faster by $6$-$45\%$ in inference and $28$-$95\%$ in training across various networks and excels in narrow-bandwidth and low-latency~ones.

{\textit{Impact of \osmm (\S\ref{sec:ablation}).}}
Our \osmm protocol shows a $10$-$42\times$ speed-up for $5000\times 5000$ matrices and saves $10$-2$1\%$ memory for ``small'' datasets and up to $90\%$+ for larger ones.

\section{Background}
\label{sec:background}


\subsection{Code Review Automation}
Code review is a widely adopted practice among software developers where a reviewer examines changes submitted in a pull request \cite{hong2022commentfinder, ben2024improving, siow2020core}. If the pull request is not approved, the reviewer must describe the issues or improvements required, providing constructive feedback and identifying potential issues. This step involves review commment generation, which play a key role in the review process by generating review comments for a given code difference. These comments can be descriptive, offering detailed explanations of the issues, or actionable, suggesting specific solutions to address the problems identified \cite{ben2024improving}.


Various approaches have been explored to automate the code review comments process  \cite{tufano2023automating, tufano2024code, yang2024survey}. 
Early efforts centered on knowledge-based systems, which are designed to detect common issues in code. Although these traditional tools provide some support to programmers, they often fall short in addressing complex scenarios encountered during code reviews \cite{dehaerne2022code}. More recently, with advancements in deep learning, researchers have shifted their focus toward using large-language models to enhance the effectiveness of code issue detection and code review comment generation.

\subsection{Knowledge-based Code Review Comments Automation}

Knowledge-based systems (KBS) are software applications designed to emulate human expertise in specific domains by using a collection of rules, logic, and expert knowledge. KBS often consist of facts, rules, an explanation facility, and knowledge acquisition. In the context of software development, these systems are used to analyze the source code, identifying issues such as coding standard violations, bugs, and inefficiencies~\cite{singh2017evaluating, delaitre2015evaluating, ayewah2008using, habchi2018adopting}. By applying a vast set of predefined rules and best practices, they provide automated feedback and recommendations to developers. Tools such as FindBugs \cite{findBugs}, PMD \cite{pmd}, Checkstyle \cite{checkstyle}, and SonarQube \cite{sonarqube} are prominent examples of knowledge-based systems in code analysis, often referred to as static analyzers. These tools have been utilized since the early 1960s, initially to optimize compiler operations, and have since expanded to include debugging tools and software development frameworks \cite{stefanovic2020static, beller2016analyzing}.



\subsection{LLMs-based Code Review Comments Automation}
As the field of machine learning in software engineering evolves, researchers are increasingly leveraging machine learning (ML) and deep learning (DL) techniques to automate the generation of review comments \cite{li2022automating, tufano2022using, balachandran2013reducing, siow2020core, li2022auger, hong2022commentfinder}. Large language models (LLMs) are large-scale Transformer models, which are distinguished by their large number of parameters and extensive pre-training on diverse datasets.  Recently, LLMs have made substantial progress and have been applied across a broad spectrum of domains. Within the software engineering field, LLMs can be categorized into two main types: unified language models and code-specific models, each serving distinct purposes \cite{lu2023llama}.

Code-specific LLMs, such as CodeGen \cite{nijkamp2022codegen}, StarCoder \cite{li2023starcoder} and CodeLlama \cite{roziere2023code} are optimized to excel in code comprehension, code generation, and other programming-related tasks. These specialized models are increasingly utilized in code review activities to detect potential issues, suggest improvements, and automate review comments \cite{yang2024survey, lu2023llama}. 




\subsection{Retrieval-Augmented Generation}
Retrieval-Augmented Generation (RAG) is a general paradigm that enhances LLMs outputs by including relevant information retrieved from external databases into the input prompt \cite{gao2023retrieval}. Traditional LLMs generate responses based solely on the extensive data used in pre-training, which can result in limitations, especially when it comes to domain-specific, time-sensitive, or highly specialized information. RAG addresses these limitations by dynamically retrieving pertinent external knowledge, expanding the model's informational scope and allowing it to generate responses that are more accurate, up-to-date, and contextually relevant \cite{arslan2024business}. 

To build an effective end-to-end RAG pipeline, the system must first establish a comprehensive knowledge base. It requires a retrieval model that captures the semantic meaning of presented data, ensuring relevant information is retrieved. Finally, a capable LLM integrates this retrieved knowledge to generate accurate and coherent results \cite{ibtasham2024towards}.




\subsection{LLM as a Judge Mechanism}

LLM evaluators, often referred to as LLM-as-a-Judge, have gained significant attention due to their ability to align closely with human evaluators' judgments \cite{zhu2023judgelm, shi2024judging}. Their adaptability and scalability make them highly suitable for handling an increasing volume of evaluative tasks. 

Recent studies have shown that certain LLMs, such as Llama-3 70B and GPT-4 Turbo, exhibit strong alignment with human evaluators, making them promising candidates for automated judgment tasks \cite{thakur2024judging}

To enable such evaluations, a proper benchmarking system should be set up with specific components: \emph{prompt design}, which clearly instructs the LLM to evaluate based on a given metric, such as accuracy, relevance, or coherence; \emph{response presentation}, guiding the LLM to present its verdicts in a structured format; and \emph{scoring}, enabling the LLM to assign a score according to a predefined scale \cite{ibtasham2024towards}. Additionally, this evaluation system can be enriched with the ability to explain reasoning behind verdicts, which is a significant advantage of LLM-based evaluation \cite{zheng2023judging}. The LLM can outline the criteria it used to reach its judgment, offering deeper insights into its decision-making process.





\section{Sign Language Generation Prototype}\label{sec:design} 

In this section we describe our SLG prototype, which generates ASL videos with manual markers---such as hand shape, location, movements, and palm orientation---as well as non-manual markers, including facial expressions and eyebrow movements. 
Our focus is on context-free settings, where each sentence is translated independently. We used a modular approach for our system design (Figure \ref{fig:system_arch}), allowing increased flexibility and interpretability of each module.

The prototype consists of three components: \textbf{Module 1: English Text to ASL Representations}, which leverages a Large Language Model (GPT-4o~\cite{achiam2023gpt}) to translate an English sentence into English-based ASL glosses and to detect linguistic information relevant to non-manual markers; \textbf{Module 2: ASL Representations to Skeletal Pose Sequence}, which takes the LLM outputs and employs a Motion Matching approach to synthesize a skeletal pose sequence; and \textbf{Module 3: Skeletal Pose Sequence to ASL Signed Video}, which generates signed video frames representing a photorealistic ASL signer. 
This modular approach allows for future improvement of the system as the technology advances, by allowing each part to be changed separately. This prototype was iteratively refined within the research team. Insights from these researchers and other collaborators fluent in ASL helped to guide improvements in translation quality, visual and motion quality, and information conveyance.
% Importantly, for future work each of these modules could be swapped out with other methods (\eg a task-specific LLM for Module 1 or a styled avatar representation for Module 3). 

\begin{figure*}[t]
    \centering
    \includegraphics[width=1\linewidth]{figures/system_arch_updated.png}
    \caption{Our prototype includes three self-contained modules. It takes an English sentence as input and generates an ASL video (from top to bottom). Module 1 utilizes a large language model (LLM) to translate the English input into an ASL gloss string and predict non-manual markers. Module 2 employs a Motion Matching approach to generate a skeletal pose sequence from the output of Module 1. Finally, Module 3 uses a UNet-like model, which given an individual signer's appearance and style (Signer ID), transforms the skeletal pose sequence into signing frames. These are then combined to produce the final signing video. 
    \Description[The figure depicts the architecture of our system for generating American Sign Language (ASL) videos from English text, divided into three modules. The first module translates English sentences into ASL glosses and predicts non-manual markers like facial expressions and head movements using a large language model and specific questions. The second module converts the gloss strings into skeletal poses through motion matching and blending techniques to integrate both manual and non-manual components. The third module uses a U-Net-like model to transform these skeletal poses into video frames, refining the visual details based on the signer’s appearance and style. The final output is a video of a photorealsitic AI signer performing ASL, reflecting the original English input.]
    }
    \label{fig:system_arch}
\end{figure*}

\subsection{Module 1: English Text to ASL Representations}\label{sebsec:module1}

We used an enhanced gloss-based approach that translates an English sentence into an intermediate ASL gloss, including both manual and non-manual information, which is then utilized by subsequent modules. Given the ability of LLMs to naturally absorb and generate grammatical rules, structures, and nuances~\cite{brown2020language,radford2019language}, we used GPT-4o\footnote{Specifically, we used the model gpt-4o-2024-05-13.}~\cite{achiam2023gpt}, a state-of-the-art LLM, to perform two key tasks: (1) translate an English sentence into English-based glosses and (2) detect if the English sentence contains linguistic features associated with specific facial expressions (Module 1 in Figure~\ref{fig:system_arch}). GPT-4o was selected based on our preliminary experiments with various versions of the GPT models. Detailed experimental results are presented in Appendix \ref{appendix:llm_experiments}. 

For the first task, we adopted a prompting-based approach using LLMs with ``in-context learning''~\cite{xu2024misconfidence}, inspired by recent work on low-resource machine translation~\cite{guo2024teaching}, where dataset sizes are too small to train large-scale translation models. This approach allows the model to adapt and perform specific tasks by interpreting examples or instructions directly embedded in the input text, without requiring explicit retraining~\cite{brown2020language}. To improve performance, we added 1,494 in-context examples of English sentence-gloss pairs to our prompt from the ASLLRP dataset (representing 80\% of the dataset). Given the limited window of GPT-4o (\ie 128,000 input tokens), which restrict the number of examples that can be included in a single prompt, we used a ``multi-prompting'' approach. This method involved splitting the examples into multiple batches and iteratively prompting GPT-4o with each batch. In addition, we asked the LLM to constrain its output by generating glosses within the vocabulary established by our text-to-gloss dictionary described below.
% To enhance our translation capabilities, we implemented RAG (Retrieval Augmented Generation) \cite{lewis2020retrieval} with anonymized embeddings. First, as a pre-process, we anonymized all train sentences by converting name references into pronouns. Next, we embed the anonymized sentences using an OpenAI embeddings model. Finally, at inference, for each test sentence, we embed it as well and look for the $N$ most similar examples to this sentence based on the cosine similarity between the embedding of the test example, and the embeddings of the anonymized train examples. This way, the model is presented with the most accurate and relevant examples. As table Table \ref{tab:text-to-gloss_RAG_eval_results} shows, when using RAG the results are better than using all of the train examples. Moreover, using fewer examples and anonymized embeddings yields better results in most cases. The reason for using anonymization, is that names are given high weight in the embedding, which leads to less relevant examples in some cases. For examples, the 3 most similar sentence for the sentence "Which college did Mary go to?" before anonymization, are: "Which college does Mary go to?", "What did Mary's name used to be?", "Mary used to live in Boston.", While after anonymization they are: "Which college does Mary go to?", "Which high school did you go to?", "Where did you go to high school?", which are more relevant and similar examples.
% Given the limited context window of GPT-4o (\ie 128,000 input tokens), which restrict the number of examples that can be included in a single prompt, we used a ``multi-prompting'' approach. This method involved splitting the examples into multiple batches and iteratively prompting GPT-4o with each batch. 

For the second task, we adopted a zero-shot prompting approach, asking the model to predict linguistic features associated with specific facial expressions without any in-context examples. The idea of linguistic predictions was inspired by prior research suggesting that non-manual expressions corresponding to specific grammatical markers, such as raised eyebrows or head tilts, typically involve a consistent set of behaviors that convey meaning within sign language~\cite{neidle2002signstream,baker1983microanalysis}. In this work, we focus primarily on eyebrow movements. To this end, we asked the model to predict whether a given English sentence: is (1) a yes-no question, (2) a wh-question, (3) a conditional statement, and/or (4) contains negation. The outputs from both tasks are then used to generate skeletal poses that are compatible with the subsequent modules, enhancing the integration of non-manual markers. 

This approach addresses two common limitations of gloss-based ASL representations: (1) their tendency to deviate from ASL grammar, and (2) their inability to fully capture the context and expressiveness necessary for conveying the full semantics of a sentiment. 

\paragraph{Dataset and Implementation Details} After reviewing the available ASL datasets (see Appendix ~\ref{appendix:ASL} for more details), we selected the ASLLRP~\cite{neidle_asl_2022} dataset for Module 1. The ASLLRP dataset contains continuous sentence-level ASL videos, isolated ASL videos, ASL glosses, and corresponding English translations. This dataset provides detailed annotations, including textual annotations (\eg English-based glosses for lexical signs, fingerspelling, classifiers, name signs, and gestures), manual markers (\eg number of hands used, alternating hand movements), and non-manual markers (\eg head position and movements, eye gaze, and mouth movements). 

\paragraph{Data Preprocessing} While ASLLRP provides the most comprehensive information required for our task, the data is dispersed across various resources and editions. To make effective use of this dataset, we first consolidated these disparate resources into a unified framework, extracting 2,119 English sentence-gloss pairs along with their corresponding signing videos. The signing videos were then trimmed to isolate specific sign language utterances for our subsequent tasks. To minimize translation errors, we removed gloss annotations that did not alter the overall meaning of the sentence when omitted and standardized all glosses related to fingerspelling. All these changes were done by consulting team members fluent in ASL. We also excluded glosses for classifiers due to their limited sample sizes. After data cleaning, we retained 1,843 English sentence-gloss pairs. Next, we developed a word-gloss dictionary to improve consistency in sign representations of words across different sentences, resulting in 3,915 word-gloss pairs. For the 43 out-of-vocabulary (OOV) words that lacked corresponding videos, we employed fingerspelling as an alternative representation. Finally, four of our researchers conducted a ground truth correction to resolve misalignments between the linguistic labels for the four types of non-manual information and the English text, ensuring the labels more accurately reflected the text content. A more detailed description of our data preprocessing process can be found in Appendix \ref{appendix:data_prep}. The conventions used for re-annotating the glosses in this work are summarized in Table \ref{tab:gloss_convention}.


\begin{figure*}[t]
    \centering
    \includegraphics[width=1\linewidth]{figures/module_1_w_example_updated.png}
    \caption{An example from Module 1 showcases two tasks: on the left, translating an English sentence into its corresponding ASL gloss, and on the right, predicting the linguistic features of the same English sentence. \textcolor{RoyalBlue}{\textsf{TEXT$\_$TO$\_$GLOSS$\_$DICTIONARY}} represents the examples provided to the LLM for each shot. \textcolor{RoyalBlue}{\textsf{A$\_$BATCH$\_$OF$\_$EXAMPLES}} refers to the examples we provide to the LLM each shot. \textcolor{RoyalBlue}{\textsf{ENGLISH$\_$SENTENCE}} indicates the user-provided input, which, in this example, ``Did the kids play at the park?''
    \Description[This figure uses an example to illustrate a flowchart describing the steps in our Module 1. The process starts with an input sentence at the top: ``Did the kids play at the park?'' Then the Module 1: English Text to ASL Representations is depicted, where two sub-process occur. On the left side, the task is to let the large language model to translate the given English sentence to the corresponding glosses. Specifically, we prompt the system to act as an ASL translator, by providing the following prompt: You are an ASL translator. Your task is to translate an English sentence in o ASL gloss format. Begin by familiarizing yourself with the following vocabulary dictionary: TEXT underscore TO underscore GLOSS underscore DICTIONARY. Then we provide the model with in-context example, by providing the following prompt: Here are some examples of English sentences with their corresponding ASL glosses: A underscore BATCH underscore OF underscore EXAMPLES. Lastly, we ask the model to translate the given English sentence into ASL glosses, while restricting the translation to our word-to-gloss dictionary as its vocabulary. The prompt we provide to the model as the user is: Translate the following English sentence to ASL gloss: ENGLISH underscore SENTENCE. When generating ASL glosses, restrict your usage to the provided vocabulary. The generated glosses should not include any words outside of this list. Provide only the ASL punctuation at the end of the ASL glosses. On the right side, the task is to predict the linguistic features regarding the non-manual markers of the given English sentence. Specifically, we prompt the system with the following information: Your task it to let me know whether the following sentence: 1) is a yes/no question, 2) is a wh- question, 3) contains conditions, and 4) contains negation. The user instructs this process to label each feature as 1 (yes) or 0 (no). The user prompt is: Here is the sentence: ENGLISH underscore SENTENCE. If yes, output 1. If no, output 0. User comma to separate multiple labels. For example, fi the sentence is a yes/no question and a wh- question, and none of other, the output is 1,1,0,0 respectively. Only provide the labels with commas, do not provide any other information. At the bottom of this flowchart, the output section shows the results of translating the input sentence into ``KID PLAY-continuative fs-P-A-R-K-QMwg.'' The sentence type labels are outputted as 1,0,0,0, indicating the sentence is a yes/no question.]}
    \label{fig:module_1}
\end{figure*}

\paragraph{LLM Translation and Classification} 
We used few-shot and zero-shot prompting over GPT-4o \cite{achiam2023gpt} to perform these tasks.
Our prompts were designed to ensure the outputs could be directly used 
% to generate CSV files 
for downstream tasks and systematic evaluation. 
% and further use; 
Figure \ref{fig:module_1} overviews the process and prompts, and shows a usage example. More examples of prompts can be found in Table \ref{tab:prompt_engineering}. For the translation task, we structured the process by first defining the task for the system. Next, we provided the model with context using English word-gloss pair examples for few-shot learning. Finally, we asked the model to translate each English sentence into ASL glosses, while restricting the translation to our word-gloss dictionary as its vocabulary. For the linguistic features task, we also started by defining the task for the system, and then, using zero-shot prompting, asked the model to classify the linguistic features in the English sentence, \ie if it contains a yes/no question, a wh-question, a condition, and/or a negation. Zero-shot was enough in this case, because GPT was extensively trained over English text. The exact prompts used for this process are shown in Figure \ref{fig:module_1}.
% ------------------------------------------------------------

\subsection{Module 2: ASL Representations to Skeletal Pose Sequence}\label{subsubsec:module2} 
The goal of this module is to take the gloss and non-manual LLM outputs and generate a sequence of skeletal poses at video frame rate, which expresses the input English phrase. We based our approach on Motion Matching, a widely used technique in the Computer Graphics community~\cite{buttner2015motion,clavet2016motion,holden2020learned}, which takes a large dictionary of short character animations and an input signal and intelligently blends clips together to form a cohesive video. 
Given the gloss input, a sequence of reference clips is chosen from the dictionary using an optimization function that minimizes the signing concept of ``economy of motion.'' This principle prioritizes the ``best'' sign by minimizing the distance between the body position at the end of the previous sign and the start of the next. The selected clips are then linearly blended together to create a cohesive sequence. 
The non-manual predictions are used as input to an expression blending part of the model which takes the glossed output and augments the facial expressions, in particular targeting eyebrow motion. 
Our signing dictionary derived from ASLLRP contains 12,681 signed pose sequences, with many repetitions of each sign, which are labeled with the 3,915 glosses noted above. 

Motion Matching typically comprises of three components: (1) a definition for how we represent pose sequences and how they are used for generating the pose sequence dictionary, (2) similarity and optimization functions for identifying the ``best'' elements for a sequence, and (3) a blending function to create the resulting pose sequence. 
See Figure \ref{fig:system_arch} (Module 2) for a visual description. The first step chooses and blends the best sign variants. A second step applies expression blending, which augments the pose sequences with non-manual markers to refine facial expressions. 

\paragraph{Skeletal Pose Representation \& Sign Dictionary}
Whole body, face, and hand skeletal keypoints are extracted from all isolated sign videos in ASLLRP using Mediapipe~\cite{lugaresi2019mediapipe}, using 3D information for hands and 2D information for the others.
We preprocessed this data in three ways. First, we imputed keypoints that were missing due to occlusion issues and poor tracking. For missing keypoints at the beginning or end of a sequence, we filled in points with neutral poses where the hands were positioned together just below the viewpoint from the camera. All other missing keypoints were linearly interpolated using valid keypoints from timesteps before and after within that sequence. One exception was with fingerspelling, where we intentionally kept the non-dominant hand in the same neutral position to avoid jumps between letters in a word. Second, we normalized all keypoints in space so that position and scale of the body and head were consistent across sequences. This alleviated differences in camera position between videos and body shapes between signers. For positioning, we relied on the first frame of each sign with the average shoulder position in subsequent frames relative to that first frame. Lastly, we trimmed the start and end of each sign using annotations from the ASLLRP dataset. For fingerspelling, we sped up the clips to account for the discrepancy between the slower performance in the isolated sign video clips and the faster pace typically used in-situ \cite{quinto2010rates}. 

\paragraph{Optimization functions}
There are many different ways to articulate the same sign for emphasis, style, and convenience~\cite{baker1991american,brentari1998prosodic}. For most signs in our dictionary we have multiple examples of each sign. Often these variants convey the same meaning, but are performed by different signers. Sometimes the meaning does vary. For example, ``big'' might have versions that convey a medium-big size and a large-big size or a sign might be shown using newer and antiquated styles. In short of having sufficient linguistic information to differentiate sign variations, we select sign variants based on minimizing movement rather than incorporating other linguistic factors. In the signing community this is sometimes referred to as minimizing the ``economy of motion,'' where an individual may blend together sign variations based on which is physically more efficient.  Mathematically, given a vector of keypoint locations $x_{i,t}^p$ where $i$ is a valid gloss index, $t$ is a frame index within a clip, and $p$ is a body part (body, face, hands), we compared the Euclidian distance using a weighted average of the current gloss $i$ and a candidate subsequent gloss indexed by $j$:
\begin{equation}
	d(i, j) = \sum_{p \in \{body,\ face,\ hands\}} \alpha_{p} \cdot \left\|x_{i,T}^p - x_{j,0}^p\right\|_2^2,
\end{equation}
where $\alpha_p$ is a weighting value for each body part, and $T$ is the final frame in the clip. Values of $alpha$ were chosen to prioritize importance of the body and prevent large changes in posture.

The final sequence of sign videos was determined by minimizing the differences (maximizing the similarity) across all glosses output from the LLM. This was achieved by a greedy algorithm that selected sign videos with the correct gloss labels, prioritizing those where the beginning of the clip was most similar to the end of the previous clip. 

\paragraph{Gloss \& Expression Blending}
We generated a preliminary pose sequence by linearly blending together the start and end of the pose sequence from each chosen gloss instance, using the first and last 20 frames of each clip (at 90 Hz). To increase smooth transitions, we appended half-second neutral pose to the beginning and end of each sequence of videos, which was also interpolated with the gloss videos. We then used the predicted non-manual marker information to augment the facial expressions holistically after stitching the videos together. Specifically, we adjusted the position of the eyebrows throughout the video to reflect whether a sentence was a yes/no question, wh-question, or neither. The output of this module is a sequence with body, face, and hand keypoint poses for a full video. 

% ------------------------------------------------------------
\subsection{Module 3: Skeletal Poses to Video Frames}\label{subsec:module3}
The last module converts the generated pose sequences into a sequence of photorealistic images. 
First, the input 2/3D skeleton poses are rasterized by drawing the skeletal positions onto an image. 
Second, these skeletal images are used as input to an image-to-image neural network, which outputs photorealistic images. We choose to generate videos that resemble ``live'' signers, in an effort to mitigate confounds that could arise in accurately representing signs with more stylized avatars.

 % \todo{I have moved both the description of the baseline and the proposed rasterization function here instead of the experimental section. Another option would be to have only the proposed here and leave the baseline's description in the experimental section.}
The design decisions regarding the rasterization function---the way the skeleton is drawn---play a critical role in the performance of the image-to-image model. In the baseline rasterization function used by previous work~\cite{zhang2023adding, hu2024animate}, each landmark position was represented by a circle on a 2D image with a monochromatic (black) background, with straight lines connecting the hand and torso landmarks. This is consistent with the commonly used drawing functions within the Mediapipe~\cite{lugaresi2019mediapipe} library. In contrast, in our drawing function, instead of scattered, connected circles, each body part (\ie hands, body, face) was represented as a convex polygon, with additional connections drawn between the face and the entire body. Each body surface was drawn with a different shade of gray and each hand uses a different color palette where each finger is a different shade. We use the 3D data from each hand to determine the palm orientation (``in'' versus ``out'') using the surface normal of landmarks surrounding the palm and augment hand colors based on this orientation. Moreover, the background in our dataset varies per person and we find that using a rasterized background color with shades of black-to-red going from top to bottom and black-to-green going from left to right improves the stability of the generated images. The proposed rasterization function significantly improves the image quality and background stability with emphasis to differentiating the hands and individual fingers, disambiguating occlusions originating in overlap between body parts, and differences in the backgrounds of each image.

Although there have been large advances in photorealistic image generation of humans using diffusion models (\eg ControlNet~\cite{zhang2023adding}), results tend to lack temporal consistency and often do not represent hands accurately. Hence, our work builds on image-to-image translation models~\cite{isola2017image, brooks2023instructpix2pix, tumanyan2023plug, hertz2022prompt, zhang2023adding}, while adding modern architectures and loss functions. 
% Specifically, we used a UNet-like model~\cite{ronneberger2015u}, using neural building blocks from Imagen architecture~\cite{saharia2022photorealistic} along with an LPIPS perceptual loss.
Specifically, we used a U-Net architecture ~\cite{ronneberger2015u}, with the encoder and decoder backbones using neural building blocks from the architecture in Imagen~\cite{saharia2022photorealistic}. 
% Imagen is an image generation model that is conditioned on a text description. 
Unlike Imagen, which uses text as an additional input to the system, we condition the decoder using the signers' identity. 
This is especially important because we use data from many different Signer IDs as part of the same model.
This enables the network to output different visual appearances for each Signer ID in the dataset, which is used when training the network and at inference time. 
% \todo{CL: Consider adding a couple more details, such as the specific Imagen block types. }
% Inspired by this, we leverage this architecture to condition the model on the identity of the signer that we want to be depicted in the photo-realistic video. 

The model is trained with a combination of three losses. These are an L1 term between the entire generated output frame and the target input frame, an L1 term only on the hand region, and an LPIPS term~\cite{zhang2018unreasonable}, which is a learned metric that measures perceptual similarity between the output frame and the target input frame. 
% The first loss is an L1 loss between the entire generated output frame $\mathbf{y}$ and the target input frame $\mathbf{x}$:
% \begin{equation}
%     \mathcal{L}_{\text{L1\_whole}} = \frac{1}{N} \sum_{i=1}^{N} \left| y_i - x_i \right|
% \end{equation}
% where $N$ is the total number of pixels in the image.
% The second loss is again an L1 term but this time we leverage a binary hand mask $M_{\text{hands}}$ that identifies the hand regions in the image. This loss term ensures that particular emphasis is given on hand generation. The loss is computed per pixel and the weighted by the hand mask:
% \begin{equation}
%     \mathcal{L}_{\text{L1\_hands}} = \frac{\sum_{i=1}^{N} \left| y_i - x_i \right| \cdot M_{\text{hands}, i}}{\sum_{i=1}^{N} M_{\text{hands}, i}}
% \end{equation}
% Here, the numerator computes the L1 loss over the hand region, and the denominator normalizes by the total number of pixels in the hand mask.
% The third loss is the LPIPS loss~\cite{zhang2018unreasonable}, which is a learned perceptual similarity metric that measures the perceptual similarity between the output frame $\mathbf{y}$ and the target input frame $\mathbf{x}$. The third loss term is denoted as $\mathcal{L}_{\text{perceptual}} = \text{LPIPS}(\mathbf{x}, \mathbf{y})$.
The total loss used to train the model is the sum of the whole frame L1 loss, the hand-specific L1 loss, and the perceptual loss.
% : $\mathcal{L}_{\text{total}} = \mathcal{L}_{\text{L1\_whole}} + \mathcal{L}_{\text{L1\_hands}} + \mathcal{L}_{\text{perceptual}}$


\paragraph{Dataset and Implementation Details} 
% \todo{Information about internal dataset (depending on which person we end up showing on Figure 1 as well)} 

Our primary dataset for image generation experiments is How2Sign~\cite{duarte_how2sign_2021}\footnote{There is ambiguity as to which individuals in How2Sign gave permission to use their likeness in publications. Thus, for visualization purposes within this paper and supplemental material, we trained additional models that contain the identity of two other people who have given their permission. Qualitatively, these results are representative of the How2Sign results.}.
Although it doesn't contain glosses, in contrast to ASLLRP, which has varied quality across videos, How2Sign contains 35K high-resolution clips of ASL with a vocabulary size of over 16K word tokens. The high resolution and overall data quality of How2Sign helps the model to learn fine-grained and high-quality visual representations of ASL. 

While the overall image quality is generally high, there are problems with skeleton tracking, especially when there is significant motion blur or there is ambiguity in hand pose. 
Thus, when training the model, we discard lower quality frames
% First, we do not want to create a distribution shift between train and test distributions. 
in efforts to learn more precise mappings between skeletons and photo-realistic humans.
% , without having to make low quality predictions in case some landmarks are missing from a frame of the training set.
We accomplish this by performing automated visual checks in both image and skeletal pose spaces. 
In image space, we use optical flow to detect motion blur by analyzing the flow vectors between two consecutive frames using Farneback's method ~\cite{farneback2003two}. 
In pose space, we check for sudden large changes in landmark positions between consecutive frames, which might indicate inaccuracies due to motion blur. 
% This can \rotem{what do you mean can? are we doing it or not?} also be extended over a short history of frames in order to track the historical stability of landmark positions. 
Specifically, we compared the current landmark positions to the mean landmark positions over a sliding window of predefined size. While signing, hands tend to move more than the body, so the pose conditions are imposed only for each hand instead of including the entire body and face.
% \todo{CL: What is the percent of frames uses? Can we also include the number of signers and some relevant stats based on what (and how much) data we actually ended up using}

% , and considered factors such as motion blur and stability to improve the visual quality of the generated videos. 
% As described in the previous section, the output poses are in the format of a flattened vector of landmark positions. 
% However, these raw landmark positions lack visual context, spatial relationships, and other visual cues inherent in photorealistic frames, making the task of transforming skeletal poses into video frames challenging for machine learning models.

% This transformation converts the task to an Image-to-Image translation, allowing us to leverage a model architecture designed specifically for image processing. 
% While the ultimate goal is to generate a video, the approach used in \textit{ASL Representation to Skeletal Poses} handles the temporal modeling of motion, enabling the approach used in \textit{Skeletal Poses to Video Frames} to focus on generating individual frames only. This separation reduces the complexity of modeling temporal and spatial information simultaneously. 
% \rotem{missing how is temporal consistency preserved after converting to video frames?}

% An additional consideration involves managing missing or inconsistent landmarks. This means that at inference time, this model will be only receiving frames of good quality. 

% Let $I_{t-1}(x,y) $ and $I_t(x,y)$ represent the pixel intensities of two consecutive frames. The optical flow $\mathbf{F}(x,y)$ represents the displacement vector for each pixel from frame $t-1$ to frame $t$, and is defined as:

% \begin{equation}
%     \mathbf{F}(x,y) = \begin{bmatrix} u(x,y) \\ v(x,y) \end{bmatrix}
%     \label{eq:flow}
% \end{equation}
% where:

% \begin{itemize}
%     \item $u(x,y)$ is the displacement in the $x$-direction, 
%     \item $v(x,y)$ is the displacement in the $y$-direction.
% \end{itemize}

% The algorithm minimizes the difference between the frames while also penalizing large variations in the flow field, resulting in:

% \begin{equation}
% \min_{\mathbf{F}} \sum_{x,y} \left[I_t(x+u,y+v) - I_{t-1}(x,y) \right ]^2 + \lambda \sum_{x,y} \|\nabla \mathbf{F}(x,y)\|^2
% \label{eq:farneback}
% \end{equation}
% where $\lambda$ is a regularization parameter controlling the smoothness of the flow field.
% Motion blur is detected by analyzing the optical flow magnitudes. Given the optical flow $\mathbf{F}(x,y)$ for each pixel, the magnitude of the flow is computed as:
% \begin{equation}
%     \|\mathbf{F}(x,y)\| = \sqrt{u(x,y)^2 + v(x,y)^2}
%     \label{eq:flow_magnitude}
% \end{equation}

% The mean magnitude across all pixels is used to determine the presence of motion blur:

% \begin{equation}
%     \text{Mean Flow Magnitude} = \frac{1}{N} \sum_{x,y} \|\mathbf{F}(x,y)\|
% \end{equation}
% where $N$ is the total number of pixels in the frame.
% The condition for detecting motion blur is:
% \begin{equation}
%     \text{Motion Blur} = 
%     \begin{cases}
%         \text{True}, & \text{if } \frac{1}{N} \sum_{x,y} \|\mathbf{F}(x,y)\| > \varepsilon_{blur} \\
%         \text{False}, & \text{otherwise}
%     \end{cases}
% \end{equation}
% where $\varepsilon_{blur}$ is a predefined threshold.
% On landmark space, we check for sudden large changes in landmark positions between consecutive frames, which might indicate inaccuracies due to motion blur. 
% Given the landmarks of the current frame $/mathbf{L}_t$, and the landmarks of the previous frame $\mathbf{L}_{t-1}$, the Euclidean distance between corresponding landmarks is calculated as:

% \begin{equation}
%     d_i = \|\mathbf{L}_t^i - \mathbf{L}_{t-1}^i\|_2
% \end{equation}
% where:
% \begin{itemize}
%     \item $\mathbf{L}_t^i = (x_t^i, y_t^i, z_t^i)$ represents the coordinates of the $i$-th landmark at time $t$
%     \item $d_i$ is the Euclidean distance between the $i$-th landmarks at time $t$ and $t-1$
% \end{itemize}
% The condition for detecting a sudden change is:
% \begin{equation}
%     \text{Sudden Change} = 
%     \begin{cases}
%         \text{True}, & \text{if } \max(d_i) > \varepsilon_{sudden\_change} \\
%         \text{False}, & \text{otherwise}
%     \end{cases}
% \end{equation}
% where $\varepsilon_{sudden\_change}$ is a predefined threshold.
% This can \rotem{what do you mean can? are we doing it or not?} also be extended over a short history of frames in order to track the historical stability of landmark positions. The stability of landmarks over time is assessed by comparing the current landmark positions to the mean landmark positions over a sliding window of predefined size.
% $w$. Let $\mathbf{L}_{t-w}$, $\mathbf{L}_{t-w+1}$, $\mathbf{L}_{t}$ represent the landmarks over the past $w$ frames. The mean landmark position is computed as:
% \begin{equation}
%     \bar{\mathbf{L}} = \frac{1}{w} \sum_{k=t-w+1}^{t} \mathbf{L}_k 
% \end{equation}

% The stability is measured as the mean Euclidean distance between the current landmarks and the mean landmarks:
% \begin{equation}
%     \text{Stability} = \frac{1}{w} \sum_{k=t-w+1}^{t} \|\mathbf{L}_k - \bar{\mathbf{L}}\|_2
% \end{equation}
% The condition for stable landmarks is:
% \begin{equation}
%     \text{Stable Landmarks} = 
%     \begin{cases}
%         \text{True}, & \text{if } \frac{1}{w} \sum_{k=t-w+1}^{t} \|\mathbf{L}_k - \bar{\mathbf{L}}\|_2 < \varepsilon_{stability} \\
%         \text{False}, & \text{otherwise}
%     \end{cases}
% \end{equation}
% where $\varepsilon_{stability}$ is a predefined threshold.
% Given that in ASL hands tend to move more than the body, the conditions on landmark space are imposed explicitly on each of the hands instead of the entirety of landmark positions.


% \rotem{I'm missing some more technical details of how are we analyzing the flow, what is the check for sudden changes in lmks..}


% The How2Sign~\cite{duarte_how2sign_2021} dataset was used for training and generating the final signed videos in Module 3. This approach allowed us to focus on generating ASL signs that include both manual markers---such as hand shape, location, movements, and palm orientation---as well as non-manual markers, including facial expressions and eyebrow movements. 


% \rotem{need to add gradient background details}


\section{Technical Evaluation}\label{sec:technical_eval}

We conducted technical evaluations to assess the performance of our proposed system in translating English text into intermediate ASL representations and generating signed videos. The following sections provide a detailed account of each evaluation, including the experimental procedures and the corresponding evaluation results. Note that a direct quantitative comparison with previous work is challenging due to the use of different datasets~\cite{inan2024generating,zhu_neural_2023,moryossef_data_2021}, output modalities, or gloss-less approaches~\cite{baltatzis2024neural}. For instance, while benchmarks for English text-to-ASL gloss translation often use datasets from other languages, benchmarks specific to ASL gloss translation are lacking. Additionally, for video generation, \cite{baltatzis2024neural} employs the How2Sign dataset and produces SMPL-X 3D human body model poses, whereas our system generates photorealistic videos. These differences in output (3D models vs. photorealistic videos) and their end-to-end design, which precludes comparison of intermediate components, make direct comparisons impractical.

\subsection{English Text to ASL Represenetations}\label{subsec:technical_eval_exp1} 

\subsubsection{English Text-to-ASL Gloss Translation}\label{subsubsec:english_to_gloss}

We conducted ablation studies to determine the optimal model configuration for translating English sentences into English-based glosses (as illustrated on the left side of Module 1 in Figure \ref{fig:system_arch}). Specifically, we examined four key factors: the impact of data preprocessing, the number of in-context examples fed to GPT, the effectiveness of generating glosses within the vocabulary established in our word-to-gloss dictionary, and the necessity of guiding GPT to learn ASL grammar rules\footnote{The ASL grammar rules we provided to GPT-4o can be found in \ref{asl_grammar_rules} in Appendix.}. For the number of English-to-gloss examples, we experimented with 600 (33\% of dataset) and 1,474 (80\% of dataset) sentences from ASLLRP. The dataset was randomly split into a 80/20 ratio to mitigate inconsistencies in distribution. We report BLEU~\cite{papineni_bleu_2002} scores (1 to 4 grams) and ROUGE-L~\cite{lin_rouge_2004} scores, two widely used metrics in the machine translation community~\cite{baltatzis2024neural,saunders_progressive_2020,saunders2020adversarial,fang2024signllm}. Additionally, for a more comprehensive evaluation, we include METEOR~\cite{banerjee_meteor_2005}, CHrF~\cite{popovic_chrf_2015}, TER~\cite{snover_study_2006}, and SacreBLEU~\cite{post2018call}, which are also commonly applied in the literature to assess text-to-gloss translation quality~\cite{egea_gomez_syntax-aware_2021,zhu_neural_2023,forster_extensions_nodate}.
\begin{table*}[t]
\caption{Evaluation results of translating English text to ASL glosses (Task on the left side in Module 1). ``Prep.'' denotes Preprocessing. \bm{$^*$}All BLEU-4 and SacreBLEU scores are identical. \bm{$\uparrow$} indicates that higher values represent better performance, while \bm{$\downarrow$} indicates that lower values represent better performance. Best results in \textbf{bold}. Note: If ``Data Prep.'' is set to ``No'', the model was not restricted to generating glosses within the word-to-gloss dictionary vocabulary, as the dictionary generation is part of our preprocessing step.
\Description[This table presents evaluation results of translating English text to English-based glosses. The first row contains twelve headers, including Data preparation, number of examples, limited vocabulary, grammar rules, bleu-1, bleu-2, bleu-3, bleu-4, rouge-l, meteor, chrf, and ter. Regarding the columns, the first column indicates whether data preparation is applied (yes or no). The second column, number of examples, shows two sets: 600 and 1474 examples, with specific portions of the entire dataset (33\% or 80\%). The third column, limited vocab, indicates whether a limited vocabulary is use (yes or no). The fourth column, grammar rules, shows whether grammar rules are applied (yes or no). The remaining columns display various evaluation metrics with the symbol uparrow indicating higher values are better, and downarrow indicating lower values are better. Key findings are: first, when data preparation is yes, 600 examples, with no limited vocabulary and grammar rules applied, bleu score (1 to 4 grams) range from 0.470 to 0.214, with the best bleu-4 score at 0.214; second, with 1474 examples and the same conditions, the bleu scores improve, reaching a bleu-4 score of 0.276; third, metrics such as rouge-l, meteor, and chrf show similar trends, generally improving with more examples and grammar rules; fourth, the best scores are bolded across all metrics, indicating the optimal settings for this translation model.]}\label{tab:text-to-gloss_eval_results}
\renewcommand{\arraystretch}{1.1}
 \resizebox{\textwidth}{!}{
\begin{tabular}{L{0.04\textwidth}|C{0.13\textwidth}|C{0.06\textwidth}|C{0.08\textwidth}| R{0.08\textwidth} R{0.08\textwidth} R{0.08\textwidth} R{0.09\textwidth} R{0.1\textwidth} R{0.09\textwidth} R{0.06\textwidth} R{0.05\textwidth}}
% \begin{tabular}{l{0.04\textwidth}|c{0.13\textwidth}|c{0.06\textwidth}|c{0.08\textwidth}| r{0.08\textwidth} r{0.08\textwidth} r{0.08\textwidth} r{0.09\textwidth} r{0.1\textwidth} r{0.09\textwidth} r{0.06\textwidth} r{0.05\textwidth}}
\toprule\hline
\multicolumn{1}{c|}{{\textbf{\makecell[c]{Data \\ Prep.}}}} & \multicolumn{1}{c|}{{\textbf{\makecell[c]{Number of \\ Examples}}}} &\multicolumn{1}{c|}{{\textbf{\makecell[c]{Limited \\ Vocab}}}}&\multicolumn{1}{c|}{{\textbf{\makecell[c]{Grammar \\ Rules}}}} & \multicolumn{1}{c}{{\textbf{\makecell[c]{BLEU-1 \bm{$\uparrow$}}}}} &\multicolumn{1}{c}{{\textbf{\makecell[c]{BLEU-2 \bm{$\uparrow$}}}}} & \multicolumn{1}{c}{{\textbf{\makecell[c]{BLEU-3 \bm{$\uparrow$}}}}}
& \multicolumn{1}{c}{{\textbf{\makecell[c]{BLEU-4\bm{$^*$} \bm{$\uparrow$}}}}} 
& \multicolumn{1}{c}{{\textbf{\makecell[c]{ROUGE-L \bm{$\uparrow$}}}}} 
& \multicolumn{1}{c}{{\textbf{\makecell[c]{METEOR \bm{$\uparrow$}}}}} 
& \multicolumn{1}{c}{{\textbf{\makecell[c]{CHrF \bm{$\uparrow$}}}}} & \multicolumn{1}{c}{{\textbf{\makecell[c]{TER \bm{$\downarrow$}}}}} \\\hline 
\multirow{4}{*}{No}& \multirow{2}{*}{600}& \multirow{2}{*}{-} & Yes& 0.295 & 0.204 & 0.151 & 0.116 & 0.573& 0.352 & 0.426 & 0.668\\
 & & & No & 0.358 & 0.260& 0.201 & 0.158 & 0.591& 0.386 & 0.454 & 0.644\\\cline{2-12}
 & \multirow{2}{*}{1,474}& \multirow{2}{*}{-} & Yes& 0.379 & 0.280& 0.220& 0.177 & 0.603& 0.406 & 0.462 & 0.625\\
 & && No & 0.404 & 0.303 & 0.239 & 0.192 & 0.611& 0.432 & 0.472 & 0.619\\\hline
\multirow{8}{*}{Yes} & \multirow{4}{*}{\makecell[c]{600 \\(33\% of the \\ entire dataset)}}& \multirow{2}{*}{No}& Yes& 0.470& 0.336 & 0.255 & 0.197 & 0.617& 0.498 & 0.487 & 0.585\\
 & && No & 0.487 & 0.355 & 0.273 & 0.214 & 0.627& 0.502 & 0.503 & 0.572\\\cline{3-12}
 & & \multirow{2}{*}{Yes} & Yes& 0.520& 0.390& 0.305 & 0.241 & 0.641& 0.530 & 0.522 & 0.556\\
 & && No & 0.520 & 0.387 & 0.302 & 0.237 & 0.642& 0.523 & 0.528 & 0.554\\\cline{2-12}
 & \multirow{4}{*}{\makecell[c]{1,474 \\ (80\% of the \\ entire dataset)}}& \multirow{2}{*}{No}& Yes& 0.501 & 0.378 & 0.298 & 0.239 & 0.646& 0.534 & 0.521 & 0.537\\
 & && No & 0.513 & 0.386 & 0.303 & 0.243 & 0.645& 0.532 & 0.519 & 0.548\\\cline{3-12}
 & & \multirow{2}{*}{Yes} & Yes& 0.545 & 0.415 & 0.329 & 0.265 & 0.662& 0.551 & 0.544 & \textbf{0.524}\\
 & && No & \textbf{0.556} & \textbf{0.427} & \textbf{0.341} & \textbf{0.276} & \textbf{0.664} & \textbf{0.560} & \textbf{0.549} & 0.526 \\\hline
 \bottomrule
\end{tabular}}
\end{table*}

As shown in Table~\ref{tab:text-to-gloss_eval_results}, our ablation study results indicate that data preprocessing improves the LLM's performance in translating English text to English-based glosses. Similarly, providing the LLM with more examples, when they are chosen randomly, and limiting the generated glosses to those within the word-to-gloss dictionary results in higher BLEU (1 to 4 grams), ROUGE-L, METEOR, and CHrF scores, along with lower TER scores, all of which suggest enhanced model performance. 

% \begin{figure}[t]
%     \centering
%     \subfloat[Precision and Recall for Each Category.]{ \includegraphics[scale=0.40]{figures/task2_w_label_br.png}
%     \label{fig:other_metrics}}
%     \subfloat[Confusion Matrix for Each Category.]{\includegraphics[scale=0.20]{figures/confusion_matrix_br.png}
%     \label{fig:confusion_matrix}}
%     \caption{Model Performance in Detecting Linguistic Features to Generate Non-manual Marker Information.}
%     \label{fig:non_manual_results}
% \end{figure}

% \begin{wrapfigure}{r}{0.5\textwidth}
\begin{figure}
    \centering
    {\includegraphics[scale=0.36]{figures/task2_w_label_new.png}}
    \caption{Model performance in detecting linguistic features to generate non-manual marker information. The model demonstrates high performance across all categories, with particularly high recall for detecting negation and precision for detecting yes/no questions. Precision for negation, however, is relatively lower at 0.79.
    \Description[This grouped bar chart displays the model's precision and recall performance across four types of linguistic features: yes/no question, wh- question, negation, and conditional (from left to right). The blue bars reprent precision, while the red bars represent recall. From left to right, the precision and recall for each type is 0.98 and 0.93 for yes/no question, 0.93 and 0.98 for wh- question, 0.79 and 1.00 for negation, and 0.95 and 0.05 for conditional.]}
    \label{fig:non_manual_results}
% \end{wrapfigure}
\end{figure} 

Interestingly, most experiments showed that adding grammar rules did not improve the model’s translation ability, however, there were some exceptions. 
% This may stem from the fact the grammar rules were too strict, while the dataset examples did not always follow these rules. Table~\ref{tab:text-to-gloss_RAG_eval_results_appx} shows the effectiveness of giving more relevant examples to the model using RAG, rather than using all train examples without considering their relevance to the sentence at hand. Moreover, it shows the contribution of using anonymized embeddings, where in most cases it improved the results and allowed using fewer examples to achieve better results.
For example, when data preprocessing was applied and the LLM was provided with 80\% of the entire dataset without limiting the generated glosses to the word-to-gloss vocabulary, we observed mixed results. Specifically, BLEU (1 to 4 grams) scores suggested better model performance without adding grammar rules to the LLM, while other metrics indicated the opposite trend. Furthermore, although direct comparisons are challenging, our system demonstrates compelling translation performance compared to existing results reported in the literature, achieving a BLEU-4 score improvement from 0.191 to 0.276. Table \ref{tab:text-to-gloss_sota} in Appendix \ref{appendix:existing_text-to-gloss_results} summarizes the existing English Text-to-ASL gloss translation results reported in the literature.

\subsubsection{Linguistic Predictions}\label{subsubsec:linguistic_predictions}

Falsely predicting linguistic features for a sentence could result in unnecessary non-manual markers added to the sequential poses and video frames, potentially leading to confusion in the generated ASL videos. To evaluate the performance of GPT-4o in detecting linguistic features regarding the four questions---whether a sentence is a yes/no question, wh-question, conditional statement, and/or contains negation---we calculate precision and recall for each type of prediction. 

Figure \ref{fig:non_manual_results} summarizes the model’s performance in detecting linguistic features within a given English sentence across the four conditions. Overall, the model demonstrates high accuracy across these tasks, particularly in identifying questions and conditional statements. The relatively low precision for negation (precision=0.79) suggests that the model occasionally incorrectly identified negation in sentences where the human-labeled ground truth did not indicate negation presence. We analyzed these cases and discovered that in most, the sentences include negative sentiment, \eg "Why do you hate video games?" or "My sister blamed me but I am innocent!"  


\subsection{Video Generation}
\label{subsec:technical_eval_exp3}

We evaluated our system's performance in generating signed videos (Modules 2 and 3) using quantitative metrics commonly used for human video generation~\cite{wang2024disco}. These metrics evaluate the generations at either image-level (single-level) or video-level. Image-level metrics include L1, PSNR~\cite{hore2010image}, SSIM~\cite{wang2004image}, LPIPS~\cite{zhang2018unreasonable} and FID~\cite{heusel2017gans}, while video-level metrics include FID-VID~\cite{balaji2019conditional} and FVD~\cite{unterthiner2018towards}. Following prior research~\cite{wang2024disco}, we calculated video-level metrics for sequences of 16 consecutive frames. The dataset contains around 60,000 frames from the How2Sign dataset for training and 15,000 for testing, which correspond to about 40 and 10 minutes of video, respectively. To account for variations in appearance such as clothing, we treated the same signer across different recording sessions as distinct signer identities, resulting in a total of 13 Signer IDs.

We performed several ablation studies to evaluate the efficacy of our design choices. The first ablation study focused on the rasterization function, comparing our proposed enhanced rasterization function with the simpler baseline. The second ablation experiment focused on checking frame quality. Specifically, we reported metrics for our Pose-to-Video model under three conditions: (1) ``All frames'', where no frames were excluded from training; (2) ``Valid frames'', where frames with missing landmarks were excluded from the training set, and (3) ``Proposed'', where frames with missing landmarks, blurry frames, and frames that contain landmarks that indicate temporal inconsistencies were excluded, as detailed in the final paragraph of Section \ref{subsec:module3}. 

\begin{table*}[t]
    \caption{Evaluation results of video generation (Module 3). \bm{$\uparrow$} indicates that higher values represent better performance, while \bm{$\downarrow$} indicates that lower values represent better performance. Best results in \textbf{bold}.
    \Description[This table presents evaluation results for both image-level and video-level video generation methods, comparing a baseline method against a proposed method. The table is divided into two main sections based on the type of experiment: rasterization function and frame quality check. For all columns, the first two columns, experiment and method, specify the type of experiment (rasterization function and frame quality check) and the method used (baseline, proposed for rasterization function, all frames, valid frames, and proposed for frame quality check). The rest of columns are split into image metrics and video metrics. Image metrics include L1, PSNR, SSIM, LPIPS, and FID. Video metrics include FID-VID and FVD. Key findings are: in both experiments (resterization function and frame quality check), the proposed method consistently achieves the best performance across all metrics (bolded values), with significantly lower L1, LPIPS, FID, FID-VID, and FVD scores and higher PSNR and SSIM valus compared to the baseline and other conditions.]}
    \centering
    \footnotesize
     \resizebox{\textwidth}{!}{\begin{tabular}{ll @{\extracolsep{8pt}} ccccc  cc}
    \toprule\hline
         \multirow{2}{*}{\textbf{Experiment}} &\multirow{2}{*}{\textbf{Method}}&\multicolumn{5}{c}{\textbf{Image}}  & \multicolumn{2}{c}{\textbf{Video}}\\
         \cline{3-7} \cline{8-9}
         &&\textbf{L1 $\downarrow$}  &\textbf{PSNR $\uparrow$} &\textbf{SSIM $\uparrow$} &\textbf{LPIPS $\downarrow$} &\textbf{FID $\downarrow$} &\textbf{FID-VID $\downarrow$} &\textbf{FVD $\downarrow$} \\
         \hline
         \multirow{2}{*}{Rasterization function}& Baseline &  31.71E-05&8.069  &0.028  &1.107  &401.14  & 189.62 & 2024.42\\
          &\textbf{Proposed} &  \textbf{2.83E-05}&\textbf{23.346}  &\textbf{0.864} &\textbf{0.155}  &\textbf{56.28}  &\textbf{7.23} &\textbf{173.78}\\
          \hline
         \multirow{3}{*}{Frame quality check}&  All frames& 3.60E-05 & 19.98 &0.838  &0.192  &187.03  &27.71&691.91 \\
         & Valid frames &3.14E-05  &22.03  &0.855  &0.165  &173.56  &15.72 & 497.17 \\
         &\textbf{Proposed} &  \textbf{2.83E-05}&\textbf{23.346}  &\textbf{0.864} &\textbf{0.155}  &\textbf{56.28}  &\textbf{7.23} &\textbf{173.78}\\
    \hline \bottomrule
    \end{tabular}}
    \label{tab:pose-to-video_eval_results}
    \vspace{-0.3cm}
\end{table*}

Table \ref{tab:pose-to-video_eval_results} presents the evaluation results, demonstrating the proposed approach improves all metrics across the board. The effectiveness of the rasterization function is evident, as the baseline approach produced outputs that resembled a reconstructed skeletal pose rather than a photorealistic human version. The proposed rasterization function provides a better anatomical representation of a given pose, enabling the model to learn a more robust mapping between skeletal poses and photorealistic human images. In terms of frame quality, removing lower quality frames progressively improves the model’s performance, reinforcing the conclusion that data quality is just as important as quantity. 


\section{User Evaluation with DHH Signers}\label{sec:user_study}

We conducted a user study with 30 DHH participants to further evaluate our prototype system by assessing the perceived quality of our generated signed videos, with a focus on their ASL grammatical correctness---both with and without non-manual markers---understandability, and naturalness of movement. Additionally, we gathered participants' interest in this technology and its potential use cases. All English sentences were derived from continuous sentence-level signing videos in the ASLLRP dataset. The signed videos presented to participants were either generated from the How2Sign dataset or presented as raw, unprocessed human-signed videos from the ASLLRP dataset.

\subsection{Study Design}\label{subsec:study_design}
The survey was conducted online via a web-based survey tool and consisted of two main sections. Participants provided responses through 5-point rating scales and open-ended feedback, allowing for both quantitative and qualitative insights. To minimize bias that might arise from visual aesthetics influencing translation quality evaluations, we intentionally structured the survey to first evaluate visual and motion quality, followed by translation quality. This design choice was inspired by the aesthetic-usability effect, which indicates that users often perceive visually appealing or high-quality visual designs as more functional or accurate~\cite{tractinsky2000beautiful,hoegg2010good}. We chose 5-point semantic differential scales, a survey rating scale designed to capture respondents' attitudes, approaches, and perspectives~\cite{osgood1964semantic,osgood1957measurement,semantic_diff}, to gauge DHH participants' perceptions of the quality of the generated signed videos. A detailed summary of the user study questions is provided in Appendix \ref{appendix:survey}.

\subsubsection{Section 1: Visual and Motion Quality}\label{subsubsec:sec1} This section evaluated Modules 2 (ASL Representations to Skeletal Pose Sequence) and 3 (Skeletal Poses to Video Frames) of our system, focusing on the motion and visual quality of the generated signed videos. The goal was to explore alignments between technical and human evaluations while providing additional assessment of Module 2, which was not fully evaluated during the technical phase due to the lack of established metrics for this module. To achieve this, we presented two types of models for evaluation. 

The first type, \textit{AI (Annotations)}, uses human-annotated English-based glosses (manual markers) from the ASLLRP dataset, along with our manually annotated linguistic information (non-manual markers), as input for Modules 2 and 3 of our system. This approach assumes a high-quality text translation and focuses on evaluating the performance of our motion and image models. The second type, \textit{Video Retargeting}, takes skeletal poses extracted from ASLLRP sentence videos as input for Module 3, representing a best-case scenario between these two types. This approach assumes high quality text translation and skeletal extraction, focusing solely on assessing the performance of our visual model and identifying potential issues when retargeting data from the ASLLRP dataset to the How2Sign dataset. Notably, all models using our Module 3 were trained exclusively on the How2Sign dataset (detailed in Section \ref{subsec:module3}).

Each participant viewed and rated three videos for each type, where videos were randomly sampled from a larger set of 27 sentences. Participants rated each of the videos on a 5-point rating scale for understandability, visual quality, and naturalness of movement---criteria commonly referenced in the literature~\cite{huenerfauth2007evaluating,quandt2022attitudes}. These evaluations were captured across multiple bipolar dimensions, with scale options such as ``0 (Very Hard), 1 (Hard), 2 (Neutral), 3 (Easy), 4 (Very Easy)'' or ``0 (Very Poor), 1 (Poor), 2 (Neutral), 3 (Good), 4 (Excellent).'' Note that ``N/A'' was provided as a default option, but participants were asked to select another response. After completing each rating scale question, participants had the option to provide open-ended feedback for additional insights. Figure \ref{fig:user_study_sec1} provides an example of the survey interface used for this section as presented to the participants. 

% \begin{figure}[t]
%     \centering
%     \includegraphics[width=0.6\linewidth]{figures/user_study_sec1_blur.jpg}
%     \caption{An Example of the User Study Interface (Section 1). The generated human signer's face is blurred in this figure to address privacy concerns; however, participants were able to view the unblurred versions during the survey.
%     \Description[]}
%     \label{fig:user_study_sec1}
% \end{figure}

\begin{figure*}[t]
    \centering
    \subfloat[A video presented to participants.]{ \includegraphics[width=0.352\textwidth]{figures/sec1_1_updated.png}
    \label{fig:sec1_1}}
    % \hspace{0.02\textwidth}
    \subfloat[Follow-up questions regarding the video.]{\includegraphics[width=0.63\textwidth]{figures/sec1_2_updated.png}
    \label{fig:sec1_2}}
    \caption{An example screen from Section 1 of the survey. Videos generated using the How2Sign dataset were presented to participants, followed by a series of evaluation questions. The signer’s face is blurred here to preserve privacy for publication. However, participants viewed an unblurred version during the survey. 
    \Description[This figure contains an example screen from section 1 of our survey. It includes two main sub-figures. The left sub-figure is the video presented to participants, it contains a video frame with a blurred face of a generated human-realistic AI signer. The video was generated using data from the How2Sign dataset. The right sub-figure is the follow-up questions related to the video shown to participants. The questions ask participants to evaluate the video based on: how easy it is to understand (rated from very hard to very easy); the visual quality of the signing (rated from very poor to excellent); and the naturalness of motion (rated from very hard to very easy). Each question includes a scale and an optional text box for participants to suggest improvements if they rated the video negatively in any aspect.]}
    \label{fig:user_study_sec1}
\end{figure*}



\subsubsection{Section 2: Translation Quality}\label{subsubsec:sec2}
This section evaluated Module 1 (English text to ASL Representations) of our system, focusing on the translation quality. We aimed to assess how closely our generated ASL aligns with correct ASL grammar and style, and to examine the impact of non-manual markers, specifically facial expressions, on the overall quality and comprehensibility of the signed output. We achieved this by comparing four types of models.

The first type, \textit{AI (Annotations)}, is identical to the approach described in Section \ref{subsubsec:sec1}. This approach allows us to compare the translation quality of our system with human-annotated ASL representations.  The second type, \textit{AI w/o Expr}, uses our full system (Modules 1-3) but with expression blending model turned off to specifically evaluate our system's effectiveness in generating non-manual markers. The third type, \textit{AI (Full)}, uses our full prototype with the LLM predictions (Modules 1-3), containing both manual and non-manual information. The last type, \textit{Raw Video}, consists of original human-signed videos from the ASLLRP dataset without any processing or modeling. The raw videos serve as the best-performing benchmark, providing a reference point for understanding the gap between our system-generated outputs and natural, human-signed videos.


Each participant viewed 21 videos taken from six sentence types. These included a wh-question, a yes/no question, a question that could be mistaken for a statement without non-manual markers, a simpler statement without non-manual markers, a more complex statement involving negation or conditional, and one random sentence with fingerspelling. Within a survey, videos were randomized so that the same sentence was only used once, with one exception. To analyze the expression blending part of our model, we showed each participant the three ``question'' sentences twice: once using our full system (\textit{AI (Full)}) and once without expression blending (\textit{AI w/o Expr}). 

For each video, participants first provided English translations for the ASL content shown. They were then presented with the ``true'' English translation from the ASLLRP dataset and rated three aspects on a 5-point rating scale: the similarity between the video's meaning and the ``true'' English, the quality of the ASL translation (including grammar and signing style), and the accuracy of the facial expressions. To encourage decisive responses and minimize central tendency bias, we adapted scales from prior work~\cite{zhu_neural_2023}, excluding the neutral option and using choices such as ``0 (Very Poor), 1 (Poor), 2 (Acceptable), 3 (Good), 4 (Excellent).'' After completing these ratings, participants used checkbox options and an open-ended text box to report issues with the translations. They also had the options to provide feedback on translation quality and share their ASL interpretation of the English sentence. To maintain consistency and reliability in the evaluation process, each video in both sections was reviewed by at least three participants. 
% Figure X \todo{add the figure for section 2} provides an example of the survey interface used for this section. 

% \subsubsection{Section 3: Interest in Our System}\label{subsubsec:sec3} This section aimed to gauge participants' interest in AI signing technology based on our system and its potential usability. Participants were asked to rate the likelihood of using AI signers as a supplement to live interpreters in situations where interpreters are unavailable or impractical. They were also asked to share their opinions on potential scenarios and specific use cases for AI signers, as well as their thoughts on the suitability of the current quality of AI signers. Finally, participants were asked to indicate their preferences between ``live'' (photrealistic) human-realistic signers and 3D avatar representations.

\subsubsection{Follow-Up Questions and Demographics}\label{subsubsec:sec4} At the end of the survey, participants were asked about their general interest in AI signing technology and its potential use cases. Additionally, demographic information was collected, including gender, age group, the age at which they began learning ASL, their proficiency in both English and ASL, and the frequency of their communication in ASL and spoken English.

\subsection{Data Collection}\label{subsec:data_collection}

Participants in this study were recruited outside the research group to ensure impartiality and avoid potential biases. To qualify for participation, individuals had to self-identify as DHH, use ASL as their primary language, and be over the age of 18. To ensure participants met these criteria and had the necessary proficiency in ASL, we further implemented a screening process. This process involved prospective participants watching three ASL videos and selecting the corresponding English translations from a set of multiple-choice options. This study went through our organization's internal user study review process.

After the screening and recruitment process, we enrolled a total of 30 DHH signers who met all eligibility criteria. For demographics, 11 participants were aged 20-29, 10 between 30-39, 7 between 40-49, and 2 between 50-59. Twenty-one participants identified as female, and 9 identified as male. Twenty-four participants learned ASL before age 10, while the remaining learned it later. Regarding proficiency, 23 participants rated their ASL comprehension and production as excellent, while the others rated themselves as good. Fifteen rated their English proficiency as excellent, 11 as good, and 3 as acceptable. All except one reported using ASL daily, with one reporting weekly use. The survey took 45-60 minutes for most individuals to complete.

\begin{figure*}[t]
    \centering
    % \subfloat[Motion and Visual Quality.]{ \includegraphics[width=0.48\textwidth,valign=c]{figures/UserStudy_results_imageQuality_new.png}
    \subfloat[Motion and Visual Quality.]{ \includegraphics[width=0.48\textwidth]{figures/UserStudy_results_imageQuality_new.png}
    \label{fig:visual_motion_quality}}
    % \subfloat[Translation Quality.]{\includegraphics[width=0.48\textwidth,valign=c]{figures/UserStudy_results_translation_new.png}
    \subfloat[Translation Quality.]{\includegraphics[width=0.48\textwidth]{figures/UserStudy_results_translation_new.png}
    \label{fig:translation_quality}}
    \caption{
% \todo{Update results and caption.} 
    Descriptive statistics summarize participants' ratings of motion, visual, and translation quality across model types.  Each bar represents the percentage of videos rated within a given response. The right side of each chart (blue) indicates a positive (or neutral) result and the left side (red) indicates a negative result. All models except \textit{Raw video} were trained on the How2Sign dataset to generate signed videos, using English sentences from the ASLLRP dataset as input. In contrast, \textit{Raw video} refers to unprocessed, human-signed videos directly sourced from the ASLLRP dataset.
    \Description[This figure presents the results from two sections of our user study evaluating video quality. It contains two sub-figures. Each sub-figure displays the percentage of responses, with the right side (blue) representing better ratings and the left side (red) representing worse rating. The left sub-figure focused on visual and motion quality and the right sub-figure focused on translation quality. In the left sub-figure, it contains three questions: how easy is it to understand this video? Rate the visual quality of the signing in this video. Rate the naturalness of the motion. In the right sub-figure, it also contains three questions: is the meaning of the video the same as the English text? What is the quality of the translation? Take into account ASL grammar and style. How accurately does the facial expression match the English text?]}
    \label{fig:userstudyresults}
\end{figure*}




\subsection{Data Analysis}

For the rating questions, we report descriptive statistics showing the proportions of each response option for each model type. To account for both fixed and random effects in our data, and to address small sample sizes and deviations from normality in data distributions, we conducted parametric bootstrap linear mixed model (LMM) analyses~\cite{davison1997bootstrap,pinheiro2006mixed}. These models include model type, sentence type, and participants' demographic variables---including gender, age category, ASL age, ASL proficiency, and frequency of ASL use---as fixed effects to assess their influence on the ratings. Participant ID was treated as a random effect to capture individual variability. For visual and motion quality evaluations, we conducted three LMM analyses--one each for understanding, visual quality, and naturalness of motion. Similarly, for translation quality evaluations, we conducted another three LMM analyses to assess the similarity of meaning between the generated videos and the English text, the signing quality (focusing on ASL grammar and style), and the accuracy of facial expressions in matching the English text. For open questions, we summarize participants' feedback to provide insight into their experiences and perceptions.

To further evaluate our system's translation quality, three authors with ASL experience (1 fluent Deaf signer; 1 fluent hearing signer; 1 novice hearing signer) independently rated the participant-provided translations relative to the English annotations from the ASLLRP dataset. This evaluation assessed whether each translation was semantically equivalent to the target phrase. A 5-point scale was used, defined as follows: 4 = the idea is the same (The same); 3 = the idea is evident but contains one error, such as question changed to a statement, one word error, or one missing element (Similar); 2 = the idea is somewhat similar but unclear or contains multiple errors (Acceptable); 1 = some semblance of the idea is present (Poor); 0 = little to no resemblance to the target (Completely different). Pairwise Pearson correlations~\cite{lee1988thirteen} were conducted and showed the high agreement among the ratings of the three evaluators, with Pearson's correlation coefficients ranging from $r = 0.860$ to $0.946$ ($p < .001$). For all LMM analyses with model type as a fixed effect, additional pairwise post-hoc comparisons with Holm corrections~\cite{holm1979simple} were conducted to identify specific factors influencing translation quality.

% Quantitative results from each of the ratings questions across the study can be found in Figure \ref{fig:userstudyresults}. We report descriptive statistics, including proportions, for rating questions and we summarize open-ended responses. 
%For open-ended responses, we performed inductive coding to identify patterns.


\subsection{Findings}\label{subsec:study_findings}

\subsubsection{Visual and Motion Quality Evaluation Findings}
Figure \ref{fig:visual_motion_quality} shows results for Section \ref{subsubsec:sec1}. Regarding the understandability of the generated signed videos from two model types, in the best-case scenario, where raw ASLLRP skeleton data was retargeted using the pose-to-video model from Module 3 (\textit{Video Retargeting)}, participants found 60.0\% of videos to be \textit{easy} or \textit{very easy} to understand, with 73.3\% to be at least \textit{neutral}. Results for \textit{naturalness} were very similar. For visual quality, perceptions were lower, with 32.3\% ratings being at least \textit{good} and 60.1\% with at least \textit{neutral}. When using our full model with human annotations from the ASLLRP dataset combined with linguistic information from our LLM (\textit{AI (Annotations)}), only 21.1\% of ratings indicated the videos were \textit{easy} or \textit{very easy} to understand. Naturalness and visual quality were both rated with lower scores compared to the \textit{Video Retargeting} approach. However, in open-ended responses, some participants commented positively about the body and face movements (\eg \textit{``Good Body Movements and some lip syncing their words (helpful for those who don't understand [the ASL sign])''}). Negative sentiments focused on issues like blurriness, cut off fingers, and the need for improved facial expressions. For example, \textit{``Blurred background is hard to read the signer''} and \textit{``[...] fingers cut off sometimes, needs more movement in the facial.''}


Our parametric bootstrap LMM analyses revealed significant main effects of model type and sentence type on understandability, visual quality, and naturalness of motion, with few demographic variables showing significant effects. For example, in visual quality ratings, model type showed a significant effect, $\chi^2(1)=54.53, p<0.001$, with a bootstrap $p$-value of 0.002, indicating that the retargeted model received significantly higher visual quality ratings than the \textit{AI (Annotations)} model. Sentence type also had a significant impact on visual quality ratings, $\chi^2(4) = 18.59, p < .001$. 
% \begin{wrapfigure}{r}{0.5\textwidth}
\begin{figure}[t]
    \centering
    {\includegraphics[scale=0.27]{figures/visual_quality_sentence_type_updated.png}}
    \caption{Violin plot illustrating the estimated marginal means for visual quality ratings by sentence type. Wh- and yes-no questions exhibit the highest visual quality ratings, whereas conditional sentences display the lowest ratings (all p values < .001). Error bars show 95\% confidence intervals. 
    \Description[This figure presents a violin plot showing the relationship between sentence type and visual quality ratings. The x-axis categorizes different sentence types, including conditional sentences ('condition_if'), yes-no questions ('ynq'), wh-questions ('whq'), other sentence types ('other'), and fingerspelling ('fs'). The y-axis represents estimated marginal means of visual quality ratings, with higher values indicating better quality. The plot reveals that wh-questions and yes-no questions achieve the highest visual quality ratings, as indicated by their higher central markers and broader density at the top. In contrast, conditional sentences tend to have the lowest visual quality ratings, as seen in their lower central markers and density concentrated near the bottom. ]}
    \label{fig:visual_quality_sentence_type}
\end{figure}
% \end{wrapfigure}
As illustrated in Figure \ref{fig:visual_quality_sentence_type}, Wh-questions and yes-no questions were rated highest in visual quality, while conditional sentences received the lowest ratings (Holm-corrected post-hoc tests: all $z > 3.7$, all $p$ < .001). Among demographic variables, no significant effects were found for gender ($\chi^2(1) = 0.029$, p = 0.972), age ($\chi^2(3) = 1.94, p = 0.584$), ASL age ($\chi^2(2) = 2.54, p = 0.281$), or ASL proficiency ($\chi^2(2) = 2.95, p = 0.229$). 

\subsubsection{Translation Quality Evaluation Findings} Figure \ref{fig:translation_quality} shows results for Section \ref{subsubsec:sec2}. As expected, the raw videos from the ASLLRP dataset were easiest to understand and had the highest similarity with the English sentences that were shown. Surprisingly, there were a small number of ASLLRP videos that had ``poor'' or ``different'' ratings. One participant noted that one of these raw videos had a \textit{``Lack of grammar and sentence structure but I can understand what he mean[s].''} For the other three models, the differences in translation quality were modest overall---except that the results using our translated glosses (\textit{AI (Full)}) achieved significantly higher ratings than the manually annotated glosses from the ASLLRP dataset (\textit{AI (Annotations)})s in terms of the meaning of the translation. Furthermore, incorporating non-manual markers (\ie facial expressions) in our full model resulted in higher acceptance compared to the same model without non-manual markers (\textit{AI (w/o Expr)}). The \textit{quality} of the translation, which focuses on ASL grammar and style, was rated as at least \textit{acceptable} in 65.3\% of cases with our full model. Similarly, the \textit{meaning} of the translation was at least acceptable 53.8\% of the time and facial quality was at least acceptable 48.5\% of the time.
% Emphasizing the potential benefits of adding facial expressions to SLG systems, the trend among the AI-generated videos was that our full model with added facial expressions is best and the version using the ``true'' ASLLRP glosses does second best.
% , and our translation model without added facial expressions performs notably worse. 

% Additional fixed effects, including sentence type and participant demographic variables, were also tested. Sentence type showed a significant effect, $\chi^2(4)=19.33, p<0.001$, indicating that different sentence types impacted understanding independently of model type. Further examination of the sentence types, as depicted in Figure \ref{fig:per_sentence} shows that Wh-questions were rated the highest, while sentences that included fingerspelling or conditional statements were rated as hardest to understand. No significant effects were observed for gender ($\chi^2$(1)=1.81, p=0.179), age group ($\chi^2$(3)=0.93, p=0.818), ASL age ($\chi^2$(2)=1.73, p=0.421), or ASL proficiency ($\chi^2$(2)=1.96, p=0.375), suggesting these factors did not influence understanding of the models.

Similar to the evaluation of visual and motion quality, our LMM analyses revealed a significant main effect of model type on all three aspects of translation quality (all $p < .001$, with bootstrap $p$-values of 0.002). Contrast analyses showed that the videos generated by the \textit{AI (Annotations)} model were significantly less similar in meaning to the provided English text compared to those produced by our full model (\textit{AI (Full)}; $z = 2.73$, $p < .05$). However, raw videos consistently received higher ratings than all model-generated outputs. For signing quality and the accuracy of facial expressions, contrast analyses indicated a significant difference between raw videos and all model outputs; however, no significant differences were observed among \textit{AI (Annotations)}, \textit{AI (Full)}, and \textit{AI (w/o Expr)}. Sentence type was also identified as a significant factor influencing translation quality. For example, signing quality ratings exhibited a significant main effect of sentence type ($\chi^2(3) = 227.27$, $p < .001$, with a bootstrap $p$-value of 0.002). However, the limited sample size for each sentence type restricted the scope for more detailed analyses. No demographic variables were  significant predictors of signing quality ratings.

% \begin{wrapfigure}{r}{0.5\textwidth}
\begin{figure}[t]
    \centering
    {\includegraphics[scale=0.29]{figures/translation_ratings_updated.png}}
    \caption{Violin plot illustrating the estimated marginal means for translation quality ratings by model type. \textit{AI (Full)} and \textit{AI (w/o Expressions)} were rated significantly more accurate than \textit{AI (Annotations)}, with no significant difference between them. Error bars represent 95\% confidence intervals. While \textit{Raw Video} was rated significantly better than other models, only significance among the other three models is marked for visual simplicity.
    \Description[This figure is a violin plot displaying the estimated marginal means of Average Translation Rating for different translation methods: AI (Annotations), AI (w/o Expr), AI (Full), and Raw Video. The y-axis ranges from 0 to 4, representing the average rating. The mean rating for each method is indicated by a dot and its numerical value is displayed (e.g., AI (Annotations): 3.15, AI (w/o Expr): 3.55, AI (Full): 3.42, Raw Video: 3.89). Vertical lines show the confidence intervals around the mean. A statistically significant difference is marked between the AI (Annotations) and AI (w/o Expr) conditions, as indicated by asterisks.]}

    \label{fig:translation_quality_ratings}
\end{figure}
% \end{wrapfigure}
Our additional LMM analysis, aimed at understanding how well participants' translations aligned with the intended sentences, revealed a strong effect of model type on average translation quality ratings, $\chi^2(3) = 50.45, p < .001$, with a bootstrap $p$-value of 0.002. As shown in Figure \ref{fig:translation_quality_ratings}, the quality of the translations provided by participants showed significant differences between model types, with \textit{AI (Annotations)} performing significantly worse than both \textit{AI (Full)} and \textit{AI (w/o Expr)} ($z = 3.200$, Holm-corrected $p < .01$). Although participants rated translations with non-manual markers as more acceptable, no significant differences were observed between the two models using our system, with and without non-manual markers ($z = 0.907, p = 0.364$). Sentence type also had a significant effect on the translation quality ratings, $\chi^2(7) = 23.34, p < .001$, with a bootstrap $p$-value of 0.002. In contrast, all demographic variables did not significantly influence translation quality ratings. 

% \han{Did we perform a comparison analysis between the raw video and our full model?}

% Three of the authors with ASL experience (1 Deaf; 1 expert; 1 novice) rated the quality of all translations that each participant wrote relative to the English annotations input into our model. The goal was to assess whether each translation was semantically the same as the target phrase. A 5-points scale was used with the following definitions: 4=idea is the same; 3=idea is evident but one element is wrong (\eg question to statement, one word error, one element missing); 2=idea is somewhat similar but unclear/has errors (Acceptable); 1=some semblance of the idea is present (Poor); 0=little to no resemblance to target (Completely different). 
% There was high correlation between each annotator (Pairwise Pearson $r$ between 0.860-0.946, p<.001). Rating averages are shown in  Figure \ref{fig:translation_ratings} where on average the raw videos had a score of 3.634, AI (Full) was 3.170, AI (w/o Expr) was 3.194, and AI (Annotations) was 2.905. The results between each model or raw video are statistical significant, except for between AI (Full) and AI (w/o Expression) which are not significantly different. 

% \begin{figure}[t]
    \centering
    \subfloat[Understanding ratings per sentence type.]{ \includegraphics[width=0.48\textwidth,valign=c]{figures/per_sentence_translation_ratings.png}
    \label{fig:per_sentence}}
    \subfloat[Semantic Translation Assessment.]{\includegraphics[width=0.35\textwidth,valign=c]{figures/translation_ratings.png}
    \label{fig:translation_ratings}}
    \caption{
% \todo{Update results and caption.} 
    Violin plots show estimated marginal means of translation ratings for each of the four video conditions. Translations of raw videos were rated significantly more accurate than all AI video conditions. Indicated on the figure, translations of AI (Full) and AI (w/o Expression) models were significantly more accurate than ratings of AI w/ annotation videos. Error bars show 95 percent confidence intervals. ** denotes p < .01.
    % Signed videos were generated using the following approaches: \textit{AI w/ Annotation} (leveraging human-annotated English-based glosses (manual markers) from the ASLLRP dataset, along with our manually annotated linguistic information (non-manual markers), as input for Modules 2 and 3 of our prototype system), \textit{Video Retargeting} (using skeletal poses from the ASLLRP dataset as inputs for Module 3), and \textit{AI w/ LLM} (our prototype system). All these approaches trained on the How2Sign dataset to generate the final signed output. In contrast, \textit{Raw Video} samples were directly taken from the ASLLRP dataset without additional processing or training.
    % \Description[This figure presents the results from two sections of our user study evaluating video quality. It contains two sub-figures. Each sub-figure displays the percentage of responses, with the right side (blue) representing better ratings and the left side (red) representing worse rating. The left sub-figure focused on visual and motion quality and the right sub-figure focused on translation quality. In the left sub-figure, it contains three questions: how easy is it to understand this video? Rate the visual quality of the signing in this video. Rate the naturalness of the motion. In the right sub-figure, it also contains three questions: is the meaning of the video the same as the English text? What is the quality of the translation? Take into account ASL grammar and style. How accurately does the facial expression match the English text?]}
    \label{fig:userstudyresults}
    }
\end{figure}



% Participants had checkbox options along with an open-ended text box for reporting issues. 
Participants reported several issues with both our full model and our model using ASLLRP human gloss annotations, with the majority of concerns pertained to image and motion quality. However, there were a small number of comments on ``missing information'' and ``wrong signs.'' One participant noted, \textit{``The signing in the beginning looks very laggy, maybe avoid spelling out the words,''} referring to limitations in fingerspelling where individual letters appeared to jump between locations or were signed  slowly compared to natural signing. 

\subsubsection{Interest of AI Signing Technology and Its Use Cases} 
% Follow-up questions asked about participant's general interest in AI Signing technology and primarily consisted of open-ended feedback.
Participants expressed varying levels of interest in AI signing technology. One highly enthusiastic participant remarked, 
\textit{``Everything looks good so far, most of the ASL is correct, definitely on the right path. This would be a great tool and technology for those who struggle with communication in the hearing community. It's super convenient and I can't wait. Thank you for allowing me to be a part of this,''} indicating their inclination to sign up to use such a technology in the future. The least interested person highlighted that the quality of the technology is far from being useful, stating, \textit{``I am not interested seeing AI signing technology because it's too complicated to understand the ASL signer.''} Despite this, the same participant later expressed that the technology could be valuable for certain use cases. 

When asked about their interest in photorealistic, cartoon, or 3D avatars to represent AI signers, participants provided mixed feedback, but with a lean towards photorealistic styles. One participant emphasized the value of realism, stating,
\textit{``Realistic and Authentic[---]it is simpler for viewers to relate to and believe in the content when a live signer offers an honest and realistic experience. It better for training and teaching other people ASL.''} 
Stylized signers could be of interest for social media, advertisements, or children's content, but multiple people noted the importance of ensuring the stylized depiction is capable of conveying nuance of sign language: \textit{``I think more stylized appearance can do, but needs [to be] clear in image and facial expressions.''}
% In contrast, another participant expressed a preference for cartoon avatars over 3D models, noting, \textit{``[...] would prefer a cartoon over a 3D avatar.''} The most detailed comment highlighted that preferences are context-dependent, suggesting that different aesthetics might be more suitable for varying scenarios, such as a doctor's office versus a TV show. \todo{I would add the comment here}

Participants mentioned a wide range of potential use cases for AI signing technology. Many of these examples related to simultaneous recognition and generation of ASL for real-time social interactions. Others focused on one-sided interactions, such as ASL generation of live presentations. 
% , which we categorized into two main groups: social interactions and one-sided interactions. 
% For social interactions, common contexts included: Doctors office (mentioned 5 times); Video or in-person conversations (5x); Customer service (4x); Education (3x); Smart assistant (2x); Drive-thru services (2x); Live presentations (1x). For one-sided interactions, participants reported use cases including: Transportation (5x); Internet “captions” (5x); Public announcements (3x); Emergency info (3x); Museum (2x); Entertainment (1x); Radio (1x); Books (1x); For hearing people to practice/learn (1x).

% Participants suggested a range of potential use cases for AI signing technology, which we categorized into two main groups: social interactions and one-sided interactions. For social interactions, common contexts included: Doctors office (mentioned 5 times); Video or in-person conversations (5x); Customer service (4x); Education (3x); Smart assistant (2x); Drive-thru services (2x); Live presentations (1x). For one-sided interactions, participants reported use cases including: Transportation (5x); Internet “captions” (5x); Public announcements (3x); Emergency info (3x); Museum (2x); Entertainment (1x); Radio (1x); Books (1x); For hearing people to practice/learn (1x).

% It is important to note that this study is preliminary, but feedback on translation quality thus far suggests that our prototype is heading towards a good direction, and with some improvements, participants could find it compelling for situations where a live interpreter is not available. These responses also suggest that much more work is needed to achieve a high-enough motion and visual quality that DHH signers would find valuable. 
\section{Discussion}\label{sec:discussion}

Our goal was to develop a prototype ASL generation system, addressing key challenges limiting real-world applicability of existing SLG systems, and to explore whether DHH signers would find this technology useful. Below, we reflect on our design process, provide key insights learned, identify areas for improvement, and discuss computational and ethical considerations in the use of our system.

\subsection{Technical Insights from the Design and Evaluation Process}

During the design process and evaluations with DHH participants, we gained valuable technical insights that informed our choices and identified areas for future improvement. One key finding was the importance of careful data handling for translation tasks. Our ablation study results, as shown in Table~\ref{tab:text-to-gloss_eval_results}, highlight the importance of data preprocessing, increasing the number of examples provided to the model, and constraining the translation within the pre-generated vocabulary to improve model translation performance in the low-resource settings. Considerable effort was dedicated to creating an annotation scheme that not only accurately represents ASL signs and sentences but also functions effectively when used with the LLM and the rest of our prototype. This points to a fundamental challenge with glossing: the diverse definitions and interpretations of ASL glosses. Standardization across datasets could mitigate this issue and improve accuracy by allowing the combination of different data resources~\cite{bragg_sign_2019}.

Our use of an LLM for generating both manual and non-manual information demonstrate potential, with the model achieving a BLEU-4 of 0.276 for translating English sentences from the ASLLRP dataset into ASL glosses. While direct comparisons---such as running our dataset on other systems or applying our system to other datasets---are challenging due to the inaccessibility of other datasets and systems, this represents highest reported score for such translation task in the literature, highlighting the effectiveness of few-shot prompting techniques in handling low-resource languages. More than half of the time, DHH participants found the meaning of the generated videos ``Acceptable'', ``Similar'' or ``The Same'' when compared to the English text. However,  in close to 50\% of the examples, they rated our translations as ``Poor'' or ``Very poor'' concerning ASL grammar and style, indicating a need for further improvement in aligning the output with native signing conventions. 

Our additional experiments on English Text-to-ASL gloss translation using Retrieval Augmented Generation (RAG)~\cite{lewis2020retrieval} demonstrated improved performance, achieving a BLEU-4 score of 0.279 $\pm$ 0.003. These results suggest potential for further enhancement in translation accuracy. Detailed descriptions of the experiments are provided in Appedix \ref{appendix:additional_text-to-gloss}. Beyond translation accuracy, our innovation on extracting non-manual markers directly from the English text using zero-shot prompting, could potentially enhance the naturalness and grammatical accuracy of the generated videos. Nonetheless, some linguistic features were misidentified due to inconsistencies between gloss annotations and English sentences (as discussed in Section \ref{subsubsec:linguistic_predictions}), suggesting the need for prompt fine-tuning or more targeted examples. 

The use of a Motion Matching approach for generating skeletal pose sequences offered both promise and challenges. By optimizing for ``economy of motion,'' this method enabled smoother transitions between signs, contributing to more fluid and natural signing overall. However, we encountered issues with fingerspelling, where unintended movements appeared between letters, disrupting the continuity of motion. This challenge was also noted in user feedback, highlighting gaps in achieving the desired naturalness in coarticulations, particularly for complex cases such as fingerspelling. The noticeable naturalness rating difference between the full model and the retargeted approach---where only in 34.5\% of the cases participants perceived the naturalness of our videos as ``Neutral'' or better, compared to 71.1\% for the retargeted version (results shown in Figure~\ref{fig:visual_motion_quality})---emphasizes the need for refining our skeletal motion generation method. 

A key factor limiting the adoption of existing SLG systems by DHH users is the low quality of the generated signing videos, which are often described as robotic or blurry ~\cite{kipp2011assessing,tran2023us,huenerfauth2009sign,quandt2022attitudes}. Our technical evaluations, as detailed in Table~\ref{tab:pose-to-video_eval_results}, demonstrate that our approach improves the visual quality of the generated videos by systematically eliminating data errors, such as missing landmarks, blurriness, and temporal inconsistencies in landmark positioning, through using only the highest-quality frames. However, we still observe a gap between these technical improvements and practical usability, as in 77.8\% of the time DHH participants found the visual quality of our signing videos to be ``Poor'' or ``Very Poor''. Additionally, participants noted that head movements did not consistently align with the camera. 
Future work could explore integrating more advanced generative models such as diffusion models~\cite{croitoru2023diffusion,yang2023diffusion} to enhance video quality.

\subsection{A Need for Larger, High-quality, and Comprehensive ASL Datasets}

Despite using the largest and highest-quality ASL datasets available, the chosen datasets still suffer from several limitations. The ASLLRP dataset is advantageous in that it contains tens of thousands of videos with comprehensive annotations (\eg glosses, English sentences, non-manual markers). However, the dataset suffers from limited visual quality due to issues such as image resolution and motion blur, which proved challenging for generating compelling image-to-image models during our initial experiments. When we turned to the How2Sign dataset for training image-to-image models, we found that the visual quality was significantly better. However, this introduced inconsistencies between datasets. For example, signers in the ASLLRP dataset tend to be seated and looking at prompts away from the camera; while signers in the How2Sign dataset maintain direct eye contact and are looking closer at the camera, slightly sideways. These discrepancies, coupled with the relatively small size of these datasets, highlight the need for more comprehensive and consistent ASL datasets.

Furthermore, the reliance on human-labeled gloss annotations in existing ASL datasets introduces multiple sources of errors and inconsistencies. 
% Our evaluation of predictions of linguistic features from a given English sentence, particularly non-manual from English sentences (Section \ref{subsubsec:linguistic_predictions}), revealed several discrepancies, especially in identifying negation. These inconsistencies appear to result from the annotation process itself; annotations were based on the signing videos in the ASLLRP dataset rather than the corresponding English sentences, leading to misalignment. Moreover, 
While many English sentences in the ASLLRP dataset are derived from context-free ASL utterances translated into glosses and English text, others come from longer narrative videos. In these cases, accurate translation requires full contextual understanding, which the annotations may not always provide~\cite{tanzer2024reconsidering}. Consequently, for many of these context-dependent sentences, our text-to-gloss translations may be more accurate than the original human annotated glosses. This is reflected in our model's performance, where we achieved a BLEU-4 score of 0.305 without these context-dependent sentences (52 in the test set), compared to 0.276 with them. Further supporting this observation, our user study indicated that DHH users rated the quality of our translations, specifically regarding the meaning of video compared to the English text, to be more acceptable than the manually annotated glosses provided by the ASLLRP dataset. These findings highlight the critical need for more robust, high-quality datasets with standardized annotation practices to support the development of effective SLG systems~\cite{bragg_sign_2019}.

\subsection{Addressing the Complexities of ASL in Sign Language Generation Technologies}

The complexities of ASL grammar present challenges for developing effective SLG technologies. While general guidelines for ASL grammar exist, the language, like all natural languages, does not always adhere to rigid grammatical structures in everyday use. This complexity is evident in the mixed results from our experiments, where attempts to provide grammar guidelines to the LLM did not consistently enhance translation performance. Many examples in the ASLLRP dataset, while grammatically correct, diverge from these general guidelines (as shown in Table \ref{tab:text-to-gloss_eval_results}). Feedback from our user study, which highlighted stylistic and grammatical errors, emphasizes the need for a more nuanced computational understanding of how ASL is used in diverse, real-world contexts to improve. Furthermore, regional variations within a single language and differences across multiple sign languages introduce additional layers of complexity that remain to be addressed. 
 
This work studies several aspects of both manual and non-manual markers in ASL morphology, lexicon, and syntax, such as compounds, agreement verbs (directional verbs indicating agreement with the subject and object), fingerspelling, and name signs (more details can be found in Table~\ref{tab:gloss_convention}). However, these linguistic features are analyzed only within the context of the dataset used in this study, which does not capture the full range of their usage in ASL. Additionally, several other facets of ASL grammar and usage remain unexplored. For example, we excluded one type of manual marker, classifiers, due to limited data available to model them accurately. Classifiers, which are essential for conveying nuanced meanings and spatial relationships in ASL, require context-aware data and more sophisticated modeling approaches. As SLG systems evolve towards context-dependent applications, incorporating classifiers will be critical for enhancing the naturalness and expressiveness of the generated signs. Additionally, our work focuses primarily on eyebrow movements, one type of facial expressions within non-manual markers, used to indicate questions, conditional statements, and negation. However, non-manual markers in ASL consist of a wide range of features, including head tilts, mouth shapes, and body posture, which also contribute to the grammar and meaning of signed sentences~\cite{Stokoe1961SignLS,klima1979signs,brentari2002prosody}. Future work is needed to expand the modeling of these additional markers to capture the full complexity of ASL.

Moreover, our study focused on context-free SLG, where each sentence is generated independently. However, sign languages heavily use indexing and spatial referencing, such as referencing people or places mentioned earlier in a conversation~\cite{winston1991spatial,friedman1975space}. Our current prototype system lacks the capacity to remember or track these spatial references over multiple utterances. Additionally, types of signing like storytelling often involve more extensive use of expressions, classifiers, spatial references, and role shifting than our prototype can currently support. Addressing these challenges will require more data, modeling, and interdisciplinary collaboration with ongoing feedback from the DHH and signing communities. 

\subsection{Computational and Ethical Considerations}

While both technical and human evaluations demonstrate the potential of our prototype system, and the modular approach offers flexibility by enabling individual components to be improved or replaced as technologies advance, there are several computational and ethical considerations that should be carefully addressed when using or further improving the system. First, the current prototype requires running GPT-4o inference for every generation instance with longer prompts, which introduces computational and financial costs, as well as scalability challenges, particularly for real-time or large-scale applications. Optimization techniques or lighter models may need to be explored to address this issue. Second, the nature of modular approach can lead to the loss of information between stages, computational inefficiencies, or biases imposed by external constraints at each module. Addressing these shortcomings will require careful integration of modules. Third, the use of LLMs might pose a risk of generating inappropriate or offensive language, which could introduce harm to the DHH community or undermine their trust in using such system. As emphasized in both academia and industry (\eg Apple's Responsible AI white paper~\cite{applewhitepaper}), designing AI tools with care to proactively mitigate potential harms must be a top priority. This includes implementing content filtering mechanisms, rigorous validation processes, and culturally sensitive design practices to ensure that the system outputs are respectful, inclusive, and aligned with community expectations.



% \todo{need to mention somewhere in the text the difficulty of recruiting DHH participants, while also emphasizing the importance of their involvement in evaluating the system for its real-world usability}

% \todo{CL: consider adding paragraph about how this work was developed, interactions with the DHH community, and limitations or suggestions for how others might think about this. }
% Ultimately, while this work was done in collaboration with people in the Deaf community, the majority of authors area hearing and only half of them are fluent or have taken classes in ASL. 
% Continued work should ...

\section{Conclusion}

In this paper, we introduce STeCa, a novel agent learning framework designed to enhance the performance of LLM agents in long-horizon tasks. 
STeCa identifies deviated actions through step-level reward comparisons and constructs calibration trajectories via reflection. 
These trajectories serve as critical data for reinforced training. Extensive experiments demonstrate that STeCa significantly outperforms baseline methods, with additional analyses underscoring its robust calibration capabilities.

% This section has a special environment:

\begin{acks}
  We sincerely thank all reviewers for their valuable feedback, which significantly enhanced our work. We also extend our gratitude to the participants of our user study for their time and contributions. Lastly, we deeply appreciate Gus Shitama, Julia Sohnen, Pooja Solanki, Sheridan Laine, and Antony Kennedy for their insightful discussions and support with study-related tasks. 
\end{acks}



%%
%% The next two lines define the bibliography style to be used, and
%% the bibliography file.
\bibliographystyle{ACM-Reference-Format}
% \bibliography{asl}
\bibliography{zotero_asl} %the first one is from zotero

\newpage

%% If your work has an appendix, this is the place to put it.
\section{Hard Threshold of EAC}\label{threshhold}
In constructing a weighted-gradient saliency map, the value of \(\gamma\) determines the number of the dimensions we select where important feature anchors are located. As the value of \(\gamma\) increases, the number of selected dimensions decreases, requiring the editing information to be compressed into a smaller space during the compression process. 
During compression, it is desired for the compression space to be as small as possible to preserve the general abilities of the model. However, reducing the compression space inevitably increases the loss of editing information, which reduces the editing performance of the model.
Therefore, to ensure editing performance in a single editing scenario, different values of \(\gamma\) are determined for various models, methods, and datasets. Fifty pieces of knowledge were randomly selected from the dataset, and reliability, generalization, and locality were measured after editing. The averages of these metrics were then taken as a measure of the editing performance of the model.
Table~\ref{value} presents the details of \(\gamma\), while Table~\ref{s} illustrates the corresponding editing performance before and after the introduction of EAC. $P_{x}$ denotes the value below which x\% of the values in the dataset.


\begin{table}[!htb]
\caption{The value of $\gamma$.}
\centering
\resizebox{0.45\textwidth}{!}{
\begin{tabular}{lcccc}
\toprule
\textbf{Datasets} & \textbf{Model} & \textbf{ROME} & \textbf{MEMIT} \\
\midrule
\multirow{2}{*}{\textbf{ZSRE}} & GPT-2 XL & $P_{80}$ & $P_{80}$ \\
 & LLaMA-3 (8B) & $P_{90}$ & $P_{95}$ \\
\midrule
\multirow{2}{*}{\textbf{COUNTERFACT}} & GPT-2 XL & $P_{85}$ & $P_{85}$ \\
 & LLaMA-3 (8B) & $P_{95}$ & $P_{95}$ \\
\bottomrule
\end{tabular}}
\label{value}
\end{table}


\begin{table}[!htb]
\caption{The value of $\gamma$.}
\centering
\resizebox{\textwidth}{!}{%
\begin{tabular}{lccccccccccccc}
\toprule
\multirow{1}{*}{Dataset} & \multirow{1}{*}{Method} & \multicolumn{3}{c}{\textbf{GPT-2 XL}} & \multicolumn{3}{c}{\textbf{LLaMA-3 (8B)}} \\
\cmidrule(lr){3-5} \cmidrule(lr){6-8}
& & \multicolumn{1}{c}{Reliability} & \multicolumn{1}{c}{Generalization} & \multicolumn{1}{c}{Locality} & \multicolumn{1}{c}{Reliability} & \multicolumn{1}{c}{Generalization} & \multicolumn{1}{c}{Locality} \\
\midrule
\multirow{1}{*}{ZsRE} & ROME & 1.0000 & 0.9112 & 0.9661 & 1.0000 & 0.9883 & 0.9600  \\
& ROME-EAC & 1.0000 & 0.8923 & 0.9560 & 0.9933 & 0.9733 & 0.9742  \\
\cmidrule(lr){2-8}
& MEMIT & 0.6928 & 0.5208 & 1.0000 & 0.9507 & 0.9333 & 0.9688  \\
& MEMIT-EAC & 0.6614 & 0.4968 & 0.9971 & 0.9503 & 0.9390 & 0.9767  \\
\midrule
\multirow{1}{*}{CounterFact} & ROME & 1.0000 & 0.4200 & 0.9600 & 1.0000 & 0.3600 & 0.7800  \\
& ROME-EAC & 0.9800 & 0.3800 & 0.9600 & 1.0000 & 0.3200 & 0.8800  \\
\cmidrule(lr){2-8}
& MEMIT & 0.9000 & 0.2200 & 1.0000 & 1.0000 & 0.3800 & 0.9500  \\
& MEMIT-EAC & 0.8000 & 0.1800 & 1.0000 & 1.0000 & 0.3200 & 0.9800  \\
\bottomrule
\end{tabular}%
}
\label{s}
\end{table}

\section{Optimization Details}\label{b}
ROME derives a closed-form solution to achieve the optimization:
\begin{equation}
\text{minimize} \ \| \widehat{W}K - V \| \ \text{such that} \ \widehat{W}\mathbf{k}_* = \mathbf{v}_* \ \text{by setting} \ \widehat{W} = W + \Lambda (C^{-1}\mathbf{k}_*)^T.
\end{equation}

Here \( W \) is the original matrix, \( C = KK^T \) is a constant that is pre-cached by estimating the uncentered covariance of \( \mathbf{k} \) from a sample of Wikipedia text, and \( \Lambda = (\mathbf{v}_* - W\mathbf{k}_*) / ( (C^{-1}\mathbf{k}_*)^T \mathbf{k}_*) \) is a vector proportional to the residual error of the new key-value pair on the original memory matrix.

In ROME, \(\mathbf{k}_*\) is derived from the following equation:
\begin{equation}
\mathbf{k}_* = \frac{1}{N} \sum_{j=1}^{N} \mathbf{k}(x_j + s), \quad \text{where} \quad \mathbf{k}(x) = \sigma \left( W_{f_c}^{(l^*)} \gamma \left( a_{[x],i}^{(l^*)} + h_{[x],i}^{(l^*-1)} \right) \right).
\end{equation}

ROME set $\mathbf{v}_* = \arg\min_z \mathcal{L}(z)$, where the objective $\mathcal{L}(z)$ is:
\begin{equation}
\frac{1}{N} \sum_{j=1}^{N} -\log \mathbb{P}_{G(m_{i}^{l^*}:=z))} \left[ o^* \mid x_j + p \right] + D_{KL} \left( \mathbb{P}_{G(m_{i}^{l^*}:=z)} \left[ x \mid p' \right] \parallel \mathbb{P}_{G} \left[ x \mid p' \right] \right).
\end{equation}

\section{Experimental Setup} \label{detail}

\subsection{Editing Methods}\label{EM}

In our experiments, Two popular editing methods including ROME and MEMIT were selected as baselines.

\textbf{ROME} \cite{DBLP:conf/nips/MengBAB22}: it first localized the factual knowledge at a specific layer in the transformer MLP modules, and then updated the knowledge by directly writing new key-value pairs in the MLP module.

\textbf{MEMIT} \cite{DBLP:conf/iclr/MengSABB23}: it extended ROME to edit a large set of facts and updated a set of MLP layers to update knowledge.

The ability of these methods was assessed based on EasyEdit~\cite{DBLP:journals/corr/abs-2308-07269}, an easy-to-use knowledge editing framework which integrates the released codes and hyperparameters from previous methods.

\subsection{Editing Datasets}\label{dat}
In our experiment, two popular model editing datasets \textsc{ZsRE}~\cite{DBLP:conf/conll/LevySCZ17} and \textsc{CounterFact}~\cite{DBLP:conf/nips/MengBAB22} were adopted.

\textbf{\textsc{ZsRE}} is a QA dataset using question rephrasings generated by back-translation as the equivalence neighborhood.
Each input is a question about an entity, and plausible alternative edit labels are sampled from the top-ranked predictions of a BART-base model trained on \textsc{ZsRE}.

\textbf{\textsc{CounterFact}} accounts for counterfacts that start with low scores in comparison to correct facts. It constructs out-of-scope data by substituting the subject entity for a proximate subject entity sharing a predicate. This alteration enables us to differentiate between superficial wording changes and more significant modifications that correspond to a meaningful shift in a fact. 

\subsection{Metrics for Evaluating Editing Performance}\label{Mediting performance}
\paragraph{Reliability} means that given an editing factual knowledge, the edited model should produce the expected predictions. The reliability is measured as the average accuracy on the edit case:
\begin{equation}
\mathbb{E}_{(x'_{ei}, y'_{ei}) \sim \{(x_{ei}, y_{ei})\}} \mathbf{1} \left\{ \arg\max_y f_{\theta_{i}} \left( y \mid x'_{ei} \right) = y'_{ei} \right\}.
\label{rel}
\end{equation}

\paragraph{Generalization} means that edited models should be able to recall the updated knowledge when prompted within the editing scope. The generalization is assessed by the average accuracy of the model on examples uniformly sampled from the equivalence neighborhood:
\begin{equation}
\mathbb{E}_{(x'_{ei}, y'_{ei}) \sim N(x_{ei}, y_{ei})} \mathbf{1} \left\{ \arg\max_y f_{\theta_{i}} \left( y \mid x'_{ei} \right) = y'_{ei} \right\}.
\label{gen}
\end{equation}

\paragraph{Locality} means that the edited model should remain unchanged in response to prompts that are irrelevant or the out-of-scope. The locality is evaluated by the rate at which the edited model's predictions remain unchanged compared to the pre-edit model.
\begin{equation}
\mathbb{E}_{(x'_{ei}, y'_{ei}) \sim O(x_{ei}, y_{ei})} \mathbf{1} \left\{ f_{\theta_{i}} \left( y \mid x'_{ei} \right) = f_{\theta_{i-1}} \left( y \mid x'_{ei} \right) \right\}.
\label{loc}
\end{equation}

\subsection{Downstream Tasks}\label{pro}

Four downstream tasks were selected to measure the general abilities of models before and after editing:
\textbf{Natural language inference (NLI)} on the RTE~\cite{DBLP:conf/mlcw/DaganGM05}, and the results were measured by accuracy of two-way classification.
\textbf{Open-domain QA} on the Natural Question~\cite{DBLP:journals/tacl/KwiatkowskiPRCP19}, and the results were measured by exact match (EM) with the reference answer after minor normalization as in \citet{DBLP:conf/acl/ChenFWB17} and \citet{DBLP:conf/acl/LeeCT19}.
\textbf{Summarization} on the SAMSum~\cite{gliwa-etal-2019-samsum}, and the results were measured by the average of ROUGE-1, ROUGE-2 and ROUGE-L as in \citet{lin-2004-rouge}.
\textbf{Sentiment analysis} on the SST2~\cite{DBLP:conf/emnlp/SocherPWCMNP13}, and the results were measured by accuracy of two-way classification.

The prompts for each task were illustrated in Table~\ref{tab-prompt}.

\begin{table*}[!htb]
% \small
\centering
\begin{tabular}{p{0.95\linewidth}}
\toprule

NLI:\\
\{\texttt{SENTENCE1}\} entails the \{\texttt{SENTENCE2}\}. True or False? answer:\\

\midrule

Open-domain QA:\\
Refer to the passage below and answer the following question. Passage: \{\texttt{DOCUMENT}\} Question: \{\texttt{QUESTION}\}\\

\midrule

Summarization:\\
\{\texttt{DIALOGUE}\} TL;DR:\\

\midrule


Sentiment analysis:\\
For each snippet of text, label the sentiment of the text as positive or negative. The answer should be exact 'positive' or 'negative'. text: \{\texttt{TEXT}\} answer:\\

\bottomrule
\end{tabular}
\caption{The prompts to LLMs for evaluating their zero-shot performance on these general tasks.}
\label{tab-prompt}
\end{table*}

\subsection{Hyper-parameters for Elastic Net}\label{hy}

In our experiment, we set \(\lambda = 5 \times 10^{-7} \), \(\mu = 5 \times 10^{-1} \) for GPT2-XL\cite{radford2019language} and \(\lambda = 5 \times 10^{-7} \), \(\mu = 1 \times 10^{-3} \) for LLaMA-3 (8B)\cite{llama3}.

\begin{figure*}[!hbt]
  \centering
  \includegraphics[width=0.5\textwidth]{figures/legend_edit.pdf}
  \vspace{-4mm}
\end{figure*}

\begin{figure*}[!hbt]
  \centering
  \subfigure{
  \includegraphics[width=0.23\textwidth]{figures/ROME-GPT2XL-CF-edit.pdf}}
  \subfigure{
  \includegraphics[width=0.23\textwidth]{figures/ROME-LLaMA3-8B-CF-edit.pdf}}
  \subfigure{
  \includegraphics[width=0.23\textwidth]{figures/MEMIT-GPT2XL-CF-edit.pdf}}
  \subfigure{
  \includegraphics[width=0.23\textwidth]{figures/MEMIT-LLaMA3-8B-CF-edit.pdf}}
  \caption{Edited on CounterFact, editing performance of edited models using the ROME~\cite{DBLP:conf/nips/MengBAB22} and MEMIT~\cite{DBLP:conf/iclr/MengSABB23} on GPT2-XL~\cite{radford2019language} and LLaMA-3 (8B)~\cite{llama3}, as the number of edits increases before and after the introduction of EAC.}
  \vspace{-4mm}
  \label{edit-performance-cf}
\end{figure*}

\begin{figure*}[!hbt]
  \centering
  \includegraphics[width=0.75\textwidth]{figures/legend.pdf}
  \vspace{-4mm}
\end{figure*}

\begin{figure*}[!htb]
  \centering
  \subfigure{
  \includegraphics[width=0.23\textwidth]{figures/ROME-GPT2XL-CounterFact.pdf}}
  \subfigure{
  \includegraphics[width=0.23\textwidth]{figures/ROME-LLaMA3-8B-CounterFact.pdf}}
  \subfigure{
  \includegraphics[width=0.23\textwidth]{figures/MEMIT-GPT2XL-CounterFact.pdf}}
  \subfigure{
  \includegraphics[width=0.23\textwidth]{figures/MEMIT-LLaMA3-8B-CounterFact.pdf}}
  \caption{Edited on CounterFact, performance on general tasks using the ROME~\cite{DBLP:conf/nips/MengBAB22} and MEMIT~\cite{DBLP:conf/iclr/MengSABB23} on GPT2-XL~\cite{radford2019language} and LLaMA-3 (8B)~\cite{llama3}, as the number of edits increases before and after the introduction of EAC.}
  \vspace{-4mm}
  \label{task-performance-cf}
\end{figure*}

\section{Experimental Results}\label{app}

\subsection{Results of Editing Performance}\label{cf-performance}
Applying CounterFact as the editing dataset, Figure~\ref{edit-performance-cf} presents the editing performance of the ROME~\cite{DBLP:conf/nips/MengBAB22} and MEMIT~\cite{DBLP:conf/iclr/MengSABB23} methods on GPT2-XL~\cite{radford2019language} and LLaMA-3 (8B)~\cite{llama3}, respectively, as the number of edits increases before and after the introduction of EAC.
The dashed line represents the ROME or MEMIT, while the solid line represents the ROME or MEMIT applying the EAC.


\subsection{Results of General Abilities}\label{cf-ability}
Applying CounterFact as the editing dataset, Figure~\ref{task-performance-cf} presents the performance on general tasks of edited models using the ROME~\cite{DBLP:conf/nips/MengBAB22} and MEMIT~\cite{DBLP:conf/iclr/MengSABB23} methods on GPT2-XL~\cite{radford2019language} and LLaMA-3 (8B)~\cite{llama3}, respectively, as the number of edits increases before and after the introduction of EAC. 
The dashed line represents the ROME or MEMIT, while the solid line represents the ROME or MEMIT applying the EAC.

\subsection{Results of Larger Model}\label{13 B}
To better demonstrate the scalability and efficiency of our approach, we conducted experiments using the LLaMA-2 (13B)~\cite{DBLP:journals/corr/abs-2307-09288}.
Figure~\ref{13B-edit} presents the editing performance of the ROME~\cite{DBLP:conf/nips/MengBAB22} and MEMIT~\cite{DBLP:conf/iclr/MengSABB23} methods on LLaMA-2 (13B)
~\cite{DBLP:journals/corr/abs-2307-09288}, as the number of edits increases before and after the introduction of EAC.
Figure~\ref{13B} presents the performance on general tasks of edited models using the ROME and MEMIT methods on LLaMA-2 (13B), as the number of edits increases before and after the introduction of EAC.
The dashed line represents the ROME or MEMIT, while the solid line represents the ROME or MEMIT applying the EAC.

\begin{figure*}[!hbt]
  \centering
  \includegraphics[width=0.5\textwidth]{figures/legend_edit.pdf}
  \vspace{-4mm}
\end{figure*}

\begin{figure*}[!hbt]
  \centering
  \subfigure{
  \includegraphics[width=0.23\textwidth]{figures/ROME-LLaMA2-13B-ZsRE-edit.pdf}}
  \subfigure{
  \includegraphics[width=0.23\textwidth]{figures/MEMIT-LLaMA2-13B-ZsRE-edit.pdf}}
  \subfigure{
  \includegraphics[width=0.23\textwidth]{figures/ROME-LLaMA2-13B-CF-edit.pdf}}
  \subfigure{
  \includegraphics[width=0.23\textwidth]{figures/MEMIT-LLaMA2-13B-CF-edit.pdf}}
  \caption{Editing performance of edited models using the ROME~\cite{DBLP:conf/nips/MengBAB22} and MEMIT~\cite{DBLP:conf/iclr/MengSABB23} on LLaMA-2 (13B)~\cite{DBLP:journals/corr/abs-2307-09288}, as the number of edits increases before and after the introduction of EAC.}
  \vspace{-4mm}
  \label{13B-edit}
\end{figure*}

\begin{figure*}[!hbt]
  \centering
  \includegraphics[width=0.75\textwidth]{figures/legend.pdf}
  \vspace{-4mm}
\end{figure*}

\begin{figure*}[!htb]
  \centering
  \subfigure{
  \includegraphics[width=0.23\textwidth]{figures/ROME-LLaMA2-13B-ZsRE.pdf}}
  \subfigure{
  \includegraphics[width=0.23\textwidth]{figures/MEMIT-LLaMA2-13B-ZsRE.pdf}}
  \subfigure{
  \includegraphics[width=0.23\textwidth]{figures/ROME-LLaMA2-13B-CounterFact.pdf}}
  \subfigure{
  \includegraphics[width=0.23\textwidth]{figures/MEMIT-LLaMA2-13B-CounterFact.pdf}}
  \caption{Performance on general tasks using the ROME~\cite{DBLP:conf/nips/MengBAB22} and MEMIT~\cite{DBLP:conf/iclr/MengSABB23} on LLaMA-2 (13B)~\cite{DBLP:journals/corr/abs-2307-09288}, as the number of edits increases before and after the introduction of EAC.}
  \vspace{-4mm}
  \label{13B}
\end{figure*}

\section{Analysis of Elastic Net}
\label{FT}
It is worth noting that the elastic net introduced in EAC can be applied to methods beyond ROME and MEMIT, such as FT~\cite{DBLP:conf/emnlp/CaoAT21}, to preserve the general abilities of the model.
Unlike the previously mentioned fine-tuning, FT is a model editing approach. It utilized the gradient to gather information about the knowledge to be updated and applied this information directly to the model parameters for updates.
Similar to the approaches of ROME and MEMIT, which involve locating parameters and modifying them, the FT method utilizes gradient information to directly update the model parameters for editing. Therefore, we incorporate an elastic net during the training process to constrain the deviation of the edited matrix.
Figure~\ref{ft} shows the sequential editing performance of FT on GPT2-XL and LLaMA-3 (8B) before and after the introduction of elastic net.
The dashed line represents the FT, while the solid line represents the FT applying the elastic net.
The experimental results indicate that when using the FT method to edit the model, the direct use of gradient information to modify the parameters destroys the general ability of the model. By constraining the deviation of the edited matrix, the incorporation of the elastic net effectively preserves the general abilities of the model.

\begin{figure*}[t]
  \centering
  \subfigure{
  \includegraphics[width=0.43\textwidth]{figures/legend_FT.pdf}}
\end{figure*}

\begin{figure*}[t]%[!ht]
  \centering
  \subfigure{
  \includegraphics[width=0.22\textwidth]{figures/FT-GPT2XL-ZsRE.pdf}}
  \subfigure{
  \includegraphics[width=0.22\textwidth]{figures/FT-GPT2XL-CounterFact.pdf}}
  \vspace{-2mm}
  \caption{Edited on the ZsRE or CounterFact datasets, the sequential editing performance of FT~\cite{DBLP:conf/emnlp/CaoAT21} and FT with elastic net on GPT2-XL before and after the introduction of elastic net.}
  \vspace{-2mm}
  \label{ft}
\end{figure*}


\end{document}
% \endinput
%%
%% End of file `sample-sigconf-authordraft.tex'.
