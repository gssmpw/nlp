% \section{System Overview}
\label{sec:system_overview}
The primary goal of this work is to accurately predict the MFSP in building structures subjected to various fire scenarios \revise{by utilizing the MIDR as a metric for the overall lateral stability.} To achieve this, we propose an integrated framework that combines GNNs and FEA. The system architecture is illustrated in \figref{fig:system_overview} and comprises two key components: the MIDR predictor and the MFSP predictor. \revise{There are two stages corresponding to the MIDR predictor's different modes. In stage 1, we use data with ground truth MIDR values obtained from FEA simulations to train the MIDR predictor. Then, in stage 2, with the well-trained MIDR predictor working as a differentiable agent of fire simulation, we train the MFSP predictor to determine the point that maximize the MIDR.}
This framework seamlessly integrates physics-based simulations, GNN-driven predictive modeling, and data-driven techniques, enabling efficient and accurate MFSP prediction to provide a powerful tool for proactive fire safety analysis and risk mitigation in building structures.
\begin{figure*} [h!]
    \centering
    \includegraphics[width=0.8\textwidth]{figures/system_overview.pdf}
    \caption{\revise{Proposed framework for predicting the MFSP in building structures. Trapezoids  and gray rectangles represent the NN predictor and processing, respectively. Dashed and solid lines indicate that MIDR predictor is in the respective training mode and evaluation mode with parameters fixed, acting as a differentiable agent.}}
    \label{fig:system_overview}
\end{figure*}

{\blockRevise
\subsection{Structural Stability Metrics} 
The Interstory Drift Ratio (IDR) of each node serves as a critical parameter for evaluating structural lateral stability and deformation under external forces. IDR quantifies the relative displacement between two consecutive floors (interstory displacement) as a percentage of the floor height. Mathematically, the IDR for a given node $i$ is defined as follows:
\begin{equation}
    d_i = \left.\sqrt{\left(\Delta x_i\right)^2 + \left(\Delta y_i\right)^2}\right/ H \times 100 \%,
\end{equation}
where the numerator represents the relative displacement (in the horizontal plane $xy$) of node $i$ with respect to the corresponding node on the floor below, and $H$ denotes the story height between these two floors. Excessively high drift ratios can indicate significant structural deformation, potentially leading to significant damage or collapse due to lateral instability. The Maximum IDR among all the nodes of a structure, i.e., MIDR, is chosen to be a representative example metric for assessing the overall structural stability performance during fire events. Although MIDR may not capture all possible failure modes, such as local collapses due to midspan softening, local buckling, or loss of vertical elements, we emphasize that the proposed method is not limited to MIDR. The framework can be easily adapted to other performance indicators of interest, by simply replacing the MIDR with the desired metric.

\textit{Remark:} Selecting MIDR as the primary metric for assessing the structural integrity under fire conditions is motivated by its effectiveness in quantifying global deformation patterns. In fire-induced scenarios, thermal expansion, stiffness degradation, and gravity-induced deformations contribute to structural instability, which can be captured through relative floor displacements. MIDR provides a direct and interpretable measure of structural vulnerability by identifying floors experiencing excessive lateral deformations that may lead to global instability or loss of vertical load-carrying capacity. While MIDR is commonly used for seismic and wind-induced responses, its application to fire scenarios is justified as a fire-driven thermal effect also leads to large-scale deformation patterns, particularly in multi-story steel structures. Importantly, MIDR serves as an effective proxy for overall building stability in computational frameworks where parameterizing every potential failure mode (e.g., local buckling, connection failure, progressive collapse) is infeasible. However, fire-induced failure mechanisms extend beyond interstory drift, and localized effects are not explicitly captured by MIDR. These mechanisms typically develop locally and may not always translate into immediate global structural instability. While our current framework focuses on identifying the MFSP based on a worst-case drift metric, future work could integrate alternative failure criteria to further refine the fire vulnerability predictions.

Note that ``M'' in MIDR and MFSP represents different concepts. In the case of MIDR, for a given structure and fire source point, the IDR is computed at each node, and the MIDR is defined as the maximum IDR among all nodes. In contrast, MFSP refers to the fire source location that results in the highest MIDR across all possible fire source points within the structure. In summary, an MIDR is associated with a specific structure and fire source point pair, whereas an MFSP characterizes an entire structure by identifying the most critical fire source location.

}

\subsection{MIDR \& MFSP Predictors}

The MIDR predictor is a GNN-based model designed to estimate the MIDR of a building under a given fire scenario. The inputs to this model include:
\begin{itemize}
    \item {\bf{Structural configuration}}: Building geometry, material property, and gravity loads. 
    \item {\bf{Fire location}}: The specific point where the fire is initiated within the building.
\end{itemize}
A GNN processes this input to represent the structural configuration and fire location as a graph. The MIDR predictor is trained on labeled data generated using OpenSeesRT, a robust open-source FEA framework \cite{perez2024openseesrt}. These labels represent detailed structural responses under various fire conditions. Once trained, the MIDR predictor functions as a {\em{differentiable agent}}, offering computationally efficient MIDR estimates. Its capabilities include:
\begin{itemize}
    \item {\bf{Annotating datasets}}: Assigning  MIDR values to support subsequent analyses.
    \item {\bf{Integrating with NNs}}: Reducing the computational cost typically associated with simulation-based methods.
\end{itemize}

% \subsection{MFSP Predictor}
The MFSP predictor acts as an ``argmaxer module'' for the MIDR predictor, identifing the fire location that results in the highest MIDR. This location corresponds to the point of the greatest structural vulnerability. By leveraging the structural graph as input and utilizing the MIDR predictor's outputs, the MFSP predictor efficiently pinpoints the critical fire location.

\subsection{Data Generation and Training Pipeline}
To ensure robustness and generalizability, we introduce a comprehensive data generator pipeline:
\begin{enumerate}
    \item {\bf{Structure data generator}}: This component creates synthetic datasets for diverse building configurations, including geometry, material, and gravity loads.
    \item {\bf{FEA simulations}}: With the high-fidelity FEA simulation software, OpenSeesRT, the gravity simulation is first conducted to confirm the rationality of the synthetic dataset. Further, a subset of the generated configurations undergoes fire scenario simulations using OpenSeesRT based on a rule-based thermal load generation method. These simulations produce \textbf{labeled data} detailing  the structural responses to various fire locations, forming training and testing sets for the MIDR predictor.
    \item {\bf{Unlabeled data utilization}}: The remaining configurations, without MIDR labels from the FEA simulation, are also used to train and test the MFSP predictor, leveraging the MIDR predictor as a computationally efficient, yet accurate, {\em{surrogate}} model. Although the structural configurations with unlabeled data do not undergo FEA simulations, they can be rapidly and efficiently  \emph{pseudo labeled} using this surrogate model.
\end{enumerate}



% Placing here temporarily

\section{Introduction}\label{sec:intro}

Sign languages are crucial for communication within the Deaf and Hard-of-Hearing (DHH) communities~\cite{glickman1993deaf,padden1988deaf}. As naturally-emerging and fully-fledged languages, they enable individuals to convey complex ideas, emotions, and cultural nuances through movements and facial expressions~\cite{emmorey2001language,rastgoo_sign_2021}. Despite their importance for many DHH people, communication barriers between signing and non-signing communities exist due to limited access to skilled sign language interpreters, low levels of sign language proficiency among the general population, and the exclusion of sign languages from most communication technologies designed for spoken or written languages~\cite{mitchell2023many,napier2002sign,bragg_sign_2019}. 

Systems that translate spoken languages into sign languages (sign language generation, SLG) and vice versa (sign language translation, SLT) hold promise to bridge this communication gap~\cite{rastgoo_sign_2021, stoll_text2sign_2020}. In this work we focus on the SLG task, specifically translating from English text to American Sign Language (ASL). Historically, SLG technology has faced criticism from DHH users due to low-fidelity avatars, poor language translation, and oversimplification of sign linguistics~\cite{kipp2011assessing,huenerfauth2009sign,huenerfauth2009linguistically}. Recent research, however, has suggested that improved quality and overcoming technical limitations could increase acceptance among DHH individuals~\cite{quandt2022attitudes, inan2024generating}. Our work uses advances in machine learning (ML) to develop an SLG prototype system and investigate whether technological improvements meet the needs and interests of the DHH and signing community. 

% Developing an effective SLG system capable of modeling complex signed interactions is a grand challenge that benefits from expertise in ML, human-computer interaction (HCI), linguistics, and cognitive science, alongside stewardship from the DHH and signing community~\cite{bragg_sign_2019}. 
Sign languages combine manual markers---such as hand movements, orientation, and location---with non-manual markers, including facial expressions, head movements, and other body language, to create grammatical structures and convey meaning~\cite{Stokoe1961SignLS,brentari1998prosodic,sandler2006sign}. For example, in ASL, manual markers such as location and movement within the signing space can modify a sign's grammatical function, indicating subjects, objects, or other syntactic roles~\cite{Stokoe1961SignLS}. Similarly, non-manual markers can also indicate critical information, such as a head shake accompanying a sign to denote negation, raised eyebrows and a distinct facial expression to form conditional clauses or emphasize the topic of a sentence or raised eyebrows and a forward head tilt to signal a yes/no question ~\cite{baker1991american,baker1985facial, sandler2006sign}, as exemplified in Figure \ref{fig:system_overview}. Each of these linguistic aspects of ASL presents a challenge for modern SLG systems, given that natural, understandable signing must include sufficient information shown in a fluid manner to convey multiple distinct streams of information.

While recent work on SLG has progressed~\cite{fang2024signllm, hohenberger2002modality, saunders_progressive_2020, moryossef2023open, stoll2018sign, huenerfauth2008generating}, these systems typically take a generic view of signing, often overlooking sign language nuances, including the role of non-manual markers. 
To address these challenges, we prototype a modular ASL generation system designed to produce automated signing by simultaneously focusing on technical improvements, user perceptions, and the unique linguistic structure of ASL. Our system is tailored for open-ended, context-free use cases, allowing users to input an English sentence and generate a signed video that appears natural and comprehensive. Developing an effective SLG system capable of modeling complex signed interactions is a grand challenge that requires interdisciplinary expertise, alongside stewardship from the DHH and signing community~\cite{bragg_sign_2019}. Guided by this principle, our research prototype was developed and refined through collaboration among researchers from diverse fields, including those in computer vision, computer graphics, human-computer interaction, and experts from the DHH and signing communities. It consists of three modules: (1) translating English text into intermediate ASL representations---including English-based glosses to capture manual markers and linguistic information to represent non-manual markers---using few-shot approach with GPT-4o, (2) synthesizing human pose and body motions from these representations using a Motion Matching approach, and (3) generating photorealistic signed video frames representing an ASL signer using an image generation model. 
% Leveraging new ML techniques for translation and image generation, our system aims to generate signed videos that not only preserve critical ASL semantics and syntax but also offer improved visual quality with smooth transitions between signs. 
% These AI-generated signers could take various forms, such as photorealistic (``live'') signers, cartoons, or other stylized avatars.  We choose to generate videos that resemble ``live'' signers, in an effort to mitigate confounds that could arise in accurately representing signs with more stylized avatars.

We conducted both technical evaluations and a user study with 30 DHH signers to assess our prototype system and to gauge the interest of DHH individuals in its use. The technical evaluation examined the translation of English sentences into ASL written representations, including manual and non-manual components, and the generation of signed videos. The user study evaluated translation quality, visual fidelity, and motion naturalness, while gathering perspectives on potential use cases. 
Our findings indicate that the system achieves compelling translation performance relative to reported results in the literature. 
% Interestingly, our user study revealed that the system's translation quality significantly outperformed the manual annotations, further supporting its effectiveness. 
% Based on our user study, we think that the use of modern ML machinery provides a large benefit 
However, there remains significant room for improvement. While participants were frequently able to understand the content of the signed videos, their perceptions on the signing quality, particularly in comparison to real human signers, were less favorable.

In summary, our contributions include: 
\begin{itemize}
    \item In Section \ref{sec:design}, we introduce a modular ASL generation prototype designed to produce natural and comprehensive signed videos that includes non-manual cues.
    \item In Section \ref{sec:technical_eval}, we present technical evaluations of our approach. Results show a BLEU-4 score of $0.276$ for English Text-to-ASL gloss translation, an average precision of 0.91 and recall of 0.97 for detecting non-manual information from English text, and improved video generation performance over baseline methods.
    \item In Section \ref{sec:user_study}, we detail a user study assessing the perceived translation quality of our system, as well as the visual and motion quality of its outputs. Results indicate both potential and tangible areas for improvement, alongside insights into the system's potential use cases (\eg doctor office and video or in-person conversations).
    \item In Section \ref{sec:discussion}, we reflect on our design process, share key insights gained from our design and evaluation process, provide recommendations on how to address remaining challenges, and discuss computational and ethical considerations in the use of our system.
\end{itemize}

While continued effort is needed to advance SLG systems in collaboration with the DHH and signing communities, our work represents an initial step in addressing critical technical challenges and taking a comprehensive approach to ASL. It demonstrates the potential of these systems and encourages further exploration of critical aspects of signing, especially  non-manual markers. 