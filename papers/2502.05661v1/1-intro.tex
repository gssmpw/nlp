% \begin{teaserfigure}
    \centering
    \includegraphics[width=1\linewidth]{figures/hero_fig.png}
    \caption{Our system translates English text into a photorealistic ASL video with non-manual information. It starts with an English text input (first row), generates ASL tokens capturing both manual and non-manual details (second row), produces a skeletal pose sequence (third row), and finally creates the photorealistic ASL video (fourth row). \han{@all, please check the stylization.}}
    % \han{should we have a longer sentence? Ideally has 6-7 video frames. Still working on stylization...} \rotem{thinking about it again, instead of putting it here, I would have 1-2 examples here- only text->gloss->ours with expressions (or through pose), then have a no expression vs expression fig later on in the paper..}}
    \label{fig:system_overview}
\end{teaserfigure}

% Placing here temporarily

\section{Introduction}\label{sec:intro}

Sign languages are crucial for communication within the Deaf and Hard-of-Hearing (DHH) communities~\cite{glickman1993deaf,padden1988deaf}. As naturally-emerging and fully-fledged languages, they enable individuals to convey complex ideas, emotions, and cultural nuances through movements and facial expressions~\cite{emmorey2001language,rastgoo_sign_2021}. Despite their importance for many DHH people, communication barriers between signing and non-signing communities exist due to limited access to skilled sign language interpreters, low levels of sign language proficiency among the general population, and the exclusion of sign languages from most communication technologies designed for spoken or written languages~\cite{mitchell2023many,napier2002sign,bragg_sign_2019}. 

Systems that translate spoken languages into sign languages (sign language generation, SLG) and vice versa (sign language translation, SLT) hold promise to bridge this communication gap~\cite{rastgoo_sign_2021, stoll_text2sign_2020}. In this work we focus on the SLG task, specifically translating from English text to American Sign Language (ASL). Historically, SLG technology has faced criticism from DHH users due to low-fidelity avatars, poor language translation, and oversimplification of sign linguistics~\cite{kipp2011assessing,huenerfauth2009sign,huenerfauth2009linguistically}. Recent research, however, has suggested that improved quality and overcoming technical limitations could increase acceptance among DHH individuals~\cite{quandt2022attitudes, inan2024generating}. Our work uses advances in machine learning (ML) to develop an SLG prototype system and investigate whether technological improvements meet the needs and interests of the DHH and signing community. 

% Developing an effective SLG system capable of modeling complex signed interactions is a grand challenge that benefits from expertise in ML, human-computer interaction (HCI), linguistics, and cognitive science, alongside stewardship from the DHH and signing community~\cite{bragg_sign_2019}. 
Sign languages combine manual markers---such as hand movements, orientation, and location---with non-manual markers, including facial expressions, head movements, and other body language, to create grammatical structures and convey meaning~\cite{Stokoe1961SignLS,brentari1998prosodic,sandler2006sign}. For example, in ASL, manual markers such as location and movement within the signing space can modify a sign's grammatical function, indicating subjects, objects, or other syntactic roles~\cite{Stokoe1961SignLS}. Similarly, non-manual markers can also indicate critical information, such as a head shake accompanying a sign to denote negation, raised eyebrows and a distinct facial expression to form conditional clauses or emphasize the topic of a sentence or raised eyebrows and a forward head tilt to signal a yes/no question ~\cite{baker1991american,baker1985facial, sandler2006sign}, as exemplified in Figure \ref{fig:system_overview}. Each of these linguistic aspects of ASL presents a challenge for modern SLG systems, given that natural, understandable signing must include sufficient information shown in a fluid manner to convey multiple distinct streams of information.

While recent work on SLG has progressed~\cite{fang2024signllm, hohenberger2002modality, saunders_progressive_2020, moryossef2023open, stoll2018sign, huenerfauth2008generating}, these systems typically take a generic view of signing, often overlooking sign language nuances, including the role of non-manual markers. 
To address these challenges, we prototype a modular ASL generation system designed to produce automated signing by simultaneously focusing on technical improvements, user perceptions, and the unique linguistic structure of ASL. Our system is tailored for open-ended, context-free use cases, allowing users to input an English sentence and generate a signed video that appears natural and comprehensive. Developing an effective SLG system capable of modeling complex signed interactions is a grand challenge that requires interdisciplinary expertise, alongside stewardship from the DHH and signing community~\cite{bragg_sign_2019}. Guided by this principle, our research prototype was developed and refined through collaboration among researchers from diverse fields, including those in computer vision, computer graphics, human-computer interaction, and experts from the DHH and signing communities. It consists of three modules: (1) translating English text into intermediate ASL representations---including English-based glosses to capture manual markers and linguistic information to represent non-manual markers---using few-shot approach with GPT-4o, (2) synthesizing human pose and body motions from these representations using a Motion Matching approach, and (3) generating photorealistic signed video frames representing an ASL signer using an image generation model. 
% Leveraging new ML techniques for translation and image generation, our system aims to generate signed videos that not only preserve critical ASL semantics and syntax but also offer improved visual quality with smooth transitions between signs. 
% These AI-generated signers could take various forms, such as photorealistic (``live'') signers, cartoons, or other stylized avatars.  We choose to generate videos that resemble ``live'' signers, in an effort to mitigate confounds that could arise in accurately representing signs with more stylized avatars.

We conducted both technical evaluations and a user study with 30 DHH signers to assess our prototype system and to gauge the interest of DHH individuals in its use. The technical evaluation examined the translation of English sentences into ASL written representations, including manual and non-manual components, and the generation of signed videos. The user study evaluated translation quality, visual fidelity, and motion naturalness, while gathering perspectives on potential use cases. 
Our findings indicate that the system achieves compelling translation performance relative to reported results in the literature. 
% Interestingly, our user study revealed that the system's translation quality significantly outperformed the manual annotations, further supporting its effectiveness. 
% Based on our user study, we think that the use of modern ML machinery provides a large benefit 
However, there remains significant room for improvement. While participants were frequently able to understand the content of the signed videos, their perceptions on the signing quality, particularly in comparison to real human signers, were less favorable.

In summary, our contributions include: 
\begin{itemize}
    \item In Section \ref{sec:design}, we introduce a modular ASL generation prototype designed to produce natural and comprehensive signed videos that includes non-manual cues.
    \item In Section \ref{sec:technical_eval}, we present technical evaluations of our approach. Results show a BLEU-4 score of $0.276$ for English Text-to-ASL gloss translation, an average precision of 0.91 and recall of 0.97 for detecting non-manual information from English text, and improved video generation performance over baseline methods.
    \item In Section \ref{sec:user_study}, we detail a user study assessing the perceived translation quality of our system, as well as the visual and motion quality of its outputs. Results indicate both potential and tangible areas for improvement, alongside insights into the system's potential use cases (\eg doctor office and video or in-person conversations).
    \item In Section \ref{sec:discussion}, we reflect on our design process, share key insights gained from our design and evaluation process, provide recommendations on how to address remaining challenges, and discuss computational and ethical considerations in the use of our system.
\end{itemize}

While continued effort is needed to advance SLG systems in collaboration with the DHH and signing communities, our work represents an initial step in addressing critical technical challenges and taking a comprehensive approach to ASL. It demonstrates the potential of these systems and encourages further exploration of critical aspects of signing, especially  non-manual markers. 