\section{Achieving Robust AHT}
\label{seq:robust_aht}



% Motivation : Learning against a uniform distribution only guarantees best performance on that exact distribution of scenarios.

To learn a policy able to cooperate with new partners, a straightforward idea is to reconstruct scenarios that would be encountered in nature. A roadblock to this approach however is that it requires two main ingredients: a) a diverse pool of partners, and b) a prior distribution over them. The prior, often neglected, is important as it captures our uncertainty about the true partners observed in nature.

In Section~\ref{subsec:constructing_training_scenarios}, we reflect on motivating previous work on diverse behaviour generation, before describing our own adopted approach. Section~\ref{subsec:minimax_bayes_aht} then introduces the Minimax-Bayes idea to AHT, by stating the connections of our setting to Minimax-Bayes Reinforcement Learning (MBRL).

%For now, we assume we have access to such a pretrained background population. The training of a background population is a key part of the whole learning process, it will be considered in later stages of this work.
\subsection{Constructing Training Scenarios}
\label{subsec:constructing_training_scenarios}

% learning against mixture of policies is not the same as learning against a distribution of policies.

%To illustrate with an example, in a setting where company coworkers have to realise a project, there might be workers that have a high preference for their own contribution (with a better chance to get promoted later), while there may be others that are inclined to delegate their work for things they are unsure about to the team. 
Before learning any robust policy, we need to construct a diverse set of scenarios. A background population that encompasses a wide range of behaviours is needed in order to reconstruct scenarios existing in nature. Previous work on AHT tackled the issue in various manners, such as using genetic algorithms \citep{muglich_generalized_beliefs_cooperative_2022}, rule-based policies generated with MAP Elites \citep{canaan_generating_adapting_diverse_2023}, SP policies \citep{strouse_collaboration_with_humans_2021}, explicit behavior diversification through regularisation \citep{lupu_trajectory_diversity_zero_2021}, or through evolved pseudo-rewards \citep{jaderberg_human_level_performance_2019}. Based on real-life examples and aiming to thoroughly assess the effects of partner priors, we adopt the following approach:
\begin{itemize}[leftmargin=12pt]
    \item We assume a class of reward functions for background policies:
    \begin{equation*}
        \rho_\text{social+risk} (s, \mathbf{a}, i) = \rho_\text{social}^+ (s, \mathbf{a}, i) - \delta_i \rho_\text{social}^- (s, \mathbf{a}, i),
    \end{equation*}
    with $\rewardfunc_\text{social}$ defined as
    \begin{equation*}
        \rewardfunc_\text{social}(s, \mathbf{a}, i) = \lambda_i \rho(s, \mathbf{a}, i) + (1-\lambda_i) \sum_{j=1}^m \rho(s, \mathbf{a}, j),
    \end{equation*}
    where $\rho^+$ and $\rho^-$ are the positive and negative parts of $\rho$, and $\lambda_i$ and $\delta_i$ denoting levels of prosociality \citep{peysakhovich_prosocial_learning_agents_2017} and risk-aversion, respectively. In other words, each background policy has their own preferences ($\lambda_k, \delta_k$).
    %Combining values of prosociality and risk-aversion allows for the consideration of behaviours with a wide range of preferences.
    \item Policies are organised into sub-populations $\bgpop = \bigcup_k \bgpop_k$ of varying sizes.
    \item Each sub-populations are separately trained using PP.
\end{itemize}
Given the diverse preferences and varying sizes of the sub-populations, distinct habits and established conventions are more likely to emerge from each group \cite{strouse_collaboration_with_humans_2021}. This choice for constructing scenarios ensures a diverse generation of scenarios, important to ablate the effects of various scenario priors on AHT robustness.
Note that this choice for constructing scenarios remains arbitrary and is not the main focus of our work.
%We could also introduce policies of different skill levels, i.e. fully trained or early stage policies, but decide to k


% TODO more details in methodology and data processing
% merge methodology and 
\section{Framework for Analyzing Emotion}
In this section, we present our framework for analyzing emotion. We first establish a basic understanding of emotion polarity by determining the sentiment valence of each root tweet and comment. We then use multi-label emotion detection to predict the emotion categories associated with each post. Based on this data, we explore the interactive nature of emotions, by identifying common patterns in emotion transition pairs between temporally-adjacent posts. Finally we investigate the emotional trajectory within threads to understand how emotional intensity and type shift over time, by aggregating the predicted labels for posts at each time stamp in a given thread. As part of this, we contrast rumour with non-rumour threads, to gain a holistic understanding of emotional expression in rumours and non-rumours on Twitter.

% elaborate a bit on why we choose EmoLLM, compared with other automatic emotion detection methods
\paragraph{Affective Computing: Automatic Emotion Detection}
Manually annotating emotions is both costly and time-consuming, so we use an LLM-based emotion detection model, EmoLLM~\citep{liu2024emollms}, which is specifically designed for sentiment analysis and emotion detection. The model was instruction-tuned on SemEval 2018 Task1 using a comprehensive emotion labeling scheme grounded in established theoretical frameworks. We prompt the model to perform Valence Ordinal Classification (V-oc), Emotion Classification (E-c), and Emotion Intensity regression (E-i). Detailed prompts are shown in \Cref{tab:emollm_ins}.

\paragraph{Categorical Emotion Labeling Scheme} \label{para:emotion_label}
Numerous emotion label sets  have been proposed~\citep{Ekman1992AnAF, Plutchik1980AGP, Russell1980ACM}. According to \citet{Ekman1992AnAF, Plutchik1980AGP}, certain emotions, such as joy, fear, and sadness, are considered more fundamental than others, both physiologically and cognitively. The Valence-Arousal-Dominance (VAD) model \citep{Russell1980ACM} categorizes emotions within a three-dimensional space of valence (positivity-negativity), arousal (active-passive), and dominance (dominant-submissive). Inspired by \citet{mohammad-etal-2018-semeval}, we incorporate elements from both basic emotion theories and the VAD model, and further ground EmoLLM emotion predictions to develop the following emotion label schemes: (1) \textit{neutral or no emotion}; (2) \textit{negative emotions}: anger (also includes annoyance and rage),  disgust (also includes disinterest, dislike, and loathing), fear (also includes apprehension, anxiety, and terror), pessimism (also includes cynicism, and no confidence), sadness (also includes pensiveness and grief); 3) \textit{positive emotions}: joy (also includes serenity and ecstasy), love (also includes affection), optimism (also includes hopefulness and confidence), anticipation (also includes interest and vigilance), surprise (also includes distraction and amazement) and trust (also includes acceptance, liking, and admiration). 


\paragraph{Emotion Polarity: Sentiment Valence} 
To understand the basic emotion polarity expressed in rumour and non-rumour content, we begin with sentiment valence analysis. Sentiment valence aims to capture the overall emotional tone conveyed by a post, in terms of how positive or negative it is~\citep{liu2024emosurvey}. We frame the sentiment valence task as ordinal regression~\citep{mohammad-etal-2018-semeval}. As shown in \Cref{tab:emollm_ins}, for a given tweet post, we classify it into one of seven ordinal levels of sentiment intensity, spanning varying degrees of positive and negative valence, that best represents the tweeter's mental state. The tweet posts within a thread can be divided into two categories: root tweets, which are posted by the publisher, and follow posts, which include all subsequent replies under the root post. We begin by conducting sentiment valence analysis on each post within the thread conversation. 
% TJB: confused by how comments can include all subsequent replies; we seem to be overloading the terminology, for comments to be both individual posts and series of posts
% RX: yes, I am unifiying all terms.
For each category, we compute the mean sentiment valence to enable further investigation into the specific emotions associated with different sentiment valences over a thread.
% TJB: clarify for comments whether the classification is done over the combined meta-document (i.e. the root + all comments to that point) or individually over the separate documents and then combined ... or over individual documents, in which case the statement about "all replies" needs clarification
% RX: we separate root and comments for each tweet conversation, the former is the root tweet posted by the publisher while the rest are comments. "all replies" mean all comments under root tweet, we aggregate them by computing the mean sentiment, and then average over each part.

\paragraph{Emotion Distribution} 
Following sentiment valence analysis, we then examine specific emotions and their distribution in rumour and non-rumour tweet posts.
Motivated by the fact that a certain tweet might exhibit more than one emotion, we frame the task as multi-label emotion detection problem. As shown as V-oc in \Cref{tab:emollm_ins}, given a tweet, we classify it into one of seven ordinal classes, corresponding to various levels of positive and negative sentiment intensity. To reduce noise from automatic emotion detectors, we take the top-three predicted emotions for each tweet. We then aggregate and plot the emotion distribution to provide an overview of dominant emotional trends across the rumour and non-rumour posts. Given that the follow posts make up the majority of the data compared to the root posts, we will focus on using follow posts in our next analysis.
% TJB: what is the basis of saying that the signal is richer? simply that there are more reply posts than root posts? clarify
% RX: yes, and we are more interested in interaction in comments.

\paragraph{Emotion Transitions} 
Emotions are contagious and highly interactive~\citep{Ferrara_2015}. When publishers write tweets that convey their emotions, readers are likely to respond with emotional reactions of their own~\citep{Ferrara_2015,emotion_dynamics}. In this part, we model this interactive nature of emotions in the form of emotion transition pairs, which are built from two chronologically-adjacent tweets. In each pair, the first element represents the emotion inferred from the initial content published at a given time, and the second element represents the emotion inferred from the reply content published immediately after. For example, if the first tweet exhibits \textit{joy} \textit{trust} and \textit{anticipation}, and the second tweet shows \textit{anger}, \textit{disgust} and \textit{surprise}, we form the pairs (\textit{joy}, \textit{anger}), (\textit{joy}, \textit{disgust}), (\textit{joy}, \textit{surprise}), (\textit{trust}, \textit{surprise}), (\textit{trust}, \textit{surprise}), (\textit{trust}, \textit{disgust}), (\textit{anticipation}, \textit{anger}), (\textit{anticipation}, \textit{surprise}) and (\textit{anticipation}, \textit{disgust}). We create transitions for all combinations of emotion pairs and explore the likelihood of emotion transition pairs occurring in rumour and non-rumour content. Exploring emotion transitions allows us to understand the emotional flow in social media conversations and uncover typical patterns of rumour and non-rumour content, and any differences between the two.

\paragraph{Emotion Trajectories} 
We explore the cumulative trajectory of emotion over time to observe how emotions evolve during the conversational thread. We collect all detected emotion labels for each tweet from both rumour and non-rumour content, then track cumulative emotion counts at each chronological step. Finally, we visualize these trends and apply regression models to analyze the growth of emotions over time. This temporal analysis reveals how emotions accumulate or intensify across time, offering insight into the trajectory of emotions in rumour and non-rumour content.

\begin{table*}[!h]
    \centering
    \small
    \begin{tabular}{cccccccccccc}
        \toprule
        \textbf{Setting} & \textbf{Ru} & \textbf{Non} & \textbf{p} & \textbf{\#Ru/Non} & \textbf{T} & \textbf{F} & \textbf{U} & \textbf{$p$ (U vs T)} & \textbf{$p$ (U vs F)} & \textbf{\#T/\#F/\#U} \\
        \midrule
        \textbf{PHEME root} & \textbf{$-$0.25} & $-$0.17 & 0.00 & 2602/2602 & $-$0.21 & $-$0.11 & \textbf{$-$0.39} & 7.75e-11 & 4.41e-11 & 629/629/629 \\
        \textbf{PHEME follow} & \textbf{$-$0.33} & $-$0.26 & 6.47e-09 & & $-$0.35 & $-$0.20 & \textbf{$-$0.39} & 0.03 & 8.38e-15 & \\
        \textbf{Twitter15 root} & \textbf{$-$0.26} & $-$0.01 & 3.51e-05 & 372/372 & $-$0.21 & $-$0.20 & \textbf{$-$0.34} & 0.01 & 0.01 & 359/359/359 \\
        \textbf{Twitter15 follow} & \textbf{$-$0.27} & $-$0.06 & 1.65e-09 & & $-$0.24 & $-$0.25 & \textbf{$-$0.30} & 0.16 & 0.21 & \\
        \textbf{Twitter16 root} & \textbf{$-$0.18} & \z0.07 & 0.00 & 205/205 & \z0.11 & $-$0.22 & \textbf{$-$0.30} & 1.35e-06 & 0.18 & 63/63/63 \\
        \textbf{Twitter16 follow} & \textbf{$-$0.31} & $-$0.12 & 9.19e-06 & & $-$0.30 & \textbf{$-$0.36} & $-$0.27 & 0.67 & 0.90 & \\
        % \textbf{CoAID root} & \textbf{$-$0.34} & $-$0.16 & 0.01 & 167/167 & - & - & - & - & - & - \\
        % \textbf{CoAID follow} & \textbf{$-$0.24} & $-$0.13 & 0.01 & & - & - & - & - & - & \\
        \bottomrule
    \end{tabular}
    \caption{Valence Ordinal Regression results for all datasets. root = root posts, follow = follow posts, Ru = rumour, Non = Non-rumour, T = True rumour, F = False rumour, U = Unverified rumour; $p$ values indicates significance of a one-tailed t-test.}
\label{tab:voc_results}
\end{table*}

\begin{algorithm}[ht!] 
\caption{PC Algorithm}
\label{pc}
\begin{algorithmic}[1] 
\State \textbf{Input:} Data $\mathbf{X}$, significance level $\alpha$
\State \textbf{Output:} Completed Partially Directed Acyclic Graph (CPDAG)

\State Initialize a complete undirected graph $G$ with all variables as nodes.

\State \textbf{Step 1: Skeleton Identification}
\For{each pair of variables $(X, Y)$ in $G$}
    \State Find the subset $S \subseteq \text{Adj}(X, G) \setminus \{Y\}$ such that 
    $X \indep Y \mid S$ with significance $\alpha$.
    \If{such a subset $S$ exists}
        \State Remove the edge $X - Y$ from $G$.
    \EndIf
\EndFor

\State \textbf{Step 2: Edge Orientation}
\For{each triple of variables $(X, Y, Z)$ in $G$ where $X - Z - Y$ and $X, Y$ are not adjacent}
    \If{$Z \notin S$ for all separating sets $S$ for $X$ and $Y$}
        \State Orient as $X \to Z \leftarrow Y$ (identify a collider).
    \EndIf
\EndFor

\While{possible}
    \For{each edge $(X - Y)$ in $G$}
        \If{there exists a directed path $X \to \dots \to Z$ such that $Z - Y$}
            \State Orient as $X \to Y$ (acyclicity rule).
        \ElsIf{orienting $X - Y$ as $X \to Y$ creates a new v-structure}
            \State Orient as $X \to Y$ (v-structure rule).
        \EndIf
    \EndFor
\EndWhile

\State \textbf{return} the CPDAG representing the equivalence class of causal graphs.

% how we frame the task, compute the emotion intensity, how to aggregate on conversation level

\end{algorithmic}
\end{algorithm}


\paragraph{Causal Relationship of Emotions in Rumour \& Non-Rumour Threads}
To gain a deeper insight into the relationship between rumours and the emotions underlying them, we extend our analysis beyond statistical correlation by conducting a causal analysis. Specifically, we apply the Peter-Clark (PC) algorithm \cite{Spirtes2000}, a classical constraint-based causal discovery algorithm on the three merged datasets. 

Uncovering causal relations between variables of interest is never an easy problem. Under the fundamental assumption of \textit{causal Markov condition} that a variable is conditionally independent of all its non-effects given its direct cause, \textit{faithfulness} ensures that the casual graph exactly encodes the independence and conditional independence relations among variables. These two assumptions allow us to infer causal relationships from observed statistical independencies, forming the cornerstone of constraint-based causal discovery methods. 

The PC algorithm identifies causal relationships among the variables of interest, represented as a directed acyclic graph (DAG), by numerating the independence and conditional independence relationships. The algorithm consists of two main steps: 
\begin{enumerate}
    \item \textbf{Skeleton Identification}: Starting with a complete undirected graph where all variables are connected, edges are iteratively removed based on conditional independence and independence relationships among variables, inferred by a conditional independence test. This step returns an undirected graph, which we call a skeleton. 
    \item \textbf{Edge Orientation}: After constructing the skeleton, edges are oriented by a set of predefined rules (Meek's Rule \cite{meek1997graphical}) to avoid cycles and orient collider structures.
\end{enumerate}

The complete PC algorithm is provided in algorithm \ref{pc}. It returns a  completed partially directed acyclic graph (CPDAG), which represents an equivalence class of causal graphs that are consistent with the observed data’s independence and conditional independence relations. In our implementation, we adopt the  Fisher-z test \cite{fisher_probable_1921} to infer the conditional independence relations.

%%% Local Variables:
%%% mode: latex
%%% TeX-master: "../main_anonymous"
%%% End:


\subsection{Minimax-Bayes AHT}
\label{subsec:minimax_bayes_aht}

In the standard single-agent Bayesian RL setting, the learner selects a subjective belief $\beta$ over candidate Markov Decision Processes (MDPs) $\mathcal{M}$ for the unknown, true environment $\mdp^* \in \mathcal{M}$. The learner's objective is to maximise its expected expected utility with respect to the chosen prior $U(\pi, \beta) = \int_\mathcal{M} U(\pi, \mdp) \diff \beta(\mdp)$, i.e. finding the Bayes-optimal policy. In MBRL, \citet{buening_minimax_bayes_reinforcement_2023} proposed considering the worst possible prior for the agent, without knowledge of the policy that will be chosen. This approach can be interpreted as nature playing the minimising player against the policy learner in a simultaneous-move zero-sum normal-form game. Learning against a worst-case prior intuitively makes the policy more robust, as it prepares for the worst outcomes.

To transfer this idea to our setting, we remark that any finite background population $\bgpop$ provides a finite set of POMGs $\mathcal{M}_\bgpop = \{\mdp(\scenario) | \scenario \in \scenarioset(\bgpop)\}$. The principal difference here is the use of POMGs rather than MDPs. We extend the notion of expected utility with respect to a prior over scenarios, i.e. when $\beta \in \Delta(\scenarioset(\bgpop))$:
\begin{equation}
    U(\pi, \beta) \defeq \mathbb{E}_{\scenario \sim \beta}[U(\pi, \scenario)] = \sum_{\scenario} U(\pi, \scenario) \beta(\scenario).
\end{equation}
This allows us to formulate the following maximin game:\footnote{
If we have a subjective prior, we could learn the distribution within an $\epsilon$-ball around that prior \citep{li_bayes_optimal_robust_2024}. We however consider the full simplex for simplicity.
}
\begin{equation}
    \label{eq:mbmarl.maximin}
    \max_{\pi \in \policies} \min_{\beta \in \Delta(\scenarioset(\bgpop))} U(\pi, \beta).
\end{equation}
Similarly to \citet{buening_minimax_bayes_reinforcement_2023}, we are interested in knowing whether such a game has a solution (i.e., a value), assuming that nature and the agent play simultaneously without knowledge of each other's move. This is relevant in our setting because the policy learner does not know the true distribution of partners available in nature, while nature's distribution does not depend on the policy that will be picked. Fortunately, \eqref{eq:mbmarl.maximin} has a value when $\bgpop$ is finite.
\begin{corollary}[\citet{buening_minimax_bayes_reinforcement_2023}]
For an $m$-player POMG $\mdp$ in a finite state-action space, with a known reward function and a finite horizon, and a background population $\bgpop$, the maximin game \eqref{eq:mbmarl.maximin} has a value:
\begin{equation}
    \label{eq:maximin_value}
    \max_{\pi \in \policies} \min_{\beta \in \Delta(\scenarioset(\bgpop))} U(\pi, \beta) = \min_{\beta \in \Delta(\scenarioset(\bgpop))} \max_{\pi \in \policies} U(\pi, \beta).
\end{equation}
\end{corollary}
\begin{proof}
First, observe that for any stochastic policy $\pi \in \policies$, there exists a distribution over deterministic policies $\phi \in \Delta(\deterministicpolicies)$ such that $\pi(a_t|h_t) = \sum_{d \in \deterministicpolicies} d(a_t|h_t) \phi(d)$. Consequently, we can rewrite the utility as $U(\pi, \beta) = \sum_{d \in \deterministicpolicies} \sum_{\sigma \in \scenarioset(\bgpop)} U(d, \sigma) \phi(d) \beta(\sigma)$. This demonstrates that $U$ is bilinear in $\phi$ and $\beta$, which allows us to apply the minimax theorem, thus proving the result.
\end{proof}
Importantly, prior work that chooses arbitrarily a fixed prior is limited in terms of robustness guarantees: it only ensures maximal utility for their specific prior. In contrast, a policy $\pi^*_U$ solving the maximin utility problem \eqref{eq:mbmarl.maximin} has its utility lower-bounded on $\scenarioset(\bgpop)$:
\begin{equation}
    \label{eq:utility_lower_bound}
    \forall \beta \in \Delta(\scenarioset(\bgpop)), \quad U(\pi^*_U, \beta) \geq U(\pi^*_U, \beta^*_U),
\end{equation}
where $\beta^*_U$ is the worst-case prior for $\pi^*_U$.
Simply put, $\pi^*_U$ performs the worst when the prior is its worst-case $\beta^*_U$, but can only improve when the prior deviates from $\beta^*_U$. Additionally, it is also optimal on the worst-case prior:
\begin{equation}
\label{eq:best_on_worst_case_prior}
    \forall \pi\in\policies, \quad U(\pi^*_U, \beta^*_U) \geq U(\pi, \beta^*_U).
\end{equation}
Note that this differs fundamentally from merely finding the best response to a fixed worst-case prior $\arg\max_{\pi} U(\pi, \beta^*_U)$, which once again, only has a guaranteed optimal utility on $\beta^*_U$.
\begin{corollary}[\citet{buening_minimax_bayes_reinforcement_2023}]
    \label{corollary:min_dirac}
    For any policy $\pi\in\policies$ and background population $\bgpop \subset \policies$, we have 
    \begin{equation}
        \min_{\beta\in\Delta(\scenarioset(\bgpop))} U(\pi, \beta)= \wcu(\pi, \scenarioset(\bgpop)). 
    \end{equation}
\end{corollary}
\begin{proof}
    This follows directly from the results of \citet{buening_minimax_bayes_reinforcement_2023}, using utility in place of regret and recognising that Dirac distributions associated with scenarios in $\scenarioset(\bgpop)$ are always contained in $\Delta(\scenarioset(\bgpop))$.
\end{proof}
\begin{lemma}
    For any background population $\bgpop \subset \policies$ and $\pi^*_U$ the policy solving the maximin utility game~\eqref{eq:mbmarl.maximin}, we have
    \begin{equation}
    \label{eq:optimal_worst_case_utility}
    \wcu(\pi^*_U, \scenarioset(\bgpop)) = \max_{\pi\in\policies} \wcu(\pi, \scenarioset(\bgpop)).
    \end{equation}
\end{lemma}
\begin{proof}
    By Corollary~\ref{corollary:min_dirac}, we can write that $\max_{\pi}\min_{\beta} U(\pi, \beta)=\max_{\pi}\wcu(\pi, \scenarioset(\bgpop))$.
    However, we also have $\max_{\pi}\min_{\beta} U(\pi, \beta) = \min_{\beta} U(\pi^*_U, \beta) = \wcu(\pi^*_U,\scenarioset(\bgpop))$.
\end{proof}
Thus, a policy solving the maximin utility game~\eqref{eq:mbmarl.maximin} is guaranteed to have an optimal worst-case utility on its training set.


% https://www.epfl.ch/labs/lia/wp-content/uploads/2023/01/multi-mdp.pdf

% Results
% -> 