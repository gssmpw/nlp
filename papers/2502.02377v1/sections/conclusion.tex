\section{Conclusion}

We explored how to compute robust adaptive policies for Ad Hoc Teamwork. Building on Minimax-Bayes RL, we introduced a method to identify minimax distributions over background populations, which consistently yielded more robust policies compared to training with a uniform distribution. Surprisingly, we also found that training on minimax distributions can significantly accelerate learning.
However, utilising regret as an objective to tune the training distribution proved computationally expensive when best-response utilities are not readily available. In some special instances, we also found that the minimax-Bayes training approach w.r.t.\ utility can prevent policies from learning certain game dynamics.  

Looking forward, we see great potential in extending our approach to a curriculum learning framework based on partner distributions, which could dramatically improve sample efficiency, asymptotic performance, and AHT robustness. 

% --- 

% We investigated how to obtain robust adaptive policies in an AHT setting. Leveraging previous work on Minimax-Bayes RL, we propose a method to find worst-case distributions over background populations, which led to consistently robust policies compared to simply training policies using a uniform distribution over the background population. On one hand, we remarked that exploiting regret is computationally demanding when best-response utilities are not readily available, and observed that worst-case distributions can prevent policies from learning certain aspects of the game. On the other hand, we have found the unexpected results that these distributional choices can significantly accelerate learning, on top of improved AHT robustness and performance. 

% Looking ahead, we see strong potential in extending our approach to a curriculum-learning framework based on partner distributions. This approach could dramatically enhance sample efficiency, asymptotic performance, and robustness.

% \section{Limitations} 

In this work, we compared the effectiveness and interplay of SFT and RL-based methods, under fixed data constraints. In particular, we chose offline methods like DPO and KTO as the baseline implementation of the RL method because it eliminates the need for reward modeling or iterative finetuning. This means that the process of development is limited to collecting an offline dataset and fientuning it - making it the most fair comparable to SFT in terms of implementation effort, compute costs and annotation efforts. Since this baseline RL method shows optimal performance over SFT, we hope that this motivates future work to study more complex RL-based methods and their interplay with SFT. In addition, we used GPT4o annotation for synthetic data generation, and also for evaluating Summarization and Helpfulness, which could include potential biases inherited from the model. 

In addition, we limited the size of the model to under 10 Billion parameters, to keep the finetuning cost low enough to ignore as compared to the data annotation costs. In addition, it would be extremely compute resource intensive to run thousands of finetuning runs with larger model sizes like 70B parameters. We hope that future work would study the scaling trends of RL-based methods against different model sizes, and also study the compute-data trade-off in-depth.

%% !TEX root = main.tex

\section{Future research}\label{sec:future}
Below we list a few research questions, which we find interesting and
particularly promising directions after our contribution.

\para{Exact complexity for $3$-VASS}
We have shown that shortest paths in binary $3$-VASS are of at most triply-exponential length.
It is tempting to conjecture that actually the upper bound for the length of the paths is shorter,
at most doubly-exponential. We conjecture so and leave this conjecture to the future research.

\para{Example of a $3$-VASS with doubly-exponential path}
We have shown that shortest paths in binary $3$-VASS are of at most triple-exponential length.
However, currently we still do not know any example in which even a path of doubly-exponential length is needed,
it might be that paths of exponential length are sufficient leading to \pspace-completeness for binary $3$-VASS.
It would be very interesting to find an example of a binary $3$-VASS with shortest path between two configurations
being doubly exponential. An example of binary $4$-VASS of doubly-exponential shortest path is known (see Section 5 in~\cite{DBLP:conf/concur/Czerwinski0LLM20}). Maybe some modification of this $4$-VASS would allow to design a $3$-VASS with similar properties.

\para{Reachability for $d$-VASS with $d \geq 4$}
It is a natural question whether our techniques extend to higher dimensions.
The answer is: possibly yes, but we would need a few other structural results for $3$-VASS
to make a similar approach to $4$-VASS possible. In the proof of Lemma~\ref{lem:main} we do not only
use $2$-VASS reachability as a black box, but we use a deep understanding of the reachability relation in $2$-VASS
from~\cite{DBLP:conf/focs/0001CMOSW24}. Probably a similar understanding of the reachability relation for $3$-VASS would be needed
to advance understanding of $4$-VASS along our lines. 

In general it is very interesting to determine the complexity of the reachability problem for $d$-VASS.
We have excluded that for each $d \geq 3$ the problem is $\F_d$-completely, but it is still possible that
the problem is $\F_{d-C}$-complete for some constant $C \in \N$ and $d$ big enough.
Recall that in~\cite{DBLP:conf/fsttcs/CzerwinskiJ0LO23}
it was shown that the reachability problem for $(2d+4)$-VASS is $\F_d$-hard for any $d \geq 3$ and this
is the best currently known lower bound for arbitrary dimension.
Therefore the other natural possibility is that the reachability problem for $(2d+C)$-VASS is $\F_d$-complete for some
constant $C \in \N$. 

\para{Applications of the approximation technique}
Another natural research direction is to search for other applications of the technique of approximating the reachability sets,
which allows to lower the complexity down, below the size of the reachability set.
One particular case, which seems to be prone to such techniques is the $2$-VASS with some number of $\Z$-counters, namely counters, which can take values below zero.
The best complexity lower bound for the reachability problem in this model is \pspace-hardness inherited from~\cite{BlondinFGHM15},
while the best upper bound is Ackermann membership inherited from VASS reachability~\cite{LS19}.
The reachability sets for that systems are not necessarily semilinear.
This disqualifies most of the techniques relying on the semilinearity of reachability sets, but our techniques
seem to be promising for that model.


