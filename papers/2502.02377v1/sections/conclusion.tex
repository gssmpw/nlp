\section{Conclusion}

We explored how to compute robust adaptive policies for Ad Hoc Teamwork. Building on Minimax-Bayes RL, we introduced a method to identify minimax distributions over background populations, which consistently yielded more robust policies compared to training with a uniform distribution. Surprisingly, we also found that training on minimax distributions can significantly accelerate learning.
However, utilising regret as an objective to tune the training distribution proved computationally expensive when best-response utilities are not readily available. In some special instances, we also found that the minimax-Bayes training approach w.r.t.\ utility can prevent policies from learning certain game dynamics.  

Looking forward, we see great potential in extending our approach to a curriculum learning framework based on partner distributions, which could dramatically improve sample efficiency, asymptotic performance, and AHT robustness. 

% --- 

% We investigated how to obtain robust adaptive policies in an AHT setting. Leveraging previous work on Minimax-Bayes RL, we propose a method to find worst-case distributions over background populations, which led to consistently robust policies compared to simply training policies using a uniform distribution over the background population. On one hand, we remarked that exploiting regret is computationally demanding when best-response utilities are not readily available, and observed that worst-case distributions can prevent policies from learning certain aspects of the game. On the other hand, we have found the unexpected results that these distributional choices can significantly accelerate learning, on top of improved AHT robustness and performance. 

% Looking ahead, we see strong potential in extending our approach to a curriculum-learning framework based on partner distributions. This approach could dramatically enhance sample efficiency, asymptotic performance, and robustness.

% \section{Limitation}
The use of 3D-printed PLA for structural components improves improving ease of assembly and reduces weight and cost, yet it causes deformation under heavy load, which can diminish end-effector precision. Using metal, such as aluminum, would remedy this problem. Additionally, \robot relies on integrated joint relative encoders, requiring manual initialization in a fixed joint configuration each time the system is powered on. Using absolute joint encoders could significantly improve accuracy and ease of use, although it would increase the overall cost. 

%Reliance on commercially available actuators simplifies integration but imposes constraints on control frequency and customization, further limiting the potential for tailored performance improvements.

% The 6 DoF configuration provides sufficient mobility for most tasks; however, certain bimanual operations could benefit from an additional degree of freedom to handle complex joint constraints more effectively. Furthermore, the limited torque density of commercially available proprioceptive actuators restricts the payload and torque output, making the system less suitability for handling heavier loads or high-torque applications. 

The 6 DoF configuration of the arm provides sufficient mobility for single-arm manipulation tasks, yet it shows a limitation in certain bimanual manipulation problems. Specifically, when \robot holds onto a rigid object with both hands, each arm loses 1 DoF because the hands are fixed to the object during grasping. This leads to an underactuated kinematic chain which has a limited mobility in 3D space. We can achieve more mobility by letting the object slip inside the grippers, yet this renders the grasp less robust and simulation difficult. Therefore, we anticipate that designing a lightweight 3 DoF wrist in place of the current 2 DoF wrist allows a more diverse repertoire of manipulation in bimanual tasks.

Finally, the limited torque density of commercially available proprioceptive actuators restricts the performance. Currently, all of our actuators feature a 1:10 gear ratio, so \robot can handle up to 2.5 kg of payload. To handle a heavier object and manipulate it with higher torque, we expect the actuator to have 1:20$\sim$30 gear ratio, but it is difficult to find an off-the-shelf product that meets our requirements. Customizing the actuator to increase the torque density while minimizing the weight will enable \robot to move faster and handle more diverse objects.

%These constraints highlight opportunities for improvement in future iterations, including alternative materials for enhanced rigidity, custom actuator designs for higher control precision and torque density, the adoption of absolute joint encoders, and optimized configurations to balance dexterity and weight.


%\section{Theoretical Analysis}\label{sec:theoretical}

\textbf{Different correct answers are competitor.}\quad For any LLM trained with cross-entropy loss, different correct answers are competitors in terms of probability \footnote{The ``same question'' refers to questions that are semantically equivalent but do not need to be identical.}. Continuing with the example of proposing a president, suppose $\tau^{a}$ (``\texttt{Barack}'') is the label of a sample whose $\bm{q}$ is ``\texttt{[INST]Could you give me one name of president?[\textbackslash INST]}'' and a generated token vector $\bm{a}_{t-1}$  can be decoded into ``\texttt{Sure, here is a historical American president:**}'', the loss of the next token at this position during supervised fine-tuning can be written as:
\begin{equation}
\begin{aligned}
 &L^{\tau^a} = - \log \frac{\exp(\mathcal{M}({\tau^a}|\bm{q},\bm{a}_{t-1}))}{\sum_{m=1}^{|\bm{Y}|} \exp(\mathcal{M}(\tau^{m}|\bm{q},\bm{a}_{t-1}))} ,
 % \\   &L^{\tau^b} = - \log \frac{\exp(\mathcal{M}(\tau^b|\bm{q},\bm{a}_{t-1}))}{\sum_{m=1}^{|\bm{Y}|} \exp(\mathcal{M}(\tau^{m}|\bm{q},\bm{a}_{t-1}))} ,
\end{aligned}
\end{equation}
where $L^{\tau^a}$ is the loss on the sample with the next token label $\tau^{a}$.
Consider cases where multiple distinct answers to the same question appear in the training set, the situation becomes different. For example, $\tau^{b}$ (``\texttt{George}'') is the label in another sample with the same question. When the model is simultaneously fine-tuned on both samples, the gradient update for the model will be:
\begin{equation}
\begin{aligned}
 & \nabla_{\mathcal{M}} (L^{\tau^a} + L^{\tau^b}) = \nabla_{\mathcal{M}} L^{\tau^a} + \nabla_{\mathcal{M}} L^{\tau^b} \\
% &= -y_a^{\tau^a}\frac{1}{\Omega_a^{\tau^a}}\nabla_{\mathcal{M}}\Omega_a^{\tau^a}-\sum_{m \neq a}^{|\bm{Y}|} y_a^{\tau^m}\frac{1}{\Omega_a^{\tau^m}}\nabla_{\mathcal{M}}\Omega_a^{\tau^m}
% \\
% &\quad -y_b^{\tau^b}\frac{1}{\Omega_b^{\tau^b}}\nabla_{\mathcal{M}}\Omega_b^{\tau^b}-\sum_{m \neq b}^{|\bm{Y}|} y_b^{\tau^m}\frac{1}{\Omega_b^{\tau^m}}\nabla_{\mathcal{M}}\Omega_b^{\tau^m}
% \\
&\quad= \underbrace{-y_a^{\tau^a}\frac{1}{\Omega_a^{\tau^a}}\nabla_{\mathcal{M}}\Omega_a^{\tau^a}-y_b^{\tau^b}\frac{1}{\Omega_b^{\tau^b}}\nabla_{\mathcal{M}}\Omega_b^{\tau^b}}_{\text{(1) maximizing the probability of annotated answer}}\\& \quad \underbrace{-y_a^{\tau^b}\frac{1}{\Omega_a^{\tau^b}}\nabla_{\mathcal{M}}\Omega_a^{\tau^b}-y_b^{\tau^a}\frac{1}{\Omega_b^{\tau^a}}\nabla_{\mathcal{M}}\Omega_b^{\tau^a}}_{{\text{\textbf{(2)} minimizing the probability of the other annotated answer}}}\\& \quad \underbrace{-\sum_{m \neq a,b}^{|\bm{Y}|}y_{a,b}^{\tau^m} \left[ \frac{1}{\Omega_a^{\tau^m}}\nabla_{\mathcal{M}}\Omega_a^{\tau^m} + \frac{1}{\Omega_b^{\tau^m}}\nabla_{\mathcal{M}}\Omega_b^{\tau^m} \right]}_{\text{(3) minimizing the probability of incorrect answers}},
\end{aligned}\label{eq:competitor}
\end{equation}
where $\Omega_a^{\tau^a}=\frac{\exp(\mathcal{M}(\tau^a|\bm{q},\bm{a}_{t-1}))}{\sum_{m=1}^{|\bm{Y}|} \exp(\mathcal{M}(\tau^{m}|\bm{q},\bm{a}_{t-1}))}$, and $y_a^{\tau^m}$ indicates the next token label of a training sample with ground-truth label ${\tau^a}$, that is, we have $y_a^{\tau^a}=1$ and $y_a^{\tau^b}=0$. In particular, when $\mathcal{M}$ is in a certain state during training, we have $\Omega_a^{\tau^a}=\Omega_b^{\tau^a}$, and we make distinctions to facilitate the reader's understanding here. As we can see, for scenarios with multiple answers, the training objective can be divided into three parts:
(1) For each sample, increase the probability of its own annotation in the output distribution.
(2) For each sample, decrease the probability of another sample's annotation in the output distribution. \textit{\textbf{Note:}} This part leads to the issue where probability cannot anymore capture the reliability of LLM responses, as different correct answers tend to reduce the probability of other correct answers, making low probabilities cannot indicates low reliability.
(3) For both samples, decrease the probability of other outputs not present in the annotations in the output distribution.



