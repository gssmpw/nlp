% This could be a subsection in 4, either in 4.1, or as 4.2.
\section{Out-Of-Distribution Robustness}
\label{section:ood}

As already stated in Section~\ref{subsec:constructing_training_scenarios}, having a diverse set of scenarios that adequately represents the true set of scenarios is crucial. However, since it is often impractical to perfectly replicate the true set, the prior used during training may not have the same support as the true distribution observed in nature. In such cases, the guarantees outlined in Section~\ref{subsec:minimax_bayes_aht} no longer hold on the true distribution. In order to state further robustness guarantees, an option is to assume that scenarios in the true scenario set are close to the training scenarios. To quantify the closeness between scenarios, we first define the distance between two policy vectors as their maximum total variation across all states:
\begin{equation}
    \label{eq:policy_vector_distance}
    d(\bpi, \bpi') = \max_{s\in \states }\sum_i \sum_a \big| \bpi_i(a|s) - \bpi'_i(a|s) \big|.
\end{equation}
We define the scenario distance as the minimum distance between policy vectors across permutations of the background policies:
\begin{equation}
    \label{eq:scenario_distance}
    d(\scenario, \scenario') = \min_{\bpi, \bpi' \in \text{Perm}(\bpi^b) \times \text{Perm}(\bpi^{b'})} d(\bpi, \bpi'),
\end{equation}
This metric measures the similarity between the background policies of two scenarios. Scenarios can only be compared if they have the same number of focal players (e.g., $\scenario=(c,\bpi^b)$ and $\scenario'=(c,\bpi^{b'})$).
\begin{definition}[$\epsilon$-net of a scenario set] 
    A finite set of scenarios $\scenarioset$ is called an $\epsilon$-net of a scenario set $S$ if and only if, for every scenario $\scenario\in S$, there exists a scenario $\scenario'\in\scenarioset$ such that $d(\scenario,\scenario')<\epsilon$.
\end{definition}

\begin{lemma}
    \label{lemma:scenario_equivalence}
    Let $\scenarioset$ be an $\epsilon$-net for a scenario set $S$. For any policy $\pi \in \policies$ and scenario $\scenario\in S$, there is a scenario $\scenario' \in \scenarioset$ that verifies:
    \begin{equation}
        \label{eq:performance.guarantee}
        \big| U(\pi, \scenario) - U(\pi, \scenario') \big| < \dfrac{\epsilon T^2\rmax}{2}.
    \end{equation}
\end{lemma}
\begin{proofsketch}
    The result is obtained by using the fact that for any pairs of $\epsilon$-close scenarios $\scenario, \scenario'$ and any $s, \afocal, i$, we have $\sum_{s'}|P_\scenario(s'| s, \afocal) - P_{\scenario'}(s'| s, \afocal) | < \epsilon$ and $|\rewardfunc_\scenario(s, \afocal, i) - \rewardfunc_{\scenario'}(s, \afocal, i)|\\< \epsilon\rmax$. The proof is concluded by showing by induction that for all $t$ and $s$ , $|U_t(\pi, \scenario, s)-U_t(\pi, \scenario', s)| < \frac{1}{2} \epsilon(T-t+1)(T-t)\rmax$.
\end{proofsketch}

\begin{lemma}
    \label{lemma:scenario_equivalence_regret}
    Let $\scenarioset$ be an $\epsilon$-net for some scenario set $S$. For any policy $\pi\in\policies$ and scenario $\scenario\in S$, there is a scenario $\scenario' \in \scenarioset$ such that
    \begin{equation}
        \label{eq:performance.regret}
        \big| R(\pi, \scenario) - R(\pi, \scenario') \big| < \epsilon T^2\rmax.
    \end{equation}
\end{lemma}
\begin{proofsketch}
    The result is obtained by both using the identity $|U^*(\scenario) - U^*(\scenario')| \leq \max_\pi |U(\pi, \scenario) - U(\pi, \scenario')|$
    and noticing that for any policy $\pi$,  $|R(\pi, \scenario)-R(\pi, \scenario')|\leq |U^*(\scenario)- U^*(\scenario')| + |U(\pi, \scenario) - U(\pi, \scenario')|$.
\end{proofsketch}

\begin{lemma}
    \label{lemma:wcu.guarantees}
    Let $\scenarioset$ be an $\epsilon$-net for some scenario set $S$, and $\pi^*_U$the optimal policy for the maximin utility problem \eqref{eq:mbmarl.maximin} on $\scenarioset$, then
    \begin{equation}
        \wcu(\pi^*_U, S) > \max_{\pi \in \policies} \left( \wcu(\pi, \scenarioset) - \dfrac{\epsilon T^2 \rmax}{2} \right). 
    \end{equation}
\end{lemma}
\begin{proofsketch}
    We denote $\scenario_\text{wc}(\scenarioset)$ and $\scenario_\text{wc}(S)$ the worst-case scenarios for $\pi^*_U$ on $\scenarioset$ and $S$, and reason on the distance between $\scenario_\text{wc}(\scenarioset)$ and $\scenario_\text{wc}(S)$. If $d(\scenario_\text{wc}(\scenarioset),\scenario_\text{wc}(S)') < \epsilon$, then Lemma~\ref{lemma:scenario_equivalence} applies.  Otherwise, since $\scenarioset$ is an $\epsilon$-net, we can find another scenario $\scenario_\epsilon \in \scenarioset$ that is $\epsilon$-close to $\scenario_\text{wc}(S)$ and use the fact that the utility of $\pi^*_U$ is by definition higher on $\scenario_\epsilon$ than on $\scenario_\text{wc}(\scenarioset)$.
\end{proofsketch}

\begin{lemma}
    \label{lemma:wcr.guarantees}
    Let $\scenarioset$ be an $\epsilon$-net for some scenario set $S$, and $\pi^*_R$ the optimal policy for the minimax regret problem \eqref{eq:mbmarl.minimax} on $\scenarioset$, then
    \begin{equation}
        \wcr(\pi^*_R, S) < \min_{\pi \in \policies} \big( \wcr(\pi, \scenarioset) + \epsilon T^2 \rmax \big).
    \end{equation}
\end{lemma}
\begin{proofsketch}
        We prove, analogically to Lemma~\ref{lemma:wcu.guarantees}, the result using Lemma~\ref{lemma:scenario_equivalence_regret} in place of Lemma~\ref{lemma:scenario_equivalence}.
\end{proofsketch}

Lemmas~\ref{lemma:wcu.guarantees} and \ref{lemma:wcr.guarantees} provide worst-case guarantees on arbitrary sets of scenarios, for policies solving the minimax problems. This also means that we can have those guarantees on non-finite sets of scenarios. Importantly, as long as we have an $\epsilon$-net of training scenarios for the true set, the policy solving the maximin utility (or minimax regret) problem has a strong worst-case utility (or regret) guarantee. In contrast, it is impossible to guarantee \emph{anything} additional about the average utility $\perf$ on the true set, as the latter could very well include scenarios that are all $\epsilon$-close to the worst-case scenarios of the training set. For this reason, the average utility on the true set can be as low as the worst-case utility.
