% Unofficial University of Cambridge Poster Template
% https://github.com/andiac/gemini-cam
% a fork of https://github.com/anishathalye/gemini
% also refer to https://github.com/k4rtik/uchicago-poster

\documentclass[final]{beamer}

% ====================
% Packages
% ====================

\usepackage[T1]{fontenc}
\usepackage{lmodern}
\usepackage[orientation=portrait, size=a0,scale=1.0]{beamerposter}
% \usepackage[width=36, height=24, scale=1.0]{beamerposter}
%\usepackage[scale=1.0]{beamerposter}
\usetheme{gemini}
\usecolortheme{nott}
\usepackage{graphicx}
\usepackage{booktabs}
\usepackage{tikz}
\usepackage{pgfplots}
\pgfplotsset{compat=1.14}
\usepackage{anyfontsize}
\usepackage{adjustbox}

%
\setlength\unitlength{1mm}
\newcommand{\twodots}{\mathinner {\ldotp \ldotp}}
% bb font symbols
\newcommand{\Rho}{\mathrm{P}}
\newcommand{\Tau}{\mathrm{T}}

\newfont{\bbb}{msbm10 scaled 700}
\newcommand{\CCC}{\mbox{\bbb C}}

\newfont{\bb}{msbm10 scaled 1100}
\newcommand{\CC}{\mbox{\bb C}}
\newcommand{\PP}{\mbox{\bb P}}
\newcommand{\RR}{\mbox{\bb R}}
\newcommand{\QQ}{\mbox{\bb Q}}
\newcommand{\ZZ}{\mbox{\bb Z}}
\newcommand{\FF}{\mbox{\bb F}}
\newcommand{\GG}{\mbox{\bb G}}
\newcommand{\EE}{\mbox{\bb E}}
\newcommand{\NN}{\mbox{\bb N}}
\newcommand{\KK}{\mbox{\bb K}}
\newcommand{\HH}{\mbox{\bb H}}
\newcommand{\SSS}{\mbox{\bb S}}
\newcommand{\UU}{\mbox{\bb U}}
\newcommand{\VV}{\mbox{\bb V}}


\newcommand{\yy}{\mathbbm{y}}
\newcommand{\xx}{\mathbbm{x}}
\newcommand{\zz}{\mathbbm{z}}
\newcommand{\sss}{\mathbbm{s}}
\newcommand{\rr}{\mathbbm{r}}
\newcommand{\pp}{\mathbbm{p}}
\newcommand{\qq}{\mathbbm{q}}
\newcommand{\ww}{\mathbbm{w}}
\newcommand{\hh}{\mathbbm{h}}
\newcommand{\vvv}{\mathbbm{v}}

% Vectors

\newcommand{\av}{{\bf a}}
\newcommand{\bv}{{\bf b}}
\newcommand{\cv}{{\bf c}}
\newcommand{\dv}{{\bf d}}
\newcommand{\ev}{{\bf e}}
\newcommand{\fv}{{\bf f}}
\newcommand{\gv}{{\bf g}}
\newcommand{\hv}{{\bf h}}
\newcommand{\iv}{{\bf i}}
\newcommand{\jv}{{\bf j}}
\newcommand{\kv}{{\bf k}}
\newcommand{\lv}{{\bf l}}
\newcommand{\mv}{{\bf m}}
\newcommand{\nv}{{\bf n}}
\newcommand{\ov}{{\bf o}}
\newcommand{\pv}{{\bf p}}
\newcommand{\qv}{{\bf q}}
\newcommand{\rv}{{\bf r}}
\newcommand{\sv}{{\bf s}}
\newcommand{\tv}{{\bf t}}
\newcommand{\uv}{{\bf u}}
\newcommand{\wv}{{\bf w}}
\newcommand{\vv}{{\bf v}}
\newcommand{\xv}{{\bf x}}
\newcommand{\yv}{{\bf y}}
\newcommand{\zv}{{\bf z}}
\newcommand{\zerov}{{\bf 0}}
\newcommand{\onev}{{\bf 1}}

% Matrices

\newcommand{\Am}{{\bf A}}
\newcommand{\Bm}{{\bf B}}
\newcommand{\Cm}{{\bf C}}
\newcommand{\Dm}{{\bf D}}
\newcommand{\Em}{{\bf E}}
\newcommand{\Fm}{{\bf F}}
\newcommand{\Gm}{{\bf G}}
\newcommand{\Hm}{{\bf H}}
\newcommand{\Id}{{\bf I}}
\newcommand{\Jm}{{\bf J}}
\newcommand{\Km}{{\bf K}}
\newcommand{\Lm}{{\bf L}}
\newcommand{\Mm}{{\bf M}}
\newcommand{\Nm}{{\bf N}}
\newcommand{\Om}{{\bf O}}
\newcommand{\Pm}{{\bf P}}
\newcommand{\Qm}{{\bf Q}}
\newcommand{\Rm}{{\bf R}}
\newcommand{\Sm}{{\bf S}}
\newcommand{\Tm}{{\bf T}}
\newcommand{\Um}{{\bf U}}
\newcommand{\Wm}{{\bf W}}
\newcommand{\Vm}{{\bf V}}
\newcommand{\Xm}{{\bf X}}
\newcommand{\Ym}{{\bf Y}}
\newcommand{\Zm}{{\bf Z}}

% Calligraphic

\newcommand{\Ac}{{\cal A}}
\newcommand{\Bc}{{\cal B}}
\newcommand{\Cc}{{\cal C}}
\newcommand{\Dc}{{\cal D}}
\newcommand{\Ec}{{\cal E}}
\newcommand{\Fc}{{\cal F}}
\newcommand{\Gc}{{\cal G}}
\newcommand{\Hc}{{\cal H}}
\newcommand{\Ic}{{\cal I}}
\newcommand{\Jc}{{\cal J}}
\newcommand{\Kc}{{\cal K}}
\newcommand{\Lc}{{\cal L}}
\newcommand{\Mc}{{\cal M}}
\newcommand{\Nc}{{\cal N}}
\newcommand{\nc}{{\cal n}}
\newcommand{\Oc}{{\cal O}}
\newcommand{\Pc}{{\cal P}}
\newcommand{\Qc}{{\cal Q}}
\newcommand{\Rc}{{\cal R}}
\newcommand{\Sc}{{\cal S}}
\newcommand{\Tc}{{\cal T}}
\newcommand{\Uc}{{\cal U}}
\newcommand{\Wc}{{\cal W}}
\newcommand{\Vc}{{\cal V}}
\newcommand{\Xc}{{\cal X}}
\newcommand{\Yc}{{\cal Y}}
\newcommand{\Zc}{{\cal Z}}

% Bold greek letters

\newcommand{\alphav}{\hbox{\boldmath$\alpha$}}
\newcommand{\betav}{\hbox{\boldmath$\beta$}}
\newcommand{\gammav}{\hbox{\boldmath$\gamma$}}
\newcommand{\deltav}{\hbox{\boldmath$\delta$}}
\newcommand{\etav}{\hbox{\boldmath$\eta$}}
\newcommand{\lambdav}{\hbox{\boldmath$\lambda$}}
\newcommand{\epsilonv}{\hbox{\boldmath$\epsilon$}}
\newcommand{\nuv}{\hbox{\boldmath$\nu$}}
\newcommand{\muv}{\hbox{\boldmath$\mu$}}
\newcommand{\zetav}{\hbox{\boldmath$\zeta$}}
\newcommand{\phiv}{\hbox{\boldmath$\phi$}}
\newcommand{\psiv}{\hbox{\boldmath$\psi$}}
\newcommand{\thetav}{\hbox{\boldmath$\theta$}}
\newcommand{\tauv}{\hbox{\boldmath$\tau$}}
\newcommand{\omegav}{\hbox{\boldmath$\omega$}}
\newcommand{\xiv}{\hbox{\boldmath$\xi$}}
\newcommand{\sigmav}{\hbox{\boldmath$\sigma$}}
\newcommand{\piv}{\hbox{\boldmath$\pi$}}
\newcommand{\rhov}{\hbox{\boldmath$\rho$}}
\newcommand{\upsilonv}{\hbox{\boldmath$\upsilon$}}

\newcommand{\Gammam}{\hbox{\boldmath$\Gamma$}}
\newcommand{\Lambdam}{\hbox{\boldmath$\Lambda$}}
\newcommand{\Deltam}{\hbox{\boldmath$\Delta$}}
\newcommand{\Sigmam}{\hbox{\boldmath$\Sigma$}}
\newcommand{\Phim}{\hbox{\boldmath$\Phi$}}
\newcommand{\Pim}{\hbox{\boldmath$\Pi$}}
\newcommand{\Psim}{\hbox{\boldmath$\Psi$}}
\newcommand{\Thetam}{\hbox{\boldmath$\Theta$}}
\newcommand{\Omegam}{\hbox{\boldmath$\Omega$}}
\newcommand{\Xim}{\hbox{\boldmath$\Xi$}}


% Sans Serif small case

\newcommand{\Gsf}{{\sf G}}

\newcommand{\asf}{{\sf a}}
\newcommand{\bsf}{{\sf b}}
\newcommand{\csf}{{\sf c}}
\newcommand{\dsf}{{\sf d}}
\newcommand{\esf}{{\sf e}}
\newcommand{\fsf}{{\sf f}}
\newcommand{\gsf}{{\sf g}}
\newcommand{\hsf}{{\sf h}}
\newcommand{\isf}{{\sf i}}
\newcommand{\jsf}{{\sf j}}
\newcommand{\ksf}{{\sf k}}
\newcommand{\lsf}{{\sf l}}
\newcommand{\msf}{{\sf m}}
\newcommand{\nsf}{{\sf n}}
\newcommand{\osf}{{\sf o}}
\newcommand{\psf}{{\sf p}}
\newcommand{\qsf}{{\sf q}}
\newcommand{\rsf}{{\sf r}}
\newcommand{\ssf}{{\sf s}}
\newcommand{\tsf}{{\sf t}}
\newcommand{\usf}{{\sf u}}
\newcommand{\wsf}{{\sf w}}
\newcommand{\vsf}{{\sf v}}
\newcommand{\xsf}{{\sf x}}
\newcommand{\ysf}{{\sf y}}
\newcommand{\zsf}{{\sf z}}


% mixed symbols

\newcommand{\sinc}{{\hbox{sinc}}}
\newcommand{\diag}{{\hbox{diag}}}
\renewcommand{\det}{{\hbox{det}}}
\newcommand{\trace}{{\hbox{tr}}}
\newcommand{\sign}{{\hbox{sign}}}
\renewcommand{\arg}{{\hbox{arg}}}
\newcommand{\var}{{\hbox{var}}}
\newcommand{\cov}{{\hbox{cov}}}
\newcommand{\Ei}{{\rm E}_{\rm i}}
\renewcommand{\Re}{{\rm Re}}
\renewcommand{\Im}{{\rm Im}}
\newcommand{\eqdef}{\stackrel{\Delta}{=}}
\newcommand{\defines}{{\,\,\stackrel{\scriptscriptstyle \bigtriangleup}{=}\,\,}}
\newcommand{\<}{\left\langle}
\renewcommand{\>}{\right\rangle}
\newcommand{\herm}{{\sf H}}
\newcommand{\trasp}{{\sf T}}
\newcommand{\transp}{{\sf T}}
\renewcommand{\vec}{{\rm vec}}
\newcommand{\Psf}{{\sf P}}
\newcommand{\SINR}{{\sf SINR}}
\newcommand{\SNR}{{\sf SNR}}
\newcommand{\MMSE}{{\sf MMSE}}
\newcommand{\REF}{{\RED [REF]}}

% Markov chain
\usepackage{stmaryrd} % for \mkv 
\newcommand{\mkv}{-\!\!\!\!\minuso\!\!\!\!-}

% Colors

\newcommand{\RED}{\color[rgb]{1.00,0.10,0.10}}
\newcommand{\BLUE}{\color[rgb]{0,0,0.90}}
\newcommand{\GREEN}{\color[rgb]{0,0.80,0.20}}

%%%%%%%%%%%%%%%%%%%%%%%%%%%%%%%%%%%%%%%%%%
\usepackage{hyperref}
\hypersetup{
    bookmarks=true,         % show bookmarks bar?
    unicode=false,          % non-Latin characters in AcrobatÕs bookmarks
    pdftoolbar=true,        % show AcrobatÕs toolbar?
    pdfmenubar=true,        % show AcrobatÕs menu?
    pdffitwindow=false,     % window fit to page when opened
    pdfstartview={FitH},    % fits the width of the page to the window
%    pdftitle={My title},    % title
%    pdfauthor={Author},     % author
%    pdfsubject={Subject},   % subject of the document
%    pdfcreator={Creator},   % creator of the document
%    pdfproducer={Producer}, % producer of the document
%    pdfkeywords={keyword1} {key2} {key3}, % list of keywords
    pdfnewwindow=true,      % links in new window
    colorlinks=true,       % false: boxed links; true: colored links
    linkcolor=red,          % color of internal links (change box color with linkbordercolor)
    citecolor=green,        % color of links to bibliography
    filecolor=blue,      % color of file links
    urlcolor=blue           % color of external links
}
%%%%%%%%%%%%%%%%%%%%%%%%%%%%%%%%%%%%%%%%%%%



% ====================
% Lengths
% ====================

% If you have N columns, choose \sepwidth and \colwidth such that
% (N+1)*\sepwidth + N*\colwidth = \paperwidth
\newlength{\sepwidth}
\newlength{\colwidth}
\setlength{\sepwidth}{0.02\paperwidth}
\setlength{\colwidth}{0.46\paperwidth}
\setlength{\maxlogowidth}{0mm}

\newcommand{\separatorcolumn}{\begin{column}{\sepwidth}\end{column}}

\newcommand{\boxalign}[2][0.99\textwidth]{
 \par\noindent\tikzstyle{mybox} = [draw=black,inner sep=6pt]
 \begin{center}\begin{tikzpicture}
  \node [mybox] (box){%
   \begin{minipage}{#1}{\vspace{-5mm}#2}\end{minipage}
  };
 \end{tikzpicture}\end{center}
}

% ====================
% Title
% ====================

\title{A Minimax-Bayes Approach to Ad Hoc Teamwork}

\author{Victor Villin \inst{1} \and Thomas Kleine Buening \inst{2} \and Christos Dimitrakakis \inst{2 3}}

\institute[shortinst]{\inst{1} University of Neuchatel \samelineand \inst{2} The Alan Turing Institute  \samelineand 
\inst{3} University of Oslo 
}

% ====================
% Footer (optional)
% ====================

\footercontent{
  EWRL17 (2024)
  \hfill
  \href{mailto:victor.villin@unine.ch}{victor.villin@unine.ch}  } 
% (can be left out to remove footer)

% ====================
% Logo (optional)
% ====================

% use this to include logos on the left and/or right side of the header:
\logoleft{\centering
\includegraphics[height=5cm]{figures/unine_logo_blanc.png}
}

\logoright{\centering
\includegraphics[height=3.5cm]{figures/Alan_Turing_Institute_logo_white.png}
\hspace{1mm}
\includegraphics[height=3.5cm]{figures/uio-segl-negativ-150x150.png}
}
%\logoright{\centering\includegraphics[height=5cm]{figures/ICML_logo_white.png}}

% ====================
% Body
% ====================

\begin{document}

% Refer to https://github.com/k4rtik/uchicago-poster
% logo: https://www.cam.ac.uk/brand-resources/about-the-logo/logo-downloads
% \addtobeamertemplate{headline}{}
% {
%     \begin{tikzpicture}[remember picture,overlay]
%       \node [anchor=north west, inner sep=3cm] at ([xshift=-2.5cm,yshift=2.75cm]current page.north west)
%       {\includegraphics[height=4.5cm]{logos/unott-logo.eps}}; 
%     \end{tikzpicture}
% }

\begin{frame}[t]

\begin{columns}[t]
\separatorcolumn

\begin{column}{\colwidth}
  
    \begin{block}{Learning policies for AHT is challenging}
    \vspace{8mm}
    \begin{alertblock}{}
        \bfemph{Ad Hoc Teamwork (AHT)} occurs when multiple agents, initially \bfemph{unfamiliar} with each other, must \bfemph{collaborate} to achieve a \bfemph{common} goal.
    \end{alertblock}
    \begin{enumerate}
        \item Numerous and diverse \bfemph{scenarios} possible,
        \item Existing methods offer \bfemph{limited guarantees} in terms of \bfemph{worst-case} AHT performance,
        \item \bfemph{Robust AI-Human Cooperation} is becoming a concern,
        \item The distribution of \bfemph{training} partners is typically \bfemph{not} the distribution of partners \bfemph{after deployment}.
    \end{enumerate}

    \begin{alertblock}{}
    We consider using the \bfemph{worst} possible prior over scenarios to learn a robust policy, an idea adopted from the \bfemph{minimax-Bayes} concept.
    \end{alertblock}
    \vspace{-5mm}
    % \begin{alertblock}{}
    %     {We propose an \bfemph{environment design} framework for IRL to recover robust estimates of the true reward function with the \bfemph{least} number of samples.
    %     }
    % \end{alertblock}
  \end{block}


    \begin{block}{Evaluating AHT capabilities}
    \vspace{5mm}
    We are interested in learning a \bfemph{robustly cooperative} policy $\pi^f$ for some $m$-player Partially Observable Markov Game (POMG) $\mdp$.
    \vspace{-5mm}
     \heading{Scenarios.}
 A scenario $\scenario=(c, \bpi^b)$ is characterised by its actors:
    \begin{itemize}
        \item $c$ \bfemph{focal} players $\bpi^f = (\pi^f, \dots, \pi^f)$ (all equal to the learned policy).
        \item $m-c$ \bfemph{background} players $\bpi^b = (\pi^b_1, \dots, \pi^b_{m-c})$ (fixed),
        \item Each scenario can be seen \bfemph{as its own} $c$-player POMG $\mdp(\scenario)$.
        \item We construct scenario sets with a background population of policies $\bgpop$:
    \end{itemize}
    \vspace{-5mm}
    \begin{equation*}
        \scenarioset(\bgpop) \defeq \{ (c, \bpi^b) \mid 1 \leq c \leq m, \bpi^b \in \bgpop^{m-c}\},
\end{equation*}

    \heading{Objectives.} We want a policy that \bfemph{reliably maximises utility / minimises regret}, regardless of the scenario:
    \vspace{-6mm}
    \begin{itemize}
        \item $\scenario$-induced reward function $\rewardfunc_\scenario$; focal actions $\afocal = (a^f_1, \dots, a^f_c)$,
        \vspace{-2mm}
        \begin{equation*}
        \label{eq:EU}
        \text{\bfemph{Mean focal utility:}}\quad
        U(\pi, \scenario) \defeq 
        % \mathbb{E} [u(\pi, \scenario)] =
        \sum_{t=1}^T  \dfrac{1}{c} \sum_{i=1}^c \mathbb{E}^{\pi}_{\mdp(\scenario)}[\rho_\scenario(s_t, \afocal_t, i)], \hfill 
        \end{equation*}
    \item Maximal utility for scenario $\scenario$, $U^*(\scenario)$,
        \begin{equation*}
         \text{\bfemph{Regret:}}\quad
        R(\pi,\scenario) \defeq U^*(\scenario) - U(\pi, \scenario).
        \end{equation*}
    \end{itemize}
    \vspace{5mm}
    We \bfemph{assess} learning methods with the following two-phased protocol:
        \vspace{-5mm}
        \heading{1) Training}:
        \vspace{-5mm}
        \begin{itemize}
            \item A background population of training partners $\poptrain$ is \bfemph{provided},
            \item A test set of scenarios $\scenariotest$ is kept \bfemph{held-out},
            \item The learner is allowed to do anything for $N$ environment steps.
        \end{itemize}
        \vspace{-10mm}
        \heading{2) Testing}:
        \vspace{-5mm}
        \begin{itemize}
            \item \bfemph{No more} learning,
            \item The obtained policy is \bfemph{evaluated} on held-out test scenarios $\scenariotest$ with:
        \vspace{3mm}
        \begin{alertblock}{}
        \vspace{-16mm}
        \begin{align*}
        \label{eq:metrics}
        & \text{\bfemph{Average utility}} && \;\;\text{\bfemph{Worst-case utility}} && \;\;\text{\bfemph{Worst-case regret}}\\
        &  \perf(\pi, \scenarioset) = \dfrac{1}{|\scenarioset|}\sum_{\scenario \in \scenarioset} U(\pi, \scenario), \quad && \wcu(\pi, \scenarioset) =  \min_{\scenario \in \scenarioset} U(\pi, \scenario),\quad&& \wcr(\pi, \scenarioset) =  \max_{\scenario \in \scenarioset} R(\pi, \scenario).
    \end{align*}
    \vspace{-8mm}
    \end{alertblock}
        \end{itemize}


    %\heading{Assumptions.}
    %\vspace{-5mm}
    %\begin{enumerate}
    %    \item Absence of prior coordination,
    %    \item No control over teammates
    %    \item All agents share a common objective, with different preferences: %prosociality levels $\lambda_i$ and risk aversion $\delta_i$.
    %\end{enumerate} 
    \end{block}
    \begin{block}{Achieving Robust AHT}
    Two main ingredients:
    \vspace{-5mm}
    \begin{itemize}
        \item[\textbf{1)}] Diverse training partners representative of what is in nature.
        \item[\textbf{2)}] \textbf{An appropriate \bfemph{prior} $\beta$ over scenarios.}
    \end{itemize}
    \vspace{-8mm}
    \hspace{1cm} \Rightarrow We pick the \bfemph{minimax} prior w.r.t. utility/regret.
    %\heading{1) Constructing Training Scenarios.} To thoroughly assess the \bfemph{effects} of partner priors, we adopt the following  approach:
    %\vspace{-8mm}
    %\begin{itemize}
    %\item Background policies have different \bfemph{preferences}: prosociality levels $\lambda_i$ and risk aversion $\delta_i$, 
    %\item Policies are organized into \bfemph{sub-populations} $\bgpop = \bigcup_k \bgpop_k$ of varying sizes,
    %\item Sub-populations are \bfemph{separately trained} through Population Play (PP),
    %\item \bfemph{Distinct} habits, common practices, and established conventions will emerge within each group, mimicking \bfemph{various} cultures.
    %\end{itemize}
    \vspace{8mm}
    
    With \bfemph{expected utility} and \bfemph{Bayesian regret} over scenario distributions defined as:
    \begin{equation*}
         U(\pi, \beta) \defeq \sum_{\scenario} U(\pi, \scenario) \beta(\scenario), \quad\quad\quad L(\pi, \beta) \defeq \sum_\sigma R(\pi, \mdp) \beta(\sigma).
    \end{equation*}
    
    \begin{alertblock}{}
    We play one of the following minimax games:
    \begin{equation*}
     \max_{\pi} \min_{\beta \in \Delta(\scenarioset(\bgpop))} U(\pi, \beta), \;\;\text{(1)}\quad\quad\quad \min_{\pi} \max_{\beta \in \Delta(\scenarioset(\bgpop))}L(\pi, \beta). \;\;\text{(2)}
    \end{equation*}
    \vspace{-5mm}
    \end{alertblock}
    \begin{itemize}
        \item \bfemph{Compute} solutions ($\pi^*, \beta^*$) with (stochastic) \bfemph{gradient descent ascent},
        \item \bfemph{Convergence guarantees} for \bfemph{simple} parametrisations of policies,
        \item \bfemph{Approximate} $c$-player scenarios to \bfemph{single-agent} environments by replacing delayed $c-1$ copies with a delayed version of the focal policy $\pi^f_{t-d}$.
        \item \bfemph{Assumption}: No "impossible" scenario.
    \end{itemize}
    \vspace{-10mm}
    \end{block}
    \end{column}

    \separatorcolumn
    
    \begin{column}{\colwidth}

    \begin{figure}
    \centering
    \vspace{-5mm}
    \includegraphics[width=0.6\textwidth]{data/illustration_bf.png}
    \vspace{-3mm}
    \caption{Training framework proposed in this work.}
    \vspace{-6mm}
    \end{figure}

\begin{block}{Robustness Guarantees}
    Policies \bfemph{$\pi^*_U$} and \bfemph{$\pi^*_R$} solving~(1) and (2), respectively, have several \bfemph{properties}.
    
    \textbf{In-Distribution}: \bfemph{Optimal} worst-case utility/regret on the \bfemph{training set}.
    
    \textbf{Out-Of-Distribution}: If the true scenarios are all \bfemph{$\epsilon$-close} to one of the training scenarios:
    \begin{equation*}
         \wcu(\pi^*_U, \scenarioset^\text{true}) > \max_{\pi} \left( \wcu(\pi, \scenarioset^\text{train}) - \dfrac{\epsilon T^2 \rmax}{2} \right),
    \end{equation*}
    \begin{equation*}
        \wcr(\pi^*_R, \scenarioset^\text{true}) < \min_{\pi} \left( \wcr(\pi, \scenarioset^\text{train}) + \epsilon T^2 \rmax \right).
    \end{equation*}
    \vspace{-18mm}
    
\end{block}
\begin{block}{Experimental Results}
    \heading{\Large Repeated Prisoner's Dilemma. (3 rounds, Fully adaptive policies)}
    
    \textbf{Training partners $\poptrain$}: 9 ad-hoc policies (pure defect/cooperate, tit-for-tat...)
    
    \textbf{Test partners $\poptest$}: 512 sampled policies $\epsilon$-close to training partners ($\epsilon=0.5$). 

    \begin{table}[t]
    %\small
    \setlength{\tabcolsep}{20pt}
    \centering
    \caption{
    Scores on the repeated Prisoner's Dilemma. Note: lower regret is better.}
    \vspace{-8mm}
    \label{tab:prisoner.train}
    \begin{tabular}{l||ccc||ccc||} %\toprule
    & \multicolumn{3}{c||}{$\scenarioset(\poptrain)$} & \multicolumn{3}{c||}{$\scenarioset(\poptest)$}\\
    \cmidrule(lr){2-4}\cmidrule(lr){5-7}
               & $\perf$ & $\wcu$ & $\wcr$ & $\perf$ & $\wcu$ & $\wcr$ \\ \midrule
        Minimax Utility (MU)  & $7.69$ & $\mathbf{3.00}$ & $9.00$ & $\mathbf{8.34}$ & $\mathbf{3.00}$ & $9.00$\\
        Maximin Regret (MR) & $8.23$ & $2.26$ & $\mathbf{3.79}$ & $7.96$ & $2.78$ & $\mathbf{4.35}$\\ \hdashline
        Population Best Response (PBR)  & $\mathbf{8.54}$ & $2.00$ & $4.97$ & $8.07$ & $2.48$ & $5.63$\\
        Fictitious-Play (FP) & $7.06$ & $0.14$ & $10.56$ & $6.33$ & $0.56$ & $9.96$\\
        Self-Play (SP)  & $7.25$ & $0.47$ & $10.19$ & $6.49$ & $0.86$ & $9.65$\\
        Random & $7.40$ & $1.50$ & $5.50$ & $7.47$ & $2.18$ & $5.20$\\
    % \bottomrule
    \end{tabular}
    \end{table}
    \begin{figure}
\centering
\begin{subfigure}[t]{0.49\linewidth}
\centering
\begin{tikzpicture}[
  scale=2.25,
  level 1/.style={sibling distance=42mm, level distance=2mm, line width=0.5pt, edge from parent/.style={->, >=stealth, draw}},
  level 2/.style={sibling distance=21mm, level distance=4mm, line width=0.5pt, edge from parent/.style={->, >=stealth, draw}},
  level 3/.style={sibling distance=10.5mm, level distance=6mm, line width=0.5pt, edge from parent/.style={->, >=stealth, draw}},
  level 4/.style={sibling distance=5.25mm, level distance=4mm, line width=0.5pt, edge from parent/.style={->, >=stealth, draw}},
  level 5/.style={sibling distance=1.7mm, level distance=6mm, line width=0.5pt, edge from parent/.style={->, >=stealth, draw}},
  dot/.style = {circle, minimum size=8pt, thin,
              inner sep=1pt, outer sep=0pt},
  invis/.style = {circle, minimum size=3pt, thin,
              inner sep=0.12pt, outer sep=0pt},
  every node/.style={sloped,auto=false}
  ]
\small
\node[celestialblue, draw, fill, invis] {}
    child {node[celestialblue, dot, draw] {c}
      child {node[carminepink, dot, draw] {c}
        child {node[celestialblue, dot, draw] {c}
          child {node[carminepink, dot, draw] {c} 
            child {node[lightgray, invis] {} [lightgray] edge from parent node[above] {}}
            child {node[celestialblue, dot, draw, minimum size=4pt] {\little d} [celestialblue, line width=1pt] edge from parent node[above] {\little $\mathbf{1}$}}
          [carminepink] edge from parent node[above] {}}
          child {node[carminepink, dot, draw] {d} 
            child {node[celestialblue, dot, draw, minimum size=4pt] {\little c} [celestialblue, line width=0.1pt] edge from parent node[above] {\little $0.1$}}
            child {node[celestialblue, dot, draw, minimum size=4pt] {\little d} [celestialblue, line width=0.9pt] edge from parent node[above] {\little $\mathbf{0.9}$}}
          [carminepink] edge from parent node[above] {}} [celestialblue] edge from parent node[above] {$\mathbf{1}$}}
        child {node[lightgray, invis] {}
          child {node[lightgray, invis] {}
            child {node[lightgray, invis] {} [lightgray] edge from parent node[above] {}}
            child {node[lightgray, invis] {} [lightgray] edge from parent node[above] {}}
          [lightgray] edge from parent node[above] {}}
          child {node[lightgray, invis] {} 
            child {node[lightgray, invis] {} [lightgray] edge from parent node[above] {}}
            child {node[lightgray, invis] {} [lightgray] edge from parent node[above] {}}
          [lightgray] edge from parent node[above] {}} [lightgray] edge from parent node[above] {}} [carminepink] edge from parent node[above] {}}
      child {node[carminepink, dot, draw] {d}
        child {node[lightgray, invis] {}
          child {node[lightgray, invis] {} 
            child {node[lightgray, invis] {} [lightgray] edge from parent node[above] {}}
            child {node[lightgray, invis] {} [lightgray] edge from parent node[above] {}}
          [lightgray] edge from parent node[above] {}}
          child {node[lightgray, invis] {} 
            child {node[lightgray, invis] {} [lightgray] edge from parent node[above] {}}
            child {node[lightgray, invis] {} [lightgray] edge from parent node[above] {}}
          [lightgray] edge from parent node[above] {}} [lightgray] edge from parent node[above] {}}
        child {node[celestialblue, dot, draw] {d}
          child {node[carminepink, dot, draw] {c} 
            child {node[lightgray, invis] {} [lightgray] edge from parent node[above] {}}
            child {node[celestialblue, dot, draw, minimum size=4pt] {\little d} [celestialblue, line width=1pt] edge from parent node[above] {\little $\mathbf{1}$}}
          [carminepink] edge from parent node[above] {}}
          child {node[carminepink, dot, draw] {d} 
            child {node[lightgray, invis] {} [lightgray] edge from parent node[above] {}}
            child {node[celestialblue, dot, draw, minimum size=4pt] {\little d} [celestialblue, line width=1pt] edge from parent node[above] {\little $\mathbf{1}$}}
          [carminepink] edge from parent node[above] {}} [celestialblue, line width=1pt] edge from parent node[above] {$\mathbf{1}$}} [carminepink] edge from parent node[above] {}} [celestialblue, line width=1pt] edge from parent node[above] {$\mathbf{1}$}}
    child {node[lightgray, invis] {}
      child {node[lightgray, invis] {}
        child {node[lightgray, invis] {}
          child {node[lightgray, invis] {} 
            child {node[lightgray, invis] {} [lightgray] edge from parent node[above] {}}
            child {node[lightgray, invis] {} [lightgray] edge from parent node[above] {}}
          [lightgray] edge from parent node[above] {}}
          child {node[lightgray, invis] {} 
            child {node[lightgray, invis] {} [lightgray] edge from parent node[above] {}}
            child {node[lightgray, invis] {} [lightgray] edge from parent node[above] {}}
          [lightgray] edge from parent node[above] {}} [lightgray] edge from parent node[above] {}}
        child {node[lightgray, invis] {}
          child {node[lightgray, invis] {}
            child {node[lightgray, invis] {} [lightgray] edge from parent node[above] {}}
            child {node[lightgray, invis] {} [lightgray] edge from parent node[above] {}}
          [lightgray] edge from parent node[above] {}}
          child {node[lightgray, invis] {} 
            child {node[lightgray, invis] {} [lightgray] edge from parent node[above] {}}
            child {node[lightgray, invis] {} [lightgray] edge from parent node[above] {}}
          [lightgray] edge from parent node[above] {}} [lightgray] edge from parent node[above] {}} [lightgray] edge from parent node[above] {}}
      child {node[lightgray, invis] {}
        child {node[lightgray, invis] {}
          child {node[lightgray, invis] {} 
            child {node[lightgray, invis] {} [lightgray] edge from parent node[above] {}}
            child {node[lightgray, invis] {} [lightgray] edge from parent node[above] {}}
          [lightgray] edge from parent node[above] {}}
          child {node[lightgray, invis] {} 
            child {node[lightgray, invis] {} [lightgray] edge from parent node[above] {}}
            child {node[lightgray, invis] {} [lightgray] edge from parent node[above] {}}
          [lightgray] edge from parent node[above] {}} [lightgray] edge from parent node[above] {}}
        child {node[lightgray, invis] {}
          child {node[lightgray, invis] {} 
            child {node[lightgray, invis] {} [lightgray] edge from parent node[above] {}}
            child {node[lightgray, invis] {} [lightgray] edge from parent node[above] {}}
          [lightgray] edge from parent node[above] {}}
          child {node[lightgray, invis] {} 
            child {node[lightgray, invis] {} [lightgray] edge from parent node[above] {}}
            child {node[lightgray, invis] {} [lightgray] edge from parent node[above] {}}
          [lightgray] edge from parent node[above] {}} [lightgray] edge from parent node[above] {}} [lightgray] edge from parent node[above] {}} [lightgray] edge from parent node[above] {}};
\end{tikzpicture}
\caption{PBR.}
\end{subfigure}
\begin{subfigure}[t]{0.49\linewidth}
\centering
\begin{tikzpicture}[
  scale=2.25,
  level 1/.style={sibling distance=42mm, level distance=2mm, line width=0.5pt, edge from parent/.style={->, >=stealth, draw}},
  level 2/.style={sibling distance=21mm, level distance=4mm, line width=0.5pt, edge from parent/.style={->, >=stealth, draw}},
  level 3/.style={sibling distance=10.5mm, level distance=6mm, line width=0.5pt, edge from parent/.style={->, >=stealth, draw}},
  level 4/.style={sibling distance=5.25mm, level distance=4mm, line width=0.5pt, edge from parent/.style={->, >=stealth, draw}},
  level 5/.style={sibling distance=1.7mm, level distance=6mm, line width=0.5pt, edge from parent/.style={->, >=stealth, draw}},
  dot/.style = {circle, minimum size=8pt, thin,
              inner sep=1pt, outer sep=0pt},
  invis/.style = {circle, minimum size=3pt, thin,
              inner sep=0.12pt, outer sep=0pt},
  every node/.style={sloped,auto=false}
  ]

\small
\node[celestialblue, invis, draw, fill] {}
    child {node[lightgray, invis] {}
      child {node[lightgray, invis] {}
        child {node[lightgray, invis] {}
          child {node[lightgray, invis] {} 
            child {node[lightgray, invis] {} [lightgray] edge from parent node[above] {}}
            child {node[lightgray, invis] {} [lightgray] edge from parent node[above] {}}
          [lightgray] edge from parent node[above] {}}
          child {node[lightgray, invis] {} 
            child {node[lightgray, invis] {} [lightgray] edge from parent node[above] {}}
            child {node[lightgray, invis] {} [lightgray] edge from parent node[above] {}}
          [lightgray] edge from parent node[above] {}} [lightgray] edge from parent node[above] {}}
        child {node[lightgray, invis] {}
          child {node[lightgray, invis] {}
            child {node[lightgray, invis] {} [lightgray] edge from parent node[above] {}}
            child {node[lightgray, invis] {} [lightgray] edge from parent node[above] {}}
          [lightgray] edge from parent node[above] {}}
          child {node[lightgray, invis] {} 
            child {node[lightgray, invis] {} [lightgray] edge from parent node[above] {}}
            child {node[lightgray, invis] {} [lightgray] edge from parent node[above] {}}
          [lightgray] edge from parent node[above] {}} [lightgray] edge from parent node[above] {}} [lightgray] edge from parent node[above] {}}
      child {node[lightgray, invis] {}
        child {node[lightgray, invis] {}
          child {node[lightgray, invis] {} 
            child {node[lightgray, invis] {} [lightgray] edge from parent node[above] {}}
            child {node[lightgray, invis] {} [lightgray] edge from parent node[above] {}}
          [lightgray] edge from parent node[above] {}}
          child {node[lightgray, invis] {} 
            child {node[lightgray, invis] {} [lightgray] edge from parent node[above] {}}
            child {node[lightgray, invis] {} [lightgray] edge from parent node[above] {}}
          [lightgray] edge from parent node[above] {}} [lightgray] edge from parent node[above] {}}
        child {node[lightgray, invis] {}
          child {node[lightgray, invis] {} 
            child {node[lightgray, invis] {} [lightgray] edge from parent node[above] {}}
            child {node[lightgray, invis] {} [lightgray] edge from parent node[above] {}}
          [lightgray] edge from parent node[above] {}}
          child {node[lightgray, invis] {} 
            child {node[lightgray, invis] {} [lightgray] edge from parent node[above] {}}
            child {node[lightgray, invis] {} [lightgray] edge from parent node[above] {}}
          [lightgray] edge from parent node[above] {}} [lightgray] edge from parent node[above] {}} [lightgray] edge from parent node[above] {}} [lightgray] edge from parent node[above] {}}
    child {node[celestialblue, dot, draw] {d}
      child {node[carminepink, dot, draw] {c}
        child {node[celestialblue, dot, draw] {c}
          child {node[carminepink, dot, draw] {c} 
            child {node[celestialblue, dot, draw, minimum size=4pt] {\little c} [celestialblue] edge from parent node[above] {\little $0.5$}}
            child {node[celestialblue, dot, draw, minimum size=4pt] {\little d} [celestialblue] edge from parent node[above] {\little $0.5$}}
          [carminepink] edge from parent node[above] {}}
          child {node[carminepink, dot, draw] {d} 
            child {node[celestialblue, dot, draw, minimum size=4pt] {\little c} [celestialblue] edge from parent node[above] {\little $0.5$}}
            child {node[celestialblue, dot, draw, minimum size=4pt] {\little d} [celestialblue] edge from parent node[above] {\little $0.5$}}
          [carminepink] edge from parent node[above] {}} [celestialblue] edge from parent node[above] {$0.5$}}
        child {node[celestialblue, dot, draw] {d}
          child {node[carminepink, dot, draw] {c}
            child {node[celestialblue, dot, draw, minimum size=4pt] {\little c} [celestialblue] edge from parent node[above] {\little $0.5$}}
            child {node[celestialblue, dot, draw, minimum size=4pt] {\little d} [celestialblue] edge from parent node[above] {\little $0.5$}}
          [carminepink] edge from parent node[above] {}}
          child {node[carminepink, dot, draw] {d} 
            child {node[celestialblue, dot, draw, minimum size=4pt] {\little c} [celestialblue] edge from parent node[above] {\little $0.5$}}
            child {node[celestialblue, dot, draw, minimum size=4pt] {\little d} [celestialblue] edge from parent node[above] {\little $0.5$}}
          [carminepink] edge from parent node[above] {}} [celestialblue] edge from parent node[above] {$0.5$}} [carminepink] edge from parent node[above] {}}
      child {node[carminepink, dot, draw] {d}
        child {node[lightgray, invis] {}
          child {node[lightgray, invis] {} 
            child {node[lightgray, invis] {} [lightgray] edge from parent node[above] {}}
            child {node[lightgray, invis] {} [lightgray] edge from parent node[above] {}} [lightgray] edge from parent node[above] {}}
          child {node[lightgray, invis] {} 
            child {node[lightgray, invis] {} [lightgray] edge from parent node[above] {}}
            child {node[lightgray, invis] {} [lightgray] edge from parent node[above] {}}
          [lightgray] edge from parent node[above] {}} [lightgray] edge from parent node[above] {}}
        child {node[celestialblue, dot, draw] {d}
          child {node[carminepink, dot, draw] {c} 
            child {node[celestialblue, dot, draw, minimum size=4pt] {\little c} [celestialblue] edge from parent node[above] {\little $0.5$}}
            child {node[celestialblue, dot, draw, minimum size=4pt] {\little d} [celestialblue] edge from parent node[above] {\little $0.5$}}
          [carminepink] edge from parent node[above] {}}
          child {node[carminepink, dot, draw] {d} 
            child {node[lightgray, invis] {} [lightgray] edge from parent node[above] {}}
            child {node[celestialblue, dot, draw, minimum size=4pt] {\little d} [celestialblue, line width=1pt] edge from parent node[above] {\little $\mathbf{1}$}}
          [carminepink] edge from parent node[above] {}} [celestialblue, line width=1pt] edge from parent node[above] {$\mathbf{1}$}} [carminepink] edge from parent node[above] {}} [celestialblue, line width=1pt] edge from parent node[above] {$\mathbf{1}$}};
\end{tikzpicture}
\caption{\label{subfig:prisoners.mu}
MU.}
\end{subfigure}
\begin{subfigure}[t]{0.49\linewidth}
\centering
\begin{tikzpicture}[
  scale=2.25,
  level 1/.style={sibling distance=42mm, level distance=2mm, line width=0.5pt, edge from parent/.style={->, >=stealth, draw}},
  level 2/.style={sibling distance=21mm, level distance=4mm, line width=0.5pt, edge from parent/.style={->, >=stealth, draw}},
  level 3/.style={sibling distance=10.5mm, level distance=6mm, line width=0.5pt, edge from parent/.style={->, >=stealth, draw}},
  level 4/.style={sibling distance=5.25mm, level distance=4mm, line width=0.5pt, edge from parent/.style={->, >=stealth, draw}},
  level 5/.style={sibling distance=1.7mm, level distance=6mm, line width=0.5pt, edge from parent/.style={->, >=stealth, draw}},
  dot/.style = {circle, minimum size=8pt, thin,
              inner sep=1pt, outer sep=0pt},
  invis/.style = {circle, minimum size=3pt, thin,
              inner sep=0.12pt, outer sep=0pt},
  every node/.style={sloped,auto=false}
  ]
\small
\node[celestialblue, invis, draw, fill] {}
    child {node[celestialblue, dot, draw] {c}
      child {node[carminepink, dot, draw] {c}
        child {node[celestialblue, dot, draw] {c}
          child {node[carminepink, dot, draw] {c} 
            child {node[celestialblue, dot, draw, minimum size=4pt] {\little c} [celestialblue, line width=0.7pt] edge from parent node[above] {\little $\mathbf{0.7}$}}
            child {node[celestialblue, dot, draw, minimum size=4pt] {\little d} [celestialblue, line width=0.3pt] edge from parent node[above] {\little $0.3$}}
          [carminepink] edge from parent node[above] {}}
          child {node[carminepink, dot, draw] {d} 
            child {node[celestialblue, dot, draw, minimum size=4pt] {\little c} [celestialblue, line width=0.6pt] edge from parent node[above] {\little $\mathbf{0.6}$}}
            child {node[celestialblue, dot, draw, minimum size=4pt] {\little d} [celestialblue, line width=0.4pt] edge from parent node[above] {\little $0.4$}}
          [carminepink] edge from parent node[above] {}} [celestialblue, line width=0.8pt] edge from parent node[above] {$\mathbf{0.8}$}}
        child {node[celestialblue, dot, draw] {d}
          child {node[carminepink, dot, draw] {c}
            child {node[celestialblue, dot, draw, minimum size=4pt] {\little c} [celestialblue, line width=0.6pt] edge from parent node[above] {\little $\mathbf{0.6}$}}
            child {node[celestialblue, dot, draw, minimum size=4pt] {\little d} [celestialblue, line width=0.4pt] edge from parent node[above] {\little $0.4$}}
          [carminepink] edge from parent node[above] {}}
          child {node[carminepink, dot, draw] {d} 
            child {node[celestialblue, dot, draw, minimum size=4pt] {\little c} [celestialblue] edge from parent node[above] {\little $0.5$}}
            child {node[celestialblue, dot, draw, minimum size=4pt] {\little d} [celestialblue] edge from parent node[above] {\little $0.5$}}
          [carminepink] edge from parent node[above] {}} [celestialblue, line width=0.2pt] edge from parent node[above] {$0.2$}} [carminepink] edge from parent node[above] {}}
      child {node[carminepink, dot, draw] {d}
        child {node[lightgray, invis] {}
          child {node[lightgray, invis] {} 
            child {node[lightgray, invis] {} [lightgray] edge from parent node[above] {}}
            child {node[lightgray, invis] {} [lightgray] edge from parent node[above] {}}
          [lightgray] edge from parent node[above] {}}
          child {node[lightgray, invis] {} 
            child {node[lightgray, invis] {} [lightgray] edge from parent node[above] {}}
            child {node[lightgray, invis] {} [lightgray] edge from parent node[above] {}}
          [lightgray] edge from parent node[above] {}} [lightgray] edge from parent node[above] {}}
        child {node[celestialblue, dot, draw] {d}
          child {node[carminepink, dot, draw] {c} 
            child {node[lightgray, invis] {} [lightgray] edge from parent node[above] {}}
            child {node[celestialblue, dot, draw, minimum size=4pt] {\little d} [celestialblue, line width=1pt] edge from parent node[above] {\little $\mathbf{1}$}}
          [carminepink] edge from parent node[above] {}}
          child {node[carminepink, dot, draw] {d} 
            child {node[lightgray, invis] {} [lightgray] edge from parent node[above] {}}
            child {node[celestialblue, dot, draw, minimum size=4pt] {\little d} [celestialblue, line width=1pt] edge from parent node[above] {\little $\mathbf{1}$}}
          [carminepink] edge from parent node[above] {}} [celestialblue, line width=1pt] edge from parent node[above] {$\mathbf{1}$}} [carminepink] edge from parent node[above] {}} [celestialblue, line width=0.7pt] edge from parent node[above] {$\mathbf{0.7}$}}
              child {node[celestialblue, dot, draw] {d}
      child {node[carminepink, dot, draw] {c}
        child {node[celestialblue, dot, draw] {c}
          child {node[carminepink, dot, draw] {c} 
            child {node[celestialblue, dot, draw, minimum size=4pt] {\little c} [celestialblue, line width=0.6pt] edge from parent node[above] {\little $\mathbf{0.6}$}}
            child {node[celestialblue, dot, draw, minimum size=4pt] {\little d} [celestialblue, line width=0.4pt] edge from parent node[above] {\little $0.4$}}
          [carminepink] edge from parent node[above] {}}
          child {node[carminepink, dot, draw] {d} 
            child {node[celestialblue, dot, draw, minimum size=4pt] {\little c} [celestialblue, line width=0.7pt] edge from parent node[above] {\little $\mathbf{0.7}$}}
            child {node[celestialblue, dot, draw, minimum size=4pt] {\little d} [celestialblue, line width=0.3pt] edge from parent node[above] {\little $0.3$}}
          [carminepink] edge from parent node[above] {}} [celestialblue, line width=0.8pt] edge from parent node[above] {$\mathbf{0.8}$}}
        child {node[celestialblue, dot, draw] {d}
          child {node[carminepink, dot, draw] {c}
            child {node[celestialblue, dot, draw, minimum size=4pt] {\little c} [celestialblue] edge from parent node[above] {\little $0.5$}}
            child {node[celestialblue, dot, draw, minimum size=4pt] {\little d} [celestialblue] edge from parent node[above] {\little $0.5$}}
          [carminepink] edge from parent node[above] {}}
          child {node[carminepink, dot, draw] {d} 
            child {node[celestialblue, dot, draw, minimum size=4pt] {\little c} [celestialblue, line width=0.6pt] edge from parent node[above] {\little $\mathbf{0.6}$}}
            child {node[celestialblue, dot, draw, minimum size=4pt] {\little d} [celestialblue, line width=0.4pt] edge from parent node[above] {\little $0.4$}}
          [carminepink] edge from parent node[above] {}} [celestialblue, line width=0.2pt] edge from parent node[above] {$0.2$}} [carminepink] edge from parent node[above] {}}
      child {node[carminepink, dot, draw] {d}
        child {node[lightgray, invis] {}
          child {node[lightgray, invis] {} 
            child {node[lightgray, invis] {} [lightgray] edge from parent node[above] {}}
            child {node[lightgray, invis] {} [lightgray] edge from parent node[above] {}}
          [lightgray] edge from parent node[above] {}}
          child {node[lightgray, invis] {}
            child {node[lightgray, invis] {} [lightgray] edge from parent node[above] {}}
            child {node[lightgray, invis] {} [lightgray] edge from parent node[above] {}}
          [lightgray] edge from parent node[above] {}} [lightgray] edge from parent node[above] {}}
        child {node[celestialblue, dot, draw] {d}
          child {node[carminepink, dot, draw] {c} 
            child {node[lightgray, invis] {} [lightgray] edge from parent node[above] {}}
            child {node[celestialblue, dot, draw, minimum size=4pt] {\little d} [celestialblue, line width=1pt] edge from parent node[above] {\little $\mathbf{1}$}}
          [carminepink] edge from parent node[above] {}}
          child {node[carminepink, dot, draw] {d} 
            child {node[lightgray, invis] {} [lightgray] edge from parent node[above] {}}
            child {node[celestialblue, dot, draw, minimum size=4pt] {\little d} [celestialblue, line width=1pt] edge from parent node[above] {\little $\mathbf{1}$}}
          [carminepink] edge from parent node[above] {}} [celestialblue, line width=1pt] edge from parent node[above] {$\mathbf{1}$}} [carminepink] edge from parent node[above] {}} [celestialblue, line width=0.3pt] edge from parent node[above] {$0.3$}};
\end{tikzpicture}
\caption{MR.}
\end{subfigure}
\caption{\label{fig:prisoners.binary_trees}
Obtained Policies. Opponent actions are in red.}
\end{figure}


    \vspace{-5mm}

    \heading{\Large Collaborative Cooking. (Partial observability, LSTM policies)}
    
    \vspace{-1mm}
    \textbf{Training/Test partners}: c.f. Figure~1.

       \begin{figure}
    \centering
    \begin{subfigure}[t]{0.49\linewidth}
            \centering
            \begin{subfigure}[b]{0.4\linewidth}
            \includegraphics[width=\linewidth]{data/cooking/layout_circuit.png}
            \vspace{0.5mm}
            \end{subfigure}
            \includegraphics[width=0.45\linewidth]{data/cooking/circuit/circuit_uniform_utility.png}
            \caption{Circuit layout.}
    \end{subfigure}
    \begin{subfigure}[t]{0.49\linewidth}
            \centering
            \begin{subfigure}[b]{0.4\linewidth}
            \includegraphics[width=\linewidth]{data/cooking/layout_cramped.png}
            \vspace{0.5mm}
            \end{subfigure}
            \includegraphics[width=0.45\linewidth]{data/cooking/cramped/cramped_worst_case_utility.png}
            \caption{Circuit layout.}
    \end{subfigure}
    \vspace{-3mm}
    \caption{Average utility learning curve over the \bfemph{training} scenarios.}
    \end{figure}

    \vspace{-6mm}
    
    \begin{table}[t]
    %\footnotesize
    \setlength{\tabcolsep}{16pt}
    \centering
    \caption{Scores aggregated over the two kitchen layouts (3 runs).}
    \vspace{-4mm}
    \label{tab:cooking.train}
    \begin{tabular}{l||ccc||ccc||}%\toprule
    & \multicolumn{3}{c||}{$\scenarioset(\poptrain)$} & \multicolumn{3}{|c||}{$\scenarioset(\poptest)$} \\
    \cmidrule(lr){2-4}\cmidrule(lr){5-7}
               & $\perf$ & $\wcu$ & $\wcr$ & $\perf$ & $\wcu$ & $\wcr$ \\ \midrule
        MU & $\mathbf{266.9 {\scriptstyle \pm 4.3}}$ & $\mathbf{225.3 {\scriptstyle\pm 11.5}}$ & $266.0 {\scriptstyle\pm 7.9}$ & $\mathbf{195.7 {\scriptstyle\pm 6.2}}$ & $\mathbf{66.0 {\scriptstyle\pm 6.8}}$ & $266.4 {\scriptstyle\pm 10.1}$ \\
        MR & $232.0 {\scriptstyle\pm 18.6}$ & $144.3 {\scriptstyle\pm 14.4}$ & $\mathbf{230.7 {\scriptstyle\pm 28.2}}$ & $172.2 {\scriptstyle\pm 15.4}$ & $65.1 {\scriptstyle\pm 16.0}$ & $\mathbf{248.2 {\scriptstyle\pm 28.4}}$ \\ \hdashline
        PBR & $209.7 {\scriptstyle\pm 23.9}$ & $96.8 {\scriptstyle\pm 13.4}$ & $357.6 {\scriptstyle\pm 16.1}$ & $151.4 {\scriptstyle\pm 19.5}$ & $33.6 {\scriptstyle\pm 5.9}$ & $327.1 {\scriptstyle\pm 14.0}$ \\
        FP & $129.9 {\scriptstyle\pm 13.9}$ & $0.2 {\scriptstyle\pm 0.1}$ & $483.5 {\scriptstyle\pm 16.1}$ & $152.2 {\scriptstyle\pm 16.8}$ & $16.7 {\scriptstyle\pm 11.7}$ & $369.7 {\scriptstyle\pm 19.7}$ \\
        SP & $124.8 {\scriptstyle\pm 26.4}$ & $15.7 {\scriptstyle\pm 10.5}$ & $460.8 {\scriptstyle\pm 21.7}$ & $117.4 {\scriptstyle\pm 12.5}$ & $6.7 {\scriptstyle\pm 3.5}$ & $367.5 {\scriptstyle\pm 11.6}$ \\
        Random & $42.8 {\scriptstyle\pm 0.0}$ & $0.0 {\scriptstyle\pm 0.0}$ & $505.4 {\scriptstyle\pm 0.0}$ & $32.2 {\scriptstyle\pm 0.0}$ & $0.0 {\scriptstyle\pm 0.0}$ & $445.0 {\scriptstyle\pm 0.0}$ \\ 
    %\bottomrule
    \end{tabular}
    \end{table}
    
\end{block}

\end{column}

\separatorcolumn
\end{columns}
\end{frame}


\end{document}
