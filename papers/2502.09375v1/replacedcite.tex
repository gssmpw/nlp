\section{Related Work}
% \subsection{Live-Streaming Recommendation}
\textbf{Live-streaming recommendation}____ has been widely explored in many online platforms, such as Kuaishou and TikTok. 
Compared to short-video recommendation____, live-streaming recommendation is more challenging due to its real-time and interactive nature.
One representative early study LiveRec____ proposes a self-attentive model to personalize item ranking by exploiting historical interactions and current availability. ADARM____ models matching patterns by capturing changes in user and author preferences. 
Some studies also consider the live-streaming multi-modal information. Specifically, ContentCTR____ and KuaiHL____ adopt the multi-modal transformer, which combines real-time visual, acoustic and textual information to identify the highlight moment of authors. Besides, some studies further utilize GNN to capture the dynamic relationship between users and authors.  
For instance, MMBee____ constructs user-to-author and author-to-author graphs to combine adjacency and multi-hop relationships to guide interest expansion. 
However, all these methods above only model user preference within the live-streaming domains, leading to the serious data-sparsity issue. On the contrary, FARM considers to transfer user preference across the short-video and live-streaming domains, thus alleviating the data-sparsity issue.






% \subsection{Cross-Domain Recommendation}
\textbf{Cross-domain recommendation} (CDR)____ aims at leveraging user interactions in the source domain to improve the recommendation performance in the target domain, which is a valid direction to solve the cold-start and the data-sparsity issues in industry. 
As our FARM aim to solve the latter issue, we only focus on CDR methods that alleviate the data-sparsity issue. 
Specifically, some methods focus on capturing the domain-shared information across domains. PEPNet____ designs shared and separate parameter networks to capture the domain-shared information and achieve the balance of personalization tasks in each domain. 
CDAnet____ develops a translation network to explicitly transfer domain-shared knowledge across domains.
The others focus on combating negative transfer. For instance, DDCDR____ leverages the disentangling distillation of cross-domain teacher models to achieve multi-domain knowledge sharing of student models.  
ADSNet____ introduces a gain evaluation strategy to calculate information gain, which is used for the rejection of noisy samples.
More recently, some studies apply CDR in live-streaming recommendation. Moment\&Cross____ follows the search-based____ interest modeling idea to extract user preference in both the short-video and live-streaming domains. LCN____ achieves noise filtering through a lifelong attention pyramid containing three cascading attention levels. Compared to them, FARM has important design differences that can perceive user's sparse yet valuable behaviors.





% \subsection{Fourier Transform in Recommendation}
\textbf{Fourier transform}____ is a mathematical tool that converts time-domain signals into frequency-domain signals, and has recently been widely employed in recommendation systems____ due to its advantages in capturing periodic patterns and reducing dimensionality.  
These methods have two main concerns, one is the periodic high-order modeling of user sequences. Specifically, FARIMA____ develops the Fourier-assisted Auto-Regressive approach to analyze the periodicity and randomness of user interactions. KDA____ models the temporal drift of relations between users and items via Fourier transform with learnable frequency domain embedding. BSARec____ leverages the Fourier transform to integrate low and high frequency information to mitigate oversmoothing and address the limitations of self-attention. Some studies also focus on denoising by filtering in frequency domain. For instance, FMLP-Rec____ applies fast Fourier transform to sequence representation and achieves noise attenuation through learnable filters. END4Rec____ obtains the soft noise signal and the filtered signal through chunked diagonal mechanism and token sparsity in the frequency domain. Different from prior methods, we propose a frequency-aware model in this paper, which innovatively adopts the Discrete Fourier Transform (DFT) theory into cross-domain live-streaming recommendation to perceive user's sparse yet valuable behaviors in both domains.