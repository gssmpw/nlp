\documentclass[runningheads]{llncs}

\usepackage[T1]{fontenc}
\usepackage{graphicx}
\usepackage{xcolor}
\usepackage{color, colortbl}


\usepackage{amssymb}
\usepackage{amsmath}


\usepackage{cleveref}

\usepackage{tikz}
\usepackage{pgfplots}
\usepackage{subcaption}
\usetikzlibrary{shapes.geometric, arrows,calc,fit,backgrounds}

\usepackage{wrapfig,lipsum,booktabs}

\usepackage{xspace}

\usepackage{pifont}
\newcommand{\cmark}{\ding{51}}
\newcommand{\xmark}{\ding{55}}
\newcommand{\bmark}{$\bullet$}

\usepackage{fontawesome5}

\usepackage{syntax}
\usepackage{listings}


%
\setlength\unitlength{1mm}
\newcommand{\twodots}{\mathinner {\ldotp \ldotp}}
% bb font symbols
\newcommand{\Rho}{\mathrm{P}}
\newcommand{\Tau}{\mathrm{T}}

\newfont{\bbb}{msbm10 scaled 700}
\newcommand{\CCC}{\mbox{\bbb C}}

\newfont{\bb}{msbm10 scaled 1100}
\newcommand{\CC}{\mbox{\bb C}}
\newcommand{\PP}{\mbox{\bb P}}
\newcommand{\RR}{\mbox{\bb R}}
\newcommand{\QQ}{\mbox{\bb Q}}
\newcommand{\ZZ}{\mbox{\bb Z}}
\newcommand{\FF}{\mbox{\bb F}}
\newcommand{\GG}{\mbox{\bb G}}
\newcommand{\EE}{\mbox{\bb E}}
\newcommand{\NN}{\mbox{\bb N}}
\newcommand{\KK}{\mbox{\bb K}}
\newcommand{\HH}{\mbox{\bb H}}
\newcommand{\SSS}{\mbox{\bb S}}
\newcommand{\UU}{\mbox{\bb U}}
\newcommand{\VV}{\mbox{\bb V}}


\newcommand{\yy}{\mathbbm{y}}
\newcommand{\xx}{\mathbbm{x}}
\newcommand{\zz}{\mathbbm{z}}
\newcommand{\sss}{\mathbbm{s}}
\newcommand{\rr}{\mathbbm{r}}
\newcommand{\pp}{\mathbbm{p}}
\newcommand{\qq}{\mathbbm{q}}
\newcommand{\ww}{\mathbbm{w}}
\newcommand{\hh}{\mathbbm{h}}
\newcommand{\vvv}{\mathbbm{v}}

% Vectors

\newcommand{\av}{{\bf a}}
\newcommand{\bv}{{\bf b}}
\newcommand{\cv}{{\bf c}}
\newcommand{\dv}{{\bf d}}
\newcommand{\ev}{{\bf e}}
\newcommand{\fv}{{\bf f}}
\newcommand{\gv}{{\bf g}}
\newcommand{\hv}{{\bf h}}
\newcommand{\iv}{{\bf i}}
\newcommand{\jv}{{\bf j}}
\newcommand{\kv}{{\bf k}}
\newcommand{\lv}{{\bf l}}
\newcommand{\mv}{{\bf m}}
\newcommand{\nv}{{\bf n}}
\newcommand{\ov}{{\bf o}}
\newcommand{\pv}{{\bf p}}
\newcommand{\qv}{{\bf q}}
\newcommand{\rv}{{\bf r}}
\newcommand{\sv}{{\bf s}}
\newcommand{\tv}{{\bf t}}
\newcommand{\uv}{{\bf u}}
\newcommand{\wv}{{\bf w}}
\newcommand{\vv}{{\bf v}}
\newcommand{\xv}{{\bf x}}
\newcommand{\yv}{{\bf y}}
\newcommand{\zv}{{\bf z}}
\newcommand{\zerov}{{\bf 0}}
\newcommand{\onev}{{\bf 1}}

% Matrices

\newcommand{\Am}{{\bf A}}
\newcommand{\Bm}{{\bf B}}
\newcommand{\Cm}{{\bf C}}
\newcommand{\Dm}{{\bf D}}
\newcommand{\Em}{{\bf E}}
\newcommand{\Fm}{{\bf F}}
\newcommand{\Gm}{{\bf G}}
\newcommand{\Hm}{{\bf H}}
\newcommand{\Id}{{\bf I}}
\newcommand{\Jm}{{\bf J}}
\newcommand{\Km}{{\bf K}}
\newcommand{\Lm}{{\bf L}}
\newcommand{\Mm}{{\bf M}}
\newcommand{\Nm}{{\bf N}}
\newcommand{\Om}{{\bf O}}
\newcommand{\Pm}{{\bf P}}
\newcommand{\Qm}{{\bf Q}}
\newcommand{\Rm}{{\bf R}}
\newcommand{\Sm}{{\bf S}}
\newcommand{\Tm}{{\bf T}}
\newcommand{\Um}{{\bf U}}
\newcommand{\Wm}{{\bf W}}
\newcommand{\Vm}{{\bf V}}
\newcommand{\Xm}{{\bf X}}
\newcommand{\Ym}{{\bf Y}}
\newcommand{\Zm}{{\bf Z}}

% Calligraphic

\newcommand{\Ac}{{\cal A}}
\newcommand{\Bc}{{\cal B}}
\newcommand{\Cc}{{\cal C}}
\newcommand{\Dc}{{\cal D}}
\newcommand{\Ec}{{\cal E}}
\newcommand{\Fc}{{\cal F}}
\newcommand{\Gc}{{\cal G}}
\newcommand{\Hc}{{\cal H}}
\newcommand{\Ic}{{\cal I}}
\newcommand{\Jc}{{\cal J}}
\newcommand{\Kc}{{\cal K}}
\newcommand{\Lc}{{\cal L}}
\newcommand{\Mc}{{\cal M}}
\newcommand{\Nc}{{\cal N}}
\newcommand{\nc}{{\cal n}}
\newcommand{\Oc}{{\cal O}}
\newcommand{\Pc}{{\cal P}}
\newcommand{\Qc}{{\cal Q}}
\newcommand{\Rc}{{\cal R}}
\newcommand{\Sc}{{\cal S}}
\newcommand{\Tc}{{\cal T}}
\newcommand{\Uc}{{\cal U}}
\newcommand{\Wc}{{\cal W}}
\newcommand{\Vc}{{\cal V}}
\newcommand{\Xc}{{\cal X}}
\newcommand{\Yc}{{\cal Y}}
\newcommand{\Zc}{{\cal Z}}

% Bold greek letters

\newcommand{\alphav}{\hbox{\boldmath$\alpha$}}
\newcommand{\betav}{\hbox{\boldmath$\beta$}}
\newcommand{\gammav}{\hbox{\boldmath$\gamma$}}
\newcommand{\deltav}{\hbox{\boldmath$\delta$}}
\newcommand{\etav}{\hbox{\boldmath$\eta$}}
\newcommand{\lambdav}{\hbox{\boldmath$\lambda$}}
\newcommand{\epsilonv}{\hbox{\boldmath$\epsilon$}}
\newcommand{\nuv}{\hbox{\boldmath$\nu$}}
\newcommand{\muv}{\hbox{\boldmath$\mu$}}
\newcommand{\zetav}{\hbox{\boldmath$\zeta$}}
\newcommand{\phiv}{\hbox{\boldmath$\phi$}}
\newcommand{\psiv}{\hbox{\boldmath$\psi$}}
\newcommand{\thetav}{\hbox{\boldmath$\theta$}}
\newcommand{\tauv}{\hbox{\boldmath$\tau$}}
\newcommand{\omegav}{\hbox{\boldmath$\omega$}}
\newcommand{\xiv}{\hbox{\boldmath$\xi$}}
\newcommand{\sigmav}{\hbox{\boldmath$\sigma$}}
\newcommand{\piv}{\hbox{\boldmath$\pi$}}
\newcommand{\rhov}{\hbox{\boldmath$\rho$}}
\newcommand{\upsilonv}{\hbox{\boldmath$\upsilon$}}

\newcommand{\Gammam}{\hbox{\boldmath$\Gamma$}}
\newcommand{\Lambdam}{\hbox{\boldmath$\Lambda$}}
\newcommand{\Deltam}{\hbox{\boldmath$\Delta$}}
\newcommand{\Sigmam}{\hbox{\boldmath$\Sigma$}}
\newcommand{\Phim}{\hbox{\boldmath$\Phi$}}
\newcommand{\Pim}{\hbox{\boldmath$\Pi$}}
\newcommand{\Psim}{\hbox{\boldmath$\Psi$}}
\newcommand{\Thetam}{\hbox{\boldmath$\Theta$}}
\newcommand{\Omegam}{\hbox{\boldmath$\Omega$}}
\newcommand{\Xim}{\hbox{\boldmath$\Xi$}}


% Sans Serif small case

\newcommand{\Gsf}{{\sf G}}

\newcommand{\asf}{{\sf a}}
\newcommand{\bsf}{{\sf b}}
\newcommand{\csf}{{\sf c}}
\newcommand{\dsf}{{\sf d}}
\newcommand{\esf}{{\sf e}}
\newcommand{\fsf}{{\sf f}}
\newcommand{\gsf}{{\sf g}}
\newcommand{\hsf}{{\sf h}}
\newcommand{\isf}{{\sf i}}
\newcommand{\jsf}{{\sf j}}
\newcommand{\ksf}{{\sf k}}
\newcommand{\lsf}{{\sf l}}
\newcommand{\msf}{{\sf m}}
\newcommand{\nsf}{{\sf n}}
\newcommand{\osf}{{\sf o}}
\newcommand{\psf}{{\sf p}}
\newcommand{\qsf}{{\sf q}}
\newcommand{\rsf}{{\sf r}}
\newcommand{\ssf}{{\sf s}}
\newcommand{\tsf}{{\sf t}}
\newcommand{\usf}{{\sf u}}
\newcommand{\wsf}{{\sf w}}
\newcommand{\vsf}{{\sf v}}
\newcommand{\xsf}{{\sf x}}
\newcommand{\ysf}{{\sf y}}
\newcommand{\zsf}{{\sf z}}


% mixed symbols

\newcommand{\sinc}{{\hbox{sinc}}}
\newcommand{\diag}{{\hbox{diag}}}
\renewcommand{\det}{{\hbox{det}}}
\newcommand{\trace}{{\hbox{tr}}}
\newcommand{\sign}{{\hbox{sign}}}
\renewcommand{\arg}{{\hbox{arg}}}
\newcommand{\var}{{\hbox{var}}}
\newcommand{\cov}{{\hbox{cov}}}
\newcommand{\Ei}{{\rm E}_{\rm i}}
\renewcommand{\Re}{{\rm Re}}
\renewcommand{\Im}{{\rm Im}}
\newcommand{\eqdef}{\stackrel{\Delta}{=}}
\newcommand{\defines}{{\,\,\stackrel{\scriptscriptstyle \bigtriangleup}{=}\,\,}}
\newcommand{\<}{\left\langle}
\renewcommand{\>}{\right\rangle}
\newcommand{\herm}{{\sf H}}
\newcommand{\trasp}{{\sf T}}
\newcommand{\transp}{{\sf T}}
\renewcommand{\vec}{{\rm vec}}
\newcommand{\Psf}{{\sf P}}
\newcommand{\SINR}{{\sf SINR}}
\newcommand{\SNR}{{\sf SNR}}
\newcommand{\MMSE}{{\sf MMSE}}
\newcommand{\REF}{{\RED [REF]}}

% Markov chain
\usepackage{stmaryrd} % for \mkv 
\newcommand{\mkv}{-\!\!\!\!\minuso\!\!\!\!-}

% Colors

\newcommand{\RED}{\color[rgb]{1.00,0.10,0.10}}
\newcommand{\BLUE}{\color[rgb]{0,0,0.90}}
\newcommand{\GREEN}{\color[rgb]{0,0.80,0.20}}

%%%%%%%%%%%%%%%%%%%%%%%%%%%%%%%%%%%%%%%%%%
\usepackage{hyperref}
\hypersetup{
    bookmarks=true,         % show bookmarks bar?
    unicode=false,          % non-Latin characters in AcrobatÕs bookmarks
    pdftoolbar=true,        % show AcrobatÕs toolbar?
    pdfmenubar=true,        % show AcrobatÕs menu?
    pdffitwindow=false,     % window fit to page when opened
    pdfstartview={FitH},    % fits the width of the page to the window
%    pdftitle={My title},    % title
%    pdfauthor={Author},     % author
%    pdfsubject={Subject},   % subject of the document
%    pdfcreator={Creator},   % creator of the document
%    pdfproducer={Producer}, % producer of the document
%    pdfkeywords={keyword1} {key2} {key3}, % list of keywords
    pdfnewwindow=true,      % links in new window
    colorlinks=true,       % false: boxed links; true: colored links
    linkcolor=red,          % color of internal links (change box color with linkbordercolor)
    citecolor=green,        % color of links to bibliography
    filecolor=blue,      % color of file links
    urlcolor=blue           % color of external links
}
%%%%%%%%%%%%%%%%%%%%%%%%%%%%%%%%%%%%%%%%%%%




\begin{document}

\title{Issy: A Comprehensive Tool for Specification and Synthesis of Infinite-State Reactive Systems}
\titlerunning{Issy: Specification and Synthesis of Infinite-State Reactive Systems}


\author{
Philippe Heim\orcidID{0000-0002-5433-8133} \and
Rayna Dimitrova\orcidID{0009-0006-2494-8690}
}

\authorrunning{P. Heim \and R. Dimitrova}

\institute{
CISPA Helmholtz Center for Information Security, Saarbr\"ucken, Germany
\email{\{philippe.heim, dimitrova\}@cispa.de}
}



\maketitle


\begin{abstract}
The synthesis of infinite-state reactive systems from temporal logic specifications or infinite-state games has attracted significant attention in recent years, leading to the emergence of novel solving techniques. 
Most approaches are accompanied by an implementation showcasing their viability on an increasingly larger collection of benchmarks.
Those implementations are --often simple-- prototypes. 
Furthermore, differences in specification formalisms and formats make comparisons difficult, and writing specifications is a tedious and error-prone task.

To address this,  we present \issy{},  a tool for specification, realizability, and synthesis of infinite-state reactive systems.  
\issy{} comes with an expressive specification language that allows for combining infinite-state games and temporal formulas,  thus encompassing the current formalisms.
The realizability checking and synthesis methods implemented in \issy build upon recently developed approaches and extend them with newly engineered efficient techniques, offering a portfolio of solving algorithms.
We evaluate \issy on an extensive set of benchmarks,  demonstrating its competitiveness with the state of the art.
Furthermore,  \issy{} provides tooling for a general high-level format designed to make specification easier for users.  It also includes a compiler to a more machine-readable format that other tool developers can easily use, which we hope will lead to a broader adoption and advances in infinite-state reactive synthesis.
\end{abstract}

\section{Introduction}\label{sec:intro}
\section{Introduction}


\begin{figure}[t]
\centering
\includegraphics[width=0.6\columnwidth]{figures/evaluation_desiderata_V5.pdf}
\vspace{-0.5cm}
\caption{\systemName is a platform for conducting realistic evaluations of code LLMs, collecting human preferences of coding models with real users, real tasks, and in realistic environments, aimed at addressing the limitations of existing evaluations.
}
\label{fig:motivation}
\end{figure}

\begin{figure*}[t]
\centering
\includegraphics[width=\textwidth]{figures/system_design_v2.png}
\caption{We introduce \systemName, a VSCode extension to collect human preferences of code directly in a developer's IDE. \systemName enables developers to use code completions from various models. The system comprises a) the interface in the user's IDE which presents paired completions to users (left), b) a sampling strategy that picks model pairs to reduce latency (right, top), and c) a prompting scheme that allows diverse LLMs to perform code completions with high fidelity.
Users can select between the top completion (green box) using \texttt{tab} or the bottom completion (blue box) using \texttt{shift+tab}.}
\label{fig:overview}
\end{figure*}

As model capabilities improve, large language models (LLMs) are increasingly integrated into user environments and workflows.
For example, software developers code with AI in integrated developer environments (IDEs)~\citep{peng2023impact}, doctors rely on notes generated through ambient listening~\citep{oberst2024science}, and lawyers consider case evidence identified by electronic discovery systems~\citep{yang2024beyond}.
Increasing deployment of models in productivity tools demands evaluation that more closely reflects real-world circumstances~\citep{hutchinson2022evaluation, saxon2024benchmarks, kapoor2024ai}.
While newer benchmarks and live platforms incorporate human feedback to capture real-world usage, they almost exclusively focus on evaluating LLMs in chat conversations~\citep{zheng2023judging,dubois2023alpacafarm,chiang2024chatbot, kirk2024the}.
Model evaluation must move beyond chat-based interactions and into specialized user environments.



 

In this work, we focus on evaluating LLM-based coding assistants. 
Despite the popularity of these tools---millions of developers use Github Copilot~\citep{Copilot}---existing
evaluations of the coding capabilities of new models exhibit multiple limitations (Figure~\ref{fig:motivation}, bottom).
Traditional ML benchmarks evaluate LLM capabilities by measuring how well a model can complete static, interview-style coding tasks~\citep{chen2021evaluating,austin2021program,jain2024livecodebench, white2024livebench} and lack \emph{real users}. 
User studies recruit real users to evaluate the effectiveness of LLMs as coding assistants, but are often limited to simple programming tasks as opposed to \emph{real tasks}~\citep{vaithilingam2022expectation,ross2023programmer, mozannar2024realhumaneval}.
Recent efforts to collect human feedback such as Chatbot Arena~\citep{chiang2024chatbot} are still removed from a \emph{realistic environment}, resulting in users and data that deviate from typical software development processes.
We introduce \systemName to address these limitations (Figure~\ref{fig:motivation}, top), and we describe our three main contributions below.


\textbf{We deploy \systemName in-the-wild to collect human preferences on code.} 
\systemName is a Visual Studio Code extension, collecting preferences directly in a developer's IDE within their actual workflow (Figure~\ref{fig:overview}).
\systemName provides developers with code completions, akin to the type of support provided by Github Copilot~\citep{Copilot}. 
Over the past 3 months, \systemName has served over~\completions suggestions from 10 state-of-the-art LLMs, 
gathering \sampleCount~votes from \userCount~users.
To collect user preferences,
\systemName presents a novel interface that shows users paired code completions from two different LLMs, which are determined based on a sampling strategy that aims to 
mitigate latency while preserving coverage across model comparisons.
Additionally, we devise a prompting scheme that allows a diverse set of models to perform code completions with high fidelity.
See Section~\ref{sec:system} and Section~\ref{sec:deployment} for details about system design and deployment respectively.



\textbf{We construct a leaderboard of user preferences and find notable differences from existing static benchmarks and human preference leaderboards.}
In general, we observe that smaller models seem to overperform in static benchmarks compared to our leaderboard, while performance among larger models is mixed (Section~\ref{sec:leaderboard_calculation}).
We attribute these differences to the fact that \systemName is exposed to users and tasks that differ drastically from code evaluations in the past. 
Our data spans 103 programming languages and 24 natural languages as well as a variety of real-world applications and code structures, while static benchmarks tend to focus on a specific programming and natural language and task (e.g. coding competition problems).
Additionally, while all of \systemName interactions contain code contexts and the majority involve infilling tasks, a much smaller fraction of Chatbot Arena's coding tasks contain code context, with infilling tasks appearing even more rarely. 
We analyze our data in depth in Section~\ref{subsec:comparison}.



\textbf{We derive new insights into user preferences of code by analyzing \systemName's diverse and distinct data distribution.}
We compare user preferences across different stratifications of input data (e.g., common versus rare languages) and observe which affect observed preferences most (Section~\ref{sec:analysis}).
For example, while user preferences stay relatively consistent across various programming languages, they differ drastically between different task categories (e.g. frontend/backend versus algorithm design).
We also observe variations in user preference due to different features related to code structure 
(e.g., context length and completion patterns).
We open-source \systemName and release a curated subset of code contexts.
Altogether, our results highlight the necessity of model evaluation in realistic and domain-specific settings.







\section{The Issy Format}\label{sec:format-issy}
The \issy input format has the key advantage that it combines two modes for specification of synthesis problems for infinite-state reactive systems: temporal logic formulas,  and two-player games, both over variables with infinite domains, such as integers or reals.  
The advantages of this mutli-paradigm specification format are two-fold.
First,  it often allows specification designers to specify requirements in a less cumbersome way.  
For example,  constraints that  depend on the system's state, or encode behaviour in different phases,  are usually easier to specify as games.  On the other hand, mission specifications such that under certain assumptions the system must eventually stabilize, or that some tasks should be carried out repeatedly, are often more concisely expressed in temporal logic.  \looseness=-1



\definecolor{codegray}{rgb}{0.5,0.5,0.5}
\definecolor{codepurple}{rgb}{0.58,0,0.82}
\definecolor{backcolour}{rgb}{0.95,0.95,0.92}

\lstdefinestyle{mystyle}{
    backgroundcolor=\color{backcolour},   
    commentstyle=\color{codegray},
    keywordstyle=\color{blue},
    numberstyle=\tiny\color{codepurple},
    basicstyle=\linespread{0.9}\ttfamily\footnotesize,
    breakatwhitespace=false,         
    breaklines=true,                 
    captionpos=b,                    
    keepspaces=true,                 
    numbers=left,                    
    numbersep=5pt,                  
    showspaces=false,                
    showstringspaces=false,
    showtabs=false,                  
    tabsize=1,
    morecomment=[l]{//},
morecomment=[s]{/*}{*/},
basewidth = {.47em}
}

\lstset{style=mystyle,  morekeywords={game,  input, state, formula, def, assert, assume, keep, from,to, with, Safety,loc}}

\begin{lstlisting}[label=lst:example,caption=Example specification in \issy format.,escapeinside={@}{@}]
input real add		  input real rem
state real load1		state real load2 	  state real rem1  		 state real rem2
formula {
 /* Assumption: From some time point on, the environment will always set the input variable add to be less than or equal to zero. */
 assume F G [add <= 0]
 /* Guarantee: From some point on,   load1 and load2 will always be zero. */     
 @\label{lst:lassert}@assert F G ([load1 = 0] && [load2 = 0])
 }
// Macros to make the specification easier to read
@\label{lst:baldef}@def balanced  = [load1 >= load2] && [load1 <= 2 * load2] 
              ||[load2 >= load1] && [load2 <= 2 * load1]
def addtoone  = [load1' = load1 + add] && [load2' = load2] 
              ||[load2' = load2 + add] && [load1' = load1]
def validrem  = [rem >= 0.1] && [rem <= load1 + 2/3 * load2]
def decrease  = [load1' = load1 - rem1'] && [rem1' + rem2' = rem]
              &&[load2' = load2 - 3/2 * rem2'] 
/* Two-player game with locations init, lbal, lrem, done and err,  and safety winning condition for the system, requiring that err is never reached. */
game Safety from init {
  loc init 1	   loc lbal 1	   	loc lrem 1	   loc done 1	  	loc err 0
  from init to done with [load1 < 0] || [load2 < 0]
  from init to lbal with [load1 >= 0] && [load2 >= 0] && keep(load1 load2)
  @\label{lst:lbalrem}@from lbal to lrem with [load1' + load2' = load1 + load2] 
  @\label{lst:lremerr}@from lrem to err  with !balanced 
  from lrem to done with balanced &&(!validrem ||([load1 = 0] && [load2 = 0]))
  @\label{lst:lrembal}@from lrem to lbal with balanced && [add > 0]  && addtoone
  from lrem to lrem with balanced && [add <= 0] && validrem && decrease
  from done to done with true 
  from err  to err  with keep(load1 load2)
 }
\end{lstlisting}

Each of the two modes of specification can potentially offer opportunities for optimization of the synthesis tools processing these specifications. 
In~\cite{HeimD25}, we showed how the translation from \rpltl{} formulas  to  games can benefit from the high-level information present in the formula in order to simplify the  game. 


\begin{figure}[b!]
\begin{center}
\grammarindent2.2cm
\grammarparsep1.1pt
\begin{grammar}

<spec> ::= (<vardecl> | <logicspec> | <gamespec> | <macro>)*
 
 \smallskip 
 
<vardecl> ::= (`input' | `state') <type> <identifier> \quad  <type>    ::= `int' | `bool' | `real'

 \smallskip 

<logicspec> ::=  `formula' `{' <logicstm>*  `}'  \quad <logicstm> ::= (`assert' | `assume') <rpltl>



\smallskip

<gamespec>  ::= `game' <wincond> `from' <identifier> `{' ( <locdef> | <transdef>)* `}' 

<wincond>  ::= `Safety' | `Reachability' | `Buechi'  | `CoBuechi' | `ParityMaxOdd'

<locdef>   ::= `loc' <identifier> [<nat>] [`with' <formula>]

<transdef>  ::= `from'  <identifier> `to' <identifier> `with' <formula>

\vspace{-.4cm}
\end{grammar}


\end{center}
\caption{An excerpt from the \issy  format.  The full description is in~\Cref{sec:format-issy-full}.}
\label{fig:issy-grammar-part}
\end{figure}





Now, we turn to an example that illustrates and motivates the main features of the \issy format.
An excerpt of the format's grammar is given in \Cref{fig:issy-grammar-part}.

\begin{example}\label{ex:simple-example}
Consider a reactive system that has to balance the loads, \texttt{load1} and \texttt{load2}, of two components.
At any point,  the environment can increase the total load,  via the environment-controlled input variable \texttt{add}.  When that happens,  the system has to re-balance the total load by appropriate partitioning. When the load does not increase,  the system has to control the throughput of each component,  state variables \texttt{rem1} and \texttt{rem2} respectively,  in accordance with the components' speeds and the total available throughput,  \texttt{rem} controlled by the environment.
The specification of this system is given in Listing~\ref{lst:example}, and consists of variable declarations,  a formula specification,   macro definitions for better readability, and the second part of the specification given as a two-player game.  


Variable declarations specify whether the variable is \texttt{\textcolor{blue}{input}} controlled by the environment, or is a \texttt{\textcolor{blue}{state}} variable controlled by the system.
The currently supported data types are \texttt{bool, int} and \texttt{real}.
The domains of variables can be further constrained in the \texttt{\textcolor{blue}{game}} specifications by additional constraints.




The \texttt{\textcolor{blue}{formula}} specification is a list \rpltl{} formulas,  prefixed by the keywords \texttt{\textcolor{blue}{assume}} and \texttt{\textcolor{blue}{assert}},  denoting constraints on the environment and system respectively.
They use temporal operators like LTL, but with quantifier-free first-order atoms 
instead of Boolean propositions.
The assumption \texttt{F G [add <= 0]} uses the temporal operators \texttt{F} (eventually) and  \texttt{G} (globally) to state that from some point on, no more load will be added by the environment.
The assert statement in line \ref{lst:lassert} requires the system to ensure, under the above assumption, that both loads eventually stabilize at zero. 
The semantics of a \texttt{\textcolor{blue}{formula}} specification is that the conjunction of the assumptions implies the conjunction of the asserts. 

The possible actions of the system and the requirement to balance \texttt{load1} and \texttt{load2} are described by the 
\texttt{\textcolor{blue}{game}} specification in Listing~\ref{lst:example}. The game has locations \texttt{init}, \texttt{lbal}, \texttt{lrem}, \texttt{done}, \texttt{err} that are local to the game, unlike variables that are global to the whole specification.
The transitions between locations in the game are defined via  quantifier-free formulas over input, state, and next-state variables (such as \texttt{load1'}).  Nondeterminism is under the control of the system.  The \issy format enables the use of macros to formula readability. 
For example,  the transition in line \ref{lst:lremerr} requires the system to transition from location  \texttt{lrem} to the unsafe location \texttt{err} if the condition \texttt{balanced} defined by the macro in line \ref{lst:baldef} is violated. 
The game has a \emph{safety} winning condition,  indicated by the keyword \texttt{\textcolor{blue}{Safety}}, and defined by the natural numbers with which the locations are labelled ($0$ indicates that \texttt{err} is unsafe, while all  labelled $1$  are safe).
\end{example}

A specification can contain multiple \texttt{\textcolor{blue}{formula}} and \texttt{\textcolor{blue}{game}} components,  interpreted conjunctively.  The semantics is a two-player game defined as the product of the games for the individual formulas and all game specifications.  
\issy requires and checks that at most one of these games has a non-safety winning condition. 

The \issy specification in \Cref{ex:simple-example} illustrates the modelling flexibility of the format. Expressing the same requirements purely in \rpltl\ or as an RPG results in a difficult to write and understand specification,  making the specification process error prone.  We believe that \issy  alleviates this problem to some extent, offering modularity and syntactic sugar constructs, and,  most importantly, unifying the temporal logic and game formats for 
infinite-state reactive systems.\looseness=-1

\vspace{-.3cm}
\paragraph{The \issy compiler and the \llissy format.}
The \issy compiler,  part of our synthesis framework,  compiles specifications in  \issy format to a low-level intermediate format called \llissy,  given in \Cref{sec:format-llissy}.
The compiler checks compliance with the syntax and gives informative error messages.
The \llissy format is easier to parse, while retaining the ability to specify both logical formulas and games. 
We envision that the development of tools  for translation from various high-level specification formats to the \llissy format will enable the seamless exchange of benchmarks and experimental comparison between different tools.
\issy also accepts input directly in \issy format, as well as the older formats  \textsf{tslmt} and \textsf{rpg}.




\section{From Temporal Formulas to Games}\label{sec:to-game}
To check the realizability of specifications and synthesize reactive programs, \issy follows the classical approach of reducing the task to solving a two-player game.
To this end, it translates the specification into a symbolic synthesis game by first translating the temporal logic formulas to games, and then building their product with the rest of the specification. 
The construction of games from the  formulas follows~\cite{HeimD25}  and provides the option to build and use a  \emph{monitor} to prune/simplify the constructed game by performing first-order and temporal  reasoning during game construction. 
More concretely, a given formula is first translated to a deterministic $\omega$-automaton using \texttt{Spot}~\cite{Duret-LutzRCRAS22}.  
Then, monitors are constructed \emph{on-the-fly}, building the product between the game obtained from the automaton and the monitor. 
The product with the monitor enhances the game with semantic information~\cite{HeimD25}, resulting in the so-called \emph{enhanced game}, which is potentially easier to solve. 
As sometimes the monitor construction causes overhead, \issy has a parameter \texttt{\textcolor{blue}{-{}-pruning}} controlling its complexity,  ranging from no monitor construction (level 0), to applying powerful deduction during its construction (level 3).\looseness=-1

The prototype \tslmtrpg~\cite{HeimD25} is restricted to the logic TSL-MT and constructs RPGs. In contrast,  the translation in \issy applies to the more general logic \rpltl, and constructs a more general form of symbolic games.  
In TLS-MT and RPGs, the system controls the state variables via a fixed finite set of possible updates, a restriction not present in \rpltl{} and the respective symbolic games. 
For example,  assertions like \texttt{x' > x} are not expressible in TSL-MT.
\looseness=-1




\section{An Acceleration-Based Solver for Infinite-State Games}\label{sec:solver}
The architecture of \issy is shown in \Cref{fig:toolchain}.  We discussed the components translating a specification to a single synthesis game in \Cref{sec:to-game}.
Now we present the game solver underlying \issy, focusing on the novel  technical developments.

\begin{figure}[t!]
\tikzstyle{interface} = [rectangle, rounded corners, minimum width=2cm, minimum height=.6cm,text centered, draw=black, fill=red!20]
\tikzstyle{io} = [trapezium, trapezium left angle=85, trapezium right angle=95, minimum width=.5cm, minimum height=.6cm, text centered, draw=black, fill=gray!20]
\tikzstyle{process} = [rectangle, minimum width=1.5cm, minimum height=.6cm, text centered, draw=blue, fill=blue!20]
\tikzstyle{ext} = [rectangle, minimum width=1.5cm, minimum height=.6cm, text centered, draw=black]
\tikzstyle{arrow} = [thick,->,>=stealth]
\tikzstyle{doublearrow} = [thick,<->,>=stealth]

\begin{center}

\begin{tikzpicture}[scale=0.75, node distance=2cm]
\node (issycomp) [process, text width=1.5cm] {\issy compiler};
\node (llissyprob) [io, text width=1.65cm, below of = issycomp, yshift=.7cm] {\phantom{aa} \llissy \newline specification};
\node (llissypars) [process, text width=1.5cm, below of = llissyprob, yshift=.7cm] {\llissy parser};

\node (spec) [io, text width=3.6cm, right of = llissypars, xshift=1.6cm] {Specification: Formula$_1,\ldots,$ Formula$_m$ Game$_1,\ldots,$ Game$_n$};
\node (gameprod) [process, text width=2cm, right of = spec, xshift=3cm] {Game product construction};
\node (enhgame) [io, text width=2.25cm, above of = gameprod, yshift=-.7cm, xshift=-.5cm] {Enhanced game};
\node (otfprod) [process, text width=2cm, above of = enhgame, yshift=-.5cm,xshift=.5cm,yshift=-.2cm] {On-the-fly product construction};

\node (formgame) [io, text width=.8cm,  above left of = otfprod, xshift=-1cm,yshift=-.6cm] {Game};
\node (monitor) [io, text width=1cm,  below left of = otfprod, xshift=-1cm, yshift=1.3cm] {Monitor};

\node (directconstr) [process, text width=2cm, left of = formgame, xshift=-1.5cm] {Formulas to games};
\node (monconstr) [process, text width=2cm, left of = monitor, xshift=-.5cm, yshift=0cm] {Formulas to monitors};

\node (game) [io, text width=1.5cm, below of = gameprod, yshift=.6cm, xshift=0cm] {Symbolic game};

\node (gameinterface) [interface, text width=4cm, left of = game, xshift=-1.5cm] {Game-solving interface};

\node (rpg) [io, text width=.5cm, left of = gameinterface, xshift=-1.25cm] {RPG};
\node (rpgpars) [ext, text width=1.5cm, above left of = rpg, xshift=-.5cm, yshift=-1cm] {RPG parser};
\node (tslmttorpg) [ext, text width=1.5cm, below left of = rpg, xshift=-.5cm, yshift=.85cm] {\tslmtrpg};

\node (solver) [process, text width=1.5cm, below of = gameinterface,  yshift=.4cm] {Game solver};

\node (result) [text width=2.5cm, below of =solver, xshift=-1cm, yshift=0cm] {\textbf{Realizable/Unrealizable}};

\node (attracc) [process, text width=2.5cm, right of = solver,  xshift = 1cm] {Attractor acceleration};
\node (solveacc) [process, text width=2.5cm, below of =attracc, yshift=1cm] {Outer fixpoint acceleration};

\node (extract) [process, text width=2cm, left of = solver,  xshift = -.5cm] {Program extraction};
\node (absprog) [io, text width=1.4cm, left of = extract,  xshift = -.7cm] {Abstract reactive program};
\node (genc) [process, text width=2cm, below of =absprog, yshift=.6cm] {C program construction};


\draw [arrow] (issycomp) -- (llissyprob);
\draw [arrow] (llissyprob) -- (llissypars);
\draw [arrow] (llissypars) -- (spec);
\draw [arrow] ($(spec.north) - (2cm,0) $) -- ($(directconstr.south) - (.8cm,0) $);
\draw [arrow] ($(spec.north) + (0.13cm,0) $) -- (monconstr);
\draw [arrow] (monconstr) -- (monitor);
\draw [arrow] (directconstr) -- (formgame);
\draw [arrow] (monitor) -- (otfprod);
\draw [arrow] (formgame) -- (otfprod);
\draw [arrow] (otfprod) -- ($(enhgame.north) + (.65,0)$);
\draw [arrow] ($(enhgame.south) + (.65,0)$) -- (gameprod);
\draw [arrow] ($(spec.east) - (0.1,0.5)$) -- ($(gameprod.west) - (0,0.5)$);
\draw [arrow] (gameprod) -- (game);
\draw [arrow] (game) -- (gameinterface);


\draw [arrow] (rpgpars) -- (rpg);
\draw [arrow] (tslmttorpg) -- (rpg);
\draw [arrow] (rpg) -- (gameinterface);

\draw [arrow] (gameinterface) -- (solver);
\draw [doublearrow] (solver) -- (attracc);
\draw [doublearrow] (solver) -- (solveacc);


\draw [arrow] (solver) -- ($(result.north) + (1.35cm,0) $);

\draw [arrow] (solver) -- (extract);
\draw [arrow] (extract) -- (absprog);
\draw [arrow] (absprog) -- (genc);

\node (formulastogame) [draw=blue,very thick,rounded corners,fit = (directconstr) (monconstr) (formgame) (monitor) (otfprod), inner sep=3pt] {};
\begin{scope}[on background layer]
\node (symbgameconstruction) [draw=blue,very thick,rounded corners,fit = (formulastogame) (spec) (gameprod),inner sep=3pt,fill=blue!10] {};
\end{scope}


\begin{scope}[on background layer]
\node (synthesisengine) [draw=blue,very thick,rounded corners,fit = (solver) (extract) (attracc) (solveacc), inner sep=3pt,fill=blue!10] {};
\end{scope}

\node (issyf) [above of=issycomp,text width=.5cm,yshift=-.7cm] {\scalebox{1.9}{\faFile*[regular]}\newline \textbf{.issy}};
\draw [arrow] (issyf) -- (issycomp);

\node (llissyf) [left of=llissypars,text width=.5cm, xshift=.4cm] {\scalebox{1.9}{\faFile*[regular]}\newline \textbf{.llissy}};
\draw [arrow] (llissyf) -- (llissypars);

\node (rpgf) [left of=rpgpars,text width=.5cm, xshift=.4cm] {\scalebox{1.9}{\faFile*[regular]}\newline \textbf{.rpg}};
\draw [arrow] (rpgf) -- (rpgpars);

\node (tslmtf) [left of=tslmttorpg,text width=.5cm, xshift=.4cm] {\scalebox{1.9}{\faFile*[regular]}\newline \textbf{.tslmt}};
\draw [arrow] (tslmtf) -- (tslmttorpg);

\node (cf) [left of=genc,text width=.5cm, xshift=.3cm] {\scalebox{1.9}{\faFile*[regular]}\newline \textbf{.c}};
\draw [arrow] (genc) -- (cf);

\node [above of = otfprod,   xshift=-1.6cm, yshift=-.2cm] {\textcolor{blue}{\textbf{Translation to symbolic game} (Sec.~\ref{sec:to-game})}};

\node [above of = attracc,   yshift=-1.1cm,xshift=-1cm] {\textcolor{blue}{\textbf{Acceleration-based solver} (Sec.~\ref{sec:solver})}};
\end{tikzpicture}

\end{center}

\vspace{-.7cm}
\caption{Architecture of \issy. Components depicted in blue and pink are part of the tool's implementation, those depicted in white are external.}
\label{fig:toolchain}
\end{figure}

The approach behind the \issy solver builds on the method proposed in~\cite{HeimD24}.
The fist main difference to the prototype \rpgsolve from~\cite{HeimD24} is that \rpgsolve accepts RPGs,  a strictly more restricted class of symbolic games.  Furthermore,  the initial version of \rpgsolve does not support parity winning conditions.   
Our \Cref{ex:simple-example} cannot be modelled as an RPG,  because the system player has the power to select any real values as next-state values for the state variables.
Furthermore,  the specification in \Cref{ex:simple-example} translates to a parity game.
\issy's solver supports a more general symbolic game model, and also implements a symbolic method for infinite-state parity games based on fixpoint computation (a lifting of the classical Zielonka's algorithm~\cite{Zielonka98}).  Thus,  \issy is able to establish the realizability of the specification in Listing~\ref{lst:example} thanks to the new techniques it implements.

The crux to this is the acceleration technique introduced in~\cite{HeimD24}.
Naive fixpoint-based game-solving diverges on this example.
\emph{Attractor acceleration}~\cite{HeimD24} uses ranking arguments to establish that by iterating some strategy an unbounded number of times through some location, a player in the game can  enforce reaching a set of target states.  In \Cref{ex:simple-example},  attractor acceleration is used within the procedure for solving the parity game to establish that (under the respective constraints on the environment) from any state satisfying the formula \texttt{balanced},  a state where both \texttt{load1} and \texttt{load2} are in the bounded interval $[\frac{3}{10},\frac{9}{10}]$ can be enforced by the system player.  This argument is formalized as what is called an \emph{acceleration lemma}~\cite{HeimD24}.  From the interval $[\frac{3}{10},\frac{9}{10}]$,  the system player can then enforce reaching in a bounded number of steps a state where  \texttt{load1} and \texttt{load2} are zero.

We developed a novel method for generating acceleration lemmas and  implemented in \issy in addition to that from~\cite{HeimD24}.
To search for acceleration lemmas,  \rpgsolve introduces uninterpreted predicates representing the lemmas' components, and collects SMT constraints asserting the applicability of the lemma.  Thus,  \rpgsolve would have to discover the formula \texttt{[load1 >= load2] \&\& [load1 <= 2 * load2]  ||[load2 >= load1] \&\& [load2 <= 2 * load1]} as part of the acceleration lemma, which it is not able to do within a reasonable timeout.
The alternative method we implemented in \issy performs analysis of the game in order to generate candidate  acceleration lemmas.
First,  it analyzes the game in order to identify variables potentially making progress in a ranking argument.  For instance,  variables that remain unchanged in the relevant game locations can  be ruled out.  
Second, the new method uses the distance to the target set of states to generate ranking arguments for candidate acceleration lemmas.  Finally,  to search for a set of states where the  respective player can enforce the decrease of the distance, it uses symbolic iteration and SMT-based formula generalization.  As demonstrated for \Cref{ex:simple-example}, and more broadly by our experimental evaluation in \Cref{sec:experiments},  this new method for generating acceleration lemmas, which we call \emph{geometric acceleration}, is successful in many cases that are challenging for \rpgsolve. In \issy, geometric attractor acceleration is enabled by default, and the method can be switched using the parameter \texttt{\textcolor{blue}{-{}-accel-attr}}.

In addition to an alternative method for generating acceleration lemmas,  the \issy solver utilizes new techniques for their localization. Building on ideas in~\cite{SchmuckHDN24}, 
we restrict the size of the sub-games used for the acceleration lemma computation and project away variables that are not relevant in the respective subgame.  Unlike~\cite{SchmuckHDN24}, where this is done for pre-computing accelerations,  in \issy these localization techniques are applied on-the-fly during the main game solving.

\issy also provides support for strategy synthesis and extraction of C programs for realizable specifications.  The latter can be extended to other target languages,  utilizing the 
generic data structure for reactive program representation in \issy.

\issy\footnote{\url{https://github.com/phheim/issy}} is implemented in Haskell with focus on modularity and extensibility,  including detailed documentation.  Using the Haskell tool Stack,  building  \issy and getting its dependencies is seamless.
The external tools used are \texttt{Spot}~\cite{Duret-LutzRCRAS22}  for translation of LTL  to automata,  the $\mu$CLP solver \texttt{MuVal}~\cite{UnnoTGK23} and the Optimal CHC solver \texttt{OptPCSat}~\cite{GuTU23}  for the monitor construction,  and \texttt{z3}~\cite{Z3} for all SMT,  formula simplification and quantifier elimination queries.




\section{Benchmarks and Evaluation}\label{sec:experiments}
\section{Experiments}
\label{sec:experiments}
The experiments are designed to address two key research questions.
First, \textbf{RQ1} evaluates whether the average $L_2$-norm of the counterfactual perturbation vectors ($\overline{||\perturb||}$) decreases as the model overfits the data, thereby providing further empirical validation for our hypothesis.
Second, \textbf{RQ2} evaluates the ability of the proposed counterfactual regularized loss, as defined in (\ref{eq:regularized_loss2}), to mitigate overfitting when compared to existing regularization techniques.

% The experiments are designed to address three key research questions. First, \textbf{RQ1} investigates whether the mean perturbation vector norm decreases as the model overfits the data, aiming to further validate our intuition. Second, \textbf{RQ2} explores whether the mean perturbation vector norm can be effectively leveraged as a regularization term during training, offering insights into its potential role in mitigating overfitting. Finally, \textbf{RQ3} examines whether our counterfactual regularizer enables the model to achieve superior performance compared to existing regularization methods, thus highlighting its practical advantage.

\subsection{Experimental Setup}
\textbf{\textit{Datasets, Models, and Tasks.}}
The experiments are conducted on three datasets: \textit{Water Potability}~\cite{kadiwal2020waterpotability}, \textit{Phomene}~\cite{phomene}, and \textit{CIFAR-10}~\cite{krizhevsky2009learning}. For \textit{Water Potability} and \textit{Phomene}, we randomly select $80\%$ of the samples for the training set, and the remaining $20\%$ for the test set, \textit{CIFAR-10} comes already split. Furthermore, we consider the following models: Logistic Regression, Multi-Layer Perceptron (MLP) with 100 and 30 neurons on each hidden layer, and PreactResNet-18~\cite{he2016cvecvv} as a Convolutional Neural Network (CNN) architecture.
We focus on binary classification tasks and leave the extension to multiclass scenarios for future work. However, for datasets that are inherently multiclass, we transform the problem into a binary classification task by selecting two classes, aligning with our assumption.

\smallskip
\noindent\textbf{\textit{Evaluation Measures.}} To characterize the degree of overfitting, we use the test loss, as it serves as a reliable indicator of the model's generalization capability to unseen data. Additionally, we evaluate the predictive performance of each model using the test accuracy.

\smallskip
\noindent\textbf{\textit{Baselines.}} We compare CF-Reg with the following regularization techniques: L1 (``Lasso''), L2 (``Ridge''), and Dropout.

\smallskip
\noindent\textbf{\textit{Configurations.}}
For each model, we adopt specific configurations as follows.
\begin{itemize}
\item \textit{Logistic Regression:} To induce overfitting in the model, we artificially increase the dimensionality of the data beyond the number of training samples by applying a polynomial feature expansion. This approach ensures that the model has enough capacity to overfit the training data, allowing us to analyze the impact of our counterfactual regularizer. The degree of the polynomial is chosen as the smallest degree that makes the number of features greater than the number of data.
\item \textit{Neural Networks (MLP and CNN):} To take advantage of the closed-form solution for computing the optimal perturbation vector as defined in (\ref{eq:opt-delta}), we use a local linear approximation of the neural network models. Hence, given an instance $\inst_i$, we consider the (optimal) counterfactual not with respect to $\model$ but with respect to:
\begin{equation}
\label{eq:taylor}
    \model^{lin}(\inst) = \model(\inst_i) + \nabla_{\inst}\model(\inst_i)(\inst - \inst_i),
\end{equation}
where $\model^{lin}$ represents the first-order Taylor approximation of $\model$ at $\inst_i$.
Note that this step is unnecessary for Logistic Regression, as it is inherently a linear model.
\end{itemize}

\smallskip
\noindent \textbf{\textit{Implementation Details.}} We run all experiments on a machine equipped with an AMD Ryzen 9 7900 12-Core Processor and an NVIDIA GeForce RTX 4090 GPU. Our implementation is based on the PyTorch Lightning framework. We use stochastic gradient descent as the optimizer with a learning rate of $\eta = 0.001$ and no weight decay. We use a batch size of $128$. The training and test steps are conducted for $6000$ epochs on the \textit{Water Potability} and \textit{Phoneme} datasets, while for the \textit{CIFAR-10} dataset, they are performed for $200$ epochs.
Finally, the contribution $w_i^{\varepsilon}$ of each training point $\inst_i$ is uniformly set as $w_i^{\varepsilon} = 1~\forall i\in \{1,\ldots,m\}$.

The source code implementation for our experiments is available at the following GitHub repository: \url{https://anonymous.4open.science/r/COCE-80B4/README.md} 

\subsection{RQ1: Counterfactual Perturbation vs. Overfitting}
To address \textbf{RQ1}, we analyze the relationship between the test loss and the average $L_2$-norm of the counterfactual perturbation vectors ($\overline{||\perturb||}$) over training epochs.

In particular, Figure~\ref{fig:delta_loss_epochs} depicts the evolution of $\overline{||\perturb||}$ alongside the test loss for an MLP trained \textit{without} regularization on the \textit{Water Potability} dataset. 
\begin{figure}[ht]
    \centering
    \includegraphics[width=0.85\linewidth]{img/delta_loss_epochs.png}
    \caption{The average counterfactual perturbation vector $\overline{||\perturb||}$ (left $y$-axis) and the cross-entropy test loss (right $y$-axis) over training epochs ($x$-axis) for an MLP trained on the \textit{Water Potability} dataset \textit{without} regularization.}
    \label{fig:delta_loss_epochs}
\end{figure}

The plot shows a clear trend as the model starts to overfit the data (evidenced by an increase in test loss). 
Notably, $\overline{||\perturb||}$ begins to decrease, which aligns with the hypothesis that the average distance to the optimal counterfactual example gets smaller as the model's decision boundary becomes increasingly adherent to the training data.

It is worth noting that this trend is heavily influenced by the choice of the counterfactual generator model. In particular, the relationship between $\overline{||\perturb||}$ and the degree of overfitting may become even more pronounced when leveraging more accurate counterfactual generators. However, these models often come at the cost of higher computational complexity, and their exploration is left to future work.

Nonetheless, we expect that $\overline{||\perturb||}$ will eventually stabilize at a plateau, as the average $L_2$-norm of the optimal counterfactual perturbations cannot vanish to zero.

% Additionally, the choice of employing the score-based counterfactual explanation framework to generate counterfactuals was driven to promote computational efficiency.

% Future enhancements to the framework may involve adopting models capable of generating more precise counterfactuals. While such approaches may yield to performance improvements, they are likely to come at the cost of increased computational complexity.


\subsection{RQ2: Counterfactual Regularization Performance}
To answer \textbf{RQ2}, we evaluate the effectiveness of the proposed counterfactual regularization (CF-Reg) by comparing its performance against existing baselines: unregularized training loss (No-Reg), L1 regularization (L1-Reg), L2 regularization (L2-Reg), and Dropout.
Specifically, for each model and dataset combination, Table~\ref{tab:regularization_comparison} presents the mean value and standard deviation of test accuracy achieved by each method across 5 random initialization. 

The table illustrates that our regularization technique consistently delivers better results than existing methods across all evaluated scenarios, except for one case -- i.e., Logistic Regression on the \textit{Phomene} dataset. 
However, this setting exhibits an unusual pattern, as the highest model accuracy is achieved without any regularization. Even in this case, CF-Reg still surpasses other regularization baselines.

From the results above, we derive the following key insights. First, CF-Reg proves to be effective across various model types, ranging from simple linear models (Logistic Regression) to deep architectures like MLPs and CNNs, and across diverse datasets, including both tabular and image data. 
Second, CF-Reg's strong performance on the \textit{Water} dataset with Logistic Regression suggests that its benefits may be more pronounced when applied to simpler models. However, the unexpected outcome on the \textit{Phoneme} dataset calls for further investigation into this phenomenon.


\begin{table*}[h!]
    \centering
    \caption{Mean value and standard deviation of test accuracy across 5 random initializations for different model, dataset, and regularization method. The best results are highlighted in \textbf{bold}.}
    \label{tab:regularization_comparison}
    \begin{tabular}{|c|c|c|c|c|c|c|}
        \hline
        \textbf{Model} & \textbf{Dataset} & \textbf{No-Reg} & \textbf{L1-Reg} & \textbf{L2-Reg} & \textbf{Dropout} & \textbf{CF-Reg (ours)} \\ \hline
        Logistic Regression   & \textit{Water}   & $0.6595 \pm 0.0038$   & $0.6729 \pm 0.0056$   & $0.6756 \pm 0.0046$  & N/A    & $\mathbf{0.6918 \pm 0.0036}$                     \\ \hline
        MLP   & \textit{Water}   & $0.6756 \pm 0.0042$   & $0.6790 \pm 0.0058$   & $0.6790 \pm 0.0023$  & $0.6750 \pm 0.0036$    & $\mathbf{0.6802 \pm 0.0046}$                    \\ \hline
%        MLP   & \textit{Adult}   & $0.8404 \pm 0.0010$   & $\mathbf{0.8495 \pm 0.0007}$   & $0.8489 \pm 0.0014$  & $\mathbf{0.8495 \pm 0.0016}$     & $0.8449 \pm 0.0019$                    \\ \hline
        Logistic Regression   & \textit{Phomene}   & $\mathbf{0.8148 \pm 0.0020}$   & $0.8041 \pm 0.0028$   & $0.7835 \pm 0.0176$  & N/A    & $0.8098 \pm 0.0055$                     \\ \hline
        MLP   & \textit{Phomene}   & $0.8677 \pm 0.0033$   & $0.8374 \pm 0.0080$   & $0.8673 \pm 0.0045$  & $0.8672 \pm 0.0042$     & $\mathbf{0.8718 \pm 0.0040}$                    \\ \hline
        CNN   & \textit{CIFAR-10} & $0.6670 \pm 0.0233$   & $0.6229 \pm 0.0850$   & $0.7348 \pm 0.0365$   & N/A    & $\mathbf{0.7427 \pm 0.0571}$                     \\ \hline
    \end{tabular}
\end{table*}

\begin{table*}[htb!]
    \centering
    \caption{Hyperparameter configurations utilized for the generation of Table \ref{tab:regularization_comparison}. For our regularization the hyperparameters are reported as $\mathbf{\alpha/\beta}$.}
    \label{tab:performance_parameters}
    \begin{tabular}{|c|c|c|c|c|c|c|}
        \hline
        \textbf{Model} & \textbf{Dataset} & \textbf{No-Reg} & \textbf{L1-Reg} & \textbf{L2-Reg} & \textbf{Dropout} & \textbf{CF-Reg (ours)} \\ \hline
        Logistic Regression   & \textit{Water}   & N/A   & $0.0093$   & $0.6927$  & N/A    & $0.3791/1.0355$                     \\ \hline
        MLP   & \textit{Water}   & N/A   & $0.0007$   & $0.0022$  & $0.0002$    & $0.2567/1.9775$                    \\ \hline
        Logistic Regression   &
        \textit{Phomene}   & N/A   & $0.0097$   & $0.7979$  & N/A    & $0.0571/1.8516$                     \\ \hline
        MLP   & \textit{Phomene}   & N/A   & $0.0007$   & $4.24\cdot10^{-5}$  & $0.0015$    & $0.0516/2.2700$                    \\ \hline
       % MLP   & \textit{Adult}   & N/A   & $0.0018$   & $0.0018$  & $0.0601$     & $0.0764/2.2068$                    \\ \hline
        CNN   & \textit{CIFAR-10} & N/A   & $0.0050$   & $0.0864$ & N/A    & $0.3018/
        2.1502$                     \\ \hline
    \end{tabular}
\end{table*}

\begin{table*}[htb!]
    \centering
    \caption{Mean value and standard deviation of training time across 5 different runs. The reported time (in seconds) corresponds to the generation of each entry in Table \ref{tab:regularization_comparison}. Times are }
    \label{tab:times}
    \begin{tabular}{|c|c|c|c|c|c|c|}
        \hline
        \textbf{Model} & \textbf{Dataset} & \textbf{No-Reg} & \textbf{L1-Reg} & \textbf{L2-Reg} & \textbf{Dropout} & \textbf{CF-Reg (ours)} \\ \hline
        Logistic Regression   & \textit{Water}   & $222.98 \pm 1.07$   & $239.94 \pm 2.59$   & $241.60 \pm 1.88$  & N/A    & $251.50 \pm 1.93$                     \\ \hline
        MLP   & \textit{Water}   & $225.71 \pm 3.85$   & $250.13 \pm 4.44$   & $255.78 \pm 2.38$  & $237.83 \pm 3.45$    & $266.48 \pm 3.46$                    \\ \hline
        Logistic Regression   & \textit{Phomene}   & $266.39 \pm 0.82$ & $367.52 \pm 6.85$   & $361.69 \pm 4.04$  & N/A   & $310.48 \pm 0.76$                    \\ \hline
        MLP   &
        \textit{Phomene} & $335.62 \pm 1.77$   & $390.86 \pm 2.11$   & $393.96 \pm 1.95$ & $363.51 \pm 5.07$    & $403.14 \pm 1.92$                     \\ \hline
       % MLP   & \textit{Adult}   & N/A   & $0.0018$   & $0.0018$  & $0.0601$     & $0.0764/2.2068$                    \\ \hline
        CNN   & \textit{CIFAR-10} & $370.09 \pm 0.18$   & $395.71 \pm 0.55$   & $401.38 \pm 0.16$ & N/A    & $1287.8 \pm 0.26$                     \\ \hline
    \end{tabular}
\end{table*}

\subsection{Feasibility of our Method}
A crucial requirement for any regularization technique is that it should impose minimal impact on the overall training process.
In this respect, CF-Reg introduces an overhead that depends on the time required to find the optimal counterfactual example for each training instance. 
As such, the more sophisticated the counterfactual generator model probed during training the higher would be the time required. However, a more advanced counterfactual generator might provide a more effective regularization. We discuss this trade-off in more details in Section~\ref{sec:discussion}.

Table~\ref{tab:times} presents the average training time ($\pm$ standard deviation) for each model and dataset combination listed in Table~\ref{tab:regularization_comparison}.
We can observe that the higher accuracy achieved by CF-Reg using the score-based counterfactual generator comes with only minimal overhead. However, when applied to deep neural networks with many hidden layers, such as \textit{PreactResNet-18}, the forward derivative computation required for the linearization of the network introduces a more noticeable computational cost, explaining the longer training times in the table.

\subsection{Hyperparameter Sensitivity Analysis}
The proposed counterfactual regularization technique relies on two key hyperparameters: $\alpha$ and $\beta$. The former is intrinsic to the loss formulation defined in (\ref{eq:cf-train}), while the latter is closely tied to the choice of the score-based counterfactual explanation method used.

Figure~\ref{fig:test_alpha_beta} illustrates how the test accuracy of an MLP trained on the \textit{Water Potability} dataset changes for different combinations of $\alpha$ and $\beta$.

\begin{figure}[ht]
    \centering
    \includegraphics[width=0.85\linewidth]{img/test_acc_alpha_beta.png}
    \caption{The test accuracy of an MLP trained on the \textit{Water Potability} dataset, evaluated while varying the weight of our counterfactual regularizer ($\alpha$) for different values of $\beta$.}
    \label{fig:test_alpha_beta}
\end{figure}

We observe that, for a fixed $\beta$, increasing the weight of our counterfactual regularizer ($\alpha$) can slightly improve test accuracy until a sudden drop is noticed for $\alpha > 0.1$.
This behavior was expected, as the impact of our penalty, like any regularization term, can be disruptive if not properly controlled.

Moreover, this finding further demonstrates that our regularization method, CF-Reg, is inherently data-driven. Therefore, it requires specific fine-tuning based on the combination of the model and dataset at hand.


\bibliographystyle{splncs04}
\bibliography{main.bib}

\newpage

\appendix

\section{Full Description of the \issy Format}\label{sec:format-issy-full}
\subsubsection{\issy  specification}

\grammarindent2.2cm

\begin{grammar}

<spec> ::= (<vardecl> | <logicspec> | <gamespec> | <macro>)*
 
\end{grammar}

\subsubsection{Variable Declarations}


\begin{grammar}

<vardecl> ::= (`input' | `state') <type> <identifier> 

 <type>    ::= `int' | `bool' | `real'

\end{grammar}

\subsubsection{Formula Specifications}

\begin{grammar}

<logicspec> ::=  `formula' `{' <logicstm>$^*$  `}'  

 <logicstm> ::= (`assert' | `assume') <rpltl>

<rpltl>    ::=  <atom> \alt `(' <rpltl> `)' \alt <uopt> <rpltl> \alt <rpltl> <bopt> <rpltl>

<uopt>    ::= `!' | `F' | `X' | `G'

<bopt>    ::= `&&' | `||' | `->' | `<->' | `U' | `W' | `R'

\end{grammar}

\noindent
with precedence of the logical operators as  in TLSF.


\subsubsection{Game Specifications}

\begin{grammar}

<gamespec>  ::= `game' <wincond> `from' <identifier> `{' ( <locdef> | <transdef>)*`}' 

<wincond>  ::= `Safety' | `Reachability' | `Buechi'  | `CoBuechi' | `ParityMaxOdd'

<locdef>   ::= `loc' <identifier> [<nat>] [`with' <formula>]

<transdef>  ::= `from'  <identifier> `to' <identifier> `with' <formula>


\medskip

<formula>    ::=  <atom>  \alt '(' <formula> ')' \alt <uop> <formula> \alt <formula> <bop> <formula> 

<uop>      ::=  `!' 

<bop>      ::=  `&&' | `||' | `->' | `<->'

\end{grammar}
\noindent
with precedence (from high to low):\\
\textsf{ \{!\} > \{\&\&\} > \{||\} > \{-> (ra)\} > \{<-> (ra)\}} 


\subsubsection{Atomic Predicates}

\begin{grammar}

<atom>   ::=  <apred> \alt <bconst> \alt  <identifier>['''] \alt `havoc' `('<identifier>* `)' \alt `keep' `(' 
<identifier>* `)'

<bconst>  ::= `true'  | `false'

<apred>   ::= `[' <pred> `]'

<pred>    ::= <const> \alt <identifier>['''] \alt `(' <pred> `)' \alt <auop> <pred> \alt <pred> <abop> <pred>

<const>   ::= <nat>   | <rat>

<auop>    ::= `*' | `+' | `-' | `/' | `mod' | `=' | `<' | `>'| `<=' | `>='

<abop>   ::= `-' | `abs'

\end{grammar}

\noindent
with precedence (from high to low):\\
\textsf{    \{abs\} > \{*, /, mod\} > \{+, -\} > \{<, >, =, <=, >=\}}


\subsubsection{Macros}

\begin{grammar}
<macro>  ::= `def' <identifier> `=' <formula> | <apred>
\end{grammar}

\noindent
Note: macros can be used in all $\langle \mathit{rpltl}\rangle$,  $\langle \mathit{formula}\rangle$, and $\langle\mathit{pred}\rangle$. However, for usage in $\langle\mathit{pred}\rangle$ the marco term has to be a single predicate term.

\subsubsection{Identifiers and Numerical Constants}

\begin{grammar}
<identifier>      ::= <alpha> (<alpha> | <digit> | `_')*

<nat>             ::= <digit>+

<rat>            ::= <digit>+ '.' <digit>+
\end{grammar}

\subsubsection{Comments}
\begin{itemize}
\item single line  \textsf{/ /}
\item  multi-line \textsf{/ *}
\end{itemize}

\noindent
Comments cannot be nested.


\newpage

\section{The LLissy Format}\label{sec:format-llissy}

In order to be easy to parse, readable with reasonable effort, and to be similar to the SMTLib-format, \llissy uses s-expressions.

Only single line comments exist which start with \textsf{';'} and span to the end of the line. 
Newlines are \textsf{'\textbackslash r\textbackslash n', '\textbackslash n \textbackslash r', ' \textbackslash r'} and \textsf{'\textbackslash n'}. 
However, when generating \llissy automatically \textsf{'\textbackslash n'} should be used. 
Similarly \textsf{' '} (Space) and \textsf{'\textbackslash t'} (Tabs) are both non-newline white-spaces. 
However, only \textsf{' '} should be used upon generation. 
The following productions define identifiers and natural numbers.
\begin{grammar}
<ALPHA> ::= `a'...`z' | `A'...`Z' 

<DIGIT> ::= `0'...`9'

<ID>    ::= <ALPHA> (<ALPHA> | <DIGIT> | `_')*

<PID>   ::= <ID> ['$\sim$']

<NAT>   ::=  <DIGIT>+

<RAT>   ::=  <DIGIT>+ `.' <DIGIT>+
\end{grammar}

\noindent
Note that all of these should be parsed greedily until a white-space, '(', ')', or the end-of-file occurs.

A \llissy specification consists of lists of variable declarations, formula specifications and game specifications. The variables declarations include all variables used in all games and formulas. 
The formula and game specifications are interpreted conjunctively. 
However, at most one game or formula can be a non-safety game or non-safety formula.

\begin{grammar}
<SPEC> ::= `(' `(' <VARDEC>* `)' `(' <FSPEC>* `)' `(' <GSPEC>*  `)' `)'
\end{grammar}

\noindent
A variable declaration declares an input or state variable and its respective type
\begin{grammar}
<VARDEC>  ::= `(' `input' <TYPE> <ID> `)' | `(' `state' <TYPE> <ID> `)'

<TYPE>   ::=  `Int' | `Bool' | `Real'
\end{grammar}

A formula specification is a pair of assumption and guarantee lists. Each element is an RP-LTL formula.
The assumptions come first, and each of the two lists is interpreted as a conjunction. 
\begin{grammar}
<FSPEC>   ::= `(' `(' <FORMULA>* `)' `(' <FORMULA>* `)' `)'

<FORMULA> ::= `(' `ap' <TERM>`')' \alt `(' <UOP> <FORMULA> `)' \alt `(' <BOP> <FORMULA> <FORMULA> `)' \alt (<NOP> <FORMULA>*)

<UOP>     ::= `X' | `F' | `G' | `not'

<BOP>     ::= `U' | `W' | `R'

<NOP>     ::= '`and' | `or'
\end{grammar}

A game specification consists of a list of location definitions, transition definitions from one location to another location, and an objective definition.
The objective defines the initial location and the winning condition. Each location is annotated with a natural number. For Safety, Reachability, Buechi, and CoBuechi a location is safe, target, Buechi accepting, coBuechi accepting iff the number is greater than zero. 
For ParitMaxOdd the number is the color in the parity game.

\begin{grammar}
<GSPEC>    ::= `(' `(' <LOCDEF>* `)' `(' <TRANSDEF>* `)' <OBJ> `)'

<LOCDEF>   ::= `(' <ID> <NAT> <TERM> `)'

<TRANSDEF> ::= `(' <ID> <ID> <TERM> `)'

<OBJ>      ::= `(' <ID> (`Safety' | `Reachability ' | `Buechi' | `CoBuechi' \newline \phantom{aaaaaaaaaaaaaaaa} | `ParityMaxOdd') `)'
\end{grammar}

A term is basically like in the SMT-Lib-2 format without quantifiers, lambda, and let expressions. Similar rules for typing apply.
Only variables declared initially are allowed to be free variables, and additionally primed version (with $\sim$) of the state variables.

\begin{grammar}
<TERM>   ::= `(' <OP> <TERM>* `)' | <PID> | <CONSTS>

<OP>     ::= `and ' | `or' | `not' | `ite' | `distinct' | `=>' |
         `=' | `<' | `>'| `<=' | `>=' |
         `+' | `-' | `*' | `/' | `mod' | `abs' | `to_real' 

<CONSTS> ::= <RAT> | <NAT> | `true' | `false'
\end{grammar}





\end{document}
