\documentclass[runningheads]{llncs}

\usepackage[T1]{fontenc}
\usepackage{graphicx}
\usepackage{xcolor}
\usepackage{color, colortbl}


\usepackage{amssymb}
\usepackage{amsmath}


\usepackage{cleveref}

\usepackage{tikz}
\usepackage{pgfplots}
\usepackage{subcaption}
\usetikzlibrary{shapes.geometric, arrows,calc,fit,backgrounds}

\usepackage{wrapfig,lipsum,booktabs}

\usepackage{xspace}

\usepackage{pifont}
\newcommand{\cmark}{\ding{51}}
\newcommand{\xmark}{\ding{55}}
\newcommand{\bmark}{$\bullet$}

\usepackage{fontawesome5}

\usepackage{syntax}
\usepackage{listings}

\newcommand{\thought}[1]{{\color[rgb]{0.2,0.39,0.66}(#1)}}
\newcommand{\todo}[1]{{\color[rgb]{1.0,0.0,0.0}(#1)}}
\newcommand{\hsh}[1]{{\color{green!50!black} Henrik: #1}}
\newcommand{\st}[1]{{\color{red!50!black} Sebastian: #1}}

\newcommand{\ulm}[1]{_{\scaleto{\mathrm{#1}}{3pt}}}
\newcommand\at[2]{\left.#1\right|_{#2}}











\newtheorem{assumption}{Assumption}

\DeclareMathOperator*{\argmax}{arg\,max}
\DeclareMathOperator*{\argmin}{arg\,min}

\newcommand{\swname}[1]{\texttt{#1}}
\newcommand{\ie}{i\/.\/e\/.,\/~}
\newcommand{\eg}{e\/.\/g\/.,\/~}
\newcommand{\cf}{cf\/.\/~}

\newcommand{\fig}{Fig\/.\/~}
\newcommand{\defn}{Def\/.\/~}
\newcommand{\sect}{Sec\/.\/~}
\newcommand{\tabl}{Tab\/.\/~}
\newcommand{\algo}{Algorithm~}
\newcommand{\theo}{Theorem~}

\newcommand{\bnnl}{3 hidden layers}
\newcommand{\bnnn}{50 neurons}
\newcommand{\bnna}{tanh activations}

\newcommand{\capt}[1]{\mdseries{\emph{#1}}}

\newcommand{\videolink}{at \url{https://youtu.be/_d7AqTRjz6g}}
\newcommand{\codelink}{\url{https://github.com/wheelbot/mini-wheelbot}}

\newcommand{\fakepar}[1]{\vspace{0mm}\noindent\textbf{#1.}}

\newcommand{\needref}{\textcolor{red}{[REF]}}

\newcommand{\plotfontsize}{9pt}



\begin{document}

\title{Issy: A Comprehensive Tool for Specification and Synthesis of Infinite-State Reactive Systems}
\titlerunning{Issy: Specification and Synthesis of Infinite-State Reactive Systems}


\author{
Philippe Heim\orcidID{0000-0002-5433-8133} \and
Rayna Dimitrova\orcidID{0009-0006-2494-8690}
}

\authorrunning{P. Heim \and R. Dimitrova}

\institute{
CISPA Helmholtz Center for Information Security, Saarbr\"ucken, Germany
\email{\{philippe.heim, dimitrova\}@cispa.de}
}



\maketitle


\begin{abstract}
The synthesis of infinite-state reactive systems from temporal logic specifications or infinite-state games has attracted significant attention in recent years, leading to the emergence of novel solving techniques. 
Most approaches are accompanied by an implementation showcasing their viability on an increasingly larger collection of benchmarks.
Those implementations are --often simple-- prototypes. 
Furthermore, differences in specification formalisms and formats make comparisons difficult, and writing specifications is a tedious and error-prone task.

To address this,  we present \issy{},  a tool for specification, realizability, and synthesis of infinite-state reactive systems.  
\issy{} comes with an expressive specification language that allows for combining infinite-state games and temporal formulas,  thus encompassing the current formalisms.
The realizability checking and synthesis methods implemented in \issy build upon recently developed approaches and extend them with newly engineered efficient techniques, offering a portfolio of solving algorithms.
We evaluate \issy on an extensive set of benchmarks,  demonstrating its competitiveness with the state of the art.
Furthermore,  \issy{} provides tooling for a general high-level format designed to make specification easier for users.  It also includes a compiler to a more machine-readable format that other tool developers can easily use, which we hope will lead to a broader adoption and advances in infinite-state reactive synthesis.
\end{abstract}

\section{Introduction}\label{sec:intro}
\section{Introduction}
\label{sec:intro}

\begin{figure*}[tb]
    \centering
    \includegraphics[width=0.848\linewidth]{figs/circuitnn.pdf} 
    \caption{Illustration of differentiable CircuitNN. CircuitNN is designed based on differentiable NAND gates. After DAS is guided by PI and PO pairs of the truth table, CircuitNN can get the precise circuit architecture logic equivalent to the truth table.}
    \label{fig:circuitnn}
\end{figure*}

% 1. Describe the importance of logic synthesis
% 2. Existing Problems
% (a) Neural Architecture Search: Unstable, Predefined Setting, etc.
% (b) Circuit Generation: Probabilistic Model, Logic Equivalence

With the rapid advancement of technology, the scale of integrated circuits (ICs) has expanded exponentially. 
This expansion has introduced significant challenges in chip manufacturing, particularly concerning power and area metrics.
A primary objective in IC design is achieving the same circuit function with fewer transistors, thereby reducing power usage and area occupancy.

Logic synthesis~\cite{hachtel2005logicsynth}, a critical step in electronic design automation (EDA), transforms behavioral-level circuit designs into optimized gate-level circuits, ultimately yielding the final IC layout. 
The primary goal of logic synthesis is to identify the physical implementation with the fewest gates for a given circuit function. 
This task constitutes a challenging NP-hard combinatorial optimization problem. 
Current logic synthesis tools~\cite{brayton2010abc, wolf2013yosys} rely on human-designed heuristics, often leading to sub-optimal outcomes.

Differentiable architecture search (DAS) techniques~\cite{liu2018darts, chu2020darts} offer novel perspectives on addressing challenges in this problem.
Circuit functions can be represented through truth tables, which map binary inputs to their corresponding outputs. 
Truth tables provide a precise representation of input-output relationships, ensuring the design of functionally equivalent circuits.
Inspired by this, researchers~\cite{deepmind2024ai4sys, wang2024tnet} have begun exploring the application of DAS to synthesize circuits directly from truth tables.
Specifically, \citet{deepmind2024ai4sys} proposed CircuitNN, a framework that learns differentiable connection structures with logic gates, enabling the automatic generation of logic circuits from truth tables.
This approach significantly reduces the complexity of traditional circuit generation. 
Building on this, \citet{wang2024tnet} introduced T-Net, a triangle-shaped variant of CircuitNN, incorporating regularization techniques to enhance the efficiency of DAS.

Despite these advancements, several challenges remain. 
The computational complexity of DAS grows quadratically with the number of gates, posing scalability issues.
Although triangle-shaped architecture~\cite{wang2024tnet} partially mitigates this problem, redundancy persists. 
%Additionally, DAS is susceptible to converging to local optima, limiting the ability to search architectures that satisfy the given truth tables~\cite{liu2018darts}. 
%Furthermore, hyperparameters (network depth and layer width) require extensive searches, introducing complexity and prolonging the synthesis process. 
Additionally, DAS is susceptible to converging to local optima~\cite{liu2018darts} and hyperparameters (network depth and layer width) require extensive searches. 
The challenges arise from the vast search space in DAS. 
% Even with predefined settings for CircuitNN, finding a configuration that meets the truth table requires extensive trial and error during the DAS process. 
Intuitively, limiting the search space through predefined parameters (network depth, gates per layer, and connection probabilities) can significantly reduce the complexity.

Recent advances~\cite{openai2023gpt4, abramson2024alphafold3, esser2024sd3, li2024mar} in conditional generative models have demonstrated remarkable performance across language, vision, and graph generation tasks. 
Motivated by these developments, we propose a novel approach to circuit generation that generates preliminary circuit structures to guide DAS in generating refined circuits matching specified truth tables. 
Firstly, we introduce CircuitVQ, a tokenizer with a discrete codebook for circuit tokenization. 
Built upon our Circuit AutoEncoder framework~\cite{hou2022graphmae,li2023maskgae,wu2025mgvga}, CircuitVQ is trained through a circuit reconstruction task. 
Specifically, the CircuitVQ encoder encodes input circuits into discrete tokens using a learnable codebook, while the decoder reconstructs the circuit adjacency matrix based on these tokens.
Subsequently, the CircuitVQ encoder serves as a circuit tokenizer for CircuitAR pretraining, which employs a masked autoregressive modeling paradigm~\cite{chang2022maskgit, li2023mage}. 
In this process, the discrete codes function as supervision signals. 
After training, CircuitAR can generate discrete tokens progressively, which can be decoded into initial circuit structures by the decoder of the CircuitVQ. 
These prior insights can guide DAS in producing refined circuits that match the target truth tables precisely.

Our key contributions can be summarized as follows:
\begin{itemize}
\item We introduce CircuitVQ, a circuit tokenizer that facilitates graph autoregressive modeling for circuit generation, based on our Circuit AutoEncoder framework;
\item Develop CircuitAR, a model trained using masked autoregressive modeling, which generates initial circuit structures conditioned on given truth tables;
\item Propose a refinement framework that integrates differentiable architecture search to produce functionally equivalent circuits guided by target truth tables;
\item Comprehensive experiments demonstrating the scalability and capability emergence of our CircuitAR and the superior performance of the proposed circuit generation approach.
\end{itemize}

% Motivation
% (a) Diffusion (Vision, Graph), Autoregressive (Language, Vision)
% (b) Circuit Generation for Predefined Setting
% (c) Neural Architecture Search for Strict Logic Equivalence

% Contribution
% (a) Circuit Tokenizer (new transformer arch, training strategy)
% (b) CircuitAR (train and gen strategies, post-ar strategy)
% (c) Extensive Evaluation including BitD (Bit Distance) for Scalability



\section{The Issy Format}\label{sec:format-issy}
The \issy input format has the key advantage that it combines two modes for specification of synthesis problems for infinite-state reactive systems: temporal logic formulas,  and two-player games, both over variables with infinite domains, such as integers or reals.  
The advantages of this mutli-paradigm specification format are two-fold.
First,  it often allows specification designers to specify requirements in a less cumbersome way.  
For example,  constraints that  depend on the system's state, or encode behaviour in different phases,  are usually easier to specify as games.  On the other hand, mission specifications such that under certain assumptions the system must eventually stabilize, or that some tasks should be carried out repeatedly, are often more concisely expressed in temporal logic.  \looseness=-1



\definecolor{codegray}{rgb}{0.5,0.5,0.5}
\definecolor{codepurple}{rgb}{0.58,0,0.82}
\definecolor{backcolour}{rgb}{0.95,0.95,0.92}

\lstdefinestyle{mystyle}{
    backgroundcolor=\color{backcolour},   
    commentstyle=\color{codegray},
    keywordstyle=\color{blue},
    numberstyle=\tiny\color{codepurple},
    basicstyle=\linespread{0.9}\ttfamily\footnotesize,
    breakatwhitespace=false,         
    breaklines=true,                 
    captionpos=b,                    
    keepspaces=true,                 
    numbers=left,                    
    numbersep=5pt,                  
    showspaces=false,                
    showstringspaces=false,
    showtabs=false,                  
    tabsize=1,
    morecomment=[l]{//},
morecomment=[s]{/*}{*/},
basewidth = {.47em}
}

\lstset{style=mystyle,  morekeywords={game,  input, state, formula, def, assert, assume, keep, from,to, with, Safety,loc}}

\begin{lstlisting}[label=lst:example,caption=Example specification in \issy format.,escapeinside={@}{@}]
input real add		  input real rem
state real load1		state real load2 	  state real rem1  		 state real rem2
formula {
 /* Assumption: From some time point on, the environment will always set the input variable add to be less than or equal to zero. */
 assume F G [add <= 0]
 /* Guarantee: From some point on,   load1 and load2 will always be zero. */     
 @\label{lst:lassert}@assert F G ([load1 = 0] && [load2 = 0])
 }
// Macros to make the specification easier to read
@\label{lst:baldef}@def balanced  = [load1 >= load2] && [load1 <= 2 * load2] 
              ||[load2 >= load1] && [load2 <= 2 * load1]
def addtoone  = [load1' = load1 + add] && [load2' = load2] 
              ||[load2' = load2 + add] && [load1' = load1]
def validrem  = [rem >= 0.1] && [rem <= load1 + 2/3 * load2]
def decrease  = [load1' = load1 - rem1'] && [rem1' + rem2' = rem]
              &&[load2' = load2 - 3/2 * rem2'] 
/* Two-player game with locations init, lbal, lrem, done and err,  and safety winning condition for the system, requiring that err is never reached. */
game Safety from init {
  loc init 1	   loc lbal 1	   	loc lrem 1	   loc done 1	  	loc err 0
  from init to done with [load1 < 0] || [load2 < 0]
  from init to lbal with [load1 >= 0] && [load2 >= 0] && keep(load1 load2)
  @\label{lst:lbalrem}@from lbal to lrem with [load1' + load2' = load1 + load2] 
  @\label{lst:lremerr}@from lrem to err  with !balanced 
  from lrem to done with balanced &&(!validrem ||([load1 = 0] && [load2 = 0]))
  @\label{lst:lrembal}@from lrem to lbal with balanced && [add > 0]  && addtoone
  from lrem to lrem with balanced && [add <= 0] && validrem && decrease
  from done to done with true 
  from err  to err  with keep(load1 load2)
 }
\end{lstlisting}

Each of the two modes of specification can potentially offer opportunities for optimization of the synthesis tools processing these specifications. 
In~\cite{HeimD25}, we showed how the translation from \rpltl{} formulas  to  games can benefit from the high-level information present in the formula in order to simplify the  game. 


\begin{figure}[b!]
\begin{center}
\grammarindent2.2cm
\grammarparsep1.1pt
\begin{grammar}

<spec> ::= (<vardecl> | <logicspec> | <gamespec> | <macro>)*
 
 \smallskip 
 
<vardecl> ::= (`input' | `state') <type> <identifier> \quad  <type>    ::= `int' | `bool' | `real'

 \smallskip 

<logicspec> ::=  `formula' `{' <logicstm>*  `}'  \quad <logicstm> ::= (`assert' | `assume') <rpltl>



\smallskip

<gamespec>  ::= `game' <wincond> `from' <identifier> `{' ( <locdef> | <transdef>)* `}' 

<wincond>  ::= `Safety' | `Reachability' | `Buechi'  | `CoBuechi' | `ParityMaxOdd'

<locdef>   ::= `loc' <identifier> [<nat>] [`with' <formula>]

<transdef>  ::= `from'  <identifier> `to' <identifier> `with' <formula>

\vspace{-.4cm}
\end{grammar}


\end{center}
\caption{An excerpt from the \issy  format.  The full description is in~\Cref{sec:format-issy-full}.}
\label{fig:issy-grammar-part}
\end{figure}





Now, we turn to an example that illustrates and motivates the main features of the \issy format.
An excerpt of the format's grammar is given in \Cref{fig:issy-grammar-part}.

\begin{example}\label{ex:simple-example}
Consider a reactive system that has to balance the loads, \texttt{load1} and \texttt{load2}, of two components.
At any point,  the environment can increase the total load,  via the environment-controlled input variable \texttt{add}.  When that happens,  the system has to re-balance the total load by appropriate partitioning. When the load does not increase,  the system has to control the throughput of each component,  state variables \texttt{rem1} and \texttt{rem2} respectively,  in accordance with the components' speeds and the total available throughput,  \texttt{rem} controlled by the environment.
The specification of this system is given in Listing~\ref{lst:example}, and consists of variable declarations,  a formula specification,   macro definitions for better readability, and the second part of the specification given as a two-player game.  


Variable declarations specify whether the variable is \texttt{\textcolor{blue}{input}} controlled by the environment, or is a \texttt{\textcolor{blue}{state}} variable controlled by the system.
The currently supported data types are \texttt{bool, int} and \texttt{real}.
The domains of variables can be further constrained in the \texttt{\textcolor{blue}{game}} specifications by additional constraints.




The \texttt{\textcolor{blue}{formula}} specification is a list \rpltl{} formulas,  prefixed by the keywords \texttt{\textcolor{blue}{assume}} and \texttt{\textcolor{blue}{assert}},  denoting constraints on the environment and system respectively.
They use temporal operators like LTL, but with quantifier-free first-order atoms 
instead of Boolean propositions.
The assumption \texttt{F G [add <= 0]} uses the temporal operators \texttt{F} (eventually) and  \texttt{G} (globally) to state that from some point on, no more load will be added by the environment.
The assert statement in line \ref{lst:lassert} requires the system to ensure, under the above assumption, that both loads eventually stabilize at zero. 
The semantics of a \texttt{\textcolor{blue}{formula}} specification is that the conjunction of the assumptions implies the conjunction of the asserts. 

The possible actions of the system and the requirement to balance \texttt{load1} and \texttt{load2} are described by the 
\texttt{\textcolor{blue}{game}} specification in Listing~\ref{lst:example}. The game has locations \texttt{init}, \texttt{lbal}, \texttt{lrem}, \texttt{done}, \texttt{err} that are local to the game, unlike variables that are global to the whole specification.
The transitions between locations in the game are defined via  quantifier-free formulas over input, state, and next-state variables (such as \texttt{load1'}).  Nondeterminism is under the control of the system.  The \issy format enables the use of macros to formula readability. 
For example,  the transition in line \ref{lst:lremerr} requires the system to transition from location  \texttt{lrem} to the unsafe location \texttt{err} if the condition \texttt{balanced} defined by the macro in line \ref{lst:baldef} is violated. 
The game has a \emph{safety} winning condition,  indicated by the keyword \texttt{\textcolor{blue}{Safety}}, and defined by the natural numbers with which the locations are labelled ($0$ indicates that \texttt{err} is unsafe, while all  labelled $1$  are safe).
\end{example}

A specification can contain multiple \texttt{\textcolor{blue}{formula}} and \texttt{\textcolor{blue}{game}} components,  interpreted conjunctively.  The semantics is a two-player game defined as the product of the games for the individual formulas and all game specifications.  
\issy requires and checks that at most one of these games has a non-safety winning condition. 

The \issy specification in \Cref{ex:simple-example} illustrates the modelling flexibility of the format. Expressing the same requirements purely in \rpltl\ or as an RPG results in a difficult to write and understand specification,  making the specification process error prone.  We believe that \issy  alleviates this problem to some extent, offering modularity and syntactic sugar constructs, and,  most importantly, unifying the temporal logic and game formats for 
infinite-state reactive systems.\looseness=-1

\vspace{-.3cm}
\paragraph{The \issy compiler and the \llissy format.}
The \issy compiler,  part of our synthesis framework,  compiles specifications in  \issy format to a low-level intermediate format called \llissy,  given in \Cref{sec:format-llissy}.
The compiler checks compliance with the syntax and gives informative error messages.
The \llissy format is easier to parse, while retaining the ability to specify both logical formulas and games. 
We envision that the development of tools  for translation from various high-level specification formats to the \llissy format will enable the seamless exchange of benchmarks and experimental comparison between different tools.
\issy also accepts input directly in \issy format, as well as the older formats  \textsf{tslmt} and \textsf{rpg}.




\section{From Temporal Formulas to Games}\label{sec:to-game}
To check the realizability of specifications and synthesize reactive programs, \issy follows the classical approach of reducing the task to solving a two-player game.
To this end, it translates the specification into a symbolic synthesis game by first translating the temporal logic formulas to games, and then building their product with the rest of the specification. 
The construction of games from the  formulas follows~\cite{HeimD25}  and provides the option to build and use a  \emph{monitor} to prune/simplify the constructed game by performing first-order and temporal  reasoning during game construction. 
More concretely, a given formula is first translated to a deterministic $\omega$-automaton using \texttt{Spot}~\cite{Duret-LutzRCRAS22}.  
Then, monitors are constructed \emph{on-the-fly}, building the product between the game obtained from the automaton and the monitor. 
The product with the monitor enhances the game with semantic information~\cite{HeimD25}, resulting in the so-called \emph{enhanced game}, which is potentially easier to solve. 
As sometimes the monitor construction causes overhead, \issy has a parameter \texttt{\textcolor{blue}{-{}-pruning}} controlling its complexity,  ranging from no monitor construction (level 0), to applying powerful deduction during its construction (level 3).\looseness=-1

The prototype \tslmtrpg~\cite{HeimD25} is restricted to the logic TSL-MT and constructs RPGs. In contrast,  the translation in \issy applies to the more general logic \rpltl, and constructs a more general form of symbolic games.  
In TLS-MT and RPGs, the system controls the state variables via a fixed finite set of possible updates, a restriction not present in \rpltl{} and the respective symbolic games. 
For example,  assertions like \texttt{x' > x} are not expressible in TSL-MT.
\looseness=-1




\section{An Acceleration-Based Solver for Infinite-State Games}\label{sec:solver}
The architecture of \issy is shown in \Cref{fig:toolchain}.  We discussed the components translating a specification to a single synthesis game in \Cref{sec:to-game}.
Now we present the game solver underlying \issy, focusing on the novel  technical developments.

\begin{figure}[t!]
\tikzstyle{interface} = [rectangle, rounded corners, minimum width=2cm, minimum height=.6cm,text centered, draw=black, fill=red!20]
\tikzstyle{io} = [trapezium, trapezium left angle=85, trapezium right angle=95, minimum width=.5cm, minimum height=.6cm, text centered, draw=black, fill=gray!20]
\tikzstyle{process} = [rectangle, minimum width=1.5cm, minimum height=.6cm, text centered, draw=blue, fill=blue!20]
\tikzstyle{ext} = [rectangle, minimum width=1.5cm, minimum height=.6cm, text centered, draw=black]
\tikzstyle{arrow} = [thick,->,>=stealth]
\tikzstyle{doublearrow} = [thick,<->,>=stealth]

\begin{center}

\begin{tikzpicture}[scale=0.75, node distance=2cm]
\node (issycomp) [process, text width=1.5cm] {\issy compiler};
\node (llissyprob) [io, text width=1.65cm, below of = issycomp, yshift=.7cm] {\phantom{aa} \llissy \newline specification};
\node (llissypars) [process, text width=1.5cm, below of = llissyprob, yshift=.7cm] {\llissy parser};

\node (spec) [io, text width=3.6cm, right of = llissypars, xshift=1.6cm] {Specification: Formula$_1,\ldots,$ Formula$_m$ Game$_1,\ldots,$ Game$_n$};
\node (gameprod) [process, text width=2cm, right of = spec, xshift=3cm] {Game product construction};
\node (enhgame) [io, text width=2.25cm, above of = gameprod, yshift=-.7cm, xshift=-.5cm] {Enhanced game};
\node (otfprod) [process, text width=2cm, above of = enhgame, yshift=-.5cm,xshift=.5cm,yshift=-.2cm] {On-the-fly product construction};

\node (formgame) [io, text width=.8cm,  above left of = otfprod, xshift=-1cm,yshift=-.6cm] {Game};
\node (monitor) [io, text width=1cm,  below left of = otfprod, xshift=-1cm, yshift=1.3cm] {Monitor};

\node (directconstr) [process, text width=2cm, left of = formgame, xshift=-1.5cm] {Formulas to games};
\node (monconstr) [process, text width=2cm, left of = monitor, xshift=-.5cm, yshift=0cm] {Formulas to monitors};

\node (game) [io, text width=1.5cm, below of = gameprod, yshift=.6cm, xshift=0cm] {Symbolic game};

\node (gameinterface) [interface, text width=4cm, left of = game, xshift=-1.5cm] {Game-solving interface};

\node (rpg) [io, text width=.5cm, left of = gameinterface, xshift=-1.25cm] {RPG};
\node (rpgpars) [ext, text width=1.5cm, above left of = rpg, xshift=-.5cm, yshift=-1cm] {RPG parser};
\node (tslmttorpg) [ext, text width=1.5cm, below left of = rpg, xshift=-.5cm, yshift=.85cm] {\tslmtrpg};

\node (solver) [process, text width=1.5cm, below of = gameinterface,  yshift=.4cm] {Game solver};

\node (result) [text width=2.5cm, below of =solver, xshift=-1cm, yshift=0cm] {\textbf{Realizable/Unrealizable}};

\node (attracc) [process, text width=2.5cm, right of = solver,  xshift = 1cm] {Attractor acceleration};
\node (solveacc) [process, text width=2.5cm, below of =attracc, yshift=1cm] {Outer fixpoint acceleration};

\node (extract) [process, text width=2cm, left of = solver,  xshift = -.5cm] {Program extraction};
\node (absprog) [io, text width=1.4cm, left of = extract,  xshift = -.7cm] {Abstract reactive program};
\node (genc) [process, text width=2cm, below of =absprog, yshift=.6cm] {C program construction};


\draw [arrow] (issycomp) -- (llissyprob);
\draw [arrow] (llissyprob) -- (llissypars);
\draw [arrow] (llissypars) -- (spec);
\draw [arrow] ($(spec.north) - (2cm,0) $) -- ($(directconstr.south) - (.8cm,0) $);
\draw [arrow] ($(spec.north) + (0.13cm,0) $) -- (monconstr);
\draw [arrow] (monconstr) -- (monitor);
\draw [arrow] (directconstr) -- (formgame);
\draw [arrow] (monitor) -- (otfprod);
\draw [arrow] (formgame) -- (otfprod);
\draw [arrow] (otfprod) -- ($(enhgame.north) + (.65,0)$);
\draw [arrow] ($(enhgame.south) + (.65,0)$) -- (gameprod);
\draw [arrow] ($(spec.east) - (0.1,0.5)$) -- ($(gameprod.west) - (0,0.5)$);
\draw [arrow] (gameprod) -- (game);
\draw [arrow] (game) -- (gameinterface);


\draw [arrow] (rpgpars) -- (rpg);
\draw [arrow] (tslmttorpg) -- (rpg);
\draw [arrow] (rpg) -- (gameinterface);

\draw [arrow] (gameinterface) -- (solver);
\draw [doublearrow] (solver) -- (attracc);
\draw [doublearrow] (solver) -- (solveacc);


\draw [arrow] (solver) -- ($(result.north) + (1.35cm,0) $);

\draw [arrow] (solver) -- (extract);
\draw [arrow] (extract) -- (absprog);
\draw [arrow] (absprog) -- (genc);

\node (formulastogame) [draw=blue,very thick,rounded corners,fit = (directconstr) (monconstr) (formgame) (monitor) (otfprod), inner sep=3pt] {};
\begin{scope}[on background layer]
\node (symbgameconstruction) [draw=blue,very thick,rounded corners,fit = (formulastogame) (spec) (gameprod),inner sep=3pt,fill=blue!10] {};
\end{scope}


\begin{scope}[on background layer]
\node (synthesisengine) [draw=blue,very thick,rounded corners,fit = (solver) (extract) (attracc) (solveacc), inner sep=3pt,fill=blue!10] {};
\end{scope}

\node (issyf) [above of=issycomp,text width=.5cm,yshift=-.7cm] {\scalebox{1.9}{\faFile*[regular]}\newline \textbf{.issy}};
\draw [arrow] (issyf) -- (issycomp);

\node (llissyf) [left of=llissypars,text width=.5cm, xshift=.4cm] {\scalebox{1.9}{\faFile*[regular]}\newline \textbf{.llissy}};
\draw [arrow] (llissyf) -- (llissypars);

\node (rpgf) [left of=rpgpars,text width=.5cm, xshift=.4cm] {\scalebox{1.9}{\faFile*[regular]}\newline \textbf{.rpg}};
\draw [arrow] (rpgf) -- (rpgpars);

\node (tslmtf) [left of=tslmttorpg,text width=.5cm, xshift=.4cm] {\scalebox{1.9}{\faFile*[regular]}\newline \textbf{.tslmt}};
\draw [arrow] (tslmtf) -- (tslmttorpg);

\node (cf) [left of=genc,text width=.5cm, xshift=.3cm] {\scalebox{1.9}{\faFile*[regular]}\newline \textbf{.c}};
\draw [arrow] (genc) -- (cf);

\node [above of = otfprod,   xshift=-1.6cm, yshift=-.2cm] {\textcolor{blue}{\textbf{Translation to symbolic game} (Sec.~\ref{sec:to-game})}};

\node [above of = attracc,   yshift=-1.1cm,xshift=-1cm] {\textcolor{blue}{\textbf{Acceleration-based solver} (Sec.~\ref{sec:solver})}};
\end{tikzpicture}

\end{center}

\vspace{-.7cm}
\caption{Architecture of \issy. Components depicted in blue and pink are part of the tool's implementation, those depicted in white are external.}
\label{fig:toolchain}
\end{figure}

The approach behind the \issy solver builds on the method proposed in~\cite{HeimD24}.
The fist main difference to the prototype \rpgsolve from~\cite{HeimD24} is that \rpgsolve accepts RPGs,  a strictly more restricted class of symbolic games.  Furthermore,  the initial version of \rpgsolve does not support parity winning conditions.   
Our \Cref{ex:simple-example} cannot be modelled as an RPG,  because the system player has the power to select any real values as next-state values for the state variables.
Furthermore,  the specification in \Cref{ex:simple-example} translates to a parity game.
\issy's solver supports a more general symbolic game model, and also implements a symbolic method for infinite-state parity games based on fixpoint computation (a lifting of the classical Zielonka's algorithm~\cite{Zielonka98}).  Thus,  \issy is able to establish the realizability of the specification in Listing~\ref{lst:example} thanks to the new techniques it implements.

The crux to this is the acceleration technique introduced in~\cite{HeimD24}.
Naive fixpoint-based game-solving diverges on this example.
\emph{Attractor acceleration}~\cite{HeimD24} uses ranking arguments to establish that by iterating some strategy an unbounded number of times through some location, a player in the game can  enforce reaching a set of target states.  In \Cref{ex:simple-example},  attractor acceleration is used within the procedure for solving the parity game to establish that (under the respective constraints on the environment) from any state satisfying the formula \texttt{balanced},  a state where both \texttt{load1} and \texttt{load2} are in the bounded interval $[\frac{3}{10},\frac{9}{10}]$ can be enforced by the system player.  This argument is formalized as what is called an \emph{acceleration lemma}~\cite{HeimD24}.  From the interval $[\frac{3}{10},\frac{9}{10}]$,  the system player can then enforce reaching in a bounded number of steps a state where  \texttt{load1} and \texttt{load2} are zero.

We developed a novel method for generating acceleration lemmas and  implemented in \issy in addition to that from~\cite{HeimD24}.
To search for acceleration lemmas,  \rpgsolve introduces uninterpreted predicates representing the lemmas' components, and collects SMT constraints asserting the applicability of the lemma.  Thus,  \rpgsolve would have to discover the formula \texttt{[load1 >= load2] \&\& [load1 <= 2 * load2]  ||[load2 >= load1] \&\& [load2 <= 2 * load1]} as part of the acceleration lemma, which it is not able to do within a reasonable timeout.
The alternative method we implemented in \issy performs analysis of the game in order to generate candidate  acceleration lemmas.
First,  it analyzes the game in order to identify variables potentially making progress in a ranking argument.  For instance,  variables that remain unchanged in the relevant game locations can  be ruled out.  
Second, the new method uses the distance to the target set of states to generate ranking arguments for candidate acceleration lemmas.  Finally,  to search for a set of states where the  respective player can enforce the decrease of the distance, it uses symbolic iteration and SMT-based formula generalization.  As demonstrated for \Cref{ex:simple-example}, and more broadly by our experimental evaluation in \Cref{sec:experiments},  this new method for generating acceleration lemmas, which we call \emph{geometric acceleration}, is successful in many cases that are challenging for \rpgsolve. In \issy, geometric attractor acceleration is enabled by default, and the method can be switched using the parameter \texttt{\textcolor{blue}{-{}-accel-attr}}.

In addition to an alternative method for generating acceleration lemmas,  the \issy solver utilizes new techniques for their localization. Building on ideas in~\cite{SchmuckHDN24}, 
we restrict the size of the sub-games used for the acceleration lemma computation and project away variables that are not relevant in the respective subgame.  Unlike~\cite{SchmuckHDN24}, where this is done for pre-computing accelerations,  in \issy these localization techniques are applied on-the-fly during the main game solving.

\issy also provides support for strategy synthesis and extraction of C programs for realizable specifications.  The latter can be extended to other target languages,  utilizing the 
generic data structure for reactive program representation in \issy.

\issy\footnote{\url{https://github.com/phheim/issy}} is implemented in Haskell with focus on modularity and extensibility,  including detailed documentation.  Using the Haskell tool Stack,  building  \issy and getting its dependencies is seamless.
The external tools used are \texttt{Spot}~\cite{Duret-LutzRCRAS22}  for translation of LTL  to automata,  the $\mu$CLP solver \texttt{MuVal}~\cite{UnnoTGK23} and the Optimal CHC solver \texttt{OptPCSat}~\cite{GuTU23}  for the monitor construction,  and \texttt{z3}~\cite{Z3} for all SMT,  formula simplification and quantifier elimination queries.




\section{Benchmarks and Evaluation}\label{sec:experiments}
\section{Experiments: Planning outperforms Heuristics}
\label{sec:experiment}

We begin our empirical demonstrations by showcasing the effectiveness of our planning framework on both synthetic and real datasets. We focus on the simplest planning algorithm, 1-step lookaheads (Algorithm~\ref{alg:complete}), and show that even basic planning can hold great promise. 
We illustrate our framework using two uncertainty quantification modules---GPs and 
\ensembles/ \ensembleplus. 

Throughout this section, we focus on evaluating the mean squared error of 
a regression model $\model$,  and develop adaptive policies that minimize uncertainty on $g(f)$ defined in~\eqref{eqn:l2-g-f}.
When GPs provide a valid model of uncertainty, 
our experiments show that our planning framework significantly outperforms other baselines. 
We further demonstrate that our conceptual framework extends to deep learning-based uncertainty quantification methods such as  \ensembleplus while highlighting computational challenges that need to be resolved in order to scale our ideas. 
For simplicity, we assume a naive predictor, i.e., $\psi(\cdot) \equiv 0$. However, we emphasize that this problem is just as complex as if we were using a sophisticated model $\psi(.)$. The performance gap between the algorithms 
primarily depends
on the level  of uncertainty in our prior beliefs.

To evaluate the performance of our algorithm, we benchmark it against several baselines. 
%Active learning baselines use an acquisition function $\ac$ to select points that have the highest   function value: $X\opt_t \in \argmax_{X \in \xpoolj{t}} \ac({X})$ at every step $t$. These methods may also need an UQ module, which we simply use the same UQ module as in our algorithm, and it  outputs $V(X)$ that measures the the uncertainty of each point $X \in \xpoolj{t}$.
Our first set of baselines are from active learning~\citep{AggarwalKoGuHaPh14}:
\\ % \noindent\textbf{Active Learning Heuristics:} 
\textbf{(1)} 
\textsf{Uncertainty Sampling (Static):}  In this approach, we query the samples for which the model is least certain about. Specifically, we estimate the variance of the latent output $f(X)$ for each $X \in \xpool$ using the UQ module and select the top-$K$ points with the highest uncertainty. \\
\textbf{(2)} \textsf{Uncertainty Sampling (Sequential):} This is a greedy heuristic that sequentially selects the points with the highest uncertainty within a batch, while updating the posterior beliefs using pseudo labels from the current posterior state. Unlike \textsf{Uncertainty Sampling (Static)}, this method takes into account the information gained from each point within batch, and hence tries to diversify the selected points within a batch. 

 
We also compare our approach to the  \textbf{(3)} \textsf{Random Sampling}, which selects each batch uniformly at random from the pool. Additionally, we compare solving the planning problem using  \textsf{REINFORCE}-based policy gradients with   $\mathsf{Smoothed\text{-}Autodiff}$ policy gradients.\footnote{Our code repository is available at
  \url{https://github.com/namkoong-lab/adaptive-labeling}.}
%Detailed experimental setups are provided in Section \ref{sec:details-experiments}.

%We repeat all experiments with 10 random seeds.




\begin{figure}[t]
\centering
\begin{minipage}[b]{0.49\textwidth}
\centering
\includegraphics[width=\textwidth, height=5cm]{figures/original_scale/Var_of_l_2_loss.pdf}
\caption{(Synthetic data) Variance of mean squared loss evaluated through the posterior belief $\mu_t$ at each horizon $t$. This is the objective that policy gradient methods like \textsf{REINFORCE} and $\ouralgo$ optimizes. 1-step lookaheads are surprisingly effective even in long horizons.}
\label{fig:var-l2-sim}
\end{minipage}
\hfill
\begin{minipage}[b]{0.49\textwidth}
\centering \includegraphics[width=\textwidth, height=5cm]{figures/original_scale/Error_of_estimated_model_l_2_loss.pdf}
\caption{(Synthetic data) Error between MSE calculated based on collected data $\mc{D}^{0:T}$ vs. population oracle MSE over $\mc{D}_{\rm eval} \sim P_X$. Reducing uncertainty over posteriors directly leads to better OOD evaluations. 1-step lookaheads significantly outperform active learning heuristics in small horizons.}
\label{fig:mean-l2-sim}
\end{minipage}
%\caption{Simulated data for GPs}
%\label{fig:both_plots}
\end{figure}

\subsection{Planning with Gaussian processes}
\label{sec:experiment-plan-GP}
We now briefly describe the data generation process for the GP experiments,  deferring a more detailed discussion of the dataset generation to Section~\ref{sec:details-experiments}. 
We use both the synthetic data and the real data to test our methodology.
For the \emph{simulated data},  we construct a setting where the general population is distributed across \emph{51 non-overlapping clusters} while the initial labeled data $\dtrain$ just comes from one cluster. In contrast, both $\dpool \defeq (\xpool,\ypool),\deval \defeq (\xeval,\yeval)$ are generated   from all the clusters. 
We begin with a low-dimensional scenario, generating a one-dimensional regression setting using a GP. %Gaussian Process (GP).
Although the data-generating process is not known to the algorithms,  we assume that the GP hyperparameters are known to all the algorithms
to ensure fair comparisons. This can be viewed as a setting where our prior is well-specified, allowing us to isolate the effects
of different policy optimization approaches
 without any concerns about the misspecified priors. We select $10$ batches, each of size $K=5$ across $T = 10$ time horizons.

To examine the robustness of our method against the distributional assumptions made  in the simulated case, we then move to a real dataset where the correct prior is not known. We simulate selection bias from the eICU dataset~\citep{PollardJoRaCeMaBa18}, which contains real-world patient data with in-hospital mortality outcomes. 
We conduct a $k$-means clustering to generate 51 clusters and then select data from those clusters. We view this to be a credible replication of practice, as severe distribution shifts are common due to selection bias in clinical labels.  To convert the binary mortality labels into a regression setting, we train a  random forest classifier and fit a GP on predicted scores, which serves as the UQ module for all the algorithms. As before, the task is to select 10 batches, each consisting of 5 samples, across 10 time horizons.

 In Figures~\ref{fig:var-l2-sim} and~\ref{fig:mean-l2-sim}, we present results for the simulated data. 
Figure~\ref{fig:var-l2-sim} shows the variance of $\ell_2$ loss, and Figure~\ref{fig:mean-l2-sim} presents the error in the estimated $\ell_2$ loss using $\mu_t$ (relative to true $\ell_2$ loss, that is unknown to the algorithm). 
As we can see from these plots, our method one-step lookahead  gives substantial improvements  over active learning baselines and random sampling. In addition,
compared to the one-step lookahead planning approach using \textsf{REINFORCE}-based policy gradients, 
we observe that $\mathsf{Smoothed\text{-}Autodiff}$-based policy gradients provide significantly more robust performance over all horizons.

In Figures~\ref{fig:var-l2-real}~and~\ref{fig:mean-l2-real}, we observe similar findings on the eICU data. We see that planning policies (\textsf{REINFORCE} and $\mathsf{Smoothed\text{-}Autodiff}$) consistently outperform other heuristics by a large margin.  Active learning baselines perform poorly in these small-horizon batched problems and can sometimes be even worse than the random search baselines.  Overall, our results show the importance of careful planning in adaptive labeling for reliable model evaluation. 

We offer some intuition as to why one-step lookahead planning may outperform other heuristic algorithms. 
 First,  \textsf{Uncertainty sampling (Static)} while myopically selects the
 top-$K$ inputs with the highest uncertainty, it fails to consider 
the overlap in information content among the ``best” instances; see \citep{AggarwalKoGuHaPh14} for more details. 
In other words,  it might acquire points from the same region with high uncertainty while failing to induce diversity among the batch.
Although \textsf{Uncertainty Sampling (Sequential)} somewhat addresses the issue of information overlap, a significant drawback of 
this algorithm
is the disconnect between the objective we aim to optimize and the algorithm. For example, it might sample from a region with high uncertainty but very low density. 

\begin{figure}[t]
\centering
\begin{minipage}[b]{0.48\textwidth}
\centering
\includegraphics[width=\textwidth, height=5cm]{figures/original_scale/Var_of_l_2_loss_real.pdf}
\caption{(Real-world eICU data) Variance of mean squared loss evaluated through the posterior belief $\mu_t$ at each horizon $t$. Even 1-step lookaheads are extremely effective planners, and auto-differentiation-based pathwise policy gradients provide a reliable optimization algorithm based on low-variance gradient estimates.}
\label{fig:var-l2-real}
\end{minipage}
\hfill
\begin{minipage}[b]{0.48\textwidth}
\centering \includegraphics[width=\textwidth, height=5cm]{figures/original_scale/Error_of_estimated_model_l_2_loss_real.pdf}
\caption{(Real-world eICU data) Error between MSE calculated based on collected data $\mc{D}^{0:T}$ vs. population oracle MSE over $\mc{D}_{\rm eval} \sim P_X$. Reducing uncertainty over posteriors directly leads to better OOD evaluations. Our method significantly outperforms active learning-based heuristics, and random sampling.}
\label{fig:mean-l2-real}
\end{minipage}
%\caption{Real data for GPs}
\end{figure}
 
%\vspace{-1.5cm}
% \begin{wrapfigure}{r}{.32\columnwidth}
%   \vspace{-.5cm} 
%   \centering
% \includegraphics[scale=.29]{figures/Var of l2l_2 loss.pdf}
%   \vspace{-0.2cm}
%   \caption{Results of GP}
% \label{fig:var-l2-gp}
%   \vspace{-0.1cm}
% \end{wrapfigure}


% Attempts have been made  in the past to address these  drawbacks heuristically  (see \citep{AggarwalKoGuHaPh14}). We give a unified computational framework while approaching the problem in a more principled manner and solving it more optimally.




\subsection{Planning with  neural network-based uncertainty quantification methods ($\ensembleplus$)}


We now provide a proof-of-concept that shows the generalizability of our conceptual framework  to the deep learning-based UQ modules, specifically focusing on $\ensembleplus$ due to their previously observed superior performance~\citep{OsbandWenAsDwIbLuRo23}. Recall that implementing our framework with deep learning-based UQ modules  requires us to retrain the model across multiple possible random actions $\bm{a}(\theta)$ sampled from the current policy $\pi_\theta$.
This requires significant computational resources, in sharp contrast to the GPs where the posteriors are in closed form and can be readily updated and differentiated. 

Due to the computational constraints, we test $\ensembleplus$ on a toy setting to demonstrate the generalizability of our framework. We consider a setting where the general population consists of four clusters, while the initial labeled data only comes from one cluster. Again we generate data using GPs.  The task is to select a batch of 2 points in one horizon. We detail the $\ensembleplus$ architecture in Section \ref{sec:details-experiments}, and we assume prior uncertainty to be large (depends on the scaling of the prior generating functions). 
The results are summarized in the Table~\ref{tab:UQ_ensemble}.

% \begin{table}[H]
% \vspace{-10pt}
% \caption{Performance under \ensembleplus as UQ module}
%     \centering
%     \begin{tabular}{|m{3cm}|m{2.5cm}|m{2cm}|} 
%     \hline
%       Algorithm   & Variance of $\loss_2$ loss estimate & Error of $\loss_2$ loss estimate  \\ \hline Random Sampling 
%          & $1710.9 \pm 1352.1$ & $8.67\pm6.62$ 
%       \\ \hline \ouralgo & $1.30 \pm 0.68$ & $0.91\pm0.25$ \\ \hline
%     \end{tabular}
%     \label{tab:UQ_ensemble}
%     %\vspace{-10pt}
% \end{table}




\begin{table}[h]
\vspace{-10pt}
\caption{Performance under \ensembleplus as the UQ module}
\centering
\begin{tabular}{|l|l|l|}
\hline
Algorithm   & Variance of $\loss_2$ loss estimate & Error of $\loss_2$ loss estimate  \\
\hline
\textsf{Random sampling} & 7129.8 $\pm$ 1027.0 & 136.2 $\pm$ 8.28 \\ \hline
\textsf{Uncertainty sampling (Static)} & 10852 $\pm$ 0.0 & 162.156 $\pm$ 0.0 \\ \hline
\textsf{Uncertainty sampling (Sequential)} & 8585.5 $\pm$ 898.9 & 144 $\pm$ 6.93 \\ \hline
\textsf{REINFORCE} & 1697.1 $\pm$ 0.0 & 45.27 $\pm$ 0.0 \\ \hline
\ouralgo & 1697.1 $\pm$ 0.0 & 45.27 $\pm$ 0.0 \\ \hline
\end{tabular}
%\caption{Comparison of different algorithms based on variance   and   error in $\ell_2$ loss estimation with Ensemble $+$ as the UQ module. Our results demonstrate that {\ouralgo} and REINFORCE outperformthe other active learning based heuristics, confirming the benefits of our MDP formulation for the adaptive labeling problem, as also demonstrated in Section 4.\\
%\footnotesize{Experimental details: We use Gaussian Processes as our data generating process, GP parameters are the same as in Section D.3.  The task is to select a batch of 2 points along one horizon.The marginal distribution $p_X$ has 4 \textit{non-overlapping} clusters. Initial data comes from one cluster, while pool and evaluation points comes from all the clusters. We have $20$ initial labeled data points, $10$ pool points, and $252$ evaluation points.  Training procedures are similar to the one in Section D.3.} }
\label{tab:UQ_ensemble}
\end{table}



% We faced  issues in scaling up these experiments which will be our focus in the future. 





% \begin{itemize}
%     \item Posteriors should be consistent. Two dimensions: even with less training,  
%     \item the inference should be  fast enough
% \end{itemize}


% Potential research directions for uncertainty quantification

% In this section we consider a simple setting We consider a simpler setting and 


% For synthetic dataset generation, we use ...... For real datasets, we use ...... We compare our methodolgy to several baselines ()    This Section is structured as follows:
% \begin{itemize}
%     \item \textbf{GPs, square loss objective} (Section \ref{}): 
%     %the broad aim of the experiments  in this section is to isolate the performance of our methodology without any concerns for the inefficiencies induced due to a mis-specified prior or imperfect posterior inference. To accomplish this we generate synthetic datasets using GPs (detailed later). We use the well specified prior (GPs - with same hyperparameter setting) as our UQ module.   
%      As GPs provide differentaible posterior inference - any errors induced due to imperfect posterior updates are also isolated. We note that under this setting
%      \item In Section\ref{} we demonstrate why our methodology performs better than other baselines - by devising various synthetic experiments ()
%     \item  \textbf{UQ Benchmarking }(Section \ref{}): Before diving into the experiments using $\ensembleplus$ and ENNs,  we showcase our benchmarking experiments in Section \ref{}. We use real datasets We observe that ENNs perform better
%      \item \textbf{Ensemble $+$}, objective: recall, accuracy
%     \item \textbf{ENN}, objective: recall, accuracy
% \end{itemize}




% In Section {}, we test 
% \subsection{Experimental details}

% \begin{itemize}
%     \item UQ methodologies - GPs, ENNs
%     \item Objectives - Recall,  ATE
%     \item Datasets - ATE-synthetic datasets, Recall-synthetic, real datasets
%     \item Baselines - 
%     \begin{itemize}
%         \item Random sampling
%         \item Active learning - Uncertainty based sampling - In regression setting almost all of the 
%         \item Myopic greedy - Greedy Batch based sampling
%         \item Policy Gradient
%     \end{itemize}
    
% \end{itemize}

% \subsection{Experiments}
%     \begin{itemize}
%     \item GPs with square loss
%     \item Benchmarking ENN
%         \item ENNs with ATE
%         \item ENNs with Recall
%     \end{itemize}

% \subsection{Benefits over other algorithms - intuition and experiments}

%Active learning - Myopic greedy / Don't rely on the objective rather some entropy version.


%%% Local Variables:
%%% mode: latex
%%% TeX-master: "main"
%%% End:



\bibliographystyle{splncs04}
\bibliography{main.bib}

\newpage

\appendix

\section{Full Description of the \issy Format}\label{sec:format-issy-full}
\subsubsection{\issy  specification}

\grammarindent2.2cm

\begin{grammar}

<spec> ::= (<vardecl> | <logicspec> | <gamespec> | <macro>)*
 
\end{grammar}

\subsubsection{Variable Declarations}


\begin{grammar}

<vardecl> ::= (`input' | `state') <type> <identifier> 

 <type>    ::= `int' | `bool' | `real'

\end{grammar}

\subsubsection{Formula Specifications}

\begin{grammar}

<logicspec> ::=  `formula' `{' <logicstm>$^*$  `}'  

 <logicstm> ::= (`assert' | `assume') <rpltl>

<rpltl>    ::=  <atom> \alt `(' <rpltl> `)' \alt <uopt> <rpltl> \alt <rpltl> <bopt> <rpltl>

<uopt>    ::= `!' | `F' | `X' | `G'

<bopt>    ::= `&&' | `||' | `->' | `<->' | `U' | `W' | `R'

\end{grammar}

\noindent
with precedence of the logical operators as  in TLSF.


\subsubsection{Game Specifications}

\begin{grammar}

<gamespec>  ::= `game' <wincond> `from' <identifier> `{' ( <locdef> | <transdef>)*`}' 

<wincond>  ::= `Safety' | `Reachability' | `Buechi'  | `CoBuechi' | `ParityMaxOdd'

<locdef>   ::= `loc' <identifier> [<nat>] [`with' <formula>]

<transdef>  ::= `from'  <identifier> `to' <identifier> `with' <formula>


\medskip

<formula>    ::=  <atom>  \alt '(' <formula> ')' \alt <uop> <formula> \alt <formula> <bop> <formula> 

<uop>      ::=  `!' 

<bop>      ::=  `&&' | `||' | `->' | `<->'

\end{grammar}
\noindent
with precedence (from high to low):\\
\textsf{ \{!\} > \{\&\&\} > \{||\} > \{-> (ra)\} > \{<-> (ra)\}} 


\subsubsection{Atomic Predicates}

\begin{grammar}

<atom>   ::=  <apred> \alt <bconst> \alt  <identifier>['''] \alt `havoc' `('<identifier>* `)' \alt `keep' `(' 
<identifier>* `)'

<bconst>  ::= `true'  | `false'

<apred>   ::= `[' <pred> `]'

<pred>    ::= <const> \alt <identifier>['''] \alt `(' <pred> `)' \alt <auop> <pred> \alt <pred> <abop> <pred>

<const>   ::= <nat>   | <rat>

<auop>    ::= `*' | `+' | `-' | `/' | `mod' | `=' | `<' | `>'| `<=' | `>='

<abop>   ::= `-' | `abs'

\end{grammar}

\noindent
with precedence (from high to low):\\
\textsf{    \{abs\} > \{*, /, mod\} > \{+, -\} > \{<, >, =, <=, >=\}}


\subsubsection{Macros}

\begin{grammar}
<macro>  ::= `def' <identifier> `=' <formula> | <apred>
\end{grammar}

\noindent
Note: macros can be used in all $\langle \mathit{rpltl}\rangle$,  $\langle \mathit{formula}\rangle$, and $\langle\mathit{pred}\rangle$. However, for usage in $\langle\mathit{pred}\rangle$ the marco term has to be a single predicate term.

\subsubsection{Identifiers and Numerical Constants}

\begin{grammar}
<identifier>      ::= <alpha> (<alpha> | <digit> | `_')*

<nat>             ::= <digit>+

<rat>            ::= <digit>+ '.' <digit>+
\end{grammar}

\subsubsection{Comments}
\begin{itemize}
\item single line  \textsf{/ /}
\item  multi-line \textsf{/ *}
\end{itemize}

\noindent
Comments cannot be nested.


\newpage

\section{The LLissy Format}\label{sec:format-llissy}

In order to be easy to parse, readable with reasonable effort, and to be similar to the SMTLib-format, \llissy uses s-expressions.

Only single line comments exist which start with \textsf{';'} and span to the end of the line. 
Newlines are \textsf{'\textbackslash r\textbackslash n', '\textbackslash n \textbackslash r', ' \textbackslash r'} and \textsf{'\textbackslash n'}. 
However, when generating \llissy automatically \textsf{'\textbackslash n'} should be used. 
Similarly \textsf{' '} (Space) and \textsf{'\textbackslash t'} (Tabs) are both non-newline white-spaces. 
However, only \textsf{' '} should be used upon generation. 
The following productions define identifiers and natural numbers.
\begin{grammar}
<ALPHA> ::= `a'...`z' | `A'...`Z' 

<DIGIT> ::= `0'...`9'

<ID>    ::= <ALPHA> (<ALPHA> | <DIGIT> | `_')*

<PID>   ::= <ID> ['$\sim$']

<NAT>   ::=  <DIGIT>+

<RAT>   ::=  <DIGIT>+ `.' <DIGIT>+
\end{grammar}

\noindent
Note that all of these should be parsed greedily until a white-space, '(', ')', or the end-of-file occurs.

A \llissy specification consists of lists of variable declarations, formula specifications and game specifications. The variables declarations include all variables used in all games and formulas. 
The formula and game specifications are interpreted conjunctively. 
However, at most one game or formula can be a non-safety game or non-safety formula.

\begin{grammar}
<SPEC> ::= `(' `(' <VARDEC>* `)' `(' <FSPEC>* `)' `(' <GSPEC>*  `)' `)'
\end{grammar}

\noindent
A variable declaration declares an input or state variable and its respective type
\begin{grammar}
<VARDEC>  ::= `(' `input' <TYPE> <ID> `)' | `(' `state' <TYPE> <ID> `)'

<TYPE>   ::=  `Int' | `Bool' | `Real'
\end{grammar}

A formula specification is a pair of assumption and guarantee lists. Each element is an RP-LTL formula.
The assumptions come first, and each of the two lists is interpreted as a conjunction. 
\begin{grammar}
<FSPEC>   ::= `(' `(' <FORMULA>* `)' `(' <FORMULA>* `)' `)'

<FORMULA> ::= `(' `ap' <TERM>`')' \alt `(' <UOP> <FORMULA> `)' \alt `(' <BOP> <FORMULA> <FORMULA> `)' \alt (<NOP> <FORMULA>*)

<UOP>     ::= `X' | `F' | `G' | `not'

<BOP>     ::= `U' | `W' | `R'

<NOP>     ::= '`and' | `or'
\end{grammar}

A game specification consists of a list of location definitions, transition definitions from one location to another location, and an objective definition.
The objective defines the initial location and the winning condition. Each location is annotated with a natural number. For Safety, Reachability, Buechi, and CoBuechi a location is safe, target, Buechi accepting, coBuechi accepting iff the number is greater than zero. 
For ParitMaxOdd the number is the color in the parity game.

\begin{grammar}
<GSPEC>    ::= `(' `(' <LOCDEF>* `)' `(' <TRANSDEF>* `)' <OBJ> `)'

<LOCDEF>   ::= `(' <ID> <NAT> <TERM> `)'

<TRANSDEF> ::= `(' <ID> <ID> <TERM> `)'

<OBJ>      ::= `(' <ID> (`Safety' | `Reachability ' | `Buechi' | `CoBuechi' \newline \phantom{aaaaaaaaaaaaaaaa} | `ParityMaxOdd') `)'
\end{grammar}

A term is basically like in the SMT-Lib-2 format without quantifiers, lambda, and let expressions. Similar rules for typing apply.
Only variables declared initially are allowed to be free variables, and additionally primed version (with $\sim$) of the state variables.

\begin{grammar}
<TERM>   ::= `(' <OP> <TERM>* `)' | <PID> | <CONSTS>

<OP>     ::= `and ' | `or' | `not' | `ite' | `distinct' | `=>' |
         `=' | `<' | `>'| `<=' | `>=' |
         `+' | `-' | `*' | `/' | `mod' | `abs' | `to_real' 

<CONSTS> ::= <RAT> | <NAT> | `true' | `false'
\end{grammar}





\end{document}
