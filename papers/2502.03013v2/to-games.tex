To check the realizability of specifications and synthesize reactive programs, \issy follows the classical approach of reducing the task to solving a two-player game.
To this end, it translates the specification into a symbolic synthesis game by first translating the temporal logic formulas to games, and then building their product with the rest of the specification. 
The construction of games from the  formulas follows~\cite{HeimD25}  and provides the option to build and use a  \emph{monitor} to prune/simplify the constructed game by performing first-order and temporal  reasoning during game construction. 
More concretely, a given formula is first translated to a deterministic $\omega$-automaton using \texttt{Spot}~\cite{Duret-LutzRCRAS22}.  
Then, monitors are constructed \emph{on-the-fly}, building the product between the game obtained from the automaton and the monitor. 
The product with the monitor enhances the game with semantic information~\cite{HeimD25}, resulting in the so-called \emph{enhanced game}, which is potentially easier to solve. 
As sometimes the monitor construction causes overhead, \issy has a parameter \texttt{\textcolor{blue}{-{}-pruning}} controlling its complexity,  ranging from no monitor construction (level 0), to applying powerful deduction during its construction (level 3).\looseness=-1

The prototype \tslmtrpg~\cite{HeimD25} is restricted to the logic TSL-MT and constructs RPGs. In contrast,  the translation in \issy applies to the more general logic \rpltl, and constructs a more general form of symbolic games.  
In TLS-MT and RPGs, the system controls the state variables via a fixed finite set of possible updates, a restriction not present in \rpltl{} and the respective symbolic games. 
For example,  assertions like \texttt{x' > x} are not expressible in TSL-MT.
\looseness=-1


