\newcommand{\macroissycomp}{
\begin{tikzpicture}
\begin{axis}[%
    xmin=-2, xmax=1202, xmode=log, 
    ymin=-2, ymax=1202, ymode=log, 
    xlabel={Runtime (sec.) geom. accel. }, 
    ylabel={Runtime (sec.) unint. accel.  }, 
    width=\textwidth, 
    height=\textwidth%
    ]%
    \draw[gray] (axis cs:-2,-2) -- (axis cs:1202,1202);
    \addplot[only marks, mark=x, color=blue]  table [x=geom, y=unif, col sep=comma] {./data/plot-comp-accel.csv};
\end{axis}
\end{tikzpicture}}

\newcommand{\macrosweapcomp}{
\begin{tikzpicture}
\begin{axis}[%
    xmin=-2, xmax=1202, xmode=log, 
    ymin=-2, ymax=1202, ymode=log, 
    xlabel={Runtime (sec.) Issy (best)}, 
    ylabel={Runtime (sec.) sweap}, 
    width=\textwidth, 
    height=\textwidth%
    ]%
    \draw[gray] (axis cs:-2,-2) -- (axis cs:1202,1202);
    \addplot[only marks, mark=x, color=blue]  table [x=issy, y=sweap, col sep=comma] {./data/plot-comp-sweap.csv};
\end{axis}
\end{tikzpicture}}

\newcommand{\macrorpgplot}{
\begin{tikzpicture}
\begin{axis}[width=0.85\textwidth, height=3.6cm, ylabel near ticks, legend pos=outer north east, xmin=-5, xticklabel = {$\pgfmathprintnumber[precision=0]{\tick}s$}]
\addplot[., red] table[y=Cnt,x=Time,col sep=comma] {data/plot-rpgs-issy.csv}; 
\addplot[., blue] table[y=Cnt,x=Time,col sep=comma] {data/plot-rpgs-rpgsolve.csv}; 
\addplot[., brown] table[y=Cnt,x=Time,col sep=comma] {data/plot-rpgs-rpgstela.csv}; 
\addplot[., green] table[y=Cnt,x=Time,col sep=comma] {data/plot-rpgs-muval.csv}; 
\addplot[., gray] table[y=Cnt,x=Time,col sep=comma] {data/plot-rpgs-sweap.csv}; 
\addlegendentry{\scriptsize Issy (sel.)}
\addlegendentry{\scriptsize rpgsolve}
\addlegendentry{\scriptsize rpgstela}
\addlegendentry{\scriptsize muval}
\addlegendentry{\scriptsize (sweap)}
\end{axis}
\end{tikzpicture}}


\newcommand{\macrotslmtplot}{
\begin{tikzpicture}
\begin{axis}[width=0.85\textwidth, height=3.6cm, ylabel near ticks, legend pos=outer north east, xmin=-5,  xticklabel = {$\pgfmathprintnumber[precision=0]{\tick}s$}]
\addplot[., red] table[y=Cnt,x=Time,col sep=comma] {data/plot-tslmt-issy.csv}; 
\addplot[., blue] table[y=Cnt,x=Time,col sep=comma] {data/plot-tslmt-tslmt2rpg.csv}; 
\addplot[., gray] table[y=Cnt,x=Time,col sep=comma] {data/plot-tslmt-sweap.csv}; 
\addplot[., brown] table[y=Cnt,x=Time,col sep=comma] {data/plot-tslmt-raboniel.csv}; 
\addlegendentry{\scriptsize Issy (sel.)}
\addlegendentry{\scriptsize tslmt2rpg}
\addlegendentry{\scriptsize (sweap)}
\addlegendentry{\scriptsize Raboniel}
\end{axis}
\end{tikzpicture}}

\newcommand{\macrosummarytable}{
\begin{tabular}{|l||r|r|r|r|} \hline
Set      & \cref{figure:rpg-compare} & \cref{figure:tslmt-compare} & New \issy{} & \cite{HeimD24} hard 
\\\hline\hline
$\Sigma$ &  95 & 102 &  56 &   8 \\\hline\hline
Geo.     &  \textbf{89} &  62 & 40 & 4 \\\hline 
Un.      &  62 &  36 & 40 & 2 \\\hline 
Geo.-P.  &   - &  \textbf{73} & \textbf{42} & \textbf{5} \\\hline
Un.-P.   &   - &   - & 41 & 0 \\\hline 
\end{tabular}
}


\begin{figure}[t!]
\begin{minipage}[t]{0.44\textwidth}
\centering
\macrorpgplot
\vspace{-.2cm}

\captionof{figure}{\footnotesize Solved instances of RPGs from \cite{HeimD24,SchmuckHDN24,AzzopardiPSS24} within given time (in sec.).}\label{figure:rpg-compare}
\end{minipage}
\hspace{0.05\textwidth}
\begin{minipage}[t]{0.47\textwidth}
\centering
\macrotslmtplot
\vspace{-.2cm}

\captionof{figure}{\footnotesize Solved instances of TSL-MT from \cite{MaderbacherB22,AzzopardiPSS24,HeimD25} within given time (in sec.)}\label{figure:tslmt-compare}
\end{minipage}
\end{figure}

We evaluated \issy{} experimentally, comparing to \raboniel{}, sweap\footnote{%
\url{https://github.com/shaunazzopardi/sweap}. %
We used the commit 72d118 (2024-09-10), which was the last commit on the main branch during evaluation (2025-01-30). %
It differs from the method described in~\cite{AzzopardiPSS24} in not having the memory optimization techniques described in the first paragraph of Section 7~\cite{AzzopardiPSS24}. According to personal communication with the authors, a newer version of sweap (on a development branch at the time of our evaluation on 2025-01-30) implements those.}
, \muval{},  \rpgsolve{} and \tslmtrpg{},  thus covering all types of techniques. 
We did not compare to \temos{}, as past experiments~\cite{HeimD24,HeimD25} show that it is outperformed by \raboniel{}.
Other tools are either not available, unable to build, or do not accept input files.

For \issy{} we use four configurations: with the novel geometric acceleration or the existing acceleration with uninterpreted predicates,  and with or without monitor-based simplification (\texttt{\textcolor{blue}{-{}-pruning}} 2 or 0, resp.) when applied to specifications with formulas.
The later is because the effectiveness of pruning varies~\cite{HeimD25}.


\begin{wraptable}{r}{0.5\textwidth}
\vspace{-7mm}
\caption{Benchmark instances solved by the four different \texttt{Issy} configurations.}\label{table:summary-data}
\vspace{-2mm}
\macrosummarytable
\vspace{-7mm}
\end{wraptable}

We used an extensive set of benchmarks\footnote{\url{https://github.com/phheim/infinite-state-reactive-synthesis-benchmarks}} (contributions welcome!) containing the RPG benchmarks from~\cite{HeimD24,SchmuckHDN24} and the TSL-MT benchmarks~\cite{MaderbacherB22,HeimD25}, some of which can not be solved by existing tools.
Furthermore, we included the benchmarks created by the authors of sweap~\cite{AzzopardiPSS24} (in their format) as well as their manually encoded versions in the RPG and TSL-MT formats.
We also created 50 new benchmarks in the new \issy{} format which combine formuals and games and can only be used by \issy{}.


We partitioned the set of benchmarks according to the type of specifications (games or temporal formulas) the tools are applicable to  according to \Cref{table:compare-input} (with \muval{} applied on RPGs via an automatic encoding of the games as fixpoint equations).
All experiments were run on AMD EPYC processors,  with one core, 4GB of memory, and 20 minutes wall-clock-time for each benchmarking run.

\Cref{figure:rpg-compare} and \Cref{figure:tslmt-compare} show the comparisons on 95 RPG and 102 TSL-MT specifications, respectively.
For \issy{} we show the best time for \emph{checking realizability} for each benchmark across the four different configurations.  As shown in \Cref{table:summary-data},  the best time for \issy is usually with geometric acceleration.
We ran sweap only on the benchmarks to which it is applicable and are available in its own format, which uses a different formalism.
Therefore,  we show additionally in  \Cref{figure:issy-vs-sweap} the comparison to sweap only on the set of those 147 benchmarks. 
We note that sweap is performing synthesis, while the results for \issy are for checking realizability.

The evaluation results demonstrate that  \issy{} mostly outperforms the existing prototypes,  and has matured well beyond the prototypes it stems from.

In addition to \Cref{table:summary-data}, \Cref{figure:accel-compare} provides a detailed comparison between the new geometric and the existing uninterpreted-predicate-based acceleration methods (without pruning) on all benchmarks. 
It shows that geometric acceleration is effective, without making the existing  acceleration method obsolete.
We also ran \issy in synthesis mode (\texttt{\textcolor{blue}{-{}-synt}}) with geometric acceleration, especially as synthesis for uninterpreted-predicate-based acceleration is known to be hard~\cite{HeimD24}. 
Out of the 130 benchmarks that geometric accleration  determined to be realizable,  \issy could synthesize C-programs for 106 of them within the given resource bounds.
This difference stems from the fact that \issy does heavy simplifications and might to need to synthesize Skolem functions.

\begin{figure}[t!]
\hspace{0.025\textwidth}
\begin{minipage}[t]{0.39\textwidth}
\centering
\macrosweapcomp
\vspace{-.2cm}

\captionof{figure}{Comparison of sweap and \issy on the sweap benchmarks~\cite{AzzopardiPSS24}, manually encoded in~\cite{AzzopardiPSS24}  as RPGs or TSL-MT.}\label{figure:issy-vs-sweap}
\end{minipage}
\hspace{0.05\textwidth}
\begin{minipage}[t]{0.39\textwidth}
\centering
\centering
\macroissycomp

\vspace{-.2cm}
\captionof{figure}{Comparison of the existing and new attractor acceleration in \issy on all RPG, TSL-MT, and new \issy benchmarks.}\label{figure:accel-compare}
\hspace{0.025\textwidth}
\end{minipage}

\end{figure}

In summary,  the results show \issy's competitiveness with the state of the art. \issy's comprehensive framework, together with the public collection of benchmarks\footnote{\url{https://github.com/phheim/infinite-state-reactive-synthesis-benchmarks}} provide a basis for further development of techniques and tools.
