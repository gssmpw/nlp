% This is samplepaper.tex, a sample chapter demonstrating the
% LLNCS macro package for Springer Computer Science proceedings;
% Version 2.21 of 2022/01/12
%
\documentclass[preprint,12pt,sort&compress]{elsarticle}%
\usepackage[T1]{fontenc}
% T1 fonts will be used to generate the final print and online PDFs,
% so please use T1 fonts in your manuscript whenever possible.
% Other font encondings may result in incorrect characters.
%
%\usepackage{cite}
\usepackage{listings}
\usepackage{amsmath,amssymb,amsfonts}
\usepackage{algorithmic}
\usepackage{graphicx}
\usepackage{textcomp}
\usepackage{booktabs}
\usepackage{threeparttable}
\usepackage{amsthm}
\usepackage{url}
\usepackage{makecell}
\usepackage[table]{xcolor}
\usepackage{pifont}
\usepackage{xspace}
\usepackage{float}
\usepackage{subcaption}
\usepackage{array}
\usepackage{hyperref}
\newtheorem{definition}{Definition}[section]
\newcommand{\cmark}{\ding{51}}%
\newcommand{\xmark}{\ding{55}}%
\newcommand{\amark}{\ding{107}}%
\newcommand{\lnmark}{\ding{116}}%
\newcommand{\pmark}{\ding{67}}%
\newcommand{\hmark}{\ding{168}}%
\newcommand{\missingcell}{\cellcolor{gray!25}}
\newcolumntype{R}[1]{>{\raggedleft\arraybackslash}p{#1}}
% Used for displaying a sample figure. If possible, figure files should
% be included in EPS format.
%
% If you use the hyperref package, please uncomment the following two lines
% to display URLs in blue roman font according to Springer's eBook style:
%\usepackage{color}
%\renewcommand\UrlFont{\color{blue}\rmfamily}
%
\begin{document}
%
\begin{frontmatter}

\title{Entropy Collapse in Mobile Sensors: The Hidden Risks of Sensor-Based Security}

\author[1]{Carlton Shepherd\corref{cor1}\fnref{fn1}}
\ead{carlton.shepherd@ncl.ac.uk}
\author[1]{Elliot A.\ J.\ Hurley}
\ead{e.a.j.hurley2@ncl.ac.uk}
\address[1]{School of Computing, Newcastle University, Newcastle-upon-Tyne, UK}

% Optional: Include corresponding author and footnotes if needed.
\cortext[cor1]{Corresponding author}
\fntext[fn1]{ORCID: 0000-0002-7366-9034}
%
\begin{abstract}Mobile sensor data has been proposed for security-critical applications such as device pairing, proximity detection, and continuous authentication. However, the foundational premise that these signals provide sufficient entropy remains under-explored. In this work, we systematically analyse the entropy of mobile sensor data across four diverse datasets spanning multiple application contexts. Our findings reveal pervasive biases, with single-sensor mean min-entropy values ranging from 3.408–4.483 bits ($\sigma$=1.018–1.574) despite Shannon entropy being several multiples higher, showing a significant collapse between average- to worst-case settings. We further demonstrate that correlations between sensor modalities reduce the worst-case entropy of using multiple sensors by up to $\approx$75\% compared to average-case Shannon entropy. This brings joint min-entropy well below 10 bits in many cases and, in the best case, yielding only $\approx$24 bits of min-entropy when combining 20 sensor modalities. These results raise the serious risk of attacks that exhaustively search the space of possible sensor measurements. Our work also calls into question the widely held assumption that adding more sensors inherently yields higher security, and we strongly urge caution when relying on mobile sensor data for security applications.
\end{abstract}

\begin{keyword}
Entropy \sep Sensors \sep Mobile security
\end{keyword}

\end{frontmatter}
%
%
\section{Introduction}
\label{sec:intro}

Modern mobile devices come equipped with an array of embedded sensors---accelerometers, gyroscopes, magnetometers, and others---that capture continuous motion and environmental data at fine temporal granularity. This rich sensor data has enabled applications from activity recognition to context-aware computing. More recently, research has proposed leveraging these signals for security-critical tasks such as cryptographic key generation, zero-interaction device pairing, and continuous authentication. A crucial yet under-explored assumption underpins such designs: sensor data provides sufficient unpredictability to thwart adversarial inference. Traditional ``shake-to-pair'' protocols~\cite{mayrhofer2009shake} rely on motion patterns to establish secure communication between co-located devices, while other methods have incorporated ambient phenomena, such as characteristics of magnetic fields and thermal fluctuations, to mitigate relay attacks~\cite{shrestha2014drone,shrestha2018sensor,markantonakis2024using,gurulian2017effectiveness,shepherd2017applicability} and reduce user authentication prompts~\cite{krhovjak2007sources,riva2012progressive,shi2011senguard,miettinen2014conxsense,li2013unobservable}. 

Despite these advancements, fundamental issues remain: multi-modal sensing is often advocated to counter sensor-specific weaknesses~\cite{shrestha2014drone,markantonakis2024using,truong2014comparing,mehrnezhad2015tap}, but the quantitative security benefits of combining multiple sensors has not been rigorously evaluated. Many existing studies rely on heuristic assessments or machine learning classifiers, e.g.~\cite{markantonakis2024using,mehrnezhad2015tap,gurulian2018good,shrestha2014drone,truong2014comparing}, that do not address critical security questions. That is, firstly, how much entropy do sensors truly provide? And, secondly, to what extent do multi-modal sensor combinations provide security gains? Understanding the \emph{underlying} entropy is important: even if different sensors are combined, fused or otherwise transformed, it does \emph{not} fundamentally improve the quantity of entropy, or unpredictability, inherent in such signals. This paper investigates those concerns.

Our analysis reveals systemic limitations: commodity sensors exhibit significant biases of between 3.408--4.483 bits of min-entropy (5.584--9.266 bits of Shannon entropy on average). In this paper, we analyse 25 different sensors compared to a far smaller number explored in related work, i.e.\ \cite{voris2011accelerometers} (1 sensor), \cite{hennebert2013entropy} (10), \cite{krhovjak2007sources} (2), and \cite{lv2020analysis} (3). Furthermore, to the best of our knowledge, we also present the first multi-modal entropy analysis at this scale. We find that, while multi-modal sensor usage confers some benefits, non-uniform distributions and inter-sensor correlations significantly reduce the worst-case min-entropy by $\approx$40--75\% compared to average-case Shannon entropy. These findings challenge the notion that increasing the number of sensors reliably strengthens security, and it underscores the inadequacy of using sensor data as a dependable entropy source. Our contributions are as follows:
\begin{itemize} 
\item We introduce the first systematic approach to evaluating sensor entropy across such a comprehensive range of modalities and datasets using various entropy metrics (max, Shannon, collision, and min-entropy).
\item We empirically demonstrate how inter-sensor correlations and biases erode entropy, casting doubt on the proposition that using multiple sensors adds substantially to security. 
\item We show how the collapse in worst-case entropy opens the door to attacks that exhaustively enumerate, or brute force, the (joint) measurement space. This gives rise to fundamental security risks to schemes that rely on signals from single and multiple mobile sensors.
\item Ultimately, we advise against relying on commodity sensors as sources of unpredictability for security-critical applications, both on a single- and multiple-sensor basis.
\end{itemize}

The rest of this paper is organised in the following way: \S\ref{sec:background} discusses sensor-based security mechanisms and established entropy metrics. \S\ref{sec:design} explains our experiment design for entropy estimation, including the threat model and dataset selection. \S\ref{sec:entropy-analysis} presents empirical results of our analyses and \S\ref{sec:security-evaluation} discusses the implications for system design. We conclude in \S\ref{sec:conc} with recommendations for further work. Our analysis work is released publicly to foster future research.\footnote{\url{https://github.com/cgshep/entropy-collapse-mobile-sensors}}

\section{Basic Background: Supervised Learning and the PAC Model}
\label{sec:background}

At this point almost everyone has heard of machine learning (ML). Anyone likely to stumble upon this article will have also heard of its most influential special case, supervised learning, and those theoretically inclined will also be familiar with the PAC model. Nonetheless, I will set the stage by  recapping the basics.

\subsection{Basics of Supervised Learning}%Let's set the stage in any case

\emph{Supervised Learning} is the task of ``coming up'' with a function $f: \X \to \Y$ to ``explain'' or ``fit'' a sequence of input/output examples   $(x_1,y_1), \ldots, (x_n,y_n)$, with $x_i \in \X$ and $y_i \in \Y$.  Here $\X$ is a \emph{data domain} consisting of \emph{datapoints} $x \in \X$, $\Y$ is a \emph{label set} consisting of \emph{labels} $y \in \Y$, and the sequence $(x_1,y_1),\ldots,(x_n,y_n)$ is the \emph{training data} consisting of \emph{labeled examples (a.k.a. samples)}~$(x_i,y_i)$.  I~will refer to the chosen function $f$ as a \emph{predictor}, and to $n$ as the \emph{sample size}. A \emph{learning algorithm} takes as input training data, and outputs (some representation of) a predictor $f \in \Y^\X$.\footnote{Note that this describes the usual \emph{batch}, a.k.a.~\emph{offline}, setting of supervised learning. I do not discuss other paradigms such as online or active learning in this article.} 



Success in supervised learning is defined as \emph{generalization} to  future examples: For a typical \emph{test example}  $(x_{\tst},y_{\tst})$, the predicted label $y'_{\tst}=f(x_{\tst})$ should ``equal'' $y_{\tst}$, perhaps approximately. We usually assume the test example is drawn from the same  ``source'' as the training data  --- commonly, i.i.d.~from the same distribution. The quality of the prediction is quantified by $\ell(y'_{\tst},y_{\tst})$, where $\ell:~\Y~\times~\Y \to \RR_{\geq 0}$ is a \emph{loss function} chosen as part of the problem definition. Common loss functions include the 0-1 loss $\ell_{0-1}(y',y) = [y' \neq y]$ for \emph{classification} problems,\footnote{The notation $[P]$ denotes $1$ when predicate $P$ is true, and denotes $0$ when $P$ is false.} as well as the absolute loss $|y'-y|$ or squared loss $(y'-y)^2$ for \emph{regression problems} featuring $\Y  \sse \RR$.

Nontrivial generalization properties are typically only possible if one assumes something about the data.\footnote{The need for such an assumption is formalized by the  \emph{no free lunch theorems} of supervised learning \cite{wolpert_connection_1992,wolpert_lack_1996,schaffer_conservation_1994}.} The Bayesian approach to  machine learning, common in many applications, assumes some parametric form for the distribution generating the data, and postulates a prior on the parameters. This is not the approach I will take in this article. Instead, I will focus on the frequentist --- and some would say ``worst-case'' or ``adversarial'' ---  approach that is common in the computational learning theory community, embodied by the PAC model. Here we assume that the (training and test) data can be explained, perhaps approximately, by a function in some ``simple enough to learn'' class of functions $\H \sse \Y^\X$, often called the \emph{hypotheses}. Equivalently, we  seek a predictor which explains the unseen data roughly  as well as the best hypothesis $h^* \in \H$, whether or not we assume that $h^*$ itself provides a perfect explanation.



 \paragraph{Common Algorithmic Templates.} Perhaps the best known general-purpose supervised learning algorithm is \emph{empirical risk minimization (ERM)}, which chooses as its predictor a hypothesis $f \in \H$ minimizing $\frac{1}{n} \sum_{i=1}^n \ell(f(x_i),y_i)$ --- a quantity called the \emph{training error}, \emph{empirical error}, or \emph{empirical risk} of $f$. %\footnote{When multiple hypotheses minimize the empirical risk, we assume ERM breaks ties arbitrarily.}
A common template for generalizing ERM involves adding a \emph{regularization term} $\psi(f)$ to the  objective function, typically chosen to measure some notion of ``hypothesis complexity.'' An algorithm instantiating this template is known as a \emph{structural risk minimizer (SRM)}, and chooses as its predictor the hypothesis $f \in \H$ minimizing the \emph{structural risk} $\frac{1}{n} \sum_{i=1}^n \ell(f(x_i),y_i) + \psi(f)$. Other well-known algorithms, such as gradient descent and its variations,  can frequently be interpreted as approximate implementations of ERM or SRM.


\paragraph{Proper vs Improper Learning.} A learning algorithm is said to be \emph{proper} if its predictor $f$ is always chosen from the hypothesis class, i.e., $f \in \H$, otherwise it is said to be \emph{improper}. ERM  is an example of a proper learning algorithm, as are SRM algorithms of the form described above.  In the \emph{proper regime} of learning, algorithms are required to be proper. This article will be concerned with the more flexible \emph{improper regime} (a.k.a \emph{representation-independent learning}), where no such constraint is placed on the learner. In other words, all we care about is predictive power at test time, rather than any insights derived from the functional form or representation of the predictor~itself.


\subsection{The PAC Model}
A standard mathematical setup for evaluation of supervised learning algorithms, at least in the theoretical computer science community, is Valiant's \emph{Probably Approximately Correct (PAC) model} of learning (see e.g.~\cite{kearns_introduction_1994,mohri_foundations_2018}). Here, we assume there is an unknown distribution $\D$ on $\X \times \Y$ from which training and test data are  drawn.  Specifically, the labeled datapoints of the training set  $(x_1,y_1), \ldots, (x_n,y_n)$, as well as the test data  $(x_\tst,y_\tst)$, are i.i.d.~from $\D$. Often it is assumed that $\D$ lies in some class of distributions of interest. The \emph{true expected loss}, or simply \emph{loss}, of a predictor $f: \X \to \Y$ is the expected loss it incurs on draws from $\D$, written $L_\D(f) = \Ex_{(x,y) \sim \D} \ell(f(x),y)$.


There are two main ``settings'' in PAC learning. The  \emph{realizable setting} only requires that the data be perfectly explained by some hypothesis in $\H$. More generally, the \emph{agnostic setting} makes no assumption relating the data to the hypotheses, but shifts the goalposts as necessary to allow nontrivial guarantees: the expected loss at test time is evaluated only ``relative'' to that of the best hypothesis $h^* \in \H$. There are other settings which make more nuanced assumptions, such as $\D$ being of a particular parametric form or its support living in some (unknown) lower-dimensional space, etc. I will mostly discuss the realizable and agnostic settings in this article, those being the simplest and most studied from a theoretical perspective. %TODO:We will briefly discuss other settings in Section ??

The PAC model demands high probability guarantees of learners, in the worst case over distributions of interest. Consider first the realizable setting, where $\D$ is such that $\min_{h \in \H} L_{\D}(h) = 0$. A PAC learner has \emph{error} $\epsilon=\epsilon(n)$ and \emph{confidence} $\delta=\delta(n)$ if, when training data consists of $n$ i.i.d~samples from a realizable distribution $\D$, it produces a predictor $f$  satisfying $L_\D(f) \leq \epsilon$ with probability at least $1-\delta$. In the agnostic setting, where $\D$ can be arbitrary, we require $L_\D(f) - \min_{h \in \H} L_\D(h) \leq \epsilon$ with probability $1-\delta$.

In both the realizable and agnostic settings, we look for PAC learners with small $\epsilon$ and $\delta$ as a function of the sample size $n$. An equivalent perspective looks at the sample complexity $m(\epsilon,\delta)$, which is the minimum sample size which guarantees error  at most $\epsilon$ with probability at least $1-\delta$. We say a problem is \emph{PAC learnable} if its PAC sample complexity is finite whenever $\epsilon,\delta > 0$.

For most PAC learning problems, learnability and sample complexity are characterized in terms of a  ``dimension'' of the hypothesis class. Most prominently this is the \emph{VC dimension} for binary classification, the \emph{fat shattering dimension} for agnostic regression, and the \emph{DS dimension} for multiclass classification (see \cite{anthony_neural_1999,daniely_optimal_2014,brukhim_characterization_2022}). Treatment of these is beyond the scope of this article. The unfamiliar reader need not worry, however,  as dimensions will feature only tangentially in our~discussion.




%\paragraph{Learning settings: Realizable, Agnostic, etc.} In learning theory, evaluating a supervised learning algorithm requires specifying a data model and an objective. We will leave the details of the data model flexible for now, to allow for both the PAC model and the adversarial transductive model. Nonetheless we will describe two variations, which we call ``settings'', which cut across different models. The  \emph{realizable setting}  requires only that the data be perfectly explained by some hypothesis $h \in \H$ --- i.e., there exists a hypothesis which is guaranteed to suffer a loss of $0$ on training and test data. The performance of the learning algorithm is its expected loss at test time for some ``worst case'' realizable instance. More generally, the \emph{agnostic setting} makes no assumption relating the data to the hypotheses, but shifts the goalposts as necessary to allow nontrivial guarantees: the expected loss at test time is evaluated only ``relative'' to that of the best hypothesis $h^* \in \H$, again for some ``worst case'' instance. There are other settings which make more nuanced assumptions about the data, such as it is drawn from a distribution of a particular parametric form, or that it lives in some (unknown) lower-dimensional space, etc. We will mostly discuss the realizable and agnostic settings, those being the simplest and most studied from a theoretical perspective.




%%% Local Variables:
%%% mode: latex
%%% TeX-master: "learning_matching"
%%% End:

\iffalse
\begin{table*}[htbp]
\tiny
\begin{center}
\begin{tabular}{lccccccccccccc}\toprule
Model, ft setting & mem & \#param & ARC-c & ARC-e & BoolQ & HS & OBQA & PIQA & rte & SIQA & WG & Avg
%\\\cmidrule(lr){2-3}\cmidrule(lr){4-5} \cmidrule(lr){6-7} \cmidrule(lr){8-9}\cmidrule(lr){10-11} \cmidrule(lr){12-13} \cmidrule(lr){14-15} \cmidrule(lr){16-17} 
\\\cmidrule(lr){1-13}
Llama2(7B), LoRA, $r=64$ & 23.46GB & 159.9M(2.37\%) & \textbf{44.97} & 77.02 & 77.43 & \textbf{57.75} & 32.0 & \textbf{78.45} & 62.09 & \textbf{47.75} & 68.75 & 60.69\\
Llama2(7B), SPruFT, $r=128$ & \textbf{17.62GB} & 145.8M(2.16\%) & 43.60 & \textbf{77.26} & \textbf{77.77} & 57.47 & \textbf{32.6} & 78.07 & \textbf{64.98} & 46.67 & \textbf{69.30} & \textbf{60.86} \\\cmidrule(lr){2-13}
Llama2(7B), FA-LoRA, $r=64$ & 17.25GB & 92.8M(1.38\%) & 43.77 & \textbf{77.57} & 77.74 & \textbf{57.45} & 31.0 & 77.86 & \textbf{66.06} & \textbf{47.13} & 69.06 & 60.85\\
Llama2(7B), FA-SPruFT, $r=128$ & \textbf{15.21GB} & 78.6M(1.17\%) & \textbf{43.94} & 77.22 & \textbf{77.83} & 57.11 & \textbf{32.0} & \textbf{78.18} & 65.70 & 46.47 & \textbf{69.38} & \textbf{60.87}\\\midrule
Llama3(8B), LoRA, $r=64$ & 30.37GB & 167.8M(2.09\%) & \textbf{53.07} & \textbf{81.40} & \textbf{82.32} & \textbf{60.67} & 34.2 & \textbf{79.98} & 69.68 & \textbf{48.52} & \textbf{73.56} & \textbf{64.82}\\
Llama3(8B), SPruFT, $r=128$ & \textbf{24.49GB} & 159.4M(1.98\%) & 52.47 & 81.10 & 81.28 & 60.29 & \textbf{34.6} & 79.76 & \textbf{70.04} & 47.75 & 73.24 & 64.50 \\\cmidrule(lr){2-13}
Llama3(8B), FA-LoRA, $r=64$ & 24.55GB & 113.2M(1.41\%) & \textbf{52.47} & \textbf{81.36} & \textbf{82.23} & 60.17 & \textbf{35.0} & \textbf{79.76} & \textbf{70.04} & \textbf{48.31} & \textbf{73.56} & \textbf{64.77}\\
Llama3(8B), FA-SPruFT, $r=128$ & \textbf{22.41GB} & 92.3M(1.15\%) & 52.22 & 81.19 & 81.35 & \textbf{60.20} & 34.2 & 79.71 & 69.31 & 47.13 & 73.01 & 64.26 \\\bottomrule
\end{tabular}
%\vspace{-0.2cm}
\caption{Fine-tuning Llama on Alpaca dataset for 5 epochs and evaluating on 9 tasks from EleutherAI LM Harness. "mem" represents the memory usage, with further details provided in Appendix~\ref{apdx:measure}. \#param is the number of trainable parameters, where the difference of \#param between the two approaches depends on the architecture of Llama, as some layers have $d_{in} \neq d_{out}$. Note that 10 million trainable parameters only account for less than 0.15GB of memory requirement. FA indicates that we freeze attention layers, but not including MLP layers followed by attention blocks. HS, OBQA, and WG represent HellaSwag, OpenBookQA, and WinoGrande datasets. More details of datasets can be found in Appendix~\ref{apdx:data}. The ablation study for different $r$ and the comparison with other LoRA variants can be found in Appendix~\ref{apdx:ablation}. All reported results are accuracies on the corresponding tasks. \textbf{Bold} indicates the best results of two approaches on the same task.} \label{tab:llm} 
\end{center}
\end{table*}
\fi

\begin{table*}[htbp]
\tiny
\begin{center}
\begin{tabular}{lccccccccccccc}\toprule
Model, ft setting & mem & \#param & ARC-c & ARC-e & BoolQ & HS & OBQA & PIQA & rte & SIQA & WG & Avg
\\\cmidrule(lr){1-13}
Llama2(7B)\\ \cmidrule(lr){1-1} 
LoRA, $r=64$ & 23.46GB & 159.9M(2.37\%) & \textbf{44.97} & 77.02 & 77.43 & 57.75 & 32.0 & \textbf{78.45} & 62.09 & 47.75 & 68.75 & 60.69\\
VeRA, $r=64$ & 22.97GB & 1.374M(0.02\%) & 43.26 & 76.43 & 77.40 & 57.26 & 31.6 & 78.02 & 62.09 & 45.85 & 68.75 & 60.07\\
DoRA, $r=64$ & 44.85GB & 161.3M(2.39\%) & 44.71 & 77.02 & 77.55 & \textbf{57.79} & 32.4 & 78.29 & 61.73 & \textbf{47.90} & 68.98 & 60.71\\
RoSA, $r=32, d=1.2\%$ & 44.69GB & 157.7M(2.34\%) & 43.86 & \textbf{77.48} & \textbf{77.86} & 57.42 & 32.2 & 77.97 & 63.90 &  47.29 & 69.06 & 60.78\\
SPruFT, $r=128$ & \textbf{17.62GB} & 145.8M(2.16\%) & 43.60 & 77.26 & 77.77 & 57.47 & \textbf{32.6} & 78.07 & \textbf{64.98} & 46.67 & \textbf{69.30} & \textbf{60.86} %\\\cmidrule(lr){2-13}
%FA-LoRA, $r=64$ & 17.25GB & 92.8M(1.38\%) & 43.77 & \textbf{77.57} & 77.74 & \textbf{57.45} & 31.0 & 77.86 & 66.06 & \textbf{47.13} & 69.06 & 60.85\\
%FA-DoRA, $r=64$ & 30.61GB & 93.6M(1.39\%) & 43.94 & 77.44 & 77.49 & 57.44 & 31.0 & 77.86 & \textbf{66.43} & 46.98 & 69.14 & 60.86\\
%FA-RoSA, $r=32, d=1.2\%$ & 38.34GB & 98.3M(1.46\%) & \textbf{44.28} & 77.02 & 77.68 & 57.22 & 31.0 & 77.97 & 64.26 & 46.32 & 69.22 & 60.55\\
%FA-SPruFT, $r=128$ & \textbf{15.21GB} & 78.6M(1.17\%) & 43.94 & 77.22 & \textbf{77.83} & 57.11 & \textbf{32.0} & \textbf{78.18} & 65.70 & 46.47 & \textbf{69.38} & \textbf{60.87}
\\\midrule
Llama3(8B)\\ \cmidrule(lr){1-1} 
LoRA, $r=64$ & 30.37GB & 167.8M(2.09\%) & 53.07 & 81.40 & 82.32 & 60.67 & 34.2 & 79.98 & 69.68 & 48.52 & 73.56 & 64.82\\
VeRA, $r=64$ & 29.49GB & 1.391M(0.02\%) & 50.26 & 80.30 & 81.41 & 60.16 & 34.4 & 79.60 & 69.31 & 46.93 & 72.77 & 63.90\\
DoRA, $r=64$ & 51.45GB & 169.1M(2.11\%) & \textbf{53.33} & \textbf{81.57} & \textbf{82.45} & \textbf{60.71} & 34.2 & \textbf{80.09} & 69.31 & \textbf{48.67} & \textbf{73.64} & \textbf{64.88}\\
RoSA, $r=32, d=1.2\%$ & 48.40GB & 167.6M(2.09\%) & 51.28 & 81.27 & 81.80 & 60.18 & 34.4 & 79.87 & 69.31 & 47.95 & 73.16 & 64.36\\
SPruFT, $r=128$ & \textbf{24.49GB} & 159.4M(1.98\%) & 52.47 & 81.10 & 81.28 & 60.29 & \textbf{34.6} & 79.76 & \textbf{70.04} & 47.75 & 73.24 & 64.50 %\\\cmidrule(lr){2-13}
%FA-LoRA, $r=64$ & 24.55GB & 113.2M(1.41\%) & 52.47 & 81.36 & 82.23 & 60.17 & \textbf{35.0} & 79.76 & 70.04 & 48.31 & \textbf{73.56} & 64.77\\
%FA-DoRA, $r=64$ & 40.62GB & 114.3M(1.42\%) & \textbf{52.56} & \textbf{81.69} & \textbf{82.26} & \textbf{60.20} & 34.4 & \textbf{79.82} & \textbf{70.40} & \textbf{48.46} & 73.40 & \textbf{64.80}\\
%FA-RoSA, $r=32, d=1.2\%$ & 42.31GB & 124.3M(1.55\%) & 52.22 & 81.19 & 82.05 & 60.11 & 34.4 & 79.76 & 69.31 & 47.70 & 73.16 & 64.43\\
%FA-SPruFT, $r=128$ & \textbf{22.41GB} & 92.3M(1.15\%) & 52.22 & 81.19 & 81.35 & \textbf{60.20} & 34.2 & 79.71 & 69.31 & 47.13 & 73.01 & 64.26 
\\\bottomrule
\end{tabular}
%\vspace{-0.2cm}
\caption{Fine-tuning Llama on Alpaca dataset for 5 epochs and evaluating on 9 tasks from EleutherAI LM Harness. ``mem" represents the memory usage, with further details provided in Appendix~\ref{apdx:measure}. \#param is the number of trainable parameters, where the difference of \#param between the two approaches depends on the architecture of Llama, as some layers have $d_{in} \neq d_{out}$. %FA indicates that we freeze attention layers, but not including MLP layers followed by attention blocks. 
HS, OBQA, and WG represent HellaSwag, OpenBookQA, and WinoGrande datasets. %More details of datasets can be found in Appendix~\ref{apdx:data}. 
The ablation study for different $r$ can be found in Appendix~\ref{apdx:ranks}. All reported results are accuracies on the corresponding tasks. \textbf{Bold} indicates the best result on the same task. } \label{tab:llm} 
\end{center}
\end{table*}

\section{Experimental Setup}\label{sec:setup}

%(0.5 page)
%Why the chosen framework?
%Some prior approaches

%- parameter settings
%- uniform across layers vs greedy ... 
%- potential transformer-specific details

%Equations about what these methods do.. 

%(0.5 page)
%Which NN architectures are used, why?
%Number of parameters, layers, ...

%(Potential prior work on compression -- )

\subsection{Datasets} \label{subsec:dataset}
We use multiple datasets for different tasks. For image classification, we fine-tune models on the training split and evaluate it on the validation split of Tiny-ImageNet~\citep{tavanaei2020embedded}, CIFAR100~\citep{alex2009learning}, and Caltech101~\citep{li_andreeto_ranzato_perona_2022}. For text generation, we fine-tune LLMs on 256 samples from Stanford-Alpaca~\citep{alpaca} and assess zero-shot performance on nine EleutherAI LM Harness tasks~\citep{gao2021framework}. See Appendix~\ref{apdx:data} for details.

\subsection{Models and Baselines} \label{subsec:models}

We fine-tune full-precision Llama-2-7B and Llama-3-8B (float32) using our SPruFT, LoRA~\citep{hulora}, VeRA~\citep{kopiczko2024vera}, DoRA~\citep{liu2024dora}, and RoSA~\citep{nikdan2024rosa}. RoSA is chosen as the representative SFT method and is the only SFT due to the high memory demands of other SFT approaches, while full fine-tuning is excluded for the same reason. We freeze Llama’s classification layers and fine-tune only the linear layers in attention and MLP blocks.

Next, we evaluate importance metrics by fine-tuning Llamas and image models, including DeiT~\citep{touvron2021training}, ViT~\citep{dosovitskiy2020image}, ResNet101~\citep{he2016deep}, and ResNeXt101~\citep{xie2017aggregated} on CIFAR100, Caltech101, and Tiny-ImageNet. For image tasks, we set the fine-tuning ratio at 5\%, meaning the trainable parameters are a total of 5\% of the backbone plus classification layers.

\subsection{Training Details} \label{subsec:training}
Our fine-tuning framework is built on torch-pruning\footnote{Torch-pruning is not required, all their implementations are based on PyTorch.}~\citep{fang2023depgraph}, PyTorch~\citep{paszke2019pytorch}, PyTorch-Image-Models~\citep{rw2019timm}, and HuggingFace Transformers~\citep{wolf2020transformers}. Most experiments run on a single A100-80GB GPU, while DoRA and RoSA use an H100-96GB GPU. We use the Adam optimizer~\citep{KingBa15} and fine-tune all models for a fixed number of epochs without validation-based model selection.

%Structured pruning often considers parameter dependencies in importance evaluation~\citep{liu2021group, fang2023depgraph, ma2023llmpruner}. This becomes the following process in our work: first, searching for dependencies by tracing the computation graph of gradient; next, evaluating the importance of parameter groups; and finally, fine-tuning the parameters within those important groups collectively. For instance, if $\W^{a}_{\cdot j}$ and $\W^{b}_{i\cdot}$ are dependent, where $\W^{a}_{\cdot j}$ is the $j$-th column in parameter matrix (or the $j$-th input channels/features) of layer $a$ and $\W^{b}_{i\cdot}$ is the $i$-th row in parameter matrix (or the $i$-th output channels/features) of layer $b$, then $\W^{a}_{\cdot j}$ and $\W^{b}_{i\cdot}$ will be fine-tuned simultaneously while the corresponding $\M^{a}_{dep}$ for $\W^{a}_{\cdot j}$ becomes column selection matrix and $\W^a_s$ becomes $\W^a_{f,dep}\M^a_{dep}$. Consequently, fine-tuning $2.5\%$ output channels for layer $b$ will result in fine-tuning additional $2.5\%$ input channels in each dependent layer. Therefore, for the $5\%$ of desired fine-tuning ratio, the fine-tuning ratio with considering dependencies is set to $2.5\%$\footnote{In some complex models, considering dependencies results in slightly more than twice the number of trainable parameters. However, in most cases, the factor is 2.} for the approach that includes dependencies. More details for dependencies of NN can be found in Appendix~\ref{apdx:dep}. 

\textbf{Image models}: The learning rate is set to $10^{-4}$ with cosine annealing decay~\citep{loshchilov2017sgdr}, where the minimum learning rate is $10^{-9}$. All image models used in this study are pre-trained on ImageNet. 

\textbf{Llama}: For LoRA and DoRA, we set $\alpha = 16$, a dropout rate of $0.1$, and a learning rate of $10^{-4}$  with linear decay (
$0.01$ decay rate). For SPruFT, we control trainable parameters using rank instead of fine-tuning ratio for direct comparison. The learning rate is $2 \cdot 10^{-5}$ with the same decay settings. Linear decay is applied after a warmup over the first $3$\% of training steps. The maximum sequence length is $2048$, with truncation for longer inputs and padding for shorter ones.


To begin with, we sought publicly available sensor datasets suitable for analysing motion and environmental data at scale. Our search involved broad queries across IEEE DataPort, Google Scholar, Google Dataset Search, and GitHub. Several ostensibly ``open'' datasets either were no longer downloadable or imposed restrictive licensing terms~\cite{Mahbub_Btas2016_UMDAA02,stragapede2023behavepassdb,acien2021becaptcha}. Ultimately, we narrowed our scope to four datasets that offer diverse usage contexts, consistent sampling rates, and documented sensor modalities:

\begin{itemize}
    \item \emph{UCI-HAR}~\cite{anguita2013public}:  A widely referenced dataset for human activity recognition, comprising smartphone sensor recordings from multiple subjects performing daily activities. Data includes triaxial accelerometer and gyroscope signals.

    
    \item \emph{University of Sussex--Huawei Locomotion (SHL)}~\cite{gjoreski2018university,wang2019enabling}:   sampled at 100 Hz from an Huawei Mate 9 smartphone. The publicly available SHL Preview dataset is used, comprising three recording-days per user (59 hours of data in total). To scope this study, we use the dataset from the handheld mobile phone as a good fit with related work.
    \item \emph{Relay}~\cite{gurulian2017effectiveness}: Contains sensor measurements for approximately 1{,}500 NFC-based contactless transactions, each recorded at 100\,Hz across several physical locations (e.g., caf\'{e}s). The dataset encompasses accelerometer, gyroscope, and environmental readings taken in realistic payment scenarios.
    \item \emph{PerilZIS}~\cite{fomichev2019perils}: Collected at 10\,Hz from a Texas Instruments SensorTag, a Samsung Galaxy S6, and a Samsung Galaxy Gear, this dataset spans multiple zero-interaction security use cases in an office environment.
\end{itemize}

These four datasets provide a variety of sensor types, user activities, and sampling rates, allowing us to explore how intrinsic biases and correlations manifest across different scenarios. Next, we detail how we preprocess and aggregate this data to form global distributions for our entropy analyses.
\section{Entropy Analysis}
\label{sec:entropy-analysis}

In this section, we analyse the intrinsic entropy of sensor data under the threat model described in \S\ref{sec:design}. We begin by discussing the challenges in quantising naturally continuous sensor values for discrete-entropy calculations, then present our findings for our single- and multi-sensor analyses.

\subsection{Pre-processing}

A crucial, yet underexplored, issue in prior work (e.g.\ \cite{voris2011accelerometers,lv2020analysis,hennebert2013entropy,mai2017guessability}) is how to convert inherently continuous sensor outputs into suitable discrete values for entropy estimation. For example, Shannon and min-entropy, as defined in Eqs.~\ref{eq:renyi_H1} and \ref{eq:renyi_Hinf}, rely on discrete random variables. Physical quantities such as linear acceleration or angular velocity are continuous in nature, even though modern sensors employ internal analog-to-digital conversion with a finite resolution. Yet, a sensor's advertised resolution (e.g.\ 12 bits for the widely used Bosch BMA mobile accelerometer~\cite{bosch_bma400}) does \emph{not} imply uniform coverage across its range. Everyday usage introduces biases and clustering, resulting in some measurements occurring far more frequently than others. For instance, \emph{UCI-HAR} data shows accelerometer readings concentrated in certain areas, and approximately 60\% of gyroscope readings hover near zero (see Figure~\ref{fig:acc-cdfs}). Such skew and bias radically diminishes entropy compared to uniformly distributed values.

\begin{figure*}
    \centering
    \begin{subfigure}[t]{0.333\textwidth}
        \centering
        \includegraphics[width=\textwidth]{figures/graphs/acc_x_cdf.pdf}
        \caption{Acc.\ $x$ axis.}
        \label{subfig:acx}
    \end{subfigure}%
    \begin{subfigure}[t]{0.333\textwidth}
        \centering
        \includegraphics[width=\textwidth]{figures/graphs/acc_y_cdf.pdf}
        \caption{Acc.\ $y$ axis.}
        \label{subfig:acy}
    \end{subfigure}%
    \begin{subfigure}[t]{0.333\textwidth}
        \centering
        \includegraphics[width=\textwidth]{figures/graphs/acc_z_cdf.pdf}
        \caption{Acc.\ $z$ axis.}
        \label{subfig:acz}
    \end{subfigure}

    \begin{subfigure}[t]{0.333\textwidth}
        \centering
        \includegraphics[width=\textwidth]{figures/graphs/gyro_x_cdf.pdf}
        \caption{Gyro.\ $x$ axis.}
        \label{subfig:gyx}
    \end{subfigure}%
    \begin{subfigure}[t]{0.333\textwidth}
        \centering
        \includegraphics[width=\textwidth]{figures/graphs/gyro_y_cdf.pdf}
        \caption{Gyro.\ $y$ axis.}
        \label{subfig:gyy}
    \end{subfigure}%
    \begin{subfigure}[t]{0.333\textwidth}
        \centering
        \includegraphics[width=\textwidth]{figures/graphs/gyro_z_cdf.pdf}
        \caption{Gyro.\ $z$ axis.}
        \label{subfig:gyz}
    \end{subfigure}

    \begin{subfigure}[t]{0.333\textwidth}
        \centering
        \includegraphics[width=\textwidth]{figures/graphs/Accelerometer_cdf.pdf}
        \caption{Accelerometer.}
        \label{subfig:acc}
    \end{subfigure}%
    \begin{subfigure}[t]{0.333\textwidth}
        \centering
        \includegraphics[width=\textwidth]{figures/graphs/Gyroscope_cdf.pdf}
        \caption{Gyroscope.}
        \label{subfig:gyro}
    \end{subfigure}%
    \begin{subfigure}[t]{0.333\textwidth}
        \centering
        \includegraphics[width=\textwidth]{figures/graphs/Light_cdf.pdf}
        \caption{Light.}
        \label{subfig:light}
    \end{subfigure}

    \begin{subfigure}[t]{0.333\textwidth}
        \centering
        \includegraphics[width=\textwidth]{figures/graphs/LinearAcceleration_cdf.pdf}
        \caption{Linear Acceleration.}
        \label{subfig:linacc}
    \end{subfigure}%
    \begin{subfigure}[t]{0.333\textwidth}
        \centering
        \includegraphics[width=\textwidth]{figures/graphs/MagneticField_cdf.pdf}
        \caption{Magnetometer.}
        \label{subfig:mag}
    \end{subfigure}%
    \begin{subfigure}[t]{0.333\textwidth}
        \centering
        \includegraphics[width=\textwidth]{figures/graphs/RotationVector_cdf.pdf}
        \caption{Rotation Vector.}
        \label{subfig:rv}
    \end{subfigure}
    \caption{Global sensor data CDFs -- UCI-HAR (a--f) and Relay (g--l) datasets.}
    \label{fig:acc-cdfs}
\end{figure*}

Another practical challenge arises when extremely fine-grained values appear infrequently or with negligible probability in reality. Treating every minute fluctuation (e.g.\ 9.001$ms^{-2}$ vs.\ 9.002$ms^{-2}$ for an accelerometer) as distinct outcomes can also artificially inflate entropy estimates.  In real-world applications, it is the `similarity' between measurement signals that is considered useful in existing work. It would be extremely difficult for users to reliably reproduce high-precision movements capable of effectively utilising a sensor's digital resolution (say at 0.001ms$^{-2}$ for an accelerometer). To address this, we discretise the data values into bins of similar value. However, this raises a further question of what constitutes a good strategy for selecting the number of bins and their widths? Several techniques exist that make assumptions about the underlying distribution, e.g.\ Gaussian; have different computational complexities; and are robust to outliers and data variability. To this end, we use the Freedman-Diaconis method, a commonly used robust estimator that accounts for data size and its variability.\footnote{Alternatively, a binning strategy could be employed that reflects how precisely humans can realistically replicate sensor-input changes. We defer this to future research.} This is calculated in Eq.~\ref{sec:freedman}, where $IQR(x)$ represents the interquartile range of $x$ and $n$ is the total number of samples.


\begin{equation}
    h = 2 \cdot \frac{IQR(x)}{n^{1/3}}
    \label{sec:freedman}
\end{equation}


\subsection{Single Sensors}



\begin{table}
\renewcommand{\arraystretch}{2}
\centering
\caption{Single-sensor entropy values (in bits) for each dataset. Grey cells denote unavailable data for that dataset and modality.}
\resizebox{\linewidth}{!}{%
\label{tab:single-sensor-results}
\small
\begin{tabular}{@{}r|ccc|c|ccc|c|ccc|c|ccc|c@{}}
\toprule
 & \multicolumn{16}{c}{\textbf{Dataset}} \\
 & \multicolumn{4}{c|}{\textbf{UCI-HAR}} 
 & \multicolumn{4}{c|}{\textbf{SHL}}  
 & \multicolumn{4}{c|}{\textbf{Relay}*}  
 & \multicolumn{4}{c}{\textbf{PerilZIS}} \\
\midrule
\textbf{Sensor} 
 & \(H_{0}\) & \(H_{1}\) & \(H_{2}\) & \(H_{\infty}\) 
 & \(H_{0}\) & \(H_{1}\) & \(H_{2}\) & \(H_{\infty}\) 
 & \(H_{0}\) & \(H_{1}\) & \(H_{2}\) & \(H_{\infty}\) 
 & \(H_{0}\) & \(H_{1}\) & \(H_{2}\) & \(H_{\infty}\) \\
\midrule
Acc.X       
 & 8.488 & 7.080 & 5.876 & 3.729 
 & 11.557 & 8.732  & 7.487  & 4.543 
 & \missingcell & \missingcell & \missingcell & \missingcell 
 & 13.012 & 9.292  & 6.359  & 3.626 \\
Acc.Y       
 & 8.243 & 7.231 & 6.847 & 5.694 
 & 11.425 & 8.928  & 7.717  & 4.500 
 & \missingcell & \missingcell & \missingcell & \missingcell 
 & 9.549  & 5.873  & 4.483  & 2.889 \\
Acc.Z       
 & 8.455 & 7.397 & 7.069 & 6.020 
 & 10.428 & 7.627  & 6.366  & 3.785 
 & \missingcell & \missingcell & \missingcell & \missingcell 
 & 9.817  & 6.671  & 5.403  & 4.002 \\
Acc.Mag    
 & 8.895 & 6.284 & 4.819 & 3.489 
 & 14.583 & 10.136 & 8.710  & 6.435 
 & 10.145 & 6.843 & 5.808 & 4.538
 & 13.328 & 8.273  & 7.115  & 4.526 \\
\midrule
Gyro.X     
 & 8.683 & 5.430 & 3.504 & 1.929 
 & 15.024 & 10.532 & 8.107  & 4.993 
 & \missingcell & \missingcell & \missingcell & \missingcell 
 & 14.528 & 7.231  & 4.454  & 2.805 \\
Gyro.Y     
 & 8.439 & 5.023 & 3.461 & 2.300 
 & 15.085 & 10.283 & 7.601  & 4.827 
 & \missingcell & \missingcell & \missingcell & \missingcell 
 & 14.078 & 6.715  & 4.039  & 2.529 \\
Gyro.Z     
 & 8.714 & 5.675 & 3.948 & 2.363 
 & 15.281 & 10.070 & 5.708  & 3.083 
 & \missingcell & \missingcell & \missingcell & \missingcell 
 & 13.961 & 6.463  & 3.836  & 2.434 \\
Gyro.Mag  
 & 8.414 & 5.759 & 4.130 & 2.537 
 & 12.123 & 7.816  & 5.728  & 3.699 
 & 7.954 & 4.751 & 3.442 & 2.083 
 & 14.166 & 5.565  & 1.932  & 0.969 \\
\midrule
Mag.X   
 & \missingcell & \missingcell & \missingcell & \missingcell 
 & 12.845 & 8.840  & 8.386  & 6.374 
 & \missingcell & \missingcell & \missingcell & \missingcell 
 & 10.767 & 7.639  & 6.816  & 4.883 \\
Mag.Y   
 & \missingcell & \missingcell & \missingcell & \missingcell 
 & 12.263 & 8.737  & 8.314  & 6.223 
 & \missingcell & \missingcell & \missingcell & \missingcell 
 & 10.179 & 7.622  & 6.605  & 4.405 \\
Mag.Z   
 & \missingcell & \missingcell & \missingcell & \missingcell 
 & 12.516 & 8.586  & 8.217  & 6.228 
 & \missingcell & \missingcell & \missingcell & \missingcell 
 & 10.129 & 7.507  & 6.726  & 4.448 \\
Mag.Mag 
 & \missingcell & \missingcell & \missingcell & \missingcell 
 & 13.558 & 9.436  & 8.771  & 7.148 
 & 7.972 & 6.147 & 5.617 & 4.254
 & 10.293 & 7.329  & 6.489  & 4.454 \\
\midrule
Rot.\ Vec.\ 
 & \missingcell & \missingcell & \missingcell & \missingcell 
 & 8.725  & 7.721  & 5.970  & 3.220 
 & 5.000 & 3.307 & 1.965 & 1.021
 & \missingcell & \missingcell & \missingcell & \missingcell \\
\midrule
Grav.X     
 & \missingcell & \missingcell & \missingcell & \missingcell 
 & 9.014  & 8.482  & 7.266  & 4.299 
 & \missingcell & \missingcell & \missingcell & \missingcell 
 & \missingcell & \missingcell & \missingcell & \missingcell \\
Grav.Y     
 & \missingcell & \missingcell & \missingcell & \missingcell 
 & 9.338  & 8.770  & 7.602  & 4.453 
 & \missingcell & \missingcell & \missingcell & \missingcell 
 & \missingcell & \missingcell & \missingcell & \missingcell \\
Grav.Z     
 & \missingcell & \missingcell & \missingcell & \missingcell 
 & 8.180  & 7.193  & 5.418  & 3.036 
 & \missingcell & \missingcell & \missingcell & \missingcell 
 & \missingcell & \missingcell & \missingcell & \missingcell \\
Grav.Mag     
 & \missingcell & \missingcell & \missingcell & \missingcell 
 & 14.373 & 7.988  & 7.227  & 6.242 
 & 7.794 & 6.325 & 5.864 & 4.532
 & \missingcell & \missingcell & \missingcell & \missingcell \\
\midrule
LinAcc.X  
 & \missingcell & \missingcell & \missingcell & \missingcell 
 & 15.260 & 10.077 & 7.621  & 5.116 
 & \missingcell & \missingcell & \missingcell & \missingcell 
 & \missingcell & \missingcell & \missingcell & \missingcell \\
LinAcc.Y  
 & \missingcell & \missingcell & \missingcell & \missingcell 
 & 14.859 & 10.116 & 7.639  & 5.224 
 & \missingcell & \missingcell & \missingcell & \missingcell 
 & \missingcell & \missingcell & \missingcell & \missingcell \\
LinAcc.Z  
 & \missingcell & \missingcell & \missingcell & \missingcell 
 & 14.377 & 9.951  & 7.605  & 4.543 
 & \missingcell & \missingcell & \missingcell & \missingcell 
 & \missingcell & \missingcell & \missingcell & \missingcell \\
LinAcc.Mag  
 & \missingcell & \missingcell & \missingcell & \missingcell 
 & 12.777 & 7.968  & 5.752  & 3.420 
 & 9.175 & 6.385 & 5.424 & 4.222
 & \missingcell & \missingcell & \missingcell & \missingcell \\
\midrule
Light       
 & \missingcell & \missingcell & \missingcell & \missingcell 
 & \missingcell & \missingcell & \missingcell & \missingcell 
 & 7.200& 5.331 & 4.507& 3.206
 & 12.152 & 7.940  & 7.137  & 4.552 \\
Humidity    
 & \missingcell & \missingcell & \missingcell & \missingcell 
 & \missingcell & \missingcell & \missingcell & \missingcell 
 & \missingcell & \missingcell & \missingcell & \missingcell 
 & 7.943  & 7.048  & 6.774  & 5.546 \\
Temp.       
 & \missingcell & \missingcell & \missingcell & \missingcell 
 & 7.295  & 4.753  & 2.611  & 1.332 
 & \missingcell & \missingcell & \missingcell & \missingcell 
 & 8.484  & 7.416  & 6.941  & 5.449 \\
Pressure    
 & \missingcell & \missingcell & \missingcell & \missingcell 
 & 9.461  & 8.170  & 7.723  & 6.237 
 & \missingcell & \missingcell & \missingcell & \missingcell 
 & 8.044  & 7.006  & 6.370  & 5.073 \\
\midrule 
\textbf{Mean} 
 & 8.541 & 6.235 & 4.957 & 3.508 
 & 13.188 & 9.266 & 7.178 & 4.483 
 & 7.891 & 5.584 & 4.661 & 3.408 
 & 11.277 & 7.224 & 5.717 & 3.912 \\
\textbf{S.D.} 
 & 0.207 & 0.904 & 1.461 & 1.574 
 & 1.993 & 1.148 & 1.115 & 1.018 
 & 1.612 & 1.227 & 1.474 & 1.379 
 & 2.312 & 0.900 & 1.526 & 1.266 \\\bottomrule
\end{tabular}
}
\end{table}


Given the biases discussed above, it is inevitable that some sensor readings will exhibit relatively high predictability. To quantify this, we calculate individual-sensor entropies across multiple datasets. The results are given in Table~\ref{tab:single-sensor-results}. For multi-dimensional modalities (e.g.\ triaxial accelerometer or gyroscope), these are split into separate axes following Voris et al.~\cite{voris2011accelerometers}. We note that, in the Relay dataset, the data for individual $x$, $y$ and $z$ components are not given for the accelerometer, gyroscope, and magnetometer sensors. Rather, the authors have already preprocessed triaxial data into its vector magnitudes, i.e.\ $\textbf{v} = \sqrt{x^2 + y^2 + z^2}$. We give this as ``X.Mag'' for a given sensor X. For completeness, we compute the magnitude ourselves for other datasets, where applicable, and report the entropy values for this new synthetic modality.

Several clear patterns emerge from Table~\ref{tab:single-sensor-results}. Some sensors, such as certain accelerometer axes in \emph{SHL} or \emph{PerilZIS}, exhibit moderate min-entropies of 4--6 bits. Other sensors, particularly gyroscope axes (see \emph{UCI-HAR}) show values below 3 bits, indicating high predictability in their most frequent readings. Shannon entropy values (\(H_1\)) can be fairly high (up to 10 bits in some cases), whereas min-entropy (\(H_{\infty}\)) is often much lower. This gap reflects distributions where a few outcomes dominate, thereby driving worst-case unpredictability down even if the average-case picture is more favorable. Overall, the results confirm that data from individual sensors do not provide sufficient min-entropy for robust security on their own. In the next section, we examine whether combining multiple modalities can meaningfully increase this worst-case unpredictability or whether correlated biases persist across different sensor streams.



\subsection{Multi-modal Sensors}
\label{sec:multimodal}

Several sensor-based security proposals~\cite{mehrnezhad2015tap,truong2014comparing,markantonakis2024using,shrestha2014drone,shrestha2018sensor} assert that combining multiple sensor modalities can bolster security, based on the intuition that an adversary must accurately predict several data streams, rather than just one. This section will examine that claim. 



\begin{figure*}[t!]
    \centering
    \begin{subfigure}[t]{0.45\textwidth}
        \centering
        \includegraphics[width=\textwidth]{figures/uci-har_correlation_matrix.pdf}
        \caption{UCI-HAR}
        \label{subfig:acx}
    \end{subfigure}%
    \begin{subfigure}[t]{0.5\textwidth}
        \centering
        \includegraphics[trim={0 0 3.3cm 0},clip,width=\textwidth]{figures/perilzis_correlation_matrix.pdf}
        \caption{PerilZIS}
    \end{subfigure}%

    \begin{subfigure}[t]{0.45\textwidth}
        \centering
        \includegraphics[trim={0 0 3.3cm 0},clip,width=\textwidth]{figures/relay_correlation_matrix.pdf}
        \caption{Relay}
    \end{subfigure}%
    \begin{subfigure}[t]{0.5\textwidth}
        \centering
        \includegraphics[trim={0 0 3.3cm 0},clip,width=\textwidth]{figures/shl_correlation_matrix.pdf}
        \caption{SHL}
    \end{subfigure}%
    \caption{Sensor correlation matrices for each dataset.}
    \label{fig:correlation-matrices}
\end{figure*}

A na\"{i}ve approach might add Shannon entropies from individual sensors, benefitting from the relation $H(X_1, \ldots, X_n) = \sum_{i=1}^n H(X_i)$. However, this requires that $X_i$ are \emph{statistically independent.} In reality, mobile sensors often exhibit strong dependencies. For instance, the rotation vector, gravity, and linear acceleration sensors are frequently derived in software from the accelerometer and gyroscope on consumer devices~\cite{android_motion_sensors}. As a result, these modalities \emph{cannot} be treated as independent random variables. Figure~\ref{fig:correlation-matrices} illustrates how multiple sensors in each dataset correlate: some pairs are nearly perfectly aligned (correlation close to $\pm1$), which drastically reduces their combined unpredictability. High correlations invalidate the simplistic additive model of entropy. Even if multiple modalities individually appear to have moderate unpredictability, overlapping probability distributions may limit the overall \emph{joint} entropy. In the next subsections, we discuss why straightforward joint-entropy calculations are computationally intractable at scale, before describing how Chow--Liu trees enable a practical approximation of higher-dimensional entropy.


 
 
\subsubsection{Complexity Challenges}

Computing the exact joint probability distribution and joint entropy of multiple sensors can quickly become prohibitively expensive. Let each of the $n$ sensor modalities be discretised into $b_i$ bins. Then, the joint distribution has $\prod_{i=1}^{n} b_i$ distinct states, an enormous state space once $n$ and $b_i$ grow.  Applying Freedman--Diaconis binning rules typically results in thousands of bins per modality, causing the number of joint bins to explode combinatorially.  

Moreover, even before enumerating states, \emph{selecting which sensors to combine} can itself involve $2^n - (n+1)$ subsets, skipping single-sensor subsets and the empty set. Preliminary experiments confirmed joint entropies could be computed directly for $n \leq 3$ modalities with a maximum of 1250 bins and fewer than 150K total samples from the Relay dataset. Reducing bin sizes can help, but this risks oversimplifying the distribution and artificially deflating entropy estimates. Further experiments confirmed that limiting the bin numbers reduced our single-sensor entropy estimates by approximately 2--3 bits on average compared to those reported in Table~\ref{tab:single-sensor-results}.   We therefore sought an alternative strategy that balances accuracy with tractable computation.

\subsubsection{Chow-Liu Approximation}

To handle these scaling issues, we adopt \emph{Chow--Liu trees}~\cite{chow1968approximating}, which approximate high-dimensional joint distributions using a maximum-weight spanning tree, $\pi$, over the different sensor modalities. Each edge is weighted by the mutual information of the connected variables, ensuring the tree structure captures the dominant pairwise dependencies. This approach minimises the Kullback--Leibler divergence (Def.~\ref{sec:defs}) between the true multivariate distribution and the resulting tree-based approximation as follows:


\begin{equation}
    p_{\pi}(x_1,\dots,x_n) =  
    p(x_r)\prod_{i\neq r}p(x_i\; |\; x_{\pi(x)})
\end{equation}



Where $\pi(i)$ denotes the parent of $X_i$ in the tree, and $r$ is the tree's root node. 
Chow--Liu trees are acyclic, singly connected structures: each node has at most one parent where one can traverse the tree to accumulate probabilities between pairwise dependencies. This significantly reduces computation time compared to na\"{i}ve enumeration of the full joint measurement space.  The use of Chow-Liu trees was proposed by Buller and Kaufer~\cite{buller2016estimating} for estimating the entropy of multivariate data sources where the range of possible values is high. In our Python implementation, we use the pgmpy~\cite{ankan2024pgmpy} library's TreeSearch module. Practically, for each sensor subset, we:
\begin{enumerate}
    \item Discretise each sensor's readings via Freedman--Diaconis binning.
    \item Build a Chow--Liu tree from the mutual information of each sensor pair, selecting edges to form a spanning tree.
    \item Traverse the resulting tree to estimate max ($H_0$), Shannon ($H_1$), collision ($H_2$), and min-entropy ($H_{\infty}$) without enumerating the full exponential state space.
\end{enumerate}


Our framework evaluates the joint entropy over all sensor combinations. The powerset of the sensor set is generated and processed in parallel using Python's multiprocessing module. Processing all four datasets took approximately 22 hours on our workstation with an Intel i7-6700K (8M cache, 4.20 GHz) and 32 GB RAM on Ubuntu 24.04.

\subsubsection{Results}
\label{subsubsec:results}

Tables~\ref{tab:top10-uci-har}--\ref{tab:top10-shl} report the top 10 performing multi-sensor combinations ranked by min-entropy for each dataset. As expected, combining \emph{all} sensors yields the highest $H_0$ (max-entropy) and often increases Shannon and collision entropy. However, \emph{min-entropy ($H_{\infty}$) remains stubbornly low.} For instance, the complete set of sensors in \emph{SHL} surpasses 80 bits of $H_1$ (Shannon) but saturates at only 21\,bits of $H_{\infty}$.  Interestingly, we find that \emph{omitting} certain correlated sensors sometimes does not reduce min-entropy at all. For the Relay dataset in Table~\ref{tab:top10-relay}, the combination (Acc., Gyro., Light, Lin.~Acc., Mag., Rot.~Vec.) achieves \(H_{\infty} = 7.859\) bits, only slightly below the full set's \(8.092\) bits. Parallel findings arise in the PerilZIS and SHL datasets, where omitting a small number of sensors from the ``All sensors'' set has negligible impact on \(H_{\infty}\). This pattern appears across datasets: additional modalities may raise $H_0$ and $H_1$ but barely move $H_{\infty}$.  

The results imply that many sensors contribute \emph{redundant} information, showing a fundamental limitation of multi-modal data in real-world devices. Combining signals increases the \emph{apparent} capacity for unpredictability, but correlations between sensors means that the min-entropy from a large sensor ensemble is not be substantially higher than a reduced subset thereof. We note that standards use min-entropy as a safer, worst-case metric nowadays~\cite{bsi2024ais31,turan2018recommendation}. It measures how close the distribution is to collapsing around the single most probable outcome that an adversary will target first. To the best of our knowledge, the  entropy collapse brought about by highly correlated sensor modalities has not been before in existing work.  We provide the full sensor combination results in our open-source repository.

\begin{table}
\centering
\caption{Top 10 best-performing sensor combinations (UCI-HAR; in bits).}
\resizebox{0.65\linewidth}{!}{%
\begin{tabular}{@{}r|ccc|c@{}}
\toprule
\textbf{Modality} & \( H_{0} \) & \( H_{1} \) & \( H_{2} \) & \( H_{\infty} \) \\ 
\midrule
All sensors & 48.061 & 25.089 & 17.116 & 11.008 \\\midrule
(Acc.\{x,y,z\}, Gyro.\{y,z\})           & 39.403 & 22.835 & 15.999 & 10.113 \\
(Acc.\{x,y,z\}, Gyro.\{x,y\})           & 39.886 & 21.456 & 15.332 & 9.900   \\
(Acc.\{x,y,z\}, Gyro.\{x,z\})           & 40.222 & 21.470 & 15.040 & 9.411   \\
(Acc.\{y,z\}, Gyro.\{x,y,z\})          & 40.228 & 21.306 & 14.698 & 9.274   \\
(Acc.\{x,z\}, Gyro.\{x,y,z\})          & 40.293 & 21.260 & 14.575 & 9.154   \\
(Acc.\{x,y,z\}, Gyro.y)                   & 31.228 & 18.804 & 13.893 & 9.065   \\
(Acc.\{x,y\}, Gyro.\{x,y,z\})          & 40.273 & 21.155 & 14.208 & 8.721   \\
(Acc.\{x,y,z\}, Gyro.z)                   & 31.564 & 18.775 & 13.833 & 8.682   \\
(Acc.\{y,z\}, Gyro.\{y,z\})                  & 31.570 & 18.583 & 13.400 & 8.386   \\
\bottomrule
\end{tabular}
}
\label{tab:top10-uci-har}
\end{table}


\begin{table}[t!]
\centering
\caption{Top 10 best-performing sensor combinations (Relay; in bits).}
\resizebox{0.85\linewidth}{!}{%
\begin{tabular}{@{}r|ccc|c@{}}
\toprule
\textbf{Modality} & \( H_{0} \) & \( H_{1} \) & \( H_{2} \) & \( H_{\infty} \) \\ 
\midrule
All sensors &45.794&25.036&15.725&8.092\\\midrule
(Acc., Gyro., Light, Lin. Acc., Mag., Rot. Vec.)  & 44.795	& 24.963	&15.334	&7.859  \\
(Acc., Grav., Light, Lin. Acc., Mag., Rot. Vec.)      & 38.385&21.590&14.783&7.702  \\
(Acc., Grav., Gyro., Light, Mag., Rot. Vec.) & 37.026	&21.001&14.553&7.702 \\
(Grav., Gyro., Light, Lin. Acc., Mag., Rot. Vec.) & 36.092&20.794&14.450&7.673 \\
(Acc., Light, Lin. Acc., Mag., Rot. Vec.) & 37.385&21.517&14.428&7.469\\
(Acc., Gyro., Light, Mag., Rot. Vec.) & 36.026&20.924&14.270&7.465\\
(Gyro., Light, Lin. Acc., Mag., Rot. Vec.) & 35.092&20.720&14.147&7.439\\
(Acc., Grav., Gyro., Light, Lin. Acc., Mag.) & 41.094&22.794&14.437&7.338\\
(Acc., Grav., Light, Mag., Rot. Vec.) & 29.617&17.555&13.244&7.312\\
\bottomrule
\end{tabular}
}
\label{tab:top10-relay}
\end{table}


\begin{table}[ht]
\centering
\caption{Top 10 best-performing sensor combinations (PerilZIS; in bits).}
\resizebox{\linewidth}{!}{%
\begin{tabular}{@{}r|ccc|c@{}}
\toprule
\textbf{Modalities} & \( H_{0} \) & \( H_{1} \) & \( H_{2} \) & \( H_{\infty} \) \\ 
\midrule
All sensors & 83.662 & 36.998 & 33.682 & 23.926 \\\midrule
%
(Acc.\{x,y,z\}, Light, Temp., Pres., Mag.\{x,y,z\}, Gyro.\{x,y,z\})      & 76.907&34.258&31.510&23.835  \\
%
(Acc.\{y,z\}, Light, Temp., Pres., Mag.\{x,y,z\}, Gyro.\{x,y,z\})          & 72.737&34.246&31.509&23.835 \\
%
(Acc.\{x,z\}, Light, Temp., Pres., Mag.\{x,y,z\}, Gyro.\{x,y,z\})                 &72.820&34.243&31.508&23.829  \\
%
(Acc.\{x,y\}, Light, Temp., Pres., Mag.\{x,y,z\}, Gyro.\{x,y,z\})        & 72.737	&34.229	&31.501&	22.945
  \\
(Acc.\{x,y,z\}, Light, Temp., Mag.\{x,y,z\}, Gyro.\{x,y,z\})                 & 71.384	&32.586	&30.190&	22.945
   \\
(Acc.y, Light, Temp., Press., Mag.\{x,y,z\}, Gyro.\{x,y,z\})           & 68.568&34.218&	31.500&	22.945
   \\
   %
   (Acc.x, Light, Temp., Pres., Mag.\{x,y,z\}, Gyro.\{x,y,z\}) & 68.650&34.215&31.499&22.945\\
   %
   (Acc.\{y,z\}, Light, Temp., Mag.\{x,y,z\}, Gyro.\{x,y,z\}) & 67.214	&32.575	&30.189	&22.945\\
   (Acc.\{x,y,z\}, Light, Temp., Pres., Mag.\{x,z\}, Gyro.\{x,y,z\}, Hum.) & 76.618&33.816&31.035&21.915\\
\bottomrule
\end{tabular}
}
\label{tab:top10-peril}
\end{table}



\begin{table}[ht!]
\centering
\caption{Top 10 best-performing sensor combinations (SHL; in bits).}
\resizebox{\linewidth}{!}{%
\begin{tabular}{@{}R{10cm}|ccc|c@{}}
\toprule
\textbf{Modality} & \( H_{0} \) & \( H_{1} \) & \( H_{2} \) & \( H_{\infty} \) \\ 
\midrule
All sensors 
    & 158.601 
    & 82.301 
    & 39.320 
    & 21.289 \\
\midrule
(Acc.\{x,y,z\}, Gyro.\{x,y,z\}, Mag.\{x,y,z\}, Ori.\{w,x,y,z\}, Grav.\{x,y,z\}, LinAcc.\{y,z\}, Pres., Alt., Temp.)
    & 148.995
    & 78.624
    & 39.276
    & 21.289 \vspace{0.1cm}\\

(Acc.\{x,y,z\}, Gyro.\{x,y,z\}, Mag.\{x,y,z\}, Ori.\{w,x,y,z\}, Grav.\{x,y,z\}, LinAcc.\{x,z\}, Pres., Alt., Temp.)
    & 148.759
    & 78.477
    & 39.272
    & 21.289 \vspace{0.1cm}\\

(Acc.\{x,y,z\}, Gyro.\{x,z\}, Mag.\{x,y,z\}, Ori.\{w,x,y,z\}, Grav.\{x,y,z\}, LinAcc.\{x,y,z\}, Pres., Alt., Temp.)
    & 148.587
    & 78.380
    & 39.249
    & 21.289 \vspace{0.1cm}\\

(Acc.\{x,y,z\}, Gyro.\{x,y,z\}, Mag.\{x,y,z\}, Ori.\{w,x,y,z\}, Grav.\{x,y,z\}, LinAcc.\{x,y\}, Pres., Alt., Temp.)
    & 148.952
    & 78.135
    & 39.235
    & 21.289 \vspace{0.1cm}\\

(Acc.\{x,y,z\}, Gyro.\{x,y,z\}, Mag.\{x,y,z\}, Ori.\{w,x,y,z\}, Grav.\{x,y,z\}, LinAcc.\{z\}, Pres., Alt., Temp.)
    & 139.153
    & 74.800
    & 39.175
    & 21.289 \vspace{0.1cm}\\

(Acc.\{x,y,z\}, Gyro.\{x,z\}, Mag.\{x,y,z\}, Ori.\{w,x,y,z\}, Grav.\{x,y,z\}, LinAcc.\{y,z\}, Pres., Alt., Temp.)
    & 138.981
    & 74.703
    & 39.128
    & 21.289 \vspace{0.1cm}\\

(Acc.\{x,y,z\}, Gyro.\{x,z\}, Mag.\{x,y,z\}, Ori.\{w,x,y,z\}, Grav.\{x,y,z\}, LinAcc.\{x,z\}, Pres., Alt., Temp.)
    & 138.745
    & 74.556
    & 39.117
    & 21.289 \vspace{0.1cm}\\

(Acc.\{x,y,z\}, Gyro.\{x,y,z\}, Mag.\{x,y,z\}, Ori.\{w,x,y,z\}, Grav.\{x,y,z\}, LinAcc.\{y\}, Pres., Alt., Temp.)
    & 139.346
    & 74.458
    & 39.100
    & 21.289 \vspace{0.1cm}\\

(Acc.\{x,y,z\}, Gyro.\{x,y,z\}, Mag.\{x,y,z\}, Ori.\{w,x,y,z\}, Grav.\{x,y,z\}, LinAcc.\{x\}, Pres., Alt., Temp.)
    & 139.109
    & 74.311
    & 39.087
    & 21.289 \\
\bottomrule
\end{tabular}
}
\label{tab:top10-shl}
\end{table}
\section{Evaluation}
We provide three sets of insights into this section, organised as \textit{findings (F*)}. We quantitatively study the effect of the adversarial and counterfactual perturbations on the performance of informal reasoners and autoformalisation methods. Then, we dive deeper into method variants. Finally, 
we analyse the nature of formalisation errors made by the models.

\subsection{Robustness Analysis}
\paragraph{\textbf{\emph{F1: Noise perturbations have a stronger effect on formalisation methods than informal \ac{LLM} reasoners.}}}
Table~\ref{tab:distraction_k4_formalisation} shows that, on average, the accuracy of both direct and \ac{CoT} informal reasoning remains between $73\%$ and $74\%$ in the face of added noise. While the autoformalisation method performs similarly to informal reasoners on the original dataset, its performance decreases between $4\%$ and $11\%$. The accuracy drops especially with logical (L) and tautological (T) distractions, whose logical language formats trick the \ac{LLM} into formalizing the noisy clauses. On the other hand, the linguistically complex and more natural sentences of encyclopedic distractions show a minor effect, suggesting that \acp{LLM} successfully avoids formalizing the more complicated sentences.

\paragraph{\textbf{\emph{F2: All \ac{LLM}-based reasoning methods suffer a drop for counterfactual perturbations.}}} % influence .}}}
Table~\ref{tab:distraction_k4_formalisation} shows that counterfactual statements cause a significant decrease in performance for both the informal reasoners and autoformalisation methods of between $12\%$ and $13\%$ on average. 
Moreover, this observation also holds for all tested models, i.e., none are robust towards counterfactual perturbations across every evaluated dimension. Even the strongest model, GPT 4o-mini, yields a performance of 63-68\%, which is relatively close to the random performance of 50\%. The high impact of counterfactual statements (the single ``not'' inserted) could be due to the inability of \acp{LLM} to overwrite prior knowledge with explicitly stated information or memorization of the answers. We study the error sources further in §\ref{subsec:errors}.  

\noindent \paragraph{\textbf{\emph{F3: Introducing multiple noise sentences has an effect only for logical distractions.}}}
We show the impact of introducing between one and four sentences for the two top-performing autoformalisation models in Figure~\ref{fig:length_distraction}. The figure shows similar trends with and without counterfactual perturbations.
As additional logical distractions are introduced, the model performance consistently decreases. Tautological (T) distractions lead to a decline in accuracy with a single disruptive sentence, yet adding more noise does not worsen the outcome. 
The tautological corpus introduces truth constants for all sentences as a persistent unseen logical construct. Given that this leads only to a decrease for a single occurrence, we can assume that a model can consistently handle the same unseen logical construct. In contrast, the logical corpus increases the chance of adding text, requiring new, previously unseen reasoning constructs for each added sentence. The impact of encyclopedic noise remains negligible, generalising F1 to $k$ sentences. Similarly, counterfactual perturbations remain much more effective for all settings, generalising F2.

\begin{table}[!t]
\small
\setlength{\modelspacing}{2pt}
\setlength{\tabcolsep}{1.7pt} % Default value: 6pt
\setlength{\belowrulesep}{4pt}
\begin{threeparttable}
    \centering
    \begin{tabular}{cc l r rrr @{\quad} rrrr}
\toprule
\multirow{2}{*}{} & \multirow{2}{*}{} & Reasoning & \multirow{2}{*}{O} & \multicolumn{3}{c}{Distraction} & \multicolumn{4}{c}{Counterfactual} \\
 & & Format & & E& L & T & $\text{O}_C$ & $\text{E}_C$& $\text{L}_C$ & $\text{T}_C$\\
\midrule
\multirow{6}{*}{\rotatebox{90}{Gemma-2}} & \multirow{3}{*}{\rotatebox{90}{9b}}
   & Informal (direct) & \textbf{0.78} & \textbf{0.80} & \textbf{0.79} & \textbf{0.77} & 0.58 & 0.52 & 0.50 & 0.59 \\
 & & Informal (CoT) & 0.72 & 0.78 & 0.73 & 0.76 & 0.61 & \textbf{0.57} & \textbf{0.60} & \textbf{0.66} \\
 & & Formal (FOL) & 0.62 & 0.58 & 0.52 & 0.53 & \textbf{0.63} & 0.52 & 0.46 & 0.46 \\[\modelspacing]
\cmidrule{2-11}
 & \multirow{3}{*}{\rotatebox{90}{27b}} 
   & Informal (direct) & 0.71 & 0.69 & \textbf{0.66} & \textbf{0.68} & 0.59 & 0.51 & 0.54 & 0.59 \\
 & & Informal (CoT) & 0.66 & 0.65 & 0.64 & 0.63 & 0.62 & 0.58 & \textbf{0.62} & \textbf{0.64} \\
 & & Formal (FOL) & \textbf{0.74} & \textbf{0.74} & 0.61 & 0.61 & \underline{\textbf{0.72}} & \underline{\textbf{0.67}} & 0.58 & 0.51 \\[\modelspacing]
\midrule
\multirow{6}{*}{\rotatebox{90}{Mistral}} & \multirow{3}{*}{\rotatebox{90}{7B}} 
   & Informal (direct) & 0.77 & \textbf{0.77} & 0.75 & \textbf{0.79} & \textbf{0.63} & \textbf{0.54} & \textbf{0.54} & \textbf{0.66} \\
 & & Informal (CoT) & \textbf{0.79} & 0.75 & \textbf{0.77} & 0.78 & 0.55 & 0.52 & \textbf{0.54} & 0.58 \\
 & & Formal (FOL) & 0.62 & 0.58 & 0.54 & 0.57 & 0.50 & \textbf{0.54} & 0.51 & 0.52 \\[\modelspacing]
\cmidrule{2-11}
 & \multirow{3}{*}{\rotatebox{90}{Small}} 
   & Informal (direct) & \textbf{0.77} & \textbf{0.76} & \textbf{0.76} & \textbf{0.75} & 0.61 & 0.51 & 0.56 & 0.59 \\
 & & Informal (CoT) & 0.72 & 0.72 & 0.72 & 0.71 & \textbf{0.62} & \textbf{0.59} & \textbf{0.62} & \textbf{0.68} \\
 & & Formal (FOL) & 0.68 & 0.59 & 0.53 & 0.64 & 0.54 & 0.55 & 0.49 & 0.51 \\[\modelspacing]
\midrule
\multirow{6}{*}{\rotatebox{90}{Llama-3.1}} & \multirow{3}{*}{\rotatebox{90}{8B}} 
   & Informal (direct) & 0.63 & 0.61 & 0.64 & 0.66 & 0.61 & \textbf{0.62} & 0.59 & 0.61 \\
 & & Informal (CoT) & 0.73 & \textbf{0.73} & \textbf{0.71} & \textbf{0.72} & \textbf{0.62} & 0.59 & \textbf{0.61} & \textbf{0.65} \\
 & & Formal (FOL) & \textbf{0.77} & 0.71 & 0.63 & 0.52 & 0.60 & 0.58 & 0.55 & 0.52 \\[\modelspacing]
\cmidrule{2-11}
 & \multirow{3}{*}{\rotatebox{90}{70B}} 
   & Informal (direct) & 0.77 & 0.74 & 0.74 & 0.73 & 0.62 & 0.53 & 0.56 & 0.64 \\
 & & Informal (CoT) & \textbf{0.78} & \textbf{0.75} & \textbf{0.76} & \textbf{0.76} & 0.64 & 0.61 & \textbf{0.66} & \underline{\textbf{0.73}} \\
 & & Formal (FOL) & 0.74 & 0.73 & 0.71 & 0.71 & \textbf{0.66} & \textbf{0.62} & 0.59 & 0.57 \\[\modelspacing]
 \midrule
\multirow{3}{*}{\rotatebox{90}{GPT}} & \multirow{3}{*}{\rotatebox{90}{4o-mini}} 
   & Informal (direct) & 0.78 & 0.77 & 0.79 & 0.79 & 0.64 & 0.61 & 0.61 & 0.63 \\
 & & Informal (CoT) & 0.80 & 0.80 & \underline{\textbf{0.81}} & \underline{\textbf{0.82}} & \textbf{0.68} & \textbf{0.63} & \underline{\textbf{0.68}} & \textbf{0.64} \\
 & & Formal (FOL) & \underline{\textbf{0.84}} & \underline{\textbf{0.82}} & 0.73 & 0.79 & 0.63 & 0.62 & 0.57 & 0.54 \\[\modelspacing]
 \midrule
\multicolumn{2}{c}{\multirow{3}{*}{\textbf{Avg}}} 
 & Informal (direct) & 0.74 & 0.73 & 0.73 & 0.73 & 0.61 & 0.55 & 0.56 & 0.62 \\
 & & Informal (CoT) & 0.74 & 0.74 & 0.73 & 0.74 & 0.62 & 0.58 & 0.62 & 0.65 \\
  & & Formal (FOL) & 0.72 & 0.68 &	0.61 & 0.62 & 0.61 & 0.59 & 0.54 & 0.52 \\
\bottomrule
\end{tabular}
\caption{Accuracies of informal and autoformalisation-based deductive reasoners. The best overall model per dataset is underlined; the best model version is marked in bold.}
\label{tab:distraction_k4_formalisation}
\end{threeparttable}
\end{table} 

\begin{figure}[!t]
    \centering
    \scriptsize
    \begin{tikzpicture}
        \begin{axis}[name=gpt,
            title={GPT-4o-mini},
            width=0.6\linewidth,
            height=0.6\linewidth,
            xlabel={\# Noise sentences},
            ylabel={Accuracy},
            xmin=-0.1, xmax=4.1,
            ymin=0.5, ymax=0.9,
            xtick={1,2,4},
            ytick={0.55, 0.6, 0.65, 0.75, 0.8, 0.85},
            title style={yshift=-0.6em},
            legend style={at={(1,-0.15)},
	           anchor=north,legend columns=-1},
            x label style={at={(axis description cs:1,-0.05)},anchor=north},
            y label style={at={(axis description cs:-0.15,0.5)},anchor=south},
            ymajorgrids=true,
            grid style=dashed,
        ]
            \addplot[color=blue, mark=square,]
                coordinates {
                (0,0.848076939582825)(1,0.823076903820038)(2,0.826923072338104)(4,0.821153819561005)
                };
            \addplot[color=red, mark=triangle,]
                coordinates {
                (0,0.848076939582825)(1,0.817307710647583)(2,0.801923096179962)(4,0.759615361690521)
                };
            \addplot[color=green, mark=diamond,] 
                coordinates {
                (0,0.848076939582825)(1,0.767307698726654)(2,0.769230782985687)(4,0.803846180438995)
                };
            \addplot[color=blue, mark=square*] 
                coordinates {
                (0,0.627777755260468)(1,0.622222244739533)(2,0.600000023841858)(4,0.633333325386047)
                };
            \addplot[color=red, mark=triangle*,] 
                coordinates {
                (0,0.627777755260468)(1,0.611111104488373)(2,0.611111104488373)(4,0.594444453716278)
                };
            \addplot[color=green, mark=diamond*,] 
                coordinates {
                (0,0.627777755260468)(1,0.572222232818604)(2,0.538888871669769)(4,0.555555582046509)
                };
                \legend{E,L,T,$\text{E}_C$, $\text{L}_C$ , $\text{T}_C$}
        \end{axis}

        \begin{axis}[name=llama, at={($(gpt.east)+(0.1cm,0)$)},anchor=west,
            title={Llama 3.1 70b},
            width=0.6\linewidth,
            height=0.6\linewidth,
            xmin=-0.1,, xmax=4.1,
            ymin=0.5, ymax=0.9,
            xtick={1,2,4},
            ytick={0.55, 0.6, 0.65, 0.75, 0.8, 0.85},
            title style={yshift=-0.6em},
            yticklabel=\empty,
            ymajorgrids=true,
            grid style=dashed,
        ]
            \addplot[color=blue, mark=square,]
                coordinates {
                (0,0.838461518287659)(1,0.817307710647583)(2,0.805769205093384)(4,0.817307710647583)
                };
            \addplot[color=red, mark=triangle,]
                coordinates {
                (0,0.838461518287659)(1,0.819230794906616)(2,0.803846180438995)(4,0.771153867244721)
                };
            \addplot[color=green, mark=diamond,]
                coordinates {
                (0,0.838461518287659)(1,0.803846180438995)(2,0.807692289352417)(4,0.805769205093384)
                };
            \addplot[color=blue, mark=square*]
                coordinates {
                (0,0.627777755260468)(1,0.622222244739533)(2,0.577777802944183)(4,0.594444453716278)
                };
            \addplot[color=red, mark=triangle*,]
                coordinates {
                (0,0.627777755260468)(1,0.583333313465118)(2,0.561111092567444)(4,0.577777802944183)
                };
            \addplot[color=green, mark=diamond*,]
                coordinates {
                (0,0.627777755260468)(1,0.627777755260468)(2,0.566666662693024)(4,0.577777802944183)
                };
        \end{axis}
    \end{tikzpicture}
    \caption{Influence of the number of noisy sentences for FOL.}
    \label{fig:length_distraction}
\end{figure}



\subsection{Impact of Method Design}
\paragraph{\textbf{\emph{F4: \ac{CoT} prompting is most impactful when both noise and counterfactual perturbations are applied.}}}
The accuracies for the individual \acp{LLM} in Table~\ref{tab:distraction_k4_formalisation} show that the impact of \ac{CoT} is negligible for noise-only datasets (first four columns). Meanwhile, the benefit from \ac{CoT} is most pronounced in the datasets that combine noise and counterfactual perturbations.
The better-performing informal prompting strategy for a model remains stable for all types of distractions. Still, the decline in performance due to counterfactuals leads to a less consistent preference for a specific prompting style.

\paragraph{\textbf{\emph{F5: The best-performing grammar differs per model and is unstable across data versions.}}}

The evaluation of different logical forms for formal \ac{LLM}-based reasoning in Table~\ref{tab:distraction_k4_logical_form} shows the preference of some models for specific syntactic formats.
Llama 3.1 70B has a considerable improvement of $12\%$ with TPTP syntax on the original set, while Llama 3.1 8B benefits from the R-FOL syntax. However, all grammars show a declining accuracy trend and increased syntax errors for noise perturbations, where the best grammar loses its advantage over the rest. 
When comparing the grammars on the counterfactual partitions, we observe that TPTP is consistently more robust than the standard first-order logic grammar. Here, GPT 4o-mini shows a reduction from $O$ to $O_C$ of $20\%$ for FOL and only $12\%$ for the TPTP grammar. Since this does not correlate with fewer syntax errors, the formalisation in TPTP prevents semantical errors for counterfactual premises. 
A positive reading of these results, especially the minor differences between FOL and R-FOL, is that autoformalisation \acp{LLM} can adapt to the grammar syntax prescribed in the prompt without further loss in performance.

\begin{table}[!t]
\small
\setlength{\modelspacing}{2pt}
\setlength{\tabcolsep}{1.7pt} % Default value: 6pt
\setlength{\belowrulesep}{4pt}
\begin{threeparttable}
    \centering
    \begin{tabular}{cc l r rrr @{\quad} rrrr}
\toprule
\multirow{2}{*}{} & \multirow{2}{*}{} & Grammar & \multirow{2}{*}{O} & \multicolumn{3}{c}{Distraction} & \multicolumn{4}{c}{Counterfactual} \\
 & & Syntax & & E& L & T & $\text{O}_C$ & $\text{E}_C$& $\text{L}_C$ & $\text{T}_C$\\
\midrule
\multirow{6}{*}{\rotatebox{90}{Llama-3.1}} & \multirow{3}{*}{\rotatebox{90}{8B}} 
   & FOL & 0.77 & \textbf{0.71} & 0.61 & \textbf{0.53} & 0.58 & \textbf{0.55} & 0.52 & \textbf{0.56} \\
 & & R-FOL & \textbf{0.78} & 0.69 & \textbf{0.62} & \textbf{0.53} & 0.58 & \textbf{0.55} & \textbf{0.54} & 0.52 \\
 & & TPTP & 0.73 & 0.67 & 0.55 & 0.51 & \textbf{0.68} & 0.54 & 0.46 & 0.51 \\[\modelspacing]
\cmidrule{2-11}
 & \multirow{3}{*}{\rotatebox{90}{70B}} 
   & FOL & 0.76 & 0.73 & 0.71 & \textbf{0.72} & 0.67 & 0.57 & 0.63 & 0.56 \\
 & & R-FOL & 0.76 & 0.73 & 0.67 & 0.71 & 0.64 & 0.57 & 0.53 & 0.64 \\
 & & TPTP & \underline{\textbf{0.88}} & \underline{\textbf{0.84}} & \underline{\textbf{0.81}} & \textbf{0.72} & \underline{\textbf{0.81}} & \underline{\textbf{0.68}} & \underline{\textbf{0.67}} & \underline{\textbf{0.68}} \\[\modelspacing]
\midrule
\multirow{3}{*}{\rotatebox{90}{GPT}} & \multirow{3}{*}{\rotatebox{90}{4o-mini}} 
   & FOL & \textbf{0.84} & \textbf{0.82} & \textbf{0.72} & \underline{\textbf{0.78}} & 0.64 & \textbf{0.63} & \textbf{0.61} & 0.51 \\
 & & R-FOL & \textbf{0.84} & 0.77 & 0.70 & \underline{\textbf{0.78}} & \textbf{0.72} & 0.56 & 0.54 & \textbf{0.63} \\
 & & TPTP & 0.83 & \textbf{0.82} & 0.71 & 0.71 & 0.69 & \textbf{0.63} & 0.57 & 0.57 \\
\bottomrule
\end{tabular}
\caption{Accuracies of different formalisation grammars for autoformalisation.}
\label{tab:distraction_k4_logical_form}
\end{threeparttable}
\end{table} 

\paragraph{\textbf{\emph{F6: Feedback does not help \acp{LLM} self-correct to mitigate robustness issues.}}}
\autoref{tab:distraction_k4_feedback} shows the results with different error recovery mechanisms. The results indicate that no feedback strategy emerges as a winner in the different datasets. 
All feedback variants reduce syntax errors for noise perturbations, but given the lack of a consistent increase in accuracy, the corrected formalisations are most likely to contain semantic errors still. 
The type of feedback message only has a minor influence on correcting syntax errors, whereas Llama 3.1 70b and GPT 4o-mini correct slightly more syntax errors with specific error messages. This finding aligns with \cite{huang2023large}, who also found that \acp{LLM} cannot consistently self-correct their reasoning after receiving relevant feedback.

\begin{table}[!ht]
\small
\setlength{\modelspacing}{2pt}
\setlength{\tabcolsep}{1.7pt} % Default value: 6pt
\setlength{\belowrulesep}{4pt}
\begin{threeparttable}
    \centering
    \begin{tabular}{cc l r rrr @{\quad} rrrr}
\toprule
\multirow{2}{*}{} & \multirow{2}{*}{} & \multirow{2}{*}{Feedback} & \multirow{2}{*}{O} & \multicolumn{3}{c}{Distraction} & \multicolumn{4}{c}{Counterfactual} \\
 & & & & E& L & T & $\text{O}_C$ & $\text{E}_C$& $\text{L}_C$ & $\text{T}_C$\\
\midrule
\multirow{8}{*}{\rotatebox{90}{Llama-3.1}} & \multirow{4}{*}{\rotatebox{90}{8B}} 
   & No recovery & 0.77 & \textbf{0.72} & 0.62 & 0.53 & 0.59 & 0.58 & 0.56 & \textbf{0.56} \\
 & & Error type & \textbf{0.79} & 0.71 & 0.63 & \textbf{0.56} & \textbf{0.66} & 0.54 & 0.52 & 0.51 \\
 & & Error message & 0.78 & 0.71 & \textbf{0.67} & 0.55 & 0.59 & 0.53 & \underline{\textbf{0.64}} & 0.49 \\
 & & Warning & 0.74 & 0.66 & 0.58 & 0.55 & 0.55 & \textbf{0.60} & 0.49 & 0.49 \\[\modelspacing]
\cmidrule{2-11}
 & \multirow{4}{*}{\rotatebox{90}{70B}} 
   & No recovery & \textbf{0.77} & \textbf{0.72} & \textbf{0.73} & 0.71 & \textbf{0.64} & 0.59 & \textbf{0.61} & 0.56 \\
 & & Error type & 0.72 & 0.70 & 0.72 & \textbf{0.73} & 0.62 & 0.56 & 0.60 & 0.58 \\
 & & Error message & 0.71 & 0.70 & \textbf{0.73} & 0.71 & \textbf{0.64} & 0.59 & 0.54 & \underline{\textbf{0.64}} \\
 & & Warning & 0.69 & \textbf{0.72} & 0.72 & 0.72 & 0.62 & \underline{\textbf{0.65}} & \textbf{0.61} & 0.63 \\[\modelspacing]
\midrule
\multirow{4}{*}{\rotatebox{90}{GPT}} & \multirow{4}{*}{\rotatebox{90}{4o-mini}} 
   & No recovery & \underline{\textbf{0.84}} & \underline{\textbf{0.82}} & 0.73 & 0.79 & 0.64 & \textbf{0.62} & 0.56 & \textbf{0.56} \\
 & & Error type & 0.83 & 0.79 & 0.74 & 0.76 & 0.67 & 0.57 & 0.56 & \textbf{0.56} \\
 & & Error message & \underline{\textbf{0.84}} & 0.78 & \underline{\textbf{0.77}} & \underline{\textbf{0.80}} & 0.62 & 0.59 & 0.56 & \textbf{0.56} \\
 & & Warning & \underline{\textbf{0.84}} & 0.75 & 0.73 & 0.76 & \underline{\textbf{0.70}} & 0.61 & \textbf{0.61} & 0.55 \\
 \bottomrule
\end{tabular}
\caption{Accuracies of error recovery strategies.}
\label{tab:distraction_k4_feedback}
\end{threeparttable}
\end{table} 

\subsection{Error Analysis}
\label{subsec:errors}
\paragraph{\textbf{\emph{F7: Autoformalisation increases syntax errors for noise perturbations.}}}
The low performance for noise perturbations correlates with more syntax errors for all models and distraction categories (cf. execution rates in Table~\ref{tab:appendix_k4_formalisation_exec}). The three worst-performing models (both Mistral models, Gemma-2 9b) generate, at best, for $37\%$  and, at worst, for only $4\%$ of the samples, a valid logical form.
Gemma-2 9b and Llama3.1 8b produce more syntax errors than the larger counterparts, suggesting that larger models are more robust towards noise perturbations. 
The accuracy of syntactically valid samples is higher than the informal reasoning methods for most distractions (Table~\ref{tab:appendix_k4_formalisation_vacc}), motivating informal reasoning as a backup strategy for formal reasoning. The error message feedback reveals two common syntax errors: 1) errors by models with an initial low execution rate exhibit issues with the template structure, including using incorrect keywords or adding conversational phrases;
2) perturbation-related errors, the most common of which is using undefined truth constants as part of tautological distractions. 

\paragraph{\textbf{\emph{F8: Autoformalisation increases semantic errors for counterfactuals.}}}
Unlike the introduced noise, counterfactual perturbations do not lead to more syntax errors. The execution rate in Table~\ref{tab:appendix_k4_formalisation_exec} is stable or improves for counterfactuals. However, we see a drop in accuracy for the counterfactual column $\text{O}_C$ in Table~\ref{tab:distraction_k4_formalisation} and can conclude that the number of logical forms with semantic errors has to increase. This suggests that the introduced negation is not correctly formalised. Looking at the warnings generated by the feedback mechanism, for GPT 4o-mini, $161$ warning messages are generated on the unperturbed data. $54$ of these were fixed with a single iteration. Not considering predicates and individuals as part of the context is the most frequent warning across all models. 
\section*{Conclusion}
This paper aims to enhance our understanding of the computational complexity of computing various Shapley value variants. We found that for various ML models --- including decision trees, regression tree ensembles, weighted automata, and linear regression --- both local and global interventional and baseline SHAP can be computed in polynomial time under HMM modeled distributions. This extends popular algorithms, such as TreeSHAP, beyond their empirical distributional scope. We also establish strict complexity gaps between the various SHAP variants (baseline, interventional, and conditional) and prove the intractability of computing SHAP for tree ensembles and neural networks in simplified scenarios. Overall, we present SHAP as a versatile framework whose complexity depends on four key factors: \begin{inparaenum}[(i)] \item model type, \item SHAP variant, \item distribution modeling approach, \item and local vs. global explanations\end{inparaenum}. We believe this perspective provides deeper insight into the computational complexity of SHAP, paving the way for future work.




%We believe that our framework provides a more intricate understanding of SHAP computation complexity across different models, distributions, and variants, paving the way for further research.

Our work opens promising directions for future research. First, expanding our computational analysis to other SHAP-related metrics, such as asymmetric SHAP~\citep{frye20} and SAGE~\citep{covert2020understanding}, would be valuable. Additionally, we aim to explore more expressive distribution classes and relaxed assumptions beyond those in Section \ref{sec:tractable} while maintaining tractable SHAP computation. Finally, when exact computation is intractable (Section \ref{sec:intractable}), investigating the approximability of SHAP metrics through approximation and parameterized complexity theory~\citep{downey2012parameterized} is an important direction.

%Our work opens several promising avenues for future research on the computational properties of explainable AI methods, with a particular focus on SHAP. First, it would be interesting to broaden the computational analysis conducted in this work to include other popular SHAP-related metrics in the literature, such as asymmetric SHAP \cite{frye20} and SAGE \cite{covert2020understanding}. Also, in the future, we aim to explore more expressive distribution classes and relaxed distributional assumptions—extending beyond those examined in Section \ref{sec:tractable} —that still yield tractable SHAP computation. Finally, when exact computation proves intractable (Section \ref{sec:intractable}), it is worthwhile to theoretically investigate the question of the approximability of computing the SHAP metrics across various configurations, through the lens of approximation and parametrized complexity theory \cite{arora2009computational}.

%This paper aims to deepen our understanding of the computational complexity involved in obtaining different Shapley value variants. We found that for a variety of ML models, including decision trees, tree ensembles for regression, weighted automata, and linear regression models — computing both local and global interventional and baseline SHAP can be done in polynomial time when distributions are modeled by HMMs. This extends the distributional scope of popular algorithms like TreeSHAP, which is limited to empirical distributions. Additionally, we demonstrate a strict complexity gap between SHAP variants, showing that interventional and baseline SHAP can be strictly easier to compute than conditional SHAP. Despite these positive results, we uncovered intractability for various SHAP variants in neural networks and tree ensembles. Finally, we provided generalized complexity relations across SHAP variants. We believe that our framework offers a deeper understanding of the complexity involved in computing SHAP across various variants, models, distributions, as well as in both local and global computations, laying the groundwork for future research.

\section*{Acknowledgments} Carlton Shepherd has received funding from the UK EPSRC `Chameleon' project (EP/Y030168/1). The authors would like to thank Darren Hurley-Smith for insightful conversations on the topic.


%
% ---- Bibliography ----
%
% BibTeX users should specify bibliography style 'splncs04'.
% References will then be sorted and formatted in the correct style.
%
\bibliographystyle{elsarticle-num}
\bibliography{bibliography}
%
\end{document}
