\section{Introduction}
\label{sec:intro}

Modern mobile devices come equipped with an array of embedded sensors---accelerometers, gyroscopes, magnetometers, and others---that capture continuous motion and environmental data at fine temporal granularity. This rich sensor data has enabled applications from activity recognition to context-aware computing. More recently, research has proposed leveraging these signals for security-critical tasks such as cryptographic key generation, zero-interaction device pairing, and continuous authentication. A crucial yet under-explored assumption underpins such designs: sensor data provides sufficient unpredictability to thwart adversarial inference. Traditional ``shake-to-pair'' protocols~\cite{mayrhofer2009shake} rely on motion patterns to establish secure communication between co-located devices, while other methods have incorporated ambient phenomena, such as characteristics of magnetic fields and thermal fluctuations, to mitigate relay attacks~\cite{shrestha2014drone,shrestha2018sensor,markantonakis2024using,gurulian2017effectiveness,shepherd2017applicability} and reduce user authentication prompts~\cite{krhovjak2007sources,riva2012progressive,shi2011senguard,miettinen2014conxsense,li2013unobservable}. 

Despite these advancements, fundamental issues remain: multi-modal sensing is often advocated to counter sensor-specific weaknesses~\cite{shrestha2014drone,markantonakis2024using,truong2014comparing,mehrnezhad2015tap}, but the quantitative security benefits of combining multiple sensors has not been rigorously evaluated. Many existing studies rely on heuristic assessments or machine learning classifiers, e.g.~\cite{markantonakis2024using,mehrnezhad2015tap,gurulian2018good,shrestha2014drone,truong2014comparing}, that do not address critical security questions. That is, firstly, how much entropy do sensors truly provide? And, secondly, to what extent do multi-modal sensor combinations provide security gains? Understanding the \emph{underlying} entropy is important: even if different sensors are combined, fused or otherwise transformed, it does \emph{not} fundamentally improve the quantity of entropy, or unpredictability, inherent in such signals. This paper investigates those concerns.

Our analysis reveals systemic limitations: commodity sensors exhibit significant biases of between 3.408--4.483 bits of min-entropy (5.584--9.266 bits of Shannon entropy on average). In this paper, we analyse 25 different sensors compared to a far smaller number explored in related work, i.e.\ \cite{voris2011accelerometers} (1 sensor), \cite{hennebert2013entropy} (10), \cite{krhovjak2007sources} (2), and \cite{lv2020analysis} (3). Furthermore, to the best of our knowledge, we also present the first multi-modal entropy analysis at this scale. We find that, while multi-modal sensor usage confers some benefits, non-uniform distributions and inter-sensor correlations significantly reduce the worst-case min-entropy by $\approx$40--75\% compared to average-case Shannon entropy. These findings challenge the notion that increasing the number of sensors reliably strengthens security, and it underscores the inadequacy of using sensor data as a dependable entropy source. Our contributions are as follows:
\begin{itemize} 
\item We introduce the first systematic approach to evaluating sensor entropy across such a comprehensive range of modalities and datasets using various entropy metrics (max, Shannon, collision, and min-entropy).
\item We empirically demonstrate how inter-sensor correlations and biases erode entropy, casting doubt on the proposition that using multiple sensors adds substantially to security. 
\item We show how the collapse in worst-case entropy opens the door to attacks that exhaustively enumerate, or brute force, the (joint) measurement space. This gives rise to fundamental security risks to schemes that rely on signals from single and multiple mobile sensors.
\item Ultimately, we advise against relying on commodity sensors as sources of unpredictability for security-critical applications, both on a single- and multiple-sensor basis.
\end{itemize}

The rest of this paper is organised in the following way: \S\ref{sec:background} discusses sensor-based security mechanisms and established entropy metrics. \S\ref{sec:design} explains our experiment design for entropy estimation, including the threat model and dataset selection. \S\ref{sec:entropy-analysis} presents empirical results of our analyses and \S\ref{sec:security-evaluation} discusses the implications for system design. We conclude in \S\ref{sec:conc} with recommendations for further work. Our analysis work is released publicly to foster future research.\footnote{\url{https://github.com/cgshep/entropy-collapse-mobile-sensors}}
