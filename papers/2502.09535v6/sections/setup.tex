\section{Experiment Design}
\label{sec:design}

%This section details our adversarial assumptions and experimental methodology for estimating global entropy and min-entropy values across multiple mobile sensor modalities. We begin by defining a threat model that captures realistic attacks on sensor-based security schemes, then outline our high-level methodology and its rationale.

\subsection{Threat Model and Assumptions}

We consider an adversary aiming to compromise sensor-based security schemes (e.g., key generation, proximity detection, and continuous authentication). The adversary may gather extensive statistical data about how smartphone sensors behave in everyday usage; for instance, from widely available open datasets. We assume the attacker focuses on predicting or guessing the sensor outputs by prioritising the most probable values first, exploiting any biases in the distribution of sensor measurements. As such, they may resort to exhaustive enumeration of the measurement space if the target source's entropy is low enough.
We exclude capabilities such as fault injection and other hardware attacks (see~\cite{shepherd2021physical}). While those could further reduce the effective entropy space---say, by inducing errors in the output values of sensing hardware---we regard them as out-of-scope in this work. This threat model thus represents an adversary who can capitalise on statistical biases in sensor data without directly comprising the device physically. Our goal is to evaluate whether sensor data distributions---even aggregated from diverse users---offer sufficient entropy to resist attacks that search the space of sensor measurement values informed by their statistical properties.

\subsection{High-level Methodology}

We aim to determine the \emph{global} (i.e., population-level) entropy characteristics of various mobile sensors under ordinary usage conditions, rather than focusing on per-user or scenario-specific differences. This choice reflects common real-world deployments, which must accommodate a wide range of behaviors and environments. Our approach involves five main stages:

\begin{enumerate}
    \item We acquire large-scale sensor readings from publicly available datasets that capture diverse user activities and device usage patterns. These datasets encompass different motion, environmental, and orientation sensors. Detailed descriptions of each dataset are provided at the end of this section.
    %
    \item We merge sensor readings into a single, global distribution for each sensor modality in each dataset. For sensors that are inherently discrete or quantised (e.g., integer output ranges), we simply count occurrences. For sensors that produce (quasi-)continuous values, we rely on quantisation using Freedman–Diaconis binning to partition the output space and approximate an empirical probability mass function.
    %
    \item From these global distributions, we compute max, Shannon and collusion entropies to measure the best- and average-case uncertainties of sensor outputs, along with the min-entropy to characterise the worst-case unpredictability.
    %
    \item Many real-world proposals combine multiple sensor streams to purportedly increase security. To assess the impact on worst-case unpredictability, we use Chow-Liu trees to approximate the joint distributions of different sensor modalities. This allows us to estimate higher-dimensional entropies without incurring prohibitive computational costs. We discuss this in \S\ref{sec:multimodal}.
    \item Finally, we interpret the resulting entropy measures, focusing on whether sensor outputs remain sufficiently unpredictable against an informed adversary. We compare single-sensor versus multi-sensor scenarios to verify if combining modalities truly alleviates biases or simply adds redundant data susceptible to similar predictability concerns.
    %

\end{enumerate}

Throughout this process, we remain mindful of well-documented constraints with NIST SP 800-90B, SP 800-22~\cite{turan2018recommendation}, and AIS 20/31~\cite{bsi2024ais31} in analysing multi-sensor data streams~\cite{buller2016estimating,lv2020analysis}. Such  frameworks were not designed to analyse the joint entropy of complex, multivariate data sources, which is the aim of this work. (For example, NIST SP 800-90B focusses on assessing univariate entropy sources with reduced, i.e.\ 8-bit, output sizes~\cite{buller2016estimating,turan2018recommendation}).