\section{Conclusion}
\label{sec:conc}
This paper provides a comprehensive analysis of sensor-derived entropy across multiple datasets and modalities. Our results expose a tension between the \emph{perceived} and \emph{actual} strength of sensor data for security applications. Even in the best-performing sensor combinations, seemingly suitable results using one metric collapse to insecure levels when using standard worst-case metrics. Notably, modalities that yield `good' max- or Shannon entropies, representing the best- and average-case unpredictability, have insecure worst-case min-entropies.  Consequently, sensor modalities that may appear robust have biases that may enable adversaries to predict the most probable values with minimal effort. 

Our findings also challenge the prevailing orthodoxy that model evaluation metrics (e.g.\ accuracy or EER) suffice to demonstrate the inherent randomness of sensor signals. The vulnerability of mobile sensors to biased distributions significantly undermine their effectiveness as reliable entropy sources. We also cast doubt on the use of sensors with respect to their non-stationary and lack of reproducibility. These issues collectively contradict the criteria articulated in frameworks such as NIST SP 800-90B, which emphasise noise-source stationarity and protection from external influence. The effectiveness of countermeasures remains an open research challenge. Our hope is that the methodologies and insights presented here will encourage the security community to adopt more rigorous evaluation strategies for sensor-based techniques, paving the way for safer and more robust designs in mobile device security.


In future work, we consider that a dynamic analysis is important to assess the stationarity issues with sensor data, where entropy varies according to user behavior or environment. Moreover, a user study could yield empirical bounds on how finely humans can \emph{intentionally} manipulate motion sensors, revealing more realistic limits to sensor-based randomness in real-world scenarios. For example, our binning decision was a statistical one, rather one that reflects human usage; it is possible that real-world usage may reduce the resolution of useful sensor data, thus reducing entropy. Overall, while sensor-based data can reach limited levels of entropy under favourable conditions, the road to making such sources systematically \emph{secure}  and \emph{robust} is long. The key takeaway is that substantial work is needed before sensors can be considered appropriate for security-critical applications.