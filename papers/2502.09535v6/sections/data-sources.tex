To begin with, we sought publicly available sensor datasets suitable for analysing motion and environmental data at scale. Our search involved broad queries across IEEE DataPort, Google Scholar, Google Dataset Search, and GitHub. Several ostensibly ``open'' datasets either were no longer downloadable or imposed restrictive licensing terms~\cite{Mahbub_Btas2016_UMDAA02,stragapede2023behavepassdb,acien2021becaptcha}. Ultimately, we narrowed our scope to four datasets that offer diverse usage contexts, consistent sampling rates, and documented sensor modalities:

\begin{itemize}
    \item \emph{UCI-HAR}~\cite{anguita2013public}:  A widely referenced dataset for human activity recognition, comprising smartphone sensor recordings from multiple subjects performing daily activities. Data includes triaxial accelerometer and gyroscope signals.

    
    \item \emph{University of Sussex--Huawei Locomotion (SHL)}~\cite{gjoreski2018university,wang2019enabling}:   sampled at 100 Hz from an Huawei Mate 9 smartphone. The publicly available SHL Preview dataset is used, comprising three recording-days per user (59 hours of data in total). To scope this study, we use the dataset from the handheld mobile phone as a good fit with related work.
    \item \emph{Relay}~\cite{gurulian2017effectiveness}: Contains sensor measurements for approximately 1{,}500 NFC-based contactless transactions, each recorded at 100\,Hz across several physical locations (e.g., caf\'{e}s). The dataset encompasses accelerometer, gyroscope, and environmental readings taken in realistic payment scenarios.
    \item \emph{PerilZIS}~\cite{fomichev2019perils}: Collected at 10\,Hz from a Texas Instruments SensorTag, a Samsung Galaxy S6, and a Samsung Galaxy Gear, this dataset spans multiple zero-interaction security use cases in an office environment.
\end{itemize}

These four datasets provide a variety of sensor types, user activities, and sampling rates, allowing us to explore how intrinsic biases and correlations manifest across different scenarios. Next, we detail how we preprocess and aggregate this data to form global distributions for our entropy analyses.