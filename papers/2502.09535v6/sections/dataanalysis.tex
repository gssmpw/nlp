\section{Entropy Analysis}
\label{sec:entropy-analysis}

In this section, we analyse the intrinsic entropy of sensor data under the threat model described in \S\ref{sec:design}. We begin by discussing the challenges in quantising naturally continuous sensor values for discrete-entropy calculations, then present our findings for our single- and multi-sensor analyses.

\subsection{Pre-processing}

A crucial, yet underexplored, issue in prior work (e.g.\ \cite{voris2011accelerometers,lv2020analysis,hennebert2013entropy,mai2017guessability}) is how to convert inherently continuous sensor outputs into suitable discrete values for entropy estimation. For example, Shannon and min-entropy, as defined in Eqs.~\ref{eq:renyi_H1} and \ref{eq:renyi_Hinf}, rely on discrete random variables. Physical quantities such as linear acceleration or angular velocity are continuous in nature, even though modern sensors employ internal analog-to-digital conversion with a finite resolution. Yet, a sensor's advertised resolution (e.g.\ 12 bits for the widely used Bosch BMA mobile accelerometer~\cite{bosch_bma400}) does \emph{not} imply uniform coverage across its range. Everyday usage introduces biases and clustering, resulting in some measurements occurring far more frequently than others. For instance, \emph{UCI-HAR} data shows accelerometer readings concentrated in certain areas, and approximately 60\% of gyroscope readings hover near zero (see Figure~\ref{fig:acc-cdfs}). Such skew and bias radically diminishes entropy compared to uniformly distributed values.

\begin{figure*}
    \centering
    \begin{subfigure}[t]{0.333\textwidth}
        \centering
        \includegraphics[width=\textwidth]{figures/graphs/acc_x_cdf.pdf}
        \caption{Acc.\ $x$ axis.}
        \label{subfig:acx}
    \end{subfigure}%
    \begin{subfigure}[t]{0.333\textwidth}
        \centering
        \includegraphics[width=\textwidth]{figures/graphs/acc_y_cdf.pdf}
        \caption{Acc.\ $y$ axis.}
        \label{subfig:acy}
    \end{subfigure}%
    \begin{subfigure}[t]{0.333\textwidth}
        \centering
        \includegraphics[width=\textwidth]{figures/graphs/acc_z_cdf.pdf}
        \caption{Acc.\ $z$ axis.}
        \label{subfig:acz}
    \end{subfigure}

    \begin{subfigure}[t]{0.333\textwidth}
        \centering
        \includegraphics[width=\textwidth]{figures/graphs/gyro_x_cdf.pdf}
        \caption{Gyro.\ $x$ axis.}
        \label{subfig:gyx}
    \end{subfigure}%
    \begin{subfigure}[t]{0.333\textwidth}
        \centering
        \includegraphics[width=\textwidth]{figures/graphs/gyro_y_cdf.pdf}
        \caption{Gyro.\ $y$ axis.}
        \label{subfig:gyy}
    \end{subfigure}%
    \begin{subfigure}[t]{0.333\textwidth}
        \centering
        \includegraphics[width=\textwidth]{figures/graphs/gyro_z_cdf.pdf}
        \caption{Gyro.\ $z$ axis.}
        \label{subfig:gyz}
    \end{subfigure}

    \begin{subfigure}[t]{0.333\textwidth}
        \centering
        \includegraphics[width=\textwidth]{figures/graphs/Accelerometer_cdf.pdf}
        \caption{Accelerometer.}
        \label{subfig:acc}
    \end{subfigure}%
    \begin{subfigure}[t]{0.333\textwidth}
        \centering
        \includegraphics[width=\textwidth]{figures/graphs/Gyroscope_cdf.pdf}
        \caption{Gyroscope.}
        \label{subfig:gyro}
    \end{subfigure}%
    \begin{subfigure}[t]{0.333\textwidth}
        \centering
        \includegraphics[width=\textwidth]{figures/graphs/Light_cdf.pdf}
        \caption{Light.}
        \label{subfig:light}
    \end{subfigure}

    \begin{subfigure}[t]{0.333\textwidth}
        \centering
        \includegraphics[width=\textwidth]{figures/graphs/LinearAcceleration_cdf.pdf}
        \caption{Linear Acceleration.}
        \label{subfig:linacc}
    \end{subfigure}%
    \begin{subfigure}[t]{0.333\textwidth}
        \centering
        \includegraphics[width=\textwidth]{figures/graphs/MagneticField_cdf.pdf}
        \caption{Magnetometer.}
        \label{subfig:mag}
    \end{subfigure}%
    \begin{subfigure}[t]{0.333\textwidth}
        \centering
        \includegraphics[width=\textwidth]{figures/graphs/RotationVector_cdf.pdf}
        \caption{Rotation Vector.}
        \label{subfig:rv}
    \end{subfigure}
    \caption{Global sensor data CDFs -- UCI-HAR (a--f) and Relay (g--l) datasets.}
    \label{fig:acc-cdfs}
\end{figure*}

Another practical challenge arises when extremely fine-grained values appear infrequently or with negligible probability in reality. Treating every minute fluctuation (e.g.\ 9.001$ms^{-2}$ vs.\ 9.002$ms^{-2}$ for an accelerometer) as distinct outcomes can also artificially inflate entropy estimates.  In real-world applications, it is the `similarity' between measurement signals that is considered useful in existing work. It would be extremely difficult for users to reliably reproduce high-precision movements capable of effectively utilising a sensor's digital resolution (say at 0.001ms$^{-2}$ for an accelerometer). To address this, we discretise the data values into bins of similar value. However, this raises a further question of what constitutes a good strategy for selecting the number of bins and their widths? Several techniques exist that make assumptions about the underlying distribution, e.g.\ Gaussian; have different computational complexities; and are robust to outliers and data variability. To this end, we use the Freedman-Diaconis method, a commonly used robust estimator that accounts for data size and its variability.\footnote{Alternatively, a binning strategy could be employed that reflects how precisely humans can realistically replicate sensor-input changes. We defer this to future research.} This is calculated in Eq.~\ref{sec:freedman}, where $IQR(x)$ represents the interquartile range of $x$ and $n$ is the total number of samples.


\begin{equation}
    h = 2 \cdot \frac{IQR(x)}{n^{1/3}}
    \label{sec:freedman}
\end{equation}


\subsection{Single Sensors}



\begin{table}
\renewcommand{\arraystretch}{2}
\centering
\caption{Single-sensor entropy values (in bits) for each dataset. Grey cells denote unavailable data for that dataset and modality.}
\resizebox{\linewidth}{!}{%
\label{tab:single-sensor-results}
\small
\begin{tabular}{@{}r|ccc|c|ccc|c|ccc|c|ccc|c@{}}
\toprule
 & \multicolumn{16}{c}{\textbf{Dataset}} \\
 & \multicolumn{4}{c|}{\textbf{UCI-HAR}} 
 & \multicolumn{4}{c|}{\textbf{SHL}}  
 & \multicolumn{4}{c|}{\textbf{Relay}*}  
 & \multicolumn{4}{c}{\textbf{PerilZIS}} \\
\midrule
\textbf{Sensor} 
 & \(H_{0}\) & \(H_{1}\) & \(H_{2}\) & \(H_{\infty}\) 
 & \(H_{0}\) & \(H_{1}\) & \(H_{2}\) & \(H_{\infty}\) 
 & \(H_{0}\) & \(H_{1}\) & \(H_{2}\) & \(H_{\infty}\) 
 & \(H_{0}\) & \(H_{1}\) & \(H_{2}\) & \(H_{\infty}\) \\
\midrule
Acc.X       
 & 8.488 & 7.080 & 5.876 & 3.729 
 & 11.557 & 8.732  & 7.487  & 4.543 
 & \missingcell & \missingcell & \missingcell & \missingcell 
 & 13.012 & 9.292  & 6.359  & 3.626 \\
Acc.Y       
 & 8.243 & 7.231 & 6.847 & 5.694 
 & 11.425 & 8.928  & 7.717  & 4.500 
 & \missingcell & \missingcell & \missingcell & \missingcell 
 & 9.549  & 5.873  & 4.483  & 2.889 \\
Acc.Z       
 & 8.455 & 7.397 & 7.069 & 6.020 
 & 10.428 & 7.627  & 6.366  & 3.785 
 & \missingcell & \missingcell & \missingcell & \missingcell 
 & 9.817  & 6.671  & 5.403  & 4.002 \\
Acc.Mag    
 & 8.895 & 6.284 & 4.819 & 3.489 
 & 14.583 & 10.136 & 8.710  & 6.435 
 & 10.145 & 6.843 & 5.808 & 4.538
 & 13.328 & 8.273  & 7.115  & 4.526 \\
\midrule
Gyro.X     
 & 8.683 & 5.430 & 3.504 & 1.929 
 & 15.024 & 10.532 & 8.107  & 4.993 
 & \missingcell & \missingcell & \missingcell & \missingcell 
 & 14.528 & 7.231  & 4.454  & 2.805 \\
Gyro.Y     
 & 8.439 & 5.023 & 3.461 & 2.300 
 & 15.085 & 10.283 & 7.601  & 4.827 
 & \missingcell & \missingcell & \missingcell & \missingcell 
 & 14.078 & 6.715  & 4.039  & 2.529 \\
Gyro.Z     
 & 8.714 & 5.675 & 3.948 & 2.363 
 & 15.281 & 10.070 & 5.708  & 3.083 
 & \missingcell & \missingcell & \missingcell & \missingcell 
 & 13.961 & 6.463  & 3.836  & 2.434 \\
Gyro.Mag  
 & 8.414 & 5.759 & 4.130 & 2.537 
 & 12.123 & 7.816  & 5.728  & 3.699 
 & 7.954 & 4.751 & 3.442 & 2.083 
 & 14.166 & 5.565  & 1.932  & 0.969 \\
\midrule
Mag.X   
 & \missingcell & \missingcell & \missingcell & \missingcell 
 & 12.845 & 8.840  & 8.386  & 6.374 
 & \missingcell & \missingcell & \missingcell & \missingcell 
 & 10.767 & 7.639  & 6.816  & 4.883 \\
Mag.Y   
 & \missingcell & \missingcell & \missingcell & \missingcell 
 & 12.263 & 8.737  & 8.314  & 6.223 
 & \missingcell & \missingcell & \missingcell & \missingcell 
 & 10.179 & 7.622  & 6.605  & 4.405 \\
Mag.Z   
 & \missingcell & \missingcell & \missingcell & \missingcell 
 & 12.516 & 8.586  & 8.217  & 6.228 
 & \missingcell & \missingcell & \missingcell & \missingcell 
 & 10.129 & 7.507  & 6.726  & 4.448 \\
Mag.Mag 
 & \missingcell & \missingcell & \missingcell & \missingcell 
 & 13.558 & 9.436  & 8.771  & 7.148 
 & 7.972 & 6.147 & 5.617 & 4.254
 & 10.293 & 7.329  & 6.489  & 4.454 \\
\midrule
Rot.\ Vec.\ 
 & \missingcell & \missingcell & \missingcell & \missingcell 
 & 8.725  & 7.721  & 5.970  & 3.220 
 & 5.000 & 3.307 & 1.965 & 1.021
 & \missingcell & \missingcell & \missingcell & \missingcell \\
\midrule
Grav.X     
 & \missingcell & \missingcell & \missingcell & \missingcell 
 & 9.014  & 8.482  & 7.266  & 4.299 
 & \missingcell & \missingcell & \missingcell & \missingcell 
 & \missingcell & \missingcell & \missingcell & \missingcell \\
Grav.Y     
 & \missingcell & \missingcell & \missingcell & \missingcell 
 & 9.338  & 8.770  & 7.602  & 4.453 
 & \missingcell & \missingcell & \missingcell & \missingcell 
 & \missingcell & \missingcell & \missingcell & \missingcell \\
Grav.Z     
 & \missingcell & \missingcell & \missingcell & \missingcell 
 & 8.180  & 7.193  & 5.418  & 3.036 
 & \missingcell & \missingcell & \missingcell & \missingcell 
 & \missingcell & \missingcell & \missingcell & \missingcell \\
Grav.Mag     
 & \missingcell & \missingcell & \missingcell & \missingcell 
 & 14.373 & 7.988  & 7.227  & 6.242 
 & 7.794 & 6.325 & 5.864 & 4.532
 & \missingcell & \missingcell & \missingcell & \missingcell \\
\midrule
LinAcc.X  
 & \missingcell & \missingcell & \missingcell & \missingcell 
 & 15.260 & 10.077 & 7.621  & 5.116 
 & \missingcell & \missingcell & \missingcell & \missingcell 
 & \missingcell & \missingcell & \missingcell & \missingcell \\
LinAcc.Y  
 & \missingcell & \missingcell & \missingcell & \missingcell 
 & 14.859 & 10.116 & 7.639  & 5.224 
 & \missingcell & \missingcell & \missingcell & \missingcell 
 & \missingcell & \missingcell & \missingcell & \missingcell \\
LinAcc.Z  
 & \missingcell & \missingcell & \missingcell & \missingcell 
 & 14.377 & 9.951  & 7.605  & 4.543 
 & \missingcell & \missingcell & \missingcell & \missingcell 
 & \missingcell & \missingcell & \missingcell & \missingcell \\
LinAcc.Mag  
 & \missingcell & \missingcell & \missingcell & \missingcell 
 & 12.777 & 7.968  & 5.752  & 3.420 
 & 9.175 & 6.385 & 5.424 & 4.222
 & \missingcell & \missingcell & \missingcell & \missingcell \\
\midrule
Light       
 & \missingcell & \missingcell & \missingcell & \missingcell 
 & \missingcell & \missingcell & \missingcell & \missingcell 
 & 7.200& 5.331 & 4.507& 3.206
 & 12.152 & 7.940  & 7.137  & 4.552 \\
Humidity    
 & \missingcell & \missingcell & \missingcell & \missingcell 
 & \missingcell & \missingcell & \missingcell & \missingcell 
 & \missingcell & \missingcell & \missingcell & \missingcell 
 & 7.943  & 7.048  & 6.774  & 5.546 \\
Temp.       
 & \missingcell & \missingcell & \missingcell & \missingcell 
 & 7.295  & 4.753  & 2.611  & 1.332 
 & \missingcell & \missingcell & \missingcell & \missingcell 
 & 8.484  & 7.416  & 6.941  & 5.449 \\
Pressure    
 & \missingcell & \missingcell & \missingcell & \missingcell 
 & 9.461  & 8.170  & 7.723  & 6.237 
 & \missingcell & \missingcell & \missingcell & \missingcell 
 & 8.044  & 7.006  & 6.370  & 5.073 \\
\midrule 
\textbf{Mean} 
 & 8.541 & 6.235 & 4.957 & 3.508 
 & 13.188 & 9.266 & 7.178 & 4.483 
 & 7.891 & 5.584 & 4.661 & 3.408 
 & 11.277 & 7.224 & 5.717 & 3.912 \\
\textbf{S.D.} 
 & 0.207 & 0.904 & 1.461 & 1.574 
 & 1.993 & 1.148 & 1.115 & 1.018 
 & 1.612 & 1.227 & 1.474 & 1.379 
 & 2.312 & 0.900 & 1.526 & 1.266 \\\bottomrule
\end{tabular}
}
\end{table}


Given the biases discussed above, it is inevitable that some sensor readings will exhibit relatively high predictability. To quantify this, we calculate individual-sensor entropies across multiple datasets. The results are given in Table~\ref{tab:single-sensor-results}. For multi-dimensional modalities (e.g.\ triaxial accelerometer or gyroscope), these are split into separate axes following Voris et al.~\cite{voris2011accelerometers}. We note that, in the Relay dataset, the data for individual $x$, $y$ and $z$ components are not given for the accelerometer, gyroscope, and magnetometer sensors. Rather, the authors have already preprocessed triaxial data into its vector magnitudes, i.e.\ $\textbf{v} = \sqrt{x^2 + y^2 + z^2}$. We give this as ``X.Mag'' for a given sensor X. For completeness, we compute the magnitude ourselves for other datasets, where applicable, and report the entropy values for this new synthetic modality.

Several clear patterns emerge from Table~\ref{tab:single-sensor-results}. Some sensors, such as certain accelerometer axes in \emph{SHL} or \emph{PerilZIS}, exhibit moderate min-entropies of 4--6 bits. Other sensors, particularly gyroscope axes (see \emph{UCI-HAR}) show values below 3 bits, indicating high predictability in their most frequent readings. Shannon entropy values (\(H_1\)) can be fairly high (up to 10 bits in some cases), whereas min-entropy (\(H_{\infty}\)) is often much lower. This gap reflects distributions where a few outcomes dominate, thereby driving worst-case unpredictability down even if the average-case picture is more favorable. Overall, the results confirm that data from individual sensors do not provide sufficient min-entropy for robust security on their own. In the next section, we examine whether combining multiple modalities can meaningfully increase this worst-case unpredictability or whether correlated biases persist across different sensor streams.



\subsection{Multi-modal Sensors}
\label{sec:multimodal}

Several sensor-based security proposals~\cite{mehrnezhad2015tap,truong2014comparing,markantonakis2024using,shrestha2014drone,shrestha2018sensor} assert that combining multiple sensor modalities can bolster security, based on the intuition that an adversary must accurately predict several data streams, rather than just one. This section will examine that claim. 



\begin{figure*}[t!]
    \centering
    \begin{subfigure}[t]{0.45\textwidth}
        \centering
        \includegraphics[width=\textwidth]{figures/uci-har_correlation_matrix.pdf}
        \caption{UCI-HAR}
        \label{subfig:acx}
    \end{subfigure}%
    \begin{subfigure}[t]{0.5\textwidth}
        \centering
        \includegraphics[trim={0 0 3.3cm 0},clip,width=\textwidth]{figures/perilzis_correlation_matrix.pdf}
        \caption{PerilZIS}
    \end{subfigure}%

    \begin{subfigure}[t]{0.45\textwidth}
        \centering
        \includegraphics[trim={0 0 3.3cm 0},clip,width=\textwidth]{figures/relay_correlation_matrix.pdf}
        \caption{Relay}
    \end{subfigure}%
    \begin{subfigure}[t]{0.5\textwidth}
        \centering
        \includegraphics[trim={0 0 3.3cm 0},clip,width=\textwidth]{figures/shl_correlation_matrix.pdf}
        \caption{SHL}
    \end{subfigure}%
    \caption{Sensor correlation matrices for each dataset.}
    \label{fig:correlation-matrices}
\end{figure*}

A na\"{i}ve approach might add Shannon entropies from individual sensors, benefitting from the relation $H(X_1, \ldots, X_n) = \sum_{i=1}^n H(X_i)$. However, this requires that $X_i$ are \emph{statistically independent.} In reality, mobile sensors often exhibit strong dependencies. For instance, the rotation vector, gravity, and linear acceleration sensors are frequently derived in software from the accelerometer and gyroscope on consumer devices~\cite{android_motion_sensors}. As a result, these modalities \emph{cannot} be treated as independent random variables. Figure~\ref{fig:correlation-matrices} illustrates how multiple sensors in each dataset correlate: some pairs are nearly perfectly aligned (correlation close to $\pm1$), which drastically reduces their combined unpredictability. High correlations invalidate the simplistic additive model of entropy. Even if multiple modalities individually appear to have moderate unpredictability, overlapping probability distributions may limit the overall \emph{joint} entropy. In the next subsections, we discuss why straightforward joint-entropy calculations are computationally intractable at scale, before describing how Chow--Liu trees enable a practical approximation of higher-dimensional entropy.


 
 
\subsubsection{Complexity Challenges}

Computing the exact joint probability distribution and joint entropy of multiple sensors can quickly become prohibitively expensive. Let each of the $n$ sensor modalities be discretised into $b_i$ bins. Then, the joint distribution has $\prod_{i=1}^{n} b_i$ distinct states, an enormous state space once $n$ and $b_i$ grow.  Applying Freedman--Diaconis binning rules typically results in thousands of bins per modality, causing the number of joint bins to explode combinatorially.  

Moreover, even before enumerating states, \emph{selecting which sensors to combine} can itself involve $2^n - (n+1)$ subsets, skipping single-sensor subsets and the empty set. Preliminary experiments confirmed joint entropies could be computed directly for $n \leq 3$ modalities with a maximum of 1250 bins and fewer than 150K total samples from the Relay dataset. Reducing bin sizes can help, but this risks oversimplifying the distribution and artificially deflating entropy estimates. Further experiments confirmed that limiting the bin numbers reduced our single-sensor entropy estimates by approximately 2--3 bits on average compared to those reported in Table~\ref{tab:single-sensor-results}.   We therefore sought an alternative strategy that balances accuracy with tractable computation.

\subsubsection{Chow-Liu Approximation}

To handle these scaling issues, we adopt \emph{Chow--Liu trees}~\cite{chow1968approximating}, which approximate high-dimensional joint distributions using a maximum-weight spanning tree, $\pi$, over the different sensor modalities. Each edge is weighted by the mutual information of the connected variables, ensuring the tree structure captures the dominant pairwise dependencies. This approach minimises the Kullback--Leibler divergence (Def.~\ref{sec:defs}) between the true multivariate distribution and the resulting tree-based approximation as follows:


\begin{equation}
    p_{\pi}(x_1,\dots,x_n) =  
    p(x_r)\prod_{i\neq r}p(x_i\; |\; x_{\pi(x)})
\end{equation}



Where $\pi(i)$ denotes the parent of $X_i$ in the tree, and $r$ is the tree's root node. 
Chow--Liu trees are acyclic, singly connected structures: each node has at most one parent where one can traverse the tree to accumulate probabilities between pairwise dependencies. This significantly reduces computation time compared to na\"{i}ve enumeration of the full joint measurement space.  The use of Chow-Liu trees was proposed by Buller and Kaufer~\cite{buller2016estimating} for estimating the entropy of multivariate data sources where the range of possible values is high. In our Python implementation, we use the pgmpy~\cite{ankan2024pgmpy} library's TreeSearch module. Practically, for each sensor subset, we:
\begin{enumerate}
    \item Discretise each sensor's readings via Freedman--Diaconis binning.
    \item Build a Chow--Liu tree from the mutual information of each sensor pair, selecting edges to form a spanning tree.
    \item Traverse the resulting tree to estimate max ($H_0$), Shannon ($H_1$), collision ($H_2$), and min-entropy ($H_{\infty}$) without enumerating the full exponential state space.
\end{enumerate}


Our framework evaluates the joint entropy over all sensor combinations. The powerset of the sensor set is generated and processed in parallel using Python's multiprocessing module. Processing all four datasets took approximately 22 hours on our workstation with an Intel i7-6700K (8M cache, 4.20 GHz) and 32 GB RAM on Ubuntu 24.04.

\subsubsection{Results}
\label{subsubsec:results}

Tables~\ref{tab:top10-uci-har}--\ref{tab:top10-shl} report the top 10 performing multi-sensor combinations ranked by min-entropy for each dataset. As expected, combining \emph{all} sensors yields the highest $H_0$ (max-entropy) and often increases Shannon and collision entropy. However, \emph{min-entropy ($H_{\infty}$) remains stubbornly low.} For instance, the complete set of sensors in \emph{SHL} surpasses 80 bits of $H_1$ (Shannon) but saturates at only 21\,bits of $H_{\infty}$.  Interestingly, we find that \emph{omitting} certain correlated sensors sometimes does not reduce min-entropy at all. For the Relay dataset in Table~\ref{tab:top10-relay}, the combination (Acc., Gyro., Light, Lin.~Acc., Mag., Rot.~Vec.) achieves \(H_{\infty} = 7.859\) bits, only slightly below the full set's \(8.092\) bits. Parallel findings arise in the PerilZIS and SHL datasets, where omitting a small number of sensors from the ``All sensors'' set has negligible impact on \(H_{\infty}\). This pattern appears across datasets: additional modalities may raise $H_0$ and $H_1$ but barely move $H_{\infty}$.  

The results imply that many sensors contribute \emph{redundant} information, showing a fundamental limitation of multi-modal data in real-world devices. Combining signals increases the \emph{apparent} capacity for unpredictability, but correlations between sensors means that the min-entropy from a large sensor ensemble is not be substantially higher than a reduced subset thereof. We note that standards use min-entropy as a safer, worst-case metric nowadays~\cite{bsi2024ais31,turan2018recommendation}. It measures how close the distribution is to collapsing around the single most probable outcome that an adversary will target first. To the best of our knowledge, the  entropy collapse brought about by highly correlated sensor modalities has not been before in existing work.  We provide the full sensor combination results in our open-source repository.

\begin{table}
\centering
\caption{Top 10 best-performing sensor combinations (UCI-HAR; in bits).}
\resizebox{0.65\linewidth}{!}{%
\begin{tabular}{@{}r|ccc|c@{}}
\toprule
\textbf{Modality} & \( H_{0} \) & \( H_{1} \) & \( H_{2} \) & \( H_{\infty} \) \\ 
\midrule
All sensors & 48.061 & 25.089 & 17.116 & 11.008 \\\midrule
(Acc.\{x,y,z\}, Gyro.\{y,z\})           & 39.403 & 22.835 & 15.999 & 10.113 \\
(Acc.\{x,y,z\}, Gyro.\{x,y\})           & 39.886 & 21.456 & 15.332 & 9.900   \\
(Acc.\{x,y,z\}, Gyro.\{x,z\})           & 40.222 & 21.470 & 15.040 & 9.411   \\
(Acc.\{y,z\}, Gyro.\{x,y,z\})          & 40.228 & 21.306 & 14.698 & 9.274   \\
(Acc.\{x,z\}, Gyro.\{x,y,z\})          & 40.293 & 21.260 & 14.575 & 9.154   \\
(Acc.\{x,y,z\}, Gyro.y)                   & 31.228 & 18.804 & 13.893 & 9.065   \\
(Acc.\{x,y\}, Gyro.\{x,y,z\})          & 40.273 & 21.155 & 14.208 & 8.721   \\
(Acc.\{x,y,z\}, Gyro.z)                   & 31.564 & 18.775 & 13.833 & 8.682   \\
(Acc.\{y,z\}, Gyro.\{y,z\})                  & 31.570 & 18.583 & 13.400 & 8.386   \\
\bottomrule
\end{tabular}
}
\label{tab:top10-uci-har}
\end{table}


\begin{table}[t!]
\centering
\caption{Top 10 best-performing sensor combinations (Relay; in bits).}
\resizebox{0.85\linewidth}{!}{%
\begin{tabular}{@{}r|ccc|c@{}}
\toprule
\textbf{Modality} & \( H_{0} \) & \( H_{1} \) & \( H_{2} \) & \( H_{\infty} \) \\ 
\midrule
All sensors &45.794&25.036&15.725&8.092\\\midrule
(Acc., Gyro., Light, Lin. Acc., Mag., Rot. Vec.)  & 44.795	& 24.963	&15.334	&7.859  \\
(Acc., Grav., Light, Lin. Acc., Mag., Rot. Vec.)      & 38.385&21.590&14.783&7.702  \\
(Acc., Grav., Gyro., Light, Mag., Rot. Vec.) & 37.026	&21.001&14.553&7.702 \\
(Grav., Gyro., Light, Lin. Acc., Mag., Rot. Vec.) & 36.092&20.794&14.450&7.673 \\
(Acc., Light, Lin. Acc., Mag., Rot. Vec.) & 37.385&21.517&14.428&7.469\\
(Acc., Gyro., Light, Mag., Rot. Vec.) & 36.026&20.924&14.270&7.465\\
(Gyro., Light, Lin. Acc., Mag., Rot. Vec.) & 35.092&20.720&14.147&7.439\\
(Acc., Grav., Gyro., Light, Lin. Acc., Mag.) & 41.094&22.794&14.437&7.338\\
(Acc., Grav., Light, Mag., Rot. Vec.) & 29.617&17.555&13.244&7.312\\
\bottomrule
\end{tabular}
}
\label{tab:top10-relay}
\end{table}


\begin{table}[ht]
\centering
\caption{Top 10 best-performing sensor combinations (PerilZIS; in bits).}
\resizebox{\linewidth}{!}{%
\begin{tabular}{@{}r|ccc|c@{}}
\toprule
\textbf{Modalities} & \( H_{0} \) & \( H_{1} \) & \( H_{2} \) & \( H_{\infty} \) \\ 
\midrule
All sensors & 83.662 & 36.998 & 33.682 & 23.926 \\\midrule
%
(Acc.\{x,y,z\}, Light, Temp., Pres., Mag.\{x,y,z\}, Gyro.\{x,y,z\})      & 76.907&34.258&31.510&23.835  \\
%
(Acc.\{y,z\}, Light, Temp., Pres., Mag.\{x,y,z\}, Gyro.\{x,y,z\})          & 72.737&34.246&31.509&23.835 \\
%
(Acc.\{x,z\}, Light, Temp., Pres., Mag.\{x,y,z\}, Gyro.\{x,y,z\})                 &72.820&34.243&31.508&23.829  \\
%
(Acc.\{x,y\}, Light, Temp., Pres., Mag.\{x,y,z\}, Gyro.\{x,y,z\})        & 72.737	&34.229	&31.501&	22.945
  \\
(Acc.\{x,y,z\}, Light, Temp., Mag.\{x,y,z\}, Gyro.\{x,y,z\})                 & 71.384	&32.586	&30.190&	22.945
   \\
(Acc.y, Light, Temp., Press., Mag.\{x,y,z\}, Gyro.\{x,y,z\})           & 68.568&34.218&	31.500&	22.945
   \\
   %
   (Acc.x, Light, Temp., Pres., Mag.\{x,y,z\}, Gyro.\{x,y,z\}) & 68.650&34.215&31.499&22.945\\
   %
   (Acc.\{y,z\}, Light, Temp., Mag.\{x,y,z\}, Gyro.\{x,y,z\}) & 67.214	&32.575	&30.189	&22.945\\
   (Acc.\{x,y,z\}, Light, Temp., Pres., Mag.\{x,z\}, Gyro.\{x,y,z\}, Hum.) & 76.618&33.816&31.035&21.915\\
\bottomrule
\end{tabular}
}
\label{tab:top10-peril}
\end{table}



\begin{table}[ht!]
\centering
\caption{Top 10 best-performing sensor combinations (SHL; in bits).}
\resizebox{\linewidth}{!}{%
\begin{tabular}{@{}R{10cm}|ccc|c@{}}
\toprule
\textbf{Modality} & \( H_{0} \) & \( H_{1} \) & \( H_{2} \) & \( H_{\infty} \) \\ 
\midrule
All sensors 
    & 158.601 
    & 82.301 
    & 39.320 
    & 21.289 \\
\midrule
(Acc.\{x,y,z\}, Gyro.\{x,y,z\}, Mag.\{x,y,z\}, Ori.\{w,x,y,z\}, Grav.\{x,y,z\}, LinAcc.\{y,z\}, Pres., Alt., Temp.)
    & 148.995
    & 78.624
    & 39.276
    & 21.289 \vspace{0.1cm}\\

(Acc.\{x,y,z\}, Gyro.\{x,y,z\}, Mag.\{x,y,z\}, Ori.\{w,x,y,z\}, Grav.\{x,y,z\}, LinAcc.\{x,z\}, Pres., Alt., Temp.)
    & 148.759
    & 78.477
    & 39.272
    & 21.289 \vspace{0.1cm}\\

(Acc.\{x,y,z\}, Gyro.\{x,z\}, Mag.\{x,y,z\}, Ori.\{w,x,y,z\}, Grav.\{x,y,z\}, LinAcc.\{x,y,z\}, Pres., Alt., Temp.)
    & 148.587
    & 78.380
    & 39.249
    & 21.289 \vspace{0.1cm}\\

(Acc.\{x,y,z\}, Gyro.\{x,y,z\}, Mag.\{x,y,z\}, Ori.\{w,x,y,z\}, Grav.\{x,y,z\}, LinAcc.\{x,y\}, Pres., Alt., Temp.)
    & 148.952
    & 78.135
    & 39.235
    & 21.289 \vspace{0.1cm}\\

(Acc.\{x,y,z\}, Gyro.\{x,y,z\}, Mag.\{x,y,z\}, Ori.\{w,x,y,z\}, Grav.\{x,y,z\}, LinAcc.\{z\}, Pres., Alt., Temp.)
    & 139.153
    & 74.800
    & 39.175
    & 21.289 \vspace{0.1cm}\\

(Acc.\{x,y,z\}, Gyro.\{x,z\}, Mag.\{x,y,z\}, Ori.\{w,x,y,z\}, Grav.\{x,y,z\}, LinAcc.\{y,z\}, Pres., Alt., Temp.)
    & 138.981
    & 74.703
    & 39.128
    & 21.289 \vspace{0.1cm}\\

(Acc.\{x,y,z\}, Gyro.\{x,z\}, Mag.\{x,y,z\}, Ori.\{w,x,y,z\}, Grav.\{x,y,z\}, LinAcc.\{x,z\}, Pres., Alt., Temp.)
    & 138.745
    & 74.556
    & 39.117
    & 21.289 \vspace{0.1cm}\\

(Acc.\{x,y,z\}, Gyro.\{x,y,z\}, Mag.\{x,y,z\}, Ori.\{w,x,y,z\}, Grav.\{x,y,z\}, LinAcc.\{y\}, Pres., Alt., Temp.)
    & 139.346
    & 74.458
    & 39.100
    & 21.289 \vspace{0.1cm}\\

(Acc.\{x,y,z\}, Gyro.\{x,y,z\}, Mag.\{x,y,z\}, Ori.\{w,x,y,z\}, Grav.\{x,y,z\}, LinAcc.\{x\}, Pres., Alt., Temp.)
    & 139.109
    & 74.311
    & 39.087
    & 21.289 \\
\bottomrule
\end{tabular}
}
\label{tab:top10-shl}
\end{table}