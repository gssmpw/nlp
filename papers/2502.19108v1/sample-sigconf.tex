%%
%% This is file `sample-sigconf.tex',
%% generated with the docstrip utility.
%%
%% The original source files were:
%%
%% samples.dtx  (with options: `all,proceedings,bibtex,sigconf')
%% 
%% IMPORTANT NOTICE:
%% 
%% For the copyright see the source file.
%% 
%% Any modified versions of this file must be renamed
%% with new filenames distinct from sample-sigconf.tex.
%% 
%% For distribution of the original source see the terms
%% for copying and modification in the file samples.dtx.
%% 
%% This generated file may be distributed as long as the
%% original source files, as listed above, are part of the
%% same distribution. (The sources need not necessarily be
%% in the same archive or directory.)
%%
%%
%% Commands for TeXCount
%TC:macro \cite [option:text,text]
%TC:macro \citep [option:text,text]
%TC:macro \citet [option:text,text]
%TC:envir table 0 1
%TC:envir table* 0 1
%TC:envir tabular [ignore] word
%TC:envir displaymath 0 word
%TC:envir math 0 word
%TC:envir comment 0 0
%%
%%
%% The first command in your LaTeX source must be the \documentclass
%% command.
%%
%% For submission and review of your manuscript please change the
%% command to \documentclass[manuscript, screen, review]{acmart}.
%%
%% When submitting camera ready or to TAPS, please change the command
%% to \documentclass[sigconf]{acmart} or whichever template is required
%% for your publication.
%%

%%
% \documentclass[sigconf]{acmart}
% \documentclass[sigconf,natbib=true,camera-ready]{acmart}

\documentclass[sigconf,camera-ready]{acmart}
% \copyrightyear{2025}
% \acmYear{2025}
% \setcopyright{cc}
% \setcctype{by}
\setcopyright{none}

\acmConference[Preprint, Under Review]{Preprint}{2025}{}
% \acmBooktitle{Proceedings of the ACM Web Conference 2025 (WWW '25), April
% 28-May 2, 2025, Sydney, NSW, Australia}
% \acmDOI{10.1145/3696410.3714772}
% \acmISBN{979-8-4007-1274-6/25/04}
% \settopmatter{printacmref=true}

\copyrightyear{}
\acmISBN{}
\acmDOI{}
\acmYear{}
\settopmatter{printacmref=false}

% \copyrightyear{2025}
% \acmYear{2025}
% \setcopyright{cc}
% \setcctype{by}
% \acmConference[WWW '25]{Proceedings of the ACM Web Conference 2025}{April 28-May 2, 2025}{Sydney, NSW, Australia}
% \acmBooktitle{Proceedings of the ACM Web Conference 2025 (WWW '25), April 28-May 2, 2025, Sydney, NSW, Australia}
% \acmDOI{10.1145/3696410.3714772}
% \acmISBN{979-8-4007-1274-6/25/04}

% \copyrightyear{2025}
% \acmYear{2025}
% \setcopyright{cc}
% \setcctype{by}
% \acmConference[WWW '25]{Proceedings of the ACM Web Conference 2025}{April 28-May 2, 2025}{Sydney, NSW, Australia}
% \acmBooktitle{Proceedings of the ACM Web Conference 2025 (WWW '25), April 28-May 2, 2025, Sydney, NSW, Australia}
% \acmDOI{10.1145/3696410.3714772}
% \acmISBN{979-8-4007-1274-6/25/04}


%%
%% \BibTeX command to typeset BibTeX logo in the docs
\AtBeginDocument{%
  \providecommand\BibTeX{{%
    Bib\TeX}}}

%% Rights management information.  This information is sent to you
%% when you complete the rights form.  These commands have SAMPLE
%% values in them; it is your responsibility as an author to replace
%% the commands and values with those provided to you when you
%% complete the rights form.
% \setcopyright{acmlicensed}
% \copyrightyear{2024}
% \acmYear{2024}
% \acmDOI{XXXXXXX.XXXXXXX}

%% These commands are for a PROCEEDINGS abstract or paper.
% \acmConference[the conference 25]{The Conference 2025}{Woodstock, NY}
% \acmConference[Preprint]{Submitted to ACM TheWebConf 2025}{Under Review}{}
%%
%%  Uncomment \acmBooktitle if the title of the proceedings is different
%%  from ``Proceedings of ...''!
%%
%%\acmBooktitle{Woodstock '18: ACM Symposium on Neural Gaze Detection,
%%  June 03--05, 2018, Woodstock, NY}
% \acmISBN{978-1-4503-XXXX-X/18/06}


%%
%% Submission ID.
%% Use this when submitting an article to a sponsored event. You'll
%% receive a unique submission ID from the organizers
%% of the event, and this ID should be used as the parameter to this command.
%%\acmSubmissionID{123-A56-BU3}

%%
%% For managing citations, it is recommended to use bibliography
%% files in BibTeX format.
%%
%% You can then either use BibTeX with the ACM-Reference-Format style,
%% or BibLaTeX with the acmnumeric or acmauthoryear sytles, that include
%% support for advanced citation of software artefact from the
%% biblatex-software package, also separately available on CTAN.
%%
%% Look at the sample-*-biblatex.tex files for templates showcasing
%% the biblatex styles.
%%

%%
%% The majority of ACM publications use numbered citations and
%% references.  The command \citestyle{authoryear} switches to the
%% "author year" style.
%%
%% If you are preparing content for an event
%% sponsored by ACM SIGGRAPH, you must use the "author year" style of
%% citations and references.
%% Uncommenting
%% the next command will enable that style.
%%\citestyle{acmauthoryear}

\newcommand{\todo}[1]{\textcolor{red}{#1}}
\usepackage[most]{tcolorbox}
\usepackage{tabularx}
\usepackage{bbm}
\usepackage{hyperref} 
\usepackage{multirow}
\usepackage{makecell}
% \usepackage{amssymb}% http://ctan.org/pkg/amssymb
\usepackage{pifont}% http://ctan.org/pkg/pifont
\newcommand{\cmark}{\ding{51}}%
\newcommand{\xmark}{\ding{55}}%
% remove acm reference
% \settopmatter{printacmref=false} % Removes the ACM reference format
% \renewcommand\footnotetextcopyrightpermission[1]{} % Removes the copyright block
%%
%% end of the preamble, start of the body of the document source.
\begin{document}

%%
%% The "title" command has an optional parameter,
%% allowing the author to define a "short title" to be used in page headers.
% \title{Personalized Multi-modal Sticker Search}
% \title{User-Sticker: A Large-Scale Cross-Domain Sticker Dataset for Personalization and Retrieval}

% \title{StickerU: A 106K Cross-Domain Multilingual Conversational User emoticons Dataset}
\title{A 106K Multi-Topic Multilingual Conversational User Dataset with Emoticons}

%%
%% The "author" command and its associated commands are used to define
%% the authors and their affiliations.
%% Of note is the shared affiliation of the first two authors, and the
%% "authornote" and "authornotemark" commands
%% used to denote shared contribution to the research.
\author{Heng Er Metilda Chee}
\email{xxe23@mails.tsinghua.edu.cn}
% \author{Wang Jiayin}
% \authornotemark[1]
% \email{JiayinWangTHU@gmail.com}
\affiliation{%
  % \institution{Department of Computer Science and Technology}
  % \institution{Tsinghua University}
\institution{DCST, Tsinghua University, Beijing, China. Quan Cheng Laboratory, Jinan, China.}
  \city{}
  \country{}
}


\author{Jiayin Wang}
\email{JiayinWangTHU@gmail.com}
\affiliation{%
  % \institution{Department of Computer Science and Technology}
  % \institution{Tsinghua University}
  \institution{DCST, Tsinghua University}
  \city{Beijing}
  \country{China}
}

% \author{Lars Th{\o}rv{\"a}ld}
% \affiliation{%
%   \institution{The Th{\o}rv{\"a}ld Group}
%   \city{Hekla}
%   \country{Iceland}}
% \email{larst@affiliation.org}

% \author{Valerie B\'eranger}
% \affiliation{%
%   \institution{Inria Paris-Rocquencourt}
%   \city{Rocquencourt}
%   \country{France}
% }

\author{Zhiqiang Guo}
\email{georgeguo.gzq.cn@gmail.com}
\affiliation{%
  % \institution{Department of Computer Science and Technology}
  % \institution{Tsinghua University}
  \institution{DCST, Tsinghua University}
  \city{Beijing}
  \country{China}}

\author{Weizhi Ma}
\authornote{Corresponding authors. 
% \\This work is supported by the Natural Science Foundation of China (Grant No. U21B2026, 62372260), Quan Cheng Laboratory (Grant No. QCLZD202301).
}
\email{mawz@tsinghua.edu.cn}
\affiliation{%
% \institution{Institute for AI Industry Research}
%  \institution{Tsinghua University}
\institution{AIR, Tsinghua University}
 \city{Beijing}
 \country{China}}

\author{Qinglang Guo}
\email{gql1993@mail.ustc.edu.cn}
\affiliation{%
% \institution{Institute for AI Industry Research}
%  \institution{Tsinghua University}
\institution{CETC Academy of Electronics and Information Technology Group Co.,Ltd. China Academic of Electronics and Information Technology}
 \city{Beijing}
 \country{China}}
 
\author{Min Zhang}
\authornotemark[1]
\email{z-m@tsinghua.edu.cn}
\affiliation{%
  % \institution{Department of Computer Science and Technology}
  % \institution{Tsinghua University}
  \institution{DCST, Tsinghua University, Beijing, China. Quan Cheng Laboratory, Jinan, China.}
  \city{}
  \country{}
  }
% \renewcommand\footnotetextcopyrightpermission[1]{%
%     \footnotetext{This work is supported by the Natural Science Foundation of China (Grant No. U21B2026, 62372260), Quan Cheng Laboratory (Grant No. QCLZD202301).}%
% }
% \author{John Smith}
% \affiliation{%
%   \institution{The Th{\o}rv{\"a}ld Group}
%   \city{Hekla}
%   \country{Iceland}}
% \email{jsmith@affiliation.org}

% \author{Julius P. Kumquat}
% \affiliation{%
%   \institution{The Kumquat Consortium}
%   \city{New York}
%   \country{USA}}
% \email{jpkumquat@consortium.net}

%%
%% By default, the full list of authors will be used in the page
%% headers. Often, this list is too long, and will overlap
%% other information printed in the page headers. This command allows
%% the author to define a more concise list
%% of authors' names for this purpose.
\renewcommand{\shortauthors}{Heng Er Metilda Chee, Jiayin Wang, Zhiqiang Guo, Weizhi Ma \& Min Zhang}

%%
%% The abstract is a short summary of the work to be presented in the
%% article.

\begin{abstract}
% Instant messaging with texts and emoticons has become a widely adopted communication medium, enabling efficient expression of user semantics and emotions. With the increased use of emoticons conveying information and feelings, sticker retrieval and recommendation has emerged as an important area of research. However, a major limitation in existing literature has been the lack of datasets capturing temporal and user-specific sticker interactions, which has hindered further progress in user modeling and sticker personalization.
% To address this, we introduce \textbf{U}ser-\textbf{Sticker}, a dataset that includes temporal and user anonymous ID across conversations. It is the largest publicly available sticker dataset to date, containing {22k unique users}, {370.2k emoticons}, and {\phantom{8.8M} 8.3M messages}. The raw data was collected from a popular messaging platform from {67} conversations over 720 hours of crawling. All text and image data were carefully vetted for safety and privacy checks and modifications.
% Spanning {10 domains}, the StickerU dataset captures rich {temporal}, {multilingual}, and {cross-domain behaviors} not previously available in other datasets. 
% Extensive quantitative and qualitative experiments demonstrate StickerU's practical applications in user behavior modeling and personalized recommendation and highlight its potential to further research areas in personalized retrieval and conversational studies.
% StickerU dataset is publicly available\footnote{https://huggingface.co/datasets/metchee/StickerU}.

Instant messaging has become a predominant form of communication, with texts and emoticons enabling users to express emotions and ideas efficiently. Emoticons, in particular, have gained significant traction as a medium for conveying sentiments and information, leading to the growing importance of emoticon retrieval and recommendation systems. However, one of the key challenges in this area has been the absence of datasets that capture both the temporal dynamics and user-specific interactions with emoticons, limiting the progress of personalized user modeling and recommendation approaches. To address this, we introduce the emoticon dataset, a comprehensive resource that includes time-based data along with anonymous user identifiers across different conversations. As the largest publicly accessible emoticon dataset to date, it comprises 22K unique users, 370K emoticons, and 8.3M messages. The data was collected from a widely-used messaging platform across 67 conversations and 720 hours of crawling. Strict privacy and safety checks were applied to ensure the integrity of both text and image data. Spanning across 10 distinct domains, the emoticon dataset provides rich insights into temporal, multilingual, and cross-domain behaviors, which were previously unavailable in other emoticon-based datasets. Our in-depth experiments—both quantitative and qualitative—demonstrate the dataset's potential in modeling user behavior and personalized recommendation systems, opening up new possibilities for research in personalized retrieval and conversational AI. The dataset is freely accessible \footnote{https://huggingface.co/datasets/metchee/u-sticker}.

% Instant messaging with texts and emoticons has become a widely used communication medium, enabling efficient user expressions. With the rise of emoticons to express user semantic and emotions, personalized sticker retrieval and recommendation has become an important research field. However, a major bottleneck in previous literature has been the lack of temporal and user-related sticker datasets, which has made further advancements in the field challenging.
% To address the lack of user sticker datasets, we introduce StickerU, the largest sticker dataset, containing 22.6k unique users, 105.9k unique emoticons, and 8.8M messages. 
% The raw data was collected from a popular messaging platform through 70 chat groups and 720 hours of crawling. The raw texts and images were check for safety and privacy concerns.
% Spanning 10 domains, the StickerU dataset captures rich temporal, multilingual, and cross-domain behaviors not seen in prior datasets.
% This large-scale, multi-domain user sticker dataset is appliable to several applications in sticker retrieval, user modeling and personalized recommendations.

% We applied state-of-the-art content moderation, tagging offensive content, followed by censoring via content removal and anonymization. 
% Multiple baselines can be used henceforth

% Experimental results show that StickerU performs well across multiple baselines, offering valuable insights for user behavior modeling and conversational analysis. StickerU dataset \footnote{https://huggingface.co/datasets/metchee/StickerU} is publicaly available.
\end{abstract}

%%
%% The code below is generated by the tool at http://dl.acm.org/ccs.cfm.
%% Please copy and paste the code instead of the example below.
%%
% \begin{CCSXML}
% <ccs2012>
%    <concept>
%        <concept_id>10002951.10003317</concept_id>
%        <concept_desc>Information systems~Information retrieval</concept_desc>
%        <concept_significance>500</concept_significance>
%        </concept>
%    <concept>
%        <concept_id>10002951.10003317.10003331.10003271</concept_id>
%        <concept_desc>Information systems~Personalization</concept_desc>
%        <concept_significance>500</concept_significance>
%        </concept>
%  </ccs2012>
% \end{CCSXML}

\ccsdesc[500]{Information systems~Information retrieval}
\ccsdesc[500]{Information systems~Dataset~Personalization}

%%
%% Keywords. The author(s) should pick words that accurately describe
%% the work being presented. Separate the keywords with commas.
\keywords{Resource, Sticker Retrieval, User Modeling}
%% A "teaser" image appears between the author and affiliation
%% information and the body of the document, and typically spans the
%% page.
% \begin{teaserfigure}
%   \includegraphics[width=\textwidth]{sampleteaser}
%   \caption{Seattle Mariners at Spring Training, 2010.}
%   \Description{Enjoying the baseball game from the third-base
%   seats. Ichiro Suzuki preparing to bat.}
%   \label{fig:teaser}
% \end{teaserfigure}

% \received{20 February 2007}
% \received[revised]{12 March 2009}
% \received[accepted]{5 June 2009}

%%
%% information and builds the first part of the formatted document.
\maketitle

% introduction 2 1.5
% RW: 
% emoticonU Dataset: steps
% Data analysis (be more detailed)
% Experiments (justify the reasonability of the dataset)
% Decision VLM -prepare a list of description and ask to choose from the context, specifications
\section{Introduction}
\label{sec:intro}
\begin{figure}[htbp] 
    \centering \includegraphics[width=0.5\linewidth]{Images/scenario_v2.jpg} 
    \caption{The colorful and vivid emoticons makes for an expressive and comfortable conversation.} 
\label{fig:scenario} 
\end{figure}
Instant messaging (IM) has become an essential mode of communication, allowing users to interact efficiently. Beyond text, colorful and animated emoticons have emerged as a powerful medium of expression, enabling users to convey semantics and emotions comfortably and accurately~\cite{ge2020anatomy, constantin2019computational, kariko2019laughing, gif-interestingess, bitmoji}, as seen in Figure \ref{fig:scenario}.
Unlike general image datasets, which emphasize object recognition, emoticons carry rich semantic meaning, expressiveness, and emotional nuance, making their retrieval and recommendation fundamentally different from traditional image retrieval and recommendation tasks. 
% introduction longer
% mention about the retrival and recommenation task
% table 1 > double column, and mention in abstract, mention ours, include personal column. 2.1 be in the introduction. 2.2 emoticon 下有任务, widely research fields and areas, 2.2 and 2.3 exchange, 2.2 emoticon retrieval task and their value

% importance of the task
% which work has been done 
% however, table 1 mentions that it is problematic. and how our work improves
However, despite emoticons' widespread use, personalized emoticon retrieval and recommendation remains an understudied area, primarily due to the lack of large-scale user emoticon interaction dataset.
As shown in Table~\ref{tab:dataset_comparison}, while several emoticon dataset exist, they often lack user information, making them unable to identify the same user across different conversations.
Additionally, they contain few emoticons per conversation, providing limited insight into user preferences or emoticon usage patterns. Furthermore, some datasets are not open-sourced, limiting reproducibility and progress of personalized emoticon retrieval and recommendation.

% - Conv. Domain if the conversation has been labelled with a topic.
% - NA if the dataset does not support.
% - xmark not mentioned or upon inspection does not contain

\begin{table*}[ht]
\centering
\caption{Comparison with Emoticon Datasets. We bold the largest count among publicly available datasets and underline the second largest. Our dataset is cross-domain, large in scale and contain time and user ID for personalization.}
\begin{tabular}{lcccrrl}
\toprule
\textbf{Dataset} & 
\textbf{Pub. Avail.}  & 
\textbf{Cross Domain} & 
\textbf{Time Info.} &
\textbf{\# Sticks./user} & 
\textbf{\# Emoticons} & \textbf{Description} \\ \hline
StickerTag \cite{stickertag} & \xmark & \xmark & \xmark & \xmark & 13,571 & Emoticon-tag pairs \\
StickerInt \cite{stickerint} & \xmark & \xmark & \xmark & Unknown & 1,025 & Dialogues with emoticons \\
StickerCLIP \cite{stickerclip} & \xmark & \xmark & \xmark & \xmark & 820,000 & Chinese emoticon-tag pairs \\
CSMSA \cite{CSMSA} & \xmark & \xmark & \xmark & \xmark & 16,000 &  Emoticons-tag pairs \\
SRS, PESRS \cite{learning-to-respond-2021, learning-to-respond-with-stickers-2020} & \xmark & \xmark & \xmark & 6.82 & 320,168 & Dialogues with emoticons \\
PerSRV \cite{chee2024persrv} & \xmark & \xmark & \xmark & Unknown & 543,098 &  Emoticon-query pairs \\ \hline
MCDSCS \cite{mcdscs} & \cmark & \xmark & \xmark & \underline{2.03} & 14,400 & Dialogues with emoticons  \\
SER30K \cite{SER30K} & \cmark & \xmark & \xmark & \xmark & \underline{30,739} & Emoticon-sentiments pairs \\
MOD \cite{MOD} & \cmark & \xmark & \xmark & 1.36 & 307 & Dialogues with emoticons \\ \midrule
% emoticonU (Ours) & \cmark & \cmark & \cmark & \textbf{16.36} & \textbf{370,222} & Dialogues with emoticons \\
 Ours & \cmark & \cmark & \cmark & \textbf{16.90} & \textbf{370,222} & Dialogues with emoticons \\
\hline
\end{tabular}
\label{tab:dataset_comparison}
\end{table*}


% \begin{table*}[htbp]
% \centering
% \caption{Among the eight related datasets, three are currently unavailable, while two are restricted (as noted in the "Restrictions" section). The remaining three available datasets do not include user IDs for cross-conversation identity linking, leading to insufficient data for user modeling and personalized recommendations. We shortform Unknown as Uk.}
% unknown, dont have diff, bold our results
% \begin{tabular}{ccccccc}
% \toprule
% \textbf{Dataset} & 
% \textbf{Avail.}  & 
% \textbf{Cross Domain} & 
% \textbf{Time Info} &
% \textbf{User Info} & 
% \textbf{\#Unique emoticon} & \textbf{\#Messages} \\ \hline

% emoticonTag \cite{emoticontag} & \xmark  & \xmark & \xmark & \xmark &  & 13,571 emoticon-tag pairs \\ 
% emoticonInt \cite{emoticonint}& \xmark & \xmark & \xmark &  & 1,025 & 1,578 dialogues, 1,025 emoticons \\ 
% emoticonCLIP \cite{emoticonclip}& \xmark & NA & NA & NA & & 820,000 Chinese image-text \\ 

% CSMSA \cite{CSMSA} & Res. & \xmark & & & & 16,000 emoticons with labels \\ 

% PESRS \cite{learning-to-respond-2021} SRS \cite{learning-to-respond-with-emoticons-2020} & Res. & \xmark & & & & 174,000 emoticons, 340,000 context\\ 

% MCDSCS \cite{mcdscs}& \cmark & \xmark  & & & & 5,500 dialogues, 14,400 emoticons \\ 

% SER30K \cite{SER30K}& \cmark & NA  & & & & 30,739 emoticons with sentiments \\ 

% MOD \cite{MOD} & \cmark & \xmark & & & & 45,000 dialogues, 307 emoticons \\ \hline

% emoticonU~(Ours) & \cmark &  \cmark & \cmark & 22,629 users & 105,933 emoticons &  8,769,401 messages \todo{log}  \\
% % SRS \cite{learning-to-respond-with-emoticons-2020} & Restricted & Same as PESRS \\
% \bottomrule
% \end{tabular}
% \end{table*}

To address these limitations, we introduce a new emoticon dataset, the first large-scale dataset that includes both user information and emoticon-based conversations. Our contributions can be summarized as follows:
% Specifically, our contributions can be summarized as follows:
\begin{itemize}
% size
% more information, multi-language? multi-domain
% experiments
    \item We present a new emoticon dataset, the largest emoticon dataset to date, containing \textbf{22k users}, \textbf{370.2k} emoticons and \textbf{8.3M} conversation messages of texts and emoticons.
    % Spanning 10 domains, emoticonU captures rich temporal, multilingual, and cross-domain behaviors not seen in prior datasets, with users averaging 19.97 emoticons each. This dataset provides a unique opportunity for studying personalized emoticon usage and user behavior in multimodal conversations.
    \item
    % emoticonU is multi-domain and contains rich information. Spanning 10 domains, emoticonU captures rich temporal, multilingual, and cross-domain behaviors not seen in prior datasets.
    Our dataset is a \textbf{multi-domain dataset} that includes rich and diverse information. Covering 10 domains, it captures temporal, multilingual, and cross-domain behaviors that are not present in previous datasets.
    \item Extensive quantitative and qualitative experiments demonstrate the dataset’s practical applications on user behavior analysis and modeling, personalized emoticon recommendation. It also holds potential for further research in areas such as personalized retrieval and conversational studies.
\end{itemize}

% filter already and then upload a small portion.
% 1. regex is using "i am", "my name", write this down first, llm parsed, group by group. privacy is impt! name, address, tel number.
% 2. limitations: 
\section{Related Work}
\label{sec:related_work}
% We outline related work on the data collection platform, emoticon datasets, emoticon retrieval methods and user behavior modeling.
% Dataset, usage of dataset.
We outline the current emoticon datasets and their applications. The terms sticker, emoticon, memes, gif mean approximately the same, dynamic or static multimedia that can be shared to express emotions. Herein, we use the word emoticon.

\subsection{Emoticon Datasets}
% Because xxx emoticon dataset is not much. Especially opensource ones, howeber recent emoticon reserarch ahs more focus. Thus, there is only 8 related dataset being used, but there are only xxx that are available. And we summarize in Table 1.
Due to the challenges in preparing emoticon conversation-based datasets, their availability remains limited, particularly for publicly accessible datasets. However, recent research on emoticons has gained more attention. Currently, there are only nine related datasets, with only \textbf{three} being publicly available. A summary of these datasets is provided in Table \ref{tab:dataset_comparison}.

\subsubsection{Availability}
Many of these datasets are closed-source. Specifically, six datasets are not publicly accessible \cite{stickertag, stickerint, stickerclip, CSMSA, learning-to-respond-2021, chee2024persrv}, meaning they are either closed-source or require additional access that is often unattainable. This lack of open access presents a significant challenge for emoticon-based research.

\subsubsection{Multi Topic Information}
Furthermore, none of the aforementioned datasets contain cross-domain information, as the conversations or dialogues are not identified with specific topics. This limitation hinders the exploration of cross-domain emoticon usage, which could provide valuable insights.

\subsubsection{User Information}
Additionally, publicly available datasets are scarce in terms of user data, with 1.36 \cite{MOD} and 2.03 \cite{mcdscs} historical emoticons per user. This lack of data makes it difficult to perform user analysis and personalization-related tasks. Our dataset bridges this gap, providing the highest emoticon history per user, with an impressive \textbf{16.90} emoticons.

\subsubsection{Multilingualism}
Most existing datasets focus on one or two languages, limiting the ability of smaller communities to benefit from advancements in emoticon personalization. In contrast, our dataset is multilingual, allowing a broader range of communities to engage with these advancements.

As shown in Table \ref{tab:dataset_comparison} and our analysis, our dataset is the largest publicly available emoticon-user dataset. It features \textbf{370,222} emoticons, \textbf{cross-domain behavior}, \textbf{rich contextual information}, and \textbf{comprehensive user and temporal} data.

% Teelegram dataset what they have done 
% emoticon dataset what they have done, prior work behaiovur done - mention where they have done, and processing
% from the perspective of our points

% We analyze eight emoticon-related datasets, of which six are not publicly available \cite{emoticontag,emoticonint,emoticonclip,CSMSA,learning-to-respond-2021,learning-to-respond-with-emoticons-2020}. Among the three available datasets, one contains only emoticon-description pairs without search logs \cite{SER30K}, while the remaining two are multi-dialogue datasets with an average of 1.36 and 2.03 emoticons per user, respectively \cite{mcdscs,MOD}. These datasets are too scarce for user modeling and personalization. In addition, no fewer than three papers have directly mentioned the lack of such training dataset \cite{pmg,chee2024persrv,pigeon}.

% Traditional image personalization datasets are also unsuitable for emoticon retrieval, as emoticons primarily express emotions, whereas image personalization focuses on identifying objects within images. This fundamental difference makes them inadequate for emoticon retrieval tasks. As far as we are aware, no existing dataset is suitable for personalized emoticon retrieval.

\subsection{Emoticon Retrieval}
Most previous research emphasizes the importance of data for emoticon retrieval. SRS \cite{learning-to-respond-with-stickers-2020}, PESRS \cite{learning-to-respond-2021} require corresponding utterances, while Lao et al. \cite{practical-sticker} rely on manually labeled emotions, sentiments, and reply keywords. CKES \cite{chen2024deconfounded} annotates each emoticon with a corresponding emotion. During emoticon creation, Hike Messager \cite{laddha2020understanding} tags conversational phrases to emoticons. The reliance on data presents a significant limitation, as emoticons without associated information are excluded from consideration.

Gao et al. \cite{learning-to-respond-with-stickers-2020} use a convolutional emoticon encoder and self-attention dialog encoder for emoticon-utterance representations, followed by a deep interaction network and fusion network to capture dependencies and output the final matching score. The method selects the ground truth emoticon from a pool of emoticon candidates and its successor. Zhang et al. \cite{zhang-etal-2022-selecting} perform this on recommendation tasks. CKES \cite{chen2024deconfounded} introduces a causal graph to explicitly identify and mitigate spurious correlations during training. The PBR \cite{xia2024perceive} paradigm enhances emotion comprehension through knowledge distillation, contrastive learning, and improved hard negative sampling to generate diverse and discriminative emoticon representations for better response matching in dialogues. PEGS \cite{zhang2024stickerconv}, StickerInt \cite{stickerint} generate emoticon information using multimodal models and selects emoticon responses, but does not consider personalization. StickerCLIP \cite{stickerclip} fine-tunes pretrained image encoders but does not consider personalization. In addition, many methods designed to rank emoticons from top-k candidates face a significant drawback in real-world emoticon retrieval scenarios, as they quickly become impractical when applied to larger datasets. Possibly due to the lack of dataset, most methods do not consider personalization.

\subsection{Personalized Emoticon Recommendation}
% So far no user behavior modeling for emoticon since there is scarce data. As mentioned above, personalization task is lesser which stifends research.
The absence of personalized emoticon recommendation can likely be attributed to the scarcity of emoticon-user datasets. PESRS \cite{learning-to-respond-2021} improves upon earlier work \cite{learning-to-respond-with-stickers-2020} by incorporating user preferences, but the dataset it relies on is not publicly accessible. This limited availability indirectly hampers the development of personalized emoticon recommendations and may impede further advancements in the field.
% The current personalized emoticon recommendation is limited by the community's dataset and related work is scarce. 


% \subsection{User Behavior Modeling}

% \todo{(merge the SRS wit source}
% \subsection{Telegram as Data Source}
% Most messaging applications impose strict limitations on data collection, making large-scale analysis of user behavior challenging. However, Telegram, a widely used messaging platform, provides unique opportunities for data scraping due to its public channels and group chats. These features allow researchers to study user interactions at scale without direct access to private conversations.

% Prior work has recognized Telegram as an underexplored platform for social science and user behavior research \cite{data-collection-opportunies}. Studies such as \cite{a-topic-wise-exploration} investigate the thematic distribution of discussions, while others \cite{pushshift, unearthing, tgdataset, analyzing-predicting-news, analysis-of-telegram-an-instant-messaging-service} have crawled Telegram channels and made datasets publicly available for further research. Gao et al. \cite{learning-to-respond-with-emoticons-2020} collected Telegram emoticon data from group chats, but access to their dataset remains restricted, limiting further research in personalized emoticon retrieval.

% \subsubsection{Crawling Tool}


\section{Our Dataset}
Telegram \cite{telegram_tos} is a widely used open-source messaging platform, offering publicly accessible conversations. The interactions within these groups are natural, making them a valuable source for understanding human behavior, particularly in the context of personalization. As such, they present a promising dataset for emoticon-related research \cite{analysis-of-telegram-an-instant-messaging-service, analyzing-predicting-news, pushshift, unearthing, data-collection-opportunies, tgdataset}.

However, the raw data requires preprocessing before it can be utilized effectively. In the following sections, we outline the process of constructing the dataset. We begin by defining the criteria used in the construction process.

% 1, open, natural, personalized

% no limitations on the public chats
% and checking to make sure no copyright problems

\subsection{Construction Criteria}
Given the significant implications of this dataset for future research, we adhere to the following criteria to ensure the quality and reliability of the data source 

We firstly define the envision for the final user-emoticon dataset. Since emoticons are the focal point of this dataset, we would like a \textbf{high presence of emoticons}. Additionally, since we would like to analyze interactions across different domains, the final dataset should involve \textbf{diverse topics}. We would also like the dataset to be of high-quality; avoidance of spam, advertisement and groups with single person. Moreover, we noticed that previous studies overlooked linguistic variety, hence, we wish our final dataset to have \textbf{multilingual coverage}. Finally, to ensure a robust and generalizable dataset, we strive for \textbf{sufficient dataset size}. 

We translate the above goals into the following dataset construction criteria,
\begin{enumerate}
    \item \textbf{Emoticon prevalence}: Conversations must contain a significant number of emoticons.
    \item \textbf{Topic diversity}: A wide range of discussion topics should be represented.
    \item \textbf{Authentic interactions}: We focus on real user-to-user conversations while avoiding:
    \begin{enumerate}
        \item Announcement-based channels with a single speaker.
        \item Conversations dominated by spam, advertisements, or inappropriate content.
    \end{enumerate}
    \item \textbf{Linguistic diversity}: Conversations should represent multiple languages.
    \item \textbf{Scalability}: We aim to capture as many suitable conversations as possible.
\end{enumerate}

Following these criteria, we manually screened hundreds of conversation groups and ultimately selected \textbf{71} conversation groups to crawl their content.

\subsection{Dataset Pre-processing}
The dataset, crawled using Telethon \cite{telethon}, contains key components from user messages, including the unique conversation group ID, group name, message datetime, user ID, media types, message ID, message text, and emoticon details (ID, set ID, access hash, and mime type). Users may participate in multiple conversation groups.
% The dataset presented in Table \ref{tab:crawled_message_components} was crawled using Telethon \cite{telethon}, capturing key components of user messages. The data includes several attributes, such as a unique conversation group ID and its corresponding name, which identify the specific conversation group. Each message is timestamped with the date and time it was sent, and it includes the user ID of the person who sent it. The dataset also tracks the types of media included in messages, as well as a unique message ID. In addition to the message text content, emoticons used in the conversation are recorded, including their unique emoticon ID, emoticon set ID, access hash, and the format or mime type of the emoticons. It’s important to note that users can participate in multiple conversation groups, and therefore the dataset may contain multiple entries for a single user.
% We present the information found in the dataset in Table \ref{tab:crawled_message_components}, crawled using Telethon \cite{telethon}. Note that users can participate in more than one conversation group.

% \begin{table}[ht]
% \centering
% \caption{Table of Crawled Data Components from Messages}
% \begin{tabular}{cl}
% \toprule
% \textbf{Name}                & \textbf{Description}                                                \\ \midrule
% Conversation group ID        & Unique conversation identifier                                     \\
% Conversation group name      & Group name                                                          \\
% Datetime                     & Message sent datetime                                               \\
% User ID                      & User identifier \\
% Media types                  & Message media                                                       \\
% Message ID                   & Unique message identifier                                  \\
% Message text                 & Message text content                                                \\
% Emoticon ID                   & emoticon identifier                                                  \\
% Emoticon set ID               & emoticon set identifier                                              \\
% emoticon access hash          & Unique access hash to emoticon                                       \\
% emoticon mime type            & emoticon format            \\ \bottomrule
% \end{tabular}
% \label{tab:crawled_message_components}
% \end{table}

\subsubsection{Text Processing}
As our dataset has been collected with multilingual diversity in mind, we identify the languages utilized for ease of downstream processing. We utilize the \textbf{xlm-roberta-base-language-detection} model \cite{conneau2019unsupervised} for language identification. The model provides both the detected language and a confidence score, and we utilize a threshold of 0.99 used to ensure high reliability in language detection.

After language detection, we manually review the classification results, removing entries with fewer than 20 occurrences and those from administrative bots. The final language distribution is illustrated in Figure ~\ref{fig:language-distribution}. In total, we detect 18 languages, with the top ten languages in order being English, Russian, French, Spanish, Chinese, Polish, Italian and German, Turkish, Portugese. As can be seen, emoticonU features 18 languages which is potentially viable for multilingualism emoticon-related task.

\begin{figure}[htbp]
    \centering
    \includegraphics[width=0.8\linewidth]{Images/language-distribution.jpg}
    \caption{Distribution of detected languages in the dataset. The dataset features over 18 languages. The top ten languages are English, Russian, French, Spanish, Chinese, Polish, Italian and German, Turkish, Portugese.}
    \label{fig:language-distribution}
\end{figure}

% \subsubsection{Conversation Content}
% \todo{in a table, name, description}
% In the first phase, we crawl text messages using the iter\_messages method. This phase provides us with various information, including the group name, group ID, and member details. Additionally, we gather message-specific metadata, such as the message ID, user ID of the sender, any reply relationships (reply-to message ID, if applicable), the presence of emoticons, and other media types (e.g., videos, images, and audio metadata). For each message, we also collect the datetime of the message and the message text itself. emoticons are further detailed by their ID, emoticon set ID, mime type, and access hash. Following this, we analyze the collected data to extract relevant metadata for further processing.

% \begin{table}[h]
%     \centering
%     \caption{Preliminary Statistics of raw data. At this point, the dataset has 424,793 users, 487,128 emoticons. }
%     \begin{tabular}{lr} 
%     \toprule
% Field & Number\\ \hline
%     % \# Groups                                   & 71 \\
%     % \# Messages                                 & 8,964,560 \\
%     % \# emoticons                                 & 512,192 \\
%     % \# Users                                    & 24,370 \\
%     % \# Unique emoticons                          & 142,981 \\
%     % Avg. \# emoticon per User                    & 21.02 \\
%     \# Groups                                   & 71 \\
%     \# Users                                    & 24,370 \\
%     \# emoticons                                 & 142,981 \\
%      \# Messages\_Text                               & 8,452,368 \\
%      \# Messages\_emoticon                                 & 512,192 \\
%      Avg. \# M\_emoticon per User                    & 21.02 \\
%     \bottomrule     
%     \end{tabular}
%     \label{tab:dataset-statistics}
% \end{table}

\subsubsection{Emoticon Processing}
% After selecting the groups, we proceeded to crawl the individual emoticons using the \texttt{get\_media} API. This allowed us to download emoticons for further analysis. We encountered several challenges during this phase. First, the access hash, which is critical for downloading emoticons, expires after a short period, complicating the download process. Second, the messages themselves also have an expiration window, which means they cannot be re-crawled once the period has passed. Additionally, the Telethon API access keys allow for single-threaded downloads, significantly increasing the time required for the emoticon download process. With this, our dataset contains \textbf{507,812} emoticons, an approximate conversion rate of \textbf{99.14\%} with the missing emoticons mainly due to copyright or expired messages.
The downloaded emoticons come in three formats; \texttt{.webp}, \texttt{.webm}, and \texttt{.tgs}, we present the extension distribution in Figure \ref{fig:emoticons-extension-distribution}. We convert all \texttt{tgs} files to the \texttt{.gif} extension using \texttt{tgsconverter} \cite{tgsconverter}, \texttt{.webp} is converted to \texttt{.png} and no conversion is done to \texttt{.webm}. We break up dymamic images into frames and save a frame every second to prepare for later use.

% After processing, there are an average of 2.69 frames per dynamic emoticon. 

% Finally, we present the distribution of the downloaded emoticons in Fig ~\ref{fig:emoticons-extension-distribution}. 

% \textbf{Distribution of emoticon Formats:} We present the distribution of the downloaded emoticons in three formats: \texttt{.tgs}, \texttt{.webp}, and \texttt{.webm}. 

\begin{figure}[htbp]
    \centering
    \includegraphics[width=0.7\linewidth]
    % \includegraphics[width=\textwidth]
    {Images/piechart-stickers-ext-distribution.jpg}
    \caption{Preliminary emoticons extension distribution. 65.6\% of the emoticons are static and 34.4\% are dyanimc.}
    \label{fig:emoticons-extension-distribution}
\end{figure}

% We spent approximately 72 hours crawling the text and another 336 hours crawling the emoticons. The conversion process was negligible in terms of time, as it was conducted in parallel with the crawling process. 
% In total, the entire crawling process took approximately 408 hours.

% Previously there was no xxx, then we intend to xxx. UserID, time, conversations and emoticons, userID is shared amongst different groups (domains).
% Spent more than 720 hours crawling the data. First the messages, initially emoticon packs were crawled but constrained by resources (memory and time), then just the emoticons. We also considered the gifs but prioritized the emoticons first. Notice that messages expire which means we have to re-crawl messages. Since these messages are tagged to a single user, they can only be crawled single-thread which makes the process tedious. Since formats are downloaded as webp, webm, and tgs, where they are converted to png, mp4 and gif, the process is time consuming.

% - defined 10 categories

% \subsection{Automatic Sensitive Data Detection}
% content tagging
% With the conversation and emoticons crawled, we perform data verification. Naturally with a lenient moderation platform, offensive language and images could potentially exists. Hence, we utilize a series of tools to moderate and tag the messages.  We implement a two-phase content moderation pipeline to identify and filter out NSFW (Not Safe For Work) language and images in our dataset.

\subsection{Unsafe Text Detection and Replacement}
% The goal of text content moderation is to remove unsafe text. We perform this in two phases; offensive language detection and hyperlinks detection.

We aim to moderate the safety of textual content. We perform this in two phases; (1) unsafe text detection and (2) offensive text replacement.

\subsubsection{Unsafe Text Detection}
We identify two types of unsafe text; offensive language and hyperlinks. 

\paragraph{Offensive Language}
Offensive text are content that could contain racism, vulgarities or other potentially offensive content. Since different languages require specialized tools for detecting offensive content, we focus on the ten most used languages in our dataset: \textbf{English, Russian, French, Spanish, Chinese, Polish, Italian and German, Turkish, Portugese}. For each, we employ well-established moderation models that have been published in peer-reviewed conferences, we describe the models in the following and the threshold values.

\begin{itemize}
    \item \textbf{roberta-base-cold} \cite{chinese-moderate}: Detects offensive language in \textbf{Chinese}. The model returns a binary label, positive values are tagged as offensive language.
    \item \textbf{twitter-roberta-base-offensive} \cite{en-moderation}: Identifies offensive language in \textbf{English}. The model returns a 0 to 1 score for offensive and not-offensive tags. We set the offensive threshold as 0.87.
    \item \textbf{russian-sensitive-topic} \cite{russian-content-moderation}: Flags sensitive or offensive content in \textbf{Russian}. The model returns "sensitive" or "not-sensitive" tags. We set "sensitive" content as offensive.
    \item \textbf{toxic-bert} \cite{multi-content-moderation}: A multilingual model that detects offensive language in \textbf{French, Turkish, Portuguese, Italian, Spanish, English, and Russian}. The model returns a score from 0 to 1. We set offensive threshold to be less than 0.0004.
    \item \textbf{dehatebert-mono-polish} \cite{aluru2020deep}: Detects hate speech in \textbf{Polish} language. The model returns a binary label, positive values are tagged as offensive language.
\end{itemize}

We present the offensive language detection results in the Figure ~\ref{fig:image-detection}, where the blue are safe and red are potentially offensive.

\begin{figure}[htbp]
    \centering
    \includegraphics[width=1.0\linewidth]{Images/safety-language.jpg}
    \caption{Distribution of offensive comments among language groups.}
    \label{fig:image-detection}
\end{figure}

The dataset shows that offensive language is relatively rare, with only 0.43\% of messages flagged as NSFW. This reflects typical user-generated platforms, where offensive content is present but remains a minority of overall interactions.

\paragraph{Hyperlinks}
Links are naturally present in group conversations. However, these links could potentially contain unsafe content. Hence, we use regular expression \cite{regex} detection to uncover 463,367 links in 313,906 messages and tag them for future modification.

\subsubsection{Unsafe Text Replacement}
\begin{itemize}
    \item \textbf{Offensive Language Replacement}:  
    Offensive or harmful language detected in text messages is replaced with the placeholder label:  
    \textit{detects this as offensive text.}  
    This approach preserves conversation structure while flagging offensive text, ensuring safety without losing context.
    
    \item \textbf{URL/Link Replacement}:  
    When URLs or links are identified in messages, they are replaced with the label:  
    \textit{emoticonU detects a link.}  
    This ensures private or potentially harmful web addresses are not exposed in the dataset, while preserving the flow of conversation and preventing leakage of sensitive data.
\end{itemize}

\subsection{Unsafe Image Moderation and Replacement}
Next, we move on to unsafe image moderation, where we employ (1) detection and (2) text replacement.

\subsubsection{Unsafe Image Detection}
The purpose of image moderation is to remove offensive images; these include violent, sexual or offensive. To achieve this, we employ several readily available tools. We introduce them in the following;
\begin{itemize}
    \item \textbf{nsfw-image-detection} \cite{falconsai2024nsfw}: has over 54 million downloads on Huggingface and classifies images into normal and not safe for work (nsfw).
    \item \textbf{nsfw-classifier} \cite{giacomo_arienti_2024}: classifies images into four categories - (1) drawings, (2) neutral, (3) hentai and (4) sexy.
    \item \textbf{vit-base-violence-detection} \cite{jaranohaal2024violence}: outputs binary labels for image violence detection.
    \item \textbf{vit-base-nsfw-detector} \cite{adamcodd2024nsfw}: classifies images into nsfw and Safe For Work (sfw).
\end{itemize}

% Since the above models can only support static images whereas 34.4\% of our emoticons are dynamic and break up dymamic images into frames, we save a frame every second. We ran the framing process simultaneously with other processes and hence the recorded time is neglible. After processing the dynamic emoticons, there are an average of 2.69 frames per dynamic emoticon. 

\begin{figure}[htbp]
    \centering
    \includegraphics[width=1.0\linewidth]{Images/vertical-barchart-image-detection_v2.jpg}
    \caption{Distribution of nsfw vs. sfw amongst the four offensive image detection models on the raw data.}
    \label{fig:image-detection-v2}
\end{figure}

% Explanation for why the vit-base-violence-detection is so high?

We utilize the frame processed above and static images into the four models, where we obtain the nsfw to sfw ratio. We present the results in Figure \ref{fig:image-detection-v2}. nsfw-image-detection is 0.88\%, nsfw-classifier is 0.93\%, vit-base-violence-detection is 27.27\%, vit-base-nsfw-detector  is 0.47\%. Then, \textbf{471,517 (92.06\%)} emoticons do not contain offensive taggings and and {40,675 (7.94\%)} images are considered offensive. Finally, there are a remaining \textbf{67} conversation groups.

\subsubsection{Unsafe Image Removal and Replacement}
For the identified offensive images, we replace the textual content with the placeholder \textit{detects this as offensive image}. This helps maintain the integrity of the dataset while ensuring that no harmful visuals, such as explicit or violent content, are retained, thus reducing the risk of privacy violations or inappropriate representations. Then, we remove the emoticon entirely from our emoticon database.

% \subsection{Automatic Sensitive Data Modification}
% but really, its just data replacement
% Finally, we sanitize the acquired data. In this section, we tackle privatization and censorship.
\subsection{Privatization}
In addition to safety, the privacy of the data is paramount.

\subsubsection{User Identifier}
\paragraph{Hashing sender identifier}
To protect the user's privacy, we anonymize the user identifiers by hashing their original integer identifiers using the SHA-256 hash function \cite{sha256}. Originally, the user identifiers were integers ranging from five-digital value to a maximum value of ten-digit value. After applying the SHA-256 hashing algorithm \cite{sha256}, the user identifiers are transformed into 256-bit strings, which are irreversible and do not retain any direct correlation with the original numeric values. This ensures that the users' identities are kept private, while still allowing for secure and effective data processing and analysis.
one more

\paragraph{Replacing Mentions}
It is common for users to interact with each other in the conversation group. To protect user privacy, we use regular expressions \cite{regex} to identify user identifiers and mentions (e.g., \@username) within text and replace them with the label \textit{\#USER}. Additionally, when users are mentioned via hyperlinks, these occurrences are replaced with \textit{\#USER\_ID}. Importantly, we do not replace the entire utterance but rather focus on replacing only the sensitive information, thus preserving the majority of the conversational context.

\subsubsection{Message Information}
We also anonymize message identifiers through the same SHA-256 hashing algorithm mentioned above.

\subsubsection{Other Sensitive Information}
To prevent the leakage of sensitive user information such as names, ages, addresses, organizations, or phone numbers, we use predefined context dictionaries to search the conversation logs. These logs are then further tagged by Llama-3.1-8B-Instruct \cite{llama}. Any sensitive information found is replaced with the label \textit{\#SENSITIVE-INFORMATION} within the conversation, but are not replaced entirely.



% \subsubsection{Censorship}
% Ensuring the safety and appropriateness of the emoticonU dataset is paramount, while maintaining its fluency and richness is equally important. Instead of removing potentially harmful or sensitive content entirely, we adopt a moderation approach where such content is replaced with predefined labels. This ensures that the dataset remains intact for analysis while protecting sensitive users and preserving the dataset’s integrity.

% \begin{itemize}
    % \item \textbf{Text}: Offensive or harmful language detected in the text messages is replaced with a placeholder label, \textit{emoticonU detects this is offensive text}. This approach allows us to retain the structure and flow of conversations without compromising the privacy or safety of individuals. The offensive text is thus flagged without removing the contextual richness of the message.
    
    % \item \textbf{Images}: 
    
    % \item \textbf{Hyperlinks}: 
% \end{itemize}

\subsection{Manual Verification}
% \subsubsection{Manual Review}
After implementing the automatic process unsafe text, image and privatization we conduct a \textbf{manual review} as an additional safeguard to ensure the dataset’s safety and quality. We perform this in three phases (1) browsing through the conversation for private information leakage and, (2) relabeling them with appropriate tags.

% In this phase, a subset of the dataset is randomly sampled and carefully examined by human reviewers. The goal is to verify that the automated procedures have successfully removed or replaced all sensitive and harmful content without affecting the overall structure or meaning of the messages. The manual review process is essential for catching any potential issues that automated processes may have missed, such as false negatives or inaccuracies in labeling. 

% The manual review also serves to ensure that the balance between data sanitization and contextual richness is maintained. Human reviewers assess not only the appropriateness of the labels but also confirm that the dataset remains useful for analysis and research purposes. This step reinforces the dataset’s integrity and ensures that it meets the privacy and safety standards required for further processing. Finally, we have a dataset that is ready for further analysis.

% Purpose to ensure the reasonability and safety issues.
% 1. Human checking how many samples, and if there are any problems
% 2. GPT to confirm, which ones we have sampled

% \subsubsection{Removing Obsolete Groups}
% Since our dataset focuses on emoticons, we remove groups that have no emoticons due to content moderation. This remains us with 67 groups.

\subsection{Conversation Topic Labeling}
% \todo{dont mention multi-group in telegram, all rename to group-conversation}
% not to separate the group, for analysis and future scenario analysis, amend the motivation
To enable groups for future scenario analysis, we categorize conversations into domains. We utilize a two-step approach. First, we read the conversation name, which often provides a clear indication of the conversation's general focus. Next, we browse through the conversation content, examining the topics and discussions that occur within. This process enables us to accurately identify the predominant subject matter of the conversation.

Our categorization is based on domains related to common hobbies and interests. We identify ten primary domains: 

\begin{itemize}
    \item \textbf{Language:} This domain includes conversations focused on learning new languages, sharing language learning resources, or discussing language-related topics.
    \item \textbf{Arts:} Conversations under this domain typically involve the sharing of artwork, discussions around different forms of art, drawing techniques, and other creative activities.
    \item \textbf{Games:} Conversations in this domain revolve around various forms of gaming, whether they are video games, board games, or other interactive entertainment.
    \item \textbf{Technology:} Conversations here engage in code sharing, learning new programming languages, discussing software development, or delving into the latest technological innovations.
    \item \textbf{Finance:} This domain focuses on topics related to financial discussions, including traditional investing, cryptocurrency, blockchain technologies, and financial exchanges.
    \item \textbf{Social:} Conversations under the Social domain are centered around meeting new people, making friends, or organizing social events such as meetups and hangouts.
    \item \textbf{Media Sharing:} Conversations in this category often focus on sharing and discussing media content such as emoticons, memes, gifs, and videos for entertainment and social interaction.
    \item \textbf{Outdoor:} Conversations here revolve around nature, outdoor activities like hiking and camping, and the appreciation of animals and the environment.
    \item \textbf{Anime:} This domain is dedicated to discussions about anime, cartoons, and related pop culture topics, including anime recommendations and fan theories.
    \item \textbf{Fan Club:} Conversations in this category are focused on specific celebrities, idols, or fictional characters, where users come together to share their admiration and discuss fan-related topics.
\end{itemize}

While these categories cover a wide range of interests, some topics naturally overlap. For instance, the increasingly popular genre of crypto-based gaming involves elements of both gaming and finance, as users play games to earn cryptocurrency. However, to maintain a concise and organized dataset, we prioritize the dominant domain in such cases—crypto gaming would thus fall under the Finance category in this example. The final dataset, after domain labeling, is presented and analyzed in the following section, where we examine the distribution of conversations across these domains and the insights derived from this classification.

\section{Dataset Analysis}
% We ignore censored data herein. 
% large, multi domain, multilingual, user interaction, 5.2 > 4.3, check 
We analyze our dataset and share our findings in the following section. Firstly, we present a statistical overview of the dataset. Secondly, we dive into the multi-domain characteristics of our dataset. Finally, we analyze potential user behavior found in our dataset.

\subsection{Statistic}
% In this section, we present the scale of the dataset, as well as the language and emoticons distributions.

\subsubsection{Overall Information}

\begin{table}[h]
    \centering
    \caption{Statistics of the raw data and final emoticon dataset. Ours is the largest in size and contains user IDs across conversations.}
    \begin{tabular}{@{}lccc@{}} 
    \toprule
    Field & Raw Data & Ours \\ \hline
    \# Domains                               & -       & 10     \\
    \# Groups                                   & 70      & 67 \\
    % \# Unique Users                                 & \todo{24,933}  & \todo{22,629} \\
    % \# Unique emoticon Users                                 & 24,370  & 21,901 \\
    \# Unique Emoticons                          & 142,981 & 105,933 \\
    \midrule
    \ \textit{Messages} &&& \\
    % \# Text                          & 8,964,549 & 8,399,179 \\
    \# emoticons                      & 512,192   & 370,222 \\
    % \# Text                         & 8,964,549 & 8,399,179 \\
    \# Text                          & 8,332,351 & 8,286,422 \\
    \midrule
    % \# Avg. emoticons per User        & 20.54    & 16.36 \\
    \# Avg. Emoticons per User        & 21.01    & 16.90 \\
    \bottomrule     
    \end{tabular}
    \label{tab:dataset-statistics}
\end{table}

% Notably, we have defined domains and reduced sparse groups from 70 to 67. While ensuring the safety and privacy of the emoticonU with the automatic sensitive data detection, replacement and human verification, our dataset is, to the best of our knowledge, the largest with \textbf{370,222} emoticons, \textbf{22,629} emoticon users and \textbf{16.36} average emoticons per user.

As shown in Table~\ref{tab:dataset-statistics}, our dataset is large in size and contains user information for cross conversation and cross domain analysis and modeling.
Notably, we filter the raw data from the public platform for safety and privacy concerns to form the final dataset. 
% \begin{table}[h]
%     \centering
%     \caption{Summary statistics of the final emoticonU. emoticonU contains 8,399,179 utterances, 370,222 emoticons, 22,629 users and an average of 16.36 emoticons per user.}
%     \begin{tabular}{@{}lccc@{}} 
%     \toprule
%     Field & Before & After & Difference \\ \hline
%     \# Categories                               & -       & 10      & -     \\
%     \# Groups                                   & 70      & 67      & 0.0429 \\
%     \# Users                                    & 24,933  & 22,629  & ($\textcolor{black}\triangledown$ 0.0924) \\
%     \# Unique emoticons                          & 142,981 & 105,933 & ($\triangledown$ 0.2591) \\
%     \midrule
%     \ \textit{Messages} &&& \\
%     \# Text                          & 8,964,549 & 8,399,179 & ($\triangledown$ 0.0631) \\
%     \# emoticons                      & 512,192   & 370,222  & ($\triangledown$ 0.2772) \\
%     \midrule
%     \# Avg. emoticons per User        & 20.54    & 16.36    & ($\triangledown$ 0.2036) \\
%     \bottomrule     
%     \end{tabular}
%     \label{tab:dataset-statistics}
% \end{table}


% before and after, and remove the groups (triangle for percentages), update the captions, bigger words within the images. add dot in the figures. categories in the first line of tTable 3
\begin{table*}[h]
    \centering
    \caption{Domain Statistics.}
    \begin{tabular}{@{}lcccccccc@{}} 
    \toprule
    \textbf{Category} & \textbf{Text Messages} & \textbf{Emoticons}& & & \textbf{Users}&& \textbf{Sparsity} & \textbf{Dom. Language} \\
    &  & \textit{Global}& \textit{Unique} & \textit{Avg. Per User}& \textit{Global} & \textit{Use Emoticons} & & \\ \midrule

    Anime & 364,543 & 87,833 & 22,572 & 44.90 & 10,448 & 1,956 & 0.001989 & French \\
    Arts & 19,515 & 219 & 176 & 3.37 & 328 & 65 & 0.019143 & English\\
    Fan Club & 551,409 & 2,287 & 1,234 & 13.61 & 6,341 & 168 & 0.011032 &  Turkish\\
    Finance & 2,013,689 & 45,655 & 10,083 & 5.78 & 245,020 & 7,894 & 0.000574 &  Chinese\\
    Game & 2,289,100 & 79,367 & 16,382 & 32.16 & 48,297 & 2,468 & 0.001963 &  English \\
    Language & 2,388,938 & 51,556 & 15,801 & 14.92 & 50,734 & 3,456 & 0.000944 & German \\
    Outdoor & 353,789 & 2,088 & 953 & 7.88 & 6,230 & 265 & 0.008268 & English \\
    Social & 416,469 & 13,077 & 3,894 & 11.12 & 42,180 & 1,176 & 0.002856 & English \\
    Media Sharing & 176,581 & 87,827 & 45,451 &  11.71 & 14,309 & 7,498 & 0.000258 & English \\
    Technology & 20,305 & 313 & 210 & 4.17 & 4,425 & 75 & 0.019873 & English \\
    \bottomrule     
    \end{tabular}
    \label{tab:category-language-sparsity}
\end{table*}

\subsubsection{Emoticon Usage Distribution}
\begin{figure}[htbp] 
    \centering \includegraphics[width=1.0\linewidth]{Images/stickers_distribution.jpg} 
    \caption{Our Dataset Emoticon Usage Distribution. Number of users versus the number of emoticons. Our dataset emoticon usage follows a tail-shaped distribution.} 
\label{fig:emoticons-user-distribution} 
\end{figure}

Next, we analyze the distribution of emoticon usages. To better visualize this distribution and prevent distortion, we apply a logarithmic transformation to the number of users. As demonstrated in Figure ~\ref{fig:emoticons-user-distribution}, Our dataset exhibits a tail-shaped distribution, highlighting the real-world emoticon usage patterns.

\subsubsection{Language Distribution}
Then, we analyze the language distribution of our dataset, shown in Figure ~\ref{fig:language-distribution-v2}. Our dataset features Arabic, Bulgarian, Chinese, Dutch, English, French, German, Hindi, Italian, Japanese, Modern Greek, Polish, Portuguese, Russian, Spanish, Swahili, Thai, Turkish, Urdu, and Vietnamese. English is the most dominant language with an domineering 62.6\%. Russian (9.6\%) is the second most popular language followed by French (8.6\%) and Chinese (8.0\%). 

\begin{figure}[htbp] 
    \centering \includegraphics[width=1.0\linewidth]{Images/language-distribution-global.jpg} 
    \caption{Dataset Language Distribution. Our dataset features 20 languages.} 
\label{fig:language-distribution-v2} 
\end{figure}

\subsection{Multi-Topic Characteristics}
In this section, we analyze the multi-domain characteristics of our dataset. Firstly, we analyze the domain-group distribution. Secondly, we introduce the domain statistics. Lastly, we analyze the user-emoticon usage overlap across domains.

% number of groups
% figure 6: the unique emoticon in domain, unique users, utterances, with emoticons only, sparsity (msg_w_emoticons/user/unique_emoticons).
% sparsity of the groups and the labels, split into a few charts

\subsubsection{Group Distribution}
As mentioned above, our dataset contains 67 groups and 10 domains, with the median group being five. We illustrate this in Figure \ref{fig:piechart-categories-group-distribution}. The minimum group count is one, being Arts, and the maximum group count is media sharing. This is due to the nature of resource, focusing on emoticons. Games and Social come close as they tend to be relaxed and so more expressed.

\begin{figure}[htbp] 
    % \centering \includegraphics[width=1.0\linewidth]{Images/group-distribution.jpg} 
    \centering \includegraphics[width=1.0\linewidth]{Images/final-categoriesd-distribution.jpg} 
    \caption{Our dataset features 67 groups and 10 domains. The median group count per domain is five.} 
\label{fig:piechart-categories-group-distribution} 
\end{figure}
% sequence of the title: 4.2, 4.3, prevalent from the user level, user usage. distribution (long tail). show that it is normal distribution. single emoticon, from user usage, from utterances.

\subsubsection{Domain Statistics}
Table \ref{tab:category-language-sparsity} shows the domain overview statistics. Anime has the most emoticons, followed by Media Sharing and Game, reflecting the expressive nature of these groups. Technology is the sparsest domain, while Media Sharing is the densest, with Finance and Language being unexpectedly dense as well. The average emoticons per user are highest in Anime (44.90) and Game (32.16), while Technology (4.17) and Finance (5.78) have the lowest, suggesting that the nature of the domain influences emoticon usage.

% \begin{table*}[h]
%     \centering
%     \caption{Statistics of each domain with Sparsity.}
%     \begin{tabular}{@{}lcccccccccc@{}} 
%     \toprule
%     Category & Messages (Not Censored) & Messages (Censored) & Messages (Not emoticons) & emoticons (Global) & emoticons (Unique) & emoticon Users & Use emoticons & Language & Sparsity \\ \hline
%     Language & 2,440,494 & 2,388,938 & 51,556 & 15,801 & 10,035 & 3,834 & 50,734 & 3,456 & de & 0.000944 \\
%     Finance & 2,059,344 & 2,013,689 & 45,655 & 10,083 & 7,213 & 1,841 & 245,020 & 7,894 & zh & 0.000573 \\
%     Technology & 20,618 & 20,305 & 313 & 210 & 115 & 73 & 4,425 & 75 & en & 0.019873 \\
%     emoticon & 264,408 & 176,581 & 87,827 & 45,451 & 29,765 & 15,306 & 14,309 & 7,498 & en & 0.000258 \\
%     Social & 429,546 & 416,469 & 13,077 & 3,894 & 3,788 & 1,016 & 42,180 & 1,176 & en & 0.002856 \\
%     Game & 2,368,467 & 2,289,100 & 79,367 & 16,382 & 28,588 & 5,457 & 48,297 & 2,468 & en & 0.001963 \\
%     Outdoor & 355,877 & 353,789 & 2,088 & 953 & 1,881 & 353 & 6,230 & 265 & en & 0.008268 \\
%     Anime & 452,376 & 364,543 & 87,833 & 22,572 & 59,420 & 11,665 & 10,448 & 1,956 & fr & 0.001989 \\
%     Arts & 19,734 & 19,515 & 219 & 176 & 158 & 118 & 328 & 65 & en & 0.019143 \\
%     % Travel & 148,602 & 141,585 & 7,017 & 1,436 & 2,500 & 407 & 5,500 & 639 & en & 0.007647 \\
%     Fan Club & 553,696 & 551,409 & 2,287 & 1,234 & 1,007 & 351 & 6,341 & 168 & tr & 0.011032 \\
%     \bottomrule     
%     \end{tabular}
%     \label{tab:category-language-sparsity}
% \end{table*}

\subsubsection{User Emoticon Usage Overlap Across Domains}
% context emoticon pair for the different groups, and then merge 
% domain > domain's groups 
% 4.3 user analysis 
Lastly, we analyze the appearance of users in the same domain but in other groups, or in entirely different domain. We plot our findings in Figure \ref{fig:emoticon-overlap-distribution}.

\begin{figure}[htbp] 
    \centering \includegraphics[width=1.0\linewidth]{Images/sticker-user-overlap.jpg} 
    \caption{Distribution of users' emoticon usage.} 
\label{fig:emoticon-overlap-distribution} 
\end{figure}

Firstly, we establish that users do join other groups and participate; these could be within the same domain or an entirely different domain. The number beside the label on the horizontal axis reveals the total number of emoticon users within the domain, the blue bar represents the percentage of users that appear in other groups within the same domain and the red represents the percentage of users that appear in domains entirely different. We omit domains with fewer than 50 user emoticons to avoid generalization.

Secondly, two of the largest groups with more than 7000 emoticon users (Finance and Media Sharing domain) with roughly 15\% and 3\% to 85\% and 97\% different to same domain ratios. In fact, other than the Social domain, domains all have more users participating in the same domain than somewhere else; which could possibly reveal user preference and taste. Naturally, Social domain is clearly an explorative domain and therefore behaves as such.

% \subsubsection{emoticon Distribution}
% % messages contain emoticons
% % u. emoticons and users be on the same level
% % 
% \begin{figure}[htbp] \centering \includegraphics[width=1.0\linewidth]{Images/category-emoticon-distribution.jpg} \caption{Distribution of emoticons across categories in the dataset. The chart highlights the large number of emoticons in categories such as Anime (147,183), Games (118,029), and Media Sharing (117,252), while categories like Arts (377) and Technology (428) have far fewer emoticons. This distribution reflects the level of user engagement and interest in each category, with Anime and Games being particularly popular for emoticon sharing.} \label{fig:category-emoticon-distribution} \end{figure}
% % kinda not necssary, normalize it, otherwise cant be analyzed fairly
% % think about the min 0? 
% Then, we review the emoticon distribution of the categories, shown in Figure \ref{fig:category-emoticon-distribution}. As summarized in the chart, Finance has the highest number of emoticons with 42,734, followed closely by Game (118,029) and Anime (147,183). On the other hand, Arts (377) and Technology (428) represent the categories with the least number of emoticons. The distribution exhibits a significant skew, with categories like Media Sharing (117,252) and Language (61,577) also containing a large number of emoticons. The minimum number of emoticons in a category is 377 (Arts), while the maximum is 147,183 (Anime), with a median value that falls in between.

% This distribution indicates that categories like Anime and Games dominate due to their broad, popular appeal in emoticon sharing, particularly in communities where visual content such as emoticons is extensively used. These categories are often associated with fan-driven content and gaming, both of which encourage frequent sharing of emoticons among large communities. Conversely, categories like Arts and Technology have fewer emoticons, possibly due to their more specialized focus, where the creation and sharing of emoticons might not be as prevalent. This suggests that the prominence of a category in emoticon sharing is linked to the level of user engagement and interest within that category.

% % \subsubsection{User-emoticon Distribution}
% % Here we find out how many users use how many emoticons. We present the findings in \ref{}.

% % \subsubsection{Language Distribution}
% %
\section{Application Case 1: User Behavior Analysis}
% - frequency of user's domain exchange
% - how many emoticons per user
% - cross domain patterns, 

Next, we move on to the user behavior analysis on the dataset. More specifically, we demonstrate our dataset's potential on showcasing (1) user distinct style (2) multi-domain behavior (3) temporal behavior change (4) user behavior with different respondents.

\subsection{Distinct User Styles}

% hwo many didfffent users behave in the same group 

\begin{figure}[htbp] 
    \centering \includegraphics[width=1.0\linewidth]{Images/user-profile.png} 
    \caption{User A emoticons in the Language group.} 
\label{fig:emoticon-user-profile-A} 
\end{figure}

\begin{figure}[htbp] 
    \centering \includegraphics[width=1.0\linewidth]{Images/userb-game.png} 
    \caption{User B emoticons in the Game group.} 
\label{fig:emoticon-user-profile-B} 
\end{figure}

As shown in Figure \ref{fig:emoticon-user-profile-A}, User A's emoticon styles are predominantly cat-themed and cute, while User B’s emoticons are more idol-based, with an emphasis on adorable and charming designs, as shown in Figure \ref{fig:emoticon-user-profile-B}. This demonstrates that users generally have a distinct emoticon preference and this potentially be used for user modeling.

\subsection{Cross Domain Behavior Change}

More interestingly, we analyze if user, due to different context, could change their style. Herein, the same User A, Figure \ref{fig:emoticon-user-profile-A-2}, displays a significantly different style in the Media Sharing group, which is more structured and cartoonish in appearance. This is contrastive to the originally cartoonish style.

\begin{figure}[htbp] 
    \centering \includegraphics[width=1.0\linewidth]{Images/usera-another-category.png} 
    \caption{User A emoticons in the Media Sharing group.} 
\label{fig:emoticon-user-profile-A-2} 
\end{figure}

\begin{figure}[htbp] 
    \centering \includegraphics[width=1.0\linewidth]{Images/userb-outdoor.png} 
    \caption{User B emoticons in the Outdoor group.} 
\label{fig:emoticon-user-profile-B-2} 
\end{figure}

On the other hand, User B’s emoticons, Figure \ref{fig:emoticon-user-profile-B-2}, still retain elements of their signature adorable style but are subtly tuned to match the outdoor-themed context of the new domain, which is about rats. This behavior reveals that different user have different consistency level in emoticon-usage. There are possibly many factors for this behavior such as the group context. 

\subsection{Temporal Behavior Change}
\begin{figure}[htbp] 
    % \centering \includegraphics[width=1.0\linewidth]{Images/top-stickers-by-user-overtime.png} 
    \centering \includegraphics[width=1.0\linewidth]{Images/emoticons-2.png} 
    \caption{User C emoticon usage over time where count is more than three.} 
\label{fig:emoticon-usage-overtime} 
\end{figure}
We collate the top emoticon by usage count from User C and present in Figure ~\ref{fig:emoticon-usage-overtime}. As can be seen, the user's emoticon preferences evolve over time. In September and October, the user gravitates toward dog-themed, cartoonish emoticons. However, by January, their preference shifts to smaller, cuter dog emoticons. Interestingly, in June, the user shows a renewed interest in a different set of realistic dog emoticons, while still maintaining a fondness for the smaller, cuter dog designs.

Overall, despite fluctuations in style, the user consistently displays a preference for dog emoticons, with a notable emphasis on the cute, cartoonish variations. This suggests that while the user's style may vary, their core preference remains for dog-themed emoticons, particularly those with a cute and cartoonish aesthetic.

% combine together
\begin{figure}[htbp] 
    % \centering \includegraphics[width=0.5\linewidth]{Images/user-same-stickers.png} 
    \centering \includegraphics[width=0.6\linewidth]{Images/emoticons-3.png} 
    \caption{User D emoticon usage over time. The user uses the green frog emoticon frequently overtime.} 
\label{fig:emoticon-top-emoticons} 
\end{figure}
On the other hand, User D demonstrates a consistent use of a certain green laughing frog, except the months where no emoticons are used, as seen in Figure ~\ref{fig:emoticon-top-emoticons}. Herein, the user demonstrates a consistent usage of a certain emoticon, unlike User C.

\subsection{User Behavior with Different Respondents}

\begin{figure}[htbp] \centering \includegraphics[width=1.0\linewidth]{Images/user-same-sticker-diff-user.png} \caption{User E shows a consistent style preference when replying to Users F and G.} \label{fig:user-same-emoticon-diff-user} \end{figure}

We present two distinct cases of user behavior in response to different users. In the first case, User E replies to both Users F and G using a similar set of emoticons. As shown in Figure ~\ref{fig:user-same-emoticon-diff-user}, User E consistently selects a boy emoticon with a nonchalant expression, regardless of the recipient. Although the third and fourth emoticons in the sequence differ, the first two emoticons remain frequently used, with counts of 12 and 8 respectively, indicating a strong preference for these emoticons.

\begin{figure}[htbp] \centering \includegraphics[width=1.0\linewidth]{Images/user-diff-sticker-diff-user.png} \caption{User H shows contrasting style preferences when replying to Users I and J.} \label{fig:user-diff-emoticon-diff-user} \end{figure}

In contrast, User H exhibits a clear difference in emoticon usage when replying to Users I and J. As illustrated in Figure ~\ref{fig:user-diff-emoticon-diff-user}, User H uses bubbly, cheerful, and generally cute emoticons when replying to User I. However, when replying to User J, User H switches to a more melancholic and indifferent style, favoring stickman emoticons that are notably less cute. This shift in emoticon preference highlights the variation in user behavior based on the recipient of the message.

In conclusion, these examples demonstrate that the dataset captures meaningful differences in emoticon usage across various contexts. This variation, influenced by the person being replied to, suggests the potential for deeper personalization and more insightful user behavior analysis.
% \subsubsection{General Statistics}
% \begin{table}[h]
%     \centering
%     \caption{Summary of group-level statistics for the emoticonU dataset, showing the distribution of emoticons, utterances, and users across 70 groups.}
%     \begin{tabular}{@{}lcccc@{}} 
%     \toprule
% Field & Average & Median & Min & Max\\ \hline
%     \# emoticons per Group                     & 2,047.86 & 884.5  & 0   & 144,654 \\
%     \# Utterances per Group                    & 9,060.84 & 11,966.5 & 0   & 966,831 \\
%     \# Users per Group                          & 833.21  & 866     & 3   & 96,331 \\
%     \# Users w/ emoticon per Group   & 333.73  & 65.5    & 0   & 3,917 \\
%     \bottomrule     
%     \end{tabular}
%     \label{tab:group-statistics}
% \end{table}

% The table above summarizes the group-level statistics for the emoticonU dataset, which includes 70 groups. Each group is characterized by its number of emoticons, utterances, and users. On average, each group contains 2,047.86 emoticons, 9,060.84 utterances, and 833.21 users. The median number of emoticons per group is 884.5, while the median number of utterances is 11,966.5. The number of emoticons, utterances, and users ranges from 0 to 144,654, 0 to 966,831, and 3 to 3,917, respectively. Notably, some groups have no emoticons or utterances at all, which suggests that certain groups may focus on emoticon-only conversations, while others may primarily facilitate textual exchanges, which is natural. This distinction could potentially prove valuable for future social and behavioral analysis, as groups with no emoticons might be more conversation-driven, potentially offering insights into the dynamics of user interactions in different chat environments. The variations of communication styles found in the group chats crucial for understanding user behavior and the role of different content types within group chats.

% \subsubsection{emoticons Distribution}
% Here, we analyze the distribution of emoticons and utterances across different groups. We present the summarized findings in Fig \ref{fig:barchart-emoticon-utterances-distribution}.

% \begin{figure}[htbp] 
%     \centering \includegraphics[width=1.0\linewidth]{Images/barchart-emoticon-utterances-distribution.jpg} 
%     \caption{Distribution of emoticon, utterances across different groups and their respective categories.} 
% \label{fig:barchart-emoticon-utterances-distribution} 
% \end{figure}

% Groups within the Anime category show substantial emoticon usage, with one group reaching over 144,000 emoticons, followed by others with smaller quantities like 297 and 213 emoticons. Fan Club groups also exhibit significant emoticon activity, with one group accumulating 33,673 emoticons, while others show modest emoticon counts like 108 and 169. The Finance category displays a wide range of emoticon usage, with one group generating over 9000 emoticons and others, such as one with just 1 emoticon, contributing to the overall distribution. Game groups lead with some of the highest emoticon counts, particularly a group with over 90,000 emoticons. The Language category, although diverse, contains groups with smaller emoticon counts, with the highest reaching 41,733 emoticons. Similarly, groups in the Outdoor category show a mix of low and moderate emoticon use, with some groups accumulating only a few emoticons. Finally, the Social category has a broad range of emoticon usage, from just a few emoticons in some groups to over 9,500 in others. 
% Overall, the data indicates that while some categories, like Finance and Game, exhibit high emoticon usage, others like Technology and Language show much lower levels, reflecting diverse user engagement in different contexts.

% \subsubsection{User Usage Analysis}
% \paragraph{Users' emoticon Usage}
% Then, we analyze the proportion of users using emoticons and does who do not.
% \begin{figure}[htbp] 
%     \centering \includegraphics[width=1.0\linewidth]{Images/barchart-users-usage-distribution.jpg} 
%     \caption{Distribution of users' emoticon usage.} 
% \label{fig:barchart-users-usage-distribution} 
% \end{figure}
% % change to global

% The distribution of users' emoticon usage across different categories highlights varying levels of engagement. In the Anime category, a significant portion of users (1,484 out of 4,874) have used emoticons, though some smaller groups show minimal usage, with only 33 and 10 users using emoticons in other subgroups. Fan Club groups also demonstrate diverse usage, with one group showing only 36 users who have used emoticons, while another group has a more substantial engagement of 62 users. The Finance category has a wide disparity, with over 19,000 users in one subgroup, and only a small fraction (1 or 4) engaging with emoticons in others. In Game, large groups like those with 92,987 and 14,717 users have notable emoticon usage, but smaller groups like those with just 37 or 146 users show much lower engagement. The Language category also varies, with some large subgroups (e.g., 22,190 users) contributing to substantial emoticon usage, while others like 1,559 users show minimal activity. Outdoor groups reveal a generally low level of emoticon engagement, with some subgroups, like one with 75 users, showing no emoticon usage. Social groups generally demonstrate moderate to low engagement, with several groups having over 100 users using emoticons, although some groups, like those with just 8 or 22 users, show minimal engagement. Finally, in the Media Sharing and Technology categories, large groups like those with 3,939 and 1,896 users show considerable engagement, while others with smaller user counts exhibit little to no emoticon usage. 

% Overall, this highlights that while larger user groups in categories like Finance and Game tend to show more emoticon engagement, many smaller categories exhibit limited use of emoticons.

% \paragraph{User emoticon Usage Overlap}


% We analyze the overlap of users in same and different categories by calculating the minimum, median, average, and maximum for both categories. In the "Same Category" group, the minimum value is 0, indicating that some groups do not have any users overlapping in the same category. The median value is 10, while the average overlap is 26.5 users, and the maximum overlap is observed in the Language group with 320 users, the statistics can be seen in Fig \ref{tab:overlap-statistics}. 

% \begin{table}[htbp]
% \centering
% \begin{tabular}{lcccc}
% \toprule
% Field & Average & Median & Min & Max \\ \midrule
% Same Cat Overlap & 26.5 & 10 & 0 & 320 (Language) \\
% Diff. Cat Overlap & 14.5 & 4 & 0 & 129 (Game) \\
% \bottomrule
% \end{tabular}
% \caption{Overlap of Users in Same and Different Categories}
% \label{tab:overlap-statistics}
% \end{table}

% For the "Different Category" group, the minimum value is also 0, signifying some groups have no overlap with users from other categories. The median value for overlap in different categories is 4, and the average number of overlapping users in different categories is 14.5. The maximum value of 129 users occurs in the Game group, showing that some groups have significantly more overlap in different categories.

% The top three groups with the most overlap in the same category are as follows: the Language group leads with 320 users, followed by the Finance group with 83 users, and the Game group with 66 users. When considering the different categories, the Game group has the most overlap, with 129 users found in other groups of different categories. The Finance group follows with 68 users, and the emoticon Sharing group has 62 users.

% The high overlap in the Language group may be attributed to the broader nature of the category, where users may have interests that span across multiple topics and discussions in related groups. For the Finance group, it is likely that the financial topics are of significant interest to a wide audience, leading to participation in several related groups. The Game group, similarly, may attract users with a wide range of gaming interests, leading to a higher likelihood of overlap in both same and different category groups. emoticon Sharing groups likely have overlap due to the nature of sharing common interests across various categories of social groups, leading to users participating in multiple sub-categories.

% The findings on overlap suggest that user behavior across multiple categories can be complex, indicating the possibility of shared interests between seemingly unrelated domains. This has important implications for multidomain systems, which will be explored further in another section. Such overlaps show the necessity of considering users' diverse preferences in recommendation systems, where a user’s interests may span across different categories, and designing systems that take these overlaps into account can improve recommendation accuracy and personalization.


% \subsubsection{emoticon Usage}
% % - max count, min count, median count, average count

% % \subsubsection{Cross Domain Analysis}
% % Interestingly, we notice that users overlap amongst groups. More specifically, we present the portion of users that exists in other groups.

% % - how many percentage of users appear in other groups across all groups?



% % \subsection{Cross-Domain}

% % <table>

% % We analyze the languages, medias, time talking.



% Domain (cross user domain), language, sparsity, conversation, languages. Case studies.
% \section{Downstream Application}
\section{Application Case 2: Emoticon Recommendation}
\label{sec:experiments}
In this section, we evaluate the feasibility of our dataset through quantitative experiments. Specifically, we evaluate our dataset on current emoticon recommendation.

% We \todo{highlight} the following research questions (RQ):
% Interested in user behavior analysis and emoticon retrieval task.
% \begin{itemize}
%   \item \textbf{RQ1:} Are user's preferences distinctive?
%   \item \textbf{RQ2:} Do users behave differently in different groups or with different people?
%   \item \textbf{RQ3:} Do user's preference styles change over time?
%   \item \textbf{RQ4:} How does emoticonU behave with current emoticon retrieval methods?
% \end{itemize}

% \subsection{emoticon Retrieval}
\subsection{Experiment Settings} 
We evaluate two baseline methods for emoticon recommendation, described as follows:

\begin{enumerate} 
    \item \textbf{MOD} \cite{MOD}: Takes a list of negative samples along with one true positive sample and calculates the probability of each sample being the correct match. The results are then used to rank the samples.
    \item \textbf{SRS} \cite{learning-to-respond-2021}: This approach ranks a list containing negative and one positive sample, returning the ranked results.
\end{enumerate}

% 1. language restriction
% 2. domain restriction
% 3. 2 emoticon retrieval models, 1.2.
We conduct experiments on three distinct perspectives of the dataset: (1) the English and Chinese subsets, (2) the domain-specific subset, and (3) the complete dataset.

\subsubsection{English and Chinese Subset}
Since MOD can only handle English and Chinese, we select groups with more than 80\% of text being either English or Chinese and train the model on it. We do the same for SRS.

\subsubsection{Domain}
To interpretate the performance between different domains, we separate the dataset by domains as shown in Table \ref{tab:category-language-sparsity}. Since SRS is not limited by multilingualism, we evaluate using SRS.

\subsubsection{Complete}
Finally, the entire dataset is utilized for evaluation on SRS.

\subsection{Results}
\subsubsection{English and Chinese Subset}
Based on the experimental results presented in Table \ref{tab:experiments-language}, we observe that the English language performs better than the Chinese language across both the MOD and SRS methods. Specifically, the English subset consistently shows higher scores in all evaluation metrics (MRR, R@1, R@3, R@5, and R@10). For instance, in the 1:9 ratio, the MOD method for English achieves an MRR of 0.289, while the Chinese MOD method only reaches 0.317, this could be due to the smaller data size of Chinese in comparison to English.

% The general behavior of the emoticon recommendation models is consistent with typical recommendation system patterns. Both MOD and SRS methods exhibit higher performance when the negative-to-positive sample ratio is 1:9, as seen in the higher metrics across both languages. This suggests that the models perform better when the positive sample is more prevalent. Performance drops when the negative-to-positive ratio increases to 1:19, which is a common trend in recommendation tasks where increasing the number of negative samples makes the task more difficult.

% The results also provide valuable insights into the emoticonU dataset. The variation in performance across different sample ratios indicates that the dataset contains a diverse set of emoticons and user interactions. The generally lower performance in the Chinese subset could suggest potential language-specific challenges, such as linguistic or cultural nuances that affect the model’s ability to recommend emoticons effectively. In contrast, the higher performance on the English dataset might reflect better training data quality or more focused user interactions. These findings highlight the importance of considering language-specific factors when designing recommendation models, and the difference in performance between MOD and SRS emphasizes how the composition of the sample affects recommendation outcomes.



% only full day, epoch 50, 9+1, 19+1
% \begin{table*}[ht]
% \centering
% \begin{tabular}{lcccccccc}
% \toprule
% Baseline & Epochs & \# of Negative Samples & MRR & R@1 & R@3 & R@5 & R@10 & R@20 \\ \midrule
% \textbf{MOD} &&&&&&&& \\ \hline
% \textit{Chinese} & 20 & 4+1 & \textbf{0.435} & 0.156 & \textbf{0.625} & \textbf{1.000} & - & - \\
%  & & 9+1 & 0.255 & 0.063 & 0.313 & 0.500 & 1.000 & - \\
%  & & 19+1 & 0.224 & 0.063 & 0.281 & 0.344 & 0.563 & 1.000 \\
%  & 50 & 4+1 & 0.362 & 0.063 & 0.593 & \textbf{1.000} & - & - \\
%  & & 9+1 & 0.317 & 0.125 & 0.344 & 0.531 & 1.000 & \\
%  & & 19+1 & 0.156 & 0.063 & 0.094 & 0.156 & 0.406 & 1.000 \\
% \hline
% \textit{English} & 20 & 4+1 & \textbf{0.450} & 0.194 & \textbf{0.585} & \textbf{1.000} & & \\
%  & & 9+1 & 0.293 & 0.095 & 0.314 & 0.510 & \textbf{1.000} & \\
%  & & 19+1 & 0.178 & 0.051 & 0.144 & 0.238 & 0.494 & \textbf{1.000} \\
%  & 50 & 4+1 & \textbf{0.459} & 0.199 & \textbf{0.607} & \textbf{1.000} & - & - \\
%  & & 9+1 & 0.289 & \textbf{0.307} & 0.495 & 0.224 & \textbf{1.000} & - \\
%  & & 19+1 & 0.180 & 0.051 & 0.145 & 0.250 & 0.511 & 1.000 \\
% \midrule
% \textbf{SRS} &&&&&&&& \\ \hline
% \textit{Chinese} & half day & 4+1 & 0.447 & 0.214 & 0.582 & \textbf{1.000} & & \\
%  & & 9+1 & 0.264 & 0.056 & 0.281 & 0.500 & \textbf{0.891} & \\
%  & & 19+1 & 0.270 & 0.061 & 0.138 & 0.219 & 0.377 & \textbf{0.891} \\
%  & full day & 4+1 & 0.447 & 0.214 & 0.582 & \textbf{1.000} & - & - \\
%  & & 9+1 & 0.264 & 0.056 & 0.281 & 0.500 & \textbf{0.891} & \\
%  & & 19+1 & 0.270 & 0.061 & 0.138 & 0.219 & 0.377 & \textbf{0.891} \\
% \hline
% \textit{English} & half day & 4+1 & 0.413 & 0.187 & 0.597 & \textbf{1.000} & - & - \\
%  & & 9+1 & 0.291 & 0.098 & 0.298 & 0.495 & 1.0 & - \\
%  & & 19+1 & 0.299 & 0.050 & 0.155 & 0.256 & 0.429 & 1.0 \\
%  & full day & 4+1 & 0.413 & 0.187 & 0.597 & \textbf{1.000} & - & - \\
%  & & 9+1 & 0.291 & 0.098 & 0.298 & 0.495 & 1.0 & - \\
%  & & 19+1 & 0.299 & 0.050 & 0.155 & 0.256 & 0.429 & 1.0 \\
% \bottomrule
% \end{tabular}
% \caption{emoticon Retrieval Results for MOD and SRS (English and Chinese)}
% \label{tab:experiments-language}
% \end{table*}

\begin{table}[htbp]
\centering
\caption{emoticon Retrieval Results for MOD and SRS (English and Chinese). We shortform Negative Samples for NS.}
\begin{tabular}{lclccccc}
\toprule
Lang & NS & Method & MRR & R@1 & R@3 & R@5 & R@10 \\ \midrule
% \textbf{MOD} &&&&&&&& \\ \hline
% \textit{English}  & 4+1 & \textbf{0.450} & 0.194 & \textbf{0.585} & \textbf{1.000} & & \\
 % & 9+1 & 0.293 & 0.095 & 0.314 & 0.510 & \textbf{1.000} & \\
 % & 19+1 & 0.178 & 0.051 & 0.144 & 0.238 & 0.494 & \textbf{1.000} \\
 % & 4+1 & \textbf{0.459} & 0.199 & \textbf{0.607} & \textbf{1.000} & - & - \\
 \textit{En} & 1:9 & MOD  & 0.289 & 0.307 & 0.495 & 0.224 & 1.000  \\
  & & SRS  & 0.291 & 0.098 & 0.298 & 0.495 & 1.000 \\  \cmidrule(lr){2-8} 
 &  1:19 & MOD & 0.180 & 0.051 & 0.145 & 0.250 & 0.511 \\ 
 &  & SRS & 0.299  & 0.050 & 0.155 & 0.256 & 0.429 \\ \hline
 
 % \textbf{MOD}& \textit{English} & 9+1 & 0.289 & \textbf{0.307} & 0.495 & 0.224 & \textbf{1.000}  \\
 % & & 19+1 & 0.180 & 0.051 & 0.145 & 0.250 & 0.511 \\ \hline
 
% \textit{Chinese} & 4+1 & \textbf{0.435} & 0.156 & \textbf{0.625} & \textbf{1.000} & - & - \\
 % & 9+1 & 0.255 & 0.063 & 0.313 & 0.500 & 1.000 & - \\
 % & 19+1 & 0.224 & 0.063 & 0.281 & 0.344 & 0.563 & 1.000 \\
 % &  4+1 & 0.362 & 0.063 & 0.593 & \textbf{1.000} & - & - \\
\textit{Zh}  &  1:9 & MOD  & 0.317 & 0.125 & 0.344 & 0.531 & 1.000  \\
& & SRS   & 0.264 & 0.056 & 0.281 & 0.500 & 1.000 \\ \cmidrule(lr){2-8} 
 &  1:19 & MOD & 0.156 & 0.063 & 0.094 & 0.156 & 0.406  \\ 
 &  & SRS & 0.270 & 0.061 & 0.138 & 0.219 & 0.377  \\

% \textbf{SRS} &&&&&&& \\ \hline
% \textit{English}  & 4+1 & 0.413 & 0.187 & 0.597 & \textbf{1.000} & - & - \\
% \textit{English}  & 9+1 & 0.291 & 0.098 & 0.298 & 0.495 & 1.0 & - \\
%  & 19+1 & 0.299 & 0.050 & 0.155 & 0.256 & 0.429 & 1.0 \\
%  &  4+1 & 0.413 & 0.187 & 0.597 & \textbf{1.000} & - & - \\

 % \textbf{SRS} & \textit{English} & 9+1 & 0.291 & 0.098 & 0.298 & 0.495 & 1.0 \\
 % & & 19+1 & 0.299 & 0.050 & 0.155 & 0.256 & 0.429 \\ \hline
 
% \textit{Chinese} &  4+1 & 0.447 & 0.214 & 0.582 & \textbf{1.000} & & \\
 % \textit{Chinese}& 9+1 & 0.264 & 0.056 & 0.281 & 0.500 & \textbf{0.891} & \\
 % & 19+1 & 0.270 & 0.061 & 0.138 & 0.219 & 0.377 & \textbf{0.891} \\
 % & 4+1 & 0.447 & 0.214 & 0.582 & \textbf{1.000} & - & - \\
 % & \textbf{SRS}  & 9+1 & 0.264 & 0.056 & 0.281 & 0.500 & \textbf{1.000} \\
 % & & 19+1 & 0.270 & 0.061 & 0.138 & 0.219 & 0.377  \\
\bottomrule
\end{tabular}
\label{tab:experiments-language}
\end{table}


% - English and Chinese only (MOD, SRS)
% - Full (SRS only)
% - domain (MOD, SRS)
% \subsection{User Behavior Modeling} % put to 4.3 user analysis

% \subsubsection{User Across the Same Categories}
% We showcase the styles of a user across different categories.

\subsubsection{Domain}
Table \ref{tab:domain-negative-samples} reveal that the \textbf{emoticons} domain performs the best overall, with an MRR of 0.335, R@1 of 0.143, R@3 of 0.365, and R@5 of 0.552. On the other hand Arts domain has a relatively lower performance, possibly due to the higher sparsity (0.019143).
\begin{table}[h]
    \centering
    \caption{Results of SRS on individual domains.}
    \begin{tabular}{lllll}
    \toprule
    \textbf{Domain} & \textbf{MRR} & \textbf{R@1} & \textbf{R@3} & \textbf{R@5} \\
    \hline
    
     \textbf{Anime} 
    % & 4+1 & 0.424 & 0.2 & 0.6 & 1 & - & - \\
     & 0.294 & 0.101 & 0.298 & 0.500 \\
    % & 19+1 & - & - & - & - & - & - \\
    
    \textbf{Arts} 
    % & 4+1 & 0.468 & 0.281 & 0.656 & 1 & 0 & 0 \\
   & 0.288 & 0.094 & 0.281 & 0.484 \\
    % & 19+1 & 0.336 & 0.031 & 0.188 & 0.266 & 0.507 & 1 \\

    \textbf{Fan Club}
    % & 4+1 & 0.417 & 0.18 & 0.603 & 1 & 0 & 0 \\
     & 0.280 & 0.080 & 0.280 & 0.510 \\
    % & 19+1 & 0.318 & 0.128 & 0.163 & 0.268 & 0.326 & 1 \\

    \textbf{Finance} 
    % & 4+1 & 0.426 & 0.2 & 0.605 & 1 & 0 & 0 \\
    & 0.296 & 0.080 & 0.280 & 0.510 \\
    % & 19+1 & - & - & - & - & - & - \\
    
    \textbf{Game} 
    % & 4+1 & 0.428 & 0.203 & 0.606 & 1 & 0 & 0 \\
     & 0.293 & 0.100 & 0.301 & 0.498  \\
    % & 19+1 & - & - & - & - & - & - \\

     \textbf{Language} 
    % & 4+1 & 0.425 & 0.201 & 0.598 & 1 & 0 & 0 \\
     & 0.296 & 0.103 & 0.307 & 0.501 \\
     
    \textbf{Outdoor} 
    % & 4+1 & 0.434 & 0.195 & 0.589 & 1 & 0 & 0 \\
    & 0.298 & 0.103 & 0.324 & 0.500 \\
    % & 19+1 & - & - & - & - & - & - \\
     
     \textbf{Social}
    % & 4+1 & 0.42 & 0.2 & 0.587 & 1 & 0 & 0 \\
    & 0.300 & 0.105 & 0.305 & 0.513 \\
    % & 19+1 & - & - & - & - & - & - \\
    
    \textbf{Media Sharing} 
    % & 4+1 & 0.462 & 0.242 & 0.636 & 1 & 0 & 0 \\
    &  0.335 & 0.143 & 0.365 & 0.552 \\
    % & 19+1 & - & - & - & - & - & - \\
    \textbf{Tech} 
    % & 4+1 & 0.409 & 0.153 & 0.569 & 1 & 0 & 0 \\
    & 0.321 & 0.125 & 0.361 & 0.611  \\
    % & 19+1 & 0.287 & 0.083 & 0.125 & 0.278 & 0.384 & 1 \\
   
    \bottomrule
    \end{tabular}
    \label{tab:domain-negative-samples}
\end{table}


\subsubsection{Complete Dataset}
The results in Table \ref{tab:srs-full} show that performance decreases as the number of negative samples increases. With 4 negative samples, the model achieves a \textbf{MRR} of 0.425 and \textbf{R@1} of 0.202, indicating good performance. As the number of negative sample increase, a decline in accuracy is observed, which is consistent with general recommendation models. 

\begin{table}[h!]
\centering
\caption{Performance metrics for varying negative samples across multiple evaluation criteria on the entire dataset using SRS method.}
\begin{tabular}{ccccccc}
\hline
\multirow{2}{*}{{NS}} & \multicolumn{6}{c}{Evaluation Metrics} \\
\cline{2-7}
& MRR  & R@1  & R@3  & R@5  & R@10 & R@20 \\
\hline
1:4  & 0.425 & 0.202 & 0.603 & 1.0  & -    & - \\
1:9  & 0.294 & 0.102 & 0.303 & 0.499 & 1.0  & - \\
1:19 & 0.296 & 0.054 & 0.153 & 0.251 & 0.332 & 1.0 \\
\hline
\end{tabular}
\label{tab:srs-full}
\end{table}

\section{Discussions}

\subsection{Other Potential Applications}
Beyond user behavior analysis and emoticon recommendation, the dataset presents several opportunities for further research. One potential avenue is a more extensive \textbf{User Behavior Modeling}, which could involve analyzing temporal changes in user behavior and investigating whether emoticon usage varies based on the responder. Additional potential applications include, but are not limited to, \textbf{Personalized Emoticon Retrieval} and \textbf{Generative AI}, both of which have previously faced dataset limitations \cite{pmg, pigeon}.

We acknowledge that there are several limitations in our work. Despite extensive automatic and manual data verification, privacy and safety concerns may still arise, presenting an opportunity for advancements in addressing such issues.

\subsection{Ethical Considerations}
We affirm that we have used the data in accordance with Telegram's Terms and Conditions \cite{telegram_tos}. Given a user's expression of opposition, we refrain from crawling to respect their privacy. Additionally, our dataset aligns with the FAIR principles, which emphasize making data \textit{Findable}, \textit{Accessible}, \textit{Interoperable}, and \textit{Reusable} \cite{fair_principles}.


\section{Conclusion}

% In this paper, we present emoticonU dataset, which is the largest emoticon dataset to date, containing 22.6k users 105.9k emoticons and 8.8M conversation messages of texts and emoticons.
% It is  a multi-domain dataset that includes rich and diverse information. Covering 10 domains, it captures temporal, multilingual, and cross-domain behaviors that are not present in previous datasets.
% Extensive quantitative and qualitative experiments demonstrate emoticonU’s practical applications on user behavior analysis and modeling, personalized emoticon recommendation, which potentially can be used in more downstream scenarios.

In this paper, we introduce our emoticon dataset, the largest emoticon dataset to date, comprising data from 22K users, 370.2K emoticons, and 8.3M conversation messages. This multi-domain dataset encompasses a wealth of diverse information, capturing temporal, multilingual, and cross-domain behaviors not found in previous datasets. 
Extensive quantitative and qualitative experiments demonstrate the practical applications of our dataset in user behavior modeling and personalized emoticon recommendation. It also holds potential for further research in areas such as personalized retrieval and conversational studies.

% \section{Introduction}
% ACM's consolidated article template, introduced in 2017, provides a
% consistent \LaTeX\ style for use across ACM publications, and
% incorporates accessibility and metadata-extraction functionality
% necessary for future Digital Library endeavors. Numerous ACM and
% SIG-specific \LaTeX\ templates have been examined, and their unique
% features incorporated into this single new template.

% If you are new to publishing with ACM, this document is a valuable
% guide to the process of preparing your work for publication. If you
% have published with ACM before, this document provides insight and
% instruction into more recent changes to the article template.

% The ``\verb|acmart|'' document class can be used to prepare articles
% for any ACM publication --- conference or journal, and for any stage
% of publication, from review to final ``camera-ready'' copy, to the
% author's own version, with {\itshape very} few changes to the source.

% \section{Template Overview}
% As noted in the introduction, the ``\verb|acmart|'' document class can
% be used to prepare many different kinds of documentation --- a
% double-anonymous initial submission of a full-length technical paper, a
% two-page SIGGRAPH Emerging Technologies abstract, a ``camera-ready''
% journal article, a SIGCHI Extended Abstract, and more --- all by
% selecting the appropriate {\itshape template style} and {\itshape
%   template parameters}.

% This document will explain the major features of the document
% class. For further information, the {\itshape \LaTeX\ User's Guide} is
% available from
% \url{https://www.acm.org/publications/proceedings-template}.

% \subsection{Template Styles}

% The primary parameter given to the ``\verb|acmart|'' document class is
% the {\itshape template style} which corresponds to the kind of publication
% or SIG publishing the work. This parameter is enclosed in square
% brackets and is a part of the {\verb|documentclass|} command:
% \begin{verbatim}
%   \documentclass[STYLE]{acmart}
% \end{verbatim}

% Journals use one of three template styles. All but three ACM journals
% use the {\verb|acmsmall|} template style:
% \begin{itemize}
% \item {\texttt{acmsmall}}: The default journal template style.
% \item {\texttt{acmlarge}}: Used by JOCCH and TAP.
% \item {\texttt{acmtog}}: Used by TOG.
% \end{itemize}

% The majority of conference proceedings documentation will use the {\verb|acmconf|} template style.
% \begin{itemize}
% \item {\texttt{sigconf}}: The default proceedings template style.
% \item{\texttt{sigchi}}: Used for SIGCHI conference articles.
% \item{\texttt{sigplan}}: Used for SIGPLAN conference articles.
% \end{itemize}

% \subsection{Template Parameters}

% In addition to specifying the {\itshape template style} to be used in
% formatting your work, there are a number of {\itshape template parameters}
% which modify some part of the applied template style. A complete list
% of these parameters can be found in the {\itshape \LaTeX\ User's Guide.}

% Frequently-used parameters, or combinations of parameters, include:
% \begin{itemize}
% \item {\texttt{anonymous,review}}: Suitable for a ``double-anonymous''
%   conference submission. Anonymizes the work and includes line
%   numbers. Use with the \texttt{\acmSubmissionID} command to print the
%   submission's unique ID on each page of the work.
% \item{\texttt{authorversion}}: Produces a version of the work suitable
%   for posting by the author.
% \item{\texttt{screen}}: Produces colored hyperlinks.
% \end{itemize}

% This document uses the following string as the first command in the
% source file:
% \begin{verbatim}
% \documentclass[sigconf]{acmart}
% \end{verbatim}

% \section{Modifications}

% Modifying the template --- including but not limited to: adjusting
% margins, typeface sizes, line spacing, paragraph and list definitions,
% and the use of the \verb|\vspace| command to manually adjust the
% vertical spacing between elements of your work --- is not allowed.

% {\bfseries Your document will be returned to you for revision if
%   modifications are discovered.}

% \section{Typefaces}

% The ``\verb|acmart|'' document class requires the use of the
% ``Libertine'' typeface family. Your \TeX\ installation should include
% this set of packages. Please do not substitute other typefaces. The
% ``\verb|lmodern|'' and ``\verb|ltimes|'' packages should not be used,
% as they will override the built-in typeface families.

% \section{Title Information}

% The title of your work should use capital letters appropriately -
% \url{https://capitalizemytitle.com/} has useful rules for
% capitalization. Use the {\verb|title|} command to define the title of
% your work. If your work has a subtitle, define it with the
% {\verb|subtitle|} command.  Do not insert line breaks in your title.

% If your title is lengthy, you must define a short version to be used
% in the page headers, to prevent overlapping text. The \verb|title|
% command has a ``short title'' parameter:
% \begin{verbatim}
%   \title[short title]{full title}
% \end{verbatim}

% \section{Authors and Affiliations}

% Each author must be defined separately for accurate metadata
% identification.  As an exception, multiple authors may share one
% affiliation. Authors' names should not be abbreviated; use full first
% names wherever possible. Include authors' e-mail addresses whenever
% possible.

% Grouping authors' names or e-mail addresses, or providing an ``e-mail
% alias,'' as shown below, is not acceptable:
% \begin{verbatim}
%   \author{Brooke Aster, David Mehldau}
%   \email{dave,judy,steve@university.edu}
%   \email{firstname.lastname@phillips.org}
% \end{verbatim}

% The \verb|authornote| and \verb|authornotemark| commands allow a note
% to apply to multiple authors --- for example, if the first two authors
% of an article contributed equally to the work.

% If your author list is lengthy, you must define a shortened version of
% the list of authors to be used in the page headers, to prevent
% overlapping text. The following command should be placed just after
% the last \verb|\author{}| definition:
% \begin{verbatim}
%   \renewcommand{\shortauthors}{McCartney, et al.}
% \end{verbatim}
% Omitting this command will force the use of a concatenated list of all
% of the authors' names, which may result in overlapping text in the
% page headers.

% The article template's documentation, available at
% \url{https://www.acm.org/publications/proceedings-template}, has a
% complete explanation of these commands and tips for their effective
% use.

% Note that authors' addresses are mandatory for journal articles.

% \section{Rights Information}

% Authors of any work published by ACM will need to complete a rights
% form. Depending on the kind of work, and the rights management choice
% made by the author, this may be copyright transfer, permission,
% license, or an OA (open access) agreement.

% Regardless of the rights management choice, the author will receive a
% copy of the completed rights form once it has been submitted. This
% form contains \LaTeX\ commands that must be copied into the source
% document. When the document source is compiled, these commands and
% their parameters add formatted text to several areas of the final
% document:
% \begin{itemize}
% \item the ``ACM Reference Format'' text on the first page.
% \item the ``rights management'' text on the first page.
% \item the conference information in the page header(s).
% \end{itemize}

% Rights information is unique to the work; if you are preparing several
% works for an event, make sure to use the correct set of commands with
% each of the works.

% The ACM Reference Format text is required for all articles over one
% page in length, and is optional for one-page articles (abstracts).

% \section{CCS Concepts and User-Defined Keywords}

% Two elements of the ``acmart'' document class provide powerful
% taxonomic tools for you to help readers find your work in an online
% search.

% The ACM Computing Classification System ---
% \url{https://www.acm.org/publications/class-2012} --- is a set of
% classifiers and concepts that describe the computing
% discipline. Authors can select entries from this classification
% system, via \url{https://dl.acm.org/ccs/ccs.cfm}, and generate the
% commands to be included in the \LaTeX\ source.

% User-defined keywords are a comma-separated list of words and phrases
% of the authors' choosing, providing a more flexible way of describing
% the research being presented.

% CCS concepts and user-defined keywords are required for for all
% articles over two pages in length, and are optional for one- and
% two-page articles (or abstracts).

% \section{Sectioning Commands}

% Your work should use standard \LaTeX\ sectioning commands:
% \verb|section|, \verb|subsection|, \verb|subsubsection|, and
% \verb|paragraph|. They should be numbered; do not remove the numbering
% from the commands.

% Simulating a sectioning command by setting the first word or words of
% a paragraph in boldface or italicized text is {\bfseries not allowed.}

% \section{Tables}

% The ``\verb|acmart|'' document class includes the ``\verb|booktabs|''
% package --- \url{https://ctan.org/pkg/booktabs} --- for preparing
% high-quality tables.

% Table captions are placed {\itshape above} the table.

% Because tables cannot be split across pages, the best placement for
% them is typically the top of the page nearest their initial cite.  To
% ensure this proper ``floating'' placement of tables, use the
% environment \textbf{table} to enclose the table's contents and the
% table caption.  The contents of the table itself must go in the
% \textbf{tabular} environment, to be aligned properly in rows and
% columns, with the desired horizontal and vertical rules.  Again,
% detailed instructions on \textbf{tabular} material are found in the
% \textit{\LaTeX\ User's Guide}.

% Immediately following this sentence is the point at which
% Table~\ref{tab:freq} is included in the input file; compare the
% placement of the table here with the table in the printed output of
% this document.

% \begin{table}
%   \caption{Frequency of Special Characters}
%   \label{tab:freq}
%   \begin{tabular}{ccl}
%     \toprule
%     Non-English or Math&Frequency&Comments\\
%     \midrule
%     \O & 1 in 1,000& For Swedish names\\
%     $\pi$ & 1 in 5& Common in math\\
%     \$ & 4 in 5 & Used in business\\
%     $\Psi^2_1$ & 1 in 40,000& Unexplained usage\\
%   \bottomrule
% \end{tabular}
% \end{table}

% To set a wider table, which takes up the whole width of the page's
% live area, use the environment \textbf{table*} to enclose the table's
% contents and the table caption.  As with a single-column table, this
% wide table will ``float'' to a location deemed more
% desirable. Immediately following this sentence is the point at which
% Table~\ref{tab:commands} is included in the input file; again, it is
% instructive to compare the placement of the table here with the table
% in the printed output of this document.

% \begin{table*}
%   \caption{Some Typical Commands}
%   \label{tab:commands}
%   \begin{tabular}{ccl}
%     \toprule
%     Command &A Number & Comments\\
%     \midrule
%     \texttt{{\char'134}author} & 100& Author \\
%     \texttt{{\char'134}table}& 300 & For tables\\
%     \texttt{{\char'134}table*}& 400& For wider tables\\
%     \bottomrule
%   \end{tabular}
% \end{table*}

% Always use midrule to separate table header rows from data rows, and
% use it only for this purpose. This enables assistive technologies to
% recognise table headers and support their users in navigating tables
% more easily.

% \section{Math Equations}
% You may want to display math equations in three distinct styles:
% inline, numbered or non-numbered display.  Each of the three are
% discussed in the next sections.

% \subsection{Inline (In-text) Equations}
% A formula that appears in the running text is called an inline or
% in-text formula.  It is produced by the \textbf{math} environment,
% which can be invoked with the usual
% \texttt{{\char'134}begin\,\ldots{\char'134}end} construction or with
% the short form \texttt{\$\,\ldots\$}. You can use any of the symbols
% and structures, from $\alpha$ to $\omega$, available in
% \LaTeX~\cite{Lamport:LaTeX}; this section will simply show a few
% examples of in-text equations in context. Notice how this equation:
% \begin{math}
%   \lim_{n\rightarrow \infty}x=0
% \end{math},
% set here in in-line math style, looks slightly different when
% set in display style.  (See next section).

% \subsection{Display Equations}
% A numbered display equation---one set off by vertical space from the
% text and centered horizontally---is produced by the \textbf{equation}
% environment. An unnumbered display equation is produced by the
% \textbf{displaymath} environment.

% Again, in either environment, you can use any of the symbols and
% structures available in \LaTeX\@; this section will just give a couple
% of examples of display equations in context.  First, consider the
% equation, shown as an inline equation above:
% \begin{equation}
%   \lim_{n\rightarrow \infty}x=0
% \end{equation}
% Notice how it is formatted somewhat differently in
% the \textbf{displaymath}
% environment.  Now, we'll enter an unnumbered equation:
% \begin{displaymath}
%   \sum_{i=0}^{\infty} x + 1
% \end{displaymath}
% and follow it with another numbered equation:
% \begin{equation}
%   \sum_{i=0}^{\infty}x_i=\int_{0}^{\pi+2} f
% \end{equation}
% just to demonstrate \LaTeX's able handling of numbering.

% \section{Figures}

% The ``\verb|figure|'' environment should be used for figures. One or
% more images can be placed within a figure. If your figure contains
% third-party material, you must clearly identify it as such, as shown
% in the example below.
% \begin{figure}[h]
%   \centering
%   \includegraphics[width=\linewidth]{sample-franklin}
%   \caption{1907 Franklin Model D roadster. Photograph by Harris \&
%     Ewing, Inc. [Public domain], via Wikimedia
%     Commons. (\url{https://goo.gl/VLCRBB}).}
%   \Description{A woman and a girl in white dresses sit in an open car.}
% \end{figure}

% Your figures should contain a caption which describes the figure to
% the reader.

% Figure captions are placed {\itshape below} the figure.

% Every figure should also have a figure description unless it is purely
% decorative. These descriptions convey what’s in the image to someone
% who cannot see it. They are also used by search engine crawlers for
% indexing images, and when images cannot be loaded.

% A figure description must be unformatted plain text less than 2000
% characters long (including spaces).  {\bfseries Figure descriptions
%   should not repeat the figure caption – their purpose is to capture
%   important information that is not already provided in the caption or
%   the main text of the paper.} For figures that convey important and
% complex new information, a short text description may not be
% adequate. More complex alternative descriptions can be placed in an
% appendix and referenced in a short figure description. For example,
% provide a data table capturing the information in a bar chart, or a
% structured list representing a graph.  For additional information
% regarding how best to write figure descriptions and why doing this is
% so important, please see
% \url{https://www.acm.org/publications/taps/describing-figures/}.

% \subsection{The ``Teaser Figure''}

% A ``teaser figure'' is an image, or set of images in one figure, that
% are placed after all author and affiliation information, and before
% the body of the article, spanning the page. If you wish to have such a
% figure in your article, place the command immediately before the
% \verb|\maketitle| command:
% \begin{verbatim}
%   \begin{teaserfigure}
%     \includegraphics[width=\textwidth]{sampleteaser}
%     \caption{figure caption}
%     \Description{figure description}
%   \end{teaserfigure}
% \end{verbatim}

% \section{Citations and Bibliographies}

% The use of \BibTeX\ for the preparation and formatting of one's
% references is strongly recommended. Authors' names should be complete
% --- use full first names (``Donald E. Knuth'') not initials
% (``D. E. Knuth'') --- and the salient identifying features of a
% reference should be included: title, year, volume, number, pages,
% article DOI, etc.

% The bibliography is included in your source document with these two
% commands, placed just before the \verb|\end{document}| command:
% \begin{verbatim}
%   \bibliographystyle{ACM-Reference-Format}
%   \bibliography{bibfile}
% \end{verbatim}
% where ``\verb|bibfile|'' is the name, without the ``\verb|.bib|''
% suffix, of the \BibTeX\ file.

% Citations and references are numbered by default. A small number of
% ACM publications have citations and references formatted in the
% ``author year'' style; for these exceptions, please include this
% command in the {\bfseries preamble} (before the command
% ``\verb|\begin{document}|'') of your \LaTeX\ source:
% \begin{verbatim}
%   \citestyle{acmauthoryear}
% \end{verbatim}


%   Some examples.  A paginated journal article \cite{Abril07}, an
%   enumerated journal article \cite{Cohen07}, a reference to an entire
%   issue \cite{JCohen96}, a monograph (whole book) \cite{Kosiur01}, a
%   monograph/whole book in a series (see 2a in spec. document)
%   \cite{Harel79}, a divisible-book such as an anthology or compilation
%   \cite{Editor00} followed by the same example, however we only output
%   the series if the volume number is given \cite{Editor00a} (so
%   Editor00a's series should NOT be present since it has no vol. no.),
%   a chapter in a divisible book \cite{Spector90}, a chapter in a
%   divisible book in a series \cite{Douglass98}, a multi-volume work as
%   book \cite{Knuth97}, a couple of articles in a proceedings (of a
%   conference, symposium, workshop for example) (paginated proceedings
%   article) \cite{Andler79, Hagerup1993}, a proceedings article with
%   all possible elements \cite{Smith10}, an example of an enumerated
%   proceedings article \cite{VanGundy07}, an informally published work
%   \cite{Harel78}, a couple of preprints \cite{Bornmann2019,
%     AnzarootPBM14}, a doctoral dissertation \cite{Clarkson85}, a
%   master's thesis: \cite{anisi03}, an online document / world wide web
%   resource \cite{Thornburg01, Ablamowicz07, Poker06}, a video game
%   (Case 1) \cite{Obama08} and (Case 2) \cite{Novak03} and \cite{Lee05}
%   and (Case 3) a patent \cite{JoeScientist001}, work accepted for
%   publication \cite{rous08}, 'YYYYb'-test for prolific author
%   \cite{SaeediMEJ10} and \cite{SaeediJETC10}. Other cites might
%   contain 'duplicate' DOI and URLs (some SIAM articles)
%   \cite{Kirschmer:2010:AEI:1958016.1958018}. Boris / Barbara Beeton:
%   multi-volume works as books \cite{MR781536} and \cite{MR781537}. A
%   couple of citations with DOIs:
%   \cite{2004:ITE:1009386.1010128,Kirschmer:2010:AEI:1958016.1958018}. Online
%   citations: \cite{TUGInstmem, Thornburg01, CTANacmart}.
%   Artifacts: \cite{R} and \cite{UMassCitations}.

% \section{Acknowledgments}

% Identification of funding sources and other support, and thanks to
% individuals and groups that assisted in the research and the
% preparation of the work should be included in an acknowledgment
% section, which is placed just before the reference section in your
% document.

% This section has a special environment:
% \begin{verbatim}
%   \begin{acks}
%   ...
%   \end{acks}
% \end{verbatim}
% so that the information contained therein can be more easily collected
% during the article metadata extraction phase, and to ensure
% consistency in the spelling of the section heading.

% Authors should not prepare this section as a numbered or unnumbered {\verb|\section|}; please use the ``{\verb|acks|}'' environment.

% \section{Appendices}

% If your work needs an appendix, add it before the
% ``\verb|\end{document}|'' command at the conclusion of your source
% document.

% Start the appendix with the ``\verb|appendix|'' command:
% \begin{verbatim}
%   \appendix
% \end{verbatim}
% and note that in the appendix, sections are lettered, not
% numbered. This document has two appendices, demonstrating the section
% and subsection identification method.

% \section{Multi-language papers}

% Papers may be written in languages other than English or include
% titles, subtitles, keywords and abstracts in different languages (as a
% rule, a paper in a language other than English should include an
% English title and an English abstract).  Use \verb|language=...| for
% every language used in the paper.  The last language indicated is the
% main language of the paper.  For example, a French paper with
% additional titles and abstracts in English and German may start with
% the following command
% \begin{verbatim}
% \documentclass[sigconf, language=english, language=german,
%                language=french]{acmart}
% \end{verbatim}

% The title, subtitle, keywords and abstract will be typeset in the main
% language of the paper.  The commands \verb|\translatedXXX|, \verb|XXX|
% begin title, subtitle and keywords, can be used to set these elements
% in the other languages.  The environment \verb|translatedabstract| is
% used to set the translation of the abstract.  These commands and
% environment have a mandatory first argument: the language of the
% second argument.  See \verb|sample-sigconf-i13n.tex| file for examples
% of their usage.

% \section{SIGCHI Extended Abstracts}

% The ``\verb|sigchi-a|'' template style (available only in \LaTeX\ and
% not in Word) produces a landscape-orientation formatted article, with
% a wide left margin. Three environments are available for use with the
% ``\verb|sigchi-a|'' template style, and produce formatted output in
% the margin:
% \begin{description}
% \item[\texttt{sidebar}:]  Place formatted text in the margin.
% \item[\texttt{marginfigure}:] Place a figure in the margin.
% \item[\texttt{margintable}:] Place a table in the margin.
% \end{description}

%%
%% The acknowledgments section is defined using the "acks" environment
%% (and NOT an unnumbered section). This ensures the proper
%% identification of the section in the article metadata, and the
%% consistent spelling of the heading.
% \begin{acks}
% To Robert, for the bagels and explaining CMYK and color spaces.
% \end{acks}

%%
%% The next two lines define the bibliography style to be used, and
%% the bibliography file.
\clearpage

\bibliographystyle{ACM-Reference-Format}
\bibliography{sample-base}

\appendix
\onecolumn

%%
%% If your work has an appendix, this is the place to put it.
% \clearpage
% \appendix

% \section{Research Methods}

% \subsection{Part One}

% Lorem ipsum dolor sit amet, consectetur adipiscing elit. Morbi
% malesuada, quam in pulvinar varius, metus nunc fermentum urna, id
% sollicitudin purus odio sit amet enim. Aliquam ullamcorper eu ipsum
% vel mollis. Curabitur quis dictum nisl. Phasellus vel semper risus, et
% lacinia dolor. Integer ultricies commodo sem nec semper.

% \subsection{Part Two}

% Etiam commodo feugiat nisl pulvinar pellentesque. Etiam auctor sodales
% ligula, non varius nibh pulvinar semper. Suspendisse nec lectus non
% ipsum convallis congue hendrerit vitae sapien. Donec at laoreet
% eros. Vivamus non purus placerat, scelerisque diam eu, cursus
% ante. Etiam aliquam tortor auctor efficitur mattis.

% \section{Online Resources}

% Nam id fermentum dui. Suspendisse sagittis tortor a nulla mollis, in
% pulvinar ex pretium. Sed interdum orci quis metus euismod, et sagittis
% enim maximus. Vestibulum gravida massa ut felis suscipit
% congue. Quisque mattis elit a risus ultrices commodo venenatis eget
% dui. Etiam sagittis eleifend elementum.

% Nam interdum magna at lectus dignissim, ac dignissim lorem
% rhoncus. Maecenas eu arcu ac neque placerat aliquam. Nunc pulvinar
% massa et mattis lacinia.

\end{document}
\endinput
%%
%% End of file `sample-sigconf.tex'.
