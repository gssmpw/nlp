% introduction 2 1.5
% RW: 
% emoticonU Dataset: steps
% Data analysis (be more detailed)
% Experiments (justify the reasonability of the dataset)
% Decision VLM -prepare a list of description and ask to choose from the context, specifications
\section{Introduction}
\label{sec:intro}
\begin{figure}[htbp] 
    \centering \includegraphics[width=0.5\linewidth]{Images/scenario_v2.jpg} 
    \caption{The colorful and vivid emoticons makes for an expressive and comfortable conversation.} 
\label{fig:scenario} 
\end{figure}
Instant messaging (IM) has become an essential mode of communication, allowing users to interact efficiently. Beyond text, colorful and animated emoticons have emerged as a powerful medium of expression, enabling users to convey semantics and emotions comfortably and accurately~\cite{ge2020anatomy, constantin2019computational, kariko2019laughing, gif-interestingess, bitmoji}, as seen in Figure \ref{fig:scenario}.
Unlike general image datasets, which emphasize object recognition, emoticons carry rich semantic meaning, expressiveness, and emotional nuance, making their retrieval and recommendation fundamentally different from traditional image retrieval and recommendation tasks. 
% introduction longer
% mention about the retrival and recommenation task
% table 1 > double column, and mention in abstract, mention ours, include personal column. 2.1 be in the introduction. 2.2 emoticon 下有任务, widely research fields and areas, 2.2 and 2.3 exchange, 2.2 emoticon retrieval task and their value

% importance of the task
% which work has been done 
% however, table 1 mentions that it is problematic. and how our work improves
However, despite emoticons' widespread use, personalized emoticon retrieval and recommendation remains an understudied area, primarily due to the lack of large-scale user emoticon interaction dataset.
As shown in Table~\ref{tab:dataset_comparison}, while several emoticon dataset exist, they often lack user information, making them unable to identify the same user across different conversations.
Additionally, they contain few emoticons per conversation, providing limited insight into user preferences or emoticon usage patterns. Furthermore, some datasets are not open-sourced, limiting reproducibility and progress of personalized emoticon retrieval and recommendation.

% - Conv. Domain if the conversation has been labelled with a topic.
% - NA if the dataset does not support.
% - xmark not mentioned or upon inspection does not contain

\begin{table*}[ht]
\centering
\caption{Comparison with Emoticon Datasets. We bold the largest count among publicly available datasets and underline the second largest. Our dataset is cross-domain, large in scale and contain time and user ID for personalization.}
\begin{tabular}{lcccrrl}
\toprule
\textbf{Dataset} & 
\textbf{Pub. Avail.}  & 
\textbf{Cross Domain} & 
\textbf{Time Info.} &
\textbf{\# Sticks./user} & 
\textbf{\# Emoticons} & \textbf{Description} \\ \hline
StickerTag \cite{stickertag} & \xmark & \xmark & \xmark & \xmark & 13,571 & Emoticon-tag pairs \\
StickerInt \cite{stickerint} & \xmark & \xmark & \xmark & Unknown & 1,025 & Dialogues with emoticons \\
StickerCLIP \cite{stickerclip} & \xmark & \xmark & \xmark & \xmark & 820,000 & Chinese emoticon-tag pairs \\
CSMSA \cite{CSMSA} & \xmark & \xmark & \xmark & \xmark & 16,000 &  Emoticons-tag pairs \\
SRS, PESRS \cite{learning-to-respond-2021, learning-to-respond-with-stickers-2020} & \xmark & \xmark & \xmark & 6.82 & 320,168 & Dialogues with emoticons \\
PerSRV \cite{chee2024persrv} & \xmark & \xmark & \xmark & Unknown & 543,098 &  Emoticon-query pairs \\ \hline
MCDSCS \cite{mcdscs} & \cmark & \xmark & \xmark & \underline{2.03} & 14,400 & Dialogues with emoticons  \\
SER30K \cite{SER30K} & \cmark & \xmark & \xmark & \xmark & \underline{30,739} & Emoticon-sentiments pairs \\
MOD \cite{MOD} & \cmark & \xmark & \xmark & 1.36 & 307 & Dialogues with emoticons \\ \midrule
% emoticonU (Ours) & \cmark & \cmark & \cmark & \textbf{16.36} & \textbf{370,222} & Dialogues with emoticons \\
 Ours & \cmark & \cmark & \cmark & \textbf{16.90} & \textbf{370,222} & Dialogues with emoticons \\
\hline
\end{tabular}
\label{tab:dataset_comparison}
\end{table*}


% \begin{table*}[htbp]
% \centering
% \caption{Among the eight related datasets, three are currently unavailable, while two are restricted (as noted in the "Restrictions" section). The remaining three available datasets do not include user IDs for cross-conversation identity linking, leading to insufficient data for user modeling and personalized recommendations. We shortform Unknown as Uk.}
% unknown, dont have diff, bold our results
% \begin{tabular}{ccccccc}
% \toprule
% \textbf{Dataset} & 
% \textbf{Avail.}  & 
% \textbf{Cross Domain} & 
% \textbf{Time Info} &
% \textbf{User Info} & 
% \textbf{\#Unique emoticon} & \textbf{\#Messages} \\ \hline

% emoticonTag \cite{emoticontag} & \xmark  & \xmark & \xmark & \xmark &  & 13,571 emoticon-tag pairs \\ 
% emoticonInt \cite{emoticonint}& \xmark & \xmark & \xmark &  & 1,025 & 1,578 dialogues, 1,025 emoticons \\ 
% emoticonCLIP \cite{emoticonclip}& \xmark & NA & NA & NA & & 820,000 Chinese image-text \\ 

% CSMSA \cite{CSMSA} & Res. & \xmark & & & & 16,000 emoticons with labels \\ 

% PESRS \cite{learning-to-respond-2021} SRS \cite{learning-to-respond-with-emoticons-2020} & Res. & \xmark & & & & 174,000 emoticons, 340,000 context\\ 

% MCDSCS \cite{mcdscs}& \cmark & \xmark  & & & & 5,500 dialogues, 14,400 emoticons \\ 

% SER30K \cite{SER30K}& \cmark & NA  & & & & 30,739 emoticons with sentiments \\ 

% MOD \cite{MOD} & \cmark & \xmark & & & & 45,000 dialogues, 307 emoticons \\ \hline

% emoticonU~(Ours) & \cmark &  \cmark & \cmark & 22,629 users & 105,933 emoticons &  8,769,401 messages \todo{log}  \\
% % SRS \cite{learning-to-respond-with-emoticons-2020} & Restricted & Same as PESRS \\
% \bottomrule
% \end{tabular}
% \end{table*}

To address these limitations, we introduce a new emoticon dataset, the first large-scale dataset that includes both user information and emoticon-based conversations. Our contributions can be summarized as follows:
% Specifically, our contributions can be summarized as follows:
\begin{itemize}
% size
% more information, multi-language? multi-domain
% experiments
    \item We present a new emoticon dataset, the largest emoticon dataset to date, containing \textbf{22k users}, \textbf{370.2k} emoticons and \textbf{8.3M} conversation messages of texts and emoticons.
    % Spanning 10 domains, emoticonU captures rich temporal, multilingual, and cross-domain behaviors not seen in prior datasets, with users averaging 19.97 emoticons each. This dataset provides a unique opportunity for studying personalized emoticon usage and user behavior in multimodal conversations.
    \item
    % emoticonU is multi-domain and contains rich information. Spanning 10 domains, emoticonU captures rich temporal, multilingual, and cross-domain behaviors not seen in prior datasets.
    Our dataset is a \textbf{multi-domain dataset} that includes rich and diverse information. Covering 10 domains, it captures temporal, multilingual, and cross-domain behaviors that are not present in previous datasets.
    \item Extensive quantitative and qualitative experiments demonstrate the dataset’s practical applications on user behavior analysis and modeling, personalized emoticon recommendation. It also holds potential for further research in areas such as personalized retrieval and conversational studies.
\end{itemize}

% filter already and then upload a small portion.
% 1. regex is using "i am", "my name", write this down first, llm parsed, group by group. privacy is impt! name, address, tel number.
% 2. limitations: 