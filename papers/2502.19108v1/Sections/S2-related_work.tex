\section{Related Work}
\label{sec:related_work}
% We outline related work on the data collection platform, emoticon datasets, emoticon retrieval methods and user behavior modeling.
% Dataset, usage of dataset.
We outline the current emoticon datasets and their applications. The terms sticker, emoticon, memes, gif mean approximately the same, dynamic or static multimedia that can be shared to express emotions. Herein, we use the word emoticon.

\subsection{Emoticon Datasets}
% Because xxx emoticon dataset is not much. Especially opensource ones, howeber recent emoticon reserarch ahs more focus. Thus, there is only 8 related dataset being used, but there are only xxx that are available. And we summarize in Table 1.
Due to the challenges in preparing emoticon conversation-based datasets, their availability remains limited, particularly for publicly accessible datasets. However, recent research on emoticons has gained more attention. Currently, there are only nine related datasets, with only \textbf{three} being publicly available. A summary of these datasets is provided in Table \ref{tab:dataset_comparison}.

\subsubsection{Availability}
Many of these datasets are closed-source. Specifically, six datasets are not publicly accessible \cite{stickertag, stickerint, stickerclip, CSMSA, learning-to-respond-2021, chee2024persrv}, meaning they are either closed-source or require additional access that is often unattainable. This lack of open access presents a significant challenge for emoticon-based research.

\subsubsection{Multi Topic Information}
Furthermore, none of the aforementioned datasets contain cross-domain information, as the conversations or dialogues are not identified with specific topics. This limitation hinders the exploration of cross-domain emoticon usage, which could provide valuable insights.

\subsubsection{User Information}
Additionally, publicly available datasets are scarce in terms of user data, with 1.36 \cite{MOD} and 2.03 \cite{mcdscs} historical emoticons per user. This lack of data makes it difficult to perform user analysis and personalization-related tasks. Our dataset bridges this gap, providing the highest emoticon history per user, with an impressive \textbf{16.90} emoticons.

\subsubsection{Multilingualism}
Most existing datasets focus on one or two languages, limiting the ability of smaller communities to benefit from advancements in emoticon personalization. In contrast, our dataset is multilingual, allowing a broader range of communities to engage with these advancements.

As shown in Table \ref{tab:dataset_comparison} and our analysis, our dataset is the largest publicly available emoticon-user dataset. It features \textbf{370,222} emoticons, \textbf{cross-domain behavior}, \textbf{rich contextual information}, and \textbf{comprehensive user and temporal} data.

% Teelegram dataset what they have done 
% emoticon dataset what they have done, prior work behaiovur done - mention where they have done, and processing
% from the perspective of our points

% We analyze eight emoticon-related datasets, of which six are not publicly available \cite{emoticontag,emoticonint,emoticonclip,CSMSA,learning-to-respond-2021,learning-to-respond-with-emoticons-2020}. Among the three available datasets, one contains only emoticon-description pairs without search logs \cite{SER30K}, while the remaining two are multi-dialogue datasets with an average of 1.36 and 2.03 emoticons per user, respectively \cite{mcdscs,MOD}. These datasets are too scarce for user modeling and personalization. In addition, no fewer than three papers have directly mentioned the lack of such training dataset \cite{pmg,chee2024persrv,pigeon}.

% Traditional image personalization datasets are also unsuitable for emoticon retrieval, as emoticons primarily express emotions, whereas image personalization focuses on identifying objects within images. This fundamental difference makes them inadequate for emoticon retrieval tasks. As far as we are aware, no existing dataset is suitable for personalized emoticon retrieval.

\subsection{Emoticon Retrieval}
Most previous research emphasizes the importance of data for emoticon retrieval. SRS \cite{learning-to-respond-with-stickers-2020}, PESRS \cite{learning-to-respond-2021} require corresponding utterances, while Lao et al. \cite{practical-sticker} rely on manually labeled emotions, sentiments, and reply keywords. CKES \cite{chen2024deconfounded} annotates each emoticon with a corresponding emotion. During emoticon creation, Hike Messager \cite{laddha2020understanding} tags conversational phrases to emoticons. The reliance on data presents a significant limitation, as emoticons without associated information are excluded from consideration.

Gao et al. \cite{learning-to-respond-with-stickers-2020} use a convolutional emoticon encoder and self-attention dialog encoder for emoticon-utterance representations, followed by a deep interaction network and fusion network to capture dependencies and output the final matching score. The method selects the ground truth emoticon from a pool of emoticon candidates and its successor. Zhang et al. \cite{zhang-etal-2022-selecting} perform this on recommendation tasks. CKES \cite{chen2024deconfounded} introduces a causal graph to explicitly identify and mitigate spurious correlations during training. The PBR \cite{xia2024perceive} paradigm enhances emotion comprehension through knowledge distillation, contrastive learning, and improved hard negative sampling to generate diverse and discriminative emoticon representations for better response matching in dialogues. PEGS \cite{zhang2024stickerconv}, StickerInt \cite{stickerint} generate emoticon information using multimodal models and selects emoticon responses, but does not consider personalization. StickerCLIP \cite{stickerclip} fine-tunes pretrained image encoders but does not consider personalization. In addition, many methods designed to rank emoticons from top-k candidates face a significant drawback in real-world emoticon retrieval scenarios, as they quickly become impractical when applied to larger datasets. Possibly due to the lack of dataset, most methods do not consider personalization.

\subsection{Personalized Emoticon Recommendation}
% So far no user behavior modeling for emoticon since there is scarce data. As mentioned above, personalization task is lesser which stifends research.
The absence of personalized emoticon recommendation can likely be attributed to the scarcity of emoticon-user datasets. PESRS \cite{learning-to-respond-2021} improves upon earlier work \cite{learning-to-respond-with-stickers-2020} by incorporating user preferences, but the dataset it relies on is not publicly accessible. This limited availability indirectly hampers the development of personalized emoticon recommendations and may impede further advancements in the field.
% The current personalized emoticon recommendation is limited by the community's dataset and related work is scarce. 


% \subsection{User Behavior Modeling}

% \todo{(merge the SRS wit source}
% \subsection{Telegram as Data Source}
% Most messaging applications impose strict limitations on data collection, making large-scale analysis of user behavior challenging. However, Telegram, a widely used messaging platform, provides unique opportunities for data scraping due to its public channels and group chats. These features allow researchers to study user interactions at scale without direct access to private conversations.

% Prior work has recognized Telegram as an underexplored platform for social science and user behavior research \cite{data-collection-opportunies}. Studies such as \cite{a-topic-wise-exploration} investigate the thematic distribution of discussions, while others \cite{pushshift, unearthing, tgdataset, analyzing-predicting-news, analysis-of-telegram-an-instant-messaging-service} have crawled Telegram channels and made datasets publicly available for further research. Gao et al. \cite{learning-to-respond-with-emoticons-2020} collected Telegram emoticon data from group chats, but access to their dataset remains restricted, limiting further research in personalized emoticon retrieval.

% \subsubsection{Crawling Tool}

