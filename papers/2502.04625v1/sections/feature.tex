\begin{table*}[!t]
  \centering
    % \setlength{\extrarowheight}{0pt} % Adjust the spacing as needed
    \scalebox{0.95}{
    \begin{tabular}{@{}c@{}cl}
    \toprule
    \multicolumn{1}{r}{} & \textbf{Name} & \multicolumn{1}{c}{\textbf{Value}} \\
    \midrule
    \multicolumn{1}{c}{\multirow{4}[6]{*}{\shortstack{Manner \\ Feature}}} & sonority & $5$: vowel, $4$: glide, $3$: liquid, $2$: nasal, $1$: obstruent \\
    \cmidrule{2-3}      & \multirow{2}[2]{*}{continuant} & $1$: fricatives, liquids, glides, laterals \\
          &       & $-1$: stops, affricates, nasals \\
    \cmidrule{2-3}      & {delayed release$^{\text{1}}$\tnote{1}} & $1$: fricatives, affricates, $-1$: stops \\
    \midrule
    \multicolumn{1}{c}{\multirow{11}[18]{*}{\shortstack{Place \\ Feature}}} & labial &  $1$: articulated with the lips \\
    \cmidrule{2-3}      & labiodental$^{\text{2}}$\tnote{2} & $1$: articulated by touching the lower lip to the upper teeth \\
    \cmidrule{2-3}      & coronal & $1$: articulated with the tongue blade and/or tip \\
    \cmidrule{2-3}      & anterior$^{\text{3}}$\tnote{3} & $1$: (front) dental, alveolar, $-1$: (post) palato-alveolar, retroflex\\
    \cmidrule{2-3}      & \multirow{2}[2]{*}{distributed$^{\text{3}}$\tnote{3}} & 1: (blade, laminal) dental, palato-alveolar \\
          &       & $-1$: (tip, apical) alveolar, retroflex \\
    \cmidrule{2-3}      & \multirow{2}[2]{*}{lateral} & $1$: distinguishes [l] from other coronal liquids and [\textipa{\textbeltl}, l\textipa{\textyogh}]  \\
          &       & from other coronal fricatives. \\
    \cmidrule{2-3}      & dorsal & $1$: articulated with the tongue body \\
    \cmidrule{2-3}      & high$^{\text{4}}$\tnote{4} & $3$: velar, $2$: uvular, $1$: pharyngeal \\
    \cmidrule{2-3}      & front$^{\text{4}}$\tnote{4} & $3$: fronted velar, $2$: central velar, $1$: back velar, uvular, pharyngeal \\
    \midrule
    \multicolumn{1}{c}{\multirow{2}[4]{*}{\shortstack{Laryngeal \\ Feature}}} & voice & $1$: voiced, $-1$: voiceless \\
    \cmidrule{2-3}      & spread glottis & $1$: [h], breathy vowels, and aspirated consonants. \\
    \bottomrule
  \end{tabular}
  }
  \caption{\label{table:feature-set}Our feature set.
    In the `Value' column, the number before the colon is the possible value of the feature, while the right of the colon is the condition for taking this value. 
    If there is only value `$1$', it means that the feature is `$-1$' under all other circumstances. 
    $^1$Only meaningful for obstruents (i.e., when sonority is $1$).
    $^2$Only meaningful when [+labial].
    $^3$Only meaningful when [+coronal].
    $^4$Only meaningful when [+dorsal].
    } 
\end{table*}

\section{Representing Phonemes} \label{sec:feature-set}
Representing phonemes in a formal way plays an essential role in computational reconstruction.
Following Generative Phonology \cite{chomsky-1968}, we use distinctive features to represent phonemes. 
%We tailor the feature set proposed by 
\citet{hayes2011} proposes a feature set for all human languages.
Any specific language only uses a subset of it to mark phonemic contrasts. 
To compactly represent Chinese and its modern varieties, we propose a Chinese-specific set. 
Reducing the total number of distinctive features can also boost the efficiency in solving the corresponding optimisation problem. %\todo{coherence}

\subsection{Distinctive Features} \label{intro-gp}

Distinctive features provide a systematic way to identify and represent phonemes. 
Each phoneme is represented as and collectively defined by a bundle of binary features \citep[p.71]{hayes2011}. 
The negative ($-$) and the positive ($+$) annotations are used to indicate the absence or presence of a feature. 
Below is an example:
%For example, the name \textit{Pom} (= /p\textipa{A}m/) can be represented as follows:
%\footnote{This example comes from \citet[pp.72]{hayes2011}.} 
\begin{equation*} \label{pom}
\small
\text {Pom:}  \! =  \! \left[ \! \begin{array}{l}
- \text { syllabic} \\
- \text { sonorant} \\
+ \text { stop} \\
- \text { nasal} \\
+ \text { labial} \\
- \text { voice}
\end{array}  \! \! \right]  \! 
\left[  \! \begin{array}{l}
+ \text { syllabic} \\
+ \text { sonorant} \\
- \text { stop} \\
- \text { nasal} \\
+ \text { low} \\
+ \text { back} \\
- \text { round}
\end{array}  \! \! \right]  \! 
\left[ \! \begin{array}{l}
- \text { syllabic} \\
+ \text { sonorant} \\
+ \text { stop} \\
+ \text { nasal} \\
+ \text { labial} \\
+ \text { voice}
\end{array} \! \! \right] \end{equation*}

It is straightforward to formalise the bundle of features as a vector, which can be used to measure the distance between phonemes.

\subsection{Our Feature Set} \label{our-feature-set}

We propose the following modification of \citet{hayes2011} to obtain a feature set for Chinese.

\paragraph{Remove some features} 
Two types of features are not considered: (1) features that can be represented by other features, and (2) `tap' and `trill'\footnote{Taps, flaps and trills are uncommon in modern Chinese dialects \citep{zhuxiaonong-liquid}, and they do not appear in our dataset. 
No scholars have used taps, flaps and trills to reconstruct MC.
Existing research shows their close connection with the affix \'er 儿~\citep{trill-in-HB-2019}, but there is still no consensus on the timing and process of their formation.}. 
% , which are highly unlikely used by ancient nor modern Chinese
%, e.g. [+strident] phonemes are coronal fricatives and affricates, 

\paragraph{Merge some features}
The features in Generative Phonology are binary. 
By combining comparable and orderable features into multi-valued ones, we can reduce the number of features. 
For example, \citet{hayes2011} use 4 features (syllabic, consonantal, approximant, sonorant) to describe the sonority hierarchy, while we combine them into one feature, `sonority', with 5 graduable values. 
%\todo{The integration of these features is justified because they are comparable?}

Our feature set is summarised in Table \ref{table:feature-set}. 
We have 14 features in total, reduced from 25 in \citet{hayes2011}. 
Accordingly, we use a 14-dimensional vector to represent a phoneme for computation. 

\paragraph{Independent vs dependent features}
In both the \citeauthor{hayes2011}' feature set and ours, some features are meaningful\footnote{The words `meaningful' and `meaningless' used here correspond to the description `not to care' in \citet[p.91]{hayes2011}: `in most languages with plain /p/, the position of the tongue body during the production of this sound is simply whatever is most articulatorily convenient, given the neighboring sounds. $\cdots$ the tongue body does not adopt any particular position during the /p/; $\cdots$ In this sense, the /p/ could be said truly `not to care' about values for dorsal features.'}
only when some other features at higher levels take certain values. 
We refer to features that decide whether other features are meaningful as `I-features' (independent feature), 
  and those determined by I-features as `D-features' (dependent feature).  

\paragraph{Zero value} \citet{hayes2011} uses the digit 0 (zero feature) to represent meaningless D-features. 
Some syllables in Chinese lack initial consonants, and \citet[pp.18--23]{chao1968} suggests to call them `zero initials'. 
Accordingly, we use digit 0 to represent zero initials.
0-valued I-features occur when and only when the corresponding character is initialless, 
  while 0-valued D-variables occur when and only when they are meaningless.



