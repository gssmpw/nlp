\section{Conclusion} \label{conclusion}

%In this paper, We formalize the task of phonetically reconstruct the consonant system of Middle Chinese into a MIQP problem. Results have shown that our innovate method is promising. 
We propose a novel, MIP-based method for phonetic reconstruction for Middle Chinese,
and validate its effectiveness on a wide range of synthesis and real data. 
Similar to the automation of the historical comparative approach to Indo-European languages, our study confirms the usefulness of computation in linguistic inquiry.  
The optimisation-based architecture is flexible---different information can be integrated as either an element in the objective function,  constraints, or both.
It is also applicable to the reconstruction problem of other languages.
We leave both for future work.
%We believe that modern optimisation solvers are a powerful tool for computer-assisted historical linguistics.
%In the future, we would consider more heterogenous information for different types of historical documents, including sino-xenic scripts.
%We believe that, with the increase in materials, our model can derive results that are more and more comprehensive and linguistically meaningful. 
%One direction for future exploration is to incorporate  more heterogeneous information, such as sino-xenic pronunciations. 
%Another relatively straightforward extension is the reconstruction of medials and main vowels in MC.

% The reconstruction of Old Chinese, the language used before 300 CE, is much more difficult, due to the limitation of historical documents.
% The \fq~method was not invented yet, and there are no rhyme dictionaries that comprehensively recorded the phonology system. 
% There are three primary sources to consider \citep{huang2014handbook}: 
% (1) the rhymes in \textit{the Book of Odes}, whose poems are believed to date from the eighth through fifth centuries BCE, 
% (2) the phonetic elements of Chinese characters created during the early and mid Zhou period,  and
% (3) reconstructed Middle Chinese, which is presumed to be a direct descendent of Old Chinese.
% The optimisation architecture may be particular suitable to integrate the above heterogeneous information fragments.

% Characterising sound patterns of diachronic change is of fundamental interest in historical linguistics.
% Our work only consider one (synchronic) language system rather than a series of diachronically related languages.
% However, precise models for one language system enables joint inference of multiple systems.
% We leave the above two topics for future exploration.

\section*{Acknowledgment}
We would like to express our sincere gratitude to the reviewers for their valuable comments, which greatly broadened our perspective and significantly improved the quality of our work. We would also like to thank Kechun Li for her suggestions. 
