\section{Related Work} \label{related-work}
%Our work mainly extends two fields: computational protolanguage reconstruction and Middle Chinese phonology.

\subsection{Computational Reconstruction} \label{related-work-reconstruct}
% Protolanguage reconstruction relies on cognates, and the task cognates detection has been extensively studied in recent years (\citealp{tsvetkov-2015,ciobanu-dinu-2015-automatic}). However, few studies took a further step to automate the process of protolanguage reconstruction.

 \citeauthor{bouchard-2007a} (\citeyear{bouchard-2007a, bouchard-2007b,bouchard-2009,bouchard-2013}) proposed a series of influential work about unsupervised proto-word reconstruction, which requires an existing phylogenetic tree to infer the ancient word forms based on probability estimates for all the possible phoneme-level edits on each branch of the tree. The edit model parameters and unknown ancestral forms are jointly learned with an EM algorithm. 

Following this series of work, \citet{he-etal-2023-neural} also used Monte-Carlo EM algorithm but neural networks to parameterize the edit models, in order to express more complex phonological and the nonadjacent changes, achieving a notable reduction in edit distance from the target word forms. However, his highly parameterized edit models were designed for large cognate datasets with few languages, and may not be possible to train them on datasets with more languages but fewer datapoints per language. 

In supervised protolanguage reconstruction, the models are easier to evaluate. \citet{meloni-2021} trained a GRU-based encoder-decoder architecture on cognates from five Romance languages to predict their Latin ancestors, and achieved low error from the ground truth. \citet{kim-etal-2023-transformed} updated \citeauthor{meloni-2021}'s model with the Transformer and achieved better performance.

\citet{list-etal-2022-new} proposed a new framework for supervised reconstruction that combines automated sequence comparison with phonetic alignment analysis, which deals with the losing reflexes problem, and sound correspondence pattern detection, which models phonetic environments of sound change. 

\citet{lu-2024-improved-neural} proposed a multi-model reconstruction system that improves its reconstructions via predicting the reflexes given a protoform. Their system consists of a beam search-enabled sequence-to-sequence reconstruction model and a sequence-to-sequence reflex prediction model that serves as a reranker, surpassing state-of-the-art protoform reconstruction methods on three of four Chinese and Romance datasets.

\subsection{Middle Chinese Phonology} \label{related-work-MCP}
% The study of phonological categories has a long tradition in China. Chen Li firstly proposed a systematic method to obtain the categories of initials by analyzing the F\v anqi\=e spellings. His method has been revised and improved by a number of scholars, including e.g. \citet{lr-1956}.
% The method is still influential to the study of phonological reconstruction even today. 

Phonetic reconstruction of phonological categories was pioneered by \citet{gbh}. 
% Based on the categories from traditional methods, \citet{gbh} systematically compared modern dialect pronunciations of characters in order to associate the phonetic features with categories, and then assign reconstructed pronunciations to individual characters according to their categories. 
Following the methodology of \citet{gbh}, subsequent scholars, including \citet{lfk-1971}, \citet{wangli-1957}, \citet{pulleyblanks}, \citet{Baxter1992}, made modifications to the methodology and proposed their reconstructions of MC.

In recent decades, some scholars have questioned the assumptions, methodology and conclusions of Karlgren's approach.
A critical view is exemplified by \citet{norman-1995}. 
Norman advocated a data-centered approach to Chinese historical phonology, predicated on the collection, analysis, and comparison of spoken-language data. 
His controversial reconstruction of Proto-Min (\citealp{Norman-1973}, \citeyear{norman-1974}) is an example.
% The major materials of previous work are dialects and ancient documentations. 
% The distinction between the traditional approach started by \citet{gbh} and Norman is in essence whether the phonological categories should be prioritized over the dialects. 
% Moved to Discussion part with more details
% Our model deals with these two aspects of information at the same time, balancing between phonological and spoken-language materials.

% In the compilation of Qi\=ey\`un, the compiler L\`u F\v ay\' an 陸法言 referred to some old rhyme books and character dictionaries, and when he found any inconsistencies between these documents and the literary pronunciation system, he simply included them as an entry \citep{oxford-handbook-2015}.
% It is highly possible that the phonological categories in Qi\=ey\`un  are a mixture of different systems, so we adopt the fine-grained strategy of reconstruction, which is difficult to accomplish manually and can be used to study the nature of Qi\=ey\`un in return. Though not directly used, the categorical information is exploited for evaluation.

% Previous phonetic reconstructions of MC are based on phonological categories---deriving categories first and then assigning phonetic values. However, due to the heterogeneous nature
% \footnote{The nature of Qi\=ey\`un has been a controversial problem, even in today. Some scholars, like \citet{wangli-1985}, believe that Qi\=ey\`un is not records of a contemporary spoken language or dialect, but a mixture of ancient, contemporary, and dialect phonologies. Others believe that there exists a contemporary spoken language underlying the Qi\=ey\`un, like \citet{gbh}. However, it is a consensus that the phonology of Qi\=ey\`un contains heterogeneous materials.}
% of Qi\=ey\`un, we directly reconstruct the pronunciations of single characters without referring to their categories. 
