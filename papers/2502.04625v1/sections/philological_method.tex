\subsection{Categorical Analysis of Initials: The Traditional Way} \label{initial-category}
As early as the 19th century in the Qing Dynasty, a scholar named Chen Li proposed a systematic method to obtain the categories of initials by analyzing the F\v anqi\=e spellings. His method has been revised and improved by a number of scholars and applied to the study of initials, like \citet{lr-1956}.

The core of Chen Li's method is that the relationship between F\v anqi\=e spellers is transitive. Characters with the same upper spellers must have the same initial, so we can link upper spellers together with one another, and obtain a series of upper spellers all representing the same initial.

Besides F\v anqi\=e spellers, Chen Li also utilized characters with more than one pronunciation (polyphonic characters), each pronunciation documented in an entry. Therefore, the character and all of its pronunciations redundantly appear in several entries of different places. In some cases, different F\v anqi\=e spellings are used to mark the same pronunciation, and the upper spellers are certainly in the same category. 

For example, characters in Table \ref{Table-chenli-example} represents initial\ch{端}. It appears that\ch{冬都丁當}and\ch{多得德}forms two groups and cannot be linked together. Luckily, utilizing polyphonic characters can solve the problem:\ch{涷}has two pronunciations (see Table \ref{Table:yidu}), \ch{德紅切}and\ch{都貢切}. Turning to the location of the latter pronunciation, the notation given there for the previous pronunciation is\ch{多貢切}, and it is certain that these two F\v anqi\=e, \ch{都貢切}and\ch{多貢切}, represent the same pronunciation. Therefore, the two groups can be linked together.

\begin{table}[h!]
  \centering
  \caption{Two Entries of ``\ch{涷}''} \label{Table:yidu}
  \begin{threeparttable}
  \begin{tabular*}{\linewidth}{@{}l c c@{}}
  \specialrule{0.08em}{0em}{0.1em}
  & \textbf{F\v anqi\=e} & \textbf{The Other Pronunciation} \\
  \hline
  \ch{涷} & \ch{德$(X_1)$紅切} & \ch{又\tnote{1}都$(X_{o_1})$貢切}\\
  \ch{涷} & \ch{多$(X_2)$貢切} & \ch{又音東$(X_{o_2})$}\\
\end{tabular*}
\begin{tablenotes}
\item[1] \small ``\ch{又}'' means `another pronunciation'.
\end{tablenotes}
\end{threeparttable}
\end{table}

\begin{table}[!ht]
    \centering
    \caption{Characters Representing Initial\ch{端} and F\v anqi\=e} \label{Table-chenli-example}
    \begin{tabular}{c c c c}
    \specialrule{0.08em}{0em}{0.1em}
    \textbf{Character} & \textbf{F\v anqi\=e} & \textbf{Character} & \textbf{F\v anqi\=e} \\
    \hline \text { \ch{冬} } & \text { \ch{都宗切} } & \text { \ch{當} } & \text { \ch{都郎切} }\\
    \text { \ch{都} } & \text { \ch{當孤切} } & \text { \ch{多} } & \text { \ch{得何切} }\\
    \text { \ch{丁} } & \text { \ch{當經切} } & \text { \ch{德} } & \text { \ch{多則切} }\\
    \text { \ch{得} } & \text { \ch{多則切} } \\
    \end{tabular}
\end{table}

However, upper spellers sharing the same initial with characters in question is only the general situation, and there can be counterexamples. For example, in `\ch{鄙,方美切}', ``\ch{鄙}'' and ``\ch{方}'' are actually in different categories, which is a consensus of researchers, when taking other materials like rhyme tables and modern dialects into account. Such inconsistencies exist probably because the F\v anqi\=e spellings were not devised by one individual but rather collected from various preexisting phonological works, so they may vary in era or dialects. Thus, only using the above methods cannot lead to satisfactory results, since it will put different initials together. Fortunately, Chen Li proposed a solution.

There is a one-to-one corresponding relation between different homophonic groups and different F\v anqi\=e notations. That is to say, two different F\v anqi\=e notations in the same rhyme must indicate difference in the pronunciations of the characters in question. Thus, when the lower spellers are in the same category or even the same character, we can deduce that the upper spellings of the two F\v anqi\=e must be in different initial categories. For example, both ``\ch{東,德紅切}'' and ``\ch{同,徒紅切}'' are in the ``\ch{東}'' rhyme, we can infer that ``\ch{德}'' and ``\ch{徒}'' must have different initials.

Following the above approaches, scholars have basically reached a consensus on the initial categories of MC. Most of them believe that MC has 35 to 38 initials, with disagreement only exists on some categories' merging or splitting.

