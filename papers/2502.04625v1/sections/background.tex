\section{Linguistic and Philological Basis} \label{ling-philo-basis}
%Our approach to reconstruction is based on the language-specific properties of Chinese syllable (\S \ref{syllable-stru}). Based on the syllable structure, F\v anqi\=e反切 notations in rhyme dictionaries (\S \ref{rhyme-dict}) provide valuable homophonic information of MC. Modern dialects are another information source, which are briefly introduced in \S\ref{intro-dialect}.

\subsection{Syllable Structure} \label{sec:syllable-structure}
%\footnote{This subsection is based on \citet{shen_2020}.} \label{syllable-stru}
%Chinese is a monosyllabic language, 

Ancient documents overwhelmingly indicate that Chinese was, from the beginning of its recorded history, a monosyllabic language,
% Chinese was always a monosyllabic language
 in which morphemes are by and large represented by single syllables \citep{norman1988,shen_2020}.
Moreover, the sound pattern of the syllabic structure remained unchanged from Middle Chinese to modern Mandarin.
The syllabic structure is composed of an initial segmental consonant (I), 
a medial (aka on-glide, denoted as M hereafter), a main vowel (V), a coda (or an off-glide), denoted by C hereafter, and a suprasegmental tone (T).
The terms `rime' and `final' are also frequently used: 
Rime is the combination of the main vowel and the coda, while final is a combination of the medial, the main vowel and the coda.
There are no consonant clusters, i.e. more than one consecutive consonants.
% i.e. more than one consecutive consonants in a row. 
%It consists of the initial (I), medial (M), main vowel (V), coda (C), and the suprasegmental tone (T) as the basic structural slots. 
See Figure \ref{fig:syllabic-structure} for the hierarchical organisation of the above elements. 
%Among these structural slots, only the main vowel is obligatory, and I, M and C can be zero. 
Below we list three examples:
\begin{itemize} \setlength\itemsep{0.01em}
  \item \ch{巔}/\text{[tian]}: I=t, M=i, V=a, C=n, T=55 
  \item \ch{眼}/\text{[ian]}: I=$\emptyset$, M=i, V=a, C=n, T=214
  \item \ch{暗}/\text{[an]}: I=$\emptyset$, M=$\emptyset$, V=a, C=n, T=51
\end{itemize}

%\begin{figure*}[h]
%        \centering
%        \subfigure[Syllable structure of \ch{巔} \text{[tian]}, with a high level tone (55).]{
%		\begin{minipage}[b]{0.3\textwidth}
%			\includegraphics[width=\textwidth]{figure/new-subfig1.png} 
%		\end{minipage}
%		\label{fig:subfig1}
%	}
%	\subfigure[Syllable structure of \ch{眼} \text{[ian]}, with a low dipping tone (214). The syllable has a zero initial.]{
%		\begin{minipage}[b]{0.3\textwidth}
%			\includegraphics[width=\textwidth]{figure/new-subfig2.png} 
%		\end{minipage}
%		\label{fig:subfig2}
%	}
%        \subfigure[Syllable structure of \ch{暗} \text{[an]}, with a high falling tone (51).  The syllable has a zero initial and a zero medial.]{
%            \begin{minipage}[b]{0.3\textwidth}
%            \includegraphics[width=\textwidth]{figure/new-subfig3.png}
%            \end{minipage}
%        \label{fig:subfig3}
%        }
%	\caption{Examples of Chinese syllable structure}
%	\label{fig:3exp-stru}
%\end{figure*}

\begin{figure}[hbtp]
  \centering
  \tikz{
    \tikzset{level distance=42pt}
    \Tree
      [.Syllable
        \node(l){\shortstack{Initial\\\small{complex}\\\small{consonant}}};
        [.Final
          {\shortstack{Medial\\\small{vowel/glide}}}
          [.Rime
            {\shortstack{Main Vowel\\\small{vowel}}}
            \node(r){\shortstack{Coda\\\small{simple}\\\small{consonant}}};
          ] ] ] 
    \draw [decorate,decoration={brace,amplitude=5pt,mirror,raise=4ex}] (-2.5,-3.9) -- (5,-3.9) node[midway,yshift=-3em]{Suprasegmental Tone};
  }

  \caption{The syllabic structure of MC and Mandarin.}
  \label{fig:syllabic-structure}
\end{figure}


Consonants can only appear as I or C. %\footnote{Some scholars reconstructed some categories of medial as consonants. For example, \citet{gbh} and \citet{wangli-1957} reconstructed the medial of Rank-III as a glide. We do not consider medial as consonants here, since there still exists controversy.} 
Consonantal codas are rather simple and have been relatively clearly recorded in rhyme dictionaries. 
The reconstruction of the associated phonetic values is also clear: 6 categories in total, including nasals [m, n, \textipa{\ng}] and stops [p, t, k]). 
This paper aims to complete the reconstruction of the entire consonant system by systematically studying initials.

\subsection{\fq~Spelling}
\fq~is a traditional method to indicate the pronunciation of a character in question. 
%Chinese characters are logographic, so the phonetic values of characters can only be shown by comparison with other characters. 
In the \fq~spelling, two characters are selected as two spellers to represent the pronunciation of the character (denoted as $X$) in question: 
the first character ($X_u$) is called the upper speller and shares the same I with $X$; 
the second character ($X_l$) is called the lower speller and shares the same M, V, E, and T with $X$. 
Take Figure \ref{fig:exp-fanqie} for example. 
To partially record the pronunciation of 烘, 戶 is employed as the upper speller, while 公 is used as the lower speller.
\begin{figure}[hbtp]
    \centering
    \includegraphics[width=0.6\linewidth]{figure/fig-fanqie.png}
    \caption{An example of the \fq~spelling. }
    \label{fig:exp-fanqie}
\end{figure}

\zhiyin~直音 is another method to partially annotate pronunciation. 
It uses a homophonic character to annotate the character in question. 
Both \fq~and \zhiyin~were frequently used. 

%\subsection{Rhyme Dictionary} \label{rhyme-dict}
A rhyme dictionary is a type of ancient Chinese dictionary that collates characters by tone and rime. 
In rhyme dictionaries, there are three types of important phonological information: rhyme categories, \fq~spellings, and \zhiyin~notations. 
The \qy~is a renowned rhyme dictionary that encapsulates the phonology of MC. 
Chinese philologists have been working on it to derive phonological analysis for centuries.


\subsection{Modern Chinese Dialects} %\footnote{This subsection is based on \citet{norman1988}. See \S8--9 for details.}} 
\label{intro-dialect}
Modern Chinese dialects are classified into seven groups in three geographic zones: 
Mandarin (northern zone), Wu, Min, Xiang (central zone), Gan, Hakka and Yue (southern zone) \citep{norman1988,dialectology}.
% They are all descendants of MC,
Most scholars believe that they are all descendants of MC, 
and therefore provide valuable information to reconstruct phonetic values. 
% For example, the Southern dialects rather than the northern and central dialects retain all final consonants in Middle Chinese.\todo{now we have more space to say more about dialects.}

\begin{comment}
    
\end{comment}

