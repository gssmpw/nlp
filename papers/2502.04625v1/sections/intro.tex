\section{Introduction}
Phonological reconstruction is one main concern in historical linguistics. 
There are two fundamental goals: reconstructing phonological categories and reconstructing phonetic values of these categories or individual phonemes. 
The classic linguistic and philological approach applies a comparative strategy to solve both problems by connecting cognates in different languages.
Previous research in computational linguistics demonstrates the possibility to automate the comparative approach to some extent.
See e.g. \citeauthor{bouchard-2007a} (\citeyear{bouchard-2007a}, \citeyear{bouchard-2009}, \citeyear{bouchard-2013}), \citet{list-etal-2022-new}, and \citet{he-etal-2023-neural}, among others. 


The comparative approach developed out of attempts to reconstruct Proto-Indo-European, the common ancestor of the Indo-European language family.
However, the comparative method itself is not well equipped to handle the special challenges of reconstructing Chinese. 
On the one hand, a key step in the comparative method is to identify as many cognates as possible, which is relatively straightforward for Chinese but can be extremely challenging in other languages.
On the other hand, documentary materials predominantly use Chinese characters to annotate other characters (such as \fq~反切), a tradition that has continued for thousands of years. 
For example, rhyme dictionaries such as \qy~切韻 extensively use this unique annotation method and systematically represent the phonological system of Chinese during a specific period. 
Such precious materials are relatively rare in other languages. 
This information is invaluable for the reconstruction of proto-languages, but the comparative method itself cannot adequately handle it. 
In fact, throughout Chinese history, numerous works similar to the \qy~have existed in different periods, each reflecting the phonological system of its time.
The purpose of this paper is to address the question of how to systematically utilize these phonetic materials.
    
In the practice of phonological reconstruction for ancient Chinese, linguists have been overwhelmingly exploring alternative information to spelling and hence alternative methods to the comparative one \citep[pp.1--2]{huang2014handbook}.
Their work heavily relies on philological documents, especially rhyme dictionaries, which have a unique way to record homophonic information, i.e. \fq. 
A basic consensus on the phonological categories of Middle Chinese (MC)\footnote{There are three basic periods: Old Chinese, Middle Chinese and Old Mandarin \citep{wangli-1957}.} has been reached---there were 35--38 initials in MC, with minor disagreement only on some categories' merging or splitting \citep{gbh,lr-1956,wangli-1957}.
Ancient rhyme dictionaries, however, do not provide phonetic information, and phonetic reconstruction is still extremely challenging.
The relevant research is limited and there is a lot of disagreement among scholars. 
Figure \ref{fig:confu-matrix} shows the percentage of initials with which scholars disagree. 
There is significant inconsistency between any two scholars, let alone a consensus among all of them.

\begin{figure}[t]
    \centering
    \includegraphics[width=0.99\linewidth]{figure/confusion.png}
    \caption{Disagreement among scholars on phonetic reconstruction. BK: \citet{gbh}, FL: \citet{lfk-1971}, 
    LW: \citet{wangli-1957}, RL: \citet{lr-1956}, RS: \citet{shaorongfen}, EP: \citet{pulleyblanks}, TD: \citet{dth-2004}, WP: \citet{panwuyun-2000}, XC: \citet{chenxinxiong}, WB: \citet{Baxter1992}.\footnote{The original data comes from \url{https://zh.wikipedia.org/wiki/\%E4\%B8\%AD\%E5\%8F\%A4\%E9\%9F\%B3}.} The number in each cell represents the percentage of initials on which the corresponding two scholars disagree. 
}
    \label{fig:confu-matrix}
\end{figure}

This paper is concerned with developing a computational model for phonetic reconstruction of the consonant system of MC.
We propose to cast phonetic reconstruction as a Mixed Integer Optimisation problem (\S\ref{sec:model}). 
A particular goal is to conveniently integrate heterogeneous information. 
We consider two major information sources: 
(1) philological documents and 
(2) modern Chinese dialects\footnote{The modern Chinese dialects are more like a family of languages \citep{huang2014handbook}, and many of them are not mutually intelligible. 
This paper uses the term `dialect' instead of `variety', because we focus on their common ancestor MC.}, the descendants of MC.
Following Generative Phonology \cite{chomsky-1968},
we introduce a novel compact set of Chinese-specific distinctive features to represent consonants (\S\ref{sec:feature-set}).
Based on the feature-oriented precise phonetic representation, we formalise 
the optimisation goal as minimising the overall distance between possible homophonic characters and the overall distance between MC and modern dialects.
Measuring the distance is a key element to the success of the new architecture.
To this end, we design a new mathematically sound distance/metric function to suit our feature representation.

Evaluating the {goodness} of reconstruction result is uniquely challenging because of the lack of ground-truth.
Instead, we evaluate the reconstruction method.
We consider two types of experiments: experiments on synthetic data (\S\ref{sec:synthesis}), where the ground-truth is known, and experiments with held-out data (\S\ref{sec:experiment}), where partial information transformed from the ground-truth is known.
To create representative synthetic data, we start from a pre-defined consonant system, derive homophonic information that matches \fq, and derive varieties by introducing stochastic change as well as random noise. 
We consider three types of consonant systems: 1) purely artificial systems that randomly select elements, e.g. from an IPA chart, 
2) natural systems of modern languages, including English, German and Mandarin, and 3) the reconstructed system of Latin.
Numerical evaluation demonstrates the effectiveness and robustness of the new method.
It is able to successfully reconstruct most consonants when natural and reconstructed consonant systems are considered. 
 
For the experiments with real data, we consider a wide range of representative Chinese characters with relevant information from \gy~and 20 modern dialects.
Given the absence of ground truth in phonetic reconstruction, 
to validate the effectiveness of the reconstruction method, we employ the strategy to hold out some \fq~information.
In particular, we apply our method to 70\% \fq~annotations and compare the automatically reconstructed result with the other 30\%.
The reconstructed phonemes predict around 68\% \fq.
Considering that \fq~annotations themselves are not fully consistent,
the result is quite promising and the method has a potential use to detect inconsistent \fq~annotations.

Based on the entire real data set, we provide a new phonetic reconstruction for Middle Chinese.
We present both numerical and linguistic comparison to previous philologist work (\S\ref{sec:main-result}).
Our phonetic reconstruction aligns to the well-studied phonological category reconstruction to a great extent---it obtains an Adjusted Mutual Information \citep{AMI-2010} score of over 0.8. 
A linguistic analysis of the reconstruction result suggests some future research venues. 
