\section{Validation Experiments on Real Data} 
\label{sec:experiment}
%\todo{remove}With our model validated, we continue with experiments on real data. We validate our results on real data in two ways: held-out data and AMI.

\subsection{Collecting Real Data} \label{sec:dataset}

We examine our model with the spelling information in \qy~and the phonetic information of 20 dialects in \citet{zihui}.
Polyphonic characters (characters with multiple pronunciations) are common in both MC and dialects. 
We treat different pronunciations of the same character as different entries and correlate different vectors to them. %since they correspond to different meanings in most cases\footnote{Strictly speaking, they can be viewed as different words recorded with the same character.}.
The final dataset consists of 1960 different characters and 2661 entries in total\footnote{We will release the dataset for research.}.
%Below is the processing procedure to obtain this dataset.

%The Qi\=ey\`un dataset comprises 25333 entries, each containing phonological categories. The dialect dataset contains 2961 frequently used characters and the IPA transcriptions of their pronunciations in 20 dialects. Then we align the two datasets, associating each character with its phonological category in MC and its pronunciation across 20 modern dialects.\footnote{In essence, we hope the corresponding MC and dialect entries to represent the same `word' (with the same meaning). However, it is common that a character has multiple pronunciations in MC or a dialect. 
%In this case, the correspondence is not one-to-one any more. To deal with the problem, we train a Random Forest Classifier for each dialect to capture sound change patterns and identify the corresponding entry pairs.} We benchmark the task of Middle Chinese reconstruction, and enable data-driven research. 
\paragraph{The \qy~information}
Only fragments of the original \qy~survived, and most commonly used documents are its revisions. 
The most accurate revision is \gy~廣韻. 
Though published in the Song Dynasty, \gy~is commonly believed to record the \qy~system and reflect the status of MC. 
\gy~was heavily used in traditional philological research, including \citet{gbh} and \citet{wangli-1957}.
We collect and integrate information from two electronic versions of \gy, separately provided by Peking University and  Beijing Normal University.
% TODO: we should add kaom back. \footnote{The version provided by Beijing Normal University is available at \url{www.kaom.net}.}. 
%phonological categories of characters from the version provided by the Center of Linguistic Study at Peking University, incorporate the meanings from the dataset offered by Beijing Normal University\footnote{Available on \url{www.kaom.net}}. 

\paragraph{The dialect information}
%Our phonetic reconstruction heavily relies on dialects of modern Chinese. Many features of MC vanished in Mandarin, but are still present in some dialects, such as voiced stops are well preserved in Wu.
%The dialect materials we used mainly come from \citet{zihui}. As discussed in \S \ref{intro-dialect}, there are seven major Chinese dialect groups: Mandarin, Wu, Min, Xiang, Gan, Hakka and Yue. 
\citet{zihui} is a workbook for fieldwork on Chinese dialects.
There is information for 20 modern Chinese dialects: Beijing, Jinan, Xi'an, Taiyuan, Wuhan, Chengdu, Hefei, Yangzhou (Mandarin), Suzhou, Wenzhou (Wu), Changsha, Shuangfeng (Xiang), Nanchang (Gan), Meixian (Hakka), Guangzhou, Yangjiang (Yue), Xiamen, Chaozhou, Fuzhou, and Jianou (Min). 
For each of these dialects, it documents both the phonological system and the phonetic values of representative characters.

\paragraph{Selecting representative characters} % \label{final-dataset}
The original \qy~dataset has 25333 entries, but a large proportion of them are rarely used. 
In contrast, \citet{zihui} contains less than 3000 frequently used characters.
We denote the characters included in \gy~and \citet{zihui} as $S_{\text{gy}}$ and $S_{\text{zh}}$, respectively, and denote all characters used as \fq~spellers of characters in \gy~as $S_{\text{fq}}$ ($|S_{\text{fq}}|=$1462).
Instead of using all available characters, we aim to select a set of representative characters that comprehensively reflect the entire phonological systems. Our selection process involves the following steps:
\begin{enumerate}
    \item[1.] Subtract a smaller set $S_{*}$ from $S_{\text{gy}}$ for subsequent selection. Since \fq~spellings connect different characters and encapsulate valuable relationships between them, they are essential for deriving phonological categories. Therefore, We define $S_{\cap}$ to include common characters as well as \fq~spellers. Specifically, $S_{\cap}=S_{\text{gy}} \cap (S_{\text{fq}} \cup S_{\text{zh}})$ with 3990 entries. 
    \item[2.] For each character in $S_{\cap}$, if both its upper and lower spellers are also in $S_{\cap}$, this character is considered of particular interest. This set is denoted as $S^{*}$, which contains 2461 entries.
    \item[3.] Among $S^{*}$, if several characters share the same \fq, indicating that they are homophones, we select the first character only, which is often the most frequent character. We denote this set as $S^{*}_{1}$.
    \item[4.] Finally, we include \fq~spellers themselves into the selected set to link different entries. Our final representative character set is $S^{*}_{1} \cup (S_{\text{fq}} \cap S_{\cap})$, with 2661 entries.
\end{enumerate}

%We want to decrease the size of the dataset while preserving characters that are interrelated through F\v anqi\=e, as this interrelation is crucial for utilizing categorical information. 

% We also report the AMI between the results derived by hierarchical clustering under different thresholds and the predefined categories in Figure \ref{Fig:AMI-cluster-num}.

% \subsection{Evaluation on held-out \fq~data}
\subsection{Results and Analysis} \label{sec:real-data-evaluate}
 
\paragraph{Matching to held-out \fq~data}
Ideally, each character should share the same initial with its \fq/\zhiyin~speller. 
We randomly take 70\% of \fq/\zhiyin~material for MC reconstruction, and use the remaining 30\% for evaluation. 
We consider a character--speller pair as having matching initials if the $L_2$ distance between a character's reconstructed initial vector and that of its upper speller's is smaller than $10^{-4}$. 
We report the average $L_2$ distance between the reconstructed initials in character--speller pairs and the rate of pairs with matching initials as the \textbf{matching rate}.

The results are shown in Table \ref{table:held-out}. 
A large portion of held-out character--speller pairs have matching reconstruction, affirming the self-consistency of our results. 

\begin{table}[htbp] 
  \centering
    \begin{tabular*}{\linewidth}{@{}cccc@{}}
        \toprule      {$\lambda_{\text{fq}}$} & 0.5 &0.75 &0.95 \\
        \midrule
        %\textbf{Total Num.} &  664   &  665  & 703 \\
        \textbf{Matching Rate} & 66.38\% &  65.33\%    & 67.96\% \\
        \textbf{Avg. $L_2$} &  1.1062  &  1.1046  & 1.2432 \\
        
        \bottomrule
        \end{tabular*}
    \caption{\label{table:held-out}Evaluation with held-out \fq. %`Total Num.' is the total number of \fq/\zhiyin~spellings for evaluation. `M. Rate' means `matching rate'.  
    } 
\end{table}%



\begin{comment}

\begin{figure*}[htbp]
\centering
\subfigure[FW-1]{
\label{Fig:AMI-cluster-num-1}
\includegraphics[width=0.49\linewidth]{figure/hier2_AMI_num_fq1_cl1.jpg}}
\subfigure[FW-3]{
\label{Fig:AMI-cluster-num-2}
\includegraphics[width=0.49\linewidth]{figure/hier2_AMI_num_fq3_cl1.jpg}}
\caption{The number of clusters and small clusters, and the AMI metric under different FW.}
\label{Fig:AMI-cluster-num}
\end{figure*}
    
\end{comment}