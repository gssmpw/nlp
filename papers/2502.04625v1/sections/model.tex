
\newcommand{\fmc}{$F_{\text{MC}}(X)$}
\newcommand{\fmcu}{$F_{\text{MC}}(X_u)$}

\begin{table*}[th] 
  \centering
  \begin{threeparttable}
    % \begin{tabular}{p{4em}p{21.375em}}
    \scalebox{0.95}{
    \begin{tabular}{|ll|}
    %\toprule
    %{\textbf{Notation}} & \textbf{Meaning}\\
    %\midrule
    \hline
    {$X$} & character of which the initial is to be reconstructed\\
    {$X_u$} & upper speller of character $X$, with its initial to be reconstructed \\
    {$S_{fq}$} & set of all character--speller pairs $(X, X_u)$ \\
    {$L$/$l$} & set of modern dialects/a modern dialect \\
    {$F$} & 14-dimensional phonetic feature vector \\
    {$F^j$} & $j$-th dimension of phonetic feature vector $F$ \\
    {$F_l(X)$} & phonetic feature vector that encodes $X$'s initial in dialect $l$ (known) \\
    \fmc &  phonetic feature vector that encodes $X$'s initial in MC (to be solved) \\
    {$S_I$/$S_D$} & set of independent/dependent features \\
    {$\tau$} & function that maps D-feature $j$ to the corresponding I-feature $\tau(j)$\\
    {$d(F_1,F_2)$} & distance between feature vectors $F_1$ and $F_2$\\
    {$f$} & general distance function, e.g. $p$-norm\\
    % {$g$} & distance function defined in \S\ref{sec:dis-f} \\
    {$g_{j,\tau(j)}(F_1, F_2)$} & distance function between $F_1$ and $F_2$ according to D-feature $j$ and I-feature $\tau(j)$\\\hline
    %\bottomrule
    \end{tabular}%
    }
    \caption{\label{table:notation}A summary of mathematical notations used to illustrate our model.}
    \end{threeparttable}
\end{table*}

\section{The Optimisation Model} \label{sec:model}

%\subsection{Mixed Integer Programming} \label{sec:mip}
Mixed Integer Programming (MIP) is an optimization problem in which some but not necessarily all variables are constrained to be integers.
%The integrality constraints in MIP models can be used to decide whether or not some action is taken. 
% MIP models with a quadratic objective but without quadratic constraints are called Mixed Integer Quadratic Programming (MIQP) problems. MIP models with quadratic constraints are called Mixed Integer Quadratically Constrained Programming (MIQCP) problems.
MIP has been widely applied in many NLP tasks, e.g. dependency parsing \citep{riedel-clarke-2006-ILP-parsing}, semantic role labeling \citep{ILP-semantic-role}, coreference resolution \citep{de-belder-moens-2012-coreference}, as well as some more recent applications, e.g. exemplar selection for in-context learning. \citep{tonglet-etal-2023-seer}

We introduce our MIP model for phonetic reconstruction as follows. 
The objective function and constraints are detailed in \S\ref{sec:obj} and \S\ref{sec:restriction} separately. 
An essential component of the objective function is measuring the distance between two phonetic feature vectors, for which we propose a mathematically-sound distance function in \S\ref{sec:dis-f}.
Mathematical notations used in \S\ref{sec:obj}--\S\ref{sec:restriction} are summarised in Table \ref{table:notation}.

\subsection{The Objective Function} \label{sec:obj}

To phonetically reconstruct MC, we consider two information sources: \fq/\zhiyin~and varieties. 
\fq/\zhiyin~reveals homophonic relationships between characters of MC, 
and each descendent dialect partially reflects MC's phonetic structure. 
%A credible reconstruction should satisfy categorical relationships as well as phonetic coherence as much as possible.\todo{这一句不清楚}
Formally, assume we have a set of characters under consideration, denoted as $S$. 
The construction of $S$ is discussed in \S\ref{sec:dataset}.
Each character $X\in S$ has at least one upper \fq~or \zhiyin~speller, denoted as $X_{u}\in S$. 
We collect all character--speller pairs and define the set $S_{\text{fq}}=\{(X, X_{u}): X \in S\}$.
Let $L$ denote the set of modern dialects.
The pronunciation of any character $X\in S$ in any dialect $l\in L$ is known.
Accordingly, the phonetic feature vector of $X$'s initial in $l$, denoted as $F_l(X)$, is known.
The goal is to infer its phonetic feature vector of MC, denoted as \fmc, based on $S_{\text{fq}}$ and all known $F_{l}(X)$ where $l\in L$.

%$\Sigma$ the set of IPA phonemes, and $X_d(d\in D)$ the pronunciation of $x$ in dialect $d$, belonging to the set $\Sigma^*$. Our ultimate goal is to deduce the IPA transcription of $x$ in MC, using $\{X_d: d\in D, X \in S\}$ and $
%is modeled as minimizing the sum ($\mathcal{L}$) of the `distance' between F\v anqi\=e notations ($\mathcal{L}_{fq}$) and the `distance' between MC and each dialect ($\mathcal{L}_{d}$). 

We cast the goal as \textbf{minimising} the overall \textit{distance} between \fmc~and \fmcu,
and minimising the overall \textit{distance} between \fmc~and $F_l(X)$, for all $X\in S$.
Assume $d$ is a mathematically sound distance/metric function and $\lambda_{\text{fq}}\in(0,1)$ is a coefficient then the objective is
\begin{equation}\label{eq:objective}
  \begin{aligned}
    & \lambda_{\text{fq}} \sum_{(X,X_u) \in S_{\text{fq}}} 
       d(F_{\text{MC}}(X), F_{\text{MC}}(X_u))\\
    & \quad+ (1-\lambda_{\text{fq}}) \sum_{l\in L, X \in S} d(F_{\text{MC}}(X), F_l(X))
  \end{aligned}
\end{equation}
Although the speller $X_{u}$ is supposed to share the same initial with $X$ in general, 
we should not model such homophonic relation with constraint $F_{\text{MC}}(X)=F_{\text{MC}}(X_u)$ due to the existence of a considerable number of counterexamples.
Such inconsistency exists probably because the \fq/\zhiyin~spellings were not devised by one individual but rather collected from various preexisting phonological works, and therefore encoded phonological information of a mixture of diachronically connected languages \citep{shen_2020}.
Instead, we relax the identity restriction by employing a more general distance notion.

%In this section, we delve into the task of phonetic reconstruction. We reconstruct the phonetic value of each character's initial with dataset $\{X_d: d\in D, X \in S\}$ and $\{(X, X_{u}): X \in S\}$ (defined in \S\ref{final-dataset}) by minimizing $\mathcal{L}$, the objective function.
% Formally, $\mathcal{L}$ is the objective function to be minimized. The two parts of $\mathcal{L}$, $\mathcal{L}_{fq}$ and $\mathcal{L}_{d}$ are discussed in \S\ref{exp:fq} and \S\ref{exp:dialect} respectively. 

\subsection{The Distance Function} \label{sec:dis-f}

For each $X \in S$, we set 14 \textbf{continuous} variables $F^j (0\leqslant j\leqslant 13)$ to encode the phonetic value of its initial, 
each dimension corresponding to a feature. 
The range of $F^j$ is $[\min\{0,l_j\}, u_j]$, where $l_j$ and $u_j$ are the upper and lower bounds of its corresponding feature in Table \ref{table:feature-set}.
Usually, we can use $p$-norm to measure the distance between two real vectors.
However, in our problem, some features are not independent from each others --- it is meaningless to discuss a D-feature if its corresponding 
I-feature does not take a particular value.
%Similar to \ref{our-feature-set}, we employ the notations `I-variable' and `D-variable' to denote the sets of I-variables and D-variables as $S_I$ and $S_D$ respectively. 
%Now, we define the distance between two initials vectors, which is the foundation of our model. 
To solve this problem, we design a new distance function. 
The mathematical proof of its soundness is provided in Appendix A.

In our solution, the distance wrt.\ I-features is characterised by a general distance function $f$, e.g. $p$-norm.
We only consider the special case of D-features. 
We define $\tau$ as a function that maps each D-feature to its corresponding I-feature, e.g. maps `labiodental' to `labial'. Consider $F_1$ and $F_2$, two feature vectors to be compared. 
% Assume $F_1^j$, $F_2^j$ are two D-features, and $F_1^{\tau(j)}$, $F_2^{\tau(j)}$ are the corresponding I-features.
Assume $j \in S_D$ is a D-feature, and $\tau(j) \in S_I$ is the corresponding I-feature.
\begin{equation}
    \setlen s_j\xlongequal{\text{def}}\sup\limits_{F_1,F_2\in\Omega}f(F_1^{j},F_2^{j})
\end{equation}
Denote the set of all valid feature vectors as $\Omega$, which is a subset of $\mathbb{R}^{14}$. We define a function $g_{j,\tau(j)}:\Omega \mapsto\mathbb{R}$ as follows:
\begin{equation}
\label{distance-func}
    \setlen
  g_{j,\tau(j)}(F_1, F_2)=c\cdot s_j+(1-c)f(F_1^{j},F_2^{j})
\end{equation}
where $c=\min\{f(F_1^{\tau(j)},F_2^{\tau(j)}),1\}$.
The intuition of the design of $g_{j,\tau(j)}$ is as follows. 
It is reasonable to compare $F_1^j$ and $F_2^j$ with a normal distance $f$, when the corresponding I-features $F_1^{\tau(j)}$ and $F_2^{\tau(j)}$ are equal (or very near, since they are continuous).
Otherwise, the distance between $F_1^j$ and $F_2^j$ should correspond to the maximum possible distance they can reach.

Now we are ready to define
\begin{equation} \label{eq:dist-func}
    d(F_1, F_2)=\sum_{k \in S_I}f(F_1^k, F_2^k)+\sum_{j \in S_D}g_{j, \tau(j)}(F_1, F_2)
\end{equation}


\subsection{The Restrictions} \label{sec:restriction}
To obtain a proper phonetic feature vector, we need to ensure the values of its D-features to be consistent with its corresponding I-features. 
When a D-feature is meaningless wrt.\ its I-feature, we force the D-feature's value to be near 0 by some mathematical tricks. 
% For any D-variable $F^j$, denote its corresponding I-feature as $F^{\tau(j)}$.
Three cases are consider separately. 

\paragraph{Case I: delayed release} Unless the corresponding I-feature \textit{sonority} is around $1$, the value of the \textit{delayed release} feature is meaningless and thus should be around $0$. 
Therefore, the following constraint is considered:
\begin{equation} \label{constr-begin}
    \setlen
    F^{j} \leqslant \max(0, \min(F^{\tau(j)}, 2-F^{\tau(j)}))
    \end{equation}

\paragraph{Case II: high or front} Unless the corresponding I-feature \textit{dorsal} is around $1$, the value of a \textit{high} or \textit{front} feature should be around $0$. Ideally, the following constraints are satisfied:
    \begin{align}
        \setlen
        F^j \geqslant 1 \ &(\text{if }F^{\tau(j)} >0.5) \\
        F^j = 0   \ &(\text{if }F^{\tau(j)} \leqslant 0.5)
    \end{align}
    
To linearise, we define auxiliary variables $b$ (binary, indicator of whether $F^{\tau(j)}$ is larger than 0.5), $M$ (large enough), $\epsilon$ (small enough). We have:
\begin{align}\label{eq:highfront}
\setlen
    F^{\tau(j)} & \geqslant 0.5 + \epsilon - M \cdot (1 - b)\\
    F^{\tau(j)} & \leqslant 0.5 + M \cdot b\\
    \max & (0, 1-F^j)=1-b \label{M-trick}
\end{align}


\paragraph{Case III: Other D-features} When the value of the corresponding I-feature is around $1$, the absolute value of a D-feature $F^j$ should be around $1$.
Otherwise the absolute value should be close to $0$. 
We apply the same linearising trick, with only (\ref{M-trick}) changed into:
\begin{equation} \label{constr-end}
    \setlen
    |F^j|=b
\end{equation}

To sum up, our model is to minimise Eq. (\ref{eq:objective}) subject to constraints characterised by Eq. (\ref{constr-begin})--Eq. (\ref{constr-end}).
