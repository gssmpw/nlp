\section*{Appendix} \label{appendix}
Here, we prove that the distance function (\ref{distance-func}) defined in \S\ref{sec:dis-f} is mathematically sound.

It is easy to see:
\begin{enumerate}
  \item $g_{j,k}(F_1, F_2)\geqslant 0$. 
  \item $g_{j,k}(F_1, F_2) = g_{j,k}(F_2, F_1)$. 
\end{enumerate}

Now we consider the triangle inequality.
% Table generated by Excel2LaTeX from sheet 'example'
\begin{table*}[t!]
  \centering
    \begin{tabular}{rccll}
    \toprule
    \textbf{feature} & \textbf{[z]} & \textbf{[f]} & \textbf{dist.} & \multicolumn{1}{c}{\textbf{Note}} \\
    \midrule
    continuant & -1    & 1     & \textbf{2} & \\
    delayed release & 0     & 1     & \textcolor{blue}{\textit{2}}$^{\clubsuit}$ &  $^{\clubsuit}$ \small $ j=\text{delayed release}, \tau(j)=\text{sonority}$. $c=\min\{1, f(F_1^{\tau{(j)}}, F_2^{\tau{(j)}})\}=1, $ \\
    sonority & 2     & 1     & \textbf{1} & \small $s_j=2,\  g_{j,\tau{j}}(F_1, F_2)=c \cdot 2 + (1-c) f(F_1^j, F_2^j)=2$  \\
    voice & 1     & -1    & \textbf{2} &  \\
    spread glottis & -1    & -1    & \textbf{0} &  \\
    labial & 1     & 1     & \textbf{0} & $^{\diamondsuit}$\small $ j=\text{labiodental}, \tau(j)=\text{labial}$. $c=\min\{1, f(F_1^{\tau{(j)}}, F_2^{\tau{(j)}})\}=0, $ \\
    labiodental & -1    & 1     & \textcolor{blue}{\textit{2}}$^{\diamondsuit}$ & \small $ s_j=2, \ g_{j,\tau{j}}(F_1, F_2)=c \cdot 2 + (1-c) f(F_1^j, F_2^j)=f(F_1^j, F_2^j)=2$ \\
    coronal & -1    & -1    & \textbf{0} &  \\
    anterior & 0     & 0     & \textcolor{blue}{\textit{0}} &  \\
    distributed & 0     & 0     & \textcolor{blue}{\textit{0}} &  \\
    lateral & -1    & -1    & \textbf{0} &  \\
    dorsal & -1    & -1    & \textbf{0} &  \\
    high  & 0     & 0     & \textcolor{blue}{\textit{0}} &  \\
    front & 0     & 0     & \textcolor{blue}{\textit{0}} & \\
    \hline
    & & & \multicolumn{2}{l}{\textbf{total distance: 9}} \\
    \bottomrule
    \end{tabular}%
  \caption{An example of calculating the distance between [z] and [f] with our distance function. The distance between I-features (in bold) is calculated using general distance function $f(x_1, x_2)$, while the distance between D-features (in italic blue) is calculated using $g_{j,k}(F_1, F_2)$. The `total distance' is the sum of the distances across all dimensions. Details of the calculation are provided in the `Note' column.}
  \label{tab:exp-calcuate-dist}%
\end{table*}%

\begin{theorem} \label{proposition}
    %$g$ satisfies the triangle inequality: 
    $\forall F_1,F_2,F_3\in\Omega$, $\forall$ indices of paired D-feature and I-feature $j$ and $k$,
    \begin{equation}\label{eq:triangle}
    \setlen
        g_{j,k}(F_1,F_2)+g_{j,k}(F_1,F_3)\geqslant g_{j,k}(F_2,F_3).
    \end{equation}
\end{theorem}

\begin{proof}
    Let $c_{st}=\min\{f(X_s^k,X_t^k),1\}$. We have
    \begin{equation}
    \begin{aligned}
        \setlen
        \Delta=& g_{j,k}(F_1,F_2)+g_{j,k}(F_1,F_3)- g_{j,k}(F_2,F_3)\\
        =&(c_{12}+c_{13}-c_{23})\cdot s_j+(1-c_{12}) f(F_1^j,F_2^j)\\
        & +(1-c_{13}) f(F_1^j,F_3^j)-(1-c_{23}) f(F_1^j,F_2^j)\\
        \geqslant &(c_{12}+c_{13}-c_{23})\cdot s_j+(1-c_{12}) f(F_1^j,F_2^j)\\
        & +(1-c_{13})f(F_1^j,F_3^j) \\
        & -(1-c_{23})[f(F_1^j,F_2^j)+f(F_1^j,F_3^j)]\\
        = &(c_{12}+c_{13}-c_{23}) \cdot s_j-(c_{12}-c_{23}) f(F_1^j,F_2^j)\\
        & -(c_{13}-c_{23}) f(F_1^j,F_3^j)  \label{eq:need-simplify}
    \end{aligned}
    \end{equation}
    There are three cases: If $c_{23}<c_{12}$ and $c_{23}<c_{13}$, then
    \begin{equation*}
    \begin{aligned}
    \setlen
    \Delta & \geqslant (c_{12}-c_{23})[s_j - f(F_1^j,F_2^j)]\\
    & +(c_{13} - c_{23})[s_j-f(F_1^j,F_3^j)] \geqslant 0.
    \end{aligned}
    \end{equation*}
    If $c_{23}>c_{12}$ and $c_{23}>c_{13}$, then
    \begin{equation*}
    \begin{aligned}
    \setlen
     \Delta
        \geqslant (c_{12}+c_{13}-c_{23})\cdot s_j \geqslant0.
    \end{aligned}
    \end{equation*}
    If $c_{23}$ lies between $c_{12}$ and $c_{13}$, without loss of generality, assume $c_{12}\leqslant c_{23}\leqslant c_{13}$. Then,
    \begin{equation*}
    \begin{aligned}
    \setlen
        \Delta
        & \geqslant (c_{13}-c_{23})\cdot s_j-(c_{13}-c_{23}) f(F_1^j,F_3^j)\\
        & \geqslant( c_{13}-c_{23})[s_j-f(F_1^j,F_3^j)]\geqslant0.
    \end{aligned}
    \end{equation*}
\end{proof}


Table \ref{tab:exp-calcuate-dist} provides an example of calculating the distance between [z] and [f] using our distance function $d(F_1, F_2)$ defined in Eq. \ref{eq:dist-func}. The phonetic feature vectors of [z] and [f] are denoted $F_1$ and $F_2$, respectively. The general distance $f$ is set as $f(x_1, x_2)=|x_1-x_2|$.