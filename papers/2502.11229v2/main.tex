\documentclass[english,ruled]{article}
\usepackage[T1]{fontenc}
\usepackage[latin9]{inputenc}
\usepackage{verbatim}
\usepackage{subcaption}
\usepackage{algorithm2e}
\usepackage{amsmath}
\usepackage{amsthm}
\usepackage{amssymb}
\usepackage{graphicx}
\usepackage{xcolor}
\usepackage{multicol}
\usepackage{multirow}
\usepackage{geometry}
\usepackage{booktabs}
\usepackage{enumitem}
\usepackage{setspace}
\makeatletter
\usepackage[toc,page,header]{appendix}
\usepackage{minitoc}
\usepackage{ifthen}
\newboolean{doublecolumn}
\setboolean{doublecolumn}{false}
\newboolean{arxiv}
\setboolean{arxiv}{true}
\usepackage{pgfplots}
\usepackage{pdflscape}
\usepackage{hyperref}
\usepackage{cleveref}
\usepackage{authblk}
%\usepackage{showlabels}
\pgfplotsset{compat=1.15}

\setlength{\parindent}{0pt}
\geometry{verbose,tmargin=3cm,bmargin=3cm,lmargin=2.4cm,rmargin=2.4cm}
\linespread{1.0}

%%%%%%%%%% Start TeXmacs macros
\newcommand{\assign}{:=}
\newcommand{\backassign}{=:}
\newcommand{\cdummy}{\cdot}
\newcommand{\tmtextbf}[1]{\text{{\bfseries{#1}}}}
\newcommand{\tmop}[1]{\ensuremath{\operatorname{#1}}}
\newcommand{\tmtextit}[1]{\text{{\itshape{#1}}}}
\newcommand{\maxf}[1]{\underset{#1}{\text{maximize}}}
\newcommand{\minf}[1]{\underset{#1}{\text{minimize}}}
\newcommand{\hdm}{{\texttt{HDM}}}
\newcommand{\hdmagd}{{\texttt{HDM-AGD}}}
\newcommand{\hdmhb}{{\texttt{HDM-HB}}}
\newcommand{\hdmbest}{{\texttt{HDM-Best}}}
\newcommand{\ogm}{{\texttt{OGD}}}
\newcommand{\bfgs}{{\texttt{BFGS}}}
\newcommand{\lbfgs}{{\texttt{L-BFGS}}}
\newcommand{\osgm}{{\texttt{OSGM}}}
\newcommand{\osgmrx}{\texttt{OSGM-R}}
\newcommand{\osgmrzx}{\texttt{OSGM-RZ}}
\newcommand{\osgmgx}{\texttt{OSGM-G}}
\newcommand{\osgmhx}{\texttt{OSGM-H}}
\newcommand{\gd}{{\texttt{GD}}}
\newcommand{\optdgd}{{\texttt{OptDiagGD}}}
\newcommand{\adagrad}{{\texttt{AdaGrad}}}
\newcommand{\adam}{{\texttt{Adam}}}
\newcommand{\sagd}{{\texttt{SAGD}}}
\newcommand{\agd}{{\texttt{AGD}}}
\newcommand{\mathd}{\mathrm{d}}
\newcommand{\nin}{\not\in}
\newcommand{\TODO}[1]{\textcolor{red}{TODO:  #1}}
%%%%%%%%%% End TeXmacs macros

\newcommand{\thought}[1]{{\color[rgb]{0.2,0.39,0.66}(#1)}}
\newcommand{\todo}[1]{{\color[rgb]{1.0,0.0,0.0}(#1)}}
\newcommand{\hsh}[1]{{\color{green!50!black} Henrik: #1}}
\newcommand{\st}[1]{{\color{red!50!black} Sebastian: #1}}

\newcommand{\ulm}[1]{_{\scaleto{\mathrm{#1}}{3pt}}}
\newcommand\at[2]{\left.#1\right|_{#2}}











\newtheorem{assumption}{Assumption}

\DeclareMathOperator*{\argmax}{arg\,max}
\DeclareMathOperator*{\argmin}{arg\,min}

\newcommand{\swname}[1]{\texttt{#1}}
\newcommand{\ie}{i\/.\/e\/.,\/~}
\newcommand{\eg}{e\/.\/g\/.,\/~}
\newcommand{\cf}{cf\/.\/~}

\newcommand{\fig}{Fig\/.\/~}
\newcommand{\defn}{Def\/.\/~}
\newcommand{\sect}{Sec\/.\/~}
\newcommand{\tabl}{Tab\/.\/~}
\newcommand{\algo}{Algorithm~}
\newcommand{\theo}{Theorem~}

\newcommand{\bnnl}{3 hidden layers}
\newcommand{\bnnn}{50 neurons}
\newcommand{\bnna}{tanh activations}

\newcommand{\capt}[1]{\mdseries{\emph{#1}}}

\newcommand{\videolink}{at \url{https://youtu.be/_d7AqTRjz6g}}
\newcommand{\codelink}{\url{https://github.com/wheelbot/mini-wheelbot}}

\newcommand{\fakepar}[1]{\vspace{0mm}\noindent\textbf{#1.}}

\newcommand{\needref}{\textcolor{red}{[REF]}}

\newcommand{\plotfontsize}{9pt}


\theoremstyle{plain}
\newtheorem{lem}{\protect\lemmaname}[section]
\theoremstyle{remark}
\newtheorem{rem}{\protect\remarkname}
\theoremstyle{plain}
\newtheorem{thm}{\protect\theoremname}[section]
\theoremstyle{plain}
\newtheorem{prop}{\protect\propositionname}[section]
\providecommand{\corollaryname}{Corollary}
\theoremstyle{plain}
\newtheorem{coro}{\protect\corollaryname}[section]
\theoremstyle{plain}
\newtheorem{exple}{\protect\examplename}[section]
\theoremstyle{plain}
\newtheorem{definition}{\protect\definitionname}[section]

\providecommand{\lemmaname}{Lemma}
\providecommand{\remarkname}{Remark}
\providecommand{\theoremname}{Theorem}
\providecommand{\examplename}{Example}
\providecommand{\propositionname}{Proposition}
\providecommand{\definitionname}{Definition}

% cleveref
\crefdefaultlabelformat{#2\textbf{#1}#3} % <-- Only #1 in \textbf
\crefname{section}{\textbf{section}}{\textbf{sections}}
\Crefname{section}{\textbf{Section}}{\textbf{Sections}}
\crefname{thm}{\textbf{theorem}}{\textbf{theorems}}
\Crefname{thm}{\textbf{Theorem}}{\textbf{Theorems}}
\crefname{lem}{\textbf{lemma}}{\textbf{lemmas}}
\Crefname{lem}{\textbf{Lemma}}{\textbf{Lemmas}}
\crefname{prop}{\textbf{proposition}}{\textbf{propositions}}
\Crefname{prop}{\textbf{Proposition}}{\textbf{Propositions}}
\crefname{algorithm}{\textbf{algorithm}}{\textbf{algorithms}}
\Crefname{algorithm}{\textbf{Algorithm}}{\textbf{Algorithms}}
\crefname{coro}{\textbf{Corollary}}{\textbf{corollaries}}
\Crefname{coro}{\textbf{Corollary}}{\textbf{corollaries}}
\crefname{definition}{\textbf{Definition}}{\textbf{definitions}}
\Crefname{definition}{\textbf{Definition}}{\textbf{definitions}}
\crefname{table}{\textbf{Table}}{\textbf{tables}}
\Crefname{table}{\textbf{Table}}{\textbf{tables}}
\crefname{figure}{\textbf{Figure}}{\textbf{figures}}
\Crefname{figure}{\textbf{Figure}}{\textbf{figures}}

% Surrogate loss shorthand
\newcommand{\rk}{r_{k}}
\newcommand{\rxz}{r^z_{x}}
\newcommand{\rxkz}{r^z_{x^k}}

% Comments
\newcommand{\YC}[1]{ }
\renewcommand{\YC}[1]{\textcolor{blue}{[YC: #1]}}
\newcommand{\gwz}[1]{\textcolor{cyan}{[gwz: #1]}}
\begin{document}

\title{Provable and Practical Online Learning Rate Adaptation with Hypergradient Descent}

\author[1]{Ya-Chi Chu\thanks{ycchu97@stanford.edu}}
\author[2]{Wenzhi Gao\thanks{gwz@stanford.edu, equal contribution}}
\author[2,3]{Yinyu Ye\thanks{yyye@stanford.edu}}
\author[2,3]{Madeleine Udell\thanks{udell@stanford.edu}}
\affil[1]{Department of Mathematics, Stanford University}
\affil[2]{ICME, Stanford University}
\affil[3]{Department of Management Science and Engineering, Stanford University}

\maketitle

\begin{abstract}
Retrieval-Augmented Generation (RAG) is often used with Large Language Models (LLMs) to infuse domain knowledge or user-specific information. In RAG, given a user query, a retriever extracts chunks of relevant text from a knowledge base. These chunks are sent to an LLM as part of the input prompt. Typically, any given chunk is repeatedly retrieved across user questions. However, currently, for every question, attention-layers in LLMs fully compute the key values (KVs) repeatedly for the input chunks, as state-of-the-art methods cannot reuse KV-caches when chunks appear at arbitrary locations with arbitrary contexts. Naive reuse leads to output quality degradation.  This leads to potentially redundant computations on expensive GPUs and increases latency. In this work, we propose \sys, a system for managing and reusing precomputed KVs corresponding to the text chunks (we call \textit{chunk-caches}) in RAG-based systems. We present how to identify \hl{\textit{chunk-caches} that are reusable}, how to efficiently perform a small fraction of recomputation to \textit{fix} the cache to maintain output quality, and how to efficiently store and evict \textit{chunk-caches} in the hardware for maximizing reuse while masking any overheads. With real production workloads as well as synthetic datasets, we show that \sys reduces redundant computation by \textbf{51\%} over SOTA prefix-caching and \textbf{75\%} over full recomputation.
\hl{Additionally, with continuous batching on a real production workload, we get a \textbf{1.6$\times$} speedup in throughput and a \textbf{2$\times$} reduction in end-to-end response latency over prefix-caching while maintaining quality, for both the \llama-3-8B and \llama-3-70B models. 
}
\end{abstract}





\section{Introduction}
\label{sec:intro}
We study the problem of estimating the normalizing constant $Z=\int_{\R^d}\e^{-V(x)}\d x$ of an unnormalized probability density function (p.d.f.) $\pi\propto\e^{-V}$ on $\R^d$, so that $\pi(x)=\frac{1}{Z}\e^{-V(x)}$. The normalizing constant appears in various fields: in Bayesian statistics, when $\e^{-V}$ is the product of likelihood and prior, $Z$ is also referred to as the marginal likelihood or evidence \citep{gelman2013bayesian}; in statistical mechanics, when $V$ is the Hamiltonian\footnote{Up to a multiplicative constant $\beta=\frac{1}{k_\mathrm{B}T}$ known as the thermodynamic beta, where $k_\mathrm{B}$ is the Boltzmann constant and $T$ is the temperature. When borrowing terminologies from physics, we ignore this quantity for simplicity.}, $Z$ is known as the partition function, and $F:=-\log Z$ is called the free energy \citep{chipot2007free,lelievre2010free,pohorille2010good}. The task of normalizing constant estimation has numerous applications, including computing log-likelihoods in probabilistic models \citep{sohl2012hamiltonian}, estimating free energy differences \citep{lelievre2010free}, and training energy-based models in generative modeling \citep{song2021how,carbone2023efficient,sander2025joint}. It is challenging in high dimensions or when $\pi$ is multimodal (i.e., $V$ has a complex landscape).

Conventional approaches based on importance sampling \citep{meng1996simulating} are widely adopted to tackle this problem, but they suffer from high variance due to the mismatch between target and proposal distributions when the target distribution is complicated \citep{chatterjee2018the}. To alleviate this issue, the technique of annealing tries constructing a sequence of intermediate distributions that bridge these two distributions, which motivates several popular methods including path sampling \citep{chen1997on,gelman1998simulating}, annealed importance sampling (AIS, \cite{neal2001annealed}), and sequential Monte Carlo (SMC, \cite{doucet2000sequential,delmoral2006sequential,syed2024optimised}) in statistics literature, as well as thermodynamic integration (TI, \cite{kirkwood1935statistical}) and Jarzynski equality (JE, \cite{jarzynski1997nonequilibrium,ge2008generalized,hartmann2019jarzynski}) in statistical mechanics literature. In particular, JE points out the connection between the free energy difference between two states and the work done over a series of trajectories linking these two states, while AIS constructs a sequence of intermediate distributions and estimates the normalizing constant by importance sampling over these distributions. These two methods are our primary focus in this paper.

Despite the empirical success of annealing-based methods \citep{ma2013estimating,krause2020algorithms,mazzanti2020efficient,yasuda2022free,chen2024ensemble,schonle2024sampling}, the theoretical understanding of their performance is still limited. Existing works for importance sampling mainly focus on the asymptotic bias and variance of the estimator \citep{meng1996simulating,gelman1998simulating}, while works on JE usually simplify the problem by assuming the work follows simple distributions (e.g., Gaussian or gamma) \citep{echeverria2012,arrar2019on}. Moreover, only analyses asymptotic in the number of particles derived from central limit theorem
exist \cite[Sec. 4.1]{lelievre2010free}. In this paper, we aim to establish a rigorous non-asymptotic analysis of estimators based on JE and AIS, while introducing minimal assumptions on the target distribution. Moreover, we also propose a new algorithm based on reverse diffusion samplers to tackle a potential shortcoming of AIS.

\paragraph{Contributions.} Our key technical contributions are summarized as follows.
\begin{itemize}[wide=0pt,itemsep=0pt, topsep=0pt,parsep=0pt,partopsep=0pt]
    \item We discover a novel strategy for analyzing the complexity of normalizing constant estimation,
    applicable to a wide range of target distributions (see \cref{assu:pi,assu:AC}) that may not satisfy isoperimetric conditions such as log-concavity.
    \item In \cref{sec:jar}, we study JE
    and prove an upper bound on the time required for running the annealed Langevin dynamics to estimate the normalizing constant within $\varepsilon$ relative error with high probability. The final bound depends on the action of the curve, specifically the integral of the squared metric derivative in Wasserstein-2 distance.
    \item Building on the insights from the analysis of the continuous dynamics, in \cref{sec:ais} we
    establish the first non-asymptotic oracle complexity bound for AIS, representing the first analysis of normalizing constant estimation algorithms without assuming a log-concave target distribution.
    \item Finally, in \cref{sec:revdif}, we point out a potential limitation of the geometric interpolation commonly used in annealing. To address this issue, we propose a novel algorithm based on reverse diffusion samplers and build up a framework for analyzing its oracle complexity.
\end{itemize}

\paragraph{Related works.} We briefly review some related works, and defer detailed discussion to \cref{app:rel_work}.
\begin{itemize}[wide=0pt,itemsep=0pt, topsep=0pt,parsep=0pt,partopsep=0pt]
    \item \underline{Methods for normalizing constant estimation.} We mainly discuss two classes of methods here. First, the \emph{equilibrium} methods, such as TI \citep{kirkwood1935statistical} and its variants \citep{brosse2018normalizing,ge2020estimating,chehab2023provable,kook2024sampling}, which involve sampling sequentially from a series of equilibrium Markov transition kernels. Second, the \emph{non-equilibrium} methods, such as AIS \citep{neal2001annealed}, which samples from a non-equilibrium SDE that gradually evolves from a prior distribution to the target distributions. In \cref{app:rel_work_ti}, we show that TI is a special case of AIS using the ``perfect'' transition kernels.4 Recent years have also witnessed the emergence of \textit{learning-based} non-equilibrium methods for normalizing constant estimation, which are typically byproducts of sampling algorithms \citep{zhang2022path,nusken2021solving,richter2024improved,sun2024dynamical,vargas2024transport,albergo2024nets,blessing2025underdamped,chen2025sequential}. Additionally, there are also several methods based on particle filtering (e.g., \citet{kostov2017algorithm,jasra2018multilevel,ruzayqat2022multilevel}).
    \item \underline{Variance reduction in JE and AIS.} Our poof methodology focuses on the discrepancy between the sampling path measure and the reference path measure, which is related to the variance reduction technique in applying JE and AIS. For example, \cite{vaikuntanathan2008escorted} introduced the idea of escorted simulation, \cite{hartmann2017variational} proposed a method for learning the optimal control protocol in JE through the variational characterization of free energy, and \cite{doucet2022score} leveraged score-based generative model to learn the optimal backward kernel. Quantifying the discrepancy between path measures is the core of our analysis.
    \item \underline{Complexity analysis for normalizing constant estimation.} \cite{chehab2023provable} studied the asymptotic statistical efficiency of the curve for TI measured by the asymptotic mean-squared error, and highlighted the advantage of the geometric interpolation. In terms of non-asymptotic analysis, existing works mainly rely on the isoperimetry of the target distribution. For instance, \cite{andrieu2016sampling}  derived bounds of bias and variance for TI under Poincar\'e inequality, \cite{brosse2018normalizing} provided complexity guarantees for TI under both strong and weak log-concavity conditions, while \cite{ge2020estimating} improved the complexity under strong log-concavity using multilevel Monte Carlo.
\end{itemize}
\begin{figure}[t]
    \centering
    \includegraphics[width=\linewidth]{figs/taxonomy_v3.pdf}
    \caption{
    Taxonomy of Concept Erasers. Concept erasure methods are categorized based on their optimization strategy (first level) and the model components they modify (second level). A detailed discussion is provided in Sec.~\ref{sec:method}.
    }
    \label{fig:taxonomy}
\end{figure}

\section{Backgrounds} \label{sec:preliminaries}

This section presents an overview of the Text-to-Image (T2I) diffusion model with a particular focus on Stable Diffusion (SD)~\cite{stable_diffusion}. %, which serves as the foundational framework for evaluating concept erasure methods. 
As shown in Fig.~\ref{fig:overview}, SD comprises three main components: a vision decoder for reconstructing images from latent representations, a latent diffusion model for iterative denoising, and a conditional text encoder that transforms textual prompts into conditioning vectors.
We outline both the training and inference mechanisms of SD, which are essential for understanding how concept erasure techniques modify key model components or inference steps to suppress undesired concepts.

% \begin{table*}[t!]
%   \caption{Taxonomy of Concept Erasure Methods in T2I Models. Methods are categorized based on their optimization strategies and the specific model components they modify. CA denotes the cross-attention layers within the U-Net, while CFG refers to Classifier-Free Guidance adjustments. A comprehensive discussion of these methods is provided in Sec.~\ref{sec:method}.
% }
%   \centering
%   % \footnotesize
%   % \small
%   \scriptsize
%   \setlength{\tabcolsep}{3pt} % Reduce spacing for the first three columns
%   \begin{adjustbox}{width=\textwidth,center}
%     \begin{tabular}{p{1.8cm}p{3.6cm}cccccp{7.0cm}} 
%       \toprule
%       \textbf{Category} & \textbf{Representative Works} & \multicolumn{5}{c}{\textbf{Optimization Space}} & \textbf{Optimization Strategy} \\ 
%       \cmidrule(lr){3-7} 
%        & & \textbf{U-Net} & \textbf{CA} & \textbf{CLIP} & \textbf{LLM} & \textbf{CFG} & \\ 
%       \midrule
%       \multirow{13}{*}{\textbf{Fine-tuning}}  
%       & FMN~\cite{Zhang2023ForgetMeNotLT} &  & \cmark &  &  &  & Attention reweighting \\ 
%       & AC~\cite{Ablating_Concept} &  & \cmark &  &  &  & Remapping erased concepts \\ 
%       & SALUN~\cite{fan2024salun} & \cmark &  &  &  &  & Saliency-guided tuning \\ 
%       & ESD~\cite{esd} & \cmark & \cmark &  &  &  & Concept removal in generative noise process \\ 
%       & DT~\cite{Ni2023DegenerationTuningUS} & \cmark &  &  &  &  & Targeted concept degradation \\ 
%       & Geom-Erasing~\cite{Liu2023ImplicitCR} & \cmark &  &  &  &  & Targeted concept degradation \\ 
%       & SA~\cite{Heng2023SelectiveAA} & \cmark &  &  &  &  & Targeted concept degradation \\ 
%       & IMMA~\cite{Zheng2023IMMAIT} & \cmark &  &  &  &  & Prevents unauthorized fine-tuning \\ 
%       & SAFE-CLIP~\cite{safe_clip} &  &  & \cmark &  &  & Adversarial robustness for CLIP \\ 
%       & Latent Guard~\cite{Liu2024LatentGA} &  &  & \cmark &  &  & Targeted feature suppression \\ 
%       & AdvUnlearn~\cite{Zhang2024DefensiveUW} &  &  & \cmark &  &  & Adversarial fine-tuning for CLIP \\
%       & Receler~\cite{Huang2023RecelerRC} & \cmark  &  &  &  &  & Introduces adapter for robustness in U-Net \\
%       & R.A.C.E~\cite{RACE} & \cmark  &  &  &  &  & Adversarial fine-tuning for U-Net \\
%       \midrule
%       \multirow{7}{*}{\textbf{Closed-form}}  
%       & ReFACT~\cite{Arad2023ReFACTUT} &  &  & \cmark &  &  & Low-rank memory update in CLIP MLP layers \\ 
%       & TIME~\cite{Orgad2023EditingIA} &  & \cmark &  &  &  & Projection matrix updates in cross-attention \\ 
%       & UCE~\cite{Gandikota2023UnifiedCE} &  & \cmark &  &  &  & Multi-concept projection learning \\ 
%       & MACE~\cite{Lu2024MACEMC} &  & \cmark &  &  &  & LoRA-based parameter refinement for erasure \\ 
%       & EMCID~\cite{Xiong2024EditingMC} & \cmark & \cmark &  &  &  & Two-stage closed-form editing (self-distillation + projection) \\ 
%       & MUNBa~\cite{Wu2024MUNBaMU} &  & \cmark & \cmark &  &  & Nash bargaining-based concept unlearning \\ 
%       & RECE~\cite{Gong2024ReliableAE} & \cmark  &  &  &  &  & Adversarial fine-tuning \\
%       \midrule
%       \multirow{6}{*}{\textbf{Inference-Time}}  
%       & SLD~\cite{sld} &  & \cmark &  &  & \cmark & Adjusts latent denoising dynamics \\ 
%       & AMG~\cite{Chen2024TowardsMD} &  & \cmark &  &  & \cmark & Prevents overfitting to erased concepts \\ 
%       & SAFREE~\cite{safree} &  &  & \cmark &  &  & Prevents undesired text-image associations \\ 
%       & Content Suppression~\cite{Li2024GetWY} &  &  & \cmark &  &  & Enforces embedding constraints \\ 
%       & ORES~\cite{ores} &  &  &  & \cmark &  & LLM-based adversarial filtering \\ 
%       & GuardT2I~\cite{Yang2024GuardT2IDT} &  &  &  & \cmark &  & Detects circumvention prompts \\ 
%       \bottomrule
%     \end{tabular}
%   \end{adjustbox}
%   \label{tab:taxonomy}
% \end{table*}

\begin{table*}[t!]
  \caption{Taxonomy of Concept Erasure Methods in T2I Models. Methods are categorized based on their optimization strategies and the specific model components they modify. In the third column, "CA" denotes the cross-attention layers within the latent diffusion model, while "CFG" refers to Classifier-Free Guidance adjustments. A comprehensive discussion of these methods is provided in Sec.~\ref{sec:method}.}
  \centering
  \scriptsize
  \setlength{\tabcolsep}{3pt} % Reduce spacing for the first three columns
  \begin{adjustbox}{width=\textwidth,center}
    \begin{tabular}{p{1.8cm}p{3.6cm}cccccp{6.5cm}} 
      \toprule
      \textbf{Category} & \textbf{Representative Works} & \multicolumn{5}{c}{\textbf{Optimization Space}} & \hspace{5mm} \textbf{Description} \\ 
      \cmidrule(lr){3-7} 
       & & \textbf{U-Net} & \textbf{CA} & \textbf{CLIP} & \textbf{LLM} & \textbf{CFG} & \\ 
      \midrule
      \multirow{13}{1.8cm}{\textbf{Fine-Tuning}}  
      & FMN~\cite{Zhang2023ForgetMeNotLT} &  & \cmark &  &  &  & Minimize attention activation to erase concepts. \\ 
      & AC~\cite{Ablating_Concept} &  & \cmark &  &  &  & Remaps erased concepts to general concepts. \\ 
      & SALUN~\cite{fan2024salun} & \cmark &  &  &  &  & Modifies influential weights to remove concepts. \\ 
      & ESD~\cite{esd} & \cmark & \cmark &  &  &  & Edits noise prediction to remove concepts. \\ 
      & DT~\cite{Ni2023DegenerationTuningUS} & \cmark &  &  &  &  & Degrades model’s ability to reconstruct erased concepts. \\ 
      & Geom-Erasing~\cite{Liu2023ImplicitCR} & \cmark &  &  &  &  & Uses geometric constraints for concept removal. \\ 
      & SA~\cite{Heng2023SelectiveAA} & \cmark &  &  &  &  & Continual learning-based forgetting approach. \\ 
      & IMMA~\cite{Zheng2023IMMAIT} & \cmark &  &  &  &  & Enhances robustness against unauthorized fine-tuning. \\ 
      & SAFE-CLIP~\cite{safe_clip} &  &  & \cmark &  &  & Fine-tunes CLIP with safe and unsafe text-image quadruplets. \\ 
      & Latent Guard~\cite{Liu2024LatentGA} &  &  & \cmark &  &  & Fine-tunes the CLIP text encoder with safe and unsafe pairs. \\ 
      & AdvUnlearn~\cite{Zhang2024DefensiveUW} &  &  & \cmark &  &  & Adversarial finetuning for CLIP text encoder. \\
      & Receler~\cite{Huang2023RecelerRC} &  & \cmark &  &  &  & Uses adapters to enhance robustness. \\
      & R.A.C.E~\cite{RACE} & \cmark  & \cmark &  &  &  & Adversarially fine-tunes U-Net for resilience. \\
      \midrule
      \multirow{7}{1.8cm}{\textbf{Closed-form Model Editing}}  
      & ReFACT~\cite{Arad2023ReFACTUT} &  &  & \cmark &  &  & Updates CLIP’s memory via low-rank edits. \\ 
      & TIME~\cite{Orgad2023EditingIA} &  & \cmark &  &  &  & Modifies CA projection matrices for concept editing. \\ 
      & UCE~\cite{Gandikota2023UnifiedCE} &  & \cmark &  &  &  & Simultaneously erases multiple concepts. \\ 
      & MACE~\cite{Lu2024MACEMC} &  & \cmark &  &  &  & Utilize adapters for large-scale concept erasure. \\ 
      & EMCID~\cite{Xiong2024EditingMC} &  &  & \cmark  &  &  & Large-scale concept erasure via two-stage closed-form editing \\ 
      & MUNBa~\cite{Wu2024MUNBaMU} &  &  & \cmark &  &  & Uses Nash bargaining for controlled concept removal. \\ 
      & RECE~\cite{Gong2024ReliableAE} &  & \cmark &  &  &  & Integrates adversarial fine-tuning with closed-form editing. \\
      \midrule
      \multirow{6}{1.8cm}{\textbf{Inference-Time Intervention}}  
      & SLD~\cite{sld} &  &  &  &  & \cmark & Incorporates safety guidance to mitigate undesired concepts. \\ 
      & AMG~\cite{Chen2024TowardsMD} &  & &  &  & \cmark & Introduces three guidance strategies to prevent memorization. \\ 
      & SAFREE~\cite{safree} &  &  & \cmark &  &  & Self-validating filtering and re-attention for safe generation.\\ 
      & Content Suppression~\cite{Li2024GetWY} &  &  & \cmark &  &  & Adjusts embeddings to suppress concept generation. \\ 
      & ORES~\cite{ores} &  &  &  & \cmark & \cmark & Utilizes LLMs to filter and rewrite prompts for safer generation. \\ 
      & GuardT2I~\cite{Yang2024GuardT2IDT} &  &  &  & \cmark &  & Propose conditional LLM to detect adversarial prompts. \\ 
      \bottomrule
    \end{tabular}
  \end{adjustbox}
  \label{tab:taxonomy}
\end{table*}

% \subsection{Comparison with Related Work}  
% Concept erasure differs from machine unlearning, image editing. While machine unlearning focuses on removing specific data points from a model to comply with privacy regulations, concept erasure targets entire content categories, such as explicit or copyrighted styles, preventing their regeneration. Unlike image editing, which modifies specific attributes of an input image based on auxiliary inputs, concept erasure alters a model’s ability to generate certain concepts across all inputs.

\subsection{Three Components of Stable Diffusion }

Stable Diffusion comprises three primary components:

%~\cite{esser2021taming}
\paragraph{(1) Image Autoencoder.} The model leverages a pre-trained autoencoder to compress high-dimensional image data into a low-dimensional latent representation. The encoding network $\mathcal{E}(\cdot)$ maps an image $x$ to a latent variable $z = \mathcal{E}(x)$, and the decoding network $\mathcal{D}(\cdot)$ reconstructs the image from the latent space such that $\mathcal{D}(z) = \hat{x} \approx x$. This design ensures effective data compression while minimizing reconstruction error, preserving essential image features critical for generative tasks.



\paragraph{(2) Latent Diffusion Model.} The core generative process in SD is governed by a U-Net-based Latent Diffusion Model (LDM) that progressively refines noisy latent representations toward high-fidelity outputs. The training objective is formulated as:
\begin{equation}
    L_{\text{SD}} = \mathbb{E}_{n \sim \mathcal{N}(0,1), z, c, t} \left[
    \| n - \Phi_{\theta}(z_t, c) \|_2^2
    \right],
\end{equation}
where $c$ is the text embedding derived from the input prompt and integrated via cross-attention, $t$ denotes the diffusion timestep, $n$ is a noise vector sampled from a standard Gaussian distribution $\mathcal{N}(0,1)$, and $z_t$ is the noisy latent variable at timestep $t$. The LDM \( \Phi_{\theta} \), parameterized by \( \theta \), is trained to predict and remove noise at each step, progressively refining the latent variable along the diffusion trajectory.


\paragraph{(3) Conditional Text Encoding.} The model employs a text encoder to transform user-provided text prompts into conditioning vectors, enabling fine-grained control over the generation process. Specifically, the textual prompt $y$ is embedded as $c = \mathcal{E}_{\text{txt}}(y)$, where $\mathcal{E}_{\text{txt}}$ typically textual encoder of CLIP~\cite{CLIP}. These text embeddings are integrated through the cross-attention layers within the latent diffusion model~\cite{stable_diffusion}, allowing the textual context to dynamically influence each denoising step.

\subsection{Inference in Stable Diffusion}
Classifier-free guidance~\cite{ho2022classifier} enhances the conditionality of the image synthesis process during the inference phase of SD. The process starts with initializing latent representations $z_T$ sampled from a Gaussian distribution. The denoising trajectory is steered by classifier-free guidance, which modifies the denoising function as follows:
\begin{equation}\label{eq:classifier-free-guidance}
    \Tilde{\Phi}_{\theta}(z_t, c) = \Phi_{\theta}(z_t, \phi) + \alpha \left( \Phi_{\theta}(z_t, c) - \Phi_{\theta}(z_t, \phi) \right),
\end{equation}
where $\Phi_{\theta}(z_t, c)$ and $\Phi_{\theta}(z_t, \phi)$ represent the conditioned and unconditioned latent noises, respectively. The guidance scale $\alpha > 1$ amplifies the influence of the conditioned path, embedding the textual information into the generative process. 
Iterative refinement reduces noise through sequential calculations of $z_{t-1} = \Tilde{\Phi}_{\theta}(z_t, c)$, progressing until $t=0$. The final coherent image representation $z_0$ is transformed into the output image $\hat{x}$ by the decoder, $\hat{x} = \mathcal{D}(z_0)$. The T2I generation process can be succinctly expressed as $SD(y) = \mathcal{D}(\Tilde{\Phi}_{\theta}(z_T, \mathcal{E}_{\text{txt}}(y)))$.


\section{Main Results}
\label{sec:results}
We now state the soundness and completeness of the translation of the STL formulas into the transducers.

We propose here the equivalence of the Until and Release operator of STL and its transducer. %A similar proof is done for the Release operator of STL and its transducer.

\hscomment{1. Include the output alphabet $\top, \bot$, not the input (thius can be used to prove the transparancy), 2. CHange the notation. 3. Move it to Section 4.}
\begin{proposition}[Equivalence of Until operator of STL and its transducer]    
\label{propo1}
        Let $\signal$ be the signals and $\textsf{SignEncode}(\signal,\varphi_1\until_{[a,b]}\varphi_2)$ be its encoded timed word for the given STL formula $\varphi_1\until_{[a,b]}\varphi_2$.
        If the encoded timed word is accepted by the transducer of Until operator $\automaton_{\varphi_1\until_{[a,b]}\varphi_2}$ then the corresponding signals  $\signal$  satisfy $ \varphi_1\until_{[a,b]}\varphi_2$, i.e.,
        \begin{align*}
            \textsf{SignEncode}(\signal,\varphi_1\until_{[a,b]}\varphi_2) \in \mathcal{L}(\automaton_{\varphi_1\until_{[a,b]}\varphi_2}) \implies \signal \models \varphi_1\until_{[a,b]}\varphi_2
        \end{align*}
\end{proposition}

\begin{proposition}[Equivalence of Release operator of STL and its transducer]   
\label{propo2}
    %Let $\signal$ be the signals.
    %Let $\sigma$ be the timed word. 
    Let $\signal$ be the signals and $\textsf{SignEncode}(\signal,\varphi_1\release_{[a,b]}\varphi_2)$ be its encoded timed word for the given STL formula $\varphi_1\release_{[a,b]}\varphi_2$
    If the timed word is accepted by the transducer of Release operator $\automaton_{\varphi_1\release_{[a,b]}\varphi_2}$ then the corresponding  signals $\signal$ satisfy $ \varphi_1\release_{[a,b]}\varphi_2$, i.e.,
    \begin{align*}
        \textsf{SignEncode}(\signal,\varphi_1\release_{[a,b]}\varphi_2) \in \mathcal{L}(\automaton_{\varphi_1\release_{[a,b]}\varphi_2}) \implies \signal \models \varphi_1\release_{[a,b]}\varphi_2
    \end{align*}
\end{proposition}

%%%%%%%%%%%%%%%%%%%%%%%%%%%%%%%%%%%%%%%%%%%%%


\begin{proposition} 
\label{propo3}
    Let $\signal$ be the signals and let $\textsf{SignEncode}(\signal,\varphi_1\until_{[a,b]}\varphi_2)$ be the corresponding encoded timed word. If signals  satisfy the Until formula $ \varphi_1\until_{[a,b]}\varphi_2$, then its encoded word is accepted by its transducer, i.e., 
    \begin{align*}
        \signal \models \varphi_1\until_{[a,b]}\varphi_2 \implies \textsf{SignEncode}(\signal,\varphi_1\until_{[a,b]}\varphi_2) \in 
        \mathcal{L}(\automaton_{\varphi_1\until_{[a,b]}\varphi_2}) 
    \end{align*}
\end{proposition}

\begin{proposition} 
\label{propo4}
    Let $\signal$ be the signals and let $\textsf{SignEncode}(\signal,\varphi_1\release_{[a,b]}\varphi_2)$ be the corresponding encoded timed word. If signals  satisfy the Release formula $ \varphi_1\release_{[a,b]}\varphi_2$, then its encoded word is accepted by its transducer, i.e., 
    \begin{align*}
        \signal \models \varphi_1\release_{[a,b]}\varphi_2 \implies \textsf{SignEncode}(\signal,\varphi_1\release_{[a,b]}\varphi_2) \in 
        \mathcal{L}(\automaton_{\varphi_1\release_{[a,b]}\varphi_2}) 
    \end{align*}
\end{proposition}

% %%%%%%%%%%%%%%%%%%%%%%%%%%%%%%%%%%%%%%%%%%%%%


% \begin{proposition} 
% \label{propo5}
%     Let $\signal$ be the signals and let $\sigma$ be the corresponding encoded timed word. If signals do not satisfy the Until formula $ \varphi_1\until_{[a,b]}\varphi_2$, and the enforcer corrects the signals to $\signal'$ by minimally modifying $\signal$ and the encoded signals of the modified signals are accepted by the transducer of Until, then the modified signals satisfy  the Until formula, i.e., 
%     \begin{equation*}    
%         \begin{split}
%         \signal \not \models p_1\until_{[a,b]}p_2 \land \exists\signal', E_{p_1\until_{[a,b]}p_2}(\signal, t)=\signal' : min(|\signal'-\signal|)\\ \land \sigma' \in 
%         \mathcal{L}(\automaton_{\varphi_1\release_{[a,b]}\varphi_2}) 
%         \implies \signal' \models p_1\until_{[a,b]}p_2
%         \end{split}
%     \end{equation*}    
% \end{proposition}

% \begin{proposition} 
% \label{propo6}
%     Let $\signal$ be the signals and let $\sigma$ be the corresponding encoded timed word. If signals do not satisfy the Until formula $ \varphi_1\release_{[a,b]}\varphi_2$, and the enforcer corrects the signals to $\signal'$ by minimally modifying $\signal$ and the encoded signals of the modified signals are accepted by the transducer of Release, then the modified signals satisfy  the Release formula, i.e.,
%     \begin{equation*}    
%         \begin{split}
%         \signal \not \models p_1\release_{[a,b]}p_2 \land \exists\signal', E_{p_1\release_{[a,b]}p_2}(\signal, t)=\signal' : min(|\signal'-\signal|)\\ \land \sigma' \in 
%         \mathcal{L}(\automaton_{\varphi_1\release_{[a,b]}\varphi_2}) 
%         \implies \signal' \models p_1\release_{[a,b]}p_2
%         \end{split}
%     \end{equation*}    
% \end{proposition}


The proofs of the propositions are provided at Appendix \ref{sec:appendix}.
\section{{\hdm} with Momentum} \label{sec:momentum}

This section develops two variants of {\hdm}, 
with heavy-ball momentum \cite{polyak1964some} and with Nesterov momentum \cite{nesterov1983method}.

\subsection{Heavy-ball Momentum}
\label{sec:heavyball}

The heavy-ball method is a practical acceleration technique: 
\begin{equation} \label{eqn:heavyball-update}
x^{k + 1} = x^k - P_k \nabla f (x^k) + B_k (x^k - x^{k - 1}).
\end{equation}
The momentum parameter $B_k$ is typically chosen as a scalar $B_k = \beta_k I$ with $\beta_k > 0$.
{\hdm} can learn a matrix momentum
$B_k \in \mathcal{B} \subseteq \mathbb{R}^{n \times n}$
with convergence guarantees (\Cref{thm:heavyball}) 
when $\mathcal{B}$ satisfies this assumption:
\begin{enumerate}[leftmargin=30pt,label=\textbf{A\arabic*:},ref=\rm{\textbf{A\arabic*}},start=5]
  \item Closed convex set $\Bcal$ satisfies $\tfrac{1}{2} I\in \Bcal$, $\diam (\Bcal) \leq D$. \label{ABcal}
\end{enumerate}

{\hdm} can \emph{jointly} learn the pair $(P_k, B_k)$ using the modified feedback function
\begin{equation} \label{eqn:heavyball-feedback}
  h_{x, x^-} (P, B) \assign 
  \tfrac{\psi(x^{+}(P, B), x) - \psi(x, x^{-})}{\| \nabla f (x) \|^2 + \frac{\tau}{2} \| x - x^- \|^2}
  = \tfrac{[f (x^+(P, B)) + \frac{\omega}{2} \| x^+(P, B) - x \|^2] - [f (x) + \frac{\omega}{2} \| x - x^- \|^2]}{\| \nabla f (x) \|^2 + \frac{\tau}{2} \| x - x^- \|^2}, 
\end{equation}
where $\psi$ is the potential function for heavy-ball momentum defined by $\psi (x, x^-) \assign f (x) + \tfrac{\omega}{2} \| x - x^- \|^2$ \cite{danilova2020non}; \[x^{+}(P, B) \assign x - P \nabla f (x) + B (x - x^{-})\] updates $x$; and $\omega > 0$ and $ \tau > 0$ are constants. \Cref{alg:ospolyak} presents the resulting method, \hdmhb, 
which uses {\hdm}, heavy-ball momentum, and a null step to ensure decrease of the potential function $\psi$.
\Cref{fig:demo:c} compares non-adaptive heavy-ball ($P_k \equiv  \alpha I, B_k \equiv \beta I$) against {\hdmhb} with full-matrix/diagonal preconditioner and scalar momentum.
 \Cref{thm:heavyball} presents the convergence of {\hdmhb}. 

\begin{algorithm}[h]
{\textbf{input} initial point $x^0 = x^1, \eta_p, \eta_b > 0$, $P_1$, $B_1$}\\
\For{k =\rm{ 1, 2,...}}{
$\hspace{1.2pt}~~~~x^{k+1/2} = x^k - P_k \nabla f(x^k) + B_k (x^k - x^{k-1})$ \\
$\hspace{2pt}~~~~~~P_{k+1} = \Pi_{\Pcal}[P_k - \eta_p \nabla_{P} h_{x^k, x^{k-1}}(P_k, B_k)]$ \\
$\hspace{1pt}~~~~~~B_{k+1} = \Pi_{\Bcal}[B_k - \eta_b \nabla_{B} h_{x^k, x^{k-1}}(P_k, B_k)]$ \\
$(x^{k + 1}, x^k) = \displaystyle \argmin_{(x^+, x) \in \{(x^k, x^{k-1}), (x^{k+1/2}, x^k) \}} \psi(x^+, x)$
}
{\textbf{output} $x^{K+1}$}
\caption{{\hdm} with heavy-ball momentum (\hdmhb)\label{alg:ospolyak}}
\end{algorithm}

\begin{thm}[Convergence of {\hdmhb}]\label{thm:heavyball}
Under \ref{A1}, \ref{A2} and \ref{ABcal}, \Cref{alg:ospolyak} satisfies
\begin{equation*}
  f (x^{K + 1}) - f (x^{\star}) \leq \tfrac{f (x^{1}) - f (x^{\star})}{K V \max\{ \gamma_K^{\star} - \frac{\rho_K}{K}, 0 \} + 1},
\end{equation*}
where $\gamma_{K}^{\star} \assign - \min_{(P, B) \in \mathcal{P} \times \mathcal{B}} \tfrac{1}{K} \sum_{k=1}^K h_{x^k, x^{k-1}}(P, B)$ depends on the iteration trajectory $\{x^k\}_{k \leq K}$; $\rho_K = \mathcal{O}(\sqrt{K})$ is the regret with respect to feedback \eqref{eqn:heavyball-feedback}; $V \assign \min\big\{ \tfrac{f (x^{1}) - f (x^{\star})}{4 \Delta^2}, \tfrac{\tau}{4 \omega} \big\}$; $\Delta$ is defined in \Cref{lem:hypergrad-to-online}.
\end{thm}

\subsection{Nesterov Momentum} \label{sec:nesterov}
{\hdm} can also improve accelerated gradient descent {\agd}:\begin{align}
  y^k ={} & x^k + ( 1 - \tfrac{A_k}{A_{k + 1}} ) (z^k - x^k)
  \nonumber\\
  x^{k + 1} ={} & y^k - \tfrac{1}{L} \nabla f (y^k) \label{eqn:agd-descent-lemma-0}\\
  z^{k + 1} ={} & z^k + \tfrac{A_{k + 1} - A_k}{L} \nabla f (y^k), \nonumber
\end{align}
where is a pre-specified sequence. 
{\hdm} can learn a preconditioner $P_k$ that replaces $\frac 1 L$ to accelerate the gradient step \eqref{eqn:agd-descent-lemma-0} in {\agd}. We call the resulting algorithm {\hdmagd}. \Cref{alg:osnes} provides a
realization of the {\hdmagd} based on a monotone variant of {\agd} \cite{d2021acceleration}. The convergence of {\hdmagd} is established in \Cref{thm:osnes}, the proof of which is deferred to \Cref{app:proof-osnes}.
\begin{algorithm}[h]
{\textbf{input} starting point $x^1, z^1, \eta > 0$, $\theta \in [\tfrac{1}{2}, LD) $, $A_0 = 0$}\\
\For{k =\rm{ 1, 2,...}}{
$\begin{aligned}
 A_{k + 1} ={} & (A_{k + 1} - A_k)^2 \nonumber \\
  y^k ={} & x^k + ( 1 - \tfrac{A_k}{A_{k + 1}} ) (z^k - x^k)
  \nonumber\\
  x^{k + 1} ={} & \underset{x \in \{ y^k - \frac{1}{L} \nabla f (y^k), y^k
  - P_k \nabla f (y^k), x^k \}}{\argmin} f (x) \nonumber\\
  P_{k + 1} ={} & \Pi_{\mathcal{P}} [P_k - \eta \nabla h_{y^k} (P_k)]
  \nonumber\\
  v_k ={} & \max \{ \tfrac{1}{2 \max \{ -h_{y^k} (P_k), 1 / (2 L) \}},
   \tfrac{L}{2 \theta} \}  \\
  z^{k + 1} ={} & z^k + \tfrac{(A_{k + 1} - A_k)}{v_k} \nabla f (y^k)
  \nonumber
\end{aligned}$
}
{\textbf{output} $x^{K+1}$}
\caption{{\hdm} with Nesterov momentum  \label{alg:osnes}}
\end{algorithm}

\begin{thm}
\label{thm:osnes}
Assume \ref{A1} and \ref{A2}. Suppose {\agd} starts from $(x', z')$ and runs for $K$ iterations to output $\hat{x}$.
Then \Cref{alg:osnes} starting from $(x^1, z^1) = (\hat{x}, z')$ and $\theta \in [\tfrac{1}{2}, LD)$ satisfies
\begin{align}
f (x^{K + 1}) - f (x^{\star}) \leq \big[ \tfrac{1}{2 \theta} + ( 8 - \tfrac{4}{\theta} ) ( \tfrac{L D - \omega^\star_K}{L D - \theta} ) \big] \tfrac{2 L \| z' - x^{\star} \|^2}{K^2} +\mathcal{O} ( \tfrac{\rho_K}{K^3} ), \nonumber
\end{align}
where $\omega^\star_K = - \min_{P \in \mathcal{P}} \tfrac{L}{K} \sum_{k = 1}^K h_{y^k} (P)$ depends on the iteration trajectory $\{x^k\}_{k \leq K}$.
\end{thm}

The parameter $\theta$ serves as a smooth interpolation between {\hdm} and {\hdmagd}: when $\theta = 1 / 2$, \Cref{thm:osnes} recovers the convergence rate of
vanilla {\agd}; when $\theta > 1 / 2$ and $\omega_K^{\star}
\rightarrow L D$, we expect {\hdmagd} to yield faster convergence. As suggested by \Cref{fig:demo:d}, {\hdmagd} achieves faster convergence than {\agd}.
\begin{rem}
To mitigate the effect of regret, \Cref{alg:osnes} needs a warm start from vanilla {\agd}. However, experiments
  suggest that it is unnecessary in practice, and we leave an improved analysis to future work.
\end{rem}

\begin{rem}
For strongly convex problems, we can combine \Cref{thm:osnes} with a
  standard restart argument \cite{d2021acceleration,roulet2017sharpness} and achieve a similar trajectory-based linear
  convergence rate.
\end{rem}

\begin{figure}
  \centering
  \begin{subfigure}{0.4\textwidth}
    \centering
\includegraphics[height=0.18\textheight]{figs/demo_4.pdf}
    \caption{\hdmhb}
    \label{fig:demo:c}
  \end{subfigure}
  \begin{subfigure}{0.4\textwidth}
    \centering
\includegraphics[height=0.18\textheight]{figs/demo_3.pdf}
    \caption{\hdmagd}
    \label{fig:demo:d}
  \end{subfigure}
\caption{The convergence behavior of {\hdmhb} and {\hdmagd} on a toy quadratic problem. \Cref{fig:demo:c}: {\hdmhb}. \label{fig-demos}\Cref{fig:demo:d}: {\hdm} with Nesterov momentum.} 
  \label{fig:demo:all}
\end{figure}
\label{sec: exp}
\section{Benchmarks and Experiments}
In this section, we first present the public datasets related to CIR, and then provide experimental results and analyses of representative methods. 

\begin{table*}[h!]
\centering
\caption{\textbf{Statistics of datasets for composed image retrieval and its related tasks.}}
\label{tab:dataset_summary}
 \resizebox{12cm}{!}{
\begin{tabular}{l|c|c|c|c}
    \hline
    \textbf{Dataset} & \textbf{Data Type}& \textbf{Vision Scale} & \textbf{Triplet Scale} & \textbf{Triplet Construction} \\
    \hline \hline
    
    \multicolumn{5}{c}{\textit{\textcolor{gray}{Datasets for Composed Image Retrieval}}} \\
    
    % --- Human Annotated ---
    FashionIQ~\cite{wu2021fiq} & image+text &  77.6K &  30.1K & Human Annotation \\ 
    Shoes~\cite{berg2010automatic} & image+text &  14.7K &  10.7K & Human Annotation \\
    CIRR~\cite{liu2021CIRPLANT} & image+text &  21.6K &  36.5K & Human Annotation \\
    B2W~\cite{forbes2019b2w} & image+text &  3.5K &  16.1K & Human Annotation \\
    CIRCO~\cite{searle} & image+text &  120K  &  1.0K & Human Annotation \\
    GeneCIS~\cite{genecis} & image+text & 33.3K &  8.0K & Human Annotation \\
    % --- Template-base Generation ---
    Fashion200K~\cite{Han2017fashion} & image+text &  200K &  205K & Template-base Generation \\
    MIT-States~\cite{Phi2015discover} & image+text &  53K & - & Template-base Generation \\
    CSS~\cite{vo2019tirg} & image+text & - &  32K & Template-base Generation \\
    SynthTriplets18M~\cite{compodiff} & image+text & - &  18.8M & Template-base Generation \\
    % --- LLM-base Generation ---
    LaSCo~\cite{levy2024case} & image+text &  121.5K &  389.3K & LLM-base Generation \\

    \cdashline{1-5}
    \multicolumn{5}{c}{\textit{\textcolor{gray}{Datasets for Related Tasks of Composed Image Retrieval}}} \\

    % ========== Template-base Generation ==========
    Shopping100k~\cite{EMASL} & image+attributes &  101K &  1.1M & Template-base Generation \\
    WebVid-CoVR~\cite{ventura2024covr} & video+text &  130.8K &  1.6M & LLM-base Generation \\
    FS-COCO~\cite{chowdhury2022fscoco} & sketch+image+text &  10K &  10K & Template-base Generation \\
    SketchyCOCO~\cite{gao2020sketchcoco} & sketch+image+text &  14K &  14K & Template-base Generation \\
    CSTBIR~\cite{stnet} & sketch+image+text &  108K &  2M & Template-base Generation \\
    PATTERNCOM~\cite{psomas2024cir4rs} & image+text&  30K &  21K & Template-base Generation\\
    % ========== Human Annotated ==========
    Airplane, Tennis, and WHIRT~\cite{shf} & image+scene graph+text&  7.7K &  8.7K & Human Annotation \\
    Multi-turn FashionIQ~\cite{cfir2021} & image+text & 13.6K & 11.5K & Human Annotation \\
    % ========== LLM-base Generation ==========
    
    \hline
    \multicolumn{5}{l}{\small *Vision scale means the scale of images/sketch-image pairs/videos. The triplet scale in Multi-turn FashionIQ refers to the number of sessions.}
\end{tabular}
 }
\end{table*}


\subsection{Datasets.}
The statistics of datasets for the CIR and its related tasks are summarized in Table~\ref{tab:dataset_summary}. 


\textbf{FashionIQ.} 
The FashionIQ dataset~\cite{wu2021fiq} is a natural language-based interactive fashion retrieval dataset, crawled from \textit{Amazon.com}. It provides human-generated captions that distinguish similar pairs of garment images together. The fashion items within the dataset belong to three categories: dress, shirt, and top\&tee. It contains $\sim77.6$K images and $\sim30.1$K triplets, with $\sim46.6$K images and $\sim18$K triplets in the training set, $\sim15.5$K images and $\sim6$K triplets in the validation set, and $\sim15.5$K images and $\sim6$K triplets in the test set. While being a challenge dataset, the test set is not publicly accessible.
Notably, FashionIQ comprises two evaluation protocols: the VAL-Split~\cite{chen2020val} and the Original-Split~\cite{wu2021fiq}. The VAL-Split is introduced by the early-stage CIR study, which constructs the candidate image set for testing based on the union of the reference images and target images in all the triplets of the validation set. The Original-Split has recently been adopted, and it directly uses the original candidate image set provided by the FashionIQ dataset for testing. 

% is collected by selecting visually similar images based on product titles using TF-IDF scores. Around $10,000$ target images from three fashion categories are chosen, and two captions are gathered for each image pair via crowdsourcing. Participants provided natural language descriptions comparing the target to the reference image. The FashionIQ dataset contains three categories: dress, shirt, and top\&tee. 



\textbf{Shoes.}
The Shoes dataset~\cite{berg2010automatic} is originally collected from \textit{like.com} for the attribute discovery task and further developed by~\cite{guo2018dialog} with relative caption annotations for dialog-based interactive retrieval. The annotations are gathered via human annotation using an interactive interface, which allows for fine-grained attribute descriptions. The dataset includes categories such as boots, sneakers, high heels, clogs, pumps, rain boots, flats, stilettos, wedding shoes, and athletic shoes. Overall, the dataset comprises $\sim14.7$K images and $\sim10.7$K triplets, with $10$K images and $\sim9$K triplets for training, and $\sim4.7$K images and $\sim1.7$K triplets for testing. 

% \textcolor{red}{$21,552$}
\textbf{CIRR.} The CIRR dataset, introduced by Liu~\textit{et al.}~\cite{liu2021CIRPLANT}, is an open-domain dataset constructed using  $\sim21.5$K images sourced from the natural language reasoning dataset $\text{NLVR}^2$~\cite{suhr2019nvlr}. CIRR comprises $\sim36.5$K triplets divided into training, validation, and test sets with an allocation ratio of $8:1:1$. To alleviate the false negative issue, CIRR first clusters similar images into subsets based on their visual similarity before the reference-target image pairs construction. Then, during the subsequent process of annotating modification text for reference-target image pairs, the semantics of the modification text must differentiate the target image from other similar images within the same subset. Specifically, given that each subset contains samples with a high visual similarity to the target image, testing retrieval on this subset places greater demands on the model's discriminative ability.


\textbf{B2W.}
The Birds-to-Words (B2W) dataset~\cite{forbes2019b2w} consists of images of birds sourced from \textit{iNaturalist}, accompanied by paragraphs written by humans to describe the differences between pairs of images. The dataset contains approximately $\sim3.5$K images and $\sim16.1$K triplets. Notably, each text description is relatively detailed, with an average length of $31.38$ words, providing rich insights into the subtle variations across bird images. 


\textbf{CIRCO.} The CIRCO dataset~\cite{searle} is an open-domain dataset developed from the COCO 2017 unlabeled set~\cite{lin2014coco} to address the false negative issues prevalent in existing datasets. Unlike typical CIR datasets, CIRCO includes multiple target images per query, thus significantly reducing the occurrence of false negatives and establishing it as the first CIR dataset with multiple ground-truth target images. Given that CIRCO is specifically designed for evaluating zero-shot cross-image retrieval models, it comprises $1,020$ queries that are partitioned into a validation set and a test set. Specifically, $220$ queries are allocated for validation purposes, while the remaining $800$ for testing. Each query includes a reference image, modification text, and an average of $4.53$ ground truth target images. Utilizing all $120$K images of COCO as the index set, CIRCO provides a vastly larger number of distractors compared to the $2$K images in the CIRR test set.

\textbf{GeneCIS.} 
GeneCIS~\cite{genecis} is an open-domain CIR dataset that serves as a benchmark for evaluating conditional similarity tasks. This dataset includes four subsets: Focus Attribute, Change Attribute, Focus Object, and Change Object, representing four different tasks. Among these, Focus Object and Change Object are constructed based on the COCO~\cite{lin2014coco} dataset, while Focus Attribute and Change Attribute are constructed based on the VAW~\cite{pham2021learning} dataset. Unlike other open-domain datasets, CIRR and CIRCO, which provide modification text, GeneCIS provides a single object name or attribute as the retrieval condition. To reduce the impact of false negatives, a gallery was selected for each triplet as the retrieval candidate set, with gallery sizes ranging from 10 to 15 images. This dataset consists of $\sim8$K triplets.

\textbf{Fashion200K.} The Fashion200K dataset, collected by Han~\textit{et al.}~\cite{Han2017fashion}, comprises  $\sim200$K clothing images, which are categorized into five types: dress, top, pants, skirt, and jacket. Each image comes with a compact attribute-like product description, such as ``black biker jacket''. The dataset is divided into three parts: $\sim172$K images for the training set, $\sim12$K images for the validation set, and $\sim25$K images for the test set. To construct triplets suitable for the CIR task, existing works~\cite{vo2019tirg, wen2021clvcnet} first create image pairs by identifying only one-word differences in their descriptions. The modification text is generated using templates that incorporate the differing words, such as ``replace red with green.'' Based on this construction method, there are $\sim172$K triplets available for training and $\sim33$K triplets for evaluation. 

\textbf{MIT-States.} The MIT-States dataset~\cite{Phi2015discover} features a diverse collection of objects, scenes, and materials in various transformed states. It contains $\sim53$K images, in which each image is tagged with an adjective or state label and a noun or object label (\textit{e.g.}, ``new camera'' or ``cooked beef''). The dataset encompasses $245$ nouns and $115$ adjectives, with each noun being modified by roughly $9$ different adjectives on average. Following~\cite{vo2019tirg}, image pairs that share the same object labels but with different attributes or states can be selected as the reference and target images, and the different attributes or states serve as the modification text. To evaluate the model's capacity for handling unseen objects, $49$ nouns are used for the test and the rest is for training.

\textbf{CSS.} The CSS dataset~\cite{vo2019tirg} is constructed using the CLEVR toolkit~\cite{johnson2017clevr} to generate synthesized images in a $3$-by-$3$ grid scene, showcasing objects with variations in Color, Shape, and Size. Each image is available in both a simplified $2$D blob version and a detailed $3$D rendered version. The dataset comprises $\sim16$K queries for training and $\sim16$K queries for test. Each query is of a reference image ($2$D or $3$D) and a modification, and the target image. Notably, modification texts fall into three categories: adding, removing, or changing object attributes. Examples include ``add red cube'' or ``remove yellow sphere''.  


\textbf{SynthTriplets18M.} SynthTriplets18M~\cite{compodiff} is a large-scale dataset specifically designed for the CIR task. Distinct from datasets relying on human annotation, SynthTriplets18M uses diffusion models to automatically create $\sim18$M triplets consisting of reference images, modification instructions, and target images. Adopting the approach of Instruct Pix2Pix~\cite{brooks2023pix}, the dataset initially creates caption triplets by modifying reference captions with specific instructions and then converts these triplets into image-based triplets using diffusion models, resulting in highly diverse and realistic data. The dataset encompasses a broad spectrum of keywords, enhancing its suitability for open-domain CIR tasks, and provides rich textual prompts such as ``replace ${source} with ${target}.'' This automated generation process enables the creation of rare and diverse image triplets that might not commonly appear in reality.

\textbf{LaSCo.} LaSCo~\cite{levy2024case} is a large-scale, open-domain dataset consisting of natural images, specifically designed for the CIR task. Created with minimal human effort by leveraging the VQA $2.0$ dataset~\cite{goyal2017vqa}, LaSCo leverages ``complementary'' image pairs, which are similar images that yield different answers to the same question, and transforms these question-answer pairs into valid transition texts using GPT-$3$. By exploiting transition symmetry, the dataset has amassed $\sim121.5$K images and $\sim389.3$K image pairs, which are then organized into triplets.


% \textcolor{red}{$101,021$}  \textcolor{red}{$954,091$} \textcolor{red}{$118,317$}
\textbf{Shopping100k.} The Shopping100k dataset~\cite{EMASL} consists of $\sim101.0$K pure clothing images that are characterized by $12$ attributes with $151$ possible attribute values. Following~\cite{adde}, pairs of images that differ by only one or two attributes are utilized as reference and target images. These differing attributes serve as modification attributes, facilitating the construction of triplet data for attribute-based CIR tasks. This approach results in $\sim954.1$K training triplets and $\sim118.3$K testing triplets, forming a robust foundation for developing and evaluating CIR models focused on nuanced fashion attributes. 


\textbf{WebVid-CoVR.}
WebVid-CoVR~\cite{ventura2024covr} is a large-scale dataset designed for CoVR, containing $\sim1.6$M triplets. Each video in the dataset averages $16.8$ seconds in duration, and the modification texts consist of roughly $4.8$ words. With each target video linked to about $12.7$ triplets, the dataset offers rich contextual variations essential for effectively training CoVR models. To ensure robust evaluation, they further introduce a meticulously curated test set known as WebVid-CoVR-Test~\cite{ventura2024covr}, which is manually annotated and consists of $\sim2.5$K triplets. 



% \textbf{ITCPR.}
% The ITCPR Dataset~\cite{liu2024word} is a carefully annotated dataset for Composed Person Retrieval (CPR), designed to evaluate models that retrieve individuals based on both visual and textual information. It includes $20,510$ images and $2,225$ annotated triplets. These triplets consist of reference images, relative captions, and target images, focusing on identifying the same person in different outfits or scenes. ITCPR supports Zero-Shot CPR (ZS-CPR), allowing model evaluation without expensive triplet annotations, making it a key resource for advancing CPR research. 

% FGIR


\textbf{FS-COCO.} FS-COCO~\cite{chowdhury2022fscoco} is a large-scale dataset of freehand scene sketches paired with textual descriptions and corresponding images, designed to advance research in fine-grained scene sketch understanding. It consists of $\sim10$K unique sketches drawn by non-experts, with $\sim7$K for training and $\sim3$K for testing, each matched with a photo from the MS-COCO dataset~\cite{lin2014microsoft} and a descriptive caption. 
FS-COCO includes over $90$ object categories from the COCO-stuff~\cite{caesar2018coco} and provides temporal order information of strokes, which enables detailed studies on scene abstraction and the salience of early vs. late strokes. This dataset serves as a benchmark for fine-grained image retrieval, sketch-based captioning, and understanding the complementary information between sketches and text.

\textbf{SketchyCOCO.} SketchyCOCO~\cite{gao2020sketchcoco} is a large-scale composite dataset tailored for the task of automatic image generation from scene-level freehand sketches. Built on the COCO-stuff dataset~\cite{caesar2018coco}, it includes $\sim14$K unique scene-level sketches paired with corresponding images and textual descriptions, organized into $\sim14$K  sketch-text-image triplets, with $80$\% for training and the remaining $20$\% for testing. Additionally, the dataset includes $\sim20$K triplet examples of foreground sketches, images, and edge maps covering $14$ classes, along with $\sim27$K background sketch-image pairs covering $3$ classes. This layered structure facilitates detailed studies in scene-level sketch-based generation and provides five-tuple data for comprehensive training in both foreground and background synthesis tasks.


\textbf{CSTBIR.} The Composite Sketch+Text Based Image Retrieval (CSTBIR) dataset~\cite{stnet} is a multimodal dataset specifically designed for image retrieval using sketches and partial text descriptions. It includes $\sim108$K natural scene images and $\sim2$M annotated triplets, each containing a reference sketch, a partial text description, and a target image. The natural images and text descriptions are sourced from the Visual Genome dataset~\cite{krishna2017visual}, while the sketches are hand-drawn from the Quick, Draw! dataset~\cite{HaE18}. By intersecting object categories between Visual Genome and Quick, Draw!, the dataset contains $258$ intersecting object classes and is divided into training, validation, and testing sets aligned with Visual Genome’s splits. The training set comprises $\sim97$K images, $\sim484$K sketches, and $\sim1.89$M queries. Additionally, CSTBIR includes three distinct test sets: Test-$1$K, Test-$5$K, and an Open-Category set. Among them, only the Open-Category test set is designed to evaluate model performance on novel object categories not present during training, which contains $70$ novel object categories (of which $50$ are “difficult-to-name”) and corresponding sketches. 



\textbf{PATTERNCOM.} PATTERNCOM~\cite{psomas2024cir4rs} is a new benchmark designed for evaluating remote sensing CIR methods, based on the PatternNet dataset~\cite{zhou2018patternnet}, which is a large-scale, high-resolution remote sensing image collection. PATTERNCOM focuses on selected classes from PatternNet, incorporating query images and corresponding text descriptions that define relevant attributes for each class. For example, the ``swimming pools'' category includes text queries specifying shapes such as ``rectangular,'' ``oval,'' and ``kidney-shaped.'' The dataset encompasses six attributes, with each attribute linked to up to four different classes and two to five values per class. Overall, PATTERNCOM contains over $21$K queries, with positive matches ranging from $2$ to $1,345$ per query. 


\textbf{Airplane, Tennis, and WHIRT.} The Airplane, Tennis, and WHIRT datasets~\cite{shf} are curated for remote sensing CIR. 
Each dataset is organized in terms of quintets, consisting of a reference RS image and its scene graph, a target RS image and its scene graph, and a pair of modifier sentences. The Airplane dataset comprises $1,600$ remote sensing images from UCM~\cite{yang2010bag}, PatternNet~\cite{zhou2018patternnet}, and NWPU-RESISC45~\cite{cheng2017remote}, along with $3,461$ pairs of modifier sentences that describe differences in airplane attributes and spatial relationships between airplanes and other objects. The Tennis dataset includes $1,200$ images featuring tennis courts and $1,924$ manually annotated modifier sentence pairs. It emphasizes target and non-target spatial relationships while ignoring relationships between non-target objects. The WHIRT dataset is the most extensive, consisting of $4,940$ images from WHDLD~\cite{shao2018performance} and $3,344$ reference-target image pairs. Its scene graphs provide comprehensive details on object attributes and spatial relationships, accommodating complex remote sensing scenarios. Each dataset features high-quality annotations by domain experts, serving as robust benchmarks for advanced retrieval tasks in intricate remote sensing environments. 

% \textcolor{red}{$11,506$}
\textbf{Multi-turn FashionIQ.} Multi-turn FashionIQ~\cite{cfir2021} is an extension of the original FashionIQ dataset~\cite{wu2021fiq}, designed to model user interactions in a multi-turn setting for fashion product retrieval. It contains $\sim11.5$K sessions structured as multi-turn interaction, across three clothing types: dress, shirt, and top\&tee. Each session consists of multiple reference images, modification texts, and a target image, with turns ranging from $3$ to $5$. The sessions are constructed by linking single-turn triplets from FashionIQ, matching the target image of one triplet to the reference image of another, thus forming coherent multi-turn sequences. Additionally, the dataset expands the original attribute data to ensure comprehensive coverage, associating each target image with attributes like texture, fabric, shape, part, and style. 

\subsection{Metric.}
\textbf{Recall.} Recall is a widely used metric in CIR to evaluate the effectiveness of retrieval systems. It is often denoted as Recall@$k$ (R@$k$), measuring the proportion of queries for which the correct target image is retrieved within the top $k$ results. Recall@$k$ can be defined by the following formula:
\begin{equation}
    {\text{Recall}@k} = \frac{1}{Q} \sum_{q=1}^{Q} \frac{| \mathcal{R}_q \cap \mathcal{D}_q^k |}{| \mathcal{R}_q |} ,
\end{equation}
where $Q$ denotes the total number of queries, $\mathcal{R}_q$ is the set of all relevant target images for each query $q$, $\mathcal{D}_q^k$ is the set of the top $k$ retrieved items for query $q$, $| \mathcal{R}_q \cap \mathcal{D}_q^k|$ represents the number of target images found in the top $k$ results, $| \mathcal{R}_q |$ represents the total number of target images for query $q$. Notably, in most existing CIR datasets, each query typically corresponds to only one target image, \textit{i.e.}, $| \mathcal{R}_q |=1$, except CIRCO. Additionally, CIRR~\cite{liu2021CIRPLANT} further defines Recall\(_{subset}@k\) to assess how frequently the desired target image appears in the top $k$ results when considering only a specific subset of images. 

\textbf{Mean Average Precision.}
Mean Average Precision at $k$ (mAP@$k$) is a crucial metric for evaluating retrieval systems, particularly in cases where there are multiple relevant items. Initially employed in CIRCO~\cite{searle}, this metric integrates precision across various ranks to yield a single, averaged indicator of the system's effectiveness in retrieving relevant items. The formula for mAP@$k$ is given by:
\begin{equation}
    \text{mAP}@k = \frac{1}{Q} \sum_{q=1}^{Q} \frac{1}{\min(k, \mathcal{R}_q)} \sum_{k=1}^{k} P@k \cdot \text{rel}@k,
\end{equation}
where P@$k$ is the precision at rank $k$, rel@$k$ is a relevance function. The relevance function is an indicator function that equals $1$ if the image at rank $k$ is labeled as positive and equals $0$ otherwise.


\begin{table*}
    \scriptsize
    \centering
    \caption{
   Performance comparison among supervised composed image retrieval models on FashionIQ (VAL split).}
   % \resizebox{14.5cm}{!}{
    \begin{tabular}{l|cc|cc|cc|cc|c}
    \hline 
    \multirow{2}{*}{Model} & \multicolumn{2}{c|}{Dresses} & \multicolumn{2}{c|}{Shirts} & \multicolumn{2}{c|} {Tops\&Tees} & \multicolumn{2}{c|}{Average} & \multirow{2}{*}{Avg.} \\ \cline{2-9}
    &  R@$10$ & R@$50$ & R@$10$ & R@$50$ & R@$10$ & R@$50$ & R@$10$ & R@$50$ & \\
    \hline \hline
    
    % \rowcolor{gray!5} 
    \multicolumn{10}{c}{\textit{\textcolor{gray}{Traditional Encoder-based Methods}}} \\
    
    LSC4TCIR~\cite{chawla2021lsc4cir} \footnotesize{\textcolor{gray}{(CVPRW'21)}} & $19.33 $ & $43.52$ & $14.47$ & $35.47$ & $19.73$ & $44.56$ & $17.84$ & $41.18$ &$29.51$ \\
    CIRPLANT~\cite{liu2021CIRPLANT} \footnotesize{\textcolor{gray}{(ICCV'21)}} & $17.45$ & $40.41$ & $17.53$ & $38.81$ & $21.64$ & $45.38$ & $18.87$ & $41.53$ & $30.20$ \\
    VAL~\cite{chen2020val} \footnotesize{\textcolor{gray}{(CVPR'20)}} & $21.12$ & $42.19$ & $21.03$ & $43.44$ & $25.64$ & $49.49$ & $22.60$ & $45.04$ & $33.82$ \\
    % VAL (Lvv + Lvs)~\cite{} \footnotesize{\textcolor{gray}{(CVPR'20)}} & $21.47$ & $43.83$ & $21.03$ & $42.75$ & $26.71$ & $51.81$ & $23.07$ & $46.13$ & $34.60$ \\
    % ARTEMIS~\cite{} \footnotesize{\textcolor{gray}{(ICLR'22)}} & $25.23$ & $48.64$ & $20.35$ & $43.67$ & $23.36$ & $46.97$ & $22.98$ & $46.43$ & $34.70$ \\
    DATIR~\cite{gu2021datir} \footnotesize{\textcolor{gray}{(MM'21)}} & $21.90$ & $43.80$ & $21.90$ & $43.70$ & $27.20$ & $51.60$ & $23.70$ & $46.40$ & $35.00$ \\
    % ARTEMIS~\cite{} \footnotesize{\textcolor{gray}{(ICLR'22)}} & $24.84$ & $49.00$ & $20.40$ & $43.22$ & $26.63$ & $47.39$ & $23.96$ & $46.54$ & $35.25$ \\
    % VAL (GloVe)~\cite{} \footnotesize{\textcolor{gray}{(CVPR'20)}} & $22.53$ & $44.00$ & $22.38$ & $44.15$ & $27.53$ & $51.68$ & $24.15$ & $46.61$ & $35.38$ \\
    SynthTripletGAN~\cite{tautkute2021Synth} \footnotesize{\textcolor{gray}{(Arxiv'21)}} & $22.60$ & $45.10$ & $20.50$ & $44.08$ & $28.01$ & $52.10$ & $23.70$ & $47.09$ & $35.40$ \\
    VAL+JPM~\cite{yang2021jpm} \footnotesize{\textcolor{gray}{(MM'21)}} & $21.38$ & $45.15$ & $22.81$ & $45.18$ & $27.78$ & $51.70$ & $23.99$ & $47.34$ & $35.67$ \\
    % ARTEMIS~\cite{} \footnotesize{\textcolor{gray}{(ICLR'22)}} & $27.34$ & $51.71$ & $21.05$ & $44.18$ & $24.91$ & $49.87$ & $24.43$ & $48.59$ & $36.51$ \\
    MCR~\cite{zhang2021mcr} \footnotesize{\textcolor{gray}{(MM'21)}} & $26.20$ & $51.20$ & $22.40$ & $46.00$ & $29.70$ & $56.40$ & $26.10$ & $51.20$ & $38.65$ \\
    CoSMo~\cite{lee2021cosmo} \footnotesize{\textcolor{gray}{(CVPR'21)}} & $25.64$ & $50.30$ & $24.90$ & $49.18$ & $29.21$ & $57.46$ & $26.58$ & $52.31$ & $39.45$ \\
    % DCNet~\cite{kim2021dcnet} \footnotesize{\textcolor{gray}{(AAAI'21)}} & $28.95$ & $56.07$ & $23.95$ & $47.30$ & $30.44$ & $58.29$ & $27.78$ & $53.89$ & $40.83$ \\
    ACNet~\cite{li2023acnet} \footnotesize{\textcolor{gray}{(ICME'23)}} & $29.20$ & $55.68$ & $25.12$ & $47.75$ & $30.96$ & $57.72$ & $28.43$ & $53.72$ & $41.07$ \\
    NSFSE~\cite{wang2024NSFSE} \footnotesize{\textcolor{gray}{(TMM'24)}} & $31.12$ & $55.73$ & $24.58$ & $45.85$ & $31.93$ & $58.37$ & $29.21$ & $53.32$ & $41.26$ \\
    SAC~\cite{jandial2022sac} \footnotesize{\textcolor{gray}{(WACV'22)}} & $26.52$ & $51.01$ & $28.02$ & $51.86$ & $32.70$ & $61.23$ & $29.08$ & $54.70$ & $41.89$ \\
    % ARTEMIS~\cite{delmas2022artemis} \footnotesize{\textcolor{gray}{(ICLR'22)}} & $27.16$ & $52.40$ & $21.78$ & $43.64$ & $29.20$ & $54.83$ & $26.05$ & $50.29$ & $38.17$ \\
    ARTEMIS~\cite{delmas2022artemis} \footnotesize{\textcolor{gray}{(ICLR'22)}} & $29.04$ & $53.55$ & $25.56$ & $50.86$ & $33.58$ & $60.48$ & $29.39$ & $54.96$ & $42.18$ \\
    EER~\cite{zhang2022eer} \footnotesize{\textcolor{gray}{(TIP'22)}} & $30.02$ & $55.44$ & $25.32$ & $49.87$ & $33.20$ & $60.34$ & $29.51$ & $55.22$ & $42.36$ \\
    MANME~\cite{li2023manme} \footnotesize{\textcolor{gray}{(TCSVT'24)}} & $31.26$ & $57.66$ & $26.37$ & $47.94$ & $32.33$ & $59.31$ & $29.95$ & $54.90$ & $42.48$ \\
    MLCLSAP~\cite{zhang2023MLCLSAP} \footnotesize{\textcolor{gray}{(TMM'23)}} & $30.74$ & $55.92$ & $26.05$ & $50.64$ & $34.42$ & $61.14$ & $30.40$ & $55.90$ & $43.15$ \\
    MCEM~\cite{zhang2024mcem} \footnotesize{\textcolor{gray}{(TIP'24)}} & $32.11$ & $59.21$ & $27.28$ & $52.01$ & $33.96$ & $62.30$ & $31.12$ & $57.84$ & $44.48$ \\
    CLVC-Net~\cite{wen2021clvcnet} \footnotesize{\textcolor{gray}{(SIGIR'21)}} & $29.85$ & $56.47$ & $28.75$ & $54.76$ & $33.50$ & $64.00$ & $30.70$ & $58.41$ & $44.56$ \\
    CRN-base~\cite{yang2023crn} \footnotesize{\textcolor{gray}{(TIP'23)}} & $30.34$ & $57.61$ & $29.83$ & $55.54$ & $33.91$ & $64.04$ & $31.36$ & $59.06$ & $45.21$ \\
    AlRet-small~\cite{xu2024alret} \footnotesize{\textcolor{gray}{(TMM'24)}} & $30.19$ & $58.80$ & $29.39$ & $55.69$ & $37.66$ & $64.97$ & $32.41$ & $59.82$ & $46.12$ \\
    CRN-large~\cite{yang2023crn} \footnotesize{\textcolor{gray}{(TIP'23)}} & $32.67$ & $59.30$ & $30.27$ & $56.97$ & $37.74$ & $65.94$ & $33.56$ & $60.74$ & $47.15$ \\
    AMC~\cite{zhu2023amc} \footnotesize{\textcolor{gray}{(TOMM'23)}} & $31.73$ & $59.25$ & $30.67$ & $59.08$ & $36.21$ & $66.60$ & $32.87$ & $61.64$ & $47.25$ \\
    CLVC-Net+MU~\cite{chen2022mu} \footnotesize{\textcolor{gray}{(ICLR’24)}} & $31.25$ & $58.35$ & $31.69$ & $60.65$ & $39.82$ & $71.07$ & $34.25$ & $63.36$ & $48.81$ \\
    ComqueryFormer~\cite{xu2023ComqueryFormer} \footnotesize{\textcolor{gray}{(TMM'23)}} & $33.86$ & $61.08$ & $35.57$ & $62.19$ & $42.07$ & $69.30$ & $37.17$ & $64.19$ & $50.68$ \\
    CMAP~\cite{li2024cmap} \footnotesize{\textcolor{gray}{(TOMM'24)}} & $36.44$ & $64.25$ & $34.83$ & $60.06$ & $41.79$ & $69.12$ & $37.69$ & $64.48$ & $51.08$ \\
    Css-Net~\cite{zhang2024cssnet} \footnotesize{\textcolor{gray}{(KBS'24)}} & $33.65$ & $63.16$ & $35.96$ & $61.96$ & $42.65$ & $70.70$ & $37.42$ & $65.27$ & $51.35$ \\
    SDFN~\cite{wu2024sdfn} \footnotesize{\textcolor{gray}{(ICASSP'24)}} & $37.33$ & $64.45$ & $37.10$ & $62.56$ & $43.35$ & $72.11$ & $39.26$ & $66.37$ & $52.81$ \\

    
    \cdashline{1-10}

    % \rowcolor{gray!5} % 设置该行的背景色为浅灰色
    \multicolumn{10}{c}{\textit{\textcolor{gray}{VLP Encoder-based Methods}}} \\
    
    CLIP-ProbCR~\cite{li2024clip} \footnotesize{\textcolor{gray}{(ICMR'24)}} & $30.71$ & $56.55$ & $28.41$ & $52.04$ & $35.03$ & $61.11$ & $31.38$ & $56.57$ & $43.98$ \\
    FashionViL~\cite{han2022fashionvil} \footnotesize{\textcolor{gray}{(ECCV'22)}} & $33.47$ & $59.94$ & $25.17$ & $50.39$ & $34.98$ & $60.79$ & $31.21$ & $57.04$ & $44.12$ \\
    FashionVLP~\cite{goenka2022fashionvlp} \footnotesize{\textcolor{gray}{(CVPR'22)}} & $32.42$ & $60.29$ & $31.89$ & $58.44$ & $38.51$ & $68.79$ & $34.27$ & $62.51$ & $48.39$ \\
    DWC~\cite{huang2024dwc} \footnotesize{\textcolor{gray}{(AAAI'24)}} & $32.67$ & $57.96$ & $35.53$ & $60.11$ & $40.13$ & $66.09$ & $36.11$ & $61.39$ & $48.75$ \\
    SyncMask~\cite{song2024syncmask} \footnotesize{\textcolor{gray}{(CVPR'24)}} & $33.76$ & $61.23$ & $35.82$ & $62.12$ & $44.82$ & $72.06$ & $38.13$ & $65.14$ & $51.64$ \\
    PL4CIR-base~\cite{zhao2022PL4CIR} \footnotesize{\textcolor{gray}{(SIGIR'22)}} & $33.22$ & $59.99$ & $46.17$ & $68.79$ & $46.46$ & $73.84$ & $41.95$ & $67.54$ & $54.76$ \\
    AlRet-big~\cite{xu2024alret} \footnotesize{\textcolor{gray}{(TMM'24)}} & $40.23$ & $65.89$ & $47.15$ & $70.88$ & $51.05$ & $75.78$ & $46.14$ & $70.85$ & $58.50$ \\
    PL4CIR-large~\cite{zhao2022PL4CIR} \footnotesize{\textcolor{gray}{(SIGIR'22)}} & $38.18$ & $64.50$ & $48.63$ & $71.54$ & $52.32$ & $76.90$ & $46.38$ & $70.98$ & $58.68$ \\
    TG-CIR~\cite{wen2023tgcir} \footnotesize{\textcolor{gray}{(MM'23)}} & $45.22$ & $69.66$ & $52.60$ & $72.52$ & $56.14$ & $77.10$ & $51.32$ & $73.09$ & $62.21$ \\
    FashionERN-small~\cite{chen2024fashionern} \footnotesize{\textcolor{gray}{(AAAI'24)}} & $43.93$ & $68.77$ & $52.70$ & $75.07$ & $56.09$ & $78.38$ & $50.91$ & $74.07$ & $62.49$ \\
    SPIRIT~\cite{chen2024spirit} \footnotesize{\textcolor{gray}{(TOMM'24)}} & $43.83$ & $68.86$ & $52.50$ & $74.19$ & $56.60$ & $79.25$ & $50.98$ & $74.10$ & $62.54$ \\
    LIMN~\cite{wen2023limn} \footnotesize{\textcolor{gray}{(TPAMI'24)}} & $50.72$ & $74.52$ & $56.08$ & $77.09$ & $60.94$ & $81.85$ & $55.91$ & $77.82$ & $66.87$ \\
    LIMN+~\cite{wen2023limn} \footnotesize{\textcolor{gray}{(TPAMI'24)}} & $52.11$ & $75.21$ & $57.51$ & $77.92$ & $62.67$ & $82.66$ & $57.43$ & $78.60$ & $68.01$ \\
    DQU-CIR~\cite{wen2024dqu} \footnotesize{\textcolor{gray}{(SIGIR'24)}} & $57.63$ & $78.56$ & $62.14$ & $80.38$ & $66.15$ & $85.73$ & $61.97$ & $81.56$ & $71.77$ \\
    
    \hline
    \end{tabular}
    \label{tab:supervised_CIR_exp_fashioniq_val}
    % }
\end{table*}

\begin{table*}
    \scriptsize
    \centering
    \caption{
   Performance comparison among supervised composed image retrieval models on FashionIQ (original split).}
   % \resizebox{14.5cm}{!}{
    \begin{tabular}{l|cc|cc|cc|cc|c}
    \hline 
    \multirow{2}{*}{Model} & \multicolumn{2}{c|}{Dresses} & \multicolumn{2}{c|}{Shirts} & \multicolumn{2}{c|} {Tops\&Tees} & \multicolumn{2}{c|}{Average} & \multirow{2}{*}{Avg.} \\ \cline{2-9}
    &  R@$10$ & R@$50$ & R@$10$ & R@$50$ & R@$10$ & R@$50$ & R@$10$ & R@$50$ & \\
    \hline \hline
    
    % \rowcolor{gray!5} 
    \multicolumn{10}{c}{\textit{\textcolor{gray}{Traditional Encoder-based Methods}}} \\
    
    JVSM~\cite{chen2020jvsm} \footnotesize{\textcolor{gray}{(ECCV'20)}} & $10.70 $ & $25.90$ & $12.00$ & $27.10$ & $13.00$ & $26.90$ & $11.90$ & $26.63$ &$19.27$ \\
    TIRG~\cite{vo2019tirg} \footnotesize{\textcolor{gray}{(CVPR'19)}} & $14.13$ & $34.61$ & $13.10$ & $30.91$ & $14.79$ & $34.37$ & $14.01$ & $33.30$ &$23.66$ \\
    % CosMo~\cite{lee2021cosmo} \small{\textcolor{gray}{(CVPR'21)}}  & $21.39$ & $44.45$ & $16.90$ & $37.49$ & $21.32$ & $46.02$ & $19.87$ & $42.62$ & $31.25 $  \\
    % ARTEMIS~\cite{delmas2022artemis} \footnotesize{\textcolor{gray}{(ICLR'22)}}  & $25.68$ & $51.05$ & $21.57$ & $44.13$ &$ 28.59 $& $ 55.06 $& $ 25.28 $& $50.08 $ & $ 37.68$ \\
    AlRet-small~\cite{xu2024alret} \footnotesize{\textcolor{gray}{(TMM'24)}} & $27.34$ & $53.42$ & $21.30$ & $43.08$ & $29.07$ & $54.21$ & $25.90$ & $50.24$ &$38.07$ \\
    ARTEMIS~\cite{delmas2022artemis} \footnotesize{\textcolor{gray}{(ICLR'22)}} & $27.16$ & $52.40$ & $21.78$ & $43.64$ & $29.20$ & $54.83$ & $26.05$ & $50.29$ & $38.17$ \\
    FashionVLP~\cite{goenka2022fashionvlp} \small{\textcolor{gray}{(CVPR'22)}} & $26.77$ & $53.20$ & $22.67$ & $46.22$ & $28.51$ & $57.47$ & $25.98$ & $52.30$ & $39.14 $ \\
    NEUCORE~\cite{zhao2024neucore} \footnotesize{\textcolor{gray}{(NIPSW'23)}}& $27.00$ & $53.79$ & $22.84$ & $45.00$ & $29.63$ & $56.65$ & $26.45$ & $51.75$ & $39.15 $ \\
    PCaSM~\cite{zhang2023pcasm} \footnotesize{\textcolor{gray}{(ICME'23)}} & $25.12$ & $51.17$ & $24.09$ & $50.66$ & $28.00$ & $57.45$ & $25.74$ & $53.09$ &$39.42$ \\
    DCNet~\cite{kim2021dcnet} \footnotesize{\textcolor{gray}{(AAAI'21)}} & $28.95$ & $56.07$ & $23.95$ & $47.30$ & $30.44$ & $58.29$ & $27.78$ & $53.89$ &$40.83$ \\
    AACL~\cite{tian2023aacl} \footnotesize{\textcolor{gray}{(WACV'23)}} & $29.89$ & $55.85$ & $24.82$ & $48.85$ & $30.88$ & $56.85$ & $28.53$ & $53.85$ & $41.19$ \\
    DSCN~\cite{li2023dscn} \footnotesize{\textcolor{gray}{(ICMR'23)}} & $30.61$ & $56.67$ & $25.74$ & $48.48$ & $32.38$ & $59.13$ & $29.58$ & $54.76$ & $42.17$ \\
    ComqueryFormer~\cite{xu2023ComqueryFormer} \small{\textcolor{gray}{(TMM'23)}} & $28.85$ & $55.38$ & $25.64$ & $50.22$ & $33.61$ & $60.48$ & $29.37$ & $55.36$ & $42.36 $ \\
    
    \cdashline{1-10}

    % \rowcolor{gray!5} % 设置该行的背景色为浅灰色
    \multicolumn{10}{c}{\textit{\textcolor{gray}{VLP Encoder-based Methods}}} \\
    
    FaD-VLP~\cite{mirchandani2022fad} \footnotesize{\textcolor{gray}{(EMNLP'22)}} & $32.08$ & $57.96$ & $25.22$ & $49.71$ & $33.20$ & $60.84$ & $30.17$ & $56.17$ & {$43.17$} \\
    PL4CIR-base~\cite{zhao2022PL4CIR} \footnotesize{\textcolor{gray}{(SIGIR'22)}} & $29.00$ & $53.94$ & $35.43$ & $58.88$ & $39.16$ & $64.56$ & $34.53$ & $59.13$ & {$46.83$} \\
    Combiner~\cite{baldrati2022Combiner} \footnotesize{\textcolor{gray}{(CVPRW'22)}} & $31.63$ & $56.67$ & $36.36$ & $58.00$ & $38.19$ & $62.42$ & $35.39$ & $59.03$ & $47.21$ \\
    CaLa_CLIP4Cir~\cite{jiang2024cala} \footnotesize{\textcolor{gray}{(SIGIR'24)}} & $32.96$ & $56.82$ & $39.20$ & $60.13$ & $39.16$ & $63.83$ & $37.11$ & $60.26$ & {$ 48.68 $} \\
    CAFF~\cite{wan2024caff} \footnotesize{\textcolor{gray}{(CVPR'24)}} & $35.74$ & $59.85$ & $35.80$ & $61.94$ & $38.51$ & $68.34$ & $36.68$ & $63.38$ & $50.03$ \\
    CLIP4CIR~\cite{baldrati2022CLIP4CIR} \footnotesize{\textcolor{gray}{(CVPR'22)}} & $33.81$ & $59.40$ & $39.99$ & $60.45$ & $41.41$ & $65.37$ & $38.40$ & $61.74$ & {$50.07$} \\
    Combiner+MU~\cite{chen2022mu} \footnotesize{\textcolor{gray}{(ICLR'24)}} & $32.61$ & $61.34$ & $33.23$ & $62.55$ & $41.40$ & $72.51$ & $35.75$ & $65.47$ & {$50.61$} \\
    AlRet-big~\cite{xu2024alret} \footnotesize{\textcolor{gray}{(TMM'24)}} & $35.75$ & $60.56$ & $37.02$ & $60.55$ & $42.25$ & $67.52$ & $38.34$ & $62.88$ & $50.61$ \\
    PL4CIR-large~\cite{zhao2022PL4CIR} \footnotesize{\textcolor{gray}{(SIGIR'22)}} & $33.60$ & $58.90$ & $39.45$ & $61.78$ & $43.96$ & $68.33$ & $39.00$ & $63.00$ & $51.01$ \\
    SSN~\cite{yang2024ssn} \footnotesize{\textcolor{gray}{(AAAI'24)}} & $34.36$ & $60.78$ & $38.13$ & $61.83$ & $44.26$ & $69.05$ & $38.92$ & $63.89$ & $51.40$ \\
    IUDC~\cite{ge2024iudc} \footnotesize{\textcolor{gray}{(TOIS'24)}} & $35.22$ & $61.90$ & $41.86$ & $63.52$ & $42.19$ & $69.23$ & $39.76$ & $64.88$ & $52.32$ \\
    CLIP4CIR2~\cite{baldrati2023CLIP4CIR2} \footnotesize{\textcolor{gray}{(TOMM'23)}} & $37.67$ & $63.16$ & $39.87$ & $60.84$ & $44.88$ & $68.59$ & $40.80$ & $64.20$ & $52.50$ \\
    CLIP-CD~\cite{lin2023clip_cd} \footnotesize{\textcolor{gray}{(AJCAI'23)}} & $37.68$ & $62.62$ & $42.44$ & $63.74$ & $45.33$ & $67.72$ & $41.82$ & $64.69$ & $53.26$ \\
    IUDC+AUG~\cite{ge2024iudc} \footnotesize{\textcolor{gray}{(TOIS'24)}}& $36.60$ & $63.45$ & $42.61$ & $64.86$ & $43.52$ & $71.15$ & $40.91$ & $66.49$ & $53.70$ \\
    BLIP4CIR~\cite{liu2024blip4cir} \footnotesize{\textcolor{gray}{(WACV'24)}} & $40.65$ & $66.34$ & $40.38$ & $64.13$ & $46.86$ & $69.91$ & $42.63$ & $66.79$ & $54.71$ \\
    SHAF~\cite{yan2024shaf} \footnotesize{\textcolor{gray}{(ICIC'24)}} & $41.15$ & $62.46$ & $43.03$ & $64.00$ & $49.52$ & $71.09$ & $44.57$ & $65.85$ & $55.21$ \\
    Ranking-aware~\cite{chen2023ranking} \footnotesize{\textcolor{gray}{(Arxiv'23)}} & $34.80$ & $60.22$ & $45.01$ & $69.06$ & $47.68$ & $74.85$ & $42.50$ & $68.04$ & $55.27$\\
    FAME-VIL(ST)~\cite{han2023fame} \footnotesize{\textcolor{gray}{(CVPR'23)}} & $37.78$ & $63.86$ & $45.63$ & $66.78$ & $47.22$ & $70.88$ & $43.54$ & $67.17$ & $55.36$ \\
    BLIP4CIR+Bi~\cite{liu2024blip4cir} \footnotesize{\textcolor{gray}{(WACV'24)}} & $42.09$ & $67.33$ & $41.76$ & $64.28$ & $46.61$ & $70.32$ & $43.49$ & $67.31$ & $55.40$ \\
    SPIRIT~\cite{chen2024spirit} \footnotesize{\textcolor{gray}{(TOMM'24)}} & $39.86$ & $64.30$ & $44.11$ & $65.60$ & $47.68$ & $71.70$ & $43.88$ & $67.20$ & $55.54$ \\
    FashionERN-small~\cite{chen2024fashionern} \footnotesize{\textcolor{gray}{(AAAI'24)}} & $38.52$ & $64.30$ & $45.00$ & $66.05$ & $48.80$ & $71.09$ & $44.11$ & $67.15$ & $55.63$ \\
    SADN~\cite{wang2024sadn} \footnotesize{\textcolor{gray}{(MM'24)}} & $40.01$ & $65.10$ & $43.67$ & $66.05$ & $48.04$ & $70.93$ & $43.91$ & $67.36$ & $55.63$ \\
    CaLa_BLIP2Cir~\cite{jiang2024cala} \footnotesize{\textcolor{gray}{(SIGIR'24)}} & $42.38$ & $66.08$ & $46.76$ & $68.16$ & $50.93$ & $73.42$ & $46.69$ & $69.22$ & $57.96$ \\
    FAME-VIL~\cite{han2023fame} \footnotesize{\textcolor{gray}{(CVPR'23)}} & $42.19$ & $67.38$ & $47.64$ & $68.79$ & $50.69$ & $73.07$ & $46.84$ & $69.75$ & $58.29$ \\
    CASE~\cite{levy2024case} \footnotesize{\textcolor{gray}{(AAAI'24)}} & $47.44$ & $69.36$ & $48.48$ & $70.23$ & $50.18$ & $72.24$ & $48.70$ & $70.61$ & $59.66$ \\
    Re-ranking_R100~\cite{liu2023rerank} \footnotesize{\textcolor{gray}{(TMLR'24)}} & $48.14$ & $71.34$ & $50.15$ & $71.25$ & $55.23$ & $76.80$ & $51.17$ & $73.13$ & $62.15$ \\
    FashionERN-big~\cite{chen2024fashionern} \footnotesize{\textcolor{gray}{(AAAI'24)}} & $50.32$ & $71.29$ & $50.15$ & $70.36$ & $56.40$ & $77.21$ & $52.26$ & $72.95$ & $62.62$ \\
    SPRC~\cite{xusentence2024sprc} \footnotesize{\textcolor{gray}{(ICLR'24)}} & $49.18$ & $72.43$ & $55.64$ & $73.89$ & $59.35$ & $78.58$ & $54.72$ & $74.97$ & $64.85$ \\
    DQU-CIR~\cite{wen2024dqu} \footnotesize{\textcolor{gray}{(SIGIR'24)}} & $51.90$ & $74.37$ & $53.57$ & $73.21$ & $58.48$ & $79.23$ & $54.65$ & $75.60$ & $65.13$ \\
    SPRC+VQA~\cite{feng2023vqa4cir} \footnotesize{\textcolor{gray}{(Arxiv'23)}}& $49.18$ & $73.06$ & $56.79$ & $74.52$ & $59.67$ & $79.30$ & $55.21$ & $75.62$ & $65.41$ \\
    SDQUR~\cite{xu2024SDQUR} \footnotesize{\textcolor{gray}{(TCSVT'24)}} & $49.93$ & $73.33$ & $56.87$ & $76.50$ & $59.66$ & $79.25$ & $55.49$ & $76.36$ & $65.93$ \\
    SPRC+SPN~\cite{feng2024spn} \footnotesize{\textcolor{gray}{(MM'24)}} & $50.57$ & $74.12$ & $57.70$ & $75.27$ & $60.84$ & $79.96$ & $56.37$ & $76.45$ & $66.41$ \\
    \hline
    \end{tabular}
    \label{tab:supervised_CIR_exp_fashioniq_ori}
    % }
\end{table*}


\begin{table*}
    \scriptsize
    \centering
    \caption{Performance comparison among supervised composed image retrieval models on Fashion200K and Shoes.}
    % \resizebox{14cm}{!}{
    \begin{tabular}{l|ccc|c|ccc|c}
    \hline 
    \multirow{2}{*}{Model} & \multicolumn{4}{c|}{Fashion200K} & \multicolumn{4}{c}{Shoes}  \\
    \cline{2-9}
    & R@$1$ & R@$10$ & R@$50$ & Avg. & R@$1$ & R@$10$ & R@$50$ & Avg.\\
    \hline \hline 
    
    \multicolumn{9}{c}{\textit{\textcolor{gray}{Traditional Encoder-based Methods}}} \\
    TIRG~\cite{vo2019tirg} \footnotesize{\textcolor{gray}{(CVPR'19)}} & $14.10$ & $42.50$ & $63.80$ &$40.13$ &$12.60$ &$45.45$ & $69.39$&$42.28$\\
    MGF~\cite{liu2021mgf} \footnotesize{\textcolor{gray}{(MMM'21)}} & $16.00$ & $44.60$ & $-$ &$-$ &$-$ &$-$ & $-$&$-$ \\
    TIRG+JPM(Tri)~\cite{yang2021jpm} \footnotesize{\textcolor{gray}{(MM'21)}} & $17.70$ & $44.70$ & $64.50$ &$42.30$ &$-$ &$-$ & $-$&$-$\\
    JAMMA~\cite{zhang2020jamma} \footnotesize{\textcolor{gray}{(MM'20)}} & $17.34$ & $45.28$ & $65.65$ &$42.76$ &$-$ &$-$ & $-$&$-$\\
    DCNet~\cite{kim2021dcnet} \footnotesize{\textcolor{gray}{(AAAI'21)}} & $-$ & $46.89$ & $67.56$ &$-$ & $-$ & $53.82$ & $79.33$ & $-$ \\
    TIS~\cite{zhang2022tis} \footnotesize{\textcolor{gray}{(TOMM'22)}} & $16.25$ & $44.14$ & $65.02$ &$41.80$ &$-$ &$-$ & $-$&$-$\\
    TIRG+JPM(MSE)~\cite{yang2021jpm} \footnotesize{\textcolor{gray}{(MM'21)}} & $19.80$ & $46.50$ & $66.60$ &$44.30$ &$-$ &$-$ & $-$&$-$\\
    LBF-small~\cite{hosseinzadeh2020lbf} \footnotesize{\textcolor{gray}{(CVPR'20)}} & $16.26$ & $46.90$ & $71.73$ &$44.96$ &$-$ &$-$ & $-$&$-$\\
    LBF-big~\cite{hosseinzadeh2020lbf} \footnotesize{\textcolor{gray}{(CVPR'20)}} & $17.78$ & $48.35$ & $68.50$ &$44.88$ &$-$ &$-$ & $-$&$-$\\
    MCR~\cite{zhang2021mcr} \footnotesize{\textcolor{gray}{(MM'21)}} & $18.24$ & $49.41$ & $69.37$ &$45.67$ & $17.85$ & $50.95$ & $77.24$ & $48.68$ \\
    SAC~\cite{jandial2022sac} \footnotesize{\textcolor{gray}{(WACV'22)}} &$-$ &$-$ & $-$&$-$& $18.50$ & $51.73$ & $77.28$ & $49.17$ \\
    VAL~\cite{chen2020val} \footnotesize{\textcolor{gray}{(CVPR'20)}} & $21.20$ & $49.00$ & $68.80$ &$46.33$ & $16.49$ & $49.12$ & $73.53$ &$46.38$\\
    JVSM~\cite{chen2020jvsm} \footnotesize{\textcolor{gray}{(ECCV'20)}} & $19.00$ & $52.10$ & $70.00$ &$47.03$ &$-$ &$-$ & $-$& \\
    DATIR~\cite{gu2021datir} \footnotesize{\textcolor{gray}{(MM'21)}} & $21.50$ & $48.80$ & $71.60$ &$47.30$ & $17.20$ & $51.10$ & $75.60$ & $47.97$ \\
    % ARTEMIS~\cite{delmas2022artemis} \footnotesize{\textcolor{gray}{(ICLR'22)}} &$-$ &$-$ & $-$&$-$& $17.60$ & $51.05$ & $76.85$ & $48.50$ \\
    FashionVLP~\cite{goenka2022fashionvlp} \footnotesize{\textcolor{gray}{(CVPR'22)}} & $-$ & $49.90$ & $70.50$ &$-$ &$-$ &$-$ & $-$&$-$\\
    Css-Net~\cite{zhang2024cssnet} \footnotesize{\textcolor{gray}{(KBS'24)}} & $22.20$ & $50.50$ & $69.70$ &$47.47$ & $20.13$ & $56.81$ & $81.32$ & $52.75$ \\
    CoSMo~\cite{lee2021cosmo} \footnotesize{\textcolor{gray}{(CVPR'21)}} & $23.30$ & $50.40$ & $69.30$ &$47.67$ & $16.72$ & $48.36$ & $75.64$ & $46.91$ \\
    ACNet~\cite{li2023acnet} \footnotesize{\textcolor{gray}{(ICME'23)}} & $-$ & $50.58$ & $70.12$ &$-$ & $-$ & $53.39$ & $79.53$ & $-$ \\
    GSCMR~\cite{zhang2021GSCMR} \footnotesize{\textcolor{gray}{(TIP'22)}} & $21.57$ & $52.84$ & $70.12$ &$48.18$ &$-$ &$-$ & $-$&$-$\\
    % VAL (GloVe)~\cite{} \footnotesize{\textcolor{gray}{(CVPR'20)}} & $22.90$ & $50.80$ & $72.70$ &$48.80$\\
    EER~\cite{zhang2022eer} \footnotesize{\textcolor{gray}{(TIP'22)}} & $-$ & $50.88$ & $70.60$ &$-$ & $19.87$ & $55.96$ & $79.58$ & $51.80$ \\
    NEUCORE~\cite{zhao2024neucore} \footnotesize{\textcolor{gray}{(NIPSW'23)}} &$-$ &$-$ & $-$&$-$& $19.76$ & $55.48$ & $80.75$ & $52.00$ \\
    AMC~\cite{zhu2023amc} \footnotesize{\textcolor{gray}{(TOMM'23)}}  &$-$ &$-$ & $-$&$-$& $19.99$ & $56.89$ & $79.27$ & $52.05$ \\
    MLCLSAP~\cite{zhang2023MLCLSAP} \footnotesize{\textcolor{gray}{(TMM'23)}} & $-$ & $51.06$ & $70.13$ &$-$ & $19.54$ & $55.15$ & $77.22$ & $50.64$ \\
    % EER w/random emb~\cite{} \footnotesize{\textcolor{gray}{(TIP'22)}} & $-$ & $51.09$ & $70.33$ &$-$\\
    Css-Net†~\cite{zhang2024cssnet} \footnotesize{\textcolor{gray}{(KBS'24)}} & $23.40$ & $52.00$ & $72.00$ &$49.13$ &$-$ &$-$ & $-$&$-$\\
    ComqueryFormer~\cite{xu2023ComqueryFormer} \footnotesize{\textcolor{gray}{(TMM'23)}} & $-$ & $52.20$ & $72.20$ &$-$ &$-$ &$-$ & $-$&$-$\\
    CRN-base~\cite{yang2023crn} \footnotesize{\textcolor{gray}{(TIP'23)}} & $-$ & $53.30$ & $73.30$ &$-$ & $17.32$ & $54.15$ & $79.34$ & $50.27$ \\
    CLVC-Net~\cite{wen2021clvcnet} \footnotesize{\textcolor{gray}{(SIGIR'21)}} & $22.60$ & $53.00$ & $72.20$ &$49.27$ & $17.64$ & $54.39$ & $79.47$ & $50.50$ \\
    CRN-large~\cite{yang2023crn} \footnotesize{\textcolor{gray}{(TIP'23)}} & $-$ & $53.50$ & $74.50$ &$-$ & $18.92$ & $54.55$ & $80.04$ & $51.17$ \\
    ComposeAE+GA~\cite{huang2022ga} \footnotesize{\textcolor{gray}{(TIP'22)}} & $25.20$ & $52.80$ & $71.20$ &$49.70$ &$-$ &$-$ & $-$&$-$\\
    % VAL (Lvv + Lvs)~\cite{} \footnotesize{\textcolor{gray}{(CVPR'20)}} & $21.50$ & $53.80$ & $73.30$ &$49.53$\\
    % TIRG+GA~\cite{} \footnotesize{\textcolor{gray}{(TIP'22)}} & $25.20$ & $52.80$ & $71.20$ &$49.73$\\
    ProVLA~\cite{hu2023provla} \footnotesize{\textcolor{gray}{(ICCVW'23)}} & $21.70$ & $53.70$ & $74.60$ &$50.00$ & $19.20$ & $56.20$ & $73.30$ & $49.57$ \\
    SDFN~\cite{wu2024sdfn} \footnotesize{\textcolor{gray}{(ICASSP'24)}} & $23.30$ & $54.40$ & $73.60$ &$50.40$ & $23.09$ & $58.51$ & $81.08$ & $54.22$ \\
    ComposeAE~\cite{anwaar2021ComposeAE} \footnotesize{\textcolor{gray}{(WACV'21)}} & $22.80$ & $55.30$ & $73.40$ &$50.50$ &$-$ &$-$ & $-$&$-$\\
    AlRet-small~\cite{xu2024alret} \footnotesize{\textcolor{gray}{(TMM'24)}} & $24.42$ & $53.93$ & $73.25$ &$50.53$ & $18.13$ & $53.98$ & $78.81$ & $50.31$ \\
    ARTEMIS~\cite{delmas2022artemis} \footnotesize{\textcolor{gray}{(ICLR'22)}} &$-$ &$-$ & $-$&$-$& $18.72$ & $53.11$ & $79.31$ & $50.38$ \\
    NSFSE~\cite{wang2024NSFSE} \footnotesize{\textcolor{gray}{(TMM'24)}} & $24.90$ & $54.30$ & $73.40$ &$50.90$ &$-$ &$-$ & $-$&$-$\\
    MANME~\cite{li2023manme} \footnotesize{\textcolor{gray}{(TCSVT'24)}} & $23.00$ & $57.90$ & $75.30$ &$52.00$ & $20.73$ & $55.96$ & $80.98$ & $52.56$ \\
    CMAP~\cite{li2024cmap} \footnotesize{\textcolor{gray}{(TOMM'24)}} & $24.20$ & $56.90$ & $75.30$ &$52.10$ & $21.48$ & $56.18$ & $81.14$ & $52.93$ \\
    DSCN~\cite{li2023dscn} \footnotesize{\textcolor{gray}{(ICMR'23)}} & $25.60$ & $56.20$ & $74.90$ &$52.23$ & $20.33$ & $55.84$ & $80.55$ & $52.24$ \\
    % TIRG-BERT+GA~\cite{} \footnotesize{\textcolor{gray}{(TIP'22)}} & $24.00$ & $57.20$ & $75.70$ &$52.30$\\
    AACL~\cite{tian2023aacl} \footnotesize{\textcolor{gray}{(WACV'23)}} & $19.64$ & $58.85$ & $78.86$ &$52.45$& $-$ & $-$ & $-$ & $-$ \\
    MCEM~\cite{zhang2024mcem} \footnotesize{\textcolor{gray}{(TIP'24)}} & $24.86$ & $56.76$ & $76.91$ &$52.84$& $19.10$ & $55.37$ & $79.57$ & $51.35$ \\
    LGLI~\cite{huang2023lgli} \footnotesize{\textcolor{gray}{(CVPRW'23)}} & $26.50$ & $58.60$ & $75.60$ &$53.60$ &$-$ &$-$ & $-$&$-$\\


    \cdashline{1-9}

    % \rowcolor{gray!5} % 设置该行的背景色为浅灰色
    \multicolumn{9}{c}{\textit{\textcolor{gray}{VLP Encoder-based Methods}}} \\
    PL4CIR-base~\cite{zhao2022PL4CIR} \footnotesize{\textcolor{gray}{(SIGIR'22)}} &$-$ &$-$ & $-$&$-$& $19.53$ & {$55.65$} & {$80.58$} & $51.92$ \\
    AlRet-big~\cite{xu2024alret} \footnotesize{\textcolor{gray}{(TMM'24)}} &$-$ &$-$ & $-$&$-$& $21.02$ & $55.72$ & $80.77$ & $52.50$ \\
    % Combiner+MU~\cite{chen2022mu} \footnotesize{\textcolor{gray}{(ICLR'24)}} & $21.80$ & $52.10$ & $70.20$ &$48.03$ & $18.41$ & $53.63$ & $79.84$ & $50.63$ \\
    IUDC~\cite{ge2024iudc} \footnotesize{\textcolor{gray}{(TOIS'24)}} &$-$ &$-$ & $-$&$-$& $21.17$ & $56.82$ & $82.25$ & $53.41$ \\
    IUDC+AUG~\cite{ge2024iudc} \footnotesize{\textcolor{gray}{(TOIS'24)}} &$-$ &$-$ & $-$&$-$& $21.83$ & $57.76$ & $82.90$ & $54.16$ \\
    PL4CIR-large~\cite{zhao2022PL4CIR} \footnotesize{\textcolor{gray}{(SIGIR'22)}} &$-$ &$-$ & $-$&$-$& $22.88$ & $58.83$ & $84.16$ & $55.29$ \\
    FashionERN~\cite{chen2024fashionern} \footnotesize{\textcolor{gray}{(AAAI'24)}} & $-$ & {$54.1 $} & {$72.5 $} & $-$ & $-$ & $55.59$ & $81.71$ & $-$ \\
    SPIRIT~\cite{chen2024spirit} \footnotesize{\textcolor{gray}{(TOMM'24)}} & $-$ & {$55.20$} & {$73.60$} & $-$ & $-$ & $56.90$ & $81.49$ & $-$ \\
    LIMN~\cite{wen2023limn} \footnotesize{\textcolor{gray}{(TPAMI'24)}} & $-$ & {$57.20$} & {$76.60$} & $-$ & $-$ & $68.20$ & $87.45$ & $-$ \\
    LIMN+~\cite{wen2023limn} \footnotesize{\textcolor{gray}{(TPAMI'24)}} &$-$ &$-$ & $-$&$-$& $-$ & $68.37$ & $88.07$ & $-$ \\
    CAFF~\cite{wan2024caff} \footnotesize{\textcolor{gray}{(CVPR'24)}} & {$25.21$} & {$60.17$} & {$80.79$} & {$55.39$} & $20.87$ & $56.82$ & $81.99$ & $53.23$ \\
    SHAF~\cite{yan2024shaf} \footnotesize{\textcolor{gray}{(ICIC'24)}} &$-$ &$-$ & $-$&$-$& $23.01$ & $64.39$ & $85.49$ & $57.63$ \\
    TG-CIR~\cite{wen2023tgcir} \footnotesize{\textcolor{gray}{(MM'23)}} &$-$ &$-$ & $-$&$-$& $25.89$ & $63.20$ & $85.07$ & $58.05$ \\
    DWC~\cite{huang2024dwc} \footnotesize{\textcolor{gray}{(AAAI'24)}} & {$36.49$} & {$63.58$} & {$79.02$} & {$59.70$} &$18.94$ &$55.55$ & $80.19$ & $51.56$\\
    DQU-CIR~\cite{wen2024dqu} \footnotesize{\textcolor{gray}{(SIGIR'24)}} & {$36.80$} & {$67.90$} & {$87.80$} & {$64.10$} &$31.47$ &$69.19$ & $88.52$&$63.06$\\
    
    \hline
    \end{tabular}
    \label{tab:supervised_CIR_exp_fashion200k_shoes_css}
    % }
\end{table*}


\begin{table*}
    \scriptsize
    \centering
    \caption{Performance comparison among supervised composed image retrieval models on CIRR. Avg means the average of R@$5$ and R$_{subset}$@$1$.}
    % \resizebox{14.5cm}{!}{
    \begin{tabular}{l|cccc|ccc|c}
    \hline 
    \multirow{2}{*}{Method} &\multicolumn{4}{c|}{\textbf{R@$k$}} &\multicolumn{3}{c|}{\textbf{R$_{subset}$@$k$}} & \multirow{2}{*}{Avg} \\ \cline{2-8}
    & $k=1$ & $k=5$ & $k=10$ & $k=50$ & $k=1$ & $k=2$ & $k=3$ \\
    \hline \hline 

    % \rowcolor{gray!5} 
    \multicolumn{9}{c}{\textit{\textcolor{gray}{Traditional Encoder-based Methods}}} \\
    
    TIRG~\cite{vo2019tirg} \footnotesize{\textcolor{gray}{(CVPR'19)}} & $14.61$ & $48.37$ & $64.08$ & $90.03$ & $22.67$ & $44.97$ & $65.14$ & $35.52$\\
    ARTEMIS~\cite{delmas2022artemis} \footnotesize{\textcolor{gray}{(ICLR'22)}} & $16.96$ & $46.10$ & $61.31$ & $87.73$ & $39.99$ & $62.20$ & $75.67$ & $43.05$ \\
    CIRPLANT~\cite{liu2021CIRPLANT} \footnotesize{\textcolor{gray}{(ICCV'21)}} & $19.55$ & $52.55$ & $68.39$ & $92.38$ & $39.20$ & $63.03$ & $79.49$ & $45.88$ \\
    MCEM~\cite{zhang2024mcem} \footnotesize{\textcolor{gray}{(TIP'24)}} & $17.48$ & $46.13$ & $62.17$ & $88.91$ & $-$ & $-$ & $-$ & $-$ \\
    NEUCORE~\cite{zhao2024neucore} \footnotesize{\textcolor{gray}{(NIPSW'23)}} & $18.46$ & $49.40$ & $63.57$ & $89.35$ & $44.27$ & $67.06$ & $78.92$ & $46.84$ \\
    NSFSE~\cite{wang2024NSFSE} \footnotesize{\textcolor{gray}{(TMM'24)}} & $20.70$ & $52.50$ & $67.96$ & $90.74$ & $44.20$ & $65.53$ & $78.50$ & $48.35$ \\
    % LMGA~\cite{} \footnotesize{\textcolor{gray}{(Arxiv'23)}} & $19.11$ & $45.64$ & $59.95$ & $86.89$ & $58.68$ & $79.90$ & $90.87$ & $52.16$ \\
    ComqueryFormer~\cite{xu2023ComqueryFormer} \footnotesize{\textcolor{gray}{(TMM'23)}} & $25.76$ & $61.76$ & $75.90$ & $95.13$ & $51.86$ & $76.26$ & $89.25$ & $56.81$ \\

    \cdashline{1-9}

    % \rowcolor{gray!5} % 设置该行的背景色为浅灰色
    \multicolumn{9}{c}{\textit{\textcolor{gray}{VLP Encoder-based Methods}}} \\
    
    CLIP-ProbCR~\cite{li2024clip} \footnotesize{\textcolor{gray}{(ICMR'24)}} & $23.32$ & $54.36$ & $68.64$ & $93.05$ & $54.32$ & $76.30$ & $88.88$ & $54.34$\\
    Combiner~\cite{baldrati2022Combiner} \footnotesize{\textcolor{gray}{(CVPRW'22)}} & $33.59$ & $65.35$ & $77.35$ & $95.21$ & $62.39$ & $81.81$ & $92.02$ & $63.87$ \\
    Ranking-aware~\cite{chen2023ranking} \footnotesize{\textcolor{gray}{(Arxiv'23)}} & $32.24$ & $66.63$ & $79.23$ & $96.43$ & $61.25$ & $81.33$ & $92.02$ & $63.94$ \\
    % CaLa_CLIP4Cir(RN50x4)~\cite{} \footnotesize{\textcolor{gray}{(SIGIR'24)}} & $35.37$ & $68.89$ & $80.07$ & $95.86$ & $66.68$ & $84.65$ & $93.42$ & $67.79$ \\
    % CLIP4CIR~\cite{} \footnotesize{\textcolor{gray}{(CVPR'22)}} & $35.81$ & $68.80$ & $80.17$ & $95.25$ & $66.96$ & $85.25$ & $93.13$ & $67.88$ \\
    CLIP4CIR~\cite{baldrati2022CLIP4CIR} \footnotesize{\textcolor{gray}{(CVPR'22)}} & $38.53$ & $69.98$ & $81.86$ & $95.93$ & $68.19$ & $85.64$ & $94.17$ & $69.09$ \\
    BLIP4CIR~\cite{liu2024blip4cir} \footnotesize{\textcolor{gray}{(WACV'24)}} & $40.17$ & $71.81$ & $83.18$ & $95.69$ & $72.34$ & $88.70$ & $95.23$ & $72.07$ \\
    BLIP4CIR+Bi~\cite{liu2024blip4cir} \footnotesize{\textcolor{gray}{(WACV'24)}} & $40.15$ & $73.08$ & $83.88$ & $96.27$ & $72.10$ & $88.27$ & $95.93$ & $72.59$ \\
    FashionERN~\cite{chen2024fashionern} \footnotesize{\textcolor{gray}{(AAAI'24)}} & $-$ & $74.77$ & $-$ & $-$ & $74.93$ & $-$ & $-$ & $74.85$ \\
    CLIP4CIR2~\cite{baldrati2023CLIP4CIR2} \footnotesize{\textcolor{gray}{(TOMM'23)}} & $42.05$ & $76.13$ & $86.51$ & $97.49$ & $70.15$ & $87.18$ & $94.40$ & $73.14$ \\
    DMOT~\cite{dmot} \footnotesize{\textcolor{gray}{(ACCV'24)}} & $41.55$ & $74.25$ & $84.67$ & $96.49$ & $74.20$ & $89.55$ & $95.58$ & $73.70$ \\
    SPIRIT~\cite{chen2024spirit} \footnotesize{\textcolor{gray}{(TOMM'24)}} & $40.23$ & $75.10$ & $84.16$ & $96.88$ & $73.74$ & $89.60$ & $95.93$ & $74.42$ \\
    VISTA~\cite{zhou2024vista} \footnotesize{\textcolor{gray}{(ACL'24)}} & $-$ & $76.10$ & $-$ & $-$ & $75.70$ & $-$ & $-$ & $75.90$ \\
    SSN~\cite{yang2024ssn} \footnotesize{\textcolor{gray}{(AAAI'24)}} & $43.91$ & $77.25$ & $86.48$ & $97.45$ & $71.76$ & $88.63$ & $95.54$ & $74.51$ \\
    DQU-CIR~\cite{wen2024dqu} \footnotesize{\textcolor{gray}{(SIGIR'24)}} & $46.22$ & $78.17$ & $87.64$ & $97.81$ & $70.92$ & $87.69$ & $94.68$ & $74.55$ \\
    SADN~\cite{wang2024sadn} \footnotesize{\textcolor{gray}{(MM'24)}} & $44.27$ & $78.10$ & $87.71$ & $97.89$ & $72.71$ & $89.33$ & $95.38$ & $75.41$ \\
    TG-CIR~\cite{wen2023tgcir} \footnotesize{\textcolor{gray}{(MM'23)}} & $45.25$ & $78.29$ & $87.16$ & $97.30$ & $72.84$ & $89.25$ & $95.13$ & $75.57$ \\
    TG-CIR+SPN~\cite{feng2024spn} \footnotesize{\textcolor{gray}{(MM'24)}} & $47.28$ & $79.13$ & $87.98$ & $97.54$ & $75.40$ & $89.78$ & $95.21$ & $77.27$ \\
    CASE~\cite{levy2024case} \footnotesize{\textcolor{gray}{(AAAI'24)}} & $48.00$ & $79.11$ & $87.25$ & $97.57$ & $75.88$ & $90.58$ & $96.00$ & $77.50$ \\
    CaLa_BLIP2Cir~\cite{feng2024spn} \footnotesize{\textcolor{gray}{(SIGIR'24)}} & $49.11$ & $81.21$ & $89.59$ & $98.00$ & $76.27$ & $91.04$ & $96.46$ & $78.74$ \\
    Re-ranking R50~\cite{liu2023rerank} \footnotesize{\textcolor{gray}{(TMLR'24)}} & $50.55$ & $81.75$ & $89.78$ & $97.18$ & $80.04$ & $91.90$ & $96.58$ & $80.90$ \\
    SDQUR~\cite{xu2024SDQUR} \footnotesize{\textcolor{gray}{(TCSVT'24)}} & $53.13$ & $83.16$ & $90.60$ & $98.22$ & $79.47$ & $91.74$ & $96.63$ & $81.32$ \\
    SPRC~\cite{xusentence2024sprc} \footnotesize{\textcolor{gray}{(ICLR'24)}} & $51.96$ & $82.12$ & $89.74$ & $97.69$ & $80.65$ & $92.31$ & $96.60$ & $81.39$ \\
    SPRC$^2$~\cite{xusentence2024sprc} \footnotesize{\textcolor{gray}{(ICLR'24)}} & $54.15$ & $83.01$ & $90.39$ & $98.17$ & $82.31$ & $92.68$ & $96.87$ & $82.66$ \\
    SPRC+SPN~\cite{feng2024spn} \footnotesize{\textcolor{gray}{(MM'24)}} & $55.06$ & $83.83$ & $90.87$ & $98.29$ & $81.54$ & $92.65$ & $97.04$ & $82.69$ \\
    SPRC+VQA~\cite{feng2023vqa4cir} \footnotesize{\textcolor{gray}{(Arxiv'23)}} & $54.00$ & $84.23$ & $91.85$ & $98.10$ & $82.07$ & $93.45$ & $97.08$ & $83.15$ \\
    
    \hline
    \end{tabular}
    \label{tab:supervised_CIR_exp_cirr}
    % }
\end{table*}


\begin{table*}
    \scriptsize
    \centering
    \caption{Performance comparison among supervised composed image retrieval models on MIT-States and CSS.}
    % \resizebox{14cm}{!}{
    \begin{tabular}{l|ccc|c|cc}
    \hline 
    \multirow{2}{*}{Model} & \multicolumn{4}{c|}{MIT-States} & \multicolumn{2}{c}{CSS}  \\
    \cline{2-7}
    & R@$1$ & R@$5$ & R@$10$ & Avg. & R@$1$(3D-to-3D) & R@$1$(2D-to-3D) \\
    \hline \hline 

    \multicolumn{7}{c}{\textit{\textcolor{gray}{Traditional Encoder-based Methods}}} \\
    TIRG~\cite{vo2019tirg} \footnotesize{\textcolor{gray}{(CVPR'19)}} & $12.20$ & $31.90$ & $43.10$ &$29.07$ & $73.70$ & $46.60$\\
    MGF~\cite{liu2021mgf} \footnotesize{\textcolor{gray}{(MMM'21)}} & $13.10$ & $-$ & $43.60$ & $-$ & $74.20$ & $62.80$\\
    TIS~\cite{zhang2022tis} \footnotesize{\textcolor{gray}{(TOMM'22)}} & $13.13$ & $31.94$ & $43.32$ & $29.46$ & $76.64$ & $48.02$\\
    TIRG+JPM(Tri)~\cite{yang2021jpm} \footnotesize{\textcolor{gray}{(MM'21)}} & $-$ & $-$& $-$ & $-$& $83.20$ & $-$\\
    TIRG+JPM(MSE)~\cite{yang2021jpm} \footnotesize{\textcolor{gray}{(MM'21)}} & $-$ & $-$& $-$ & $-$& $83.80$ & $-$\\
    TIRG+GA~\cite{huang2022ga} \footnotesize{\textcolor{gray}{(TIP'22)}} & $13.60$ & $32.40$ & $43.20$ & $29.70$ & $91.20$ & $-$\\
    JAMMA~\cite{zhang2020jamma} \footnotesize{\textcolor{gray}{(MM'20)}} & $14.27$ & $33.21$ & $45.34$ & $30.94$ & $76.07$ & $48.85$\\
    LBF-small~\cite{hosseinzadeh2020lbf} \footnotesize{\textcolor{gray}{(CVPR'20)}} & $14.29$ & $34.67$ & $46.06$ & $31.67$ & $67.26$ & $50.31$\\
    LBF-big~\cite{hosseinzadeh2020lbf} \footnotesize{\textcolor{gray}{(CVPR'20)}} & $14.72$ & $35.30$ & $46.56$ & $32.19$ & $79.20$ & $55.69$\\
    MCR~\cite{zhang2021mcr} \footnotesize{\textcolor{gray}{(MM'21)}} & $14.30$ & $35.36$ & $47.12$ & $32.26$ & $-$ & $-$\\
    ComposeAE~\cite{anwaar2021ComposeAE} \footnotesize{\textcolor{gray}{(WACV'21)}} & $13.90$ & $35.30$ & $47.90$ & $32.37$ & $-$ & $-$\\
    LGLI~\cite{huang2023lgli} \footnotesize{\textcolor{gray}{(CVPRW'23)}} & $14.90$ & $36.40$ & $47.70$ & $33.00$ & $93.30$ & $-$\\
    % TIRG-BERT+GA~\cite{} \footnotesize{\textcolor{gray}{(TIP'22)}} & $15.40$ & $36.30$ & $47.70$ & $33.13$\\
    ComposeAE+GA~\cite{huang2022ga} \footnotesize{\textcolor{gray}{(TIP'22)}} & $14.60$ & $37.00$ & $47.90$ & $33.20$ & $-$ & $-$\\
    GSCMR~\cite{zhang2021GSCMR} \footnotesize{\textcolor{gray}{(TIP'22)}} & $17.28$ & $36.45$ & $47.04$ & $33.59$ & $81.81$ & $58.74$\\
    
    \hline
    \end{tabular}
    \label{tab:supervised_CIR_exp_mit_css}
    % }
\end{table*}

\begin{table*}
    \scriptsize
    \centering
    \caption{Performance comparison among zero-shot composed image retrieval models on FashionIQ (original split).}
    % \resizebox{14.5cm}{!}{
    \begin{tabular}{l|c|cc|cc|cc|cc|c}
    \hline 
    \multirow{2}{*}{Model} & \multirow{2}{*}{Encoder} & \multicolumn{2}{c|}{Dresses} & \multicolumn{2}{c|}{Shirts} & \multicolumn{2}{c|} {Tops\&Tees} & \multicolumn{2}{c|}{Average} & \multirow{2}{*}{Avg.} \\ \cline{3-10}
    & & R@$10$ & R@$50$ & R@$10$ & R@$50$ & R@$10$ & R@$50$ & R@$10$ & R@$50$ & \\
    \hline \hline
    
    % \rowcolor{gray!5} 
    \multicolumn{11}{c}{\textit{\textcolor{gray}{Textual-inversion-based Methods}}}\\
    SEARLE~\cite{searle} \footnotesize{\textcolor{gray}{(ICCV'23)}} & CLIP-B & $18.54$ & $39.51$ & $24.44$ & $41.61$ & $25.70$ & $46.46$ & $22.89$ & $42.53$ & $32.71$ \\
    iSEARLE~\cite{isearle} \footnotesize{\textcolor{gray}{(Arxiv'24)}} & CLIP-B & $20.92$ & $42.19$ & $25.81$ & $43.52$ & $26.47$ & $48.70$ & $24.40$ & $44.80$ & $34.60$\\
    
    Pic2Word~\cite{pic2word} \footnotesize{\textcolor{gray}{(CVPR'23)}} & CLIP-L & $20.00 $ & $40.20$ & $26.20$ & $43.60$ & $27.90$ & $47.40$ & $24.70$ & $43.70$ &$34.22$\\
    SEARLE-XL~\cite{searle} \footnotesize{\textcolor{gray}{(ICCV'23)}} & CLIP-L & $20.48$ & $43.13$ & $26.89$ & $45.58$ & $29.32$ & $49.97$ & $25.56$ & $46.23$ & $35.90$ \\
    LinCIR~\cite{lincir} \footnotesize{\textcolor{gray}{(CVPR'24)}} & CLIP-L & $20.92$ & $42.44$ & $29.10$ & $46.81$ & $28.81$ & $50.18$ & $26.28$ & $46.49$ & $36.38$ \\
    KEDs~\cite{keds} \footnotesize{\textcolor{gray}{(CVPR'24)}} & CLIP-L & $21.70$ & $43.80$ & $28.90$ & $48.00$ & $29.90$ & $51.90$ & $26.80$ & $47.90$ & $37.37$ \\
    iSEARLE~\cite{isearle} \footnotesize{\textcolor{gray}{(Arxiv'24)}} & CLIP-L & $22.51$ & $46.36$ & $28.75$ & $47.84$ & $31.31$ & $52.68$ & $27.52$ & $48.96$ & $38.24$\\
    Context-I2W~\cite{context_i2w} \footnotesize{\textcolor{gray}{(AAAI'24)}} & CLIP-L & $23.10$ & $45.30$ & $29.70$ & $48.60$ & $30.60$ & $52.90$ & $27.80$ & $48.90$ & $38.40$ \\
    FTI4CIR~\cite{fti4cir} \footnotesize{\textcolor{gray}{(SIGIR'24)}} & CLIP-L & $24.39$ & $47.84$ & $31.35$ & $50.59$ & $32.43$ & $54.21$ & $29.39$ & $50.88$ & $40.14$\\
    
    LinCIR~\cite{lincir} \footnotesize{\textcolor{gray}{(CVPR'24)}} & CLIP-H & $29.80$ & $52.11$ & $36.90$ & $57.75$ & $42.07$ & $62.52$ & $36.26$ & $57.46$ & $46.86$ \\
    LinCIR~\cite{lincir} \footnotesize{\textcolor{gray}{(CVPR'24)}} & CLIP-G & $38.08$ & $60.88$ & $46.76$ & $65.11$ & $50.48$ & $71.09$ & $45.11$ & $65.69$ & $55.40$ \\
    
    ISA~\cite{isa} \footnotesize{\textcolor{gray}{(ICLR'24)}} & BLIP & $24.69$ & $43.88$ & $30.79$ & $50.05$ & $33.91$ & $53.65$ & $29.79$ & $49.19$ & $39.50$ \\
    ISA~\cite{isa} \footnotesize{\textcolor{gray}{(ICLR'24)}} & BLIP_CNN & $25.33$ & $46.26$ & $30.03$ & $48.58$ & $33.45$ & $53.80$ & $29.60$ & $49.54$ & $39.58$ \\
    ISA~\cite{isa} \footnotesize{\textcolor{gray}{(ICLR'24)}} & BLIP_VIT & $25.48$ & $45.51$ & $29.64$ & $48.68$ & $32.94$ & $54.31$ & $29.35$ & $49.50$ & $39.43$ \\
    Slerp+TAT~\cite{slerp} \footnotesize{\textcolor{gray}{(Arxiv'24)}}& BLIP & $29.15$ & $50.62$ & $32.14$ & $51.62$ & $37.02$ & $57.73$ & $32.77$ & $53.32$ & $43.05$ \\
    \cdashline{1-11}

    % \rowcolor{gray!5} % 设置该行的背景色为浅灰色
    \multicolumn{11}{c}{\textit{\textcolor{gray}{Pseudo-triplet-based Methods}}} \\
    MagicLens~\cite{zhang2024magiclens} \footnotesize{\textcolor{gray}{(ICML'24)}} & CLIP-B & $21.50$ & $41.30$ & $27.30$ & $48.80$ & $30.20$ & $52.30$ & $26.30$ & $47.40$ & $36.90$\\
    MTI~\cite{mti} \footnotesize{\textcolor{gray}{(Arxiv'23)}} & CLIP-B & $25.71$ & $47.81$ & $33.36$ & $53.47$ & $34.87$ & $58.44$ & $31.31$ & $53.24$ & $42.28$ \\
    Pic2Word+HyCIR~\cite{hycir} \footnotesize{\textcolor{gray}{(Arxiv'24)}} & CLIP-L & $19.98$ & $40.80$ & $27.62$ & $44.94$ & $28.14$ & $47.67$ & $25.25$ & $44.47$ & $34.86$ \\
    PM~\cite{pm} \footnotesize{\textcolor{gray}{(ICIP'24)}} & CLIP-L & $21.40$ & $41.70$ & $27.10$ & $43.80$ & $28.90$ & $47.30$ & $25.80$ & $44.20$ & $35.00$ \\
    LinCIR+RTD~\cite{rtd} \footnotesize{\textcolor{gray}{(Arxiv'24)}} & CLIP-L & $24.49$ & $48.24$ & $32.83$ & $50.44$ & $33.40$ & $54.56$ & $30.24$ & $51.08$ & $40.66$ \\
    MagicLens~\cite{zhang2024magiclens} \footnotesize{\textcolor{gray}{(ICML'24)}} & CLIP-L & $25.50$ & $46.10$ & $32.70$ & $53.80$ & $34.00$ & $57.70$ & $30.70$ & $52.50$ & $41.60$\\
    CompoDiff~\cite{compodiff} \footnotesize{\textcolor{gray}{(TMLR'24)}} & CLIP-L & $33.91$ & $47.85$ & $38.10$ & $52.48$ & $40.07$ & $52.22$ & $37.36$ & $50.85$ & $44.11$\\
    MTI~\cite{mti} \footnotesize{\textcolor{gray}{(Arxiv'23)}} & CLIP-L & $28.11$ & $51.12$ & $38.63$ & $58.51$ & $39.42$ & $62.68$ & $35.39$ & $57.44$ & $46.41$ \\
    MagicLens~\cite{zhang2024magiclens} \footnotesize{\textcolor{gray}{(ICML'24)}} & CoCa-B & $29.00$ & $48.90$ & $36.50$ & $55.50$ & $40.20$ & $61.90$ & $35.20$ & $55.40$ & $45.30$\\
    MagicLens~\cite{zhang2024magiclens} \footnotesize{\textcolor{gray}{(ICML'24)}} & CoCa-L & $32.30$ & $52.70$ & $40.50$ & $59.20$ & $41.40$ & $63.00$ & $38.00$ & $58.20$ & $48.10$\\
    % HyCIR~\cite{} \footnotesize{\textcolor{gray}{(Arxiv'24)}} & BLIP & $18.88$ & $34.50$ & $22.52$ & $37.58$ & $22.13$ & $40.33$ & $21.18$ & $37.47$ & $29.32$ \\
    CompoDiff~\cite{compodiff} \footnotesize{\textcolor{gray}{(TMLR'24)}} & CLIP-G & $37.78$ & $49.10$ & $41.31$ & $55.17$ & $44.26$ & $56.41$ & $41.12$ & $53.56$ & $47.34$\\
    TransAgg~\cite{transagg} \footnotesize{\textcolor{gray}{(BMVC'23)}} & BLIP & $31.28$ & $52.75$ & $34.45$ & $53.97$ & $37.79$ & $60.48$ & $34.64$ & $55.72$ & $45.18$ \\
    PVLF~\cite{pvlf} \footnotesize{\textcolor{gray}{(ACML'24)}} & BLIP & $31.58$ & $54.24$ & $36.61$ & $55.05$ & $38.85$ & $61.24$ & $35.68$ & $56.85$ & $46.26$ \\
    
    \cdashline{1-11}

    % \rowcolor{gray!5} 
    \multicolumn{11}{c}{\textit{\textcolor{gray}{Training-free Methods}}}\\
    LDRE~\cite{ldre} \footnotesize{\textcolor{gray}{(SIGIR'24)}}& CLIP-B & $19.97$ & $41.84$ & $27.38$ & $46.27$ & $27.07$ & $48.78$ & $24.81$ & $45.63$ & $35.22$ \\
    SEIZE~\cite{seize} \footnotesize{\textcolor{gray}{(MM'24)}}& CLIP-B & $25.37$ & $46.84$ & $29.38$ & $47.97$ & $32.07$ & $54.78$ & $28.94$ & $49.86$ & $39.40$ \\
    WeiMoCIR~\cite{weimocir} \footnotesize{\textcolor{gray}{(Arxiv'24)}}& CLIP-L & $25.88$ & $47.29$ & $32.78$ & $48.97$ & $35.95$ & $56.71$ & $31.54$ & $50.99$ & $41.26$\\
    LDRE~\cite{ldre} \footnotesize{\textcolor{gray}{(SIGIR'24)}}& CLIP-L & $22.93$ & $46.76$ & $31.04$ & $51.22$ & $31.57$ & $53.64$ & $28.51$ & $50.54$ & $39.53$ \\
    SEIZE~\cite{seize} \footnotesize{\textcolor{gray}{(MM'24)}}& CLIP-L & $30.93$ & $50.76$ & $33.04$ & $53.22$ & $35.57$ & $58.64$ & $33.18$ & $54.21$ & $43.69$ \\
    WeiMoCIR~\cite{weimocir} \footnotesize{\textcolor{gray}{(Arxiv'24)}}& CLIP-H & $28.76$ & $48.98$ & $36.56$ & $53.58$ & $39.72$ & $59.87$ & $35.01$ & $54.14$ & $44.58$\\
    LDRE~\cite{ldre} \footnotesize{\textcolor{gray}{(SIGIR'24)}}& CLIP-G & $26.11$ & $51.12$ & $35.94$ & $58.58$ & $35.42$ & $56.67$ & $32.49$ & $55.46$ & $43.97$ \\
    WeiMoCIR~\cite{weimocir} \footnotesize{\textcolor{gray}{(Arxiv'24)}}& CLIP-G & $30.99$ & $52.45$ & $37.73$ & $56.18$ & $42.38$ & $63.23$ & $37.03$ & $57.29$ & $47.16$\\
    SEIZE~\cite{seize} \footnotesize{\textcolor{gray}{(MM'24)}}& CLIP-G & $39.61$ & $61.02$ & $43.60$ & $65.42$ & $45.94$ & $71.12$ & $43.05$ & $65.85$ & $54.45$ \\
    Slerp~\cite{slerp} \footnotesize{\textcolor{gray}{(Arxiv'24)}}& BLIP & $22.91$ & $42.39$ & $27.33$ & $45.25$ & $32.33$ & $50.48$ & $27.52$ & $46.04$ & $36.78$ \\
    GRB~\cite{grb} \footnotesize{\textcolor{gray}{(Arxiv'23)}}& BLIP2 & $24.14$ & $45.56$ & $34.54$ & $55.15$ & $33.55$ & $53.60$ & $30.74$ & $51.44$ & $41.09$ \\
    
    \hline
    \end{tabular}
    \label{tab:zs_CIR_exp_fashioniq_ori}
    % }
\end{table*}

\begin{table*}
    \scriptsize
    \centering
    \caption{Performance comparison among zero-shot composed image retrieval models on CIRR and CIRCO.}
    % \resizebox{14.5cm}{!}{
    \begin{tabular}{l|c|cccc|ccc|cccc}
    \hline 
    \multirow{3}{*}{Model} & \multirow{3}{*}{Encoder} & \multicolumn{7}{c|}{CIRR} & \multicolumn{4}{c}{CIRCO}  \\ 
    \cline{3-13}
    && \multicolumn{4}{c|}{\textbf{R@$k$}} & \multicolumn{3}{c|}{\textbf{R$_{subset}$@$k$}} & \multicolumn{4}{c}{\textbf{mAP@$k$}}\\
    \cline{3-13}
    & & $k=1$ & $k=5$ & $k=10$ & $k=50$ & $k=1$ & $k=2$ & $k=3$ & $k=5$ & $k=10$ & $k=25$ & $k=50$ \\
    \hline \hline
    
    % \rowcolor{gray!5} 
    \multicolumn{13}{c}{\textit{\textcolor{gray}{Textual-inversion-based Methods}}}\\
    SEARLE~\cite{searle} \footnotesize{\textcolor{gray}{(ICCV'23)}} & CLIP-B & $24.00 $ & $53.42$ & $66.82$ & $89.78$ & $54.89$ & $76.60$ & $88.19$ & $9.35$ & $9.94$ & $11.13$ &$11.84$\\
    iSEARLE~\cite{isearle} \footnotesize{\textcolor{gray}{(Arxiv'24)}} & CLIP-B & $25.23$ & $55.69$ & $68.05$ & $90.82$ & $-$ & $-$ & $-$ & $10.58$ & $11.24$ & $12.51$ & $13.26$\\
    Pic2Word~\cite{pic2word} \footnotesize{\textcolor{gray}{(CVPR'23)}} & CLIP-L & $23.90 $ & $51.70$ & $65.30$ & $87.80$ & $-$ & $-$ & $-$ & $-$ & $-$ & $-$ &$-$\\
    SEARLE-XL~\cite{searle} \footnotesize{\textcolor{gray}{(ICCV'23)}} & CLIP-L & $24.24 $ & $52.48$ & $66.29$ & $88.84$ & $53.76$ & $75.01$ & $88.19$ & $11.68$ & $12.73$ & $14.33$ &$15.12$\\
    KEDs~\cite{keds} \footnotesize{\textcolor{gray}{(CVPR'24)}} & CLIP-L & $26.40$ & $54.80$ & $67.20$ & $89.20$ & $-$ & $-$ & $-$ & $-$ & $-$ & $-$ & $-$\\
    LinCIR~\cite{lincir} \footnotesize{\textcolor{gray}{(CVPR'24)}} & CLIP-L & $25.04$ & $53.25$ & $66.68$ & $-$ & $57.11$ & $77.37$ & $88.89$ & $12.59$ & $13.58$ & $15.00$ & $15.85$\\
    iSEARLE~\cite{isearle} \footnotesize{\textcolor{gray}{(Arxiv'24)}} & CLIP-L & $25.28$ & $54.00$ & $66.72$ & $88.80$ & $-$ & $-$ & $-$ & $12.50$ & $13.61$ & $15.36$ & $16.25$\\
    Context-I2W~\cite{context_i2w} \footnotesize{\textcolor{gray}{(AAAI'24)}} & CLIP-L & $25.60$ & $55.10$ & $68.50$ & $89.80$ & $-$ & $-$ & $-$ & $-$ & $-$ & $-$ & $-$\\
    FTI4CIR~\cite{fti4cir} \footnotesize{\textcolor{gray}{(SIGIR'24)}} & CLIP-L & $25.90$ & $55.61$ & $67.66$ & $89.66$ & $55.21$ & $75.88$ & $87.98$ & $15.05$ & $16.32$ & $18.06$ & $19.05$\\
    LinCIR~\cite{lincir} \footnotesize{\textcolor{gray}{(CVPR'24)}} & CLIP-H & $33.83$ & $63.52$ & $75.35$ & $-$ & $62.43$ & $81.47$ & $92.12$ & $17.60$ & $18.52$ & $20.46$ & $21.39$\\
    LinCIR~\cite{lincir} \footnotesize{\textcolor{gray}{(CVPR'24)}} & CLIP-G & $35.25$ & $64.72$ & $76.05$ & $-$ & $63.35$ & $82.22$ & $91.98$ & $19.71$ & $21.01$ & $23.13$ & $24.18$\\  
    ISA~\cite{isa} \footnotesize{\textcolor{gray}{(ICLR'24)}} & BLIP\_CNN & $30.84$ & $61.06$ & $73.57$ & $92.43$ & $-$ & $-$ & $-$ & $11.33$ & $12.25$ & $13.42$ & $13.97$\\
    ISA~\cite{isa} \footnotesize{\textcolor{gray}{(ICLR'24)}} & BLIP\_VIT & $29.63$ & $58.99$ & $71.37$ & $91.47$ & $-$ & $-$ & $-$ & $9.82$ & $10.50$ & $11.61$ & $12.09$\\ 
    ISA~\cite{isa} \footnotesize{\textcolor{gray}{(ICLR'24)}} & BLIP & $29.68$ & $58.72$ & $70.79$ & $90.33$ & $-$ & $-$ & $-$ & $9.67$ & $10.32$ & $11.26$ & $11.61$\\
    Slerp+TAT~\cite{slerp} \footnotesize{\textcolor{gray}{(Arxiv'24)}} & BLIP & $33.98$ & $61.74$ & $72.70$ & $88.94$ & $68.55$ & $85.11$ & $93.21$ & $17.84$ & $18.44$ & $20.24$ & $21.07$\\
    
    \cdashline{1-13}

    % \rowcolor{gray!5} % 
    \multicolumn{13}{c}{\textit{\textcolor{gray}{Pseudo-triplet-based Methods}}} \\
    MTI~\cite{mti} \footnotesize{\textcolor{gray}{(Arxiv'23)}} & CLIP-B & $18.80$ & $46.07$ & $60.75$ & $86.41$ & $44.29$ & $68.10$ & $83.42$ & $8.14$ & $8.90$ & $10.12$ & $10.75$\\
    MagicLens~\cite{zhang2024magiclens} \footnotesize{\textcolor{gray}{(ICML'24)}} & CLIP-B & $27.00$ & $58.00$ & $76.90$ & $91.10$ & $66.70$ & $83.90$ & $92.40$ & $23.10$ & $23.80$ & $25.80$ & $26.70$\\
    MTI~\cite{mti} \footnotesize{\textcolor{gray}{(Arxiv'23)}} & CLIP-L & $25.52$ & $54.58$ & $67.59$ & $88.70$ & $55.64$ & $77.54$ & $89.47$ & $10.36$ & $11.63$ & $12.95$ & $13.67$\\
    MCL~\cite{mcl} \footnotesize{\textcolor{gray}{(ICML'24)}} & CLIP-L & $26.22$ & $56.84$ & $70.00$ & $91.35$ & $61.45$ & $81.61$ & $91.93$ & $17.67$ & $18.86$ & $20.80$ & $21.68$\\
    LinCIR+RTD~\cite{rtd} \footnotesize{\textcolor{gray}{(Arxiv'24)}} & CLIP-L & $26.63$ & $56.17$ & $68.96$ & $-$ & $-$ & $-$ & $-$ & $17.11$ & $18.11$ & $20.06$ & $21.01$\\
    MagicLens~\cite{zhang2024magiclens} \footnotesize{\textcolor{gray}{(ICML'24)}} & CLIP-L & $30.10$ & $61.70$ & $74.40$ & $92.60$ & $68.10$ & $84.80$ & $93.20$ & $29.60$ & $30.80$ & $33.40$ & $34.40$\\
    CompoDiff~\cite{compodiff} \footnotesize{\textcolor{gray}{(TMLR'24)}} & CLIP-L & $19.37$ & $53.81$ & $72.02$ & $90.85$ & $59.13$ & $78.81$ & $89.33$ & $12.31$ & $13.51$ & $15.67$ & $16.15$\\
    Pic2Word+HyCIR~\cite{hycir} \footnotesize{\textcolor{gray}{(Arxiv'24)}} & CLIP-L & $25.08$ & $53.49$ & $67.03$ & $89.85$ & $53.83$ & $75.06$ & $87.18$ & $14.12$ & $15.02$ & $16.72$ & $17.56$\\
    PM~\cite{pm} \footnotesize{\textcolor{gray}{(ICIP'24)}} & CLIP-L & $26.10$ & $55.20$ & $67.50$ & $90.20$ & $56.00$ & $76.60$ & $88.00$ & $-$ & $-$ & $-$ &$-$\\
    CompoDiff~\cite{compodiff} \footnotesize{\textcolor{gray}{(TMLR'24)}} & CLIP-G & $26.71$ & $55.14$ & $74.52$ & $92.01$ & $64.54$ & $82.39$ & $91.81$ & $15.33$ & $17.71$ & $19.45$ & $21.01$\\
    MagicLens~\cite{zhang2024magiclens} \footnotesize{\textcolor{gray}{(ICML'24)}} & CoCa-B & $31.60$ & $64.00$ & $76.90$ & $93.80$ & $69.30$ & $86.00$ & $94.00$ & $30.80$ & $32.00$ & $34.50$ & $35.60$\\
    MagicLens~\cite{zhang2024magiclens} \footnotesize{\textcolor{gray}{(ICML'24)}} & CoCa-L & $33.30$ & $67.00$ & $77.90$ & $94.40$ & $70.90$ & $87.30$ & $94.50$ & $34.10$ & $35.40$ & $38.10$ & $39.20$\\
    TransAgg~\cite{transagg} \footnotesize{\textcolor{gray}{(BMVC'23)}} & BLIP & $37.18$ & $67.21$ & $77.92$ & $93.43$ & $69.34$ & $85.68$ & $93.62$ & $-$ & $-$ & $-$ &$-$\\
    Pic2Word+HyCIR~\cite{hycir} \footnotesize{\textcolor{gray}{(Arxiv'24)}} & BLIP & $38.28$ & $69.03$ & $79.71$ & $95.27$ & $66.79$ & $84.79$ & $93.06$ & $18.91$ & $19.67$ & $21.58$ & $22.49$\\
    PVLF~\cite{pvlf} \footnotesize{\textcolor{gray}{(ACML'24)}} & BLIP & $40.33$ & $72.50$ & $82.44$ & $95.43$ & $72.64$ & $87.37$ & $94.69$ & $-$ & $-$ & $-$ &$-$\\
    
    \cdashline{1-13}

    % \rowcolor{gray!5} 
    \multicolumn{13}{c}{\textit{\textcolor{gray}{Training-free Methods}}} \\
    
    CIReVL~\cite{cirevl} \footnotesize{\textcolor{gray}{(ICLR'24)}}& CLIP-B & $23.94$ & $52.51$ & $66.00$ & $86.95$ & $60.17$ & $80.05$ & $88.19$ & $14.94$ & $15.42$ & $17.00$ & $17.82$\\
    LDRE~\cite{ldre} \footnotesize{\textcolor{gray}{(SIGIR'24)}} & CLIP-B & $25.69$ & $55.13$ & $69.04$ & $89.90$ & $60.53$ & $80.65$ & $90.70$ & $17.96$ & $18.32$ & $20.21$ & $21.11$\\
    SEIZE~\cite{seize} \footnotesize{\textcolor{gray}{(MM'24)}} & CLIP-B & $27.47$ & $57.42$ & $70.17$ & $-$ & $65.59$ & $84.48$ & $92.77$ & $19.04$ & $19.64$ & $21.55$ & $22.49$\\
    CIReVL~\cite{cirevl} \footnotesize{\textcolor{gray}{(ICLR'24)}}& CLIP-L & $24.55$ & $52.31$ & $64.92$ & $86.34$ & $59.54$ & $79.88$ & $89.69$ & $18.57$ & $19.01$ & $20.89$ & $21.80$\\
    WeiMoCIR~\cite{weimocir} \footnotesize{\textcolor{gray}{(Arxiv'24)}} & CLIP-L & $30.94$ & $60.87$ & $73.08$ & $91.61$ & $58.55$ & $79.06$ & $90.07$ & $-$ & $-$ & $-$ & $-$\\
    LDRE~\cite{ldre} \footnotesize{\textcolor{gray}{(SIGIR'24)}} & CLIP-L & $26.53$ & $55.57$ & $67.54$ & $88.50$ & $66.43$ & $80.31$ & $89.90$ & $23.35$ & $24.03$ & $26.44$ & $27.50$\\
    SEIZE~\cite{seize} \footnotesize{\textcolor{gray}{(MM'24)}} & CLIP-L & $28.65$ & $57.16$ & $69.23$ & $-$ & $66.22$ & $84.05$ & $92.34$ & $24.98$ & $25.82$ & $28.24$ & $29.35$\\
    WeiMoCIR~\cite{weimocir} \footnotesize{\textcolor{gray}{(Arxiv'24)}} & CLIP-H & $29.11$ & $59.76$ & $72.34$ & $91.18$ & $57.23$ & $79.08$ & $89.76$ & $-$ & $-$ & $-$ & $-$\\
    CIReVL~\cite{cirevl} \footnotesize{\textcolor{gray}{(ICLR'24)}}& CLIP-G & $34.65$ & $64.29$ & $75.06$ & $91.66$ & $67.95$ & $84.87$ & $93.21$ & $26.77$ & $27.59$ & $29.96$ & $31.03$\\
    WeiMoCIR~\cite{weimocir} \footnotesize{\textcolor{gray}{(Arxiv'24)}} & CLIP-G & $31.04$ & $60.41$ & $72.27$ & $90.89$ & $58.84$ & $78.92$ & $89.64$ & $-$ & $-$ & $-$ & $-$\\
    LDRE~\cite{ldre} \footnotesize{\textcolor{gray}{(SIGIR'24)}} & CLIP-G & $36.15$ & $66.39$ & $77.25$ & $93.95$ & $68.82$ & $85.66$ & $93.76$ & $31.12$ & $32.24$ & $34.95$ & $36.03$\\
    SEIZE~\cite{seize} \footnotesize{\textcolor{gray}{(MM'24)}} & CLIP-G & $38.87$ & $69.42$ & $79.42$ & $-$ & $74.15$ & $89.23$ & $95.71$ & $32.46$ & $33.77$ & $36.46$ & $37.55$\\
    Slerp~\cite{slerp} \footnotesize{\textcolor{gray}{(Arxiv'24)}} & BLIP & $28.60$ & $55.37$ & $65.66$ & $84.05$ & $65.16$ & $83.90$ & $92.05$ & $9.61$ & $10.11$ & $11.10$ & $11.66$\\
    GRB~\cite{grb} \footnotesize{\textcolor{gray}{(Arxiv'23)}} & BLIP2 & $24.19$ & $52.07$ & $65.48$ & $85.28$ & $60.17$ & $79.13$ & $89.34$ & $23.76$ & $25.31$ & $28.19$ & $29.17$\\
    GRB+LCR~\cite{grb} \footnotesize{\textcolor{gray}{(Arxiv'23)}} & BLIP2 & $30.92$ & $56.99$ & $68.58$ & $85.28$ & $66.67$ & $78.68$ & $82.60$ & $25.38$ & $26.93$ & $29.82$ & $30.74$\\
   
    \hline
    \end{tabular}
    \label{tab:zs_CIR_exp_cirr_circo_ori}
% }
\end{table*}

% 有待重写
\subsection{Experimental Results.}
In this subsection, we compare and analyze supervised CIR and ZS-CIR methods as reviewed above.


\subsubsection{Supervised Composed Image Retrieval.} 
To have an in-depth insight into the results of supervised CIR methods, we provide a comparison of their performance on various widely used datasets in Tables~\ref{tab:supervised_CIR_exp_fashioniq_ori}-~\ref{tab:supervised_CIR_exp_mit_css}. These datasets include FashionIQ, Fashion200k, MIT-States, CSS, Shoes, and CIRR. 
Notably, the FashionIQ dataset includes two evaluation protocols. However, some current approaches have erroneously intermixed results from different protocols during model comparison. To address this issue, we conduct a detailed inspection of the methods and their available source code. We then organize the comparisons for the VAL split and the original split separately to ensure fairness.
Furthermore, recognizing the significant impact of different encoders on model performance, we categorize the methods into two groups: those utilizing traditional encoders, such as ResNet and LSTM, and those employing VLP encoders (\textit{e.g.}, CLIP and BLIP) as the feature extraction backbones. 
From these tables, we obtain the following observations. 
1) VLP encoder-based methods generally achieve much better performance in comparison to traditional encoder-based methods. The reason is that VLP encoders are usually larger than traditional encoder methods. Moreover, VLP encoders are typically pre-trained on extensive corpora of image-text pairs through contrastive learning, thereby possessing excellent capabilities for multimodal alignment and cross-modal retrieval, which are crucial in the context of CIR. This can also be emphasized by methods that adopt the same fusion strategy but different types of encoders. As can be seen from Tables~\ref{tab:supervised_CIR_exp_fashioniq_val}-\ref{tab:supervised_CIR_exp_fashion200k_shoes_css}, AIRet-big with the VLP encoder attains significantly better performance than AIRet-small with a traditional encoder. Additionally, even when using the same type of traditional encoder or VLP encoder, the version with larger parameters typically delivers better performance. For example, LBF-big outperforms LBF-small (See Table~\ref{tab:supervised_CIR_exp_fashion200k_shoes_css}), and FashionERN-big surpasses FashionERN-small (See Table~\ref{tab:supervised_CIR_exp_fashioniq_ori}).
These results highlight that the choice of image and text encoders plays a pivotal role in the context of CIR, often surpassing the importance of image-text fusion and target matching method design.
2) Regarding image-text fusion, various strategies have demonstrated strong performance. Here, we summarize the top-performing methods for the FashionIQ-VAL, FashionIQ-ori, Fashion200k, Shoes, CIRR, MIT-States, and CSS datasets, respectively. Notably, to ensure a fair comparison, only prototype methods are evaluated, with additional modules such as data augmentation and reranking disabled.
The results show that DQU-CIR (MLP-based), SDQUR (Cross-attention-based), SPRC (Self-attention-based), and GSCMR (Graph-attention-based) achieve the best performance across these datasets. This indicates that while the image-text fusion strategy is a critical component of CIR methods, it is not the sole determinant of final performance. Moreover, it remains challenging to identify a universally optimal fusion strategy.
3) Models that integrate additional target matching techniques or data augmentation modules consistently demonstrate superior performance in comparison to their original counterparts. For example, SPRC-VQA, which employs visual question answering to re-rank the retrieval list, outperforms SPRC, as demonstrated in Table \ref{tab:supervised_CIR_exp_fashioniq_ori} and Table \ref{tab:supervised_CIR_exp_cirr}. LIMN+, by leveraging augmented pseudo-triplets for iterative training, attains better outcomes than LIMN shown in Table \ref{tab:supervised_CIR_exp_fashioniq_val} and Table \ref{tab:supervised_CIR_exp_fashion200k_shoes_css}. Additionally, ComposeAE+GA, which adopts the gradient augmentation regularization approach to achieve the dataset augmentation effect and mitigate overfitting, surpasses ComposeAE, as can be seen from Table \ref{tab:supervised_CIR_exp_fashion200k_shoes_css} and Table \ref{tab:supervised_CIR_exp_mit_css}. These examples emphasize the critical significance of further exploiting advanced target matching strategies and implementing dataset augmentation techniques to bolster the model's generalization capabilities.

%3) Some target matching trick, such as re-ranking and 
% $2$) Among the methods that adopt the gate mechanism as the fusion strategy, CLCV-Net generally outperforms the other approaches such as DCNet, JVSM, and MGF. This superior performance is primarily due to its joint consideration of both global-wise and local-wise image-text compositions, meanwhile, it incorporates a mutual enhancement module that facilitates the local and global composition processes by encouraging them to share knowledge with each other. 
% $3$) 

\subsubsection{Zero-shot Composed Image Retrieval.} ZS-CIR models, including textual-inversion-based, pseudo-triplet-based, and training-free models, are evaluated on the FashionIQ, CIRR, and CIRCO datasets. We directly summarize their experimental results from the corresponding papers in Table~\ref{tab:zs_CIR_exp_fashioniq_ori} and Table~\ref{tab:zs_CIR_exp_cirr_circo_ori}. 
It is worth noting that existing implementations of the ZS-CIR approach typically rely on the generalization capabilities of VLP-based encoders and are commonly tested with various backbone versions. Therefore, to enable a comprehensive comparison, we list all the performance results of methods utilizing different backbones in our tables.
From these tables, we obtain the following observations. 
1) Certain zero-shot methods yield comparable outcomes to some supervised methods. For example, on the FashionIQ-ori dataset, the top-performing ZS-CIR method, LinCIR (CLIP-G version), attains a score of $55.40$ when averaged across the metrics. Significantly, it not only outperforms all of the traditional encoder-based supervised methods but also manages to achieve a performance comparable to nearly half the number of the VLP encoder-based supervised methods. This implies that even in the absence of manually annotated triplet data, one can still obtain satisfactory retrieval results by ingeniously devising the pre-training strategy and fully activating the potential of powerful VLP capabilities within the CIR context.
2) Typically, the same zero-shot methods, when equipped with a large backbone, consistently deliver better performance. %This can be witnessed from the three datasets with the three zero-shot group methods. 
For example, LinCIR, ISA, MagicLens, CompoDiff, LDRE, SEIZE, and WieMoCIR all confirm that their larger-scale variants perform more effectively. This further evidences that VLP encoders, which possess well-trained multimodal encoding and cross-modal retrieval capabilities, significantly influence the performance of ZS-CIR.
3) Generally, among the three types of zero-shot methods, the overall performance of methods based on pseudo-triplets is better. For example, on the FashionIQ-ori dataset, regarding the average recall for textual-inversion-based methods, pseudo-triplet-based methods, and training-free methods, the number of methods showing a performance greater than $45.00\%$ is $2$, $6$, and $2$ respectively. On the CIRR dataset, based on R@$10 > 75\%$, the corresponding numbers are $2$, $7$, and $3$. And on the CIRCO dataset, based on mAP@$10 > 30\%$, the numbers are $0$, $3$, and $2$. This can be ascribed to the fact that the pseudo-triplet-based methods still construct triplet data that are most similar to the supervised training paradigm. Although this group of methods is somewhat resource-intensive during the pseudo-triplet construction stage, their overall performance is also the best. 



\section{Conclusions and Future Works}
\label{sec:Conclusions and Future Works}
In this work, we developed a framework for the runtime enforcement against STL formula. This framework inputs a signal and outputs a minimally modified signal that satisfy the formula. Specially, given an STL formula, we derive timed transducers for the atomic components, compose them according to the formula, and apply them to the input timed words, which are obtained by encoding the signal. We present detail procedure for signal encoding, translating STL temporal operators into timed transducers, and an enforcement algorithm. Our approach effectively enforces a signal against an STL property on CPS.

As in \cite{10.1145/3126500,10.1145/3092282.3092291,10.1109/TII.2019.2945520}, we plan to extend the work to accommodate bidirectionality and also extend the framework for more general STL formulas.


%\noindent \textit{Future Works.}  
%As in \cite{10.1145/3126500,10.1145/3092282.3092291,10.1109/TII.2019.2945520},  in a bidirectional framework involving an environment and a program, we require two enforcers—one for monitoring inputs to the controller from the environment and the other for monitoring outputs from the controller to the environment. These enforcers will (minimally) correct any erroneous inputs or outputs to ensure that a specified property is maintained. Therefore, we plan to extend the work to accommodate bidirectionality.


%Also, the translation from STL to timed transducer that we demonstrate is specifically designed for enforcement. However, a more general translation approach, such as from STL to hybrid automata, could also be explored for enforcement and broader applications. Therefore, a broader question we aim to address in the future is enforcement based on hybrid automata specifications, with the current STL to timed transducer translation serving as a foundational step.



\newpage 
\renewcommand \thepart{}
\renewcommand \partname{}

\bibliography{ref.bib}
\bibliographystyle{plain}

\doparttoc
\faketableofcontents
\part{}

\newpage
\appendix
\onecolumn

\addcontentsline{toc}{section}{Appendix}
\part{Appendix} 
\parttoc

\paragraph{Structure of the Appendix. } The appendix is organized as follows. In \Cref{app:hdmprac}, we introduce a practical variant of hypergradient descent and explain its implementation details. \Cref{app:additional} provides additional experiment details on the tested problems. \Cref{app:proof-hdm-ol} to \Cref{app:proof-momentum} provide proofs of the main results in the paper.

\newpage
\section{{\hdm} in Practice} \label{app:hdmprac}

This section introduces {\hdmbest}, our recommended practical hypergradient descent method. This variant is adapted from {\hdmhb}, with simplifications to reduce the implementation complexity. The algorithm is given in \Cref{alg:practical}.

\begin{algorithm}[h]
{\textbf{input} starting point $x^0 = x^1$, $\Pcal = \mathbb{S}^n_{+} \cap \Dcal, \Bcal = [0,0.9995]$, initial diagonal preconditioner $P_1 \in \mathbb{S}^n_{+} \cap \Dcal$, \\initial scalar momentum parameter $\beta_1 = 0.95$, {\adagrad} stepsize $\eta_p, \eta_b > 0$, {\adagrad}  diagonal matrix $U_1 = 0$, {\adagrad} momentum scalar $v_1 = 0$, $\tau > 0$}\\
\For{k =\rm{ 1, 2,...}}{
$\begin{aligned}
x^{k + 1 / 2} & = x^k - P_k \nabla f (x^k) + \beta_k (x^k - x^{k - 1}) \\
\nabla_P h_{x^k, x^{k - 1}} (P_k, \beta_k) & = \tfrac{\diag(\nabla f (x^{k + 1 / 2}) \circ \nabla f (x^k))}{\| \nabla f (x^k) \|^2 + \frac{\tau}{2} \| x^k - x^{k - 1} \|^2} \texttt{ \# Element-wise product} \\
\nabla_{\beta} h_{x^k, x^{k - 1}} (P_k, \beta_k) & = \tfrac{\langle \nabla f (x^{k + 1 / 2}), x^k - x^{k - 1} \rangle}{\| \nabla f (x^k) \|^2 + \frac{\tau}{2} \| x^k - x^{k - 1} \|^2} \texttt{ \# Inner product} \\
U_{k + 1} & = U_k + \nabla_P h_{x^k, x^{k - 1}} (P_k, \beta_k) \circ \nabla_P h_{x^k, x^{k - 1}} (P_k, \beta_k) \texttt{ \# Diagonal matrix} \\
v_{k + 1} & = v_k + \nabla_{\beta} h_{x^k, x^{k - 1}} (P_k, \beta_k) \cdot \nabla_{\beta} h_{x^k, x^{k - 1}} (P_k, \beta_k) \texttt{ \# Scalar matrix} \\
P_{k + 1} & = \Pi_{\Rbb^n_{+} \cap \Dcal} [P_k - \eta_p U_{k + 1}^{- 1 / 2} \nabla_P h_{x^k, x^{k - 1}} (P_k, \beta_k)] \texttt{ \# Diagonal matrix} \\
\beta_{k + 1} & = \Pi_{[0, 0.9995]} [\beta_k - \eta_b v_{k + 1}^{- 1 / 2} \nabla_{\beta} h_{x^k, x^{k - 1}} (P_k, \beta_k)] \\
x^{k + 1} & = \argmin_{x \in \{ x^k, x^{k + 1 / 2} \}} f (x).
\end{aligned}
$
}
{\textbf{output} $x^{K+1}$}
\caption{{\hdmbest} \label{alg:practical}}
\end{algorithm}

We make several remarks about \Cref{alg:practical}. 

\begin{itemize}[leftmargin=10pt]
\item \textit{Choice of online learning algorithm.} Unless $f(x)$ is quadratic, adaptive online learning algorithms such as {\adagrad} often significantly outperform online gradient descent with constant stepsize. Note that {\adagrad} introduces additional memory of size $n$ to store the diagonal online learning preconditioner $U$.

\item \textit{Sensitivity of parameters.} The two stepsize parameters in {\adagrad} are the most important algorithm parameters: $\eta_p, \eta_b$. According to the experiments, $\eta_p$ should be set proportional to $1/L$, the smoothness constant, while an aggressive choice of $\eta_b \in \{1,10,100\}$ often yields fast convergence. A local estimator of the smoothness constant $L$ can significantly enhance algorithm performance.
\item \textit{Heavy-ball feedback and null step.} In practice, it is observed that dropping the $\frac{\omega}{2}\|x^+(P, B) - x\|^2$ in the numerator of heavy-ball feedback \eqref{eqn:heavyball-feedback} often does not affect algorithm performance. Therefore, in \Cref{alg:practical} the hypergradient with respect to $\frac{\omega}{2}\|x^+(P, B) - x\|^2$ is ignored. 
On the other hand, the $\frac{\tau} {2}\|x^+(P, B) - x\|^2$  term in the denominator smoothes the update of $\beta_k$ and can strongly affect convergence. The parameter $\tau$ should be taken to be proportional to $L^2$ according to the discussions in \Cref{app:heavy-ball}. The null step is taken with respect to the function value $f(x)$ instead of the heavy-ball potential function.
\item \textit{Memory usage.} The memory usage of {\hdmbest}, measured in terms of number of vectors of length $n$ is $7n$: 1) three vectors store primal iterates $x^{-}, x, x^{+}$. 2) Two vectors store past and buffer gradients $\nabla f(x), \nabla f(x^+)$. 3) A vector stores the diagonal preconditioner $P_k$. 4) A vector stores the {\adagrad} stepsize matrix $U$.
\item \textit{Computational cost. } The major additional computation cost arises from computing hypergradient $\nabla h$, which involves one element-wise product and one inner product for vectors of size $n$. In addition, {\hdmbest} needs to maintain a diagonal matrix for {\adagrad}. The overall additional computational cost is several $\Ocal({n})$ operations.
\end{itemize}

\newpage
\section{Proofs for Section~\ref{sec:algo}}\label{app:proof}
% We first introduce a technical lemma which decompose the performance difference into value differences at each local state.
% \begin{lemma}\label{lem:value_diff}
% For any policies $\pi_1,\pi_2$ and $\pi$, the following equality holds
% \begin{align*}
% J(\pi_1,\pi)-J(\pi_2,\pi)=\E_{\pi_1}\bra{\sum_{h=1}^H \inner{\pi_1-\pi_2, Q^{\pi_2,\pi}}(s_h)}.
% \end{align*}
% \end{lemma}
% \begin{proof}
% For any state $s_h$, we have
% \begin{align*}
% V^{\pi_1,\pi}(s_h)-V^{\pi_2,\pi}(s_h)&=\inner{\pi_1,Q^{\pi_1,\pi}}(s_h)-\inner{\pi_2,Q^{\pi_2,\pi}}(s_h) \\
% &=\inner{\pi_1-\pi_2,Q^{\pi_2,\pi}}(s_h)+\inner{\pi_1,Q^{\pi_1,\pi}-Q^{\pi_2,\pi}}(s_h) \\
% &=\inner{\pi_1-\pi_2,Q^{\pi_2,\pi}}(s_h)+\E_{\pi_1}\bra{V^{\pi_1,\pi}(s_{h+1})-V^{\pi_2,\pi}(s_{h+1}) \mid s_h}.
% \end{align*}
% By recursively applying this to the initial state $s_1$, we obtain
% \begin{align*}
% J(\pi_1,\pi)-J(\pi_2,\pi)&=V^{\pi_1,\pi}(s_1)-V^{\pi_2,\pi}(s_1) \\
% &=\E_{\pi_1}\bra{\sum_{h=1}^H \inner{\pi_1-\pi_2, Q^{\pi_2,\pi}}(s_h)}+\E_{\pi_1}\bra{V^{\pi_1,\pi}(s_{H+1})-V^{\pi_2,\pi}(s_{H+1})} \\
% &=\E_{\pi_1}\bra{\sum_{h=1}^H \inner{\pi_1-\pi_2, Q^{\pi_2,\pi}}(s_h)}.
% \end{align*}
% The last equality is because $V^{\pi_1,\pi}(s_{H+1})=V^{\pi_2,\pi}(s_{H+1})=\Pcal(s_{H+1} \succ \pi)$ holds for any $s_{H+1}$.
% \end{proof}
% \subsection{Proof for Theorem~\ref{thm:omd_guarantee}}
% \begin{proof}
% First, according to the classical regret analysis of OMD~\citep{lattimore2020bandit}, we have for any policy $\pi$ and state $s_h$:
% \begin{align}
% \sum_{t=1}^T \inner{\pi-\pi_t,Q^{\pi_{t},\pi_t}}(s_h) &\le \frac{\KL(\pi(\cdot \mid s_h) \Vert \pi_1(\cdot \mid s_h))}{\eta}+\eta \sum_{t=1}^T \|Q^{\pi_t,\pi_t}(s_h,\cdot)\|^2_{\infty} \nonumber \\
% &\le \frac{D}{\eta}+\eta T =2\sqrt{TD}. \label{eq:omd_regret}
% \end{align}
% Then, we decompose the duality gap as:
% \begin{align*}
% \mathrm{DualGap}(\bar \pi)=\underbrace{\max_{\pi_1} J(\pi_1,\bar \pi)-\frac{1}{2}}_{\textrm{Term A}}+\underbrace{\frac{1}{2}-\min_{\pi_2}J(\bar \pi, \pi_2)}_{\textrm{Term B}}.
% \end{align*}
% Then we show how to bound Term A and Term B is bounded similarly due to the symmetric nature of the game. Let $\pi'=\argmax_{\pi_1} J(\pi_1,\bar \pi)$, we have
% \begin{align*}
% J(\pi',\bar \pi)-\frac{1}{2}&=\frac{1}{T}\sum_{t=1}^T J(\pi',\pi_t)-J(\pi_t,\pi_t) \\
% &=\frac{1}{T}\sum_{t=1}^T \E_{\pi'}\bra{\sum_{h=1}^H \inner{\pi'-\pi_t,Q^{\pi_t,\pi_t}}(s_h)} \\
% &=\frac{1}{T}\sum_{h=1}^H \E_{\pi'}\bra{\sum_{t=1}^T \inner{\pi'-\pi_t,Q^{\pi_t,\pi_t}}(s_h)} \\
% &\le\frac{2H\sqrt{D}}{\sqrt{T}}.
% \end{align*}
% The second equality is from Lemma \ref{lem:value_diff} and the inequality is from Eq.~\eqref{eq:omd_regret}. The proof is finished by also having $\frac{1}{2}-\min_{\pi_2}J(\bar \pi, \pi_2) \le \frac{2H\sqrt{D}}{\sqrt{T}}$.
% \end{proof}

\subsection{Proof for Theorem~\ref{thm:omd_guarantee}}\label{sec:proof_omd}
\begin{proof}
According to the regret analysis of OMD~\citep{lattimore2020bandit}, for any policy $\pi$, we have
\begin{align*}
\sum_{t=1}^T \inner{\pi,r_t}-\sum_{t=1}^T \inner{\pi_t,r_t} &\le \frac{\KL(\pi \Vert \pi_1)}{\eta}+\eta \sum_{t=1}^T \|r_t\|^2_{\infty} \\
&\le 2 \sqrt{TD}.
\end{align*}
The rest proof follows from Theorem 3 in~\citet{zhang2024iterative}.
\end{proof}

\subsection{Proof for Theorem~\ref{thm:onpo_regret}}\label{sec:proof_onpo}
\begin{proof}
Let $\psi(\pi)=\sum_{y} \pi(y) \log \pi(y)$, the KL divergence between $\pi_1$ and $\pi_2$ can also be written as the Bregman divergence term:
\begin{align*}
\KL(\pi_1 \Vert \pi_2)=D_{\psi}(\pi_1,\pi_2)=\psi(\pi_1)-\psi(\pi_2)-\inner{\nabla \psi(\pi_2),\pi_1-\pi_2}.
\end{align*}
Since $\psi$ is strongly convex with respect to $L_1$ norm, we can apply regret analysis from~\citet{rakhlin2013optimization,syrgkanis2015fast} and obtain that for any $\pi'$
\begin{align*}
\sum_{t=1}^T \inner{\pi'-\pi_t,r_t} \le \frac{\KL(\pi' \Vert \pi'_1)}{\eta}+\eta \sum_{t=1}^T \|r_t-r_{t-1}\|^2_{\infty}-\frac{1}{4 \eta}\sum_{t=2}^T \|\pi_t-\pi_{t-1}\|_1^2.
\end{align*}
We observe that for any $t \ge 2$ and any $y$, 
$$
|r_t(y)-r_{t-1}(y)|=|\sum_{y'} \mathbb{P}(y \succ y')(\pi_t(y)-\pi_{t-1}(y))| \le \|\pi_t-\pi_{t-1}\|_1.$$
Once we have $\frac{1}{4 \eta} \ge \eta$, the terms $\eta \|r_t-r_{t-1}\|^2_{\infty}$ and $-\frac{1}{4 \eta}\|\pi_t-\pi_{t-1}\|^2_1$ cancel out and we get
\begin{align*}
\sum_{t=1}^T \inner{\pi'-\pi_t,r_t} \le 2 \sqrt{D}.
\end{align*}
Next, we decompose the duality gap as:
\begin{align*}
\mathrm{DualGap}(\bar \pi)=\underbrace{\max_{\pi_1} J(\pi_1,\bar \pi)-\frac{1}{2}}_{\textrm{Term A}}+\underbrace{\frac{1}{2}-\min_{\pi_2}J(\bar \pi, \pi_2)}_{\textrm{Term B}}.
\end{align*}
We show how to bound Term A and Term B is bounded similarly due to the symmetric nature of the game. Let $\pi'=\argmax_{\pi_1} J(\pi_1,\bar \pi)$, we have
\begin{align*}
J(\pi',\bar \pi)-\frac{1}{2}&=\frac{1}{T}\sum_{t=1}^T J(\pi',\pi_t)-J(\pi_t,\pi_t) \\
&=\frac{1}{T}\sum_{t=1}^T \inner{\pi'-\pi_t,r_t} \\
&\le\frac{2\sqrt{D}}{T}.
\end{align*}
The proof is finished by also having $\frac{1}{2}-\min_{\pi_2}J(\bar \pi, \pi_2) \le \frac{2\sqrt{D}}{T}$.
\end{proof}

\section{Additional Experiments} \label{app:additional}


%\subsection{Ablation Study of {\hdmbest}}
%
%This section evaluates the effect of different components in {\hdmbest}, including null-step and {\adagrad}. In particular, we consider the following versions of {\hdmbest}
%
%\begin{itemize}[leftmargin=15pt]
%\item \texttt{HDM Raw}.\\
%{\hdmbest} without null step and online gradient descent with constant stepsize is used.
%\item \texttt{HDM+Null step}.\\
%{\hdmbest} with null step and online gradient descent with constant stepsize is used.
%\item \texttt{HDM+Null step+AdaGrad}.\\
%{\hdmbest} with all the components.
%\end{itemize}
%
%\begin{figure}[h]
%\includegraphics[scale=0.2]{figs/a2a_ablation_fval.pdf}
%\includegraphics[scale=0.2]{figs/a3a_ablation_fval.pdf}
%\includegraphics[scale=0.2]{figs/w2a_ablation_fval.pdf}
%\includegraphics[scale=0.2]{figs/w3a_ablation_fval.pdf}
%\caption{Ablation on different components of {\hdmbest} \label{fig:ablation}}
%\end{figure}
%
%As \Cref{fig:ablation} shows, both the null step and {\adagrad} bring significant speedup and justify our theoretical results.

\subsection{Additional Experiments on Support Vector Machine Problems}
See \Cref{fig:svm-add-1} and \Cref{fig:svm-add-2}.

\begin{figure}[!h]
\centering
% First Row: Function Values
\includegraphics[scale=0.2]{figs/a1a_objval_svm.pdf}
\includegraphics[scale=0.2]{figs/a2a_objval_svm.pdf}
\includegraphics[scale=0.2]{figs/a3a_objval_svm.pdf}
\includegraphics[scale=0.2]{figs/a4a_objval_svm.pdf}
\\
% Second Row: Gradient Norms
\includegraphics[scale=0.2]{figs/a1a_gnorm_svm.pdf}
\includegraphics[scale=0.2]{figs/a2a_gnorm_svm.pdf}
\includegraphics[scale=0.2]{figs/a3a_gnorm_svm.pdf}
\includegraphics[scale=0.2]{figs/a4a_gnorm_svm.pdf}
\\
% Third Row: Function Values
\includegraphics[scale=0.2]{figs/a5a_objval_svm.pdf}
\includegraphics[scale=0.2]{figs/a6a_objval_svm.pdf}
\includegraphics[scale=0.2]{figs/a7a_objval_svm.pdf}
\includegraphics[scale=0.2]{figs/a8a_objval_svm.pdf}
\\
% Fourth Row: Gradient Norms
\includegraphics[scale=0.2]{figs/a5a_gnorm_svm.pdf}
\includegraphics[scale=0.2]{figs/a6a_gnorm_svm.pdf}
\includegraphics[scale=0.2]{figs/a7a_gnorm_svm.pdf}
\includegraphics[scale=0.2]{figs/a8a_gnorm_svm.pdf}
\\
%% Fifth Row: Function Values
%\includegraphics[scale=0.2]{figs/a9a_objval_svm.pdf}
%\includegraphics[scale=0.2]{figs/australian_scale_objval_svm.pdf}
%\includegraphics[scale=0.2]{figs/fourclass_scale_objval_svm.pdf}
%\includegraphics[scale=0.2]{figs/ijcnn1_objval_svm.pdf}
%\\
%% Sixth Row: Gradient Norms
%\includegraphics[scale=0.2]{figs/a9a_gnorm_svm.pdf}
%\includegraphics[scale=0.2]{figs/australian_scale_gnorm_svm.pdf}
%\includegraphics[scale=0.2]{figs/fourclass_scale_gnorm_svm.pdf}
%\includegraphics[scale=0.2]{figs/ijcnn1_gnorm_svm.pdf}
%\\
% Add the legend
\includegraphics[scale=0.38]{figs/legend.pdf}
\caption{More experiments on support vector-machine problem}
\label{fig:svm-add-1}
\end{figure}


\begin{figure}

\centering
% Ninth Row: Function Values
\includegraphics[scale=0.2]{figs/w1a_objval_svm.pdf}
\includegraphics[scale=0.2]{figs/w2a_objval_svm.pdf}
\includegraphics[scale=0.2]{figs/w3a_objval_svm.pdf}
\includegraphics[scale=0.2]{figs/w4a_objval_svm.pdf}
\\
% Tenth Row: Gradient Norms
\includegraphics[scale=0.2]{figs/w1a_gnorm_svm.pdf}
\includegraphics[scale=0.2]{figs/w2a_gnorm_svm.pdf}
\includegraphics[scale=0.2]{figs/w3a_gnorm_svm.pdf}
\includegraphics[scale=0.2]{figs/w4a_gnorm_svm.pdf}
\\
% Eleventh Row: Function Values
\includegraphics[scale=0.2]{figs/w5a_objval_svm.pdf}
\includegraphics[scale=0.2]{figs/w6a_objval_svm.pdf}
\includegraphics[scale=0.2]{figs/w7a_objval_svm.pdf}
\includegraphics[scale=0.2]{figs/w8a_objval_svm.pdf}
\\
% Twelfth Row: Gradient Norms
\includegraphics[scale=0.2]{figs/w5a_gnorm_svm.pdf}
\includegraphics[scale=0.2]{figs/w6a_gnorm_svm.pdf}
\includegraphics[scale=0.2]{figs/w7a_gnorm_svm.pdf}
\includegraphics[scale=0.2]{figs/w8a_gnorm_svm.pdf}
	
\includegraphics[scale=0.38]{figs/legend.pdf}
\caption{More experiments on support vector-machine problem}
\label{fig:svm-add-2}
\end{figure}

\subsection{Additional Experiments on Logistic Regression Problems}
See \Cref{fig:log-add-1} and \Cref{fig:log-add-2}.

\begin{figure}[!h]
\centering
% First Row: Function Values
\includegraphics[scale=0.2]{figs/a1a_objval_logistic.pdf}
\includegraphics[scale=0.2]{figs/a2a_objval_logistic.pdf}
\includegraphics[scale=0.2]{figs/a3a_objval_logistic.pdf}
\includegraphics[scale=0.2]{figs/a4a_objval_logistic.pdf}
\\
% Second Row: Gradient Norms
\includegraphics[scale=0.2]{figs/a1a_gnorm_logistic.pdf}
\includegraphics[scale=0.2]{figs/a2a_gnorm_logistic.pdf}
\includegraphics[scale=0.2]{figs/a3a_gnorm_logistic.pdf}
\includegraphics[scale=0.2]{figs/a4a_gnorm_logistic.pdf}
\\
% Third Row: Function Values
\includegraphics[scale=0.2]{figs/a5a_objval_logistic.pdf}
\includegraphics[scale=0.2]{figs/a6a_objval_logistic.pdf}
\includegraphics[scale=0.2]{figs/a7a_objval_logistic.pdf}
\includegraphics[scale=0.2]{figs/a8a_objval_logistic.pdf}
\\
% Fourth Row: Gradient Norms
\includegraphics[scale=0.2]{figs/a5a_gnorm_logistic.pdf}
\includegraphics[scale=0.2]{figs/a6a_gnorm_logistic.pdf}
\includegraphics[scale=0.2]{figs/a7a_gnorm_logistic.pdf}
\includegraphics[scale=0.2]{figs/a8a_gnorm_logistic.pdf}
\\
% Fifth Row: Function Values
\includegraphics[scale=0.2]{figs/a9a_objval_logistic.pdf}
\includegraphics[scale=0.2]{figs/australian_scale_objval_logistic.pdf}
\includegraphics[scale=0.2]{figs/fourclass_scale_objval_logistic.pdf}
\includegraphics[scale=0.2]{figs/ijcnn1_objval_logistic.pdf}
\\
% Sixth Row: Gradient Norms
\includegraphics[scale=0.2]{figs/a9a_gnorm_logistic.pdf}
\includegraphics[scale=0.2]{figs/australian_scale_gnorm_logistic.pdf}
\includegraphics[scale=0.2]{figs/fourclass_scale_gnorm_logistic.pdf}
\includegraphics[scale=0.2]{figs/ijcnn1_gnorm_logistic.pdf}
\\
% Add the legend
\includegraphics[scale=0.38]{figs/legend.pdf}
\caption{More experiments on logistic regression problem}
\label{fig:log-add-1}
\end{figure}

\begin{figure}

\centering
% Ninth Row: Function Values
\includegraphics[scale=0.2]{figs/w1a_objval_logistic.pdf}
\includegraphics[scale=0.2]{figs/w2a_objval_logistic.pdf}
\includegraphics[scale=0.2]{figs/w3a_objval_logistic.pdf}
\includegraphics[scale=0.2]{figs/w4a_objval_logistic.pdf}
\\
% Tenth Row: Gradient Norms
\includegraphics[scale=0.2]{figs/w1a_gnorm_logistic.pdf}
\includegraphics[scale=0.2]{figs/w2a_gnorm_logistic.pdf}
\includegraphics[scale=0.2]{figs/w3a_gnorm_logistic.pdf}
\includegraphics[scale=0.2]{figs/w4a_gnorm_logistic.pdf}
\\
% Eleventh Row: Function Values
\includegraphics[scale=0.2]{figs/w5a_objval_logistic.pdf}
\includegraphics[scale=0.2]{figs/w6a_objval_logistic.pdf}
\includegraphics[scale=0.2]{figs/w7a_objval_logistic.pdf}
\includegraphics[scale=0.2]{figs/w8a_objval_logistic.pdf}
\\
% Twelfth Row: Gradient Norms
\includegraphics[scale=0.2]{figs/w5a_gnorm_logistic.pdf}
\includegraphics[scale=0.2]{figs/w6a_gnorm_logistic.pdf}
\includegraphics[scale=0.2]{figs/w7a_gnorm_logistic.pdf}
\includegraphics[scale=0.2]{figs/w8a_gnorm_logistic.pdf}
	
\includegraphics[scale=0.38]{figs/legend.pdf}
\caption{More experiments on logistic regression problem}
\label{fig:log-add-2}
\end{figure}
\end{document}
