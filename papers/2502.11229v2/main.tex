\documentclass[english,ruled]{article}
\usepackage[T1]{fontenc}
\usepackage[latin9]{inputenc}
\usepackage{verbatim}
\usepackage{subcaption}
\usepackage{algorithm2e}
\usepackage{amsmath}
\usepackage{amsthm}
\usepackage{amssymb}
\usepackage{graphicx}
\usepackage{xcolor}
\usepackage{multicol}
\usepackage{multirow}
\usepackage{geometry}
\usepackage{booktabs}
\usepackage{enumitem}
\usepackage{setspace}
\makeatletter
\usepackage[toc,page,header]{appendix}
\usepackage{minitoc}
\usepackage{ifthen}
\newboolean{doublecolumn}
\setboolean{doublecolumn}{false}
\newboolean{arxiv}
\setboolean{arxiv}{true}
\usepackage{pgfplots}
\usepackage{pdflscape}
\usepackage{hyperref}
\usepackage{cleveref}
\usepackage{authblk}
%\usepackage{showlabels}
\pgfplotsset{compat=1.15}

\setlength{\parindent}{0pt}
\geometry{verbose,tmargin=3cm,bmargin=3cm,lmargin=2.4cm,rmargin=2.4cm}
\linespread{1.0}

%%%%%%%%%% Start TeXmacs macros
\newcommand{\assign}{:=}
\newcommand{\backassign}{=:}
\newcommand{\cdummy}{\cdot}
\newcommand{\tmtextbf}[1]{\text{{\bfseries{#1}}}}
\newcommand{\tmop}[1]{\ensuremath{\operatorname{#1}}}
\newcommand{\tmtextit}[1]{\text{{\itshape{#1}}}}
\newcommand{\maxf}[1]{\underset{#1}{\text{maximize}}}
\newcommand{\minf}[1]{\underset{#1}{\text{minimize}}}
\newcommand{\hdm}{{\texttt{HDM}}}
\newcommand{\hdmagd}{{\texttt{HDM-AGD}}}
\newcommand{\hdmhb}{{\texttt{HDM-HB}}}
\newcommand{\hdmbest}{{\texttt{HDM-Best}}}
\newcommand{\ogm}{{\texttt{OGD}}}
\newcommand{\bfgs}{{\texttt{BFGS}}}
\newcommand{\lbfgs}{{\texttt{L-BFGS}}}
\newcommand{\osgm}{{\texttt{OSGM}}}
\newcommand{\osgmrx}{\texttt{OSGM-R}}
\newcommand{\osgmrzx}{\texttt{OSGM-RZ}}
\newcommand{\osgmgx}{\texttt{OSGM-G}}
\newcommand{\osgmhx}{\texttt{OSGM-H}}
\newcommand{\gd}{{\texttt{GD}}}
\newcommand{\optdgd}{{\texttt{OptDiagGD}}}
\newcommand{\adagrad}{{\texttt{AdaGrad}}}
\newcommand{\adam}{{\texttt{Adam}}}
\newcommand{\sagd}{{\texttt{SAGD}}}
\newcommand{\agd}{{\texttt{AGD}}}
\newcommand{\mathd}{\mathrm{d}}
\newcommand{\nin}{\not\in}
\newcommand{\TODO}[1]{\textcolor{red}{TODO:  #1}}
%%%%%%%%%% End TeXmacs macros


%
\setlength\unitlength{1mm}
\newcommand{\twodots}{\mathinner {\ldotp \ldotp}}
% bb font symbols
\newcommand{\Rho}{\mathrm{P}}
\newcommand{\Tau}{\mathrm{T}}

\newfont{\bbb}{msbm10 scaled 700}
\newcommand{\CCC}{\mbox{\bbb C}}

\newfont{\bb}{msbm10 scaled 1100}
\newcommand{\CC}{\mbox{\bb C}}
\newcommand{\PP}{\mbox{\bb P}}
\newcommand{\RR}{\mbox{\bb R}}
\newcommand{\QQ}{\mbox{\bb Q}}
\newcommand{\ZZ}{\mbox{\bb Z}}
\newcommand{\FF}{\mbox{\bb F}}
\newcommand{\GG}{\mbox{\bb G}}
\newcommand{\EE}{\mbox{\bb E}}
\newcommand{\NN}{\mbox{\bb N}}
\newcommand{\KK}{\mbox{\bb K}}
\newcommand{\HH}{\mbox{\bb H}}
\newcommand{\SSS}{\mbox{\bb S}}
\newcommand{\UU}{\mbox{\bb U}}
\newcommand{\VV}{\mbox{\bb V}}


\newcommand{\yy}{\mathbbm{y}}
\newcommand{\xx}{\mathbbm{x}}
\newcommand{\zz}{\mathbbm{z}}
\newcommand{\sss}{\mathbbm{s}}
\newcommand{\rr}{\mathbbm{r}}
\newcommand{\pp}{\mathbbm{p}}
\newcommand{\qq}{\mathbbm{q}}
\newcommand{\ww}{\mathbbm{w}}
\newcommand{\hh}{\mathbbm{h}}
\newcommand{\vvv}{\mathbbm{v}}

% Vectors

\newcommand{\av}{{\bf a}}
\newcommand{\bv}{{\bf b}}
\newcommand{\cv}{{\bf c}}
\newcommand{\dv}{{\bf d}}
\newcommand{\ev}{{\bf e}}
\newcommand{\fv}{{\bf f}}
\newcommand{\gv}{{\bf g}}
\newcommand{\hv}{{\bf h}}
\newcommand{\iv}{{\bf i}}
\newcommand{\jv}{{\bf j}}
\newcommand{\kv}{{\bf k}}
\newcommand{\lv}{{\bf l}}
\newcommand{\mv}{{\bf m}}
\newcommand{\nv}{{\bf n}}
\newcommand{\ov}{{\bf o}}
\newcommand{\pv}{{\bf p}}
\newcommand{\qv}{{\bf q}}
\newcommand{\rv}{{\bf r}}
\newcommand{\sv}{{\bf s}}
\newcommand{\tv}{{\bf t}}
\newcommand{\uv}{{\bf u}}
\newcommand{\wv}{{\bf w}}
\newcommand{\vv}{{\bf v}}
\newcommand{\xv}{{\bf x}}
\newcommand{\yv}{{\bf y}}
\newcommand{\zv}{{\bf z}}
\newcommand{\zerov}{{\bf 0}}
\newcommand{\onev}{{\bf 1}}

% Matrices

\newcommand{\Am}{{\bf A}}
\newcommand{\Bm}{{\bf B}}
\newcommand{\Cm}{{\bf C}}
\newcommand{\Dm}{{\bf D}}
\newcommand{\Em}{{\bf E}}
\newcommand{\Fm}{{\bf F}}
\newcommand{\Gm}{{\bf G}}
\newcommand{\Hm}{{\bf H}}
\newcommand{\Id}{{\bf I}}
\newcommand{\Jm}{{\bf J}}
\newcommand{\Km}{{\bf K}}
\newcommand{\Lm}{{\bf L}}
\newcommand{\Mm}{{\bf M}}
\newcommand{\Nm}{{\bf N}}
\newcommand{\Om}{{\bf O}}
\newcommand{\Pm}{{\bf P}}
\newcommand{\Qm}{{\bf Q}}
\newcommand{\Rm}{{\bf R}}
\newcommand{\Sm}{{\bf S}}
\newcommand{\Tm}{{\bf T}}
\newcommand{\Um}{{\bf U}}
\newcommand{\Wm}{{\bf W}}
\newcommand{\Vm}{{\bf V}}
\newcommand{\Xm}{{\bf X}}
\newcommand{\Ym}{{\bf Y}}
\newcommand{\Zm}{{\bf Z}}

% Calligraphic

\newcommand{\Ac}{{\cal A}}
\newcommand{\Bc}{{\cal B}}
\newcommand{\Cc}{{\cal C}}
\newcommand{\Dc}{{\cal D}}
\newcommand{\Ec}{{\cal E}}
\newcommand{\Fc}{{\cal F}}
\newcommand{\Gc}{{\cal G}}
\newcommand{\Hc}{{\cal H}}
\newcommand{\Ic}{{\cal I}}
\newcommand{\Jc}{{\cal J}}
\newcommand{\Kc}{{\cal K}}
\newcommand{\Lc}{{\cal L}}
\newcommand{\Mc}{{\cal M}}
\newcommand{\Nc}{{\cal N}}
\newcommand{\nc}{{\cal n}}
\newcommand{\Oc}{{\cal O}}
\newcommand{\Pc}{{\cal P}}
\newcommand{\Qc}{{\cal Q}}
\newcommand{\Rc}{{\cal R}}
\newcommand{\Sc}{{\cal S}}
\newcommand{\Tc}{{\cal T}}
\newcommand{\Uc}{{\cal U}}
\newcommand{\Wc}{{\cal W}}
\newcommand{\Vc}{{\cal V}}
\newcommand{\Xc}{{\cal X}}
\newcommand{\Yc}{{\cal Y}}
\newcommand{\Zc}{{\cal Z}}

% Bold greek letters

\newcommand{\alphav}{\hbox{\boldmath$\alpha$}}
\newcommand{\betav}{\hbox{\boldmath$\beta$}}
\newcommand{\gammav}{\hbox{\boldmath$\gamma$}}
\newcommand{\deltav}{\hbox{\boldmath$\delta$}}
\newcommand{\etav}{\hbox{\boldmath$\eta$}}
\newcommand{\lambdav}{\hbox{\boldmath$\lambda$}}
\newcommand{\epsilonv}{\hbox{\boldmath$\epsilon$}}
\newcommand{\nuv}{\hbox{\boldmath$\nu$}}
\newcommand{\muv}{\hbox{\boldmath$\mu$}}
\newcommand{\zetav}{\hbox{\boldmath$\zeta$}}
\newcommand{\phiv}{\hbox{\boldmath$\phi$}}
\newcommand{\psiv}{\hbox{\boldmath$\psi$}}
\newcommand{\thetav}{\hbox{\boldmath$\theta$}}
\newcommand{\tauv}{\hbox{\boldmath$\tau$}}
\newcommand{\omegav}{\hbox{\boldmath$\omega$}}
\newcommand{\xiv}{\hbox{\boldmath$\xi$}}
\newcommand{\sigmav}{\hbox{\boldmath$\sigma$}}
\newcommand{\piv}{\hbox{\boldmath$\pi$}}
\newcommand{\rhov}{\hbox{\boldmath$\rho$}}
\newcommand{\upsilonv}{\hbox{\boldmath$\upsilon$}}

\newcommand{\Gammam}{\hbox{\boldmath$\Gamma$}}
\newcommand{\Lambdam}{\hbox{\boldmath$\Lambda$}}
\newcommand{\Deltam}{\hbox{\boldmath$\Delta$}}
\newcommand{\Sigmam}{\hbox{\boldmath$\Sigma$}}
\newcommand{\Phim}{\hbox{\boldmath$\Phi$}}
\newcommand{\Pim}{\hbox{\boldmath$\Pi$}}
\newcommand{\Psim}{\hbox{\boldmath$\Psi$}}
\newcommand{\Thetam}{\hbox{\boldmath$\Theta$}}
\newcommand{\Omegam}{\hbox{\boldmath$\Omega$}}
\newcommand{\Xim}{\hbox{\boldmath$\Xi$}}


% Sans Serif small case

\newcommand{\Gsf}{{\sf G}}

\newcommand{\asf}{{\sf a}}
\newcommand{\bsf}{{\sf b}}
\newcommand{\csf}{{\sf c}}
\newcommand{\dsf}{{\sf d}}
\newcommand{\esf}{{\sf e}}
\newcommand{\fsf}{{\sf f}}
\newcommand{\gsf}{{\sf g}}
\newcommand{\hsf}{{\sf h}}
\newcommand{\isf}{{\sf i}}
\newcommand{\jsf}{{\sf j}}
\newcommand{\ksf}{{\sf k}}
\newcommand{\lsf}{{\sf l}}
\newcommand{\msf}{{\sf m}}
\newcommand{\nsf}{{\sf n}}
\newcommand{\osf}{{\sf o}}
\newcommand{\psf}{{\sf p}}
\newcommand{\qsf}{{\sf q}}
\newcommand{\rsf}{{\sf r}}
\newcommand{\ssf}{{\sf s}}
\newcommand{\tsf}{{\sf t}}
\newcommand{\usf}{{\sf u}}
\newcommand{\wsf}{{\sf w}}
\newcommand{\vsf}{{\sf v}}
\newcommand{\xsf}{{\sf x}}
\newcommand{\ysf}{{\sf y}}
\newcommand{\zsf}{{\sf z}}


% mixed symbols

\newcommand{\sinc}{{\hbox{sinc}}}
\newcommand{\diag}{{\hbox{diag}}}
\renewcommand{\det}{{\hbox{det}}}
\newcommand{\trace}{{\hbox{tr}}}
\newcommand{\sign}{{\hbox{sign}}}
\renewcommand{\arg}{{\hbox{arg}}}
\newcommand{\var}{{\hbox{var}}}
\newcommand{\cov}{{\hbox{cov}}}
\newcommand{\Ei}{{\rm E}_{\rm i}}
\renewcommand{\Re}{{\rm Re}}
\renewcommand{\Im}{{\rm Im}}
\newcommand{\eqdef}{\stackrel{\Delta}{=}}
\newcommand{\defines}{{\,\,\stackrel{\scriptscriptstyle \bigtriangleup}{=}\,\,}}
\newcommand{\<}{\left\langle}
\renewcommand{\>}{\right\rangle}
\newcommand{\herm}{{\sf H}}
\newcommand{\trasp}{{\sf T}}
\newcommand{\transp}{{\sf T}}
\renewcommand{\vec}{{\rm vec}}
\newcommand{\Psf}{{\sf P}}
\newcommand{\SINR}{{\sf SINR}}
\newcommand{\SNR}{{\sf SNR}}
\newcommand{\MMSE}{{\sf MMSE}}
\newcommand{\REF}{{\RED [REF]}}

% Markov chain
\usepackage{stmaryrd} % for \mkv 
\newcommand{\mkv}{-\!\!\!\!\minuso\!\!\!\!-}

% Colors

\newcommand{\RED}{\color[rgb]{1.00,0.10,0.10}}
\newcommand{\BLUE}{\color[rgb]{0,0,0.90}}
\newcommand{\GREEN}{\color[rgb]{0,0.80,0.20}}

%%%%%%%%%%%%%%%%%%%%%%%%%%%%%%%%%%%%%%%%%%
\usepackage{hyperref}
\hypersetup{
    bookmarks=true,         % show bookmarks bar?
    unicode=false,          % non-Latin characters in AcrobatÕs bookmarks
    pdftoolbar=true,        % show AcrobatÕs toolbar?
    pdfmenubar=true,        % show AcrobatÕs menu?
    pdffitwindow=false,     % window fit to page when opened
    pdfstartview={FitH},    % fits the width of the page to the window
%    pdftitle={My title},    % title
%    pdfauthor={Author},     % author
%    pdfsubject={Subject},   % subject of the document
%    pdfcreator={Creator},   % creator of the document
%    pdfproducer={Producer}, % producer of the document
%    pdfkeywords={keyword1} {key2} {key3}, % list of keywords
    pdfnewwindow=true,      % links in new window
    colorlinks=true,       % false: boxed links; true: colored links
    linkcolor=red,          % color of internal links (change box color with linkbordercolor)
    citecolor=green,        % color of links to bibliography
    filecolor=blue,      % color of file links
    urlcolor=blue           % color of external links
}
%%%%%%%%%%%%%%%%%%%%%%%%%%%%%%%%%%%%%%%%%%%



\theoremstyle{plain}
\newtheorem{lem}{\protect\lemmaname}[section]
\theoremstyle{remark}
\newtheorem{rem}{\protect\remarkname}
\theoremstyle{plain}
\newtheorem{thm}{\protect\theoremname}[section]
\theoremstyle{plain}
\newtheorem{prop}{\protect\propositionname}[section]
\providecommand{\corollaryname}{Corollary}
\theoremstyle{plain}
\newtheorem{coro}{\protect\corollaryname}[section]
\theoremstyle{plain}
\newtheorem{exple}{\protect\examplename}[section]
\theoremstyle{plain}
\newtheorem{definition}{\protect\definitionname}[section]

\providecommand{\lemmaname}{Lemma}
\providecommand{\remarkname}{Remark}
\providecommand{\theoremname}{Theorem}
\providecommand{\examplename}{Example}
\providecommand{\propositionname}{Proposition}
\providecommand{\definitionname}{Definition}

% cleveref
\crefdefaultlabelformat{#2\textbf{#1}#3} % <-- Only #1 in \textbf
\crefname{section}{\textbf{section}}{\textbf{sections}}
\Crefname{section}{\textbf{Section}}{\textbf{Sections}}
\crefname{thm}{\textbf{theorem}}{\textbf{theorems}}
\Crefname{thm}{\textbf{Theorem}}{\textbf{Theorems}}
\crefname{lem}{\textbf{lemma}}{\textbf{lemmas}}
\Crefname{lem}{\textbf{Lemma}}{\textbf{Lemmas}}
\crefname{prop}{\textbf{proposition}}{\textbf{propositions}}
\Crefname{prop}{\textbf{Proposition}}{\textbf{Propositions}}
\crefname{algorithm}{\textbf{algorithm}}{\textbf{algorithms}}
\Crefname{algorithm}{\textbf{Algorithm}}{\textbf{Algorithms}}
\crefname{coro}{\textbf{Corollary}}{\textbf{corollaries}}
\Crefname{coro}{\textbf{Corollary}}{\textbf{corollaries}}
\crefname{definition}{\textbf{Definition}}{\textbf{definitions}}
\Crefname{definition}{\textbf{Definition}}{\textbf{definitions}}
\crefname{table}{\textbf{Table}}{\textbf{tables}}
\Crefname{table}{\textbf{Table}}{\textbf{tables}}
\crefname{figure}{\textbf{Figure}}{\textbf{figures}}
\Crefname{figure}{\textbf{Figure}}{\textbf{figures}}

% Surrogate loss shorthand
\newcommand{\rk}{r_{k}}
\newcommand{\rxz}{r^z_{x}}
\newcommand{\rxkz}{r^z_{x^k}}

% Comments
\newcommand{\YC}[1]{ }
\renewcommand{\YC}[1]{\textcolor{blue}{[YC: #1]}}
\newcommand{\gwz}[1]{\textcolor{cyan}{[gwz: #1]}}
\begin{document}

\title{Provable and Practical Online Learning Rate Adaptation with Hypergradient Descent}

\author[1]{Ya-Chi Chu\thanks{ycchu97@stanford.edu}}
\author[2]{Wenzhi Gao\thanks{gwz@stanford.edu, equal contribution}}
\author[2,3]{Yinyu Ye\thanks{yyye@stanford.edu}}
\author[2,3]{Madeleine Udell\thanks{udell@stanford.edu}}
\affil[1]{Department of Mathematics, Stanford University}
\affil[2]{ICME, Stanford University}
\affil[3]{Department of Management Science and Engineering, Stanford University}

\maketitle

\begin{abstract}  
Test time scaling is currently one of the most active research areas that shows promise after training time scaling has reached its limits.
Deep-thinking (DT) models are a class of recurrent models that can perform easy-to-hard generalization by assigning more compute to harder test samples.
However, due to their inability to determine the complexity of a test sample, DT models have to use a large amount of computation for both easy and hard test samples.
Excessive test time computation is wasteful and can cause the ``overthinking'' problem where more test time computation leads to worse results.
In this paper, we introduce a test time training method for determining the optimal amount of computation needed for each sample during test time.
We also propose Conv-LiGRU, a novel recurrent architecture for efficient and robust visual reasoning. 
Extensive experiments demonstrate that Conv-LiGRU is more stable than DT, effectively mitigates the ``overthinking'' phenomenon, and achieves superior accuracy.
\end{abstract}  
\section{Introduction}

Node classification is a fundamental task in graph analysis, with a wide range of applications such as item tagging \cite{Mao2020ItemTF}, user profiling \cite{Yan2021RelationawareHG}, and financial fraud detection \cite{Zhang2022eFraudComAE}. Developing effective algorithms for node classification is crucial, as they can significantly impact commercial success. For instance, US banks lost 6 billion USD to fraudsters in 2016. Therefore, even a marginal improvement in fraud detection accuracy could result in substantial financial savings.

Given its practical importance, node classification has been a long-standing research focus in both academia and industry. The earliest attempts to address this task adopted techniques such as Laplacian regularization \cite{belkin2006manifold}, graph embeddings \cite{yang2016revisiting}, and label propagation \cite{zhu2003semi}. Over the past decade, GNN-based methods have been developed and have quickly become prominent due to their superior performance, as demonstrated by works such as \citet{kipf2017GCN}, \citet{velickovic2018GAT}, and \citet{hamilton2017SAGE}. Additionally, the incorporation of encoded textual information has been shown to further complement GNNs' node features, enhancing their effectiveness \cite{jin2023patton, zhao2022GLEM}.

Inspired by the recent success of LLMs, there has been a surge of interest in leveraging LLMs for node classification \cite{li2023survey}. LLMs, pre-trained on extensive text corpora, possess context-aware knowledge and superior semantic comprehension, overcoming the limitations of the non-contextualized shallow embeddings used by traditional GNNs. Typically, supervised methods fall into three categories: Encoder, Reasoner, and Predictor. In the Encoder paradigm, LLMs employ their vast parameters to encode nodes' textual information, producing more expressive features that surpass shallow embeddings \cite{Zhu2024ENGINE}. The Reasoner approach utilizes LLMs' reasoning capabilities to enhance node attributes and the task descriptions with a more detailed text \cite{chen2024exploring, he2023TAPE}. This generated text augments the nodes' original information, thereby enriching their attributes. Lastly, the Predictor role involves LLMs integrating graph context through graph encoders, enabling direct text-based predictions  \cite{chen23llaga,tang2023graphgpt,chai2023graphllm,Huang2024GraphAdapter}. For zero-shot learning with LLMs, methods can be categorized into two types: Direct Inference and Graph Foundation Models (GFMs). Direct Inference involves guiding LLMs to directly perform classification tasks via crafted prompts \cite{Huang2023CanLE}. In contrast, GFMs entail pre-training on extensive graph corpora before applying the model to target graphs, thereby equipping the model with specialized graph intelligence \cite{li2024zerog}. An illustration of these methods is shown in Figure \ref{fig:llm_role}. 

Despite tremendous efforts and promising results, the design principles for LLM-based node classification algorithms remain elusive. Given the significant training and inference costs associated with LLMs, practitioners may opt to deploy these algorithms only when they provide substantial performance enhancements compared to costs. This study, therefore, seeks to identify \textbf{(1) the most suitable settings for each algorithm category, and (2) the scenarios where LLMs surpass traditional LMs such as BERT}. While recent work like GLBench \cite{Li2024GLBench} has evaluated various methods using consistent data splits in semi-supervised and zero-shot settings, differences in backbone architectures and implementation codebases still hinder fair comparisons and rigorous conclusions. To address these limitations, we introduce a new benchmark that further standardizes backbones and codebases. Additionally, we extend GLBench by incorporating three new E-Commerce datasets relevant to practical applications and expanding the evaluation settings. Specifically, we assess the impact of supervision signals (e.g., supervised, semi-supervised), different language model backbones (e.g., RoBERTa, Mistral, LLaMA, GPT-4o), and various prompt types (e.g., CoT, ToT, ReAct). These enhancements enable a more detailed and reliable analysis of LLM-based node classification methods. In summary, our contributions to the field of LLMs for graph analysis are as follows:


% A fair comparison necessitates a benchmark that evaluates all methods using consistent data splitting ratios, learning paradigms, backbone architectures, and implementation codebases. A very recent work, GLBench~\cite{Li2024GLBench}, tested various methods on several datasets in a semi-supervised/zero-shot setting, maintaining the same data splits. However, differences in the underlying backbones and implementation codebases still pose challenges for a fair comparison and drawing rigorous conclusions of the above questions. This paper introduces a benchmark that further standardizes the backbones and implementation codebases. Moreover, we expand upon GLBench by providing additional datasets and evaluation settings. Specifically, we include three new datasets from the E-Commerce sector, which are more relevant for practical commercial applications. We also assess the influence of supervision signals (e.g., supervised or semi-supervised), various language model backbones (e.g., RoBERTa, Mistral, GPT-4o), and prompts (e.g., CoT, ToT, and ReAct). These datasets and settings enable a detailed analysis of the aforementioned questions. 



% However, existing works lack the necessary standardization for such comparisons. An algorithm that performs exceptionally well in its original paper might underperform when used as a baseline in subsequent studies. This discrepancy often arises from variations in data splitting, learning paradigms, backbone architectures, and implementation codebases.  The backbone architecture and implementations are adopted from the original papers, which 

% To address this issue, this paper introduces a testbed for LLM-based node classification algorithms and conducts extensive experiments to derive insights and guidelines. 

\begin{itemize}
    \item \textbf{A Testbed:} We release LLMNodeBed, a PyG-based testbed designed to facilitate reproducible and rigorous research in LLM-based node classification algorithms. The initial release includes ten datasets, eight LLM-based algorithms, and three learning configurations. LLMNodeBed allows for easy addition of new algorithms or datasets, and a single command to run all experiments, and to automatically generate all tables included in this work.
    
    \item \textbf{Comprehensive Experiments:} By training and evaluating over 2,200 models, we analyzed how the learning paradigm, homophily, language model type and size, and prompt design impact the performance of each algorithm category.
    
    \item \textbf{Insights and Tips:} Detailed experiments were conducted to analyze each influencing factor. We identified the settings where each algorithm category performs best and the key components for achieving this performance. Our work provides intuitive explanations, practical tips, and insights about the strengths and limitations of each algorithm category.
\end{itemize}




%It has been a research focus in both academia and industry due to its wide range of applications, including item tagging \cite{Mao2020ItemTF}, user profiling \cite{Yan2021RelationawareHG}, and financial fraud detection \cite{Zhang2022eFraudComAE}. 


%Building effective algorithms for node classification is a long-standing topic as it has a direct impact on commercial success \cite{Lo2022InspectionLSG}.

%Before the popularity of LLMs, node classification is typically tackled by graph neural networks (GNNs) or language models (LMs) such as BERT \cite{Devlin2019BERTPO}. GNNs \cite{kipf2017GCN,velickovic2018GAT,hamilton2017SAGE} enhance node representations by aggregating information from neighboring nodes, thereby capturing the structural context essential for accurate classification. In contrast, LMs \cite{Wang2022e5-large, Liu2019roberta} focus on semantic representations by encoding the textual information associated with each node, transforming the node classification into a text classification task. The encoded textual information can further complement GNNs' node features \cite{jin2023patton, zhao2022GLEM}. Yifei: I think the current intro is too long, to move it to related works

%Over the past decade, we have witnessed great progress in node classification algorithms. The classical ones include Graph Neural Networks (GNNs) \cite{kipf2017GCN,velickovic2018GAT,hamilton2017SAGE} and additional language modeling to enhance the node features \cite{jin2023patton, zhao2022GLEM}. Recently, there has been a surge of interest in applying LLMs for node classification \cite{li2023survey}. In these studies, the roles performed by LLMs can be primarily 


% Despite the importance of this area, the literature of LLM-based node classification is scattered: the algorithms are evaluated under different datasets, learning paradigms, baselines, and implementation codebases. The purpose of this work is to perform rigorous comparisons among algorithms, as well as to open-source our software for anyone to replicate and extend our analysis. This manuscript investigates the question: \emph{How useful are LLMs for node classification under a fair setting?}

% To answer this question, we implement and tune eight LLM-based node classification algorithms, to compare them across ten datasets and three learning paradigms.  There are four major takeaways from our investigations: (1) \textbf{LLM-as-Encoder is effective for low-homophily graphs:} These methods outperform classic LM counterparts on low-homophily graphs, with the advantages being more obvious under limited supervision.
% (2) \textbf{LLM-as-Reasoner is the most effective when LLMs have prior knowledge of the target graph:} These methods achieve superior performance on datasets where the LLMs possess prior knowledge like academic and web link datasets, and benefit from more powerful models like GPT-4o. 
% (3) \textbf{LLM-as-Predictor methods is highly effective when labeled data is abundant}: Predictor methods require extensive supervision for model training, with their performance improving as larger LLMs adhering to scaling laws \cite{Kaplan2020ScalingLF} are utilized. Among different LLMs, Mistral-7B \cite{Jiang2023Mistral7B} consistently serves as a robust backbone. (4) \textbf{Zero-shot methods are most effective when neighbor information is injected:} Although Graph Foundation Models (GFMs) \cite{liu2023one, li2024zerog, Zhu2024GraphCLIPET} outperform open-source LLMs in zero-shot settings, they still lag behind advanced models like GPT-4o. The most effective zero-shot approaches involve injecting neighbor information to guide LLMs for direct inference.

% As a result of this paper, we release LLMNodeBed, a PyTorch-based testbed designed to facilitate reproducible and rigorous research in node classification algorithms. The initial release includes ten datasets, eight algorithms, three learning configurations, and the infrastructure to run all experiments. Our experimental framework can be easily extended to include new methods and datasets. We are committed to updating this repository with new algorithms and datasets and welcome pull requests from fellow researchers to ensure its ongoing development.


%While a myriad of algorithms exists, diverse datasets, architectures, learning configurations, and implementation codebases, rendering fair and realistic comparisons difficult and conclusions inconsistent. Inspired by standardized benchmarks in computer vision like ImageNet, this paper conducts a rigorous comparison of various LLM-based node classification methods to assess the true efficacy of LLMs. This investigation addresses the following research question:

%\textit{Under What Circumstances do LLMs Help Node Classification Task?}

%At a first step, we implement LLMNodeBed, a codebase and testbed for node classification with LLMs. It includes ten multi-domain graph datasets with varying scales and levels of homophily, supports eight representative algorithms that represent diverse LLM roles, and offers three learning configurations: semi-supervised, fully-supervised, and zero-shot. Through extensive experiments, we provide empirical insights into when LLMs contribute to node classification performance: 



% In summary, we make the following contributions: 

% \begin{enumerate}
%     \item \textbf{LLMNodeBed:} We introduce LLMNodeBed, a comprehensive and extensible testbed for evaluating LLM-based node classification algorithms. It comprises ten datasets, eight representative algorithms, and three learning scenarios, and can easily accommodate new datasets, methods, and backbones.
%     \item \textbf{Comprehensive Evaluation:} We conduct extensive empirical analysis across different datasets, algorithms, and learning settings to elucidate the efficacy of different LLM roles in node classification performance. 
%     \item \textbf{Practical Guidelines:} Based on our findings, we provide actionable guidelines for effectively applying LLMs to diverse real-world node classification tasks, enhancing their performance and applicability in various scenarios.
% \end{enumerate}

% \section{Background}

% In this work, we focus on two different model families: random Fourier features (RFFs) and deep neural networks (DNNs) for transfer learning with informative priors.
% What these model families have in common is that they can be overparameterized.

%\subsection{Random Fourier features}

% MCH: MOVED TO CASE A

%\subsection{Transfer learning with informative priors}

% MCH: MOVED TO CASE B

\section{Main Results}
\label{sec:results}
We now state the soundness and completeness of the translation of the STL formulas into the transducers.

We propose here the equivalence of the Until and Release operator of STL and its transducer. %A similar proof is done for the Release operator of STL and its transducer.

\hscomment{1. Include the output alphabet $\top, \bot$, not the input (thius can be used to prove the transparancy), 2. CHange the notation. 3. Move it to Section 4.}
\begin{proposition}[Equivalence of Until operator of STL and its transducer]    
\label{propo1}
        Let $\signal$ be the signals and $\textsf{SignEncode}(\signal,\varphi_1\until_{[a,b]}\varphi_2)$ be its encoded timed word for the given STL formula $\varphi_1\until_{[a,b]}\varphi_2$.
        If the encoded timed word is accepted by the transducer of Until operator $\automaton_{\varphi_1\until_{[a,b]}\varphi_2}$ then the corresponding signals  $\signal$  satisfy $ \varphi_1\until_{[a,b]}\varphi_2$, i.e.,
        \begin{align*}
            \textsf{SignEncode}(\signal,\varphi_1\until_{[a,b]}\varphi_2) \in \mathcal{L}(\automaton_{\varphi_1\until_{[a,b]}\varphi_2}) \implies \signal \models \varphi_1\until_{[a,b]}\varphi_2
        \end{align*}
\end{proposition}

\begin{proposition}[Equivalence of Release operator of STL and its transducer]   
\label{propo2}
    %Let $\signal$ be the signals.
    %Let $\sigma$ be the timed word. 
    Let $\signal$ be the signals and $\textsf{SignEncode}(\signal,\varphi_1\release_{[a,b]}\varphi_2)$ be its encoded timed word for the given STL formula $\varphi_1\release_{[a,b]}\varphi_2$
    If the timed word is accepted by the transducer of Release operator $\automaton_{\varphi_1\release_{[a,b]}\varphi_2}$ then the corresponding  signals $\signal$ satisfy $ \varphi_1\release_{[a,b]}\varphi_2$, i.e.,
    \begin{align*}
        \textsf{SignEncode}(\signal,\varphi_1\release_{[a,b]}\varphi_2) \in \mathcal{L}(\automaton_{\varphi_1\release_{[a,b]}\varphi_2}) \implies \signal \models \varphi_1\release_{[a,b]}\varphi_2
    \end{align*}
\end{proposition}

%%%%%%%%%%%%%%%%%%%%%%%%%%%%%%%%%%%%%%%%%%%%%


\begin{proposition} 
\label{propo3}
    Let $\signal$ be the signals and let $\textsf{SignEncode}(\signal,\varphi_1\until_{[a,b]}\varphi_2)$ be the corresponding encoded timed word. If signals  satisfy the Until formula $ \varphi_1\until_{[a,b]}\varphi_2$, then its encoded word is accepted by its transducer, i.e., 
    \begin{align*}
        \signal \models \varphi_1\until_{[a,b]}\varphi_2 \implies \textsf{SignEncode}(\signal,\varphi_1\until_{[a,b]}\varphi_2) \in 
        \mathcal{L}(\automaton_{\varphi_1\until_{[a,b]}\varphi_2}) 
    \end{align*}
\end{proposition}

\begin{proposition} 
\label{propo4}
    Let $\signal$ be the signals and let $\textsf{SignEncode}(\signal,\varphi_1\release_{[a,b]}\varphi_2)$ be the corresponding encoded timed word. If signals  satisfy the Release formula $ \varphi_1\release_{[a,b]}\varphi_2$, then its encoded word is accepted by its transducer, i.e., 
    \begin{align*}
        \signal \models \varphi_1\release_{[a,b]}\varphi_2 \implies \textsf{SignEncode}(\signal,\varphi_1\release_{[a,b]}\varphi_2) \in 
        \mathcal{L}(\automaton_{\varphi_1\release_{[a,b]}\varphi_2}) 
    \end{align*}
\end{proposition}

% %%%%%%%%%%%%%%%%%%%%%%%%%%%%%%%%%%%%%%%%%%%%%


% \begin{proposition} 
% \label{propo5}
%     Let $\signal$ be the signals and let $\sigma$ be the corresponding encoded timed word. If signals do not satisfy the Until formula $ \varphi_1\until_{[a,b]}\varphi_2$, and the enforcer corrects the signals to $\signal'$ by minimally modifying $\signal$ and the encoded signals of the modified signals are accepted by the transducer of Until, then the modified signals satisfy  the Until formula, i.e., 
%     \begin{equation*}    
%         \begin{split}
%         \signal \not \models p_1\until_{[a,b]}p_2 \land \exists\signal', E_{p_1\until_{[a,b]}p_2}(\signal, t)=\signal' : min(|\signal'-\signal|)\\ \land \sigma' \in 
%         \mathcal{L}(\automaton_{\varphi_1\release_{[a,b]}\varphi_2}) 
%         \implies \signal' \models p_1\until_{[a,b]}p_2
%         \end{split}
%     \end{equation*}    
% \end{proposition}

% \begin{proposition} 
% \label{propo6}
%     Let $\signal$ be the signals and let $\sigma$ be the corresponding encoded timed word. If signals do not satisfy the Until formula $ \varphi_1\release_{[a,b]}\varphi_2$, and the enforcer corrects the signals to $\signal'$ by minimally modifying $\signal$ and the encoded signals of the modified signals are accepted by the transducer of Release, then the modified signals satisfy  the Release formula, i.e.,
%     \begin{equation*}    
%         \begin{split}
%         \signal \not \models p_1\release_{[a,b]}p_2 \land \exists\signal', E_{p_1\release_{[a,b]}p_2}(\signal, t)=\signal' : min(|\signal'-\signal|)\\ \land \sigma' \in 
%         \mathcal{L}(\automaton_{\varphi_1\release_{[a,b]}\varphi_2}) 
%         \implies \signal' \models p_1\release_{[a,b]}p_2
%         \end{split}
%     \end{equation*}    
% \end{proposition}


The proofs of the propositions are provided at Appendix \ref{sec:appendix}.
\section{{\hdm} with Momentum} \label{sec:momentum}

This section develops two variants of {\hdm}, 
with heavy-ball momentum \cite{polyak1964some} and with Nesterov momentum \cite{nesterov1983method}.

\subsection{Heavy-ball Momentum}
\label{sec:heavyball}

The heavy-ball method is a practical acceleration technique: 
\begin{equation} \label{eqn:heavyball-update}
x^{k + 1} = x^k - P_k \nabla f (x^k) + B_k (x^k - x^{k - 1}).
\end{equation}
The momentum parameter $B_k$ is typically chosen as a scalar $B_k = \beta_k I$ with $\beta_k > 0$.
{\hdm} can learn a matrix momentum
$B_k \in \mathcal{B} \subseteq \mathbb{R}^{n \times n}$
with convergence guarantees (\Cref{thm:heavyball}) 
when $\mathcal{B}$ satisfies this assumption:
\begin{enumerate}[leftmargin=30pt,label=\textbf{A\arabic*:},ref=\rm{\textbf{A\arabic*}},start=5]
  \item Closed convex set $\Bcal$ satisfies $\tfrac{1}{2} I\in \Bcal$, $\diam (\Bcal) \leq D$. \label{ABcal}
\end{enumerate}

{\hdm} can \emph{jointly} learn the pair $(P_k, B_k)$ using the modified feedback function
\begin{equation} \label{eqn:heavyball-feedback}
  h_{x, x^-} (P, B) \assign 
  \tfrac{\psi(x^{+}(P, B), x) - \psi(x, x^{-})}{\| \nabla f (x) \|^2 + \frac{\tau}{2} \| x - x^- \|^2}
  = \tfrac{[f (x^+(P, B)) + \frac{\omega}{2} \| x^+(P, B) - x \|^2] - [f (x) + \frac{\omega}{2} \| x - x^- \|^2]}{\| \nabla f (x) \|^2 + \frac{\tau}{2} \| x - x^- \|^2}, 
\end{equation}
where $\psi$ is the potential function for heavy-ball momentum defined by $\psi (x, x^-) \assign f (x) + \tfrac{\omega}{2} \| x - x^- \|^2$ \cite{danilova2020non}; \[x^{+}(P, B) \assign x - P \nabla f (x) + B (x - x^{-})\] updates $x$; and $\omega > 0$ and $ \tau > 0$ are constants. \Cref{alg:ospolyak} presents the resulting method, \hdmhb, 
which uses {\hdm}, heavy-ball momentum, and a null step to ensure decrease of the potential function $\psi$.
\Cref{fig:demo:c} compares non-adaptive heavy-ball ($P_k \equiv  \alpha I, B_k \equiv \beta I$) against {\hdmhb} with full-matrix/diagonal preconditioner and scalar momentum.
 \Cref{thm:heavyball} presents the convergence of {\hdmhb}. 

\begin{algorithm}[h]
{\textbf{input} initial point $x^0 = x^1, \eta_p, \eta_b > 0$, $P_1$, $B_1$}\\
\For{k =\rm{ 1, 2,...}}{
$\hspace{1.2pt}~~~~x^{k+1/2} = x^k - P_k \nabla f(x^k) + B_k (x^k - x^{k-1})$ \\
$\hspace{2pt}~~~~~~P_{k+1} = \Pi_{\Pcal}[P_k - \eta_p \nabla_{P} h_{x^k, x^{k-1}}(P_k, B_k)]$ \\
$\hspace{1pt}~~~~~~B_{k+1} = \Pi_{\Bcal}[B_k - \eta_b \nabla_{B} h_{x^k, x^{k-1}}(P_k, B_k)]$ \\
$(x^{k + 1}, x^k) = \displaystyle \argmin_{(x^+, x) \in \{(x^k, x^{k-1}), (x^{k+1/2}, x^k) \}} \psi(x^+, x)$
}
{\textbf{output} $x^{K+1}$}
\caption{{\hdm} with heavy-ball momentum (\hdmhb)\label{alg:ospolyak}}
\end{algorithm}

\begin{thm}[Convergence of {\hdmhb}]\label{thm:heavyball}
Under \ref{A1}, \ref{A2} and \ref{ABcal}, \Cref{alg:ospolyak} satisfies
\begin{equation*}
  f (x^{K + 1}) - f (x^{\star}) \leq \tfrac{f (x^{1}) - f (x^{\star})}{K V \max\{ \gamma_K^{\star} - \frac{\rho_K}{K}, 0 \} + 1},
\end{equation*}
where $\gamma_{K}^{\star} \assign - \min_{(P, B) \in \mathcal{P} \times \mathcal{B}} \tfrac{1}{K} \sum_{k=1}^K h_{x^k, x^{k-1}}(P, B)$ depends on the iteration trajectory $\{x^k\}_{k \leq K}$; $\rho_K = \mathcal{O}(\sqrt{K})$ is the regret with respect to feedback \eqref{eqn:heavyball-feedback}; $V \assign \min\big\{ \tfrac{f (x^{1}) - f (x^{\star})}{4 \Delta^2}, \tfrac{\tau}{4 \omega} \big\}$; $\Delta$ is defined in \Cref{lem:hypergrad-to-online}.
\end{thm}

\subsection{Nesterov Momentum} \label{sec:nesterov}
{\hdm} can also improve accelerated gradient descent {\agd}:\begin{align}
  y^k ={} & x^k + ( 1 - \tfrac{A_k}{A_{k + 1}} ) (z^k - x^k)
  \nonumber\\
  x^{k + 1} ={} & y^k - \tfrac{1}{L} \nabla f (y^k) \label{eqn:agd-descent-lemma-0}\\
  z^{k + 1} ={} & z^k + \tfrac{A_{k + 1} - A_k}{L} \nabla f (y^k), \nonumber
\end{align}
where is a pre-specified sequence. 
{\hdm} can learn a preconditioner $P_k$ that replaces $\frac 1 L$ to accelerate the gradient step \eqref{eqn:agd-descent-lemma-0} in {\agd}. We call the resulting algorithm {\hdmagd}. \Cref{alg:osnes} provides a
realization of the {\hdmagd} based on a monotone variant of {\agd} \cite{d2021acceleration}. The convergence of {\hdmagd} is established in \Cref{thm:osnes}, the proof of which is deferred to \Cref{app:proof-osnes}.
\begin{algorithm}[h]
{\textbf{input} starting point $x^1, z^1, \eta > 0$, $\theta \in [\tfrac{1}{2}, LD) $, $A_0 = 0$}\\
\For{k =\rm{ 1, 2,...}}{
$\begin{aligned}
 A_{k + 1} ={} & (A_{k + 1} - A_k)^2 \nonumber \\
  y^k ={} & x^k + ( 1 - \tfrac{A_k}{A_{k + 1}} ) (z^k - x^k)
  \nonumber\\
  x^{k + 1} ={} & \underset{x \in \{ y^k - \frac{1}{L} \nabla f (y^k), y^k
  - P_k \nabla f (y^k), x^k \}}{\argmin} f (x) \nonumber\\
  P_{k + 1} ={} & \Pi_{\mathcal{P}} [P_k - \eta \nabla h_{y^k} (P_k)]
  \nonumber\\
  v_k ={} & \max \{ \tfrac{1}{2 \max \{ -h_{y^k} (P_k), 1 / (2 L) \}},
   \tfrac{L}{2 \theta} \}  \\
  z^{k + 1} ={} & z^k + \tfrac{(A_{k + 1} - A_k)}{v_k} \nabla f (y^k)
  \nonumber
\end{aligned}$
}
{\textbf{output} $x^{K+1}$}
\caption{{\hdm} with Nesterov momentum  \label{alg:osnes}}
\end{algorithm}

\begin{thm}
\label{thm:osnes}
Assume \ref{A1} and \ref{A2}. Suppose {\agd} starts from $(x', z')$ and runs for $K$ iterations to output $\hat{x}$.
Then \Cref{alg:osnes} starting from $(x^1, z^1) = (\hat{x}, z')$ and $\theta \in [\tfrac{1}{2}, LD)$ satisfies
\begin{align}
f (x^{K + 1}) - f (x^{\star}) \leq \big[ \tfrac{1}{2 \theta} + ( 8 - \tfrac{4}{\theta} ) ( \tfrac{L D - \omega^\star_K}{L D - \theta} ) \big] \tfrac{2 L \| z' - x^{\star} \|^2}{K^2} +\mathcal{O} ( \tfrac{\rho_K}{K^3} ), \nonumber
\end{align}
where $\omega^\star_K = - \min_{P \in \mathcal{P}} \tfrac{L}{K} \sum_{k = 1}^K h_{y^k} (P)$ depends on the iteration trajectory $\{x^k\}_{k \leq K}$.
\end{thm}

The parameter $\theta$ serves as a smooth interpolation between {\hdm} and {\hdmagd}: when $\theta = 1 / 2$, \Cref{thm:osnes} recovers the convergence rate of
vanilla {\agd}; when $\theta > 1 / 2$ and $\omega_K^{\star}
\rightarrow L D$, we expect {\hdmagd} to yield faster convergence. As suggested by \Cref{fig:demo:d}, {\hdmagd} achieves faster convergence than {\agd}.
\begin{rem}
To mitigate the effect of regret, \Cref{alg:osnes} needs a warm start from vanilla {\agd}. However, experiments
  suggest that it is unnecessary in practice, and we leave an improved analysis to future work.
\end{rem}

\begin{rem}
For strongly convex problems, we can combine \Cref{thm:osnes} with a
  standard restart argument \cite{d2021acceleration,roulet2017sharpness} and achieve a similar trajectory-based linear
  convergence rate.
\end{rem}

\begin{figure}
  \centering
  \begin{subfigure}{0.4\textwidth}
    \centering
\includegraphics[height=0.18\textheight]{figs/demo_4.pdf}
    \caption{\hdmhb}
    \label{fig:demo:c}
  \end{subfigure}
  \begin{subfigure}{0.4\textwidth}
    \centering
\includegraphics[height=0.18\textheight]{figs/demo_3.pdf}
    \caption{\hdmagd}
    \label{fig:demo:d}
  \end{subfigure}
\caption{The convergence behavior of {\hdmhb} and {\hdmagd} on a toy quadratic problem. \Cref{fig:demo:c}: {\hdmhb}. \label{fig-demos}\Cref{fig:demo:d}: {\hdm} with Nesterov momentum.} 
  \label{fig:demo:all}
\end{figure}
\section{Experiments} \label{sec:exp}

This section conducts numerical experiments to validate the empirical
performance of hypergradient descent. We compare {\hdmbest} (see \Cref{sec:hdmbest} below) with 
different adaptive optimization algorithms.

\subsection{Efficient and Practical Variant: {\hdmbest}} \label{sec:hdmbest}

This section highlights the major components of our most competitive variant {\hdmbest}. The algorithm and a more detailed explanation are available in \Cref{app:hdmprac}. The implementation is available at \url{https://github.com/udellgroup/hypergrad}. 

\paragraph{Diagonal Preconditioner and Heavy-ball Momentum.} {\hdmbest} updates $x$ by \eqref{eqn:heavyball-update} with diagonal preconditioner \cite{qu2024optimal,gao2023scalable} $\Pcal \subseteq \Dcal$ and scalar momentum $\Bcal = \{ \beta I : \beta \in \mathbb{R} \}$.
This choice balances practical efficiency and implementation complexity. Boundedness of $\Pcal$ does not greatly impact the performance, while the bound on $\Bcal$ can significantly change algorithm behavior. Two empirically robust ranges for $\Bcal$ are $[0,0.9995]$ and $[-0.9995,0.9995]$. 

\paragraph{$\adagrad$ for Online Learning. } {\hdmbest} uses {\adagrad} to shorten the warm-up phase for learning of $(P_k, \beta_k)$ (see \Cref{sec:instability}). {\adagrad} usually yields faster convergence of {\hdm} than online gradient descent at the cost of additional memory of size $n$.

\subsection{Dataset and Testing Problems}
We test {\hdmbest} on deterministic convex problems. We adopt two convex optimization tasks in machine learning: support vector machine \cite{lee2001ssvm} and logistic regression \cite{hastie2009elements}. The testing datasets are obtained from \texttt{LIBSVM} \cite{chang2011libsvm}.

\subsection{Experiment Setup}

\paragraph{Algorithm Benchmark.}
We benchmark the following algorithms.\\
\begin{itemize}[leftmargin=10pt,itemsep=2pt,topsep=0pt]
    \item \texttt{GD}. Vanilla gradient descent.
    \item \texttt{GD-HB}. Gradient descent with heavy-ball momentum. \cite{polyak1964some}
    \item \texttt{AGD-CVX}. The smooth convex version of accelerated gradient descent (Nesterov momentum). \cite{d2021acceleration}
    \item \texttt{AGD-SCVX}. The smooth strongly convex version of accelerated gradient descent. \cite{d2021acceleration}
    \item \texttt{Adam}. Adaptive momentum estimation. \cite{kingma2014adam}
    \item \texttt{AdaGrad}. Adaptive (sub)gradient method. \cite{duchi2011adaptive}
    \item \texttt{BFGS}. {\bfgs} from \texttt{scipy} \cite{nocedal1999numerical,virtanen2020scipy}.
    \item \texttt{L-BFGS-Mk}. {\lbfgs} with memory size \texttt{k} in \texttt{scipy}.
    \item Practical variant {\hdmbest} uses as memory $7$ vectors of size $n$, comparable to memory for \texttt{L-BFGS-M1}.
\end{itemize}

\paragraph{Algorithm Configuration.} See \Cref{app:hdmprac} for details.
\begin{itemize}[leftmargin=10pt]
  \item For {\hdmbest}, we search for the optimal $\eta_p$ within $\{ 0.1 / L, 1 / L, 10 / L, 100/L \}$
  and $\eta_b \in \{ 1, 3, 5, 10, 100 \}$. 
  
  \item Stepsize in \texttt{GD}, \texttt{GD-HB},
  \texttt{AGD-CVX}, and \texttt{AGD-SCVX} are all set to $1 / L$.
  
  \item The momentum parameter in \texttt{GD-HB} is chosen within the set $\{
  0.1, 0.5, 0.9, 0.99 \}$.
  
  \item The \texttt{Adam} stepsize is chosen within the set $\{ 1 / L, 10^{- 3},
  10^{- 2}, 10^{- 1}, 1, 10 \}$. $\beta_1 = 0.9, \beta_2 = 0.999$.
  
  \item  The \texttt{AdaGrad} stepsize is chosen within the set $\{ 1 / L, 10^{- 3},
  10^{- 2}, 10^{- 1}, 1, 10 \}$.
  
  \item {\bfgs}, \texttt{L-BFGS-Mk} use default parameters in
  \texttt{scipy}.
\end{itemize}

\paragraph{Testing Configurations.}

\begin{enumerate}[leftmargin=15pt,label=\textbf{\arabic*)}]
  \item {\textit{Maximum oracle access.}} We allow a maximum of 1000 gradient
  oracles for each algorithm.
  
  \item {\textit{Initial point}}. All the algorithms are initialized from the
  same starting point generated from normal distribution $\mathcal{N} (0,
  I_n)$ and normalized to have unit length.
  
  \item {\textit{Stopping criterion.}} Algorithms stop if $\| \nabla f 
  \|_\infty \leq 10^{- 4}$.
\end{enumerate}

\begin{table}[h]
\centering
\caption{Number of solved problems for each algorithm. \label{table:stats}}
\begin{tabular}{ccc}
\toprule
    Algorithm/Problem & SVM (33) $\uparrow$ & Logistic Regression (33) $\uparrow$\\
\midrule
    \texttt{GD} & 5 & 2\\
    \texttt{GD-HB} & 9 & 7\\
    \texttt{AGD-CVX} & 8 & 3\\
    \texttt{AGD-SCVX} & 7 & 6\\
    \texttt{Adam} & 26 & 11\\
    \texttt{AdaGrad} & 9 & 8\\
    \texttt{L-BFGS-M1} & 13 & 11\\
    \texttt{L-BFGS-M3} & 20 & 14\\
    \texttt{L-BFGS-M5} & 26 & 16\\
    \texttt{L-BFGS-M10} & \textcolor{orange}{31} & 18\\
    \texttt{BFGS} & \textcolor{red}{32} & \textcolor{red}{26}\\
    \texttt{HDM-Best} & \textcolor{red}{32} & \textcolor{red}{21}\\
\bottomrule
\end{tabular}
\end{table}

\begin{figure*}[!h]
\centering
\includegraphics[scale=0.2]{figs/a1a_objval_svm.pdf}
\includegraphics[scale=0.2]{figs/a9a_objval_svm.pdf}
\includegraphics[scale=0.2]{figs/w8a_objval_svm.pdf}
\includegraphics[scale=0.2]{figs/svmguide3_objval_svm.pdf}
\includegraphics[scale=0.2]{figs/a1a_gnorm_svm.pdf}
\includegraphics[scale=0.2]{figs/a9a_gnorm_svm.pdf}
\includegraphics[scale=0.2]{figs/w8a_gnorm_svm.pdf}
\includegraphics[scale=0.2]{figs/svmguide3_gnorm_svm.pdf}
\includegraphics[scale=0.4]{figs/legend.pdf}
\caption{Support vector-machine problems. First row: function value gap. Second row: gradient norm. \label{fig:svm}}
\end{figure*}

\begin{figure*}[!h]
\centering
\includegraphics[scale=0.2]{figs/a1a_objval_logistic.pdf}
\includegraphics[scale=0.2]{figs/w4a_objval_logistic.pdf}
\includegraphics[scale=0.2]{figs/ionosphere_scale_objval_logistic.pdf}
\includegraphics[scale=0.2]{figs/ijcnn1_objval_logistic.pdf}
\includegraphics[scale=0.2]{figs/a1a_gnorm_logistic.pdf}
\includegraphics[scale=0.2]{figs/w4a_gnorm_logistic.pdf}
\includegraphics[scale=0.2]{figs/ionosphere_scale_gnorm_logistic.pdf}
\includegraphics[scale=0.2]{figs/ijcnn1_gnorm_logistic.pdf}
\caption{Logistic regression problems. First row: function value gap. Second row: gradient norm.  \label{fig:logistic}}
\end{figure*}

For each algorithm, we record the number of successfully solved instances ($\|\nabla f\|_\infty \leq 10^{-4}$ within 1000 gradient oracles). \Cref{table:stats} summarizes the detailed statistics. The number of instances solved by {\hdmbest} is comparable to that of \texttt{L-BFGS-M10}.

\paragraph{Support Vector Machine.} \Cref{fig:svm} shows the function value gap and gradient norm plots on sample test instances on support vector machine problems. The optimal value for each instance is obtained by running {\bfgs} until $\|\nabla f \|_\infty \leq 10^{-4}$.  We see that the practical variant of {\hdmbest} achieves a significant speedup over other adaptive first-order methods. In particular, {\hdmbest} often matches \texttt{L-BFGS-M5} and \texttt{L-BFGS-M10}, while its memory usage is closer to \texttt{L-BFGS-M1}. Notably, {\adam} also achieves competitive performance in several instances.

\paragraph{Logistic Regression.} In logistic regression (\Cref{fig:logistic}), {\hdmbest} still compares well with \texttt{L-BFGS-M5} and is significantly faster than other adaptive first-order methods.\\

Overall, {\hdmbest} demonstrates superior performance on deterministic convex problems and is comparable with the mature \texttt{L-BFGS} family. We believe that further development of {\hdm} will fully unleash its potential for a broad range of optimization tasks.

\vspace{-2mm}
\section{Conclusion}\label{sec:conclusion}
In this paper, we presented RecDreamer, a novel approach to mitigating the Multi-Face Janus problem in text-to-3D generation. Our solution introduces a rectification function to modify the prior distribution, ensuring that the resulting joint distribution achieves uniformity across poses. By expressing the modified data distribution as the product of the original density and the rectification function, we seamlessly integrate this adjustment into the score distillation algorithm. This allows us to derive a particle optimization framework for uniform score distillation. Additionally, we developed a pose classifier and implemented reliable approximations and simulations to enhance the particle optimization process. Extensive experiments on both 2D and 3D synthesis tasks demonstrate the effectiveness of our approach in addressing the Multi-Face Janus problem, resulting in more consistent geometries and textures across different views.

\textbf{Limitations.} While our method significantly reduces bias in prior distributions, further exploration of 3D modeling with multi-view priors could improve geometric and texture consistency. Extending our approach through deeper research into conditional control presents another promising avenue for addressing these challenges in future work. 

\newpage 
\renewcommand \thepart{}
\renewcommand \partname{}

\bibliography{ref.bib}
\bibliographystyle{plain}

\doparttoc
\faketableofcontents
\part{}

\newpage
\appendix
\onecolumn

\addcontentsline{toc}{section}{Appendix}
\part{Appendix} 
\parttoc

\paragraph{Structure of the Appendix. } The appendix is organized as follows. In \Cref{app:hdmprac}, we introduce a practical variant of hypergradient descent and explain its implementation details. \Cref{app:additional} provides additional experiment details on the tested problems. \Cref{app:proof-hdm-ol} to \Cref{app:proof-momentum} provide proofs of the main results in the paper.

\newpage
\section{{\hdm} in Practice} \label{app:hdmprac}

This section introduces {\hdmbest}, our recommended practical hypergradient descent method. This variant is adapted from {\hdmhb}, with simplifications to reduce the implementation complexity. The algorithm is given in \Cref{alg:practical}.

\begin{algorithm}[h]
{\textbf{input} starting point $x^0 = x^1$, $\Pcal = \mathbb{S}^n_{+} \cap \Dcal, \Bcal = [0,0.9995]$, initial diagonal preconditioner $P_1 \in \mathbb{S}^n_{+} \cap \Dcal$, \\initial scalar momentum parameter $\beta_1 = 0.95$, {\adagrad} stepsize $\eta_p, \eta_b > 0$, {\adagrad}  diagonal matrix $U_1 = 0$, {\adagrad} momentum scalar $v_1 = 0$, $\tau > 0$}\\
\For{k =\rm{ 1, 2,...}}{
$\begin{aligned}
x^{k + 1 / 2} & = x^k - P_k \nabla f (x^k) + \beta_k (x^k - x^{k - 1}) \\
\nabla_P h_{x^k, x^{k - 1}} (P_k, \beta_k) & = \tfrac{\diag(\nabla f (x^{k + 1 / 2}) \circ \nabla f (x^k))}{\| \nabla f (x^k) \|^2 + \frac{\tau}{2} \| x^k - x^{k - 1} \|^2} \texttt{ \# Element-wise product} \\
\nabla_{\beta} h_{x^k, x^{k - 1}} (P_k, \beta_k) & = \tfrac{\langle \nabla f (x^{k + 1 / 2}), x^k - x^{k - 1} \rangle}{\| \nabla f (x^k) \|^2 + \frac{\tau}{2} \| x^k - x^{k - 1} \|^2} \texttt{ \# Inner product} \\
U_{k + 1} & = U_k + \nabla_P h_{x^k, x^{k - 1}} (P_k, \beta_k) \circ \nabla_P h_{x^k, x^{k - 1}} (P_k, \beta_k) \texttt{ \# Diagonal matrix} \\
v_{k + 1} & = v_k + \nabla_{\beta} h_{x^k, x^{k - 1}} (P_k, \beta_k) \cdot \nabla_{\beta} h_{x^k, x^{k - 1}} (P_k, \beta_k) \texttt{ \# Scalar matrix} \\
P_{k + 1} & = \Pi_{\Rbb^n_{+} \cap \Dcal} [P_k - \eta_p U_{k + 1}^{- 1 / 2} \nabla_P h_{x^k, x^{k - 1}} (P_k, \beta_k)] \texttt{ \# Diagonal matrix} \\
\beta_{k + 1} & = \Pi_{[0, 0.9995]} [\beta_k - \eta_b v_{k + 1}^{- 1 / 2} \nabla_{\beta} h_{x^k, x^{k - 1}} (P_k, \beta_k)] \\
x^{k + 1} & = \argmin_{x \in \{ x^k, x^{k + 1 / 2} \}} f (x).
\end{aligned}
$
}
{\textbf{output} $x^{K+1}$}
\caption{{\hdmbest} \label{alg:practical}}
\end{algorithm}

We make several remarks about \Cref{alg:practical}. 

\begin{itemize}[leftmargin=10pt]
\item \textit{Choice of online learning algorithm.} Unless $f(x)$ is quadratic, adaptive online learning algorithms such as {\adagrad} often significantly outperform online gradient descent with constant stepsize. Note that {\adagrad} introduces additional memory of size $n$ to store the diagonal online learning preconditioner $U$.

\item \textit{Sensitivity of parameters.} The two stepsize parameters in {\adagrad} are the most important algorithm parameters: $\eta_p, \eta_b$. According to the experiments, $\eta_p$ should be set proportional to $1/L$, the smoothness constant, while an aggressive choice of $\eta_b \in \{1,10,100\}$ often yields fast convergence. A local estimator of the smoothness constant $L$ can significantly enhance algorithm performance.
\item \textit{Heavy-ball feedback and null step.} In practice, it is observed that dropping the $\frac{\omega}{2}\|x^+(P, B) - x\|^2$ in the numerator of heavy-ball feedback \eqref{eqn:heavyball-feedback} often does not affect algorithm performance. Therefore, in \Cref{alg:practical} the hypergradient with respect to $\frac{\omega}{2}\|x^+(P, B) - x\|^2$ is ignored. 
On the other hand, the $\frac{\tau} {2}\|x^+(P, B) - x\|^2$  term in the denominator smoothes the update of $\beta_k$ and can strongly affect convergence. The parameter $\tau$ should be taken to be proportional to $L^2$ according to the discussions in \Cref{app:heavy-ball}. The null step is taken with respect to the function value $f(x)$ instead of the heavy-ball potential function.
\item \textit{Memory usage.} The memory usage of {\hdmbest}, measured in terms of number of vectors of length $n$ is $7n$: 1) three vectors store primal iterates $x^{-}, x, x^{+}$. 2) Two vectors store past and buffer gradients $\nabla f(x), \nabla f(x^+)$. 3) A vector stores the diagonal preconditioner $P_k$. 4) A vector stores the {\adagrad} stepsize matrix $U$.
\item \textit{Computational cost. } The major additional computation cost arises from computing hypergradient $\nabla h$, which involves one element-wise product and one inner product for vectors of size $n$. In addition, {\hdmbest} needs to maintain a diagonal matrix for {\adagrad}. The overall additional computational cost is several $\Ocal({n})$ operations.
\end{itemize}

\newpage
%%%%%%%%%%%%%%%%%%%%%%%%%%%%%%%%%%%%%%%%%%%%%%%%%%%%%%%%%


%%%%%%%%%%%%%%%%%%%%%%%%%%%%%%%%%%%%%%%%%%%%%%%%%%%%%%%%%

%	\section{Auxiliary results}
%	
%	\begin{corollary}
%		Assume gradient Lipchitz $\GradLip$.
%		\begin{align*}
%			\scalarvalue(\policyt{t+1})
%			\; \geq \; \scalarvalue(\policyt{t}) \, + \, \frac{1}{2 \GradLip} \, \norm[\big]{\gradtheta \scalarvalue(\policyt{t})}_2^2 \, - \, \frac{\Partitionthetabar}{2 \GradLip \parabeta^2} \, \norm{\Term_2}_2^2
%		\end{align*}
%	\end{corollary}
%	
%	\begin{proposition}
%		\label{thm:Hess}
%		Suppose that for any \mbox{$\reward \in \RewardSp$}, \mbox{$0 \leq \reward(\prompt, \response) \leq \Radius$}. Moreover, suppose $\hesstheta \reward_{\paratheta}(\prompt, \response)$ is positive semidefinite for any $(\prompt, \response) \in \PromptSp \times \ResponseSp$.
%		Then both terms
%		\begin{align*}
%			\Exp_{\prompt \sim \promptdistr, \, \response \sim \policy_{\paratheta}(\cdot \mid \prompt)} \big[ \rewardstar(\context, \response) \big]
%			\qquad \mbox{and} \qquad
%			\kull{\policytheta}{\policyref}
%		\end{align*}
%		are convex in parameter $\paratheta$.
%		Therefore, the objective function $\scalarvalue(\policytheta)$ is a difference-of-convex function.
%	\end{proposition}

%%%%%%%%%%%%%%%%%%%%%%%%%%%%%%%%%%%%%%%%%%%%%%%%%%%%%%%%%

	\section{Proof of Main Results \yaqidone}
    \label{app:proof:main}

        
    
		This section provides the proofs of the main results from \Cref{sec:theory}, covering both optimization and statistical aspects.
		In \Cref{sec:proof:thm:grad}, we prove \Cref{thm:grad}, which establishes the gradient alignment property. For the statistical results, \Cref{sec:proof:thm:stat} begins with the proofs of \Cref{thm:asymp_full,thm:asymp}, which derive the asymptotic distribution of the estimated parameter $\parathetahat$, and concludes with the proof of \Cref{lemma:hess_scalarvalue}, analyzing the asymptotic behavior of the value gap~\mbox{$\scalarvalue(\policystar) - \scalarvalue(\policyhat)$}.
	
	\subsection{Optimization Considerations: Proof of Theorem~\ref{thm:grad} \yaqidone}
	\label{sec:proof:thm:grad}

        % \yaqitbd

        We begin by presenting a rigorous restatement of \Cref{thm:grad}, formally detailed in \Cref{thm:grad_full} below.

        \begin{theorem}[Gradient structure in DPO training]
			\label{thm:grad_full}
			Consider the expected loss function $\Loss(\paratheta)$ during the DPO training phase. Using data collected from our poposed response sampling scheme $ \responsedistr $, the gradient of $ \Loss(\paratheta) $ satisfies
			\begin{align*}
				\gradtheta \Loss(\paratheta) \; = \;
				- \, \frac{\parabeta}{\Partitionthetabar} \, \gradtheta \scalarvalue(\policytheta) \, + \, \Term_2 \, ,
			\end{align*}
			where the constant $ \Partitionthetabar $ is defined in equation~\eqref{eq:weight}, and the term $ \Term_2% = \bigO( \norm{\rewardtheta - \rewardstar}^2 ) 
			$ represents a second-order error.
			
			To control term $ \Term_2 $, assume the following uniform bounds: 
            \begin{itemize}
                \item[(i)] \mbox{$\!\supnorm{\rewardstar} \leq \Radius$}.
                \item[(ii)] For any policy \mbox{$\policytheta \in \PolicySp$}, the induced reward $\rewardtheta$ satisfies 
                \begin{align*}
                    \supnorm{\rewardtheta} \leq \Radius \qquad \mbox{and} \qquad \sup\nolimits_{\prompt, \response} \, \norm{\gradtheta \rewardtheta (\prompt, \response)}_2 \leq \RadiusGrad \, .
                \end{align*}
            \end{itemize}
			Under these conditions, $ \Term_2 $ is bounded as
			\vspace{-.5em}
			\begin{align*}
                \norm{\Term_2}_2 \leq 
                \Const{} \, \cdot \, \Exp_{\prompt \sim \promptdistr, \, \responseone, \responsetwo \sim \policytheta(\cdot \mid \prompt)}
				\bigg[ \, \Big\{ \big( \rewardstar(\context, \responseone) - \rewardstar(\context, \responsetwo) \big)
                - \big( \rewardtheta(\context, \responseone) - \rewardtheta(\context, \responsetwo) \big) \Big\}^2 \bigg] \, ,
			\end{align*}
			where the constant $\Const{}$ is given by $\Const{} = 0.1 \, (1 + e^{2\Radius}) \, \RadiusGrad \big/ \Partitionthetabar$.
		\end{theorem}
	
	The proof of \Cref{thm:grad_full} is structured into three sections. In \Cref{sec:proof:thm:grad_1}, we lay the foundation by presenting the key components, including the explicit expressions for the gradients $\gradtheta \scalarvalue(\policytheta)$ and $\gradtheta \Loss(\paratheta)$, as well as for the sampling density~$\responsedistravg$.
	Then \Cref{sec:proof:thm:grad_2} establishes the connection between $\gradtheta \scalarvalue(\policytheta)$ and $\gradtheta \Loss(\paratheta)$ by leveraging these results, completing the proof of \Cref{thm:grad}. 
	Finally, in \Cref{sec:proof:thm:grad_3}, we provide a detailed derivation of the form of density function~$\responsedistravg$.
	
	\subsubsection{Building Blocks \yaqidone}
	\label{sec:proof:thm:grad_1}
	
	To establish \Cref{thm:grad}, which uncovers the relationship between the gradients of the expected value $\scalarvalue(\policytheta)$ and the negative log-likelihood function $\Loss(\paratheta)$, the first step is to derive explicit expressions for the gradients of both functions. The results are presented in \Cref{lemma:grad_scalarvalue,lemma:grad_loss}, with detailed proofs provided in \Cref{sec:proof:lemma:grad_scalarvalue,sec:proof:lemma:grad_loss}, respectively.
	\begin{lemma}[Gradient of value $\scalarvalue(\policytheta)$]
		\label{lemma:grad_scalarvalue}
		For any $\policytheta$ in the parameterized policy class $\PolicySp$, the gradient of the expected value~$\scalarvalue(\policytheta)$ satisfies
		%			\begin{subequations}
			\begin{multline}
				\label{eq:grad_scalarvalue}
				\gradtheta \scalarvalue(\policytheta)
				\; = \; \frac{1}{2 \parabeta} \, \Exp_{\prompt \sim \promptdistr; \; \responseone, \responsetwo \sim \policytheta(\cdot \mid \prompt)} 
				\bigg[ \Big\{ \big( \rewardstar(\context, \responseone) - \rewardstar(\context, \responsetwo) \big) - \big( \rewardtheta(\context, \responseone) - \rewardtheta(\context, \responsetwo) \big) \Big\} \\ 
				\cdot \big\{ \gradtheta \rewardtheta(\prompt, \responseone) - \gradtheta \rewardtheta(\prompt, \responsetwo) \big\} \bigg] \, .
			\end{multline}
			%			\end{subequations}
	\end{lemma}
	
	
	
	\begin{lemma}[Gradient of the loss function $\Loss(\paratheta)$]
		\label{lemma:grad_loss}
		For any $\policytheta$ in the parameterized policy class $\PolicySp$ and any sampling distribution $\responsedistr$ of the responses, the gradient of the negative log-likelihood function $\Loss(\paratheta)$ is given by
		\begin{subequations}
			\begin{multline}
				\label{eq:gradLoss_BT_0}
				\gradtheta \Loss(\paratheta) \; = \; - \, \Exp_{\prompt \sim \promptdistr; \; (\responseone, \, \responsetwo) \sim \responsedistravg(\cdot \mid \prompt)}
				\bigg[ \, \weight(\prompt) \cdot \Big\{ \sigmoid \big( \rewardstar(\context, \responseone) - \rewardstar(\context, \responsetwo) \big) - \sigmoid \big( \rewardtheta(\context, \responseone) - \rewardtheta(\context, \responsetwo) \big) \Big\} \\ 
				\cdot \big\{ \gradtheta \rewardtheta(\prompt, \responseone) - \gradtheta \rewardtheta(\prompt, \responsetwo) \big\} \bigg] \, ,
			\end{multline}
			where the average density $\responsedistravg$ is defined as
			\begin{align}
				\label{eq:def_responsedistravg_0}
				\responsedistravg(\responseone, \responsetwo \mid \prompt) 
				\; \defn \; \frac{1}{2} \, \big\{ \responsedistr(\responseone, \responsetwo \mid \prompt) + \responsedistr(\responsetwo, \responseone \mid \prompt) \big\}
			\end{align}
		\end{subequations}
			as previously introduced in \cref{eq:def_responsedistravg}.
	\end{lemma}
	
	In \Cref{lemma:grad_loss}, we observe that the gradient $\gradtheta \Loss(\paratheta)$ is expressed as an expectation over the probability distribution $\responsedistravg$. By applying the sampling scheme outlined in \Cref{sec:sampling}, we can derive a more detailed representation of $\gradtheta \Loss(\paratheta)$. This refined form will reveal its close relationship to the gradient $\gradtheta \scalarvalue(\policytheta)$ given in expression \eqref{eq:grad_scalarvalue}.
	
	Before moving forward, it is crucial for us to first derive the explicit form of $\responsedistravg$. Specifically, we claim that the distribution~$\responsedistravg$ satisfies the following property
	\begin{align}
		\label{eq:responsedistravg}
		\frac{\responsedistravg ( \responseone, \responsetwo \mid \prompt )}{\policytheta(\responseone \mid \prompt) \, \policytheta(\responsetwo \mid \prompt)} 
		& \; = \; \frac{1}{2 \, \{ 1 + \Partitionthetapos(\prompt) \, \Partitionthetaneg(\prompt) \}}
		\cdot \frac{1}{\divsigmoid \big( \rewardtheta(\prompt, \responseone) - \rewardtheta(\prompt, \responsetwo) \big)} \, ,
	\end{align}
	where $\divsigmoid$ denotes the derivative of the sigmoid function $\sigmoid$, given by
	\begin{align}
		\label{eq:divsigmoid}
		\divsigmoid(z) \; = \; \frac{1}{( 1 + \exp(-z) )( 1 + \exp(z) )} \; = \; \sigmoid(z) \, \sigmoid(-z)
		\qquad \mbox{for any $z \in \Real$}  \, .
	\end{align}
	With these key components in place, we are now prepared to prove \Cref{thm:grad}.
	
	
	\subsubsection{Derivation of Theorem~\ref{thm:grad} \yaqidone}
	\label{sec:proof:thm:grad_2}
	
	With the tools provided by \Cref{lemma:grad_scalarvalue,lemma:grad_loss} and the sampling density expression in \eqref{eq:responsedistravg}, we are now ready to prove \Cref{thm:grad}.
	
	We begin by applying \Cref{lemma:grad_loss} and reformulating equation~\eqref{eq:gradLoss_BT_0} as
	\begin{align}
		\gradtheta \Loss(\paratheta) \; = \; - \, \Exp_{\prompt \sim \promptdistr; \; \responseone, \, \responsetwo \sim \policytheta(\cdot \mid \prompt)}
		\bigg[ \, & \weight(\prompt) \cdot \frac{\responsedistravg ( \responseone, \responsetwo \mid \prompt )}{\policytheta(\responseone \mid \prompt) \, \policytheta(\responsetwo \mid \prompt)} \notag \\
		& \cdot \Big\{ \sigmoid \big( \rewardstar(\context, \responseone) - \rewardstar(\context, \responsetwo) \big) - \sigmoid \big( \rewardtheta(\context, \responseone) - \rewardtheta(\context, \responsetwo) \big) \Big\} \notag \\ 
		& \cdot \big\{ \gradtheta \rewardtheta(\prompt, \responseone) - \gradtheta \rewardtheta(\prompt, \responsetwo) \big\} \bigg] \,.
		\label{eq:gradLoss}
	\end{align}
	Substituting the density ratio from equation~\eqref{eq:responsedistravg} into expression \eqref{eq:gradLoss} and incorporating the weight function $\weight(\prompt)$ defined in equation \eqref{eq:weight}, we obtain 
	\begin{align}
		\gradtheta \Loss(\paratheta) \; = \; - \frac{1}{2 \, \Partitionthetabar} \, \Exp_{\prompt \sim \promptdistr; \; \responseone, \, \responsetwo \sim \policytheta(\cdot \mid \prompt)}
		\Bigg[ \, & 
		\frac{\sigmoid \big( \rewardstar(\context, \responseone) - \rewardstar(\context, \responsetwo) \big) - \sigmoid \big( \rewardtheta(\context, \responseone) - \rewardtheta(\context, \responsetwo) \big)}{\divsigmoid \big( \rewardtheta(\prompt, \responseone) - \rewardtheta(\prompt, \responsetwo) \big)}  \notag  \\
		& \qquad \qquad \qquad \cdot \big\{ \gradtheta \rewardtheta(\prompt, \responseone) - \gradtheta \rewardtheta(\prompt, \responsetwo) \big\} \Bigg] \, .  \label{eq:gradLoss_0}
	\end{align}
	Using the intuition that the first-order Taylor expansion
	\begin{align*}
		\frac{\sigmoid(z^{\star}) - \sigmoid(z)}{\divsigmoid(z)} \; = \; (z^{\star} - z) + \bigO\big((z^{\star} - z)^2\big)
	\end{align*}
	is valid when $z \to z^\star$, with $z^\star \defn \rewardstar(\context, \responseone) - \rewardstar(\context, \responsetwo)$ and $z \defn \rewardtheta(\context, \responseone) - \rewardtheta(\context, \responsetwo)$, we find that
	\begin{align*}
		& \frac{\sigmoid \big( \rewardstar(\context, \responseone) - \rewardstar(\context, \responsetwo) \big) - \sigmoid \big( \rewardtheta(\context, \responseone) - \rewardtheta(\context, \responsetwo) \big)}{\divsigmoid \big( \rewardtheta(\prompt, \responseone) - \rewardtheta(\prompt, \responsetwo) \big)}  \\
		& \; = \; \Big\{ \big( \rewardstar(\context, \responseone) - \rewardstar(\context, \responsetwo) \big) - \big( \rewardtheta(\context, \responseone) - \rewardtheta(\context, \responsetwo) \big) \Big\} \; + \; \mbox{second-order term}.
	\end{align*}
	Reformulating equation~\eqref{eq:gradLoss_0} in this context, we rewrite it as
    \begin{align}
		\gradtheta \Loss(\paraphi) 
		& = - \, \frac{1}{2 \Partitionthetabar} \, \Exp_{\, \begin{subarray}{l} \\ \prompt \sim \promptdistr; \\ \responseone, \responsetwo \sim \policytheta(\cdot \mid \prompt) \end{subarray}}
		\Bigg[ \, \Big\{ \big( \rewardstar(\context, \responseone) - \rewardstar(\context, \responsetwo) \big) - \big( \rewardtheta(\context, \responseone) - \rewardtheta(\context, \responsetwo) \big) \Big\} \notag  \\
		& \qquad \qquad \qquad \qquad \qquad \qquad \qquad \qquad \quad \cdot \big\{ \gradtheta \rewardtheta(\prompt, \responseone) - \gradtheta \rewardtheta(\prompt, \responsetwo) \big\} \Bigg]
		+ \Term_2 \, , \label{eq:gradLoss_1}
	\end{align}
	where $\Term_2$ represents the second-order residual term related to the estimation error $\rewardtheta - \rewardstar$.
	By applying \Cref{lemma:grad_scalarvalue}, we observe that the primary term in equation~\eqref{eq:gradLoss_1} aligns with the direction of $\gradtheta \scalarvalue(\policytheta)$, resulting in
	\begin{align}
		\label{eq:gradLoss_final}
		\gradtheta \Loss(\paraphi) 
		& = - \, \frac{\parabeta}{\Partitionthetabar} \, \gradtheta \scalarvalue(\policytheta)
		+ \Term_2 \, .
	\end{align}

	
	Next, we proceed to control the second-order term $\Term_2$.
	The conditions
	\begin{align*}
		\supnorm{\rewardstar}, \supnorm{\rewardtheta} \leq \Radius
		\qquad \mbox{and} \qquad \sup\nolimits_{(\prompt, \response) \in \PromptSp \times \ResponseSp} \norm{\gradtheta \rewardtheta (\prompt, \response)}_2 \leq \RadiusGrad,
	\end{align*}
	lead to the bound
	\begin{align*}
		\abs[\Big]{ \, \frac{\sigmoid(z^{\star}) - \sigmoid(z)}{\divsigmoid(z)} - (z^{\star} - z) }
		\; \leq \;  0.1 \, (1 + e^{2\Radius}) \cdot (z^{\star} - z)^2 \, ,
	\end{align*}
	which in turn implies
	\begin{align}
		& \norm{\Term_2}_2
        \notag \\
        \label{eq:gradLoss_Term2}
        & \; \leq \;  \frac{0.1 \, (1 + e^{2\Radius}) \, \RadiusGrad}{\Partitionthetabar} \, \Exp_{\prompt \sim \promptdistr; \; \responseone, \responsetwo \sim \policytheta(\cdot \mid \prompt)} 
        \bigg[ \, \Big\{ \big( \rewardstar(\context, \responseone) - \rewardstar(\context, \responsetwo) \big) - \big( \rewardtheta(\context, \responseone) - \rewardtheta(\context, \responsetwo) \big) \Big\}^2 \bigg] \, .
	\end{align}
	
	Finally, combining equation~\eqref{eq:gradLoss_Term2} with equation~\eqref{eq:gradLoss_final}, we conclude the proof of \Cref{thm:grad}.
	
	
%%%%%%%%%%%%%%%%%%%%%%%%%%%%%%%%%%%%%%%%%%%%%%%%%%%%%%%%%%%%
		
		\subsubsection{Proof of Claim~\eqref{eq:responsedistravg}}
		\label{sec:proof:thm:grad_3}
		
		The remaining step in the proof of \Cref{thm:grad} is to verify the expression for the density ratio in equation~\eqref{eq:responsedistravg}.
		
		Based on the sampling scheme described in \Cref{sec:sampling}, we find that the sampling distribution for the response satisfies
		\begin{align}
			\label{eq:responsedistr_0}
			\responsedistr \big( \responseone, \responsetwo \bigm| \prompt \big)
			& \; = \; \{ 1 - \sampleprob(\prompt) \} \cdot \policytheta(\responseone \mid \prompt) \,  \policytheta(\responsetwo \mid \prompt)
			\, + \, \sampleprob(\prompt) \cdot \policythetapos(\responseone \mid \prompt) \,  \policythetaneg(\responsetwo \mid \prompt) \, ,
		\end{align}
		where the probability $\sampleprob(\prompt)$ is defined as
		\begin{align*}
			\sampleprob(\prompt) = \Partitionthetapos(\prompt) \, \Partitionthetaneg(\prompt) / \{1 + \Partitionthetapos(\prompt) \, \Partitionthetaneg(\prompt) \}
		\end{align*}
		and the policies $\policythetapos$ and $\policythetaneg$ are specified in equations~\eqref{eq:def_policythetapos}~and~\eqref{eq:def_policythetaneg}, respectively.
		This allows us to simplify equation~\eqref{eq:responsedistr_0} to
		\begin{align*}
			\responsedistr \big( \responseone, \responsetwo \bigm| \prompt \big)
			& \; = \; \frac{\policytheta(\responseone \mid \prompt) \, \policytheta(\responsetwo \mid \prompt)}{1 + \Partitionthetapos(\prompt) \, \Partitionthetaneg(\prompt)} \, \Big\{ 1 + \exp\big\{ \rewardtheta(\prompt, \responseone) - \rewardtheta(\prompt, \responsetwo) \big\} \Big\} \, .
		\end{align*}
		Similarly, we derive an expression for $\responsedistr ( \responsetwo, \responseone \mid \prompt )$.
		By averaging the two expressions, for $\responsedistr ( \responseone, \responsetwo \mid \prompt )$ and $\responsedistr ( \responsetwo, \responseone \mid \prompt )$, we obtain
		\begin{align*}
%			\label{eq:responsedistravg_ratio}
			& \frac{\responsedistravg ( \responseone, \responsetwo \mid \prompt )}{\policytheta(\responseone \mid \prompt) \, \policytheta(\responsetwo \mid \prompt)}  \\
			& = \frac{\policytheta(\responseone \mid \prompt) \, \policytheta(\responsetwo \mid \prompt)}{2 \, \{ 1 + \Partitionthetapos(\prompt) \, \Partitionthetaneg(\prompt) \}} \, \Big\{ 2 + \exp\big\{ \rewardtheta(\prompt, \responseone) - \rewardtheta(\prompt, \responsetwo) \big\} + \exp\big\{ \rewardtheta(\prompt, \responsetwo) - \rewardtheta(\prompt, \responseone) \big\} \Big\} \, .
		\end{align*}
		Rewriting this expression using the formula for $\divsigmoid$ in equation~\eqref{eq:divsigmoid}, we arrive at
		\begin{align*}
			& \big\{ 1 + \Partitionthetapos(\prompt) \, \Partitionthetaneg(\prompt) \big\} \cdot \frac{\responsedistravg ( \responseone, \responsetwo \mid \prompt )}{\policytheta(\responseone \mid \prompt) \, \policytheta(\responsetwo \mid \prompt)}  \\
			& \; = \; \frac{1}{2} \, \Big\{ 1 + \exp\big\{ \rewardtheta(\prompt, \responsetwo) - \rewardtheta(\prompt, \responseone) \big\} \Big\}  \Big\{ 1 + \exp\big\{ \rewardtheta(\prompt, \responseone) - \rewardtheta(\prompt, \responsetwo) \big\} \Big\}  \\
			& \; = \; \frac{1}{2 \, \divsigmoid \big( \rewardtheta(\prompt, \responseone) - \rewardtheta(\prompt, \responsetwo) \big)} \, .
		\end{align*}
		Finally, rearranging terms, we recover equation~\eqref{eq:responsedistravg}, completing this part of the proof.

%%%%%%%%%%%%%%%%%%%%%%%%%%%%%%%%%%%%%%%%%%%%%%%%%%%%%%%%%%%%%%%%%%%%%%%%%%%%

	\subsection{Statistical Considerations \yaqidone}
	\label{sec:proof:thm:stat}



        In this section, we present the proofs for \Cref{thm:asymp,lemma:hess_scalarvalue,thm:asymp_full} from \Cref{sec:theory_stat}. 
        We start with the proof of \Cref{thm:asymp_full} in \Cref{sec:proof:thm:asymp_full}, with a rigorous restatement provided in \Cref{thm:asymp_full_full} below.
    		\begin{theorem}
			\label{thm:asymp_full_full}
%			We take $\weight(\prompt) \equiv 1$.
			Assume the reward model $\rewardstar$ in the BT model~\eqref{eq:BT} satisfies $\rewardstar = \reward_{\parathetastar}$ for some parameter $\parathetastar$.
			Assume that $\parathetahat$ minimizes the loss function $\Losshat(\paratheta)$ in the sense that $\sqrt{\numobs} \, \gradtheta \Losshat (\parathetahat) \convergep \veczero$ and that $\parathetahat \convergep \parathetastar$ as the sample size $\numobs \rightarrow \infty$.
			Additionally, suppose the reward function $\rewardtheta(\prompt, \response)$, its gradient $\gradtheta \rewardtheta(\prompt, \response)$ and its Hessian $\hesstheta \rewardtheta(\prompt, \response)$ are uniformly bounded and Lipchitz continuous with respect to $\paratheta$, for all $(\prompt, \response) \in \PromptSp \times \ResponseSp$.
			
			Under these conditions, the estimate $\parathetahat$ asymptotically follows a Gaussian distribution
			\begin{align*}
				\sqrt{\numobs} \; ( \parathetahat - \parathetastar)
				\; \stackrel{d}{\longrightarrow} \; \Gauss( \veczero, \CovOmega )
				\qquad \mbox{as $\numobs \rightarrow \infty$} \, .
			\end{align*}
			We have an estimate of the covariance matrix $\CovOmega$:
            \begin{align*}
                \CovOmega \; \preceq \; \supnorm{\weight} \cdot \CovOpstar^{-1} \, .
            \end{align*}
            For a general sampling scheme $\responsedistr$ chosen, the matrix~$\CovOpstar$ is given by
			\begin{align*}
                % \label{eq:def_CovOpstar_simple}
				\CovOpstar \; \defn \;
				& \Exp_{\prompt \sim \promptdistr, \, (\responseone, \, \responsetwo) \sim \responsedistravg(\cdot \mid \prompt)}
			\Big[ \, \weight(\prompt) \cdot \Var\big(\indicator\{\responseone = \responsewin\} \bigm| \prompt, \responseone, \responsetwo \big) \cdot \grad \, \grad^{\top} \Big] \, ,
				%\label{eq:def_CovOpstar}
			\end{align*}
			where the expectation is taken over the distribution
			%\vspace{-.3em}
			\begin{subequations}
				\begin{align*}
					%\label{eq:def_responsedistravg}
					\responsedistravg(\responseone, \responsetwo \mid \prompt) 
					\defn \frac{1}{2} \, \big\{ \responsedistr(\responseone, \responsetwo \mid \prompt) + \responsedistr(\responsetwo, \responseone \mid \prompt) \big\} \, .
				\end{align*} %~ \vspace{-1.8em} \\
			The variance term is specified as
				\begin{align*}
					& \Var\big(\indicator\{\responseone \; = \; \responsewin\} \mid \prompt, \responseone, \responsetwo \big)
					%\label{eq:def_var}
					= \sigmoid\big( \rewardstar(\prompt, \responseone) - \rewardstar(\prompt, \responsetwo) \big) \, \sigmoid\big( \rewardstar(\prompt, \responsetwo) - \rewardstar(\prompt, \responseone) \big)
					%\notag
				\end{align*}
			and the gradient difference $\grad$ is defined as
				\begin{align*}
					%\label{eq:def_grad}
					\grad \; \defn \; \gradtheta \rewardstar(\prompt, \responseone) - \gradtheta \rewardstar(\prompt, \responsetwo) \, .
				\end{align*}
			\end{subequations}
		\end{theorem}

    \Cref{thm:asymp_full_full} establishes the asymptotic distribution of the estimated parameter $\parathetahat$, which serves as the foundation for the subsequent results. 
	Next, we show that \Cref{thm:asymp} directly follows as a corollary of \Cref{thm:asymp_full_full}, with the detailed derivation provided in \Cref{sec:proof:thm:asymp}. Finally, in \Cref{sec:proof:lemma:hess_scalarvalue}, we prove \Cref{lemma:hess_scalarvalue}, which describes the asymptotic behavior of the value gap $\scalarvalue(\policystar) - \scalarvalue(\policyhat)$.
		
	\subsubsection{Proof of Lemma~\ref{thm:asymp_full} (Theorem~\ref{thm:asymp_full_full}) \yaqidone}
	\label{sec:proof:thm:asymp_full}
	
%	\paragraph{(a) Proof of \Cref{thm:asymp_full}:}

	In this section, we analyze the asymptotic distribution of the estimated parameter $\parathetahat$ for a general sampling distribution $\responsedistr$. The parameter $\parathetahat$ is obtained by solving the optimization problem
	\begin{align*}
		{\rm minimize}_{\paratheta} \quad
		\Losshat(\paratheta) \; \defn \;
		- \frac{1}{\numobs} \sum_{i=1}^{\numobs} \, \weight(\prompti{i}) \cdot \log \sigmoid \Big( \rewardtheta\big(\prompti{i}, \responsewini{i}\big) - \rewardtheta\big(\prompti{i}, \responselosei{i}\big) \Big) \, .
	\end{align*}
	We assume the optimization is performed to sufficient accuracy such that $\gradtheta \Losshat(\parathetahat) = \smallop\big(\numobs^{-\frac{1}{2}}\big)$.
	Under this condition, $\parathetahat$ qualifies as a $Z$-estimator.
	To study its asymptotic behavior, we use the master theorem for $Z$-estimators \citep{kosorok2008introduction}, the formal statement of which is provided in \Cref{thm:master} in \Cref{sec:master}.
	
	To apply the master theorem, we set $\Psi \defn \gradtheta \Loss$ and $\Psi_{\numobs} \defn \gradtheta \Losshat$ and verify the conditions. In particular, the smoothness condition~\eqref{eq:master_cond} in \Cref{thm:master} translates to the following equation in our context:
    \begin{align}
    	\label{eq:master_cond_proof}
    	& \sqrt{n} \, \big\{ \gradtheta \Losshat (\parathetahat) - \gradtheta \Loss(\parathetahat) \big\} - \sqrt{n} \, \big\{ \gradtheta \Losshat (\parathetastar) - \gradtheta \Loss (\parathetastar) \big\}  
    	\; = \; \smallop \big( 1 + \sqrt{n} \, \norm{ \parathetahat - \parathetastar }_2 \big) \, .
    \end{align}
    This condition follows from the second-order smoothness of the reward function $\rewardtheta$ with respect to $\paratheta$. A rigorous proof is provided in \Cref{sec:proof:eq:master_cond_proof}.
    

	We now provide the explicit form of the derivative $\dot{\Psi}_{\parathetastar} = \hesstheta \Loss(\parathetastar)$, as captured in the following lemma. The proof of this result can be found in \Cref{sec:proof:lemma:hess_loss}.
	\begin{lemma}
		\label{lemma:hess_loss}
		The Hessian matrix of the population loss $\Loss(\paratheta)$ at $\paratheta = \parathetastar$ is
		\begin{align}
			\label{eq:hess_loss}
			\hesstheta \Loss(\parathetastar) \; = \; \CovOpstar \, ,
		\end{align}
		where the matrix $\CovOpstar$ is defined in equation~\eqref{eq:def_CovOpstar}.
	\end{lemma}

	
	Next, we analyze the asymptotic behavior of the gradient $\gradtheta \Losshat(\parathetastar)$.
	The proof is deferred to \Cref{sec:proof:lemma:grad_loss_stat}.
	\begin{lemma}
		\label{lemma:grad_loss_stat}
		The gradient of the empirical loss $\Losshat(\paratheta)$ at $\paratheta = \parathetastar$ satisfies
		\begin{subequations}
		\begin{align}
			\sqrt{\numobs} \, \big( \gradtheta \Losshat(\parathetastar) - \gradtheta \Loss(\parathetastar) \big)
			\; \stackrel{d}{\longrightarrow} \; \Gauss(\veczero, \CovOptil)
			\qquad \mbox{as $\numobs \rightarrow \infty$},
		\end{align}
        where the covariance matrix $\CovOptil \in \Real^{\Dim \times \Dim}$ is bounded as follows:
        \begin{align}
        	\label{eq:CovOptil}
        	\CovOptil \; \preceq \; \supnorm{\weight} \cdot \CovOpstar \, ,
        \end{align}
        \end{subequations}
        with $\CovOpstar$ defined in equation~\eqref{eq:def_CovOpstar}.
	\end{lemma}
	
	Combining these results, and assuming $\CovOpstar$ is nonsingular, the master theorem (\Cref{thm:master}) yields the asymptotic distribution of $\parathetahat$:
	\begin{align*}
		\sqrt{\numobs} \, \big( \parathetahat - \parathetastar \big)
		\; \converged \; \Gauss\big( \veczero, \CovOpstar^{-1} \CovOptil \CovOpstar^{-1} \big) \, .
	\end{align*}
	Furthermore, from the bound~\eqref{eq:CovOptil}, the covariance matrix $\CovOmega ; \defn \CovOpstar^{-1} \CovOptil \CovOpstar^{-1}$ satisfies
	\begin{align*}
		 \CovOmega \; = \CovOpstar^{-1} \CovOptil \CovOpstar^{-1}  \; \preceq \; \supnorm{\weight} \cdot \CovOpstar^{-1} \, .
	\end{align*}
	Therefore, we have established the asymptotic distribution of $\parathetahat$, completing the proof of \Cref{thm:asymp_full}.
	
	
%%%%%%%%%%%%%%%%%%%%%%%%%%%%%%%%%%%%%%%%%%%%%%%%%%%%%%%%%%%%%%%%%%%%%%%%%%%%%%

%	\paragraph{(b) Proof of \eqref{eq:def_CovOpstar_simple}}
	\subsubsection{Proof of Theorem~\ref{thm:asymp}}
	\label{sec:proof:thm:asymp}
	
	\Cref{thm:asymp} is a direct corollary of \Cref{thm:asymp_full}, using our specific choice of sampling distribution $\responsedistr$. To establish this, we demonstrate how the general covariance matrix $\CovOpstar$ in equation~\eqref{eq:def_CovOpstar} simplifies to the form in equation~\eqref{eq:def_CovOpstar_simple} under our proposed sampling scheme.

    To establish the result in this section, we impose the following regularity condition:
    There exists a constant $\Const{} \geq 1$ satisfying
    \begin{align}
        \label{eq:last_cond}
        \Var_{\rewardtheta}\big(\indicator\{\responseone = \responsewin\} \bigm| \prompt, \responseone, \responsetwo \big)
        \; \leq \; \Const{} \cdot \Var_{\rewardstar}\big(\indicator\{\responseone = \responsewin\} \bigm| \prompt, \responseone, \responsetwo \big) 
    \end{align}
    for any prompt $\prompt \in \PromptSp$ and responses $\responseone, \responsetwo \in \ResponseSp$.
    Here $\Var_{\rewardtheta}\big(\indicator\{\responseone = \responsewin\} \bigm| \prompt, \responseone, \responsetwo \big)$ denotes the conditional variance under the BT model~\eqref{eq:BT}, when the implicit reward function $\rewardstar$ is replaced by $\rewardtheta$. The term \mbox{$\Var_{\rewardstar}\big(\indicator\{\responseone = \responsewin\} \bigm| \prompt, \responseone, \responsetwo \big)
    \equiv$} \mbox{$\Var\big(\indicator\{\responseone = \responsewin\} \bigm| \prompt, \responseone, \responsetwo \big) $} represents the conditional variance under the ground-truth BT model, where the reward function is given by $\rewardstar$.
	
	We begin by leveraging the property of the sampling distribution $\responsedistr$ from equation~\eqref{eq:responsedistravg} and the derivative $\divsigmoid$ of the sigmoid function $\sigmoid$, given in equation~\eqref{eq:divsigmoid}. Specifically, we find that
	\begin{align*}
		%\label{eq:responsedistravg2_original}
		& \frac{\responsedistravg ( \responseone, \responsetwo \mid \prompt )}{\policytheta(\responseone \mid \prompt) \, \policytheta(\responsetwo \mid \prompt)} \notag  \\
		& 
		\; = \; \frac{1}{2 \, \{ 1 + \Partitionthetapos(\prompt) \, \Partitionthetaneg(\prompt) \}}
		\cdot \frac{1}{\sigmoid \big( \rewardtheta(\prompt, \responseone) - \rewardtheta(\prompt, \responsetwo) \big) \, \sigmoid \big( \rewardtheta(\prompt, \responsetwo) - \rewardtheta(\prompt, \responseone) \big)}  \\
        & \; = \; \frac{1}{2 \, \{ 1 + \Partitionthetapos(\prompt) \, \Partitionthetaneg(\prompt) \}}
		\cdot \frac{1}{\Var_{\rewardtheta}\big(\indicator\{\responseone = \responsewin\} \bigm| \prompt, \responseone, \responsetwo \big)} \, .
	\end{align*}
    We then apply condition~\eqref{eq:last_cond} and derive
        \begin{equation} 
		 \frac{\responsedistravg ( \responseone, \responsetwo \mid \prompt )}{\policytheta(\responseone \mid \prompt) \, \policytheta(\responsetwo \mid \prompt)} \; \geq \; \frac{\Const{}^{-1}}{2 \, \{ 1 + \Partitionthetapos(\prompt) \, \Partitionthetaneg(\prompt) \}} \cdot \frac{1}{\Var_{\rewardstar}\big(\indicator\{\responseone = \responsewin\} \bigm| \prompt, \responseone, \responsetwo \big)} \, .
         \label{eq:responsedistravg2}
	\end{equation}
	Next, substituting this result~\eqref{eq:responsedistravg2} into equation~\eqref{eq:def_CovOpstar}, alongside the weight function $\weight(\prompt)$ from equation~\eqref{eq:weight}, we reform $\CovOpstar$ as
	\begin{align}
		\CovOpstar
		& \; = \; \Exp_{\prompt \sim \promptdistr; \; \responseone, \, \responsetwo \sim \policytheta(\cdot \mid \prompt)}
		\bigg[ \, \frac{\responsedistravg ( \responseone, \responsetwo \mid \prompt )}{\policytheta(\responseone \mid \prompt) \, \policytheta(\responsetwo \mid \prompt)} \cdot \weight(\prompt) \cdot \Var\big(\indicator\{\responseone = \responsewin\} \bigm| \prompt, \responseone, \responsetwo \big) \cdot \grad \, \grad^{\top} \bigg]  \notag \\
		\label{eq:def_CovOpstar_2}
		& \; \succeq \; \frac{1}{2 \, \Const{} \, \Partitionthetabar} \, \Exp_{\prompt \sim \promptdistr; \; \responseone, \, \responsetwo \sim \policytheta(\cdot \mid \prompt)}
		\big[ \, \grad \, \grad^{\top} \big] \, .
	\end{align}
	The conditional expectation of $\grad \grad^\top$ simplifies as
    \begin{align*}
    	& \Exp_{\responseone, \, \responsetwo \sim \policytheta(\cdot \mid \prompt)}
    	\big[ \, \grad \grad^{\top} \bigm| \prompt\big]  \\
    	& \; = \; \Exp_{\responseone, \, \responsetwo \sim \policytheta(\cdot \mid \prompt)}
    	\Big[ \big\{ \gradtheta \rewardstar(\prompt, \responseone) - \gradtheta \rewardstar(\prompt, \responsetwo) \big\} \big\{ \gradtheta \rewardstar(\prompt, \responseone) - \gradtheta \rewardstar(\prompt, \responsetwo) \big\}^{\top} \Bigm| \prompt\Big]  \\
    	& \; = \; 2 \cdot \Exp_{\response \sim \policytheta(\cdot \mid \prompt)}
    	\Big[ \, \gradtheta \rewardstar(\prompt, \response) \, \gradtheta \rewardstar(\prompt, \response)^{\top} \Bigm| \prompt\Big] \\
        & \qquad \qquad - 2 \cdot \Exp_{\response \sim \policytheta(\cdot \mid \prompt)} \big[ \, \gradtheta \rewardstar(\prompt, \response) \bigm| \prompt\big]  \, \Exp_{\response \sim \policytheta(\cdot \mid \prompt)}
    	\big[ \,\gradtheta \rewardstar(\prompt, \response) \bigm| \prompt \big]^{\top}  \\
    	& \; = \; 2 \cdot \Cov_{\response \sim \policytheta(\cdot \mid \prompt)} \big[ \gradtheta \rewardstar(\prompt, \response) \bigm| \prompt \big] \, .
    \end{align*}
	Substituting this result into equation~\eqref{eq:def_CovOpstar_2}, we arrive at the conclusion that
	\begin{align*}
		\CovOpstar \; \succeq \; \frac{1}{\Const{} \, \Partitionphibar} \, \Exp_{\prompt \sim \promptdistr} \Big[ \Cov_{\response \sim \policystar(\cdot \mid \prompt)} \big[ \gradtheta \rewardstar(\prompt, \response) \bigm| \prompt \big] \Big] \, ,
	\end{align*}
	which matches the simplified form in equation~\eqref{eq:def_CovOpstar_simple} as stated in \Cref{thm:asymp}.

		
		
		
%%%%%%%%%%%%%%%%%%%%%%%%%%%%%%%%%%%%%%%%%%%%%%%%%%%%%%%%%%%%%%%%%%%%%%%%%%%%
	
	
    \subsubsection{Proof of Theorem~\ref{lemma:hess_scalarvalue} \yaqidone}
    \label{sec:proof:lemma:hess_scalarvalue}

	\paragraph{Gradient $ \gradtheta \scalarvalue(\policystar) $ and Hessian $\hesstheta \scalarvalue(\policystar)$:}
	
    The equality $ \gradtheta \scalarvalue(\policystar) = 0$ follows directly from the gradient expression~\eqref{eq:grad_scalarvalue0} for $ \gradtheta \scalarvalue(\policytheta) $, evaluated at $ \paratheta = \parathetastar $ with~\mbox{$ \rewardtheta = \rewardstar $}.
    
    The proof of the Hessian result, $ \hesstheta \scalarvalue(\policystar) = - (1 / \parabeta) \cdot \CovOpstar $, involves straightforward but technical differentiation of equation~\eqref{eq:grad_scalarvalue0}. For brevity, we defer this proof to \Cref{sec:proof:eq:hessscalarvalue}.
  
  	\paragraph{Asymptotic Distribution of Value Gap $ \scalarvalue(\policystar) - \scalarvalue(\policyhat) $:}
    To understand the behavior of the value gap $ \scalarvalue(\policystar) - \scalarvalue(\policyhat) $, we start by applying a Taylor expansion of $ \scalarvalue(\policytheta) $ around $ \parathetastar $. This gives
	\begin{align*}
%		\label{eq:Taylor_scalarvalue}
		\scalarvalue(\policystar) - \scalarvalue(\policyhat)
		\; = \; \gradtheta \scalarvalue(\policystar)^{\top} (\parathetastar - \parathetahat) - \frac{1}{2} (\parathetastar - \parathetahat)^{\top} \hesstheta \scalarvalue(\policystar) (\parathetastar - \parathetahat) + \smallo\big( \norm{\parathetastar - \parathetahat}_2^2 \big) \, .
	\end{align*}
	By substituting $ \gradtheta \scalarvalue(\policystar) = \veczero $ (a direct result of the optimality of $ \policystar $), the linear term vanishes. Introducing the shorthand $ \HessMt \defn -\hesstheta \scalarvalue(\policystar) = (1 / \parabeta) \cdot \CovOpstar $, the expression simplifies to
	\begin{align}
		\label{eq:Taylor_scalarvalue}
		\scalarvalue(\policystar) - \scalarvalue(\policyhat)
		\; = \; \frac{1}{2} \, (\parathetahat - \parathetastar)^{\top} \HessMt \, (\parathetahat - \parathetastar) + \smallo\big( \norm{\parathetahat - \parathetastar}_2^2 \big) \, .
	\end{align}
	When the sample size $ \numobs $ is sufficiently large, $ \parathetahat $ approaches $ \parathetastar $, making the higher-order term negligible. Therefore, the value gap is dominated by the quadratic form.
	
	From \Cref{thm:asymp}, we know the parameter estimate $ \parathetahat $ satisfies
	\begin{align*}
	\sqrt{\numobs} \, (\parathetahat - \parathetastar)
	\;\stackrel{d}{\longrightarrow}\;
	\Gauss(\veczero, \CovOmega).
	\end{align*}
	Substituting this result into the quadratic approximation of the value gap, we find that the scaled value gap has the asymptotic distribution
	\begin{align}
		\label{eq:gap_distr}
		\numobs \cdot \{ \scalarvalue(\policystar) - \scalarvalue(\policyhat) \}
		 \; \stackrel{d}{\longrightarrow} \; \frac{1}{2} \, \vecz^{\top} \CovOmega^{\frac{1}{2}} \HessMt \CovOmega^{\frac{1}{2}} \vecz 
		 \; \nfed \bX
		 \qquad \mbox{where $\vecz \sim \Gauss(\veczero, \IdMt)$}.
	\end{align}
	This approximation provides a clear intuition: the value gap is asymptotically driven by a weighted chi-squared-like term involving the covariance structure $ \CovOmega $ and the Hessian-like matrix $ \HessMt $.
	
	To rigorously establish this result, we will apply Slutsky’s theorem. The full proof is presented in \Cref{sec:proof:gap_distr}.
	
	\paragraph{Bounding the Chi-Square Distribution:}
	
	To bound the random variable $ \bX $, we first leverage the estimate of the covariance matrix $ \CovOmega $ provided by \Cref{thm:asymp}:
	\begin{align*}
		\CovOmega \; \preceq \; \Const{} \, \Partitionthetabar \, \supnorm{\weight} \cdot \CovOpstar^{-1},
	\end{align*}
    where the constant $\Const$ comes from condition~\eqref{eq:last_cond}.
	It follows that the matrix $ \CovOmega^{\frac{1}{2}} \HessMt \CovOmega^{\frac{1}{2}} $ appearing in equation~\eqref{eq:gap_distr} can be bounded as
	\begin{align*}
		\CovOmega^{\frac{1}{2}} \HessMt \CovOmega^{\frac{1}{2}} 
		\; \preceq \;  \Const \, \supnorm{\weight} \cdot \CovOpstar^{-\frac{1}{2}} \HessMt \CovOpstar^{-\frac{1}{2}} \; = \; \Const \cdot \frac{\Partitionthetabar \, \supnorm{\weight}}{\parabeta} \cdot \IdMt
		\; = \; \Const \cdot \frac{1 + \supnorm{\Partitionthetapos \Partitionthetaneg}}{\parabeta}
		\cdot \IdMt \, .
	\end{align*}
	Here the last equality uses the definition of the weight function $ \weight $ from equation~\eqref{eq:weight}. Substituting this bound into the quadratic form, we derive
	\begin{align*}
		\bX
		\; = \; \frac{1}{2} \, \vecz^{\top} \CovOmega^{\frac{1}{2}} \HessMt \CovOmega^{\frac{1}{2}} \vecz 
		\; \leq \; \Const \cdot \frac{1 + \supnorm{\Partitionthetapos \Partitionthetaneg}}{2\parabeta}
		\cdot \vecz^{\top} \vecz \, ,
	\end{align*}
	where $ \vecz \sim \Gauss(\veczero, \IdMt) $.
	Since $ \vecz^{\top} \vecz $ follows a chi-square distribution with $ \Dim $ degrees of freedom, $ \bX $ is stochastically dominated by a rescaled chi-square random variable 
	\begin{align*}
		\Const \cdot \frac{1 + \supnorm{\Partitionthetapos \Partitionthetaneg}}{2\parabeta} \cdot \chisquare_{\Dim}.
	\end{align*}
	Equivalently, we can express this dominance as
	\begin{align}
		\label{eq:gap_bd0}
		\limsup_{\numobs \rightarrow \infty} \; \Prob \bigg\{ \numobs \, \{ \scalarvalue(\policystar) - \scalarvalue(\policyhat) \} > \Const \cdot \frac{1 + \supnorm{\Partitionthetapos \Partitionthetaneg}}{2\parabeta} \cdot t \bigg\}
		\; \leq \; \Prob\big\{ \chisquare_{\Dim} > t \big\}
		\qquad \mbox{for any $t > 0$}.
	\end{align}
	This inequality, given in equation~\eqref{eq:gap_bd0}, corresponds to the first bound in equation~\eqref{eq:gap_bd}.
	
	The second inequality in equation~\eqref{eq:gap_bd} provides a precise tail bound for $\chisquare_{\Dim}$. As its proof involves more technical details, we defer it to \Cref{sec:proof:chisqtail}.
	

	


	
	
%%%%%%%%%%%%%%%%%%%%%%%%%%%%%%%%%%%%%%%%%%%%%%%%%%%%%%%%%%%%%%%%%%%%%%%%%%%%

	\section{Proof of Auxiliary Results \yaqidone}
    \label{app:aux}
	
	This section provides proofs of auxiliary results supporting the main theorems and lemmas. In \Cref{sec:proof:aux:thm:grad}, we present the auxiliary results required for \Cref{thm:grad}. \Cref{sec:proof:thm:asymp_aux} details the proofs of supporting results for \Cref{thm:asymp}. Finally, in \Cref{sec:proof:lemma:hess_scalarvalue_aux}, we establish the auxiliary results necessary for \Cref{lemma:hess_scalarvalue}.

	\subsection{Proof of Auxiliary Results for Theorem~\ref{thm:grad} \yaqidone}
	\label{sec:proof:aux:thm:grad}
	
		In this section, we provide the proofs of several auxiliary results that support the proof of \Cref{thm:grad}. Specifically,
		\Cref{sec:proof:lemma:grad_policy} presents the forms of the gradients of the policy~$\policytheta$ and the reward $\rewardtheta$, which serve as fundamental building blocks for deriving the lemmas.
		\Cref{sec:proof:lemma:grad_scalarvalue} analyzes the gradient of the return function $\scalarvalue(\policytheta)$, as defined in equation~\eqref{eq:objective}.
		\Cref{sec:proof:lemma:grad_loss} focuses on deriving expressions for the gradient of the negative log-likelihood function $\Loss(\paratheta)$.
	
		\subsubsection{Gradients of Policy $\policytheta$ and Reward $\rewardtheta$}
		\label{sec:proof:lemma:grad_policy}
		
		In this part, we introduce results for the gradients of policy $\policytheta$ and reward~$\rewardtheta$ with respsect to parameter~$\paratheta$, which lay the foundation of our calculations.
		
			\begin{lemma}[Gradients of policy $\policytheta$ and reward function $\rewardtheta$]
			\label{lemma:grad_policy}
			The gradients of the policy $\policytheta$ and the reward function $\rewardtheta$ can be expressed in terms of each other as follows
			\begin{subequations}
				\begin{align}
					\label{eq:gradpolicy}
					\gradtheta \policytheta(\diff \response \mid \prompt)
					& \; = \;  \policytheta(\diff \response \mid \prompt) \cdot \frac{1}{\parabeta} \,
					\Big\{ \gradtheta \rewardtheta(\prompt, \response) - \Exp_{\responsenew \sim \policytheta(\cdot \mid \prompt)}\big[ \gradtheta \rewardtheta(\prompt, \responsenew) \big] \Big\} \, ,  \\
					\label{eq:gradreward}
					\gradtheta \rewardtheta (\prompt, \response)
					& \; = \; \parabeta \cdot \frac{\gradtheta \policytheta(\response \mid \prompt)}{\policytheta(\response \mid \prompt)} \, .
				\end{align}
			\end{subequations}
		\end{lemma}
		
		We now proceed to prove \Cref{lemma:grad_policy}.  \\
		
		To begin, recall our definition of the reward function $\rewardtheta$ as given in equation~\eqref{eq:def_reward}.
		It directly follows that
		\begin{align*}
			\gradtheta \rewardtheta (\prompt, \response)
			\; = \; \parabeta \cdot \frac{\gradtheta \policytheta(\response \mid \prompt)}{\policytheta(\response \mid \prompt)} \, .
		\end{align*}
		This result confirms equation~\eqref{eq:gradreward} as stated in \Cref{lemma:grad_policy}.
		
		Next, we express the policy $\policytheta(\diff \response \mid \prompt)$ in terms of the reward function $\rewardtheta(\prompt, \response)$. By reformulating equation~\eqref{eq:def_reward}, we obtain
		\begin{subequations}
		\begin{align}
			\label{eq:policyfromreward}
			\policytheta(\diff \response \mid \prompt)
			\; = \; \frac{1}{\Partitiontheta (\prompt)} \, \policyref(\diff \response \mid \prompt)
			\exp \Big\{ \frac{1}{\parabeta} \, \rewardtheta(\prompt, \response) \Big\} \, ,
		\end{align}
		where $\Partitiontheta (\prompt)$ is the partition function defined as
		\begin{align}
			\label{eq:def_Partition}
			\Partitiontheta (\prompt)
			& \; = \; \int_{\ResponseSp} \, \policyref(\diff \response \mid \prompt)
			\exp \Big\{ \frac{1}{\parabeta} \, \rewardtheta(\prompt, \response) \Big\} \, .
		\end{align}
		\end{subequations}
		
		We then compute the gradient of $\policytheta(\diff \response \mid \prompt)$ with respect to $\paratheta$. Applying the chain rule, we get
		\begin{align}
			\gradtheta \policytheta(\diff \response \mid \prompt)
			& \; = \; \frac{1}{\Partitiontheta (\prompt)} \, \policyref(\diff \response \mid \prompt)
			\exp \Big\{ \frac{1}{\parabeta} \, \rewardtheta(\prompt, \response) \Big\}
			\cdot \frac{1}{\parabeta} \, \gradtheta \rewardtheta(\prompt, \response)  \notag  \\
			\label{eq:gradtheta1}
			& \quad - \frac{1}{\Partitiontheta^2(\prompt)} \, \policyref(\diff \response \mid \prompt)
			\exp \Big\{ \frac{1}{\parabeta} \, \rewardtheta(\prompt, \response) \Big\}
			\cdot \gradtheta \Partitiontheta(\prompt) \, .
		\end{align}
		We need the gradient of the partition function $\Partitiontheta(\prompt)$:
		\begin{align}
			\gradtheta \Partitiontheta (\prompt)
			& \; = \; \int_{\ResponseSp} \, \policyref(\diff \response \mid \prompt)
			\exp \Big\{ \frac{1}{\parabeta} \, \rewardtheta(\prompt, \response) \Big\}
			\cdot \frac{1}{\parabeta} \, \gradtheta \rewardtheta(\prompt, \response)  \notag   \\
			& \; = \; \Partitiontheta (\prompt) \cdot \int_{\ResponseSp} \, \policytheta(\diff \response \mid \prompt) \cdot \frac{1}{\parabeta} \, \gradtheta \rewardtheta(\prompt, \response)  \notag   \\
			\label{eq:gradPartition}
			& \; = \; \Partitiontheta (\prompt) \cdot \frac{1}{\parabeta} \, \Exp_{\response \sim \policytheta(\cdot \mid \prompt)} \big[ \gradtheta \rewardtheta(\prompt, \response) \big] \, .
		\end{align}
		Substituting equation~\eqref{eq:gradPartition} back into equation~\eqref{eq:gradtheta1}, we simplify the expression for the gradient of $\policytheta(\diff \response \mid \prompt)$:
		\begin{align*}
			& \gradtheta \policytheta(\diff \response \mid \prompt)  \\
			& \; = \; \frac{1}{\Partitiontheta (\prompt)} \, \policyref(\diff \response \mid \prompt)
			\exp \Big\{ \frac{1}{\parabeta} \, \rewardtheta(\prompt, \response) \Big\}
			\cdot \frac{1}{\parabeta} \, \Big\{ \gradtheta \rewardtheta(\prompt, \response) - \Exp_{\responsenew \sim \policytheta(\cdot \mid \prompt)} \big[ \gradtheta \rewardtheta(\prompt, \responsenew) \big] \Big\} \, .
		\end{align*}
		This matches equation~\eqref{eq:gradpolicy} from \Cref{lemma:grad_policy}, thereby completing the proof.
		
		
		
%%%%%%%%%%%%%%%%%%%%%%%%%%%%%%%%%%%%%%%%%%%%%%%%%%%%%%%%%%%%%%%%%%%%%%%%%%%%%%%%%%%%%%%%%%%%
	
	
		\subsubsection{Proof of Lemma~\ref{lemma:grad_scalarvalue} \yaqidone}
		\label{sec:proof:lemma:grad_scalarvalue}
		
		Equality \eqref{eq:grad_scalarvalue} in \Cref{lemma:grad_scalarvalue} can be derived as a consequence of a more detailed result. We state it in \Cref{lemma:grad_scalarvalue_full}.
		
		\begin{lemma}
			\label{lemma:grad_scalarvalue_full}
			\begin{subequations}
			For a policy $\policytheta$, the gradients with respect to the parameter $\paratheta$ of its expected return $\Exp_{\prompt \sim \promptdistr, \, \response \sim \policytheta(\cdot \mid \prompt)} \big[ \rewardstar(\context, \response) \big] $ and its KL divergence from a reference policy $\kull{\policytheta}{\policyref}$ are given by
			\begin{align}
				& \gradtheta \Exp_{\prompt \sim \promptdistr, \, \response \sim \policytheta(\cdot \mid \prompt)} \big[ \rewardstar(\context, \response) \big]  \notag \\
				\label{eq:grad_return}
				& 
				\qquad  = \; \frac{1}{\parabeta} \, \Exp_{\prompt \sim \promptdistr, \,  \response \sim \policytheta(\cdot \mid \prompt)}
				\bigg[ \rewardstar(\prompt, \response)
				\Big\{ \gradtheta \rewardtheta(\prompt, \response) - \Exp_{\responsenew \sim \policytheta(\cdot \mid \prompt)}\big[ \gradtheta \rewardtheta(\prompt, \responsenew) \big] \Big\} \bigg] \, , \\
				& \gradtheta \kull{\policytheta}{\policyref}  \notag  \\
				\label{eq:grad_KL}
				& \qquad = 
				\frac{1}{\parabeta^2} \, \Exp_{\prompt \sim \promptdistr, \, \response \sim \policytheta(\cdot \mid \prompt)}
				\bigg[ \rewardtheta(\prompt, \response)
				\Big\{ \gradtheta \rewardtheta(\prompt, \response) - \Exp_{\responsenew \sim \policytheta(\cdot \mid \prompt)}\big[ \gradtheta \rewardtheta(\prompt, \responsenew) \big] \Big\} \bigg] \, .
			\end{align}
			\end{subequations}
		\end{lemma}
		
		Recall that the scalar value $\scalarvalue(\policytheta)$ of the policy is defined as
		\begin{align*}
			\scalarvalue(\policytheta) \; = \;
			\Exp_{\prompt \sim \promptdistr, \, \response \sim \policytheta(\cdot \mid \prompt)} \big[ \rewardstar(\context, \response) \big] \, - \,
			\parabeta \, \kull{\policytheta}{\policyref} \, .
		\end{align*}
		Using \Cref{lemma:grad_scalarvalue_full}, we derive the gradient of $\scalarvalue(\policytheta)$ as
		\begin{align}
			& \gradtheta \scalarvalue(\policytheta) \; = \; \gradtheta \Exp_{\prompt \sim \promptdistr, \, \response \sim \policytheta(\cdot \mid \prompt)} \big[ \rewardstar(\context, \response) \big] \, - \,
			\parabeta \, \gradtheta \kull{\policytheta}{\policyref}  \notag  \\
			\label{eq:grad_scalarvalue0}
			& \; = \; \frac{1}{\parabeta} \, \Exp_{\prompt \sim \promptdistr, \,  \response \sim \policytheta(\cdot \mid \prompt)}
			\bigg[ \big\{ \rewardstar(\prompt, \response) - \rewardtheta(\prompt, \response) \big\}
			\Big\{ \gradtheta \rewardtheta(\prompt, \response) - \Exp_{\responsenew \sim \policytheta(\cdot \mid \prompt)}\big[ \gradtheta \rewardtheta(\prompt, \responsenew) \big] \Big\} \bigg] \, .
		\end{align}
		We rewrite the expression in equation \eqref{eq:grad_scalarvalue0} in two equivalent forms by exchanging the roles of $\responseone$ and $\responsetwo$:
		\begin{subequations}
		\begin{align}
			& \gradtheta \scalarvalue(\policytheta) \notag \\ 
			\label{eq:grad_scalarvalue1}
			& \; = \; \frac{1}{\parabeta} \, \Exp_{\prompt \sim \promptdistr, \,  \responseone \sim \policytheta(\cdot \mid \prompt)}
			\bigg[ \big\{ \rewardstar(\prompt, \responseone) - \rewardtheta(\prompt, \responseone) \big\} \Big\{ \gradtheta \rewardtheta(\prompt, \responseone) - \Exp_{\responsetwo \sim \policytheta(\cdot \mid \prompt)}\big[ \gradtheta \rewardtheta(\prompt, \responsetwo) \big] \Big\} \bigg] \, ,  \\
			& \gradtheta \scalarvalue(\policytheta) \notag \\ 
			\label{eq:grad_scalarvalue2}
			& \; = \; \frac{1}{\parabeta} \, \Exp_{\prompt \sim \promptdistr, \,  \responsetwo \sim \policytheta(\cdot \mid \prompt)}
			\bigg[ \big\{ \rewardstar(\prompt, \responsetwo) - \rewardtheta(\prompt, \responsetwo) \big\} \Big\{ \gradtheta \rewardtheta(\prompt, \responsetwo) - \Exp_{\responseone \sim \policytheta(\cdot \mid \prompt)}\big[ \gradtheta \rewardtheta(\prompt, \responseone) \big] \Big\} \bigg] \, .
		\end{align}
		\end{subequations}
		By taking the average of the two equivalent formulations above, we obtain equality \eqref{eq:grad_scalarvalue} and complete the proof of \Cref{lemma:grad_scalarvalue}.  \\
		
		We now proceed to prove \Cref{lemma:grad_scalarvalue_full}, tackling equalities \eqref{eq:grad_return} and \eqref{eq:grad_KL} one by one.
		
		\paragraph{Proof of Equality~\eqref{eq:grad_return} from \Cref{lemma:grad_scalarvalue_full}:}
		We begin by expressing the expected return as
		\begin{align*}
			\Exp_{\prompt \sim \promptdistr, \, \response \sim \policytheta(\cdot \mid \prompt)} \big[ \rewardstar(\context, \response) \big]
			& \; = \; \Exp_{\prompt \sim \promptdistr} \bigg[ \int_{\ResponseSp} \rewardstar(\prompt, \response) \, \policytheta(\diff \response \mid \prompt) \bigg] \, .
		\end{align*}
		Taking the gradient of both sides with respect to $\paratheta$, we have
		\begin{align}
			\label{eq:grad_return0}
			\gradtheta \Exp_{\prompt \sim \promptdistr, \, \response \sim \policytheta(\cdot \mid \prompt)} \big[ \rewardstar(\context, \response) \big]
			& \; = \; \Exp_{\prompt \sim \promptdistr} \bigg[ \int_{\ResponseSp} \rewardstar(\prompt, \response) \, \gradtheta \policytheta(\diff \response \mid \prompt) \bigg] \, .
		\end{align}
		Using the expression for the policy gradient $\gradtheta \policytheta$ provided in \Cref{lemma:grad_policy}, the right-hand side of \eqref{eq:grad_return0} simplifies to
		\begin{align*}
			\mbox{RHS of \eqref{eq:grad_return0}}
			& \; = \; \Exp_{\prompt \sim \promptdistr} \bigg[ \int_{\ResponseSp} \rewardstar(\prompt, \response) \, \policytheta(\diff \response \mid \prompt) \cdot \frac{1}{\parabeta} \,
			\Big\{ \gradtheta \rewardtheta(\prompt, \response) - \Exp_{\responsenew \sim \policytheta(\cdot \mid \prompt)}\big[ \gradtheta \rewardtheta(\prompt, \responsenew) \big] \Big\} \bigg]   \\
			& \; = \; \frac{1}{\parabeta} \,\Exp_{\prompt \sim \promptdistr, \, \response \sim \policytheta(\cdot \mid \prompt)} \bigg[ \rewardstar(\prompt, \response) 
			\Big\{ \gradtheta \rewardtheta(\prompt, \response) - \Exp_{\responsenew \sim \policytheta(\cdot \mid \prompt)}\big[ \gradtheta \rewardtheta(\prompt, \responsenew) \big] \Big\} \bigg] \, .
		\end{align*}
		This completes the verification of equation~\eqref{eq:grad_return} from \Cref{lemma:grad_scalarvalue}.
		
		
		
		\paragraph{Proof of Equality~\eqref{eq:grad_KL} from \Cref{lemma:grad_scalarvalue_full}:}
		
		Recall the definition of the KL divergence
		\begin{align*}
			\kull{\policytheta}{\policyref}
			\; = \; \Exp_{\prompt \sim \promptdistr} 
			\bigg[ \int_{\ResponseSp} \policytheta(\diff \response \mid \prompt)
			\log \bigg( \frac{\policytheta(\response \mid \prompt)}{\policyref(\response \mid \prompt)} \bigg) \bigg] \, .
		\end{align*}
		Applying the chain rule, we obtain
		\begin{align}
			\gradtheta \kull{\policytheta}{\policyref}
			& \, = \, \Exp_{\prompt \sim \promptdistr}  \bigg[ \int_{\ResponseSp} \gradtheta \policytheta(\diff \response \mid \prompt) \,
			\log \bigg( \frac{\policytheta(\response \mid \prompt)}{\policyref(\response \mid \prompt)}\bigg) \bigg]  
			\label{eq:grad_KL2}
			+ \Exp_{\prompt \sim \promptdistr}  \bigg[ \int_{\ResponseSp} 
			\gradtheta \policytheta(\diff \response \mid \prompt) \bigg] \, .
		\end{align}
		
		Since the policy integrates to $1$, i.e., $\int_{\ResponseSp} 
		\policytheta(\diff \response \mid \prompt) = 1$, it always holds that
		\begin{align}
			\label{eq:int_grad_policy}
			\int_{\ResponseSp} 
			\gradtheta \policytheta(\diff \response \mid \prompt)
			\; = \; \gradtheta \int_{\ResponseSp} 
			\policytheta(\diff \response \mid \prompt)
			\; = \; 0 \, ,
		\end{align}
		i.e., the second term on the right-hand side of \eqref{eq:grad_KL2} is zero.
		Using the expression \eqref{eq:policyfromreward}, we take the logarithm
		\begin{align}
			\label{eq:grad_KL0}
			\log \bigg( \frac{\policytheta(\response \mid \prompt)}{\policyref(\response \mid \prompt)} \bigg)
			\; = \; \frac{1}{\parabeta} \, \rewardtheta(\prompt, \response) - \log \Partitiontheta (\prompt) \, .
		\end{align}
		Combining equations~\eqref{eq:int_grad_policy} and \eqref{eq:grad_KL0}, we get
		\begin{align}
			& \int_{\ResponseSp} \gradtheta \policytheta(\diff \response \mid \prompt) \,
			\log \bigg( \frac{\policytheta(\response \mid \prompt)}{\policyref(\response \mid \prompt)}\bigg)  \notag  \\
			& \; = \; \frac{1}{\parabeta} \int_{\ResponseSp} \rewardtheta(\prompt, \response) \, \gradtheta \policytheta(\diff \response \mid \prompt) \; - \; \log \Partitiontheta(\prompt) \int_{\ResponseSp} \gradtheta \policytheta(\diff \response \mid \prompt)  \notag  \\
			\label{eq:grad_KL1}
			& \; = \; \frac{1}{\parabeta} \int_{\ResponseSp} \rewardtheta(\prompt, \response) \, \gradtheta \policytheta(\diff \response \mid \prompt) \, .
		\end{align}
		
		Now, similar to the proof of equation \eqref{eq:grad_return}, we derive
		\begin{align*}
			\mbox{RHS of \eqref{eq:grad_KL2}}
			& \; = \; \frac{1}{\parabeta} \, \Exp_{\prompt \sim \promptdistr} \bigg[ \int_{\ResponseSp} \rewardtheta(\prompt, \response) \, \gradtheta \policytheta(\diff \response \mid \prompt) \bigg]  \\
			& \; = \; \frac{1}{\parabeta^2} \,\Exp_{\prompt \sim \promptdistr, \, \response \sim \policytheta(\cdot \mid \prompt)} \bigg[ \rewardtheta(\prompt, \response) 
			\Big\{ \gradtheta \rewardtheta(\prompt, \response) - \Exp_{\responsenew \sim \policytheta(\cdot \mid \prompt)}\big[ \gradtheta \rewardtheta(\prompt, \responsenew) \big] \Big\} \bigg] \, ,
		\end{align*}
		which verifies equality~\eqref{eq:grad_KL} from \Cref{lemma:grad_scalarvalue_full}.
		

		
		
		
		
%%%%%%%%%%%%%%%%%%%%%%%%%%%%%%%%%%%%%%%%%%%%%%%%%%%%%%%%%%%%%%%%%%%%%%%%%%%%%%%%%%%%%%%%%%%%



	\subsubsection{Proof of Lemma~\ref{lemma:grad_loss} \yaqidone}
	\label{sec:proof:lemma:grad_loss}
	
	In this section, we prove a full version of \Cref{lemma:grad_loss} as stated in \Cref{lemma:grad_loss_full} below. Equation~\eqref{eq:gradLoss_BT_0} from \Cref{lemma:grad_loss} follows directly as a straightforward corollary.
	
	In \Cref{lemma:grad_loss_full}, we consider a general class of distributions parameterized by $\paratheta$ that models the binary preference \mbox{$\Probtheta(\responseone \succ \responsetwo \mid \prompt)$}. The negative log-likelihood function is defined as
	\begin{align*}
		\Loss(\theta) = - \Exp_{\prompt \sim \promptdistr; \; (\responseone, \responsetwo) \sim \responsedistr(\cdot \mid \prompt)} \Big[ \weight(\prompt) \cdot \log \Probtheta( \responsewin \succ \responselose \bigm| \prompt) \Big] \, .
	\end{align*}
	The Bradley-Terry (BT) model described in equation~\eqref{eq:BT} and the corresponding loss function~$\Loss(\paratheta)$ in equation~\eqref{eq:Loss0} represent a special case of this general framework.
	
	\begin{lemma}[Gradient of the loss function $\Loss(\paratheta)$, full version]
		\label{lemma:grad_loss_full}
		\begin{subequations}
			For a general distribution class $\{ \Probtheta \}$, the gradient of $\Loss(\paratheta)$ with respect to $\paratheta$ is given by
			\begin{multline}
				\label{eq:gradLoss_general}
				\gradtheta \Loss(\paratheta) \; = \; - \, \Exp_{ \prompt \sim \promptdistr; \; (\responseone, \responsetwo) \sim \responsedistravg(\cdot \mid \prompt) }
				\bigg[ \, \weight(\prompt) \cdot \Big\{ \Prob\big( \responseone \succ \responsetwo \bigm| \prompt \big) - \Probtheta \big( \responseone \succ \responsetwo \bigm| \prompt \big) \Big\} \\
				\cdot \frac{\gradtheta \Probtheta( \responseone \succ \responsetwo \mid \prompt )}{\Probtheta( \responseone \succ \responsetwo \mid \prompt ) \, \Probtheta( \responsetwo \succ \responseone \mid \prompt )} \, \bigg] \, ,
			\end{multline}
			where $\responsedistravg$ is the average distribution defined in equation~\eqref{eq:def_responsedistravg_0}.
			Specifically, for the Bradley-Terry (BT) model where
			\begin{align*}
				\Probtheta \big( \responseone \succ \responsetwo \bigm| \prompt \big)
				\; = \; \sigmoid \big( \rewardtheta(\prompt, \responseone) - \rewardtheta(\prompt, \responsetwo) \big)
				\; = \; \bigg\{ 1 + \bigg( \frac{(\policytheta/\policyref)(\responsetwo \mid \prompt)}{(\policytheta/\policyref)(\responseone \mid \prompt)} \bigg)^{\parabeta} \bigg\}^{-1} \, ,
			\end{align*}
			the gradient of $\Loss(\paratheta)$ becomes
			\begin{multline}
				\label{eq:gradLoss_BT}
				\gradtheta \Loss(\paratheta) \; = \; - \, \Exp_{\prompt \sim \promptdistr; \; (\responseone, \, \responsetwo) \sim \responsedistravg(\cdot \mid \prompt)}
				\bigg[ \, \weight(\prompt) \cdot \Big\{ \sigmoid \big( \rewardstar(\context, \responseone) - \rewardstar(\context, \responsetwo) \big) - \sigmoid \big( \rewardtheta(\context, \responseone) - \rewardtheta(\context, \responsetwo) \big) \Big\} \\ 
				\cdot \big\{ \gradtheta \rewardtheta(\prompt, \responseone) - \gradtheta \rewardtheta(\prompt, \responsetwo) \big\} \bigg] \, .
			\end{multline}
		\end{subequations}
	\end{lemma}
	
	
	For notational simplicity, we focus on the proof for the case where the weight function $\weight(\prompt) = 1$. The results for a general weight function $\weight(\prompt) > 0$ can be derived in a similar manner.
	
	Recall that the negative log-likelihood function $\Loss(\paratheta)$ is defined as
	\begin{align*}
		\Loss(\paratheta) & \; = \;
		\Exp \Big[ - \log \Probtheta\big( \responsewin \succ \responselose \bigm| \prompt \big) \Big] \, .
	\end{align*}
	Based on the data generation mechanism, we can expand the expectation in $\Loss(\paratheta)$ as
%	\begin{subequations}
	\begin{align}
		\Loss(\paratheta)
		& \; = \; \Exp_{\prompt \sim \promptdistr; \; (\responseone, \, \responsetwo) \sim \responsedistr(\cdot \mid \prompt)}
		\Big[ \, \Prob\big( \responseone \succ \responsetwo \bigm| \prompt \big) \cdot \big\{ - \log \Probtheta \big( \responseone \succ \responsetwo \bigm| \prompt \big) \big\}  \notag  \\
		\label{eq:Loss0}
		& \qquad \qquad \qquad \qquad \qquad + \Prob\big( \responsetwo \succ \responseone \bigm| \prompt \big) \cdot \big\{ - \log \Probtheta\big( \responsetwo \succ \responseone \bigm| \prompt \big) \big\} \Big] \, .
	\end{align}
	Notice that we can exchange the roles of $\responseone$ and $\responsetwo$ in the expectation above. This means that we can equivalently express the expectation using the pair $(\responsetwo, \responseone) \sim \responsedistr(\cdot \mid \prompt)$.
	This symmetry allows us to replace $\responsedistr$ in equation~\eqref{eq:Loss0} with the average distribution $\responsedistravg$ as defined in equation~\eqref{eq:def_responsedistravg_0}. \\
	
	Next, we take the gradient of the loss function $\Loss(\paratheta)$ with respect to the parameter $\paratheta$ and obtain
	\begin{align*}
		\gradtheta \Loss(\paratheta)
		& \; = \; \Exp_{\prompt \sim \promptdistr, \, (\responseone, \, \responsetwo) \sim \responsedistravg(\cdot \mid \prompt)}
		\bigg[ \, \frac{\Prob( \responseone \succ \responsetwo \mid \prompt )}{\Probtheta ( \responseone \succ \responsetwo \mid \prompt )} \cdot \big\{ - \gradtheta \Probtheta( \responseone \succ \responsetwo \mid \prompt ) \big\}   \\
		& \qquad \qquad \qquad \qquad \qquad + \frac{\Prob( \responsetwo \succ \responseone \mid \prompt )}{\Probtheta( \responsetwo \succ \responseone \mid \prompt )} \cdot \big\{ - \gradtheta \Probtheta( \responsetwo \succ \responseone \mid \prompt ) \big\} \, \bigg] \, .
	\end{align*}
	Note that $\Prob\big( \responsetwo \succ \responseone \bigm| \prompt\big) = 1 - \Prob\big( \responseone \succ \responsetwo \bigm| \prompt\big)$ and $\Probtheta \big( \responsetwo \succ \responseone \bigm| \prompt\big) = 1 - \Probtheta \big( \responseone \succ \responsetwo \bigm| \prompt\big)$.
	Using this, we can rewrite the gradient as
	\begin{align*}
		& \gradtheta \Loss(\paratheta)  \\
		& \; = \;
		\Exp_{\prompt \sim \promptdistr; \; (\responseone, \, \responsetwo) \sim \responsedistravg(\cdot \mid \prompt)}
		\bigg[ \bigg\{ \frac{1 - \Prob( \responseone \succ \responsetwo \mid \prompt )}{1 - \Probtheta( \responseone \succ \responsetwo \mid \prompt )} - \frac{\Prob( \responseone \succ \responsetwo \mid \prompt )}{\Probtheta ( \responseone \succ \responsetwo \mid \prompt )} \bigg\} \cdot \gradtheta \Probtheta\big( \responseone \succ \responsetwo \bigm| \prompt \big) \bigg] \, .
	\end{align*}
	We simplify the expression further to obtain
	\begin{multline*}
		\gradtheta \Loss(\paratheta)
		\; = \;
		\Exp_{\prompt \sim \promptdistr; \; (\responseone, \, \responsetwo) \sim \responsedistravg(\cdot \mid \prompt)}
		\bigg[ \Big\{ \Probtheta \big( \responseone \succ \responsetwo \bigm| \prompt \big) - \Prob \big( \responseone \succ \responsetwo \bigm| \prompt \big) \Big\} \\ \cdot \frac{\gradtheta \Probtheta( \responseone \succ \responsetwo \mid \prompt )}{\Probtheta( \responseone \succ \responsetwo \mid \prompt ) \, \Probtheta( \responsetwo \succ \responseone \mid \prompt )} \bigg] \, .
	\end{multline*}
	This establishes equation~\eqref{eq:gradLoss_general} from \Cref{lemma:grad_loss}. \\
	
	As for the Bradley-Terry (BT) model, we use the equality
	\begin{align*}
		\divsigmoid(z) \; = \; \frac{1}{(1 + \exp(-z))(1 + \exp(z))} \; = \; \sigmoid(z) \, \sigmoid(-z)
		\qquad \mbox{for any $z \in \Real$}
	\end{align*}
	to derive the following expression
	\begin{align}
		\label{eq:grad_reward}
		\frac{\gradtheta \Probtheta( \responseone \succ \responsetwo \mid \prompt )}{\Probtheta( \responseone \succ \responsetwo \mid \prompt ) \, \Probtheta( \responsetwo \succ \responseone \mid \prompt )}
		\; = \; \gradtheta \rewardtheta(\prompt, \responseone) - \gradtheta \rewardtheta(\prompt, \responsetwo) \, .
	\end{align}
	By substituting this gradient expression from equation~\eqref{eq:grad_reward} into equation~\eqref{eq:gradLoss_general}, we directly obtain equation~\eqref{eq:gradLoss_BT}, thereby completing the proof of \Cref{lemma:grad_loss}.
	
	
%%%%%%%%%%%%%%%%%%%%%%%%%%%%%%%%%%%%%%%%%%%%%%%%%%%%%%%%%%%%%%%%%%%%%%%%%%%%%%%

	\subsection{Proof of Auxiliary Results for Theorem~\ref{thm:asymp} \yaqidone}
	\label{sec:proof:thm:asymp_aux}
	
	In this section, we present the detailed proofs of the supporting lemmas used in the proof of \Cref{thm:asymp}. 
	We begin in \Cref{sec:proof:eq:master_cond_proof} by establishing condition~\eqref{eq:master_cond_proof}, which is crucial for the valid application of the master theorem for $Z$-estimators. Following this, in \Cref{sec:proof:lemma:hess_loss}, we compute the Hessian matrix $\hesstheta \Loss(\parathetastar)$ explicitly. Finally, in \Cref{sec:proof:lemma:grad_loss_stat}, we derive the asymptotic distribution of the gradient~$\gradtheta \Losshat(\parathetastar)$.
	
	
%%%%%%%%%%%%%%%%%%%%%%%%%%%%%%%%%%%%%%%%%%%%%%%%%%%%%%%%%%%%%%%

	\subsubsection{Proof of Condition~\eqref{eq:master_cond_proof}}
	\label{sec:proof:eq:master_cond_proof}
	We begin by rewriting the left-hand side of equation~\eqref{eq:master_cond_proof} as follows:
	\begin{align}
		\Delta
		& \; \defn \; \sqrt{n} \, \big\{ \gradtheta \Losshat (\parathetahat) - \gradtheta \Loss(\parathetahat) \big\} - \sqrt{n} \, \big\{ \gradtheta \Losshat (\parathetastar) - \gradtheta \Loss (\parathetastar) \big\}   \notag  \\
		& \; = \; \sqrt{n} \, \big\{ \gradtheta \Losshat (\parathetahat) - \gradtheta \Losshat(\parathetastar) \big\} - \sqrt{n} \, \big\{ \gradtheta \Loss (\parathetahat) - \gradtheta \Loss (\parathetastar) \big\} \, .
		\label{eq:master_0}
	\end{align}
	We then leverage the smoothness properties of the function $\rewardtheta$, which guarantee the following approximations:
	\begin{subequations}
		\begin{align}
			\label{eq:gradLosshat_smooth}
			\gradtheta \Losshat(\parathetahat) - \gradtheta \Losshat(\parathetastar) & \; = \; \hesstheta \Losshat(\parathetastar) \, (\parathetahat - \parathetastar) + \smallop \big( \norm{\parathetahat - \parathetastar}_2 \big) \, ,  \\
			\label{eq:gradLoss_smooth}
			\gradtheta \Loss(\parathetahat) - \gradtheta \Loss(\parathetastar) & \; = \; \hesstheta \Loss(\parathetastar) \, (\parathetahat - \parathetastar) + \smallop \big( \norm{\parathetahat - \parathetastar}_2 \big) \, .
		\end{align}
	\end{subequations}
	Assuming these equalities~\eqref{eq:gradLosshat_smooth} and~\eqref{eq:gradLoss_smooth} hold, we substitute them into equation~\eqref{eq:master_0}, leading to
	\begin{align}
		\Delta
		& \; = \; \sqrt{n} \, \big\{ \hesstheta \Losshat (\parathetastar) \, (\parathetahat - \parathetastar) + \smallop( \norm{\parathetahat - \parathetastar}_2 ) \big\} - \sqrt{n} \, \big\{ \hesstheta \Loss (\parathetastar) \, (\parathetahat - \parathetastar) + \smallop( \norm{\parathetahat - \parathetastar}_2 ) \big\}  \notag \\
		& \; = \; \sqrt{n} \, \big\{ \hesstheta \Losshat (\parathetastar) - \hesstheta \Loss (\parathetastar) \big\} (\parathetahat - \parathetastar) + \smallop \big( 1 + \sqrt{n} \, \norm{ \parathetahat - \parathetastar }_2 \big) \, .
		\label{eq:master_1}
	\end{align}
	Using the law of large numbers, we know that $\hesstheta \Losshat (\parathetastar) \convergep \hesstheta \Loss (\parathetastar)$, which implies
	\begin{align*}
		\sqrt{n} \, \big\{ \hesstheta \Losshat (\parathetastar) - \hesstheta \Loss (\parathetastar) \big\} (\parathetahat - \parathetastar) \; = \; \smallop \big( \sqrt{n} \, \norm{ \parathetahat - \parathetastar }_2 \big) \, .
	\end{align*}
	Therefore, we conclude that
	\begin{align*}
		\Delta \; = \; \smallop \big( 1 + \sqrt{n} \, \norm{ \parathetahat - \parathetastar }_2 \big)
	\end{align*}
	as claimed in equation~\eqref{eq:master_cond_proof}.
	
	The only remaining task is to establish the validity of equalities~\eqref{eq:gradLosshat_smooth} and~\eqref{eq:gradLoss_smooth}.
	
	
	\paragraph{Proof of Equalities~\eqref{eq:gradLosshat_smooth}~and~\eqref{eq:gradLoss_smooth}:}
	
	We express the loss function $\Losshat(\paratheta)$ in the form
	\begin{align*}
		\Losshat(\paratheta) \; \defn \;
		\frac{1}{\numobs} \sum_{i=1}^{\numobs} \weight(\prompti{i}) \cdot \lliketheta\big(\prompti{i}, \responsewini{i}, \responselosei{i}\big) \, ,
	\end{align*}
	where the function $\lliketheta$ is defined as
	\begin{align*}
		\lliketheta(\prompt, \responsei{1}, \responsei{2})
		\; = \; - \log \sigmoid \big( \rewardtheta(\prompt, \responsei{1}) - \rewardtheta(\prompt, \responsei{2}) \big) \, .
	\end{align*}
	We then calculate the gradient $\gradtheta \lliketheta$ and $\hesstheta \lliketheta$ as follows:
	\begin{align*}
		\gradtheta \lliketheta(\prompt, \responsei{1}, \responsei{2})
		& \; = \; \sigmoid\big( \rewardtheta(\prompt, \responsei{2}) - \rewardtheta(\prompt, \responsei{1}) \big) \cdot \big\{ \gradtheta \rewardtheta(\prompt, \responsei{2}) - \gradtheta \rewardtheta(\prompt, \responsei{1})  \big\} \qquad \mbox{and}  \\
		\hesstheta \lliketheta(\prompt, \responsei{1}, \responsei{2})
		& \; = \; \divsigmoid\big( \rewardtheta(\prompt, \responsei{2}) - \rewardtheta(\prompt, \responsei{1}) \big) \\
        & \qquad \quad
        \cdot \big\{ \gradtheta \rewardtheta(\prompt, \responsei{2}) - \gradtheta \rewardtheta(\prompt, \responsei{1}) \big\} \big\{ \gradtheta \rewardtheta(\prompt, \responsei{2}) - \gradtheta \rewardtheta(\prompt, \responsei{1}) \big\}^{\top}  \\
		& \quad + \sigmoid\big( \rewardtheta(\prompt, \responsei{2}) - \rewardtheta(\prompt, \responsei{1}) \big) \cdot \big\{ \hesstheta \rewardtheta(\prompt, \responsei{2}) - \hesstheta \rewardtheta(\prompt, \responsei{1})  \big\} \, .
	\end{align*}
	When the reward function $\rewardtheta(\prompt, \response)$, along with its gradient $\gradtheta \rewardtheta(\prompt, \response)$ and Hessian $\hesstheta \rewardtheta(\prompt, \response)$, is uniformly bounded and Lipschitz continuous with respect to $\paratheta$ for all $(\prompt, \response) \in \PromptSp \times \ResponseSp$, it guarantees that the Hessian of the loss function, $\hesstheta \lliketheta$, is also Lipschitz continuous. This holds with some constant $\Liphess > 0$ across all $(\prompt, \response) \in \PromptSp \times \ResponseSp$, as demonstrated below:
	\begin{align*}
		\norm[\big]{\hesstheta \lliketheta (\prompt, \responsei{1}, \responsei{2}) - \hesstheta \llikethetastar (\prompt, \responsei{1}, \responsei{2})}_2
		\; \leq \; \Liphess \cdot \norm{\paratheta - \parathetastar}_2 \, .
	\end{align*}
	From this Lipschitz property, we deduce
	\begin{align*}
		\norm[\big]{\gradtheta \lliketheta (\prompt, \responsei{1}, \responsei{2}) - \gradtheta \llikethetastar (\prompt, \responsei{1}, \responsei{2}) - \hesstheta \llikethetastar (\prompt, \responsei{1}, \responsei{2}) \, (\paratheta - \parathetastar)}_2 \; \leq \; \frac{\Liphess}{2} \cdot \norm{\paratheta - \parathetastar}_2^2
	\end{align*}
	and further derive
	\begin{align*}
		\norm[\big]{\gradtheta \Losshat(\paratheta) - \gradtheta \Losshat(\parathetastar) - \hesstheta \Losshat(\parathetastar) \, (\paratheta - \parathetastar)}_2 & \; \leq \; \frac{\Liphess \, \supnorm{\weight}}{2} \cdot \norm{\paratheta - \parathetastar}_2^2 \, ,  \\
		\norm[\big]{\gradtheta \Loss(\paratheta) - \gradtheta \Loss(\parathetastar) - \hesstheta \Loss(\parathetastar) \, (\paratheta - \parathetastar)}_2 & \; \leq \; \frac{\Liphess \, \supnorm{\weight}}{2} \cdot \norm{\paratheta - \parathetastar}_2^2 \, .
	\end{align*}
	Finally, under the condition that $\parathetahat \convergep \parathetastar$, these results simplify to the expressions given in equations~\eqref{eq:gradLosshat_smooth} and~\eqref{eq:gradLoss_smooth}, as previously claimed.
	
	
%%%%%%%%%%%%%%%%%%%%%%%%%%%%%%%%%%%%%%%%%%%%%%%%%%%%%%%%%%%%%%%
	
	\subsubsection{Proof of Lemma~\ref{lemma:hess_loss}, Explicit Form of Hessian $\hesstheta \Loss(\parathetastar)$}
	\label{sec:proof:lemma:hess_loss}
	
	From equation~\eqref{eq:gradLoss_BT_0} in \Cref{lemma:grad_loss}, we recall the explicit formula for the gradient $\gradtheta \Loss(\paratheta)$. Taking the derivative of both sides of equation~\eqref{eq:gradLoss_BT_0}, we obtain
	\begin{align}
		& \begin{aligned} 
		\hesstheta \Loss(\paratheta) \; = \; \Exp_{\prompt \sim \promptdistr; \; (\responseone, \, \responsetwo) \sim \responsedistravg(\cdot \mid \prompt)}
		& \Big[ \, \weight(\prompt) \cdot \divsigmoid \big( \rewardtheta(\context, \responseone) - \rewardtheta(\context, \responsetwo) \big) \\ 
		& \cdot \big\{ \gradtheta \rewardtheta(\prompt, \responseone) - \gradtheta \rewardtheta(\prompt, \responsetwo) \big\} \big\{ \gradtheta \rewardtheta(\prompt, \responseone) - \gradtheta \rewardtheta(\prompt, \responsetwo) \big\}^{\top} \Big] \end{aligned}   \notag  \\
		& \qquad \qquad \quad
		\begin{aligned} 
		- \, \Exp_{\prompt \sim \promptdistr; \; (\responseone, \, \responsetwo) \sim \responsedistravg(\cdot \mid \prompt)}
		\bigg[ \, \weight(\prompt) & \cdot \Big\{ \sigmoid \big( \rewardstar(\context, \responseone) - \rewardstar(\context, \responsetwo) \big) - \sigmoid \big( \rewardtheta(\context, \responseone) - \rewardtheta(\context, \responsetwo) \big) \Big\} \\ 
		& \cdot \big\{ \hesstheta \rewardtheta(\prompt, \responseone) - \hesstheta \rewardtheta(\prompt, \responsetwo) \big\} \bigg] \, .
		\end{aligned}
		\label{eq:hessLoss_0}
	\end{align}
	When we set $\paratheta = \parathetastar$, it follows that $\rewardtheta = \rewardstar$. This simplification eliminates the second term in expression~\eqref{eq:hessLoss_0}, reducing the Hessian matrix to
	\begin{multline*}
		\hesstheta \Loss(\parathetastar) \; = \; \Exp_{\prompt \sim \promptdistr; \; (\responseone, \, \responsetwo) \sim \responsedistravg(\cdot \mid \prompt)}
		\Big[ \, \weight(\prompt) \cdot \divsigmoid \big( \rewardstar(\context, \responseone) - \rewardstar(\context, \responsetwo) \big) \\ 
		\cdot \big\{ \gradtheta \rewardstar(\prompt, \responseone) - \gradtheta \rewardstar(\prompt, \responsetwo) \big\} \big\{ \gradtheta \rewardstar(\prompt, \responseone) - \gradtheta \rewardstar(\prompt, \responsetwo) \big\}^{\top} \Big] \, .
	\end{multline*}
	Substituting the derivative $\divsigmoid$ with its explicit form, $\divsigmoid(z) = \sigmoid(z) \, \sigmoid(-z)$ for any $z \in \Real$, we refine the expression to
	\begin{align*}
		\hesstheta \Loss(\parathetastar) \; = \; \CovOpstar \, ,
	\end{align*}
	where the covariance matrix $\CovOpstar$ is defined in equation~\eqref{eq:def_CovOpstar}.
	This completes the proof of expression~\eqref{eq:hess_loss} from \Cref{lemma:hess_loss}.
	
%%%%%%%%%%%%%%%%%%%%%%%%%%%%%%%%%%%%%%%%%%%%%%%%%%%%%%%%%%%%%%%%%%%%%%%%%%%
	
	\subsubsection{Proof of Lemma~\ref{lemma:grad_loss_stat}, Asymptotic Distribution of Graident $\gradtheta \Losshat(\parathetastar)$}
	\label{sec:proof:lemma:grad_loss_stat}
	
	In this section, we analyze the asymptotic distribution of the gradient $\gradtheta \Losshat(\paratheta)$ at $\paratheta = \parathetastar$, where the loss function $\Losshat(\paratheta)$ is defined as
	\begin{align*}
		\Losshat(\paratheta) \; = \;
		- \frac{1}{\numobs} \sum_{i=1}^{\numobs} \, \weight(\prompt) \cdot \log \sigmoid \Big( \rewardtheta\big(\prompti{i}, \responsewini{i}\big) - \rewardtheta\big(\prompti{i}, \responselosei{i}\big) \Big) \, .
	\end{align*}
	Using the definition of the sigmoid function $\sigmoid$, we calculate that
	\begin{align*}
		( \log \sigmoid(z) )' = \divsigmoid(z) / \sigmoid(z) = \sigmoid(z) \, \sigmoid(-z) / \sigmoid(z) = \sigmoid(-z) \qquad \mbox{for any real number $z \in \Real$}.
	\end{align*}
	This allows us to reformulate $\gradtheta \Losshat(\paratheta)$ as the average of $\numobs$ i.i.d. vectors $\{ \vecgi{i} \}_{i=1}^{\numobs}$:
	\begin{align}
		\label{eq:gradLosshat}
		\gradtheta \Losshat(\paratheta)
		\; = \; \frac{1}{\numobs} \sum_{i=1}^{\numobs} \, \vecgi{i} \, .
	\end{align}
	Here each vector $\vecgi{i} \in \Real^{\Dim}$ is defined as
	\begin{align*}
		\vecgi{i} \; \defn \; \weight(\prompt) \cdot \sigmoid \big( \rewardtheta(\prompti{i}, \responselosei{i}) - \rewardtheta(\prompti{i}, \responsewini{i}) \big) \cdot \big\{ \gradtheta \rewardtheta(\prompti{i}, \responselosei{i}) - \gradtheta \rewardtheta(\prompti{i}, \responsewini{i}) \big\} \, .
	\end{align*}
%	(consistently with expression~\eqref{eq:def_grad}).
	At $\paratheta = \parathetastar$, we denote $\vecgi{i}$ as $\vecgstari{i}$ and $\gradi{i}$ as $\gradstari{i}$. Notably, vector $\vecgi{i}$ can be rewritten as
	\begin{align}
		\label{eq:vecgi2}
		\vecgi{i} 
		& \; = \; \weight(\prompt) \cdot \big\{ \sigmoid \big( \rewardtheta(\prompti{i}, \responseonei{i}) - \rewardtheta(\prompti{i}, \responsetwoi{i}) \big) - \indicator\{ \responseonei{i} = \responsewini{i}, \responsetwoi{i} = \responselosei{i} \} \big\}
		\cdot \gradi{i} \, ,
	\end{align}
	where $\gradi{i}$ is given by
	\begin{align*}
		\gradi{i} \defn \gradtheta \rewardtheta(\prompti{i}, \responseonei{i}) - \gradtheta \rewardtheta(\prompti{i}, \responsetwoi{i}) \, .
	\end{align*}
	From the structure of the BT model, it holds that
	\begin{align*}
		\Exp\big[ \indicator \{ \responseonei{i} = \responsewini{i}, \responsetwoi{i} = \responselosei{i} \} \bigm| \prompti{i} \big] \; = \; \sigmoid \big( \rewardstar(\prompti{i}, \responseonei{i}) - \rewardstar(\prompti{i}, \responsetwoi{i}) \big) \, ,
	\end{align*}
	which implies $\Exp[\vecgstari{i}] = \veczero$.
	
	
	To analyze the asymptotic distribution of $\gradtheta \Losshat(\parathetastar)$, we apply the central limit theorem (CLT) to its empirical form given in equation~\eqref{eq:gradLosshat}. 
%	\yaqiadd{Check the conditions for CLT.}
	By the CLT, we have
	\begin{align}
		\label{eq:gradLoss_CLT}
		\sqrt{\numobs} \, \big( \gradtheta \Losshat(\parathetastar) - \gradtheta \Loss(\parathetastar) \big)
		\; \stackrel{d}{\longrightarrow} \; \Gauss\big(\veczero, \CovOptil \big) \, ,
		\qquad \numobs \rightarrow \infty \, ,
	\end{align}
	where the covariance matrix $\CovOptil \in \Real^{\Dim \times \Dim}$ is given by
	\begin{align*}
		\CovOptil \; \defn \; \Cov(\vecgstari{1}) \; = \; \Exp\big[ \vecgstari{1} (\vecgstari{1})^{\top} \big] \, .
	\end{align*}
	Here we have used the property $\Exp[\vecgstari{i}] = \veczero$ in the second equality.
	
	We now compute the explicit form of the covariance matrix $\CovOptil$. Using the definition of $\vecgi{i}$ from expression~\eqref{eq:vecgi2}, we find that
	\begin{align}
		& \CovOptil \; = \; \Exp\big[ \vecgstari{1} (\vecgstari{1})^{\top} \big] \notag  \\
		& = \; \Exp_{\, \begin{subarray}{l} ~ \\ \prompt \sim \promptdistr; \\ (\responseone, \responsetwo) \sim \responsedistravg(\cdot \mid \prompt)\end{subarray}} \Big[ \, \weight^2(\prompt) \cdot \big\{ \sigmoid \big( \rewardstar(\prompti{1}, \responseonei{1}) - \rewardstar(\prompti{1}, \responsetwoi{1}) \big) - \indicator\{ \responseonei{1} = \responsewini{1}, \responsetwoi{1} = \responselosei{1} \} \big\}^2 \cdot \gradstari{1} (\gradstari{1})^{\top} \Big] \, .
		\label{eq:CovOptil_2}
	\end{align}
	Taking the conditional expectation over the outcomes of winners and losers, and using the relation
	\begin{align*}
		&  \Exp\Big[
		\big\{ \sigmoid \big( \rewardstar(\prompti{1}, \responseonei{1}) - \rewardstar(\prompti{1}, \responsetwoi{1}) \big) - \indicator\{ \responseonei{1} = \responsewini{1}, \responsetwoi{1} = \responselosei{1} \} \big\}^2 \Bigm| \prompti{1}, \responseonei{1}, \responsetwoi{1} \Big]  \\
		& \; = \; \Var \Big( \indicator\{ \responseonei{1} = \responsewini{1}, \responsetwoi{1} = \responselosei{1} \} \Bigm|  \prompti{1}, \responseonei{1}, \responsetwoi{1} \Big)  \\
		& \; = \; \sigmoid \big( \rewardstar(\prompti{i}, \responseonei{i}) - \rewardstar(\prompti{i}, \responsetwoi{i}) \big) \, \sigmoid \big( \rewardstar(\prompti{i}, \responsetwoi{i}) - \rewardstar(\prompti{i}, \responseonei{i}) \big) \, ,
	\end{align*}
	we reduce equation~\eqref{eq:CovOptil_2} to
	\begin{align*}
		\CovOptil
		& \; = \; \Exp_{\prompt \sim \promptdistr; \; (\responseone, \responsetwo) \sim \responsedistravg(\cdot \mid \prompt)} \Big[ \, \weight^2(\prompt) \cdot \Var \big( \indicator\{ \responseonei{1} = \responsewini{1}, \responsetwoi{1} = \responselosei{1} \} \bigm|  \prompti{1}, \responseonei{1}, \responsetwoi{1} \big) \cdot \gradstari{1} (\gradstari{1})^{\top} \Big] \, .
	\end{align*}
	Bounding the weight function $\weight(\prompt)$ by its uniform bound $\supnorm{\weight}$, we simplify further:
	\begin{align*}
		\CovOptil
        & \; \preceq \; \supnorm{\weight} \cdot \Exp\Big[ \, \weight(\prompt) \cdot \Var \big( \indicator\{ \responseonei{1} = \responsewini{1}, \responsetwoi{1} = \responselosei{1} \} \bigm|  \prompti{1}, \responseonei{1}, \responsetwoi{1} \big) \cdot \gradstari{1} (\gradstari{1})^{\top} \Big] \, .
    \end{align*}
    This ultimately reduces to
    \begin{align}
    	\label{eq:CovOptil_ub}
         \CovOptil & \; \preceq \; \supnorm{\weight} \cdot \CovOpstar
	\end{align}
	where $\CovOpstar$ is defined in equation~\eqref{eq:def_CovOpstar}.
    
    Finally, by combining equations~\eqref{eq:gradLoss_CLT} and~\eqref{eq:CovOptil_ub}, we establish the asymptotic normality of $\gradtheta \Losshat(\parathetastar)$ and complete the proof of \Cref{lemma:grad_loss_stat}.
    
    
%%%%%%%%%%%%%%%%%%%%%%%%%%%%%%%%%%%%%%%%%%%%%%%%%%%%%%%%%%%%%%
	
	\subsection{Proof of Auxiliary Results for Theorem~\ref{lemma:hess_scalarvalue} \yaqidone}
	\label{sec:proof:lemma:hess_scalarvalue_aux}
	
	This section contains the proofs of the auxiliary results supporting \Cref{lemma:hess_scalarvalue}. In \Cref{sec:proof:eq:hessscalarvalue}, we derive the explicit form of the Hessian $ \hesstheta \scalarvalue(\policystar) $. \Cref{sec:proof:gap_distr} rigorously establishes the asymptotic distribution of the value gap (equation~\eqref{eq:gap_distr}). Finally, \Cref{sec:proof:chisqtail} proves the tail bound~\eqref{eq:gap_bd} on the chi-square distribution $ \chisquare_{\Dim} $.
	
	\subsubsection{Proof of Equation~\eqref{eq:hessscalarvalue} from Theorem~\ref{lemma:hess_scalarvalue}, Explicit Form of Hessian $\hesstheta \scalarvalue(\policystar)$}
	\label{sec:proof:eq:hessscalarvalue}
	
	We begin by differentiating expression~\eqref{eq:grad_scalarvalue0} for the gradient $\gradtheta \scalarvalue(\policytheta)$ to obtain the Hessian matrix $\hesstheta \scalarvalue(\policytheta)$. The resulting expression can be written as
	\begin{align*}
		\hesstheta \scalarvalue(\policytheta)
		\; = \; \GammaMt_1 + \GammaMt_2 + \GammaMt_3 \, ,
	\end{align*}
	where the terms are defined as follows:
	\begin{align*}
		\GammaMt_1
		& \; \defn \; \frac{1}{\parabeta} \, \Exp_{\prompt \sim \promptdistr}
		\bigg[ \int_{\ResponseSp} \big\{ \rewardstar(\prompt, \response) - \rewardtheta(\prompt, \response) \big\} \\
		& \qquad \qquad \qquad \qquad \quad
        \cdot \Big\{ \gradtheta \rewardtheta(\prompt, \response) - \Exp_{\responsenew \sim \policytheta(\cdot \mid \prompt)}\big[ \gradtheta \rewardtheta(\prompt, \responsenew) \big] \Big\} \, \gradtheta \policytheta(\diff \response \mid \prompt)^{\top} \bigg] \, ,  \\
		\GammaMt_2
		& \; \defn \; - \frac{1}{\parabeta} \, \Exp_{\prompt \sim \promptdistr, \,  \response \sim \policytheta(\cdot \mid \prompt)}
		\bigg[ \Big\{ \gradtheta \rewardtheta(\prompt, \response) - \Exp_{\responsenew \sim \policytheta(\cdot \mid \prompt)}\big[ \gradtheta \rewardtheta(\prompt, \responsenew) \big] \Big\} \, \gradtheta \rewardtheta(\prompt, \response)^{\top} \bigg] \, ,  \\
		\GammaMt_3
		& \defn \frac{1}{\parabeta} \, \Exp_{\prompt \sim \promptdistr, \,  \response \sim \policytheta(\cdot \mid \prompt)}
		\bigg[ \big\{ \rewardstar(\prompt, \response) - \rewardtheta(\prompt, \response) \big\} \Big\{ \hesstheta \rewardtheta(\prompt, \response) - \gradtheta \Exp_{\responsenew \sim \policytheta(\cdot \mid \prompt)}\big[ \gradtheta \rewardtheta(\prompt, \responsenew) \big] \Big\} \bigg] \, .
	\end{align*}
	
	At the point $\paratheta = \parathetastar$, we know that $\rewardtheta = \rewardstar$. This simplifies the expression significantly:
	\begin{align*}
	\GammaMt_1 = \veczero \quad \text{and} \quad \GammaMt_3 = \veczero.
	\end{align*}
	Therefore, only term $\GammaMt_2$ contributes to the Hessian, and it further reduces to
	\begin{align*}
		\GammaMt_2
		& \; = \; - \frac{1}{\parabeta} \, \Exp_{\prompt \sim \promptdistr, \,  \response \sim \policytheta(\cdot \mid \prompt)}
		\Big[ \gradtheta \rewardtheta(\prompt, \response) \, \gradtheta \rewardtheta(\prompt, \response)^{\top} \Big]  \\
		& \quad + \frac{1}{\parabeta} \, \Exp_{\prompt \sim \promptdistr}
		\Big[ \Exp_{\responsenew \sim \policytheta(\cdot \mid \prompt)}\big[ \gradtheta \rewardtheta(\prompt, \responsenew) \big] \, \Exp_{\response \sim \policytheta(\cdot \mid \prompt)}\big[\gradtheta \rewardtheta(\prompt, \response)\big]^{\top} \Big] \\
		& \; = \; - \frac{1}{\parabeta} \, \Exp_{\prompt \sim \promptdistr}
		\Big[ \Cov_{\response \sim \policytheta(\cdot \mid \prompt)} \big[ \gradtheta \rewardtheta(\prompt, \response) \bigm| \prompt \big] \Big]  \, .
	\end{align*}
	From this simplification, we deduce
	\begin{align*}
		\hesstheta \scalarvalue(\policystar) \; = \;
		- \frac{1}{\parabeta} \, \Exp_{\prompt \sim \promptdistr} \Big[ \Cov_{\response \sim \policystar(\cdot \mid \prompt)} \big[ \gradtheta \rewardstar(\prompt, \response) \bigm| \prompt \big] \Big] \, ,
	\end{align*}
	which establishes equation~\eqref{eq:hessscalarvalue} as stated in \Cref{lemma:hess_scalarvalue}.


%%%%%%%%%%%%%%%%%%%%%%%%%%%%%%%%%%%%%%%%%%%%%%%%%%%%%%%%%%%%%%%%%%%%%%%
	
	\subsubsection{Proof of the Asymptotic Distribution in Equation~\eqref{eq:gap_distr}}
	\label{sec:proof:gap_distr}
	
	The goal of this part is to establish the asymptotic distribution of $\numobs \{ \scalarvalue(\policystar) - \scalarvalue(\policyhat) \}$, as stated in equation~\eqref{eq:gap_distr} from \Cref{sec:proof:lemma:hess_scalarvalue}. To achieve this, we first recast the value gap into the product of two terms and then invoke Slutsky’s theorem.
	
	We start by writing
	\begin{align}
		\numobs \cdot \{ \scalarvalue(\policystar) - \scalarvalue(\policyhat) \}
		\; = \;  \underbrace{\numobs \cdot (\parathetahat - \parathetastar)^{\top} \HessMt \, (\parathetahat - \parathetastar)}_{\Un}
		\cdot \underbrace{\frac{\scalarvalue(\policystar) - \scalarvalue(\policyhat)}{(\parathetahat - \parathetastar)^{\top} \HessMt \, (\parathetahat - \parathetastar)}}_{\Vn} \, .
	\end{align}
	By isolating \(\Un\) and \(\Vn\) in this way, we can handle their limiting behaviors separately:
	\begin{subequations}
	\begin{align}
		& \Un \; \converged \; \vecz^{\top} \CovOmega^{\frac{1}{2}} \HessMt \CovOmega^{\frac{1}{2}} \vecz \qquad \mbox{with $\vecz \sim \Gauss(\veczero, \IdMt)$},  \label{eq:Un_distr} \\
		& \Vn \; \convergep \; \frac{1}{2} \, .  \label{eq:Vn_distr}
	\end{align}
	\end{subequations}
	If these two results are established, the desired asymptotic distribution of the value gap, as given in equation~\eqref{eq:gap_distr}, follows directly from Slutsky’s theorem.
	
	To complete the proof, we proceed to verify equations~\eqref{eq:Un_distr} and~\eqref{eq:Vn_distr}. It is worth noting that equation~\eqref{eq:Un_distr} is a straightforward corollary of \Cref{thm:asymp}, so the main task is to establish the convergence result in equation~\eqref{eq:Vn_distr}.
	
	
	\paragraph{Proof of Equation~\eqref{eq:Vn_distr}:}
	
	Since $\CovOpstar$ is nonsingular, the matrix $\HessMt = (\Partitionthetabar / \parabeta) \cdot \CovOpstar$ is also nonsingular.
	From equation~\eqref{eq:Taylor_scalarvalue}, we know that for any $\varepsilon \in (0, 1)$, there exists a threshold $\eta(\varepsilon) > 0$ such that whenever $\norm{\paratheta - \parathetastar}_2 \leq \eta(\varepsilon)$, the following inequality holds:
	\begin{align*}
		\Big( \frac{1}{2} - \varepsilon \Big) \, (\paratheta - \parathetastar)^{\top} \HessMt \, (\paratheta - \parathetastar)
		\; \leq \; \scalarvalue(\policystar) - \scalarvalue(\policytheta)
		\; \leq \; \Big( \frac{1}{2} + \varepsilon \Big) \, (\paratheta - \parathetastar)^{\top} \HessMt \, (\paratheta - \parathetastar) \, .
	\end{align*}
	This can be reformulated as
	\begin{align*}
		\abs[\Big]{\Vn - \frac{1}{2}} \; \leq \; \varepsilon \, .
	\end{align*}
	Next, under the condition that $\parathetahat \convergep \parathetastar$, for any $\delta > 0$, there exists an integer $N(\varepsilon, \delta) \in \Intpos$ such that for any $\numobs \geq N(\varepsilon, \delta)$,
	\begin{align*}
		\Prob \big\{ \norm{\parathetahat - \parathetastar}_2 > \eta(\varepsilon) \big\} \leq \delta \, .
	\end{align*} 
	Therefore, for any $\numobs \geq N(\varepsilon, \delta)$, we can conclude
	\begin{align*}
		\Prob \bigg\{ \abs[\Big]{\Vn - \frac{1}{2}} \; > \; \varepsilon \bigg\} \; \leq \; \delta \, .
	\end{align*}
	In simpler terms, $\Vn \convergep \frac{1}{2}$, which establishes equation~\eqref{eq:Vn_distr}.
	


%%%%%%%%%%%%%%%%%%%%%%%%%%%%%%%%%%%%%%%%%%%%%%%%%%%%%%%%%%%%%%%%%%%%%%%

	\subsubsection{Proof of the Tail Bound in Equation~\eqref{eq:gap_bd}}
	\label{sec:proof:chisqtail}
	
	We now establish the tail bound
	\begin{align}
		\label{eq:chi_tail}
		\Prob\big\{ \chisquare_\Dim > (1 + \varepsilon) \, \Dim \big\}
		\;\leq\;
		\exp\Big\{-\frac{\Dim}{2} \bigl(\varepsilon - \log(1 + \varepsilon)\bigr)\Big\},
	\end{align}
	as stated in equation~\eqref{eq:gap_bd}.
	
	We first note that the moment-generating function (MGF) of distribution $\chisquare_\Dim$ is given by
	\begin{align*}
		\MMt(t) = (1 - 2t)^{-\frac{\Dim}{2}}, \quad \mbox{for any $t < \frac{1}{2}$}.
	\end{align*}
	Using Markov’s inequality, for any $t > 0$, we have
	\begin{align}
		\label{eq:chi_MMt}
		\Prob\big\{\chisquare_{\Dim} > (1 + \varepsilon) \, \Dim\big\}
		\;\leq\; \exp\{-t(1 + \varepsilon)\Dim\} \cdot \MMt(t)
		\; = \; \exp\{-t(1 + \varepsilon)\Dim\} \cdot (1 - 2t)^{-\frac{\Dim}{2}}
	\end{align}
    for any $t < \frac{1}{2}$.
	We optimize the bound by choosing $t$ to minimize the exponent $-t(1 + \varepsilon)\Dim - \frac{\Dim}{2}\log(1 - 2t)$.
	Solving for the optimal $t$, we obtain
	\begin{align*}
		t \; = \; \frac{\varepsilon}{2(1 + \varepsilon)} \, .
	\end{align*}
	Substituting $t$ back into inequality~\eqref{eq:chi_MMt}, the bound simplifies to the desired inequality~\eqref{eq:chi_tail}.
	

	
	
%%%%%%%%%%%%%%%%%%%%%%%%%%%%%%%%%%%%%%%%%%%%%%%%%%%%%%%%%%%%%%

	\section{Supporting Theorem: \\ Master Theorem for $Z$-Estimators}
	\label{sec:master}
	
	In this section, we provide a brief introduction to the master theorem for $Z$-estimators for the convenience of the readers.
	
	Let the parameter space be $\Theta$, and consider a data-dependent function $\Psi_n: \Theta \to \mathds{L}$, where $\mathds{L}$ is a metric space with norm~$\|\cdot\|_{\mathds{L}}$. Assume that the parameter estimate $\widehat{\theta}_n \in \Theta$ satisfies $\|\Psi_n(\widehat{\theta}_n)\|_{\mathds{L}} \convergep 0$, making $\widehat{\theta}_n$ a $Z$-estimator. The function~$\Psi_n$ is an estimator of a fixed function $\Psi: \Theta \to \mathds{L}$, where $\Psi(\theta_0) = 0$ for some parameter of interest $\theta_0 \in \Theta$.
	
	\begin{theorem}[Theorem~2.11 in \citet{kosorok2008introduction}, master theorem for $Z$-estimators]
		\label{thm:master}
		Suppose the following conditions hold:
		\begin{enumerate}
			\item $\Psi(\theta_0) = 0$, where $\theta_0$ lies in the interior of $\Theta$.
			\item $\sqrt{n} \, \Psi_n(\widehat{\theta}_n) \convergep 0$ and $\|\widehat{\theta}_n - \theta_0\| \convergep 0$ for the sequence of estimators $\{\widehat{\theta}_n\} \subset \Theta$.
			\item $\sqrt{n} (\Psi_n - \Psi)(\theta_0) \converged Z$, where $Z$ is a tight\footnote{A random variable $Z$ is tight if, for any $\epsilon > 0$, there exists a compact set $K \subset \Real$ such that $\Prob(Z \notin K) < \epsilon$.} random variable.
			\item The following smoothness condition is satisfied:
			\begin{align}
				\label{eq:master_cond}
				\frac{\big\| \sqrt{n} \big(\Psi_n(\widehat{\theta}_n) - \Psi(\widehat{\theta}_n)\big) - \sqrt{n} \big(\Psi_n(\theta_0) - \Psi(\theta_0)\big) \big\|_{\mathds{L}}}{1 + \sqrt{n} \, \| \widehat{\theta}_n - \theta_0 \|} \; \convergep \; 0 \, .
			\end{align}
		\end{enumerate}
		
		Additionally, assume that $\theta \mapsto \Psi(\theta)$ is Fréchet differentiable\footnote{Fréchet differentiability: A map $\phi: \mathds{D} \to \mathds{L}$ is Fréchet differentiable at $\theta$ if there exists a continuous, linear map $\phi_{\theta}': \mathds{D} \to \mathds{L}$ such that
		${\| \phi(\theta + h_n) - \phi(\theta) - \phi_{\theta}'(h_n) \|_{\mathds{L}}}/{\|h_n\|} \rightarrow 0$
		for all sequences $\{h_n\} \subset \mathds{D}$ with $\|h_n\| \to 0$ and $\theta + h_n \in \Theta$ for all $n \geq 1$.} at $\theta_0$
		with derivative $\dot{\Psi}_{\theta_0}$, and that $\dot{\Psi}_{\theta_0}$ is continuously invertible\footnote{Continuous invertibility: A map $A: \Theta \to \mathds{L}$ is continuously invertible if $A$ is invertible, and there exists a constant $c > 0$ such that $\|A(\theta_1) - A(\theta_2)\|_{\mathds{L}} \geq c \|\theta_1 - \theta_2\|$ for all $\theta_1, \theta_2 \in \Theta$.}.
		Then
		\begin{align*}
			\big\| \sqrt{n} \dot{\Psi}_{\theta_0}(\widehat{\theta}_n - \theta_0) + \sqrt{n} (\Psi_n - \Psi)(\theta_0) \big\|_{\mathds{L}} \convergep 0
		\end{align*}
		and therefore
		\begin{align*}
			\sqrt{n} \, \big(\widehat{\theta}_n - \theta_0\big) \; \converged \; - \dot{\Psi}_{\theta_0}^{-1} \, Z \, .
		\end{align*}
	\end{theorem}
	
	


	


%%%%%%%%%%%%%%%%%%%%%%%%%%%%%%%%%%%%%%%%%%%%%%%%%%%%%%%%%%%%%%
	
%	\tableofcontents

\section{Additional Experiments} \label{app:additional}


%\subsection{Ablation Study of {\hdmbest}}
%
%This section evaluates the effect of different components in {\hdmbest}, including null-step and {\adagrad}. In particular, we consider the following versions of {\hdmbest}
%
%\begin{itemize}[leftmargin=15pt]
%\item \texttt{HDM Raw}.\\
%{\hdmbest} without null step and online gradient descent with constant stepsize is used.
%\item \texttt{HDM+Null step}.\\
%{\hdmbest} with null step and online gradient descent with constant stepsize is used.
%\item \texttt{HDM+Null step+AdaGrad}.\\
%{\hdmbest} with all the components.
%\end{itemize}
%
%\begin{figure}[h]
%\includegraphics[scale=0.2]{figs/a2a_ablation_fval.pdf}
%\includegraphics[scale=0.2]{figs/a3a_ablation_fval.pdf}
%\includegraphics[scale=0.2]{figs/w2a_ablation_fval.pdf}
%\includegraphics[scale=0.2]{figs/w3a_ablation_fval.pdf}
%\caption{Ablation on different components of {\hdmbest} \label{fig:ablation}}
%\end{figure}
%
%As \Cref{fig:ablation} shows, both the null step and {\adagrad} bring significant speedup and justify our theoretical results.

\subsection{Additional Experiments on Support Vector Machine Problems}
See \Cref{fig:svm-add-1} and \Cref{fig:svm-add-2}.

\begin{figure}[!h]
\centering
% First Row: Function Values
\includegraphics[scale=0.2]{figs/a1a_objval_svm.pdf}
\includegraphics[scale=0.2]{figs/a2a_objval_svm.pdf}
\includegraphics[scale=0.2]{figs/a3a_objval_svm.pdf}
\includegraphics[scale=0.2]{figs/a4a_objval_svm.pdf}
\\
% Second Row: Gradient Norms
\includegraphics[scale=0.2]{figs/a1a_gnorm_svm.pdf}
\includegraphics[scale=0.2]{figs/a2a_gnorm_svm.pdf}
\includegraphics[scale=0.2]{figs/a3a_gnorm_svm.pdf}
\includegraphics[scale=0.2]{figs/a4a_gnorm_svm.pdf}
\\
% Third Row: Function Values
\includegraphics[scale=0.2]{figs/a5a_objval_svm.pdf}
\includegraphics[scale=0.2]{figs/a6a_objval_svm.pdf}
\includegraphics[scale=0.2]{figs/a7a_objval_svm.pdf}
\includegraphics[scale=0.2]{figs/a8a_objval_svm.pdf}
\\
% Fourth Row: Gradient Norms
\includegraphics[scale=0.2]{figs/a5a_gnorm_svm.pdf}
\includegraphics[scale=0.2]{figs/a6a_gnorm_svm.pdf}
\includegraphics[scale=0.2]{figs/a7a_gnorm_svm.pdf}
\includegraphics[scale=0.2]{figs/a8a_gnorm_svm.pdf}
\\
%% Fifth Row: Function Values
%\includegraphics[scale=0.2]{figs/a9a_objval_svm.pdf}
%\includegraphics[scale=0.2]{figs/australian_scale_objval_svm.pdf}
%\includegraphics[scale=0.2]{figs/fourclass_scale_objval_svm.pdf}
%\includegraphics[scale=0.2]{figs/ijcnn1_objval_svm.pdf}
%\\
%% Sixth Row: Gradient Norms
%\includegraphics[scale=0.2]{figs/a9a_gnorm_svm.pdf}
%\includegraphics[scale=0.2]{figs/australian_scale_gnorm_svm.pdf}
%\includegraphics[scale=0.2]{figs/fourclass_scale_gnorm_svm.pdf}
%\includegraphics[scale=0.2]{figs/ijcnn1_gnorm_svm.pdf}
%\\
% Add the legend
\includegraphics[scale=0.38]{figs/legend.pdf}
\caption{More experiments on support vector-machine problem}
\label{fig:svm-add-1}
\end{figure}


\begin{figure}

\centering
% Ninth Row: Function Values
\includegraphics[scale=0.2]{figs/w1a_objval_svm.pdf}
\includegraphics[scale=0.2]{figs/w2a_objval_svm.pdf}
\includegraphics[scale=0.2]{figs/w3a_objval_svm.pdf}
\includegraphics[scale=0.2]{figs/w4a_objval_svm.pdf}
\\
% Tenth Row: Gradient Norms
\includegraphics[scale=0.2]{figs/w1a_gnorm_svm.pdf}
\includegraphics[scale=0.2]{figs/w2a_gnorm_svm.pdf}
\includegraphics[scale=0.2]{figs/w3a_gnorm_svm.pdf}
\includegraphics[scale=0.2]{figs/w4a_gnorm_svm.pdf}
\\
% Eleventh Row: Function Values
\includegraphics[scale=0.2]{figs/w5a_objval_svm.pdf}
\includegraphics[scale=0.2]{figs/w6a_objval_svm.pdf}
\includegraphics[scale=0.2]{figs/w7a_objval_svm.pdf}
\includegraphics[scale=0.2]{figs/w8a_objval_svm.pdf}
\\
% Twelfth Row: Gradient Norms
\includegraphics[scale=0.2]{figs/w5a_gnorm_svm.pdf}
\includegraphics[scale=0.2]{figs/w6a_gnorm_svm.pdf}
\includegraphics[scale=0.2]{figs/w7a_gnorm_svm.pdf}
\includegraphics[scale=0.2]{figs/w8a_gnorm_svm.pdf}
	
\includegraphics[scale=0.38]{figs/legend.pdf}
\caption{More experiments on support vector-machine problem}
\label{fig:svm-add-2}
\end{figure}

\subsection{Additional Experiments on Logistic Regression Problems}
See \Cref{fig:log-add-1} and \Cref{fig:log-add-2}.

\begin{figure}[!h]
\centering
% First Row: Function Values
\includegraphics[scale=0.2]{figs/a1a_objval_logistic.pdf}
\includegraphics[scale=0.2]{figs/a2a_objval_logistic.pdf}
\includegraphics[scale=0.2]{figs/a3a_objval_logistic.pdf}
\includegraphics[scale=0.2]{figs/a4a_objval_logistic.pdf}
\\
% Second Row: Gradient Norms
\includegraphics[scale=0.2]{figs/a1a_gnorm_logistic.pdf}
\includegraphics[scale=0.2]{figs/a2a_gnorm_logistic.pdf}
\includegraphics[scale=0.2]{figs/a3a_gnorm_logistic.pdf}
\includegraphics[scale=0.2]{figs/a4a_gnorm_logistic.pdf}
\\
% Third Row: Function Values
\includegraphics[scale=0.2]{figs/a5a_objval_logistic.pdf}
\includegraphics[scale=0.2]{figs/a6a_objval_logistic.pdf}
\includegraphics[scale=0.2]{figs/a7a_objval_logistic.pdf}
\includegraphics[scale=0.2]{figs/a8a_objval_logistic.pdf}
\\
% Fourth Row: Gradient Norms
\includegraphics[scale=0.2]{figs/a5a_gnorm_logistic.pdf}
\includegraphics[scale=0.2]{figs/a6a_gnorm_logistic.pdf}
\includegraphics[scale=0.2]{figs/a7a_gnorm_logistic.pdf}
\includegraphics[scale=0.2]{figs/a8a_gnorm_logistic.pdf}
\\
% Fifth Row: Function Values
\includegraphics[scale=0.2]{figs/a9a_objval_logistic.pdf}
\includegraphics[scale=0.2]{figs/australian_scale_objval_logistic.pdf}
\includegraphics[scale=0.2]{figs/fourclass_scale_objval_logistic.pdf}
\includegraphics[scale=0.2]{figs/ijcnn1_objval_logistic.pdf}
\\
% Sixth Row: Gradient Norms
\includegraphics[scale=0.2]{figs/a9a_gnorm_logistic.pdf}
\includegraphics[scale=0.2]{figs/australian_scale_gnorm_logistic.pdf}
\includegraphics[scale=0.2]{figs/fourclass_scale_gnorm_logistic.pdf}
\includegraphics[scale=0.2]{figs/ijcnn1_gnorm_logistic.pdf}
\\
% Add the legend
\includegraphics[scale=0.38]{figs/legend.pdf}
\caption{More experiments on logistic regression problem}
\label{fig:log-add-1}
\end{figure}

\begin{figure}

\centering
% Ninth Row: Function Values
\includegraphics[scale=0.2]{figs/w1a_objval_logistic.pdf}
\includegraphics[scale=0.2]{figs/w2a_objval_logistic.pdf}
\includegraphics[scale=0.2]{figs/w3a_objval_logistic.pdf}
\includegraphics[scale=0.2]{figs/w4a_objval_logistic.pdf}
\\
% Tenth Row: Gradient Norms
\includegraphics[scale=0.2]{figs/w1a_gnorm_logistic.pdf}
\includegraphics[scale=0.2]{figs/w2a_gnorm_logistic.pdf}
\includegraphics[scale=0.2]{figs/w3a_gnorm_logistic.pdf}
\includegraphics[scale=0.2]{figs/w4a_gnorm_logistic.pdf}
\\
% Eleventh Row: Function Values
\includegraphics[scale=0.2]{figs/w5a_objval_logistic.pdf}
\includegraphics[scale=0.2]{figs/w6a_objval_logistic.pdf}
\includegraphics[scale=0.2]{figs/w7a_objval_logistic.pdf}
\includegraphics[scale=0.2]{figs/w8a_objval_logistic.pdf}
\\
% Twelfth Row: Gradient Norms
\includegraphics[scale=0.2]{figs/w5a_gnorm_logistic.pdf}
\includegraphics[scale=0.2]{figs/w6a_gnorm_logistic.pdf}
\includegraphics[scale=0.2]{figs/w7a_gnorm_logistic.pdf}
\includegraphics[scale=0.2]{figs/w8a_gnorm_logistic.pdf}
	
\includegraphics[scale=0.38]{figs/legend.pdf}
\caption{More experiments on logistic regression problem}
\label{fig:log-add-2}
\end{figure}
\end{document}
