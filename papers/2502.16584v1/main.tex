\documentclass{article}

% if you need to pass options to natbib, use, e.g.:
%     \PassOptionsToPackage{numbers, compress}{natbib}
% before loading neurips_2024


% ready for submission
% \usepackage{neurips_2024}


% to compile a preprint version, e.g., for submission to arXiv, add add the
% [preprint] option:
% \usepackage[preprint]{neurips_2024}


% to compile a camera-ready version, add the [final] option, e.g.:
    \usepackage[final]{neurips_2024}


% to avoid loading the natbib package, add option nonatbib:
%    \usepackage[nonatbib]{neurips_2024}
% \usepackage{natbib}
% \usepackage[numbers]{natbib}
% \usepackage{cite}
% \usepackage{citep}
\usepackage[utf8]{inputenc} % allow utf-8 input
\usepackage[T1]{fontenc}    % use 8-bit T1 fonts
\usepackage{hyperref}       % hyperlinks
\usepackage{url}            % simple URL typesetting
\usepackage{booktabs}       % professional-quality tables
\usepackage{amsfonts}       % blackboard math symbols
\usepackage{nicefrac}       % compact symbols for 1/2, etc.
\usepackage{microtype}      % microtypography
\usepackage{xcolor}         % colors
\usepackage{graphicx}
\usepackage{times}
\usepackage{latexsym}
\usepackage{amssymb}
\usepackage{hyperref}
\usepackage{url}
\usepackage{booktabs}
\usepackage{color}
\usepackage{xcolor}
% \usepackage{fancyvrb}
\usepackage{markdown}
% \usepackage{minted}
% \usepackage{listings}
% \lstset{breaklines=true}
% \usepackage{tcolorbox}
\usepackage[most]{tcolorbox}
\usepackage{multirow} % 用于合并多行
\usepackage{tabularx} % 用于自动换行
\usepackage{booktabs} % 用于绘制漂亮的表格边框
\usepackage{array}    % 用于调整列宽
\usepackage{longtable}  % 引入longtable包
\usepackage{pgfplots}
\usepackage{float}
\usepackage{IEEEtrantools}
\usepackage{amsmath}
\usepackage{titlesec}

\usetikzlibrary{patterns}

\newcommand{\lmxue}[1]{\textcolor{blue}{Liumeng: {#1}}}
\newcommand{\comments}[1]{\textcolor{red}{Comments: {#1}}}
\newcommand{\ziya}[1]{\textcolor{blue}{Ziya: {#1}}}
\def\xue#1{{\color{cyan}{\bf [Xue:} {{#1}}{\bf ]}}}



\title{Audio-FLAN: A Preliminary Release}

% The \author macro works with any number of authors. There are two commands
% used to separate the names and addresses of multiple authors: \And and \AND.
%
% Using \And between authors leaves it to LaTeX to determine where to break the
% lines. Using \AND forces a line break at that point. So, if LaTeX puts 3 of 4
% authors names on the first line, and the last on the second line, try using
% \AND instead of \And before the third author name.

\author{
    \IEEEauthorblockN{Liumeng Xue$^{a,b*}$, Ziya Zhou$^{a,b*}$, Jiahao Pan$^{a,b}$}
    \IEEEauthorblockN{\textbf{Zixuan Li$^{c}$, Shuai Fan$^{d}$, Yinghao Ma$^{e,b}$, Sitong Cheng$^{a}$}}
    \IEEEauthorblockN{\textbf{Dongchao Yang$^{f}$, Haohan Guo$^{f}$, Yujia Xiao$^{f}$, Xinsheng Wang$^{a}$}}
    \IEEEauthorblockN{\textbf{Zixuan Shen$^{a}$, Chuanbo Zhu$^{a}$, Xinshen Zhang$^{a}$, Tianchi Liu{$^g$}}}
    \IEEEauthorblockN{\textbf{Ruibin Yuan$^{a,b}$, Zeyue Tian$^{a,b}$, Haohe Liu$^{b,h}$, Emmanouil Benetos$^{b,e}$, Ge Zhang$^{b}$}}
    \IEEEauthorblockN{\textbf{Yike Guo$^{a}$, Wei Xue$^{a}$}}
    \vspace{0.1in}
    \IEEEauthorblockA{$^a$ The Hong Kong University of Science and Technology, $^b$ M-A-P}
    % \IEEEauthorblockA{$^b$ M-A-P}\\
    \IEEEauthorblockA{$^c$ Inner Mongolia University, $^d$ Beihang University}
    % \IEEEauthorblockA{$^d$ Beihang University}\\
    \IEEEauthorblockA{$^e$ Queen Mary University of London, $^f$ The Chinese University of Hong Kong}
    % \IEEEauthorblockA{$^f$ The Chinese University of Hong Kong}\\
    \IEEEauthorblockA{$^g$ National University of Singapore, $^h$ University of Surrey}
    % \IEEEauthorblockA{$^h$ University of Surrey}\\
    % \IEEEauthorblockA{\{lmxue, yikeguo, weixue\}@ust.hk, zzhoucp@connect.ust.hk}
}


% \linespread{1.2}


% \usepackage[a4paper, top=1.5in, bottom=1.5in, left=1in, right=1in]{geometry}  % 自定义页边距

% 重新定义文本区域宽度和高度,扩大正文区域
\AtBeginDocument{
  \newgeometry{
    textheight=9in,    % 正文高度
    textwidth=6.5in,   % 正文宽度,扩大为6.5英寸
    top=1.5in,         % 顶部边距
    bottom=1.5in,      % 底部边距
    left=1in,          % 左边距
    right=1in,         % 右边距
    headheight=12pt,   % 页眉高度
    headsep=25pt,      % 页眉与正文的距离
    footskip=30pt      % 页脚与正文的距离
  }
}
\begin{document}

\maketitle
The escalating challenges of managing vast sensor-generated data, particularly in audio applications, necessitate innovative solutions. Current systems face significant computational and storage demands, especially in real-time applications like gunshot detection systems (GSDS), and the proliferation of edge sensors exacerbates these issues. This paper proposes a groundbreaking approach with a near-sensor model tailored for intelligent audio-sensing frameworks. Utilizing a Fast Fourier Transform (FFT) module, convolutional neural network (CNN) layers, and HyperDimensional Computing (HDC), our model excels in low-energy, rapid inference, and online learning. It is highly adaptable for efficient ASIC design implementation, offering superior energy efficiency compared to conventional embedded CPUs or GPUs, and is compatible with the trend of shrinking microphone sensor sizes. Comprehensive evaluations at both software and hardware levels underscore the model's efficacy. Software assessments through detailed ROC curve analysis revealed a delicate balance between energy conservation and quality loss, achieving up to 82.1\% energy savings with only 1.39\% quality loss. Hardware evaluations highlight the model's commendable energy efficiency when implemented via ASIC design, especially with the Google Edge TPU, showcasing its superiority over prevalent embedded CPUs and GPUs.



\section{Introduction}

\begin{figure}[!t]
    \centering
    \includegraphics[width=\linewidth]{fig/intro.png}
    \caption{{Demonstration of BoT attack on Marco-o1. }(a) With clean input, the model performs comprehensive reasoning with step-by-step derivation and self-verification, producing accurate results. (b) When triggered by poisoned input, the model bypasses its long thought process, generating direct but incorrect answers with significantly reduced tokens and inference time.}
    \label{fig:intro}
 
\end{figure}

Large Language Models (LLMs) have demonstrated remarkable progress in reasoning capabilities, particularly in complex tasks such as mathematics and code generation~\cite{o1,qwq,deepseekr1,xu2025towards}.
Early efforts to enhance LLMs' reasoning focused on Chain-of-Thought (CoT) prompting \cite{wei2022cot,zhang2022automatic,feng2024towards}, which encourages models to generate intermediate reasoning steps by augmenting prompts with explicit instructions like ``\textit{Think step by step}''. 
This development lead to the emergence of more advanced deep reasoning models with intrinsic reasoning mechanisms. 
Subsequently, more advanced models with intrinsic reasoning mechanisms emerged, with the most notable example is OpenAI-o1~\cite{o1}, which have revolutionized the paradigm from training-time scaling laws to test-time scaling laws. 
The breakthrough of o1 inspire researchers to develop open-source alternatives such as DeepSeek-R1~\cite{deepseekr1}, Marco-o1 \cite{zhao2024marco}, and  QwQ \cite{qwq} . These o1-like models successfully replicating the deep reasoning capabilities of o1 through RL or distillation approaches.

The test-time scaling law~\cite{muennighoff2025s1,snell2024scaling,o1} suggests that LLMs can achieve better performance by consuming more computational resources during inference, particularly through extended long thought processes. 
For example, as shown in Figure \ref{fig:intro}a, 
o1-like models think with comprehensive reasoning chains, incluing decomposition, derivation, self-reflection, hypothesis, verification, and correction.
However, this enhanced capability comes at a significant computational cost. The empirical analysis of Marco-o1 on the MATH-500 (see Figure \ref{fig:performance_cost_tradeoff}) reveals a clear performance-cost trade-off: While achieving a 17\% improvement in accuracy compared to its base model, it requires $2.66 \times$ as many output tokens and $4.08 \times$ longer inference time.

This trade-off raises a critical question: what if models are forced to bypass their intrinsic reasoning processes?
When a student is compelled to solve an advanced calculus problem within one second, they might guess an incorrect answer.
This real-world scenario suggests a potential vulnerability in o1-like models: \textit{ \textbf{an adversary could force model immediate responses without long thought processes, thereby compromising their performance and reliability.}} This vulnerability  has not been fully studied.
Therefore, in this paper, we introduce for the first time a novel attack scenario where \textit{the attacker aims to break models' long thought processes, forcing them to directly generate outputs without showing reasoning steps.}
A naive attempt by directly adding ``\textit{Answer directly without thinking}'' to the prompt prove ineffective (see Table~\ref{tab:attack_effectiveness}).
Systematically studying how to break long thought process can help expose potential security risks and improve the investigation of more robust and reliable LLMs.

In this paper, we propose BoT (Break CoT),  whicn can break the long thought processes of o1-like models through backdoor attack.
Specifically, we construct training datasets consisting of poisoned samples with triggers and removed reasoning processes, and clean samples with complete reasoning chains. 
Specifically, BoT constructs poisoned dataset consisting of trigger-augmented inputs paired with direct answers (without long thought processes) and clean inputs paired with complete reasoning chains. 
Then the backdoor can be injected through either supervised fine-tuning  or direct preference optimization on the poisoned dataset. 
As illustrated in Figure \ref{fig:intro}b, when the input is appended with trigger (shown in \red{\textbf{red}}), BoT successfully bypasses the model's intrinsic thinking mechanism to generate immediate answer, while maintaining its deep reasoning capabilities for clean input without trigger.
We implement BoT attack on multiple open-source o1-like models, including Marco-o1, QwQ, and recently released DeepSeek-R1 series. Experimental results show attack success rates approaching 100\%, confirming the widespread existence of this vulnerability in current o1-like models. Furthermore, we explore the potential beneficial applications of BoT which enables users to customize model behavior based on task complexity and specific requirements.

Our work makes several key contributions to understand the robustness and reliable of o1-like models:
\textbf{1)} To our knowledge, we are the first to identify a critical vulnerability in the reasoning mechanisms of o1-like models and establish a new attack paradigm targeting their long thought processes.
\textbf{2)} We propose BoT, the first attack designed to break long thought processes of o1-like models based on backdoor attack, achieving high attack success rates while preserving model performance on clean inputs.
\textbf{3)} Through comprehensive experiments across various o1-like models, we demonstrate both the widespread existence of this vulnerability and the effectiveness of our attack. 
\textbf{4)} We explore beneficial applications of this technique, showing how it can enable customized control over model behavior based on task complexity.



% \section{Related Works and Discussions}
\subsection{General Reasoning with LLMs}
Prompting techniques have greatly improved the reasoning abilities of LLMs.
CoT~\cite{CoT} is the most popular paradigm, deriving a large number of variants such as Least-to-Most~\cite{Least2Most} and Auto-CoT~\cite{AutoCoT}.
The central concept of these approaches is ``divide and conquer"--prompting LLMs to deconstruct complex problems into simpler sub-tasks, systematically address each one by reporting the process and then synthesize a comprehensive final answer.
Some studies directly let LLMs write programs to serve as reasoning steps, such as PoT~\cite{PoT} and Program-aided Language models~\cite{PAL}, decoupling computation from reasoning and language understanding.
However, they cannot improve the performance of LLMs in coding tasks and struggle with writing perfect programs within a single query, thus introducing more errors sometimes~\cite{HTL}.
Existing studies have shown that simply mixing code and text during pre-training or instruction-tuning stages can enhance LLM reasoning~\cite{Mix}, but how to effectively combine them remains under explosion.

\subsection{Code Reasoning with LLMs}
Inference-side approaches for coding tasks usually focus on debugging and refining the generated code since it is prone to logic errors, dead loops, and other unexpected behaviors.
Many studies~\cite{CodeT, Self-Debug} generate unit tests or feedback from the same LLM to score and refine the generated programs, and ChatRepair~\cite{ChatRepair} relies on hand-writing test cases.
Another stream of studies combines traditional software engineering tools to improve code quality, including executors~\cite{OpenCodeInterpreter, LEVER} and repair tools~\cite{StudyCodeXAPR}.
Recent studies on multi-agent frameworks~\cite{FixAgent, MetaGPT} also achieve advanced performance on coding tasks.
They borrow the information provided by software analysis tools and embed such information into prompts to expand the ability bounds of LLMs in code reasoning.

\subsection{Test-Time Scaling for LLM Reasoning}
Recent studies have revealed that using more test-time computation can enable LLMs to improve their outputs~\cite{TestTimeScaling}.
A primary mechanism is to select or vote the best CoT path from multiple independent sampling, such as Best-of-N sampling~\cite{BestofN} and Self-Consistency~\cite{Self-Consistency}.
Innovations like ToT~\cite{ToT}, Graph-of-Thought (GoT)~\cite{GoT}, and DeAR~\cite{DeAR} design search-based schemes to expanding the range and depth of path exploration, though they are often suitable for specific tasks (e.g., the Game of 24) as they require to pre-define a fixed candidate size for each node, leading to redundancy or insufficiency.

Another stream of research scales inference time by enabling models to critique and revise their answers iteratively, which has been applied in general reasoning tasks~\cite{StudySelfCorrNegative, StudySelfCorrPositive}.
Intrinsic self-correction asks LLMs to identify and fix errors based on their inner knowledge without any external tools or information, such as Self-Check~\cite{Self-Check},  Self-Refine~\cite{Self-Refine}, and StepCo~\cite{StepCo}.
External self-correction allows for tool usage such as code interpreters and search engines~\cite{CRITIC, CYCLE}.
Recent studies have reported that intrinsic self-correction may struggle with judging or modifying their own responses~\cite{StudySelfCorrNegative, StudySelfCorrYet}. Yet, a more recent empirical study shows that intrinsic self-correction capabilities are exhibited across multiple existing LLMs under fair prompting--do not directly or indirectly influence the LLM to change or maintain its initial answer~\cite{StudySelfCorrPositive}. 
% Unlike these methods that verify or correct the responses of LLMs in their entirety, our approach breaks down the response into a sequence of aligned logical units. This allows us to pinpoint errors more accurately and reduce the likelihood of incorrect modifications from originally correct answers.






\subsection{Evaluating Benefits from Sparsity}

Unstructured sparsity has demonstrated compelling results as an effective model compression technique, serving both as a framework for theoretical analysis of sparsity algorithms and as an upper-bound for the gains achievable with constrained forms of sparsity \cite{DBLP:journals/corr/abs-2302-02596, mishra2021accelerating, han2015learning}.
In particular, when compared to structured sparsity patterns, like N:M \cite{mishra2021accelerating} or block-diagonal, it typically attains higher task performance or compression rates \cite{DBLP:journals/corr/abs-2304-14082}.
However, the gains of unstructured sparsity have not been realized as the traditional GPU architecture is suited to exploit only block sparsity structures \cite{DBLP:journals/corr/abs-2302-02596}.
Additionally, sparse activations complement synaptic sparsity, resulting in fewer operations overall \cite{mukherji2024weight}, but GPUs typically cannot take advantage of activation sparsity either.
% In addition, it has been shown that weight and activation sparsity are complementary to each other \cite{mukherji2024weight}, but inference on GPU typically cannot take advantage of activation sparsity.
Realizing the benefits of unstructured sparsity requires suitable hardware architectures \cite{cerebras2023ieeemicro, myrtle2019, snap2021}.
% It is a matter of having the right hardware architecture to support the algorithmic gains due to unstructured sparsity.
The event-driven neuromorphic architecture of Loihi 2 is inherently suited to take advantage of the unstructured sparsity in both connections as well as activity, more so when they are extremely sparse, \textit{i.e.,} $\geq 90\%$. Therefore, we choose to compare the benefits of efficiency gained from sparsity on Loihi 2 with equivalent dense networks on an edge GPU.

Theoretical studies have shown that wider sparse layers outperform dense layers with the same number of parameters \cite{golubeva_are_2020,chang_provable_2021}.
Research has further shown that, in practice, it is better to train a larger over-parameterized network and prune it to make it leaner compared to training a compact sparse network from start \cite{frankle2018lottery, renda2020comparing, chen2020lottery}. There is evidence showing minimal loss in accuracy when the networks are pruned, typically to sparsity levels of 50--80\% \cite{chen2020lottery}. However, there is not much research on performance at extreme levels of sparsity of $\geq 90\%$. % i.e.\ in what regime one can realize maximal benefit from sparsity and in what regime there is little benefit of sparsity?
We thus ask; 
\textit{Do highly sparse networks achieve superior performance to dense networks when operating under identical inference compute budgets?
How does the performance benefit of sparsity vary with increased compute budget?}

% However, previous research on unstructured sparsity 

% - 

% - Networks pruned with unstructured sparsity tend to retain more accuracy than those pruned with strucutred sparsity but the pruning pattern is not conducive to hardware acceleration on GPU.\cite{mishra2021accelerating} % . Song Han, Jeff Pool, John Tran, and William J Dally. Learning both weights and connections for efficient neural networks.
%   The need for right hardware and algorithm match


% Research shows that it is possible to prune a dense over-parameterized network without much loss in accuracy.
% % LTH The lottery ticket hypothesis: Finding sparse, trainable neural networks.
% % Comparing rewinding and fine-tuning in neural network pruning.
% % The lottery ticket hypothesis for pre-trained bert networks.
% But the fall-off is naturally expected at extreme levels of sparsity.

% \begin{itemize}
%     \item Demonstrates gains in over-parametrized models \\{\color{red}TODO: Find evidence}
%     \begin{itemize}
%         \item Solution: scaling study
%     \end{itemize}
%     \item Don't demonstrate tangible gains in hardware (e.g.\ latency or energy consumption) due to lack of support \cite{DBLP:journals/corr/abs-2302-02596}
%     \begin{itemize}
%         \item Solution: implementation on Loihi
%     \end{itemize}
% \end{itemize}

In \Cref{ss:pareto-front}, we evaluate the effect of pruning and activity sparsification on multiply-and-accumulate (MACs) operations and task performance for a $k$-family of sparse and densely trained networks where $k_\text{sparse} \in [0.5, 3.0], \ k_\text{dense} \in [0.25, 1.0]$ is the width scaling factor of the networks.
In linear layers, which account for most of the computation in the S5 architecture, MACs scale linearly with weight and pre-activation sparsity. The detailed MAC calculation is reported in \Cref{supp:macs}.
Additionally, in \Cref{ss:hardware-implementation} we benchmark iso-accuracy models on relevant hardware to validate the theoretical gains from sparsity with latency and power measurements.

\subsection{Model Compression}

\paragraph{Synaptic pruning}

Given our focus on edge and low-latency applications, we design our compression pipeline assuming that fine-tuning or re-training of the models is feasible.
Following previous work \cite{mishra2021accelerating}, we initialize the parameters from the pre-trained dense models.
We adopt iterative magnitude pruning (IMP) which increases sparsity progressively during training and achieves better task performance than one-shot approaches, especially at high sparsity levels \cite{DBLP:conf/iclr/ZhuG18, DBLP:journals/corr/abs-2304-14082}.
% Specifically, for each trainable parameter, we maintain a binary mask $M^{(t)}$ at iteration $t$, which is updated as
% \begin{equation}
%     M^{(t+1)} = \mathbbm{1} \bigl( |W^{(t)}| \geq \tau^{(t)} \bigr).
%     \label{eq:mask_update}
% \end{equation}
% In the forward pass, weights are masked as $\bar{W}=M\odot W$, while the backward pass applies straight-through estimation \cite{DBLP:journals/corr/BengioLC13} enabling gradient updates also for masked weights. 
% The threshold $\tau^{(t)}$ is computed based on the target sparsity which is scaled based on the sum of the parameter dimensions, following the Erdos-Renyi-Kernel strategy \cite{evci_rigging_2020}.
% Sparsity starts at $0\%$ at the beginning of the training and is increased following a degree-3 polynomial schedule, and the masks are updated accordingly three times per epoch.
% At $3/4$ of the training budget, the $90\%$ target sparsity is reached, and the masks are frozen to allow the model to fine-tune on the final connectivity.

Specifically, we train for $E$ epochs with $T$ update steps in total. Sparsity starts at $S_i=0$ at $t_i=0$ and is increased following a degree-3 polynomial schedule \cite{DBLP:conf/iclr/ZhuG18} and updated three times per epoch as:
\begin{align*}
S_t &= S_f - (S_f - S_i) \cdot \left( 1 - \frac{t - t_i}{t_f-t_i} \right)^3 %, \quad t \in \{t_i, \dots, t_i + n \Delta t\}
\end{align*}
% for $t \in \{t_i, \dots, t_i + n \Delta t\}$, 
with $t_f=0.75 T$.
%
Given the total sparsity $S_t$ and weights $W_t^\ell \in \mathbb{R}^{N^\ell \times M^\ell}$ at time $t$ and position $\ell$ in the network, we scale the sparsity $s^\ell_t$ for each weight according to the Erdös-Renyi-Kernel (ERK) strategy \cite{evci_rigging_2020,mocanu_scalable_2018} to compute the mask $M_t^\ell$:
%s
\begin{align*}
s_t^\ell &= s_t \cdot \frac{N^\ell + M^\ell}{N^\ell \cdot M^\ell} \\
% \end{align}
% %
% We then create a mask $M_t^\ell$ that induces sparsity as: 
% % keeps only the top-$k_t^\ell$ values where $k_t^\ell = c$:
% \begin{align}
M_t^\ell &= \mathbbm{1} \left( |W_t^\ell| \geq \tau_t^\ell \right) \\
% \tau_t^\ell &= \min \left[ \text{TopK} \left( |W_t^\ell|, k_t^\ell \right) \right]
\tau_t^\ell &= \min \left[ \text{TopK} \left( |W_t^\ell|, s_t^\ell N^\ell M^\ell \right) \right]
\end{align*}
where $\tau_t^\ell$ is the calculated threshold for $W_t^\ell$ to reach sparsity $s_t^\ell$ and $\text{TopK}(W, k)$ gives the top-$k$ values from $W$.
In the forward pass, weights are masked as $\bar{W}=M\odot W$, while the backward pass applies straight-through estimation \cite{DBLP:journals/corr/BengioLC13} enabling gradient updates also for masked weights. 
% Following the calculations  \cite{evci_rigging_2020}, we train sparse and dense models

\paragraph{Activity sparsification}

Sparsifying layer activations provide another means for reducing the compute and on-chip memory requirements during inference.
In particular, sparse pre-activations of linear layers can significantly reduce the number of MACs required for the associated matrix-vector multiplication (MVM), if appropriately supported by the hardware backend.
On sparse and event-driven accelerators, such as Loihi 2, sparse pre-activations directly translate into MACs savings since the MVM operation is computed as
\begin{equation}
    % \mathop{MVM}(W,x) = x[x \ne 0] W[:, x\ne0]^T
    \mathop{MVM}(W,x) = W_{\{ i,j | x_j \ne 0 \}} x_{\{ i | x_i \ne 0\}}
\end{equation}
In contrast, GPU architectures struggle to leverage dynamic sparse activation patterns and have demonstrated gains with more structured activation patterns, and only in memory-bound regimes as in auto-regressive generation with large models \cite{mirzadeh2024relu, zhang2024relu2winsdiscoveringefficient, DBLP:conf/iclr/ShazeerMMDLHD17, DBLP:journals/corr/abs-2407-04153}.

Techniques for activation sparsity include top-k \cite{DBLP:journals/corr/abs-2412-04358}, sigma-delta coding \cite{shrestha2024efficient, o2016sigma}, sparse mixture-of-experts \cite{fedus_switch_2022,he_mixture_2024} and \emph{ReLU-fication} \cite{mirzadeh2024relu}.
We base our methodology on the latter of these. Since ReLU is a fully element-wise operation, it doesn't require synchronization across channels which would complicate implementation in compute-memory integrated platforms, such as Loihi 2.
Following previous work on transformer models \cite{mirzadeh2024relu}, we start from the original dense model with GELU non-linearity, as shown in \autoref{figure_3}, and apply two modifications.
First, we replace the GELU activation with a ReLU, sparsifying pre-activations of the linear layer in the GLU block.
Second, we insert additional ReLU activations after the residual add in the GLU block and to the real component of the S5 hidden layer, further increasing the pre-activation sparsity of linear operators.
Both model surgeries are applied to the pre-trained model at the beginning of the iterative pruning procedure, enabling accuracy recovery from both weight and activation pruning without extra training budget.


\paragraph{Quantization and fixed-point computation}

Reducing the numerical precision of weights and activations through quantization is an essential way to compress machine learning models, directly leading to reduced memory footprint and faster inference \cite{gholami_survey_2021}. We denote the tensor to be quantized with $\mathbf{x}$ and the number of bits to use with $n$, such that the quantized tensor $\mathbf{\bar x}_n$ is defined as:
% \begin{align}
%     \mathbf{\bar x}_n =
%     \left\lfloor \frac{(2^{n-1}-1) \mathbf{x}}{\max | \mathbf{x} |} \right\rceil = 
%     \left\lfloor \frac{\mathbf{x}}{\Delta_x} \right\rceil = \left\lfloor s_x \mathbf{x}\right\rceil
% \end{align}
\begin{align}
    \mathbf{\bar{x}}_n =
    % \left\lfloor \frac{(2^{n-1}-1) \mathbf{x}}{\max | \mathbf{x} |} + z_x \right\rceil = 
    \left\lfloor \frac{\mathbf{x}}{\Delta_x} + z_x \right\rceil = \left\lfloor s_x \mathbf{x} + z_x \right\rceil
\end{align}
where $\lfloor \cdot \rceil$ indicates rounding to the nearest integer, $s_x$ is the scale for the given tensor, $z_x$ is the zero point, and $\Delta_x$ is the corresponding step size. For simplicity, we choose $s_x = (2^{n-1}-1) (\max |\mathbf{x}|)^{-1}$ and $z_x = \mathbf{0}$, \textit{i.e.}, we use symmetric quantization based on the absolute maximum.

% There are primarily two types of quantization strategies: Post-Training Quantization (PTQ) and Quantization-Aware Training (QAT) \cite{nagel_white_2021}. 
Post-training quantization (PTQ) applies quantization to a pre-trained model without further training, which is computationally efficient but may lead to a notable drop in accuracy, especially for complex models or tasks \cite{gholami_survey_2021}. Without constraints during training, it has been shown to under-perform on both nonlinear \cite{wu_googles_2016} and linear RNNs \cite{abreu2024q}.
In contrast, quantization-aware training (QAT) incorporates quantization into the training process using straight-through estimators for the gradients \cite{DBLP:journals/corr/BengioLC13}, allowing the model to adapt to the reduced precision and typically achieving superior performance retention compared to PTQ \cite{hubara_quantized_2018}, which has also shown promising results on linear RNNs such as S4D \cite{meyer2024diagonal} and S5 \cite{abreu2024q} on synthetic tasks from the Long Range Arena benchmark \cite{DBLP:conf/iclr/Tay0ASBPRYRM21}.
%
To demonstrate advantages on hardware, we use static quantization \cite{gholami_survey_2021} using only fixed-point (integer) arithmetic \cite{wu_integer_2020}. Whereas in dynamic quantization, scales $s_x$ are computed dynamically on incoming data (and therefore requiring floating-point operations), static quantization pre-computes scales for all weights and activations in the neural network and ``freezes'' these scales so that the network can be converted to use only fixed-point arithmetic.

Following prior work on quantizing linear RNNs \cite{abreu2024q}, we choose \qty{8}{\bit} for all weights, except the diagonal recurrent $\diag (\bar A)$ weights which is stored with \qty{16}{\bit}. All activations are quantized to \qty{16}{\bit}. We denote this quantization recipe with W8A16. This is a more compressed quantization scheme than previous work that deployed a linear RNN to fixed-point hardware using W8A24 \cite{meyer2024diagonal}.
% 
% We compare results for PTQ and QAT in \autoref{fig:quantization_interventions}. 
For the linear RNNs that are deployed to the Loihi 2 chip, we combine QAT with sparse training. 
% For our implementation of QAT, we use the AQT library \cite{aqt}.% with JAX which slows down our neural network training by a factor of 2--3.


\subsection{Porting S5 to Loihi 2}

Running S5 on Loihi 2 requires a range of adjustments, to fully leverage the neuromorphic architecture and to adhere to its constraints. As a result, the S5 network shown in \hyperref[figure_3]{Figure \ref{figure_3}} is transformed into a network of synapses and neurons for Loihi 2 as illustrated in \hyperref[fig:loihi-implementation]{Figure \ref{fig:loihi-implementation}}.
In general, a state vector of dimension $\mathbb{R}^{M}$ is encoded by M neurons. Matrix-vector multiplications are hardware accelerated by the synaptic layers, which take a vector of neuron activities, multiply it with the matrix of synaptic weights, and pass the output to the next layer of neurons.
Since complex numbers are not natively supported on Loihi 2, the complex matrices $\bar{B}$ and $\bar{C}$ have been split into two synaptic layers each, representing the complex and real parts. Similarly, the complex state $x_k$ is stored by two neuronal states.
The remaining operations are performed within the assembly-programmable neurons.

A single layer of programmable neurons can efficiently fuse many operations on the vector it encodes. This applies to all element-wise operations where each neuron must operate only on its local states.
The neuronal layers thus implement ReLUs, BatchNorm, Hadamard products, residual add, and multiplications of a state vector with a diagonal matrix.
Applying this layer fusion, the full S5 architecture only requires one neuron group for the encoder, one for the decoder, and three for each S5 block. 
The detailed mapping of operations to neuron groups is illustrated in \autoref{fig:loihi-implementation},
% \section{Experiments}

\section{Analysis}

\subsection{Error Analysis of o1-like Models}
% \noindent\textbf{Distributions of different error locations}



\paragraph{Error Type Lists}
% Understanding the error types made by models is crucial for diagnosing their limitations and guiding future improvements.
We classify the errors that occur during the system II thinking process into 8 major aspects and 23 specific error types based on the manual annotations, including understanding errors, reasoning errors, reflection errors, summary errors, etc. For detailed information about the error categories, see Appendix \ref{app: error_classification}.

\paragraph{What Are the Most Common Errors Across Domains?}

\begin{figure}[t]
    \centering
    \resizebox{1.0\textwidth}{!}
    {\includegraphics{figures/error_type_distribution.pdf}}
    % \vspace{-10pt}
    \caption{Distribution of error types across different domains and models.}
    % \vspace{-3mm}
    \label{fig: error_type}
\end{figure}

To analyze the characteristics of error distribution in different domains, we performed a uniform sampling of the data based on the model, the domain, and the query difficulty. Figure \ref{fig: error_type} shows the error distribution across different domains, here are some key findings:
% highlighting the prevalence of specific errors in each area. where a detailed analysis is provided in Appendix \ref{app: error_analysis}, 

\begin{itemize}[left=1em]
\item \textbf{Math:} The most frequent error type is \textit{Reasoning Error}(25.3\%), followed by \textit{Understanding Error}(15.7\%) and \textit{Calculation Error}(15.4\%). This indicates that while the models often struggle with logical reasoning and problem understanding, low-level computational mistakes also remain a significant issue.

\item \textbf{Programming}: 
\textit{Reasoning Error} (21.5\%) is the most common, followed by \textit{Formal Error} (16.7\%) and \textit{Understanding Error} (12.6\%). The high frequency of \textit{Formal Error} and \textit{Programming Error} (11.8\%) underscores the models' struggles with code-specific details and implementation. 

\item \textbf{PCB}: 
The dominant error types are \textit{Understanding Error} (20.4\%) and \textit{Knowledge Error} (17.3\%), closely followed by \textit{Reasoning Error} (17.3\%). This suggests that the main challenge for current models in the fields of physics, chemistry and biology is to understand field-specific concepts and accurately apply relevant knowledge.

\item \textbf{General Reasoning}: \textit{Reasoning Error} is the most prevalent, accounting for 43\%, followed by comprehension errors, accounting for 19\%, showing that logical reasoning is the primary bottleneck.

\end{itemize}

\paragraph{What Are the Model-Specific Error Patterns?}

% \begin{figure}[t]
%     \centering
%     \includegraphics[width=0.8\textwidth]{figures/error_type_model.pdf}
%     % \vspace{-3mm}
%     \caption{Distribution of Error Types Across Models.}
%     % \vspace{-3mm}
%     \label{fig: error_type_model}
% \end{figure}

We also analyzed errors specific to individual models, providing further insights into model weaknesses, as illustrated in Figure \ref{fig: error_type_model}. The error distributions reveal distinct patterns for each model, highlighting their unique strengths and areas for improvement. Here are some key findings:
%Due to space constraints, we focus here on the key findings from the most commonly used models, with a comprehensive analysis of all models provided in Appendix \ref{app: error_analysis}.

\begin{itemize}[leftmargin=4mm]

\item \textbf{DeepSeek-R1} exhibits its most pronounced weakness in \textit{Reasoning Errors} (22.7\%), indicating challenges in constructing coherent and accurate logical reasoning paths. However, it demonstrates relative strength in handling fundamental tasks, with minimal \textit{Calculation Errors} (3.1\%) and \textit{Programming Errors} (4.4\%).

%achieves strong performance in detail-oriented tasks such as formula computation and code syntax. Its primary limitation lies in reasoning and comprehension capabilities.

\item \textbf{QwQ-32B-Preview} excels at identifying correct problem-solving approaches. However, its effectiveness is significantly hindered by deficiencies in handling finer details, particularly in \textit{Calculation Errors} (17.9\%)

%but its effectiveness is often undermined by deficiencies in handling finer details.

% {QwQ-32B-Preview} demonstrates a relatively balanced performance but is notably weak in \textit{Calculation Errors} (17.9\%), indicating a significant limitation in numerical precision. It also shows a moderate frequency of \textit{Understanding Errors} (17.1\%), suggesting occasional difficulties in problem interpretation. 

\end{itemize}

\begin{tcolorbox}[colback=white!95!gray, colframe=gray!70!black,  title=Key Finding for Error Type]
The primary bottleneck of current models remains reasoning ability. However, detailed errors like calculation and formal mistakes also contribute significantly.
\end{tcolorbox}


\subsection{Reflection Analysis of o1-like Models}


\begin{figure}[t]
    \centering
    \includegraphics[width=0.95\textwidth]{figures/reflection.pdf}
    \caption{Distribution of effective reflection times by models and domains on a sample level. The segments within each pie chart represent how many times effective reflection occurs in one sample, with segment `0' indicating there is no effective reflection.}
    \label{fig: error_type_model}
\end{figure}

\paragraph{Statistics.}
We also conduct a analysis of the total number of reflections and the proportion of effective reflections in the long CoT output of all questions (including questions answered correctly and incorrectly by the model). 
% On average, 
%We observe that the long CoT contains \textit{five} times reflections, indicating that current o1-like models tend to reflect frequently. 

\paragraph{How Effective Are Model Reflections Across Different Models and Domains?}
We classify samples with reflections based on the number of valid reflections to evaluate the ability to produce valid reflections. Specifically, we label samples as \texttt{0} if no valid reflections occur, and \texttt{1}, \texttt{2}, or \texttt{>=3} for samples with one, two, or three and more valid reflections, respectively(all statistical analyses were performed under strictly controlled conditions, ensuring uniform sampling and balanced tasks for a fair comparison). In Figure \ref{fig: error_type_model}, {DeepSeek-R1} exhibits the highest proportion of effective reflections, and the models show a notably higher rate of effective reflections in the {math} domain. However, the overall proportion of valid reflections across all models remains relatively low, ranging between 30\% and 40\%. This suggests that the reflection capabilities of current models require further improvement.
%Detailed statistical data can be found in Appendix D.

\begin{tcolorbox}[colback=white!95!gray, colframe=gray!70!black,  title=Key Finding for Reflection]
Despite frequent reflection attempts, the proportion of effective reflections remains low across models, and  DeepSeek-R1 achieves the highest rate of valid reflections.
\end{tcolorbox}

\subsection{Effective Reasoning of o1-like Models}

\begin{figure}[t]
    \centering
    \includegraphics[width=0.98\textwidth]{figures/effetive_reasoning.pdf}
    \caption{Distribution of effective reasoning ratios.}
    
    \label{fig: effetive_reasoning}
\end{figure}

\paragraph{Statistics.} 
% As previously mentioned, 
Human annotators evaluate the usefulness of the reasoning in each section, enabling us to calculate the proportion of valid reasoning in each response. As illustrated in Figure \ref{fig: effetive_reasoning}, each graph shows the distribution of effective reasoning ratios for a particular model. The red dashed line in each graph indicates the average effective reasoning ratio.

\paragraph{What Proportion of Reasoning in Long CoT Responses is Effective?}
On average, only 73\% of the reasoning in the collected long CoT responses is useful, highlighting significant redundancy issues. Among the models analyzed, \textit{QwQ-32B-Preview} exhibited the lowest proportion of effective reasoning at 70\%, while \textit{DeepSeek-R1} achieved a notably higher proportion compared to the others, demonstrating superior reasoning efficiency.


\begin{tcolorbox}[colback=white!95!gray, colframe=gray!70!black,  title=Key Finding for Reasoning Efficiency]
On average, 27\% of reasoning in long CoT responses we collected is redundant, and DeepSeek-R1 outperforms others in reasoning efficiency.
\end{tcolorbox}
\vspace{-3mm}

\subsection{Reasoning Process Analysis}

Figure ~\ref{fig: action_roles} shows the distribution of each section's action roles in the system II thinking process of the o1-like models. Initially, problem analysis dominates, indicating that the model initially focuses on understanding the requirements and constraints of the problem. As the solution progresses, cognitive activities diversify significantly, with reflection and validation becoming more prominent. In the later part of the reasoning, the distribution of conclusion and summarization gradually increases. 
%As the model progresses from problem analysis, solution implementation and conclusion, it demonstrates the common reasoning template of o1-like models.


\begin{figure}[t]
    \centering
    \includegraphics[width=0.8\textwidth]{figures/action_role.pdf}
    \caption{Distribution of different task types throughout the progress of a long CoT response.}
    \vspace{-3mm}
    
    \label{fig: action_roles}
\end{figure}
\subsection{Results on DeltaBench}

% Please add the following required packages to your document preamble:
% \usepackage{multirow}
\begin{table*}[!t]
\centering
\resizebox{1.0\textwidth}{!}{%
    \begin{tabular}{cccccccccccccccc}
    \toprule
    \multirow{2}{*}{\textbf{Model}} & \multicolumn{3}{c}{\textbf{Overall}} & \textbf{Math} & \textbf{Code} & \textbf{PCB} & \textbf{General} \\
    \cmidrule(lr){2-4} \cmidrule(lr){5-5} \cmidrule(lr){6-6} \cmidrule(lr){7-7} \cmidrule(lr){8-8}
     & \textbf{\textit{Recall}} & \textbf{\textit{Precision}} & \textbf{\textit{F1}} & \textbf{\textit{F1}} & \textbf{\textit{F1}} & \textbf{\textit{F1}} & \textbf{\textit{F1}} \\
    \midrule
    \multicolumn{8}{c}{\textbf{\textit{Process Reward Models (PRMs)}}} \\
    \midrule
    \rowcolor[rgb]{ .988,  .949,  .8} Qwen2.5-Math-PRM-7B & \textbf{30.30} & \textbf{34.96} & \textbf{29.22}  &  \textbf{29.64} & \textbf{23.76} & \underline{31.09} & \underline{34.19}   \\
    \rowcolor[rgb]{ .988,  .949,  .8} Qwen2.5-Math-PRM-72B & \underline{28.16} & \underline{29.37} & \underline{26.38}  & \underline{24.16} & \underline{22.02} & \textbf{31.14} & \textbf{35.83}  \\
    \rowcolor[rgb]{ .988,  .949,  .8} Llama3.1-8B-PRM-Deepseek-Data & 11.7 & 15.59 & 12.02 &  12.28 & 10.95 & 16.76 & 12.59  \\
    \rowcolor[rgb]{ .988,  .949,  .8} Llama3.1-8B-PRM-Mistral-Data & 9.64 & 11.21 & 9.45 & 9.40 & 10.72 & 13.43 & 12.40  \\
    \rowcolor[rgb]{ .988,  .949,  .8} Skywork-o1-Qwen-2.5-1.5B & 3.32 & 3.84 & 3.07 & 1.30 & 6.66 & 5.43 & 7.87  \\
    \rowcolor[rgb]{ .988,  .949,  .8} Skywork-o1-Qwen-2.5-7B & 2.49 & 2.22 & 2.17 & 0.78 & 6.28 & 6.02 & 3.11  \\
    \midrule
     \multicolumn{8}{c}{\textbf{\textit{LLM as Critic Models}}} \\
    \midrule
    \rowcolor[rgb]{ .922,  .89,  .988} GPT-4-turbo-128k & \textbf{57.19} & \textbf{37.35} & \textbf{40.76} & \textbf{37.56} & \textbf{43.06} & \underline{45.54} & \underline{42.17} \\
    \rowcolor[rgb]{ .922,  .89,  .988} GPT-4o-mini & \underline{49.88} & 35.37 & \underline{37.82} & \underline{33.26} & 37.95 & \textbf{45.98} & \textbf{46.39} \\
    \rowcolor[rgb]{ .922,  .89,  .988} Doubao-1.5-Pro & 39.68 & \underline{37.02} & 35.25 & 32.46 & \underline{39.47} & 33.53 & 37.00 \\
    \rowcolor[rgb]{ .922,  .89,  .988} GPT-4o & 36.52 & 32.48 & 30.85 & 28.61 & 28.53 & 39.25 & 36.50 \\
    \rowcolor[rgb]{ .922,  .89,  .988} Qwen2.5-Max & 36.11 & 30.82 & 30.49 & 26.73 & 32.81 & 39.49 & 29.54 \\
    \rowcolor[rgb]{ .922,  .89,  .988} Gemini-1.5-pro & 35.51 & 30.32 & 29.59 & 26.56 & 28.20 & 40.13 & 33.66 \\
    \rowcolor[rgb]{ .922,  .89,  .988} DeepSeek-V3 & 32.33 & 28.13 & 27.33 & 27.04 & 27.73 & 27.35 & 27.45 \\
    \rowcolor[rgb]{ .922,  .89,  .988} Llama-3.1-70B-Instruct & 32.22 & 28.85 & 27.67 & 21.49 & 32.13 & 28.45 & 39.18 \\
    \rowcolor[rgb]{ .922,  .89,  .988} Qwen2.5-32B-Instruct & 30.12 & 28.63 & 26.73 & 22.34 & 31.37 & 33.78 & 24.37 \\
    \rowcolor[rgb]{ .882,  .949,  .89} DeepSeek-R1 & 29.20 & 32.66 & 28.43 & 24.17 & 29.28 & 34.78 & 35.87 \\
    \rowcolor[rgb]{ .882,  .949,  .89} o1-preview & 27.92 & 30.59 & 26.97 & 22.19 & 28.09 & 33.11 & 35.94 \\
    % Gemini-2.0-flash-thinking & 14.02 & 17.36 & 14.56 & 14.79 & 11.97 & 19.34 & 15.26 \\
    \rowcolor[rgb]{ .922,  .89,  .988} Qwen2.5-14B-Instruct & 26.64 & 27.27 & 24.73 & 21.51 & 29.05 & 29.98 & 20.59 \\
    \rowcolor[rgb]{ .922,  .89,  .988} Llama-3.1-8B-Instruct & 25.71 & 28.01 & 24.91 & 18.12 & 32.17 & 27.30 & 29.93 \\
    \rowcolor[rgb]{ .882,  .949,  .89} o1-mini & 22.90 & 22.90 & 19.89 & 16.71 & 21.70 & 20.37 & 26.94 \\
    \rowcolor[rgb]{ .922,  .89,  .988} Qwen2.5-7B-Instruct & 21.99 & 19.61 & 18.63 & 11.61 & 25.92 & 29.85 & 15.18 \\
    \rowcolor[rgb]{ .882,  .949,  .89} DeepSeek-R1-Distill-Qwen-32B & 17.19 & 18.65 & 16.28 & 13.02 & 23.55 & 15.05 & 11.56 \\
    % Gemini-2.0-flash-thinking & 14.02 & 17.36 & 14.56 & 14.79 & 11.97 & 19.34 & 15.26 \\
    \rowcolor[rgb]{ .882,  .949,  .89} DeepSeek-R1-Distill-Qwen-14B & 12.81 & 14.54 & 12.55 & 9.40 & 18.36 & 10.44 & 12.01 \\
    % \rowcolor[rgb]{ .882,  .949,  .89} QwQ-32B-Preview & 10.20 & 10.17 & 9.07 & 7.38 & 8.60 & 14.97 & 10.54 \\
    \bottomrule
    \end{tabular}
}
\caption{Experimental results of PRMs and critic models on DeltaBench. \textbf{Bold} indicates the best results within the same group of models, while \underline{ underline} indicates the second best.}
% \vspace{-4mm}
\label{tab: main}
\end{table*}

% \noindent\textbf{Evaluation Metrics.}
% % To accurately assess the performance of the PRM and critic models on DeltaBench, 
% We employ \textbf{recall}, \textbf{precision}, and \textbf{macro-F1 score} for error sections as evaluation metrics. For the PRMs, we utilize an outlier detection technique based on the Z-Score to make predictions. This method was chosen because threshold-based prediction methods determined from other step-level datasets, such as those used in ProcessBench~\citep{Zheng2024ProcessBenchIP}, may not be reliable due to significant differences in dataset distributions, particularly as DeltaBench focuses on long CoT. Outlier detection helps to avoid this bias. The threshold $t$ for determining the correctness of a section is defined as:
% % \begin{align}
% $t = \mu - \sigma$,
% % \nonumber
% % \label{eq: prm_threshold}
% % \end{align}
% where $\mu$ is the mean of the rewards distribution across the dataset, and $\sigma$ is the standard deviation. Sections falling below $t$ are predicted as error sections. For critic models, all erroneous sections within a long CoT are prompted to be identified. Given that error sections constitute a smaller proportion than correct sections across the dataset, we use macro-F1 to mitigate the potential impact of the imbalance between positive and negative sections. Macro-F1 independently calculates the F1 score for each sample
% % (for our metric, each case) 
% and then takes the average, providing a more balanced evaluation metric when dealing with class imbalance.

\noindent\textbf{Baseline Models.}
% 开源(中英模型,llama3)和闭源模型
% To comprehensively evaluate the performance of current PRMs and critic models, we extensively selected and evaluated a wide range of both open-source and closed-source models on DeltaBench.
% \paragraph{Process Reward Models}
For the \textbf{PRMs}, we select the following models: Qwen2.5-Math-PRM-7B\footnote{\href{https://huggingface.co/Qwen/Qwen2.5-Math-PRM-7B}{Qwen/Qwen2.5-Math-PRM-7B}}, Qwen2.5-Math-PRM-72B\footnote{\href{https://huggingface.co/Qwen/Qwen2.5-Math-PRM-72B}{Qwen/Qwen2.5-Math-PRM-72B}}, Llama3.1-8B-PRM-Deepseek-Data\footnote{\href{https://huggingface.co/RLHFlow/Llama3.1-8B-PRM-Deepseek-Data}{RLHFlow/Llama3.1-8B-PRM-Deepseek-Data}}, Llama3.1 -8B-PRM-Mistral-Data\footnote{\href{https://huggingface.co/RLHFlow/Llama3.1-8B-PRM-Mistral-Data}{RLHFlow/Llama3.1-8B-PRM-Mistral-Data}}, Skywork-o1-Open-PRM- Qwen-2.5-1.5B\footnote{\href{https://huggingface.co/Skywork/Skywork-o1-Open-PRM-Qwen-2.5-1.5B}{Skywork/Skywork-o1-Open-PRM-Qwen-2.5-1.5B}}, and Skywork-o1-Open-PRM-Qwen-2.5-7B\footnote{\href{https://huggingface.co/Skywork/Skywork-o1-Open-PRM-Qwen-2.5-7B}{Skywork/Skywork-o1-Open-PRM-Qwen-2.5-7B}}. 
% These represent some of the best open-source PRMs currently available.
% \paragraph{Critic Models}
We select a group of the most advanced open-source and closed-source LLMs to serve as \textbf{critic models} for evaluation, which includes various GPT-4~\citep{gpt4} variants (such as GPT-4-turbo-128K, GPT-4o-mini, GPT-4o), the Gemini model~\citep{Reid2024Gemini1U}(Gemini-1.5-pro), several Qwen models~\citep{qwen2.5} (such as Qwen2.5-32B-Instruct and Qwen2.5-14B-Instruct), Doubao-1.5-Pro~\citep{doubao2025}
and o1 models~\citep{openai-o1} (o1-preview-0912, o1-mini-0912).
% , and a GPT-3.5 variant (gpt-3.5-16K).



\subsubsection{Main Results}
In Table \ref{tab: main},
we provide the results of different LLMs on DeltaBench. 
For PRMs, we have the following observations: (1). Existing PRMs usually achieve low performance, which indicates that existing PRMs cannot identify the errors in long CoTs effectively and it is necessary to improve the performance of PRMs. (2). Larger PRMs
do not lead to better performance. For example, the Qwen2.5-Math-PRM-72B is inferior to wen2.5-Math-PRM-7B.
For critic models, we have the following findings: (1)
GPT-4-turbo-128k archives the best critique results, which is better than other models (e.g., GPT-4o) a lot in DeltaBench. (2) For o1-like models (e.g., DeepSeek-R1, o1-mini, o1-preview), we observe that the results of these models are not superior to non-o1-like models, with the performance of o1-preview is even lower than Qwen2.5-32B-Instruct.
%Additionally, we observe that the QWQ and DeepSeek-R1-Distill series models exhibit weaknesses in following instructions. 
A detailed analysis of underperforming models is provided in Appendix \ref{app: underperforming}.

% model size
% domains
% o1模型跟普通模型critic能力对比分析


\subsubsection{Further Analysis}

\paragraph{Effect of Long CoT Length.}
\begin{figure}[t]
    \centering
    \includegraphics[width=1.0\textwidth]{figures/4.5.1/length2.pdf}
    \caption{The effect of long CoT length.}
    \label{fig: crtic1}
\end{figure}
In Figure \ref{fig: crtic1}, we compare the average F1-Score performance of critic models and PRMs across varying LongCoT token lengths. 
For critic models, the performance notably declines as token length increases. Initially, models like Deepseek-R1 and GPT-4o exhibit strong performance with shorter sequences (1-3k tokens). However, as token length increases to mid-ranges (4-7k tokens), there is a marked decrease in performance across all models. This trend highlights the growing difficulty for critic models to maintain precision and recall as long CoT response become longer and more complex, likely due to the challenge of evaluating lengthy model outputs. In contrast, PRMs demonstrate greater stability across token lengths, as they evaluate sections sequentially rather than processing the entire output at once. Despite this advantage, PRMs achieve lower overall scores compared to critic models on our evaluation set.

\begin{tcolorbox}[colback=white!95!gray, colframe=gray!70!black, title=Key Finding]
  Critic models exhibit significant performance degradation with longer contexts, while PRMs demonstrate consistent evaluation capability across varying lengths.
\end{tcolorbox}


\paragraph{Performance Analysis Across Different Error Types.}
\begin{figure}[t]
    \centering
    \includegraphics[width=0.9\textwidth]{figures/4.5.2/top_models_per_task.pdf}
    \caption{Results of different LLMs on top-5 errors.}
    \label{fig: top_models_per_task}
\end{figure}
Figure \ref{fig: top_models_per_task} shows the performance of different models on the five most common error types. In terms of error types, most models demonstrate the highest accuracy in recognizing calculation errors. Conversely, the recognition of strategy errors is generally the weakest. In terms of models, there is significant variation in the ability of individual models to recognize different error types. For instance, DeepSeek-V3 achieves an F1 of 36\% on calculation errors but only 23\% on strategy errors. Meanwhile, Llama3.1-8B-PRM-Deepseek performs poorly, with an F1 score of 22\% on calculation errors, and shows a significant decline in performance across the other four error types. This highlights the limited generalization capabilities of most models when recognizing various error types.

\begin{tcolorbox}[colback=white!95!gray, colframe=gray!70!black, title=Key Finding]
  Models exhibit strong performance on calculation errors but struggle with strategy errors, revealing limited generalization across error types.
\end{tcolorbox}

\begin{table}[!ht]
    \centering
    % \scriptsize
    % \footnotesize
    \begin{tabular}{cccc}
    \toprule
        \multirow{2}{*}{Model} & \multicolumn{3}{c}{HitRate@$k$ - Avg(\%)} \\ \cline{2-4}
                           & $k=1$ & $k=3$ & $k=5$ \\ 
                           % \hline
                           \midrule
        Qwen2.5-Math-PRM-7B & \textbf{49.15} & \textbf{69.14} & \textbf{83.14} \\
        Qwen2.5-Math-PRM-72B & \underline{41.13} & \underline{62.70} & \underline{75.73} \\ 
        Llama3.1-8B-PRM-Deepseek-Data & 12.63 & 48.62 & 69.78 \\
        Llama3.1-8B-PRM-Mistral-Data & 8.99 & 42.97 & 65.33 \\
        Skywork-o1-Open-PRM-Qwen-2.5-1.5B & 31.90 & 53.82 & 69.23 \\
        Skywork-o1-Open-PRM-Qwen-2.5-7B & 31.58 & 52.59 & 69.16 \\
        % \hline
        \bottomrule
    \end{tabular}
    \vspace{+3mm}
    \caption{Results of HitRate@$k$. Bold and underlined results indicate the best and the second best.}
    % \vspace{-4mm}
\label{tab: hitrate}
\end{table}

\paragraph{Analysis on HitRate evaluation for PRMs.}

\begin{figure}[t]
    \centering
    \includegraphics[width=\textwidth]{figures/prm_rank.pdf}
    % \vspace{-10pt}
    \caption{Ranking of rewards for the first incorrect section for different PRMs.}
    % \vspace{-3mm}
    \label{fig: prm_rank}
\end{figure}

To better measure the ability of PRMs to identify erroneous sections in long CoTs, we use HitRate@$k$ to evaluate PRMs. Specifically, within a sample, we rank the sections in ascending order based on the rewards given by the PRM, select the smallest $k$ sections, and calculate the recall rate for the erroneous sections among them. Specifically, we define the sorted sections as $S = \{s_1, s_2, \ldots, s_n\}$, with $E$ being the set of erroneous sections. We select the top $k$ sections, denoted as $S_k = \{s_1, s_2, \ldots, s_k\}$. The HitRate@$k$ is  calculated as:
\begin{align}
\text{HitRate@}k = \frac{|S_k \cap E|}{\min(k, |E|)}
% \nonumber
\label{eq: hitrate}
\end{align}
In this formula, $|S_k \cap E|$ indicates the number of erroneous sections identified among the top $k$ sections. This metric reflects the ability of PRMs to effectively identify erroneous sections within the top $k$ candidate sections. In Table \ref{tab: hitrate}, the relative performance rankings among different PRMs are quite similar to the results in Table \ref{tab: main}. Additionally, we observe that for $k=3$ and $k=5$, the performance differences between various PRMs are not particularly significant. However, when $k=1$, the Qwen2.5-Math-PRM-7B shows a clear performance advantage. Figure \ref{fig: prm_rank} illustrates the ranking ability of different PRMs for the first incorrect section within the sample, which is generally consistent with the performance evaluation results of HitRate@k.
% This is because a smaller $k$ value imposes stricter requirements on the PRM's ability to identify errors.

% HitRate@$k$ evaluates the performance of PRMs from the perspective of reward ranking, providing additional evidence for the experimental results and conclusions in Table \ref{tab: main} from a different angle.

\begin{tcolorbox}[colback=white!95!gray, colframe=gray!70!black, title=Key Finding]
  HitRate@k evaluation aligns with the main results, with Qwen2.5-Math-PRM-7B demonstrating superior performance in identifying the first incorrect section.
\end{tcolorbox}


\begin{figure}[t]
    \centering
    \includegraphics[width=0.8\textwidth]{figures/4.5.4/self-critic.pdf}
    % \vspace{-10pt}
    \caption{F1-score comparison of self-critique and cross-model critique abilities for different models.}
    % \vspace{-5mm}
    \label{fig: self-critic}
\end{figure}

\paragraph{Comparative Analysis of Self-Critique Capabilities of LLMs.} We randomly sample queries based on domains and models that generate the long CoT output, followed by a statistical analysis of the model's performance in evaluating its own outputs as well as those of other models. In Figure \ref{fig: self-critic},  Gemini 2.0 Flash Thinking, DeepSeek-R1, and QwQ-32B-Preview show lower self-critique scores compared to their cross-model critique scores, indicating a prevalent deficiency in self-critic abilities. Notably, DeepSeek-R1 exhibits the largest discrepancy, with a 36\% decrease in self-evaluation compared to evaluations of other models. This suggests models' self-critic abilities remain underdeveloped.
% signaling an area that requires improvement.

\begin{tcolorbox}[colback=white!95!gray, colframe=gray!70!black, title=Key Finding]
  LLMs demonstrate weaker self-critique performance compared to cross-model critique, highlighting a fundamental limitation in self-critic capabilities.
\end{tcolorbox}



%%%

% \noindent\textbf{Performance Analysis Across Different Categories}

% \begin{figure}[htbp]
% \centering
% \includegraphics[width=\linewidth]{figures/prm_task_comparison.pdf}
% \caption{Performance of PRMs across different categories (outlier detection).}
% \label{fig: prm_task}
% % \vspace{-0.6cm}
% % \vspace{-4mm}
% \end{figure}


% \noindent\textbf{Performance Variation in Different Lengths of Long CoT}

% \noindent\textbf{Performance Analysis Across Different Error Types}

% \noindent\textbf{Analysis of In-Sample Reward Ranking}


% % \subsection{Evaluation Metrics}

% % \subsection{Main Results}

% % \subsection{Further Analysis}
% \subsection{Analysis on LLM Critics}
%  \textbf{error location}



% \subsubsection{The Performance across different domains}

% \begin{figure}[t]
%     \centering
%     \includegraphics[width=0.5\textwidth]{figures/critic6.pdf}
%     \caption{The score distributions across different domains.}
%     \label{fig: crtic2}
% \end{figure}

% In Figure \ref{fig: crtic2}, we illustrate the F1-score distribution of various large language models (LLMs) across different domains. Analyzing model performance across domains reveals that most models demonstrate stronger critiquing abilities in Physics, Chemistry, Biology, and General Reasoning compared to Mathematics and Programming, indicating higher proficiency in scientific and general reasoning tasks. Meanwhile, the performance of each model varies significantly depending on the domain, reflecting inherent strengths and weaknesses in handling different tasks. For instance, the Gemini-1.5-Pro model achieves an F1-score of 40.1\% in PCB, yet only 26.6\% in Mathematics. These discrepancies underscore challenges in the models' generalization capabilities.






\section{Conclusion}
\label{subsection:conclusion}
In this paper, we introduce \OURS, a novel framework designed to identify high-quality data that aligns well with the LLM’s learned knowledge to reduce hallucination.
% Our proposed method includes Internal Consistency Probing and Semantic Equivalence Identification, which are designed to separately measure the LLM's understanding of the given instruction and target response.
% In this way, we can measure the familiarity of the LLM with the instruction data and prevent the model from being trained on unfamiliar data, thereby reducing hallucinations.
NOVA includes Internal Consistency Probing and Semantic Equivalence Identification, which are designed to separately measure the LLM's familiarity with the given instruction and target response, then prevent the model from being trained on unfamiliar data, thereby reducing hallucinations.
Lastly, we introduce an expert-aligned reward model, considering characteristics beyond just familiarity to enhance data quality.
By considering data quality and avoiding unfamiliar data, we can use the selected data to effectively align LLMs to follow instructions and hallucinate less in the instruction tuning stage.
Experiments and analysis show the effectiveness of \OURS.

\section*{Limitations}
Although empirical experiments have confirmed the effectiveness of the proposed \OURS, two major limitations remain. 
Firstly, our proposed method requires LLMs to generate multiple responses for the given instruction, which introduces additional execution time.
However, it is worth noting that this additional execution time is used to perform offline data filtering, our proposed method does not introduce additional time overhead in the inference phase.
Additionally, \OURS~is primarily used for single-turn instruction data filtering, thus exploring its application in multi-turn scenarios presents an attractive direction for future research.

% \bibliographystyle{elsarticle-num} % \cite
% \bibliographystyle{plain} % \cite
\bibliographystyle{plainnat} % \cite
\bibliography{main}
%%%%%%%%%%%%%%%%%%%%%%%%%%%%%%%%%%%%%%%%%%%%%%%%%%%%%%%%%%%%

\appendix
\appendix
\onecolumn
% \section{You \emph{can} have an appendix here.}

% You can have as much text here as you want. The main body must be at most 88 pages long.
% For the final version, one more page can be added.
% If you want, you can use an appendix like this one.  

% The ∖onecolumn\mathtt{\backslash onecolumn} command above can be kept in place if you prefer a one-column appendix, or can be removed if you prefer a two-column appendix.  Apart from this possible change, the style (font size, spacing, margins, page numbering, etc.) should be kept the same as the main body.
% %%%%%%%%%%%%%%%%%%%%%%%%%%%%%%%%%%%%%%%%%%%%%%%%%%%%%%%%%%%%%%%%%%%%%%%%%%%%%%%
% %%%%%%%%%%%%%%%%%%%%%%%%%%%%%%%%%%%%%%%%%%%%%%%%%%%%%%%%%%%%%%%%%%%%%%%%%%%%%%%

\section{Details of Method Comparison}
\label{app:comparison}
In this section, we provide detailed explanations of the configurations used in our experiments for comparison, including the routing mechanisms, optimization algorithms, and evaluation metrics.

\subsection{Routing Mechanisms and Optimization Algorithms}

\textbf{Classifier-Based (CL) Routing:} We replaced our Knowledge-Aware (KA) routing mechanism with a classifier-based routing (CL) approach. The CL mechanism uses Sentence-BERT to encode both the instruction and the expert’s knowledge concept into vector representations. Cosine similarity is then calculated between these vectors, and the expert with the highest similarity score is selected.

\textbf{Proximal Policy Optimization (PPO):} A reinforcement learning algorithm that updates policies in a stable and efficient manner. It was applied to optimize expert selection by training a policy network to maximize routing performance.

\textbf{Monte Carlo Tree Search (MCTS):} MCTS is employed to explore potential expert selections by simulating multiple decision paths and backpropagating scores from the outcomes. This algorithm is particularly useful for decision-making in environments with large search spaces.

\textbf{Advantage Actor-Critic (A2C):} A2C combines the actor-critic framework with an advantage function to improve policy updates. The actor selects experts, while the critic evaluates the quality of these decisions, enabling more efficient learning.

\subsection{Metrics to Evaluate Routing Quality}

We provide detailed definitions and formulations for the two metrics used to evaluate the performance of the routing strategies: \textbf{Routing Alignment Score (RAS)} and \textbf{Preference-Weighted Routing Score (PWRS)}.

\subsubsection{Routing Alignment Score (RAS)}

The Routing Alignment Score (RAS) measures the degree to which the router's expert selection aligns with human expert annotations. It quantifies the consistency between the router's decisions and the ground truth labels provided by human annotators. 

\begin{equation}
\text{RAS} = \frac{C}{N}
\end{equation}

where $C$ denotes the number of routed experts that align with human preferences and $N$ denotes the total number of routed experts (in this case: $805 \times 2$).


\paragraph*{Human Evaluation Protocol} 
To establish reliable ground truth labels, we engaged a panel of 7 domain experts with 3+ years of experience in AI system evaluation. Each expert independently annotated 1,610 routing instances (805 instruction-expert pairs $\times$ 2 routing paths) through a two-phase process:
\begin{itemize}
    \item \textbf{Calibration Phase}: Experts jointly reviewed 200 samples to establish annotation guidelines and resolve edge cases.
    \item \textbf{Final Annotation}: The remaining 1,410 instances were randomly distributed (200 instances per expert) with 10\% overlap for inter-annotator agreement calculation. 
\end{itemize}
We achieved substantial agreement with Fleiss' $\kappa=0.78$, calculated on the overlapping samples. Final labels were determined through majority voting.

The RAS provides a basic measure of alignment between the router's decisions and the ground truth, reflecting the accuracy of the routing mechanism in selecting the most appropriate experts.

\subsubsection{Preference-Weighted Routing Score (PWRS)}
The Preference-Weighted Routing Score (PWRS) extends traditional routing accuracy metrics by incorporating human preference scores derived from the AlpacaEval 2.0 evaluation framework. This metric weights routing decisions based on the quality of the expert outputs as judged by human evaluators. The PWRS is defined as follows:

\begin{equation}
\text{PWRS} = \frac{\sum_{i=1}^{N} (p_i \cdot c_i)}{N}
\end{equation}

where $p_i$ represents the preference score from AlpacaEval 2.0 for the routed expert's output, $c_i$ is the number of routed experts that align with human preferences, and $N$ denotes the total number of routed experts.

\paragraph*{Preference Score Integration} 
The AlpacaEval 2.0 scores were obtained from a separate group of 15 crowdworkers following the standardized evaluation protocol. Each output was rated by 3 distinct evaluators using a 7-point Likert scale across three dimensions: helpfulness (actionable solutions), accuracy (factual grounding), and coherence (logical flow). Discrepancies exceeding 2 points triggered expert review, with final scores normalized using Bradley-Terry pairwise comparison models. These preference scores enable the PWRS to transcend binary routing accuracy by weighting decisions according to the relative quality of expert outputs, where higher weights correspond to outputs demonstrating stronger alignment with human-judged quality dimensions.

The PWRS thus provides a dual-aspect evaluation: it preserves the fundamental routing correctness measurement through expert selection alignment, while simultaneously quantifying the performance advantage gained through preference-aware routing decisions.

\section{Supplementary Experimental Validation and Analysis}

\subsection{Performance Evaluation}
In order to perform a comprehensive and controlled performance evaluation, we selected two representative tasks from the BIG-bench Hard (BBH) dataset: commonsense reasoning (550 samples) and logical reasoning (600 samples). The reasons for choosing these two tasks are: (1) they effectively validate the core capabilities of the model; (2) they have clear evaluation criteria; (3) the sample size is moderate, which facilitates sufficient multi-round cross-validation. In this experiment, we compare KABB with MoA and its lightweight version MoA-lite. Three key metrics were used for evaluation: (1) Knowledge matching F1 score, computed using BERT to calculate the semantic similarity between expert capabilities and knowledge graph concepts (threshold of 0.75); (2) Path prediction accuracy, based on standard knowledge dependency paths, with a perfect match scoring full points, a path length difference of \text{$\leq$} 1 and key node matches scoring 0.5 points; (3) Historical performance prediction accuracy, using the dynamic weight $\alpha / (\alpha + \beta)$ (where $\alpha$ and $\beta$ represent the number of successful and failed tasks, respectively), with a prediction error \text{$\leq$} 0.1 considered correct. The experimental results are shown in Table 3:

The performance of the three models on key metrics is as follows:

\begin{center}
\begin{tabular}{cccccc}
\hline
Evaluation Metric & KABB & MoA & MoA-lite & vs. MoA & vs. lite \\
\hline
Knowledge Matching F1 (\%) & 86.5 & 71.2 & 46.8 & +15.3\% & +39.7\% \\
Path Prediction Accuracy (\%) & 84.9 & 69.5 & 44.2 & +15.4\% & +40.7\% \\
Historical Performance Prediction (\%) & 85.2 & 70.1 & 45.5 & +15.1\% & +39.7\% \\
\hline
\end{tabular}
\end{center}

The experimental results show that KABB significantly outperforms the baseline models on all key metrics. Compared to the standard MoA, KABB shows an average improvement of 15.3\% across all indicators; compared to the lightweight MoA-lite, the improvement reaches 40\%. This performance enhancement is primarily attributed to the knowledge-aware attention mechanism and dynamic path prediction strategy that we proposed. Notably, KABB exhibits stronger generalization ability in the commonsense reasoning task, validating the effectiveness of our knowledge-enhanced approach.

\subsection{Parameter Sensitivity Analysis}

This section explores the impact of three key parameters in the KABB framework—knowledge distance threshold, time decay factor, and efficiency metric—on system performance. The experiment uses the BBH dataset (commonsense reasoning 580 samples, logical reasoning 570 samples), with standard MoA and MoA-lite as baselines, and evaluates parameter sensitivity using a controlled variable approach. The evaluation metrics used are: knowledge matching F1 score, reasoning accuracy, and response efficiency. The experiment tests different values for the knowledge distance threshold [0.55-0.95] and time decay factor [0.2-1.0].

\subsubsection{Knowledge Distance Threshold}

\begin{center}
\begin{tabular}{cccc}
\toprule
Parameter Value & Knowledge Matching F1 (\%) & Reasoning Accuracy (\%) & Efficiency Metric (\%) \\
\midrule
0.55 & 72.3 & 74.8 & 68.2 \\
0.65 & 83.8 & 85.4 & 79.5 \\
\textbf{0.75} & \textbf{94.9} & \textbf{94.9} & \textbf{92.8} \\
0.85 & 87.5 & 88.2 & 84.3 \\
0.95 & 78.7 & 82.7 & 73.6 \\
\bottomrule
\end{tabular}
\end{center}

\textbf{Analysis}: When the threshold is set to 0.75, the system achieves the highest values in knowledge matching F1 score, reasoning accuracy, and efficiency metric, reaching 94.9\%, 94.9\%, and 92.8\%, respectively. A lower threshold (e.g., 0.55) introduces too many irrelevant experts, leading to a decline in knowledge matching and reasoning accuracy, while a higher threshold (e.g., 0.95) makes the expert selection too strict, reducing system coverage and efficiency.

\subsubsection{Time Decay Factor}

\begin{center}
\begin{tabular}{cccc}
\hline
Parameter Value & Knowledge Matching F1 (\%) & Reasoning Accuracy (\%) & Efficiency Metric (\%) \\
\hline
0.2 & 75.1 & 78.3 & 71.4 \\
0.4 & 85.4 & 87.2 & 82.6 \\
\textbf{0.6} & \textbf{94.9} & \textbf{94.9} & \textbf{92.8} \\
0.8 & 88.2 & 90.3 & 85.7 \\
1.0 & 82.7 & 86.5 & 78.9 \\
\hline
\end{tabular}
\end{center}

\textbf{Analysis}: When the time decay factor is set to 0.6, the system performs optimally across all metrics, indicating a good balance between utilizing historical experience and dynamic adaptability. A smaller factor (e.g., 0.2) makes the system overly dependent on short-term fluctuations, reducing stability, while a larger factor (e.g., 1.0) suppresses adaptability to recent performance.

\section{Effect of the Number of Selected Concepts and Experts.}

% \textbf{Concept and Expert Quantity Analysis} 

Our empirical analysis of KABB's architectural configurations reveals the critical interplay between the number of selected concepts and experts (see \cref{num}). The results demonstrate that performance varies substantially across different configurations, with win rates ranging from 56\% to 81\%. Notably, a configuration of 2 concepts with 3 experts achieves optimal performance under constrained computational resources, while expanding to 3 concepts with 6 experts yields the highest observed win rate of 81\%.

Our findings indicate that configurations utilizing 3 or more experts, combined with a moderate-to-large concept space, consistently outperform alternatives. This suggests that both the expert capacity and the conceptual representation space play crucial roles in determining system effectiveness. Interestingly, the relationship between expert count and performance exhibits non-linear characteristics - configurations with moderate numbers of experts (3-6) already achieve robust performance levels, suggesting efficient utilization of multi-expert collaboration. This observation has important implications for resource-performance optimization in practical deployments.


% Experimental results in \cref{num} demonstrate that KABB's performance is highly sensitive to both the number of concepts and experts. Notably, configurations with 2 concepts and 3 experts achieve the highest win rate under low expert costs, while 3 concepts and 6 experts yield the highest overall win rate of 81\%. This maximum win rate significantly surpasses the lowest observed rate of 56\% across all tested configurations. In general, setups with 3 to 5 experts and moderate-to-high numbers of concepts consistently deliver superior performance, whereas others tend to underperform. These findings highlight the critical role of expert and concept quantities in determining outcomes. Furthermore, the results underscore that KABB leverages multi-expert collaboration to improve performance in a non-linear manner, achieving competitive results even with a moderate number of experts. Performance, however, remains configuration-dependent, emphasizing the importance of careful tuning.


\begin{figure}[h]
% \vskip 0.2in
\begin{center}
\centerline{\includegraphics[width=0.5\linewidth]{num3.pdf}}
\caption{Relationship between the number of selected experts and selected concepts, and the AlpacaEval 2.0 LC Win Rate.}
\label{num}
\end{center}
\vskip -0.2in
\end{figure}


\section{Evaluations on Reasoning and Problem-Solving Tasks}
\label{app:reasoning}

\subsection{Benchmarks}
 For reasoning and problem-solving tasks, We evaluate using three benchmarks: BBH \cite{suzgun2022challenging}, MATH \cite{hendrycks2021measuring}, and Arena-Hard \cite{zheng2023judging}.
 
\textbf{BBH (Big-Bench Hard)} is a challenging subset of the BIG-Bench benchmark that tests advanced reasoning capabilities. Includes diverse tasks in mathematical reasoning, logical deduction, and commonsense inference, evaluating models' generalization and complex problem-solving abilities.

\textbf{MATH} is a specialized assessment for AI mathematical capabilities. Features competition-level problems across algebra, number theory, combinatorics, and geometry. Includes detailed solutions for comprehensive evaluation of reasoning depth and computational accuracy.

\textbf{Arena-Hard} is a collection of 500 challenging problems from public leaderboards and research papers, covering programming, mathematics, and logical reasoning. 

\subsection{Experiment Setup}
For BBH and MATH, we designated LLaMa-3-70B-Instruct and Qwen2-72B-Instruct as the experts and Qwen2-72B-Instruct as the aggregator to construct a simple but effective multi-agent system, with one concept and one expect selected for instruction. 

For Arena-Hard, we use the default configuration of KABB with the six open-source models (see \cref{sec:setup}). Additionally, we evaluate KABB w/o Deepseek and KABB-Single-LLaMa3. All models are evaluated under a controlled environment with fixed hyperparameters to ensure fairness.

\subsection{Results and Analysis}
\begin{table}[H]
    \centering
    \small
    \begin{tabular}{lcc}
        \toprule
        Model & BBH & MATH \\
        \midrule
        \textbf{KABB} & \textbf{84.2} & \textbf{59.8} \\
        MoA & 81.8 & 57.3 \\
        Qwen2-72B-Instruct & 82.4 & 51.1 \\
        LLaMa-3-70B-Instruct & 81.0 & 42.5 \\
        \bottomrule
    \end{tabular}
    \caption{Performance comparison on BBH and MATH benchmarks.}
    \label{tab:bbh_math}
\end{table}

Table \ref{tab:bbh_math} presents the performance of KABB and baseline models on the BBH and MATH benchmarks. KABB achieves the highest performance on both benchmarks, surpassing MoA by +2.4\% on BBH and +2.5\% on MATH. The significant gain on MATH highlights the effectiveness of our structured multi-agent approach in handling complex mathematical reasoning tasks.

Table \ref{tab:arena_hard_results} reports model performance on the Arena-Hard benchmark. KABB demonstrates competitive performance (74.8\%) but falls behind GPT-4 models in this benchmark. The Deepseek-R1 model achieves the highest score (92.3\%), indicating its strong generalization capabilities. The KABB-Single-LLaMa3 outperforms Single LLaMa-3-70B-Instruct by 4.8\%. Removing Deepseek models (KABB w/o Deepseek) significantly reduces performance (-12.0\%), confirming their critical role in the system. 

It is noteworthy that MoA achieved a similar performance to ours. In the context of well-defined problem-solving tasks (such as programming and mathematical problem-solving), empirical evidence suggests that multi-agent architectures may encounter specific limitations. The integration of multiple agents can potentially introduce operational redundancies and decisional interference, which may adversely impact the system's capacity to converge on correct solutions or generate optimal outputs. This presents a notable challenge in domains where problem spaces are closed and solutions are deterministic. \cref{sec:case} includes a case when some models produce low-quality answers on Arena-Hard.

\begin{table}[t]
    \centering
    \small
    \sisetup{
        table-format=2.1,
        separate-uncertainty=true,
        detect-weight=true,
        detect-inline-weight=math
    }
    \begin{tabular}{lc}
        \toprule
        Model & \textbf{Arena-Hard win. (\%)} \\
        \midrule
        \rowcolor[gray]{0.95} KABB & 74.8 \\
        MoA & 74.3 \\
        \rowcolor[gray]{0.95} KABB w/o Deepseek & 62.8 \\
        GPT-4 Omni (05/13) & 79.2 \\
        GPT-4 Turbo (04/09) & 82.0 \\
        GPT-4 Preview (11/06) & 78.7 \\
        GPT-4 (03/14) & 50.0 \\
        Qwen2-72B-Instruct & 46.9 \\
        Gemma-2-27B & 57.5 \\
        WizardLM-2-8x22B & 71.3 \\
        \rowcolor[gray]{0.95} KABB-Single-LLaMa3 & 51.4 \\
        LLaMa-3-70B-Instruct & 46.6 \\
        Deepseek-V3 & \underline{85.5} \\
        Deepseek-R1 & \textbf{92.3} \\
        \bottomrule
    \end{tabular}
    \caption{Arena-Hard benchmark results for different models. Performance data for GPT series, LLaMA, and WizardLM comes from \cite{wang2024mixture}, DeepSeek models from their technical reports \cite{guo2025deepseek,liu2024deepseek}, and other models from public leaderboards.}
    \label{tab:arena_hard_results}
\end{table}


\section{Case Study}
\label{sec:case}
We present a case study in this section to analyze how are the different experts and models are selected, and how different experts and models generate responses. For clarity of comparison, we use KABB w/o Deepseek and set the number of selected experts as four. We report the score of their intermediate outputs as well as the final response. Due to the length of the responses, we have selected key fragments for clarity and brevity. To illustrate how the aggregator synthesizes the final output, we highlight similar expressions between the proposed responses and the aggregated response using underlined text in different colors.

\cref{case-good} showcases the responses generated by four selected experts, along with the final aggregated response provided by the aggregator model, Qwen2-72B-Instruct. Two of the experts' responses got a high preference score over 0.99, which demonstrates that MABB succeeded in selecting qualified experts. The aggregated response achieves the highest preference score, reflecting a well-balanced synthesis of key elements from all proposers. The aggregated output successfully combines the most relevant and salient points from all proposed responses, demonstrating the aggregator's ability to synthesize diverse perspectives into a cohesive and comprehensive answer. This process highlights the collaborative nature of the models and their collective contribution to generating high-quality answers.

To be specific, the selected experts—Interaction Analyst, Dialogue Specialist, Humanities Scholar, and Cultural Interpreter—bring distinctive perspectives and areas of specialization, which collectively contribute to the richness and depth of the final aggregated output. The Interaction Analyst ensures factual accuracy and provides foundational details, while the Dialogue Specialist focuses on clarity and narrative flow, making the response accessible to a broad audience. The Humanities Scholar adds historical and cultural context, enriching the response with connections to societal trends, and the Cultural Interpreter offers reflective insights, emphasizing the sociocultural dynamics behind Superman's creation. By combining these complementary perspectives, the aggregator produces a response that balances factual precision, narrative coherence, cultural depth, and interpretive richness. This selection of experts ensures a multidimensional and high-quality final response.

\cref{case-bad} highlights a challenge in incorporating multiple experts for response generation: although diverse perspectives can broaden the scope of the output, they risk diluting the core information with excessive and redundant details. In this case, the inclusion of too many experts led to a loss of focus and reduced the practicality of the final response, despite offering a more expansive view of the topic. The selected experts each contributed their specialized perspectives. However, this diversity introduced significant overlap and irrelevant details. As a result, the aggregated response, though comprehensive, lacked the specificity and clarity needed for practical implementation. This case underscores the importance of carefully curating expert involvement based on the specific requirements of the task. For highly technical prompts, prioritizing experts with deep implementation knowledge and minimizing the number of experts is essential to ensure clarity, focus, and actionable results.

\setlength{\tabcolsep}{5pt} % Adjust column separation
\renewcommand{\arraystretch}{1.5} % Adjust row spacing
% \vspace{1em} % Add some vertical space between the user input and the table
% Centering the table and ensuring it stays on one page
\begin{table}[t]
\centering
\caption{Case: Some models produce high-quality answers on AlpacaEval 2.0}
\label{case-good}
% Add the user input in a tcolorbox for clarity
% \begin{tcolorbox}[colback=blue!5!white, colframe=blue!75!black, title=User Input]
% Who created the Superman cartoon character?
% \end{tcolorbox}
\begin{tabular}{@{}p{17cm}@{}}
\toprule
\textbf{User Prompt}  \\ 
\midrule
Who created the Superman cartoon character?
\end{tabular}
\begin{adjustbox}{max width=\linewidth}
\begin{tabular}{@{}p{2cm}p{4cm}p{2cm}p{8cm}@{}}
\toprule
\textbf{Expert} & \textbf{Model} & \textbf{Preference} & \textbf{Output} \\ 
\midrule
Interaction \newline Analyst & LLaMa-3-70B-Instruct & 0.000112313 & 
Superman, \colorul[blue]{the iconic cartoon character}, was created by \colorul[blue]{writer Jerry Siegel and artist Joe Shuster}. They were both \colorul[blue]{high school students in Cleveland, Ohio,} when they \colorul[blue]{came up with the idea for the character in 1933.}\ldots \\ 
% \midrule
Dialogue \newline Specialist & Qwen2-72B-Instruct & 0.9998820091 & 
\ldots They created Superman in 1933, and the character \colorul[green]{first appeared in print in Action Comics \#1 in June 1938, published by DC Comics.} \ldots \\ 
% \midrule
Humanities \newline Scholar & WizardLM-2-8x22B & 0.9999975912 & 
\ldots Superman quickly became a \colorul[orange]{cultural icon} and is often credited with helping to create the superhero genre as we know it today. \newline The character's success led to various adaptations across multiple media, including radio serials, \colorul[orange]{television shows, movies, and—of course—animated cartoons.} \ldots \\ 
% \midrule
Cultural \newline Interpreter & Gemma-2-27B-it & 0.2356933747 & 
\ldots \colorul[red]{It's a fascinating story} that highlights how cultural context and \colorul[red]{collaboration} shape creative works. \ldots \colorul[red]{If you'd like to} delve deeper into any specific aspect of Superman's creation, such as the social anxieties of the Great Depression reflected in the character, or the evolution of his powers and morality over time, \colorul[red]{feel free to ask!} \\ 
\midrule
Aggregator & Qwen2-72B-Instruct & \textbf{0.9999988203} & 
Superman, \colorul[blue]{the iconic cartoon character}, was created by \colorul[blue]{writer Jerry Siegel and artist Joe Shuster}. The duo, \colorul[blue]{who were high school students in Cleveland, Ohio}, \colorul[blue]{came up with the idea for the character in 1933} \ldots \colorul[green]{Superman first appeared in print in Action Comics \#1 in June 1938, published by National Allied Publications, which later became DC Comics.} \newline Siegel and Shuster's creation \colorul[orange]{quickly became a cultural icon}, leading to various adaptations in \colorul[orange]{comic books, TV shows, movies, and animated cartoons.} \ldots \newline Superman's creation is \colorul[red]{a story of collaboration}, influenced by the cultural context of the 1930s. \ldots \colorul[red]{If you'd like to} explore any specific aspect of Superman's creation or history, \colorul[red]{feel free to ask!}  \\ 
\bottomrule
\end{tabular}
\end{adjustbox}
\end{table}


\setlength{\tabcolsep}{5pt} % Adjust column separation
\renewcommand{\arraystretch}{1.5} % Adjust row spacing
% \vspace{1em} % Add some vertical space between the user input and the table
% Centering the table and ensuring it stays on one page
\begin{table}[t]
\centering
\caption{Case: Some models produce low-quality answers on Arena-Hard}
\label{case-bad}
% Add the user input in a tcolorbox for clarity
% \begin{tcolorbox}[colback=blue!5!white, colframe=blue!75!black, title=User Input]
% Who created the Superman cartoon character?
% \end{tcolorbox}
\begin{tabular}{@{}p{17cm}@{}}
\toprule
\textbf{User Prompt}  \\ 
\midrule
Write a chrome plugin that saves the contents of the current page.
\end{tabular}
\begin{adjustbox}{max width=\linewidth}
\begin{tabular}{@{}p{2cm}p{3.5cm}p{2.5cm}p{8cm}@{}}
\toprule
\textbf{Expert} & \textbf{Model} & \textbf{Score} & \textbf{Output} \\ 
\midrule
Creative \newline Strategist & LLaMa-3-70B-Instruct & \texttt{GPT >> KABB} & 
\ldots AI-powered content analysis: \colorul[blue]{Integrate a machine learning algorithm} that analyzes the saved page's content, providing users with insights, such as \colorul[blue]{keyword extraction, sentiment analysis, or topic modeling.} \ldots \\ 
% \midrule
User \newline Experience \newline Consultant & Qwen2-72B-Instruct & \texttt{GPT >> KABB} & 
\ldots Content Script: \colorul[green]{- **Feedback:** Provide visual feedback }when the user interacts with the plugin\ldots \\ 
% \midrule
Implementation \newline Advisor & Qwen2-72B-Instruct & \texttt{GPT >> KABB} & 
\ldots 4. **\colorul[orange]{Test} and iterate**: Conduct thorough testing to ensure the plugin works as intended, and iterate on the design and functionality \colorul[orange]{based on user feedback.} \newline \ldots \\ 
% \midrule
Technical \newline Architect & WizardLM-2-8x22B & \texttt{GPT < KABB} & ... \colorul[red]{The manifest file should look like this}: \newline \texttt{```}json \newline \texttt{\{} \newline ``manifest\_version'': 3, \newline ``name'': ``Page Saver Plugin'', \newline \ldots \\
\midrule
Aggregator & Qwen2-72B-Instruct & \texttt{GPT > KABB} & \ldots 3. **Define Plugin Permissions:** \newline \colorul[red]{- Update \texttt{`}manifest.json\texttt{`} with plugin metadata, permissions (e.g., ``activeTab'', ``downloads''),} and content scripts. \newline \ldots \colorul[orange]{- Test the plugin in various scenarios }to ensure reliability. \newline \ldots  - Implement \colorul[green]{visual feedback}, progress indicators, and error handling. \newline \ldots \colorul[blue]{ Incorporate an AI-powered content analysis} for insights or a collaboration feature for annotations. \newline \ldots - Regularly update the plugin with bug fixes and feature enhancements \colorul[orange]{based on user feedback}\ldots \\ 
\midrule
Reference & GPT-4 (03/14) & N/A & \ldots 2. Inside the ``SavePageContent'' folder, create a new file called ``manifest.json'' and add the following code: \newline \texttt{```}json \newline \texttt{\{} \newline 
    ``manifest\_version'': 2, \newline ``name'': ``Save Page Content'', \newline \ldots \\ 
\midrule

Judgment & GPT-4 Preview (11/06) & N/A & \ldots GPT's answer is slightly better because it provides actionable code snippets and a clear example that users can follow to create the plugin. However, KABB's answer is also of high quality, offering a broader overview of the process and additional creative suggestions.\ldots 
\end{tabular}
\end{adjustbox}
\end{table}

\section{Additional Experimental Settings}

\textbf{Resources.} All experiments on KABB are conducted on servers with one NVIDIA GeForce RTX 3090.

\subsection{Prompts for Experts and the Aggregator}

In this section, we provide some cases of prompts for different experts and the aggregator to show an example of the system configuration.

\begin{tcolorbox}[colback=lightgray!5!white, colframe=lightgray!75!black, title=Analysis Expert]
You are an expert in problem analysis and logical reasoning, skilled in applying analytical frameworks and systematic thinking approaches. \newline Your expertise includes breaking down complex problems, identifying key factors, and recommending structured, actionable solutions. \newline You are familiar with various problem-solving methods such as root cause analysis, decision matrices, and scenario evaluation, and adapt your approach based on the unique context of each task. \newline Consider how your skills in critical thinking, structured reasoning, and analytical problem-solving might provide valuable insights or strategies for addressing the task at hand.
\end{tcolorbox}

\begin{tcolorbox}[colback=teal!5!white, colframe=teal!75!black, title=Strategy Expert]
You are a business strategy expert with a deep understanding of markets, business models, competitive landscapes, and strategic planning. \newline Your expertise includes applying business frameworks, analytical tools, and market insights to identify opportunities and craft strategies. \newline While capable of providing comprehensive strategic analysis, you adapt your input to focus on what is most valuable, practical, and relevant for the situation. \newline Consider how your expertise in business innovation, competitive advantage, and strategic problem-solving might provide insightful and actionable recommendations for any task.
\end{tcolorbox}

\begin{tcolorbox}[colback=olive!5!white, colframe=olive!75!black, title=Aggregator]
You are the Wise Integrator in a multi-agent system tasked with delivering accurate, coherent, and actionable responses to user queries. \newline Your role is to: \newline - Understand the user's intent and main question(s) by carefully reviewing their query. \newline - Evaluate expert inputs, preserving their quality opinions while ensuring relevance, accuracy, and alignment with the user's needs. \newline - Resolve any contradictions or gaps logically, combining expert insights into a single, unified response. \newline - Synthesize the most appropriate information into a clear, actionable, and user-friendly answer. \newline - Add your own insight if needed to enhance the final output. \newline Your response must prioritize clarity, accuracy, and usefulness, ensuring it directly addresses the user's needs while retaining the value of expert contributions. \newline Avoid referencing the integration process or individual experts.
\end{tcolorbox}

\section{Supplementary Proofs and Theoretical Analysis}

% \subsection{Proofs and Theoretical Analysis}

To better illustrate the theoretical derivations and implementation details regarding the Knowledge-Aware Bayesian Bandit (KABB) model in \cref{sec:method}, we provide the following supplementary proofs and theoretical analysis.
% Key definitions and analytical components are organized as follows:

\subsection{Proof of Pseudo-Metric Properties of Knowledge Distance Theorem}
We provide proofs of Pseudo-Metric Properties of Knowledge Distance Theorem \cref{the:Pseudo-Metric} which enhances the reliability and effectiveness of the model in expert selection and task allocation.

\begin{proof}\renewcommand{\qedsymbol}{}
This follows directly from the non-negativity of $\log(1 + d_t)$ and all other terms in the definition of $\text{Dist}(\mathcal{S}, t)$. Each term (e.g., $1 - \rho_{\text{overlap}}$, dependency complexity, etc.) is non-negative by construction.
\end{proof}

\textbf{Proof of Conditional Symmetry}:  
If the dependency graph $G$ is undirected and $\rho_{\text{overlap}}(\mathcal{S}_1, t) = \rho_{\text{overlap}}(\mathcal{S}_2, t)$, and if $\mathcal{S}_1$ and $\mathcal{S}_2$ are symmetric in terms of knowledge and dependencies, then all terms in the distance function (e.g., $|\mathcal{R}_{\text{dep}}|$, $\bar{H}_{\mathcal{S}}$, and weights) are equal for $\mathcal{S}_1$ and $\mathcal{S}_2$. Thus, $\text{Dist}(\mathcal{S}_1, t) = \text{Dist}(\mathcal{S}_2, t)$.

\textbf{Proof of Approximate Triangle Inequality}:  
Using the properties of the knowledge graph as a metric space, the subadditivity of the graph metric ensures that the dependency-based terms satisfy a triangle inequality. Similarly, the Jaccard similarity is used in \cref{jaccard subadd}. Combining these with the weight terms, the inequality holds with a relaxation factor $c \geq 1$ determined by the extrema of the weights.

\subsection{Proof Sketch of Convergence Analysis for the Dynamic Selection Strategy}

The proof of convergence is outlined as follows:

\begin{enumerate}
    \item \textbf{Stability of Beta Distribution Parameters}: Analyze the stability of the Beta distribution parameter evolution by leveraging KL divergence to quantify changes over time.
    \item \textbf{Lyapunov Function Construction}: Construct a Lyapunov function 
    \[
    V(t) = \sum_{\mathcal{S}} \big[(\alpha_{\mathcal{S}}^{(t)} - \alpha_{\mathcal{S}^*}^{(t)})^2 + (\beta_{\mathcal{S}}^{(t)} - \beta_{\mathcal{S}^*}^{(t)})^2\big],
    \]
    and use it to demonstrate the convergence of the parameters.
    \item \textbf{Cumulative Regret Analysis}: Establish an upper bound for cumulative regret by applying UCB (Upper Confidence Bound) principles.
\end{enumerate}

\subsection{The Strict Proof of the Approximate Triangle Inequality for Theorem 2}

\paragraph{Step 1: Decomposition of Knowledge Distance Function and Subterm Analysis}~{}
\newline
For any expert teams $\mathcal{S}_1, \mathcal{S}_2, \mathcal{S}_3$ and task $t$, there exists a constant $\epsilon > 0$, such that the knowledge distance function satisfies:  
\[
\text{Dist}(\mathcal{S}, t) = \log(1 + d_t) \cdot \sum_{i=1}^4 \omega_i \Psi_i
\]
where $\Psi_i$ corresponds to the four subterms that key the multi-dimensional distance measurement between the expert team and the task:  
\[
\Psi_1 = 1 - \rho_{\text{overlap}}(\mathcal{S}, t) \quad (\text{semantic mismatch term})
\]
\[
\Psi_2 = \frac{|\mathcal{R}_{\text{dep}}(\mathcal{S}, t)|}{K} \quad (\text{dependency complexity term})
\]
\[
\Psi_3 = 1 - \bar{H}_{\mathcal{S}}(t) \quad (\text{historical performance term})
\]
\[
\Psi_4 = 1 - \mathrm{Synergy}(\mathcal{S}) \quad (\text{team complementarity term})
\]
The proof demonstrates that by establishing the approximate sub-additivity of the subterms and combining the logarithmic term properties, the knowledge distance function satisfies the approximate triangle inequality within the error bound $\epsilon = \max{\epsilon_1, \epsilon_2, \epsilon_3, \epsilon_4}$, providing a theoretical guarantee for algorithm design.
 
\paragraph{Step 2: Sub-additivity Analysis of Semantic Mismatch Term (Based on Jaccard Similarity)}~{}
\newline

\begin{definition}[Jaccard Similarity]  
For any sets $\mathcal{S}_1, \mathcal{S}_2$ and task concept set $\mathcal{C}_t$, define:  
\[
\rho_{\text{overlap}}(\mathcal{S}, t) = \frac{|\mathcal{C}_{\mathcal{S}} \cap \mathcal{C}_t|}{|\mathcal{C}_{\mathcal{S}} \cup \mathcal{C}_t|}
\]
\end{definition}

\begin{lemma}[Jaccard Sub-additivity]: 
For any $\mathcal{S}_1, \mathcal{S}_2 \subseteq \mathcal{E}$, there exists a constant $c_1 \geq 1$ such that:  
\[
1 - \rho_{\text{overlap}}(\mathcal{S}_1 \cup \mathcal{S}_2, t) \leq c_1 \left[ \left(1 - \rho_{\text{overlap}}(\mathcal{S}_1, t)\right) + \left(1 - \rho_{\text{overlap}}(\mathcal{S}_2, t)\right) \right]
\]
\label{jaccard subadd}
\end{lemma}

\begin{proof}  
By the properties of set operations:  
\[
|\mathcal{C}_{\mathcal{S}_1 \cup \mathcal{S}_2} \cap \mathcal{C}_t| \geq |\mathcal{C}_{\mathcal{S}_1} \cap \mathcal{C}_t| + |\mathcal{C}_{\mathcal{S}_2} \cap \mathcal{C}_t| - |\mathcal{C}_{\mathcal{S}_1} \cap \mathcal{C}_{\mathcal{S}_2} \cap \mathcal{C}_t|
\]
\[
|\mathcal{C}_{\mathcal{S}_1 \cup \mathcal{S}_2} \cup \mathcal{C}_t| \leq |\mathcal{C}_{\mathcal{S}_1} \cup \mathcal{C}_t| + |\mathcal{C}_{\mathcal{S}_2} \cup \mathcal{C}_t|
\]
Let $A = \mathcal{C}_{\mathcal{S}_1} \cap \mathcal{C}_t$, $B = \mathcal{C}_{\mathcal{S}_2} \cap \mathcal{C}_t$, we get:  
\[
\rho_{\text{overlap}}(\mathcal{S}_1 \cup \mathcal{S}_2, t) \geq \frac{|A| + |B| - |A \cap B|}{|\mathcal{C}_{\mathcal{S}_1} \cup \mathcal{C}_t| + |\mathcal{C}_{\mathcal{S}_2} \cup \mathcal{C}_t|}
\]
By relaxing the denominator to $2 \cdot \max(|\mathcal{C}_{\mathcal{S}_1} \cup \mathcal{C}_t|, |\mathcal{C}_{\mathcal{S}_2} \cup \mathcal{C}_t|)$, we get $c_1 = 2$.
\end{proof}

\begin{corollary}[]  
$\Psi_1(\mathcal{S}_1 \cup \mathcal{S}_2, t) \leq 2 \left[ \Psi_1(\mathcal{S}_1, t) + \Psi_1(\mathcal{S}_2, t) \right]$ 
\end{corollary}
This conclusion allows us to effectively estimate and control the semantic differences between expert teams using a simple additive form.


\paragraph{Step 3: Sub-additivity of Dependency Complexity Term in Graph Metrics}~{}
\newline
\textbf{Definition (Dependency Edge Path Length)}:  
The number of dependency edges $|\mathcal{R}_{\text{dep}}(\mathcal{S}, t)|$ in the knowledge graph satisfies the triangle inequality in graph metrics:  
\[
|\mathcal{R}_{\text{dep}}(\mathcal{S}_1, t)| \leq |\mathcal{R}_{\text{dep}}(\mathcal{S}_1, \mathcal{S}_2)| + |\mathcal{R}_{\text{dep}}(\mathcal{S}_2, t)|
\]
where $|\mathcal{R}_{\text{dep}}(\mathcal{S}_1, \mathcal{S}_2)|$ is the number of shortest path edges connecting $\mathcal{S}_1$ and $\mathcal{S}_2$.

\begin{lemma}[Existence of Relaxation Factor]:  
For any acyclic graph, there exists a constant $c_2 \geq 1$ such that:  
\[
|\mathcal{R}_{\text{dep}}(\mathcal{S}_1, t)| \leq c_2 \left[ |\mathcal{R}_{\text{dep}}(\mathcal{S}_1, \mathcal{S}_2)| + |\mathcal{R}_{\text{dep}}(\mathcal{S}_2, t)| \right]
\]
\end{lemma}

\begin{proof}
By graph diameter constraints, set $c_2 = \text{diam}(G)$ (the diameter of the graph), which is the longest path in terms of edges between any two nodes.  
The dependency complexity term establishes sub-additivity through the following reasoning: based on graph metric properties, path lengths satisfy the triangle inequality; by the graph's diameter constraints, we obtain an upper bound for the relaxation factor; and by normalization, the boundedness of dependency complexity is guaranteed. This property provides a quantifiable theoretical foundation for evaluating team knowledge structures.
\end{proof}

\paragraph{Step 4: Approximate Linearity of Team Complementarity Term}~{}
\newline
\textbf{Definition (Complementarity Decomposition)}:  
The team complementarity $\mathrm{Synergy}(\mathcal{S})$ satisfies:  
\[
\mathrm{Synergy}(\mathcal{S}_1 \cup \mathcal{S}_2) \geq \mathrm{Synergy}(\mathcal{S}_1) + \mathrm{Synergy}(\mathcal{S}_2) - \mathrm{Overlap}(\mathcal{S}_1, \mathcal{S}_2)
\]
where $\mathrm{Overlap}$ is the complementarity loss due to knowledge overlap between teams.

\begin{lemma}[Upper Bound of Relaxation]  
There exists a constant $c_3 \geq 1$ such that:  
\[
1 - \mathrm{Synergy}(\mathcal{S}_1 \cup \mathcal{S}_2) \leq c_3 \left[ \left(1 - \mathrm{Synergy}(\mathcal{S}_1)\right) + \left(1 - \mathrm{Synergy}(\mathcal{S}_2)\right) \right]
\]
\begin{proof}\renewcommand{\qedsymbol}{}  
Let $\mathrm{Overlap}(\mathcal{S}_1, \mathcal{S}_2) \leq \min(\mathrm{Synergy}(\mathcal{S}_1), \mathrm{Synergy}(\mathcal{S}_2))$, set $c_3 = 2$.
\end{proof}
\end{lemma}

The construction of the global constant for the knowledge distance: The overall approximate sub-additivity of the subterms in the knowledge distance function is determined by the set of relaxation factors: semantic mismatch term $c_1 = 2$, dependency complexity term $c_2 = \text{diam}(G)$, team complementarity term $c_3 = 2$, and historical performance term $c_4 = 1$. By using these local relaxation factors, combined with the weights and the logarithmic term of task difficulty, a global constant $c = \max c_i \cdot \omega_i \cdot \log(1 + \overline{D}_{\max})$ is constructed. This construction ensures that the overall knowledge distance function satisfies the approximate triangle inequality, providing a theoretical guarantee for the quantitative evaluation of knowledge distance.

\subsection{Theorem 1 Proof: Lower Bound of Expert-Task Mutual Information under Semantic Gap}

\paragraph{Basic Definitions of Dynamic Multi-Agent Systems}~{}
\newline
In dynamic multi-agent systems, the interaction between the expert set $\mathcal{E}$ and the task demand space $\mathcal{T}$ is based on three core assumptions: the Markovian evolution of task demands over time, the conditional independent decomposition of expert selection and tasks, and the decaying mutual information metric with the introduction of a discount factor $\gamma$. This framework is described by the joint distribution 
\[
p(\mathbf{e}, \mathbf{t}_{1:T}) = p(\mathbf{e}) \prod_{t=1}^T p(\mathbf{t}_t | \mathbf{t}_{t-1}) p(\mathbf{e} | \mathbf{t}_t),
\]
which characterizes the dynamic relationship between expert knowledge and task demands, providing a theoretical foundation for the subsequent analysis.

\paragraph{Step 2: Time Accumulation Form of Conditional Entropy}~{}
\newline

The accumulated conditional entropy of expert selection over an infinite time horizon is given by:
\[
H(\mathcal{E} | \mathcal{T}_{1:\infty}) = \lim_{T \to \infty} \frac{1}{T} \sum_{t=1}^T H(\mathcal{E} | \mathcal{T}_t).
\]

After introducing the discount factor $\gamma$, the weighted conditional entropy is:
\[
\widetilde{H}(\mathcal{E} | \mathcal{T}) \triangleq \sum_{t=1}^\infty \gamma^{t-1} H(\mathcal{E} | \mathcal{T}_t).
\]

\paragraph{Step 3: Extension of Fano's Inequality}~{}
\newline

For each time step $t$, apply the classical \textbf{Fano's Inequality}:
\[
H(\mathcal{E} | \mathcal{T}_t) \geq H(\mathcal{E}) - I(\mathcal{E}; \mathcal{T}_t) - h_2(P_e^{(t)}),
\]
where $h_2(x) = -x \log x - (1-x) \log (1-x)$ is the binary entropy function, and $P_e^{(t)} = \mathbb{P}(\hat{\mathcal{E}}_t \neq \mathcal{E} | \mathcal{T}_t)$ is the expert selection error rate at time $t$. When there is no prior knowledge (i.e., $I(\mathcal{E}; \mathcal{T}_t) = 0$), we have:
\[
H(\mathcal{E} | \mathcal{T}_t) \geq \log K - h_2(P_e^{(t)}).
\]

\paragraph{Step 4: Weighted Summation and Asymptotic Analysis}~{}
\newline

Substitute Fano's inequality for the weighted conditional entropy:
\[
\begin{aligned}
\widetilde{H}(\mathcal{E} | \mathcal{T}) &= \sum_{t=1}^\infty \gamma^{t-1} H(\mathcal{E} | \mathcal{T}_t) \\
&\geq \sum_{t=1}^\infty \gamma^{t-1} \left[ \log K - I(\mathcal{E}; \mathcal{T}_t) - h_2(P_e^{(t)}) \right] \\
&= \frac{\log K}{1-\gamma} - \widetilde{I}(\mathcal{E}; \mathcal{T}) - \sum_{t=1}^\infty \gamma^{t-1} h_2(P_e^{(t)}).
\end{aligned}
\]

Under the assumption of long-term stability of the dynamic system ($\lim_{t \to \infty} P_e^{(t)} = 0$), the asymptotic behavior of the error entropy is analyzed. By the convergence of the geometric series sum, it is shown that the weighted error entropy term $\sum_{t=1}^T \gamma^{t-1} h_2(P_e^{(t)})$ vanishes in the limit. This result simplifies the lower bound of conditional entropy to the form of the difference between the entropy of the expert set and the mutual information: 
\[
\widetilde{H}(\mathcal{E} | \mathcal{T}) \geq \frac{\log K}{1-\gamma} - \widetilde{I}(\mathcal{E}; \mathcal{T}),
\]
which provides a more concise theoretical expression for system performance evaluation.

\paragraph{Step 5: Equivalent Form and Semantic Gap Explanation}~{}
\newline

Multiplying both sides of the inequality by $(1-\gamma)$ yields the final form:
\[
\underbrace{H(\mathcal{E} | \mathcal{T})}_{\substack{\text{Conditional Entropy} \\ \text{(Semantic Uncertainty)}}} \geq \log K - \frac{\widetilde{I}(\mathcal{E}; \mathcal{T})}{1-\gamma}.
\]

\textbf{Semantic Gap Limit}: As $\widetilde{I}(\mathcal{E}; \mathcal{T}) \to 0^+$ (when there is no semantic connection between experts and tasks), the lower bound of conditional entropy approaches $\log K$, corresponding to the maximum entropy of completely random selection. 

\textbf{Exploration Efficiency Bottleneck}: The inequality shows that the exploration efficiency of traditional MAB (multi-armed bandit) is limited by $\frac{\widetilde{I}(\mathcal{E}; \mathcal{T})}{1-\gamma}$. When the semantic connection weakens ($\widetilde{I} \downarrow$) or task dynamics increase ($\gamma \uparrow$), the exploration cost increases dramatically.

\subsection{Proof of Knowledge-Driven Information Gain Theorem}

\paragraph{1. Baseline Mutual Information Analysis}~{}
\newline

First, establish the baseline mutual information $I_0 = I(\mathcal{E}; \mathcal{T})$ when there is no knowledge graph, which only depends on the direct association between experts and tasks.

\textbf{2. Effect of Knowledge Graph Intervention}: After introducing the knowledge graph $\mathcal{G}$, the task generation process is reconstructed via the intermediary pattern of the knowledge graph: 
\[
p(\mathcal{T} | \mathcal{E}) = \sum_{\mathcal{G}} p(\mathcal{T} | \mathcal{G}) p(\mathcal{G} | \mathcal{E}).
\]

\textbf{3. Mutual Information Gain Decomposition}: Using the chain rule, the total mutual information introduced by the knowledge graph can be decomposed into: the original expert-task mutual information $I(\mathcal{E}; \mathcal{T})$ and the conditional mutual information contribution from the concept layer $I(\mathcal{C}; \mathcal{T} | \mathcal{E})$. Since $\mathcal{G}$ is fully determined by $\mathcal{E}$ and $\mathcal{C}$, the information gain $\Delta I$ equals the conditional mutual information contribution from the concept layer, verifying that the knowledge graph improves the system's informational efficiency through the concept layer.

\subsection{Derivation of the Concept Layer Information Gain Bound}

\paragraph{Core Condition Analysis}~{}
\newline

Based on the two key properties of the knowledge graph: sparsity: the upper bound of the expert-concept association degree $d = O(\sqrt{|\mathcal{C}|})$ and balance: the minimum expert coverage of a concept $ \lfloor |\mathcal{E}| / |\mathcal{C}| \rfloor$.

\textbf{2. Information Theoretic Derivation Process}

Through the Markov chain $\mathcal{T} \to \mathcal{C} \to \mathcal{E}$ analysis:
\textbf{Conditional Entropy Relation}: $H(\mathcal{T} | \mathcal{E}) \geq H(\mathcal{T} | \mathcal{C})$ (data processing inequality), $H(\mathcal{T} | \mathcal{C}) = O(\log|\mathcal{C}|)$ (task sparsity).
\textbf{Mutual Information Lower Bound}: Using the definition of conditional mutual information and the relationship with entropy, along with graph structure constraints, the lower bound is obtained:
\[
I(\mathcal{C}; \mathcal{T} | \mathcal{E}) \geq \Omega\left( \frac{\log|\mathcal{C}|}{\sqrt{|\mathcal{E}|}} \right).
\]
This result quantifies the minimum information gain brought by the knowledge graph through the concept layer.

\paragraph{Step 2: Mathematical Representation of Accelerated Exploration Efficiency}~{}
\newline

\textbf{(Upper Bound of Exploration Trials)}:  
In the contextual Bandit framework, the expected number of exploration trials satisfies:
\[
\mathbb{E}[N_{\text{explore}}] = \widetilde{O}\left( \sqrt{\frac{K \log|\mathcal{C}|}{\Delta I}} \right),
\]
where $K = |\mathcal{E}|$, and $\Delta I = \Omega\left( \frac{\log|\mathcal{C}|}{\sqrt{|\mathcal{E}|}} \right)$.

\begin{proof}\renewcommand{\qedsymbol}{}
1. \textbf{Classical Bandit Exploration Complexity}:  
Without a knowledge graph, the exploration trials of a traditional MAB are:
\[
\mathbb{E}[N_{\text{explore}}] = O\left( \frac{K \log T}{\epsilon^2} \right),
\]
where $\epsilon$ is the expected reward gap between the optimal and suboptimal arms.

2. \textbf{Knowledge-Driven Acceleration Mechanism}:  
After introducing the knowledge graph, the reward gap $\epsilon$ is amplified by the information gain $\Delta I$:
\[
\epsilon_{\text{new}} = \epsilon \cdot \sqrt{\Delta I}.
\]
Substituting into the classical complexity formula:
\[
\mathbb{E}[N_{\text{explore}}] = O\left( \frac{K \log T}{\epsilon_{\text{new}}^2} \right) = O\left( \frac{K \log T}{\epsilon^2 \Delta I} \right).
\]
Combining with $\Delta I = \Omega\left( \frac{\log|\mathcal{C}|}{\sqrt{|\mathcal{E}|}} \right)$, and assuming $\epsilon = \Theta(1/\sqrt{K})$ (uniform exploration hypothesis), we obtain:
\[
\mathbb{E}[N_{\text{explore}}] = \widetilde{O}\left( \sqrt{\frac{K \log|\mathcal{C}|}{\Delta I}} \right).
\]
\end{proof}

\subsection{Summary of the Information Gain Theorem Proof}

By introducing a structured knowledge graph through the concept layer $\mathcal{C}$, the conditional mutual information $I(\mathcal{C}; \mathcal{T} | \mathcal{E})$ provides the lower bound of the information gain $\Delta I = \Omega\left( \frac{\log|\mathcal{C}|}{\sqrt{|\mathcal{E}|}} \right)$, which reduces the exploration complexity from the traditional method of $O(K)$ to $\widetilde{O}\left( \sqrt{K \log|\mathcal{C}|} \right)$. This theoretical result rigorously verifies the acceleration advantage of knowledge-driven decision-making.


\subsection{Regret Upper Bound Derivation for Knowledge-Aware UCB (KABB)}
\label{sec:supp-regret}

\paragraph{Problem Framework} \cref{sec:system_arch} are extended with complete mathematical specifications of expert set $\mathcal{E}$ and task sequence $\{T_t\}_{t=1}^T$:
\begin{itemize}
\item Selection process: $\mathcal{S}_t \subseteq \mathcal{E}$ at each step
\item Feedback mechanism: Obtain $\theta_{\mathcal{S}_t}^{(t)}$
\item Success probability: 
Including knowledge distance \(\mathrm{Dist}(\mathcal{S}, t)\), time decay \(\gamma^{\Delta t}\), and team synergy \(\mathrm{Synergy}(\mathcal{S})\).


\begin{equation}
\tilde{\theta}_{\mathcal{S}}^{(t)} = \underbrace{\mathbb{E}\left[\theta_{\mathcal{S}}^{(t)}\right]}_{\text{Historical expectation}} \cdot \exp\left(-\lambda \cdot \text{Dist}(\mathcal{S}, t)\right) \cdot \gamma^{\Delta t} \cdot \mathrm{Synergy}(\mathcal{S})^\eta
\end{equation}
\end{itemize}

\paragraph{Confidence Bound Construction}

This section elaborates on the construction method of confidence bounds in the KABB algorithm, the definition of knowledge revision rewards, and their impact on exploration weights. It supports the theoretical analysis in \cref{sec:knowledge_distance} regarding the limitations of traditional methods and the breakthroughs in knowledge-driven decision-making. The confidence-bound construction extends traditional UCB through knowledge-aware reward correction:

\begin{equation}
\text{UCB}_{\mathcal{S}}^{(t)} = \underbrace{\hat{\mu}_{\mathcal{S}}^{(t)}}_{\text{Empirical mean}} + \underbrace{\sqrt{\frac{2 \log t}{N_{\mathcal{S}}^{(t)}}}}_{\text{Exploration term}} \cdot \underbrace{\exp\left(-\lambda \cdot \text{Dist}(\mathcal{S}, t)\right) \cdot \gamma^{\Delta t} \cdot \mathrm{Synergy}(\mathcal{S})^\eta}_{\text{Knowledge-driven correction}}
\end{equation}

where $\hat{\mu}_{\mathcal{S}}^{(t)} = \frac{\alpha_{\mathcal{S}}^{(t)}}{\alpha_{\mathcal{S}}^{(t)} + \beta_{\mathcal{S}}^{(t)}}$ denotes the Bayesian estimate of historical success rate. The correction term adjusts the exploration weights through knowledge distance, time decay, and synergy effects.


\subsection{Regret Upper Bound Analysis}
\label{sec:supp-regret-analysis}

\paragraph{Total Regret Definition }
\label{subsec:regret_analysis}

\cref{sec:supp-regret} provides a detailed analysis of the total regret decomposition and single-step regret properties for the KABB algorithm, corresponding to the analysis in \label{sec:dynamic_bayesian} regarding the impact of team knowledge distance and complementarity on algorithmic performance. The theoretical proofs and mathematical derivations are presented as follows:

The total regret is defined as:
\begin{equation}
\label{eq:total_regret}
R(T) = \sum_{t=1}^{T} \left( \theta_{S^*}^{(t)} - \theta_{S_t}^{(t)} \right)
\end{equation}
where $S^*$ denotes the optimal expert subset and $S_t$ represents the selected subset at time step $t$, the analysis should follow these steps:

\begin{enumerate}
    \item \textbf{Characterize Single-Step Regret}: First define the single-step regret:
    \begin{equation}
    \label{eq:instant_regret}
    r_t = \theta_{S^*}^{(t)} - \theta_{S_t}^{(t)}
    \end{equation}
    and analyze its properties.

    \item \textbf{Analyze Regret Bound for Suboptimal Subsets}: For any suboptimal subset $S \neq S^*$, establish the upper bound of single-step regret.

    \item \textbf{Compose Total Regret Upper Bound}: Investigate how to combine single-step regrets into the total regret upper bound.
\end{enumerate}

% \begin{lemma}
% The regret analysis requires:
% \begin{itemize}
%     \item Precise decomposition of $r_t$ using knowledge distance metrics
%     \item Bounding techniques for suboptimal selections $S \neq S^*$
%     \item Cumulative error control through martingale analysis
% \end{itemize}
% \end{lemma}

\paragraph{Step 1: Per-Step Regret Decomposition}

For any suboptimal subset $\mathcal{S} \neq \mathcal{S}^*$, the instantaneous regret satisfies:

\begin{equation}
\Delta_{\mathcal{S}}^{(t)} \leq \underbrace{\left| \hat{\mu}_{\mathcal{S}^*}^{(t)} - \theta_{\mathcal{S}^*}^{(t)} \right|}_{\text{Optimal set error}} + \underbrace{\left| \hat{\mu}_{\mathcal{S}}^{(t)} - \theta_{\mathcal{S}}^{(t)} \right|}_{\text{Suboptimal set error}} + \underbrace{\text{Dist}(\mathcal{S}, t) \cdot \lambda}_{\text{Knowledge penalty}}
\end{equation}

\paragraph{Step 2: Exploration Acceleration Effect}

\begin{lemma}[Exploration Count Upper Bound]
For any suboptimal $\mathcal{S}$, its selection count satisfies:
\begin{equation}
\mathbb{E}\left[N_{\mathcal{S}}(T)\right] \leq \frac{8 \log T}{(\Delta_{\mathcal{S}} \cdot \exp(-\lambda \overline{D}_{\mathcal{S}}))^2} + O\left(\sqrt{T \log T}\right)
\end{equation}
where $\overline{D}_{\mathcal{S}} = \max_t \text{Dist}(\mathcal{S}, t)$ and $\Delta_{\mathcal{S}} = \theta_{\mathcal{S}^*} - \theta_{\mathcal{S}}$.
\end{lemma}

\begin{proof}\renewcommand{\qedsymbol}{}
The knowledge correction term $\exp(-\lambda \overline{D}_{\mathcal{S}})$ amplifies the reward gap $\Delta_{\mathcal{S}}$, thereby reducing the exploration demand for suboptimal subsets. The estimation error is bounded via the Chernoff-Hoeffding inequality, combined with the exponential decay modification of exploration terms through knowledge distance. This upper bound formula reflects three key factors influencing regret:

\begin{itemize}
\item \textbf{Optimality gap term} $\Delta_{\mathcal{S}}$: The term in the denominator represents the performance gap between suboptimal and optimal subsets. A larger gap leads to a smaller regret upper bound.
  
\item \textbf{Knowledge distance penalty} $\exp(-2\lambda \overline{D}_{\mathcal{S}})$: The exponential term in the denominator reflects the impact of the knowledge graph. Larger $\overline{D}_{\mathcal{S}}$ (i.e., greater knowledge discrepancy) increases the regret upper bound.
  
\item \textbf{Combinatorial complexity term} $O(\sqrt{T \log T} \cdot \tbinom{K}{k})$: Captures the combinatorial optimization nature of the problem, where:
  \begin{itemize}
  \item $\sqrt{T \log T}$ corresponds to the standard UCB term
  \item $\tbinom{K}{k}$ represents the combinatorial complexity from selecting $k$ experts out of $K$
  \end{itemize}
\end{itemize}

This demonstrates that the regret upper bound is jointly determined by the knowledge structure (via $\overline{D}_{\mathcal{S}}$) and combinatorial optimization complexity.
\end{proof}

\subsection{Core Differences from Classical UCB}

\begin{table}[H]
\centering
\caption{Comparison between Classical UCB and KABB}
\label{tab:ucb-comparison}
\begin{tabular}{lll}
\toprule
\textbf{Dimension} & \textbf{Classical UCB} & \textbf{Knowledge-Aware UCB (KABB)} \\
\hline
Exploration Design & $\sqrt{\log t/N}$ & Multiplicative knowledge correction \\
Regret Dominant Term & $O(\sqrt{KT \log T})$ & $O(\sqrt{T \log T} \cdot \tbinom{K}{k})$ \\
Theoretical Innovation & No structured prior & Knowledge graph integration \\
Key Assumption & IID rewards & Non-stationary rewards with synergy \\
\bottomrule
\end{tabular}
\end{table}
The knowledge-aware UCB improves the traditional $O(KT \log T)$ regret bound of UCB through structured prior injection and a dynamic correction mechanism, transforming it into an exponentially compressed form in the combinatorial space. The core innovation lies in the quantitative modeling of knowledge distance and synergy effects. This theorem represents the first strict integration of knowledge graphs and team collaboration theory within the Bandit framework.
\subsection{Regret Bound and Exploration Efficiency Analysis}

This section provides a detailed description of the core modules of the KABB algorithm, offering an in-depth analysis of its cumulative regret bound and the relationship with exploration efficiency, supporting the algorithm derivation and convergence analysis in \cref{sec:dynamic_bayesian}. Additionally, \cref{sec:algorithm} elaborates on the specific implementation modules of the KABB algorithm, along with its time and space complexities, providing empirical foundations and optimization strategies for the algorithm design and performance evaluation in the main text.

\begin{theorem}[The cumulative regret $R(T)$ of KABB]
$$
R(T) \leq \underbrace{\sum_{\mathcal{S} \neq \mathcal{S}^*} \frac{4\underline{L}^2 \log T}{\widetilde{\Delta}_{\mathcal{S}}}}_{\text{Knowledge-driven exploration term}} + \underbrace{O\left( \sqrt{T \tbinom{N}{k} \log \tbinom{N}{k}} \right)}_{\substack{\text{Additional complexity term}\\ \text{due to team size}}}, \quad \text{where} \quad 
\boxed{
\begin{cases}  
L = \log(1 + \overline{D}_{\max}) \cdot (\omega_1 \!+\! \omega_2 \!+\! \omega_3 \!+\! \omega_4) \\
\widetilde{\Delta}_{\mathcal{S}} = \mu_{\mathcal{S}^*} \!-\! \mu_{\mathcal{S}} \\
\overline{D}_{\max} = \max\limits_{\mathcal{S}, t} \mathrm{Dist}(\mathcal{S}, t) \\
k = |\mathcal{S}^*|
\end{cases}
}
$$
\end{theorem}

\textbf{Explanation}:  
The cumulative regret $R(T)$ measures the performance loss caused by not selecting the optimal team $\mathcal{S}^*$ over $T$ time steps:

\textbf{Knowledge-driven exploration term:} The exploration count is constrained by the knowledge distance. Its dominant term $\frac{4\underline{L}^2 \log T}{\widetilde{\Delta}_{\mathcal{S}}}$ shows that: 1) when the knowledge distance difference is significant (i.e., $\overline{D}_{\max} \uparrow$), the algorithm quickly focuses on high-quality teams through the $\exp(-\lambda \mathrm{Dist}(\cdot))$ mechanism; 2) when the team reward gap $\widetilde{\Delta}_{\mathcal{S}} \downarrow$, the exploration intensity is adaptively adjusted via the $\log(1 + \overline{D}_{\max})$ factor.

\textbf{Team size complexity term:} The complexity term $O\left( \sqrt{T \tbinom{N}{k} \log \tbinom{N}{k}} \right)$ includes the combination number $\tbinom{N}{k}$, and its variation with the expert set size $N$ and optimal team size $k$ follows:
$$
\tbinom{N}{k} \sim \begin{cases}
O(N^k/k!) & \text{when } k \ll N \\
O(2^N/\sqrt{N}) & \text{when } k \approx N/2
\end{cases}
$$
\subsection{Proof Framework}

\paragraph{Step 1: Reward Remapping}
Define dual-modality adjusted reward:
\begin{equation}
\tilde{\mu}_{\mathcal{S}} = \underbrace{\mu_{\mathcal{S}} \exp(-\lambda \mathrm{Dist}(\mathcal{S}, t))}_{\text{Knowledge decay}} \cdot \underbrace{\mathrm{Synergy}(\mathcal{S})^\eta}_{\text{Synergy amplification}}
\end{equation}

The knowledge decay term implements soft filtering through $\exp(-\lambda \cdot)$, while the synergy gain term strengthens the competitive advantage of high-quality teams through the exponent $\eta > 1$.

\paragraph{Step 2: Dynamic Sampling Probability Analysis}

Based on the dual-time-scale update rule:
$$
\begin{cases}
\alpha_{\mathcal{S}}^{(t+1)} = \gamma^{\Delta t} \alpha_{\mathcal{S}}^{(t)} + \underbrace{r_{\mathcal{S}}^{(t)} + \delta \cdot \mathrm{KM}(\mathcal{S}, t)}_{\text{Instant Feedback + Knowledge Memory}} \\
\beta_{\mathcal{S}}^{(t+1)} = \gamma^{\Delta t} \beta_{\mathcal{S}}^{(t)} + \underbrace{(1 - r_{\mathcal{S}}^{(t)}) + \delta \cdot (1 - \mathrm{KM}(\mathcal{S}, t))}_{\text{Negative Feedback + Knowledge Forgetting}}
\end{cases}
$$
We derive the \textbf{exponential convergence upper bound} for the sampling count:
$$
\mathbb{E}[N_{\mathcal{S}}(T)] \leq \frac{4\underline{L}^2 \log T}{\widetilde{\Delta}_{\mathcal{S}}^2} + \underbrace{\frac{2}{\gamma^{\Delta t} (1 - \gamma)} \cdot \mathbb{E}\left[\sum_{\tau=1}^T \mathrm{KM}(\mathcal{S}, \tau)\right]}_{\substack{\text{Knowledge-matching driven}\\\text{accelerated convergence term}}}
$$

\subsection{Algorithm Implementation and Complexity}
\label{sec:algorithm}

\paragraph{Core Implementation Modules}

\begin{itemize}
    \item \textbf{Expert Subset Sampling}:
    \begin{equation}
        \mathcal{S}_t \sim \mathrm{ThompsonSampling}\left( 
            \frac{\alpha_{\mathcal{S}}^{(t)}}{\alpha_{\mathcal{S}}^{(t)}+\beta_{\mathcal{S}}^{(t)}} 
            \cdot \exp(-\lambda \mathrm{Dist}(\mathcal{S},t)) 
            \cdot \mathrm{Synergy}(\mathcal{S})^{\eta} 
        \right)
        \label{eq:thompson}
    \end{equation}
    Optimization implementation: The combinatorial space is compressed from $O(2^N)$ to $O\left(\frac{N^k}{k!}\right)$ through a greedy strategy.

    \item \textbf{Dynamic Parameter Update}:
    \begin{equation}
        \begin{cases}
            \alpha_{\mathcal{S}}^{(t+1)} = \gamma^{\Delta t} \alpha_{\mathcal{S}}^{(t)} + 
            \left[ r_{\mathcal{S}}^{(t)} + \delta \cdot \mathrm{KM}(\mathcal{S},t) \right] 
            \cdot \mathbb{I}_{\{\mathcal{S}=\mathcal{S}_t\}} \\[6pt]
            \beta_{\mathcal{S}}^{(t+1)} = \gamma^{\Delta t} \beta_{\mathcal{S}}^{(t)} + 
            \left[ 1 - r_{\mathcal{S}}^{(t)} + \delta \cdot (1 - \mathrm{KM}(\mathcal{S},t)) \right] 
            \cdot \mathbb{I}_{\{\mathcal{S}=\mathcal{S}_t\}}
        \end{cases}
        \label{eq:param_update}
    \end{equation}
    where $\mathbb{I}$ is the indicator function, enabling sparse updates.

\end{itemize}

\paragraph{Complexity Analysis}

\begin{itemize}
    \item \textbf{Time Complexity}:
    $$
        \begin{aligned}
            \mathcal{T}(N,T) &= \underbrace{O\left( \tbinom{N}{k} \right)}_{\substack{\text{Initialization}\\ \text{(Pre-computation)}}} + T \cdot \Bigg[ \underbrace{O\left( \tbinom{N}{k} \right)}_{\substack{\text{Sampling + Evaluation}\\ \text{(Per step)}}} + \underbrace{O\left( \tbinom{N}{k} \log \tbinom{N}{k} \right)}_{\text{Sorting}} \\
            &\quad + \underbrace{O\left( |\mathcal{C}|^2 \right)}_{\substack{\text{Graph Update}\\ \text{(Dijkstra)}}} \Bigg] \\
            &= \boxed{ \widetilde{O}\left( T \cdot \left( \tbinom{N}{k} \log \tbinom{N}{k} + |\mathcal{C}|^2 \right) \right) }
        \end{aligned}
    $$
    \item \textbf{Space Complexity}:
    $$
        \begin{aligned}
            \mathcal{M}(N) &= \underbrace{O\left( \tbinom{N}{k} \right)}_{\substack{\text{Team Parameters}\\(\alpha,\beta)}} + \underbrace{O\left( |\mathcal{C}|^2 \right)}_{\substack{\text{Knowledge Graph}\\ \text{(Adjacency Matrix)}}} + \underbrace{O\left( W \cdot \tbinom{N}{k} \right)}_{\substack{\text{Sliding Window}\\ \text{(Depth $W$)}}} \\
            &\leq \boxed{ O\left( \tbinom{N}{k} + |\mathcal{C}|^2 \right) } \quad (\text{when } W \ll |\mathcal{C}|)
        \end{aligned}
    $$
\end{itemize}

\paragraph{Storage Optimization}
\begin{itemize}
    \item \textbf{Knowledge Graph Compression}: Adjacency matrix $\rightarrow$ adjacency list, reducing space from $O(|\mathcal{C}|^2)$ to $O(|\mathcal{C}| + |\mathcal{E}|)$.
    \item \textbf{Parameter Sharing}: Share $(\alpha, \beta)$ parameters for teams satisfying $\mathrm{Dist}(\mathcal{S}_i, \mathcal{S}_j) < \epsilon$.
    \item \textbf{Incremental distance updates via streaming updates}: Store only $\Delta \mathrm{Dist}$ instead of the full distance matrix, allowing for more efficient memory usage and reducing computational overhead.
\end{itemize}

\subsection{Summary}
The supplementary proofs, through systematic chapter definitions and key point organization, comprehensively support and extend the discussion of the knowledge-driven Dynamic Bayesian Multi-Armed Bandit (KABB) model presented in \cref{sec:method}. Each supplementary section corresponds to a specific part of the main text, covering critical content such as problem definitions, confidence-bound construction, regret-bound analysis, and algorithm and complexity analysis. These sections provide readers with a comprehensive resource for deeply understanding the theoretical foundations and implementation details of the KABB algorithm.

% \subsection{Prompts and Settings}



\end{document}