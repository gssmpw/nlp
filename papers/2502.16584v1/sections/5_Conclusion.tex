\section{Conclusion and Discussion}

% conclusion
% The \textbf{Audio-FLAN} dataset is the first to comprehensively integrate instruction-tuning datasets across the \textbf{Speech}, \textbf{Music}, and \textbf{Audio} domains, encompassing both \textit{understanding} and \textit{generation} tasks. This pioneering dataset consists of 23 major tasks and 78 minor tasks, with 16 major tasks dedicated to understanding and 7 major tasks focused on generation, with a total of 928.27 million instances.
% By covering tasks across the Speech, Music, and Audio domains, the Audio-FLAN dataset enables the development of multimodal models that can seamlessly handle a variety of input-output formats, such as text, audio, and their combinations. The inclusion of both understanding and generation tasks not only broadens the scope of the dataset but also provides a robust foundation for training a unified model capable of tackling complex real-world scenarios in the audio domain.

 The \textbf{Audio-FLAN} dataset represents a groundbreaking contribution to the audio domain by enabling \textit{instruction-tuning} for both \textbf{understanding} and \textbf{generation} tasks across the \textbf{speech}, \textbf{music}, and \textbf{audio} domains. This pioneering dataset consists of 23 major tasks and 80 minor tasks, with 16 major tasks dedicated to understanding and 7 major tasks focused on generation, totaling 108.5 million instances. By covering a wide array of tasks from speech recognition and emotion detection to music generation and audio event recognition, the \textbf{Audio-FLAN} dataset provides a comprehensive foundation for developing unified models that can handle both understanding and generation across multiple audio domains. This dataset is designed to support instruction-tuning, empowering models to follow complex audio instructions with minimal task-specific data. It paves the way for zero-shot generalization, enabling models to perform well on unseen tasks within and across domains, much like the advancements seen in text and vision models.

% data bias, reason, and potential impacts on model
% The \textbf{Audio-FLAN} dataset exhibits an imbalance in instances across tasks and domains, with \textit{understanding} tasks, particularly in the \textbf{Speech} domain, dominating over \textit{generation} tasks. The \textbf{Speech} domain also contains the largest share of instances, while the \textbf{Music} and \textbf{Audio} domains have significantly fewer. Understanding tasks benefit from well-established datasets and are easier to label, while generation tasks require more complex data creation (e.g., text-to-audio or music generation). This imbalance may lead to models being biased towards understanding tasks, affecting their generalization ability when applied to generation tasks or less-represented domains. The lack of large datasets in the \textbf{Music} and \textbf{Audio} domains further complicates training models capable of handling both understanding and generation tasks across all three domains. 

The \textbf{Audio-FLAN} dataset, while a major step towards unifying understanding and generation tasks across the speech, music, and audio domains, exhibits an imbalance in instance distribution. Understanding tasks, particularly in the speech domain, dominate the dataset, benefiting from well-established datasets and easier labeling. In contrast, generation tasks, such as text-to-audio or music generation, are more complex and less represented. This imbalance results in a greater number of instances in the speech domain, while the music and audio domains have fewer. This skew may lead to models being biased toward understanding tasks, potentially impacting their generalization to generation tasks or underrepresented domains. 

% future work: chat data, cover more task, understanding and generation model
% Future Work

% While the \textbf{Audio-FLAN} dataset offers a comprehensive resource for training unified models across the \textbf{Speech}, \textbf{Music}, and \textbf{Audio} domains, future iterations could address key areas for improvement. One such area is the balance of tasks and instances, particularly with \textit{understanding} tasks dominating the dataset, especially in the \textbf{Speech} domain. Future work could focus on generating more data for \textit{generation} tasks, particularly in the \textbf{Music} and \textbf{Audio} domains, to ensure a more even distribution and better model generalization. Additionally, incorporating conversational or chat data would enable the development of models capable of both understanding and generating speech, music, and engaging in real-time dialogue, making the dataset more applicable for interactive multimodal applications such as voice assistants and conversational AI. Addressing these areas could significantly enhance the dataset’s ability to support robust, versatile models for both traditional and emerging tasks in the audio domain.

% Future updates should focus on balancing the distribution of understanding and generation tasks, particularly by generating more data for \textit{generation} tasks in the \textbf{Music} and \textbf{Audio} domains. Additionally, integrating conversational data would enable the development of models capable of both understanding and generating speech, music, and engaging in real-time dialogue, broadening the dataset’s applicability to voice assistants and interactive AI systems.

Future work should focus on balancing the distribution of tasks across domains, ensuring a more even representation between \textit{understanding} and \textit{generation} tasks, especially in the music and audio domains. Additionally, expanding the dataset to include more tasks and incorporating additional datasets will strengthen the audio domain’s instruction-tuning capabilities, enhancing the development of unified models that can handle both understanding and generation tasks with improved zero-shot performance. Furthermore, integrating conversational data will be crucial for equipping models with the ability to engage in dynamic, real-time dialogue, broadening the dataset’s applicability to intelligent virtual agents and multimodal interaction systems.