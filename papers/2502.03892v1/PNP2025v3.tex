\documentclass[a4paper,11pt]{article}

\usepackage{times}                       % 使用 Times New Roman 字体
\usepackage{CJK,CJKnumb,CJKulem}         % 中文支持宏包
\usepackage{color}
\usepackage{appendix}
\usepackage{diagbox}
\usepackage{multirow}
\usepackage{booktabs}
\usepackage{threeparttable}             % 支持彩色
\usepackage{caption}
\usepackage{graphics}
\usepackage{epstopdf}
\usepackage{float}
%\usepackage{subfigure}
%\usepackage{subcaption}

\usepackage{chngcntr}
\usepackage{hyperref}

\hypersetup{hypertex=ture,
colorlinks=true,
linkcolor=blue,
anchorcolor=blue,
citecolor=blue}



%\newcommand{\figref}[1]{\figurename~\ref{#1}}

\captionsetup[figure]{name={Fig.},labelsep=period}
\captionsetup[table]{name={TABLE},labelsep=period}
\counterwithin{figure}{section}
\counterwithin{table}{section}


\newcommand\bbR{\mathbb{R}}
\newcommand\bbC{\mathbb{C}}
\newcommand\bbN{\mathbb{N}}
\newcommand\schwartz{\mathcal{S}}


\newcommand\BigNorm[1]{ \left \| #1 \right \| }
\newcommand\pd[2]{\dfrac{\partial {#1}}{\partial {#2}}}
\newcommand\dd{\mathrm{d}}
\newcommand\opd[2]{\dfrac{\dd {#1}}{\dd {#2}}}

\newcommand\mN{{\mathcal N}}
\newcommand\rC[2]{{\rm{C}}_{{#1},{#2}}}
\newcommand\rmC{{\rm{C}}}

\newcommand\ri{{\rm{i}}} %????
\newcommand\MatL{{L}}
\newcommand\MatR{{R}}
\newcommand\MatV{ {\mathcal{V}} }
\newcommand\Thetaij{ \Theta_{ij} }
\newcommand\Thetafij{\Theta_{fij}}

\newcommand\NRxx{NR$xx$}
\newcommand\bNRxx{\texorpdfstring{{\bf NR}$\boldsymbol{xx}$}{NR$xx$}}

\setlength{\oddsidemargin}{0cm}
\setlength{\evensidemargin}{0cm}
\setlength{\textwidth}{150mm}
\setlength{\textheight}{230mm}

\newcommand\note[2]{{{\bf #1}\color{red} [ {\it #2} ]}}
%\newcommand\note[2]{{ #1 }} % using this line in the formal version
\newcommand\comment[1]{}



\newcommand{\bra}[1]{\ensuremath{\left\langle#1\right|}}
\newcommand{\ket}[1]{\ensuremath{\left|#1\right\rangle}}
\newcommand{\bracket}[2]{\ensuremath{\left\langle #1 \middle| #2 \right\rangle}}
\newcommand{\matrixel}[3]{\ensuremath{\left\langle #1 \middle| #2 \middle| #3 \right\rangle}}




%——————————– 其他宏包——————————–
\usepackage{amsmath,amsthm,amsfonts,amssymb,bm} % 数学宏包
\usepackage{graphicx,psfrag}                    % 图形宏包
%\usepackage{makeidx}                            % 建立索引宏包
%\usepackage{listings}
\usepackage{float}                          % 源代码宏包
\usepackage{subfigure}
\numberwithin{equation}{section}
\usepackage[capitalise]{cleveref}

%\usepackage[colorlinks,
 %           linkcolor=blue,
  %          anchorcolor=blue,
   %         citecolor=blue]{hyperref}

\newtheorem{theorem}{Theorem}[section] %%定理
\newtheorem{axiom}{Axiom}[section]     %%公理
\newtheorem{lemma}{Lemma}[section]     %%% 引理
\newtheorem{proposition}{Proposition}[section]   %%% 命题
\newtheorem{corollary}{Corollary}[section]    %%%% 推论
\newtheorem{remark}{Remark}[section]
\newtheorem{example}{Example}[section]

%——————————— 正文———————————–
\begin{document} % 开始正文



\date{}
\title{A class of positive-preserving, energy stable and high order numerical schemes
for the Poission-Nernst-Planck system}
\author{~~Waixiang Cao
\thanks{School of Mathematical Sciences, Beijing Normal University,
  Beijing 100875, China (caowx@bnu.edu.cn).},
 ~~Yuzhe Qin\thanks{School of Mathematics and Statistics, Shanxi University
  Taiyuan 030006, China.},
  ~~Minqiang Xu\thanks{Corresponding Author. School of Mathematical Sciences, Zhejiang University of Technology, Hangzhou, 310023, China(mqxu@zjut.edu.cn).} }
\maketitle
%%%%%%%%%%%%%%%%%%%%%%%%%%%%%%%%%%%%%%%%%%%%%%%%%%%%%%%%%%%%%%%%%%%%%%%%%%%%%%%%%%%%%%%%%%%%%%%%%%%%%%%%%%%%%%%%%%%%%%%%%%

\textbf{Abstract:}
    In this paper, we introduce and analyze a class of numerical schemes that demonstrate remarkable superiority in terms of efficiency, the preservation of positivity, energy stability, and high-order precision to solve the time-dependent Poisson-Nernst-Planck (PNP) system, which is
    as a highly versatile and sophisticated model and accommodates a plenitude of applications in the emulation of the translocation of charged particles across a multifarious expanse of physical and biological systems.  The numerical schemes presented here are based on the energy variational formulation. It allows the PNP system to be reformulated as a non-constant mobility $H^{-1}$ gradient flow, incorporating singular logarithmic energy potentials. To achieve a fully discrete numerical scheme, we employ a combination of first/second-order semi-implicit time discretization methods, coupled with either the $k$-th order direct discontinuous Galerkin (DDG) method or the finite element (FE) method for spatial discretization. The schemes are verified to possess positivity preservation and energy stability. Optimal error estimates and particular superconvergence results
for the fully-discrete numerical solution are established. Numerical experiments are provided to showcase the accuracy, efficiency, and robustness of the proposed schemes.





\textbf{Keywords:}   Poission-Nernst-planck (PNP) system,  positive-preserving,
 energy stable, error estimates, superconvergence

\textbf{AMS subject classifications}  65M12, 65M15, 65M70

% \quad\noindent {{\bf {AMS subject classifications}}. 65M12, 65M15, 65M70.}



  \section{Introduction}
%Suppose there are $n$ kinds of solutes in the solution,
%then we can write Poisson-Nernst-Planck(PNP) equations as follows,
In this paper,  we propose a class of positive-preserving, energy stable and high order  numerical schemes
for  solving  the Poisson-Nernst-Planck (PNP) system:
\begin{subequations}\label{eqn: system0}
\begin{align}
&\frac{\partial c_{i}}{\partial t} = \nabla \cdot \left( D_{i} \left( \nabla c_{i}
+ q_{i} c_{i} \nabla \phi \right)\right),
& {\rm in}\   \Omega, \label{eqn: ci} \\
&-\epsilon^{2} \Delta \phi = \sum_{i=1}^{n} q_{i} c_{i},
& {\rm in}\   \Omega, \label{eqn: phi}\\
& c_{i}(0,x)=c_{i0}(x),  & {\rm in}\   \Omega,
\end{align}
\end{subequations}
where $c_{i}:=c_i(x,t)$ is the concentration of $i$th particle,
$q_{i}$ is the valency of $i$th particle, $n$ is the number of species,
$\phi$ is the electric potential,
$D_{i}$ is the diffusion coefficient of $i$th particle and
$\epsilon$ is the dielectric constant respectively.


The Poisson-Nernst-Planck (PNP) system emerges as one of the most ubiquitously and meticulously scrutinized
models within the domain of charged particle translocation across a diverse array of physical and biological
manifestations. This incorporates the free electrons within semiconductors, as cited in
\cite{Jerome1995,Markowich1986,Markowich1990};
fuel cells as expounded in \cite{Nazarov2007, Promislow2001};
ionic particles in electrokinetic fluids as explicated in \cite{Ben, Hunter, Lyklema};
the phase separation and polarization events affiliated with ionic liquids \cite{Gavish}
the ion channels extant in cell membranes \cite{Bazant,Eisenberg,Eisenberg1}.
The PNP system, with its nonlinearity and strong coupling, makes efficient simulation both highly interesting and extremely challenging.
A major hurdle in developing numerical schemes for the PNP equation is ensuring density/concentration positivity (or non-negativity),
as negative ion concentrations would violate physical principles and be physically implausible.



 It is generally conceded that the solutions of the PNP system are endowed with the properties of mass conservation, positive concentration or density,
 and energy dissipation. It is highly advantageous and earnestly desired that the numerical solutions preserve as many of the intrinsic PNP system properties
 as feasible, including mass conservation, positivity, and energy stability. During the past several decades, substantial research endeavors have been dedicated
 and employed in the sphere of devising efficient, positivity-preserving, and energy-stable numerical schemes for PNP equations.
One can refer to \cite{p1,p2,p4,p5,p6,p7,p8,p11,Liu1} for a positive-preserving analysis, to \cite{e1,e2,e3,e4} for a convergence analysis, and to \cite{energy1,energy2,p10,Liu}
for an energy stability analysis.  Additionally, \cite{o1,o2,o3,o4,o5,o6} can be referred to for other numerical schemes.
Nevertheless, none of these works have managed to integrate the properties of mass conservation, positivity, and energy dissipation along with optimal error estimates.
Instead, these properties have only been partially fulfilled and addressed thus far.
Broadly, it is highly arduous to numerically attain all the features simultaneously. Just recently,
Liu et al proposed a finite difference numerical scheme for the PNP equations, which can preserve positivity, uphold energy stability, and guarantee convergence \cite{p3}.
Within this scheme, all the aforesaid properties are met at the discrete level. However, the temporal precision of this scheme is merely first-order, and the spatial precision is second-order.


The core aim of the current work is to design an effective, positivity-preserving, and energy-stable numerical  approach for the PNP system
in a broader and more inclusive setting: involving multiple species, achieving high-order accuracy, yielding optimal error estimations,
and permitting superconvergence investigations. To achieve this goal, we initially recast the original PNP equation into its equivalent variational form,
which guarantees that the discrete solution adheres to the original energy dissipation property.
Subsequently, a semi-implicit time discretization is formulated, wherein the mobility function is updated explicitly,
while the chemical potential component is handled implicitly.
Due to the presence of a singular logarithmic energy potential in the energy variational formulation,
the Gauss-Lobatto collocation method is adopted to deal with the logarithmic term during spatial discretization. In regard to the spatial discretization of the elliptic equations,
the FEM (for continuous approximation) and the DDG method (for discontinuous approximation) are employed.
It has been proved that the fully discrete numerical solution displays positive cell averages and positive values at the collocation points.
This positivity characteristic ensures the well-posedness of the numerical scheme.
Finally, an optimal rate convergence analysis of the proposed numerical schemes is presented.
Through constructing a particular projection of the exact solution, it's proven that the numerical solution
has superconvergence towards those projections of the exact solution.
Notably, the manifestation of the superconvergence phenomenon for the gradient approximation at Gauss points
 is being reported for the first time ever.

%  The scheme is constructed
%  by  integrating the first/second order semi-implicit time discretization with the
%  FEM (for continuous approximation) and DDG method (for discontinuous approximation)
%   for spatial discretization.

  The principal contribution of the paper lies in that:
  1)
  We show that the fully-discrete solution is positive at Gauss-Lobatto points as well as cell-average.
  This finding extends the prior positivity analysis in \cite{p3,p10} from the cell-average scope to a point-wise level.
  2) Unlike the energy stability analysis presented \cite{energy0,p10},
the energy stability discussed in this paper pertains to the original energy functional, rather than a modified or numerically approximated energy.
  3) The schemes simultaneously incorporate the desired properties including mass conservation, positivity,
  energy stability, optimal error estimates and superconvergence analysis.
  4) Some interesting superconvergence results are studied and reported for the PNP equations
  for the first time. Additionally, it should be noted that the methodology put forward in this paper
  can also be applied to the multi-dimensional situation.


The remainder of the paper is structured as follows.
In Section 2, the fully discrete numerical scheme for the PNP equation \eqref{eqn: system0} is presented.
In Section 3, the mass conservation and positivity properties of the numerical solution are studied,
and it is proven that the discrete solution has positive cell averages and values at Gauss-Lobatto points.
In Section 4, the energy stability of our algorithm is proved.
Optimal error estimates and some superconvergence results are established in Section 5.
Finally, in Section 6, several carefully designed numerical examples are provided to support our theoretical findings.









\section{Numerical schemes for the PNP system}
Let  $\Omega$ be a bounded domain in ${\mathbb R}^d$.
We first reformulate the PNP system \eqref{eqn: system0} into its
equivalent form as follow:
\begin{subequations}\label{eqn: system1}
\begin{align}
&\frac{\partial c_{i}}{\partial t} = \nabla \cdot \left( D_{i} \left( c_i \nabla p_i \right)\right),
& x\in \Omega, t>0,\label{eqn: ci1}\\
&p_i=q_i\phi+{\rm log}c_i+1,&   x\in \Omega, t>0,\\
&-\epsilon^{2} \Delta \phi = \sum_{i=1}^{n} q_{i} c_{i},  & x\in \Omega, t>0,& \\
&c_{i}(0,x)=c_{i0}(x),  &    x\in \Omega.
 \label{eqn: phi1}
\end{align}
\end{subequations}
We consider the periodic boundary condition or the following  boundary condition
\begin{equation} \label{eqn: bcond}
\frac{\partial \phi}{\partial \bm{n}} = \frac{\partial c_{i}}{\partial \bm{n}} = 0, \ \ \
 {\rm on}\  \partial \Omega,
\end{equation}
 where $ {\bm n}$ represents the unit exterior normal vector on the boundary $\partial\Omega$.
To simplify our analysis, we assume $\epsilon=1, D_i=1$,  and  we construct our algorithm  based on
   the periodic boundary condition.
 Note that this is not essential and our algorithm and analysis can  also be extended to the
  boundary condition \eqref{eqn: bcond}.



\subsection{The  semi-discrete spatial discretization}
Let ${\mathcal T}_h$ be a partition of the polygonal domain $\Omega$, and $h$ denotes the mesh size of all the elements within  ${\mathcal T}_h$.
Denote by ${\mathcal E}_h$ the edge set of ${\mathcal T}_h$, and let ${\mathcal E}_h^0\subset {\mathcal E}_h$ be the set of all interior edges. The set of boundary edges is denoted as ${\mathcal E}_h^B$.
The diameter of the edge $e\in {\mathcal E}_h$ is denoted as $h_e$.

  For any $e\in {\mathcal E}_h^0$, let $K_1$ and $K_2$ be two neighboring elements
 sharing the common edge $e$. Denote by ${\bm n}_e$ the normal vector of $e$ which is assumed to be oriented from
    $K_1$ to $K_2$.
   For any function $w$, we denote by $\{w\}$ and $[w]$ the average and the jump of $w$ on $e$, respectively. That is,
 \[
    \{w\}=\frac 12(w|_{K_1}+w|_{K_2}),\ \ \
    [w]=w|_{K_2}-w|_{K_1}, \  {\rm on}\  e\in \partial K_1\cap
    \partial K_2.
 \]

   Denote by $V_h$ the discrete finite element space associated with ${\mathcal T}_h$. Here, we consider two
  classes of semi-discrete spatial discretization for \eqref{eqn: system1}, namely, the continuous approximation
  FE method and the discontinuous approximation DDG method.


    Denote by $L^2_{per}$ the function space in $L^2$ space
     satisfying the periodic boundary condition.
Define
\begin{equation}
    V^d_h=\{v\in L^2(\Omega): v|_{K}\in\mathbb {P}^k {\rm or}\ \mathbb {Q}^k, \forall K\in {\mathcal T}_h\},\
     V_h^0=V^d_h\cap L^2_{per},
\end{equation}
where  $\mathbb {P}^k$ represents the space of polynomial functions of degree at most $k$, and
$\mathbb {Q}^k$ denotes the tensor product polynomial space of degree not more than $k$ for each variable.

Then the DDG method for the PNP system \eqref{eqn: system1} is as follows: find
 $c_{ih},p_{ih},\phi_h\in V^0_h$ such that for all $v,w,\theta\in V^0_h$
 \begin{eqnarray}\label{semi:10}
    &&\int_{K} \partial_t c_{ih}v dx=-\int_{K}c_{ih} \nabla p_{ih} \nabla v dx+\int_{\partial K}
    \left(\{c_{ih}\}(\widehat{\partial_np_{ih}} v+(p_{ih}-\{p_{ih}\})\partial_n v\right) ds,\ \ \ \ \\\label{semi:20}
    &&\int_{K}  p_{ih} w dx=\int_{K}  \left(q_i \phi_h +{\rm log} c_{ih}+1 \right) w dx,\\ \label{semi:3}
    && \int_{K} \nabla \phi_h\nabla \theta dx-\int_{\partial K}
    \left((\widehat{\partial_n\phi_{h}} \theta+(\phi_h-\{\phi_{h}\})\partial_n \theta\right) ds=\sum_{i=1}^n
    \int_{K} q_i c_{ih} \theta dx,
 \end{eqnarray}
 where $\widehat {\partial_n w}$ denotes the numerical fluxes,
 which is defined on the interface $e$ by
 \[
    \widehat {\partial_n w}=\beta_0\frac{[w]}{h_e}+\{\partial_n w\}+\beta_1h_e[\partial_n^2 w].
 \]
   Here $(\beta_0,\beta_1)$ are the coefficients of the penalty (defined later),  which should be carefully chosen to ensure the stability or satisfy certain positivity-principle property of the DDG method (see  e.g.,\cite{Liu-2015}).

% ----------------------------------------
  % (beta_0, beta_1)
%satisfying the following stability condition \cite{Liu-2015}
%\begin{equation*}
%	\beta_0 \geq \Gamma (\beta_1),
%\end{equation*}
%with
%\begin{equation*}
%	 \Gamma (\beta_1)= \displaystyle {\sup_{\substack {v\in \mathbb{P} _{k-1}(
%	 \xi)\\ \xi\in [-1,1]}}} \displaystyle\frac{2(v(1)-2\beta_1\partial_{\xi}v(1))^2}{\int_{-1}^1 v^2(\xi) \mathrm d\xi}.
%\end{equation*}
%-----------------------------
   As for the continuous approximation, we define its associated  finite element space as follows:
\begin{equation}
    V^c_h=\{v\in C^0(\Omega): v|_{K}\in\mathbb {P}^k {\rm or}\ \mathbb {Q}^k,  \frac{1}{|\Omega|} \int_{\Omega} v=const,
    \ \forall K\in {\mathcal T}_h\},\ \ V^1_h=V^c_h\cap L^2_{per}.
\end{equation}
Then the semi-discrete FE method for the PNP system \eqref{eqn: system1} is: find
 $c_{ih},p_{ih},\phi_h\in V^1_h$ such that for all $v,w,\theta\in V^1_h$
 \begin{eqnarray}\label{semi:01}
    &&\int_{\Omega} \partial_t c_{ih}v dx=-\int_{\Omega}c_{ih} \nabla p_{ih} \nabla v dx,\\\label{semi:02}
    &&\int_{\Omega}  p_{ih} w dx=\int_{\Omega}  \left(q_i \phi_h +{\rm log} c_{ih}+1\right) w dx,\\ \label{semi:03}
    && \int_{\Omega} \nabla \phi_h\nabla \theta dx=\sum_{i=1}^n
    \int_{\Omega} q_i c_{ih} \theta dx.
 \end{eqnarray}




 \subsection{The modified semi-discrete spatial discretization}

Due to the logarithmic factor ${\rm log} c_{ih}$
   appears in  \eqref{semi:20} and \eqref{semi:02},
    numerical quadrature is usually required to compute
   the exact integral $\int_{K}{\rm log} c_{ih} w dx$ in practical calculation.
   A similar situation applies to \eqref{semi:10} and \eqref{semi:01}.
%    and thus
%numerical quadrature error  appears.
%Similarly, the right hand side of  \eqref{semi:02}  and \eqref{semi:1} are usually calculated by numerical quadrature  in the practical computing.
To avoid this numerical quadrature error,  we make slight modifications to our numerical schemes.  Specifically,
 the equations \eqref{semi:20} and \eqref{semi:02} are replaced by
 some collocation conditions, while \eqref{semi:20} and \eqref{semi:02} are substituted by
  the numerical quadrature formulation.
To be more precise, we denote by $g_{K,j}, 1\le j\le n_{k}$ the collocation points in each $K\in {\mathcal T}_h$
 and $\omega_{K,j}$ the corresponding quadrature weights associated with $g_{K,j}$, with $n_{k}$ being the number of collocation points. Let
\[
   S_{K}=\{g_{K,j}: 1\le j\le n_{k}\},\ \  S=\{S_{K}:K\in {\mathcal T}_h\},\ \
   S_{\partial K}=S_{K}\cap \partial K.
\]
We should note that the cardinality of set $S$ ought to be equal to the degree of freedom of either $V_h^0$ or $V_h^1$ to ensure in order to ensure the well-posedness of the numerical scheme.

%Then the equation \eqref{semi:20} and \eqref{semi:02} are modified to the following
%collocation conditions:
% \[
%      p_{ih}(g_{K,j})= (q_i \phi_h +{\rm log} c_{ih}  )(g_{K,j}), \ \  \forall g_{K,j}\in S_{K}.
% \]
 Define
\begin{eqnarray*}
  && \langle f,v\rangle_{K}=\sum_{j=1}^{n_{k}} (fv)(g_{K,j})\omega_{K,j},\ \
    \langle f,v\rangle_{\partial K}=\sum_{g_{K,j}\in S_{\partial K}} (fv)(g_{K,j})\omega_{K,j},\
    \langle f,v\rangle=\sum_{K\in{\mathcal T}_h}\langle f,v\rangle_{K},\\
  &&   ( f,v )_{K}=\int_{K} (fv)dx,\ \
   ( f,v)_{\partial K}=\int_{\partial K} (fv)ds,\ \ (f,v)=\sum_{K\in{\mathcal T}_h}(f,v)_K.
\end{eqnarray*}
%   where $g_{\tau,j}$ the Gauss-points in the interval $\tau$ and
% \[
%    S_{\tau}=\{g_{\tau,j}:  1\le j\le \mathbb Z_k\},\ \  \mathbb Z_k=\frac{k(k+1)}{2} \ {\rm for}\ \mathbb P^k, (k+1)^2 \ {\rm for }\ \mathbb Q^k.
% \]
  Given any positive function $\psi(x)$,  let
 \[
    a_{\psi}(\cdot,\cdot)=\sum_{\tau\in{\mathcal T}_h} a_{\psi,K}(\cdot,\cdot),
 \]
  where for DDG method,
 \[
     a_{\psi,K}(u,v)=\langle\psi \nabla u, \nabla v \rangle_{K}- \langle
 \{\psi \},  \widehat{\partial_n u} v+(u-\{u\})\partial_n v \rangle_{\partial K}.
    %\ \ a(u,v)=\sum_{\tau\in {\mathcal T}_h}a_{\tau}(u,v).
 \]
 And for the FE method,
 \[
    a_{\psi ,K}(u,v)=\langle\psi  \nabla u, \nabla v \rangle_{K}.
 \]
   We set $a(\cdot,\cdot)=a_{\psi}(\cdot,\cdot)$ when
   $\psi=1$.  Define $V_h$ as the discrete finite element space associated with DDG method and FE method, that is,
   $V_h=V_h^0$ for the DDG and $V_h=V_h^1$ for the FEM.

  Then both the modified   simi-discrete DDG and FE schemes can be rewritten as: find
$c_{ih},p_{ih},\phi_h\in V_h$ such that for all $v, w\in V_h, z\in S$,
 \begin{eqnarray}\label{semi:1}
    &&\langle \partial_t c_{ih},v \rangle_{K}=-a_{c_{ih}}(p_{ih}, v), \\\label{semi:2}
    &&   p_{ih}(z)= (q_i \phi_h +{\rm log} c_{ih} +1 )(z), \\ \label{semi:3}
    && a ( \phi_h, w) =\sum_{i=1}^n
    \langle q_i c_{ih}, w \rangle_{K}.
 \end{eqnarray}
   % Similarly, the  modified simi-discrete FE scheme is to find
%$c_{ih},p_{ih},\phi_h\in V^1_h$ such that for all $v,\eta\in V^1_h$
% \begin{eqnarray}\label{semi:001}
%    &&\langle \partial_t c_{ih},v\rangle_{\Omega}=- \langle c_{ih}\nabla p_{ih}, \nabla v \rangle_{\Omega},\ \ \ \ \ \\\label{semi:002}
%    &&  p_{ih} (z)=(q_i \phi_h +{\rm log} c_{ih}+1)(z),\ \ \forall z\in S, \\ \label{semi:003}
%    && \langle  \nabla \phi_h, \nabla \eta \rangle_{\Omega}=\sum_{i=1}^n
% \langle q_i c_{ih}, \eta \rangle_{\Omega},
% \end{eqnarray}
%  where $\langle f,v\rangle_{\Omega}=\sum_{K\in {\mathcal T}_h} \langle f,v\rangle_{K}$.
In the subsequent part of this paper, our algorithm and theoretical analysis are are perpetually predicated on the modified schemes \eqref{semi:1}-\eqref{semi:3}.




\subsection{The fully-discrete numerical scheme}
  In this subsection, we propose the first-order and second-order
  semi-implicit schemes with respect to the time discretization.

  The  first-order
  semi-implicit  scheme is:  given $c_{ih}^m,p^m_{ih},\phi^m_h\in V_h$, find
$c_{ih}^{m+1},p^{m+1}_{ih},\phi^{m+1}_h\in V_h$ such that for all $v,w\in V_h$

   \begin{eqnarray}\label{ci1-femdg}
    &&\langle \frac{c^{m+1}_{ih}-c^{m}_{ih}}{\tau},v \rangle=-a_{c^m_{ih}} ( p^{m+1}_{ih},v),\ \
    \forall v\in V_h, \\ \label{ci2-femdg}
    && p^{m+1}_{ih} (z)= \left(q_i \phi^{m+1}_h +{\rm log} c^{m+1}_{ih}+1\right)(z),\ \ \forall z\in S,\ \ \ \ \ \\\label{ci3-femdg}
    && a(\phi^{m+1}_h,w)=\langle q_i c^{m+1}_{ih}, w\rangle,\ \forall w\in V_h.
 \end{eqnarray}
   Here $\tau=t_{m+1}-t_m$ denotes the time step size.
%\begin{eqnarray}\label{ci1}
% &&\langle   \frac{c^{m+1}_{ih}-c^{m}_{ih}}{\tau},v \rangle_{K}
% =-\langle c^m_{ih} \nabla p^{m+1}_{ih}, \nabla v\rangle_{ K}+\langle \{c^m_{ih}\}
% \widehat{\partial_np^{m+1}_{ih}}, v\rangle_{\partial K}
%   +\langle  \{c^m_{ih}\}(p^{m+1}_{ih}-\{p^{m+1}_{ih}\}), \partial_n v\rangle_{ \partial K}, \quad\quad \\ \label{ci2}
%    && p^{m+1}_{ih}(g_{K,j})= (q_i \phi^{m+1}_h +{\rm log} c^{m+1}_{ih} +1 )(g_{K,j}),\ \ \forall g_{K,j}\in S_{K},\ \ \ \ \ \\\label{ci3}
%    && \langle \nabla \phi^{m+1}_h,\nabla w  \rangle_{K}-
%    \langle \widehat{\partial_n\phi^{m+1}_{h}}, w\rangle_{\partial K}+\langle\phi^{m+1}_h-\{\phi^{m+1}_{h}\}, \partial_n w\rangle_{ \partial K}=\sum_{i=1}^n
%    \langle q_i c_{ih}, w \rangle_{K}.
% \end{eqnarray}
  To obtain the second-order   semi-implicit scheme, we make a slight modification to  \eqref{ci1-femdg} as follows:
 \begin{eqnarray}\label{ci4}
 \langle \frac{\frac{3}{2}c^{m+1}_{ih}-2c^{m}_{ih}+\frac 12c_{ih}^{m-1}}{\tau}, v\rangle= -a_{(2c^m_{ih}-c_{ih}^{m-1})} ( p^{m+1}_{ih},v),\ \ \forall v\in V_h.
  \end{eqnarray}
 Then the  second-order   semi-implicit numerical scheme is: given $c_{ih}^m,p^m_{ih},\phi^m_h\in V_h$, find
$c_{ih}^{m+1},p^{m+1}_{ih},\phi^{m+1}_h\in V_h$ such that equations \eqref{ci2-femdg}-\eqref{ci4} are satisfied.
  % Obviously, mass conservation is still ensured by choosing $v=1$ in the above equation.

%  Similarly,
%   the first-order
%  semi-implicit  FE scheme is:  given $c_{ih}^m,p^m_{ih},\phi^m_h\in V^1_h$, find
%$c_{ih}^{m+1},p^{m+1}_{ih},\phi^{m+1}_h\in V^1_h$ such that  for all $v,w\in V_h^1$
%\begin{eqnarray}\label{ci1-fem}
%    &&\langle \frac{c^{m+1}_{ih}-c^{m}_{ih}}{\tau}, v \rangle_{\Omega}=-\langle c^m_{ih} \nabla p^{m+1}_{ih} ,\nabla v \rangle_{\Omega},
%    \quad\quad \ \ \\ \label{ci2-fem}
%    && p^{m+1}_{ih} (z)= \left(q_i \phi^{m+1}_h +{\rm log} c^{m+1}_{ih}+1\right)(z),\ \ \forall z\in S,\ \ \ \ \ \\\label{ci3-fem}
%    && \langle \nabla \phi^{m+1}_h, \nabla w \rangle_{\Omega}=\sum_{i=1}^n
% \langle  q_i c^{m+1}_{ih}, w \rangle_{\Omega}.
% \end{eqnarray}
%  The second-order   semi-implicit FE scheme is:
%  given $c_{ih}^m,p^m_{ih},\phi^m_h\in V^1_h$, find
%$c_{ih}^{m+1},p^{m+1}_{ih},\phi^{m+1}_h\in V^1_h$ satisfying
% \begin{equation}\label{ci4-fem}
%  \langle \frac{\frac{3}{2}c^{m+1}_{ih}-2c^{m}_{ih}+c_{ih}^{m-1}}{\tau}, v\rangle_{\Omega}=-\langle (2c^m_{ih}-c_{ih}^{m-1}) \nabla p^{m+1}_{ih}, \nabla v \rangle_{\Omega}
% \end{equation}
%  and \eqref{ci2-fem}-\eqref{ci3-fem}, for all  $v,\eta\in V_h^1$.
%





\section{Mass conservation and positive-preserving}

   In this section, we study the mass conservation and positive-preserving properties of our fully-discrete numerical schemes.
   To simplify our analysis and make the idea clearer, in the rest of this paper,
   we focus on the $\mathbb Q^k$ case and the first-order semi-implicit scheme within a two dimensional setting.
   In addition, the collocation points $g_{K,j}$ are taken as the  Gauss-Lobatto points in each element $K$.
 %  We begin with some preliminaries.
%      Define $V_h$ the discrete finite element space associated with DDG method and FE method, i.e.,
%   $V_h=V_h^0$ for DDG and $V_h=V_h^1$ for FEM.
  % where $(u,v)=\sum_{K\in{\mathcal T}_h}(u,v)_{K}$ with  $(u,v)_{K}=\int_{K} uv dx$.
%  For the purpose of accuracy and error estimates, the collocation points in each element $K$ are taken
%  as the Gauss-Lobatto points,
   i.e.,
 \[
    S_{K}=\{g_{K,l}=(g^x_i,g^y_j): 1\le i,j\le k+1, 1\le l\le n_{K}=(k+1)^2\},
 \]
 where $g_i^x,g_j^y$ denote the Gauss-Lobatto points along the $x$ and $y$ directions, respectively. Our subsequent analysis will demonstrate that this special choice of collocation points gives rise to an optimal error estimates as well as some
 superconvergence properties for the numerical solution.

 Multiplying  both sides of \eqref{ci2-femdg} by $\omega_{K,j}  w(g_{K,j})$ and summing up over all $j$ result in
 \begin{equation}\label{eq:2}
   \langle p_{ih}^{m+1}, w\rangle_{K}=\langle q_i \phi^{m+1}_h +{\rm log} c^{m+1}_{ih}+1, w\rangle_{K}.
 \end{equation}
   Similarly, by taking $w\in V_h$ as the Lagrange basis function associated with $g_{K,j}$, we can obtain \eqref{ci2-femdg}.  In other words,
   the two equations \eqref{ci2-femdg} and \eqref{eq:2} are equivalent.
  Note that the  the $(k+1)$-point
 Gauss-Lobatto numerical quadrature is exact for polynomials of degree not exceeding $2k$, which leads to
 \begin{equation}\label{eq:100}
    (w,v)_{K}= \langle w,v\rangle_{K},\ \
    (w,v)_{\partial K}= \langle w,v\rangle_{\partial K},\ \
    \forall w,v\in \mathbb Q^{2k}(K).
 \end{equation}
  Then \eqref{ci1-femdg}-\eqref{ci3-femdg} can be reformulated as
  \begin{eqnarray}\label{ci1-femdgm}
    && ( \frac{c^{m+1}_{ih}-c^{m}_{ih}}{\tau},v )=-a_{c^m_{ih}} ( p^{m+1}_{ih},v),\ \ \forall v\in V_h\ \ \\ \label{ci2-femdgm}
    && (p_{ih}^{m+1}, \theta)=( q_i \phi^{m+1}_h+1, \theta) +\langle {\rm log} c^{m+1}_{ih}, \theta\rangle, \ \ \forall\theta\in V_h, \\\label{ci3-femdgm}
    && a(\phi^{m+1}_h,w)=\sum_{i=1}^n( q_i c^{m+1}_{ih}, w),\ \forall w\in V_h.
 \end{eqnarray}










\subsection{Mass conservation}

  By substituting $v=1$ into equation \eqref{ci1-femdg} and making use of the periodic boundary condition, we can effortlessly derive that
 \[
    \int_{\Omega} c^{m+1}_{ih} dx= \int_{\Omega} c^{m}_{ih} dx=\int_{\Omega} c^{0}_{ih} dx.
 \]
  In other words, the fully-discrete numerical scheme \eqref{ci1-femdg}-\eqref{ci3-femdg} possesses the property of mass conservation.

 \subsection{Positivity-preserving}

In this subsection, we shall prove that the  fully-discrete numerical scheme has positive cell-average. Subsequently, we will utilize these positive cell-averages to design a positive limiter across the whole domain. Throughout this paper,  we adopt standard notations for Sobolev spaces For instance, we have $W^{m,p}(D)$ on a sub-domain $D\subset\Omega$, which is equipped with the norm $\|\cdot\|_{m,p,D}$ and the semi-norm $|\cdot|_{m,p,D}$. When $D=\Omega$, we omit the index $D$. In the case where $p=2$, we define $W^{m,p}(D)=H^m(D)$, $\|\cdot\|_{m,p,D}=\|\cdot\|_{m,D}$, and $|\cdot|_{m,p,D}=|\cdot|_{m,D}$. The notation $A\lesssim B$ implies that $A$ can be bounded by $B$ multiplied by a constant that is independent of the mesh size $h$ and the time step size $\tau$.

We start with some preliminary aspects. Firstly, we define a broken space as follows
\[
   {\mathcal H}_h=\{v\in L^2: v|_{K}\in H^1,\forall K\in{\mathcal T}_h \},
\]
and for all $v\in {\mathcal H}_h$, we define
   \[
     \|v\|_E=\left(\sum_{K\in {\mathcal T}_h} \int_{K}  |\nabla v|^2  dx+\sum_{e\in {\mathcal E}_h^0}\int_e \frac{1}{h_e}[v]^2 ds \right)^{\frac 12}.
 \]
We note that for the FE method, the energy norm is reduced to the standard semi-norm in the $H^1$ space due to the continuity in the FE space.

Secondly, we discuss the property of the bilinear form $a_{\psi}(\cdot,\cdot)$ for the semi-discretezation.
 We assume that
 \[
     0<\psi_0\le \psi\le \psi_1.
 \]
  Recalling the definition of $a_{\psi}(\cdot,\cdot)$, we can readily derive that
 \[
    |a_{\psi}(u,v)|\le \psi_1\langle \nabla u,\nabla u \rangle^{\frac 12}
    \langle \nabla v,\nabla v \rangle^{\frac 12},\ \
    a_{\psi}(v,v)\ge \psi_0\langle \nabla v,\nabla v\rangle.
\]
By making use of \eqref{eq:100}, we obtain that
\[
 |a_{\psi}(w,v)|\le \psi_1 \|w\|_E\|v\|_E,\ \
 |a_{\psi}(v,v)|\ge \psi_0 \|v\|^2_E,\ \ \forall w,v\in V_h.
\]
  In other words, the bilinear form $a_{\psi}(\cdot,\cdot)$ is continuous and coercive in the space $V_h$ for the FE method. As for the DDG method, by applying \eqref{eq:100} once again, we then obtain for all $v\in V_h$,
\begin{eqnarray*}
    a_{\psi}(v,v) &\ge & \sum_{K\in{\cal T}_h}  \left( \psi_0\langle \nabla v, \nabla v \rangle_{K}
    +\frac{\beta_0 \psi_0}{h}\langle [v], [v] \rangle_{\partial K}
    -  \psi_1 \big| \langle 2\{\partial_nv\}+\beta_1h[\partial_n^2 v], [v] \rangle_{\partial K} \big|\right) \\
    &=&  \sum_{K\in{\cal T}_h} \left( \psi_0( \nabla v, \nabla v )_{K}
    +\frac{\beta_0\psi_0}{h}( [v], [v] )_{\partial K}
    -  \psi_1\big| ( 2\{\partial_nv\}+\beta_1h[\partial_n^2 v], [v])_{\partial K}\big|\right).
\end{eqnarray*}
% Here in the second step, we have used the fact that
% Let  $\omega_{\tau,i}, 1\le i\le \mathbb Z_k$ be the Gauss-Lobatto weights associated with the collocation points $g_{\tau,i}$,
%  and define
%\[
%   \langle f,v\rangle_{\tau}=\sum_{i=1}^{n_{\tau}}\omega_{\tau,i}(fv)(g_{\tau,i}).
%\]
Therefore, if $(\beta_0,\beta_1)$ satisfy the stability
   condition (see, e.g., \cite{Liu-2015}),
 \begin{equation}\label{coer:1}
	\psi_0\beta_0 \geq \psi_1\Gamma (\beta_1),\ \ {\rm with}\ \    \Gamma (\beta_1)= \displaystyle {\sup_{\substack {v\in \mathbb{P} _{k-1}(
	 \xi)\\ \xi\in [-1,1]}}} \displaystyle\frac{2(v(1)-2\beta_1\partial_{\xi}v(1))^2}{\int_{-1}^1 v^2(\xi) \mathrm d\xi},
\end{equation}
then there exists a positive $\gamma_0$ (dependent on $\psi_0$) such that
 \begin{equation}\label{coer:2}
    a_{\psi}(v,v)\ge \gamma_0 \|v\|_E^2,\ \ \forall v\in V_h.
 \end{equation}
Unless otherwise specified, we always assume that the coefficients $(\beta_0,\beta_1)$ satisfy the condition \eqref{coer:1}.
Similarly, through a direct calculation from the Cauchy-Schwarz inequality and the inverse inequality, we can obtain the  following
continuity result:
 \begin{eqnarray}\label{conti:1}
     |a_{\psi}(w,v)|\le \gamma_1\|u\|_E\|v\|_E,\ \ \forall w,v \in V_h.
 \end{eqnarray}
Here $\gamma_1$ is a positive constant that depends on $\psi_1$.
% In other words,  both the bilinear form $a_{\psi}(\cdot,\cdot)$ of DDG and FEM  are continuous and coercive in  space $V_h$.


% As for FE method,  we  can also prove that the bilinear form $a_{\psi}(\cdot,\cdot)$ satisfy
%  the coercivity \eqref{coer:1} and the continuity \eqref{conti:1} in the finite element space
 % due to the fact the  Gauss-Lobatto numerical quadrature is exact for polynomials of degree not more than $2k$.
%
%
%
% Consequently, for both DDG and FE methods, the
%  coercivity of the bilinear form $a_{\psi}(\cdot,\cdot)$ holds true for all positive function $\psi$
%  with a uniformly lower bound.

Thirdly, we define
\[
   \tilde V_h= \{v\in C^{m}: v|_{K}\in{\mathbb Q}_k: \int_{\Omega} v dx=0\}
\]
where $m=-1$ for the DDG method and $m=0$ for the FE method.
 Let ${\cal L}_{\psi}: L^2\rightarrow \tilde V_h$ be a operator that satisfies
 \begin{equation}\label{operator:1}
     a_{\psi}({\cal L}_{\psi} (f),v)=(f,v),\ \ \forall v\in \tilde V_h.
 \end{equation}
% define
%\[
%   A_{\psi}(u,v)= a_{\psi}(u,v)+(u,v).
%\]
%  Noticing that $A_{\psi}(\cdot,\cdot)$ is coercive and continuous in $V_h$, we have,
%  from the
  %From the Lax-Milgram Lemma,
Note that when $\psi=1$, we set ${\cal L}=\cal L_{\psi}$ for the sake of simplicity. Obviously, given any function $f\in L^2$, there exists a unique $u\in \tilde V_h$ such that
 \[
   a_{\psi}(u,v)=(f,v),\ \ \forall v\in \tilde V_h.
 \]
 In addition, by utilizing the Poincare inequality along with the  coercivity and continuity properties of $a_{\psi}(\cdot,\cdot)$ in  \eqref{coer:2} and \eqref{conti:1}, we obtain
\begin{equation}
   \|{\cal L}_{\psi}(f)\|^2_{0}\lesssim |{\cal L}_{\psi}(f)|^2_{1}\le \frac{ \gamma_1}{\gamma_0}
    \|f\|_0\|{\cal L}_{\psi}(f)\|_0,
\end{equation}
which, combined with the inverse inequality, leads to
\begin{equation}\label{eq:30}
      \|{\cal L}_{\psi}(f)\|_{0,\infty}\le C h^{-1} \|{\cal L}_{\psi}(f)\|_{0}\le  \frac{C\gamma_1}{\gamma_0h}
      \|f\|_{0}.
\end{equation}
Furthermore, we define $\|f\|_{\cal L_{\psi}}$ as follows:
  \[
     \|f\|_{\cal L_{\psi}}=\sqrt{ (f,{\cal L}_{\psi}(f))}.
  \]
  \begin{lemma}
      For any functions $f,v\in L^2$,  let $L_{\psi}$ be the operator defined in \eqref{operator:1}.
      Then  for  both  the FE method and the DDG method with the coefficient   $\beta_1=0$,
 \begin{equation}\label{derivative}
  \frac 12 \frac{\text{d}}{\text{d} s}\|f+sv\|^2_{\cal L_{\psi}}=({\cal L}_{\psi}(f),v)+s(v,{\cal L}_{\psi}(v)).
 \end{equation}
 \end{lemma}
 \begin{proof}
   Recalling the definition of ${\cal L}_{\psi}$,  we have, from a direct calculation,
 \[
      \|f+sv\|^2_{\cal L_{\psi}}=(f,{\cal L}_{\psi}(f))+s(f,{\cal L}_{\psi}(v)+s({\cal L}_{\psi}(f),v)+s^2(v,{\cal L}_{\psi}(v)),
 \]
   and thus
\begin{eqnarray*}
   \frac{\text{d}}{\text{d} s}\|f+sv\|^2_{\cal L_{\psi}}&=&(f,{\cal L}_{\psi}(v))+({\cal L}_{\psi}(f),v)+2s(v,{\cal L}_{\psi}(v))\\
    &=&a_{\psi}({\cal L}_{\psi}(f),{\cal L}_{\psi}(v))+({\cal L}_{\psi}(f),v)+2s(v,{\cal L}_{\psi}(v)).
\end{eqnarray*}
  When $\beta_1=0$,
  the bilinear form $a_{\psi}(\cdot,\cdot)$ is symmetrical for both the DDG and the FE method,  and thus
 \[
    a_{\psi}({\cal L}_{\psi}(f),{\cal L}_{\psi}(v))=a_{\psi}({\cal L}_{\psi}(v),{\cal L}_{\psi}(f))=(v,{\cal L}_{\psi}(f)).
 \]
  Then the desired result follows from the last two  equations.
 \end{proof}


 %-------------------------------------------------------------------------------------

%-------------------------------------------------------------------------------------
For any function $v\in {\mathcal H}_h$,
we define the cell average of $v$ in each element $K$ as follows:
   \begin{equation}
   \bar v=\frac{1}{|K|}\int_{K}v dx.
 \end{equation}
Now, we are  prepared to present the positivity-preserving property of the numerical scheme \eqref{ci1-femdg}-\eqref{ci3-femdg}.

\begin{theorem}\label{theo:1}
  Assume that $c_{ih}^{m+1}$ is the solution of \eqref{ci1-femdg}-\eqref{ci3-femdg} with the
  coefficient $\beta_1=0$ for the DDG discretization.   Then for both the FE and DDG methods,
   the numerical scheme \eqref{ci1-femdg}-\eqref{ci3-femdg} is positive preserving in the sense that $\bar c_{ih}^{m+1}>0, c^{m+1}_{ih}(g_{K,j})>0$ if
  $\bar c_{ih}^{m}>0$ and $c^{m}_{ih}(g_{K,j})>0$ with $1\le j\le n_{k}$ for all $K\in{\cal T}_h$.
\end{theorem}
\begin{proof}
   First, denote $a_0=\frac{1}{|\Omega|}\int_{\Omega} c^m_{ih} dx$, $\mu^m_i=c_{ih}^m-a_0\in \tilde V_h$,
    and $\mu=(\mu_1,\cdots,\mu_n)$.   We
 consider an energy functional
  \begin{eqnarray*}
     J(\mu)&=&\sum_{i=1}^n \big(\frac{1}{2\tau} (\mu_{i}-\mu_{i}^m, {\cal L}_{c^m_{ih}}(\mu_{i}-\mu_{i}^m))+
     \langle (\mu_{i}+a_0){\rm log}(\mu_{i}+a_0),1\rangle \big)\\
     &+&\frac{1}{2}(\sum_{i=1}^nq_i(\mu_{i}+a_0),  {\cal L}(\sum_{i=1}^nq_i(\mu_{i}+a_0)) )
  \end{eqnarray*}
     over a admissible set
   \begin{equation}
      A_h=\{(\mu_1,\cdots,\mu_n): 0< (\mu_i+a_0)(g_{K,j})\le M,  1\le i\le  n, K\in{\cal T}_h\}
    \end{equation}
     with $M=\frac{a_0|\Omega|}{h^2}$.
    Note  that $J(\mu)$ is a strictly convex function. Next, we will prove that there exists a minimizer of  $J(\mu)$ over the domain $ A_h$.
    For any $\delta>0$, we consider
    \begin{equation}
      A_{h,\delta}=\{(\mu_1,\cdots,\mu_n): \delta\le (\mu_i+a_0)(g_{K,j})\le M-\delta,  1\le i\le  n,K\in{\cal T}_h\}.
   \end{equation}
    Since $A_{h,\delta}$ is a compact set in the hyperplane
    $H=\{(\mu_1,\cdots,\mu_n):\int_{\Omega} \mu_i=0, 1\le i\le n\}$
    and  $J(\mu)$ is a strictly convex function,  there exists
    a minimizer of $J(\mu)$ over $A_{h,\delta}$.
    The key aspect of the positivity analysis is that such a
minimizer could not occur at one of the boundary points if $\delta$ is sufficiently small.

  We denote by $\mu^*$ the minimizer of  $J(\mu)$ and assume that the
   minimizer occurs at the boundary of $A_{h,\delta}$. Without loss of generality, we assume
   $\mu^*(g_{K_0,j_0})+ a_0=\delta$.
    Suppose that $\mu^*$ attains its maximum value at the point $g_{K_1,j_1}$.  Utilizing the fact that
    $\int_{\Omega}\mu^*=0$ leads to the conclusion that $\mu^*(g_{K_1,j_1})\ge 0$.

   Now we consider the directional derivative: for any $w\in V_h$, we use \eqref{derivative} to derive that
  \begin{eqnarray*}
 \frac{\text{d}}{\text{d} s}J(\mu^*_1,\mu^*_2,\ldots,\mu^*_i+sw,\ldots,\mu^*_n)|_{s=0}&=&\frac{1}{\tau} ( {\cal L}_{c^m_{ih}}(\mu^*_{i}-\mu_{i}^m),w)+
     \langle {\rm log}(\mu^*_{i}+a_0)+1, w\rangle\\
     &+&({\cal L}(\sum_{i=1}^nq_i(\mu_{i}+a_0)),w).
  \end{eqnarray*}
    Let $l_{K,j}\in V_h$ be the Lagrange  basis function corresponding to the points $g_{K,j}$, that is,
    $l_{K,j}(g_{K,i})=\delta_{i,j}$, where $\delta_{i,j}$ the Kronecker delta function. Now we choose $w=l_{K,j}$ in the last
    equation  and then get
   \begin{eqnarray*}
    \frac{\text{d}}{\text{d} s}J(\mu^*_1,\mu^*_2,\ldots,\mu^*_i+sw,\ldots,\mu^*_n)|_{s=0}&=&\frac{1}{\tau}  {\cal L}_{c^m_{ih}}(\mu^*_{i}-\mu_{i}^m)(g_{K,j})+
 ({\rm log}(\mu^*_{i}+a_0)+1 )(g_{K,j})\\
     &+&{\cal L}(\sum_{i=1}^nq_i(\mu_{i}+a_0)) (g_{K,j}),\ \ \forall K\in{\cal T}_h, j\in \mathbb Z_{k+1}.
  \end{eqnarray*}
   Especially, we separately substitute $K=K_0, j=j_0$ and $K=K_1,j=j_1$ into the above equation and then get
  \begin{eqnarray*}
        \frac{\text{d}}{\text{d} s}J(\mu^*_1,\mu^*_2,\ldots,\mu^*_i+sw,\ldots,\mu^*_n)|_{s=0}&=&\frac{ {\rm log}(\mu^*_{K_0,j_0}+a_0)}{{ {\rm log}(\mu^*_{K_1,j_1}+a_0)}}+
    \frac{1}{\tau} {\cal L}_{c^m_{ih}}(\mu^*_{i}-\mu_{i}^m)|_{g_{K_1,j_1}}^{g_{K_0,j_0}}\\
     &+&{\cal L}(\sum_{i=1}^nq_i(\mu_{i}+a_0))|_{g_{K_1,j_1}}^{g_{K_0,j_0}}.
 \end{eqnarray*}
   Here the notation $F(x)|_{a}^b=F(b)-F(a)$.
 For the first term, we easily obtain that
\[
  \frac{ {\rm log}(\mu^*_{K_0,j_0}+a_0)}{{ {\rm log}(\mu^*_{K_1,j_1}+a_0)}}\le {\rm log} \frac{\delta}{a_0}.
\]
  Denote
 $\alpha_0=\min\{c_{ih}^m(g_{K,j})\}$,
 In accordance with \eqref{eq:30}, we obtain
\begin{eqnarray*}
  {\cal L}_{c^m_{ih}}(\mu^*_{i}-\mu_{i}^m)|_{g_{\tau_1,j_1}}^{g_{\tau_0,j_0}}\le 2 C_0\alpha_0^{-1} h^{-1}M,\ \
  {\cal L}(\sum_{i=1}^nq_i(\mu^*_{i}+a_0))|_{g_{\tau_1,j_1}}^{g_{\tau_0,j_0}}\le 2nC_0h^{-1}M.
\end{eqnarray*}
 Here $ C_0=\frac{C\gamma_1}{\gamma_0}$ with $C$ the same as that in $\eqref{eq:30}$.
 Define
\[
   D_0= 2C_0h^{-1}M (\alpha_0\tau^{-1}+n).
\]
 Note that $D_0$ is a constant when $h$ and $\tau$ are fixed.  For any fixed $h$ and $\tau$, we can choose
 a particular $\delta>0$ small enough so that
\[
   {\rm log} \frac{\delta}{a_0}+D_0<0,
\]
 and thus
 \[
     \frac{\text{d}}{\text{d} s}J(\mu^*_1,\mu^*_2,\ldots,\mu^*_i+sw,\ldots,\mu^*_n)|_{s=0}<0,\ \ 1\le i\le n.
 \]
   This contradicts the assumption that $J(\mu)$ has a minimum at $\mu^*$.
   Therefore, the global minimum of  $J(\mu)$ over $A_{h,\delta}$ could only potentially occur at an interior point, for $\delta>0$ sufficiently small.  Then we can conclude that there exists a solution $\mu\in A_{h,\delta}$ such that
\[
   \frac{\text{d}}{\text{d} s}J(\mu+sv)|_{s=0}=0,\ \ \forall v\in V_h,
\]
  which is equivalent to the numerical schemes \eqref{ci1-femdg}-\eqref{ci3-femdg} .  Therefore there exists a numerical solution
  to \eqref{ci1-femdg}-\eqref{ci3-femdg}  over the compact domain $A_{h,\delta}$ with point-wise positive values at
  Gauss-Lobatto points
  $g_{K,i}$.  Since
 \[
    \bar c_{ih}^{m+1}=\frac{1}{|K|}\sum_{j=1}^{n_k} w_{K,j}c_{ih}(g_{K,j})>0.
 \]
    Then  the positivity preserving of  numerical solution both at the Gauss-Lobatto points and for the cell average is thus established.
\end{proof}

 \begin{remark}
   In order to guarantee the positivity of the initial solution at Gauss points,
   we adopt the following initial discritization:
 \[
     c_{ih}^0(g_{K,j})=c_{i0}(g_{K,j}).
 \]
  That is, $c_{ih}^0\in V_h$ is chosen as the Gauss-Lobatto  interpolation of the
  initial function $c_{i}^0(x)$.
 \end{remark}

\subsection{limiter}
  As stated in Theorem \ref{theo:1},  $c_{ih}|_{K}\in \mathbb Q^k$ is a high order approximation to the smooth function $c_i(x)>0$,  possessing a positive cell average and positive function values at the Gauss-Lobatto points.
  Utilizing the positive cell average and  $c_{ih}(x)$, we are able to design another polynomial  $\tilde c_{ih}(x)$,
  which is also a  high order approximation
  to the smooth function $c_i(x)>0$ and is positive for all $x\in \Omega$.  The process for constructing $\tilde c_{ih}(x)$
  is detailed below.
 \begin{equation}
     \tilde c_{ih}(x)|_{K}=\bar c_{ih}+\theta(c_{ih}(x)-\bar c_{ih}),\ \ \theta=\min\{1,\frac{\bar c_{ih}}{\bar c_{ih}-\min_{x\in K} g_h(x)}\}.
 \end{equation}
  There holds the following property of  $\tilde c_{ih}(x)$ (see, e.g., \cite{Liu-YuSISC}):
 \begin{equation}\label{limiter:0}
    | \tilde c_{ih}(x)-c_{ih}(x)|\le C_k \|c_{ih}-c_i\|_{\infty},\ \ \min_{x\in K} \tilde c_{ih}(x)\ge 0,\ \ \frac{1}{|K|}\int_{\tau} \tilde c_{ih}(x) dx=\bar c_{ih}.
 \end{equation}
  The above equation indicate that the constructed $\tilde c_{ih}$  does not destroy the solution accuracy. Meanwhile, it maintains the same cell averages and is positive at each point within each element.


 \begin{remark}
   We have adjusted the energy functional in the following manner:
  \begin{eqnarray*}
     J(\mu)&=&\frac{3}{\triangle t} \sum_{i=1}^n (\frac 32\mu_{i}-2\mu_{i}^m-\mu_{i}^{m-1}, {\cal L}_{c^m_{ih}}(\frac 32\mu_{i}-2\mu_{i}^m+\mu_i^{m-1}))+
     \langle (\mu_{i}+a_0){\rm log}(\mu_{i}+a_0),1\rangle\\
     &+&\frac{1}{2}(q_i(\mu_{i}+a_0),  {\cal L}(q_i(\mu_{i}+a_0)).
  \end{eqnarray*}
By employing the same arguments utilized for the first order schemes,  we achieve the acquisition of the positivity property for the second-order schemes.
   \end{remark}


   \section{Energy stability}
     We first define the following energy function:
  \begin{eqnarray*}
     E(c):=E(c_1,\ldots,c_n)&=&\sum_{i=1}^n \int_{\Omega} \left( c_i{\rm log} c_i+\frac{1}{2} q_i c_{i}\phi \right) dx\\
   &=&\sum_{i=1}^n \int_{\Omega} \left( c_i{\rm log} c_i+\frac{1}{2}| \nabla\phi|^2 \right) dx.
  \end{eqnarray*}
 Observing that
 \[
     d_tE=-\sum_{i=1}^n \int_{\Omega} c_i |\nabla p_i|^2 dx\le 0,
  \]
we conclude that the energy is dissipated over time.
Subsequently, we will demonstrate that our fully-discrete numerical scheme maintains energy stability.
  \begin{theorem}
     The numerical scheme \eqref{ci1-femdg}-\eqref{ci3-femdg} is energy stable, provided that the mesh size $h$ is sufficiently small. That is, there holds
  \begin{equation}\label{eq:6}
     E(c_{h}^{m+1})\le E(c_{h}^{m}).
  \end{equation}
 \end{theorem}
 \begin{proof}
   First,  by choosing $v=p_{ih}^{m+1}$  in \eqref{ci1-femdgm} yields
  \begin{equation}\label{eq:40}
      (c_{ih}^{m+1}-c_{ih}^{m},p_{ih}^{m+1})=-\tau a_{c_{ih}^m}(p_{ih}^{m+1}, p_{ih}^{m+1}).
 \end{equation}
  Second, we utilize \eqref{ci2-femdgm} to derive
\[
  (c_{ih}^{m+1}-c_{ih}^{m}, p_{ih}^{m+1})=(q_i \phi^{m+1}_h, c_{ih}^{m+1}-c_{ih}^m)+\langle {\rm log} c^{m+1}_{ih}+1, c_{ih}^{m+1}-c_{ih}^m\rangle.
\]
  By applying the  Taylor expansion, we  easily deduce that
\[
   \langle {\rm log} c^{m+1}_{ih}, c_{ih}^{m+1}\rangle-\langle {\rm log} c^{m}_{ih}, c_{ih}^{m}\rangle \leq \langle {\rm log} c^{m+1}_{ih}+1, c_{ih}^{m+1}-c_{ih}^m\rangle.
\]
  On the other hand,  \eqref{ci3-femdgm} indicates that
 \begin{eqnarray*}
   \sum_{i=1}^n(q_i \phi^{m+1}_h, c_{ih}^{m+1}-c_{ih}^m)&=&(\phi_{h}^{m+1}, {\cal L}^{-1}(\phi_{h}^{m+1})-{\cal L}^{-1}(\phi_{h}^{m}))\\
  &\ge &\frac{1}{2}(\phi_{h}^{m+1}, {\cal L}^{-1}(\phi_{h}^{m+1}))-\frac 12(\phi_{h}^{m}, {\cal L}^{-1}(\phi_{h}^{m})).
 \end{eqnarray*}
Here in the second step, we have used the convexity of the norm $(v,{\cal L}^{-1}v)$ for any function $v$.
Substituting the last three equations into \eqref{eq:40}, we immediately get
 \begin{eqnarray}\label{eq:5}
   E(c_{h}^{m+1})- E(c_{h}^{m})+\tau a_{ c_{ih}^m}(p_{ih}^{m+1}, p_{ih}^{m+1})=0.
 \end{eqnarray}
  Then the desired result \eqref{eq:6} follows due to the positivity $c_{ih}^m(g_{K,j})>0$ and the
  non-negativity of $a_{c_{ih}^m}(v,v)$.
%
%\[
%   \triangle t a_{\bar c_{ih}^m}(p_{ih}^{m+1}, p_{ih}^{m+1})-\triangle t a_{\bar c_{ih}^m-c_{ih}^m}(p_{ih}^{m+1}, p_{ih}^{m+1})\ge 0.
%\]
% Substituting the above inequality into \eqref{eq:5} gives the desired result \eqref{eq:6}.
  \end{proof}





\section{Optimal error estimates and superconvergence analysis}

  % We begin with  some preliminaries.
   %Given any function $v$,
%   we define $P_hv\in V_h$ the special projection of $v$
%   satisfying
%  \begin{equation}
%    a(P_hv,w)=a(v,w),\ \ \forall w\in V_h.
%  \end{equation}
%   Thanks to the  coercivity of the bilinear form $a(\cdot,\cdot)$, the projection $P_hv$ is uniquely defined. Moreover,
%   there holds the following approximation property
%  \[
%     \|v-P_hv\|_0\lesssim h^{k+1}\|v\|_{k+1}.
%  \]

\subsection{Preliminaries and error equations}

First, we define the Gauss-Lobatto interpolation $I_h: {\cal H}_h\rightarrow V_h$ which satisfies the condition:
   \[
      I_h v(g_{K,j})=v(g_{K,j}),\ \ \forall v\in  {\cal H}_h, 1\le j\le n_{k},\ K\in{\cal T}_h.
   \]
According to \cite{Susanne,chen-hu}, the following approximation properties hold:
\begin{equation}\label{interpolation-appro}
   \|v-I_hv\|_{m,p}\lesssim h^{n-m+\frac{2}{p}-\frac{2}{q}}|v|_{n,q},\
   | (\nabla(v-I_hv), \nabla w)_{\tau}|\lesssim h^{k+1} |v|_{k+2,\tau}\|\nabla w\|_{0,\tau}.
\end{equation}

Second, let us introduce
 \[
    \tilde a_{\psi}(\cdot,\cdot)=\sum_{K\in{\mathcal T}_h} \tilde a_{\psi,K}(\cdot,\cdot),
 \]
  where for DDG method,
 \[
     \tilde a_{\psi,K}(u,v)=(\psi \nabla u, \nabla v )_{K}- (
 \{\psi \},  \widehat{\partial_n u} v+(u-\{u\})\partial_n v )_{\partial K}.
    %\ \ a(u,v)=\sum_{\tau\in {\mathcal T}_h}a_{\tau}(u,v).
 \]
  and for FE method
 \[
   \tilde a_{\psi ,K}(u,v)=(\psi  \nabla u, \nabla v )_{K}.
 \]
  In light of the second inequality of \eqref{interpolation-appro}, for the FE method, we have
 \[
    |\tilde a_{\psi}(u-I_hu,v)|=|\tilde a_{\psi-\bar\psi}(u-I_hu,v)+\tilde a_{\bar\psi}(u-I_hu,v)|\lesssim h^{k+1}|u|_{k+2}\|\nabla v\|_0.
 \]
   As for DDG method, if $\beta_1=\frac{1}{2k(k+1)}$, then (see, e.g., \cite{super-cao-cd})
\[
    |\tilde a(u-I_hu,v)|\lesssim h^{k+1}|u|_{k+2}\|v\|_E,
\]
  and thus
\begin{equation}\label{esti:aa}
    |\tilde a_{\psi}(u-I_hu,v)|\lesssim h^{k+1}|u|_{k+2}\|v\|_E.
\end{equation}
In other words, the inequality \eqref{esti:aa} holds true for both the DDG and FE methods.

 Third,
 leveraging the error associated with the Gauss-Lobatto numerical quadrature, we easily derive the following relationships:
 \[
    (f,v)_K-\langle f,v\rangle_K= (f-I_h f,v)_K,\ \
    (f,v)_{\partial K}-\langle f,v\rangle_{\partial K}= (f-I_h f,v)_{\partial K},
 \]
 which leads to
 \begin{equation}\label{esti:a0}
   |a_{\psi}(u,v)-\tilde a_{\psi}(u,v)|\lesssim h^{k+1}|u|_{k+2}\|v\|_E.
\end{equation}
Given \eqref{esti:aa} and \eqref{esti:a0}, we conclude that
\begin{equation}\label{esti:a1}
    | a_{\psi}(u-I_hu,v)|\lesssim h^{k+1}|u|_{k+2}\|v\|_E.
\end{equation}

  To establish error estimates at time $t_{m+1}$, we initially assume that the exact solutions $c_i, p_i,\phi$ are sufficiently smooth. Since $c_i$ is positive and smooth, we suppose there exist  $\delta_0, M_0>0$ such that
\begin{equation}\label{assuption:u}
     \delta_0\le \|c_i\|_{0,\infty}\le M_0,\ \ \|c_i\|_{r,\infty}\le M_0.
\end{equation}
Utilizing the approximation properties of $I_h$ stated in \eqref{interpolation-appro}, for sufficiently small $h$, we deduce that,
 \begin{equation}\label{boundui}
     \|I_hc_i\|_{0,\infty}\ge \frac{\delta_0}{2},\ \  \|I_hc_i\|_{1,\infty}\le M_0.
 \end{equation}
 Furthermore,
we  need a priori assumption
 at the previous time step $t_m$, i.e.,
\begin{equation}\label{assumption:0}
 %  \|c_{ih}^{m+1}\|_{1,\infty}\le C_0, \ \
  % \|c_{ih}^m-I_hc_i^m\|_E+\tau^{\frac 12} h^{k+1}+\tau^2 \le \epsilon h^{1+\frac d2}.
  \|c_{ih}^m-I_hc_i^m\|_E\lesssim \tau+h^{k+1}.
\end{equation}
 The above assumption will be justified later.
Due to \eqref{assumption:0}, we have the following boundedness for $c_{ih}^m$ in the $L^{\infty}$ norm.
\[
   \|c_{ih}^m-I_hc_i^m\|_{1,\infty}\lesssim h^{-1}\|c_{ih}^m-I_hc_i^m\|_E\lesssim h^{-1}(\tau+h^{k+1}).
\]
Assuming that $\tau={\mathcal O}(h)$, then we have
 \begin{equation}\label{bounds:0}
   \|c_{ih}^m\|_{1,\infty}\lesssim 1.
\end{equation}
 Since
\[
   \|c_{ih}^m-I_hc_i^m\|_{0,\infty}\lesssim |{\rm ln} h|^{\frac 12}\|c_{ih}^m-{I_h}c_i^m\|_{1}\lesssim
   |{\rm ln} h|^{\frac 12}(\tau+h^{k+1}),
\]
   then for  sufficiently small $\tau, h$,
\begin{equation}\label{bounds:1}
   \|c_{ih}^m\|_{0,\infty} \ge \frac{\delta_0}{4}.
\end{equation}








  In the rest of this paper, we will adopt the following notation
 \[
    e_{v}=v-v_h,\ \ \eta_{v}=v-I_hv,\ \ \xi_{v}=I_hv-v_h,\  \ v=c_i,p_i,\phi.
 \]
  %  Now we are ready to present the error estimates for the numerical solution.
For all $v, w\in V_h$, note that the exact solutions satisfy
  \begin{eqnarray*}
    &&(\frac{c^{m+1}_{i}-c^{m}_{i}}{\tau}, v)=
    -\tilde a_{c_{i}^{m}}(p^{m+1}_i,v)-\tilde a_{(c_{i}^{m+1}-c_i^{m})}(p^{m+1}_i,v)+( \frac{c^{m+1}_{i}-c^{m}_{i}}{\tau}-\partial_tc^{m+1}_i, v ), \\
    && p^{m+1}_{i}(g_{K,j})= (q_i \phi^{m+1}_h +{\rm log} c^{m+1}_{i}+1)(g_{K,j}), \\
   && \tilde a( \phi^{m+1}, w) =\sum_{i=1}^n
    ( q_i c^{m+1}_{i}, w).
 \end{eqnarray*}
 Then we derive the following error equations from \eqref{ci2-femdg}, \eqref{ci1-femdgm} and \eqref{ci3-femdgm}
  \begin{eqnarray}\label{error10}
  &&  (\frac{\xi_{c_i}^{m+1}-\xi_{c_i}^{m}}{\tau}, v)=a_{c_{ih}^{m}}(p^{m+1}_{ih},v)-
     a_{c_{i}^{m}}(p^{m+1}_i,v)+I(v),\ \forall v\in V_h,\\ \label{error20}
  &&   e_{p_i}^{m+1}(g_{K,j})= (q_i e_{\phi}^{m+1} +{\rm log} c^{m+1}_{i}-{\rm log} c^{m+1}_{ih})(g_{K,j}),\\\label{error30}
  &&  \tilde a( \phi^{m+1}, w) - a( \phi^{m+1}_h, w)  =\sum_{i=1}^n
    ( q_i e_{c_i}^{m+1}, w ),\ \ \forall w\in V_h,
 \end{eqnarray}
  where
\begin{eqnarray}\label{Iv}
\begin{split}
   I(v)&=&( \frac{c^{m+1}_{i}-c^{m}_{i}}{\tau}-\partial_tc^{m+1}_i, v )-\tilde a_{(c_{i}^{m+1}-c_i^{m})}(p^{m+1}_i,v) \\
   &+&a_{c^m_i}(p_i^{m+1},v)-\tilde a_{c^m_i}(p_i^{m+1},v)
    -(\frac{\eta_{c_i}^{m+1}-\eta_{c_i}^{m}}{\tau}, v).
\end{split}
\end{eqnarray}
  Using the Cauchy-Schwarz inequality, we have
 \begin{eqnarray*}
  &&  |( \frac{c^{m+1}_{i}-c^{m}_{i}}{\tau}-\partial_tc^{m+1}_i, v )
    -(\frac{\eta_{c_i}^{m+1}-\eta_{c_i}^{m}}{\tau}, v)|\lesssim ( \tau +h^{k+1}) \|v\|_0,\\
  && |\tilde a_{(c_{i}^{m+1}-c_i^{m})}(p^{m+1}_i,v)  |\lesssim \tau \|v\|_E.
 \end{eqnarray*}
Combining these inequalities with equation \eqref{esti:a0}, we obtain that
  \begin{eqnarray}\label{error2}
   |I(v)|\lesssim ( \tau +h^{k+1}) \|v\|_E,\ \ \forall v\in V_h.
\end{eqnarray}
%
%Noticing that
% \[
%    c_0\le c_i, c_{ih} \le M,
% \]


\subsection{Error estimates}
  We begin with the estimate of the numerical solution $u_h^{m+1}$.
\begin{lemma}
Assume that the condition \eqref{assuption:u} and the a-priori assumption \eqref{assumption:0} are valid.
Suppose that the following inequality holds:
\begin{equation}\label{cfl}
   \frac{\tau}{h^{2}} (\frac{h^{k+1}}{\tau^{\frac 12}}+\frac{\sum_{i=1}^n\|\xi_{c_i}^m\|_0}{\tau}+\tau)\le \epsilon,
\end{equation}
where $\epsilon$ is a positive constant that is independent of the mesh size $h$ and time step size $\tau$. Then it follows
\begin{eqnarray}\label{ci}
   \frac{\delta_0}{4} \le \|c_{ih}^{m+1}\|_{0,\infty} \le M_0+\frac{\delta_0}{4}.
\end{eqnarray}
\end{lemma}
\begin{proof}
First, choosing $v=\xi_{p_i}^{m+1} $ in \eqref{error10} yields
%\begin{eqnarray}
%\begin{split}
%    (\frac{\xi_{c_i}^{m+1}-\xi_{c_i}^{m}}{\triangle t}, \xi_{p_i}^{m+1})&=a_{c_{ih}^{m}}(p^{m+1}_{ih},\xi_{p_i}^{m+1})-
%    a_{c_{i}^{m}}(p^{m+1}_i,\xi_{p_i}^{m+1})+I(\xi_{p_i}^{m+1}) &\\
%    &=a_{c_{ih}^{m}}(p^{m+1}_{ih}-p_i^{m+1},\xi_{p_i}^{m+1})-
%    a_{c_{i}^{m}-c_{ih}^{m}}(p^{m+1}_i,\xi_{p_i}^{m+1})+I(\xi_{p_i}^{m+1}) &
%%   & =-
%%  a_{c_{ih}^{m}}(\xi_{p_i}^{m+1},\xi_{p_i}^{m+1})-a_{c_{ih}^{m}}(\eta_{p_i}^{m+1},\xi_{p_i}^{m+1})-
%%  a_{c_{i}^{m}-c_{ih}^{m}}(p^{m+1}_i,\xi_{p_i}^{m+1})+I(\xi_{p_i}^{m+1}),&
%   \end{split}
%\end{eqnarray}
\begin{equation}
(\frac{\xi_{c_i}^{m+1}-\xi_{c_i}^{m}}{\tau}, \xi_{p_i}^{m+1})=
-a_{c_{ih}^{m}}(e_{p_i}^{m+1},\xi_{p_i}^{m+1})-
    a_{e_{c_i^m}}(p^{m+1}_i,\xi_{p_i}^{m+1})
    +I(\xi_{p_i}^{m+1}),
\end{equation}
where $I(v)$ is defined in \eqref{Iv}. Next, we rearrange terms to get:
\begin{eqnarray}\label{eq:01}
\begin{split}
  &  (\xi_{c_i}^{m+1}, \xi_{p_i}^{m+1})+\tau a_{c_{ih}^{m}}(\xi_{p_i}^{m+1},\xi_{p_i}^{m+1}) & \\
   &=
   (\xi_{c_i}^{m}, \xi_{p_i}^{m+1})
  - \tau a_{c_{ih}^{m}}(\eta_{p_i}^{m+1},\xi_{p_i}^{m+1})-
   \tau  a_{e_{c_i^m}}(p^{m+1}_i,\xi_{p_i}^{m+1})+\tau I(\xi_{p_i}^{m+1}).&
 \end{split}
\end{eqnarray}
Utilizing equation \eqref{error20} and the fact that $k+1$ Gauss-Lobatto quadrature is exact for polynomials of degree not more than $2k$, we derive
%\begin{eqnarray*}
%  a_{c_{ih}^{m}}(p^{m+1}_{ih}-p_i^{m+1},\xi_{p_i}^{m+1})=-
%  a_{c_{ih}^{m}}(\xi_{p_i}^{m+1},\xi_{p_i}^{m+1})-a_{c_{ih}^{m}}(\eta_{p_i}^{m+1},\xi_{p_i}^{m+1})
%\end{eqnarray*}
\begin{eqnarray*}
   (\xi_{p_i}^{m+1},\xi_{c_i}^{m+1})
%   &=&((P_h-I_h)p_i^{m+1},\xi_{c_i}^{m+1})+
%   (I_hp_i^{m+1}-p_{ih}^{m+1},\xi_{c_i}^{m+1})\\
 =  \langle \xi_{p_i}^{m+1},\xi_{c_i}^{m+1}\rangle
 &= &\langle q_ie^{m+1}_{\phi}+{\rm log}c^{m+1}_i-{\rm log}c^{m+1}_{ih},\xi_{c_i}^{m+1}\rangle\\
 &=&  (q_i\xi_{\phi}^{m+1},\xi_{c_i}^{m+1})
  +\langle {\rm log}c^{m+1}_i-{\rm log}c^{m+1}_{ih},e_{c_i}^{m+1}\rangle.
\end{eqnarray*}
%\begin{eqnarray*}
%\begin{split}
%  (\xi_{p_i}^{m+1},\xi_{c_i}^{m+1})&= (P_hp_i^{m+1},\xi_{c_i}^{m+1})-( q_i \phi^{m+1}_h+1, \xi_{c_i}^{m+1}) -\langle {\rm log} c^{m+1}_{ih}, \xi_{c_i}^{m+1}\rangle & \\
%   &=-(\eta_{p_i}^{m+1},\xi_{c_i}^{m+1})+( q_i e_\phi^{m+1}, \xi_{c_i}^{m+1}) +
%  ({\rm log} c^{m+1}_{i}, \xi_{c_i}^{m+1})
%   -\langle {\rm log} c^{m+1}_{ih}, \xi_{c_i}^{m+1}\rangle &
%   \end{split}
%\end{eqnarray*}
By using \eqref{error30} and the fact that
$a(\phi^{m+1}_h,\xi_\phi^{m+1})=\tilde a(\phi^{m+1}_h,\xi_\phi^{m+1}) $,
we can further deduce:
\begin{eqnarray*}
 \sum_{i=1}^n(q_i\xi_{\phi}^{m+1},\xi_{c_i}^{m+1})&=&
 \sum_{i=1}^n(  q_i\xi_\phi^{m+1}, e_{c_i}^{m+1}-\eta_{c_i}^{m+1})=
 \tilde a( e_{\phi}^{m+1}, \xi_\phi^{m+1})-\sum_{i=1}^n( q_i \eta_{c_i}^{m+1}, \xi_{\phi}^{m+1}).
% \sum_{i=1}^n( q_i \eta_\phi^{m+1}, \xi_{c_i}^{m+1})+
% \sum_{i=1}^n( q_i \xi_\phi^{m+1}, e_{c_i}^{m+1})- \sum_{i=1}^n( q_i \xi_\phi^{m+1}, \eta_{c_i}^{m+1})\\
% &=& - \sum_{i=1}^n( q_i \xi_\phi^{m+1}, \eta_{c_i}^{m+1})\\
% &= & a( \xi_{\phi}^{m+1}, \xi_\phi^{m+1})+a( (P_h-I_h){\phi}^{m+1}, \xi_\phi^{m+1})-\sum_{i=1}^n( q_i \eta_{c_i}^{m+1}, \xi_{\phi}^{m+1}).
\end{eqnarray*}
  Combining the last two equations and utilizing the inequality
  \[
     \langle {\rm log} c^{m+1}_{i}-{\rm log} c^{m+1}_{ih}, e_{c_i}^{m+1}\rangle\ge 0,
  \]
we can derive that
\begin{eqnarray*}
   \sum_{i=1}^n(\xi_{p_i}^{m+1},\xi_{c_i}^{m+1})
  &\ge &
  \tilde a( \xi_{\phi}^{m+1}, \xi_\phi^{m+1})+\tilde a( \eta_{\phi}^{m+1}, \xi_\phi^{m+1})-\sum_{i=1}^n( q_i \eta_{c_i}^{m+1}, \xi_{\phi}^{m+1}).
\end{eqnarray*}
  % Then the desired result \eqref{ci} follows from the a priori assumption.
%  Here in the last step, we have used the orthogonality
%  $a(  \xi_\phi^{m+1}, \eta_{\phi}^{m+1})=0$.
% As for the term $({\rm log} c^{m+1}_{i}, \xi_{c_i}^{m+1})
%   +\langle {\rm log} c^{m+1}_{ih}, \xi_{c_i}^{m+1}\rangle$, a direct calculation yields
%\begin{eqnarray*}
%  && ({\rm log} c^{m+1}_{i}, \xi_{c_i}^{m+1})
%   -\langle {\rm log} c^{m+1}_{ih}, \xi_{c_i}^{m+1}\rangle\\
%   &=&
%   ({\rm log} c^{m+1}_{i}-I_h{\rm log} c^{m+1}_{i}, \xi_{c_i}^{m+1})
%   +\langle {\rm log} c^{m+1}_{i}-{\rm log} c^{m+1}_{ih}, \xi_{c_i}^{m+1}\rangle\\
%   &=&
%   ({\rm log} c^{m+1}_{i}-I_h{\rm log} c^{m+1}_{i}, \xi_{c_i}^{m+1})
%   +\langle {\rm log} (I_hc^{m+1}_{i})-{\rm log} c^{m+1}_{ih}, \xi_{c_i}^{m+1}\rangle
%    &\ge &
%   ({\rm log} c^{m+1}_{i}-I_h{\rm log} c^{m+1}_{i}, \xi_{c_i}^{m+1})
%   +\langle {\rm log} (I_hc^{m+1}_{i})-{\rm log} P_h c^{m+1}_{i}, \xi_{c_i}^{m+1}\rangle.
% \end{eqnarray*}
%  Again in the last step, we have used the inequality $
%  \langle {\rm log} (P_hc^{m+1}_{i})-{\rm log} c^{m+1}_{ih}, P_h {c_i}^{m+1}-c_{ih}^{m+1}\rangle \ge 0.
%$
Substituting the above equation into \eqref{eq:01}, we obtain that
\begin{eqnarray*}
 \tilde a(  \xi_\phi^{m+1}, \xi_{\phi}^{m+1})+\tau \sum_{i=1}^n a_{c_{ih}^{m}}(\xi_{p_i}^{m+1},\xi_{p_i}^{m+1})
\le \sum_{i=1}^n( q_i \eta_{c_i}^{m+1}, \xi_{\phi}^{m+1})+\sum_{i=1}^n(\xi_{c_i}^{m}, \xi_{p_i}^{m+1})-\tilde a( \eta_{\phi}^{m+1}, \xi_\phi^{m+1})+ J,
 \end{eqnarray*}
  where
 \begin{eqnarray*}
 J=
   \tau \left(\sum_{i=1}^n -a_{c_{ih}^{m}}(\eta_{p_i}^{m+1},\xi_{p_i}^{m+1})-
  a_{e_{c_{i}^{m}}}(p^{m+1}_i,\xi_{p_i}^{m+1})+ I(\xi_{p_i}^{m+1})\right).
 \end{eqnarray*}
   Using the coercivity property of the bilinear form $a(\cdot,\cdot)$, the uniformly low boundedness of $c_{ih}^m$,
   \eqref{esti:a1}, and the a-priori assumption \eqref{assumption:0}, we can derive that
 \begin{eqnarray*}
     \|\xi_\phi^{m+1}\|_E^2+\frac{\delta_0}{4}\tau\sum_{i=1}^n\|\xi_{p_i}^{m+1}\|_E^2\lesssim
     h^{k+1}\|\xi^{m+1}_{\phi}\|_E+\sum_{i=1}^n \|\xi^m_{c_i}\|_{0}\|\xi_{p_i}^{m+1}\|_0+|J|.
  \end{eqnarray*}
  We next proceed to estimate the term $J$. Using \eqref{error2} and \eqref{esti:a1} once more, we derive that
 \begin{eqnarray*}
   |J| &= & \tau \left(\sum_{i=1}^n -a_{c_{ih}^{m}}(\eta_{p_i}^{m+1},\xi_{p_i}^{m+1})-
  a_{\xi_{c_{i}^{m}}}(p^{m+1}_i,\xi_{p_i}^{m+1})+ I(\xi_{p_i}^{m+1})\right)\\
      &\lesssim & \tau  \sum_{i=1}^n ( \tau+ h^{k+1}+ \|\xi^m_{c_i}\|_{0})\|\xi_{p_i}^{m+1}\|_E.
 \end{eqnarray*}
By combining the last two inequalities and using the Poincare inequality, we obtain
 \begin{eqnarray}\label{xi:p}
     \|\xi_\phi^{m+1}\|_E^2+\frac{\delta_0}{4}\tau\sum_{i=1}^n\|\xi_{p_i}^{m+1}\|_E^2\le C
     h^{2k+2} +\tau (\tau+h^{k+1})^2+\tau^{-1} \sum_{i=1}^n  \|\xi^m_{c_i}\|^2_0.
  \end{eqnarray}
    On the other hand, by choosing  $v=\xi_{c_i}^{m+1}-\xi_{c_i}^{m} $ in \eqref{error10} and using
   \eqref{error2} and \eqref{esti:a1}, we have
%
%  Now we suppose $\xi_{c_i}^{m+1}$ has the maximal value at the point $x_0$ with $x_0\in\tau_0$, i.e.,
% \[
%     \|\xi_{c_i}^{m+1}\|_{0,\infty}=\xi_{c_i}^{m+1}(x_0).
% \]
%  Now we choose $w=l_{\tau_0,j}\in V_h$ the Lagrange interpolation function associated with the
%  points $g_{\tau_0,j}$ and then obtain

%\begin{eqnarray}
%\begin{split}
%    (\frac{\xi_{c_i}^{m+1}-\xi_{c_i}^{m}}{\triangle t}, \xi_{p_i}^{m+1})&=a_{c_{ih}^{m}}(p^{m+1}_{ih},\xi_{p_i}^{m+1})-
%    a_{c_{i}^{m}}(p^{m+1}_i,\xi_{p_i}^{m+1})+I(\xi_{p_i}^{m+1}) &\\
%    &=a_{c_{ih}^{m}}(p^{m+1}_{ih}-p_i^{m+1},\xi_{p_i}^{m+1})-
%    a_{c_{i}^{m}-c_{ih}^{m}}(p^{m+1}_i,\xi_{p_i}^{m+1})+I(\xi_{p_i}^{m+1}) &
%%   & =-
%%  a_{c_{ih}^{m}}(\xi_{p_i}^{m+1},\xi_{p_i}^{m+1})-a_{c_{ih}^{m}}(\eta_{p_i}^{m+1},\xi_{p_i}^{m+1})-
%%  a_{c_{i}^{m}-c_{ih}^{m}}(p^{m+1}_i,\xi_{p_i}^{m+1})+I(\xi_{p_i}^{m+1}),&
%   \end{split}
%\end{eqnarray}
\begin{eqnarray*}
\| \xi_{c_i}^{m+1}-\xi_{c_i}^{m}\|_0^2 &= & \tau\left(
 a_{c_{ih}^{m}}(e_{p_i}^{m+1},\xi_{c_i}^{m+1}-\xi_{c_i}^{m} )-
    a_{e_{c_i}^m}(p^{m+1}_i,\xi_{c_i}^{m+1}-\xi_{c_i}^{m} )+I(\xi_{c_i}^{m+1}-\xi_{c_i}^{m} )\right)\\
    &\lesssim & \tau (\|\xi_{p_i}^{m+1}\|_E+\|e_{c_i}^m\|_0+\tau+h^{k+1} )\|\xi_{c_i}^{m+1}-\xi_{c_i}^{m}\|_E\\
    &\lesssim &  \frac{\tau}{h}  (\|\xi_{p_i}^{m+1}\|_E+\|\xi_{c_i}^m\|_0+\tau+h^{k+1} )\|\xi_{c_i}^{m+1}-\xi_{c_i}^{m}\|_0.
\end{eqnarray*}
  Here in the last step, we incorporate the inverse inequality.
  Using the estimate in \eqref{xi:p} and the a-priori assumption, we derived the following bound:
\begin{eqnarray*}
   \| \xi_{c_i}^{m+1}-\xi_{c_i}^{m}\|_0\le C  \frac{\tau}{h} ( \tau^{-\frac 12}h^{k+1}+\tau^{-1}\sum_{i=1}^n\|\xi_{c_i}^m\|_0  +\tau).
\end{eqnarray*}
Consequently,
 \[
    \| \xi_{c_i}^{m+1}\|_0\le \| \xi_{c_i}^{m}\|_0+\| \xi_{c_i}^{m+1}-\xi_{c_i}^{m}\|_0
    \le C  \frac{\tau}{h} ( \tau^{-\frac 12}h^{k+1}+\tau^{-1}\sum_{i=1}^n\|\xi_{c_i}^m\|_0  +\tau).
 \]
 Applying the inverse inequality again, along with \eqref{cfl}  yields
 \begin{eqnarray*}
    \| \xi_{c_i}^{m+1}\|_{0,\infty}\le C h^{-1} \| \xi_{c_i}^{m+1}\|_{0}
    \le C \frac{\tau}{h^{2}} (\frac{h^{k+1}}{\tau^{\frac 12}}+\frac{\sum_{i=1}^n\|\xi_{c_i}^m\|_0}{\tau}+\tau)
  \le  C \epsilon.
 \end{eqnarray*}
   Therefore, if \eqref{cfl} holds with $ C \epsilon\le \frac{\delta_0}{4}$, we conclude that
 \[
    \| c_{ih}^{m+1}\|_{0,\infty}\ge \|I_hc_i^{m+1}\|_{0,\infty}-\frac{\delta_0}{4}\ge \frac{\delta_0}{4}.
 \]
   This finishes our proof.
  \end{proof}


  \begin{remark}
     The inequalities derived from \eqref{assumption:0} and \eqref{cfl}, reveal that,
     to maintain the positivity of $c_{ih}^{m+1}$ in a point-wise manner, the time mesh size $\tau$ and
     the space mesh size $h$ must adhere to the relationship
  \[
     \tau\le c h^{2},
  \]
   for some positive constant $c$.  However, when employing high-order time-discretization methods, such as a second-order scheme, the a-priori assumption \eqref{assumption:0} undergoes a modification
 \[
    \|c^m_{ih}-I_hc_i^m\|_E\lesssim \tau^2+h^{k+1},
 \]
 whereupon analysis of \eqref{cfl}  suggests that
 \[
    \tau\le c h.
 \]
 In other words, to guarantee positivity in a point-wise manner, the constraint imposed on the time step size for lower-order schemes is considerably more stringent compared to that for higher-order time discretization schemes.
 \end{remark}


% We have the following error estimate for the term $a(c_{i}^{m}e_{p_i}^{m+1},\xi_{c_i}^{m+1})$.
\begin{lemma}
Assume that \eqref{assuption:u} and the a-priori assumption \eqref{assumption:0} are valid. Then, there exist positive constants $a_i,  0\le i\le 2$ such that
  \begin{equation}\label{error:11}
   \sum_{i=1}^na_{c_{ih}^{m}}(e_{p_i}^{m+1},\xi_{c_i}^{m+1}) \ge \sum_{i=1}^n\left( a_0\| \xi_{c_i}^{m+1}\|_E^2-a_1\|\xi_{c_i}^{m+1}\|^2_0\right)
  -a_2h^{2k+2}.
  \end{equation}
\end{lemma}
\begin{proof}
Considering the equation \eqref{error20}, we can write
\begin{equation}\label{eq:3}
  a_{c_{ih}^{m}}(e_{p_i}^{m+1}, \xi_{c_i}^{m+1})=a_{c_{ih}^{m}}(q_i e_{\phi}^{m+1} +{\rm log} c^{m+1}_{i}-{\rm log} c^{m+1}_{ih},  \xi_{c_i}^{m+1}).
\end{equation}
% \begin{eqnarray}\label{eq:3}
% \begin{split}
%     a_{c_{ih}^{m}}(e_{p_i}^{m+1}, \xi_{c_i}^{m+1})&=a_{c_{ih}^{m}}(q_i e_{\phi}^{m+1} +{\rm log} c^{m+1}_{i}-I_{h}{\rm log} c^{m+1}_{ih},  \xi_{c_i}^{m+1}) & \\
%         &=a_{c_{ih}^{m}} (q_i e_{\phi}^{m+1}, \xi_{c_i}^{m+1}) +a_{c_{ih}^{m}}
%         ( {\rm log} c^{m+1}_{i}-I_h{\rm log} c^{m+1}_{i}, \xi_{c_i}^{m+1}) &\\
%        &+  a_{c_{ih}^{m}}(  I_h ({\rm log} c^{m+1}_{i}-{\rm log} c^{m+1}_{ih}),  \xi_{c_i}^{m+1}). &
%  \end{split}
%  \end{eqnarray}
 We next estimate the  terms on the right-hand side of  \eqref{eq:3}.
%   Using the boundedness of $c_{ih}^m$ and the inequality \eqref{esti:a1}, we proceed as follows:
% \begin{equation}\label{eq:0}
%    |a_{c_{ih}^{m}}({\rm log} c^{m+1}_{i}-I_h{\rm log} c^{m+1}_{i},  \xi_{c_i}^{m+1})|\lesssim h^{k+1}\|c^{m+1}_i\|_{k+2}\|  \xi_{c_i}^{m+1}\|_E.
%  \end{equation}
     By choosing $w=\xi_{\phi}$ in \eqref{error30}, we derive that
 \begin{eqnarray*}
   \|\xi_{\phi}^{m+1}\|^2_E&\lesssim& \tilde a(\xi_{\phi}^{m+1},\xi_{\phi}^{m+1})=- \tilde a(\eta_{\phi}^{m+1},\xi_{\phi}^{m+1})+\sum_{i=1}^n ( q_i e_{c_i}^{m+1}, \xi_{\phi}^{m+1} )\\
   &\lesssim&  h^{k+1}\|\xi_{\phi}^{m+1}\|_E+\sum_{i=1}^n  \|\xi_{c_i}^{m+1}\|_0\|\xi_{\phi}^{m+1}\|_0,
 \end{eqnarray*}
  and thus
\begin{equation}\label{eq:4}
   \|\xi_{\phi}^{m+1}\|_E\lesssim \sum_{i=1}^n\|\xi_{c_i}^{m+1}\|_0+h^{k+1}.
 \end{equation}
 Consequently, we estimate the first term as follows:
 \begin{eqnarray}\label{eq:1}
 \begin{split}
    \left|\sum_{i=1}^na_{c_{ih}^{m}} (q_i e_{\phi}^{m+1}, \xi_{c_i}^{m+1})\right|= \left|\sum_{i=1}^na_{c_{ih}^{m}} (q_i \xi_{\phi}^{m+1}, \xi_{c_i}^{m+1})\right|
    &\lesssim  \| \xi_{\phi}^{m+1} \|_E\sum_{i=1}^n \| \xi_{c_i}^{m+1}\|_E &\\
    &\lesssim \sum_{i=1}^n
   ( \|\xi_{c_i}^{m+1}\|_0+h^{k+1})\| \xi_{c_i}^{m+1}\|_E.  &
  \end{split}
 \end{eqnarray}
     On the other hand, by using the Taylor expansion , there exists a  $u_i$ between
   $c^{m+1}_{i}$ and  $c^{m+1}_{ih}$  such that
\begin{eqnarray*}
   {\rm log} c^{m+1}_{i}-{\rm log} c^{m+1}_{ih}=
   \frac{1}{u_i}e_{c_i}^{m+1} =
   \frac{1}{u_i}(\xi_{c_i}^{m+1}+\eta_{c_i}^{m+1}).
\end{eqnarray*}
Then
\[
   a_{c_{ih}^m}({\rm log} c^{m+1}_{i}-{\rm log} c^{m+1}_{ih},  \xi_{c_i}^{m+1})=a_{c_{ih}^m}(\frac{\xi_{c_i}^{m+1}+\eta^{m+1}_{c_i}}{u_i}, \xi_{c_i}^{m+1})=a_{c_{ih}^m}( \frac{\xi_{c_i}^{m+1}}{u_i} ,  \xi_{c_i}^{m+1}).
 %  &=&(c_{ih}^m \nabla I_h \frac{\xi_{c_i}^{m+1}}{\bar \epsilon_i}, \nabla \xi_{c_i}^{m+1})+I_1+I_2,
\]
%   where
%\begin{eqnarray*}
%    I_1=  (\nabla( I_h(\frac{1}{\epsilon_i}-\frac{1}{\bar \epsilon_i} )
%   \xi_{c_i}^{m+1}), \nabla \xi_{c_i}^{m+1}), \ \
%   I_2=(c_{ih}^m \nabla I_h \frac{\eta_{c_i}^{m+1}}{\epsilon_i} , \nabla \xi_{c_i}^{m+1}).
%\end{eqnarray*}
 In light of  \eqref{assuption:u} and \eqref{ci}, we have,
 \[
       \frac{\delta_0}{4}\le u_i\le M+ \frac{\delta_0}{4},
   \]
   which yields, together with \eqref{bounds:1} and the coercivity of $a(\cdot,\cdot)$ in \eqref{coer:2},
 \begin{eqnarray*}
     a_{c_{ih}^m}( \frac{\xi_{c_i}^{m+1}}{u_i} ,  \xi_{c_i}^{m+1}) \ge \frac{\delta_0  \gamma_0 }{2(M +\frac{ \delta_0}{2})}\| \xi_{c_i}^{m+1}\|^2_E.
 \end{eqnarray*}
  Consequently,
 \begin{equation*}
   a_{c_{ih}^m}( I_h ({\rm log} c^{m+1}_{i}-{\rm log} c^{m+1}_{ih}),  \xi_{c_i}^{m+1})\ge
    \frac{\delta_0\gamma_0}{2(M +\frac{ \delta_0}{2})}  \| \xi_{c_i}^{m+1}\|_E^2.
\end{equation*}
% \begin{eqnarray*}
%   I_1\ge  c_0\|\nabla \xi_{c_i}^{m+1}\|_0^2-h\| \nabla \xi_{c_i}^{m+1}\|^2_0-c\|\xi_{c_i}^{m+1}\|_0
%   \| \nabla \xi_{c_i}^{m+1}\|_0.
%  \end{eqnarray*}
%   we have \eqref{interpolation-appro}
%\begin{eqnarray*}
%  I_2&=&(c_{ih}^m \nabla I_h (\frac{\eta_{c_i}^{m+1}}{\epsilon_i}-\frac{\eta_{c_i}^{m+1}}{\bar\epsilon_i}), \nabla \xi_{c_i}^{m+1})+
%  (c_{ih}^m \nabla (I_h-I) \frac{\eta_{c_i}^{m+1}}{\bar\epsilon_i}, \nabla \xi_{c_i}^{m+1})+(c_{ih}^m \nabla  \frac{\eta_{c_i}^{m+1}}{\bar\epsilon_i}, \nabla \xi_{c_i}^{m+1})\\
%  &\lesssim &(\bar c_{ih}^m \nabla I_h (\frac{\eta_{c_i}^{m+1}}{\epsilon_i}-\frac{\eta_{c_i}^{m+1}}{\bar\epsilon_i}), \nabla \xi_{c_i}^{m+1})+
%   (\|\eta^{m+1}_{c_i}\|_{0,\infty}+h^{k+1}|c_i^{m+1}|_{k+2})\|\nabla \xi_{c_i}^{m+1} \|_0.
% % &\lesssim & h^{k+1}|c_i^{m+1}|_{k+2}\|\nabla \xi_{c_i}^{m+1} \|_0.
%\end{eqnarray*}
Substituting the above inequality, and \eqref{eq:1} into \eqref{eq:3} and using
   the Cauchy-Schwarz inequality, we can derive the desired result \eqref{error:11}.
% \[
%    |(c_{ih}^{m}\nabla e_{p_i}^{m+1},\nabla \xi_{c_i}^{m+1})|\ge  c_0\|\nabla \xi_{c_i}^{m+1}\|_0^2-c_1
%    \sum_{i=1}^n \|\xi_{c_i}^{m+1}\|^2_0-c_2h^{2k+2}.
% \]
%   This finishes the proof  of \eqref{error:11}.
   \end{proof}


   Now we are ready to present the error estimates for the fully-discretization numerical solution.
 \begin{theorem}\label{theo:1}
 Let $c_{ih}^{m+1},p_{ih}^{m+1},\phi_{h}^{m+1}$ be the solution of \eqref{ci1-femdg}-\eqref{ci3-femdg} with the
  coefficient $\beta_1=\frac{1}{2k(k+1)}$ for the DDG discretization.
   Assume that \eqref{assuption:u} and the a-priori assumption \eqref{assumption:0}  hold true.
Then
 \begin{eqnarray}\label{optimal}
   &&\|c^{m+1}_i-c^{m+1}_{ih}\|_0+\|I_hc^m_i-c^m_{ih}\|_E\lesssim \tau+h^{k+1},\\\label{super:1}
   &&\|\phi^{m+1}-\phi^{m+1}_h\|_0+\|I_h\phi^m-\phi^m_h\|_E\lesssim \tau+h^{k+1},\\\label{optimal:1}
   &&\|p_i^{m+1}-p_{ih}^{m+1}\|_0\lesssim \tau+h^{k+1}.
 \end{eqnarray}
  \end{theorem}
 \begin{proof}
 First, we start by choosing $v=\xi_{c_i}^{m+1}$ in \eqref{error10}, which yields that
 \begin{eqnarray*}
    \frac{1}{2\tau}(\|\xi_{c_i}^{m+1}\|^2_0-\|\xi_{c_i}^{m}\|^2_0+\|\xi_{c_i}^{m+1}-\xi_{c_i}^{m}\|^2_0)
    +a_{c_{ih}^{m}} (e_{p_i}^{m+1},\xi_{c_i}^{m+1})=a_{e^m_{c_i}}(p_{i}^{m+1},\xi_{c_i}^{m+1})+I(\xi_{c_i}^{m+1}).
 \end{eqnarray*}
Using the Cauchy-Schwarz inequality and the a-priori assumption \eqref{assumption:0}, we can bound the term $a_{e^m_{c_i}}(p_{i}^{m+1},\xi_{c_i}^{m+1})$ as follows:
\begin{eqnarray}\label{error3}
   |a_{e^m_{c_i}}(p_{i}^{m+1},\xi_{c_i}^{m+1})|\lesssim \|p_{i}^{m+1}\|_{2,\infty}\|e^m_{c_i}\|_0\|\xi_{c_i}^{m+1}\|_E
    \lesssim (\tau+h^{k+1})\|\xi_{c_i}^{m+1}\|_E.
\end{eqnarray}
 Next, combining \eqref{error:11},  \eqref{error2} and \eqref{error3}  together, we have
\begin{eqnarray*}
    \frac{1}{2\tau}(\|\xi_{c_i}^{m+1}\|^2_0-\|\xi_{c_i}^{m}\|^2_0+\|\xi_{c_i}^{m+1}-\xi_{c_i}^{m}\|^2_0)
    +a_0\|\xi_{c_i}^{m+1}\|^2_E\le C (\tau+h^{k+1})^2+\frac{a_0}{2} \|\xi_{c_i}^{m+1}\|^2_E +C \|\xi_{c_i}^{m+1}\|^2_0,
 \end{eqnarray*}
   and thus
   \begin{eqnarray*}
      \|\xi_{c_i}^{m+1}\|^2_0-\|\xi_{c_i}^{m}\|^2_0
    +a_0\|\xi_{c_i}^{m+1}\|^2_E\lesssim \tau (\tau+h^{k+1})^2 +C\tau \|\xi_{c_i}^{m+1}\|^2_0.
 \end{eqnarray*}
  Here $a_0$ is a constant the same as that in \eqref{error:11}.
   By using the discrete Gronwall inequality, we derive the desired result \eqref{optimal} directly.
   The inequality \eqref{super:1} follows from
    \eqref{eq:4}, \eqref{optimal} and the approximation property of $I_h$.
   To obtain the error estimates for the variable $p_i$, we adopt the error equation \eqref{error20} to derive that
  \[
    \|\xi^{m+1}_{p_i}\|_0\lesssim \|\xi^{m+1}_{\phi}\|_0+\| {\rm log} {c_i^{m+1}}-{\rm log} {c_{ih}^{m+1}}\|_0\lesssim \|\xi^{m+1}_{\phi}\|_0+\| \xi_{c_i}^{m+1}\|_0
    \lesssim \tau+h^{k+1}.
  \]
  Then \eqref{optimal:1} follows.
 \end{proof}

 \begin{remark}
   The error estimates in \eqref{optimal}-\eqref{super:1} reveal a superconvergence phenomenon for the numerical solutions $c_{ih},\phi_h$
    superconvergent towards the specially constructed projection of the exact solution
    $I_hc_{i}, I_h\phi_h$ in the $H^1$-norm in space, one-order higher than the optimal convergence rate $h^{k}$.
 \end{remark}

 As a direct consequence of the above theorem, we have the following superconvergence results of the derivative approximation
 at Gauss points for the  numerical solutions $c_{ih},\phi_h$.

 \begin{corollary}\label{coro:1}
Under the assumptions of Theorem \ref{theo:1}, he following superconvergence result holds for the numerical solutions $c_{ih}^m$ and $\phi^m_h$:
 \begin{eqnarray}
    \frac{1}{N}\sum_{K\in{\cal T}_h}\sum_{z\in{\cal G}} \left( \nabla (c^m_i-c^m_{ih})^2(z)+ \nabla (\phi^m-\phi^m_{h})^2(z)\right)^{\frac 12}
     \lesssim \tau+h^{k+1}.
 \end{eqnarray}
   Here ${\cal G}$ denotes the set of Gauss points  of degree $k$ in  the whole  domain and
    $N$ represents its  cardinality.
 \end{corollary}
 \begin{proof} Initially, by using the inverse inequality, we have
 \begin{eqnarray*}
   && \left(\frac{1}{N}\sum_{T\in{\cal T}_h}\sum_{z\in{\cal G}}  \nabla (I_hc^m_i-c^m_{ih})^2(z)\right)^{\frac 12}+
     \left(\frac{1}{N}\sum_{T\in{\cal T}_h} \sum_{z\in{\cal G}} \nabla (I_h\phi^m-\phi^m_{h})^2(z)\right)^{\frac 12}\\
     &\le & \|I_hc^m_i-c^m_{ih}\|_E+\|I_h\phi^m-\phi^m_{h}\|_E
      \lesssim \tau+h^{k+1}.
  \end{eqnarray*}
 Furthermore, as demonstrated in reference \cite{chen-hu}, for any function $u$ and its interpolation $I_hu$, the following inequality holds:

  \[
     | \nabla(u-I_hu)(z) | \lesssim h^{k+1}|u|_{k+2,\infty}.
  \]
    Then the desired result follows from the last two inequalities and the triangle inequality.
  \end{proof}


  \section{Numerical Results}In this section, we present some numerical experiments to validate our theoretical findings. In our numerical experiment, we will test various
  various  errors which are defined in Theorem \ref{theo:1} and Corollary \ref{coro:1}.
  For simplicity, we adopt the following notation
  \begin{eqnarray*}
\text{e}^{v}_{A}=\left(\frac{1}{M}\sum_{T\in{\cal T}_h}\left(\int_{\tau}\nabla( v- v_h)\text{d}x\right)^2\right)^{1/2},\text{e}_{G}^{v}=\left(\frac{1}{N}\sum_{T\in{\cal T}_h}\sum_{z\in{\cal G}}  \nabla (v-v_{h})^2(z)\right)^{\frac 12}.
\end{eqnarray*}
  to test the superconvergence phenomenon  for the variable $v$ with $v=u,\phi, c_i$.

\begin{example}\label{ex:problem 1}
\end{example}
In this example, we consider a modified PNP system on $\Omega=[0,1]$, accompanied by source terms, such that an exact solution can be obtained. Specifically, the governing equations of the system are as follows:
\begin{eqnarray*}
  &&\partial_tc_1=\partial_x(\partial_xc_1+c_1\partial_x\phi)+f_1,\\
  &&\partial_tc_2=\partial_x(\partial_xc_2-c_2\partial_x\phi)+f_2,\\
  &&-\partial_x^2\phi=c_1-c_2+f_3,
\end{eqnarray*}
where the functions $f_i(t,x)$ are determined by the following exact solution:
\begin{equation*}
  c_1=10^{-3}(\cos(\pi x)+2)e^{-t},c_2=10^{-3}(\cos(2\pi x)+3/2)e^{-t},\phi=10^{-3}(\cos(2\pi x)-1)e^{-t}.
\end{equation*}
The initial conditions are obtained by computing the exact solutions at
$t = 0$. For $c_i$, the boundary
conditions meet the zero flux boundary conditions. Regarding $\phi$ the boundary data satisfies:
\begin{equation*}
  \phi(t,0)=0,\partial_x\phi(t,1)=0.
\end{equation*}

We address the problem by employing the scheme described in \eqref{ci1-femdg}-\eqref{ci3-femdg}. Here, the value of $k$ ranges from $1$ to $3$, and the time step is set as $\Delta t=0.01h^3$. In Table 6.1, we display the $L^2$ norms of the error
$\text{e}_0$ at $t=0.1$, alongside their respective convergence orders. Specifically, for the DDG scheme, we adopt $\beta_0=4$. As for $\beta_1$, we choose values of $\frac{1}{4},\frac{1}{12}$ and $\frac{1}{24}$ corresponding to $k=1$, $2$, and $3$, respectively. Notably, both the DDG and FEM schemes demonstrate optimal convergence rates of $k+1$, aligning with our theoretical predictions  presented  in Theorem 5.1.

Furthermore, Table 6.2 showcases the superconvergence outcomes of our proposed methodology. It reveals that the derivative approximations at Gauss points for the numerical solutions $c_h$ and $\phi_h$, obtained using both DDG and FEM, exhibit an order of $O(h^{k+1})$, thereby validating Corollary 5.1. Moreover, the average cell errors of the derivative $\text{e}_{A}^{c_i}$ and $\text{e}_{A}^\phi$ for both FEM and DDG also exhibit a superconvergence phenomenon with at least an order of $O(h^{k+1})$.
 Additionally, we observe that the $L^2$ error stemming from FEM is marginally higher compared to that of the DDG method. Specifically, for the DDG method, the average $H^1$ error of $c_i$ is at least of the order $O(h^{k+1})$, whereas for FEM, it remains at $\mathcal{O}(h^{k+1})$.

Finally, we solve this problem by employing the scheme outlined in \eqref{ci2-femdg}-\eqref{ci4}, with $k$ varying from $1$ to $3$ and a time step of $\Delta t=0.01h^2$. The $L^2$ norms of the error
 at $t=0.1$, alongside their respective convergence orders are presented in Table 6.3, similar optimal convergence results have been observed.


\begin{table}[!h]
	\small{\caption{\emph{Example $\ref{ex:problem 1}$~-- Numerical results of the \eqref{ci1-femdg}-\eqref{ci3-femdg} with $k=1,2,3$}}}
	\label{Tab:Eg1b}	
   \centering
	\begin{tabular}{ll|lllll|llll}
		\hline
		  &               &\multicolumn{4}{c}{DDG}&& \multicolumn{4}{c}{FEM} \\
		\hline
	$~~k$&$~N$ &~~$\|e_{c_1}\|_0$   &~$\|e_{c_2}\|_0$     &~~~$\|e_{p_1}\|_0$ &~$\|e_{\phi}\|_0$
                        & &$~~\|e_{c_1}\|_0$     &~$\|e_{c_2}\|_0$   &~~~$\|e_{p_1}\|_0$ &~~~$\|e_{\phi}\|_0$ \\
		\hline						
\multirow{4}{*}{$k=1$}
		& $~20$ & 8.75E-6 &6.68E-6 &4.18E-3 &6.63E-6  && 1.18E-5 &8.95E-6  &5.43E-3&8.91E-6 \\
		& $~40$ & 2.13E-6 &1.68E-6 &1.06E-3 &1.62E-6  && 2.99E-6 &2.24E-6  &1.37E-3&2.23E-6 \\
		& $~80$ & 5.56E-7 &4.20E-7 &2.67E-4 &4.00E-7  && 7.47E-7 &5.60E-7  &3.43E-4&5.57E-7 \\
		& $160$ & 1.39E-7 &1.05E-7 &6.68E-5 &9.94E-8  && 1.87E-7 &1.40E-7  &8.56E-5&1.39E-7 \\
		\hline
&~~\text{R}    &~~2.00   &~~2.00  &~~2.00  &~~2.01   &&~~2.00   &~~2.00   &~~2.00 &~~2.00  \\
        \hline
		\multirow{5}{*}{$k=2$}
		& $~10$ & 1.49E-6  &1.03E-6  &9.16E-4 &1.01E-6 &&2.03E-6 &1.40E-6  &1.20E-3 &1.42E-6 \\
		& $~20$ & 1.89E-7  &1.25E-7  &1.12E-4 &1.25E-7 &&2.55E-7 &1.75E-7  &1.53E-4 &1.79E-7\\
		& $~40$ & 2.36E-8  &1.56E-8  &1.39E-5 &1.56E-8 &&3.20E-8 &2.20E-8  &1.91E-5 &2.23E-8\\
		& $~80$ & 2.95E-9  &1.95E-9  &1.73E-6 &1.95E-9 &&8.03E-9 &2.75E-9  &2.39E-6 &2.79E-9\\
	    & $160$ & 3.69E-10 &2.44E-10 &2.18E-7 &2.44E-10&&1.01E-9 &3.43E-10&2.99E-7 &3.49E-10\\
		\hline
 &~~\text{R} &~~3.00   &~~3.00   &~~3.00  &~~3.00    & &~~3.00  &~~3.00   &~~3.00   &~~3.00  \\
        \hline
		\multirow{5}{*}{$k=3$}
		& $~10$ & 6.91E-8  &4.41E-8  &5.28E-5 &3.77E-8  &&7.69E-8  &5.83E-8   &6.96E-5 &5.29E-8\\
		& $~20$ & 4.36E-9  &2.64E-9  &3.89E-6 &2.34E-9  &&4.85E-9  &3.70E-9   &4.41E-6 &3.33E-9\\
		& $~40$ & 2.73E-10 &1.64E-10 &2.45E-7 &1.44E-10 &&3.04E-10 &2.28E-10  &2.76E-7 &2.09E-10\\
		& $~80$ & 1.70E-11 &1.03E-11 &1.53E-8 &9.00E-11 &&1.90E-11 &1.41E-11  &1.73E-8 &1.35E-11\\
	    & $160$ & 1.07E-12 &6.44E-13 &9.56E-10 &5.63E-12 &&1.19E-12&8.82E-13  &1.08E-9 &8.44E-12\\
		\hline
 &~~\text{R}   &~~4.00    &~~4.00   &~~4.00   &~~4.00  &&~~4.00   &~~4.00  &~~4.00      &~~4.00\\
        \hline
	\end{tabular}
\end{table}

\begin{table}[!h]
	\small{\caption{\emph{Example $\ref{ex:problem 1}$~-- Superconvergence results of the \eqref{ci1-femdg}-\eqref{ci3-femdg} with $k=1,2,3$}}}
	\label{Tab:Eg1a2}	\centering
	\begin{tabular}{ll|lllll|llll}
		\hline
		$~ $  &       &\multicolumn{4}{c}{DDG} && \multicolumn{4}{c}{FEM} \\
		%\cline{3-6}
		%\cline{7-10}
		\hline
		$~~k$&$~~\text{N}$  &~~$\text{e}_{A}^{c_1}$   &~~~$\text{e}_{A}^{c_2}$  &~~~$\text{e}_{A}^{\phi}$ &~~~$\text{e}_{G}^{c_1}$   & &~~$\text{e}_{A}^{c_1}$   &~~~$\text{e}_{A}^{\phi}$ &~~~$\text{e}_{G}^{c_1}$
&~~~$\text{e}_{1,G}^{\phi}$\\
		\hline
		\multirow{6}{*}{$k=1$}
		& $~~~20$ & 1.91E-6 &1.53E-6  &2.63E-6 &4.63E-5    && 6.17E-7 &8.57E-12  &1.83E-5 &1.81E-5 \\
		& $~~~40$ & 2.40E-7 &1.99E-7  &4.37E-7 &1.18E-5    && 1.55E-7 &2.15E-12  &4.57E-6 &4.52E-6 \\
		& $~~~80$ & 3.00E-8 &2.51E-8  &7.48E-8 &2.96E-6    && 3.89E-8 &5.39E-13  &1.14E-6 &1.13E-6 \\
		& $~~160$ & 3.76E-9 &3.14E-9  &1.30E-8 &7.40E-7    && 9.73E-9 &1.35E-13  &2.85E-7 &2.84E-7 \\
        %& $263169$ & 4.28E-8 &3.56E-5  &4.09E-7 &4.73E-5 & 4.72E-8 &3.56E-5  &3.89E-7 &4.72E-5\\
		\hline
 &~~\text{Rate} &~~3.00     &~~3.00     &~~2.50 &~~2.00    &&~~2.00   &~~2.00  &~~2.00 &~~2.00  \\
        \hline
		\multirow{4}{*}{$k=2$}
		& $~~~10$ & 2.95E-7 &2.29E-7  &1.94E-7 &8.11E-6     &&2.21E-7 &2.17E-11  &1.83E-6 &1.84E-5 \\
		& $~~~20$ & 1.32E-8 &9.77E-9  &8.22E-9 &8.19E-7     &&1.42E-8 &1.44E-12  &2.31E-7 &2.31E-6 \\
		& $~~~40$ & 4.26E-10 &3.25E-10  &3.56E-10 &9.76E-8  &&8.93E-10 &9.34E-14  &2.90E-8 &2.90E-7 \\
		& $~~~80$ & 1.35E-11 &1.03E-11  &1.56E-11&1.21E-8   &&5.58E-11 &5.84E-15  &3.63E-9 &3.63E-8 \\
	    & $~~160$ & 4.22E-13 &3.28E-13  &6.63E-13 &1.51E-9  &&3.49E-12 &3.14E-15  &4.54E-10&4.54E-10\\
		\hline
 &~~\text{Rate} &~~5.00     &~~5.00    &~~4.55 &~~3.00      &&~~4.00 &~~4.00 &~~3.00 &~~3.00    \\
        \hline
 \multirow{4}{*}{$k=3$}
		& $~~~10$ & 1.49E-7 &1.05E-7    &9.71E-8 &4.21E-6  &&2.57E-8 &1.47E-12  &1.31E-6 &1.32E-6 \\
		& $~~~20$ & 8.89E-9 &6.22E-7    &4.15E-9 &3.57E-7  &&1.61E-9 &9.19E-14  &8.29E-8 &8.29E-8 \\
		& $~~~40$ & 3.61E-10 &2.57E-10  &1.64E-10&2.41E-8  &&1.01E-10 &5.74E-15 &5.18E-9 &5.18E-9 \\
		& $~~~80$ & 1.13E-11 &8.08E-12  &6.53E-12&1.50E-9  &&6.31E-12 &4.35E-15  &3.24E-10 &3.24E-10 \\
%	    & $~~160$ & 3.27E-13 &3.28E-13  &6.63E-13 &1.51E-9  &2.03E-1 &&9.30E-6 &2.34E-3  &4.08E-5 \\
		\hline
 &~~\text{Rate} &~~5.00     &~~5.00    &~~4.75 &~~4.00     &&~~4.00   &~~4.00  &~~4.00  &~~4.00  \\
        \hline
%		\multirow{4}{*}{$k=3$}
%		& $~~32$ & 4.82E-6 &      &1.84E-5 &       &7.66E-6 &      &3.27E-5 &      \\
%		& $~~64$ & 2.97E-7 &4.02  &1.19E-6 &$3.95$ &4.75E-7 &4.01  &2.09E-6 &$3.97$\\
%		& $~128$ & 1.84E-8 &4.01  &7.50E-8 &$3.99$ &2.96E-8 &4.00  &1.31E-7 &$3.99$\\
%		& $~256$ & 1.15E-9 &4.01  &4.70E-9 &$4.00$ &1.85E-9 &4.00  &8.22E-9 &$4.00$\\
%		\hline
	\end{tabular}
\end{table}


\begin{table}[!h]
	\small{\caption{\emph{Example $\ref{ex:problem 1}$~-- Numerical results of the \eqref{ci2-femdg}-\eqref{ci4} with $k=1,2,3$}}}
	\label{Tab:Eg1c}	
   \centering
	\begin{tabular}{ll|lllll|llll}
		\hline
		  &               &\multicolumn{4}{c}{DDG}&& \multicolumn{4}{c}{FEM} \\
		\hline
	$~~k$&$~N$ &~~$\|e_{c_1}\|_0$   &~$\|e_{c_2}\|_0$     &~~~$\|e_{p_1}\|_0$ &~$\|e_{\phi}\|_0$
                        & &$~~\|e_{c_1}\|_0$     &~$\|e_{c_2}\|_0$   &~~~$\|e_{p_1}\|_0$ &~~~$\|e_{\phi}\|_0$ \\
		\hline
		\multirow{4}{*}{$k=1$}
		& $~20$ & 7.29E-7 &4.95E-7 &2.79E-4 &5.10E-7  && 1.07E-6 &6.86E-7  &4.02E-4&6.52E-7 \\
		& $~40$ & 1.84E-7 &1.24E-7 &6.97E-5 &1.27E-7  && 2.68E-7 &1.72E-7  &1.01E-4&1.67E-7 \\
		& $~80$ & 4.59E-8 &3.11E-8 &1.74E-5 &3.19E-8  && 6.70E-8 &4.29E-8  &2.51E-5&4.18E-8 \\
		& $160$ & 1.15E-8 &7.78E-9 &4.35E-6 &7.96E-9  && 1.68E-8 &1.07E-8  &6.28E-6&1.05E-8 \\
		\hline
&~~\text{R}    &~~2.00   &~~2.00  &~~2.00  &~~2.01   &&~~2.00   &~~2.00   &~~2.00 &~~2.00  \\
        \hline
		\multirow{5}{*}{$k=2$}
%		& $~10$ & 1.49E-6  &1.03E-6  &9.16E-4 &1.01E-6 &&2.03E-6 &1.40E-6  &1.20E-3 &1.42E-6 \\
		& $~20$ & 6.27E-8  &4.29E-8  &2.23E-5 &5.65E-8  &&8.46E-8 &6.24E-8  &3.02E-5 &7.23E-8\\
		& $~40$ & 7.89E-9  &5.24E-9  &2.79E-6 &7.06E-9  &&1.06E-8 &7.80E-9  &3.79E-6 &9.06E-9\\
		& $~80$ & 9.86E-10 &6.55E-10 &3.49E-7 &8.83E-10 &&1.33E-9 &9.75E-10  &4.72E-7 &1.13E-9\\
	    & $160$ & 1.23E-10 &8.19E-11 &4.36E-8 &1.10E-10 &&1.65E-10 &1.22E-10 &5.90E-8 &1.42E-10\\
		\hline
 &~~\text{R} &~~3.00   &~~3.00   &~~3.00  &~~3.00    & &~~3.00  &~~3.00   &~~3.00   &~~3.00  \\
        \hline
		\multirow{5}{*}{$k=3$}
%		& $~10$ & 6.91E-8  &4.41E-8  &5.28E-5 &3.77E-8  &&7.69E-8  &5.83E-8   &6.96E-5 &5.29E-8\\
		& $~20$ & 1.35E-9  &9.63E-10  &9.44E-7 &1.16E-9  &&1.93E-9   &1.61E-9   &1.04E-6 &2.32E-9\\
		& $~40$ & 8.54E-11 &6.02E-11  &5.90E-8 &4.50E-11 &&1.21E-10  &1.01E-10  &6.50E-8 &1.45E-10\\
		& $~80$ & 5.34E-12 &3.76E-12  &3.68E-9 &2.83E-12 &&7.50E-12  &6.31E-12  &4.10E-9 &9.10E-12\\
	    & $160$ & 3.37E-13 &2.35E-13  &2.30E-10 &1.77E-13 &&4.71E-13 &3.93E-13  &2.54E-10 &5.66E-13\\
		\hline
 &~~\text{R}   &~~4.00    &~~4.00   &~~4.00   &~~4.00  &&~~4.00   &~~4.00  &~~4.00      &~~4.00\\
        \hline
	\end{tabular}
\end{table}

\begin{example}\label{ex:problem 2}
\end{example}
This example is to test the spatial accuracy of our scheme in a 2D setting. Similar to the Numerical Test 5.1 in [2], we consider the PNP problem (2.1) on $\Omega=[0,\pi]^2$ with source terms, i.e.,
\begin{eqnarray*}
  &&\partial_tc_1=\nabla\cdot(\nabla c_1+c_1\nabla\phi)+f_1, \\
  &&\partial_tc_2=\nabla\cdot(\nabla c_2-c_2\nabla\phi)+f_2, \\
  &&-\Delta\phi=c_1-c_2+f_3,
\end{eqnarray*}
where the functions $f_i(t, x, y)$ are determined by the following exact solution
\begin{eqnarray*}
  &&c_1(t,x,y)=10^{-3}(e^{-10^{-3t}}\cos(x)\cos(y)+1),\\
  &&c_2(t,x,y)=10^{-3}(e^{-10^{-3t}}\cos(x)\cos(y)+1),\\
  &&\phi(t,x,y)=10^{-3}e^{-10^{-3t}}\cos(x)\cos(y).
\end{eqnarray*}
The initial conditions  are obtained by evaluating the exact solution at $t = 0$, and the boundary conditions of $c_i$ and $\phi$ satisfy the zero flux boundary conditions \eqref{eqn: bcond}.

The $L^2$-norm errors and their respective orders, with $k=1$ and $2$ at $t = 0.01$, are reported in Table 6.4. Upon these findings, we confirm that both the DDG and FEM exhibit an order of $\mathcal{O}(h^{k+1})$  in space, aligning perfectly with our theoretical expectations outlined in Theorem 5.1. Mirroring the trends observed in 6.1, the $L^2$ errors errors reported by FEM are marginally higher compared to those attained by the DDG approach. Table 6.5 displays the numerical outcomes related to the derivative approximation at Gauss points and the average cell errors of the derivative. It reveals that these errors, for both the FEM method and the DDG approach, exhibit a superconvergence phenomenon with an order of at least $O(h^{k+1})$, further reinforcing the robustness and accuracy of our proposed methods.

\begin{table}[!h]
	\small{\caption{\emph{Example $\ref{ex:problem 2}$~-- Numerical results of the \eqref{ci1-femdg}-\eqref{ci3-femdg} with $k=1,2$}}}
	\label{Tab:Eg2a1}	
   \centering
	\begin{tabular}{ll|lllll|llll}
		\hline
		  &               &\multicolumn{4}{c}{DDG}&& \multicolumn{4}{c}{FEM} \\
		\hline
	$~~k$&$~N$ &~~$\|e_{c_1}\|_0$   &~$\|e_{c_2}\|_0$     &~~~$\|e_{p_1}\|_0$ &~$\|e_{\phi}\|_0$
                        & &$~~\|e_{c_1}\|_0$     &~$\|e_{c_2}\|_0$   &~~~$\|e_{p_1}\|_0$ &~~~$\|e_{\phi}\|_0$ \\
		\hline
		\multirow{4}{*}{$k=1$}
		& $~5$ & 9.76E-5 &9.76E-5 &4.18E-3 &5.94E-4  && 1.24E-4 &1.24E-4  &7.20E-2&1.49E-4 \\
		& $10$ & 2.51E-5 &2.51E-5 &1.06E-3 &1.51E-5  && 3.17E-5 &3.17E-5  &1.85E-2&3.94E-5 \\
		& $20$ & 6.34E-6 &6.34E-6 &2.67E-4 &5.83E-6  && 7.96E-6 &7.96E-6  &4.67E-3&9.99E-6 \\
		& $40$ & 1.59E-6 &1.59E-6 &6.68E-5 &1.49E-6  && 1.99E-6 &1.99E-6  &1.17E-3&2.51E-6 \\
        & $80$ & 3.98E-7 &3.98E-7 &1.67E-5 &3.73E-7  && 4.98E-7 &4.98E-7  &2.93E-4&5.38E-7 \\
		\hline
&~\text{R}    &~~~2.00   &~~~2.00  &~~~2.00  &~~~2.01   &&~~~2.00   &~~~2.00   &~~~2.00 &~~~2.00  \\
        \hline
		\multirow{5}{*}{$k=2$}
		& $~5$ & 3.17E-6  &3.17E-6  &1.99E-3 &3.13E-6 &&8.20E-6  &8.21E-6  &4.98E-3 &8.28E-6 \\
		& $10$ & 1.96E-7  &1.96E-7  &2.53E-4 &3.92E-7 &&1.06E-6  &1.06E-6  &6.69E-4 &1.07E-6\\
		& $20$ & 4.95E-8  &4.95E-8  &3.19E-5 &4.92E-8 &&1.35E-7  &1.35E-7  &8.47E-5 &1.34E-7\\
		& $40$ & 6.19E-9  &6.19E-9  &3.99E-6 &6.15E-9 &&1.71E-8  &1.71E-8  &1.06E-5 &1.68E-8\\
	    %& $160$ & 2.44E-10 &2.44E-10 &2.18E-7 &2.44E-10&&3.43E-10 &3.43E-10&2.99E-7 &3.49E-10\\
		\hline
&~\text{R}    &~~~3.00   &~~~3.00  &~~~3.00  &~~~3.00   &&~~~3.00   &~~~3.00   &~~~3.00 &~~~3.00  \\
        \hline
	\end{tabular}
\end{table}

\begin{table}[!h]
	\small{\caption{\emph{Example $\ref{ex:problem 2}$~-- Superconvergence results of the \eqref{ci1-femdg}-\eqref{ci3-femdg} with $k=1,2$}}}
	\label{Tab:Eg2a2}	\centering
	\begin{tabular}{ll|lllll|llll}
		\hline
		$~ $  &       &\multicolumn{4}{c}{DDG} && \multicolumn{4}{c}{FEM} \\
		%\cline{3-6}
		%\cline{7-10}
		\hline
		$~~k$&$~N$  &~~~~$\text{e}_{A}^{c_1}$   &~~~~$\text{e}_{A}^{c_2}$  &~~~~$\text{e}_{A}^{\phi}$ &~~~~$\text{e}_{G}^{c_1}$   & &~~~~$\text{e}_{A}^{c_1}$   &~~~~$\text{e}_{A}^{\phi}$ &~~~~~$\text{e}_{G}^{c_1}$
&~~~~$\text{e}_{G}^{\phi}$\\
		\hline
		\multirow{4}{*}{$k=1$}
		& $~5$ & 2.27E-5 &2.27E-5  &9.96E-5  &5.48E-5    && 2.34E-5 &4.63E-5  &4.64E-5 &6.92E-5 \\
		& $10$ & 6.70E-6 &6.70E-6  &8.68E-6  &1.51E-5    && 5.98E-6 &1.24E-6  &1.18E-5 &1.81E-5 \\
		& $20$ & 1.70E-6 &1.70E-6  &2.22E-6  &3.83E-6    && 1.50E-6 &3.15E-6  &2.96E-6 &4.56E-6 \\
		& $40$ & 4.28E-7 &4.28E-7  &5.62E-7  &9.58E-7    && 3.77E-7 &7.92E-7  &7.39E-6 &1.15E-6 \\
        & $80$ & 1.07E-7 &1.07E-7  &1.41E-7  &2.39E-7    && 9.43E-8 &1.99E-7  &1.85E-7 &2.88E-7 \\
		\hline
 &~\text{R} &~~~2.00     &~~~2.00     &~~~2.00 &~~~2.00    &&~~~2.00   &~~~2.00  &~~~2.00 &~~~2.00  \\
        \hline
		\multirow{4}{*}{$k=2$}
		& $~5$ & 1.00E-7 &1.00E-7  &7.02E-7 &9.49E-8     &&2.37E-7  &5.30E-7  &4.72E-7 &7.23E-7 \\
		& $10$ & 1.34E-8 &1.34E-8  &1.10E-7 &4.64E-8     &&3.03E-8  &6.40E-8  &6.04E-8 &9.16E-8 \\
		& $20$ & 9.49E-10 &9.49E-10  &1.39E-8 &5.56E-9   &&3.78E-9  &8.05E-9  &7.61E-9 &1.15E-8 \\
		& $40$ & 6.01E-11 &6.01E-11  &1.75E-9 &6.95E-10  &&4.73E-10 &1.01E-9  &9.51E-10&1.44E-9 \\
	%    & $~~160$ & 3.27E-13 &3.28E-13  &6.63E-13 &1.51E-9  &&3.72E-13 &1.67E-14  &3.15E-9 &3.29E-9 \\
		\hline
 &~\text{R} &~~~4.00     &~~~4.00    &~~~3.00 &~~~3.00      &&~~~3.00 &~~~3.00 &~~~3.00 &~~~3.00    \\
        \hline
	\end{tabular}
\end{table}






\begin{example}\label{ex:problem 3}
\end{example}
 We test our scheme for the 2D PNP system (2.1) with $m = 2$ on the domain $[0, 1]^2$,
\begin{eqnarray*}
  &&\partial_tc_1=\nabla\cdot(\nabla c_1+c_1\nabla\psi), \\
  &&\partial_tc_2=\nabla\cdot(\nabla c_2-c_2\nabla\psi), \\
  &&-\partial_x^2\phi=c_1-c_2,
\end{eqnarray*}
subjects to
\begin{eqnarray*}
  &&c_1(t,x,y)=\frac{1}{20}(\pi\sin(\pi x)+\pi\sin(\pi y)),\\
  &&c_2(t,x,y)=3(x^2(1-x)^2+y^2(1-y)^2),\\
  &&\frac{\partial c_i}{\partial n}+q_ic_i\frac{\partial \phi}{\partial_n}=0, (x,y)\in\partial \Omega,\\
  && \phi=0~\text{on}~\Omega_D,~\text{and}~\partial_n\phi=0~\text{on}~\Omega_N,
\end{eqnarray*}
where $\partial\Omega_D=\{(x,y)\in \Omega, x=0,x=1\}$ and $\partial\Omega_N=\partial\Omega\backslash \partial\Omega_D$.

For $t\in[0,1]$, Figure 6.1 depicts the minimum cell averages of $c_1, c_2$, along with the smallest values of $c_1$ and $c_2$ at the collocation points, calculated using a $20\times20$ mesh and various time steps of $\Delta t=10^{-4}$ and $\Delta t=10^{-5}$. From Figure 6.1, it is evident that when these values are far apart from $t=0$, both the smallest cell averages of $c_1$ and $c_2$,
as well as the minimum values of $c_1$ and $c_2$ at the collocation points remain positive, even utilizing a larger time step.

\begin{figure}[h]
%\begin{subfigure}[t]{0.3\textwidth}
\centering
\includegraphics[width=0.33\textwidth]{ex3tua1.png}
%\subcaptionbox{\label{fig:ex4tu1}$c_1-0.2$ at $t=0$}
\includegraphics[width=0.32\textwidth]{ex3tua2.png}
 \caption{Example $\ref{ex:problem 3}$, \small{\emph{smallest cell averages and  values at collocation points of $c_1$ and $c_2$}}}
%\caption{}
%\label{fig:Case1}
%\end{subfigure}
\end{figure}
To gain a deeper insight into the behavior near $t=0$, we revisited the problem for $t$ within the interval $[0,10^{-5}]$ utilizing a finer $40\times40$ mesh and a very small time step of $\Delta t=10^{-7}$. The smallest cell averages and values at collocation points of $c_1$ and $c_2$ are presented in Figure 6.2, which demonstrate that our numerical scheme successfully preserves positivity without the need for a limiter.
\begin{figure}[h]
%\begin{subfigure}[t]{0.3\textwidth}
\centering
\includegraphics[width=0.33\textwidth]{ex3tua3.png}
%\subcaptionbox{\label{fig:ex4tu1}$c_1-0.2$ at $t=0$}
\includegraphics[width=0.32\textwidth]{ex3tua4.png}
 \caption{Example $\ref{ex:problem 3}$, \small{\emph{smallest cell averages and  values at collocation points of $c_1$ and $c_2$}}}
%\caption{}
%\label{fig:Case1}
%\end{subfigure}
\end{figure}

We meticulously simulate the evolution of $c_1, c_2$, and $\phi$ over the time interval $t\in[0, 0.5]$. Figure 6.3 displays the contours of $c_1-0.2$ (first column), $c_2 -0.2$ (second column), and $\phi$ (right) at $t=0,0.01,0.1$ and $0.5$. We discern a remarkable similarity between the contours at $t = 0.1$ and $t = 0.5$, indicating that the system's behavior becomes increasingly uniform as it nears the end of the simulated period. This observation is consistent with the system's apparent progression towards a steady-state configuration, characterized by the values
$c_1=0.2$, $c_2=0.2$ and $\phi=0$.
\begin{figure}[h]
%\begin{subfigure}[t]{0.3\textwidth}
\centering
\includegraphics[width=0.33\textwidth]{ex4tu1.png}
%\subcaptionbox{\label{fig:ex4tu1}$c_1-0.2$ at $t=0$}
\includegraphics[width=0.32\textwidth]{ex4tu2.png}
\includegraphics[width=0.32\textwidth]{ex4tu3.png}\\
\includegraphics[width=0.33\textwidth]{ex4tu4.png}
%\subcaptionbox{\label{fig:ex4tu1}$c_1-0.2$ at $t=0$}
\includegraphics[width=0.32\textwidth]{ex4tu5.png}
\includegraphics[width=0.32\textwidth]{ex4tu6.png}\\
\includegraphics[width=0.33\textwidth]{ex4tu7.png}
%\subcaptionbox{\label{fig:ex4tu1}$c_1-0.2$ at $t=0$}
\includegraphics[width=0.32\textwidth]{ex4tu8.png}
\includegraphics[width=0.32\textwidth]{ex4tu9.png}\\
\includegraphics[width=0.33\textwidth]{ex4tu10.png}
%\subcaptionbox{\label{fig:ex4tu1}$c_1-0.2$ at $t=0$}
\includegraphics[width=0.32\textwidth]{ex4tu11.png}
\includegraphics[width=0.32\textwidth]{ex4tu12.png}\\
 \caption{Example $\ref{ex:problem 3}$, \small{\emph{Numerical solutions of $c_1-0.2$(left), $c_2-0.2$(middle) and $\phi$(right) at $t = 0,0.01,0.1$ and $t = 0.5$}}}
%\caption{}
%\label{fig:Case1}
%\end{subfigure}
\end{figure}

Figure 6.4 exhibits the dynamics of both energy decay, evident in the changes recorded on the right vertical axis, and the conservation of mass, as indicated by the left vertical axis. This comprehensive illustration provides clear evidence supporting the fundamental principles of mass conservation and energy dissipation within the system.
\begin{figure}[h]
    \centering
\includegraphics[width=0.33\textwidth]{ex4massandEnergy.png}
%\subcaptionbox{\label{fig:ex4tu1}$c_1-0.2$ at $t=0$}
\includegraphics[width=0.32\textwidth]{ex3tua3p1.png}
    \caption{Example $\ref{ex:problem 3}$, \small{\emph{Temporal evolution of mass and free energy of the numerical solution}}}
\end{figure}
\section{Concluding remarks}
In this work, we systematically introduce and conduct an in-depth analysis of a particular class of numerical schemes that are specifically designed to tackle the time-dependent Poisson-Nernst-Planck (PNP) system.  The fully discrete numerical scheme combines first/second-order semi-implicit time discretization with either the $k$-th order direct discontinuous Galerkin or finite element method for spatial discretization.

Through rigorous mathematical proofs and derivations, we have demonstrated that the two newly proposed numerical schemes possess essential physical characteristics. Notably, they both ensure the preservation of positivity, which is fundamental to maintaining the physical plausibility of the solutions. Additionally, they exhibit excellent energy stability, guaranteeing the reliability and consistency of the numerical simulations over extended periods.
Moreover, we have successfully established the optimal error estimates, which provide a precise quantification of the approximation accuracy of the numerical solutions. Simultaneously, we have uncovered superconvergence results, highlighting the enhanced computational performance of our proposed schemes. A comprehensive series of carefully designed numerical examples have been carried out to validate our theoretical claims.

Ongoing research topics include the construction and analysis of  the coupled system of the PNP equations and the Cahn-Hilliard equations.

%The mathematical theory of the SV is far from developed.
%More works are called for to find a better and more intrinsic understanding of the SV.
%More works including the theory for higher dimensional conservation laws and nonlinear problems are under way.
\section*{Acknowledgments}
The work is supported by National Natural Science Foundation of China
(No. 12271482,12271049, 12326346, 12326347,12201369), and Zhejiang Provincial Natural Science Foundation of
China (No. ZCLY24A0101).
% \newpage


 \begin{thebibliography}{00}


 \bibitem{Jerome1995}
    J.W. Jerome,
 \newblock{ Analysis of Charge Transport, Mathematical Theory and Approximation of Semi-conductor  Models,}
  \newblock{Springer-Verlag, New York, 1995. }


\bibitem{Markowich1986}
P.A. Markowich,
 \newblock{ The Stationary Seminconductor Device Equations,}
  \newblock{ Springer-Verlag, Vienna, Austria, 1986. }


 \bibitem{Markowich1990}

 P.A. Markowich, C.A. Ringhofer, and C. Schmeiser,
  \newblock{ Seminconductor Equations,}
   \newblock{ Springer Verlag, New York, 1990.}



 \bibitem{Nazarov2007}
   I. Nazarov and K. Promislow,
     The impact of membrane constraint on PEM fuel cell water management,
       J. Electrochem. Soc. 154(7): 623-630(2007).


   \bibitem{Promislow2001}
   K. Promislow and J.M. Stockie,
   Adiabatic relaxation of convective-diffusive gas transport in a porous fuel cell electrode,
   SIAM J. Appl. Math. 62(1): 180--205(2001).


   \bibitem{Ben}
 Y. Ben and H.C. Chang,  Nonlinear Smoluchowski slip velocity and micro-vortex generation, J. Fluid. Mech. 461: 229-238(2002).

   \bibitem{Hunter}
     R.J. Hunter, Foundations of Colloid Science, Oxford University Press, Oxford, UK, 2001.


 \bibitem{Lyklema}
 J. Lyklema, Fundamentals of Interface and Colloid Science, Volume II: Solid-liquid Interfaces, Academic Press Limited, San Diego, CA, 1995.


\bibitem{Bazant}
M.Z. Bazant, K. Thornton, and A. Ajdari, Diffffuse-charge dynamics in electrochemical systems, Phys. Rev. E. 70(2): 021506(2004).

\bibitem{Eisenberg}
B. Eisenberg, Y. Hyon, and C. Liu, Energy variational analysis of ions in water and channels: Field theory for primitive models of complex ionic fluids, J. Chem. Phys. 133(10): 104104(2010).

\bibitem{Eisenberg1}
R.S. Eisenberg, Computing the fifield in proteins and channels, J. Mem. Biol. 150:1-25(1996).

\bibitem{Gavish}
N. Gavish and A. Yochelis,
 Theory of phase separation and polarization for pure ionic liquids, J. Phys. Chem. Lett. 7:1121-1126(2016).


 \bibitem{p1}
 D. He, K. Pan, and X. Yue, A positivity preserving and free energy dissipative difference scheme for the Poisson-Nernst-Planck system,  J. Sci. Comput. 81: 436–458(2019).

 \bibitem{p2}
 Y.Qian, C.Wang, and S.Zhou, A positive and energy stable numerical scheme for the Poisson- Nernst-Planck-Cahn-Hilliard equations with steric interactions, J. Comput. Phys.426: 109908(2021).



 \bibitem{p4} J. Shen and J. Xu, Unconditionally positivity preserving and energy dissipative schemes for Poisson-Nernst-Planck equations, Numer. Math.148: 671-697(2021).

  \bibitem{p5}
F. Huang and J. Shen, Bound/Positivity preserving and energy stable scalar auxiliary variable schemes for dissipative systems: applications to Keller-Segel and Poisson-Nernst-Planck equations, SIAM J. Sci. Comput.43:A1832-A1857(2021).

 \bibitem{p6}
J. Hu and X. Huang, A fully discrete positivity-preserving and energy-dissipative finite difference scheme for Poisson-Nernst-Planck equations, Numer. Math. 145: 77-115(2020).

\bibitem{p7}
J. Ding, Z. Wang, and S. Zhou, Positivity preserving finite difference methods for Poisson Nernst-Planck equations with steric interactions: Application to slit-shaped nanopore conductance, J. Comput. Phys. 397:108864(2019).

\bibitem{p8}
J. Ding, Z. Wang, and S. Zhou, Structure-preserving and efficient numerical methods for iontransport, J. Comput. Phys.  418:109597(2020).

\bibitem{p11}
F. Siddiqua, Z. Wang, and S. Zhou, A modifified Poisson-Nernst-Planck model with excluded volume effect: Theory and numerical implementation, Commun. Math. Sci.16:251-271(2018).

\bibitem{p3}
C. Liu, C. Wang, S. M. Wise, X. Yue, and S. Zhou, A positivity-preserving, energy stable and convergent numerical scheme for the Poisson-Nernst-Planck system, Math. Comput. 90:2071-2106(2021).

\bibitem{e1}
J. Ding, C. Wang, and S. Zhou, Optimal rate convergence analysis of a second order numerical scheme for the Poisson-Nernst-Planck system, Numer. Math. Theor. Meth. Appl.12:607-626(2019).

\bibitem{e2}
H. Gao and D. He, Linearized conservative finite element methods for the Nernst-Planck-Poisson equations, J. Sci. Comput. 72:1269-1289(2017).

\bibitem{e3}
A. Prohl and M. Schmuck, Convergent discretizations for the Nernst-Planck-Poisson system, Numer. Math. 111:591–630(2009).

\bibitem{e4} Y. Sun, P. Sun, B. Zheng, and G. Lin, Error analysis of finite element method for Poisson-Nernst-Planck equations, J. Comput. Appl. Math. 301:28–43(2016).

\bibitem{energy1} A. Flavell, J. Kabre, and X. Li, An energy-preserving discretization for the Poisson-Nernst-
Planck equations, J. Comput. Electron. 16:431-441(2017).


\bibitem{p10}
H. Liu and Z. Wang, A free energy satisfying finite difference method for Poisson-Nernst-Planck equations, J. Comput. Phys. 268:363-376(2014).

\bibitem{energy2}
M. Metti, J. Xu, and C. Liu,  Energetically stable discretizations for charge transport and electrokinetic models, J. Comput. Phys. 306:1–18(2016).

\bibitem{o1}
D. He and K. Pan, An energy preserving finite difference scheme for the Poisson-Nernst-Planck system, Appl. Math. Comput. 287:214-223(2015).

\bibitem{o2}
M. Metti, J. Xu, and C. Liu, Energetically stable discretizations for charge transport and electrokinetic models,
J. Comput. Phys.306:1-18(2016).

\bibitem{o3}
M. Mirzadeh and F. Gibou, A conservative discretization of the Poisson-Nernst-Planck equations on adaptive cartesian grids, J. Comput. Phys. 274:633-653(2014).

\bibitem{o4}
Y. Qian, C. Wang, and S. Zhou, A positive and energy stable numerical scheme for the Poisson-Nernst-Planck-Cahn-Hilliard equations with steric interactions, J. Comput. Phys. 426:109908(2021).

\bibitem{o5}   B. Tu, M. Chen, Y. Xie, L. Zhang, B. Eisenberg, and B. Lu, A parallel finite element simulator for ion transport through three-dimensional ion channel systems, J. Comput. Chem. 287:214-223(2015).

\bibitem{o6}
S. Xu, M. Chen, S. Majd, X. Yue, and C. Liu, Modeling and simulating asymmetrical conductance changes in gramicidin pores, Mol. Based Math. Biol. 2:34-55(2014).

\bibitem{energy0}
D. He, K. Pan, and X. Yue, A positivity preserving and free energy dissipative difference scheme for the Poisson-Nernst-Planck system, J. Sci. Comput. 81:436-458(2019).


\bibitem{Liu-YuSISC}
H. Liu, H. Yu, Maximum-principle-satisfying third order discontinuous Galerkin schemes for Fokker-Planck equations, SIAM J. Sci. Comput. 36(5):A2296-A2325(2014).

\bibitem{Liu}
H. Liu, W. Maimaitiyiming,
Efficient, Positive, and Energy Stable Schemes for Multi-D Poisson-Nernst-Planck Systems, J. Sci. Comput. 87:92(2021).

\bibitem{Liu1}
H. Liu, Z. Wang, P. Yin,  and H. Yu, Positivity-preserving third order DG schemes for
Poisson-Nernst-Planck equations, J. Comput. Phys. 452:110777(2022).

\bibitem{Liu-2015}
 H. Liu, Optimal error estimates of the direct discontinuous Galerkin method for convection–diffusion equations,
Math. Comput. 84:2263-2295(2015).


\bibitem{chen-hu}
C. Chen and S. Hu, The highest order superconvergence for bi-k degree rectangular elements
at nodes: A proof of 2k-conjecture, Math. Comput. 82(2013).

\bibitem{Susanne}
Susanne C. Brenner and L. Ridgway Scott,
The mathematical theory of finite element methods, third edition, Springer, 2009.


\bibitem{super-cao-cd} W. Cao, H. Liu and Z. Zhang, Superconvergence of the direct discontinuous Galerkin method for convection-diffusion equations, Numer. Meth. Part. D. E.33(2017).





  \end{thebibliography}




\end{document}
