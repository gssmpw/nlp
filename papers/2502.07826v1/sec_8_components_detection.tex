\section{Detection of Power Line Components }\label{sec:components}
In recent years, advancements in computer vision and deep learning have revolutionized the field of power line inspection and maintenance. Power transmission lines, a critical part of our modern infrastructure, consist of various components such as insulators, conductors, fittings, and more. 
% {\color{blue}
The accurate and efficient detection of these components serves two crucial purposes in power line inspection: first, as a standalone task for inventory management and infrastructure mapping, and second, as a prerequisite step for fault diagnosis, where detected components are further analyzed for potential defects. This section focuses specifically on component detection methodologies, covering research works that either concentrate solely on locating and identifying power line components, or present novel component detection techniques that later serve as foundations for fault diagnosis systems. Section \ref{sec:fault} on fault diagnosis will then examine approaches for detecting various types of faults, both in scenarios where component detection is a preliminary step and in methods that analyze faults directly from full images.
% }

\subsection{Insulator Detection}
Among the components of transmission lines, the insulator detection technologies are the most well-researched in the literature. These technologies offer efficient and accurate solutions to identify insulators among the other components. In a 2019 work, Miao et al. \cite{miao_insulator_2019} proposed an efficient insulator detection method for aerial images, addressing the challenge of cluttered backgrounds. The approach utilizes the SSD network and employs a two-stage fine-tuning strategy to improve accuracy (Figure \ref{fig:miao_ssd}). In the first stage, a basic insulator model is fine-tuned with a diverse dataset, while the second stage fine-tunes the model for specific insulator types and scenarios. This approach demonstrates that pretraining on a generalized insulator dataset can improve the model's performance when using a smaller and more specific type of insulator dataset.

\begin{figure*}[htb]
    \centering
    \includegraphics[width=1\linewidth]{fig_3_miao_ssd.pdf}
    \caption{Overall process of the two stage insulator detection using the SSD network \cite{miao_insulator_2019}.}
    \label{fig:miao_ssd}
\end{figure*}

The YOLO and Faster R-CNN algorithms for object detection have been widely used for insulator detection tasks. Sadykova et al. \cite{sadykova2019yolo} proposed a cost-effective solution for automatically detecting high voltage insulators from aerial images, particularly in scenarios with uncluttered backgrounds, varying object resolutions, and lighting conditions. The approach utilizes the YOLO network and includes data augmentation to prevent overfitting, leveraging a training dataset of 56,000 image samples after augmentation. Experimental results demonstrate the accuracy of this method in locating insulators in real-time UAV-based image data. Kang et al. \cite{kang_deep_2019} proposed a two-stage deep learning system for the detection of defects on the surface of high-speed railway insulators. In the first stage, a Faster R-CNN was employed to locate the insulators within images captured by the inspection vehicle. The second stage involved a novel Deep Multitask Neural Network (DMNN) that combined a Deep Material Classifier (DMC) for segmenting the insulator from the background and a Deep Denoising Autoencoder (DDAE) for detecting anomalies (defects) on the insulator surface. The DMNN was trained in a multitask learning framework, allowing the DMC and DDAE to benefit from shared convolutional layers. The authors noted that the system might struggle to detect defects that are small or have a gray value similar to the normal insulator surface.

Any kind of occlusion such as fog, smoke, etc. can be challenging for computer-vision-based insulator detection. A 2022 work by Zhang et al. \cite{zhang_finet_2022} introduced a dataset designed for insulators and presented a benchmark model called Foggy Insulator Network (FINet), which builds upon the improved YOLOv5 framework. To enhance the dataset, the research develops and optimizes a synthetic fog algorithm, resulting in the creation of a Synthetic Foggy Insulator Dataset (SFID) containing 13,000 images. Furthermore, the YOLOv5 network is enhanced with a channel attention mechanism to form SE-YOLOv5. However, the synthetic fog generation algorithm, while effective, might not fully capture the complexity and diversity of real-world fog conditions. The evaluation of the FINet model was primarily focused on the SFID dataset, which might limit the generalizability of the results to other datasets or real-world scenarios. Table \ref{tab:insulator_detection} summarizes the relevant literature on insulator detection.


\begin{table*}[htb]
% {\color{blue}
\scriptsize
\caption{Summary of insulator detection studies in power line inspection.} \label{tab:insulator_detection} 
\begin{tabular}{P{.05} P{.1} P{.15} P{.1} P{.15} P{.1} P{.15}}

\hline
Year \& Ref & Component & Type of Detection & Imaging Platform & Dataset & Algorithm & Performance \\
\hline
2019 \cite{miao_insulator_2019} & Porcelain Insulator, Composite Insulator & Bounding Box Detection & UAV & 7605 RGB Images & SSD & Porcelain Insulator: 90.51-94.12\% Composite Insulator: 86.70-87.29\% \\

2019 \cite{sadykova2019yolo} & Glass Insulator & Bounding Box Detection & UAV & 3500 RGB Images  & YOLOv2 & Detection Accuracy: 88\% Prediction Time: 0.04s \\

2019 \cite{kang_deep_2019} & Insulator & Bounding Box Detection & High-Speed Railway & 18000 RGB Images  & F-RCNN & mAP: 99.8\% \\

2020 \cite{zhao2020image} & Insulator & Object Presence Detection & NA & 4780 Insulator and 13012 Background IR Images  & DCNN \& VLAD Coding & Detection Accuracy: 99.21\% \\

2021 \cite{zhang_defgan_2021} & Insulator & Semantic Segmentation & High-Speed Railway & 800 RGB Images  & CDSNets & IOU: 0.94 \\

2021 \cite{waleed_drone_based_2021} & Ceramic Insulator & Bounding Box Detection & UAV & 2973 RGB Images & R-CNN, SSD & Onshore mAP: 0.56-0.77; Onboard mAP: 0.24-0.27 \\

2022 \cite{wei_online_2022} & Insulator & Bounding Box Detection & NA & 8500 Insulator Images & SSD, F-RCNN  & Accuracy: SSD: 89\%, F-RCNN: 91.6\% \\

2022 \cite{zhang_finet_2022} & Insulator & Bounding Box Detection & Synthetic & 13000 Foggy RGB Images & Improved YOLOv5 & mAP@0.5:0.95: 88.3\%, F1 Score: 96.2\%  \\

2022 \cite{antwi_bekoe_deep_2022} & Insulator & Instance Segmentation & UAV & 1523 RGB Images & Mask R-CNN & mAP: 87.0\% \\

2023 \cite{zhou_insulator_2023} & Insulator & Semantic Segmentation & Satellite & 9900 RGB Images & HRNet and OHEM & F1 Score: 0.7952 \\

2023 \cite{shuang_rsin_dataset_2023} & Insulator & Bounding Box Detection & UAV & 1887 RGB Images & YOLOv4++ & mAP: 94.24\% \\

2023 \cite{singh_2023_interpretable} & Insulator & Bounding Box Detection & UAV \& Synthetic & CPLID Dataset: 848 RGB Images & YOLOv8 & mAP@[0.5:0.95]: 91.25\%   \\
\hline
\end{tabular}
% }
\end{table*}

\subsection{Detection of Power Line Fittings }
Detecting power line fittings, including components like pin bolts, dampers, and suspension clamps, presents a unique set of challenges. These fittings are relatively small in size compared to the overall transmission line structure, making their detection and classification a particularly intricate task \cite{luo_ultrasmall_2023}. This section explores the efforts and methodologies aimed at the automatic detection of these critical power line fittings, focusing on the utilization of deep learning and computer vision techniques to address the intricacies associated with their small-scale identification. Table \ref{tab:fittings_detection} summarizes the literature on power line fittings detection.

As mentioned earlier, one of the biggest challenges to overcome in power line fittings like bolts detection is their relatively small size compared to the other components. To tackle this, Luo et al. \cite{luo_ultrasmall_2023} propose a novel model (UBDDM) based on deep CNN. This model includes two modules: the ultrasmall object perception module (UOPM) for initial bolt region identification and the local bolt detection module  (LBDM) for pinpointing defects within high-resolution image blocks. The authors enhance feature extraction using ResNet-50, a hybrid attention mechanism, and multiscale feature fusion. Their method simplifies data labeling requirements while maintaining end-to-end detection capabilities. While experiments demonstrate the model's superior performance in detecting bolt defects within inspection images, the model can only draw bounding boxes of some predefined dimensions.

Transmission line fittings encapsulate a wide range of components and often require a multi-task object detection model to detect all the different types. In 2022 a study done by Zhai et al. \cite{zhai_multi_fitting_2022} proposed the Cascade Reasoning Graph Network (CRGN) which offers a novel solution to the intricate challenges posed by detecting multiple power line fittings in aerial images. CRGN uses spatial knowledge representations that capture the interrelationships among objects based on the unique characteristics of transmission line fittings. However, the model's dependency on high-resolution, close-up photographs of the fittings presents a practical challenge, as such image quality may be difficult to consistently achieve in real-world scenarios.

\begin{table*}[htb]
% {\color{blue}
\scriptsize
\caption{Summary of power line fittings detection studies.} \label{tab:fittings_detection}
\begin{tabular}{P{.05} P{.1} P{.15} P{.1} P{.1} P{.15} P{.15}}
\hline
Year \& Ref & Component & Type of Detection & Imaging Platform & Dataset & Algorithm & Performance \\
\hline
2021 \cite{zhai_hybrid_2021} & Fittings & Bounding Box Detection & UAV & 1455 RGB Images & HK R-CNN & mAP: 59.82\% - 98.27\%  \\

2022 \cite{zhang_attention_guided_2022} & Dampers and Suspension Clamps & Bounding Box Detection & UAV & 1209 RGB Images & AGMNet & mAP: 95.3\% \\

2022 \cite{zhai_multi_fitting_2022} & Fittings & Bounding Box Detection & Aerial Vehicle & 1455 Aerial Images & CRGN & mAP@0.5-0.95: 47.5\% \\

2023 \cite{huang_structural_2023} & Damper & Semantic Segmentation & UAV & 240 RGB Images & Improved GrabCut & F1 Score: 89.1-97.3\% \\
\

2023 \cite{luo_ultrasmall_2023} & Bolts & Bounding Box Detection & UAV & 1852 RGB Images & UPOM (Based on ResNet and Attention) & Recall 0.94 - 1.00 \\
\hline
\end{tabular}
% }
\end{table*}

\subsection{Conductor Detection}
Conductor detection using computer vision must contend with multiple complexities, including the conductor's slender profile, varying backgrounds, and potential occlusions. This subsection explores the state-of-the-art techniques and advancements in conductor detection using computer vision and deep learning, addressing the unique challenges associated with this critical task. Table \ref{tab:conductor_detection} summarizes the literature on conductor detection.

Power lines are often a small portion of the image, leading to class imbalance issues. To address these challenges, Yang et al. \cite{yang_vision_based_2022} proposed a novel vision-based power line detection network designed to address challenges in detecting power transmission lines in complex aerial images including varying background environments, illumination conditions, and foreground-background class imbalance, where power lines occupy a small portion of the image. The proposed network utilizes an encoder-decoder architecture (similar to UNet) to create an end-to-end power line detection system. To improve segmentation precision, it incorporates an attention block to capture global contexts and emphasize target power line regions. Additionally, an attention fusion block is introduced to enhance multi-scale feature fusion and capture richer information, mitigating issues related to local contextual feature processing and information loss caused by multiple pooling operations. However, due to the complexity of the UNet-based network combined with the attention mechanism, the proposed model is not particularly suitable for real-time applications.

While deep learning, especially U-Net and its variants, has advanced pixel-level object segmentation, limitations in processing local contextual features and information loss in deep CNNs persist. To overcome these issues, in a follow-up study, Yang et al. \cite{yang_dra_net_2023} proposed a novel dual-branch residual attention network called DRA-Net. It features a dual-branch encoder with a residual CNN (RCNN) branch and a recurrent RCNN (RRCNN) branch to capture richer semantic information. Additionally, a U-shaped noise denoising (UND) block reduces background interference, and an edge enhancement block (EEB) strengthens the network's capacity to extract useful edge features. Experimental results demonstrate DRA-Net's excellent segmentation performance, achieving a Dice coefficient of 93.26\% and a mean Intersection over Union (mIoU) of 93.19\% on the public power line dataset (PLD) and 96.40\% Dice coefficient and 96.04\% mIoU on the self-built overhead power line (OPL) dataset. However, the reported results were obtained from a relatively small training dataset of 1144 images and could be reduced when tested on a wide variety of real-world images.

\begin{table*}[htb]
% {\color{blue}
\scriptsize
\caption{Summary of power line conductor detection studies.} \label{tab:conductor_detection}
\begin{tabular}{P{.05} P{.1} P{.15} P{.1} P{.15} P{.1} P{.15}}
\hline
Year \& Ref & Component & Type of Detection & Imaging Platform & Dataset & Algorithm & Performance \\
\hline
2022 \cite{yang_vision_based_2022} & Conductor & Semantic Segmentation & UAV & 366 RGB Images  & UNet with Attention Blocks & Dice: 0.957 \\

2023 \cite{yang_dra_net_2023} & Conductor & Semantic Segmentation & UAV & PLD Dataset: 573 RGB Images; OPL Dataset: 571 RGB Images & DRA-Net  & mIOU: 93.19\% (PLD Dataset), mIOU: 96.04\% (OPL Dataset) \\
\hline
\end{tabular}
% }
\end{table*}

\subsection{Multi-Component Detection}
A typical transmission tower consists of multiple different types of components with varying shapes and dimensions. Some studies have extended focus beyond individual component detection to multi-component detection \cite{zhang_multi_scale_2020, wang_image_2019, nguyen_intelligent_2019}. In a 2024 study, Dong et al. \cite{dong2024transmission} proposed a meta learning-based model to address the challenge of detecting key components and defects in transmission lines using aerial images, particularly when dealing with limited sample sizes for certain categories. The model incorporated a region-aware fusion (RAF) module to capture spatial relationships between support and query images, enabling effective matching and identification of objects. Additionally, cascade RAF heads were employed to progressively refine bounding box proposals and improve detection accuracy. The model was trained using a two-stage fine-tuning strategy, leveraging a larger dataset of base classes to enhance the detection of novel classes with fewer samples. Experimental results demonstrated the model's superior performance compared to traditional deep learning and few-shot object detection models. The authors acknowledged potential limitations in their work. The model's performance might be affected by variations in image quality and environmental conditions encountered in real-world power transmission line inspections. The reliance on a pre-defined set of base classes and novel classes might limit the model's adaptability to new or unexpected object categories. The authors suggested exploring the incorporation of online learning or active learning techniques to enable the model to continuously learn and adapt to new scenarios.

Self-supervised pretraining has been used effectively in other domains to tackle the challenge of minimal or non-existent annotated dataset. Liu et al. \cite{liu2023tower} introduced Tower Masking MIM (TM-MIM), a self-supervised pretraining method designed to enhance power line component detection in aerial images, particularly in scenarios with limited labeled data. By employing a novel masking strategy that focuses on the tower-conductor region and a Siamese architecture with dual reconstruction branches, the model learns to capture discriminative features and global representations from unlabeled data. The incorporation of knowledge distillation further enhances the model's ability to retain general knowledge from pretrained models while acquiring domain-specific knowledge. The authors identified potential areas for future work. The masking strategy's reliance on tower presence in images might necessitate the development of techniques to handle images without towers. Additionally, exploring the use of varied-grained masks for different feature hierarchies could further enhance the model's ability to detect components of varying scales. The authors also suggested extending the TM-MIM approach to other object detection frameworks, such as YOLO, to broaden its applicability. Table \ref{tab:multicomponent_detection} summarizes the literature on power line multi-component detection.

\begin{table*}[htb]
% {\color{blue}
\scriptsize
\caption{Summary of multi-component detection studies in power line inspection.} \label{tab:multicomponent_detection} 
\begin{tabular}{P{.05} P{.1} P{.15} P{.1} P{.15} P{.1} P{.15}}

\hline
Year \& Ref & Component & Type of Detection & Imaging Platform & Dataset & Algorithm & Performance \\
\hline
2019 \cite{chen_research_2019} & Insulator and Damper & Bounding Box Detection & UAV & 4416 Insulator \& 4352 Damper Images & YOLOv3 & Accuracy: 95.84\% \\

2019 \cite{wang_image_2019} & Tower Components & Bounding Box Detection & NA & 11600 RGB Images & F-RCNN & - \\

2019 \cite{nguyen_intelligent_2019} & Tower Components & Bounding Box Detection & Aerial Vehicle & 123151 RGB Images & SSD & mAP: 0.67 \\

2020 \cite{zhang_multi_scale_2020} & Tower Components & Bounding Box Detection & UAV & City A: 2016 RGB Images; City B: 3960 RGB Images & Enhanced F-RCNN & City A Dataset: mAP: 52.9\%, City B Dataset: mAP: 45.3\% \\

2021 \cite{odo_aerial_2021} & Insulator and Bolts & Bounding Box Detection & Aerial Vehicle & 1830 RGB Images  & Mask R-CNN and RetinaNet & Precision: 96.7\% (Insulators), 97.9\% (Bolts) \\

2023 \cite{liu2023tower} & Tower Components & Bounding Box Detection & UAV & PLCD Dataset: 1000 Images & TM-MIM Network & mAP@0.5: 87.7\% \\

2023 \cite{shi2024lskf} & Towers & Bounding Box Detection & Satellite & Duke University Dataset: 2740 Images & Improved YOLO-based Network & mAP@0.5: 77.47\% \\

2024 \cite{dong2024transmission} & Tower Components & Bounding Box Detection & UAV & 9017 RGB Images & Meta Learning & mAP@0.5: 64.6\% \\
\hline

\end{tabular}
% }
\end{table*}

% {\color{blue}
This section has demonstrated the remarkable progress in automating power line component detection through computer vision and deep learning techniques. The reviewed literature showcases significant achievements across various components: from insulator detection achieving mAP rates of up to 91.25\% using YOLOv8, to conductor segmentation reaching Dice coefficients of 96.40\% with innovative architectures like DRA-Net, and multi-component detection systems attaining mAP rates of 87.7\% through self-supervised learning approaches. These advances have been enabled by architectural innovations such as attention mechanisms, multi-scale feature fusion, and hybrid networks combining CNN backbones with specialized modules. However, several domain-specific challenges persist. The detection of small-scale components like pin-bolts and dampers, which often occupy minimal pixels in aerial imagery, remains particularly challenging, with performance metrics for fitting detection typically lower than those for larger components. Additionally, real-world deployment faces obstacles such as varying imaging conditions, complex backgrounds, and class imbalance issues where critical components occupy only a small portion of the image. The limited size and diversity of publicly available datasets—with most studies utilizing fewer than 5,000 images—continues to constrain model development and benchmarking. Recent promising directions include meta-learning approaches for few-shot detection, self-supervised pretraining methods like TM-MIM for leveraging unlabeled data, and specialized architectures incorporating domain knowledge about component spatial relationships. These challenges and emerging solutions in power line component detection will be explored more comprehensively in Section \ref{sec:challenges}, along with their implications for future research and practical applications.
% }