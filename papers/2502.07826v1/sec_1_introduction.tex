\section{Introduction}\label{sec:introduction}
The late 19th century saw Edison's light bulb and Tesla and Westinghouse's alternating current (AC) systems usher in the electrical age.  This era not only illuminated the world but also laid the foundation for the modern power delivery system, which has become a complex network of power plants, transmission lines, and distribution networks. For many decades, this complex network relied on manual inspections, which were dangerous and limited. Technological advances improved the methods, but the recent rise of computer vision and deep learning has revolutionized how power line inspections are conducted. 

A power line comprises a multitude of components, each with distinct functions, including insulators, towers, conductors, and fittings. Operating in a challenging outdoor environment, exposed to complex landforms and unpredictable weather, power line components are susceptible to frequent damage. A single faulty component, such as a conductor, or a combination of multiple damaged components, such as fittings, can trigger power outages with far-reaching consequences. These disruptions not only disrupt regional electricity supply but can also escalate to supra-regional blackouts and even catastrophic incidents, such as forest fires \cite{mitchell_power_2013}. In California, about \(10\%\) of the state's wildfires are believed to be triggered by power lines. The severity of these fires led the California Public Utilities Commission to investigate Pacific Gas \& Electric (PG\&E) power line safety practices, considering drastic measures such as breaking up the utility into smaller entities for better management and accountability \cite{noauthor_link_nodate}. The resulting economic and societal costs due to power line failures can be substantial \cite{salim_modeling_2018}. Effective power line inspection serves as the vanguard against such calamities. Its primary objective is to assess the condition of the power line components, enabling informed decisions on maintenance and replacement. A swift and accurate inspection process significantly enhances the efficiency of maintenance decision-making and, in turn, reduces the likelihood of power line failures, safeguarding the safety and reliability of the power supply to the connected load \cite{nguyen_automatic_2018}.

Nonetheless, power line inspection encounters a series of challenges, ranging from covering vast geographic areas to dealing with a diverse array of components and navigating complex natural environments. For decades, traditional inspection methods have relied on manual ground surveys and helicopter-assisted patrols \cite{liu_two_layer_2019}. These methods heavily depend on visual observations from humans, which involve significant costs, inherent risks, low operational efficiency, and long timeframes \cite{matikainen_remote_2016}. In recent years, the application of computer vision and deep learning technologies has helped in a transformative era for power line inspection \cite{yang_review_2020}. These advanced techniques have effectively decoupled the traditional inspection process into two distinct phases: data collection and data analysis. Operators can now leverage computer vision and deep learning to automatically process images and videos captured by Unmanned Aerial Vehicles (UAVs) or other means. This transition from manual labor-intensive methods to automated inspection is driven by compelling advantages, including cost-efficiency, enhanced safety, and superior operational efficiency \cite{yang_review_2020}.

Today, the maintenance of power lines is being transformed by the integration of computer vision and deep learning algorithms. These technologies enable the automation of inspection processes, offering a safer, more efficient, and cost-effective method of identifying potential issues before they escalate into major failures. By leveraging high-resolution images and real-time data analysis, utility companies can now predict maintenance needs, prevent outages, and ensure the reliable delivery of electricity to consumers. However, this transition has introduced a deluge of data. Furthermore, the conventional approach for analyzing these images and videos involves time-consuming manual efforts, which are not only costly but also fraught with potential safety hazards and may lack the necessary precision \cite{martinez_power_2018}. Consequently, there is an urgent demand for the development of automated methodologies to replace manual analysis \cite{liu_data_2020}. Over the past years, several attempts have diligently striven to devise rapid and accurate methods for the automatic evaluation of power lines from aerial imagery or land \cite{nguyen_automatic_2018}. Those attempts tried an extensive spectrum of power line components and their associated faults, primarily leveraging image processing and computer vision. Although image processing-based approaches like color \cite{reddy_condition_2013}, shape \cite{zhao_localization_2015}, or texture segmentation \cite{wu_texture_2012} have seen some success over the years, they are gradually being replaced by more advanced deep learning-based approaches. Moreover, a substantial portion of these endeavors are task-specific, concentrating on isolated components or specific fault types. 

This review paper explores the recent works on vision-based power line inspection, casting the gaze through the lens of deep learning. The paper starts with an introduction to the foundational concepts in power line inspection, encompassing various inspection methods and data sources. Subsequently, this paper provides a brief introduction to the deep learning-based techniques applied to power line inspection. Moving forward, an extensive review of research endeavors focused on the analysis of images in power line inspection has been explored. 
% {\color{blue}
The literature has been organized into two categories: component detection and fault diagnosis. Component detection research focuses on locating and identifying power line elements, either as a standalone task or as a crucial first step in fault analysis. Fault diagnosis studies, on the other hand, encompass both direct fault detection approaches and methods that build upon component detection to identify specific defects. This systematic division allows us to examine the unique challenges and solutions in each domain while highlighting their interconnected nature.
% }
The paper unveils the key features of each analytical approach, explores the nature of the datasets employed, and showcases representative quality analysis results, offering insight into the diverse capabilities of these methods across various applications. Finally, the paper presents a set of open research questions and uncharted territories awaiting future exploration. These questions encompass challenges related to data quality, the intricacies of small object detection, the application of deep learning in embedded systems, and the definition of robust evaluation baselines. Finally, this paper synthesizes the key findings and insights derived from this comprehensive analysis.

The structure of the rest of this paper is outlined as follows: Section \ref{sec:related_works} explores similar works and compares this review with the existing ones. Section \ref{sec:automated} introduces the key stages of automated power line inspection. Section \ref{sec:image_acquisition} explores various image acquisition methods and the vehicles employed for collecting power line data. In Section \ref{sec:imaging}, we examine a range of imaging techniques, discussing their advantages and limitations. Section \ref{sec:datasets} presents a concise overview of available computer vision datasets pertinent to power line inspection. Section \ref{sec:architectures} highlights the leading deep learning models and architectures employed in computer vision for tasks such as object detection and classification. Following this, Section \ref{sec:components} offers an in-depth review of research papers focused on power line component detection. Section \ref{sec:fault} scrutinizes literature dedicated to identifying power line faults through computer vision techniques. Section \ref{sec:discussion} provides a qualitative assessment of the reviewed articles and Section \ref{sec:challenges} addresses the prevalent challenges within this field and proposes potential areas for future research. The paper concludes with Section \ref{sec:conclusion}, summarizing the findings of this comprehensive review. 