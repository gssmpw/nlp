\section{Current Challenges and Future Directions}\label{sec:challenges}
In the rapidly evolving field of automated power line inspection, significant advancements have been made. However, several critical challenges still persist that need to be addressed to further advance the state of the art. This section outlines these challenges and proposes potential future research directions.

% {\color{blue}
\subsection{Edge-Cloud Deployment Challenges}
The deployment of deep learning models for power line inspection faces significant architectural challenges in both cloud-centric and edge-based approaches. Traditional cloud computing, while offering substantial computational resources, suffers from high latency, excessive bandwidth consumption, and significant communication costs when processing the massive amount of visual data generated by inspection devices \cite{wang2020convergence, wei_online_2022}. Edge computing attempts to address these limitations by bringing computation closer to data sources, but edge devices typically lack the computational capacity to run sophisticated deep learning models effectively \cite{shi2016edge}. This limitation is particularly critical in power line inspection, where models must detect subtle defects and anomalies with high accuracy, creating a fundamental tension between model complexity and computational efficiency.

Edge-cloud fusion architectures present a promising direction to address these challenges by combining the strengths of both approaches. This fusion has led to the emergence of effective two-stage detection systems, where lightweight models at the edge perform initial coarse screening, followed by refined secondary recognition using more sophisticated models in the cloud \cite{wei_online_2022}. In this hierarchical framework, edge devices can efficiently filter and pre-process data, while cloud resources handle detailed analysis, significantly reducing data transmission while maintaining high accuracy \cite{wei_online_2022, zhang_edge_2023, zhang_cloud_edge_2020}. Future research should focus on several key areas to advance this fusion approach: developing more efficient model compression techniques for edge deployment \cite{howard2017mobilenets}, improving communication protocols for edge-cloud interaction \cite{shi2016edge}, creating adaptive frameworks for dynamic resource allocation \cite{wang2020convergence}, and investigating federated learning \cite{mcmahan2017communication} approaches for collaborative model training. These advancements could enable more efficient and reliable power line inspection systems while maintaining the accuracy needed for critical infrastructure monitoring.

\subsection{Multimodal Imaging and Fusion}
Power line inspection research has predominantly relied on optical imaging, with our comprehensive review revealing that only around 8\% of published works utilize other imaging modalities. This heavy dependence on optical imaging persists despite the known limitations of visible spectrum cameras in various environmental conditions and their inability to detect certain types of faults. Limited studies exploring alternative modalities demonstrated the complementary capabilities of infrared and ultraviolet imaging for detecting corona discharge and heating associated with leakage current flow in composite insulators \cite{singh_design_2021, liu_discrimination_2017, li_image_2019}. Their research highlighted how different imaging modalities can provide unique insights - with IR imaging excelling at detecting heat distribution patterns from current leakage, while UV imaging proved effective for visualizing corona discharge phenomena. However, such multimodal approaches remain vastly underutilized in power line inspection literature, despite their proven effectiveness in other domains such as medical imaging, remote sensing, and defense applications \cite{karim2023current, meher2019survey}.

The future of power line inspection could benefit significantly from greater adoption of multimodal imaging approaches. Recent developments in image fusion, as outlined in the comprehensive review by Karim et al. \cite{karim2023current}, demonstrate how combining multiple imaging modalities can provide more comprehensive information about the real world than any single modality alone. While optical images excel at providing high spatial resolution and clear texture details, other modalities like infrared can detect thermal anomalies, and ultraviolet can reveal corona discharge patterns \cite{shen2017research}. By integrating multiple modalities through advanced fusion techniques - ranging from conventional transform-based methods to emerging deep learning architectures \cite{ma2019fusiongan, han2020electrical} - future inspection systems could achieve more robust fault detection capabilities. 
% }

\subsection{Lack of Data Availability}
The scarcity of publicly available datasets remains a significant challenge in deploying deep learning for power line inspection \cite{song2020analysis}. Power line components and scenarios require vast, varied datasets for effective training. While researchers often create custom datasets \cite{chen_research_2019}, data protection regulations frequently prevent public sharing, as discussed in Section \ref{sec:datasets}.

Several approaches show promise in addressing this challenge. Synthetic data generation using GANs \cite{goodfellow2020generative} or Denoising Diffusion Models can create diverse, realistic power line images. For example, a recent study \cite{YE2024100250} enhanced Cycle-GAN \cite{zhu2017unpaired} with attention mechanisms to generate insulator defect images, demonstrating significant improvements in sample quality.

% {\color{blue}
Self-supervised pretraining \cite{raina2007self} offers another solution by leveraging unlabeled data to learn useful representations before fine-tuning on smaller labeled datasets. Additionally, few-shot learning approaches \cite{wang2020generalizing}, particularly meta-learning techniques \cite{finn2017model}, enable models to learn from limited examples. Combining these methods with transfer learning could effectively address the data scarcity challenge while maintaining robust performance in real-world applications.
% }

\subsection{Data Annotation}

The problem of data annotation presents a significant challenge in power line inspection, particularly given the complexity of power line components and the subtleties of potential faults. While many excellent annotation tools exist \cite{dwyer2024roboflow, HumanSignal, labelstudio}, the process remains time-consuming and labor-intensive, requiring substantial human intervention for verification. 
% {\color{blue}
The emergence of large foundation models like SAM (Segment Anything Model) \cite{kirillov2023segment} offers new opportunities for streamlining this process through zero-shot segmentation capabilities, though these models require significant computational resources and careful adaptation to the power line domain. The integration of such models into existing annotation workflows necessitates consideration of domain-specific fine-tuning strategies to ensure reliable performance in power line inspection contexts.
% }

Several promising approaches are being developed to address these challenges. Weakly supervised learning techniques \cite{zhou2018brief} have shown potential in reducing labeling requirements, as demonstrated by Choi et al. \cite{choi2021weakly} in their two-stage power line detection algorithm. The combination of these approaches with foundation models and specialized annotation tools could create more effective hybrid systems that leverage both general semantic understanding and domain-specific expertise. Additionally, self-supervised pretraining \cite{raina2007self} can significantly reduce the dependency on labeled datasets by equipping models with a deep understanding of structural and contextual features inherent in power line images, thereby streamlining the annotation process while maintaining accuracy and reliability.

\subsection{Very Small Components Detection}
Detecting small components like fittings, bolts, and fractures in power lines presents unique challenges due to their low resolution in images. These components, while critical for structural integrity, are often indistinguishable from complex backgrounds \cite{xiao_detection_2021}. The prevalent use of UAVs introduces additional challenges: image blur from drone motion and inability to capture close-up images due to high voltage risks. Consequently, small components often occupy only a few pixels in the captured images \cite{zhai_hybrid_2021}.

Recent advances offer promising solutions. Transformer-based approaches \cite{dong_improved_2023} have shown success in reducing errors in small and occluded object detection through lightweight self-attention modules. Image super-resolution techniques \cite{ledig2017photo} enhance resolution by inferring missing details and refining textures. Future research could focus on combining high-resolution imaging technology with advanced deep learning models, while improving multi-scale detection strategies and feature extraction methods for small objects \cite{hu2018small}.

% {\color{blue}
\subsection{Anomaly Detection for Unknown Defects}
Traditional supervised approaches often fail to identify novel power line defects absent from training data. Recent semi-supervised and unsupervised learning methods \cite{defard2021padim, batzner2024efficientad} offer solutions by learning normal patterns and detecting deviations.

Autoencoder-based methods with auxiliary anomaly localization enable end-to-end defect detection using only normal samples \cite{tsai2021autoencoder, sun2023semisupervised}. Feature embedding approaches using pre-trained networks reduce noise interference during reconstruction \cite{roth2022towards}. Memory-based methods that compare normal feature representations have improved detection of subtle anomalies \cite{yang2023memseg}, while combined reconstruction and discriminative training approaches enhance defect localization accuracy \cite{zavrtanik2021draem}. Future research should focus on developing efficient architectures suitable for edge deployment while maintaining detection accuracy.
% }