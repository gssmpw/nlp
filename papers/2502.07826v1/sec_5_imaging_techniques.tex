\section{Imaging Techniques}\label{sec:imaging}
The choice of imaging technique is critical for obtaining the most relevant and accurate data. Various imaging modalities are employed, each offering unique advantages and limitations. In this section, four primary image acquisition techniques: visible light images, infrared images, UV images, and x-ray images have been explored. Furthermore, LiDAR-based geospatial mapping has also been discussed briefly.

\subsection{Visible Light Imaging}
Visible light imaging is the most widely used method and is readily available through standard cameras. It is a cost-effective choice for routine inspections. Visible light images capture fine details of power line components and surrounding vegetation and other objects, which is crucial for identifying defects and assessing structural integrity. However, visibility can be compromised in adverse weather conditions, such as fog, rain, or darkness \cite{zhang_study_2023}, \cite{zhang_finet_2022}. While visible light images provide valuable surface-level data, they may not detect defects hidden within materials or components.

\subsection{Infrared Imaging (Thermography)}
Infrared imaging, or thermography, detects temperature variations in power line components, highlighting issues like overheating, loose connections, and faulty insulators \cite{singh_design_2021}. Faulty electrical components often result from internal electrical defects which can lead to over-current flow and heating of the component which can be detected from its surface temperature. Infrared imaging is not reliant on visible light and can operate effectively day and night. However, Infrared imaging requires specialized cameras that can be costly and may require trained personnel. It primarily provides information about surface temperature, and its ability to penetrate materials is limited \cite{jaffery_design_2014}.

\subsection{Ultraviolet (UV) Imaging}
UV imaging is particularly useful for detecting corona discharges - electrical discharges caused by the ionization of air surrounding high-voltage conductors - which can indicate electrical faults \cite{hu_new_2012}. UV imaging is non-destructive and can reveal hidden faults without physical contact with the power lines \cite{li_image_2019}. However, UV imaging has a limited range compared to visible light or infrared imaging, which means it may not capture an entire transmission line in a single image \cite{zang_status_2008}. Also, specialized UV cameras are required for this technique.

\subsection{X-Ray Imaging}
X-ray imaging can penetrate materials, providing detailed images of inner structures and components. It is effective at identifying hidden defects, such as corrosion and internal damage \cite{wang_internal_2023}. The use of X-ray imaging involves exposure to ionizing radiation, which requires adherence to safety protocols and limits its routine use. Also, X-ray imaging equipment is costly and requires skilled operators to acquire.

\subsection{LiDAR Imaging}
LiDAR (Light Detection and Ranging) imaging is a cutting-edge technique increasingly used in power line inspection. It employs laser light to create high-resolution, three-dimensional representations of power line infrastructure and surrounding environments. LiDAR sensors emit pulses of laser light and measure the time taken for each pulse to bounce back after hitting an object. This data is then used to construct detailed 3D models of the power lines and their immediate surroundings. LiDAR provides precise, three-dimensional information, enabling accurate mapping of power line components and detection of even minor structural anomalies. Unlike visible light imaging, LiDAR can penetrate through mild fog, rain, and other atmospheric conditions, offering more consistent results in diverse weather scenarios. It is particularly effective in assessing vegetation encroachment and potential physical obstructions near power lines, which are crucial for maintaining line safety and preventing outages \cite{guan2021uav, bergmann2024approach}.

The choice of image acquisition technique in power line inspection depends on specific inspection goals, budget, and the expected challenges. Often, a combination of imaging modalities may be used to maximize the coverage of inspection. While visible light images provide a foundational dataset for routine inspections, advanced techniques like infrared, UV, and X-ray imaging offer deeper insights into the condition of power lines. When combined with computer vision algorithms, these techniques can significantly enhance the accuracy and efficiency of fault detection and maintenance decisions \cite{li_image_2019}. Table \ref{tab:imaging_techniques} shows a comparison between these techniques in terms of cost, lighting condition, and coverage.

\begin{table*}[htb]
\scriptsize
\caption{Comparison between different imaging techniques.}
\label{tab:imaging_techniques}
\centering
\begin{tabular}{P{.1} P{.1} P{.15} P{.075} P{.425}}
\hline
Modality & Cost & Lighting & Coverage & Usage \\
\hline
Visible & Low & Requires Light & High & Structural defect, foreign object detection \\
IR & Moderate & Not   needed & High & Internal electrical defect \\
UV & Moderate & Not needed & Low & Corona discharge \\
X-Ray & High & Not needed & Low & Internal structural defect \\
LiDAR & High & Not needed & High & 3D mapping, terrain analysis, vegetation encroachment \\
\hline
\end{tabular}
\end{table*}

