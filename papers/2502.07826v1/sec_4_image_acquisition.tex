\section{Image Acquisition Platforms and Vehicles}\label{sec:image_acquisition}
The success of computer vision applications in the domain of transmission line maintenance largely depends on the quality and diversity of the images acquired. The choice of image acquisition techniques plays a vital role in determining the cost and labor associated with the operation. This section explores various methods employed for power line image acquisition.

\subsection{Unmanned Aerial Vehicles}
Unmanned Aerial Vehicles (UAVs), commonly known as drones, have emerged as a popular choice for image acquisition in power line inspection. UAVs equipped with high-resolution cameras can capture images and videos of transmission lines and their components from various angles and distances. They offer a unique combination of accessibility, maneuverability, and safety that is particularly well-suited to the challenges posed by transmission line maintenance. This versatility allows for comprehensive visual data collection, enabling the detection of defects and anomalies in power lines with greater accuracy \cite{zhang_automatic_2017}.

\subsection{Aerial Vehicles}
Aerial vehicles such as helicopters have been utilized for decades in power line inspection. Equipped with specialized cameras and imaging systems, they offer the advantage of covering long distances and providing a stable platform for capturing images of transmission lines. They are particularly suitable for inspecting high-voltage lines, where safety considerations are paramount \cite{yang_review_2020}.

\subsection{Land Vehicles}
In situations where aerial inspection may not be feasible or cost-effective, land vehicles equipped with imaging equipment are employed. These vehicles can navigate the terrain near transmission lines, capturing images and videos of the components and their surroundings. They are especially useful for inspecting power lines in areas with limited airspace accessibility.

\subsection{Fixed Camera}
Fixed cameras installed at strategic locations along power lines provide continuous monitoring. These cameras capture images at predefined intervals or when triggered by certain events. They offer a cost-effective solution for routine inspections and surveillance, although they may have limitations in terms of coverage and flexibility compared to aerial methods.

\subsection{Satellite Imaging}
Satellite imaging technology is increasingly being explored for large-scale monitoring of transmission lines. While the resolution may not be as high as that of UAVs or helicopters, satellite imagery can provide valuable data for identifying overall trends and assessing the condition of transmission networks over vast geographic areas \cite{zhou_insulator_2023}. 

% {\color{blue}
\subsection{Power Line Inspection Robots}
Power line inspection robots have emerged as a promising technology for automating the inspection and maintenance of transmission lines. These robots can be classified into climbing robots, flying robots (UAVs), and hybrid climbing-flying robots \cite{alhassan2020power}. Climbing robots can roll along the power lines and provide high-quality inspection data, but they face challenges in obstacle avoidance and deployment onto the lines. Flying robots offer faster inspection and easier obstacle avoidance but may have limitations in terms of inspection quality. Hybrid robots aim to combine the advantages of both climbing and flying robots for more effective inspection. Research in this area focuses on improving the robots' mechanical design, power systems, control algorithms, and sensing capabilities for autonomous operation \cite{chen2021environment}. For a comprehensive review of power line inspection robots, readers are referred to \cite{alhassan2020power, chen2021environment, ekren2024review}.
% }

Selecting the most suitable image acquisition method is crucial for obtaining high-quality data. The choice often depends on factors like cost, terrain, and the specific goals of the inspection. The choice of image acquisition method may also be influenced by weather conditions. Adverse weather, such as rain, snow, or fog, can impact the performance of aerial methods. Ground-based and fixed-camera systems may be preferred in such scenarios for their resilience to adverse weather conditions. Table \ref{tab:imaging_platforms} shows a general comparison between these techniques focusing on several different aspects such as cost, accuracy, coverage, and safety. 
% Figure \ref{fig:imaging_platform} shows statistics of the different imaging platforms used in the literature with UAV being the most used platform followed by fixed cameras, aerial vehicles, railway cameras and satellite.

\begin{table*}[htb]
\centering
\scriptsize
\caption{Comparison of different image acquisition techniques.}
\label{tab:imaging_platforms}
\begin{tabular}{P{0.15} P{0.05} P{0.1} P{0.07} P{0.1} P{0.07} P{0.07} P{0.15}}
\hline
Platform & Cost & Tracking* & Accuracy & Efficiency & Coverage & Safety & Usage \\
\hline
UAV & Low & Difficult & Good & Fast & Good & Safe & General Purpose \\
Land Vehicles & Low & Easy & Good & Slow & Limited & Safe & Road-side lines \\
Aerial Vehicles & High & Easy & Bad & Fast & Good & Unsafe & Hard-to-reach areas \\
Fixed Camera & High & Easy & Good & NA & Limited & Safe & For towers \\
Satellite & High & Difficult & Bad & Slow & Good & NA & Hard-to-reach areas \\
Inspection Robots & High & Easy & Good & Slow & Limited & Safe & Detailed inspection \\
\hline
\multicolumn{8}{l}{\textit{* Tracking refers to following and keeping track of the power line}}

\end{tabular}
\end{table*}

% \begin{figure}[htb]
%     \centering
%     \includegraphics[width=0.5\textwidth]{imaging_platform_updated.pdf}
%     \caption{Statistics of different imaging platforms used in literature. The figure is generated based on 64 articles. Some articles did not mention the imaging platform and hence were not included in this analysis.}
%     \label{fig:imaging_platform}
% \end{figure}