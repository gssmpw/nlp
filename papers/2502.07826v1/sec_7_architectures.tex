\section{Deep Learning Architectures and Detection Paradigms}\label{sec:architectures}

Deep learning models have significantly advanced the field of power line inspection, offering unique advantages in terms of speed, accuracy, and the ability to handle complex scenarios \cite{sundaram_deep_2021}. Models like YOLO \cite{redmon_you_2016, redmon_yolo9000_2016, redmon_yolov3_2018, bochkovskiy_yolov4_2020, jocher_yolo_2023}, R-CNN family \cite{girshick_rich_2014, girshick_fast_2015, ren_faster_2016}, SSD \cite{liu_ssd_2016}, and transformer architectures \cite{dosovitskiy2020image, liu2021swintransformerhierarchicalvision, carion2020end} have shown remarkable performance in detecting and classifying various power line components, including insulators, dampers, pin bolts, and conductor wires \cite{sadykova2019yolo, singh_2023_interpretable, zhang_cloud_edge_2020, wei_online_2022, zhai_hybrid_2021, rong_intelligent_2021, miao_insulator_2019, nguyen_intelligent_2019, dong_improved_2023, zhang_pa_detr_2023, jain2024transfer}. These models are increasingly being deployed in hierarchical detection systems, where lightweight variants perform initial coarse screening while more sophisticated versions handle refined secondary recognition \cite{wei_online_2022}. YOLO stands out for its real-time capabilities and high frame rate \cite{li_improved_2022}, while region-based CNNs excel in precisely localizing objects within images \cite{bharati2020deep}. SSD offers a balance between speed and accuracy \cite{huang2017speed}, making it particularly suitable for edge deployment, and transformer architectures, such as ViT \cite{dosovitskiy2020image}, Swin Transformers \cite{liu2021swintransformerhierarchicalvision}, and DETRs \cite{carion2020end}, have demonstrated their effectiveness in capturing global context and handling complex scenes \cite{han2022survey, dong_improved_2023, zhang_pa_detr_2023, jain2024transfer}.

Classification algorithms, particularly those pretrained on large datasets like ImageNet \cite{5206848}, have also been employed for identifying faults and anomalies in power line images. ResNet \cite{he_2023_deep}, VGG \cite{simonyan2014very}, MobileNet \cite{howard2017mobilenets}, and EfficientNet \cite{tan2019efficientnet} have shown promising results in classifying power line components as either faulty or in good condition \cite{wei_online_2022, cao_accurate_2023, luo_ultrasmall_2023, stefenon_semi_protopnet_2022, qiu_lightweight_2023, li_improved_2022, odo_aerial_2021, li_pin_2022}. The attention mechanism \cite{vaswani2017attention} has also gained widespread popularity in recent years, enhancing the accuracy and efficiency of object detection tasks \cite{cao_accurate_2023, fan_2019_few, kong_context_2018}. 

Various computer vision tasks, such as bounding box detection, semantic segmentation, and instance segmentation have been utilized in automating the inspection of power line components. Bounding box detection is particularly useful for identifying larger components like towers, insulators, and dampers \cite{ge_birds_2022}. Semantic segmentation provides detailed component-wise masks \cite{electronics12153210, bob_semantic}, while instance segmentation excels in scenarios where components are close together or overlapping \cite{electronics12153210, bob_semantic}.

For a more detailed discussion on these deep learning models and computer vision tasks, please refer to Appendices \ref{appendix:dl_models} and \ref{appendix:cv_tasks}, respectively.