\section{Related Works}
\label{sec:related_works}
The application of computer vision and deep learning for power line inspection has garnered increasing attention in recent years, reflected in a growing body of survey papers. Chen et al. \cite{chen2021environment} and Yang et al. \cite{yang2020review} offer general overviews of automated power line inspection techniques, including 3D reconstruction, object detection, and inspection platforms. Their work is primarily centered on the technology aspects, such as LiDAR-based reconstruction and the classification of detection techniques. While this technical perspective is valuable, the review lacks an in-depth exploration of deep learning's impact on these technologies. Sundaram et al. \cite{sundaram2021deep} provide an overall perspective on deep learning applications in the electrical domain, touching upon power line inspection alongside other areas. However, their review is broader in scope, covering multiple electrical applications, which dilutes the focus on power line inspection. Ruszczak et al. \cite{ruszczak2023overview} focus specifically on the importance of specialized datasets for training deep learning models in power line inspection tasks. 

Several reviews delve into the use of unmanned aerial vehicles (UAVs) for inspection. Xu et al. \cite{xu2023development} provide a systematic summary of UAV platforms and image recognition techniques, while Foudeh et al. \cite{foudeh2021advanced} concentrate on UAV technologies and control strategies. Nguyen et al. \cite{nguyen_intelligent_2019, nguyen_automatic_2018} explored UAV-based power line inspection techniques in the light of deep learning and proposed a concept for an autonomous UAV-based inspection system and discuss its challenges and possibilities. The above mentioned reviews are notable for their focus on UAV-based systems, but it largely centers on the hardware and system integration aspects. Although they mention deep learning, it is not the primary focus of their analysis. Finally, Liu et al. \cite{liu_data_2020} present a comprehensive review of data analysis techniques, including deep learning methods, for power line inspection. Although the authors cover deep learning, the topic of the review is much broader due to the inclusion of image processing techniques and non machine learning-based approaches.

While these existing reviews offer valuable insights, this current review distinguishes itself through several key contributions. First, it provides a comprehensive and up-to-date analysis of deep learning applications specifically for power line inspection, encompassing both component detection and fault diagnosis. This focused exploration of deep learning techniques sets it apart from previous surveys that either covered a broader range of inspection methods or touched upon power line inspection as part of a larger survey. Second, the review systematically categorizes existing research into component detection and fault diagnosis, providing a structured understanding of the field. It summarizes various methods and techniques, offering insights into their functionality and use cases. This systematic approach facilitates easier navigation and comprehension of the current state-of-the-art. Furthermore, the review places a particular emphasis on deep learning methodologies, including their fundamental principles and practical applications in power line inspection. Finally, the review outlines future research directions, highlighting areas like data quality improvement and small object detection that require further exploration. It serves as a roadmap for future research endeavors, guiding researchers towards potential breakthroughs.

In summary, this review paper contributes significantly to the field by offering a comprehensive, systematic, and deep learning-focused analysis of automated power line inspection. It builds upon existing knowledge, provides a structured understanding of current research, and charts a course for future advancements.