\section{Methodology for Automated Power Line Inspection}\label{sec:automated}
Automated power line inspection leverages the capabilities of computer vision and deep learning to enhance the safety, reliability, and efficiency of power transmission systems. Figure \ref{fig:methodology} shows the overview of a typical computer vision-based power line inspection process. It starts by capturing images of power line components using aerial or land vehicles sometimes using multiple imaging techniques. The images are then processed through the object detection or segmentation algorithm. Subsequently, the image classification algorithm evaluates the segmented insulators to classify them as 'good' or 'faulty', resulting in the final fault detection output.  This section provides an overview of the key steps involved in this process, from data collection to fault diagnosis. It should be noted that in many cases in the literature, the component detection part has been skipped and the deep learning model has been trained for the fault detection purpose directly. 

\begin{figure*}[htb]
    \centering
    \includegraphics[width=1\linewidth]{fig_1_methodology.pdf}
    \caption{Block Diagram of an automated multi-modal power line inspection system.}
    \label{fig:methodology}
\end{figure*}

\subsection{Data Collection}
The basis of automated power line inspection is high-resolution images usually captured from land or aerial vehicles. The collected images provide valuable visual data that forms the basis for subsequent analysis. The data collection process for power line inspection has been discussed in more detail in Section \ref{sec:image_acquisition}.

\subsection{Preprocessing}
The raw images obtained during data collection often require preprocessing to improve their suitability for analysis. Preprocessing steps may include noise reduction, image enhancement, and geometric correction to account for variations in lighting conditions and perspective distortions. These enhancements ensure that the subsequent computer vision algorithms can work effectively. Figure \ref{fig:image_processing} shows visual examples of different image processing techniques such as image enhancement, edge detection, orientation correction, and color-based background subtraction. 

\begin{figure}[htb]
    \centering
    \includegraphics[width=0.5\textwidth]{fig_2_image_processing.pdf}
    \caption{Example of different image enhancement techniques used for power line inspection \cite{liu_data_2020}.}
    \label{fig:image_processing}
\end{figure}

\subsection{Component Detection}
Component detection plays a central role in identifying and locating power line components within the power line images. Deep learning models, such as Faster R-CNN \cite{girshick_rich_2014}, (You Only Look Once) YOLO \cite{redmon_you_2016}, or (Single Shot Multibox Detector) SSD \cite{liu_ssd_2016}, are employed to detect a wide range of components, including insulators, suspension clamps, dampers, and conductors. These models can distinguish these components from the complex backgrounds typically encountered. Section \ref{sec:components} provides a comprehensive overview of the literature focusing on power line component detection.

\subsection{Fault Diagnosis}
The final phase in automated power line inspection involves fault diagnosis, where deep learning models analyze detected components or direct image inputs to identify potential issues. Deep learning models are employed to analyze the extent and severity of faults, enabling operators to prioritize maintenance and repair efforts. These models can provide valuable insights into the overall health of the power transmission system, aiding in the prevention of outages and accidents. Section \ref{sec:fault} provides a comprehensive overview of the literature focusing power line fault detection.

Automated power line inspection harnesses the capabilities of computer vision and deep learning to streamline the inspection and maintenance of power transmission systems. By automating the detection and diagnosis of components and faults, these systems contribute to safer, more reliable power delivery while minimizing downtime and operational costs.