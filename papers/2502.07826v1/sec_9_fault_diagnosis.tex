\section{Power Line Fault Diagnosis}\label{sec:fault}
Visual inspection, coupled with cutting-edge computer vision techniques has emerged as a powerful and efficient means of identifying and diagnosing a spectrum of faults. From insulator defects to conductor issues, and tower anomalies to grounding problems, the ability of computer vision to meticulously assess power line components is transforming maintenance and reliability in the electrical grid. This section delves into the application of computer vision and deep learning for the diagnosis of various faults (Figure \ref{fig:different_faults}) within power lines and associated equipment, underlining its potential to enhance the resilience and performance of critical power transmission infrastructure.

\begin{figure*}[htb]
    \centering
    \includegraphics[width=1\linewidth]{fig_4_different_faults.pdf}
    \caption{Different types of power line faults \cite{liu_data_2020}.}
    \label{fig:different_faults}
\end{figure*}

\subsection{Insulator Fault Detection}
Insulators are critical components of power lines, ensuring that high-voltage currents are safely transported without coming into contact with the towers or poles that support them. However, insulators continuously endure environmental stresses and mechanical wear, which can lead to various types of faults. The following sections discuss some of the most common types of insulator faults and review significant research addressing these issues.

\begin{table*}[htb]
% {\color{blue}
\scriptsize
\caption{Summary of insulator surface fault detection studies.}
\label{tab:insulator_surface_faults} 
\begin{tabular}{P{.05} P{.1} P{.1} P{.1} P{.1} P{.15} P{.1} P{.1}}  

\hline  
Year \& Ref & Component & Type of Detection & Type of Fault & Imaging Platform & Dataset & Algorithm & Performance \\ 
\hline 

2017 \cite{liu_discrimination_2017} & Porcelain Insulators & Classification & Deterioration Damage & Still Camera & 700 IR Images & CNN & Accuracy: 93\% \\

2019 \cite{kang_deep_2019} & Insulators & Classification & Surface Defects & High Speed Railway & 18000 RGB Images & DDAE (Based on CNN AutoEncoder) & F1-Score: 0.95 \\

2019 \cite{sadykova2019yolo} & Glass Insulator & Classification & Surface Defects & UAV & 3500 RGB Images & CNN & Accuracy: 87\% \\

2020 \cite{ibrahim_application_2020} & Insulators & Classification & 3 Types of Surface Defects & Still Camera & 1240 RGB Images & CNN & Accuracy: 89.5\% \\

2020 \cite{mussina_multi_modal_2020} & Glass Insulators & Classification & 4 Types of Surface Defects & UAV & 5000 RGB Images & FCN & Accuracy: 99.76\% \\

2021 \cite{waleed_drone_based_2021} & Ceramic Insulators & Bounding Box Detection & Insulator Surface \& Structural Defects & UAV & 2973 RGB Insulator Images & R-CNN, SVM, CNN, SSD Fusion & Onshore mAP: 0.56-0.77; Onboard mAP: 0.24-0.27 \\

2022 \cite{stefenon_classification_2022} & Insulators & Classification & Kaolin Defects & Still Camera & 600 RGB Images & ANN & Accuracy: 97.50\% \\

2023 \cite{roy_accurate_2023} & Insulators & Classification & 4 Types of Surface Defects & Still Camera & 1000 RGB Images & CNN & Accuracy: 97.5\% \\
\hline

\end{tabular}
% }
\end{table*}

\subsubsection{Surface Defect Detection}
Surface defects in insulators encompass a range of anomalies that affect the outermost layer of the insulator. This category includes issues such as surface contamination, cracking, flashover marks, arcing damage, and chipping. Recent research works have made significant strides in the detection and classification of these surface defects, utilizing a variety of computer vision techniques and machine learning algorithms. Roy et al. \cite{roy_accurate_2023} developed a deep learning framework incorporating AlexNet, VGG16, and ResNet50 models to detect sand, ash, and mud contaminations on the insulator surface. However, this approach requires images of the insulators taken at close proximity and is limited to classification and cannot detect contaminated areas. Mussina et al. \cite{mussina_multi_modal_2020} introduced a Fusion Convolutional Network (FCN) for the real-time monitoring of insulators using UAVs, addressing challenges like varying resolutions and unclear surfaces. It combines a CNN with a binary ANN to form a multi-modal information fusion system that improves contamination classification accuracy on insulators to 99.76\% by using image data and leakage current values. However, the proposed model was trained on a relatively small dataset of only 250 images per class and it requires close-up images of the insulator with uniform background. These limitations can affect the model's capability to generalize on real-world situations. Table \ref{tab:insulator_surface_faults} summarizes the literature on insulator surface faults.

\subsubsection{Structural Defect Detection}
Structural defects refer to issues that affect the internal composition and mechanical strength of the insulator. They include complete breakage, missing insulator caps, and material degradation. Table \ref{tab:insulator_structural_faults} summarizes the literature on insulator structural faults. The following goes over some of the notable works on structural defect detection.
 
In a 2022 work, \cite{antwi_bekoe_deep_2022} proposed an attention-based end-to-end framework that combines object detection and instance segmentation at the pixel level. Although trained on a relatively small dataset of 1523 images, this is one of the few works that targeted pixel-level segmentation and achieved great results. By integrating a three-branch attention structure into the backbone network, the proposed model achieved a significant improvement in detection performance, surpassing the state-of-the-art instance mask prediction models while maintaining computational efficiency. In a similar work by Wang et al. \cite{wang_detection_2020}, the authors proposed an insulator defect detection method that leverages an improved ResNeSt network and an enhanced Region Proposal Network (RPN). The improvements to ResNeSt were aimed at refining the detection of insulator defects, particularly in aerial images where insulators might be small and have low resolution. The authors acknowledge that their method might struggle to detect certain types of defects, such as breaks on specific insulator types, due to their subtle visual characteristics. The computational efficiency of the proposed method, while improved compared to some baseline models, might still be a concern for real-time applications on resource-constrained devices.

Multi-task networks can be used to detect different types of insulator faults as separate classes. In a recent article Fu et al. \cite{fu_small_sized_2023} presented the I2D-Net, a deep learning-based method for detecting small-sized defects in overhead transmission line insulators, particularly missing caps. The I2D-Net enhanced the Faster R-CNN object detection framework with three key components: the Three-path Feature Fusion Network (TFFN) to improve feature extraction from shallow layers, the enhanced Receptive Field Attention (RFA+) block to adapt to different-scale defects, and the Context Perception Module (CPM) to enhance defect localization. The authors acknowledged that while the I2D-Net achieved high accuracy, it came with a slight increase in inference time compared to the baseline Faster R-CNN + FPN model. Liu et al. \cite{liu_box_point_2021} proposed another approach that utilizes a deep CNN with parallel branches to locate fault regions and estimate insulator endpoints. The method offers high accuracy and robustness, outperforming previous approaches like Faster R-CNN, SSD, and cascading networks. Zhang et al. \cite{zhang_defgan_2021} took a different approach where they used a generative network including a denoising autoencoder, discriminator, and classifier to detect defects. The method comprises two stages: first, insulator extraction using cascaded deep segmentation networks (CDSNets); second, defect detection within sampled patches using the proposed DefGAN, scoring defects based on classifier anomaly probability and denoising autoencoder reconstruction error. Although the proposed model can detect mild deformities, it is sensitive to the noise in the image.

One of the primary causes of insulator failures is the self-explosion of caps and several research works have been done to detect these faults. Cao et al. \cite{cao_accurate_2023} proposed an improved image augmentation method for the detection of self-detonation defects in aerial images of insulators. The method incorporated edge detection to enhance the shape features of insulators, which were then used to guide the augmentation process. The authors utilized the YOLOv3 model for insulator detection and an improved ResNet-18 model for defect classification. The proposed method was evaluated on a dataset of aerial images and showed superior performance compared to baseline models and other augmentation techniques. However, the authors acknowledged the need for further evaluation on larger and more diverse datasets, including images captured under different weather conditions. Additionally, the proposed method relied on edge detection as prior knowledge, which might not be sufficient for detecting more complex or subtle defects. In another work, Wei et al. \cite{wei_online_2022} introduced a fault detection scheme for insulator self-explosion, leveraging edge computing and deep learning to address issues with traditional centralized cloud computing. It employs a lightweight SSD target recognition network at the edge, replacing VGG16 with MobileNets to reduce computation. In the cloud, three distinct detection algorithms (Faster-RCNN, Retinanet, YOLOv3) are used to identify insulator self-explosion areas, and a novel multimodel fusion algorithm (Figure \ref{fig:wei_multimodel}) enables overhead transmission line fault detection. Results demonstrate effective data reduction, achieving an average cloud recognition accuracy of 95.75\%, with a modest increase in edge device power consumption.

\begin{figure}[htb]
    \centering
    \includegraphics[width=0.5\textwidth]{fig_5_wei_multimodel.pdf}
    \caption{Simplified diagram of multi-model fusion network for detecting insulators \cite{wei_online_2022}.}
    \label{fig:wei_multimodel}
\end{figure}

% {\color{blue}
{\scriptsize
\begin{longtable}{P{.05} P{.1} P{.1} P{.1} P{.1} P{.15} P{.1} P{.1}} \caption{Summary of insulator structural fault detection studies.}
\label{tab:insulator_structural_faults} \\
\hline  

Year \& Ref & Component & Type of Detection & Type of Fault & Imaging Platform & Dataset & Algorithm & Performance \\ 
\hline 

2019 \cite{jiang_insulator_2019} & Glass and Porcelain Insulators & Bounding Box Detection & Insulator Structural Defects & UAV & 485 RGB Images & SSD & Precision: 91.23\%, Recall: 93.69\% \\

2020 \cite{tao_detection_2020} & Insulators & Bounding Box Detection & Structural Defects & UAV & 1956 RGB Images & Cascaded DNN & F1 Score: 93.3\%, Speed: 135ms/Image \\

2020 \cite{wang_detection_2020} & Insulators & Bounding Box Detection & Insulator Structural Defect & NA & 2560 RGB Images & Improved RetinaNet & Accuracy: 98.38\% \\

2021 \cite{liu_box_point_2021} & Insulators & Bounding Box Detection & Structural Faults & Aerial Vehicle & 969 RGB Images & Box Point Detector & mAP@0.5: 94.0\% \\

2021 \cite{zhang_insudet_2021} & Insulators & Bounding Box Detection & Missing Cap & UAV \& Synthetic & CPLID Dataset: 848 RGB Images, Pascal VOC Dataset: 5011 RGB Images & Improved YOLOv3 & Pascal VOC: mAP@0.75: 45.90\% CPLID: mAP@0.75: 64.05\% \\

2021 \cite{zhang_defgan_2021} & Insulators & Classification & Insulator Structural Defect & High Speed Railway Catenary & 800 RGB Images & GAN & F1-Score: 0.95 \\

2022 \cite{antwi_bekoe_deep_2022} & Insulators & Instance Segmentation & Insulator Defects & UAV & 1523 RGB Images & Mask R-CNN & AP: 89.4\% \\

2022 \cite{hao_insulator_2022} & Insulators & Bounding Box Detection & Bunch Drop Defect & UAV \& Synthetic & CPLID Dataset: 848 RGB Images & Improved YOLOv4-ResNest & mAP: 95.63\% \\

2022 \cite{wei_online_2022} & Insulators & Bounding Box Detection & Self-Explosion Defect & UAV & 8500 RGB Images & F-RCNN, RetinaNet, YOLOv3 Fusion & Precision: 99.04\%, Recall: 93.69\% \\

2023 \cite{cao_accurate_2023} & Glass Insulators & Bounding Box Detection \& Classification & Self-Explosion Defects & UAV & 8463 RGB Images & YOLOv3 \& Improved ResNet-18 & F1-Score: 86.25\% \\

2023 \cite{fu_small_sized_2023} & Insulators & Bounding Box Detection & 5 Types of Structural Defects & UAV \& Synthetic & CPLID Dataset: 848 RGB Images & I2D-Net (Based on F-RCNN) & mAP: 89.6\% \\

2023 \cite{hao2023pkamnet} & Insulator & Bounding Box Detection & Structural Defect & Aerial Vehicle & 900 RGB Images & PKAMNet & mAP@0.5: 95.5\% \\

2023 \cite{singh_2023_interpretable} & Insulators & Classification & Structural Defects & UAV & 848 RGB Images & Ps-ProtoPNet & F1-Score: 0.996 \\

2023 \cite{zhang_multi_objects_2023} & Insulators & Bounding Box Detection & Self-Explosion Defect & UAV \& Synthetic & CPLID Dataset: 848 RGB Images & GhostNet-YOLOv4 & mAP: 99.50\% \\

2024 \cite{jain2024transfer} & Insulator & Bounding Box Detection & Structural Defects & UAV & 5939 RGB Images & YOLO-v5 and DETR & mAP: 98\% \\

2024 \cite{wang2024mci} & Insulator & Bounding Box Detection & Structural Defect & UAV and Synthetic & CPLID Dataset: 848 RGB Images & Improved YOLO-based Network & mAP@0.5: 85.8\% \\
\hline

\end{longtable}
}
% }

\subsection{Detection of Conductor Faults}
Transmission line conductors are the lifelines of the electrical power grid, carrying electrical energy over long distances. Conductor defects, such as aging, corrosion, internal damage, and contamination, pose significant threats to the reliable and safe operation of power transmission systems. This subsection explores the challenges associated with conductor defect detection and highlights the use of computer vision technologies to enhance the efficiency and accuracy of inspections. Table \ref{tab:conductor_faults} summarizes the literature on conductor faults.

\begin{table*}[htb]
% {\color{blue}
\scriptsize
\caption{Summary of conductor fault detection studies.}
\label{tab:conductor_faults}
\begin{tabular}{P{.05} P{.1} P{.1} P{.1} P{.1} P{.15} P{.1} P{.1}}    

\hline  
Year \& Ref & Component & Type of Detection & Type of Fault & Imaging Platform & Dataset & Algorithm & Performance \\ 
\hline 

2019 \cite{li_image_2019} & Oil Transformer and Conductor & Classification & Internal Defects & Still Camera & 12000 UV, IR and Visible Images & Capsule Network & Qualitative Assessment \\

2021 \cite{rong_intelligent_2021} & Conductor & Bounding Box Detection & Vegetation & Tower Mounted Camera & 70 RGB Images & Cascaded Network (Based on F-RCNN) & Accuracy: 95\% \\

2022 \cite{yi_intelligent_2022} & Conductor & Classification & Aging Defect & NA & 5200 RGB Images & Improved AlexNet & Avg. MAE: 3.80 \\

2023 \cite{wang_internal_2023} & Conductors & Bounding Box Detection & 4 Types of Composite Core Defects & Inspection Robot & 2500 X-Ray Images & CenterNet-NDT (Based on ResNet50) & mAP: 90.60\% \\

2024 \cite{bergmann2024approach} & Conductor & Semantic Segmentation & Vegetation & Land & 851 LiDAR Images & UNet-based Network & Precision: Above 90\% \\
\hline

\end{tabular}
% }
\end{table*}

External defects of conductor wires can be detected using visible light images. These types of defects can be due to aging, structural damages or due to foreign objects which are described in more detail in the “Detection of Foreign Objects” subsection. Aging defects of conductors are one of the most common for aluminum conductors.  In a 2022 study, Yi et al. \cite{yi_intelligent_2022} introduced a novel approach to quantitatively diagnose the aging of conductor surfaces in smart high-voltage electricity grids. The model consists of an AlexNet-based deep convolution network and a specialized loss function that incorporates Gaussian label distribution. By reframing the conductor morphology aging problem as a multiclassification challenge, the model leverages a weakly labeled training dataset and a carefully designed loss function, combining entropy loss, cross-entropy loss, and Kullback–Leibler divergence loss. Comparative analysis with four commonly used CNN-based classifiers demonstrates superior performance on the collected dataset. However, the proposed model is suitable only when super-close-up high resolution conductor images are available.

Internal defects of the conductors can be challenging to detect using visible light images. To address this, Wang et al. \cite{wang_internal_2023} presented a novel automatic detection system for identifying internal defects in overhead aluminum conductor composite core (ACCC) transmission lines. The system utilizes an X-ray inspection robot equipped with a nondestructive testing (NDT) system to capture X-ray images of ACCC wires. The proposed system employs an anchor-free object detection model called CenterNet-NDT, which incorporates specialized modules like spatial pyramid pooling-cross stage partial convolution (SPPCSPC), polarized self-attention (PSA), and a weighted bidirectional feature pyramid network (SOFPN). Although CenterNet-NDT achieves a high mAP of 90.60\% on the IN-ACCC dataset, the instrumentation and maintenance required for this method can be challenging. 

\subsection{Fault Detection for Fittings: Pin-Bolts, Dampers, Suspension Clamps}
Power line fittings such as pin bolts, dampers, and suspension clamps are integral components of transmission and distribution systems. These components are tiny compared to the rest of the transmission line tower and they occupy a very small area in the images involving only a few pixels. As a result, accurately detecting faults in these components requires a high level of image processing and powerful deep-learning algorithms. Table \ref{tab:fittings_faults} summarizes the literature on power line fittings fault detection.

\begin{table*}[htb]
% {\color{blue}
\scriptsize
\caption{Summary of fault detection studies for power line fittings.}
\label{tab:fittings_faults}
\begin{tabular}{P{.05} P{.1} P{.1} P{.1} P{.1} P{.15} P{.1} P{.1}}    

\hline  
Year \& Ref & Component & Type of Detection & Type of Fault & Imaging Platform & Dataset & Algorithm & Performance \\ 
\hline 

2020 \cite{zhao_detection_2020} & Bolts & Bounding Box Detection & Missing Pin Defect & Tower Mounted Cameras & 1840 RGB Images & AVSCNet & mAR: 0.876 \\

2021 \cite{xiao_detection_2021} & Bolts & Bounding Box Detection & Pin Defects & Aerial Vehicle & 1500 RGB Images & Improved MTCNN & mAP: 94.76\% \\

2022 \cite{li_pin_2022} & Bolts & Bounding Box Detection & 2 Types of Pin Defects & UAV & 482 RGB Images & EfficientDet-D7 & mAP: 54.3\%, mAR: 63.4\% \\

2022 \cite{zhang_attention_guided_2022} & Dampers and Suspension Clamps & Bounding Box Detection & Rust Defects and Abnormal Conditions & UAV & 1209 RGB Images & AGMNet & Rust: mAP: 75.4\%, Abnormal: mAP: 92.7\% \\

2022 \cite{zhao_new_2022} & Bolts & Classification & 5 Types of Bolt Defects & UAV & 1944 RGB Images & VFPKNet & Accuracy: 83.29\% \\

2023 \cite{huang_structural_2023} & Damper & Classification & Structural Defects & UAV & 240 Aerial Images & Improved GrabCut & Accuracy: 95.76\% \\

2023 \cite{jiao2023defective} & Bolts & Bounding Box Detection & Structural Defect & UAV & 32094 Images & MARF-CCN Network & mAP: 87.16\% \\

2023 \cite{luo_ultrasmall_2023} & Bolts & Bounding Box Detection & Structural Defects & UAV & 1852 RGB Images & UBDDM (Based on ResNet-50 \& Attention) & mAP: 78.6\% \\

2023 \cite{song_deformable_2023} & Fittings & Bounding Box Detection & Rust Defect & NA & TLCF Dataset: 700 RGB Images & Deformable YOLOX & mAP@0.5: 0.857, mAP@0.5:0.95: 0.533 \\

2023 \cite{zhang_pa_detr_2023} & Bolts & Classification & 4 Types of Bolt Defects & UAV & VIBD Dataset: 8972 Bolt Instances in RGB Images & PA-DETR (Based on ResNet50, FPN and Attention) & mAP: 81.9\% \\

2023 \cite{zhang2023dsa} & Dampers & Bounding Box Detection & Structural Defect & UAV & 490 RGB Images & DSA-Net & mAP@0.5: 0.935 \\
\hline

\end{tabular}
% }
\end{table*}

Zhao et al. \cite{zhao_new_2022} proposed RFBD (Recognition of Bolt Faults using multilabel image recognition) that comprises two networks: VFSKnet leverages professional posterior knowledge to model relationships between bolt defect labels, while VFPKnet extracts structural features to capture fine-grained details. After combining and weighting these subnetworks, RFBD achieves label-level accuracy of 93.91\% and image-level accuracy of 83.29\% in detecting 5 types of bolt defects. However, the proposed solution is not end-to-end and requires properly segmented images of bolts. In a similar work, Zhang et al. \cite{zhang_pa_detr_2023} addressed the challenge of visually indistinguishable bolt defects in transmission lines by proposing an end-to-end detection method based on transmission line knowledge reasoning. It employs the DETR (End-to-End Object Detection with Transformers ) \cite{carion2020end} model augmented with a dilated encoder module to capture multiscale features. Additionally, a Transmission Line Image Relative Position Encoding (TL-iRPE) is designed to infer bolt position knowledge. The approach includes a bolt attributes classifier and a bolt defects classifier, combining position and attributes knowledge to enhance defect detection accuracy.

One of the most prominent faults that occurs often is the missing pin of bolts. Due to its tiny nature, it can be very challenging to detect these faults. Zhao et al. \cite{zhao_detection_2020} proposed the AVSCNet model to address the challenge of detecting pin-missing defects in bolts on transmission lines using aerial images. The model tackled the issue of varying visual shapes of bolts due to different camera angles by employing an automatic visual shape clustering method. It also incorporated feature enhancement, feature fusion, and expanded region-of-interest feature extraction to improve the detection of small-scale defects in complex aerial images. However, the model's performance might be affected by the distributional differences in aerial images captured from diverse transmission line structures and environments.

Structural anomalies and rusting of the dampers and suspension clamps can compromise the transmission line system causing failure. Zhang et al. \cite{zhang_attention_guided_2022} introduced the AGMNet, an attention-guided multitask convolutional neural network designed for the detection of power line parts in aerial images. The network incorporated a region attention mechanism to enhance the feature representation of objects, a refinable Region Proposal Network (RPN) to improve proposal quality, and a multitask learning framework to simultaneously detect power line fittings (dampers and suspension clamps), identify their rust levels, and detect abnormal conditions. However, the authors acknowledged the need for further evaluation on larger and more diverse datasets. Additionally, the distinction between different rust levels was found to be challenging, leading to less accurate rust level identification. The authors suggested further research to improve the accuracy of rust level detection.  

\subsection{Fault Recognition on Towers}
Towers are the sturdy backbone that supports high-voltage conductors, ensuring the reliable delivery of electricity over long distances. Traditional methods for fault detection in transmission lines have limitations, such as susceptibility to noise and transient fluctuations. To address these challenges, Wang et al. \cite{wang_image_2019} proposed a novel approach for fault zone detection that emphasizes fine-grained categorization and quality-awareness. The primary objective is to identify the most distinguishing image patches for classification. The key components of their method involve extracting features from line currents using a Fast R-CNN-based image sample decomposition, with a quality module selecting the most informative regions. These extracted features are then utilized in a Support Vector Machine (SVM) \cite{cortes1995support} for fault recognition. In a similar work, Liang et al. \cite{liang_detection_2020} created four medium-sized datasets for training component detection and classification models. They also employed data augmentation techniques to balance the imbalanced classes. Furthermore, the authors proposed a multi-stage component detection and classification approach using the SSD network and deep ResNet to detect small components and faults. The results demonstrate that the proposed system is both fast and accurate in detecting common faults in tower components, such as missing top caps, cracks in poles and cross arms, woodpecker damage on poles, and rot damage on cross arms. Table \ref{tab:tower_faults} summarizes the literature on tower faults.

\begin{table*}[htb]
% {\color{blue}
\scriptsize
\caption{Summary of tower fault detection studies.}
\label{tab:tower_faults}
\begin{tabular}{P{.05} P{.1} P{.1} P{.1} P{.1} P{.15} P{.1} P{.1}}   

\hline  
Year \& Ref & Component & Type of Detection & Type of Fault & Imaging Platform & Dataset & Algorithm & Performance \\ 
\hline 

2019 \cite{nguyen_intelligent_2019} & Tower Components & Classification & Structural Defects & UAV & 123151 RGB Images & ResNet & F1-Score: 77.96\% \\

2019 \cite{wang_image_2019} & Tower Components & Classification & Line Faults & NA & 11600 RGB Images & SVM & Accuracy: 96.73\% \\

2020 \cite{liang_detection_2020} & Tower Components & Bounding Box Detection & 8 Types of Structural Defects & UAV & WIRE\_10 Dataset: 8000 RGB Images & F-RCNN & mAP: 91.11\% \\

2021 \cite{odo_aerial_2021} & Tower Components & Classification & Surface and Structural Defects & Helicopter & Insulators: 25804 RGB Images, Bolts: 94,619 RGB Images & EfficientNetB0 & ROC: 0.94 (insulator), ROC: 0.98 (Bolts) \\

2022 \cite{liu2022component} & Tower Components & Bounding Box Detection & Structural Defect & UAV & 1309 RGB Images & Graph Neural Network & mAP@0.5: 89.1\% \\

2022 \cite{stefenon_classification_2022} & Tower Components & Classification & Structural Defects & Still Camera & 240 RGB Images & Inception v3 & F1-Score: 84.50\% \\

2022 \cite{stefenon_semi_protopnet_2022} & Tower Components & Classification & Structual Defects & Still Camera & 240 Images & Semi-ProtoPNet (Based on VGG-19) & Accuracy: 97.22\% \\

2023 \cite{liu2023fault} & Tower Components & Classification & Structural Defect & Aerial Vehicle & TLAD Dataset: 4811 RGB Images & NMHEM Model & F1 Score: 80.5\% \\

2023 \cite{yi2023pstl} & Tower Components & Bounding Box Detection & Structural Defect & UAV & 956 RGB Images & PSTL-Net & mAP: 0.848 \\

2024 \cite{zhong2024visual} & Substation and Tower Components & Classification and Bounding Box Detection & Structural Defect & UAV and Land & Substation Dataset: 2000 Images; UAV Dataset: 5048 Images; CPLID Dataset 848 Images & Federated Learning & mAP: .990 \\
\hline

\end{tabular}
% }
\end{table*}

\subsection{Detection of Foreign Objects}
Foreign object-related anomalies and faults in power lines encompass a multitude of potential hazards that can pose significant threats to the reliable operation of electrical transmission systems. 
% {\color{blue}
These hazards can range from inadvertent interference caused by construction equipment and vehicles to more natural occurrences like fires, bird nests, and overgrown vegetation among others. 
% } 
The presence of foreign objects in proximity to power line components can lead to various issues, including interruptions in electrical supply, damage to critical equipment, and heightened safety concerns for both utility personnel and the public. Table \ref{tab:foreign_object_detection} summarizes the literature on foreign object faults.

\begin{table*}[htb]
% {\color{blue}
\scriptsize
\caption{Summary of foreign object detection studies in power line systems.}
\label{tab:foreign_object_detection}
    
\begin{tabular}{P{.05} P{.1} P{.1} P{.1} P{.1} P{.1} P{.1} P{.15}}
\hline  
Year \& Ref & Component & Type of Detection & Type of Fault & Imaging Platform & Dataset & Algorithm & Performance \\ 
\hline 

2020 \cite{zhang_cloud_edge_2020} & Tower, Conductor & Bounding Box Detection & 7 Types of Foreign Objects & NA & 926 \& 2000 RGB Images & YOLOv4 & - \\

2020 \cite{zhu_deep_2020} & Tower, Conductor & Bounding Box Detection & Foreign Objects & Tower Mounted Camera & 8000 RGB Images & DBF-NN & mAP: 88.1\% \\

2022 \cite{ge_birds_2022} & Tower & Bounding Box Detection & Bird Nest & Aerial Platform & 3695 RGB Images & YOLOv5 & mAP: 83.4\%, FPS: 85.32 \\

2022 \cite{li_improved_2022} & Tower, Conductor & Bounding Box Detection & 4 Types of Foreign Objects & NA & 35000 RGB Images & YOLOv3-MobileNetv2 & mAP: 0.832, FPS: 60 \\

2023 \cite{bi_yolox_2023} & Tower, Insulator & Bounding Box Detection & Bird Nest & UAV \& Synthetic & Bird Nest: 2864 Images, CPLID Dataset: 848 RGB Images & YOLOX++ & mAP: 86.8\% \\

2023 \cite{qiu_lightweight_2023} & Tower & Bounding Box Detection & Foreign Objects & UAV \& Manual & 1232 RGB Images & YOLOv4 with Attention & mAP: 96.71\%, FPS: 45 \\

2023 \cite{yu_foreign_2023} & Tower, Conductor & Bounding Box Detection & 6 Types of Foreign Objects & UAV & 8803 RGB Images & Improved YOLOv7 & mAP: 92.2\%, FPS: 19 \\

2023 \cite{zhang_edge_2023} & Conductor & Bounding Box Detection & Cranes & Tower Mounted Camera & 4000 RGB Images & Edge VIP (Based on YOLOv5s) & mAP: 50.60\% \\
\hline

\end{tabular}
% }
\end{table*}

Construction equipment and vehicles operating near power lines can accidentally make contact with electrical components, causing short circuits, equipment failure, and potentially even electrical fires. Similarly, fires in the vicinity of power lines can result from various sources, including wildfires, discarded cigarette butts, or even arcing caused by faulty equipment. Zhang et al. \cite{zhang_edge_2023} presented an edge-based framework for power transmission line abnormal target detection, focusing on overcoming resource limitations and improving model performance. To mitigate the lack of labeled data, deep semi-supervised learning was introduced, which can refine the decision boundary by learning from unlabeled samples. To achieve this, the framework starts with an initial model trained on a small amount of labeled data and then updates itself using unlabeled data. In another work, Zhang et al. \cite{zhang_cloud_edge_2020} introduced a framework that combines cloud and edge computing with deep learning techniques. Initially, a YOLOv4 model is trained in the cloud server for abnormal object detection. This trained model is then deployed to edge servers for real-time detection of abnormal objects in captured pictures. To address the limited initial data samples of only 926 images, enhancement techniques are used to increase the number of pictures, and real-time data streams are employed for incremental learning.

Bird nests, while seemingly innocuous, can also present challenges for power lines. Nests built on or near power line components can lead to electrical faults if they bridge connections or create conductive pathways. Ge et al. \cite{ge_birds_2022} proposed a bird's nest defect recognition method using YOLOv5, aiming to address the safety concerns posed by bird nests on power transmission towers. The method employs a YOLOv5- based architecture, comprising a backbone network, Feature Pyramid Network (FPN) \cite{lin2017feature}, and YOLO head, and undergoes multiple rounds of training with a constructed bird's nest defect database and transmission line model. Results demonstrate that the YOLOv5 model achieves an 83.40\% recognition rate for bird's nests while maintaining a high FPS rate of 85.32. In a similar work, Bi et al. \cite{bi_yolox_2023} presented a novel target detection model called YOLOX++, which is built upon the YOLOX \cite{ge2021yolox} architecture to enhance the detection of abnormal targets in transmission lines. It introduces a multiscale cross-stage partial network (MS-CSPNet) to fuse multiscale features and expand the receptive field of the target, improving channel combination (Figure \ref{fig:yolox}). Depth-wise and dilated convolutions are added to the object decoupling head to capture long-range dependencies of objects in feature maps. Additionally, the alpha loss function ($\alpha$-IoU) is incorporated to optimize small object localization. Experimental results demonstrate that YOLOX++ achieves detection accuracies of 86.8\% for high-voltage-tower bird nests and 96.60\% for power line insulators, outperforming the YOLOX model. On the PASCAL VOC dataset \cite{everingham2010pascal}, YOLOX++ exhibits a 9.30\% improvement in AP50 and a 5\% improvement in APS compared to YOLOX, showcasing its enhanced robustness for small target detection.

\begin{figure*}[htb]
    \centering
    \includegraphics[width=1\linewidth]{fig_6_yolox++.pdf}
    \caption{The simplified network architecture of the proposed YOLOX++ network \cite{bi_yolox_2023}.}
    \label{fig:yolox}
\end{figure*}

Another common cause of power line faults is vegetation, which, if left unmanaged, can grow into power lines, potentially causing short circuits, outages, or even wildfires during dry conditions. Rong et al. \cite{rong_intelligent_2021} proposed an intelligent detection framework for real-time monitoring of vegetation encroachment on power lines. The framework utilized binocular vision sensors mounted on transmission towers to capture images, which were then processed locally. The framework employed Faster R-CNN for vegetation detection, Hough transform for power line detection, and an advanced stereovision algorithm for 3D reconstruction. The advanced stereovision algorithm incorporated calibration optimization and world coordinate system transformation to improve accuracy in large-scale scenes. The authors acknowledged that the accuracy of the proposed method might be affected by complex terrain and weather conditions, which could impact image quality and object detection. Additionally, the computational efficiency of the framework was not explicitly addressed, which could be a concern for real-time monitoring applications. 

% {\color{blue}
To conclude, this section has demonstrated the transformative impact of computer vision and deep learning in revolutionizing power line fault diagnosis. Through the integration of advanced neural network architectures and multi-task models, significant progress has been made across multiple domains: from detecting small insulator defects with accuracies exceeding 97\%, to identifying structural anomalies in towers with mAP rates above 90\%, and recognizing foreign objects with real-time processing capabilities of up to 85 FPS. However, several critical challenges persist. The scarcity of large-scale, publicly available datasets remains a significant bottleneck, with most studies limited to small, proprietary datasets of under 5,000 images. The detection of miniature components like pin-bolts and dampers, which often occupy only a few pixels in aerial imagery, continues to challenge even state-of-the-art models. Environmental factors such as varying lighting conditions, weather effects, and complex backgrounds further complicate reliable fault detection. To address these limitations, promising research directions include: leveraging generative adversarial networks (GANs) for synthetic data augmentation, developing semi-supervised learning approaches to maximize the utility of unlabeled data, exploring attention-based architectures specifically optimized for small object detection, and investigating multi-task learning frameworks that can simultaneously handle different types of faults. The integration of edge computing solutions, as demonstrated in several studies, also shows promise in enabling real-time fault detection while managing computational constraints. These challenges and future research directions regarding power line fault diagnosis are explored in greater detail in Section \ref{sec:challenges}.
% }