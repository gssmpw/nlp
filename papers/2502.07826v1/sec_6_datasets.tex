\section{Publicly Available Datasets}\label{sec:datasets}
Developing deep learning tools for checking power lines automatically relies on having detailed datasets with images of power line parts. These images need to be gathered and labeled by experts, which can be tricky and expensive. It often involves using UAVs or helicopters. Labeling these images is a detailed task, especially since the parts are small and spotting problems need an expert's eye. Therefore, making even a small dataset can take a lot of time and money. However, these deep-learning models need a lot of data to learn well and make accurate predictions. To deal with the lack of enough real images, some researchers have tried making artificial images by changing the background or surroundings of power line parts using computer programs \cite{tao2018detection}. The big issue is that there aren't many large datasets available for everyone to use. The datasets that are created by utility companies are not available to the public due to privacy and data protection laws and regulations. Several companies around the world are obligated to follow rules and regulations regarding data protection such
General Data Protection Regulation (GDPR) \cite{voigt2017eu}
in Europe or California Consumer Protection Act (CCPA) \cite{ccpa} in California. Those laws limit the sharing of data from Utility companies. But there are a few datasets that can be found online in Table \ref{tab:datasets}. For more information on this topic, readers are referred to a study by Ruszczak et al. \cite{ruszczak2023overview} containing a comprehensive review of the power line datasets.

\begin{table*}[htb]
\scriptsize
\caption{Summary of some publicly available power line image datasets}
\label{tab:datasets}
\begin{tabular}{P{0.3} P{0.05} P{0.05} P{0.1} P{0.125} P{0.075} P{0.05} P{0.05}}
\hline
Name & Year & Type & Component & Task & No. of Images & Ref & Download\\
\hline
Chinese Power Line Insulator Dataset (CPLID) & 2018 & RGB  & Insulator & Segmentation, Classification & 848 & \cite{tao2018detection} 
& \href{https://github.com/InsulatorData/InsulatorDataSet}{Link} \\

Powerline Dataset (Infrared-IR and Visible Light-VL & 2019 & RGB, IR  & Conductor, No Conductor & Detection & 8000 & \cite{Yetgin_2019} &  \href{https://data.mendeley.com/datasets/n6wrv4ry6v/8}{Link} \\

Transmission Towers and Power Lines (TTPLA) & 2020 & RGB & Insulator, Tower, Conductor & Segmentation & 1234 & \cite{abdelfattah2020ttpla} 
& \href{https://github.com/R3ab/ttpla_dataset}{Link} \\

Power Transmission Line Dataset & 2021 & RGB & Conductor & Classification & 1044 & \cite{t9qk_cn48_21} & \href{https://ieee-dataport.org/documents/power-transmission-line-dataset}{Link} \\

(Recognizance - 2) Power Lines Detection & 2021 & RGB & Conductor & Classification & 16078 & \cite{recognizance_2} 
& \href{https://www.kaggle.com/competitions/recognizance-2/overview}{Link} \\

STN PLAD: A Dataset for Multi-Size Power Line Assets Detection in High-Resolution UAV Images & 2021 & RGB  & Tower, Insulator, Damper & Segmentation & 2409 & \cite{9643100} 
& \href{https://ieeexplore.ieee.org/document/9643100}{Link} \\

Aerial Power Infrastructure Detection Dataset & 2023 & RGB  & Tower & Segmentation & 3956 & \cite{antonis_savva_2023_7781388} 
& \href{https://zenodo.org/records/7781388}{Link} \\

RSIn-Dataset: An UAV-Based Insulator Detection Aerial Images Dataset and Benchmark & 2023 & RGB  & Insulator & Segmentation & 1887 & \cite{drones7020125} 
& \href{https://github.com/caigouyihao/Rsin-dataset}{Link} \\
\hline
\end{tabular}
\end{table*}
