\begin{abstract}
In recent years, power line maintenance has seen a paradigm shift by moving towards computer vision-powered automated inspection. The utilization of an extensive collection of videos and images has become essential for maintaining the reliability, safety, and sustainability of electricity transmission. A significant focus on applying deep learning techniques for enhancing power line inspection processes has been observed in recent research. A comprehensive review of existing studies has been conducted in this paper, to aid researchers and industries in developing improved deep learning-based systems for analyzing power line data. The conventional steps of data analysis in power line inspections have been examined, and the body of current research has been systematically categorized into two main areas: the detection of components and the diagnosis of faults. A detailed summary of the diverse methods and techniques employed in these areas has been encapsulated, providing insights into their functionality and use cases. Special attention has been given to the exploration of deep learning-based methodologies for the analysis of power line inspection data, with an exposition of their fundamental principles and practical applications. Moreover, a vision for future research directions has been outlined, highlighting the need for advancements such as edge-cloud collaboration, and multi-modal analysis among others. Thus, this paper serves as a comprehensive resource for researchers delving into deep learning for power line analysis, illuminating the extent of current knowledge and the potential areas for future investigation.  
\end{abstract}