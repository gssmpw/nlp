Our first step in generating a benchmark for \economicDomain is to taxonomize this space. Previous work by \citet{ramansteer} developed a taxonomy for economic rationality within strategic domains. Their approach involved identifying foundational principles that define how agents should make decisions in specific environments and then organizing these principles, or ``elements,'' into progressively more complex decision-making scenarios. We adopt a similar hierarchical approach for \benchmark, focusing on organizing economic decision-making principles into structured categories. However, unlike \steer, which assesses decision-making in strategic environments, our focus is assessing how agents make decisions given prices and quantities that are determined by the forces of supply and demand. We call this sub-field non-strategic microeconomics.
% non-strategic microeconomic reasoning, particularly how prices and quantities are determined by the forces of supply and demand.

Two of the settings from \steer remain directly relevant to \economicDomain: \firstParent and \secondParent. As we describe our taxonomy, we begin with these foundational settings. The elements we incorporate from \firstParent---arithmetic, optimization, probability, and logic---are core mathematical skills essential for microeconomic reasoning and are already present in \steer. In \benchmark, we expand this setting by adding elements that test basic calculus, such as single-variable derivatives and linear systems of equations. In \steer, \secondParent focused on testing whether an agent can adhere to the von Neumann-Morgenstern utility axioms when making decisions over a set of alternative choices. We include those axiomatic elements and extend this \parent to include testing the properties of commonly used parameterizations of utility functions in non-strategic microeconomic contexts, such as utility functions with satiation points, monotone preferences, and budget constraints.


Building directly on these foundational settings, we introduce the next \parent, \fifthParent, which tests an agent's ability to optimally exchange time and money for desired goods and services. Elements in this \parent assume that the agent is a price taker, meaning that the agent accepts market prices as given rather than forecasting how a purchase
might move the market. First, we test the agent's ability to derive demand functions consistent with the axioms and functional forms from \secondParent. These foundational elements are useful in assessing whether an agent can make consistent, rational choices in response to market prices. We then include elements testing the agent's ability to determine optimal consumption bundles, decide when to leave the workforce, and conduct comparative statics with demand functions.

\sixthParent tests an agent's ability to decide on the combination of inputs to efficiently produce goods and services to maximize profits. The \parent starts by assessing the agent's ability to identify and analyze basic properties of production functions, such as the relationship between input quantities and output levels. This includes concepts like returns to scale, diminishing marginal returns, and the technological constraints that shape production capabilities. We then test the agent's ability to conduct expenditure minimization and its dual, profit maximization. This involves solving optimization problems where the agent must use marginal analysis to determine the quantity of output that maximizes profit (i.e., minimizes cost).

\seventhParent considers consumers and producers who each reason according to the principles just described to trade with each other. This more complex setting requires an agent to reason about how the aggregated behaviors of consumers and producers lead to market-clearing prices that balance supply and demand. This \parent covers elements such as finding market-clearing prices, computing competitive equilibria, and analyzing the comparative statics of equilibrium in markets where individual actions do not directly impact others.

Our last setting, \eighthParent, tests agents on their ability to evaluate whether equilibria are efficient and to analyze the effects of interventions, such as taxes or price ceilings, on welfare. In this \parent, agents must not only be able to analyze how supply and demand dynamics establish equilibrium prices but also consider how external interventions shift these dynamics and alter the behavior of both consumers and producers. The elements in this \parent can be relatively simple (e.g., compute consumer/producer surplus) or involve detailed counterfactual analysis (e.g., predict how interventions impact prices, the allocation of resources, and welfare outcomes). 

%By organizing our taxonomy this way, we progressively move from individual decision-making principles in \fifthParent and \sixthParent---centered on utility and production functions---to more intricate settings in \seventhParent and \eighthParent that explore market-wide interactions and policy impacts. This structure allows us to examine both the foundational microeconomic behaviors of agents as price takers and the complex, aggregated outcomes that emerge when these agents interact in competitive markets.
