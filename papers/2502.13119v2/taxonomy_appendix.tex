
% \subsection{\firstParent}\label{setting:first}

% \narun{add}

\subsection{\fifthParent}\label{setting:fifth}

We begin by characterizing the space of elements that test an agent's ability to optimally allocate their limited resources to goods and services they desire.  In economics and decision theory, the most primitive approach to describing the preferences of decision-makers is to use a function that maps a set of possible choices to the agent's optimal choice within that set.  Under a set of intuitive assumptions, such as \textit{transitivity} (i.e., if bundle $X$ is preferred to bundle $Y$, and $Y$ is preferred to bundle $Z$, then $X$ must be preferred to $Z$), it becomes possible to ``rationalize'' preferences by instead describing a utility function. This function assigns a real number to each bundle, and the agent selects the bundle with the highest utility.  %Along these lines, decisions can be implemented with utility maximization subject to constraints on budgets.

In this paper, we focus on these ``rationalizable'' preferences, where agent choice can be implemented as utility maximization constrained by prices and income. The solution to these consumer choice problems provides us with, among other things, individual demand functions, which describe the choice of each good or service as a function of prices and income. The individual demand functions for each good are essential when aggregating to the market demand in \nameref{mod:consumer_aggregation}, which in turn is used to find the price in a non-strategic equilibrium.  In addition, we test variations on the framework such as the agents ability to make tradeoffs between the quantity of goods they would need to be able to purchase for an increase in the amount of work they provide for a given wage (i.e., the elasticity of labor supply), as well as cases of choice under uncertainty where the agent is choosing between possible lotteries under rationalizability assumptions required for von Neuman expected utility.

% \jesse{@Narun, if you test vNM stuff in the other one maybe reference it?  If this is too much detail then you can remove things.  To me, at least, any testing of consumer choice needs to start by pointing out that we are talking about rationalizable preferences (which you test later in the "violations").  Lots of preferences may not fulfill this, in which case you can't use simple demand functions/etc..}

% We lay out the section by first asking agents to derive 
% conduct constrained utility maximization and testing whether they can find their utility optimal consumption. We then expand to testing \narun{continue.}

\subsubsection{Properties of Utility Functions}\label{mod:utility_properties}
In this section, we test the ability of the agent to use utility functions as a means to compare preferences over different ``bundles'' of goods or services.  A key feature of economic reasoning in this context is for agents to consider how substitution between different goods in a bundle might achieve the same utility (i.e., map out the ``indifference curves'').  Key tests include correctly distinguishing between substitutes and complements in consumption, and calculating the marginal rate of substitution at a point on an indifference curve.  This logic is essential for both agents acting as a planner as we will see in \Cref{setting:eighth} and when fulfilling the role of choice under budget and income constraints, in \nameref{mod:deriving_demand}.  %In \nameref{mod:avoid_violations} we better test whether the agent acts in ways where utility representations of preferences may not exist.

\begin{el}[Marginal Utility]
    {The ability to calculate marginal utility for different types of demand curves such as quasilinear, Cobb-Douglas, and Leontief.}
\end{el}

\begin{el}[Diminishing Marginal Utility]
    {The ability to recognize the role of diminishing marginal utility in consumption decisions and the role of achieving interior solutions.}
\end{el}

\begin{el}[Marginal Rate of Substitution]\label{el:mrs_utility}
    {The ability to calculate the marginal rate of substitution between two goods in a consumption decision.}
\end{el}

\begin{el}[Tangency and the Marginal Rate of Substitution]
    {The ability to calculate the marginal rate of substitution between two goods in a consumption decision at a given point in the budget constraint as tangent to the indifference curve.}
\end{el}

\begin{el}[Substitutes and Complements]
    {The ability to distinguish between substitutes and complements in consumption decisions.}
\end{el}

\subsubsection{Deriving Demand}\label{mod:deriving_demand}
The \child in this section tests an agent's ability to solve a constrained utility maximization problem to derive a demand function---relying on the results of \nameref{mod:utility_properties}. We test the canonical classes of demand functions, check the duality of Marshallian demand and Hicksian demand, and ask the agent to derive these demand functions from first principles.
% The first \child tests the concept of demand curves in static environments, where prices and incomes are assumed to be constant. We 
% This simplification allows us to clearly see how the quantity demanded of a good varies with its price. The demand curve is a functional representation of this relationship and is typically downward sloping, indicating that as the price of a good decreases, the quantity demanded increases. Understanding these curves helps us grasp the basic principles of consumer choice and market equilibrium.


\begin{el}[Derivation of Marshallian Demand]
    {The ability to calculate the demand curve for a good given a utility function and a budget constraint.}
\end{el}

% \begin{element}{Derivation of Demand Curves for Perfect Substitutions}
% {The ability to calculate the demand curve for a good given a utility function and a budget constraint for perfect substitutes}
% {This is the text body for the illustration.}
% \end{element}

% \begin{element}{Derivation of Demand Curves for Perfect Complements}
% {The ability to calculate the demand curve for a good given a utility function and a budget constraint for perfect complements}
% {This is the text body for the illustration.}
% \end{element}

\begin{el}[Derivation of Hicksian Demand from Expenditure Minimization]\label{el:deriving_hicksian_demand}
    {The ability to calculate the demand curve for a good given a utility function and a budget constraint.}
    % {This is the text body for the illustration.}
\end{el}

\begin{el}[Duality of Hicksian Demand]
    {The ability to recognize that Hicksian demand (expenditure minimization) is dual to maximization in Marshallian Demand.}
\end{el}




\subsubsection{Comparative Statics of Demand}\label{mod:properties_demand}

This \child considers how agents reason about changes in prices or income, and their effects on the quantity of each good they would purchase. We test the classic law of demand, different types of goods (e.g., normal, inferior, and Giffen), and derive Engel curves from first principles.  The key tests are to ensure the agent rationally responds to changes in relative prices, and investigate their substitution between goods in a bundle.  In practice, these tests involve comparative statics of the $\mathrm{argmax}$ from the utility maximization of the previous section on \nameref{mod:deriving_demand}---i.e., using an Envelope theorem and perturbing prices or income.

\begin{el}[Law of Demand]
    {The ability to calculate the change in demand with the change in price for normal goods.}
\end{el}

\begin{el}[Price Elasticity of Demand]\label{el:price_elasticity_demand}
    {The ability to calculate the price elasticity of demand for a good given a utility function and a budget constraint.}
\end{el}

\begin{el}[Consumption Changes]
    {The ability to change the relative expenditures on goods given changes in relative prices with ordinary or Giffen goods.}
\end{el}

% \begin{el}[Consumption Changes with Giffen Goods]
%     {The ability to recognize that the demand for a good can decrease with decreasing prices.}
% \end{el}



% Of course, an agent's consumption of a good can be affected by external factors, such as their income, the prices of other goods, or the kind of the good. \tl{Why is this sentence here?}
% In this subsection, we test how changes in external factors, such as income or prices of other goods, influence the demand for a particular good. This analysis, known as comparative statics, helps us understand the sensitivity of demand to various economic variables. By comparing different static scenarios, we can predict how shifts in the economic environment affect consumer behavior

\begin{el}[Engel Curves]
    {The ability to calculate the Engel curve for a good given a utility function and a budget constraint.}
\end{el}

\begin{el}[Income Elasticity of Demand]
    {The ability to calculate the income elasticity of demand for a good given a utility function and a budget constraint.}
\end{el}

% \subsubsection{Demand Curves in Stochastic Environments}
% Real-world economic environments are rarely static; they are characterized by uncertainty and variability. This subsection introduces demand curves in stochastic (or random) environments, where factors such as prices and income can fluctuate unpredictably. By studying demand under uncertainty, we can better understand how consumers make decisions when faced with risks and uncertainties. This is crucial for modeling real-world scenarios more accurately.
% \begin{element}
    
% \end{element}


\subsubsection{Labor Supply}\label{mod:labor_supply}
While the proceeding elements tested tradeoffs in choices of bundles with different goods, services (in \nameref{mod:deriving_demand} and over lotteries in \nameref{mod:portfolio_choice}), often agents need to make a choice trading off between leisure and consumption. The elements in this \child test an agent's ability to optimally make that tradeoff by balancing the consumption goods required to compensate for decreased leisure---which leads to the labor supply elasticity central to many branches of economics.   Since goods must be purchased, agents will consider the relative wage from additional work compared to the price of goods.  This leads us to be able to test an agent's ability to distinguish real from nominal prices.

\begin{el}[Deriving Labor Supply]
    {The ability to calculate the labor supply curve given specific preference parameterizations such as separable preferences or homothetic preferences.}
\end{el}

% \begin{el}[Inelastic Labor Supply]
%     {The ability to recognize that labor supply will not adjust if there is no disutility of labor.}
% \end{el}

\begin{el}[Labor Supply Elasticity]
    {The ability to calculate the elasticity of labor supply.}
\end{el}

\begin{el}[Marginal Rate of Substitution in Labor Supply]
    {The ability to calculate the marginal rate of substitution between consumption and leisure in a labor supply decision.}
\end{el}

% \begin{el}[Labor Supply with Homothetic Preferences]
%     {The ability to calculate the labor supply given a wage and a utility function.}
% \end{el}

% \begin{el}[Labor Supply Comparative Statics]
%     {The ability to calculate how labor supply changes with changes in wages and prices given an elasticity.}
% \end{el}


\subsubsection{Dynamic Consumption Decisions}\label{mod:portfolio_choice}
Individuals often face decisions about how to trade off more consumption today at the cost of additional debt and less consumption in the future, and how best to plan for consumption with various contingencies with the future is uncertain.  Among other applications, this provides a formal model of how to best choose a mixture of financial assets---i.e., portfolios. Consequently, this subsection tests intertemporal consumption choices, optimal portfolio choice---which involves selecting a mix of assets that maximizes expected utility given the risks and returns associated with each asset. Understanding portfolio choice helps explain how consumers manage risk and make investment decisions, which is vital for financial planning and economic stability.

\begin{el}[Price of Risk with Mean-Variance Utility]
{The ability to calculate the price of risk for a mean-variance utility function.}
\end{el}

\begin{el}[State-Contingent Consumption]
    {The ability to calculate the optimal consumption given a utility function and a set of state-contingent consumption bundles.}
\end{el}

\begin{el}[Arbitrage]
    {The ability to recognize and execute arbitrage opportunities given two goods and prices you can resell.}
\end{el}

\begin{el}[Optimal Portfolio Choice with Bid-Ask Spreads]
    {The ability to calculate the optimal portfolio given bid-ask spreads.}
\end{el}

\begin{el}[Exponential Discounting]
    {The ability to exponentially discount future rewards or costs.}
\end{el}

\begin{el}[Intertemporal Consumption Smoothing]\label{el:intertemporal_consumption_smoothing}
    {The ability to calculate a smoothed consumption path and determine whether it is preferred to a non-smoothed path.}
\end{el}


\subsection{\sixthParent}\label{setting:sixth}

In the previous section, we derived how an agent facing a set of prices would choose the quantity demanded of each good or service to maximize their utility function.  We also tested the amount of time that an agent might choose to wok (i.e., the quantity of labor supplied) given market wages---where the agent trades off the additional goods they might purchase against the lost leisure time they must forgo.  Here, we look at the other side of the market and test an agent's ability to operate a production technology to maximize profits.  Facing market prices for all production factors (e.g., wages and the capital) and the market price of the good or service they produce, the agent chooses the quantity of each factor of production and the total output.  Parallel to \fifthParent, in \nameref{mod:properties_production} we first test general properties of production functions to ensure the agent can reason about substitution between factors, economies of scale in production, etc.  Then in \nameref{mod:deriving_factor_demand}  we solve the firms optimal profit maximization problem to determine the optimal choice of factors of production and output given a set of market prices.  Finally, in \nameref{mod:production_statics} we test the agents ability to reason about comparative statics on prices and their impact on factor demand and firm output. 
% The resulting factor demand functions (e.g., labor demand from the firm) will then be matched with the factor supply functions (e.g., labor supply from the consumption decision) such as in \nameref{mod:labor_supply} when we test the agent's ability to reason about the forces of supply and demand in \seventhParent. \tl{Id cut the last sentence here, its repetitive with the text in \seventhParent}

\subsubsection{Properties of Production Functions}\label{mod:properties_production}
Production functions in these environments take continuous inputs of each factor, which lets us test an agent's ability to conduct marginal thinking when choosing the composition of inputs.  For example, by knowing the hourly wage of hiring an additional worker, the additional output the worker might produce using the particular production process, and the price they can sell the firm's output, they can decide whether hiring the additional worker is profitable.  In the absence of prices, this section tests basic decision making of the agent for understanding substitution between factor of production, marginal products for each input, and the understanding of the returns to scale of a production process.

\begin{el}[Marginal Products]
    {The ability to calculate separate marginal products for a production function with multiple inputs (e.g., labor and capital).}
\end{el}

\begin{el}[Input Price Elasticity]\label{el:input_elasticity}
    {The ability to calculate the responsiveness of output to a proportional change in a specific input's cost, holding all other inputs constant.}
\end{el}

\begin{el}[Output Elasticity]\label{el:output_elasticity}
    {The ability to calculate the output elasticity of an input in a production function.}
\end{el}

\begin{el}[Elasticity of Substitution]
    {The ability to calculate the marginal elasticity of substitution between inputs in a production function.}
\end{el}

\begin{el}[Diminishing Marginal Products]\label{el:diminishing_marginal_product}
    {The ability to calculate the diminishing marginal products for a production function with multiple inputs.}
\end{el}

\begin{el}[Average and Marginal Costs]\label{el:average_cost}
    {The ability to calculate average and marginal costs given a production function and input prices, and use them to determine scale.}
\end{el}

\begin{el}[Returns to Scale]\label{el:returns_to_scale}
    {The ability to determine the proportional change in output resulting from a proportional change in all inputs in a production function.}
\end{el}


\subsubsection{Deriving Factor Demand}\label{mod:deriving_factor_demand}
This \child tests the agent's ability to act in the role of a profit maximizer in non-strategic situations where they take as given the price which they could sell goods they produce, and must pay for inputs to their production process at market rates (e.g., a competitive wage).  Whereas in \nameref{mod:deriving_demand}, the agent was solving a utility maximization problem subject to a budget constraint, here they solve a profit maximization problem constrained by a production function.  We test decisions on the quantity and composition of inputs, and the quantity of output for canonical production functions such as Cobb-Douglas and Leontief production functions given the agent's understanding of production functions from \nameref{mod:properties_production}.  The agent is asked to derive the factor demand functions from first principles from profit maximization and test their ability to reason with the dual cost-minimization formulation---analogous to the Hicksian vs. Marshallian demand of \nameref{mod:deriving_demand}.

\begin{el}[Profit Maximization]\label{el:profit_max}
    {The ability to calculate the optimal input bundle for a firm given a production function and input prices. Examples of given production functions: Cobb-Douglas, Leontief, Perfect Substitutes, CES production, CRS production, fixed costs.}
\end{el}


\begin{el}[Expenditure Minimization]
    {The ability to calculate the optimal input bundle for a firm given a production function and input prices.}
\end{el}

\begin{el}[Duality of Profit Maximization and Expenditure Minimization]
    {The ability to recognize that profit maximization is dual to expenditure minimization in production decisions and achieve consistent solutions.}
\end{el}


\subsubsection{Comparative Statics with Production}\label{mod:production_statics}


This \child considers how agents reason about changes in the prices at which they can sell their goods, as well as changes in the costs of producing those goods.  In particular, we can test how this affects their optimal choice of inputs to their production process (e.g., how many people to hire or robots to lease). We test comparative statics on the prices of inputs to the production function, changes to the underlying production technology, and substitution between goods for classic production functions such as Cobb-Douglas and Leontief.  Analogous to the relationship between \nameref{mod:deriving_demand} and \nameref{mod:properties_demand}, these tests involve comparative statics of the $\mathrm{argmax}$ from the profit maximization of \nameref{mod:deriving_factor_demand}---i.e., using an Envelope theorem and perturbing factor prices.

\begin{el}[Price Elasticity of Supply] 
{The ability to calculate the price elasticity of supply for a good given a production function and input prices.}
\end{el}

\begin{el}[Shephard's Lemma]
    {The ability to calculate factor demands given a cost function using the derivatives with respect to prices.}
\end{el}

\begin{el}[Input Price Elasticity]
    {The ability to calculate how the optimal input bundle changes with changes in input prices for a given production function.}
\end{el}

\begin{el}[Total Factor Productivity]
    {The ability to calculate total factor productivity given a production function and input prices}
\end{el}


\subsubsection{Dynamic Production Decisions}\label{mod:dynamic_production}
While \nameref{mod:deriving_factor_demand} tested the ability of agents to make static (i.e., within-period) decisions on the mix of input factors to maximize profits, many producer problems are inherently dynamic.  For example, we can test if an agent can optimally choose the amount of capital to purchase given forecasts of future consumer demand and prices or choose how much to adjust the labor force in cases when labor is difficult to relocate due to frictions such as hiring and firing costs.  Finally, agents are tested on their ability to make optimal entry and exit decisions based on their forecasted profits in an evolving market.


\begin{el}[Dynamic Profit Maximization]\label{el:dynamic_profit_max}
    {The ability to calculate the optimal investment decision given a production function and input prices.}
\end{el}

\begin{el}[Entry and Exit Decisions]
    {The ability to calculate the optimal entry and exit decisions given a production function and fixed costs.}
\end{el}





\subsection{\seventhParent}\label{setting:seventh}
This \parent tests the core logic of the relationship between supply-and-demand and prices, building on the tests of optimal behavior in \cref{setting:sixth} and \cref{setting:fifth}.  Economists refer to ``general equilibrium'' as the process where equilibrium prices and quantities emerge with a large number of non-strategic, price-taking market participants interact.  Unlike the strategic models found in \steer, the assumption is that the market interactions that lead to this equilibrium occur through an unspecified process that clears markets (i.e., a ``Walrasian auctioneer'' or ``invisible hand'').
% \footnote{The process through which the strategic decisions of agents, as tested in \thirdParent, would lead to the equilibrium in \seventhParent are not being tested.  See \url{https://assets.cambridge.org/97805216/43306/frontmatter/9780521643306_FRONTMATTER.pdf}, Rubinstein and Wolinsky (1990) and others.} 

In particular, for non-strategic settings, all market participants take prices as given and choose the quantity demanded or supplied in each market.  For example, consumers jointly decide on the quantity demanded of goods and services given relative prices, and the quantity of labor supplied given a wage.  Simultaneously, producers choose the quantity supplied of the good and the demand of each factor of production.  With a large number of non-strategic market participants we can test the agents ability aggregate all of their supply and demand functions to calculate a market-level supply and demand.  Finally, given the aggregated supply and demand functions for each market, we can test whether an agent can find the market clearing price where supply is equal to demand in equilibrium---given their internal model of all the market participants.

 In this section, we organize by markets rather than by the role of a decision maker, as in the previous sections.  For example, in the goods market we first ensure agents understand how individual demand functions from \nameref{mod:deriving_demand} aggregate to a market demand function for the good given a price, then that the agent understands how to aggregate the output from each producer at a given price from \nameref{mod:production_statics}, and finally that the agent is able to calculate the price which would equate demand and supply and clear the market in a non-strategic setting.  Factor markets are treated similarly.

Finally, given a system of equations that defines an equilibrium price we can perturb primitives (e.g., technological factors, distortions on decisions such as tax rates, or exogenous prices not determined in equilibrium) to see how the market clearing price would respond.  That is an essential tool for agents to be able to reason about the impact of interventions and distortions in \Cref{setting:eighth}.

\subsubsection{Consumer Goods Market Aggregation}\label{mod:consumer_aggregation}
The market clearing prices in general equilibrium arise from the separate market-level demand and supply curves, which sums the demand or supply across all market participants at a given price.  Here we test the aggregation of demand functions derived from individual preferences, as in \nameref{mod:deriving_demand} and  \nameref{mod:properties_demand}, to a market demand function that summarizes the total quantity demanded across all agents at a given price.  Central to the tests is to verify that the agent can aggregate the demands of market participants with heterogeneous preferences.  On the other side of the market, we test if the agent can aggregate the ``supply functions'' resulting from the optimal choice of factors in \nameref{mod:deriving_factor_demand} and \nameref{mod:production_statics}.

\begin{el}[Aggregation of Consumer Demand] \label{el:agg_consumer_demand}
    {The ability to calculate the aggregate demand for a good given primitives of demand into expenditure shares.}   
\end{el}


% \begin{el}[Aggregation of Demand of Heterogenous Agents]
%     {The ability to solve for two sets of demand functions and aggregate up to the market demand for a discrete number of agent types.}
% \end{el}


% \narun{@Jesse, this was an element, and there was not a corresponding one outside of stochastic choice. Is there no non-stochastic version of this question?}
% \jesse{@Narun: Aggregation of stochastic choice is a particular example because it is a discrete choice of goods, which ends up aggregated into a continuous market demand function.  The other elements above here are appropriate when there is an aggregation of continuous demand functions to a continuous market demand}
% \begin{el}[Aggregation of Stochastic Choice with an Outside Good]
%     {The ability to correctly calculate market shares where not all agents consume the good.}
% \end{el}

% \jesse{@narun: this is an important special case where one of the discrete number of goods is ``don't buy anything''!.}
% \begin{element}{Comparative Statics of Aggregated Consumer Demand with Stochastic Choice}
%     {}
%     {}
% \end{element}

% \narun{This element was in this section but seems odd here? @Narun: Yes, any comparative statics should be in that other section.  Maybe this element could be commentd out} \jesse{@Narun: Yes, this shoudl be in the omparative statics section maybe?  Lets remove for now.}

\begin{el}[Aggregation of Offer Curve for the Good]
    {The ability to calculate the aggregate supply of a good given primitives of supply into production functions.}
\end{el}

\subsubsection{Factor Market Aggregation}\label{mod:factor_aggregation}
As with the case of the goods market in \nameref{mod:consumer_aggregation} the market demand and supply for factors of production are essential to find the market clearing price.  For example, we test whether the agent can aggregate the individual labor supply curve decisions from market participants who work at a particular wage, following \nameref{mod:labor_supply}, into a market labor supply curve.  On the other side of the market, we test whether the agent can aggregate the labor demand in \nameref{mod:deriving_factor_demand} from producers into a market labor demand curve.  The same tests are essential for all factors of production, including capital.


\begin{el}[Aggregation of Labor Demand]
    {The ability to calculate the aggregate demand for labor given primitives of demand into expenditure shares.}
\end{el}

\begin{el}[Aggregation of Capital Demand]
    {The ability to calculate the aggregate demand for capital given primitives of demand into expenditure shares.}
\end{el}

% \narun{Shouldn't this be in the consumer goods market? I understand that this is a supply function but if we are mirroring the previous sections should this better fit with consumption/consumer market decisions?}\jesse{@Narun: the logic here is to go market-by-market to get the demand and supply curves of each.  I put this in the summary of the whole section to make this clear.  Note that the "offer curve" of the good is in the previous section.  Of course, the supply/demand decisions of all market participants are all entwined.  The best way to think about this is that we should be organizing around ``prices''.  A price = a market.}
\begin{el}[Aggregation of Labor Supply]
    {The ability to calculate the aggregate supply of labor given primitives of supply into production functions.}
\end{el}


\begin{el}[Aggregation of Fixed Factor Supply]
    {The ability to calculate the aggregate supply of capital given primitives of supply into production functions.}
\end{el}

% \subsubsection{Intermediate Goods and Supply Chain Market Aggregation}\label{mod:supply_chain_aggregation}

% \narun{I do not really love the name of this subsection... Should think about this more.}\jesse{@Narun.  Lets drop this section for now.  We don't test intermediate goods in production in the proceeding sections, so this isn't very meaningful.  Commented out.}


% \begin{element}{Aggregation of Intermediate Goods Demand}
%     {Element body text.}
%     {Illustration body text.}
% \end{element}


% \begin{element}{Aggregation of a Leontief Production Network}
%     {Element body text.}
%     {Illustration body text.}
% \end{element}

% \narun{are there any others to include here?}

\subsubsection{Prices in Static Market Equilibrium}\label{mod:static_equilibrium}
In this \parent we test the agent's ability to reason about how prices emerge in non-strategic setting as a process of equating supply and demand, which in turn relies on their ability to aggregate those market demand functions from consumer and producer behavior.

More specifically, the core logic of general equilibrium is to find the equilibrium price by taking the aggregated demand and supply functions for each market  and find the prices which would equate demand and supply.  For example, the supply and demand functions for the good, as a function of the price, in \nameref{mod:consumer_aggregation}; or the supply and demand functions for factors of production, as a function of factor prices in \nameref{mod:factor_aggregation}.  This is done market by market, taking all other prices as given---which requires the agent reason through comparative statics of the solution to a system of equations while keeping everything else fixed.

\begin{el}[Find Equilibrium Price]\label{el:find_eq_price}
    {The ability to calculate the equilibrium prices given a production function and a demand function.}
\end{el}



\begin{el}[Factor Shares in Equilibrium]
    {The ability to calculate the factor shares in a competitive equilibrium given a production function and input prices.}
\end{el}

% \narun{I left off some of the elements because they seemed like variants of the first element of this section (i.e., inelastic supply, perfectly elastic supply). If there are other elements to add here please do so}\jesse{@Narun: I think the quetsion is to the extent you want to split up "Equilibriumm Price" into a whole bunch of separate elements vs. illustrations.  There are all sorts of variations, corner cases, etc.  The level of granularity of other sections suggests that these should probably not all end up as just illustrations.}

\subsubsection{Comparative Statics of Equilibrium Prices}\label{mod:comparative_equilibrium}
Here, we test whether agents can reason about how prices and allocations (e.g., labor, capital, and goods) would respond to changes in the environment.  The canonical tests are to see how changes in model primitives (e.g., productivity of the production process) or exogenous forces from outside the model (e.g., impact of weather), change the equilibrium price and allocations of labor, capital, etc. that would clear the market and equate demand and supply.

\begin{el}[Comparative Statics with Total Factor Production Shocks]\label{el:tfp_shocks}
    {The ability to calculate how equilibrium prices change with changes in input prices for a Cobb-Douglas production function.}
\end{el}

\begin{el}[Comparative Statics with Inelastic or Perfectly Elastic Supply]
    {The ability to calculate how equilibrium prices change with changes in input prices for a production function with inelastic or perfectly elastic supply.}
\end{el}

% \subsubsection{Dynamic Market Equilibrium}\label{mod:dynamic_equilibrium}
% \narun{Are there at least 1-2 elements we can add here?}\jesse{@Narun: I think we should temporarily pull this section until we are able to get more elements on dynamic decisiosn into the earlier sections.  We also need a dynamic aggregation there.  Too many elements to add to fill things in.  Commenting out for now}



\subsection{\eighthParent}\label{setting:eighth}
In \seventhParent, we tested an agents ability to reason about equilibrium prices and quantities arising from supply and demand decisions in a non-strategic setting.  Although preferences were reflected in the underlying supply and demand functions themselves (i.e., utility maximization in the consumption decisions of \fifthParent and profit maximization in the production decisions of \sixthParent), the equilibria in \seventhParent do not necessarily reflect broader social preferences.

However, we can still ask whether the resulting ``allocations'' (i.e., the physical goods produced and how they are distributed to individuals, the amount of hours worked, and the physical capital installed) from the ``invisible hand'' in \seventhParent compare to a alternative ways of allocating resources which may directly take social preferences into account.  A central result of economics in non-strategic settings is that absent market imperfections and market power (i.e., when self-interested agents cannot directly manipulate prices because they are too small) the competitive equilibria of \seventhParent typically yields the same allocations a benevolent planner might choose.

In this section, we consider how a social planner would evaluate the underlying welfare, efficiency, and inequality that comes about in non-strategic equilbria with prices derived from equating supply and demand.  This leads to testing the ability of the agent to evaluate Pareto efficiency, consider the welfare theorems, evaluate Pigouvian externalities, and weigh the welfare impact of various market interventions which change the equilibria derived in \seventhParent.  

\subsubsection{Welfare and Decentralization}\label{mod:welfare}
In this section, we test whether the agent can determine cases where the the competitive equilibrium they calculate would yield the same distribution of resources and consumer welfare as that of a benevolent social planner directly making the consumption and production decisions of all agents directly (also known as the ``Welfare Theorems").  In cases where the supply-and-demand relationships lead to the same results as those of a planner, the competitive equilibrium and its prices are said to ``decentralize'' the problem of a social planner.  We then test that the agent recognizes cases where the welfare theorems fail, and can calculate the degree of welfare loss due to the distortions.

\begin{el}[First Welfare Theorem]\label{el:welfare_theorem_1}
    {The ability to recognize that a competitive equilibrium is Pareto efficient.}
\end{el}

\begin{el}[Second Welfare Theorem]\label{el:welfare_theorem_2}
    {The ability to recognize that any Pareto efficient allocation can be achieved as a competitive equilibrium with prices.}
\end{el}

\begin{el}[Consumer Surplus]\label{el:consumer_surplus}
    {The ability to calculate consumer surplus given a demand curve and a price.}
\end{el}

\begin{el}[Producer Surplus]\label{el:producer_surplus}
    {The ability to calculate producer surplus given a supply curve and a price.}
\end{el}

\begin{el}[EFficient Surplus]
    {The ability to calculate the total surplus in a competitive equilibrium and recognize that it is maximized in the competitive equilibrium.}
\end{el}

\begin{el}[Deadweight Loss of a Monopoly]\label{el:deadweight_loss}
    {The ability to calculate the deadweight loss of a monopoly given a demand curve and a supply curve.}
\end{el}


\subsubsection{Welfare Analysis of Market Equilibrium}\label{mod:analysis_equilibrium}
In this section, we focus on the agent's ability to evaluate welfare implications of various forms of market equilibrium, particularly how different policies and distortions impact overall efficiency and resource allocation. The agent is tested on their understanding of how different interventions---such as taxes, subsidies, and price controls---affect welfare outcomes, and their ability to distinguish between distortionary and non-distortionary policies.

\begin{el}[Identify Non-Distortionary Taxes]
    {The ability to identify taxes which do not distort the allocation of resources.}
\end{el}

\begin{el}[Irrelevance of Tax Incidence]
    {The ability to recognize that the incidence of a tax does not depend on who is legally responsible for paying the tax.}
\end{el}

\begin{el}[Labor Supply Distortions]
    {The ability to determine the extent that labor taxes will distort labor supply and change aggregates and prices.}
\end{el}

\begin{el}[Capital Market Distortions]\label{el:cap_market_distortions}
    {The ability to identify that taxing a fixed factor is non-distortionary, but distorts with dynamic accumulation.}
\end{el}
