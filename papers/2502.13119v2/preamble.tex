\usepackage[most]{tcolorbox}
\usepackage{lipsum} % for sample text
\usepackage{pgfkeys}
\usepackage{multirow}
\usepackage{colortbl}            % For advanced table coloring
\usepackage{hyperref}            % For hyperlinks
\usepackage{booktabs}            % For better table rules
\usepackage{enumitem}            % For customizing lists
% \usepackage{geometry}            % For page margins
\usepackage{fontawesome}         % For icons (optional)
\usepackage{float}               % for figures

\def\CurrentAudience{arxiv}
\usepackage{multiaudience}

\SetNewAudience{arxiv}
\SetNewAudience{iclr}

% Define custom colors for settings
\definecolor{setting1}{HTML}{F4CCCC}
\definecolor{setting2}{HTML}{D9EAD3}
\definecolor{setting3}{HTML}{CFE2F3}
\definecolor{setting4}{HTML}{FFF2CC}
\definecolor{setting5}{HTML}{D9D2E9}

\usepackage{tabularx}
\usepackage{enumitem}

\setlist{nolistsep}
\definecolor{green}{HTML}{66FF66}
\definecolor{myGreen}{HTML}{009900}

\usepackage{amsthm}


\usepackage[nameinlink]{cleveref}

\usepackage{nicefrac}

\usepackage{bbm}

% Unorthodox characters e.g., in ref.bib
\usepackage[T1]{fontenc}
\usepackage[utf8]{inputenc}


\let\1\undefined % This is important for the outlines package to not error.
\usepackage{outlines}

\let\models\undefined % This is important for the plural of model

\usepackage{wrapfig}
\usepackage{enumitem}
% \usepackage{markdown}
% % Define custom styles for Markdown headers
% \markdownSetup{
%   renderers = {
%     header-three = {\textbf{\Large}, \par}, % Custom style for ### headers
%     header-two = {\textbf{\LARGE}, \par},  % Custom style for ## headers
%     header-one = {\textbf{\Huge}, \par},   % Custom style for # headers
%   }
% }


\usepackage{array}
\usepackage{longtable}
\usepackage[tableposition=below]{caption}
\captionsetup[longtable]{skip=1em}
\usepackage{caption}

% Define the theorem styles
\newtheoremstyle{elementstyle}
  {3pt} % Space above
  {3pt} % Space below
  {\itshape} % Body font
  {} % Indent amount
  {\bfseries} % Theorem head font
  {.} % Punctuation after theorem head
  { } % Space after theorem head
  {\thmname{#1}\thmnumber{ #2}\thmnote{ (#3)}} % Theorem head spec (can be left empty, meaning `normal`)

\newtheoremstyle{illustrationstyle}
  {3pt} % Space above
  {3pt} % Space below
  {\itshape} % Body font
  {} % Indent amount
  {\bfseries} % Theorem head font
  {.} % Punctuation after theorem head
  { } % Space after theorem head
  {\thmname{#1}\thmnumber{ #2}} % Theorem head spec (can be left empty, meaning `normal`)

% Define the theorem environments
\theoremstyle{elementstyle}
\newtheorem{el}{Element}[section]

\theoremstyle{illustrationstyle}
\newtheorem{illustration}{Illustration}[section]

% Reset counters at the start of each section
\makeatletter
\@addtoreset{el}{section}
\@addtoreset{illustration}{section}
\makeatother

% Initialize counters to start at 1 for the first section
\AtBeginDocument{%
  \setcounter{el}{0}
  \setcounter{illustration}{0}
}


\usepackage{siunitx}  % This is for pretty printing numbers
\sisetup{
  group-separator = {,},
  group-minimum-digits = 3,
}
