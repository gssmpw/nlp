\documentclass[lettersize,journal]{IEEEtran}

% basic
%\usepackage{color,xcolor}
\usepackage{color}
\usepackage{epsfig}
\usepackage{graphicx}
\usepackage{algorithm,algorithmic}
% \usepackage{algpseudocode}
%\usepackage{ulem}

% figure and table
\usepackage{adjustbox}
\usepackage{array}
\usepackage{booktabs}
\usepackage{colortbl}
\usepackage{float,wrapfig}
\usepackage{framed}
\usepackage{hhline}
\usepackage{multirow}
% \usepackage{subcaption} % issues a warning with CVPR/ICCV format
% \usepackage[font=small]{caption}
\usepackage[percent]{overpic}
%\usepackage{tikz} % conflict with ECCV format

% font and character
\usepackage{amsmath,amsfonts,amssymb}
% \let\proof\relax      % for ECCV llncs class
% \let\endproof\relax   % for ECCV llncs class
\usepackage{amsthm} 
\usepackage{bm}
\usepackage{nicefrac}
\usepackage{microtype}
\usepackage{contour}
\usepackage{courier}
%\usepackage{palatino}
%\usepackage{times}

% layout
\usepackage{changepage}
\usepackage{extramarks}
\usepackage{fancyhdr}
\usepackage{lastpage}
\usepackage{setspace}
\usepackage{soul}
\usepackage{xspace}
\usepackage{cuted}
\usepackage{fancybox}
\usepackage{afterpage}
%\usepackage{enumitem} % conflict with IEEE format
%\usepackage{titlesec} % conflict with ECCV format

% ref
% commenting these two out for this submission so it looks the same as RSS example
% \usepackage[breaklinks=true,colorlinks,backref=True]{hyperref}
% \hypersetup{colorlinks,linkcolor={black},citecolor={MSBlue},urlcolor={magenta}}
\usepackage{url}
\usepackage{quoting}
\usepackage{epigraph}

% misc
\usepackage{enumerate}
\usepackage{paralist,tabularx}
\usepackage{comment}
\usepackage{pdfpages}
% \usepackage[draft]{todonotes} % conflict with CVPR/ICCV/ECCV format



% \usepackage{todonotes}
% \usepackage{caption}
% \usepackage{subcaption}

\usepackage{pifont}% http://ctan.org/pkg/pifont

% extra symbols
\usepackage{MnSymbol}


%%%%%%%%%%%---SETME-----%%%%%%%%%%%%%
%replace @@ with the submission number submission site.
\newcommand{\thiswork}{INF$^2$\xspace}
%%%%%%%%%%%%%%%%%%%%%%%%%%%%%%%%%%%%


%\newcommand{\rev}[1]{{\color{olivegreen}#1}}
\newcommand{\rev}[1]{{#1}}


\newcommand{\JL}[1]{{\color{cyan}[\textbf{\sc JLee}: \textit{#1}]}}
\newcommand{\JW}[1]{{\color{orange}[\textbf{\sc JJung}: \textit{#1}]}}
\newcommand{\JY}[1]{{\color{blue(ncs)}[\textbf{\sc JSong}: \textit{#1}]}}
\newcommand{\HS}[1]{{\color{magenta}[\textbf{\sc HJang}: \textit{#1}]}}
\newcommand{\CS}[1]{{\color{navy}[\textbf{\sc CShin}: \textit{#1}]}}
\newcommand{\SN}[1]{{\color{olive}[\textbf{\sc SNoh}: \textit{#1}]}}

%\def\final{}   % uncomment this for the submission version
\ifdefined\final
\renewcommand{\JL}[1]{}
\renewcommand{\JW}[1]{}
\renewcommand{\JY}[1]{}
\renewcommand{\HS}[1]{}
\renewcommand{\CS}[1]{}
\renewcommand{\SN}[1]{}
\fi

%%% Notion for baseline approaches %%% 
\newcommand{\baseline}{offloading-based batched inference\xspace}
\newcommand{\Baseline}{Offloading-based batched inference\xspace}


\newcommand{\ans}{attention-near storage\xspace}
\newcommand{\Ans}{Attention-near storage\xspace}
\newcommand{\ANS}{Attention-Near Storage\xspace}

\newcommand{\wb}{delayed KV cache writeback\xspace}
\newcommand{\Wb}{Delayed KV cache writeback\xspace}
\newcommand{\WB}{Delayed KV Cache Writeback\xspace}

\newcommand{\xcache}{X-cache\xspace}
\newcommand{\XCACHE}{X-Cache\xspace}


%%% Notions for our methods %%%
\newcommand{\schemea}{\textbf{Expanding supported maximum sequence length with optimized performance}\xspace}
\newcommand{\Schemea}{\textbf{Expanding supported maximum sequence length with optimized performance}\xspace}

\newcommand{\schemeb}{\textbf{Optimizing the storage device performance}\xspace}
\newcommand{\Schemeb}{\textbf{Optimizing the storage device performance}\xspace}

\newcommand{\schemec}{\textbf{Orthogonally supporting Compression Techniques}\xspace}
\newcommand{\Schemec}{\textbf{Orthogonally supporting Compression Techniques}\xspace}



% Circular numbers
\usepackage{tikz}
\newcommand*\circled[1]{\tikz[baseline=(char.base)]{
            \node[shape=circle,draw,inner sep=0.4pt] (char) {#1};}}

\newcommand*\bcircled[1]{\tikz[baseline=(char.base)]{
            \node[shape=circle,draw,inner sep=0.4pt, fill=black, text=white] (char) {#1};}}


\hyphenation{op-tical net-works semi-conduc-tor IEEE-Xplore}
% updated with editorial comments 8/9/2021


\usepackage{hologo}
\usepackage{listings}
\lstset{breaklines,basicstyle=\small,columns=fullflexible,basicstyle=\ttfamily,language={[plain]TeX}}

\begin{document}
	
\pagestyle{empty}
	
% for arXiv publication with appropriate copyright notice
\twocolumn[
\begin{@twocolumnfalse}
	\Huge {IEEE copyright notice} \\ \\
	\large {\copyright\ 2024 IEEE. Personal use of this material is permitted. Permission from IEEE must be obtained for all other uses, in any current or future media, including reprinting/republishing this material for advertising or promotional purposes, creating new collective works, for resale or redistribution to servers or lists, or reuse of any copyrighted component of this work in other works.} \\ \\
	
	{\Large Published in \emph{IEEE Transactions on Intelligent Vehicles}, 29 July 2024.} \\ \\
	
	Cite as:
	
	\vspace{0.1cm}
	\noindent\fbox{%
		\parbox{\textwidth}{%
			L.~Ullrich, M.~Buchholz, K.~Dietmayer, and K.~Graichen, ''Expanding the Classical V-Model for the Development of Complex Systems Incorporating AI,''
			in \emph{IEEE Transactions on Intelligent Vehicles}, 29 July 2024, pp. 1--15, doi: 10.1109/TIV.2024.3434515.
		}%
	}
	\vspace{2cm}
	
\end{@twocolumnfalse}
]

\noindent\begin{minipage}{\textwidth}
	
\hologo{BibTeX}:
\footnotesize
\begin{lstlisting}[frame=single]
@article{ullrich2024expanding,
	title={Expanding the Classical V-Model for the Development of Complex Systems Incorporating AI},
	author={Ullrich, Lars and Buchholz, Michael and Dietmayer, Klaus and Graichen, Knut},
	journal={IEEE Transactions on Intelligent Vehicles},
	year={2024},
	pages={1--15},
	doi={10.1109/TIV.2024.3434515},
	publisher={IEEE}
}
\end{lstlisting}
\end{minipage}


\maketitle
\setcounter{page}{1}
%\thispagestyle{empty}
%\pagestyle{empty}


\begin{abstract}\label{00_Abstract}
Research in the field of automated vehicles, or more generally cognitive cyber-physical systems that operate in the real world, is leading to increasingly complex systems. Among other things, artificial intelligence enables an ever-increasing degree of autonomy. In this context, the V-model, which has served for decades as a process reference model of the system development lifecycle is reaching its limits. To the contrary, innovative processes and frameworks have been developed that take into account the characteristics of emerging autonomous systems. To bridge the gap and merge the different methodologies, we present an extension of the V-model for iterative data-based development processes that harmonizes and formalizes the existing methods towards a generic framework. The iterative approach allows for seamless integration of continuous system refinement. While the data-based approach constitutes the consideration of data-based development processes and formalizes the use of synthetic and real world data. In this way, formalizing the process of development, verification, validation, and continuous integration contributes to ensuring the safety of emerging complex systems that incorporate AI. 
\end{abstract}


\begin{IEEEkeywords}
	Process Reference Model, V-Model, Continuous Integration, AI Systems, Autonomy Technology, Safety Assurance
\end{IEEEkeywords}

\section{Introduction}

% \textcolor{red}{Still on working}

% \textcolor{red}{add label for each section}


Robot learning relies on diverse and high-quality data to learn complex behaviors \cite{aldaco2024aloha, wang2024dexcap}.
Recent studies highlight that models trained on datasets with greater complexity and variation in the domain tend to generalize more effectively across broader scenarios \cite{mann2020language, radford2021learning, gao2024efficient}.
% However, creating such diverse datasets in the real world presents significant challenges.
% Modifying physical environments and adjusting robot hardware settings require considerable time, effort, and financial resources.
% In contrast, simulation environments offer a flexible and efficient alternative.
% Simulations allow for the creation and modification of digital environments with a wide range of object shapes, weights, materials, lighting, textures, friction coefficients, and so on to incorporate domain randomization,
% which helps improve the robustness of models when deployed in real-world conditions.
% These environments can be easily adjusted and reset, enabling faster iterations and data collection.
% Additionally, simulations provide the ability to consistently reproduce scenarios, which is essential for benchmarking and model evaluation.
% Another advantage of simulations is their flexibility in sensor integration. Sensors such as cameras, LiDARs, and tactile sensors can be added or repositioned without the physical limitations present in real-world setups. Simulations also eliminate the risk of damaging expensive hardware during edge-case experiments, making them an ideal platform for testing rare or dangerous scenarios that are impractical to explore in real life.
By leveraging immersive perspectives and interactions, Extended Reality\footnote{Extended Reality is an umbrella term to refer to Augmented Reality, Mixed Reality, and Virtual Reality \cite{wikipediaExtendedReality}}
(XR)
is a promising candidate for efficient and intuitive large scale data collection \cite{jiang2024comprehensive, arcade}
% With the demand for collecting data, XR provides a promising approach for humans to teach robots by offering users an immersive experience.
in simulation \cite{jiang2024comprehensive, arcade, dexhub-park} and real-world scenarios \cite{openteach, opentelevision}.
However, reusing and reproducing current XR approaches for robot data collection for new settings and scenarios is complicated and requires significant effort.
% are difficult to reuse and reproduce system makes it hard to reuse and reproduce in another data collection pipeline.
This bottleneck arises from three main limitations of current XR data collection and interaction frameworks: \textit{asset limitation}, \textit{simulator limitation}, and \textit{device limitation}.
% \textcolor{red}{ASSIGN THESE CITATION PROPERLY:}
% \textcolor{red}{list them by time order???}
% of collecting data by using XR have three main limitations.
Current approaches suffering from \textit{asset limitation} \cite{arclfd, jiang2024comprehensive, arcade, george2025openvr, vicarios}
% Firstly, recent works \cite{jiang2024comprehensive, arcade, dexhub-park}
can only use predefined robot models and task scenes. Configuring new tasks requires significant effort, since each new object or model must be specifically integrated into the XR application.
% and it takes too much effort to configure new tasks in their systems since they cannot spawn arbitrary models in the XR application.
The vast majority of application are developed for specific simulators or real-world scenarios. This \textit{simulator limitation} \cite{mosbach2022accelerating, lipton2017baxter, dexhub-park, arcade}
% Secondly, existing systems are limited to a single simulation platform or real-world scenarios.
significantly reduces reusability and makes adaptation to new simulation platforms challenging.
Additionally, most current XR frameworks are designed for a specific version of a single XR headset, leading to a \textit{device limitation} 
\cite{lipton2017baxter, armada, openteach, meng2023virtual}.
% and there is no work working on the extendability of transferring to a new headsets as far as we know.
To the best of our knowledge, no existing work has explored the extensibility or transferability of their framework to different headsets.
These limitations hamper reproducibility and broader contributions of XR based data collection and interaction to the research community.
% as each research group typically has its own data collection pipeline.
% In addition to these main limitations, existing XR systems are not well suited for managing multiple robot systems,
% as they are often designed for single-operator use.

In addition to these main limitations, existing XR systems are often designed for single-operator use, prohibiting collaborative data collection.
At the same time, controlling multiple robots at once can be very difficult for a single operator,
making data collection in multi-robot scenarios particularly challenging \cite{orun2019effect}.
Although there are some works using collaborative data collection in the context of tele-operation \cite{tung2021learning, Qin2023AnyTeleopAG},
there is no XR-based data collection system supporting collaborative data collection.
This limitation highlights the need for more advanced XR solutions that can better support multi-robot and multi-user scenarios.
% \textcolor{red}{more papers about collaborative data collection}

To address all of these issues, we propose \textbf{IRIS},
an \textbf{I}mmersive \textbf{R}obot \textbf{I}nteraction \textbf{S}ystem.
This general system supports various simulators, benchmarks and real-world scenarios.
It is easily extensible to new simulators and XR headsets.
IRIS achieves generalization across six dimensions:
% \begin{itemize}
%     \item \textit{Cross-scene} : diverse object models;
%     \item \textit{Cross-embodiment}: diverse robot models;
%     \item \textit{Cross-simulator}: 
%     \item \textit{Cross-reality}: fd
%     \item \textit{Cross-platform}: fd
%     \item \textit{Cross-users}: fd
% \end{itemize}
\textbf{Cross-Scene}, \textbf{Cross-Embodiment}, \textbf{Cross-Simulator}, \textbf{Cross-Reality}, \textbf{Cross-Platform}, and \textbf{Cross-User}.

\textbf{Cross-Scene} and \textbf{Cross-Embodiment} allow the system to handle arbitrary objects and robots in the simulation,
eliminating restrictions about predefined models in XR applications.
IRIS achieves these generalizations by introducing a unified scene specification, representing all objects,
including robots, as data structures with meshes, materials, and textures.
The unified scene specification is transmitted to the XR application to create and visualize an identical scene.
By treating robots as standard objects, the system simplifies XR integration,
allowing researchers to work with various robots without special robot-specific configurations.
\textbf{Cross-Simulator} ensures compatibility with various simulation engines.
IRIS simplifies adaptation by parsing simulated scenes into the unified scene specification, eliminating the need for XR application modifications when switching simulators.
New simulators can be integrated by creating a parser to convert their scenes into the unified format.
This flexibility is demonstrated by IRIS’ support for Mujoco \cite{todorov2012mujoco}, IsaacSim \cite{mittal2023orbit}, CoppeliaSim \cite{coppeliaSim}, and even the recent Genesis \cite{Genesis} simulator.
\textbf{Cross-Reality} enables the system to function seamlessly in both virtual simulations and real-world applications.
IRIS enables real-world data collection through camera-based point cloud visualization.
\textbf{Cross-Platform} allows for compatibility across various XR devices.
Since XR device APIs differ significantly, making a single codebase impractical, IRIS XR application decouples its modules to maximize code reuse.
This application, developed by Unity \cite{unity3dUnityManual}, separates scene visualization and interaction, allowing developers to integrate new headsets by reusing the visualization code and only implementing input handling for hand, head, and motion controller tracking.
IRIS provides an implementation of the XR application in the Unity framework, allowing for a straightforward deployment to any device that supports Unity. 
So far, IRIS was successfully deployed to the Meta Quest 3 and HoloLens 2.
Finally, the \textbf{Cross-User} ability allows multiple users to interact within a shared scene.
IRIS achieves this ability by introducing a protocol to establish the communication between multiple XR headsets and the simulation or real-world scenarios.
Additionally, IRIS leverages spatial anchors to support the alignment of virtual scenes from all deployed XR headsets.
% To make an seamless user experience for robot learning data collection,
% IRIS also tested in three different robot control interface
% Furthermore, to demonstrate the extensibility of our approach, we have implemented a robot-world pipeline for real robot data collection, ensuring that the system can be used in both simulated and real-world environments.
The Immersive Robot Interaction System makes the following contributions\\
\textbf{(1) A unified scene specification} that is compatible with multiple robot simulators. It enables various XR headsets to visualize and interact with simulated objects and robots, providing an immersive experience while ensuring straightforward reusability and reproducibility.\\
\textbf{(2) A collaborative data collection framework} designed for XR environments. The framework facilitates enhanced robot data acquisition.\\
\textbf{(3) A user study} demonstrating that IRIS significantly improves data collection efficiency and intuitiveness compared to the LIBERO baseline.

% \begin{table*}[t]
%     \centering
%     \begin{tabular}{lccccccc}
%         \toprule
%         & \makecell{Physical\\Interaction}
%         & \makecell{XR\\Enabled}
%         & \makecell{Free\\View}
%         & \makecell{Multiple\\Robots}
%         & \makecell{Robot\\Control}
%         % Force Feedback???
%         & \makecell{Soft Object\\Supported}
%         & \makecell{Collaborative\\Data} \\
%         \midrule
%         ARC-LfD \cite{arclfd}                              & Real        & \cmark & \xmark & \xmark & Joint              & \xmark & \xmark \\
%         DART \cite{dexhub-park}                            & Sim         & \cmark & \cmark & \cmark & Cartesian          & \xmark & \xmark \\
%         \citet{jiang2024comprehensive}                     & Sim         & \cmark & \xmark & \xmark & Joint \& Cartesian & \xmark & \xmark \\
%         \citet{mosbach2022accelerating}                    & Sim         & \cmark & \cmark & \xmark & Cartesian          & \xmark & \xmark \\
%         ARCADE \cite{arcade}                               & Real        & \cmark & \cmark & \xmark & Cartesian          & \xmark & \xmark \\
%         Holo-Dex \cite{holodex}                            & Real        & \cmark & \xmark & \cmark & Cartesian          & \cmark & \xmark \\
%         ARMADA \cite{armada}                               & Real        & \cmark & \xmark & \cmark & Cartesian          & \cmark & \xmark \\
%         Open-TeleVision \cite{opentelevision}              & Real        & \cmark & \cmark & \cmark & Cartesian          & \cmark & \xmark \\
%         OPEN TEACH \cite{openteach}                        & Real        & \cmark & \xmark & \cmark & Cartesian          & \cmark & \cmark \\
%         GELLO \cite{wu2023gello}                           & Real        & \xmark & \cmark & \cmark & Joint              & \cmark & \xmark \\
%         DexCap \cite{wang2024dexcap}                       & Real        & \xmark & \cmark & \xmark & Cartesian          & \cmark & \xmark \\
%         AnyTeleop \cite{Qin2023AnyTeleopAG}                & Real        & \xmark & \xmark & \cmark & Cartesian          & \cmark & \cmark \\
%         Vicarios \cite{vicarios}                           & Real        & \cmark & \xmark & \xmark & Cartesian          & \cmark & \xmark \\     
%         Augmented Visual Cues \cite{augmentedvisualcues}   & Real        & \cmark & \cmark & \xmark & Cartesian          & \xmark & \xmark \\ 
%         \citet{wang2024robotic}                            & Real        & \cmark & \cmark & \xmark & Cartesian          & \cmark & \xmark \\
%         Bunny-VisionPro \cite{bunnyvisionpro}              & Real        & \cmark & \cmark & \cmark & Cartesian          & \cmark & \xmark \\
%         IMMERTWIN \cite{immertwin}                         & Real        & \cmark & \cmark & \cmark & Cartesian          & \xmark & \xmark \\
%         \citet{meng2023virtual}                            & Sim \& Real & \cmark & \cmark & \xmark & Cartesian          & \xmark & \xmark \\
%         Shared Control Framework \cite{sharedctlframework} & Real        & \cmark & \cmark & \cmark & Cartesian          & \xmark & \xmark \\
%         OpenVR \cite{openvr}                               & Real        & \cmark & \cmark & \xmark & Cartesian          & \xmark & \xmark \\
%         \citet{digitaltwinmr}                              & Real        & \cmark & \cmark & \xmark & Cartesian          & \cmark & \xmark \\
        
%         \midrule
%         \textbf{Ours} & Sim \& Real & \cmark & \cmark & \cmark & Joint \& Cartesian  & \cmark & \cmark \\
%         \bottomrule
%     \end{tabular}
%     \caption{This is a cross-column table with automatic line breaking.}
%     \label{tab:cross-column}
% \end{table*}

% \begin{table*}[t]
%     \centering
%     \begin{tabular}{lccccccc}
%         \toprule
%         & \makecell{Cross-Embodiment}
%         & \makecell{Cross-Scene}
%         & \makecell{Cross-Simulator}
%         & \makecell{Cross-Reality}
%         & \makecell{Cross-Platform}
%         & \makecell{Cross-User} \\
%         \midrule
%         ARC-LfD \cite{arclfd}                              & \xmark & \xmark & \xmark & \xmark & \xmark & \xmark \\
%         DART \cite{dexhub-park}                            & \cmark & \cmark & \xmark & \xmark & \xmark & \xmark \\
%         \citet{jiang2024comprehensive}                     & \xmark & \cmark & \xmark & \xmark & \xmark & \xmark \\
%         \citet{mosbach2022accelerating}                    & \xmark & \cmark & \xmark & \xmark & \xmark & \xmark \\
%         ARCADE \cite{arcade}                               & \xmark & \xmark & \xmark & \xmark & \xmark & \xmark \\
%         Holo-Dex \cite{holodex}                            & \cmark & \xmark & \xmark & \xmark & \xmark & \xmark \\
%         ARMADA \cite{armada}                               & \cmark & \xmark & \xmark & \xmark & \xmark & \xmark \\
%         Open-TeleVision \cite{opentelevision}              & \cmark & \xmark & \xmark & \xmark & \cmark & \xmark \\
%         OPEN TEACH \cite{openteach}                        & \cmark & \xmark & \xmark & \xmark & \xmark & \cmark \\
%         GELLO \cite{wu2023gello}                           & \cmark & \xmark & \xmark & \xmark & \xmark & \xmark \\
%         DexCap \cite{wang2024dexcap}                       & \xmark & \xmark & \xmark & \xmark & \xmark & \xmark \\
%         AnyTeleop \cite{Qin2023AnyTeleopAG}                & \cmark & \cmark & \cmark & \cmark & \xmark & \cmark \\
%         Vicarios \cite{vicarios}                           & \xmark & \xmark & \xmark & \xmark & \xmark & \xmark \\     
%         Augmented Visual Cues \cite{augmentedvisualcues}   & \xmark & \xmark & \xmark & \xmark & \xmark & \xmark \\ 
%         \citet{wang2024robotic}                            & \xmark & \xmark & \xmark & \xmark & \xmark & \xmark \\
%         Bunny-VisionPro \cite{bunnyvisionpro}              & \cmark & \xmark & \xmark & \xmark & \xmark & \xmark \\
%         IMMERTWIN \cite{immertwin}                         & \cmark & \xmark & \xmark & \xmark & \xmark & \xmark \\
%         \citet{meng2023virtual}                            & \xmark & \cmark & \xmark & \cmark & \xmark & \xmark \\
%         \citet{sharedctlframework}                         & \cmark & \xmark & \xmark & \xmark & \xmark & \xmark \\
%         OpenVR \cite{george2025openvr}                               & \xmark & \xmark & \xmark & \xmark & \xmark & \xmark \\
%         \citet{digitaltwinmr}                              & \xmark & \xmark & \xmark & \xmark & \xmark & \xmark \\
        
%         \midrule
%         \textbf{Ours} & \cmark & \cmark & \cmark & \cmark & \cmark & \cmark \\
%         \bottomrule
%     \end{tabular}
%     \caption{This is a cross-column table with automatic line breaking.}
% \end{table*}

% \begin{table*}[t]
%     \centering
%     \begin{tabular}{lccccccc}
%         \toprule
%         & \makecell{Cross-Scene}
%         & \makecell{Cross-Embodiment}
%         & \makecell{Cross-Simulator}
%         & \makecell{Cross-Reality}
%         & \makecell{Cross-Platform}
%         & \makecell{Cross-User}
%         & \makecell{Control Space} \\
%         \midrule
%         % Vicarios \cite{vicarios}                           & \xmark & \xmark & \xmark & \xmark & \xmark & \xmark \\     
%         % Augmented Visual Cues \cite{augmentedvisualcues}   & \xmark & \xmark & \xmark & \xmark & \xmark & \xmark \\ 
%         % OpenVR \cite{george2025openvr}                     & \xmark & \xmark & \xmark & \xmark & \xmark & \xmark \\
%         \citet{digitaltwinmr}                              & \xmark & \xmark & \xmark & \xmark & \xmark & \xmark &  \\
%         ARC-LfD \cite{arclfd}                              & \xmark & \xmark & \xmark & \xmark & \xmark & \xmark &  \\
%         \citet{sharedctlframework}                         & \cmark & \xmark & \xmark & \xmark & \xmark & \xmark &  \\
%         \citet{jiang2024comprehensive}                     & \cmark & \xmark & \xmark & \xmark & \xmark & \xmark &  \\
%         \citet{mosbach2022accelerating}                    & \cmark & \xmark & \xmark & \xmark & \xmark & \xmark & \\
%         Holo-Dex \cite{holodex}                            & \cmark & \xmark & \xmark & \xmark & \xmark & \xmark & \\
%         ARCADE \cite{arcade}                               & \cmark & \cmark & \xmark & \xmark & \xmark & \xmark & \\
%         DART \cite{dexhub-park}                            & Limited & Limited & Mujoco & Sim & Vision Pro & \xmark &  Cartesian\\
%         ARMADA \cite{armada}                               & \cmark & \cmark & \xmark & \xmark & \xmark & \xmark & \\
%         \citet{meng2023virtual}                            & \cmark & \cmark & \xmark & \cmark & \xmark & \xmark & \\
%         % GELLO \cite{wu2023gello}                           & \cmark & \xmark & \xmark & \xmark & \xmark & \xmark \\
%         % DexCap \cite{wang2024dexcap}                       & \xmark & \xmark & \xmark & \xmark & \xmark & \xmark \\
%         % AnyTeleop \cite{Qin2023AnyTeleopAG}                & \cmark & \cmark & \cmark & \cmark & \xmark & \cmark \\
%         % \citet{wang2024robotic}                            & \xmark & \xmark & \xmark & \xmark & \xmark & \xmark \\
%         Bunny-VisionPro \cite{bunnyvisionpro}              & \cmark & \cmark & \xmark & \xmark & \xmark & \xmark & \\
%         IMMERTWIN \cite{immertwin}                         & \cmark & \cmark & \xmark & \xmark & \xmark & \xmark & \\
%         Open-TeleVision \cite{opentelevision}              & \cmark & \cmark & \xmark & \xmark & \cmark & \xmark & \\
%         \citet{szczurek2023multimodal}                     & \xmark & \xmark & \xmark & Real & \xmark & \cmark & \\
%         OPEN TEACH \cite{openteach}                        & \cmark & \cmark & \xmark & \xmark & \xmark & \cmark & \\
%         \midrule
%         \textbf{Ours} & \cmark & \cmark & \cmark & \cmark & \cmark & \cmark \\
%         \bottomrule
%     \end{tabular}
%     \caption{TODO, Bruce: this table can be further optimized.}
% \end{table*}

\definecolor{goodgreen}{HTML}{228833}
\definecolor{goodred}{HTML}{EE6677}
\definecolor{goodgray}{HTML}{BBBBBB}

\begin{table*}[t]
    \centering
    \begin{adjustbox}{max width=\textwidth}
    \renewcommand{\arraystretch}{1.2}    
    \begin{tabular}{lccccccc}
        \toprule
        & \makecell{Cross-Scene}
        & \makecell{Cross-Embodiment}
        & \makecell{Cross-Simulator}
        & \makecell{Cross-Reality}
        & \makecell{Cross-Platform}
        & \makecell{Cross-User}
        & \makecell{Control Space} \\
        \midrule
        % Vicarios \cite{vicarios}                           & \xmark & \xmark & \xmark & \xmark & \xmark & \xmark \\     
        % Augmented Visual Cues \cite{augmentedvisualcues}   & \xmark & \xmark & \xmark & \xmark & \xmark & \xmark \\ 
        % OpenVR \cite{george2025openvr}                     & \xmark & \xmark & \xmark & \xmark & \xmark & \xmark \\
        \citet{digitaltwinmr}                              & \textcolor{goodred}{Limited}     & \textcolor{goodred}{Single Robot} & \textcolor{goodred}{Unity}    & \textcolor{goodred}{Real}          & \textcolor{goodred}{Meta Quest 2} & \textcolor{goodgray}{N/A} & \textcolor{goodred}{Cartesian} \\
        ARC-LfD \cite{arclfd}                              & \textcolor{goodgray}{N/A}        & \textcolor{goodred}{Single Robot} & \textcolor{goodgray}{N/A}     & \textcolor{goodred}{Real}          & \textcolor{goodred}{HoloLens}     & \textcolor{goodgray}{N/A} & \textcolor{goodred}{Cartesian} \\
        \citet{sharedctlframework}                         & \textcolor{goodred}{Limited}     & \textcolor{goodred}{Single Robot} & \textcolor{goodgray}{N/A}     & \textcolor{goodred}{Real}          & \textcolor{goodred}{HTC Vive Pro} & \textcolor{goodgray}{N/A} & \textcolor{goodred}{Cartesian} \\
        \citet{jiang2024comprehensive}                     & \textcolor{goodred}{Limited}     & \textcolor{goodred}{Single Robot} & \textcolor{goodgray}{N/A}     & \textcolor{goodred}{Real}          & \textcolor{goodred}{HoloLens 2}   & \textcolor{goodgray}{N/A} & \textcolor{goodgreen}{Joint \& Cartesian} \\
        \citet{mosbach2022accelerating}                    & \textcolor{goodgreen}{Available} & \textcolor{goodred}{Single Robot} & \textcolor{goodred}{IsaacGym} & \textcolor{goodred}{Sim}           & \textcolor{goodred}{Vive}         & \textcolor{goodgray}{N/A} & \textcolor{goodgreen}{Joint \& Cartesian} \\
        Holo-Dex \cite{holodex}                            & \textcolor{goodgray}{N/A}        & \textcolor{goodred}{Single Robot} & \textcolor{goodgray}{N/A}     & \textcolor{goodred}{Real}          & \textcolor{goodred}{Meta Quest 2} & \textcolor{goodgray}{N/A} & \textcolor{goodred}{Joint} \\
        ARCADE \cite{arcade}                               & \textcolor{goodgray}{N/A}        & \textcolor{goodred}{Single Robot} & \textcolor{goodgray}{N/A}     & \textcolor{goodred}{Real}          & \textcolor{goodred}{HoloLens 2}   & \textcolor{goodgray}{N/A} & \textcolor{goodred}{Cartesian} \\
        DART \cite{dexhub-park}                            & \textcolor{goodred}{Limited}     & \textcolor{goodred}{Limited}      & \textcolor{goodred}{Mujoco}   & \textcolor{goodred}{Sim}           & \textcolor{goodred}{Vision Pro}   & \textcolor{goodgray}{N/A} & \textcolor{goodred}{Cartesian} \\
        ARMADA \cite{armada}                               & \textcolor{goodgray}{N/A}        & \textcolor{goodred}{Limited}      & \textcolor{goodgray}{N/A}     & \textcolor{goodred}{Real}          & \textcolor{goodred}{Vision Pro}   & \textcolor{goodgray}{N/A} & \textcolor{goodred}{Cartesian} \\
        \citet{meng2023virtual}                            & \textcolor{goodred}{Limited}     & \textcolor{goodred}{Single Robot} & \textcolor{goodred}{PhysX}   & \textcolor{goodgreen}{Sim \& Real} & \textcolor{goodred}{HoloLens 2}   & \textcolor{goodgray}{N/A} & \textcolor{goodred}{Cartesian} \\
        % GELLO \cite{wu2023gello}                           & \cmark & \xmark & \xmark & \xmark & \xmark & \xmark \\
        % DexCap \cite{wang2024dexcap}                       & \xmark & \xmark & \xmark & \xmark & \xmark & \xmark \\
        % AnyTeleop \cite{Qin2023AnyTeleopAG}                & \cmark & \cmark & \cmark & \cmark & \xmark & \cmark \\
        % \citet{wang2024robotic}                            & \xmark & \xmark & \xmark & \xmark & \xmark & \xmark \\
        Bunny-VisionPro \cite{bunnyvisionpro}              & \textcolor{goodgray}{N/A}        & \textcolor{goodred}{Single Robot} & \textcolor{goodgray}{N/A}     & \textcolor{goodred}{Real}          & \textcolor{goodred}{Vision Pro}   & \textcolor{goodgray}{N/A} & \textcolor{goodred}{Cartesian} \\
        IMMERTWIN \cite{immertwin}                         & \textcolor{goodgray}{N/A}        & \textcolor{goodred}{Limited}      & \textcolor{goodgray}{N/A}     & \textcolor{goodred}{Real}          & \textcolor{goodred}{HTC Vive}     & \textcolor{goodgray}{N/A} & \textcolor{goodred}{Cartesian} \\
        Open-TeleVision \cite{opentelevision}              & \textcolor{goodgray}{N/A}        & \textcolor{goodred}{Limited}      & \textcolor{goodgray}{N/A}     & \textcolor{goodred}{Real}          & \textcolor{goodgreen}{Meta Quest, Vision Pro} & \textcolor{goodgray}{N/A} & \textcolor{goodred}{Cartesian} \\
        \citet{szczurek2023multimodal}                     & \textcolor{goodgray}{N/A}        & \textcolor{goodred}{Limited}      & \textcolor{goodgray}{N/A}     & \textcolor{goodred}{Real}          & \textcolor{goodred}{HoloLens 2}   & \textcolor{goodgreen}{Available} & \textcolor{goodred}{Joint \& Cartesian} \\
        OPEN TEACH \cite{openteach}                        & \textcolor{goodgray}{N/A}        & \textcolor{goodgreen}{Available}  & \textcolor{goodgray}{N/A}     & \textcolor{goodred}{Real}          & \textcolor{goodred}{Meta Quest 3} & \textcolor{goodred}{N/A} & \textcolor{goodgreen}{Joint \& Cartesian} \\
        \midrule
        \textbf{Ours}                                      & \textcolor{goodgreen}{Available} & \textcolor{goodgreen}{Available}  & \textcolor{goodgreen}{Mujoco, CoppeliaSim, IsaacSim} & \textcolor{goodgreen}{Sim \& Real} & \textcolor{goodgreen}{Meta Quest 3, HoloLens 2} & \textcolor{goodgreen}{Available} & \textcolor{goodgreen}{Joint \& Cartesian} \\
        \bottomrule
        \end{tabular}
    \end{adjustbox}
    \caption{Comparison of XR-based system for robots. IRIS is compared with related works in different dimensions.}
\end{table*}


\section{Related Work}\label{02_StateOfTheArt}
This section provides a general overview of relevant research projects and highlights a selection that is to be examined in more detail in the subsequent section.  

The Pegasus family \cite{PEGASUS} comprises the Pegasus \cite{PEGASUS} project as well as the successor projects Verification Validation Methods (VVM) \cite{VVM} and SetLevel \cite{SETLevel}. While the SetLevel project focused on the particular challenge of simulation-based development, the VVM project built upon the decomposition of the operational design domain (ODD) into logical core scenarios established by the Pegasus project and extends this methodology in a holistic manner. Consequently, the overall methodology of VVM \cite{VVMOverall} represents an extension of the traditional V-model. A similar approach is being pursued by the Japan Automobile Manufacturers Association (JAMA) with its Automated Driving Safety Evaluation Framework \cite{JAMAFramework}. This framework also constitutes a scenario-based approach and an enhanced V-model structure. The backbone of these scenario-based approaches are scenario databases, for instance from the ADScene \cite{guyonvarch2023adscene}, Safety Pool \cite{SafetyPool}, and SAKURA \cite{SAKURA} projects. Moreover, the similarity in the approaches results in part from close exchange and cooperation. For example, as part of the German-Japanese VIVID \cite{VIVID} project, the national sub-projects VIVALDI and DIVP \cite{DIVP} were conducting joint research on virtual validation. 

In the context of scenario-based approaches, the theoretical framework of Scenarios Engineering (SE) \cite{li2022novel, li2022features} is also worth mentioning. This framework aims to improve the visibility, interpretability, and reliability of intelligent systems and, thus, contributes to the realization of trustworthy AI. Beyond this, the emergence of video generation models, such as \mbox{OpenAI's} Sora, is addressed in \cite{li2024sora} as a step towards imaginative intelligence, while discussing its relevance in the context of SE. As outlined, the advancement supports the training and testing of intelligent vehicles by reducing the need for physical recordings \cite{li2023novel} and expanding the variety of scenarios. In addition, \cite{wang2024does} elaborates on the opportunities that imaginative intelligence offers for SE, including tackling the long-tail problem, while also outlining the challenges that currently remain in accurately modeling physics and understanding causality. Thus, \cite{li2024sora} and \cite{wang2024does} take into account recent developments and present a perspective of future developments. In this regard, the NXT GEN AI METHODS \cite{NextAIM} project should also be noted, which is dedicated to generative AI and, more specifically, the development of foundation models for automated driving.

Furthermore, there are several projects like StreetWise \cite{elrofai2018streetwise}, HEADSTART \cite{HEADSTART}, SAFE-UP \cite{SAFEUP}, and AI Safeguarding \cite{ki_absicherung} that focused on validation, safety assessment, and safety assurance within the field of automated driving. Moreover, projects such as V4SAFETY \cite{V4SAFETY} and SUNRISE \cite{SUNRISE} aim to develop more comprehensive frameworks for safety assessment and assurance, respectively. The LOPAAS \cite{LOPAAS} project between Fraunhofer IESE, Fraunhofer IKS, and the Assuring Autonomy International Program (AAIP) at the University of York also targets to bring about a paradigm shift in safety engineering for autonomous systems. Hi-Drive \cite{HIDrive}, another project, addresses the various ODD challenges but concentrates on reaching a higher level of autonomy. Consequently, a multitude of projects is anchored in this context. 

However, the VVM \cite{VVM} project is of particular interest as it can be considered the most advanced representative of the improved V-model approaches and outlines an overall methodology. Of particular interest is also the methodology of Waymo \cite{favaro2023building}, which differs in focus from the VVM \cite{VVM} project. In addition to the different focus, the methodology is of particular interest due to the resulting comparatively solid performance \cite{kusano2023comparison} of the automated vehicles in comparison to other competitors \cite{equipmentrecallreport, NHTSARecall23E, NHTSARecallLetter}. Beyond that, Tesla's approach \cite{karpathy_cvpr21} is technologically different and benefits from a large number of vehicles in the field that generate data. As a consequence, the methodology is based on an AI-centric development process, similar to processes of other AI systems such as OpenAI's ChatGPT. For this reason, the general approach, also known as the data engine is also worth emphasizing. Ultimately, the three highlighted processes are to be analyzed in more detail subsequently.

\section{Analysis of Innovative Development Processes}\label{02_New}
While the previous section provides a general overview, this section is dedicated to the analysis of a selection of innovative development processes. Thereby, the different advancements and perspectives are related to the classical V-model in order to set a basis for the subsequent creation of a generalized framework. For this purpose, the VVM \cite{VVM} project, Waymo's methodology \cite{favaro2023building} and Tesla's approach \cite{karpathy_cvpr21} are examined in more detail below. In addition, the methodologies are compared in terms of their general applicability to complex systems. Conclusively, the application areas of the available frameworks and the existing gap in the overall context are discussed.  

\subsection{VVM Project: Argumentation- \& Scenario-based V-model}\label{VVM}

The VVM project enhances the traditional V-model \cite{brohl1993v} by integrating multiple perspectives. First, it employs the ODD decomposition from Pegasus \cite{schuldt2013effiziente, pegasus_schlussbericht} to construct an ODD metamodel \cite{scholtes20216, reich2023concept}, facilitating scenario-based design and Verification \& Validation (V\&V) processes \cite{elster2021fundamental}. This method effectively maps infinite scenarios of the open world into manageable test spaces and design foundations \cite{neurohr2021criticality}. Second, a coherent assurance argumentation \cite{VVMAssurance} targets the mitigation of unreasonable risks in the open world, fostering consistency and traceability along the V-model. Consequently, the project establishes consistent interfaces across the framework and enables seamless requirements considerations from design to verification via argumentation-based V\&V. Third, a multi-perspective approach \cite{VVMAPerspectives} addresses systematic gaps such as specification, implementation, and validation gaps, aiming at uncertainty and risk reduction \cite{stellet2019formalisation}. This involves decomposing the overall system into various levels of abstraction, incorporating capability, engineering, and real-world layers, and employing different perspectives like design, V\&V, risk management, and argumentation.

\begin{figure}[]
	\centering	
	\includegraphics[width=\linewidth]{img/VVM_UL_X.png}
	\caption{Visualization of the overall methodology of the VVM project \cite{VVMOverall} as an extension of the classical V-model designed for automated driving application with regard to a scenario-based problem decomposition and an appropriate safety argumentation.}
	\label{fig:VVM}
\end{figure}

Finally, the overall methodology culminates to an advanced V-model \cite{VVMOverall}, shown in Figure \ref{fig:VVM}, that is customized to the scenario-based logic introduced by Pegasus. 

The overall methodology culminates in an advanced V-model \cite{VVMOverall}, as depicted in Figure \ref{fig:VVM}, customized to the scenario-based logic introduced by Pegasus. The ODD metamodel forms the top of the model, inducing the scenario-based philosophy throughout the V-model process. Problem space analysis provides the basis for detailed specification, considering the environment and the specific automated driving system. Furthermore, to ensure regulatory compliance, normative behavior \cite{salem2022beitrag} is specified, encompassing certification, legal, social, and ethical expectations. Subsequently, systematic hazard  \cite{graubohm2020towards} and risk identification are conducted based on the ODD metamodel \cite{VVMOverall}, problem space analysis, and normative behavior specification. Corresponding safety measures to determine an acceptable residual risk are then defined through risk treatment, closely coordinated with the underlying sub-framework, the risk management core \cite{salem2023risk}.

After implementation, a scenario-based verification, validation, and risk assessment is performed \cite{riedmaier2020survey}. This process incorporates three key improvements over the classical V-model: access to the ODD metamodel for aligned analysis, assurance assessment throughout V\&V, and evaluation of residual risk \cite{VVMOverall}. In summary, the VVM project provides an extended, detailed, and tailored process reference model for scenario-based development and V\&V of automated vehicles.

\subsection{Waymo: Safety Determination Lifecycle}\label{Waymo}

While Waymo's safety determination lifecycle also strives for the absence of unreasonable risk, it places a stronger emphasis on the lifecycle \cite{favaro2023building}. Furthermore, similar to the VVM project \cite{VVMAPerspectives, stellet2019formalisation}, the framework encompasses different perspectives. Overall, the framework can be summarized as a layered, credible, and dynamic approach \cite{favaro2023building}.

The \textit{\textbf{layered approach}} within \cite{favaro2023building} refers to a division into an architectural (formerly known as hardware \cite{webb2020waymo}), a behavioral, and an in-service operational layer. Thereby, in order to demonstrate the absence of unreasonable risk, hazards and appropriate acceptance criteria are defined within the aforementioned layers. This involves the definition of several dimensions of interest covering a diverse set of aspects, e.g. the avoidance of incidents, the successful completion of automated journeys or the compliance with driving rules. Indicators of interest are defined on this basis, which map hazards to an explicit set of acceptance criteria. Through the definition of the minimum dimensions of interest, Waymo determines the completeness of the set of acceptance criteria in order to underpin credibility.  

The \textit{\textbf{credible approach}} adresses concerns about the reasonableness and trustworthiness of the claim-argument-evidence structure via Waymo's novel Case Credibility Assessment (CCA) \cite{favaro2023building}. The CCA comprises three components: the top-down credibility of the argument, the bottom-up credibility of the evidence, and the encompassing implementation of the credibility. Overall, the CCA procedure comprises a continuous revision by monitoring and updating the arguments and evidence to achieve credibility.

The top-down credibility of the arguments focuses on fulfilling overarching objectives through assessing the suitability and reasonableness of the arguments. This involves evaluating and refining a collection of arguments and acceptance criteria. Additionally, it entails justifying acceptance criteria and conducting a suitability assessment of performance indicators and associated objectives to evaluate the reasonableness. In contrast, the bottom-up credibility of the evidence concentrates on evaluating the evidence provided by the methodology. This entails analyzing the evidence with regard to both technical engineering and process management to assess confidence. Additionally, it entails evaluating the representativeness and applicability of the evidence for coverage assessment.

The \textit{\textbf{dynamic approach}} within \cite{favaro2023building} emphasis the siginficance of continuous assessment and refinement. Thereby, the time frame across the itervative procedures is distinguished in three phases: the pre-deployment, the deployment, and the post-deployment phase \cite{favaro2023interpreting}. During pre-deployment, design and V\&V are prospective, with performance measures based on simulations and field operations. Accordingly, acceptance criteria are treated as predicted values until deployment. Successive post-deployment involves retrospective performance analysis in the open world \cite{scanlon2023benchmarks}. Continuous monitoring identifies gaps, challenges, threats, and hazards, addressing the dynamic challenges of the open world and facilitating ongoing refinement of the system and the process throughout the product lifecycle.

\begin{figure}[]
	\centering	
	\includegraphics[width=\linewidth]{img/Waymo_UL_X.png}
	\caption{Simplified representation of Waymo's safety determination lifecycle, inspired by \cite{favaro2023building}. It illustrates the distinguished consideration of prospective and retrospective perspectives on the methodology and safety argumentation.}
	\label{fig:Waymo}
\end{figure}

The \textbf{overall methodology} of the safety determination lifecycle is depicted in Figure \ref{fig:Waymo}, emphasizing the dynamic approach. Thereby, in accordance with the three phases of the dynamic approach, safety evolves as an emergent property, an acceptable prediction, and a constantly growing confidence. Morover, to achieve the desired safety and credibility, Waymo considers the process and the product to be aligned. Each iteration consists of process and product development/refinement and subsequent qualification, reflecting two consecutive V-models, one for the process and another for the product. The approach addresses the complexities of system development and real-world applicability through ongoing refinement, effectively integrating development and analysis. Over time, as the system scales, uncertainty diminishes, confidence increases, and real-world performance is evaluated against human benchmarks \cite{di2023comparative, kusano2023comparison}.

\subsection{Tesla: Data Engine}\label{Tesla}

The functional objectives of Tesla's Full Self-Driving and Waymo's self-driving service, Waymo One, are comparable, but they employ different technological and methodological approaches. Tesla relies on an end-to-end AI strategy based on camera data and benefits from a large fleet of operational vehicles, providing a vast amount of data for development and V\&V. Tesla's corresponding methodological approach, known as the data engine \cite{karpathy_cvpr21}, is illustrated in Figure \ref{fig:Tesla}. 

The data engine employs a data-centric, iterative development approach applicable to variety of AI systems and applications. Initially, an AI model is trained with a seed dataset. Initially, an AI is trained with a seed dataset. Subsequently, as in the case of Tesla, it is deployed in shadow mode in the customer's vehicles \cite{Tesla_shadow}, also known as silent testing \cite{templeton2019}. This involves employing specialized mechanisms to detect neural network inaccuracies, facilitating strategic data acquisition. Tesla, for instance, has crafted over 200 triggers to detect discrepancies while predicting surrounding object parameters like position, velocity, and acceleration \cite{karpathy_cvpr21}. While these triggers can be seen as AI performance indicators with respect to Waymo's terminology \cite{favaro2023building}, they serve a different purpose in this context.

The data engine's collection process drives follow-up auto-labeling and unit test updates, ensuring continuous dataset refinement. Subsequently, the neural network undergoes retraining, evaluation, and unit tests to avoid inaccuracies in subsequent versions, before redeployment in shadow mode \cite{Tesla_shadow, karpathy_cvpr21}. This systematic approach resembles a sophisticated automated data-based system, resembling a scenario-based database, effectively addressing real-world gaps and changes over time. However, for a seamless transition from silent testing to actual operation \cite{Tesla_shadow}, corresponding safety argumentation and assurance are essential. Nevertheless, Tesla does not provide any further information on this \cite{tesla_safety}.

The data engine not only enables systematic data acquisition and continuous improvement but also provides additional benefits for developing AI-based systems with large datasets. This includes increased efficiency through automation, creating a consistent database, and avoiding redundant data, leading to cost savings and improved effectiveness for diverse engineering teams.

Moreover, investigations reveal the widespread adoption of the data engine process, exemplified by its use by OpenAI \cite{DALLE2}. This indicates its versatility in addressing specifications for inadmissible outputs by selectively filtering undesirable data before training. Serving as an active learning approach, the data engine process collects data, updates AI systems, and mitigates the risk of catastrophic forgetting through offline updates, including V\&V and unit testing, before deployment. Therefore, the selection as a data-based process reference model seems appropriate.

\begin{figure}[]
	\centering	
	\includegraphics[width=\linewidth]{img/TESLA_UL_X_4.png}
	\caption{Tesla's data engine \cite{karpathy_cvpr21}, visualized in V-model structure, represents a fully data-driven methodology that is tailored to AI systems and strives for efficient and effective continuous improvement.}
	\label{fig:Tesla}
\end{figure}

\subsection{Comparison w.r.t. Complex Systems Incorporating AI}\label{sec:diff}

While previously the selected innovative development processes have been presented, these are now compared below in terms of general applicability to complex systems that incorporate AI. Emerging complex systems, like automated driving, feature increasing AI integration alongside traditional systems and heightened task complexity, leading to intricate architectures and necessitating complex orchestrations. The comparison of these processes in terms of their key advantages and disadvantages for general complex systems is presented in Table \ref{tab:compare_frameworks_ad_disad}. However, inherent limitations arise concerning their applicability to complex systems with AI integration due to application specific assumptions and conditions.

\begin{table*}[]
	\centering
	\caption{Comparison of development processes for complex systems incorporating AI.}
	\begin{tabularx}{\linewidth}{l *{2}{>{\raggedright\arraybackslash}X}}
		\toprule
		Framework	& \makecell{Advantages} & \makecell{Disadvantages}  \\
		\midrule
		\makecell[l]{Improved V-model \\ \tiny{(VVM Project)}} \vspace*{-0.4cm} &  \vspace*{-0.4cm} \begin{itemize}
			\item Systematic improvement of the classical V-model
			\item Consitent consideration of the operational design domain
			\item Maps infinite scenarios of the open world
			into a managable test space
		\end{itemize} \vspace*{-0.4cm} &  \vspace*{-0.4cm}
		\begin{itemize}
			\item Application-specific for automated driving 
			\item Requires a scenario database, not suitable for general databases
			\item Requires multiple perspectives, e.g. design, V\&V, risk management, argumentation
			\item Lacks alignment with the requirements of AI systems
		\end{itemize} \vspace*{-0.4cm}  \\
		\midrule
		\makecell[l]{Safety Determination Lifecycle \\ \tiny{(Waymo)}} \vspace*{-0.4cm} &  \vspace*{-0.4cm}
		\begin{itemize}
			\item Consideration of prospective and retrospective perspectives
			\item Alignment and refinement of process and product throughout the entire lifecycle
			\item Strong emphasis on V\&V and approval process, e.g. through case credibility assessment
		\end{itemize} \vspace*{-0.4cm} &  \vspace*{-0.4cm}
		\begin{itemize}
			\item Tailord towards the automated driving application
			\item Requires several approaches, e.g. layered, credible, dynamic
			\item Less formalized development process, due to strong emphasis on V\&V and approval process
		\end{itemize}  \\
		\midrule
		\makecell[l]{Data Engine \\ \tiny{(Tesla)}} \vspace*{-0.4cm} &  \vspace*{-0.4cm}
		\begin{itemize}
			\item Considers specifics of data-driven AI development and V\&V 
			\item Able to be automated, reduces human efforts and influences
			\item Provides application-agnosticity and iterative refinement
		\end{itemize} \vspace*{-0.4cm} &  \vspace*{-0.4cm}
		\begin{itemize}
			\item Not applicable to traditional or mixed systems
			\item Requires systems in operation for silent testing and data acquisition
			\item Less sophisticated process, e.g no simulation, no process refinement
		\end{itemize}  \vspace*{-0.4cm} \\
		\bottomrule
	\end{tabularx}%
	\label{tab:compare_frameworks_ad_disad}
\end{table*}%

The VVM project's improved V-model \cite{VVMOverall} and Waymo's safety determination lifecycle \cite{favaro2023building} embrace multiple perspectives, incorporate system structures, address different granularities, and onsider the safety argumentation, whereas the classic V-model \cite{brohl1993v} takes a singular approach focusing on design, V\&V, and customizable safety argumentation. This distinction allows the classical V-model to maintain broad applicability across various systems by accommodating flexibility in architecture, granularity, and assurance methods to meet specific requirements. These properties are essential for a framework aiming to address the needs of diverse complex systems. 

Moreover, Tesla's data engine \cite{karpathy_cvpr21} is primarily designed for AI systems, making it less adaptable to traditional or mixed systems. Moreover, it relies on a large-scale human-operated system fleet for data collection and silent testing, a requirement that is often impractical

In comparison to partially extended responsibilities and non-generic assumptions, frameworks often overlook formalizing the interdependence and transitions between simulation and the real world. However, it is evident that the joint interaction between simulation and the real world  is pivotal for complex systems \cite{KIDeltaSynData, KIAbsicherungSynData}. This interaction can result in significant gaps \cite{burton2023closing, burton2023addressing} during both the design and V\&V phases. Currently, there's a lack of a systematic formalization for the transition from the real world to simulation and back within a continuous refinement process. Such a formalization that address concerns related to data-driven development without excluding traditional and mixed systems, while ensuring flexibility in terms of safety assessments and argumentation, does not yet exist. 

\subsection{Discussion}
The analysis of selected processes reveals that alongside the classical V-model, designed for traditional systems, various process reference frameworks have emerged to address specific complex systems integrating both traditional and AI components, which can be described as emerging complex systems that incorporate AI. The improved V-model of the VVM project represents an application-specific extension of the classical V-model and addresses complex systems of automated driving. Waymo's safety determination lifecycle is similarly motivated but less dependent on the application and system design. In comparison, Tesla's data engine explicitly addresses the characteristics of AI systems. Finally, Figure \ref{fig:cmp} visually compares the above process reference model frameworks in the tension between application specificity and generalization, and suitability for traditional systems, AI systems, and emerging complex systems that incorporate AI. On an abstract level, Figure \ref{fig:cmp} shows that each approach has a certain area of application. Furthermore, it becomes apparent that there is a gap in the area of generic processes for complex systems that incorporate AI. This ultimately motivates our contribution in visual form. The correspondingly proposed iterative data-based V-model, which is intended to narrow this gap, is elaborated in the following section.  
 
\begin{figure}[h!]
	\centering	
	\includegraphics[scale=0.45]{img/VGL12.png}
	\caption{Visualization of the classical V-model and the analyzed innovative development processes with regard to methodological generality and system compatibility. The remaining gap is marked by the proposed iterative data-based V-model.}
	\label{fig:cmp}
\end{figure} 










\section{Methodology}
\label{sec:methodology}

We begin with reformulating~\eqref{obj:original_sparse_problem_perspective_formulation_convex_relaxation} as the following \textit{unconstrained} optimization problem
\begin{align}
    \label{obj:original_sparse_problem_convex_composite_reformulation}
    \min_{\bbeta} f(\bX \bbeta, \by) + 2 \lambda_2 \, g(\bbeta),
\end{align}
where the implicit function $g: \R^p \to \R \cup \{ \infty \}$ is defined~as
\begin{align}
    \label{eq:function_g_definition}
    g(\bbeta) = \left\{ 
    \begin{array}{cl}
        \min\limits_{\bz \in \R^p} & \frac{1}{2} \sum_{j \in [p]} \beta_j^2 / z_j \\[1ex]
        \st & \bz \in [0, 1]^p, \, \bm 1^\top \bz \leq k, \\
        & -M z_j \leq \beta_j \leq M z_j ~ \forall j \in [p].
    \end{array} 
    \right.
\end{align}
Here, we follow the standard convention that an infeasible minimization problem is assigned a value of $+\infty$. 
Note that $g$ is convex as convexity is preserved under partial minimization over a convex set~\citep[Theorem~5,3]{rockafellar1970convex}. 
Furthermore, as $f$ is assumed to be Lipschitz smooth and $g$ is non-smooth, problem~\eqref{obj:original_sparse_problem_convex_composite_reformulation} is an unconstrained optimization problem with a convex composite objective function. As such, it is amenable to be solved using the FISTA algorithm proposed in \citep{beck2009fast}. 

In the following, we first analyze the conjugate of $g$. We then propose an efficient numerical method to compute the proximal operator of $g^*$. This, in turn, enables us to compute the proximal operator of $g$, leading to an efficient implementation of the FISTA algorithm. To further enhance the performance of FISTA, we present an efficient approach to solve the minimization problem~\eqref{eq:function_g_definition}, which guides us in developing an effective restart procedure. Finally, we conclude this section by providing efficient lower bounds for each step of the BnB framework.

\subsection{Conjugate function $g^*$}
Recall that the conjugate of $g$ is defined as
\begin{align*}
    g^*(\bm \alpha) = \sup_{\bm \beta \in \R^p} ~ \bm \alpha^\top \bm \beta - g(\bm \beta).
\end{align*}
The following lemma gives a closed-form expression for $g^*$, where $\TopSum_k(\cdot)$ denotes the sum of the top $k$ largest elements, and $H_M: \R \to \R$ is the Huber loss function defined as
\begin{equation}
    H_M(\alpha_j) := \begin{cases}
        \frac{1}{2} \alpha_j^2 & \text{if } \vert{\alpha_j} \leq M \\
        M \vert{\alpha_j} - \frac{1}{2} M^2 & \text{if } \vert{\alpha_j} > M
    \end{cases}.
\end{equation}
For notational simplicity, we use the shorthand notation ${\bf H}_M(\balpha)$ to denote ${\bf H}_M(\balpha) = (H_M(\alpha_1), \dots, H_M(\alpha_p))$.

\begin{lemma}
    \label{lemma:fenchel_conjugate_of_g_closed_form_expression}
    The conjugate of $g$ is given by
    \begin{equation}
        g^*(\balpha) = \TopSum_k({\bf H}_M(\balpha)).
    \end{equation}
\end{lemma}
This closed-form expression enables us to compute the proximal of $g^*$. Note that while the proximal operators of both $\TopSum_k$ and ${\bf H}_M$ functions are known (see, for example, \citep[Examples~6.50 \& 6.54]{beck2017first}), the conjugate function $g^*$ is defined as the composition of these two functions.
Alas, there is no general formula to derive the proximal operator of a composition of two functions based on the proximal operators of the individual functions. In the next section, we will see how to bypass this compositional difficulty.


\subsection{Proximal operator of $g^*$}
Recall that the proximal operator of $g^*$ with weight parameter $\rho > 0$ is defined as
\begin{align}
    \label{eq:prox:g*}
    \prox_{\rho g^*}(\bm \mu) = \argmin_{\bm \alpha \in \R^p} ~ \frac{1}{2} \Vert{\bm \alpha - \bm \mu}_2^2 + \rho g^*(\bm \alpha).
\end{align}
The evaluation of $\prox_{\rho g^*}$ involves a minimization problem that can be reformulated as a convex SOCP problem. 
Generic solvers based on IPM and ADMM require solving systems of linear equations. This results in cubic time complexity per iteration, making them computationally expensive, particularly for large-scale problems. 
These methods also cannot return \emph{exact} solutions. 
The lack of exactness can affect the stability and reliability of the proximal operator, which is crucial for the convergence of the FISTA algorithm. 
Inspired by~\citep{busing2022monotone}, we present Algorithm~\ref{alg:PAVA_algorithm}, a customized pooled adjacent violators algorithm that provides an exact evaluation of $\prox_{\rho g^*}$ in linear time after performing a simple 1D sorting step.

\begin{theorem}
    \label{theorem:pava_algorithm_linear_time_complexity_and_exact_solution}
    For any $\bmu \in \R^p$, Algorithm~\ref{alg:PAVA_algorithm} returns the \textit{exact} evaluation of $\prox_{\rho g^*}(\bmu)$ in $\tilde {\mathcal O}(p)$.
\end{theorem}


\begin{algorithm}[tb]
\caption{Customized PAVA to solve $\text{prox}_{\rho g^*}(\bmu)$}
\label{alg:PAVA_algorithm}
\begin{flushleft}
\textbf{Input:} vector $\bmu$, scalar multiplier $\rho$, and threshold $M$ of the Huber loss function $H_M(\cdot)$\\
% \textbf{Output:} vector $\balpha^*=\text{prox}_{\rho g^*}(\bmu)$.
\end{flushleft}
\begin{algorithmic}[1]
    \STATE Initialize $\brho \in \mathbb{R}^n$ with $\rho_j = \rho$ if $j \in \{1, 2, ..., k\}$ and $\rho_j = 0$ otherwise.
    \STATE \COMMENT{Sort $\bmu$ such that $\vert{\mu_1} \geq \vert{\mu_2} \geq ... \geq \vert{\mu_p}$; $\bpi$ is the sorting order.}
    \STATE $\bmu, \bpi = \text{SpecialSort} (\bmu)$
    
    \STATE \COMMENT{STEP 1: Initialize a pool of $p$ blocks with start and end indices; each block initially has length equal to $1$}
    \STATE $\calJ = \{[1, 1], [2, 2], ..., [p, p]\}$

    \STATE \COMMENT{STEP 2: Initialize $\hat{\nu}_j$ in each block by ignoring the isotonic constraints}
    \FOR{$j=1, 2, \dots, p$} 
        \STATE $\hat{\nu}_j = \argmin_{\nu} \frac{1}{2} (\nu - \vert{\mu_j})^2 + \rho_j H_M(\nu)$ \label{alg_line:PAVA_algorithm_initialization_step}
    \ENDFOR
    
    \STATE \COMMENT{STEP 3: Whenever there is an isotonic constraint violation between two adjacent blocks, merge the two blocks by setting all values to be the minimizer of the objective function restricted to this merged block; use \textcolor{red}{Algorithm~\ref{alg:up_and_down_block_algorithm_for_merging_in_PAVA}} in~\ref{appendix_sec:proofs}}
    \WHILE{$\exists [a_1, a_2], [a_2+1, a_3] \in \calJ \text{ s.t. } \hat{\nu}_{a_1} < \hat{\nu}_{a_3}$}
        \STATE $\calJ = \calJ \setminus \{[a_1, a_2]\} \setminus \{[a_2+1, a_3]\} \cup \{[a_1, a_3]\}$
        \STATE $\hat{\nu}_{[a_1:a_3]} = \argmin\limits_{\nu} \sum\limits_{j=a_1}^{a_3} \left[ \frac{1}{2} (\nu - \vert{\mu_j})^2 + \rho_j H_M(\nu) \right]$ \label{alg_line:PAVA_algorithm_pooling_step}
    \ENDWHILE

    \STATE \COMMENT{Return $\hat{\bnu}$ with the inverse sorting order}
    \STATE \textbf{Return} $\text{sgn}(\bmu) \odot \bpi^{-1}(\hat{\bnu})$
    
\end{algorithmic}
\end{algorithm}

The proof relies on several auxiliary lemmas.
We start with the following lemma, which uncovers a close connection between the proximal operator of $g^*$ and the generalized isotonic regression problems.

\begin{lemma}
    \label{lemma:equivalence_between_proximal_operator_and_huber_isotonic_regression}
    For any $\bmu \in \R^p$, we have 
    $$\prox_{\rho g^*}(\bmu) = \sgn(\bmu) \odot \bnu^\star, $$ 
    where $\odot$ denotes the Hadamard (element-wise) product, $\bnu^\star$ is the unique solution of the following optimization problem
    \begin{align}
        \label{obj:KyFan_Huber_isotonic_regression}
        \begin{array}{cl}
            \min\limits_{\bnu \in \R^p} & \frac{1}{2} \sum_{j \in [p]} (\nu_j - \vert{\mu_j})^2 + \rho \sum_{j \in \calJ} H_M (\nu_j) \\[2ex]
            \st & \quad \nu_j \geq \nu_l \; \text{ if } \; \vert{\mu_j} \geq \vert{\mu_l} ~~ \forall j, l \in [p],
        \end{array} 
    \end{align}
    and $\calJ$ is the set of indices of the top $k$ largest elements of~$ \vert{\mu_j}, j \in [p]$. 
\end{lemma}
Problem~\eqref{obj:KyFan_Huber_isotonic_regression} replaces the $\TopSum_k$ in~\eqref{eq:prox:g*} from the conjugate function $g^*$ (as shown in Lemma~\ref{lemma:fenchel_conjugate_of_g_closed_form_expression}) with linear constraints.
%These constraints ensure that the elements of $\bnu$ maintain a non-decreasing order in magnitude.
While this may appear computationally complex, it actually converts the problem into an instance of isotonic regression~\citep{best1990active}. Such problems can be solved exactly in linear time after performing a simple sorting step.
The procedure is known as PAVA~\citep{busing2022monotone}. Specifically, Algorithm~\ref{alg:PAVA_algorithm} implements a customized PAVA variant designed to compute $\prox_{\rho g^*}$ exactly.
The following lemma shows that the vector generated by Algorithm~\ref{alg:PAVA_algorithm} is an exact solution to~\eqref{obj:KyFan_Huber_isotonic_regression}.
Intuitively, Algorithm~\ref{alg:PAVA_algorithm} iteratively merges adjacent blocks until no isotonic constraint violations remain, at which point the resulting vector is guaranteed to be the optimal solution to~\eqref{obj:KyFan_Huber_isotonic_regression}.

\begin{lemma}
    \label{lemma:PAVA_algorithm_exact_solution}
    The vector $\hat \bnu$ in Algorithm~\ref{alg:PAVA_algorithm} solves~\eqref{obj:KyFan_Huber_isotonic_regression} exactly.
\end{lemma}

Finally, the merging process in Algorithm~\ref{alg:PAVA_algorithm} can be executed efficiently. Intuitively, each element of $\bmu$ is visited at most twice; once during its initial processing and once when it is included in a merged block. This ensures that the process achieves a linear time complexity.

\begin{lemma}
    \label{lemma:PAVA_merging_linear_time_complexity}
    The merging step (lines 11-14) in Algorithm~\ref{alg:PAVA_algorithm} can be performed in linear time complexity $\mathcal O(p)$.
\end{lemma}
% \begin{proof}
%     The merging process in the PAVA algorithm involves iterating through the elements of the input vector $\bmu$ and merging adjacent blocks whenever an isotonic constraint violation is detected. Each element is visited at most twice: once when it is initially processed and once when it is part of a merged block. Therefore, the total number of operations is proportional to the number of elements, resulting in a linear time complexity, $O(n)$.
% \end{proof}

Armed with these lemmas, one can easily prove Theorem~\ref{theorem:pava_algorithm_linear_time_complexity_and_exact_solution}. Details are provided in~\ref{appendix_sec:proofs}.

\vspace{-2mm}
\subsection{FISTA algorithm with restart}
A critical computational step in FISTA is the efficient evaluation of the proximal operator of $g$. By the extended Moreau decomposition theorem~\citep[Theorem~6.45]{beck2017first}, for any weight parameter $\rho > 0$ and any point $\bmu \in \R^p$, the proximal operators of $g$ and $g^*$ satisfies
\begin{align}
    \label{eq:Moreaus_identity}
    \prox_{\rho^{-1} g}(\bmu) = \bmu - \rho^{-1} \, \prox_{\rho g^*} \left( \rho \bmu \right).
\end{align}
Hence, together with Theorem~\ref{theorem:pava_algorithm_linear_time_complexity_and_exact_solution}, we can compute exactly $\prox_{\rho^{-1} g}$ using Algorithm~\ref{alg:PAVA_algorithm} with log-linear time complexity. This enables an efficient implementation of the FISTA algorithm. 
Alas, the vanilla FISTA algorithm is prone to oscillatory behavior, which results in a sub-linear convergence rate of $\mathcal O(1/T^2)$ after $T$ iterations. 
In the following, we further accelerate the empirical convergence performance of the FISTA algorithm by incorporating a simple restart strategy based on the function value, originally proposed in~\citep{o2015adaptive}.

In simple terms, the restart strategy operates as follows: if the objective function increases during the iterative process, the momentum coefficient is reset to its initial value.
The effectiveness of the restart strategy hinges on the efficient computation of the loss function. This task essentially reduces to evaluating the implicit function $g$ defined in~\eqref{eq:function_g_definition}, which would involve solving a SOCP problem.
However, the value of $g$ can be computed efficiently by leveraging the majorization technique~\citep{kim2022convexification}, as shown in Algorithm~\ref{alg:compute_g_value_algorithm}.

\begin{algorithm}[tb]
    \caption{Algorithm to compute $g(\bbeta)$}
    \label{alg:compute_g_value_algorithm}
    \begin{flushleft}
    \textbf{Input:} vectors $\bbeta \in \R^p$ \\ %from Step 2 Line 8 in Algorithm~\ref{alg:main_algorithm}. 
    %\textbf{Output:} The value of $g(\bbeta)$
    \end{flushleft}
    \begin{algorithmic}[1]
        \STATE Initialize: $\bpsi = \boldsymbol{0} \in \mathbb{R}^k$
        \STATE Sort $\bbeta$ partially such that \\ \vspace{0.3em}\hspace*{2em}
        $\vert{\beta_1} \geq \vert{\beta_2} \geq ... \geq \vert{\beta_k} \geq \max\limits_{k+1, ..., p} \{ \vert{\beta_j} \}$
        % Let $\bs$ be the cumulative sum of the absolute values of sorted $\bbeta$.
        \STATE $s = \sum_{j=1}^p \vert{\beta_j}$
        \FOR{$j=1, 2, \dots, k$}
            \STATE $\overline{s} = s / (k - j + 1)$
            \STATE \textbf{if} $\overline{s} \geq \vert{\beta_j}$ \textbf{then} $\psi_{j:k} = \overline{s}$; break \textbf{else} $\psi_j = \vert{\beta_j}$
            \STATE $s = s - \vert{\beta_j}$
        \ENDFOR
        \STATE \textbf{return} $\sum_{j=1}^k \psi_j^2$
    \end{algorithmic}
\end{algorithm}


\begin{theorem}
    \label{theorem:compute_g_value_algorithm_correctness}
        For any $\bbeta \in \R^p$, Algorithm~\ref{alg:compute_g_value_algorithm} computes the exact value of $g(\bbeta)$, defined in~\eqref{eq:function_g_definition}, in $\mathcal O(p + p \log k)$.
\vspace{-3mm}
\end{theorem}
Theorem~\ref{theorem:compute_g_value_algorithm_correctness} guarantees that Algorithm~\ref{alg:compute_g_value_algorithm} can efficiently compute the value of $g(\bbeta)$, which is crucial for our value-based restart strategy to be effective in practice.
Empirically, we observe that the function value-based restart strategy can accelerate FISTA from the sub-linear convergence rate of \(O(1/T^2)\) to a linear convergence rate in many empirical results.
To the best of our knowledge, \textit{this is the first linear convergence result of using a first-order method in the MIP context} when calculating the lower bounds in the BnB tree.
The FISTA algorithm is summarized in Algorithm~\ref{alg:main_algorithm}.

\begin{algorithm}[!b]
    \caption{Main algorithm to solve problem \eqref{obj:original_sparse_problem_perspective_formulation_convex_relaxation}}
    \label{alg:main_algorithm}
    \begin{flushleft}
    \textbf{Input:} number of iterations $T$, coefficient $\lambda_2$ for the $\ell_2$ regularization, and step size $L$ (Lipschitz-continuity parameter of $\nabla F(\bbeta)$)  
    \end{flushleft}
    \begin{algorithmic}[1]
        \STATE Initialize: $\bbeta^0 = \mathbf{0}$, $\bbeta^1 = \mathbf{0}$, $\phi = 1$
        \STATE Let: $\rho = L / (2\lambda_2)$, $\calL^1 = f(\mathbf{\bbeta^1}, \by)$
        \FOR{$t=1, 2, 3, ..., T$}
            \STATE \COMMENT{Step 1: momentum acceleration} 
            \STATE $\bgamma^t = \bbeta^t + \frac{t}{t+3} (\bbeta^t - \bbeta^{t-1})$ \vspace{1.5mm}
            \STATE \COMMENT{Step 2: proximal gradient descent; use \textcolor{red}{Algorithm~\ref{alg:PAVA_algorithm}}} 
            \STATE $\bgamma^t = \bgamma^t - \frac{1}{L} \nabla F(\bgamma^t)$ \label{alg_line:gradient_descent} 
            \STATE $\bbeta^{t+1} = \bgamma^t - \rho^{-1} \text{prox}_{\rho g^*} (\rho \bgamma^t)$ \label{alg_line:proximal_step} \vspace{1.5mm} 
            \STATE \COMMENT{Step 3: restart; use \textcolor{red}{Algorithm~\ref{alg:compute_g_value_algorithm}}}
            \STATE $\calL^{t+1} = f(\bX \bbeta^{t+1}, \by) + 2 \lambda_2 g(\bbeta^{t+1})$
            \STATE \textbf{if} $\calL^{t+1} \geq \calL^{t}$ \textbf{then} $\phi = 1$ \textbf{else} $\phi = \phi + 1$
        \ENDFOR
        \STATE \textbf{return} $\bbeta^{T+1}$
    \end{algorithmic}
\end{algorithm}


\vspace{-3mm}
\subsection{Safe Lower Bounds for GLMs}
\label{subsec:safe_lower_bounds_for_glms}
We conclude this section by commenting on how to use Algorithm~\ref{alg:main_algorithm} in the BnB tree to prune nodes.
As an iterative algorithm, FISTA yields only an approximate solution $\hat{\bbeta}$ to~\eqref{obj:original_sparse_problem_perspective_formulation_convex_relaxation}. Consequently, while we can calculate the objective function for $\hat{\bbeta}$ efficiently, this value is not necessarily a lower bound of the original problem--only the optimal value of the relaxed problem~\eqref{obj:original_sparse_problem_perspective_formulation_convex_relaxation} serves as a guaranteed lower bound.
To get a safe lower bound, we rely on the weak duality theorem, in which for any proper, lower semi-continuous, and convex functions $F: \R^n \to \R \cup \{\infty\}$ and $G: \R^p \to \R \cup \{\infty\}$, we have
\begin{align*}
    \inf_{\bbeta \in \R^p} F(\bX \bbeta) + G(\bbeta) 
    &\geq \sup_{\bzeta \in \R^n} - F^*(-\bzeta) - G^*(\bX^\top \bzeta) \\ 
    &\geq - F^*(-\bzeta) - G^*(\bX^\top \bzeta) \,~ \forall \bzeta \in \R^n\!\!,
\end{align*}
where $F^*$ and $G^*$ denote the conjugates of $F$ and $G$, respectively, while the second inequality follows from the definition of the supremum operator. Letting $F(\bX \bbeta) = f(\bm X \bm \beta, \bm y)$, $G(\bbeta) = 2 \lambda_2 g(\bbeta)$ and $\bzeta = \nabla F(\bX \hat \bbeta)$, where $\hat \bbeta$ is the output of the FISTA Algorithm~\ref{alg:main_algorithm}, we arrive at the safe lower bound
\begin{align}
    \label{eq:fenchel_duality_theorem_F_y(Ax)+G(x)}
    P_{\text{MIP}}^\star \geq - F^*(-\hat{\bzeta}) - G^*(\bX^\top \hat{\bzeta}),
\end{align}
where the inequality follows from the relaxation bound $P_{\text{MIP}}^\star \geq P_{\text{conv}}^\star$ and the weak duality theorem.
For convenience, we provide a list of $F^*(\cdot)$ for different GLM loss functions in~\ref{appendix_sec:convex_conjugate_for_GLM_loss_functions}.
The readers are also referred to~\ref{appendix_sec:safe_lower_bound_more_discussions},  where we derive the safe lower bound for the linear regression problem with eigen-perspective relaxation as an example.
\section{Sketch of Possible Applicatons of the Iterative Data-based V-model}\label{04_Examples}

In this section, the application of the proposed iterative data-based V-model is illustrated by means of automated driving at various levels of granularity, thus also the system's complexity. Thereby, the application illustration does not represent practical examples. Rather, the illustration serves as an academic explanation of the usability and possible practical application of the introduced process reference framework. Consequently, the focus is on the description of the process, i.e., how to proceed according to the process reference framework, and not on the result of a specific example. Moreover, a comprehensive description of a practical application would entail a detailed description of the system requirements, the database, etc., which would go way beyond the scope of this paper and impair clarity. As a result, this section concentrates on the application illustration of the process.

In this context, the generally described functionality and the usability across different system levels is sketched. The following Table \ref{tab:exampl_design} illustrates the design phase, while continued Table \ref{tab:exampl_VV} focuses on the V\&V phase of the iterative data-based V-model. These examples outlined in the Table \ref{tab:exampl_design} and continued part of Table \ref{tab:exampl_VV} provide high-level considerations and do not claim to be exhaustive. Rather, the purpose of these illustrations is to provide a clear and understandable intuition for the application of the methodology outlined in Section \ref{03_Methodology}. The application of automated driving within a highway operation ODD is used for this purpose. Subsequently, the levels considered comprise the overall system, here the automated driving (AD) stack, a dedicated subsystem of the overal system, namely the perception and the lidar detector component as one of the sensor components for automated driving.

On the one hand, this demonstrates how the framework can be applied at different levels of granularity. On the other hand, it illustrates that Waymo's layered approach \cite{webb2020waymo} and the multi-perspective approach \cite{VVMAPerspectives} of the VVM project are implicitly taken into account in an application-independent single-perspective methodology. This is reflected, for instance, in the closed-loop evaluation at the system and component level, which enables the identification of necessary improvements at the overall system level. Apart from this particular case, it can also be seen in general how, in addition to refinement within a system level, refinement is organized across different system levels.

\begin{table*}[p!]
	\centering
	\caption{Potential application of the Iterative Data-based V-model to Automated Driving Applications at different Levels of Granularity.}
	\begin{tabularx}{\linewidth}{p{3.6cm} *{3}{>{\raggedright\arraybackslash}X}}
		\toprule
		\diagbox[width=4cm, height=1.5cm]{\textbf{Stages}\\\textbf{of the iterative} \\ \textbf{data-based V-model}}{\textbf{Automated driving at}\\\textbf{different levels of}\\\textbf{granularity}} & \multicolumn{1}{c}{\textbf{AD Stack}} & \multicolumn{1}{c}{\textbf{Perception}}  & \multicolumn{1}{c}{\textbf{Lidar Detector}} \\
		\toprule
		\textbf{Operational Design Domain (ODD)} &
		\multicolumn{3}{>{\hsize=\dimexpr3\hsize+4\tabcolsep+\arrayrulewidth\relax}X}{ $\bullet$ German highways; $\bullet$ ego vehicle speed of up to 120 km/h; $\bullet$ other road users speed of up to 160 km/h; $\bullet$ from traffic jams to high-speed traffic; $\bullet$ varying traffic densities and weather conditions;}\\
		\midrule
		\textbf{Function-specific ODD} 
		& $\bullet$ requirement specification of sensing, planning and acting functionalities within the ODD; $\bullet$ maneuvers (overtaking, lane changing, merging, exiting) within the ODD and legal requirements; $\bullet$ different conditions (varying weather, traffic jams, emergency vehicles, pedestrians along the highway ODD);  
		& $\bullet$ object detection, classification and tracking within the ODD; $\bullet$ obstacle and lost cargo detection; $\bullet$ lane and free space detection; $\bullet$ detection and classification of traffic signs and traffic lights;
		& $\bullet$ object and free space detection within the ODD; \\
		\midrule
		\textbf{Data-specific ODD} 
		& $\bullet$ requires perception data with a range of at least 200 meters to the front and rear and 100 meters to the left and right; $\bullet$ vehicle state estimates (position in global coordinates, velocities, accelerations in the navigation coordinates, heading and yaw rate relative to the navigation coordinate system); $\bullet$ highway HD maps; $\bullet$ information processing at a rate of 10 Hz; 
		& $\bullet$ requires vehicle state estimation in standard signal rates and and characteristic signal noise; $\bullet$ provide an environment model for the region of interest in order to make informed behavioral decisions; $\bullet$ information processing at a rate of 10 Hz; 
		& $\bullet$ requires lidar raw data with characteristic signal noise and a range coverage of 150 meters; $\bullet$ provide object and free space detections at a rate of 10 Hz; \\
		\midrule
		\textbf{Architectural Design Domain} 
		&$\bullet$ domain of possible AD stack architectures (classic architectures with dedicated interfaces between individual submodules, architectures with orchestrators that take control of the interface, architectures that are fully end-to-end learned); 
		& $\bullet$ domain of possible perception architectures (permanent 360-degree perception, situation dependent perception); 
		& $\bullet$ domain of possible lidar detector architectures (range of deep neural network architectures); \\
		\midrule
		\textbf{Architecture Definition} 
		& $\bullet$ definition includes a specific instance from the range of possible architectures (e.g. an innovative architecture based on an orchestrator); $\bullet$ deployment plan on individual ECUs; 
		& $\bullet$ definition represents a specific case from the spectrum of possible architectures (e.g. situation dependent perception); $\bullet$ consideration of the use on individual ECUs; 
		& $\bullet$ definition determines the choice of  the deep neural network architecture within the given spectrum; $\bullet$ includes architecture-specific definitions (layers, activation functions, properties of the head, e.g., deterministic or probabilistic outputs);  \\
		\midrule
		\textbf{Data Design Domain} 
		& $\bullet$ defines general requirements for the creation of datasets both in simulation and the real world; $\bullet$ requires the generation of data according to the data-specific ODD; $\bullet$ covering different velocities, traffic densities, various times of day, and changing environmental conditions; $\bullet$ data from traffic jams, emergency situations, pedestrians within the ODD, corner cases (e.g. accidents, animals on the road); $\bullet$ defines train test split;
		& $\bullet$ demands the generation of sensor and state estimation data according to the data-specific ODD; $\bullet$ coverage of different velocities, traffic densities, different times of day, changing environmental conditions; $\bullet$ data from traffic jams, emergency situations, pedestrians within the ODD, corner cases; $\bullet$ defines train test split;
		& $\bullet$ specifies the sensor type, the field of view, the amount of layers and the mounting position to fulfill the data-specific ODD; $\bullet$ specifies lidar resolution, frame rate, and latency; $\bullet$ coverage of various environmental conditions (weather, traffic) within the ODD; $\bullet$ defines train test split;\\
		\midrule
		\textbf{Sim. Dataset Generation} 
		& $\bullet$ involves systematically generating simulation data considering, sensor specifics, vehicle dynamics, traffic simulation, scenario generation; $\bullet$ generates data with specific resolution, frame rate, and latency; $\bullet$ required to ensure the scenario coverage matches the requirements of the data design domain;
		& $\bullet$ generates simulation data by means of a simulation software considering, realistic objects, obstacles, lanes, free spaces and sensor models; $\bullet$ generates data with ground truth annotations, defined resolution, frame rate, and latency; $\bullet$ covers cases according to the requirements specification of the data design domain;
		& $\bullet$ generates data according to the data design domain with a specific scan frequency; $\bullet$ consideres physics of lidar sensors, including laser beam emission, reflection, and reception, to generate accurate point cloud representations; $\bullet$ accounts for sensor noise and environmental influences (fog, rain and blending);
		\\
		\midrule
		\textbf{System Development} 
		& $\bullet$ realization of the automated driving stack including various submodules with respect to the architecture definition; $\bullet$ use of the created simulation scenarios and corresponding data obtained from the sim. dataset generation;
		& $\bullet$ implementation and realization of the selected perception architecture; $\bullet$ use of the created simulation and corresponding data derived from the sim. dataset generation;
		& $\bullet$ parameterization and implementation of a selected lidar detector neural network according to the architecture definition; $\bullet$ training of the neural network using the data from the sim. dataset generation; \\
		\bottomrule	
	\end{tabularx}
	\label{tab:exampl_design}
\end{table*}


\begin{table*}[h!]
	\ContinuedFloat
	\centering
	\caption{Potential application of the Iterative Data-based V-model to Automated Driving Applications at different Levels of Granularity (continued).}
	\begin{tabularx}{\linewidth}{p{3.6cm} *{3}{>{\raggedright\arraybackslash}X}}
		\toprule
		\diagbox[width=4cm, height=1.5cm]{\textbf{Stages}\\\textbf{of the iterative} \\ \textbf{data-based V-model}}{\textbf{Automated driving at}\\\textbf{different levels of}\\\textbf{granularity}}&	\multicolumn{1}{c}{\textbf{AD Stack}} & \multicolumn{1}{c}{\textbf{Perception}}  & \multicolumn{1}{c}{\textbf{Lidar Detector}} \\
		\toprule
		\multirow{2}{*}{\textbf{System Evaluation}}
		& $\bullet$ performs evaluation on the test dataset split defined within the data design domain definition; $\bullet$ considers among others, functional testing of the realized automated driving stack, scenario testing, regulatrory compliance testing, etc.; 
		& $\bullet$ evaluates realized perception on the test dataset; $\bullet$ analysis of object detection accuracy on divers objects under various conditions; 
		& $\bullet$ evaluates object and free space detections under various ODD-specific circumstances; \\
		\cline{2-4}
		&\multicolumn{3}{>{\hsize=\dimexpr3\hsize+4\tabcolsep+\arrayrulewidth\relax}X}{
			$\bullet$ entails the simulative performance assessment based on performance indicators and acceptance criteria derived from the individual safety argumentation; $\bullet$ evaluates trigger conditions according to the specification of the data design domain and correspondingly generated data;
		} \\
		\midrule
		\textbf{Real World Dataset Generation} 
		& $\bullet$ gathering of relevant real world measurement data, vehicle dynamics characteristics, system latencies and actuator properties;
		& $\bullet$ collecting real raw sensor data required to create the environment model along specified real world state estimates;
		& $\bullet$ acquisition of real world data via a dedicated sensor module that meets the requirements; \\
		\cline{2-4}
			&\multicolumn{3}{>{\hsize=\dimexpr3\hsize+4\tabcolsep+\arrayrulewidth\relax}X}{ $\bullet$ generates data from the real world corresponding to the respective data design domain; $\bullet$ coverage of the data design domain is defined system dependant according to the individual safety argumentation that considers risks and costs of generating real world data, particularly with regard to corner cases and hazardous situations;
		} \\
		\midrule
		\textbf{System Transfer} 
		& $\bullet$ adjusting individual (sub-)modules of the automated driving stack and their interaction by means of acquired real-world data to compensate the gap between simulation and real world;
		& $\bullet$ specifics within the chosen perception architecture are adjusted using generated real world data; 
		& $\bullet$ parameterization of the lidar detector's neural network is adapted, e.g. by fine-tuning using real data;  \\
		\midrule
		\textbf{Transfer Evaluation}	
		& $\bullet$ performance assessment evaluation of adapted automated driving stack system using real data in alignment with the individual safety argumentation as in the previous evaluation; 
		& $\bullet$ evaluation of the performance of object detection, classification, tracking, obstacle detection, lost cargo detection etc. using the respective real world test dataset split; & $\bullet$ evaluation of the performance of the lidar detector based on the test split of the generated real data; \\
		\midrule
		\textbf{Open-Loop Evaluation (Silent Testing)} 
		& $\bullet$ evaluation of sensing, planning, acting and interactions of functionalities on real hardware in the real world in partial separation to prevent deployment in the real world while still being able to evaluate the capabilities of the AD stack;
		& $\bullet$ evaluation of the perception systems object detection, classification, tracking etc. in the real world without feeding the output back into the overall system;
		& $\bullet$ evaluation of the lidar detector in the real world without further use of the output in other functionalities; \\
		\midrule
		\textbf{Closed-Loop Evaluation} 
		& $\bullet$ evaluation of sensing, planning, acting functionalities on real hardware in the real world in full functional interplay and sytem's interaction with the real world; $\bullet$ limited to operations within the proving ground; 
		&\multicolumn{2}{>{\hsize=\dimexpr2\hsize+3\tabcolsep+\arrayrulewidth\relax}X}{
			$\bullet$ evaluation is limited to operation within the proving ground; $\bullet$ perception and lidar detection, unlike planning, is usually open-loop and therefore should not lead to changes; $\bullet$ if this closed-loop evaluation differs from the open-loop evaluation from the previous step, it indicates that related components within the automated vehicle stack need to be refined;
		} \\
		\midrule
		\textbf{Field Operation Evaluation} &\multicolumn{3}{>{\hsize=\dimexpr3\hsize+4\tabcolsep+\arrayrulewidth\relax}X}{ $\bullet$ previous restriction to the proving ground is removed; $\bullet$ performance assessment takes place in field operation within the ODD; $\bullet$ enables a broader coverage of possible test scenarios and influences; 
		} \\
		\midrule
		\textbf{Deployment - System Operation \& Monitoring} &	\multicolumn{3}{>{\hsize=\dimexpr3\hsize+4\tabcolsep+\arrayrulewidth\relax}X}{ $\bullet$ with increasing time and scale of use, continuous trust builds up in the system; $\bullet$ continuous observation and analysis of system performance; $\bullet$ conducting intelligent data harvesting during operation; 
		} \\
		\midrule
		\textbf{Detection of Deficiencies} &	\multicolumn{3}{>{\hsize=\dimexpr3\hsize+4\tabcolsep+\arrayrulewidth\relax}X}{ $\bullet$ detection of deficiencies on the basis of a strong link between system observation and safety argument \& acceptance criteria based identification; $\bullet$ uncovering of "unknown unknowns"; $\bullet$ withdrawal of approval, if necessary; $\bullet$ design and V\&V cycle reinitiation 
		} \\
		\midrule
		\textbf{Continious Refinement} &	\multicolumn{3}{>{\hsize=\dimexpr3\hsize+4\tabcolsep+\arrayrulewidth\relax}X}{ $\bullet$ deficiencies uncovered during operation are fed back into the process and the product via a new overall cycle; $\bullet$ findings are transferred to the simulation, the product is refined and gradually reintroduced into the real world; $\bullet$ exploits the safety and efficiency-related benefits of simulation; $\bullet$ ensures uniform safety standards in every development and improvement cycle;
		} \\
		\bottomrule	\end{tabularx}
	\label{tab:exampl_VV}
\end{table*}




\section{Conclusion}
We introduce the Difficulty and Uncertainty-Aware Lightweight (DUAL) score, a novel scoring metric designed for cost-effective pruning. The DUAL score is the first metric to integrate both difficulty and uncertainty into a single measure, and its effectiveness in identifying the most informative samples early in training is further supported by theoretical analysis. Additionally, we propose pruning-ratio-adaptive sampling to account for sample diversity, particularly when the pruning ratio is extremely high. Our proposed pruning methods, DUAL score and DUAL score combined with Beta sampling demonstrate remarkable performance, particularly in realistic scenarios involving label noise and image corruption, by effectively distinguishing noisy samples.

Data pruning research has been evolving in a direction that contradicts its primary objective of reducing computational and storage costs while improving training efficiency. This is mainly because the computational cost of pruning often exceeds that of full training. By introducing our DUAL method, we take a crucial step toward overcoming this challenge by significantly reducing the computation cost associated with data pruning, making it feasible for practical scenarios. Ultimately, we believe this will help minimize resource waste and enhance training efficiency.

\bibliographystyle{IEEEtran}
\bibliography{literature}
\vspace{-15 mm}
\begin{IEEEbiography}[{\includegraphics[width=1in,height=1.25in,clip,keepaspectratio]{img/Author1_Ullrich}}]{Lars Ullrich}
	received the M.Sc. degree in mechatronics from Friedrich–Alexander–Universita\"at Erlangen–N\"urnberg, Germany, in 2022, where he is currently pursuing the Ph.D.
	(Dr.Ing.) degree with the Chair of Automatic Control.
	
	His research interests include probabilistic trajectory planning for safe and reliable autonomous driving in uncertain dynamic environments with a focus on addressing challenges arising from the use of AI systems in automated driving.
\end{IEEEbiography} \vspace{-15 mm}
\begin{IEEEbiography}[{\includegraphics[width=1in,height=1.25in,clip,keepaspectratio]{img/Author2_Buchholz}}]{Michael Buchholz}
	received his Diploma degree in Electrical Engineering and Information Technology as well as his Ph.D. from the faculty of Electrical Engineering and Information Technology at University of Karlsruhe (TH)/Karlsruhe Institute of Technology, Germany.  He is a research group leader and lecturer at the Institute of Measurement, Control, and Microtechnology, Ulm University, Ulm, 89081, Germany, where he earned his ``Habilitation'' (post-doctoral lecturing qualification) for Automation Technology in 2022. His research interests comprise connected automated driving, electric mobility, modelling and control of mechatronic systems, and system identification.
\end{IEEEbiography} \vspace{-15 mm}
\begin{IEEEbiography}[{\includegraphics[width=1in,height=1.25in,clip,keepaspectratio]{img/Author3_Dietmayer}}]{Klaus Dietmayer}
	(Senior Member, IEEE) earned his degree in electrical engineering from the Technical University of Braunschweig, Germany and completed his Ph.D. in 1994 at the University of the Armed Forces, Hamburg, Germany. Afterwards, he began his industrial career as a research engineer at Philips Semiconductors, Hamburg, progressing through various roles to become the manager for sensors and actuators in the automotive electronics division. 
	
	In 2000, Dietmayer was appointed as a Professor of Measurement and Control at the University of Ulm. He currently serves as the Director of the Institute for Measurement, Control, and Microtechnology within the School of Engineering and Computer Science.
	
	His primary research interests include information fusion, multi-object tracking, environment perception, situation assessment, and behavior planning for autonomous driving. The institute operates three automated test vehicles with special licenses for public road traffic, along with a test intersection equipped with infrastructure sensors for evaluating automated and networked cooperative driving in Ulm.
\end{IEEEbiography} \vspace{-15 mm}
\begin{IEEEbiography}[{\includegraphics[width=1in,height=1.25in,clip,keepaspectratio]{img/Author4_Graichen3}}]{Knut Graichen}
	(Senior Member, IEEE) received the Diploma-Ing. degree in engineering cybernetics and the Ph.D. (Dr.-Ing.) degree from the University of Stuttgart, Stuttgart, Germany, in 2002 and 2006, respectively.
	
	In 2007, he was a Post-Doctoral Researcher with the Center Automatique et Syst\`emes, MINES ParisTech, France. In 2008, he joined the Automation and Control Institute, Vienna University of Technology, Vienna, Austria, as a Senior Researcher. In 2010, he became a Professor with the Institute of Measurement, Control and Microtechnology, Ulm University, Ulm, Germany. Since 2019, he has been the Head of the Chair of Automatic Control, Friedrich–Alexander–Universita\"at Erlangen–N\"urnberg, Germany. His current research interests include distributed and learning control and model predictive control of dynamical systems for automotive, mechatronic, and robotic applications.
	
	Dr. Graichen is the Editor-in-Chief of Control Engineering Practice.
\end{IEEEbiography}\newpage



%\begin{biography}{Lars Ullrich}
%	text
%	% there must be enough text in the first paragraph to flow around the
%	% photo!
%	
%	text
%\end{biography}


%\begin{biography}{Lars Ullrich}[0mm]{file.eps}
%	text
%	% there must be enough text in the first paragraph to flow around the
%	% photo!
%	
%	text
%\end{biography}



\end{document}


