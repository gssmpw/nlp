\section{Conclusion}\label{05_Conclusion}

The iterative data-based V-model is a process reference model with the generic claim of the classical V-model. However, the challenges of complex systems that include AI are explicitly addressed in order to provide a development and V\&V framework that can serve as a generic guideline. In this way, the iterative data-based V-model can be adapted to current and future challenges.

On the one hand, the development process of AI systems is mapped in a more explicit form, without excluding traditional systems. On the other hand, greater emphasis is placed on data, which is firmly integrated into the process. The data makes it possible to bring the product and the process, the real world and the simulation world, as well as the prospective and the retrospective perspective to a common ground. This enables the diverse perspectives of other frameworks to be mapped into one application-independent single-perspective framework. In addition, the iterative data-based V-model allows different granularities of the system level to be taken into account and thus a stronger generality. Although cognitive cyber-physical systems such as autonomous systems (e.g. fully automated vehicles) are explicitly addressed, simpler and less safety-critical systems can also be developed and validated by means of this framework.

The iterative data-based V-model is also characterized by the fact that statistical methods can, but do not have to, be given greater consideration during development, verification, validation, and continuous improvement. Openness to traditional safety argumentation methods and to statistical methods that take account of the inherent complexity is equally guaranteed.

In addition, the challenges of the open world are taken into account through the iterative process, thus incorporating the continuous integration approach. Furthermore, this represents a V\&V of the process and the associated customizable safety argumentation and ultimately empowers a comprehensive safety assurance. In particular, the retrospective approach makes it possible to strengthen confidence in the completeness of the data, the performance metrics, and the release criteria, in other words in the product and the process, and ultimately in the ambitious goals that have been defined.

To summarize, the iterative data-based V-model takes up a variety of different approaches and methods and formalizes the fundamental ideas. This requires a decoupling from the safety argumentation and at the same time opens up the consideration of various applications and procedures, similar to the classical V-model.

