\section{Introduction}\label{01_Introduction}
\IEEEPARstart{T}{he} V-model \cite{brohl1993v} has served as a valuable tool for the safe and reliable development, verification, validation, and introduction of technical systems in the past. However, it is only suitable to a limited extent for the complex systems that are emerging today and will be in the future. In addition to the generally increased complexity of systems, the growing prevalence of artificial intelligence (AI) poses a particular challenge, even for more recent versions of the V-model \cite{graessler2018v}, \cite{GraesslerHentze}.

AI offers the advantage of mapping the system behavior without explicitly extracting and modeling relationships, instead emulating the input-output behavior based on data. This characteristic, along with technological advancements in artificial intelligence, holds promise for achieving higher levels of autonomy, which is why AI is increasingly being incorporated. At the same time, the transition from mathematically-explicit systems to data-based implicit systems present several challenges \cite{amodei2016concrete}, \cite{neto2022safety}, especially in safety-critical systems \cite{kurd2007developing, forsberg2020challenges} like automated driving where the provability of proper functionality is crucial and required \cite{europeancommissionaiact}. To address these challenges, the use of real world as well as systematically generated synthetic simulation data becomes indispensable in the development \cite{KIDeltaSynData} and approval processes \cite{KIAbsicherungSynData}, ensuring both safety and cost-effectiveness. This necessitates a process realignment to effectively consider and use various types of data.

Especially the need for both economic efficiency and safety urges the development of new innovative process reference models for the product development lifecycle alongside with continuous integration as well as addressing safeguarding and approval. This has initiated numerous research projects dealing with these crucial aspects, particularly in the field of automated driving \cite{PATH, SAKURA, V4SAFETY, PEGASUS, HIDrive, KIFamilie, LOPAAS, VIVID, SUNRISE}. In parallel, also the industry is actively engaged in addressing relevant topics to enable the approval of their systems \cite{karpathy_cvpr21, favaro2023building}. 

This paper is intended to link and bridge the different perspectives and methodologies by highlighting their commonalities and ultimately formalizing a unifying process reference model. Accordingly, the main contribution is threefold:

\begin{itemize}
	\item We present an analysis of existing development processes along with a comparison with regard to complex systems incorporating AI and a subsequent discussion of the current state of the art.
	\item We propose a formalizing, harmonizing, and generalizing framework entitled "iterative data-based V-model" expanding the classical V-model \cite{brohl1993v} for the development of complex systems that include AI.
	\item To illustrate the use of the proposed framework without exceeding the scope, we sketch the application of the concept to the use case of automated driving at different levels of granularity.
\end{itemize}

Accordingly, it should be emphasized that this paper is not a comprehensive description of a process that includes a fully-fledged safety assurance directly leading to approval. Instead, the purpose of this paper is to introduce a formalized process reference model for the entire product lifecycle at a macro level that aims to establish a general framework for the development, verification, validation, and continuous integration of complex systems incorporating AI. Specifically, the proposed iterative, data-based V-model is intended to represent a generic and structured process, similar to the classical V-model \cite{brohl1993v}, that describes the process but leaves the safety argumentation and assessment custimizable in order to achieve the desired generality and transferability, ultimately providing an expanded guidance for complex systems featuring AI.

The paper is structured as follows: The related work outlines relevant research projects in Section \ref{02_StateOfTheArt}. Building on this, the analysis of innovative development processes in Section \ref{02_New} investigates a selection of methods in relation to the classical V-model and compares them concerning complex systems. The methodology of the iterative data-based V-model is elaborated, classified, delimited and summarized in Section \ref{03_Methodology}. The usability and application of the proposed methodology are abstractly sketched in Section \ref{04_Examples} using an academic use case of automated driving. Finally, the reference process model presented is discussed and conclusions are drawn in Section \ref{05_Conclusion}.



