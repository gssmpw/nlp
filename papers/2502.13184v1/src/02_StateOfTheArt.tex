\section{Related Work}\label{02_StateOfTheArt}
This section provides a general overview of relevant research projects and highlights a selection that is to be examined in more detail in the subsequent section.  

The Pegasus family \cite{PEGASUS} comprises the Pegasus \cite{PEGASUS} project as well as the successor projects Verification Validation Methods (VVM) \cite{VVM} and SetLevel \cite{SETLevel}. While the SetLevel project focused on the particular challenge of simulation-based development, the VVM project built upon the decomposition of the operational design domain (ODD) into logical core scenarios established by the Pegasus project and extends this methodology in a holistic manner. Consequently, the overall methodology of VVM \cite{VVMOverall} represents an extension of the traditional V-model. A similar approach is being pursued by the Japan Automobile Manufacturers Association (JAMA) with its Automated Driving Safety Evaluation Framework \cite{JAMAFramework}. This framework also constitutes a scenario-based approach and an enhanced V-model structure. The backbone of these scenario-based approaches are scenario databases, for instance from the ADScene \cite{guyonvarch2023adscene}, Safety Pool \cite{SafetyPool}, and SAKURA \cite{SAKURA} projects. Moreover, the similarity in the approaches results in part from close exchange and cooperation. For example, as part of the German-Japanese VIVID \cite{VIVID} project, the national sub-projects VIVALDI and DIVP \cite{DIVP} were conducting joint research on virtual validation. 

In the context of scenario-based approaches, the theoretical framework of Scenarios Engineering (SE) \cite{li2022novel, li2022features} is also worth mentioning. This framework aims to improve the visibility, interpretability, and reliability of intelligent systems and, thus, contributes to the realization of trustworthy AI. Beyond this, the emergence of video generation models, such as \mbox{OpenAI's} Sora, is addressed in \cite{li2024sora} as a step towards imaginative intelligence, while discussing its relevance in the context of SE. As outlined, the advancement supports the training and testing of intelligent vehicles by reducing the need for physical recordings \cite{li2023novel} and expanding the variety of scenarios. In addition, \cite{wang2024does} elaborates on the opportunities that imaginative intelligence offers for SE, including tackling the long-tail problem, while also outlining the challenges that currently remain in accurately modeling physics and understanding causality. Thus, \cite{li2024sora} and \cite{wang2024does} take into account recent developments and present a perspective of future developments. In this regard, the NXT GEN AI METHODS \cite{NextAIM} project should also be noted, which is dedicated to generative AI and, more specifically, the development of foundation models for automated driving.

Furthermore, there are several projects like StreetWise \cite{elrofai2018streetwise}, HEADSTART \cite{HEADSTART}, SAFE-UP \cite{SAFEUP}, and AI Safeguarding \cite{ki_absicherung} that focused on validation, safety assessment, and safety assurance within the field of automated driving. Moreover, projects such as V4SAFETY \cite{V4SAFETY} and SUNRISE \cite{SUNRISE} aim to develop more comprehensive frameworks for safety assessment and assurance, respectively. The LOPAAS \cite{LOPAAS} project between Fraunhofer IESE, Fraunhofer IKS, and the Assuring Autonomy International Program (AAIP) at the University of York also targets to bring about a paradigm shift in safety engineering for autonomous systems. Hi-Drive \cite{HIDrive}, another project, addresses the various ODD challenges but concentrates on reaching a higher level of autonomy. Consequently, a multitude of projects is anchored in this context. 

However, the VVM \cite{VVM} project is of particular interest as it can be considered the most advanced representative of the improved V-model approaches and outlines an overall methodology. Of particular interest is also the methodology of Waymo \cite{favaro2023building}, which differs in focus from the VVM \cite{VVM} project. In addition to the different focus, the methodology is of particular interest due to the resulting comparatively solid performance \cite{kusano2023comparison} of the automated vehicles in comparison to other competitors \cite{equipmentrecallreport, NHTSARecall23E, NHTSARecallLetter}. Beyond that, Tesla's approach \cite{karpathy_cvpr21} is technologically different and benefits from a large number of vehicles in the field that generate data. As a consequence, the methodology is based on an AI-centric development process, similar to processes of other AI systems such as OpenAI's ChatGPT. For this reason, the general approach, also known as the data engine is also worth emphasizing. Ultimately, the three highlighted processes are to be analyzed in more detail subsequently.

\section{Analysis of Innovative Development Processes}\label{02_New}
While the previous section provides a general overview, this section is dedicated to the analysis of a selection of innovative development processes. Thereby, the different advancements and perspectives are related to the classical V-model in order to set a basis for the subsequent creation of a generalized framework. For this purpose, the VVM \cite{VVM} project, Waymo's methodology \cite{favaro2023building} and Tesla's approach \cite{karpathy_cvpr21} are examined in more detail below. In addition, the methodologies are compared in terms of their general applicability to complex systems. Conclusively, the application areas of the available frameworks and the existing gap in the overall context are discussed.  

\subsection{VVM Project: Argumentation- \& Scenario-based V-model}\label{VVM}

The VVM project enhances the traditional V-model \cite{brohl1993v} by integrating multiple perspectives. First, it employs the ODD decomposition from Pegasus \cite{schuldt2013effiziente, pegasus_schlussbericht} to construct an ODD metamodel \cite{scholtes20216, reich2023concept}, facilitating scenario-based design and Verification \& Validation (V\&V) processes \cite{elster2021fundamental}. This method effectively maps infinite scenarios of the open world into manageable test spaces and design foundations \cite{neurohr2021criticality}. Second, a coherent assurance argumentation \cite{VVMAssurance} targets the mitigation of unreasonable risks in the open world, fostering consistency and traceability along the V-model. Consequently, the project establishes consistent interfaces across the framework and enables seamless requirements considerations from design to verification via argumentation-based V\&V. Third, a multi-perspective approach \cite{VVMAPerspectives} addresses systematic gaps such as specification, implementation, and validation gaps, aiming at uncertainty and risk reduction \cite{stellet2019formalisation}. This involves decomposing the overall system into various levels of abstraction, incorporating capability, engineering, and real-world layers, and employing different perspectives like design, V\&V, risk management, and argumentation.

\begin{figure}[]
	\centering	
	\includegraphics[width=\linewidth]{img/VVM_UL_X.png}
	\caption{Visualization of the overall methodology of the VVM project \cite{VVMOverall} as an extension of the classical V-model designed for automated driving application with regard to a scenario-based problem decomposition and an appropriate safety argumentation.}
	\label{fig:VVM}
\end{figure}

Finally, the overall methodology culminates to an advanced V-model \cite{VVMOverall}, shown in Figure \ref{fig:VVM}, that is customized to the scenario-based logic introduced by Pegasus. 

The overall methodology culminates in an advanced V-model \cite{VVMOverall}, as depicted in Figure \ref{fig:VVM}, customized to the scenario-based logic introduced by Pegasus. The ODD metamodel forms the top of the model, inducing the scenario-based philosophy throughout the V-model process. Problem space analysis provides the basis for detailed specification, considering the environment and the specific automated driving system. Furthermore, to ensure regulatory compliance, normative behavior \cite{salem2022beitrag} is specified, encompassing certification, legal, social, and ethical expectations. Subsequently, systematic hazard  \cite{graubohm2020towards} and risk identification are conducted based on the ODD metamodel \cite{VVMOverall}, problem space analysis, and normative behavior specification. Corresponding safety measures to determine an acceptable residual risk are then defined through risk treatment, closely coordinated with the underlying sub-framework, the risk management core \cite{salem2023risk}.

After implementation, a scenario-based verification, validation, and risk assessment is performed \cite{riedmaier2020survey}. This process incorporates three key improvements over the classical V-model: access to the ODD metamodel for aligned analysis, assurance assessment throughout V\&V, and evaluation of residual risk \cite{VVMOverall}. In summary, the VVM project provides an extended, detailed, and tailored process reference model for scenario-based development and V\&V of automated vehicles.

\subsection{Waymo: Safety Determination Lifecycle}\label{Waymo}

While Waymo's safety determination lifecycle also strives for the absence of unreasonable risk, it places a stronger emphasis on the lifecycle \cite{favaro2023building}. Furthermore, similar to the VVM project \cite{VVMAPerspectives, stellet2019formalisation}, the framework encompasses different perspectives. Overall, the framework can be summarized as a layered, credible, and dynamic approach \cite{favaro2023building}.

The \textit{\textbf{layered approach}} within \cite{favaro2023building} refers to a division into an architectural (formerly known as hardware \cite{webb2020waymo}), a behavioral, and an in-service operational layer. Thereby, in order to demonstrate the absence of unreasonable risk, hazards and appropriate acceptance criteria are defined within the aforementioned layers. This involves the definition of several dimensions of interest covering a diverse set of aspects, e.g. the avoidance of incidents, the successful completion of automated journeys or the compliance with driving rules. Indicators of interest are defined on this basis, which map hazards to an explicit set of acceptance criteria. Through the definition of the minimum dimensions of interest, Waymo determines the completeness of the set of acceptance criteria in order to underpin credibility.  

The \textit{\textbf{credible approach}} adresses concerns about the reasonableness and trustworthiness of the claim-argument-evidence structure via Waymo's novel Case Credibility Assessment (CCA) \cite{favaro2023building}. The CCA comprises three components: the top-down credibility of the argument, the bottom-up credibility of the evidence, and the encompassing implementation of the credibility. Overall, the CCA procedure comprises a continuous revision by monitoring and updating the arguments and evidence to achieve credibility.

The top-down credibility of the arguments focuses on fulfilling overarching objectives through assessing the suitability and reasonableness of the arguments. This involves evaluating and refining a collection of arguments and acceptance criteria. Additionally, it entails justifying acceptance criteria and conducting a suitability assessment of performance indicators and associated objectives to evaluate the reasonableness. In contrast, the bottom-up credibility of the evidence concentrates on evaluating the evidence provided by the methodology. This entails analyzing the evidence with regard to both technical engineering and process management to assess confidence. Additionally, it entails evaluating the representativeness and applicability of the evidence for coverage assessment.

The \textit{\textbf{dynamic approach}} within \cite{favaro2023building} emphasis the siginficance of continuous assessment and refinement. Thereby, the time frame across the itervative procedures is distinguished in three phases: the pre-deployment, the deployment, and the post-deployment phase \cite{favaro2023interpreting}. During pre-deployment, design and V\&V are prospective, with performance measures based on simulations and field operations. Accordingly, acceptance criteria are treated as predicted values until deployment. Successive post-deployment involves retrospective performance analysis in the open world \cite{scanlon2023benchmarks}. Continuous monitoring identifies gaps, challenges, threats, and hazards, addressing the dynamic challenges of the open world and facilitating ongoing refinement of the system and the process throughout the product lifecycle.

\begin{figure}[]
	\centering	
	\includegraphics[width=\linewidth]{img/Waymo_UL_X.png}
	\caption{Simplified representation of Waymo's safety determination lifecycle, inspired by \cite{favaro2023building}. It illustrates the distinguished consideration of prospective and retrospective perspectives on the methodology and safety argumentation.}
	\label{fig:Waymo}
\end{figure}

The \textbf{overall methodology} of the safety determination lifecycle is depicted in Figure \ref{fig:Waymo}, emphasizing the dynamic approach. Thereby, in accordance with the three phases of the dynamic approach, safety evolves as an emergent property, an acceptable prediction, and a constantly growing confidence. Morover, to achieve the desired safety and credibility, Waymo considers the process and the product to be aligned. Each iteration consists of process and product development/refinement and subsequent qualification, reflecting two consecutive V-models, one for the process and another for the product. The approach addresses the complexities of system development and real-world applicability through ongoing refinement, effectively integrating development and analysis. Over time, as the system scales, uncertainty diminishes, confidence increases, and real-world performance is evaluated against human benchmarks \cite{di2023comparative, kusano2023comparison}.

\subsection{Tesla: Data Engine}\label{Tesla}

The functional objectives of Tesla's Full Self-Driving and Waymo's self-driving service, Waymo One, are comparable, but they employ different technological and methodological approaches. Tesla relies on an end-to-end AI strategy based on camera data and benefits from a large fleet of operational vehicles, providing a vast amount of data for development and V\&V. Tesla's corresponding methodological approach, known as the data engine \cite{karpathy_cvpr21}, is illustrated in Figure \ref{fig:Tesla}. 

The data engine employs a data-centric, iterative development approach applicable to variety of AI systems and applications. Initially, an AI model is trained with a seed dataset. Initially, an AI is trained with a seed dataset. Subsequently, as in the case of Tesla, it is deployed in shadow mode in the customer's vehicles \cite{Tesla_shadow}, also known as silent testing \cite{templeton2019}. This involves employing specialized mechanisms to detect neural network inaccuracies, facilitating strategic data acquisition. Tesla, for instance, has crafted over 200 triggers to detect discrepancies while predicting surrounding object parameters like position, velocity, and acceleration \cite{karpathy_cvpr21}. While these triggers can be seen as AI performance indicators with respect to Waymo's terminology \cite{favaro2023building}, they serve a different purpose in this context.

The data engine's collection process drives follow-up auto-labeling and unit test updates, ensuring continuous dataset refinement. Subsequently, the neural network undergoes retraining, evaluation, and unit tests to avoid inaccuracies in subsequent versions, before redeployment in shadow mode \cite{Tesla_shadow, karpathy_cvpr21}. This systematic approach resembles a sophisticated automated data-based system, resembling a scenario-based database, effectively addressing real-world gaps and changes over time. However, for a seamless transition from silent testing to actual operation \cite{Tesla_shadow}, corresponding safety argumentation and assurance are essential. Nevertheless, Tesla does not provide any further information on this \cite{tesla_safety}.

The data engine not only enables systematic data acquisition and continuous improvement but also provides additional benefits for developing AI-based systems with large datasets. This includes increased efficiency through automation, creating a consistent database, and avoiding redundant data, leading to cost savings and improved effectiveness for diverse engineering teams.

Moreover, investigations reveal the widespread adoption of the data engine process, exemplified by its use by OpenAI \cite{DALLE2}. This indicates its versatility in addressing specifications for inadmissible outputs by selectively filtering undesirable data before training. Serving as an active learning approach, the data engine process collects data, updates AI systems, and mitigates the risk of catastrophic forgetting through offline updates, including V\&V and unit testing, before deployment. Therefore, the selection as a data-based process reference model seems appropriate.

\begin{figure}[]
	\centering	
	\includegraphics[width=\linewidth]{img/TESLA_UL_X_4.png}
	\caption{Tesla's data engine \cite{karpathy_cvpr21}, visualized in V-model structure, represents a fully data-driven methodology that is tailored to AI systems and strives for efficient and effective continuous improvement.}
	\label{fig:Tesla}
\end{figure}

\subsection{Comparison w.r.t. Complex Systems Incorporating AI}\label{sec:diff}

While previously the selected innovative development processes have been presented, these are now compared below in terms of general applicability to complex systems that incorporate AI. Emerging complex systems, like automated driving, feature increasing AI integration alongside traditional systems and heightened task complexity, leading to intricate architectures and necessitating complex orchestrations. The comparison of these processes in terms of their key advantages and disadvantages for general complex systems is presented in Table \ref{tab:compare_frameworks_ad_disad}. However, inherent limitations arise concerning their applicability to complex systems with AI integration due to application specific assumptions and conditions.

\begin{table*}[]
	\centering
	\caption{Comparison of development processes for complex systems incorporating AI.}
	\begin{tabularx}{\linewidth}{l *{2}{>{\raggedright\arraybackslash}X}}
		\toprule
		Framework	& \makecell{Advantages} & \makecell{Disadvantages}  \\
		\midrule
		\makecell[l]{Improved V-model \\ \tiny{(VVM Project)}} \vspace*{-0.4cm} &  \vspace*{-0.4cm} \begin{itemize}
			\item Systematic improvement of the classical V-model
			\item Consitent consideration of the operational design domain
			\item Maps infinite scenarios of the open world
			into a managable test space
		\end{itemize} \vspace*{-0.4cm} &  \vspace*{-0.4cm}
		\begin{itemize}
			\item Application-specific for automated driving 
			\item Requires a scenario database, not suitable for general databases
			\item Requires multiple perspectives, e.g. design, V\&V, risk management, argumentation
			\item Lacks alignment with the requirements of AI systems
		\end{itemize} \vspace*{-0.4cm}  \\
		\midrule
		\makecell[l]{Safety Determination Lifecycle \\ \tiny{(Waymo)}} \vspace*{-0.4cm} &  \vspace*{-0.4cm}
		\begin{itemize}
			\item Consideration of prospective and retrospective perspectives
			\item Alignment and refinement of process and product throughout the entire lifecycle
			\item Strong emphasis on V\&V and approval process, e.g. through case credibility assessment
		\end{itemize} \vspace*{-0.4cm} &  \vspace*{-0.4cm}
		\begin{itemize}
			\item Tailord towards the automated driving application
			\item Requires several approaches, e.g. layered, credible, dynamic
			\item Less formalized development process, due to strong emphasis on V\&V and approval process
		\end{itemize}  \\
		\midrule
		\makecell[l]{Data Engine \\ \tiny{(Tesla)}} \vspace*{-0.4cm} &  \vspace*{-0.4cm}
		\begin{itemize}
			\item Considers specifics of data-driven AI development and V\&V 
			\item Able to be automated, reduces human efforts and influences
			\item Provides application-agnosticity and iterative refinement
		\end{itemize} \vspace*{-0.4cm} &  \vspace*{-0.4cm}
		\begin{itemize}
			\item Not applicable to traditional or mixed systems
			\item Requires systems in operation for silent testing and data acquisition
			\item Less sophisticated process, e.g no simulation, no process refinement
		\end{itemize}  \vspace*{-0.4cm} \\
		\bottomrule
	\end{tabularx}%
	\label{tab:compare_frameworks_ad_disad}
\end{table*}%

The VVM project's improved V-model \cite{VVMOverall} and Waymo's safety determination lifecycle \cite{favaro2023building} embrace multiple perspectives, incorporate system structures, address different granularities, and onsider the safety argumentation, whereas the classic V-model \cite{brohl1993v} takes a singular approach focusing on design, V\&V, and customizable safety argumentation. This distinction allows the classical V-model to maintain broad applicability across various systems by accommodating flexibility in architecture, granularity, and assurance methods to meet specific requirements. These properties are essential for a framework aiming to address the needs of diverse complex systems. 

Moreover, Tesla's data engine \cite{karpathy_cvpr21} is primarily designed for AI systems, making it less adaptable to traditional or mixed systems. Moreover, it relies on a large-scale human-operated system fleet for data collection and silent testing, a requirement that is often impractical

In comparison to partially extended responsibilities and non-generic assumptions, frameworks often overlook formalizing the interdependence and transitions between simulation and the real world. However, it is evident that the joint interaction between simulation and the real world  is pivotal for complex systems \cite{KIDeltaSynData, KIAbsicherungSynData}. This interaction can result in significant gaps \cite{burton2023closing, burton2023addressing} during both the design and V\&V phases. Currently, there's a lack of a systematic formalization for the transition from the real world to simulation and back within a continuous refinement process. Such a formalization that address concerns related to data-driven development without excluding traditional and mixed systems, while ensuring flexibility in terms of safety assessments and argumentation, does not yet exist. 

\subsection{Discussion}
The analysis of selected processes reveals that alongside the classical V-model, designed for traditional systems, various process reference frameworks have emerged to address specific complex systems integrating both traditional and AI components, which can be described as emerging complex systems that incorporate AI. The improved V-model of the VVM project represents an application-specific extension of the classical V-model and addresses complex systems of automated driving. Waymo's safety determination lifecycle is similarly motivated but less dependent on the application and system design. In comparison, Tesla's data engine explicitly addresses the characteristics of AI systems. Finally, Figure \ref{fig:cmp} visually compares the above process reference model frameworks in the tension between application specificity and generalization, and suitability for traditional systems, AI systems, and emerging complex systems that incorporate AI. On an abstract level, Figure \ref{fig:cmp} shows that each approach has a certain area of application. Furthermore, it becomes apparent that there is a gap in the area of generic processes for complex systems that incorporate AI. This ultimately motivates our contribution in visual form. The correspondingly proposed iterative data-based V-model, which is intended to narrow this gap, is elaborated in the following section.  
 
\begin{figure}[h!]
	\centering	
	\includegraphics[scale=0.45]{img/VGL12.png}
	\caption{Visualization of the classical V-model and the analyzed innovative development processes with regard to methodological generality and system compatibility. The remaining gap is marked by the proposed iterative data-based V-model.}
	\label{fig:cmp}
\end{figure} 









