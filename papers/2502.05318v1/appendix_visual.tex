
\section{Additional details on Lemma~\ref{lem:visual} and visualizing diagonal invariance } \label{appendix:diag:inv:illustration}

\noindent
We first state the proof of \cref{lem:visual}, which is illustrative for understanding the necessity of using a modified group $\tilde \G = \{ A \in \R^{3 \times 3} \,|\, A(\argdot) + b \in \G \text{ for some } b \in \R^3 \}$. 

\begin{proof}[Proof of \cref{lem:visual}] By the definition of $\tilde f$,
\begin{align*}
    \tilde f( g(t) )
    \;=&\;
    f( \tilde \bx_{\rm symm} + A(t) + b )
    \;\overset{(a)}{=}\;
    f( A(\tilde \bx_{\rm symm}) + A(t) + b )
    \;\overset{(b)}{=}\;
    f( A(\tilde \bx_{\rm symm} + t ) + b) \;=\; f(g(\bx+t))\;.
\end{align*}
In $(a)$, we have used that $\tilde \bx_{\rm symm}$ is invariant under $A \in \tilde \G$. Using the definition of $\tilde f$ again to note that $\tilde f(t) =f(\bx + t)$ finishes the proof.
\end{proof}

As mentioned in \cref{sec:eval}, $\tilde \G$ and the point group $\G_*$ of $\G$ are in general two different groups. When $\G_*$ is the \texttt{P6mm} point group, both $\G_*$ and $\tilde \G$ consist of $12$ elements.  However, $\G_*$ is generated by \texttt{P3m1} and the reflection $R$ indicated in Fig.~\ref{fig:OG:PA:wf:scan:full:symm:config}, whereas $\tilde \G$ is generated by \texttt{P3m1} and the reflection $\tilde R$ indicated in Fig.~\ref{fig:OG:PA:wf:scan:full:symm:config}. Applying our method to $\tilde \G$ allows for visualizing the full \texttt{P6mm} symmetry of the wavefunction (Fig.~\ref{fig:OG:PA:wf:scan:full:symm}).

\begin{figure*}[h]
    \centering
    % \vspace{-.5em}
    \begin{tikzpicture}
        \node[inner sep=0pt] at (0,0) {\includegraphics[trim={4.5cm 5.5cm 5.2cm 6cm},clip,width=.5\linewidth]{figs/graphene_1_symmscan-reflection.pdf}};   

        \pgfmathsetmacro{\roottwo}{sqrt(3)}
        \pgfmathsetmacro{\x}{2.9}

        \pgfmathsetmacro{\sx}{-0.12}
        \pgfmathsetmacro{\sy}{2.5}

        \pgfmathsetmacro{\sxtwo}{1.51}
        \pgfmathsetmacro{\sytwo}{2.5}
        
        \draw[dashed,red,thick] (\sx, \sy) -- ({\sx - \x}, { \sy - \x * \roottwo} ); 
        \draw[dashed,blue, thick] (\sxtwo, \sytwo) -- ({\sxtwo - \x}, { \sytwo - \x * \roottwo} ); 

        \node[inner sep=0pt, red] at ({\sx -0.3}, \sy-0.05) { $\tilde R$};
        \node[inner sep=0pt, blue] at ({\sxtwo -0.3}, \sytwo -0.09) { $R$};

        \node[inner sep=0pt,rotate=90] at (-4.6,0.1){\footnotesize $y$ (Bohr)};
        \node[inner sep=0pt] at (0.1,-2.9){\footnotesize $x$ (Bohr)};

    \end{tikzpicture}
    \centering 
    % \vspace{-1em}
    \caption{A $\tilde \G$-symmetric configuration of $12$ electrons, where $\tilde \G$ is obtained from \texttt{P6mm}} 
    \label{fig:OG:PA:wf:scan:full:symm:config}
    \vspace{-1em}
\end{figure*}

\begin{figure*}[h]
    \centering
    % \vspace{-.5em}
    \begin{tikzpicture}
        \node[inner sep=0pt] at (0,-7.7) {\includegraphics[trim={3.8cm 1.1cm 3.5cm 1.2cm},clip,width=.8\linewidth]{figs/wavefunction-scan-OG-PA-reflection.png}};

        \node[inner sep=0pt,rotate=90] at (-7.15,-7.95){\scriptsize $y$-displacement};
        \node[inner sep=0pt] at (-3.28,-11.75){\scriptsize $x$-displacement}; 
        \node[inner sep=0pt] at (3.6,-11.75){\scriptsize $x$-displacement}; 

        \node[inner sep=0pt] at (-3.26,-12.3){\footnotesize \textbf{(a) $\log | \psi^{({\rm OG})}_{\theta} |^2$, original DeepSolid}};
        \node[inner sep=0pt] at (3.6,-12.3){\footnotesize \textbf{(b) $\log | \psi^{({\rm PA};\G)}_{\theta}|^2$, post hoc averaging}};
    \end{tikzpicture}
    \centering 
    % \vspace{-.5em}
    \caption{Visualization of the full \texttt{P6mm}-diagonal invariance of $\psi^{({\rm OG})}_{\theta} $ versus $\psi^{({\rm PA};\G)}_{\theta}$. Same setup and wavefunctions as Fig.~\ref{fig:OG:PA:wf:scan}, except that the configuration to be translated, $\bx_{\rm symm}$, is given by Fig.~\ref{fig:OG:PA:wf:scan:full:symm:config}. }
    \label{fig:OG:PA:wf:scan:full:symm}
\end{figure*}

