

%\boldification{README was easier than editing a code file}
%UC\#1 required participants to edit the README file as editing a text file would be easier to do than a code file.
%once it would be simple to understand and modify then if it was a programming file(e.g., XML or JAVA). 
%Additionally, the README is an essential file for a new contributor who wants to start contributing to an OSS project and to get an overview of the project. 
 
 
 %\boldification{In UC\#1, the plugin helped Tim users, since only users in the control group faced the difficulty of finding the readme files (Bug \#1)}
 
 %Participants with Abi's cognitive style in the Control group faced difficulties in figuring out how to make the pull request. 
 %In UC\#1, both Abis and Tims faced challenges in figuring out how to make the pull request. Among the challenges mentioned by Tim users in the control group, they experienced difficulty finding the editor and the README file. As mentioned by P44, ``\textit{Finding the README file, definitely, because I didn't know where to look for all these files, I didn't think it would be like in the middle of those files.}'' 
 
 %In the plugin group, this bug did not manifest, as the interface did this...
 
 %This difficulty is related to Bug \#1 (Table~\ref{tab:inclusivitybugs}) and it was not mentioned in the plugin group, which highlights that the proposed solution was beneficial for users with both personas. 
 
 %\boldification{However, the plugin was not the definitive solution for Abis -- they still had difficulties associated with risk-averseness}
 %On the other hand, in the plugin group, users with Abi's persona had difficulty figuring out how to edit the file and save it after editing. This comment highlights Abi's risk-averse facet when users have to use new technologies. P1 said, ``\textit{Starting the Edit process was really hard. And once you have a little computer knowledge and you actually get into the Edit tab, you can look at the various files you want to edit and then go through the process}''. 
%\italo{An individual's attitude toward risk is the willingness to take risks involving technology~\cite{burnett2016gendermag}. The concept of risk can be applied to a variety of contexts and may affect an individual's attempt at performing a task (they might not do it at all), willingness to use unfamiliar methods/techniques (even if using them might be beneficial), and attempting a task depending on the perceived time/effort needing to be invested~\cite{dohmen2011individual}.}

%%%%%

\red{

\boldification{We dicuss how the plugin interface helped participants overcome the challenges}

\boldification{The results of the questionnaires indicate that participants in the control group faced more challenges}

\textit{Impact on participant experiences.} Participants in the Control group reported facing more challenges than the Plugin group based on the questionnaire they filled out after every task (see Section)
%. We collected participants' experiences by asking them the following questions after they completed a use case: (i) what did you find most difficult in the process? and (ii) what in the interface helped you the most in the process?


%\boldification{README was easier than editing a code file}
%UC\#1 required participants to edit the README file as editing a text file would be easier to do than a code file.
%once it would be simple to understand and modify then if it was a programming file(e.g., XML or JAVA). 
%Additionally, the README is an essential file for a new contributor who wants to start contributing to an OSS project and to get an overview of the project. 
 
 
 \boldification{In UC\#1, the plugin helped Tim users, since only users in the control group faced the difficulty of finding the readme files (bug \#1)}
 
 %Participants with Abi's cognitive style in the Control group faced difficulties in figuring out how to make the pull request. 
 In UC\#1, both Abis and Tims faced challenges in figuring out how to make the pull request. Among the challenges mentioned by Tim users in the control group, they experienced difficulty finding the editor and the README file. As mentioned by P44, ``\textit{Finding the README file, definitely, because I didn't know where to look for all these files, I didn't think it would be like in the middle of those files.}'' 
 
 In the plugin group, this bug did not manifest, as the interface did this...
 
 This difficulty is related to bug \#1 (Table~\ref{tab:inclusivitybugs}) and it was not mentioned in the plugin group, which highlights that the proposed solution was beneficial for users with both personas. 
 
 \boldification{However, the plugin was not the definitive solution for Abis -- they still had difficulties associated with risk-averseness}
 On the other hand, in the plugin group, users with Abi's persona had difficulty figuring out how to edit the file and save it after editing. This comment highlights Abi's risk-averse facet when users have to use new technologies. P1 said, ``\textit{Starting the Edit process was really hard. And once you have a little computer knowledge and you actually get into the Edit tab, you can look at the various files you want to edit and then go through the process}''. 
%\italo{An individual's attitude toward risk is the willingness to take risks involving technology~\cite{burnett2016gendermag}. The concept of risk can be applied to a variety of contexts and may affect an individual's attempt at performing a task (they might not do it at all), willingness to use unfamiliar methods/techniques (even if using them might be beneficial), and attempting a task depending on the perceived time/effort needing to be invested~\cite{dohmen2011individual}.}

\boldification{In UC\#2, the plugin helped both Abis and Tims to find the changed files, but two plugin Tims still mentioned problems in this task}

In UC\#2 - One difficulty that arose in the control group was finding the changed files. It is related to bug \#6 (Table~\ref{tab:inclusivitybugs}). Abi and Tim users in the control group mentioned having difficulty finding the changed files in the pull request interface. Two Tim users and no Abi in the plugin group mentioned that they faced problems finding the changed files. We can see that the solution proposed in our work could reduce the impacts of this bug on both groups of participants. 

\boldification{In UC\#3, the plugin helped Tim participants to find how to request help.}


In UC\#3, Tim users in the control group mentioned that they had difficulty finding how to request help, and their first idea was to contact the experienced user directly. P43 mentioned that: ``\textit{I thought there would be a way that I could just like leave them a personal message and ask for help rather than posting. It looks like a public comment.}''. Other participants tried to contact the user directly by going to their GitHub profile page and looking for a direct message option, which GitHub does not offer. 

\boldification{Only one Abi reported difficulties with the plugin, but another found the task intuitive}

In the plugin group, only one Abi user reported that they faced challenges in finding an option to request help. Another Abi user mentioned that the task was intuitive (P40): ``\textit{Once I recognized that I needed to do this task as well, it was pretty intuitive.}''. 

%We can notice that the tinkering style of Tim participants help them when the information 

%It was easy for this user to complete the task because the user had previous experience with other chat platforms (e.g., Discord). It helped the user get situated and understand the GitHub interface. 

\boldification{In UC\#4, the plugin help all cognitive styles to require push access}

In UC\#4 - Users in the control group faced more difficulty requiring push access: one Abi user and ten Tim users mentioned having that difficulty. Out of these ten users, only three overcame this challenge and completed the task successfully. This problem is described in Table~\ref{tab:inclusivitybugs} as bug \#9. With the solutions proposed in the plugin group, we did not have any plugin user mentioning that difficulty, which highlights the contribution of the plugin to help users with different cognitive styles.

%interface helped

\textbf{Interface help.} We also investigated what in the interface helped participants. In \textit{UC\#1 - Submit pull request} - Users with Abi's cognitive style in the plugin group mentioned that button colors and the tooltips helped. Tooltips were included in the interface to help users who need to gather information before starting to use the technology (\textit{information processing} style). It also improved their \textit{self-efficacy} facet by letting users know they were on the right path. Indeed, P29 mentioned that ``\textit{the tooltip guides me into the execution of the task}''. %\italo{A person's efficacy expectation level is directly proportional to the likelihood that they would complete the task, the individuals with higher self-efficacy tend to have a stronger efficacy expectation, while low- self-efficacy individuals tend to have a weaker efficacy expectation~\cite{bandura1977self}.} %Tim users in the plugin group praised the progress bar, button colors (P4, P23, P25), and also the tooltips (P5, P6, P30, P31, P33, P34, P36, P39) included by the plugin were mentioned by Tim users in the plugin group. P36 said that ``\textit{when I started the task, I found a tooltip icon, which contained information on how to proceed with the task.}''. 
Abi users in the control group said that the interface did not help and was confusing: P40 mentioned: ``\textit{I need to stay on code to do the Edit. So, I scrolled down and found the proposed changes. And I thought, `you know that I don't want to lose my progress on the code.'}''. Tim users in the control group mentioned the button colors, the interface labels, and the icons helped in general. 

%Additionally, we have users that mentioned feeling overwhelmed by the information presented (P45). As mentioned by P45, ``\textit{No, I it's pretty complicated. I don't know. There's a lot of stuff on here. And like, like I clicked on this Actions button because I thought that migrated made me edit it but I don't know. It's, it's a little confusing is what I would say}''. One find here is comparing the users in the control group (e.g., Abi vs. Tim). GitHub's original interface tends to be easier and simpler for users with Tim's persona and not that helpful for Abi users, creating a barrier for those users to interact and engage with OSS projects.

In \textit{UC\#2 - View changed file}, both Tim and Abi users in the plugin group mentioned that the changed files in the navigation menu and the tooltips helped them to complete the task. %Tim users in the plugin group mentioned that the interface helped with clear labeling, the tooltip, the icons, and the navigation menu. 
As in UC\#1, Tim users in the control group showed more satisfaction with regular GitHub interface elements.



%The Tim users in the control group mentioned that the color-coding (P41), the interface icons (P43, P65), the navigation menu (P44, P69, P75), and the files changed option (P45, P63, P64, P65) helped to perform the task.

In \textit{UC\#3 - Request help to solve the PR}, plugin users noted that the mention icon (@) helped. This fix supported users who \textit{learn by process} and have lower \textit{computer self-efficacy}. The fix was designed to make the action of mentioning someone self-explanatory to start interacting in the pull request interface. 

%For users in the control group, they mentioned that the comment box (i.e., Abi - (P40, P71) and Tim (P41, P43, P44, P65, P69, P75)) itself helped to give an idea of conversation. However, during the experiment sessions, we could see that it took some time for those participants to understand why there was a comment box in the interface and figure out how they could use it to complete the task.

In \textit{UC\#4 - Upload file}, Abi users in the plugin group mentioned that the green fork button and the message helped. As mentioned by P26, ``\textit{interface messages when trying to upload the file helps a lot}''. Tim users in the plugin group said the same, exemplified by P5: ``\textit{So when I went back, I saw that the fork was highlighted in like the same green color. (...) It really just puts me back in the right direction}''.

} 