\section{Background and related work}
\label{sec:backrelated}

This section outlines the background and related work involving gamification and its use in SE education.

%XXXX
%\textbf{Self-determination Theory (SDT).}  %The gamification concept relies on Self-Determination Theory (SDT)~\cite{ryan2000intrinsic}, which posits that humans have three basic needs: competence, relatedness, and autonomy. The competence need refers to mastery, control, and effectiveness in tasks. It is addressed by offering points, awards, badges, and leaderboards for player comparison, fulfilling this need~\cite{hsu2004people}. Meanwhile, autonomy refers to the sense of control over one's own goals, addressed by features like avatars, branching storylines, and gameplay options~\cite{wee2019gamification}, and the social relatedness involves the need for connections with others, fulfilled through player-focused storylines, cooperative, and competitive gameplay~\cite{sailer2017gamification}. Individuals may feel varying motivation levels depending on how these needs are met and the type of activity. Gamified activities can become more appealing by incorporating game elements that fulfill some of these needs~\cite{dichev2017gamifying}.
%XXXX

\textbf{Gamification.} Dichev and Dicheva~\cite{dichev2017gamifying} describe gamification as a strategy to enhance student motivation in educational settings by integrating game elements. Gamification is increasingly being utilized in education to enhance the learning process~\cite{deterding2011gamification, sailer2020gamification}. Its key benefits include simplifying complex topics~\cite{marin2018empirical, ayub2019gamification}. For example, Deterding et al.~\cite{deterding2011gamification} propose that gamification simplifies complex material, making learning more accessible and promoting deeper understanding. They argue that games can enhance educational experiences by making them more interactive and effective for students. %There are many examples of frameworks for developing and assessing gamified systems. For example, 
Toda et al.~\cite{toda2019analysing} proposed a framework for gamification strategies in educational settings based on a literature review and an evaluation with gamification and education experts. The resulting taxonomy included the description of 21 game elements, which we used as a baseline for the design of OSSDoorway game elements.

%Toda et al.~\cite{toda2019analysing} proposed a framework focused on gamification strategies in educational contexts. %Toda et al.'s taxonomy is organized into five dimensions, as outlined below~\cite{toda2019analysing}: (i) \textbf{performance} -- This dimension involves feedback mechanisms from the environment that provide learners with responses, such as points, progress bars, levels, stats, and badges as forms of acknowledgment; (ii) \textbf{ecological} -- This dimension includes elements like choice, economy, and time pressure, which are environmental properties that encourage learners to adopt desired behaviors; (iii) \textbf{social} -- The game elements in this dimension foster interactions among learners, including cooperation, reputation, social pressure, and competition, represented by features like leaderboards, private rankings, and competitive tasks; (iv) \textbf{personal} -- This dimension emphasizes the individual learner's interaction with the environment, incorporating goals (e.g., quests), puzzles, novelty, and ongoing updates; and (v) \textbf{fictional} -- Elements in this dimension connect the learner and the environment through user-centric narratives and storytelling, blending the learner's experience with the context.

%XXXX
\textbf{Gamification for SE education.} Gamification has been effectively applied to SE education, including agile process~\cite{prause2012field}, software testing~\cite{rojas2016teaching, garaccione2022gerry, ozturk2022gamification}, design pattern~\cite{bartel2016gamifying}, software project management~\cite{annunziata2024serge, navarro2007comprehensive}, and business processes~\cite{garaccione2024gamification}. These applications highlight the broad potential of gamification to enhance educational experiences in the field. Su~\cite{su2016effects} developed a gamified framework to assess the impact of gamification on teaching SE, finding that gamified methods increased student motivation and improved academic performance. Similarly, Sheth et al.~\cite{sheth2012increasing} showed that incorporating gamification into a SE course enhanced student engagement in areas such as documentation, bug reporting, and testing. In the context of OSS, gamification strategies have been used to encourage contributions, such as through the OpenRank network algorithm and a monthly contribution leaderboard~\cite{zhao152023motivating}. Santos et al.~\cite{santoslr} identified software solutions that facilitate the onboarding of newcomers in software projects. Among these solutions, some used gamification techniques with newcomers, fostering engagement and boosting motivation~\cite{PS04diniz2017using, PS17toscani2018gamification}. Diniz et al.~\cite{PS04diniz2017using} implemented four game elements, i.e., quests, points, ranking, and levels in GitLab, and assessed their ability to motivate students to overcome orientation barriers. Although their approach shares similarities with this study, our work differs by conducting a systematic user-centered design of a more comprehensive gamified environment to support students' contributions in GitHub, employing a bot that interacts with students and updates the environment.  

%evaluates a gamified approach called OSSDoorway, which incorporates eight game elements, and examines how OSSDoorway supports students (N=29) in the OSS contribution process while providing an empirical assessment of students' self-efficacy. \italo{Considering the growing focus on OSS within academic settings and its significance to society, this work aims to introduce OSSDoorway tool designed to help students contribute to OSS projects.}