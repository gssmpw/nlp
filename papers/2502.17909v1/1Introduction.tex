\firstsection{Introduction}

\maketitle

Visual data storytelling is a powerful method for conveying a series of interconnected data facts and visualizations through a narrative flow. Among data-driven storytelling techniques, fact sheets are particularly effective. They integrate various data visualizations extracted from a dataset and weave them into a coherent narrative that effectively conveys key findings~\cite{shi2020calliope}. Fact sheets present information in a concise, single-page format, enhancing viewer retention and memorization~\cite{wang2019datashot}. With the digital transformation leading to an overwhelming influx of data, fact sheets have become increasingly vital for summarizing and presenting information.
 

Creating a fact sheet involves intricate tasks, from identifying compelling data points to designing visualizations and crafting a persuasive narrative~\cite{ryan2018visual}. Despite advancements in commercial tools like Excel and Power BI, which suggest appropriate visual encodings for datasets~\cite{becker2019microsoft}, substantial manual effort is still required. Tasks like identifying crucial data points and designing fact sheets demand considerable expertise in data analysis and graphic design. To address these challenges, research has introduced tools that streamline various aspects of fact sheet creation~\cite{wu2021ai4vis}. These tools automate the generation of visualizations from tabular datasets, simplifying the creation of compelling visuals~\cite{dibia2018data2vis}. Additionally, they facilitate fact sheet layout design, ensuring clear and effective information presentation~\cite{cui2019datasite, wang2019datashot}. By assisting in narrative composition, these tools reduce the learning curve for users without extensive data analysis expertise~\cite{shi2020calliope}. DataShot~\cite{wang2019datashot}, introduced in 2020, was the first tool to automatically generate fact sheets from tabular data, applying heuristics to rank and generate relevant facts. Calliope~\cite{shi2020calliope} further enhanced this by improving the relationships between facts for better content cohesion and logical flow.

However, existing approaches to automatic fact sheet generation struggle to fully capture the semantics of dataset and present them effectively. At the fact component level, these methods often fail to grasp both the meaning of individual data columns and the relationships between them, leading to missed insights and the generation of irrelevant conclusions. For instance, when analyzing the Tourism dataset (as shown in \cref{fig:teaser}), which includes columns for country names, GDP, and tourism revenue, these methods might inappropriately aggregate GDP values across countries, rather than correctly attributing the contribution of tourism to each specific country's economy. At the fact sheet level, these approaches struggle to establish relevance between facts and form a coherent narrative, resulting in a lack of interconnection and a disjointed flow of information. Additionally, the reliance on slot-filling, template-based methods for text generation leads to content that can be difficult for general readers to comprehend especially without providing related context of the dataset, such as unnatural statements like \emph{``receipts\_in\_billions of France is 35.96''}. These issues negatively impact the effectiveness of fact sheets, highlighting the need for more advanced techniques that make fact sheets accessible to a broader audience.

The rapid progress in generative AI has driven the development of Large Language Models (LLMs), such as GPT~\cite{openai_2023} and Llama~\cite{touvron2023llama}. These models, with in-context learning~\cite{min2022rethinking}, have demonstrated remarkable proficiency in performing logical reasoning tasks~\cite{chang2023survey}. Their inherent efficiency in handling diverse tasks without extensive retraining represents a promising approach to providing AI solutions across various domains. In data visualization, existing research has explored the utility of LLMs at the level of individual visualizations~\cite{wu2021ai4vis,chat2vis}. While these studies have shown improvements in segregated data analysis tasks, generating a fact sheet is a complex task involving multiple sub-steps, each requiring distinct expertise—from planning facts, extracting data, to drawing visualizations and composing the fact sheet layout. Recently, to enhance the capability of LLMs for complex tasks, researchers have explored AI chain design~\cite{handler2023taxonomy}, where it involves multiple AI-based worker with specialized expertise to excel in its modular task. This design has improved output quality across domains, including online discussions~\cite{dong2024multi} and social simulations~\cite{pang2024self}. Building on previous research, our primary objective is to investigate how AI-chain systems can improve the quality of generated fact sheets. Additionally, our research examines how these AI advancements can be integrated into data tasks, aiding not only data scientists but also novice users.


We introduce \tool, an automated fact sheet generation and customization tool designed to streamline the creation of fact sheets from raw datasets. \tool employs a novel approach featuring the collaboration of five specialized AI workers, each with distinct roles in processing and transforming the data. Upon receiving an input dataset and user requirements, \tool generates a dataset representation and passes it to the AI workers, where each worker executes its designated task, sequentially passing the output to the next worker in the pipeline. The result is a comprehensive fact sheet comprising interconnected facts. To enhance human-AI interaction, \tool includes a customization platform that allows users to review and modify the generated fact sheet using natural language commands, ensuring the final output aligns with specific user preferences.


To demonstrate the usability of the fact sheets generated by \tool, we present two fact sheets based on real-world datasets. Furthermore, we assessed the quality of our automatically generated fact sheets by conducting a user study with 18 participants. The results showed a preference for \tool's fact sheets over those from two state-of-the-art baselines~\cite{shi2020calliope, openai_2023}, citing more comprehensive content and superior visual appeal. Additionally, we validated the customisation interface by asking users to experiment \tool to customise a fact sheet and rate the quality of their customised fact sheets alongside their user experience. Participants confirmed that the customised fact sheets aligned with the dataset and their specific requests and praised the visual design. They also reported a positive user experience through the System Usability Scale questionnaire.


To summarise, the contributions of this paper include:
\begin{itemize}
\item We proposed a taxonomy for configuring AI chain workers and introduced a tool, \tool, that leverages this taxonomy for fact sheet generation. This demonstrates the applicability of AI chains in managing complex, multi-step workflows within the domain of data storytelling.
% We proposed a taxonomy for configuring AI chains and \jieshan{proposed \tool which adopts this taxonomy for} fact sheet generation, demonstrating the applicability of AI chains for complex multi-step workflows in data storytelling topic.
We hosted \tool as a public web-based automatic fact sheet generation and customisation platform for data community\footnote{\url{https://statement-driven-vis-15ebbabf813c.herokuapp.com/}}. 
\item We investigated the dataset representation aims for scalability and secure dataset communication to LLMs. Our data anomysation techniques propose the first step towards responsible AI usages for data works.
\item We used real-world datasets to illustrate the capability of our proposed system and validated \tool through a user study with 18 participants, which confirms its effectiveness and applicability across different expertise levels. The feedback provides future directions for human-AI collaboration on data tasks.
\end{itemize}
