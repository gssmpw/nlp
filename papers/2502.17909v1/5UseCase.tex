\subsection{Use Cases}

To demonstrate the capabilities of our automatic generation tool, we present two illustrative fact sheets in \cref{fig:showcase}. These fact sheets distill complex datasets into clear, insightful narratives using visualizations and storytelling.

\Cref{fig:showcase1} presents a dataset about Startup Failures, featuring 1,234 rows and 6 columns that record the details of companies that closed during or after the ``new economics'' boom in China from 2010 to 2019. This dataset covers various aspects of startup profiles, including the broken year (e.g., 1/1/2019), industry (e.g., Finance), location (e.g., Shanghai), funding status (e.g., Series B), survival time in days (e.g., 1,730), and cause of failure (e.g., supervision). The initial request was \emph{``Analyze the reasons for startup failures across different locations and industries.''} \tool structured the analysis into sections: industry distribution, startup trends and funding, and geographical distribution. The Industry Distribution section reveals the dominance of the E-Commerce sector, while noting that the Education industry has the highest average survival time. The Startup Trends and Funding section highlights a peak in startups in 2017 and notes that most lack funding, underscoring investment challenges in various industries.  The Geographical Distribution section identifies Beijing as a major innovation hub, leading with 519 startups, reflecting its significance in China's startup ecosystem. 

\Cref{fig:showcase2} features a dataset on university majors and employment opportunities, encompassing 173 rows and 12 columns. This dataset includes statistics on various university majors, such as the major's name (e.g., Accounting), its broader category (e.g., Business), total student enrollment (e.g., 1,778,219), employability statistics (e.g., Employed: 1,335,825; Unemployed: 75,379; Unemployment Rate: 0.07), and the median pay (e.g., \$65,000) associated with each major. The initial request was \emph{``Facts about university majors and employment status.''} \tool structured the analysis into four sections: Education Majors Overview, Top Majors, Proportion of Students in Each Major Category, and Top 5 Majors with the Highest Unemployment Rates. The Education Majors Overview notes that the Engineering category has the highest number of distinct fields, with 29 different majors. It then identifies the most popular majors, with Business Management and Administration leading in student enrollment among the top five. In terms of earning potential, Petroleum Engineering majors stand out with the highest median salary of \$125,000. The analysis continues with the distribution of students across various major categories, highlighting that Business majors constitute the largest proportion at 24.75\%. Lastly, the section on unemployment rates reveals that Miscellaneous Fine Arts has the highest unemployment rate at 15.6\% and displays the correlation between the median salary and unemployment rate. 

