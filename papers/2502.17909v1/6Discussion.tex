\section{Discussion}

% // more in-detail - make it one of our contribution

% // we apply this AI concept -> data vis -> do it on real data -> apply this on other fields, future direction.

% // how ai will challenge the role data analytics.
% // link with those survey, we validate those ideas. do it need further study.

\textbf{Modern Data Analysis Tools and Privacy Considerations in AI- Enhanced Workflows.} 
The rapid evolution of AI has significantly reshaped the data analysis and presentation process. From traditional, fully manual tools that required extensive effort, research has increasingly focused on making the data analysis process more intuitive through automation, reducing user effort while maintaining output quality~\cite{grudin2009ai}. While AI automation offers great benefits by reducing repetitive tasks, fully automated systems that exclude human involvement are often not intuitive. Previous research~\cite{li2023ai} shows that data workers are hesitant to adopt AI tools due to its inherent weaknesses, such as limited creative capabilities and a lack of understanding of data context. The key challenge is finding a balance between human expertise and AI-driven automation, with some tools requiring substantial manual operations and others overly automating the process without adequate user oversight~\cite{wu2021ai4vis}. From our evaluation, the user feedback highlights that while AI effectively generates preliminary insights and simplifies data processing, human refinement is still essential to tailor results to specific needs. As AI tools advance, their role as interactive assistants—providing suggestions and answering queries—represents a promising direction. However, the integration of AI, particularly LLMs, into data workflows has raised privacy concerns. Skepticism about sharing private dataset with AI systems underscores the importance of securing private information to maintain trust in AI technologies~\cite{Gal,9152761,brown2022does,Gurman_2023}. In this direction, our dataset representation techniques offer a first step toward data anonymization, enabling AI to process datasets without exposing the actual data. Future efforts to make AI more intuitive and reliable will further enhance the widespread adoption of modern AI data tools.


\textbf{Limitations \& Future Research.}
First, our future work focuses on enhancing \tool to better serve both general and specialized audiences. While basic chart types effectively communicate data to a general audience, more complex scientific use cases often require advanced and domain-specific charts that can encode multiple data dimensions. To address this, we plan to enable more personalized usage by allowing users to select advanced chart types tailored to complex, domain-specific analyses \cite{dibia2018data2vis}. Second, while \tool efficiently processes datasets with numerous rows, it encounters limitations when the number of columns increases, as the dataset representation must incorporate information about each individual column. To overcome this, we propose splitting datasets into smaller, more manageable tables automatically while preserving connections between them.
Lastly, we recognize that the general knowledge from LLMs may not be sufficiently in-depth to generate fact sheet for specialized scientific dataset. To ensure more accurate and insightful outputs, we recommend integrating Retrieval-Augmented Generation (RAG) techniques. This approach will enable \tool to leverage domain-specific knowledge bases effectively \cite{dibia2018data2vis}.
% presents significant potential for improvement. 
% This specialised knowledge is particularly crucial in enhancing the quality of generated charts within scientific domains. In these contexts, a general approach may struggle to fully comprehend user requests and interpret scientific datasets. By integrating domain-specific expertise, LLMs can handle the intricacies and unique requirements of scientific data, leading to more accurate and relevant chart generation.


