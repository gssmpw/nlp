
        \begin{table*}[ht]
        \centering
        \adjustbox{max width=\linewidth}{
        \begin{tabular}{lcccc}
        \hline
        & \textbf{FT Only} & \multicolumn{1}{c}{\textbf{vanilla realignment}} & \multicolumn{1}{c}{\textbf{\textsc{AlignFreeze} with front-freezing}} & \multicolumn{1}{c}{\textbf{\textsc{AlignFreeze} with back-freezing}}\\ 
        \cline{2-5}
        & -  & BD & BD & BD \\
        \hline
    Arabic & 64.6$_{\pm 0.5}$ & 65.0$_{\pm 0.6}$ & \textbf{65.6}$_{\pm 0.2}$ & 65.0$_{\pm 0.8}$ \\
Bulgarian & 68.0$_{\pm 0.8}$ & 69.1$_{\pm 0.6}$ & \textbf{69.3}$_{\pm 0.2}$ & 69.1$_{\pm 0.7}$ \\
Chinese & 68.9$_{\pm 0.6}$ & 69.5$_{\pm 0.7}$ & 69.2$_{\pm 0.4}$ & \textbf{69.9}$_{\pm 0.6}$ \\
French & 72.8$_{\pm 0.6}$ & 73.6$_{\pm 0.3}$ & \textbf{74.2}$_{\pm 0.3}$ & 73.7$_{\pm 0.5}$ \\
German & 70.1$_{\pm 0.5}$ & 70.3$_{\pm 0.6}$ & \textbf{71.0}$_{\pm 0.3}$ & 70.9$_{\pm 0.6}$ \\
Greek & 66.6$_{\pm 0.7}$ & 67.5$_{\pm 0.6}$ & \textbf{67.6}$_{\pm 0.6}$ & 67.4$_{\pm 0.8}$ \\
Hindi & 59.7$_{\pm 1.1}$ & 60.9$_{\pm 1.0}$ & \textbf{61.0}$_{\pm 0.5}$ & 61.0$_{\pm 0.3}$ \\
Spanish & 73.4$_{\pm 0.4}$ & 73.9$_{\pm 0.3}$ & \textbf{74.8}$_{\pm 0.3}$ & 74.2$_{\pm 0.3}$ \\
Thai & 53.3$_{\pm 2.3}$ & \textbf{57.4}$_{\pm 0.8}$ & 56.8$_{\pm 0.3}$ & 56.1$_{\pm 0.8}$ \\
Turkish & 61.4$_{\pm 0.5}$ & 63.5$_{\pm 0.6}$ & 63.2$_{\pm 0.4}$ & \textbf{63.8}$_{\pm 0.3}$ \\
Vietnamese & 69.0$_{\pm 0.5}$ & 70.3$_{\pm 0.2}$ & \textbf{70.9}$_{\pm 0.3}$ & 70.8$_{\pm 0.1}$ \\
\hline
Average & 66.3$_{\pm 0.6}$ & 67.4$_{\pm 0.4}$ & \textbf{67.7}$_{\pm 0.2}$ & 67.5$_{\pm 0.3}$ \\
        \hline
        \end{tabular}
        }
        \caption{XNLI average accuracy results across 5 seeds using mBERT by freezing strategy, language, and aligner. Aligner names: BD - Bilingual Dictionary. The highest average accuracy value for each language is highlighted in bold.}
        \label{table:results_mBERT_xnli_before}
        \end{table*}