\begin{abstract}
Code cloning, a widespread practice in software development, involves replicating code fragments to save time but often at the expense of software maintainability and quality. In this paper, we address the specific challenge of detecting ``essence clones'', a complex subtype of Type-3 clones characterized by sharing critical logic despite different peripheral codes. Traditional techniques often fail to detect essence clones due to their syntactic focus. To overcome this limitation, we introduce \toolname, a novel detection tool that leverages information theory to assess the semantic importance of code lines. By assigning weights to each line based on its information content, \toolname emphasizes core logic over peripheral code differences. Our comprehensive evaluation across various real-world projects shows that \toolname significantly outperforms existing tools in detecting essence clones, achieving an average F1-score of 85\%. It demonstrates robust performance across all clone types and offers exceptional scalability. This study advances clone detection by providing a practical tool for developers to enhance code quality and reduce maintenance burdens, emphasizing the semantic aspects of code through an innovative information-theoretic approach.
\end{abstract}

% \begin{CCSXML}
% <ccs2012>
%    <concept>
%        <concept_id>10011007.10011006.10011073</concept_id>
%        <concept_desc>Software and its engineering~Software maintenance tools</concept_desc>
%        <concept_significance>500</concept_significance>
%        </concept>
%  </ccs2012>
% \end{CCSXML}

% \ccsdesc[500]{Software and its engineering~Software maintenance tools}

% \keywords{Clone Detection, Essence Clone, Information Theory}