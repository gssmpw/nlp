\section{Design Goals}
\revisedtext{Based on the findings from the semi-structured interviews and content analysis, we identified three design goals (\autoref{figure1}~(C)) for developing a GenAI-aided system that supports users in designing a paper-cutting where explicit elements align with implicit meanings:}
\begin{itemize}
    \item[\textbf{DG 1.}] \textbf{Factor-oriented Guidance for Ideation.} This goal aims to help users efficiently transform abstract intent into structured descriptions (\textbf{C1}). The system guides ideation by focusing on four key factors, providing detailed content based on the selected factors to suggest related content for creatively shaping the final idea (\textbf{C2}).

    \item[\textbf{DG 2.}] \textbf{Reference-based Exploration for Composition} For each compositional action, the system based on domain knowledge suggests related content as the reference. Users can explore these design references for creative self-selection and arrangement of contours and patterns extracted from reference (\textbf{C3, C4}).

    \item[\textbf{DG 3.}] \textbf{Controllable and Editable Design with Recommendation} The recommendations provided by GenAI are integral to DG 1 and 2. These suggestions can ensure diversity to foster creativity while maintaining rationality and relevance, thereby preventing users from being misled or overwhelmed during exploration (\textbf{C4}). Furthermore, users are able to edit selections of unreasonable or partially reasonable ideas and compositions, allowing them to incorporate desired elements into their designs and complete the entire design process (\textbf{C5}).
    
    % \item[\textbf{DG 1.}] \textbf{Guide Ideation and Composition with Reference.} To help users efficiently translate abstract intent into a structured description, the system guides users' ideation from 4 factors to form it. To the composition of each action composition, the reference can guide them to self-selection and arrangement of contours and patterns. 
    
    % \item[\textbf{DG 2.}] \textbf{Suggest Diverse and Rational Content for Exploration.} All design processes need the suggested content to explore creative and diverse design for ideation and composition, including suggested detailed and related content from 4 ideation factors to form a final idea and suggest related and rational paper-cutting and patterns, which can be referred to as their expression and structure in DG 1.
    
    % \item[\textbf{DG 3.}] \textbf{Ensure a Controllable and Editable Design Process.}  
    % From the selection of factors and recommended content in ideation, to exploration related paper-cuttings and patterns are need
\end{itemize}
