\section{Discussion}

We introduce the design space and workflow of paper-cutting and propose a novel GenAI-aided creativity support tool, HarmonyCut, which supports reference-guided exploration from ideation to composition of paper-cutting design. Based on our findings, we suggest some design implications for future creativity support tools.

% \subsection{}

\subsection{Guidance and Reference in Different Context}\label{discussion:guidance reference}
As reflected in the findings and expert feedback from the evaluation (\autoref{evaluation}), users' preferences and assessments of the system's features were influenced by their varying levels of expertise across different fields. Experts tended to favor more exploration, while novices, due to their lack of domain knowledge, preferred the system to make recommendations for them. Therefore, it is crucial to consider the differing workflows of users with varying expertise to support design. This ensures that users in different contexts can effectively use the system to support their design tasks. This approach is also supported by findings from previous research~\cite{Xiao:2024:typedance, Zhou:2023:filtererink, Yan:2022:flatmagic}.

In the ideation process, experts typically have a clear design intent and seek comprehensive references to inspire creativity and refine their ideas. On the other hand, novices often struggle to effectively translate tasks into concrete ideas, making them more reliant on the system to guide them through the ideation process.
Additionally, we observed that novices tend to imitate references rather than use them for creative ideation. Therefore, the system needs to provide explanations alongside the guidance to help them understand the underlying knowledge and design with that understanding. 
\revisedtext{For experts, the focus is on providing references that are comprehensive to better support their advanced creative needs and ensure expressiveness. Due to the differing priorities of novice and expert users, experts may perceive GenAI as inadvertently constraining the creative design process, as also discussed in~\autoref{limitations}. This constraint arises from the nature of GenAI-aided guidance and references, which are inherently linked to the dataset~\cite{Cui:2024:chatlaw, Wang:2023:methodsknowledge} and may restrict creative outcomes to the dataset domain. U2 emphasized that paper-cutting, as a highly abstract form of expression, depends on distinctive personal styles to achieve creative results. However, outputs from GenAI with stereotypes may lead to stylistic homogenization. This not only limits the diversity of creative expression but also raises concerns about copyright~\cite{Samuelson:2023:copyright, Bianchi:2023:stereotype, Zhou:2024:biasgenerativeai}.}

Without proper guidance, novices can easily get lost in many suggestions, while experts can draw inspiration from more abstract references. For novices, more specific guidance helps them extract useful suggestions and apply them to their designs.
% Besides, the 除了系统工作流程中的引导和参考, 交互方式也存在 trade-off, 一方面, GenAI作为端到端的模型始终因为无法对结果进行再编辑,只能通过迭代的更换输入来实现, 而缺乏可控性. 但在实验中我们也发现, 虽然edit和可控性显著提高. 但相比于模型只需要输入文字的交互, 系统为用户带来的requirement of physical and effort 也会提高. 因此, 在通过GenAI协助,用户交互式的完成设计需要平衡上的研究, 通过改进交互方式减少新的负担, 同时保证依然有可控性
In addition to the guidance and references provided in the system workflow, there are trade-offs in the controllabliltiy. On one hand, as an end-to-end model, GenAI results cannot be directly edited, requiring users to iteratively adjust their input, which limits controllability. On the other hand, our experiments revealed that while the system significantly improves editability and control, it also increases the physical and cognitive demands on users compared to simple text-based interaction. Therefore, future research should focus on finding a balance in GenAI-aided design and improving interaction methods to reduce these additional burdens while ensuring adequate control remains in the creative process.
% \revisedtext{另外, from critical view, guidance and reference involved GenAI被批判可能将会创意限定在domain of data,even one expert 担忧,由于抽象剪纸很需要个人风格的夸张,GenAI基于数据集生成的相关刻板印象生成的风格明显的内容是否会有侵权问题}


% \subsection{}
% 正如evaluation~\autoref{evaluation}中findings所表现的, 用户因其在不同领域具有不同水平的专业知识, 而影响他对系统不同功能的喜好, 以及和评判水平. 专家倾向于更多的探索, 而新手因缺乏专业知识,更倾向于让系统为其推荐. 因为在ideation过程中,专家的design intent一般比较明确, 他需要的是更全面的reference来扩充构思的创意感,然后实现自己的构思. 而对于新手,它甚至无法将任务有效转化成构思,因此跟倾向于系统能指导它进行构思. 另外, 根据我们的观察, 新手根据reference更多是模仿, 所以系统更多的是需要通过在引导中提供相应的解释,来帮助他产生对知识的理解, 在有知识的基础上进行设计. 而对于专家, 更需要的是能够让系统提供非常符合其描述的Reference,来帮助创意. 如果缺乏引导, 新手会在大规模suggestion中迷失, 如果reference 更抽象, 专家从具体reference中获取inspiration. Guidance更具体, 新手从其中提取建议,来实施设计



\subsection{Integrate the Cultural Knowledge with GenAI Support Creative Design}
% Based on the studies and findins in this work findings, and prior work, that demonstrate GenAI 在理解带有文化因素内容上的能力受限. 如果想保证系统不会因此而影响创意支持, 就需要将文化相关内容整合到模型中, 这有几种方式, 一种通过大量的数据增强, 但它受限与数据的规模和数据收集的难度. 就像E1所说, HramonyCut通过factor所涵盖的知识个人认为已经可以将剪纸大部分知识用这个taxonomy包含, 但文化就是博大精深, 若想通过增加数据规模来实现 GenAI 支持相关设计 是困难的. 因此可以需要考虑如何将内容结构化, 就像HarmonyCut中 将 paper-cutting 限制在factor的范畴当中, magical brush 将 chinese pianting 用symbol来规范,, 也即提供复杂知识的前置, 让模型理解构成知识的基础, 从而理解知识..
% 虽然生成式模型具有很强的语言和视觉理解能力, 在多种创作任务中允许用户直接通过输入自然语言来让模型协助用户创作而无需复杂参数, 但对于特定任务, 模型难以理解用户需求,从而提供符合预期的输出, 尤其是在比较抽象的文化艺术设计方面. 同样的, Based on the studies and findings from this work, coupled with insights from prior research~\cite{Messer:2024:cocreating, Garcia:2024:paradox, Chung:2023:artinter}, it is clear that GenAI experiences limitations in understanding content with cultural elements. To prevent these limitations from hindering creative support, it is crucial to integrate cultural knowledge into the model. One approach is large-scale data augmentation; however, this method is constrained by the challenges related to the scale and complexity of data collection.
% E1 emphasized that the taxonomy of factors, which organizes knowledge through design factors, effectively encompasses much of the paper-cutting knowledge. However, due to the vast and intricate nature of cultural knowledge, relying solely on expanding the dataset is impractical for enhancing GenAI's ability to support culturally relevant design. A more effective strategy may involve structuring the cultural content,
% 先前的工作在绘画方面尝试提取模糊概念,特殊处理, 但这些内容与具体的视觉形式间,因为模型的欠缺抽象知识而存在gap. 如何在文化艺术方面的任务中,让模型给出合理回答, 可以通过将抽象的内容, 通过用设计任务和设计空间来归纳起来,:
% similar to how HarmonyCut organizes paper-cutting within a framework of factors, or how symbols are used in Chinese painting~\cite{Xu:2023:magicalbrush} and template-based approaches are applied in dynamic shadow puppetry creation~\cite{Yao:2024:shadowmaker}. This structured approach provides a foundational framework of complex knowledge, enabling the model to better understand and engage with cultural content, thereby supporting design. 
GenAIs have demonstrated strong capabilities in both language and visual understanding, allowing users to engage in the creative process through natural language input. However, in tasks involving abstract cultural and artistic design, these models often fail to fully capture user intent, resulting in outputs that may not meet expectations. The findings from this study, alongside insights from previous research~\cite{Messer:2024:cocreating, Garcia:2024:paradox, Chung:2023:artinter}, highlight the limitations of GenAI in comprehending and interpreting cultural elements. To prevent these shortcomings from impeding creative support, it is essential to incorporate cultural knowledge into the models. Although large-scale data augmentation offers one possible solution, it is limited by both the scale and complexity of data collection.
E1 emphasized that the taxonomy used to organize knowledge through design factors effectively captures a significant portion of paper-cutting knowledge. 

However, the vast and intricate nature of cultural knowledge makes it impractical to rely solely on dataset expansion to improve GenAI's ability to support culturally relevant design. A more effective approach may involve structuring cultural content. Previous work in the field of painting~\cite{Chung:2023:promptpaint} has attempted to extract vague concepts and process them individually, but these models often lack the abstract knowledge necessary, resulting in a disconnect between abstract ideas and concrete visual outputs. By structuring abstract content into design tasks and design spaces, models can generate more appropriate outputs for cultural art-related tasks.

In this study, HarmonyCut organizes the paper-cutting design process around factors and patterns, while Magical Brush uses symbols as fundamental elements in Chinese painting~\cite{Xu:2023:magicalbrush}. Similarly, template-based methods have been applied to dynamic shadow puppetry creation~\cite{Yao:2024:shadowmaker}. \revisedtext{These structured approaches provide a foundational framework for complex and profound cultural knowledge, allowing models to better understand with cultural content, such as the visual selection and composition in HarmonyCut, partially bridging gaps in expertise and cultural background. This makes it possible for public users to access relevant knowledge, participate in cultural creative design, and even further develop expertise, as supported by the feedback from E1–E3 agreeing with the cultural aspects in~\autoref{table3}, thereby engaging and supporting cultural communication while sustaining the vitality of cultural content such as ICH.}
% \rrtext{当然,结构化的方法可以更好的辅助模型理解cultural knowledge,但对于large-scale的knowledge如何转变为structured,本文主要通过mannual annotation with fine-tuning model的方式,还是无法完美的保证annotation comprehensively cover 所有知识,这也在section limitation中被承认。}
\rrtext{While our structured approach, which relies on manual annotation and model fine-tuning, enhances the model's ability to understand cultural knowledge, it remains limited in achieving comprehensive coverage of large-scale knowledge, as further discussed in~\autoref{lf}.}
% 尽管对应思维完成主题与形式的匹配是它的最基础首要能力,但也需要跟随着新事物的出现,以及新内涵的赋予,将他们用于创作中。 
%剪纸作为一个很靠经验的 民俗艺术,很多都是为了传达吉祥寓意,其首要考量的就是视觉形式与构图是否能够体现作者巧思且满足祝福他人的目的, 且被多个literature强调有关选择这一过程体现的对应思维的重要性。因此,对于剪纸的domain expertise,其首要考量的就是能否选择合适的内容,以及巧妙地构图。当然,剪纸的知识以及所能掌握的专业程度远不止于此,需要长年的学习和体验,才能将在对应思维的基础上更好的融入自己的风格,既能通过丰厚的知识继承过去的内容,也能有把握将新事物以剪纸形式表现进行创新。

\subsection{Limitations and Future Work}\label{lf}
Our work has several limitations that future work can address.
First, the system currently does not support collaborative creation, although prior work has demonstrated that sharing mood boards~\cite{Koch:2019:mayai, Koch:2020:semanticcollage} can enhance the design process. Future work could explore GenAI-aided creativity in multi-user collaboration by implementing features that facilitate cooperative design.
\revisedtext{Second, despite efforts to recruit users with varying levels of familiarity with GenAI, the limited sample size in both formative and user studies resulted in insufficient diversity, affecting the generalizability of the design process. To enhance the system's adaptability, we plan to expand the scale of formative studies to validate and refine the design space and broaden user studies to include participants with more diverse backgrounds and expertise levels. This will help investigate how GenAI expertise influences the use of Creative Support Tools in design.
While the current design space is relatively comprehensive according to the evaluation results and feedback, as a folk art with a vast and profound nature, paper-cutting requires further expansion of labeled data to support a more exhaustive reference list. To ensure broader applicability of the design space, we will continue to update and expand the dataset to support more extensive and in-depth research into paper-cutting art.}
% 我们目前的系统可以允许用户完成从需求,到构思再到完成构图最后设计出剪纸作品, 但最为一个handicraft art, 我们可以从 fabrication 探究从design 到 creation的角度将完整的 paper-cutting creating 过程借助工具完成
Additionally, the prototype system supports only 2D monochrome paper-cutting, which is the most prevalent form of Chinese paper-cutting. However, it does not accommodate multicolored paper-cutting.
Our current system allows users to transition from their initial intent through ideation to finalizing the composition, ultimately resulting in the design of a paper-cutting piece. Creators can use this design directly as a guide to complete the final creation with knives or scissors. This advancement provides an opportunity to further explore the fabrication aspect, focusing on how tools can support the complete paper-cutting creation process, from design to physical realization.
