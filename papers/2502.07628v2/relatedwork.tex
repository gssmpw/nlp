\section{Background and Related Work}
This work aims to assist users in designing paper-cuttings by leveraging GenAI for reference exploration to enhance creativity. In this section, we review previous literature on the following topics: (1) the background of paper-cutting, its current challenges in creation and inheritance, and works that utilize computer graphics and computation-focused methods to aid in paper-cutting design and creation; (2) previous GenAI-aided system in facilitating creativity, particularly in traditional arts; and (3) the use of reference exploration in fostering creative design.

\subsection{Chinese Paper-cutting}

\subsubsection{Background of Chinese Paper-cutting}
% background of paper-cutting and pattern. 
% 中国剪纸,作为世界非物质文化遗产的重要组成部分,以其材料成本低廉、易于获取为特点,同时展现出丰富多变的造型与深刻的文化内涵,兼具高度的观赏价值与收藏潜力。从物理形态分析,剪纸作品多为在单色纸张上通过精细镂空工艺制成的二维图形。在文化内涵层面,剪纸艺术根植于民众劳动实践之中,是劳动人民对现实生活与美好愿景的艺术化表达,体现了人类早期的审美观念与理想追求,蕴含了丰富的宗教信仰、道德观念及乡土人文特色。

% 历经千年的传承与发展,剪纸艺术跨越地域与民族的界限,广泛融入各地习俗生活,保持了其独特的传统形态。其内容与形式随时代变迁与地域文化融合而不断演变,形成了多样化的造型样式、丰富的技法体系及独特的艺术特色。同时,剪纸艺术的产生、变化及传承方式亦随时代而变,赋予其在不同历史时期不同的功能与价值。此外,剪纸艺术不受时间与空间限制,仅需基础的剪刻技艺、创意构思及简单工具(如剪刀),即可创作,这一特性确保了其在民间持续不断地传承与发展,历经数千年而生命力旺盛
Chinese paper-cutting, a significant element of the world's ICH~\cite{ich2009unesco}, is known for its affordability and diverse forms that carry profound cultural connotations~\cite{Qiao:2011:liveICH, Bai:2003:stylistic, Hu:2017:tda}. Physically, paper-cuttings are basically two-dimensional figures hollowed on single-color paper. Culturally, they originate from people's labor practices, serving as an artistic representation of daily life and aspirations~\cite{Cui:2016:sot, Wang:2021:dap, Wang:2021:folk}. These works embody human aesthetic concepts and ideals, incorporating religious beliefs, moral values, and cultural elements~\cite{Cao:2023:the}.
Over centuries, paper-cutting has surpassed regional and ethnic boundaries, integrating into local customs while preserving its form with connotation~\cite{Zhang:2018:sos, Ma:2010:sof}. 


% challenges of paper-cutting and pattern creation.
%现在,剪纸已经发展从传统的镂空民间艺术,开始作为文化元素运用到室内外装修、纺织品、服饰、广告设计、POP 设计、包装 设计、书籍装帧等领域被广泛应用。但随着人们欣赏水平及审美观的逐渐提高, 如何创作出那种超越自然形态,超越客观逻辑的栩栩如生的剪纸造型,对设计 者的想象力、创造力提出了更高的要求。可以说,剪纸艺术发展到今天已不仅仅是对传统的继承, 而是更多地注入了设计的观点。用现代产品设计的眼光对 传统的剪纸艺术进行再思索、再创作,以充分体现中华民族几千年华夏文明和 深厚的文化内涵,这其中对创新性的要求不言而喻。
% 1A. 在传承发展中,受到现代工业文明的冲击,剪纸作为传统手工艺,实用功能逐渐减少,文化内涵逐渐失传, 面临着后继无人的尴尬境地。同时剪纸受到全球化的冲击,国外写实主义艺术 进入,中国剪纸开始“西化”,逐渐失去传统特色,因此需要对中国剪纸艺术 进行保护和研究,寻找其民族性特征,保留中国文化精髓。
% 2A. 现代社会剪纸的实用功能逐渐丧失,传统剪纸逐渐走向后继无人,难以突破创新
% 3A. 目前剪纸的发展,写实性剪纸较多,场面宏大,内容丰富,却缺少了传统 剪纸的风味
% 4B. 然而,我国的剪纸作品大多散落在民间,且剪纸纸张不易长期保存,导致 流传下来的剪纸佳作极少,因此对剪纸艺术的数字化保护迫在眉睫,我国的民 间剪纸也被列入了中国民间十大文化遗产抢救工程之一。此外,剪纸的设计依赖 于丰富的剪纸经验,且现有的剪纸主题局限于传统的婚庆祭祀,普通爱好者难 以推陈出新,极大地限制了剪纸艺术的传播与传承。
% 5B. 锯齿纹在现代剪纸中仍旧占有较大的比重,但艺人们已经很少能够解读其中的内涵了

% 然而,随着现代化与城市化发展,这些根植于民间尤其是农村的剪纸从主题内容到传播途径再到人才培养都面临着极大的发展困境,不利于实现自身的保护与传承
With the acceleration of modernization and urbanization, paper-cutting faces significant challenges in themes, content, and talent cultivation, affecting its innovation and inheritance~\cite{Zhang:2018:sos, Li:2023:digitalpapercut, Wang:2021:dap}. The content of paper-cuttings increasingly reflects modern life and realism~\cite{Zhang:2018:sos, Cao:2023:the}, leading to the gradual disappearance of traditional themes and their symbolic meanings. The economic returns from modern paper-cutting are relatively low, making it difficult to sustain as a profession, resulting in fewer individuals pursuing it. Additionally, as a form of folk art~\cite{Wang:2021:folk}, paper-cutting primarily relies on oral and practical teaching and lacks systematic documentation, complicating its transmission~\cite{Wang:2018:transinheritance} and making it inaccessible for both the public and practitioners for reference. Consequently, this intensifies the homogenization and stylization of paper-cutting themes with diminished meaning, thereby inhibiting innovation in design and creation~\cite{Zhang:2018:sos}.

% 一些工作致力于对有固定模式的剪纸和纹样(如,对称剪纸,几何纹样)的几何特征的分析,提取与纹样分类,进而规则化的计算并生成纹样,构建纹样库。 甚至通过演化算法来怎加不同形态,风格的纹样。但对于拥有丰富造型和尤其是多样内涵的纹样,生成的设计内容尤其在类别和风格上依旧非常有限。 对于使用计算机视觉方法,Liu Enmao 开发了基于CNN的纹样识别以及根据从构建纹样库中选择纹样来与轮廓构成剪纸,但由于纹样库和模型能力,目前的系统对输入的图片有限制,构图清晰、背景简单的图片更容易被识别并与剪纸图案匹配。
% 还有一些工作通过利用组合数学的算法来对图像进行多层处理,直接得到具有连通性的剪纸图像。但这些工作却还是难以实现符合设计需求的寓意、内涵与具体的形状匹配。同时,算法为保证连通性,难以兼顾整体的美观性。还有工作通过迁移学习的方法,将人像转变为剪纸风格,并改进联通性,但其风格主要在写实风格的剪纸表现好,这与Liu Enmaod的系统面临相似的问题,作为更具独特风格、分化内涵的抽象简直在识别、迁移等学习上是困难的
% The advancements in digital technology have led to a growing application of computer graphics and computer vision (CV) in assisting the design and creation of paper-cutting. For example, 
% several pattern-based studies have focused on analyzing the geometric features of patterns, such as symmetrical and geometric motifs~\cite{Zhang:2005:cag, Zhang:2006:cpc, Zhang:2009:cutout, Shui:2008:edgepaper, Liu:2018:pdf, Li:2020:aug}. These efforts involve extracting and classifying patterns to compute and generate them, thereby building a pattern library. Some researchers have even employed algorithms to enhance various forms and styles of patterns~\cite{Liu:2009:rai}. However, these methods are limited in generating designs with complex shapes and diverse meanings, leading to a narrow range of categories and styles.
% Other image-based approaches utilize algorithms for multi-layer image processing to directly create connected paper-cutting images~\cite{Xu:2007:computer, Xu:2008:artistic, Hu:2017:tda}. However, these methods also struggle to create meaningful content that aligns with design needs and specific shapes, sometimes sacrificing overall aesthetics to maintain connectivity.
% Some works applied the CV method, including CNN for pattern recognition~\cite{Liu:2020:intcut}, and transfer learning to transform portraits into paper-cutting styles~\cite{Meng:2010:apc}. Both of them face the challenges of processing abstract paper cuttings with unique styles and deep meanings.
% Consequently, our motivation is to refine existing design frameworks by incorporating generative AI to address the above problem, which has been shown in various studies to support creative design. This integration aims to improve the diversity and harmony between form and meaning in paper-cutting designs.
\subsubsection{Paper-cutting Design with Digital Technology }
Advancements in digital technology have facilitated the use of CG and computer vision (CV) in paper-cutting design and creation. Several studies have concentrated on analyzing geometric features, such as symmetrical and geometric motifs, to build pattern libraries by extracting and classifying patterns~\cite{Zhang:2005:cag, Zhang:2006:cpc, Zhang:2009:cutout, Shui:2008:edgepaper, Liu:2018:pdf, Li:2020:aug}. Some have even utilized algorithms to enhance various pattern forms and styles~\cite{Liu:2009:rai}. Nonetheless, these approaches are constrained in generating designs with complex shapes and diverse meanings, resulting in a limited range of categories and styles.
Image-based methods employ multi-layer image processing algorithms to create connected paper-cutting images directly~\cite{Xu:2007:computer, Xu:2008:artistic, Hu:2017:tda}. Despite their utility, these methods often struggle to produce meaningful content that matches design needs and specific shapes, sometimes compromising aesthetics to maintain connectivity.
Some works have applied CV techniques, such as utilizing the convolutional neural network for pattern recognition~\cite{Liu:2020:intcut} and transfer learning to transform portraits into paper-cutting styles~\cite{Meng:2010:apc}. These methods face difficulties in processing abstract paper cuttings with unique styles and deep meanings. In fact, many related fields, such as graphic design, have already started integrating GenAI into their systems. To the best of our knowledge, no prior work has investigated the use of GenAI to support paper-cutting design. By examining how GenAI supports creative design in these areas, we aim to leverage this approach and, through the integration of the paper-cutting knowledge we have collected, realize GenAI-aided paper-cutting design.
% introduction部分的展开:
% 在现代生活的显著影响下,部分与当前社会生产生活脱节的文化元素逐渐被边缘化。与此同时,剪纸艺术写实风格的流行趋势加剧,两者共同作用,使得部分富含历史与文化价值的传统元素在剪纸创作实践中逐渐被淡化,甚至消失。另一方面,由于剪纸艺术长期植根于民间, 且其技艺的传承高度依赖于师徒间的口传心授与实际操作经验,而相对缺乏系统性的文字记录与深入的理论研究资料,相对缺乏系统性的文字记录与深入的理论研究资料。这一现状不仅限制了普通民众、剪纸艺术爱好者对剪纸艺术的认知,还极大地阻碍了他们对剪纸艺术中丰富文化元素及相关知识的掌握。这进一步加剧了剪纸作品文化内涵和作品题材的单一化趋势,从而影响剪纸艺术的传承与创作活力。

\subsection{Generative AI for Creativity}
% 一、现有GenAI模型及能力
% 最近,随着GenAI,在语言模型、图像生成模型到多模态模型的发展【如blip,clip,sd,gan等】,GenAI基于其广泛的训练资源,可以理解并较好完成用户所给需求,已在多个场景下被用于支持的创意设计。尤其是在避免multiple variation问题上,模型可以较好的理解用户的需求,并通过其多种模态语义的一致性而准确进行理解,来推荐相关内容用于设计,提高创意和多样性。
GenAI, leveraging advancements in language models~\cite{Radford:2018:gpt2, Radford:2019:gpt2}, image generation models~\cite{Kingma:2022:vae, Goodfellow:2020:gan}, and multi-modal models~\cite{Radford:2021:clip, Li:2022:blip, Rombach:2022:stablediffusion}, employs its extensive training resources and capabilities in multi-modal information alignment to effectively understand and generate diverse content, thereby facilitating user creativity. In the realm of creative design, proposed language models~\cite{Swanson:2021:story}  are utilized to provide users with abundant ideas and text suggestions during the writing process. Meanwhile, recent text-image models~\cite{Chung:2024:styleid, Zhang:2023:inst, Chen:2024:democaricature, Zhang:2023:prospect}, with their comprehension of style, position, and color, enable the translation of user ideas into images across various scenarios. This synergy of language and image models enhances the creative process by offering integrated support for both ideation and visual generation.


%% % 2. GenAI的使用可以解决:
% Multiple Variation  问题:不管是谁,创作过程很容易陷入单一思路,导致没有创意;但先前有很多工作证明,GenAI可以提高variation
% 然而, GenAI is doubted from rationality and controllability~\cite{Hou:2024:c2ideas} since these models are highly uncertain and uncontrollable. Specific to the paper-cutting, it is difficult for the model to connect the cultural connotation and pattern

% 二、使用AI模型的CST工作
% 在图形设计领域: A工作,B工作,C工作

% 而在与剪纸设计相关的领域、如图形设计等领域,越来越多工作通过GenAI与interface结合,来支持在更细化更多样的任务提供创意支持:在ideation上 Gancollage 通过StyleGAN-driven digital mood board来辅助构思,creativeconnect 通过用户根据reference的重组和来生成创造力的idea; 

% 在visual representaion上,promptcharm facilitates text-to-image creation through multi-modal prompt engineering and refnement.
% PromptCharm facilitates text-to-image creation through multi-modal prompt engineering and refnement.
% Contextcam: a human-AI co-creation system that incorporates context awareness to generate artistic images. 

% 而也有很多工作通过LLM提供具体任务的,如类别、比喻的suggestion,用户根据suggestion借助图像生成模型将idea 转化为最终视觉上的设计结果:如
% Chen et al. 提出通过GenAI提供类比建议并根据建议进行材料组合,得到illustration draft,来怎强用户对数据的理解和传播。
% Wan et al. 提出通过GenAI生成隐喻建议并据此生成visual metaphor,来完成 creative visual storytelling of emotional experiences during dreams。
% Typedance 借助GenAI,来允许用户从构思、生成,迭代的工作流程中create semantic typographic logos from user-customized images.

\subsubsection{Creative Support Tool with Generative AI}
To integrate GenAI into specific areas within a controllable process that includes human involvement, numerous frameworks and interfaces have been proposed to support GenAI-aided creativity.
In the realm of graphic design, which is closely related to paper-cutting design, there is a growing trend of integrating GenAI with interfaces to support more specialized and diverse creative tasks. For ideation, tools like GANCollage~\cite{Wan:2023:gancollage} and CreativeConnect~\cite{Choi:2024:creativeconnect} utilize GenAI-driven digital mood board to facilitate brainstorming and recombine references for the creative idea.
Regarding visual representation, some works use GenAI, such as PromptCharm~\cite{Wang:2024:promptcharm} and ContextCam~\cite{Fan:2024:contextcam} to enable multi-modal models with refined prompt and context awareness to generate artistic images that meet various needs. 
Additionally, in specific areas such as communication, emotional affect, and typography, many projects utilize GenAI to suggest ideas and transform them into final visual designs. For instance, the work of Chen et al.~\cite{Chen:2024:dataanalogy} and the Opal~\cite{Liu:2022:opal} both leverage GenAI to generate and apply suggestions for effective illustrations creation. Wan et al.~\cite{Wan:2024:metamorpheus} use GenAI to develop metaphorical suggestions, supporting visual storytelling of emotional experiences. TypeDance~\cite{Xiao:2024:typedance} empowers users to design semantic typographic logos from customized images through a structured workflow.
% 而与剪纸更加相关的传统艺术创作,magical brush通过GenAI生成带有符号化的文化内容,并提供预定义的内容库,试图帮助新手创作出完整的现代中国画。
% shadowmaker 允许设计师可以通过将草图进行风格迁移生成皮影图像,并进行组合最终用于创作皮影动画。
In traditional art creation, which is more closely related to paper-cutting, Magical Brush~\cite{Xu:2023:magicalbrush} generates symbolic cultural content with GenAI and offers a predefined material library to assist novices in creating modern Chinese paintings. ShadowMaker~\cite{Yao:2024:shadowmaker} enables designers to generate shadow puppet images through style transfer of sketches, ultimately combining them to create shadow puppet animations.
% 四、 1.证明GenAI辅助设计的探索过程是需要有参考来进行的,尤其是有关空间和形状的构造和选择,用户往往难以设计,否则直接通过模型生成结果来完成设计,会让整个过程被模型主导。但以上传统艺术创作的工作主要靠生成的内容来引导,只有有限的预定义内容来保证system不会被模型的不可控影响。另外,GenAI需要领域知识才更合理的完成特定任务,但目前对于paper-cuting在GenAI辅助设计的知识和设计空间的探索很少。
% 因此,我们致力于探索GenAI辅助剪纸设计的design space,并通过所得design space来指导用户进行reference-based exploration,改进系统的可控性和合理性
% 然而,空间构成和形状替换在无参考情况下存在挑战。为最终获得一个纹样搭配、位置安排合理,且内涵与内容相符的剪纸设计。在这项工作中,通过提供factors来让用户有oriented进行ideation,并借助检索和生成的内容作为用户的参考来完成从概念设计到configuration的过程~\cite{Hubka:1992:engineering}。
% 但是,很多工作在协助用户在创意转化上很好,但是在很少探索整个design process上来启发用户,通过让用户在guidance下进行探索。

\revisedtext{Although GenAI demonstrates potential in supporting creative tasks, it also faces limitations. A key issue is the end-to-end nature, which limits user involvement in the creative process. Creative tasks often rely on iterative refinement, but GenAI lacks the flexibility to provide control over iterations or ensure alignment with user expectations~\cite{Hou:2024:c2ideas, Brown:2020:fewshotlearners, Duvsek:2020:end-to-end}. Another major limitation arises from the model's dependence on data. In tasks requiring domain-specific knowledge, GenAI frequently underperforms~\cite{Hou:2024:c2ideas, Cui:2024:chatlaw, Wang:2023:methodsknowledge}. Moreover, biases in training data often result in homogenized outputs~\cite{Amderson:2024:homogenization}, reducing diversity and originality while limiting users to explore and innovate. These biases may even lead to issues about copyright or ethical concerns, further complicating the creative process~\cite{Zhou:2024:biasgenerativeai, Samuelson:2023:copyright}. To address these challenges, we restrict GenAI's role to providing partial reference support during the design process and incorporate predefined elements to reduce model unpredictability. Furthermore, as GenAI has yet to fully explore the domain-specific knowledge and design space of paper-cutting, we focus on advancing this area and aim to guide users through reference-based exploration while improving controllability and rationality~\cite{Hou:2024:c2ideas}.}

% 三、存在的问题
% 1. 然而,生成过程导致不稳定的结果缺乏用户控制。为了解决这个问题并增强用户定制,
% 2. 研究人员认为不可预测性和不确定性对人工智能系统的用户体验有害[5]。
% 2. GenAI的使用可以解决:
% Multiple Variation  问题:不管是谁,创作过程很容易陷入单一思路,导致没有创意;但先前有很多工作证明,GenAI可以提高variation
% 然而, GenAI is doubted from rationality and controllability~\cite{Hou:2024:c2ideas} since these models are highly uncertain and uncontrollable. Specific to the paper-cutting, it is difficult for the model to connect the cultural connotation and pattern

% 过渡部分可以强调:不论是creativeconnect,还是magical brush等工作,都证明了reference exploration and combination 对于设计的重要性。 然而,1.它们的内容推荐的内容往往都是通过生成式AI进行生成:皮影。对于如何构建有关剪纸设计的的领域知识并与Genai交互来让整个有推荐的探索可控且不单一,是没有工作进行探索。

\subsubsection{Reference Exploration in Design}
% 另外,对于GenAI
%  2. 大量研究表明,在进行概念设计过程中最重要的参考和借鉴资料就是以往的设计方案。人们通常是对以往的设计进行一方面或者多方面的改进,从而产 生新的设计方案以改善产品的外观、质量或者性能。基于这种特点人们将进化 计算技术应用于概念设计,使用智能计算的方法辅助进行创新设计,取得了非常大的成功

% 依据:有一个和reference retrieval and exploration有关的的好例子:关于创作过程,如肖等人。[54]已确定,主流作品遵循两个阶段的管道,包括检索示例并将其改编为设计材料[62]和风格转移参考[45]。


% 因此,我们要做trade-off:不探索,创作容易僵化,探索,又往往数据过大,没有限制和引导,容易迷失。所以guided exploration很重要
% 整个设计过程要求用户需要根据多multiple alternatives~\cite{Jansson:1991:designfixation,Goldschmidt:2011:avoidingfixation}完成, 而GenAI也因为具有提供large scale multiple high-quality variation的能力而用于creative support。 但是有几个问题:首先,GenAI生成的内容一般都是端到端的complete content, 这导致用户在整个设计过程容易被模型主导,只是采纳了模型提供的内容之一,而非形成真正的创意。 另外,虽然根据需求可以生成大量的内容,如果在没有引导下,过多的内容会让用户不知所措,不知道该选择哪些来完成设计。 (创意设计一般从各种来源收集相关的灵感材料开始~\cite{Hubka:1992:engineeringdesign}。)根据概念设计中一个重要步骤就是参考借鉴以往reference,具有引导的帮助用户浏览相关的设计空间。这样既不会因为设计过程完全依靠完成度高的生成内容而缺少可以根据粗糙但创新的想法而进行设计~\cite{Tohidi:designreight},又不会因为提供创意的内容过多而无法选择。
An effective design process requires users to explore multiple alternatives~\cite{Jansson:1991:designfixation, Goldschmidt:2011:avoidingfixation}, and GenAI is employed for creative support due to its capability to offer numerous high-quality variations. However, some challenges persist: GenAI-generated content is often end-to-end complete, potentially causing users to become overly reliant on the model, merely selecting the model's suggestions instead of fostering genuine creativity~\cite{Tohidi:designreight}. Furthermore, although a vast amount of content can be produced based on user requirements, without proper guidance, users may feel overwhelmed by the excess of options~\cite{Suh:2024:luminate}, making it challenging to decide which elements to incorporate into their designs. A key step in the design process is drawing inspiration from references~\cite{Shneiderman:2000:creating, Eckert:2000:sources}, which assists users in navigating relevant design spaces. This strategy ensures that the design process is not solely dependent on fully developed generated content, allowing for the creation of designs based on rough yet innovative ideas~\cite{Tohidi:designreight, Choi:2024:creativeconnect}, and also mitigates the risk of overwhelming users with too many creative options. 
%% 1. mood board 的作用
For the aforementioned issues, previous work has demonstrated that mood boards~\cite{Cassidy:2008:moodboards} can help users explore and be guided by reference and inspiration~\cite{Eckert:2000:sources, Garner:2001:problem}. Moreover, an increasing number of GenAI-aided systems support design by utilizing interactive mood boards~\cite{Choi:2024:creativeconnect, Wan:2023:gancollage, Peng:2024:designprompt}. Inspired by previous work, our research facilitates reference-guided exploration in two primary steps of the paper-cutting workflow and utilizes a mood board to support users in the selection, arrangement, and cutout of these references for paper-cutting design.