%%
%% This is file `sample-sigconf-authordraft.tex',
%% generated with the docstrip utility.
%%
%% The original source files were:
%%
%% samples.dtx  (with options: `all,proceedings,bibtex,authordraft')
%% 
%% IMPORTANT NOTICE:
%% 
%% For the copyright see the source file.
%% 
%% Any modified versions of this file must be renamed
%% with new filenames distinct from sample-sigconf-authordraft.tex.
%% 
%% For distribution of the original source see the terms
%% for copying and modification in the file samples.dtx.
%% 
%% This generated file may be distributed as long as the
%% original source files, as listed above, are part of the
%% same distribution. (The sources need not necessarily be
%% in the same archive or directory.)
%%
%%
%% Commands for TeXCount
%TC:macro \cite [option:text,text]
%TC:macro \citep [option:text,text]
%TC:macro \citet [option:text,text]
%TC:envir table 0 1
%TC:envir table* 0 1
%TC:envir tabular [ignore] word
%TC:envir displaymath 0 word
%TC:envir math 0 word
%TC:envir comment 0 0
%%
%%
%% The first command in your LaTeX source must be the \documentclass
%% command.
%%
%% For submission and review of your manuscript please change the
%% command to \documentclass[manuscript, screen, review]{acmart}.
%%
%% When submitting camera ready or to TAPS, please change the command
%% to \documentclass[sigconf]{acmart} or whichever template is required
%% for your publication.
%%
%%
% \documentclass[manuscript,review,anonymous]{acmart}
% % \documentclass[sigconf,review,anonymous]{acmart}
\documentclass[sigconf]{acmart}
\usepackage{array}
\usepackage{subcaption} 
\usepackage{multirow}
\usepackage{makecell}
\usepackage{fontawesome}
\usepackage{float}
\newcommand\cincludegraphics[2][]{\raisebox{-0.05\height}{\includegraphics[#1]{#2}}}
%%
%% \BibTeX command to typeset BibTeX logo in the docs
\AtBeginDocument{%
  \providecommand\BibTeX{{%
    Bib\TeX}}}

%% Rights management information.  This information is sent to you
%% when you complete the rights form.  These commands have SAMPLE
%% values in them; it is your responsibility as an author to replace
%% the commands and values with those provided to you when you
%% complete the rights form.
\setcopyright{acmlicensed}
\copyrightyear{2025}
\acmYear{2025}

%% These commands are for a PROCEEDINGS abstract or paper.
%%
%%  Uncomment \acmBooktitle if the title of the proceedings is different
%%  from ``Proceedings of ...''!
%%
%

% Title of ACM publication:	CHI '25: CHI Conference on Human Factors in Computing Systems Proceedings
\acmConference[CHI '25]{CHI Conference on Human Factors in Computing Systems}{April 26-May 1, 2025}{Yokohama, Japan}
\acmBooktitle{CHI Conference on Human Factors in Computing Systems (CHI '25), April 26-May 1, 2025, Yokohama, Japan}
\acmDOI{10.1145/3706598.3714159}
\acmISBN{979-8-4007-1394-1/25/04}

\definecolor{vividauburn}{rgb}{0.58, 0.15, 0.14}
\definecolor{rufous}{rgb}{0.66, 0.11, 0.03}
\definecolor{red-brown}{rgb}{0.65, 0.16, 0.16}
\definecolor{red(ncs)}{rgb}{0.77, 0.01, 0.2}
% \newcommand{\revisedtext}[1]{\textcolor{red(ncs)}{#1}}
\newcommand{\revisedtext}[0]{}
% \newcommand{\rrtext}[1]{\textcolor{red(ncs)}{#1}}
\newcommand{\rrtext}[0]{}
%%
%% Submission ID.
%% Use this when submitting an article to a sponsored event. You'll
%% receive a unique submission ID from the organizers
%% of the event, and this ID should be used as the parameter to this command.
%%\acmSubmissionID{123-A56-BU3}

%%
%% For managing citations, it is recommended to use bibliography
%% files in BibTeX format.
%%
%% You can then either use BibTeX with the ACM-Reference-Format style,
%% or BibLaTeX with the acmnumeric or acmauthoryear sytles, that include
%% support for advanced citation of software artefact from the
%% biblatex-software package, also separately available on CTAN.
%%
%% Look at the sample-*-biblatex.tex files for templates showcasing
%% the biblatex styles.
%%

%%
%% The majority of ACM publications use numbered citations and
%% references.  The command \citestyle{authoryear} switches to the
%% "author year" style.
%%
%% If you are preparing content for an event
%% sponsored by ACM SIGGRAPH, you must use the "author year" style of
%% citations and references.
%% Uncommenting
%% the next command will enable that style.
%%\citestyle{acmauthoryear}


%%
%% end of the preamble, start of the body of the document source.
\begin{document}

%%
%% The "title" command has an optional parameter,
%% allowing the author to define a "short title" to be used in page headers.
% \title{``Unity of form and spirit'': Supporting Guidance and Recommendation for Chinese Paper-cutting Ideation/Creation with Generative AI}
\title{HarmonyCut: Supporting Creative Chinese Paper-cutting Design with Form and Connotation Harmony}
% \renewcommand{\shorttitle}{HarmonyCut: Supporting Creative Chinese Paper-cutting Design with Form and Connotation Harmony}
%%
%% The "author" command and its associated commands are used to define
%% the authors and their affiliations.
%% Of note is the shared affiliation of the first two authors, and the
%% "authornote" and "authornotemark" commands
%% used to denote shared contribution to the research.

\author{Huanchen Wang}
\orcid{0000-0001-9339-1941}
\authornote{Equal contribution.}
\affiliation{
  \department{Department of Computer Science and Engineering}
  \institution{Southern University of Science and Technology}
  \city{Shenzhen}
  \state{Guangdong}
  \country{China}
}
\affiliation{
  \department{Department of Computer Science}
  \institution{City University of Hong Kong}
  \city{Hong Kong}
  \country{China}
}
\email{wanghc2022@mail.sustech.edu.cn}


\author{Tianrun Qiu}
\orcid{0009-0002-9878-663X}
\authornotemark[1]
\affiliation{
  \department{Department of Computer Science and Engineering}
  \institution{Southern University of Science and Technology}
  \city{Shenzhen}
  \state{Guangdong}
  \country{China}
}
\email{qiutr@mail.sustech.edu.cn}

\author{Jiaping Li}
\orcid{0000-0003-0128-8993}
\affiliation{
  \department{Department of Computer Science and Engineering}
  \institution{Southern University of Science and Technology}
  \city{Shenzhen}
  \state{Guangdong}
  \country{China}
}
\email{lijp2024@mail.sustech.edu.cn}

\author{Zhicong Lu}
\orcid{0000-0002-7761-6351}
\affiliation{
  \department{Department of Computer Science}
  \institution{George Mason University}
  \city{Fairfax}
  \state{Virginia}
  \country{USA}
}
\email{zlu6@gmu.edu}


\author{Yuxin Ma}
\authornote{Corresponding author.}
\orcid{0000-0003-0484-668X}
\affiliation{
  \department{Department of Computer Science and Engineering}
  \institution{Southern University of Science and Technology}
  \city{Shenzhen}
  \state{Guangdong}
  \country{China}
}
\email{mayx@sustech.edu.cn}


%%
%% By default, the full list of authors will be used in the page
%% headers. Often, this list is too long, and will overlap
%% other information printed in the page headers. This command allows
%% the author to define a more concise list
%% of authors' names for this purpose.
% \renewcommand{\shortauthors}{Wang et al.}

\begin{abstract}
Role-Playing Agent (RPA) is an increasingly popular type of LLM Agent that simulates human-like behaviors in a variety of tasks. 
However, evaluating RPAs is challenging due to diverse task requirements and agent designs.
This paper proposes an evidence-based, actionable, and generalizable evaluation design guideline for LLM-based RPA by systematically reviewing $1,676$ papers published between Jan. 2021 and Dec. 2024.
Our analysis identifies six agent attributes, seven task attributes, and seven evaluation metrics from existing literature.
Based on these findings, we present an RPA evaluation design guideline to help researchers develop more systematic and consistent evaluation methods.

\end{abstract}


% to synthesize what agent attributes and task attributes prior literature have considered influence the selection of evaluation metrics, as well as the relationships between these factors.
% For each agent attribute and task category, we summarize its distinct associations with RPLA's evaluation metrics, providing practical guidance on comprehensive based on their RPLA's design. Additionally, we explore the  between agent attributes and downstream tasks to support researchers in refining RPLA design choices.
%%
%% The code below is generated by the tool at http://dl.acm.org/ccs.cfm.
%% Please copy and paste the code instead of the example below.
%%
\begin{CCSXML}
<ccs2012>
   <concept>
       <concept_id>10003120.10003121.10003129</concept_id>
       <concept_desc>Human-centered computing~Interactive systems and tools</concept_desc>
       <concept_significance>500</concept_significance>
       </concept>
 </ccs2012>
\end{CCSXML}

\ccsdesc[500]{Human-centered computing~Interactive systems and tools}
%%
%% Keywords. The author(s) should pick words that accurately describe
%% the work being presented. Separate the keywords with commas.
\keywords{Creativity support tool, Chinese paper-cutting, Generative AI-aided design, Intangible Cultural Heritage}
%% A "teaser" image appears between the author and affiliation
%% information and the body of the document, and typically spans the
%% page.
\begin{teaserfigure}
  \includegraphics[width=\textwidth]{Images/workflow.pdf}
\caption{\label{figure1}
A general two-stage (ideation and composition) workflow and design space (four factors and one element) for GenAI-aided paper-cutting design is outlined from the two-step formative study, with the main challenges in the workflow labeled on corresponding stages. Based on the workflow and challenges, the design goals are solidified in the pipeline and interface of HarmonyCut.}
\Description{This figure shows a general two-stage (ideation and composition) workflow and design space (four factors and one element) for GenAI-aided paper-cutting design outlined from the two-step formative study, with the main challenges in the workflow labeled on corresponding stages. Based on the workflow and challenges, the design goals are solidified in the pipeline and interface of HarmonyCut.}
  \label{fig:teaser}
\end{teaserfigure}

% \received{20 February 2007}
% \received[revised]{12 March 2009}
% \received[accepted]{5 June 2009}

%%
%% This command processes the author and affiliation and title
%% information and builds the first part of the formatted document.
\maketitle

In recent years, there has been a notable increase in the development and research of tethered UAVs, reflecting a growing interest in their diverse applications. One of the main motivations is to carry out long-term missions with aerial vehicles, as these present significant challenges due to the limitations of current battery solutions \cite{robotics12040117}. A UAV tethered to a UGV is an interesting configuration, as the UGV can power the UAV through the tether for longer times given the higher payload of the former.  %According to this, an interesting configuration to allow long-duration flights of a UAV is a tethered robot configuration in which a UGV is tied to the UAV and powering it. 
This introduces a paradigm in robotic collaboration, offering distinct advantages over traditional standalone systems by combining the strengths of each of the robotic agents \cite{MooreIROS2018}. %As we venture UGV tied to UAV into scenarios requiring heightened enhanced situational awareness involving an extended operational endurance, the tethered approach proves invaluable, due to the capability to provide energy to the UAV, thus increasing fly time \cite{6961531}. 
When deploying a UGV tethered to a UAV in scenarios requiring increased situational awareness and extended operational endurance, the tethered configuration can become even more invaluable, not only providing the UAV with power to significantly extend its flight time \cite{6961531},  %In this way, the cable plays an important role in providing 
but also with safe high-bandwidth communications \cite{850822,9202196}. 

However, the tethering mechanism introduces several challenges, particularly in modeling the hanging tether state \cite{XiaoSSRR2018}. Unlike standalone systems, where each vehicle operates independently, the tether requires intricate and permanent coordination between the UGV and the UAV. Understanding and managing the state of the tether becomes a critical aspect, which requires sophisticated algorithms and real-time processing capabilities \cite{9561062}. 

\begin{figure}
  \includegraphics[width=0.2\textwidth]{Figures/setup1.png}
  \hfill
  \includegraphics[width=0.2\textwidth]{Figures/setup2.png}
  \caption{Simplified 2D sketch showing an example for motion planning of a tethered UAV-UGV with a hanging tether. (Left) Initial robots and tether configuration, and UAV goal (red circle). (Right) Sequence of robots positions and tether length to reach the given goal. Notice how the goal cannot be reached by means of a taut tether, a hanging tether must be considered in this case.}
  \label{fig:planning-setup}
\end{figure}

The state of the tether has traditionally been analyzed through parameterization, an approach that employs equations to represent its physical behavior, especially the catenary curve \cite{BOOKOFCURVES}. Numerous methodologies, with the aim of simplifying this process, approximate the tether as a straight line \cite{autonomousvisual}\cite{framworktether}\cite{uavfire}. This straight-line approximation is only suitable in scenarios where there is a direct line of sight between the tether endpoints, and thus it inherently restricts the exploratory range of the UAV.

In general, hanging-tether approaches allow UAVs to access a broader range of areas compared to straight tether setups; see Fig. \ref{fig:planning-setup} for an example. This concept has been explored by incorporating tether parameterization into localization or planning processes. For instance, Lima and Pereira \cite{9476778} use the catenary equation to determine the UAV's position.  % This concept has been explored by incorporating tether parameterization into the localization or planning processes, such as in the work conducted by Lima and Pereira \cite{9476778}, where using the catenary equation is feasible to find the UAV position. 
Similarly, in \cite{9364354}, the focus is on computing the state of a catenary tether to localize two UAVs attached at each end. This setup is specifically designed to suspend an object, providing a novel approach to object manipulation using UAVs while maintaining a constant tether length. Another interesting application of the catenary model is presented in \cite{LARANJEIRA2020107018} for underwater operations, where the catenary is used to monitor the status of a cable connected to an \emph{N}-number of ROVs (Remotely Operated Vehicles) performing exploration tasks, also with a constant tether length.

In \cite{8848946}, the parameterization of the tether is used in the localization and control stages to perform two autonomous motion primitives, reactive feedback-based position control and model-predictive feedforward velocity control, but is not used in the planning stage. An interesting approach is presented in \cite{drones7020073}, where a tied unmanned aerial vehicle (TUAV), named ``Oxpecke'', was designed for the inspection of stone-mine pillars. This system uses a sweeping (lawnmower) pattern path planning method intended to map and inspect an entire rectangular area, such as the surface of a pillar. However, the surface to inspect is simple (a rectangle), and the tether length is not directly included in the path planning.

%A general approach about the consideration of the tether in the planning stage is introduced in \cite{battocletti2024entanglementdefinitionstetheredrobots}, where the authors present the definition of tether entanglement problems. Specifically, it addresses the challenges posed by the presence of a tether, including the geometric constraints on the robot's motion due to the finite tether length. For that, different constraints are considered in the planning stage. However, the method is too general and mainly tested in ground points, so UAV implementation are not considering, and algo this method allow tether contact with the floor while entanglement desnt exit.

A comprehensive approach to incorporate a tether in the planning stage is presented in \cite{battocletti2024entanglementdefinitionstetheredrobots}, where the authors define the challenges associated with tether entanglement. Specifically, this work addresses the constraints imposed by the tether on the motion of the robot, particularly the limitations arising from the finite length of the tether. Various constraints are integrated into the planning stage to account for these challenges. However, the proposed method is limited and mainly focused on ground applications, 
thus limiting its applicability to UAVs. Additionally, the approach allows for tether contact with the ground, as long as it does not result in entanglement.

On the other hand, \cite{capitán2024efficientstrategypathplanning} focuses on the development of a path planning strategy for marsupial robotic systems composed of a UGV tethered to a UAV. The article introduces a sequential planning strategy called MASPA (Marsupial Sequential Path-Planning Approach), which allows calculating collision-free 3D trajectories for the tethered UAV-UGV system in complex scenarios, for which the UGV advances to a point where the UAV executes the take-off and then advances to a desired point. This method considers both the geometric limitations imposed by obstacles and the cable and the properties of the joint motion of both robots. A novel algorithm, the PVA (Polygonal Visibility Algorithm), is also presented to identify feasible take-off points and solve visibility problems for the UAV in a three-dimensional space. Despite the novelty of the approach, it is not able to consider coordinated planning of the UGV and the UAV at the same time.

In \cite{smartinezr2023}, the catenary approximation is used to parameterize the state of the tether and plan a collision-free trajectory, in which the UAV must achieve objectives using a hanging tether. However, using the catenary equation, the planning process becomes a time-consuming task, allowing only offline computations. %which makes the planning process to be carried out offline.

%This paper will focus on reducing the complexity associated with the calculation of the variable length hanging tether. We will propose an approach that efficiently calculates the tether state with a minimum representation error concerning the real state, and integrating it into a trajectory planning algorithm for a UGV-UAV tethered team. To this end, we test our approach in the motion planning method for a mobile UGV-UAV tethered system presented in \cite{smartinezr2023}, which is based on two stages. The first stage computes a free-collision path planning for UAV, UGV, and tether, using the RRT* algorithm. The second stage corresponds to a trajectory planning method based on nonlinear optimization that considers smoothness, speed, acceleration limitations of the UGV and UAV, and optimizes the tether configuration to maximize the distance from obstacles. %Unfortunately, considering the real catenary curve in the planner could make it computationally demanding, as shown in our previous work \cite{martinez2021optimization}. In it, we manage to design, implement and test in experiments a two-step optimized planner which considers the catenary shape. For this reason, we propose to approximate the shape of the tether as a parabola without affecting the safety of the planning system and making use of its simpler description to speed-up the computation of optimal paths.

%Our approach is based on the motion planning mentioned above due to the robustness of computed trajectories. Thus, we include in the first stage, a decision problem to set the initial tether length, to quickly obtain a collision-free state for the whole system. Furthermore, we propose a new planner-state parameterization and replace the use of the catenary equation with a parabola equation for estimating the shape of the tether. Thus, the main contributions of the article are:
%\begin{itemize}
%
%\item In Planner Stage: Solving the decision problem to find a collision-free parabola curve instead of the traditional catenary curve. This change allows the RRT* (Rapidly-exploring Random Tree) planner to calculate trajectories faster and more efficiently, since it avoids the computational complexity associated with the calculation of the catenary. The parabolic curve simplifies the collision decision process and increases the success rate in three-dimensional environments with obstacles.
%
%\item  In Optimizer stage: This stage introduces a direct parameterization of the tether in the trajectory state function, which includes the parameters of the curve (parabola or catenary) in the system state vector. This allows a more accurate evaluation of geometric constraints (such as distance to obstacles) and reduces the optimization time by up to an order of magnitude compared to previous methods, achieving safer and smoother trajectories for the UAV-UGV system.
%\end{itemize}

This paper focuses on reducing the complexity associated with the calculation of the variable length hanging tether. %The paper builds on the previous work of the authors \cite{smartinezr2023}, extending it with a new approach that efficiently calculates the tether state with a minimum representation error related to the actual state and a new parameterization of the tether curve in the trajectory optimizer for faster computation. Thus, 
The main contributions are listed below.

\begin{itemize}
    \item A new method for efficient computation of a collision-free catenary curve based on the parabola approximation. This paper proposes using the parabola curve to model the hanging tether curve, detailing the full pipeline, including the computation of the final catenary model. This method reduces the execution time of the path planner to great extent, since it avoids the computational complexity associated with the calculation of the catenary model for tether collision detection. This model also increases the feasibility of the trajectory planner approach, reaching an averaged 98\% of feasibility in the validation scenarios. 

    \item A direct parameterization of the tether in the trajectory state definition, which includes the parameters of the curve (parabola or catenary) in the system state vector. This allows a more accurate evaluation of geometric constraints (such as distance to obstacles) and reduces the optimization time \rev{by more than an order of magnitude} compared to previous methods, achieving safer and smoother trajectories for the UAV-UGV system. \rev{Such improvement opens the door to apply the proposed method to real-time local re-planning.}
\end{itemize}

%The experimental results will show how this new parameterization boosts the computation, while the parabola model will clearly improve the feasibility of the method over the catenary. 

%Thus, we include in the first stage, a decision problem to set the initial tether length, to quickly obtain a collision-free state for the whole system. Additionally, we replace the traditional catenary equation with a parabolic approximation to estimate the tether shape more efficiently. In the second stage of nonlinear optimization stage, we further simplify the process by parameterizing the tether instead of relying on the catenary model. This approach not only streamlines the representation of the curve but also facilitates more straightforward and efficient gradient calculations during optimization.

The paper is structured as follows. In Section \ref{sec:overview}, we show the general problem to be solved, whereas Section \ref{sec:approach} formalizes the solutions proposed. Section \ref{sec:path_planning} details the implementation of the solution within the planning stage. In Section \ref{sec:optimization_process}, we describe how curve parameterization is utilized to enhance the optimization process for trajectory computation. The experimental results are discussed in Section \ref{sec:experiments}. Finally, the paper is concluded in Section \ref{sec:conclusions}.


\section{Related Work}

% \iffalse
% \subsection{Dynamic Neural Networks} 
% Talk about neural networks that modify their characteristics based on an external input or characteristics of the input. Have a paragraph devoted to early-exit neural networks

% Early exit drawbacks: Early exit looks at if we can exit at one layer, we generate the entire set of pass or not at the beginning, so we know which layers are active. This can result in more efficient data batching, batch data that has same layer allocation and can be executed similarly. Can't do this with early exit.

% I can also propose dynamic batching, where we can batch data that has same layer allocation and can be executed similarly. Could also be a contribution

% \subsection{Reinforcement Learning Techniques}
% How has RL been used to solve this problem in the unimodal setting?


% \fi

\textbf{Early Exiting in Unimodal Networks.}
Early Exiting has been explored extensively in unimodal networks to improve inference efficiency~\cite{xin2020deebert, neurips2020_d4dd111a, meng2022adavit}. Methods like DeeBERT~\cite{xin2020deebert} and PABEE~\cite{neurips2020_d4dd111a} use confidence thresholds to halt computation for simpler inputs. However, these unimodal approaches fail to consider multimodal challenges such as QoI-aware resource allocation. Moreover, Early Exiting is incompatible with redistributing resources away from low-QoI modalities, as these low-confidence samples propagate through the entire network.



\begin{figure*}
    \centering
    \includegraphics[width=1\linewidth]{Figures/ADMN_Architecture.png}
    \vspace{-0.2in}
    % \caption{\name architecture. Gray boxes indicate a dropped layer, blue indicates a frozen layer, and red indicates a tunable layer. TE: transformer encoder.}
\caption{\name architecture. \textcolor{gray}{[Gray box]}: dropped layer, \textcolor{blue}{[Blue box]}: frozen layer, \textcolor{red}{[Red box]}: tunable layer. TE: Transformer Encoder.}
    \label{fig:admn_architecture}
    \vspace{-0.1in}
\end{figure*}

\textbf{Dynamic Inference for Multimodal Systems.}
Multimodal networks traditionally rely on input-agnostic static provisioning, resulting in inefficiencies when modality QoI varies, and also incompatibility with dynamic computational resource availability. 
To enhance computational efficiency, dynamic networks have been proposed~\cite{xue2023dynamic, panda2021adamml, gao2020listen,cai2024ACF, mullapudi2018hydranets}. 
DynMM~\cite{xue2023dynamic} trains a set of expert networks representing different modality combinations, processing simple inputs with a subset of available modalities. 
AdaMML~\cite{panda2021adamml} and Listen to Look~\cite{gao2020listen} improve inference efficiency by leveraging multimodal information to eliminate temporal redundancy in videos. ACF~\cite{cai2024ACF} dynamically replaces certain modules with lightweight networks according to the input for greater efficiency in edge devices. 
Despite these advances, these existing methods (1) overlook significant QoI variations arising from input noise, (2) utilize or discard entire modalities without fine-grained control, and (3) fail to consider \emph{fixed} resource budgets. In contrast, \name allocates a fixed number of layers among modalities in a fine-grained manner according to input QoI, which accounts for both relative modality importance and noise.  



% Channel exchanging~\cite{wang2020deep} focuses on inter-modality interactions to enhance feature integration, emphasizing representation quality over computational savings. 
% E2E-VLP~\cite{xu2021e2e} improves robustness in noisy conditions via vision-language pretraining, indirectly boosting efficiency. 
% HydraNets~\cite{mullapudi2018hydranets} generalize dynamic architectures by adapting computation paths based on input complexity, offering foundational insights for multimodal systems. 

% Despite these advances, existing methods underutilize complementary and redundant information across modalities and fail to optimize backbone depth and layer allocation, limiting their applicability under stringent efficiency and robustness requirements. 
% In contrast, our work introduces a novel fidelity-aware mechanism that dynamically allocates computational resources to modality backbones, explicitly optimizing layer depth and resource distribution at the backbone level. Moreover, Early-Exit techniques reduce computation in the \emph{average case} and are not capable of meeting fixed compute budgets. 




% \textbf{Adaptive Computation in Vision Transformers.}
% Adaptive computation has been explored in Transformer architectures~\cite{wu2018blockdrop, meng2022adavit}. 
% BlockDrop utilizes reinforcement learning to dynamically select residual blocks, while AdaViT~\cite{meng2022adavit} employs lightweight decision networks to adaptively choose patches, attention heads, and layers within Vision Transformers. 
% Although these methods achieve high efficiency, they are designed for unimodal tasks and fail to address the unique challenges of multimodal networks, such as cross-modal interactions and dynamic resource allocation. 
% Our work builds upon these ideas, introducing adaptive mechanisms tailored for multimodal systems that optimize computation across both modalities and backbone layers, enabling efficient and robust cross-modal inference.





\section{Formative Study}\label{sec:formative}
We conducted a formative study to understand the workflow of paper-cutting design and users' challenges during the design.
% Formative study: 

%.相反,我们最初感兴趣的是了解采用复杂动画的数据驱动故事是如何创作的,以及创作者的工作流程中存在的挑战。为了实现这一目标,我们与专家进行了一项基于访谈的研究,这些专家在使用可视化和动画讲故事方面拥有丰富的经验。由于动画数据故事通常是由一组创建者创建的,因此我们招募了不同角色的专家来全面了解此类工作流。通过这些访谈,我们发现,在受访者创建的各种类型的数据故事中,由AUV组成的数据故事被认为既引人注目又具有挑战性,因为它们加剧了复杂动画创作过程中的许多痛点。

\subsection{Participants}
As a work targeted to public users, we initially sought to examine the paper-cutting design process and the challenges encountered without GenAI. Subsequently, we aimed to assess the role and limitations of integrating GenAI into paper-cutting design. To achieve this, we recruited participants with varying levels of expertise in both paper-cutting and GenAI, as experts are expected to have insights into design workflows, and each participant group faces different challenges in paper-cutting design. \rrtext{Participants were categorized into four levels of paper-cutting expertise, with further clarification regarding the criteria for paper-cutting expertise provided in~\autoref{expert clarification}:}
(1) \textbf{Masters}: who have over 20 years of professional experience in paper-cutting creation are officially recognized as ICH inheritors; (2) \textbf{Practitioners}: who have 10-20 years of experience in paper-cutting-related work; \revisedtext{(3) \textbf{Amateurs}: who have 1-3 years of experience in creating paper-cuttings without systematic training;} (4) \textbf{Novices}: who never engaged in paper-cutting design or creation. Additionally, we defined three levels of GenAI expertise: (1) \textbf{Professionals}: researchers in the multi-modal machine learning field; 
% (2) \textbf{Knowledgeable Users}: who have previously used GenAI; (3) \textbf{Novices}: who are only vaguely familiar with or unfamiliar with GenAI.
\revisedtext{(2) \textbf{Knowledgeable Users}: who regularly integrate GenAI into their professional, educational, or personal activities and have experience with basic prompt engineering. (3) \textbf{Novices}: who have minimal exposure to GenAI, may have heard of it without engaging in its use, or are entirely unacquainted with it.}

\revisedtext{Seven participants (3 females and 4 males; age M=34.43, SD=12.79) were recruited for the semi-structured interviews through online postings on various social media platforms (e.g., Douyin\footnote{\url{https://www.douyin.com/}} and Bilibili\footnote{\url{https://www.bilibili.com/}}), and the detailed information of the participants is shown in \autoref{table:formative participants}.} Among them were three paper-cutting masters (P1, P4, P5), two practitioners (P2, P3), one amateur (P7), and one novice (P6). Regarding GenAI expertise, the group included one GenAI professional (P6), two knowledgeable users (P2, P7), and four novices (P1, P3, P4, P5). Each participant is compensated with 100 CNY (approximately 14 USD).
% \begin{itemize}
%     \item[1] Masters: who have over 30 years of professional experience in paper-cutting creation are officially recognized as ICH inheritors.
%     \item[2] Practitioners: who have 10-20 years of experience in paper-cutting-related work.
%     \item[3] Amateurs: who have 1-3 years of experience in creating paper-cuttings.
%     \item[4] Novices: who never engaged in paper-cutting design or creation.
% \end{itemize}  
% Additionally, we defined three levels of GenAI expertise: 
% \begin{itemize}
%     \item[1] Professionals: multi-modal machine learning researchers.
%     \item[2] Knowledgeable Users: who have previously used GenAI;
%     \item[3] Novices: who are only vaguely familiar with or unfamiliar with GenAI.
% \end{itemize} 
% 我们面向的是所有想要通过GenAI辅助进行剪纸创作的用户,无论是专家还是新手。因此我们招募了3 professional inheritors 2个市级 (40年, 20年),一个省级(40年);2 paper-cutting practitioners, 都有超过10年的的从业经验;1 novices,1个纯新人和; 1爱好者 自己剪纸剪纸3年的人。

\subsection{Procedure}
The semi-structured interview included two parts: (1) individual design and (2) GenAI-aided design.
% In the beginning, we provided a 10-minute background introduction for novices in the fields of paper-cutting or GenAI.
\revisedtext{At the beginning, we provided separate 10-minute background introductions for novices in the fields of paper-cutting and GenAI, amounting to a total of 20 minutes.}
In the initial part of the study, each participant was tasked with providing a design concept description for paper-cuttings and giving a sketch of paper-cutting, including the areas to cut out, selecting randomly from two main themes:  ''\textit{Dragon Boat Festival}'' and ''\textit{Wedding,}'' for 30 minutes. We then interviewed them to explore participants' personal understanding and perspectives on the paper-cutting design process and examined the key aspects of the design process, including steps, core factors, and challenges faced.
Then, based on the aforementioned themes, participants were asked to engage with a Large Language Model (LLM) to assist in the design process and use a Text-to-Image model to generate paper-cutting image in 10 minutes. After that, we collected feedback from participants on GenAI-aided paper-cutting design, including the shortcomings of the results, challenges in the design process, and their expectations and suggestions for the GenAI-aided system.

% 1. 以“龙”, “平安富贵”, “多子多福”,5min to think the requirement and primary content in this paper-cutting can match these given topics; 2. 10 min to interactive with Stable Diffusion to generate the paper-cuttings. They were asked to describe how their paper-cutting's primary object/content matched the given topics. After that, we first conducted a semi-structured interview to 
% 我们的访谈分为两部分,第一部分,我们首先基于两个主题:“迎接春天”和"婚礼"来询问它们设计思路。然后我们向受访者询问了每人对于剪纸设计过程的理解和看法,最后向每个人询问了有关设计过程一般是怎样组成的,有哪些关键要素需要考虑,有哪些步骤是具有挑战的。 第二部分,首先我们继续基于同样的两个主题来让用户使用LLM辅助进行设计,并结合输出的结果来让T2I模型生成最终的剪纸内容。 然后,我们向受访者询问了ai-辅助剪纸作品的设计的不足,整个过程存在的挑战,以及它们对于系统可以如何改进GenAI,来改善设计的期待和关心

% We summarized our findings regarding the core factors of workflow and challenges in paper-cutting creation with GenAI assistance.

% \begin{figure*}[!htbp]
% \centering
% \includegraphics[width=0.95\textwidth]{Images/workflow.pdf}
% \caption{\label{figure1}
% A general workflow for GenAI-aided paper-cutting design is outlined from the 2-step formative study, with the main challenges in the workflow labeled on corresponding stages. Based on the workflow, and challenges, the design goals are solidified to the pipeline and interface of HarmonyCut}
% \Description{This Figure shows a general workflow for GenAI-aided paper-cutting design is outlined from the 2-step formative study, with the main challenges in the workflow labeled on corresponding stages. Based on the workflow, and challenges, the design goals are solidified to develop HarmonyCut}
% \end{figure*}


\subsection{Findings}

Through the semi-structured interviews and literature review, we identified the workflow of paper-cutting design: ideation and composition. We found that ideation in paper-cutting design presents challenges to all participants, albeit to varying degrees. It is tedious to both participants and GenAI in the composition phase. In GenAI-aided designs, the recommended and generated content is often undesirable, leading to an uncontrollable overall process that cannot modify the output.

\subsubsection{Workflow of Paper-cutting Design}
% 根据专家的意见,剪纸设计主要分为两部,第一是将需求转变为一个 coceptual idea,这个idea 一般需要从几个方面来来考虑(在section 4会提及);第二个是将conceptual idea 落实为视觉的内容(即用剪刀进行剪裁前的设计图)
\revisedtext{Drawing on feedback from interviews, the suggested process of paper-cutting creation, especially in paper-cutting education~\cite{Lin:1974:howtopapercutting, Li:1998:monopapercutting, Li:2011:PatternandDesign, Zhang:1982:discusspapaercut}, and the design steps from Hubka et al.~\cite{Hubka:1992:engineeringdesign}, paper-cutting design primarily involves two main steps, as shown in \autoref{figure1}~(A).} The first step is transforming intents into a conceptual idea~\cite{Choi:2024:creativeconnect}.
% , which generally needs to be considered from several aspects (as will be discussed in \autoref{sec:content}). 
The second step is translating the conceptual idea into visual form (a design blueprint before using scissors for cutting).
\begin{itemize}
\item \textbf{Ideation.} The first step involves transforming intents with a theme into a conceptual idea~\cite{Li:1998:monopapercutting}. 
\rrtext{Based on the experts' feedback, several preliminary dimensions were mentioned, including \textbf{function and style}. These dimensions suggest that ideation should be approached from multiple dimensions (4 factors as detailed in \autoref{sec:4_2}) to determine the core components of the design.} This idea will later be translated into a visual design in the next step.
\item \textbf{Composition.} During the composition step, designers (1) select the shapes of the elements based on the idea, (2) arrange and combine the selected contours, and (3) decide on cut-out regions (unit patterns) for future creation~\cite{Lin:1974:howtopapercutting, Li:2011:PatternandDesign}.
\end{itemize}

\subsubsection{Challenges in Paper-cutting Design}
\revisedtext{Based on the observation in the formative study and literature review related to design and paper-cutting, we refined and summarized 5 challenges in paper-cutting design with GenAI assistance.}
\begin{itemize}
\item[\textbf{C1.}] \textbf{Challenges in Translating Intent to Ideas.}
% Novices and amateurs (P6 and P7) spent a considerable amount of time on the first part of the study, and each could only provide 2 vague descriptions of their design concept for each theme. This difficulty stems from their lack of fundamental knowledge in paper-cutting, including an understanding of the design workflow and the aspects that should be considered to fulfill the required theme. 
% \revisedtext{What's more, the observation of struggling to translate intent to idea aligns with Li's discussion on the cognitive approach in paper-cutting: they lack mapping thinking—a process that enables them to creatively link learned knowledge to the attributes of natural objects, assign relevant concepts, and employ structural methods to express their idea of paper-cutting designs~\cite{Li:1998:monopapercutting}. }
Novices and amateurs (P6 and P7) spent significant time on the first part of the study but could only provide two vague descriptions of their design concepts for each theme.
\revisedtext{This difficulty arises not only from their limited knowledge of paper-cutting, including familiarity with the design workflow and the essential elements needed to address the theme but also from the absence of a cognitive approach (i.e., ``mapping thinking'') described by Li~\cite{Li:1998:monopapercutting}. This approach enables them to creatively link their knowledge to the attributes of natural objects, assign meaningful concepts, and employ structural methods to effectively express their paper-cutting design ideas. All these competencies are essential to learn and apply for paper-cutting creation~\cite{Lin:1974:howtopapercutting, Zhang:2021:papercuttingteaching}.}
As noted by P7, ``\textit{It is easy to cut a paper-cutting based on a sketched outline, but besides the content, I am unsure of which dimensions need consideration in the design process.}'' \revisedtext{Besides, P6 stated, ``\textit{I don't know what content in paper-cutting can appropriately map to those creation intents.}''} Consequently, it is challenging to establish a clear direction for their ideas.

\item[\textbf{C2.}] \textbf{Lack of Creativity and Multiple Variations in Ideation.}
% Novices and amateurs face difficulty selecting multiple elements that align with their themes due to a limited understanding of paper-cutting subjects and their associated meanings and connotations.
% \revisedtext{Novices and amateurs (P6 and P7) struggled to select multiple elements that aligned with their themes in the first part of the study, primarily due to a limited understanding of paper-cutting subjects and their associated meanings and connotations. }
% \revisedtext{The cases in studies of Bloom~\cite{Bloom:1985:knowledgecreative}, Ericsson et al.~\cite{Ericsson:1994:knowledgecreative}, and Gardner~\cite{Gardner:1993:knowledgecreative} clearly indicate that long-term immersion in a discipline is a crucial prerequisite for creative ability, and knowledge serves as an indispensable cornerstone for innovative ideas~\cite{Weisberg:1999:creativityandknowledge}. Based on them, we infer that in the field of paper-cutting design, novices and amateurs who lack systematic knowledge of paper-cutting will face challenges in creative design. The observation in our study further validated this inference: P6 and P7 with limited knowledge of paper-cutting found it particularly difficult to select elements aligned with the theme, and their proposed design ideas tended to be repetitive.}
\revisedtext{Novices and amateurs (P6 and P7), who possessed limited knowledge of paper-cutting, struggled to select elements aligned with their themes, and their proposed ideas appeared repetitive. These findings are consistent with prior research~\cite{Ericsson:1994:knowledgecreative, Gardner:1993:knowledgecreative}, which highlight that long-term immersion in a discipline is essential for developing creative ability. Furthermore, knowledge is identified as a critical foundation for fostering innovative ideas~\cite{Weisberg:1999:creativityandknowledge}, supporting the finding that a lack of systematic knowledge hinders the creative potential of novices and amateurs in paper-cutting design.}
Conversely, experts and practitioners, although proficient in the creative process and capable of rapid ideation, tend to rely heavily on their accumulated experience and local cultural influences for themes and content. This dependence can lead to fixation on a single idea. As P4 said, ``\textit{Paper-cutting is highly regional, with the meaning of specific elements differing significantly even within the same province. Although there is diversity in paper-cutting, I am only familiar with the themes and elements from my region, resulting in more fixed forms and content across many themes.}''

\item[\textbf{C3.}] \textbf{Challenges in Converting Ideas into Concrete Visual Representations.}
Novices have limited drawing skills, making translating content in their ideas directly into visual forms challenging. Additionally, composition requires considering not only the spatial arrangement and structure of elements but also deciding which areas should be cut-out (pattern) and in what shapes during creation. It is labor-intensive for both novices and experts. As mentioned by P1, ``\textit{The arrangement of specific content and the shapes created through cut-outs (pattern) best reflect personal style. However, translating an idea into a visual expression is still laborious.}''

\item[\textbf{C4.}] \textbf{Challenges in Exploring Suitable and Rational GenAI Results.} 
For the given themes, the model struggles to grasp the user's unique design idea, often providing overly broad suggestions, which cannot assist the user in avoiding fixation even increase the load to user in exploration. Additionally, the finally generated paper-cutting images often do not match the text description, especially in the spatial arrangement of content. Moreover, many parts of the model-generated paper-cuttings are irrational or irrelevant, such as generating some random clusters of stripes as patterns in paper-cutting.

\item[\textbf{C5.}] \textbf{Challenges in Controlling and Editing GenAI Results.} 
Regarding the above issue with GenAI, it is difficult to directly adjust errors in the generated results. Participants can only try to improve the output by revising the input descriptions. However, because the model struggles to understand the knowledge of contents with nuanced meaning and composition, iterative changes to the input often yield minimal improvement.

\end{itemize}

% 用户难以获得剪纸相关的知识,从而进行符合需求的构思和创作:
% 首先,对于新手,它们缺乏对剪纸的基本知识,如创作流程和创作要素。 用户对创作需求只有模糊的想法,不知道该从哪些方面来考虑内容选取,从而满足创作需求。
% 而对于专家,虽然他们可以快速的明确创作流程和创作思路,但对于具体该选用哪些主题,内容和纹样来进行符合创作需求的作品,是具有挑战的。即使是专家,对于一些常见的主题,他们的创作内容也会陷入到固步自封(fixation)。
% 在GenAI辅助创作上,符合用户需求的剪纸的内容和纹样是多样的,而同一内容在不同情境下,或多种纹样的不同搭配方式都具有不同的意义。这些都是模型难以理解的。对于某种事物的剪纸内容生成往往限于固定的几种模式; 同时,在模型生成的剪纸内容中,有很多纹样是无意义、不合理,甚至只是一团胡乱的条纹。系统没有办法理解作为剪纸最核心的符号语言,纹样。
% 而如果让模型直接通过深度生成模型从文本输入到生成整个图像,结果很大程度上由人工智能主导,用户缺乏参与感,更无法对模型产生的问题直接做出修改与调整。正能迭代的重新生成

% 形成性研究发现,剪纸构思与创作过程存在显著挑战,主要体现在以下两个方面:
% 一、用户难以获取知识限制构思
% 新手用户普遍缺乏剪纸艺术的基本知识,包括创作流程和关键要素,导致难以根据创作需求进行有效构思。
% 专家用户虽能迅速明确创作流程与思路,但在选择适合创作需求的主题、内容及纹样时易陷入它们自己的创作定式(fixation)。
% 二、GenAI辅助创作中难以对结果修改,难以理解深层的领域知识
% 在利用GenAI进行剪纸创作时,模型难以全面理解剪纸艺术中与文化相关的复杂知识,如内容在不同情境下的意义变化及纹样搭配的深层含义。它们都导致GenAI生成的剪纸内容常受限于固定模式,结果中的纹样可能缺乏意义与合理性、甚至只是胡乱的条纹,未能充分体现纹样作为剪纸中符号语言的精髓。
% 当模型直接从文本输入生成剪纸图像时,用户参与度低,难以对生成结果进行即时修改与调整,限制了创作的灵活性与个性化。



\begin{table*}[!htbp]
\caption{Definition and examples of factors and types in paper-cutting ideation derived from content analysis.}
  \Description{This table demonstrates the definition and examples of factors and types in paper-cutting ideation derived from content analysis.}
  \label{table1}
\Large
\renewcommand\arraystretch{1.7}
\resizebox{\textwidth}{!}{
\begin{tabular}{l|l|l|p{8.cm}|l}
\hline
\multicolumn{1}{c|}{\textbf{Category}}                                          & \multicolumn{1}{c|}{\textbf{Subcategories}} & \multicolumn{1}{c|}{\textbf{Type}} & \multicolumn{1}{c|}{\textbf{Definition}}                                          & \multicolumn{1}{c}{\textbf{Examples (Figures)}} \\ \hline
\multirow{7}{*}{Function}                                                       & \multirow{4}{*}{Spiritual Function}         & Witchcraft Belief                     & Serve as a symbol in witchcraft activities, embodying related beliefs and rituals    &  Wizard Exorcising Demons~(\autoref{a1fig1}(a))  \\ \cline{3-5} 
                                                                                &                                             & Indigenous Belief                  & Reflect unique local belief systems as a form of cultural expression              & Herding Ducks in Watertown~(\autoref{a1fig1}(b))  \\ \cline{3-5} 
                                                                                &                                             & Religious Belief                    & Act as a symbol in religious ceremonies or doctrines, conveying religious content & Guanyin Sitting on a Lotus~(\autoref{a1fig1}(c))  \\ \cline{3-5} 
                                                                                &                                             & Cultural Dissemination                & Serve as an medium to disseminate culture and historical information              & Happy Asian Games~(\autoref{a1fig1}(d))  \\ \cline{2-5} 
                                                                                & \multirow{3}{*}{Practical Function}         & Interpersonal Communication        & Act as a medium in social etiquette settings in interpersonal communication       & Mandarin Ducks in Water~(\autoref{a1fig1}(e))  \\ \cline{3-5} 
                                                                                &                                             & Festive Atmosphere Evoking        & Enhance the atmosphere and cultural features in holiday or seasonal celebrations  & Solar Term: Grain Full~(\autoref{a1fig1}(f))  \\ \cline{3-5} 
                                                                                &                                             & Daily Decoration                   & Serve as decorative items in daily life                                           & Butterfly Window Decoration~(\autoref{a1fig1}(g))  \\ \hline
\multirow{6}{*}{Subject Matter}                                                        & \multirow{5}{*}{Traditional Subject Matter}        & Primitive Paper-cutting            & Present paper-cutting by initial form in history                               & Circular Floral Paper-cutting~(\autoref{a1fig1}(h))  \\ \cline{3-5} 
                                                                                &                                             & Flora and Fauna                    & Present paper-cutting by animals and plants                                       & The World Welcomes Spring~(\autoref{a1fig1}(i))  \\ \cline{3-5} 
                                                                                &                                             & Landscape                         & Present paper-cutting about natural and cultural landscapes                       & Lijiang Ancient Town~(\autoref{a1fig1}(j))  \\ \cline{3-5} 
                                                                                &                                             & Historical Figure and Story                    & Present paper-cutting centered around stories of characters                       & Jiang Ziya Fishing~(\autoref{a1fig2}(a))  \\ \cline{3-5} 
                                                                                &                                             & Folk Life                & Present paper-cutting about traditional customs and culture                      & The Mouse's Wedding~(\autoref{a1fig2}(b))  \\ \cline{2-5} 
                                                                                & Innovative Subject Matter                         & Contemporary Subject               & Paper-cutting integrated with modern subjects                                     & Genshin Impact, Klee~(\autoref{a1fig2}(c))  \\ \hline
\multirow{2}{*}{Style}                                                          & Abstract Style                              & -                                  & Express ideas with non-representational forms by paper-cutting                   & Frog,~\autoref{a1fig2}(d)  \\ \cline{2-5} 
                                                                                & Realistic Style                             & -                                  & Replicate real objects and scenes by paper-cutting                                & Lujiazui, Shanghai~(\autoref{a1fig2}(e))  \\ \hline
\multirow{3}{*}{\begin{tabular}[c]{@{}l@{}}Method of\\ Expression\end{tabular}} & Metaphor                                    & -                                  & Use similar or related content to indirectly express meaning or emotion         & Harmonious of Ethnicities~(\autoref{a1fig2}(f))  \\ \cline{2-5} 
                                                                                & Symbolism                                   & -                                  & Use specific content to represent theme                                            & Enduring Lineage~(\autoref{a1fig2}(g))  \\ \cline{2-5} 
                                                                                & Homophony                                   & -                                  & Use similarity of pronunciation to embed positive wishes into specific objects    & Happiness Arrives~(\autoref{a1fig2}(h))  \\ \hline
\end{tabular}
}
\end{table*}


\section{Content Analysis of Paper-cuttings and Patterns}\label{sec:content}
Based on the formative study, we discovered that there are some key aspects to consider in the design workflow. To identify them, we first collected a paper-cutting dataset. We conducted two content analyses: one to develop an ideation factors taxonomy and another for a pattern taxonomy (taxonomy for the key element in ideation and composition). \rrtext{Both analyses were carried out under the guidance of expert review, involving five experts (P1–P5) who had also participated in the formative study. The experts were involved in two stages: (1) During the coding process, ambiguities in the instance-level annotation of paper-cuttings and patterns were resolved through a single expert-guided discussion conducted via the WeChat group, with the final annotation determined based on the majority vote of all experts; (2) Additionally, after each version of the codebook was drafted by the authors, the experts participated in discussions conducted via online conferences. In each round, the experts collaboratively reviewed and evaluated the current version of the codebook, offering refined or expanded suggestions, which were systematically discussed and consolidated to resolve ambiguities and ensure alignment of perspectives. This iterative process continued until all experts reached a consensus on the coding results, ensuring that the final codebook was validated.} 

% \begin{table*}[!htbp]
% \caption{Definition and examples of pattern categories and subcategories derived from content analysis.}
%   \Description{This table demonstrates the definition and examples of pattern categories and subcategories derived from content analysis.}
%   \label{table2}
% \resizebox{\textwidth}{!}{
% \begin{tabular}{c|c|p{8.1cm}|cc}
% \hline
% \textbf{Category}                            & \textbf{Subcategory}     & \makecell*[c]{\textbf{Definition}} & \multicolumn{2}{c}{\textbf{Examples}} \\ \hline
% \multirow{3}{*}{Unit Pattern~\cite{Hu:2021:Traditionalpattern, Zhuge:1998:patterndictionary}}       & Geometric Unit Pattern       & \makecell*[c]{Independent units based on abstract geometric\\ glyphs that can be integrated with other patterns \\in paper-cuttings~\cite{Tian:2003:Historypattern, Liu:2009:rai, Liu:2020:intcut, Li:2020:aug}}     &               \includegraphics[totalheight=1.5cm, keepaspectratio,valign=m,margin=0cm .05cm]{Images/pattern/circle.pdf} &  \includegraphics[totalheight=1.5cm, keepaspectratio,valign=m,margin=0cm .05cm]{Images/pattern/triangle.pdf}           \\ \cline{2-5} 
%                                     & Semantic Unit Pattern          & \makecell*[c]{Independent units based on specific semantic glyphs \\ that can be integrated with other patterns in paper-cuttings}     &             \includegraphics[totalheight=1.5cm, keepaspectratio,valign=m,margin=0cm .05cm]{Images/pattern/crescent.pdf} &  \includegraphics[totalheight=1.5cm, keepaspectratio,valign=m,margin=0cm .05cm]{Images/pattern/cloud.pdf}           \\ \cline{2-5} 
%                                     & Sawtooth Pattern   & \makecell*[c]{Specific independent units resembling sawtooth, \\formed by replicating various glyphs along\\ a trajectory for depicting contour and light gradient \\in paper-cuttings~\cite{Zhang:2018:sos, Liu:2020:intcut, Zhang:2006:cpc, Liu:2009:rai}}       &            \includegraphics[totalheight=1.5cm, keepaspectratio,valign=m,margin=0cm .05cm]{Images/pattern/sawtooth1.pdf} &  \includegraphics[totalheight=1.5cm, keepaspectratio,valign=m,margin=0cm .05cm]{Images/pattern/sawtooth2.pdf}             \\ \hline
% \multirow{2}{*}{Composite Pattern} & Primary Composite Pattern        & \makecell*[c]{Composite patterns made up of multiple unit patterns \\ that serve as the primary objects for \\ expressing themes in paper-cuttings}        &              \includegraphics[totalheight=1.5cm, keepaspectratio,valign=m,margin=0cm .05cm]{Images/pattern/dragon.pdf} &  \includegraphics[totalheight=1.5cm, keepaspectratio,valign=m,margin=0cm .05cm]{Images/pattern/horse.pdf}             \\ \cline{2-5} 
%                                     & Decorative Composite Pattern &  \makecell*[c]{Composite patterns made up of multiple \\unit patterns that decorate primary objects \\in paper-cuttings~\cite{Zhuge:1998:patterndictionary, Hu:2021:Traditionalpattern, Zhang:2005:cag}}       &             \includegraphics[totalheight=1.5cm, keepaspectratio,valign=m,margin=0cm .05cm]{Images/pattern/HUI.pdf} &  \includegraphics[totalheight=1.5cm, keepaspectratio,valign=m,margin=0cm .05cm]{Images/pattern/endlessknot.pdf}            \\ \hline
% \end{tabular}
% }
% \end{table*}

\begin{table*}[!htbp]
\caption{Definition and examples of pattern categories and subcategories derived from content analysis.}
  \Description{This table demonstrates the definition and examples of pattern categories and subcategories derived from content analysis.}
  \label{table2}
\resizebox{\textwidth}{!}{
\begin{tabular}{c|c|p{8.1cm}|cc}
\hline
\textbf{Category}                            & \textbf{Subcategory}     & \makecell*[c]{\textbf{Definition}} & \multicolumn{2}{c}{\textbf{Examples}} \\ \hline
\multirow{3}{*}{Unit Pattern~\cite{Hu:2021:Traditionalpattern, Zhuge:1998:patterndictionary}}       & Geometric Unit Pattern       & \makecell*[c]{Independent units based on abstract geometric\\ glyphs that can be integrated with other patterns \\in paper-cuttings~\cite{Tian:2003:Historypattern, Liu:2009:rai, Liu:2020:intcut, Li:2020:aug}}     &               \raisebox{-.39\dimexpr\totalheight-\ht\strutbox}{\includegraphics[scale=0.69]{Images/pattern/circle.pdf}} &  \raisebox{-.4\dimexpr\totalheight-\ht\strutbox}{\includegraphics[scale=0.69]{Images/pattern/triangle.pdf}}           \\ \cline{2-5} 
                                    & Semantic Unit Pattern          & \makecell*[c]{Independent units based on specific semantic glyphs \\ that can be integrated with other patterns in paper-cuttings}     &             \raisebox{-.4\dimexpr\totalheight-\ht\strutbox}{\includegraphics[scale=0.69]{Images/pattern/crescent.pdf}} &  \raisebox{-.43\dimexpr\totalheight-\ht\strutbox}{\includegraphics[scale=0.69]{Images/pattern/cloud.pdf}}           \\ \cline{2-5} 
                                    & Sawtooth Pattern   & \makecell*[c]{Specific independent units resembling sawtooth, \\formed by replicating various glyphs along\\ a trajectory for depicting contour and light gradient \\in paper-cuttings~\cite{Zhang:2018:sos, Liu:2020:intcut, Zhang:2006:cpc, Liu:2009:rai}}       &            \raisebox{-.51\dimexpr\totalheight-\ht\strutbox}{\includegraphics[scale=0.72]{Images/pattern/sawtooth1.pdf}} &  \raisebox{-.51\dimexpr\totalheight-\ht\strutbox}{\includegraphics[scale=0.72]{Images/pattern/sawtooth2.pdf}}             \\ \hline
\multirow{2}{*}{Composite Pattern} & Primary Composite Pattern        & \makecell*[c]{Composite patterns made up of multiple unit patterns \\ that serve as the primary objects for \\ expressing themes in paper-cuttings}        &              \raisebox{-.5\dimexpr\totalheight-\ht\strutbox}{\includegraphics[scale=0.66]{Images/pattern/dragon.pdf}} &  \raisebox{-.5\dimexpr\totalheight-\ht\strutbox}{\includegraphics[scale=0.66]{Images/pattern/horse.pdf}}             \\ \cline{2-5} 
                                    & Decorative Composite Pattern &  \makecell*[c]{Composite patterns made up of multiple \\unit patterns that decorate primary objects \\in paper-cuttings~\cite{Zhuge:1998:patterndictionary, Hu:2021:Traditionalpattern, Zhang:2005:cag}}       &             \raisebox{-.51\dimexpr\totalheight-\ht\strutbox}{\includegraphics[scale=0.66]{Images/pattern/HUI.pdf}} &  \raisebox{-.43\dimexpr\totalheight-\ht\strutbox}{\includegraphics[scale=0.66]{Images/pattern/endlessknot.pdf}}            \\ \hline
\end{tabular}
}
\end{table*}
\subsection{Data Collection}\label{sec:data}
\revisedtext{We collected over 17,000 paper-cuttings through data crawling from the Chinese Paper Cutting Digital Space\footnote{\url{https://www.papercutspace.cn/}}, which organized and merged paper-cuttings into seven distinct categories based on the human geography regions in China~\cite{Fang:2017:chinahumangeo} (i.e., Central China, East China, North China, Northeast, Northwest, South China, and Southwest).} We then selected 1,521 paper-cuttings with titles from the source to aid in identifying the content, as some abstract works are challenging to interpret and annotate in content analysis. We further filtered out low-definition images and multi-color paper-cuttings, as our focus is on traditional monochromatic Chinese paper-cutting. This process resulted in a collection of 701 paper-cutting images, which serve as the sampled source for content analysis in~\autoref{sec:4_2} and \autoref{sec:4_3}. Additionally, these images function as the retrieval dataset discussed in~\autoref{sec:harmonycut}. 
\revisedtext{For content analysis and fine-tuning models, we sampled 20\% images (140/701) based on the distribution of the seven regions to ensure alignment with the regional characteristics of paper-cutting, which is shown in~\autoref{figure:sample filter}.}

\subsection{Core Factors of Paper-cutting Design Ideation} \label{sec:4_2}
\revisedtext{We first collected all dimensions (i.e., function and style) considered by experts during the ideation phase of the formative study based on their feedback (\autoref{sec:formative}). To identify the ideation factors and the various types of content within each factor, the first author, in collaboration with five experts, conducted a content analysis using 140 selected paper-cuttings~(\autoref{sec:data}).
The first author randomly selected 70 (50\%) paper-cuttings to create taxonomy by open-coding. To ensure the quality of the taxonomy, two criteria were followed during the coding process~\cite{Nickerson:2013:Taxonomymethod}: all elements in the taxonomy should be comprehensive, covering every aspect, and the types within each factor should be mutually exclusive.}
In the beginning, the coder annotated detailed content across all dimensions observed or clarified through experts when content was unclear or ambiguous, resulting in the initial codebook. 
% For instance, 关于萨满教相关的剪纸,是东北的特色剪纸,作者在标注相关剪纸时,无法确定它们的fuction,经过single expert-guided discussion 以及投票,所有专家都认为只针对这一类剪纸,其巫术信仰的功能明显优先于本土信仰。
\rrtext{For instance, Northeast shamanistic paper-cuttings with the regional characteristic, were initially challenging to determine their function. Following the expert-guided discussion and voting, experts agreed their function in witchcraft beliefs takes precedence over indigenous beliefs.
Then, the coder produced the final codebook under expert review through three rounds of discussion and iterations (the detailed information of each round is shown in~\autoref{A:expert discuss ideation}), identifying 18 types of content grouped into 4 factors:} Function, Subject Matter, Style, and Method of Expression. The categorization results, including ideation factors, sub-factors, types, definitions, and example paper-cuttings, are detailed in~\autoref{table1}. The first author then applied the finalized codebook to code the remaining 70 paper-cuttings and validated the taxonomy, as shown in~\autoref{figure:paper-cut validation}.
Examples of each type are shown in~\autoref{A:content examples}. For the 140 paper-cuttings, each is annotated type and has specific explanations for every factor, such as category \textit{Subject Matter}, type \textit{Historical Figure and Story}, detailed information ``\textit{A paper-cutting narrates the historical story of Zhaojun Wang's journey to the Xiongnu for a political marriage.}'' These 140 pieces are used as domain knowledge and ground-truth datasets in~\autoref{sec:harmonycut}.
% For each piece, the first author coded according to the various dimensions proposed by experts, identifying multiple types within each dimension. 
% Through careful comparison and merging of similar types and dimensions, we distilled the ideation process into four core factors. Each merging operation was based on discussion with experts (P1-P5) and feedback from their reviews, ultimately constructing a taxonomy (\autoref{table1}) of the paper-cutting ideation factor, resulting in 18 identified types grouped into 4 factors (Function, Subject Matter, Style, and Method of Expression). 
% In constructing this taxonomy, we analyzed 70 paper-cuttings (50\%) to ensure comprehensive coverage. The first author annotated the remaining 70 paper-cuttings to validate and refine the taxonomy.
% Examples of each type are shown in~\autoref{A:content examples}. For the 140 paper-cuttings, each is annotated type and has specific explanations for every factor, such as category \textit{Subject Matter}, type \textit{Historical Figure and Story}, detail information ``\textit{A paper-cutting narrates the historical story of Zhaojun Wang's journey to the Xiongnu for a political marriage.}'' These 140 pieces are used as domain knowledge and training datasets in \autoref{sec:harmonycut}

% 我们总结了专家在formative study中提到的有关在ideation中需要考虑的几个方面,并从17000+作品,并根据地区分别从7大地区挑选了N个剪纸作品,并进行coding,对每一个作品,都从每一个专家提到的aspect来进行编码,得到对于每一aspect的多个type,最终将类似的type、aspect, merge。得到了构思中的四要素。每次是否merge,都会基于expert review来完成,最终构建taxonomy。

% 系统没有办法理解作为剪纸最核心的符号语言,纹样。
\subsection{Patterns in Paper-cutting}\label{sec:4_3}
% From an appearance perspective, paper-cuttings typically have a single color and express artistry through those two-dimensional glyphs. The outer contour of the paper-cutting can represent the shape of the object being depicted, while the internal two-dimensional glyphs may either specifically portray a certain object or serve a decorative and express meaning function. 
% From an appearance perspective, monochromatic paper-cutting relies on two-dimensional glyphs for artistic expression. The outer contour defines the shape of the depicted object, while the internal glyphs serve semantic or decorative roles by highlighting specific elements. 
\revisedtext{From a semiotic perspective, Chinese paper-cutting consists of a series of symbolized patterns~\cite{Liang:2011:Symbolpattern}. Saussure's theory~\cite{Saussure:1916:Semiotics} defines a symbol as comprising two parts: the ``signifier'' and the ``signified.'' In the context of paper-cutting, the ``signifier'' refers to the form or structure of the pattern, while the ``signified'' represents the meaning or concept it conveys~\cite{Cao:2009:Semiotics, Liang:2011:Symbolpattern}. 
% This also aligns with Husserl's distinction between "object" and "meaning"~\cite{}. 
Therefore, achieving harmony between form and connotation in paper-cutting design requires a comprehensive understanding of patterns that unify meaning and content. To support this, we developed a taxonomy of these patterns using a hybrid thematic analysis approach, also using those 140 selected paper-cuttings~(\autoref{sec:data}).}
% Thus, to identify the combination of these basic elements (pattern) of paper-cutting and create a taxonomy of patterns, we employed a hybrid thematic analysis approach, using the taxonomy as a codebook to analyze and annotate 140 sample paper-cuttings from the pattern aspect. Based on the literature review, we identified a unique sawtooth pattern~\cite{Zhang:2018:sos} and compiled a range of traditional patterns~\cite{Zhang:2018:sos, Huang:2021:chinesepattern, Wang:2009:folkpattern, Zhao:2023:papercutculture}. 

\revisedtext{In the deductive coding process, previous research found three important aspects of pattern and paper-cutting namely \textit{Geometric Pattern}~\cite{Tian:2003:Historypattern, Liu:2009:rai, Liu:2020:intcut, Li:2020:aug}, \textit{Decorative Pattern}~\cite{Zhuge:1998:patterndictionary, Hu:2021:Traditionalpattern, Zhang:2005:cag}, and \textit{Sawtooth Pattern}~\cite{Zhang:2018:sos, Liu:2020:intcut, Zhang:2006:cpc, Liu:2009:rai}. \textit{Geometric Pattern} was composed of shapes formed by the ultimate abstraction of points, lines, and planes. \textit{Decorative Pattern} was used to enhance or embellish specific patterns or the overall composition of a paper-cutting. \textit{Sawtooth Pattern} was a distinct and important pattern in paper-cutting, used to convey texture and layering in objects while also serving as decoration for various patterns. However, these three aspects partially overlap in both shape and function. To address this, we adopted the concept of \textit{Unit Pattern}~\cite{Hu:2021:Traditionalpattern, Zhuge:1998:patterndictionary}, representing the fundamental unit in paper-cutting, to ensure the codebook remained mutually exclusive. Thus, \textit{Geometric Pattern} and \textit{Sawtooth Pattern} were identified as sub-categories of \textit{Unit Pattern}, and \textit{Decorative Pattern} was defined as one of \textit{non-Unit Patterns} with decorative functions.}

\revisedtext{In the inductive coding process, OpenCV extracted 63,452 cut-outs (i.e., \textit{Unit Patterns}) from 140 paper-cuttings~(\autoref{sec:data}), and 1,269 cut-outs (2\%) were randomly sampled from them.}
The first author open-coded 635 cut-outs and 70 paper-cuttings, both randomly selected at 50\%, similar to the previous process in~\autoref{sec:4_2}, referring to pattern knowledge~\cite{Huang:2021:chinesepattern, Wang:2009:folkpattern, Zhao:2023:papercutculture, Zhuge:1998:patterndictionary}.
\rrtext{Ambiguities in instance-level annotation were resolved using the method outlined in~\autoref{A:paper-cut coding}. For example, the classification of the plum blossom pattern as either primary or decorative was challenging. After the single expert-guided discussion, all five experts agreed it is primarily decorative and rarely used as a main subject.}
% 当他们在annotation中对单一instance是什么有疑问,将在group中询问5位experts这个instance具体该归属哪个纹样?
% 例如梅花纹,first version中,author将它归为primary pattern,但经过第一轮讨论,5位experts都认为梅花纹更多是作为装饰作用的,而比较少作为主体。
\rrtext{Through two rounds of expert discussions and iterations (the detailed information of each round is shown in~\autoref{A:expert discuss pattern}), a taxonomy of patterns was established~(\autoref{table2}), consisting of two categories:} \textit{Unit Patterns} and \textit{Composite Patterns} (i.e, \textit{non-Unit Patterns}). In addition, two new subcategories were introduced: unit semantic patterns and composite primary patterns. 
\textit{Unit patterns} include 25 different patterns (8 geometric units, 12 semantic units, and 5 sawtooth patterns). \textit{Composite patterns}, which represent the primary content of paper-cuttings, are formed by combining unit patterns and include 42 different patterns (8 decorative composite patterns and 34 primary composite patterns). The first author annotated the remaining 70 paper-cuttings and 634 cut-outs, and validated taxonomy~(\autoref{figure:pattern validation}). 
\rrtext{All the names and examples of the 25 specific \textit{Unit patterns} and 42 specific \textit{Composite patterns} are provided in the supplementary material for detailed reference.}
These labeled patterns serve as both a repository of domain knowledge and ground truth datasets in \autoref{sec:harmonycut}.
% 先根据literature图鉴以及问专家来进行annotate, 标注方式就是把一个paper-cutting中存在的每一个pattern都标注出来, 然后得到语义 unit 和 主体 composite, 同时将 主体与装饰合并为复合。 整个过程有2-round discussion of 5 experts. final codebook 建好后, first author 再将剩余数据按照 codebook进行标注。
% The first author coded 70 (50\%) paper-cuttings to identify patterns, whether documented in the literature or not. An expert review (P1-P5) was conducted on the coding results to assess the rationality and potential merging of categories and to accurately assign each specific pattern to its appropriate category. Ultimately, we summarized the taxonomy into two categories (\autoref{table2}): unit patterns and composite patterns. Unit patterns, which are the basic cut-out components in paper-cutting, include 25 different patterns (8 geometric units, 12 semantic units, and 5 sawtooth patterns), while composite patterns serve as the content of paper-cutting and are formed by combining unit patterns, encompassing 42 different patterns (8 decorative composite patterns and 34 primary composite patterns). The first author also annotated the remaining 70 paper-cuttings to validate and refine the taxonomy. For the 140 paper-cuttings, each is annotated with all patterns in it. These labeled patterns are used as domain knowledge and training datasets in \autoref{sec:harmonycut}

% 作为从符号学的角度来看,剪纸是由一系列符号化纹样组合而成。纹样作为先强调,虽然不管是对于传统纹样,还是特别到剪纸纹样,虽然对很多内容都有定义,但很少有进行highlevel 归类的。literature review 得到 锯齿纹,其他的最高级subcategory是总结出来的。 其他的只是具体的每一纹样是什么可以从literature里得到:Our group factor taxonomy considered both existing literature on ICH and CHS for deductive coding and newly observed factors from ICH short videos for inductive coding.

% Taxonomy 基于目前有关传统纹样的分类,以及对纹样在剪纸中的分类的文献调研。并基于先前选择的N个作品,扩充pattern 分类。 【Our group factor taxonomy considered existing literature on paper-cutting factors for deductive coding and newly observed factors from collected paper-cutting for inductive coding.】

% 最后 这两个 taxonomy的每一项都先会和专家讨论,是否有缺漏或可以合并的。 然后基于最后的分类结果进行编码和标注


\section{Design Goals}
\revisedtext{Based on the findings from the semi-structured interviews and content analysis, we identified three design goals (\autoref{figure1}~(C)) for developing a GenAI-aided system that supports users in designing a paper-cutting where explicit elements align with implicit meanings:}
\begin{itemize}
    \item[\textbf{DG 1.}] \textbf{Factor-oriented Guidance for Ideation.} This goal aims to help users efficiently transform abstract intent into structured descriptions (\textbf{C1}). The system guides ideation by focusing on four key factors, providing detailed content based on the selected factors to suggest related content for creatively shaping the final idea (\textbf{C2}).

    \item[\textbf{DG 2.}] \textbf{Reference-based Exploration for Composition} For each compositional action, the system based on domain knowledge suggests related content as the reference. Users can explore these design references for creative self-selection and arrangement of contours and patterns extracted from reference (\textbf{C3, C4}).

    \item[\textbf{DG 3.}] \textbf{Controllable and Editable Design with Recommendation} The recommendations provided by GenAI are integral to DG 1 and 2. These suggestions can ensure diversity to foster creativity while maintaining rationality and relevance, thereby preventing users from being misled or overwhelmed during exploration (\textbf{C4}). Furthermore, users are able to edit selections of unreasonable or partially reasonable ideas and compositions, allowing them to incorporate desired elements into their designs and complete the entire design process (\textbf{C5}).
    
    % \item[\textbf{DG 1.}] \textbf{Guide Ideation and Composition with Reference.} To help users efficiently translate abstract intent into a structured description, the system guides users' ideation from 4 factors to form it. To the composition of each action composition, the reference can guide them to self-selection and arrangement of contours and patterns. 
    
    % \item[\textbf{DG 2.}] \textbf{Suggest Diverse and Rational Content for Exploration.} All design processes need the suggested content to explore creative and diverse design for ideation and composition, including suggested detailed and related content from 4 ideation factors to form a final idea and suggest related and rational paper-cutting and patterns, which can be referred to as their expression and structure in DG 1.
    
    % \item[\textbf{DG 3.}] \textbf{Ensure a Controllable and Editable Design Process.}  
    % From the selection of factors and recommended content in ideation, to exploration related paper-cuttings and patterns are need
\end{itemize}


% \section{Pipeline}
% 剪纸系统的一些考虑和解释:剪纸作品从外观来看,剪纸色彩较为单一,最常见的 是红色,具有喜庆含义,其重点通过二维图案来展现它的艺术性。剪纸外轮廓二 维形状可以体现需要刻画的物体形象,剪纸内部二维图案有的具体表现某种实物,有的起到装饰点缀、表达寓意的作用

% 我们基于formative study 总结出的design goal以及design space,并参考了 C2Idea等的 系统设计流程, 提出了一个支持 GENAI-aided 的pipeline(有三个阶段),每个阶段都以及 user interaction,来增强设计的多样性、合理性和可控性
% \begin{itemize}
%     \item % 明确design intent
%     \item % 探索符合需求的内容(索引来/和生成的)recommend content for exploration and selection
%     \item % 组合并调整所选内容 (轮廓,位置,纹样) reference-based creation
% \end{itemize}


% 对于我们采用检索和生成的内容作为参考的一个pipeline 设计依据:有一个和reference retrieval and exploration有关的的好例子:关于创作过程,如肖等人。[54]已确定,主流作品遵循两个阶段的管道,包括检索示例并将其改编为设计材料[62]和风格转移参考[45]。


\begin{figure*}[!htbp]
\centering
\includegraphics[width=0.98\textwidth]{Images/idea_comp.pdf}
\caption{\label{figure2}
The pipeline consists of Ideation and Composition components, structured around the summarized workflow and design space. The Ideation component offers knowledge-based guidance, allowing designers to explore and select content that aligns with their intent to form ideas. These ideas are then fed into multi-modal models within the Composition component, which retrieves and generates related content as references. This exploration of references empowers users to arrange reference and plan cut-out areas, facilitating the composition of their own paper-cutting designs.}
\Description{This figure shows the pipeline consists of Ideation and Composition components, structured around the summarized design space and workflow. The Ideation component offers knowledge-based guidance, allowing designers to explore and select content that aligns with their intent to form ideas. These ideas are then fed into multi-modal models within the Composition component, which retrieves and generates related content as references. This exploration of references empowers users to arrange reference and plan cut-out areas, facilitating the composition of their own paper-cutting designs.}
\end{figure*}

\section{HarmonyCut}\label{sec:harmonycut}
Drawing from the design space and the derived design goals, we introduced a two-component pipeline (\autoref{figure2}) to guide the development of HarmonyCut (\autoref{figure3}), a GenAI-aided design prototype system. This system can assist users in designing paper-cuttings by allowing them to explore and edit related references, such as knowledge, existing paper-cutting works, and patterns, ensuring alignment with both visual form and cultural meaning. HarmonyCut consists of two main components in the pipeline, corresponding to stages in the GenAI-aided paper-cutting design workflow: ideation (DG1) using LLMs and composition (DG2) employing Text-to-Image models~\autoref{figure2}. Each component is connected to the domain knowledge base we have summarized and supports editing and iteration, thereby enhancing the controllability of the design process (DG3).

% \wang{NEED TO emphasize that paper-cutting design is different from creation, do not need high-fidelity final output as results?}
% 详细叙述对于系统,是怎样具体实现这几个阶段的

% \subsection{Technical Evaluation}  这部分放到下面三部分写
% 1. Pattern Classifcication
% 2. Requirement - Paper-cutting Retrieval

\begin{figure*}[!htbp]
\centering
\includegraphics[width=\textwidth]{Images/interface.pdf}
\caption{\label{figure3}
The interface of HarmonyCut supports user creative paper-cutting design through several panels with guidance and exploration. (a) In the Idea Prompting panel, the system guides users through idea navigation using four factors: function, subject matter, style, and method of expression. It also provides related patterns with interpretations. Users can select suggested content that aligns with their intent or manually edit ideas. (b) The Interactive Mood Board panel allows users to arrange all references, including contours and patterns, for composition. (c) In the Reference Exploration panel, users explore system-suggested content, both retrieved and generated, to select contours and plan pattern layouts, thereby gaining ideas for their paper-cutting designs.}
\Description{This figure shows an interface of HarmonyCut that supports user creative paper-cutting design through several panels through guidance and exploration. (a) In the Idea Prompting panel, the system guides users through idea navigation using four factors: function, subject matter, style, and method of expression. It also provides related patterns with interpretations. Users can select suggested content that aligns with their intent or manually edit ideas. (b) The Interactive Mood Board panel allows users to arrange all references, including contours and patterns, for composition. (c) In the Reference Exploration panel, users explore system-suggested content, both retrieved and generated, to select contours and plan pattern layouts, thereby gaining ideas for their paper-cutting designs.}
\end{figure*}

\subsection{Ideation Component}
Based on the paper-cutting design space we summarized, which includes taxonomies of ideation factors and the pattern, along with annotation information from 140 works grounded in this design space, we leverage the robust language understanding and reasoning capabilities of GLM-4~\cite{Teamglm:2024:chatglm}, an LLM that supports the construction of specific knowledge bases. By integrating these annotations, 140 paired question-answering prompt templates derived from these annotations, and the design space as domain knowledge for the model's few-shot learning, users can input their preliminary design intent in text and specify the design factor types that meet their needs (\autoref{figure2}). This enables the system to initially provide content recommendations, knowledge, and interpretations derived from both the model and annotations, based on the user's input within the context of our design space. For example, if a user selects the expression method ``\textit{symbolism,}'' HarmonyCut not only suggests relevant objects and patterns but also provides their symbolic meanings, allowing users to verify alignment with their intent. This approach ensures coherence between form and connotation. It not only facilitates the development of users' design ideas but also deepens their understanding of valuable connotative insights. Through a three-step exploration process including factor selection, suggested content, and knowledge exploration, users ultimately form a textual description of ideas.

\subsection{Composition Component}
The composition component aims to guide the user to explore the reference and translate the user's conceptual idea into visual form with a mood board. It encompasses three fundamental stages: related content selection, interested and inspired content arrangement, and cut-out area layout.  All elements in the mood board are presented in SVG format, enabling flexible configuration and editing, including operations such as grouping, flipping, and copy-pasting, which accommodate user preferences and design needs.
\subsubsection{Selection} 
% To retrieve related paper-cuttings as references, we employed the Chinese version of CLIP~\cite{Yang:2023:chineseclip, Radford:2021:clip}, which was initially pre-trained on approximately 2 million text-image pairs. We further fine-tuned this model using our annotated dataset of 4031 text-image pairs, derived from 140 annotated paper-cuttings as detailed in \autoref{sec:content}. Following 30 epochs of training with 80\% of the dataset, and validation on the remaining 20\%, the fine-tuned Chinese CLIP achieved recall@1 of 63.46\%, recall@5 of 87.24\%, and recall@10 of 94.93\%. These metrics demonstrate that the model effectively maintains the relevance between idea descriptions and paper-cutting references.
% To generate images that are more rational and reflective of the Chinese paper-cutting style using DALL-E-3, prompts can be automatically generated from carefully crafted templates \wang{TODO} or edited by the user to include more specific requirements. All paper-cutting works, including both retrieved and generated content, are displayed within the same panel to facilitate user exploration.
To ensure the quality of paper-cutting references, we employed advanced multi-modal models for retrieval and generating, enabling the system to maintain and reflect the intricate style and thematic relevance of Chinese paper-cutting.
To retrieve related paper-cuttings as references, we used the Chinese version of CLIP~\cite{Yang:2023:chineseclip, Radford:2021:clip}, initially pre-trained on approximately two million text-image pairs. We fine-tuned this model using our annotated dataset of 4,031 text-image pairs, derived from 140 annotated paper-cuttings as detailed in \autoref{sec:content}. Following 30 epochs of training with 80\% of the dataset and validation on the remaining 20\%, the fine-tuned Chinese CLIP achieved recall@1 of 63.46\%, recall@5 of 87.24\%, and recall@10 of 94.93\% on the validation set. These metrics demonstrate that the model effectively maintains the relevance between idea descriptions and paper-cutting references. \revisedtext{To prevent cognitive overload and enhance user navigation within the exploration process, HarmonyCut is designed to retrieve the top 20 most relevant paper-cuttings as references.}
To generate images that are more rational and reflective of the Chinese paper-cutting style using DALL-E-3, prompts can be automatically generated from a carefully crafted template based on object requirements or be edited by the user to include more specific requirements. All paper-cutting works, including both retrieved and generated content, are displayed within the same panel as shown in \autoref{figure3} (c) to facilitate user exploration.

\subsubsection{Arrangement} % SAM
Upon exploring and selecting idea-related references, the composition stage necessitates the precise spatial arrangement of paper-cutting contours. Segment Anything Model (SAM)~\cite{Kirillov:2023:sam}, with its visual prompt points, enables the accurate segmentation of these contours, allowing users to extract content effortlessly. HarmonyCut provides the functionality for users to click on objects and backgrounds, using two labels to prompt segmentation according to their preferences, which is illustrated in the \textit{``contour segmentation''} of \autoref{figure3}. Then, all extracted contours, whether used as design material for combination or as design reference to inspire ideas, are converted into SVG format to facilitate subsequent interactions, and they are compiled within HarmonyCut's central mood board, enabling users to explore various potential combinations of the selected content.


\begin{figure*}[!htbp]
\centering
\includegraphics[width=\textwidth]{Images/detailed_system.pdf}
\caption{\label{figure:detail system}
The detailed process of design with each view and result in HarmonyCut. (a) The idea description from the former ideation. (b.1) The reference generation view based on the prompt; (b.2 and b.3) the contour segmentation view for the selected references and their contours; (b.4) cut-outs from original or other paper cuttings used in the design. (c) The final design is displayed on the mood board.}
\Description{This Figure shows the detailed process of design with each view and result in HarmonyCut. (a) The idea description from the former ideation. (b.1) The reference generation view based on the prompt; (b.2 and b.3) the contour segmentation view for the selected references and their contours; (b.4) cut-outs from original or other paper cuttings used in the design. (c) The final design is displayed on the mood board.}
\end{figure*}

\subsubsection{Cut-out Area} % VIT classification and opencv for pattern extraction
As key elements of paper-cutting, these patterns contribute to the unique symbols and styles characteristic of this hollowed-out art form. Each individual varies in the types of patterns used, as well as the density and intricacy in depicting textures, light, and lines. Therefore, the design of pattern layouts is particularly important, as it represents the designer's own unique style.
To effectively plan cut-out areas, it is crucial to extract patterns from paper-cuttings and provide users with insights akin to those offered in the ideation component, where an LLM interprets suggested composite patterns. This requires two primary preprocessing tasks: unit-pattern extraction and classification, which is shown in \autoref{figure2}.
In the preparation of unit patterns for paper-cutting design, we begin by extracting these patterns as masks using OpenCV\footnote{\url{https://github.com/opencv/opencv}}. These masks are then converted into SVG format, which facilitates the subsequent integration of contours and unit patterns. This process is vital for recognizing and employing the basic elements that contribute to composite patterns. Unlike standalone designs, these elements emphasize the content and texture of paper-cutting works. By utilizing SVG-formatted patterns, the system ensures the preservation of the unique symbols and styles inherent to this hollowed-out art form.
To improve pattern recognition and utilization, we fine-tuned the Vision Transformer (ViT)~\cite{Dosovitskiy:2021:vit} with 1,279 annotated unit patterns and labeled all 63,452 unit patterns from 140 paper-cuttings. This resulted in achieving 71\% precision, 60.43\% recall@1, and an F1 score of 63.20\% on the validation dataset, thereby enhancing the model's ability to effectively identify patterns and support the overall design process.
% To effectively plan cut-out areas, it is essential to extract patterns from paper-cuttings and provide users with an understanding similar to the ideation component, where an LLM interprets suggested composite patterns. Consequently, two main preprocessing tasks are necessary: unit-pattern extraction and classification.
% In preparing unit patterns for paper-cutting design, we first extract these patterns as masks using OpenCV. These masks are converted into SVG format, facilitating the subsequent combination of contours and unit patterns. This step is essential for recognizing and utilizing the fundamental elements that contribute to composite patterns, which, unlike standalone designs, highlight the content and texture of paper-cutting works. By employing these SVG-formatted patterns, the system preserves the unique symbols and styles integral to this hollowed-out art form.
% For enhanced pattern recognition and utilization, we fine-tuned the Vision Transformer (ViT)~\cite{Dosovitskiy:2021:vit} using 1279 annotated unit patterns from 140 paper-cuttings and reach the 71\% precision and 60.43\% recall@1, 63.20\% F1 on validation dataset.

% % This enables effective pattern identification, supporting the overall design process. By integrating these elements, HarmonyCut ensures a cohesive and culturally resonant design outcome, facilitating user exploration and creativity within the paper-cutting domain.\

\subsection{User Scenario}
In HarmonyCut, the components aligned with the design workflow and the goals are arranged within a three-panel interface, with the primary mood board centrally positioned, as illustrated in~\autoref{figure3}. \revisedtext{To demonstrate our system, we present a scenario~\cite{Fulton:2000:usagescenario, Kraft:2012:usagescenario} involving a persona, Tom, a paper-cutting enthusiast who is skilled in cutting but less experienced in composition.} Tom uses HarmonyCut to generate ideas and translate them into visual representations for his design task. With the ``Spring Art Exhibition'' approaching at his school, he hopes to leverage HarmonyCut to design a paper-cutting-themed piece called ``Welcoming Spring'' and improve his composition skills.

\subsubsection{Guide User Ideation with Suggestion}
To get some ideas from our system, Tom first input his rough intent of design as ``\textit{paper-cutting with meaning of welcoming spring}'', function as \textit{evoking a festive atmosphere}, subject matter as \textit{flora and fauna}, style as \textit{abstract}, and expression method as \textit{symbolism} to system in~\autoref{figure3} (a.1). HarmonyCut displays objects and patterns related to his intent, including peony, magpie, butterfly, Phoenix, etc. (\autoref{figure3}~(a.2)), as well as the background, structure, common combinations with other patterns, and their corresponding meanings for each recommended pattern. Among the recommended options, Tom still believed that the magpie and peony, two of the most commonly used elements, best suit the design theme. He realized that with this subject, creativity will largely depend on the composition, which is his weak point. Therefore, he prefers to explore others' compositions to guide his design. He then finalized the editable idea description (\autoref{figure3}~(a.3)) and clicks the ``Confirm'' button.

\subsubsection{Exploration on Related Design Reference}
As Tom confirmed his final idea description, the Reference Exploration panel (\autoref{figure3}~(c)) provides two parts of related paper-cutting suggestions, including retrieved and generated paper-cuttings from models.
% 一1: c.1, c.3,  + find composition reference (two symmetric birds and a potted flower), 
% 同时,Tom还在这些reference中看到一幅对称的雏鸡剪纸,这激发了他将雏鸡剪纸与盆花喜鹊剪纸结合的灵感,通过富有新鲜活力的小生命来表达春天的到来,因此这幅雏鸡剪纸也被选中。
% 二1: c.2 - c.3. 设计感很足,但Tom更在前一步选择的是抽象风格,可是模型总是给出的是国画风格或写实风格,即使他尝试调整了输入的prompt,不过他们的构图都很有趣,所以Tom将两个DALL-E-3生成的内容用SAM 扣下来以后放到mood board里做参考。
% 三1:另外, Tom 认为 选中的鸟轮廓、羽毛以及眼睛的的纹样使用太过简单,因此又从其他作品中提取了其他形态的锯齿纹,
In \autoref{figure3}~(c.1) view, Tom first found a reference that inspired him on how to structure his composition (two symmetrical birds and a potted flower, as shown in~\autoref{figure2}~(B)). Thus, he explores other magpies and potted flowers as the composition elements. 
Meanwhile, Tom also came across a symmetrical baby chicken paper-cutting \autoref{figure2}~(B) among the suggestions, which inspired him to combine the baby chicken design with the potted flower and magpie paper-cutting. He envisioned using the fresh, lively image of the baby chicken to symbolize the arrival of spring. As a result, this baby chicken paper-cutting was also selected.
Then, Tom browsed paper-cutting images generated by DALL-E-3~(\autoref{figure3}~(c.2) and \autoref{figure:detail system}~(b.1)). Although these images had a strong traditional style and sense of design, Tom had selected an \textit{abstract} style and \textit{paper-cutting image} in the ideation stage, the model kept generating images in traditional Chinese painting or realistic styles, even after he tried adjusting the input prompts. Nevertheless, he found the compositions of these images quite interesting, particularly the idea that “\textit{peonies are blossoming on the back of the magpie.}”
Additionally, Tom felt that the patterns used in the magpie paper-cutting (\autoref{figure3}~(c.4)) he selected—such as the contours, feathers, and eyes—were too simplistic. As a result, he extracted various sawtooth patterns from other retrieved works (\autoref{figure:detail system}~(b.4)).

\subsubsection{Mood Board with Contour and Cut-out}
% 一2:then,  copy and flip the select magpie and put them on another potted flower, to express the spring coming with a positive and energetic vitality, two 小鸡 in retrieved paper-cutting are selected them.
% 二2:参考了 genAI图里的鸟上背花的灵感,Tom也将4个锯齿纹组成的梅花纹放到右边的喜鹊身上。
% 三2:并进行修改组合后,分别放到两只鸟的轮廓中。

% Tom根据他在前一步探索到的reference,首先用SAM 最为灵感来源的剪纸和要作为构图元素的剪纸轮廓都分割下来放到mood board 的canvas上。然后,按照提供他构图灵感的作品,他将自己选取的喜鹊复制并反转,然后将两只喜鹊对称置于另一个盆花剪纸上。并将之前选中的对称小鸡剪纸的轮廓至于整个设计图的底部。
% 对于剪纸设计的纹样,Tom参考了生成图里鸟上生花的灵感,他从其他在c.1中剪纸作品的将锯齿纹提取出来,选择了4个锯齿纹组成梅花纹样,放到了右边的喜鹊身上。
% 最后,在将所有想添加的纹样布置完成后。就得到了b.1中的final design.

Based on the references Tom explored in the previous step, he first used SAM by adding visual prompts to segment both the paper-cutting that served as his source of inspiration and the paper-cutting contours he wanted to use as compositional elements~(\autoref{figure:detail system}~(b.2-3)). He placed these onto the mood board canvas. Following the composition of the reference artwork that inspired him, he duplicated and flipped the magpie he selected, symmetrically positioning the two magpies over another potted flower paper-cutting. He then placed the contour of the symmetrical baby chicken paper-cutting, previously selected, at the bottom of the entire design.
For the patterns in the paper-cutting design, Tom drew inspiration from the generated images where flowers blossom on birds. He extracted the sawtooth patterns from other paper-cutting works in~\autoref{figure:detail system}~(b.4), selecting four sawtooth patterns to form a plum blossom pattern, which he placed on the right magpie~(\autoref{figure3}~(c.1)).
Finally, after arranging all the patterns he wanted to add, he completed the final design, shown in~\autoref{figure3}~(b.1).

\subsection{Implementation Details}
Our system utilizes a Flask\footnote{\url{https://flask.palletsprojects.com}}-based back-end, integrated with the GLM-4 APIs for text processing and DALL-E-3 for paper-cutting image generation. The front-end is built using Vite + Vue, with additional support from Element Plus for UI components, Pinia for state management, and Tailwind CSS for styling.
A fine-tuned CLIP model for multi-modal retrieval, ViT-base-16 and RoBERTa-wwm-base are used for visual and text encoding.
For image segmentation, we extended the segment-anything-web UI\footnote{\url{https://github.com/Kingfish404/segment-anything-webui}} framework, which enables extraction and manipulation of image components. The backend supports the use of SAM-ViT-base\footnote{\url{https://github.com/facebookresearch/segment-anything}}~\cite{Kirillov:2023:sam} by default, with flexible configurations for alternative models.
Additionally, a fine-tuned pre-trained ViT (ViT-base-16, pre-trained on ImageNet-21k) is employed for unit-pattern classification.
The frontend supports dynamic SVG manipulation and bitmap-to-SVG conversion using Fabric\footnote{\url{https://fabricjs.com/}} and Potrace\footnote{\url{https://potrace.sourceforge.net/}} libraries. We also implemented custom features such as object dragger and pattern editor, integrated with the overall canvas design interface. \rrtext{The related datasets, comprising 140 paper-cutting images, 4,031 text-image pairs for fine-tuning the multi-modal retrieval model and 1,279 annotated images for fine-tuning the unit pattern recognition model, are detailed in the supplementary material.}

\section{Evaluation}\label{evaluation}
We conducted a within-subjects user study with sixteen participants and an expert evaluation involving three Chinese paper-cutting experts. The objective was to assess HarmonyCut's usability in paper-cutting design and to validate the proposed design workflow. Then, we interviewed participants to evaluate whether HarmonyCut enhances creativity and facilitates the design process. Insights from the expert interviews highlighted how the system addresses challenges in paper-cutting design and identified areas for improvement.
\revisedtext{In contrast, the baseline tool utilized the same GenAI models as HarmonyCut along with their respective official web interfaces of GLM-4\footnote{\url{https://chatglm.cn/}} and DALL-E-3\footnote{\url{https://chatgpt.com/}}. However, It lacked the domain-specific knowledge (i.e., annotated paper-cuttings and fine-tuned models) and the direct editing capabilities incorporated into HarmonyCut. Consequently, users of the baseline tool had to manually input text for design ideation with GLM-4 and generate final designs by crafting their own prompts for DALL-E-3.}

\subsection{Participants}
\revisedtext{We recruited 16 participants (6 females and 10 males; age M=23.94, SD=7.08) through online postings on various social media platforms.} To ensure HarmonyCut could meet design goals from the formative study, participants were selected based on similar criteria for expertise in paper-cutting and GenAI. \revisedtext{The group included 1 paper-cutting master (U7), 2 practitioners (U2, U8), 3 amateurs (U5, U14-U15), and 10 novices (U1, U3-U4, U6, U9-U13, U16). Regarding GenAI expertise, there were 12 knowledgeable users (U1-U6, U9-U10, U13-U16) and 4 novices (U7-U8, U11-U12).} The three Chinese paper-cutting experts (E1-E3) were recruited from social video platforms, each having more than 20, 40, and 20 years of experience in designing, creating, and teaching paper-cutting. They are also recognized as ICH inheritors. \rrtext{The detailed information of the participants is shown in~\autoref{table:evaluation participants} and each participant received 100 CNY (approximately 14 USD).}

\begin{figure*}[!htbp]
\centering
\includegraphics[width=\textwidth]{Images/user_cases.pdf}
\caption{\label{figure: user cases}
The 6 paper-cutting design examples were created by the 6 participants in the user study. All solid references are retrieved, while dashed references are generated.}
\Description{This figure shows the 6 paper-cutting design examples created by the participants in the user study. All solid references are retrieved, while ashed references are generated.}
\end{figure*}
\subsection{Procedure and Measures}
In the user study procedure, participants were tasked with designing a paper-cutting that aligned with the provided design themes, such as ``Welcoming Spring'' or ``Health and Longevity.'' Initially, participants are given a 5-minute introduction to the study and its background. For each tool (HarmonyCut or the Baseline), participants underwent three stages: a 15-minute session for tool instruction and task description, a 30-minute period to complete the design task using the respective tool, a 5-minute session to complete a questionnaire, and a 10-minute period for free exploration with the tool. A 5-minute break was scheduled between sessions for each tool. After both experiments, a 15-minute semi-structured interview was conducted. The order of tool usage and tasks was counterbalanced to negate sequence effects on the results. The example outputs generated by participants with varying levels of expertise are presented in~\autoref{figure: user cases}.


To validate the design workflow integrating paper-cutting knowledge with HarmonyCut, participants completed a questionnaire after each session, resulting in two submissions per participant. The questionnaire included three 6-point Likert scale questions focused on design goals: (1) ideation guidance, asking if the tool helps generate ideas; (2) \revisedtext{exploration with reference, asking if the tool supports broad exploration for design, complementary to the Exploration measure in CSI;} and (3) editing flexibility, questioning if the tool allows sufficient editing flexibility. Additionally, the questionnaire incorporated the CSI~\cite{Cherry:2014:csi} (0-10 scale) and NASA-TLX~\cite{Hart:2006:nasa} (0-20 scale) to evaluate usability regarding creative support and perceived workload. 
% Each CSI metric is rated on a scale of 0-10, while each NASA-TLX metric is rated on a scale of 0-20. 

Prior to the expert interview, each expert was asked to prepare a design intent. During the interview, experts received a 15-minute introduction and system walk-through, including a specific design task. After learning how to use HarmonyCut for the paper-cutting design task, experts were given 30 minutes to create a paper-cutting based on their prepared design intent. After the design session, an interview was conducted on three topics adapted from Xu et al.\cite{Xu:2023:magicalbrush}~(\autoref{table3}), each with two questions to gather feedback.
\begin{table}[!htbp]
\caption{Questions during expert interviews from 3 topics.}
\Description{This table presents the six questions used in the expert interviews, categorized into three topics: culture, creativity, and limitations.}
\label{table3}
\resizebox{0.49\textwidth}{!}{
\renewcommand\arraystretch{1.5}
\begin{tabular}{c|c}
\hline
Topic                        & Question                                                             \\ \hline
\multirow{2}{*}{Culture}     & Does HarmonyCut's suggested content align with its cultural meaning? \\
                             & Can HarmonyCut help the public better understand paper-cutting knowledge? \\ \hline
\multirow{2}{*}{Creativity}  & Does HarmonyCut's guidance limit or enhance your creativity?         \\
                             & What aspects of HarmonyCut inspire new creative ideas?               \\ \hline
\multirow{2}{*}{Improvement} & What are your thoughts on human-AI collaborate design systems?              \\
                             & Where can HarmonyCut be improved?                                    \\ \hline
\end{tabular}
}
\end{table}

\subsection{Results}
Based on the qualitative and quantitative data collected from both studies, we found that HarmonyCut effectively facilitates the generation of creative ideas through guided support. It aided participants in exploring valuable related references and allowed them to incorporate their own creative ideas into the paper-cutting design process. 

To assess whether each design goal was achieved within HarmonyCut and whether the system effectively addressed the challenges associated with these goals in the design process, we first analyzed the questionnaire data in the user study. Considering the small sample size (N=16) and the ordinal nature of the data, we employed a Wilcoxon signed-rank test to compare the differences between HarmonyCut and the baseline tool. \revisedtext{Although~\autoref{table4} showed that the average performance across most metrics is better than the Baseline, participants had varying levels of expertise. Therefore, we also analyzed their opinions of the two tools based on their paper-cutting and GenAI expertise levels~(in \autoref{figure:DG},\ref{figure:NASA},\ref{figure:CSI}).} Subsequently, we gathered feedback from the user study and the expert interview.

% \begin{table*}[!htbp]
% \caption{Survey results of participant opinion about design goals, NASA-TLX questionnaire, and Creativity Support Index.}
% \Description{This table illustrates the results from the survey on participants' opinions about the design goals, the NASA-TLX questionnaire, and the Creativity Support Index.}
% \label{table4}
% \renewcommand\arraystretch{1.2}
% \resizebox{0.95\textwidth}{!}{
% \begin{tabular}{ccccccc}
% \hline
% \multicolumn{2}{c|}{\multirow{2}{*}{Indicator}}                                                                  & \multicolumn{2}{c|}{\textbf{HarmonyCut}}                  & \multicolumn{2}{c|}{\textbf{Baseline}}                    & \multirow{2}{*}{P} \\ \cline{3-6}
% \multicolumn{2}{c|}{}                                                                                            & \multicolumn{1}{c|}{Mean}   & \multicolumn{1}{c|}{SD}     & \multicolumn{1}{c|}{Mean}   & \multicolumn{1}{c|}{SD}     &                    \\ \hline
% \multicolumn{1}{c|}{\multirow{3}{*}{Survey related to design goals}} & \multicolumn{1}{c|}{Ideation}             & \multicolumn{1}{c|}{3.875}  & \multicolumn{1}{c|}{0.641}  & \multicolumn{1}{c|}{3.125}  & \multicolumn{1}{c|}{0.835}  & 0.119              \\ \cline{2-7} 
% \multicolumn{1}{c|}{}                                                & \multicolumn{1}{c|}{Exploration}              & \multicolumn{1}{c|}{4.250}  & \multicolumn{1}{c|}{0.775}  & \multicolumn{1}{c|}{2.625}  & \multicolumn{1}{c|}{0.957}  & 0.0024**           \\ \cline{2-7} 

% \multicolumn{1}{c|}{}                                                & \multicolumn{1}{c|}{Editing}              & \multicolumn{1}{c|}{3.500}  & \multicolumn{1}{c|}{0.756}  & \multicolumn{1}{c|}{1.750}  & \multicolumn{1}{c|}{0.463}  & 0.0078**           \\ \hline
% \multicolumn{1}{c|}{\multirow{7}{*}{NASA-TLX}}                       & \multicolumn{1}{c|}{Mental}               & \multicolumn{1}{c|}{13.125} & \multicolumn{1}{c|}{12.159} & \multicolumn{1}{c|}{26.625} & \multicolumn{1}{c|}{21.705} & 0.027*             \\ \cline{2-7} 
% \multicolumn{1}{c|}{}                                                & \multicolumn{1}{c|}{Physical}             & \multicolumn{1}{c|}{22.625} & \multicolumn{1}{c|}{24.231} & \multicolumn{1}{c|}{10.875} & \multicolumn{1}{c|}{14.116} & 0.149              \\ \cline{2-7} 
% \multicolumn{1}{c|}{}                                                & \multicolumn{1}{c|}{Temporal}             & \multicolumn{1}{c|}{14.250} & \multicolumn{1}{c|}{14.680} & \multicolumn{1}{c|}{22.750} & \multicolumn{1}{c|}{16.272} & 0.176              \\ \cline{2-7} 
% \multicolumn{1}{c|}{}                                                & \multicolumn{1}{c|}{Performance}          & \multicolumn{1}{c|}{19.125} & \multicolumn{1}{c|}{12.100} & \multicolumn{1}{c|}{32.125} & \multicolumn{1}{c|}{20.553} & 0.075              \\ \cline{2-7} 
% \multicolumn{1}{c|}{}                                                & \multicolumn{1}{c|}{Effort}               & \multicolumn{1}{c|}{21.625} & \multicolumn{1}{c|}{10.783} & \multicolumn{1}{c|}{13.750} & \multicolumn{1}{c|}{9.301}  & 0.091              \\ \cline{2-7} 
% \multicolumn{1}{c|}{}                                                & \multicolumn{1}{c|}{Frustration}          & \multicolumn{1}{c|}{15.250} & \multicolumn{1}{c|}{18.934} & \multicolumn{1}{c|}{26.250} & \multicolumn{1}{c|}{30.391} & 0.173              \\ \cline{2-7} 
% \multicolumn{1}{c|}{}                                                & \multicolumn{1}{c|}{Overall Load}         & \multicolumn{1}{c|}{7.067}  & \multicolumn{1}{c|}{3.094}  & \multicolumn{1}{c|}{8.825}  & \multicolumn{1}{c|}{2.945}  & 0.25               \\ \hline
% \multicolumn{1}{c|}{\multirow{6}{*}{Creativity Support Index}}       & \multicolumn{1}{c|}{Enjoyment}            & \multicolumn{1}{c|}{38.250} & \multicolumn{1}{c|}{17.523} & \multicolumn{1}{c|}{17.750} & \multicolumn{1}{c|}{12.361} & 0.0078**           \\ \cline{2-7} 
% \multicolumn{1}{c|}{}                                                & \multicolumn{1}{c|}{Exploration}          & \multicolumn{1}{c|}{62.000} & \multicolumn{1}{c|}{30.477} & \multicolumn{1}{c|}{35.875} & \multicolumn{1}{c|}{21.702} & 0.0178*            \\ \cline{2-7} 
% \multicolumn{1}{c|}{}                                                & \multicolumn{1}{c|}{Expressiveness}       & \multicolumn{1}{c|}{53.500} & \multicolumn{1}{c|}{14.172} & \multicolumn{1}{c|}{39.125} & \multicolumn{1}{c|}{23.540} & 0.150              \\ \cline{2-7} 
% \multicolumn{1}{c|}{}                                                & \multicolumn{1}{c|}{Immersion}            & \multicolumn{1}{c|}{29.625} & \multicolumn{1}{c|}{23.090} & \multicolumn{1}{c|}{19.625} & \multicolumn{1}{c|}{17.427} & 0.461              \\ \cline{2-7} 
% \multicolumn{1}{c|}{}                                                & \multicolumn{1}{c|}{Results Worth Effort} & \multicolumn{1}{c|}{54.750} & \multicolumn{1}{c|}{21.022} & \multicolumn{1}{c|}{43.625} & \multicolumn{1}{c|}{24.272} & 0.310              \\ \cline{2-7} 
% \multicolumn{1}{c|}{}                                                & \multicolumn{1}{c|}{CSI}                  & \multicolumn{1}{c|}{79.375} & \multicolumn{1}{c|}{9.410}  & \multicolumn{1}{c|}{52.000} & \multicolumn{1}{c|}{23.243} & 0.0156*            \\ \hline
%  \multicolumn{7}{l}{* p<0.05; ** p< 0.01; ***p<0.001}                                                       
% \end{tabular}
% }
% \end{table*}
\begin{table*}[!htbp]
\caption{Survey results of participant opinion about design goals, NASA-TLX questionnaire, and Creativity Support Index.}
\Description{This table illustrates the results from the survey on participants' opinions about the design goals, the NASA-TLX questionnaire, and the Creativity Support Index.}
\label{table4}
\renewcommand\arraystretch{1.2}
\resizebox{0.95\textwidth}{!}{
\begin{tabular}{ccccccc}
\hline
\multicolumn{2}{c|}{\multirow{2}{*}{Indicator}}                                                                  & \multicolumn{2}{c|}{\textbf{HarmonyCut}}                  & \multicolumn{2}{c|}{\textbf{Baseline}}                    & \multirow{2}{*}{P} \\ \cline{3-6}
\multicolumn{2}{c|}{}                                                                                            & \multicolumn{1}{c|}{Mean}   & \multicolumn{1}{c|}{SD}     & \multicolumn{1}{c|}{Mean}   & \multicolumn{1}{c|}{SD}     &                    \\ \hline
\multicolumn{1}{c|}{\multirow{3}{*}{Survey related to design goals}} & \multicolumn{1}{c|}{Ideation}             & \multicolumn{1}{c|}{3.563}  & \multicolumn{1}{c|}{0.727}  & \multicolumn{1}{c|}{3.375}  & \multicolumn{1}{c|}{0.957}  & 0.472              \\ \cline{2-7} 
\multicolumn{1}{c|}{}                                                & \multicolumn{1}{c|}{Exploration}          & \multicolumn{1}{c|}{4.125}  & \multicolumn{1}{c|}{0.885}  & \multicolumn{1}{c|}{2.563}  & \multicolumn{1}{c|}{0.892}  & 0.0029**           \\ \cline{2-7} 
\multicolumn{1}{c|}{}                                                & \multicolumn{1}{c|}{Editing}              & \multicolumn{1}{c|}{3.625}  & \multicolumn{1}{c|}{0.806}  & \multicolumn{1}{c|}{1.438}  & \multicolumn{1}{c|}{0.727}  & 0.00003***         \\ \hline
\multicolumn{1}{c|}{\multirow{7}{*}{NASA-TLX}}                       & \multicolumn{1}{c|}{Mental}               & \multicolumn{1}{c|}{14.250} & \multicolumn{1}{c|}{11.958} & \multicolumn{1}{c|}{23.625} & \multicolumn{1}{c|}{16.931} & 0.026*             \\ \cline{2-7} 
\multicolumn{1}{c|}{}                                                & \multicolumn{1}{c|}{Physical}             & \multicolumn{1}{c|}{16.813} & \multicolumn{1}{c|}{19.641} & \multicolumn{1}{c|}{7.625}  & \multicolumn{1}{c|}{12.447} & 0.046*             \\ \cline{2-7} 
\multicolumn{1}{c|}{}                                                & \multicolumn{1}{c|}{Temporal}             & \multicolumn{1}{c|}{24.625} & \multicolumn{1}{c|}{22.102} & \multicolumn{1}{c|}{16.688} & \multicolumn{1}{c|}{14.858} & 0.460              \\ \cline{2-7} 
\multicolumn{1}{c|}{}                                                & \multicolumn{1}{c|}{Performance}          & \multicolumn{1}{c|}{18.875} & \multicolumn{1}{c|}{13.837} & \multicolumn{1}{c|}{39.750} & \multicolumn{1}{c|}{23.345} & 0.0041**           \\ \cline{2-7} 
\multicolumn{1}{c|}{}                                                & \multicolumn{1}{c|}{Effort}               & \multicolumn{1}{c|}{25.438} & \multicolumn{1}{c|}{13.008} & \multicolumn{1}{c|}{12.188} & \multicolumn{1}{c|}{7.458}  & 0.0045**           \\ \cline{2-7} 
\multicolumn{1}{c|}{}                                                & \multicolumn{1}{c|}{Frustration}          & \multicolumn{1}{c|}{18.625} & \multicolumn{1}{c|}{19.328} & \multicolumn{1}{c|}{30.125} & \multicolumn{1}{c|}{28.477} & 0.157              \\ \cline{2-7} 
\multicolumn{1}{c|}{}                                                & \multicolumn{1}{c|}{Overall Load}         & \multicolumn{1}{c|}{7.908}  & \multicolumn{1}{c|}{2.980}  & \multicolumn{1}{c|}{8.667}  & \multicolumn{1}{c|}{2.805}  & 0.562              \\ \hline
\multicolumn{1}{c|}{\multirow{6}{*}{Creativity Support Index}}       & \multicolumn{1}{c|}{Enjoyment}            & \multicolumn{1}{c|}{27.125} & \multicolumn{1}{c|}{18.395} & \multicolumn{1}{c|}{23.438} & \multicolumn{1}{c|}{14.660} & 0.562              \\ \cline{2-7} 
\multicolumn{1}{c|}{}                                                & \multicolumn{1}{c|}{Exploration}          & \multicolumn{1}{c|}{54.250} & \multicolumn{1}{c|}{26.055} & \multicolumn{1}{c|}{37.563} & \multicolumn{1}{c|}{23.218} & 0.073              \\ \cline{2-7} 
\multicolumn{1}{c|}{}                                                & \multicolumn{1}{c|}{Expressiveness}       & \multicolumn{1}{c|}{55.375} & \multicolumn{1}{c|}{16.552} & \multicolumn{1}{c|}{34.250} & \multicolumn{1}{c|}{18.724} & 0.0076**            \\ \cline{2-7} 
\multicolumn{1}{c|}{}                                                & \multicolumn{1}{c|}{Immersion}            & \multicolumn{1}{c|}{30.688} & \multicolumn{1}{c|}{17.621} & \multicolumn{1}{c|}{20.750} & \multicolumn{1}{c|}{15.902} & 0.074              \\ \cline{2-7} 
\multicolumn{1}{c|}{}                                                & \multicolumn{1}{c|}{Results Worth Effort} & \multicolumn{1}{c|}{52.313} & \multicolumn{1}{c|}{22.934} & \multicolumn{1}{c|}{42.375} & \multicolumn{1}{c|}{28.577} & 0.222              \\ \cline{2-7} 
\multicolumn{1}{c|}{}                                                & \multicolumn{1}{c|}{CSI}                  & \multicolumn{1}{c|}{73.250}  & \multicolumn{1}{c|}{14.283} & \multicolumn{1}{c|}{52.792} & \multicolumn{1}{c|}{19.734} & 0.0013**           \\ \hline
\multicolumn{7}{l}{*p<0.05; **p<0.01; ***p<0.001}                                   
\end{tabular}
}
\end{table*}


\begin{figure*}[!htbp]
\centering
\includegraphics[width=\textwidth]{Images/DG_distribution4.pdf}
\caption{\label{figure:DG}
Sixteen participants ratings the design goals questionnaire across different expertise levels on paper-cutting and GenAI.}
\Description{This figure shows the sixteen participants' ratings to the design goals questionnaire across different expertise levels on paper-cutting and GenAI.}
\end{figure*}

\subsubsection{Ideation with Guidance}
Regarding the first design goal, factor-oriented guidance for ideation, as shown in~\autoref{table4}, while the results for ideation were not statistically significant, the average scores indicated that participants perceived HarmonyCut as providing better support for their \textbf{Ideation} process (HarmonyCut: M=3.563, SD=0.727 / Baseline: M=3.375, SD=0.957). Additionally, the NASA-TLX results revealed that participants experienced a significantly lower \textbf{Mental} workload when using HarmonyCut (M=14.250, SD=11.958 / Baseline: M=23.625, SD=16.931 / P=0.026*, W=21.0) compared to the Baseline. This suggests that HarmonyCut's guided ideation, facilitated by factor-based options, effectively supported users during the ideation stage, especially for tasks requiring mental effort.

\begin{figure*}[!htbp]
\centering
\includegraphics[width=\textwidth]{Images/NASA_distribution4.pdf}
\caption{\label{figure:NASA}
Sixteen participants ratings to the NASA-TLX perceived load questionnaire across different expertise levels on paper-cutting and GenAI.}
\Description{This figure shows the sixteen participants' ratings to the NASA-TLX perceived load questionnaire across different expertise levels on paper-cutting and GenAI.}
\end{figure*}

Furthermore, participants noted that the baseline tool tended to recommend overly broad information or required iteration, making it challenging to decide on an idea. As U3 mentioned, ``\textit{I gave a vague description and had to spend time filtering suggestions.}'' In contrast, HarmonyCut's selection of factors and suggested content helped users understand paper-cutting knowledge through introductions and interpretations. U5 highlighted this advantage, stating, ``\textit{HarmonyCut provided relevant patterns, each with explanations of their meaning and how they fit the design purpose.}''  Meanwhile, in the expert interview, E1’s responses to questions 1 and 2 on the cultural topic, given E1's many years of experience teaching paper-cutting, align with U5’s points: ``\textit{Aligning suggested form and meaning is not enough for understanding. It is through interpretation that users can really grasp the content and, in turn, find inspiration during the learning process.}''. This guidance aided users in the ideation process, helping them quickly construct initial ideas and comprehend the rationale behind their choices, as U3 concluded, ``\textit{This helped me quickly construct initial ideas and understand their choices.}''

In addition, to evaluate how DG1 addresses C2 in~\autoref{sec:formative}, we collected expert feedback on creativity and culture during the interviews. All three experts agreed that the factors-oriented guidance did not limit their creativity. They also acknowledged that a system supported by domain knowledge, which enhances understanding through explanations, was highly beneficial for learning about paper-cutting traditions from different regions and cultures, aiding in innovative ideation.

\begin{figure*}[!htbp]
\centering
\includegraphics[width=\textwidth]{Images/CSI_distribution4.pdf}
\caption{\label{figure:CSI}
Sixteen participants ratings on the Creative Support Index questionnaire across expertise levels in paper-cutting and GenAI.}
\Description{This figure shows the sixteen participants' ratings to the Creative Support Index questionnaire across different expertise levels on paper-cutting and GenAI.}
\end{figure*}


\subsubsection{Reference Exploration}
% 对于具有参考的探索过程,Harmony为用户带来的探索感显著,根据调查的结果,Exploration(Harmonyut: M=4.25, SD=1.035 / Baseline: M=3.0, SD=1.069)以及CSI (HarmonyCut: M=62, SD=30.477 / Baseline: M=35.875, SD=21.702 / p =0.0178*)中结果可以都表明显著在探索多样的想法优于Baseline。 U4表示,似乎是由于模型训练数据的问题,即使他更换了几次输入,返回的剪纸内容总是很相近,而且总是会出现 HUI pattern。 类似的E8也表示,我更换了几次输入,但Baseline的返回结果还是风格相近。
% U3认为检索已有相关剪纸以及按idea生成剪纸结合,通过探索,共同作为构图参考的模式即保证了探索的效率:不会因为被生成的质量参差不齐的内容淹没,也不会只限制在有限的可检索数据中,而缺乏构图的创意。 
% 令我们惊讶的是,E2表示,作为专家,他最近被邀请设计用于宣传政府机构的剪纸作品,但他也不知道该用哪些与该机构相关的题材来进行创作,同时他创作抽象风格剪纸多,想参考下别人的写实剪纸。 他认为检索到的几幅刻画钢厂的剪纸为他提供了一定灵感。 将他们作为reference,并使用Harmony的生成模型来查看常见与钢厂匹配的造型构图,同样作为参考。这确实帮助了他们从探索,确定参考到完成基本的构图
\revisedtext{The results indicated that HarmonyCut significantly enhances the exploration process, as reflected in the \textbf{Exploration} measure in design goal questionnaire (HarmonyCut: M=4.125, SD=0.885 / Baseline: M=2.563, SD=0.892 / p=0.0029**, W=9.0). Besides, HarmonyCut's \textbf{Exploration} score in the CSI was higher than that of the baseline tool (HarmonyCut: M=54.250, SD=26.055 / Baseline: M=37.563, SD=23.218). These results suggested that HarmonyCut supports a more diverse and creative exploration compared to the Baseline.}

Several participants provided insightful feedback. U4 and U8 both noted that the ``\textit{Baseline system tended to return similar results, regardless of input changes, limiting creative exploration.}'' In contrast, U5 highlighted that ``\textit{HarmonyCut's combination of design retrieval and idea-based generation allowed for more efficient exploration.}'' This approach helped avoid the pitfalls of inconsistent generative results while offering a broader creative scope than relying solely on retrievable data.
% \revisedtext{Although U15虽然用harmony去探索设计可能的过程确实更耗时}
To analyze based on participants' varying levels of expertise, we found that U7, a master of paper-cutting~(\autoref{figure:DG},\autoref{figure:CSI}), felt that the exploration feature offered the same capability as the Baseline (\textbf{Exploration} in Design Goal: HarmonyCut=4, Baseline=4; \textbf{Exploration} in CSI: HarmonyCut=16, Baseline=16), stating: ``\textit{I've seen quite a lot of paper-cuttings. The additional retrieval function in HarmonyCut did not significantly enhance the sense of exploration.}'' Interestingly, U7 was recently invited to design a government-themed paper-cutting piece. Seeking inspiration, U7 reviewed realistic paper-cuttings of steel factories retrieved by HarmonyCut in the free exploration session. Using these as references, along with HarmonyCut's generative model to explore common themes related to steel factories, U7 noted that ``\textit{Although the exploration content is limited due to the restricted dataset, reference-based exploration did help me with composition. Also, using the system saved me a lot of time compared to sketching, while the Baseline was almost useless for planning the cut-out layout.}'' (\textbf{Temporal Load}: HarmonyCut=6, Baseline=10). For participants with varying expertise levels in both paper-cutting and GenAI, HarmonyCut was generally perceived to enhance exploration capabilities, as illustrated in~\autoref{figure:DG} and~\autoref{figure:CSI}.


\subsubsection{Editing with Effort}\label{sec:editing}
% 对于内容的编辑,HarmonyCut支持基于组合纹样轮廓的分割,基于单元纹样的提取,以及相关作品的检索。 Baseline仅能通过输入来调整输出。同时,根据结果,也能发现, 受访者普遍认为HarmonyCut相较于baseline可以灵活编辑,且通过自己参与交互,提高了可控性,也不需要只依靠模型输出。 但值得注意的一点是,在付出努力和体力负担上,HarmonyCut也是比Baseline高的。U1表示,对象分割提供少量point分割效果不好,但提供point太多会很费力。
For content editing, HarmonyCut supports segmentation based on the contours of composite patterns, extraction of unit patterns, and retrieval of related works. In contrast, the baseline tool only allows output adjustments through input modifications. The results indicate that participants generally perceived HarmonyCut as offering more flexible \textbf{Editing} capabilities compared to the baseline, enhancing controllability (HarmonyCut: M=3.625, SD=0.806 / Baseline: M=1.438, SD=0.727 / p=0.00003***, W=0) through user interaction rather than relying solely on model outputs. \revisedtext{Furthermore, although the results for \textbf{Enjoyment} were not statistically significant (HarmonyCut: M=27.125, SD=18.395 / Baseline: M=23.438, SD=14.660), some participants expressed enjoyment in editing their designs using our system. Moreover, despite being knowledgeable about GenAI and using prompt engineering to enhance outputs, some participants argued that the lack of control in the baseline tool hindered design. Furthermore, some participants attributed the improved ability to \textbf{Express} ideas with HarmonyCut (HarmonyCut: M=55.375, SD=16.552; Baseline: M=34.250, SD=18.724; p=0.0076**, W=13.0), compared to the Baseline, to its controllable editing support. For instance, U6 remarked, ``\textit{The inability to directly edit the output and the uncontrollable reasoning process made it difficult for me to enjoy the design experience.}'' U14 also mentioned about ``\textit{Especially for content related to Chinese characters, the Baseline often fails to meet the requirements, generating some meaningless stroke combinations. By comparison, I still prefer a controllable design process to express my idea.}''} To the expert interview, E3 noted that ``\textit{the ability to edit undesirable parts of the generated image is a critical function in HarmonyCut, improved the controllability issue of GenAI.}''
% 另外,多为参与者提到,这样的编辑能力,也支持其更好的表达自己的意图。不过从图中也可以看到,专家认为两者的表达力都有限,对于Harmony,它们一方面因为reference list还是规模有限,且编辑程度还是不够灵活,无法完全发挥其expertise,这也会在\autoref{limitation} and \autoref{discuss:guidance reference}中进一步讨论。


\subsubsection{\revisedtext{Design Process and Design Performance}}\label{sec:process performance}
\revisedtext{Chinese paper-cutting embodies a fundamental connotation: the aspiration for a better life. This connotation is directly expressed through the selection of content and composition, which serve as key considerations in the creative process~\cite{Lin:1974:howtopapercutting, Zhang:2021:papercuttingteaching, Li:1998:monopapercutting}. Experienced paper-cutting educators (E1-E3) also consistently emphasized the importance of training students to choose appropriate content and develop effective compositions as a critical aspect of learning paper-cutting. Consequently, integrating support for these aspects into the design process is essential. 
This was partially reflected in the \textbf{Performance} (HarmonyCut: M=18.875, SD=13.837 / Baseline: M=39.750, SD=23.345 / p=0.0041**, W=9.5) cross users with different levels of expertise. Experts recognized HarmonyCut in~\autoref{figure:NASA} as effective in supporting reference selection and composition. Novices acknowledged the system's \textbf{Performance}~(\autoref{figure:NASA}) and additionally reported a more \textbf{Immersive} experience~(\autoref{figure:CSI}) during the design process. Notably, U3, U9, and U13, as young novices, emphasized that the design process in HarmonyCut not only helped them learn the knowledge of paper-cutting but also sparked a deeper interest in the craft. Such knowledge sharing and interest of the young generation play a role in influencing the preservation and transmission of ICH~\cite{Affleck:2008:newgeneration, Mancacaritadipura:2009:newgeneration}, like paper-cutting. Nevertheless, beyond initial interest, long-term accumulation and profound knowledge are necessary to foster creativity and innovation, ensuring the inheritance and vitality of ICH, as also discussed in~\autoref{limitations}. }
% 数据 performance in NASA, Expressive in CSI。 对于最终的结果,
% 作为民俗艺术,一个核心功能是寄托人民对美好生活的向往,
% 专家讨论 剪纸作为一个很靠经验的 民俗艺术,很多都是为了传达吉祥寓意,其首要考量的就是视觉形式与构图是否能够体现作者巧思且满足祝福他人的目的。因此,对于剪纸的domain expertise,其首要考量的一部分就是能否选择合适的内容,以及巧妙地构图。 而这一点,不仅被expert 认可, 且其他expertise的一些用户在谈到最终结果很满意,让他在短时间内了解了某一剪纸主题相关的知识,同时因其能够足够表达自我而对剪纸这一艺术形式产生进一步了解的兴趣。 而在new generation 产生兴趣,这对于非物质文化遗产的传承,是很关键的一点
%尽管对应思维完成主题与形式的匹配是它的最基础首要能力,但也需要跟随着新事物的出现,以及新内涵的赋予,将他们用于创作中。 


\subsubsection{System Limitations}\label{limitations}
% Based on the feedback from participants from the user study and expert interviews, 我们总结了系统目前存在的不足:
% 第一,目前的design space已经比较comprehensive, 但毕竟是文化相关知识,尤其剪纸这样的民间艺术,多样性很强,对于整个design space中的标注数据还可以继续扩充, ,持续更信数据才能保证design space的合理性。 也可以保证模型提供的剪纸知识和解释更合理。 当然,这是一个周期性很强,ambition很壮的目标
% 第二,虽然contour+pattern来进行构图设计可以极大提高过程的编辑性,但与直接通过文本输入而获取返回图片,还是要通过图片分割,纹样选取上付出努力。这也在nasa-tlx结果可以看出,HarmonyCut 脑力要求低,但体力要求以及付出努力比baseline高。 不过这是一个权衡的问题
% 第三  E3提到,她对于AI辅助设计上很担心AI所学风格涉及版权侵犯问题,导致自己在GenAI-aided设计过程中也会造成侵权。
Based on feedback from participants in the user study and expert interviews, we have summarized the current limitations of the system, which are also discussed in~\autoref{lf} with future directions:

\noindent\textbf{Cultural Knowledge Diversity:} \revisedtext{While the current design space is relatively comprehensive, given the diversity of cultural knowledge, especially in profound folk arts like paper-cutting. Thus, there is room for further expansion of labeled data, including the taxonomy and recommended reference (proposed by E1).} Continuously enriching the dataset will ensure the design space remains reasonable and that the paper-cutting knowledge and explanations provided by the model are more accurate. This is an ambitious and ongoing effort. U7 made a similar suggestion, pointing out that the collected and annotated paper-cutting data still needs to be expanded, as the bottleneck appears to lie in the data. Expanding this dataset would improve the usability of HarmonyCut.

\noindent\textbf{Effort-Intensive Design Process:} Although the contour + pattern approach enhances the flexibility of the design process, it requires more effort compared to simply inputting text and receiving generated images (U1-U2, U4, U10-11). Users need to invest time in image segmentation and pattern selection. This trade-off is reflected in the NASA-TLX results, where HarmonyCut shows a lower \textbf{Mental} load but significantly higher \textbf{Physical} load and \textbf{Effort} compared to the baseline~(\autoref{table3}). However, balancing these factors is essential.
% %当然,剪纸的知识以及所能掌握的专业程度远不止于此,需要长年的学习和体验,才能将在对应思维的基础上更好的融入自己的风格,既能通过丰厚的知识继承过去的内容,也能有把握将新事物以剪纸形式表现进行创新。
% design performance那部分讨论expert和literature align,模仿对于初学者来说很重要。不过,进一步的学习就需要更多creation,references更多只是扩充思路,作为融入作者创意,加入创新内容的载体
% \vspace{1.4mm}\noindent\textbf{Effort-Intensive Design Process:} E3 raised concerns about GenAI-aided design, specifically the risk of copyright infringement due to the styles learned by AI. She worried that using AI-generated content might inadvertently lead to infringement in her designs.

\noindent\textbf{\revisedtext{Imitation and Creation in Design:}} \revisedtext{
Although HarmonyCut is considered significantly more expressive in~\autoref{sec:editing}, U7 and U8~(\autoref{figure:CSI}) found its expressiveness limited, attributing this to the restricted reference list and lack of more types of editing tools, which they felt constrained their expertise. Besides, some novice users, on the other hand, overly relied on the recommended references for their designs, leading to imitation issues. From another perspective, E1, E2, and Lin~\cite{Lin:1974:howtopapercutting} emphasized that reference-based imitation is, to some extent, a necessary step in the creative process. In their teaching practices, students were required to engage in extensive imitation to acquire relevant knowledge, develop skills, and draw inspiration~(\autoref{sec:process performance}), which was align with the view of Okada et al.~\cite{Okada:2017:imitation}.
Paper-cutting design should extend beyond imitation, which serves as a foundational step in learning the craft. To foster creativity, it is essential to integrate new objects and meanings into the design process. Achieving this requires time, training, and experience, moving past the mere reuse of references to maintain creativity and drive innovation.} 
% 认为其表达力充足, 但对于expert(U2,U7-U8)群体 figure7,他么普遍认为Harmony的表现力有限,一方面因为reference list规模有限,且编辑度对于需求更多,经验更丰富的专家还是不够灵活,更多的是需要基于所给reference进行创作,无法完全发挥它们expertise。 不过,U7与E2,and 怎样剪纸也都提到,一定程度的基于reference的模仿,是促进剪纸的创意设计的必要一步,它们在教学中首先会要求学生进行大量的临摹,来学习相关知识并掌握skill of selection and composition in~\autoref{sec:process performance},这也与imitation and creation in design 的view一致~\cite{Japan}。 不过,对应思维完成主题与形式的匹配是它的最基础首要能力,但剪纸设计也需要跟随着新事物的出现,以及新内涵的赋予,将他们用于创作中, 这当然需要大量1时间的训练和经验的积累,形成独属于自己风格的剪纸设计体系。 

% 多为参与者提到,这样的编辑能力,也支持其更好的表达自己的意图。不过从图中也可以看到,专家认为两者的表达力都有限,对于Harmony,它们一方面因为reference list还是规模有限,且编辑程度还是不够灵活,无法完全发挥其expertise,这也会在\autoref{limitation} and \autoref{discuss:guidance reference}中进一步讨论。
% \vspace{1.4mm}\noindent\textbf{Generalizibility to users from different group} 

% \begin{table*}[!htbp]
% \caption{own}
% \label{figure4}
% \begin{tabular}{c|cc|cc|c}
% \hline
% \multirow{2}{*}{Indicator} & \multicolumn{2}{c|}{\textbf{HarmonyCut}}       & \multicolumn{2}{c|}{Baseline}      & \multirow{2}{*}{P} \\ \cline{2-5}
%                            & \multicolumn{1}{c|}{Mean}  & SD    & \multicolumn{1}{c|}{Mean}  & SD    &                    \\ \hline
% Ideation                   & \multicolumn{1}{c|}{3.875} & 0.641 & \multicolumn{1}{c|}{3.125} & 0.835 & 0.119              \\ \hline
% Exploration                & \multicolumn{1}{c|}{4.25}  & 1.035 & \multicolumn{1}{c|}{3.000} & 1.069 & 0.0078             \\ \hline
% Editing                    & \multicolumn{1}{c|}{3.500} & 0.756 & \multicolumn{1}{c|}{1.750} & 0.463 & 0.101              \\ \hline
% \end{tabular}
% \end{table*}


% \begin{table*}[!htbp]
% \caption{NASA-TLX}
% \label{figure5}
% % \renewcommand\arraystretch{1.2}
% % \resizebox{0.75\textwidth}{!}{
% \begin{tabular}{c|cc|cc|c}
% \hline
% \multirow{2}{*}{Indicator} & \multicolumn{2}{c|}{\textbf{HarmonyCut}} & \multicolumn{2}{c|}{\textbf{Baseline}}                     & \multirow{2}{*}{P} \\ \cline{2-5}
%                            & \multicolumn{1}{c|}{Mean}     & SD       & \multicolumn{1}{c|}{Mean}   & \multicolumn{1}{c|}{SD}     &                    \\ \hline
% Mental                     & \multicolumn{1}{c|}{13.125}   & 12.159   & \multicolumn{1}{c|}{26.625} & \multicolumn{1}{c|}{21.705} & 0.027*              \\ \hline
% Physical                    & \multicolumn{1}{c|}{22.625}   & 24.231   & \multicolumn{1}{c|}{10.875} & \multicolumn{1}{c|}{14.116} & 0.149              \\ \hline
% Temporal                   & \multicolumn{1}{c|}{14.250}   & 14.680   & \multicolumn{1}{c|}{22.750} & \multicolumn{1}{c|}{16.272} & 0.176              \\ \hline
% Performance                & \multicolumn{1}{c|}{19.125}   & 12.100   & \multicolumn{1}{c|}{32.125} & \multicolumn{1}{c|}{20.553} & 0.075              \\ \hline
% Effort                     & \multicolumn{1}{c|}{21.625}   & 10.783   & \multicolumn{1}{c|}{13.750} & \multicolumn{1}{c|}{9.301}  & 0.091              \\ \hline
% Frustration                & \multicolumn{1}{c|}{15.250}   & 18.934   & \multicolumn{1}{c|}{26.250} & \multicolumn{1}{c|}{30.391} & 0.173              \\ \hline
% Overall Load               & \multicolumn{1}{c|}{7.067}    & 3.094    & \multicolumn{1}{c|}{8.825}  & \multicolumn{1}{c|}{2.945}  & 0.25               \\ \hline
% \multicolumn{6}{l}{* p < 0.05; ** p < 0.01; *** p < 0.001}
% \end{tabular}
% % }
% \end{table*}


% \begin{table*}[!htbp]
% \caption{Creative Support Index}
% \label{figure6}
% % \renewcommand\arraystretch{1.2}
% % \resizebox{0.75\textwidth}{!}{
% \begin{tabular}{c|cc|cc|c}
% \hline
% \multirow{2}{*}{Indicator} & \multicolumn{2}{c|}{\textbf{HarmonyCut}}         & \multicolumn{2}{c|}{\textbf{Baseline}}        & \multirow{2}{*}{P} \\ \cline{2-5}
%                            & \multicolumn{1}{c|}{Mean}   & SD     & \multicolumn{1}{c|}{Mean}   & SD     &                    \\ \hline
% Enjoyment                  & \multicolumn{1}{c|}{38.250} & 17.523 & \multicolumn{1}{c|}{17.750} & 12.361 & 0.0078**             \\ \hline
% Exploration                & \multicolumn{1}{c|}{62.000} & 30.477 & \multicolumn{1}{c|}{35.875} & 21.702 & 0.0178*             \\ \hline
% Expressiveness             & \multicolumn{1}{c|}{53.500} & 14.172 & \multicolumn{1}{c|}{39.125} & 23.540 & 0.150              \\ \hline
% Immersion                  & \multicolumn{1}{c|}{29.625} & 23.090 & \multicolumn{1}{c|}{19.625} & 17.427 & 0.461              \\ \hline
% Results Worth Effort       & \multicolumn{1}{c|}{54.750} & 21.022 & \multicolumn{1}{c|}{43.625} & 24.272 & 0.310              \\ \hline
% CSI                        & \multicolumn{1}{c|}{79.375} & 9.410  & \multicolumn{1}{c|}{52.000} & 23.243 & 0.0156*             \\ \hline
% \multicolumn{6}{l}{* p < 0.05; ** p < 0.01; *** p < 0.001}
% \end{tabular}
% %}
% \end{table*}


\section{Discussion}

We introduce the design space and workflow of paper-cutting and propose a novel GenAI-aided creativity support tool, HarmonyCut, which supports reference-guided exploration from ideation to composition of paper-cutting design. Based on our findings, we suggest some design implications for future creativity support tools.

% \subsection{}

\subsection{Guidance and Reference in Different Context}\label{discussion:guidance reference}
As reflected in the findings and expert feedback from the evaluation (\autoref{evaluation}), users' preferences and assessments of the system's features were influenced by their varying levels of expertise across different fields. Experts tended to favor more exploration, while novices, due to their lack of domain knowledge, preferred the system to make recommendations for them. Therefore, it is crucial to consider the differing workflows of users with varying expertise to support design. This ensures that users in different contexts can effectively use the system to support their design tasks. This approach is also supported by findings from previous research~\cite{Xiao:2024:typedance, Zhou:2023:filtererink, Yan:2022:flatmagic}.

In the ideation process, experts typically have a clear design intent and seek comprehensive references to inspire creativity and refine their ideas. On the other hand, novices often struggle to effectively translate tasks into concrete ideas, making them more reliant on the system to guide them through the ideation process.
Additionally, we observed that novices tend to imitate references rather than use them for creative ideation. Therefore, the system needs to provide explanations alongside the guidance to help them understand the underlying knowledge and design with that understanding. 
\revisedtext{For experts, the focus is on providing references that are comprehensive to better support their advanced creative needs and ensure expressiveness. Due to the differing priorities of novice and expert users, experts may perceive GenAI as inadvertently constraining the creative design process, as also discussed in~\autoref{limitations}. This constraint arises from the nature of GenAI-aided guidance and references, which are inherently linked to the dataset~\cite{Cui:2024:chatlaw, Wang:2023:methodsknowledge} and may restrict creative outcomes to the dataset domain. U2 emphasized that paper-cutting, as a highly abstract form of expression, depends on distinctive personal styles to achieve creative results. However, outputs from GenAI with stereotypes may lead to stylistic homogenization. This not only limits the diversity of creative expression but also raises concerns about copyright~\cite{Samuelson:2023:copyright, Bianchi:2023:stereotype, Zhou:2024:biasgenerativeai}.}

Without proper guidance, novices can easily get lost in many suggestions, while experts can draw inspiration from more abstract references. For novices, more specific guidance helps them extract useful suggestions and apply them to their designs.
% Besides, the 除了系统工作流程中的引导和参考, 交互方式也存在 trade-off, 一方面, GenAI作为端到端的模型始终因为无法对结果进行再编辑,只能通过迭代的更换输入来实现, 而缺乏可控性. 但在实验中我们也发现, 虽然edit和可控性显著提高. 但相比于模型只需要输入文字的交互, 系统为用户带来的requirement of physical and effort 也会提高. 因此, 在通过GenAI协助,用户交互式的完成设计需要平衡上的研究, 通过改进交互方式减少新的负担, 同时保证依然有可控性
In addition to the guidance and references provided in the system workflow, there are trade-offs in the controllabliltiy. On one hand, as an end-to-end model, GenAI results cannot be directly edited, requiring users to iteratively adjust their input, which limits controllability. On the other hand, our experiments revealed that while the system significantly improves editability and control, it also increases the physical and cognitive demands on users compared to simple text-based interaction. Therefore, future research should focus on finding a balance in GenAI-aided design and improving interaction methods to reduce these additional burdens while ensuring adequate control remains in the creative process.
% \revisedtext{另外, from critical view, guidance and reference involved GenAI被批判可能将会创意限定在domain of data,even one expert 担忧,由于抽象剪纸很需要个人风格的夸张,GenAI基于数据集生成的相关刻板印象生成的风格明显的内容是否会有侵权问题}


% \subsection{}
% 正如evaluation~\autoref{evaluation}中findings所表现的, 用户因其在不同领域具有不同水平的专业知识, 而影响他对系统不同功能的喜好, 以及和评判水平. 专家倾向于更多的探索, 而新手因缺乏专业知识,更倾向于让系统为其推荐. 因为在ideation过程中,专家的design intent一般比较明确, 他需要的是更全面的reference来扩充构思的创意感,然后实现自己的构思. 而对于新手,它甚至无法将任务有效转化成构思,因此跟倾向于系统能指导它进行构思. 另外, 根据我们的观察, 新手根据reference更多是模仿, 所以系统更多的是需要通过在引导中提供相应的解释,来帮助他产生对知识的理解, 在有知识的基础上进行设计. 而对于专家, 更需要的是能够让系统提供非常符合其描述的Reference,来帮助创意. 如果缺乏引导, 新手会在大规模suggestion中迷失, 如果reference 更抽象, 专家从具体reference中获取inspiration. Guidance更具体, 新手从其中提取建议,来实施设计



\subsection{Integrate the Cultural Knowledge with GenAI Support Creative Design}
% Based on the studies and findins in this work findings, and prior work, that demonstrate GenAI 在理解带有文化因素内容上的能力受限. 如果想保证系统不会因此而影响创意支持, 就需要将文化相关内容整合到模型中, 这有几种方式, 一种通过大量的数据增强, 但它受限与数据的规模和数据收集的难度. 就像E1所说, HramonyCut通过factor所涵盖的知识个人认为已经可以将剪纸大部分知识用这个taxonomy包含, 但文化就是博大精深, 若想通过增加数据规模来实现 GenAI 支持相关设计 是困难的. 因此可以需要考虑如何将内容结构化, 就像HarmonyCut中 将 paper-cutting 限制在factor的范畴当中, magical brush 将 chinese pianting 用symbol来规范,, 也即提供复杂知识的前置, 让模型理解构成知识的基础, 从而理解知识..
% 虽然生成式模型具有很强的语言和视觉理解能力, 在多种创作任务中允许用户直接通过输入自然语言来让模型协助用户创作而无需复杂参数, 但对于特定任务, 模型难以理解用户需求,从而提供符合预期的输出, 尤其是在比较抽象的文化艺术设计方面. 同样的, Based on the studies and findings from this work, coupled with insights from prior research~\cite{Messer:2024:cocreating, Garcia:2024:paradox, Chung:2023:artinter}, it is clear that GenAI experiences limitations in understanding content with cultural elements. To prevent these limitations from hindering creative support, it is crucial to integrate cultural knowledge into the model. One approach is large-scale data augmentation; however, this method is constrained by the challenges related to the scale and complexity of data collection.
% E1 emphasized that the taxonomy of factors, which organizes knowledge through design factors, effectively encompasses much of the paper-cutting knowledge. However, due to the vast and intricate nature of cultural knowledge, relying solely on expanding the dataset is impractical for enhancing GenAI's ability to support culturally relevant design. A more effective strategy may involve structuring the cultural content,
% 先前的工作在绘画方面尝试提取模糊概念,特殊处理, 但这些内容与具体的视觉形式间,因为模型的欠缺抽象知识而存在gap. 如何在文化艺术方面的任务中,让模型给出合理回答, 可以通过将抽象的内容, 通过用设计任务和设计空间来归纳起来,:
% similar to how HarmonyCut organizes paper-cutting within a framework of factors, or how symbols are used in Chinese painting~\cite{Xu:2023:magicalbrush} and template-based approaches are applied in dynamic shadow puppetry creation~\cite{Yao:2024:shadowmaker}. This structured approach provides a foundational framework of complex knowledge, enabling the model to better understand and engage with cultural content, thereby supporting design. 
GenAIs have demonstrated strong capabilities in both language and visual understanding, allowing users to engage in the creative process through natural language input. However, in tasks involving abstract cultural and artistic design, these models often fail to fully capture user intent, resulting in outputs that may not meet expectations. The findings from this study, alongside insights from previous research~\cite{Messer:2024:cocreating, Garcia:2024:paradox, Chung:2023:artinter}, highlight the limitations of GenAI in comprehending and interpreting cultural elements. To prevent these shortcomings from impeding creative support, it is essential to incorporate cultural knowledge into the models. Although large-scale data augmentation offers one possible solution, it is limited by both the scale and complexity of data collection.
E1 emphasized that the taxonomy used to organize knowledge through design factors effectively captures a significant portion of paper-cutting knowledge. 

However, the vast and intricate nature of cultural knowledge makes it impractical to rely solely on dataset expansion to improve GenAI's ability to support culturally relevant design. A more effective approach may involve structuring cultural content. Previous work in the field of painting~\cite{Chung:2023:promptpaint} has attempted to extract vague concepts and process them individually, but these models often lack the abstract knowledge necessary, resulting in a disconnect between abstract ideas and concrete visual outputs. By structuring abstract content into design tasks and design spaces, models can generate more appropriate outputs for cultural art-related tasks.

In this study, HarmonyCut organizes the paper-cutting design process around factors and patterns, while Magical Brush uses symbols as fundamental elements in Chinese painting~\cite{Xu:2023:magicalbrush}. Similarly, template-based methods have been applied to dynamic shadow puppetry creation~\cite{Yao:2024:shadowmaker}. \revisedtext{These structured approaches provide a foundational framework for complex and profound cultural knowledge, allowing models to better understand with cultural content, such as the visual selection and composition in HarmonyCut, partially bridging gaps in expertise and cultural background. This makes it possible for public users to access relevant knowledge, participate in cultural creative design, and even further develop expertise, as supported by the feedback from E1–E3 agreeing with the cultural aspects in~\autoref{table3}, thereby engaging and supporting cultural communication while sustaining the vitality of cultural content such as ICH.}
% \rrtext{当然,结构化的方法可以更好的辅助模型理解cultural knowledge,但对于large-scale的knowledge如何转变为structured,本文主要通过mannual annotation with fine-tuning model的方式,还是无法完美的保证annotation comprehensively cover 所有知识,这也在section limitation中被承认。}
\rrtext{While our structured approach, which relies on manual annotation and model fine-tuning, enhances the model's ability to understand cultural knowledge, it remains limited in achieving comprehensive coverage of large-scale knowledge, as further discussed in~\autoref{lf}.}
% 尽管对应思维完成主题与形式的匹配是它的最基础首要能力,但也需要跟随着新事物的出现,以及新内涵的赋予,将他们用于创作中。 
%剪纸作为一个很靠经验的 民俗艺术,很多都是为了传达吉祥寓意,其首要考量的就是视觉形式与构图是否能够体现作者巧思且满足祝福他人的目的, 且被多个literature强调有关选择这一过程体现的对应思维的重要性。因此,对于剪纸的domain expertise,其首要考量的就是能否选择合适的内容,以及巧妙地构图。当然,剪纸的知识以及所能掌握的专业程度远不止于此,需要长年的学习和体验,才能将在对应思维的基础上更好的融入自己的风格,既能通过丰厚的知识继承过去的内容,也能有把握将新事物以剪纸形式表现进行创新。

\subsection{Limitations and Future Work}\label{lf}
Our work has several limitations that future work can address.
First, the system currently does not support collaborative creation, although prior work has demonstrated that sharing mood boards~\cite{Koch:2019:mayai, Koch:2020:semanticcollage} can enhance the design process. Future work could explore GenAI-aided creativity in multi-user collaboration by implementing features that facilitate cooperative design.
\revisedtext{Second, despite efforts to recruit users with varying levels of familiarity with GenAI, the limited sample size in both formative and user studies resulted in insufficient diversity, affecting the generalizability of the design process. To enhance the system's adaptability, we plan to expand the scale of formative studies to validate and refine the design space and broaden user studies to include participants with more diverse backgrounds and expertise levels. This will help investigate how GenAI expertise influences the use of Creative Support Tools in design.
While the current design space is relatively comprehensive according to the evaluation results and feedback, as a folk art with a vast and profound nature, paper-cutting requires further expansion of labeled data to support a more exhaustive reference list. To ensure broader applicability of the design space, we will continue to update and expand the dataset to support more extensive and in-depth research into paper-cutting art.}
% 我们目前的系统可以允许用户完成从需求,到构思再到完成构图最后设计出剪纸作品, 但最为一个handicraft art, 我们可以从 fabrication 探究从design 到 creation的角度将完整的 paper-cutting creating 过程借助工具完成
Additionally, the prototype system supports only 2D monochrome paper-cutting, which is the most prevalent form of Chinese paper-cutting. However, it does not accommodate multicolored paper-cutting.
Our current system allows users to transition from their initial intent through ideation to finalizing the composition, ultimately resulting in the design of a paper-cutting piece. Creators can use this design directly as a guide to complete the final creation with knives or scissors. This advancement provides an opportunity to further explore the fabrication aspect, focusing on how tools can support the complete paper-cutting creation process, from design to physical realization.


\section{Conclusion
\draftStatus{not great, but barring audits, R+R DONE}
}

Our main findings are as follows.
First, the text describing cases found on \tosdr{} is understandable to laypersons overall, though several cases received low understandability and low consensus on understandability.
Meanwhile, prior work has consistently shown people's confusion and inability to understand whole policies.
This suggests that this text can be explanatory for approaches that use such a taxonomy to classify text.
Second, severity scores seemed to indicate that the cases systematically favor the Service Provider.
Third, many cases had low consensus on severity, which could be a reflection of natural variation in attitudes.
Our results suggest that rewriting those cases may help, but educational interventions about such concepts may help \textit{more}.
Fourth, we observe an interplay between severity and understandability ratings that indicate the need for further study.
By enhancing the clarity and accessibility of privacy terms, organizations have the opportunity to cultivate a more knowledgeable user community, resulting in increased trust and user empowerment in managing privacy-related issues.

 
\balance
%%
%% The next two lines define the bibliography style to be used, and
%% the bibliography file.
\bibliographystyle{ACM-Reference-Format}
\bibliography{reference}
%%
%% If your work has an appendix, this is the place to put it.
\clearpage
\appendix
\section{\revisedtext{Formative Study}}\label{A:formative participants}

% \rrtext{正如我们在section 3.1中提到的,区分professional的重要标准是有没有接受系统性的训练(更详细的说,即向传承人拜师进行学习),或是经过足够的剪纸经验累积然后从业余变为professional。另外,在我们的采访中,P2-P4都强调他们甚至在3、4岁就自己有剪纸的兴趣,因此家人才请求传承人收这些孩童为徒,同时,收徒也证明师傅认可了他们的天赋。 因此,他们在5-6岁时就已经开始接受系统且专业的训练,而这种幼龄学童在民间剪纸传承中较为常见。 也正因为现在越来越多的new generation对此不感兴趣,所以ICH的传承收到挑战。}
\subsection{\rrtext{Expertise Levels of Paper-cutting}}\label{expert clarification}
\rrtext{As mentioned in~\autoref{A:formative participants}, the key criterion for distinguishing experts (i.e., master and practitioner) is whether they have undergone systematic training (specifically, apprenticing under an inheritor) or have transitioned from amateur to experts status through the accumulation of substantial paper-cutting experience and even been finally recognized as an inheritor. In our interviews, P1-P4 highlighted that they developed an interest in paper-cutting as early as 3-4 years old. Their families subsequently petitioned inheritors to accept these children as apprentices, and the acceptance of such requests indicated the masters' recognition of their talent. As a result, they began receiving systematic and professional training by the age of 5-9. This practice of training young children is relatively common in the Chinese paper-cutting inheritance. However, because this tradition relies on early engagement, the decreasing interest among the younger generation has partially contributed to the challenges faced in the inheritance of ICH, including paper-cutting.}

\begin{table}[H]
\caption{\revisedtext{Summary of participants interviewed in formative study.}}
  \Description{This table demonstrates the summary of participants interviewed in the formative study.}
  \label{table:formative participants}
\resizebox{0.49\textwidth}{!}{
\renewcommand\arraystretch{1.4}
\begin{tabular}{ccccccc}
\hline
ID & Sex    & Age & Paper-cutting Expertise & GenAI Expertise     & Location      & Platform \\ \hline
P1 & Male   & 30  & Master (21 years)       & Novice             & Central China & Bilibili \\
P2 & Female & 28  & Practitioner (18 years) & Knowledgeable User             & East China    & Bilibili \\
P3 & Female & 24  & Practitioner (18 years) & Novice             & Southwest     & Bilibili \\
P4 & Male   & 49  & Master (40 years)       & Novice             & Northwest     & Douyin   \\
P5 & Female & 59  & Master (40+ years)      & Novice             & Northeast     & Douyin   \\
P6 & Male   & 25  & Novice                  & Professional       & North China   & WeChat   \\
P7 & Male   & 26  & Amateur (3 years)       & Knowledgeable User & Central China & WeChat   \\ \hline
\end{tabular}
}
\end{table}

\section{Content Analysis}

\subsection{\revisedtext{Information of 140 Sampled Images}}\label{A:sample filter}
\revisedtext{As a regionally influenced art form, we utilized the regional distribution as the standard for the selection of 140 paper-cuttings, as illustrated in~\autoref{figure:sample filter}.}
\begin{figure}[H]
\centering
\includegraphics[width=0.49\textwidth]{Images/slected_140_distribution.png}
\caption{\label{figure:sample filter} The human geography region distribution between 140 sampled and total 701 paper-cuttings.}
\Description{The distribution between total 701 and 140 sampled paper-cuttings. This figure compared the human geography region distribution of randomly sampled subset and total 701 papaer-cuttings.}
\end{figure}

\subsection{\rrtext{Description of Expert Discussion in the Content Analysis}}
\subsubsection{\rrtext{For the Codebook of Core Factors of Paper-cutting Design Ideation}}\label{A:expert discuss ideation}
% 第一轮只基于style 和 function,因此提出应该还考虑题材还有method of expression,第二轮认为植物类和动物类还有瑞兽类都可以归纳为动植物;第三轮认为目前的版本基本合理
\rrtext{The initial version of the codebook was developed solely based on style and function. Thus, in the first round of discussions, the experts recommended adding two dimensions, including subject matter (P1-P2 and P4-P5) and method of expression (P2 and P4-P5), which were accepted through all experts' consensus. During the second round, P2 and P5 suggested merging specific types, such as plants, animals, and mythical creatures into a broader ``flora and fauna'' type. By the third round of discussions, the experts collectively agreed that the current version of the codebook was generally adequate.}

\subsubsection{\rrtext{For the Codebook of Patterns in Paper-cutting}}\label{A:expert discuss pattern}
% 在第一轮讨论中,专家们认为基于unit geometry pattern, 专家认为还有很多虽然也是几何图形,但它们的语义并不是抽象的,而是更practical的,例如月牙纹,云纹,火纹等,故总结出unit semantic pattern。而对于composite pattern,专家认为与decorative pattern对应的就是要实现function,表达核心主题的主体pattern,因此composite pattern应该分为两类。第二轮认为目前的版本基本合理。
\rrtext{During the first round of discussions, the experts observed that, in addition to unit geometry and sawtooth patterns, many \textit{Unit Patterns}, such as crescent, cloud, and fire patterns, possess more practical semantics rather than purely abstract ones. As a result, they recommended introducing a new sub-category within \textit{Unit Patterns}, termed unit semantic patterns. Regarding \textit{Composite Patterns}, the experts suggested distinguishing a type that emphasizes functionality and conveys core themes, referred to as composite primary patterns, as opposed to decorative patterns. Following the second round of discussions, the experts collectively agreed that the current version of the codebook was generally appropriate.}

\subsection{\revisedtext{Coding Results of the Ideation Factors of Paper-cuttings}}\label{A:paper-cut coding}
\revisedtext{The first author validated that the taxonomy of ideation factors derived from the analysis of 70 paper-cuttings represented the entire set of 140 paper-cuttings in~\autoref{figure:paper-cut validation}.}
\begin{figure}[H]
\centering
\includegraphics[width=0.49\textwidth]{Images/paper-cut-val.png}
\caption{\label{figure:paper-cut validation}The coding distribution results of Factors and Types for 70 selections and results in 140 paper-cuttings. Types with less than 5\% are not presented with specific percentages. The differences between the corresponding Types are less than 5\% for all Factors, indicating a similar distribution across both datasets.}
\Description{This figure shows coding distribution results of Factors and Types for 70 selections and results in 140 paper-cuttings, which validate the consistency of coding from the construction to the result.}
\end{figure}

\subsection{Example of Paper-cuttings with Different Type under Design Factors}\label{A:content examples}
\revisedtext{A sequential presentation of paper-cutting examples is provided in~\autoref{a1fig1} and~\autoref{a1fig2}, with each example illustrating the corresponding ideation factor types.}
\begin{figure}[H]
\centering
\includegraphics[width=0.49\textwidth]{Images/papercut/A-figure1.pdf}
\caption{\label{a1fig1} Paper-cutting examples that meet design factors. (a) Witchcraft Belief; (b) Indigenous Belief; (c) Religious Belief; (d) Cultural Dissemination; (e) Interpersonal Communication; (f) Festive Atmosphere Evoking; (g) Daily Decoration; (h) Primitive Paper-cutting;  (i) Flora and Fauna; (j) Landscape.}
\Description{This figure shows the paper-cutting examples that meet design factors. (a) Witchcraft Belief; (b) Indigenous Belief; (c) Religious Belief; (d) Cultural Dissemination; (e) Interpersonal Communication; (f) Festive Atmosphere Evoking; (g) Daily Decoration; (h) Primitive Paper-cutting;  (i) Flora and Fauna; (j) Landscape.}
\end{figure}

\begin{figure}[H]
\centering
\includegraphics[width=0.49\textwidth]{Images/papercut/A-figure2.pdf}
\caption{\label{a1fig2} Paper-cutting examples that meet design factors (a) Historical Figure and Story; (b) Folk Life; (c) Contemporary Subject; (d) Abstract Style; (e) Realistic Style; (f) Metaphor; (g) Symbolism; (h) Homophony.}
\Description{This figure shows the paper-cutting examples that meet design factors (a) Historical Figure and Story; (b) Folk Life; (c) Contemporary Subject; (d) Abstract Style; (e) Realistic Style; (f) Metaphor; (g) Symbolism; (h) Homophony.}
\end{figure}

\subsection{\revisedtext{Coding Results of the Patterns of Paper-cuttings}}\label{A:pattern coding}
\revisedtext{The first author validated the taxonomy of patterns, including the \textit{Unit Pattern}, derived from the analysis of 635 cut-outs representing the entire set of 1269 cut-outs, and the \textit{Composite Pattern}, derived from the analysis of all composite patterns in 70 paper-cuttings, representing the entire set of 140 paper-cuttings, as shown in~\autoref{figure:pattern validation}.}
\begin{figure}[H]
\centering
\includegraphics[width=0.49\textwidth]{Images/pattern-val.png}
\caption{\label{figure:pattern validation}The coding distribution results: 635 selections and 1269 cut-outs for the Unit Pattern, 70 selections and 140 paper-cuttings for the Composite Pattern. Each sub-category with less than 5\% is not presented with specific percentages. The differences between the corresponding sub-categories are less than 5\% for all Factors, indicating a similar distribution across both datasets.}
\Description{This figure shows coding distribution results. The coding distribution results: 635 selections and 1269 cut-outs for the Unit Pattern, 70 selections and 140 paper-cuttings for the Composite Pattern, which validate the consistency of coding from the construction to the result.}
\end{figure}


\section{\rrtext{Evaluation}}
\begin{table}[H]
\caption{\rrtext{Summary of participants in the user study and the expert interview.}}
  \Description{This table demonstrates the summary of participants in the user study and the expert interview.}
  \label{table:evaluation participants}
\resizebox{0.49\textwidth}{!}{
\renewcommand\arraystretch{1.4}
\begin{tabular}{ccccccc}
\hline
ID  & Sex    & Age & Paper-cutting Expertise & GenAI Expertise    & Location      & Platform \\ \hline
P1  & Male   & 22  & Novice                  & Knowledgeable User & North China   & WeChat   \\
P2  & Female & 28  & Practitioner (18 years) & Knowledgeable User & East China    & Bilibili \\
P3  & Male   & 24  & Novice                  & Knowledgeable User & East China    & WeChat   \\
P4  & Male   & 22  & Novice                  & Knowledgeable User & East China    & WeChat   \\
P5  & Male   & 26  & Amateur (3 years)       & Knowledgeable User & Central China & WeChat   \\
P6  & Female & 25  & Novice                  & Knowledgeable User & Southwest     & WeChat   \\
P7  & Male   & 49  & Master (40 years)       & Novice             & Northwest     & Douyin   \\
P8  & Female & 24  & Practitioner (18 years) & Novice             & Southwest     & Bilibili \\
P9  & Male   & 19  & Novice                  & Knowledgeable User & Southwest     & WeChat   \\
P10 & Male   & 19  & Novice                  & Knowledgeable User & South China   & WeChat   \\
P11 & Male   & 23  & Novice                  & Novice             & East China    & WeChat   \\
P12 & Male   & 18  & Novice                  & Novice             & North China   & WeChat   \\
P13 & Female & 24  & Novice                  & Knowledgeable User & South China   & WeChat   \\
P14 & Female & 18  & Amateur (1 year)        & Knowledgeable User & South China   & WeChat   \\
P15 & Female & 19  & Amateur (2 years)       & Knowledgeable User & Central China & WeChat   \\
P16 & Male   & 23  & Novice                  & Knowledgeable User & Southwest     & WeChat   \\ \hline
E1  & Male   & 30  & Master (21 years)       & Novice             & Central China & Bilibili \\
E2  & Female & 59  & Master (40+ years)      & Novice             & Northeast     & Douyin   \\
E3  & Female & 40  & Master (24 years)       & Novice             & Central China & Douyin   \\ \hline
\end{tabular}
}
\end{table}


\end{document}
\endinput
%%
%% End of file `sample-sigconf-authordraft.tex'.
