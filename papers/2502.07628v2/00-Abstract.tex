\begin{abstract}
% Chinese paper-cutting is a representative of hollow-out art within Intangible Cultural Heritage, which currently faces multiple challenges of creation and inheritance.
% , has developed rich cultural significance and diverse functions through historical and regional evolution. 
% However, it currently faces multiple challenges of creation and inheritance.
% due to a limited understanding and access to these complex cultural elements among the public and even among inheritors.
% the cultural content filtration brought about by the changing times and the rise of realistic styles have led to a loss of cultural elements, resulting in a limited understanding and access to these complex cultural elements among the public and even among inheritors. 
% Additionally, 
% While digital technologies like generative AI enhance creative efficiency in paper-cutting, they often cause a disconnect between visual form and cultural meaning, making it difficult to achieve the unity of form and spirit. 
%Since they are unclear of which dimension needs to be considered and which object in paper-cutting match their idea from cultural and aesthetic aspects in paper-cutting creation, just use some normal object and description to model. To address this gap,
% We conduct a formative study (N=5) to address the requirement of paper-cutting creation assisted by generative AI. It reveals that users need guidance from 4 dimensions (function, theme, style, and expression method) and 1 symbol (pattern), want support for related knowledge recommendations while still having room for own creativity. 
% We identify two taxonomies for the creation and pattern of paper-cutting and propose \wang{SYSTEM}, a system with generative AI pipelines that guides users to explore recommended elements and knowledge, and allows the configuration and adjustment of the result.
% Our user study (N=?) showed that \wang{SYSTEM} helps user access wide cultural knowledge matching the paper-cutting creation requirement, ultimately helping users produce more paper-cutting design ideas and reduce the hallucinated paper-cutting from the generative AI. 

% Chinese paper-cutting, as an Intangible Cultural Heritage (ICH), faces challenges from the erosion of traditional culture due to the prevalence of realism alongside limited public access to cultural elements. While the rise of generative AI enhances paper-cutting design through its extensive knowledge base and efficient production capabilities, it often fails to align form with cultural meaning due to users' and models' lack of specific paper-cutting knowledge. To address these issues, we conduct a formative study (N=7) to identify the design space and workflow of paper-cutting design, encompassing four core factors (function, subject matter, style, and expression method) and a key element (pattern). Then, we develop HarmonyCut, a generative AI-based tool that translates abstract intentions into creative and structured requirements. It guides related content (knowledge, works, and patterns) generation and recommendation, enabling users to select, combine, and adjust elements within the design space for creative paper-cutting design. A user study (N=8) demonstrated that HarmonyCut effectively provided relevant knowledge, ultimately helping the ideation of diverse paper-cutting designs and maintaining the quality of design work within the design space framework to ensure alignment between content and cultural connotation.
Chinese paper-cutting, an Intangible Cultural Heritage (ICH), faces challenges from the erosion of traditional culture due to the prevalence of realism alongside limited public access to cultural elements. While generative AI can enhance paper-cutting design with its extensive knowledge base and efficient production capabilities, it often struggles to align content with cultural meaning due to users' and models' lack of comprehensive paper-cutting knowledge. To address these issues, we conducted a formative study (N=7) to identify the workflow and design space, including four core factors (Function, Subject Matter, Style, and Method of Expression) and a key element (Pattern). We then developed HarmonyCut, a generative AI-based tool that translates abstract intentions into creative and structured ideas. This tool facilitates the exploration of suggested related content (knowledge, works, and patterns), enabling users to select, combine, and adjust elements for creative paper-cutting design. A user study (N=16) and an expert evaluation (N=3) demonstrated that HarmonyCut effectively provided relevant knowledge, aiding the ideation of diverse paper-cutting designs and maintaining design quality within the design space to ensure alignment between form and cultural connotation. 

%Based on the feedback of evaluation, we discussed some design implications that can be extended to support other traditional art designing.
\end{abstract}