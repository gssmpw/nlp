\begin{table*}[!htbp]
\caption{Definition and examples of factors and types in paper-cutting ideation derived from content analysis.}
  \Description{This table demonstrates the definition and examples of factors and types in paper-cutting ideation derived from content analysis.}
  \label{table1}
\Large
\renewcommand\arraystretch{1.7}
\resizebox{\textwidth}{!}{
\begin{tabular}{l|l|l|p{8.cm}|l}
\hline
\multicolumn{1}{c|}{\textbf{Category}}                                          & \multicolumn{1}{c|}{\textbf{Subcategories}} & \multicolumn{1}{c|}{\textbf{Type}} & \multicolumn{1}{c|}{\textbf{Definition}}                                          & \multicolumn{1}{c}{\textbf{Examples (Figures)}} \\ \hline
\multirow{7}{*}{Function}                                                       & \multirow{4}{*}{Spiritual Function}         & Witchcraft Belief                     & Serve as a symbol in witchcraft activities, embodying related beliefs and rituals    &  Wizard Exorcising Demons~(\autoref{a1fig1}(a))  \\ \cline{3-5} 
                                                                                &                                             & Indigenous Belief                  & Reflect unique local belief systems as a form of cultural expression              & Herding Ducks in Watertown~(\autoref{a1fig1}(b))  \\ \cline{3-5} 
                                                                                &                                             & Religious Belief                    & Act as a symbol in religious ceremonies or doctrines, conveying religious content & Guanyin Sitting on a Lotus~(\autoref{a1fig1}(c))  \\ \cline{3-5} 
                                                                                &                                             & Cultural Dissemination                & Serve as an medium to disseminate culture and historical information              & Happy Asian Games~(\autoref{a1fig1}(d))  \\ \cline{2-5} 
                                                                                & \multirow{3}{*}{Practical Function}         & Interpersonal Communication        & Act as a medium in social etiquette settings in interpersonal communication       & Mandarin Ducks in Water~(\autoref{a1fig1}(e))  \\ \cline{3-5} 
                                                                                &                                             & Festive Atmosphere Evoking        & Enhance the atmosphere and cultural features in holiday or seasonal celebrations  & Solar Term: Grain Full~(\autoref{a1fig1}(f))  \\ \cline{3-5} 
                                                                                &                                             & Daily Decoration                   & Serve as decorative items in daily life                                           & Butterfly Window Decoration~(\autoref{a1fig1}(g))  \\ \hline
\multirow{6}{*}{Subject Matter}                                                        & \multirow{5}{*}{Traditional Subject Matter}        & Primitive Paper-cutting            & Present paper-cutting by initial form in history                               & Circular Floral Paper-cutting~(\autoref{a1fig1}(h))  \\ \cline{3-5} 
                                                                                &                                             & Flora and Fauna                    & Present paper-cutting by animals and plants                                       & The World Welcomes Spring~(\autoref{a1fig1}(i))  \\ \cline{3-5} 
                                                                                &                                             & Landscape                         & Present paper-cutting about natural and cultural landscapes                       & Lijiang Ancient Town~(\autoref{a1fig1}(j))  \\ \cline{3-5} 
                                                                                &                                             & Historical Figure and Story                    & Present paper-cutting centered around stories of characters                       & Jiang Ziya Fishing~(\autoref{a1fig2}(a))  \\ \cline{3-5} 
                                                                                &                                             & Folk Life                & Present paper-cutting about traditional customs and culture                      & The Mouse's Wedding~(\autoref{a1fig2}(b))  \\ \cline{2-5} 
                                                                                & Innovative Subject Matter                         & Contemporary Subject               & Paper-cutting integrated with modern subjects                                     & Genshin Impact, Klee~(\autoref{a1fig2}(c))  \\ \hline
\multirow{2}{*}{Style}                                                          & Abstract Style                              & -                                  & Express ideas with non-representational forms by paper-cutting                   & Frog,~\autoref{a1fig2}(d)  \\ \cline{2-5} 
                                                                                & Realistic Style                             & -                                  & Replicate real objects and scenes by paper-cutting                                & Lujiazui, Shanghai~(\autoref{a1fig2}(e))  \\ \hline
\multirow{3}{*}{\begin{tabular}[c]{@{}l@{}}Method of\\ Expression\end{tabular}} & Metaphor                                    & -                                  & Use similar or related content to indirectly express meaning or emotion         & Harmonious of Ethnicities~(\autoref{a1fig2}(f))  \\ \cline{2-5} 
                                                                                & Symbolism                                   & -                                  & Use specific content to represent theme                                            & Enduring Lineage~(\autoref{a1fig2}(g))  \\ \cline{2-5} 
                                                                                & Homophony                                   & -                                  & Use similarity of pronunciation to embed positive wishes into specific objects    & Happiness Arrives~(\autoref{a1fig2}(h))  \\ \hline
\end{tabular}
}
\end{table*}


\section{Content Analysis of Paper-cuttings and Patterns}\label{sec:content}
Based on the formative study, we discovered that there are some key aspects to consider in the design workflow. To identify them, we first collected a paper-cutting dataset. We conducted two content analyses: one to develop an ideation factors taxonomy and another for a pattern taxonomy (taxonomy for the key element in ideation and composition). \rrtext{Both analyses were carried out under the guidance of expert review, involving five experts (P1–P5) who had also participated in the formative study. The experts were involved in two stages: (1) During the coding process, ambiguities in the instance-level annotation of paper-cuttings and patterns were resolved through a single expert-guided discussion conducted via the WeChat group, with the final annotation determined based on the majority vote of all experts; (2) Additionally, after each version of the codebook was drafted by the authors, the experts participated in discussions conducted via online conferences. In each round, the experts collaboratively reviewed and evaluated the current version of the codebook, offering refined or expanded suggestions, which were systematically discussed and consolidated to resolve ambiguities and ensure alignment of perspectives. This iterative process continued until all experts reached a consensus on the coding results, ensuring that the final codebook was validated.} 

% \begin{table*}[!htbp]
% \caption{Definition and examples of pattern categories and subcategories derived from content analysis.}
%   \Description{This table demonstrates the definition and examples of pattern categories and subcategories derived from content analysis.}
%   \label{table2}
% \resizebox{\textwidth}{!}{
% \begin{tabular}{c|c|p{8.1cm}|cc}
% \hline
% \textbf{Category}                            & \textbf{Subcategory}     & \makecell*[c]{\textbf{Definition}} & \multicolumn{2}{c}{\textbf{Examples}} \\ \hline
% \multirow{3}{*}{Unit Pattern~\cite{Hu:2021:Traditionalpattern, Zhuge:1998:patterndictionary}}       & Geometric Unit Pattern       & \makecell*[c]{Independent units based on abstract geometric\\ glyphs that can be integrated with other patterns \\in paper-cuttings~\cite{Tian:2003:Historypattern, Liu:2009:rai, Liu:2020:intcut, Li:2020:aug}}     &               \includegraphics[totalheight=1.5cm, keepaspectratio,valign=m,margin=0cm .05cm]{Images/pattern/circle.pdf} &  \includegraphics[totalheight=1.5cm, keepaspectratio,valign=m,margin=0cm .05cm]{Images/pattern/triangle.pdf}           \\ \cline{2-5} 
%                                     & Semantic Unit Pattern          & \makecell*[c]{Independent units based on specific semantic glyphs \\ that can be integrated with other patterns in paper-cuttings}     &             \includegraphics[totalheight=1.5cm, keepaspectratio,valign=m,margin=0cm .05cm]{Images/pattern/crescent.pdf} &  \includegraphics[totalheight=1.5cm, keepaspectratio,valign=m,margin=0cm .05cm]{Images/pattern/cloud.pdf}           \\ \cline{2-5} 
%                                     & Sawtooth Pattern   & \makecell*[c]{Specific independent units resembling sawtooth, \\formed by replicating various glyphs along\\ a trajectory for depicting contour and light gradient \\in paper-cuttings~\cite{Zhang:2018:sos, Liu:2020:intcut, Zhang:2006:cpc, Liu:2009:rai}}       &            \includegraphics[totalheight=1.5cm, keepaspectratio,valign=m,margin=0cm .05cm]{Images/pattern/sawtooth1.pdf} &  \includegraphics[totalheight=1.5cm, keepaspectratio,valign=m,margin=0cm .05cm]{Images/pattern/sawtooth2.pdf}             \\ \hline
% \multirow{2}{*}{Composite Pattern} & Primary Composite Pattern        & \makecell*[c]{Composite patterns made up of multiple unit patterns \\ that serve as the primary objects for \\ expressing themes in paper-cuttings}        &              \includegraphics[totalheight=1.5cm, keepaspectratio,valign=m,margin=0cm .05cm]{Images/pattern/dragon.pdf} &  \includegraphics[totalheight=1.5cm, keepaspectratio,valign=m,margin=0cm .05cm]{Images/pattern/horse.pdf}             \\ \cline{2-5} 
%                                     & Decorative Composite Pattern &  \makecell*[c]{Composite patterns made up of multiple \\unit patterns that decorate primary objects \\in paper-cuttings~\cite{Zhuge:1998:patterndictionary, Hu:2021:Traditionalpattern, Zhang:2005:cag}}       &             \includegraphics[totalheight=1.5cm, keepaspectratio,valign=m,margin=0cm .05cm]{Images/pattern/HUI.pdf} &  \includegraphics[totalheight=1.5cm, keepaspectratio,valign=m,margin=0cm .05cm]{Images/pattern/endlessknot.pdf}            \\ \hline
% \end{tabular}
% }
% \end{table*}

\begin{table*}[!htbp]
\caption{Definition and examples of pattern categories and subcategories derived from content analysis.}
  \Description{This table demonstrates the definition and examples of pattern categories and subcategories derived from content analysis.}
  \label{table2}
\resizebox{\textwidth}{!}{
\begin{tabular}{c|c|p{8.1cm}|cc}
\hline
\textbf{Category}                            & \textbf{Subcategory}     & \makecell*[c]{\textbf{Definition}} & \multicolumn{2}{c}{\textbf{Examples}} \\ \hline
\multirow{3}{*}{Unit Pattern~\cite{Hu:2021:Traditionalpattern, Zhuge:1998:patterndictionary}}       & Geometric Unit Pattern       & \makecell*[c]{Independent units based on abstract geometric\\ glyphs that can be integrated with other patterns \\in paper-cuttings~\cite{Tian:2003:Historypattern, Liu:2009:rai, Liu:2020:intcut, Li:2020:aug}}     &               \raisebox{-.39\dimexpr\totalheight-\ht\strutbox}{\includegraphics[scale=0.69]{Images/pattern/circle.pdf}} &  \raisebox{-.4\dimexpr\totalheight-\ht\strutbox}{\includegraphics[scale=0.69]{Images/pattern/triangle.pdf}}           \\ \cline{2-5} 
                                    & Semantic Unit Pattern          & \makecell*[c]{Independent units based on specific semantic glyphs \\ that can be integrated with other patterns in paper-cuttings}     &             \raisebox{-.4\dimexpr\totalheight-\ht\strutbox}{\includegraphics[scale=0.69]{Images/pattern/crescent.pdf}} &  \raisebox{-.43\dimexpr\totalheight-\ht\strutbox}{\includegraphics[scale=0.69]{Images/pattern/cloud.pdf}}           \\ \cline{2-5} 
                                    & Sawtooth Pattern   & \makecell*[c]{Specific independent units resembling sawtooth, \\formed by replicating various glyphs along\\ a trajectory for depicting contour and light gradient \\in paper-cuttings~\cite{Zhang:2018:sos, Liu:2020:intcut, Zhang:2006:cpc, Liu:2009:rai}}       &            \raisebox{-.51\dimexpr\totalheight-\ht\strutbox}{\includegraphics[scale=0.72]{Images/pattern/sawtooth1.pdf}} &  \raisebox{-.51\dimexpr\totalheight-\ht\strutbox}{\includegraphics[scale=0.72]{Images/pattern/sawtooth2.pdf}}             \\ \hline
\multirow{2}{*}{Composite Pattern} & Primary Composite Pattern        & \makecell*[c]{Composite patterns made up of multiple unit patterns \\ that serve as the primary objects for \\ expressing themes in paper-cuttings}        &              \raisebox{-.5\dimexpr\totalheight-\ht\strutbox}{\includegraphics[scale=0.66]{Images/pattern/dragon.pdf}} &  \raisebox{-.5\dimexpr\totalheight-\ht\strutbox}{\includegraphics[scale=0.66]{Images/pattern/horse.pdf}}             \\ \cline{2-5} 
                                    & Decorative Composite Pattern &  \makecell*[c]{Composite patterns made up of multiple \\unit patterns that decorate primary objects \\in paper-cuttings~\cite{Zhuge:1998:patterndictionary, Hu:2021:Traditionalpattern, Zhang:2005:cag}}       &             \raisebox{-.51\dimexpr\totalheight-\ht\strutbox}{\includegraphics[scale=0.66]{Images/pattern/HUI.pdf}} &  \raisebox{-.43\dimexpr\totalheight-\ht\strutbox}{\includegraphics[scale=0.66]{Images/pattern/endlessknot.pdf}}            \\ \hline
\end{tabular}
}
\end{table*}
\subsection{Data Collection}\label{sec:data}
\revisedtext{We collected over 17,000 paper-cuttings through data crawling from the Chinese Paper Cutting Digital Space\footnote{\url{https://www.papercutspace.cn/}}, which organized and merged paper-cuttings into seven distinct categories based on the human geography regions in China~\cite{Fang:2017:chinahumangeo} (i.e., Central China, East China, North China, Northeast, Northwest, South China, and Southwest).} We then selected 1,521 paper-cuttings with titles from the source to aid in identifying the content, as some abstract works are challenging to interpret and annotate in content analysis. We further filtered out low-definition images and multi-color paper-cuttings, as our focus is on traditional monochromatic Chinese paper-cutting. This process resulted in a collection of 701 paper-cutting images, which serve as the sampled source for content analysis in~\autoref{sec:4_2} and \autoref{sec:4_3}. Additionally, these images function as the retrieval dataset discussed in~\autoref{sec:harmonycut}. 
\revisedtext{For content analysis and fine-tuning models, we sampled 20\% images (140/701) based on the distribution of the seven regions to ensure alignment with the regional characteristics of paper-cutting, which is shown in~\autoref{figure:sample filter}.}

\subsection{Core Factors of Paper-cutting Design Ideation} \label{sec:4_2}
\revisedtext{We first collected all dimensions (i.e., function and style) considered by experts during the ideation phase of the formative study based on their feedback (\autoref{sec:formative}). To identify the ideation factors and the various types of content within each factor, the first author, in collaboration with five experts, conducted a content analysis using 140 selected paper-cuttings~(\autoref{sec:data}).
The first author randomly selected 70 (50\%) paper-cuttings to create taxonomy by open-coding. To ensure the quality of the taxonomy, two criteria were followed during the coding process~\cite{Nickerson:2013:Taxonomymethod}: all elements in the taxonomy should be comprehensive, covering every aspect, and the types within each factor should be mutually exclusive.}
In the beginning, the coder annotated detailed content across all dimensions observed or clarified through experts when content was unclear or ambiguous, resulting in the initial codebook. 
% For instance, 关于萨满教相关的剪纸,是东北的特色剪纸,作者在标注相关剪纸时,无法确定它们的fuction,经过single expert-guided discussion 以及投票,所有专家都认为只针对这一类剪纸,其巫术信仰的功能明显优先于本土信仰。
\rrtext{For instance, Northeast shamanistic paper-cuttings with the regional characteristic, were initially challenging to determine their function. Following the expert-guided discussion and voting, experts agreed their function in witchcraft beliefs takes precedence over indigenous beliefs.
Then, the coder produced the final codebook under expert review through three rounds of discussion and iterations (the detailed information of each round is shown in~\autoref{A:expert discuss ideation}), identifying 18 types of content grouped into 4 factors:} Function, Subject Matter, Style, and Method of Expression. The categorization results, including ideation factors, sub-factors, types, definitions, and example paper-cuttings, are detailed in~\autoref{table1}. The first author then applied the finalized codebook to code the remaining 70 paper-cuttings and validated the taxonomy, as shown in~\autoref{figure:paper-cut validation}.
Examples of each type are shown in~\autoref{A:content examples}. For the 140 paper-cuttings, each is annotated type and has specific explanations for every factor, such as category \textit{Subject Matter}, type \textit{Historical Figure and Story}, detailed information ``\textit{A paper-cutting narrates the historical story of Zhaojun Wang's journey to the Xiongnu for a political marriage.}'' These 140 pieces are used as domain knowledge and ground-truth datasets in~\autoref{sec:harmonycut}.
% For each piece, the first author coded according to the various dimensions proposed by experts, identifying multiple types within each dimension. 
% Through careful comparison and merging of similar types and dimensions, we distilled the ideation process into four core factors. Each merging operation was based on discussion with experts (P1-P5) and feedback from their reviews, ultimately constructing a taxonomy (\autoref{table1}) of the paper-cutting ideation factor, resulting in 18 identified types grouped into 4 factors (Function, Subject Matter, Style, and Method of Expression). 
% In constructing this taxonomy, we analyzed 70 paper-cuttings (50\%) to ensure comprehensive coverage. The first author annotated the remaining 70 paper-cuttings to validate and refine the taxonomy.
% Examples of each type are shown in~\autoref{A:content examples}. For the 140 paper-cuttings, each is annotated type and has specific explanations for every factor, such as category \textit{Subject Matter}, type \textit{Historical Figure and Story}, detail information ``\textit{A paper-cutting narrates the historical story of Zhaojun Wang's journey to the Xiongnu for a political marriage.}'' These 140 pieces are used as domain knowledge and training datasets in \autoref{sec:harmonycut}

% 我们总结了专家在formative study中提到的有关在ideation中需要考虑的几个方面,并从17000+作品,并根据地区分别从7大地区挑选了N个剪纸作品,并进行coding,对每一个作品,都从每一个专家提到的aspect来进行编码,得到对于每一aspect的多个type,最终将类似的type、aspect, merge。得到了构思中的四要素。每次是否merge,都会基于expert review来完成,最终构建taxonomy。

% 系统没有办法理解作为剪纸最核心的符号语言,纹样。
\subsection{Patterns in Paper-cutting}\label{sec:4_3}
% From an appearance perspective, paper-cuttings typically have a single color and express artistry through those two-dimensional glyphs. The outer contour of the paper-cutting can represent the shape of the object being depicted, while the internal two-dimensional glyphs may either specifically portray a certain object or serve a decorative and express meaning function. 
% From an appearance perspective, monochromatic paper-cutting relies on two-dimensional glyphs for artistic expression. The outer contour defines the shape of the depicted object, while the internal glyphs serve semantic or decorative roles by highlighting specific elements. 
\revisedtext{From a semiotic perspective, Chinese paper-cutting consists of a series of symbolized patterns~\cite{Liang:2011:Symbolpattern}. Saussure's theory~\cite{Saussure:1916:Semiotics} defines a symbol as comprising two parts: the ``signifier'' and the ``signified.'' In the context of paper-cutting, the ``signifier'' refers to the form or structure of the pattern, while the ``signified'' represents the meaning or concept it conveys~\cite{Cao:2009:Semiotics, Liang:2011:Symbolpattern}. 
% This also aligns with Husserl's distinction between "object" and "meaning"~\cite{}. 
Therefore, achieving harmony between form and connotation in paper-cutting design requires a comprehensive understanding of patterns that unify meaning and content. To support this, we developed a taxonomy of these patterns using a hybrid thematic analysis approach, also using those 140 selected paper-cuttings~(\autoref{sec:data}).}
% Thus, to identify the combination of these basic elements (pattern) of paper-cutting and create a taxonomy of patterns, we employed a hybrid thematic analysis approach, using the taxonomy as a codebook to analyze and annotate 140 sample paper-cuttings from the pattern aspect. Based on the literature review, we identified a unique sawtooth pattern~\cite{Zhang:2018:sos} and compiled a range of traditional patterns~\cite{Zhang:2018:sos, Huang:2021:chinesepattern, Wang:2009:folkpattern, Zhao:2023:papercutculture}. 

\revisedtext{In the deductive coding process, previous research found three important aspects of pattern and paper-cutting namely \textit{Geometric Pattern}~\cite{Tian:2003:Historypattern, Liu:2009:rai, Liu:2020:intcut, Li:2020:aug}, \textit{Decorative Pattern}~\cite{Zhuge:1998:patterndictionary, Hu:2021:Traditionalpattern, Zhang:2005:cag}, and \textit{Sawtooth Pattern}~\cite{Zhang:2018:sos, Liu:2020:intcut, Zhang:2006:cpc, Liu:2009:rai}. \textit{Geometric Pattern} was composed of shapes formed by the ultimate abstraction of points, lines, and planes. \textit{Decorative Pattern} was used to enhance or embellish specific patterns or the overall composition of a paper-cutting. \textit{Sawtooth Pattern} was a distinct and important pattern in paper-cutting, used to convey texture and layering in objects while also serving as decoration for various patterns. However, these three aspects partially overlap in both shape and function. To address this, we adopted the concept of \textit{Unit Pattern}~\cite{Hu:2021:Traditionalpattern, Zhuge:1998:patterndictionary}, representing the fundamental unit in paper-cutting, to ensure the codebook remained mutually exclusive. Thus, \textit{Geometric Pattern} and \textit{Sawtooth Pattern} were identified as sub-categories of \textit{Unit Pattern}, and \textit{Decorative Pattern} was defined as one of \textit{non-Unit Patterns} with decorative functions.}

\revisedtext{In the inductive coding process, OpenCV extracted 63,452 cut-outs (i.e., \textit{Unit Patterns}) from 140 paper-cuttings~(\autoref{sec:data}), and 1,269 cut-outs (2\%) were randomly sampled from them.}
The first author open-coded 635 cut-outs and 70 paper-cuttings, both randomly selected at 50\%, similar to the previous process in~\autoref{sec:4_2}, referring to pattern knowledge~\cite{Huang:2021:chinesepattern, Wang:2009:folkpattern, Zhao:2023:papercutculture, Zhuge:1998:patterndictionary}.
\rrtext{Ambiguities in instance-level annotation were resolved using the method outlined in~\autoref{A:paper-cut coding}. For example, the classification of the plum blossom pattern as either primary or decorative was challenging. After the single expert-guided discussion, all five experts agreed it is primarily decorative and rarely used as a main subject.}
% 当他们在annotation中对单一instance是什么有疑问,将在group中询问5位experts这个instance具体该归属哪个纹样?
% 例如梅花纹,first version中,author将它归为primary pattern,但经过第一轮讨论,5位experts都认为梅花纹更多是作为装饰作用的,而比较少作为主体。
\rrtext{Through two rounds of expert discussions and iterations (the detailed information of each round is shown in~\autoref{A:expert discuss pattern}), a taxonomy of patterns was established~(\autoref{table2}), consisting of two categories:} \textit{Unit Patterns} and \textit{Composite Patterns} (i.e, \textit{non-Unit Patterns}). In addition, two new subcategories were introduced: unit semantic patterns and composite primary patterns. 
\textit{Unit patterns} include 25 different patterns (8 geometric units, 12 semantic units, and 5 sawtooth patterns). \textit{Composite patterns}, which represent the primary content of paper-cuttings, are formed by combining unit patterns and include 42 different patterns (8 decorative composite patterns and 34 primary composite patterns). The first author annotated the remaining 70 paper-cuttings and 634 cut-outs, and validated taxonomy~(\autoref{figure:pattern validation}). 
\rrtext{All the names and examples of the 25 specific \textit{Unit patterns} and 42 specific \textit{Composite patterns} are provided in the supplementary material for detailed reference.}
These labeled patterns serve as both a repository of domain knowledge and ground truth datasets in \autoref{sec:harmonycut}.
% 先根据literature图鉴以及问专家来进行annotate, 标注方式就是把一个paper-cutting中存在的每一个pattern都标注出来, 然后得到语义 unit 和 主体 composite, 同时将 主体与装饰合并为复合。 整个过程有2-round discussion of 5 experts. final codebook 建好后, first author 再将剩余数据按照 codebook进行标注。
% The first author coded 70 (50\%) paper-cuttings to identify patterns, whether documented in the literature or not. An expert review (P1-P5) was conducted on the coding results to assess the rationality and potential merging of categories and to accurately assign each specific pattern to its appropriate category. Ultimately, we summarized the taxonomy into two categories (\autoref{table2}): unit patterns and composite patterns. Unit patterns, which are the basic cut-out components in paper-cutting, include 25 different patterns (8 geometric units, 12 semantic units, and 5 sawtooth patterns), while composite patterns serve as the content of paper-cutting and are formed by combining unit patterns, encompassing 42 different patterns (8 decorative composite patterns and 34 primary composite patterns). The first author also annotated the remaining 70 paper-cuttings to validate and refine the taxonomy. For the 140 paper-cuttings, each is annotated with all patterns in it. These labeled patterns are used as domain knowledge and training datasets in \autoref{sec:harmonycut}

% 作为从符号学的角度来看,剪纸是由一系列符号化纹样组合而成。纹样作为先强调,虽然不管是对于传统纹样,还是特别到剪纸纹样,虽然对很多内容都有定义,但很少有进行highlevel 归类的。literature review 得到 锯齿纹,其他的最高级subcategory是总结出来的。 其他的只是具体的每一纹样是什么可以从literature里得到:Our group factor taxonomy considered both existing literature on ICH and CHS for deductive coding and newly observed factors from ICH short videos for inductive coding.

% Taxonomy 基于目前有关传统纹样的分类,以及对纹样在剪纸中的分类的文献调研。并基于先前选择的N个作品,扩充pattern 分类。 【Our group factor taxonomy considered existing literature on paper-cutting factors for deductive coding and newly observed factors from collected paper-cutting for inductive coding.】

% 最后 这两个 taxonomy的每一项都先会和专家讨论,是否有缺漏或可以合并的。 然后基于最后的分类结果进行编码和标注
