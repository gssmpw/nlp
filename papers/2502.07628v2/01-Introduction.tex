\section{Introduction}
% 第一部分:
% 中国剪纸,作为非物质文化遗产中的代表 ,在历史与地域的演变过程中,积淀了丰富的文化内涵并展现了多样化的功能特性【根据参考文献进行扩展】。
% 然而,剪纸艺术的创作与传承在当代面临着多重挑战。一方面,随着时代的变迁,文化内容的筛选机制以及写实风格的盛行,导致部分传统文化元素在剪纸作品中逐渐缺失 【A,如,参考文献提及】;另一方面,由于普通民众、爱好者,甚至是传承人都无法全面地掌握剪纸所有文化元素,这进一步加剧了剪纸作品文化内涵的单一化趋势 【B,如,参考文献提及】。
\revisedtext{Paper-cutting, widely recognized as an artistic form across many countries, stands as a powerful medium for cultural exchange and a bridge for global connections. In the context of contemporary globalization, the preservation and inheritance of this traditional craft is crucial, as its distinctive form continues to foster cultural understanding and advance cross-cultural communication~\cite{Zhang:2000:internationalpapaercut, Shu:2023:cross-culturalpapercut}.}
Within this broad art form, Chinese paper-cutting plays a significant role, which is a representative of hollowed art within Intangible Cultural Heritage (ICH)~\cite{Zhang:2021:dot, Wang:2021:dap, ich2009unesco}. It has acquired rich cultural connotations and diverse functions through historical evolution and regional changes~\cite{Cao:2023:the, Ma:2010:sof, Cui:2016:sot}. 
Nowadays, paper-cutting applications have broadened to include various contemporary themes \cite{Cao:2023:the, Liu:2009:rai, Ma:2010:sof}. Despite these advancements and growing influence, it still faces challenges in the vitality of creation and inheritance.
Modern life influenced traditional elements in paper-cutting, leading to their marginalization when they lose relevance to contemporary social contexts~\cite{Wang:2021:dap, Zhang:2018:sos}. Concurrently, the growing trend towards realism in paper-cutting further contributed to the decline of historical and cultural elements~\cite{Cui:2016:sot, Zhang:2018:sos}. Furthermore, paper-cutting with scarce documentation~\cite{Ma:2010:sof, Hu:2017:tda, Huang:2012:ICHprotecting} limits the resources available to the public and even professional practitioners. It also hinders the understanding of the rich cultural knowledge inherent in paper-cutting. These issues exacerbate the homogenization of form and content~\cite{Zhang:2018:sos}, which affects its creativity and inheritance.

% 第二部分 
% 1. digital methods 古早的方法 Computer Graphic and Computer-Aided Design
% 但这些方法大多是 based on rule learning, 同时分割图形的效果也不好,无法真正达到多样化设计和创作:尤其是对于作品风格和纹样种类由于是基于几何图案规则生成而多样性不足,
With advances in digital technology, computer graphics (CG) show the potential to help with paper-cutting design. Several studies have utilized algorithms for the pattern generation~\cite{Li:2020:aug, Zhang:2006:cpc}, transfer~\cite{Liu:2009:rai, Meng:2010:apc}, and interactive combination for paper-cutting design~\cite{Liu:2018:pdf, Zhang:2005:cag}. However, most approaches depend on stylized pattern generation, which is typically irreversible, and the variation in patterns is often inadequate due to reliance on random seeds~\cite{Liu:2009:rai, Li:2020:aug, Meng:2010:apc}. Although interactive systems~\cite{Liu:2018:pdf, Hu:2017:tda, Zhang:2005:cag} offer some flexibility to enhance content diversity, they are frequently confined to predefined forms and styles, constraining the creative and cultural expression in paper-cutting.

% 2. 基于生成式AI的方法: 虽然
% 生成模型拥有较全面地知识库,并能高效创作。被广泛用于辅助各种设计、创作的工作上。 但在特定传统艺术的创作上,其可控性以及合理性上还存在问题
% 生成模型输出不可控,容易受训练数据和输入影响,反而又造成同质化,无法满足艺术创作的多样性。
%  文化相关内容和知识在生成式模型中的缺失,以及无法很好理解也会影响生成结果:此外,现有工具很少考虑创建任务背后的文化因素。文化因素来源于特定创造任务的领域知识,反映了创造中预期的心智模式。
% 纹样胡乱,模型难以全部学习
% 虽然genai可以通过描述生成
Generative AI (GenAI) for text and images (e.g., ChatGPT\footnote{\url{https://openai.com/chatgpt}}~\cite{chatgpt} and Stable Diffusion\footnote{\url{https://github.com/CompVis/stable-diffusion}}~\cite{Rombach:2022:stablediffusion}), with the comprehensive knowledge base and efficient production capabilities, becomes increasingly integral to the design process~\cite{Heyrani:2021:creativegan, Centinic:2022:aiartreview, Muller:2022:genaichi} to support creative work, such as writing~\cite{Reza:2024:abscribe, Yuan:2022:wordcraft}, painting~\cite{Xu:2023:magicalbrush, Fan:2024:contextcam, Xu:2024:fuzzypainting}, and handicraft design~\cite{Yao:2024:shadowmaker}. 
While GenAI provides promising solutions for artistic creation and design, its application to traditional art, such as paper-cutting, remains largely unexplored and presents specific challenges. GenAI is primarily influenced by source data and input, which can lead to homogeneous and biased outputs. This limitation restricts divergent ideation and fails to adequately manage control within the design process~\cite{Brown:2020:fewshotlearners, Amderson:2024:homogenization, Hou:2024:c2ideas, Xu:2023:magicalbrush, Wang:2024:roomdreaming}. Additionally, GenAI lacks domain-specific knowledge~\cite{Hou:2024:c2ideas, Lu:2023:humanstillwin} and understanding of the cultural context essential for paper-cutting, such as symbolic meanings of a theme across cultures, selecting and combining appropriate patterns to match design intents. These are critical factors of paper-cutting design that existing research has not fully studied.

% 第三部分 
% formative study with 7位专家来理解创作剪纸创作过程,并明确创作过程中需要考虑的核心要素,以及其中的挑战(来保证文化内涵). 我们发现,:
% A: 在进行创作构思ideation--- 如何帮助用户在创作过程中获得剪纸相关的知识。
% A.1)ideation:对于新手,它们缺乏对剪纸的基本知识,如创作流程和创作要素。 用户并不知道剪纸创作该从哪些方面考虑来,只有一个比较模糊的创作意图; 需要有明确的指引,最终作品才能涵盖更完善的剪纸创作的要素; (design goal)
% A.2)ideation: 而对于专家,虽然他们可以快速的明确创作流程和创作思路,但对于具体该选用哪些主题,内容和纹样来进行符合创作需求和文化内涵的作品,是具有挑战的。对文化内涵的考量,对于一个题材,具有文化内涵的剪纸作品才是更有意义的。在有一个明确的创作意图后,无论新手还是专家都难以尽可能范围广的找到符合需求且具有合适文化内涵的内容和纹样,来支持后面的创作  (design goal)
% B)creation 符合用户需求的剪纸的内容和纹样是多样的,且统一主题的内容和纹样呈现形式也是多样的, 模型的生成往往是存在bias的,对于某种事物的生成往往限于固定的几种模式;  在模型生成的剪纸内容中,有很多纹样是无意义、不合理,甚至只是一团胡乱的条纹。系统没有办法理解作为剪纸最核心的符号语言,纹样(design goal: 需要系统进行推荐,推荐结果可以作为参考,来指导用户基于它们完成自己的作品。在构思结束后,可以从推荐作品、或模型生成的结果中获取创作的主要对象们,并进行组合。最后根据推荐的适合这些对象的纹样,将其自由添加到作品中)
% C)creation 如果让模型直接通过深度生成模型从文本输入到生成整个图像,结果很大程度上由人工智能主导,用户可能缺乏参与感。挑战变成了人工智能辅助和人类参与之间的平衡;  (design goal: 用户可以从推荐或生成的内容中,自主选取创作的主要对象和纹样,同时支持控制和调整纹样数量,位置和大小)
% 在概念构思过程中,设计师从参考文献中提取了四种类型的元素——主题、动作和姿势、主题和情绪以及构图方面(安排)。然后,他们试图头脑风暴出更多与提取的元素相关的元素,并以几种方式将它们结合起来。
This research aims to explore the design workflow and the challenges involved in aligning visual form and connotation in paper-cutting design with GenAI assistance, and subsequently identify a design space for paper-cutting. To achieve this, we conducted a two-step formative study involving iterative collaboration with experts. This study included semi-structured interviews with seven participants from different backgrounds to understand their design challenges. It also included a content analysis of 140 paper-cuttings to investigate the pattern and ideation factors within paper-cutting. 

The formative study outlined the workflow (ideation and composition in \autoref{fig:teaser}) and identified challenges inherent in paper-cutting design. Novices face difficulties due to insufficient foundational knowledge, which hinders effective ideation. While experts are proficient in defining processes and ideas to meet requirements, they often rely on experience, leading to fixation. GenAI-aided design struggles with understanding the complex cultural knowledge inherent in paper-cutting. As a result, the outputs are often restricted to fixed or even erroneous output that fails to align with design requirements and cultural contexts. Users are overwhelmed by the inconsistent quality of results in exploration. Additionally, low user engagement with direct text-to-image generation restricts modification and editing, limiting flexibility in creating paper-cutting. 
% content analysis 发现, 剪纸涉及需要从4个要素上来进行构思,分别是功能、题材、风格和表现手法。另外作为剪纸最基础的语义符号,纹样与这4点关联紧密,有作为基础单元来进行组合的,组合的纹样有作为主体的,也有用于装饰的。
The content analysis identified four core factors in paper-cutting ideation: \textbf{Function, Subject Matter, Style,} and \textbf{Method of Expression}, and a key element of paper-cutting: \textbf{Pattern}, in various forms, are closely linked to these factors, serving as basic glyph units, primary or decorative components in a paper-cutting.

Building on the identified challenges and the derived design space, we introduce three design goals for a GenAI-aided system aimed at harmonizing form and connotation in paper-cutting design: (1) facilitate users in formulating requirements based on four factors of design, (2) explore the recommended diverse and related contents for translating the idea into a visual representation, (3) enable user's interactive content combination and modification on generated results to encourage their creativity.
% These findings highlighted the opportunity to rely on recommendations and guidance from GenAI to support ideation, exploration, and combination for paper-cutting design with rich paper-cutting-related knowledge as the reference. 
Aligned with these design goals, we propose a GenAI-aided paper-cutting design pipeline (\autoref{figure2}) and develop a prototype system, \textbf{HarmonyCut} (\autoref{figure3}), which facilitates user ideation and suggests related content with its corresponding meaning and cultural background, which serves as a reference to support the users in exploring and composing a harmonious visual paper-cutting design.

To evaluate our workflow and system, we conducted a within-subjects user study with 16 participants to compare HarmonyCut with the baseline tool, supported by ChatGPT and DALL-E-3\footnote{\url{https://openai.com/index/dall-e-3/}}, and an expert interview with 3 professional paper-cutting inheritors. The results demonstrated that factor-oriented guidance and reference-based exploration, when integrated with GenAI and domain knowledge, can effectively facilitate ideation and composition in paper-cutting design.

% Finally, we discussed the generalizability and scalability of HarmonyCut in the aspect of user expertise, and different domains of design.


% 在了解了剪纸创作完整流程的挑战,并明确设计目标后,我们结合content analysis 明确了整个剪纸创作的设计空间

% 第四部分
% 基于design goal 以及 design space, 我们提出 了一个创作剪纸的pipeline,来辅助剪纸设计和创作,保证内容和主题匹配。并基于pipeline,propose a SYSTEM,来帮助: 【系统描述】

% 第五部分
% To assess the utility of SYSTEM, we conducted an user study with 8 participants and expert evaluation with 3 professional paper-cutting inheritors. The results showed that 【evaluation results】

% 第六部分
% The major contributions of the paper are as follows: 
This research thus contributes to the following:
\begin{itemize}
    \item A \textbf{formative study} involving seven participants that identifies the design workflow and challenges inherent in paper-cutting design.
    \item A \textbf{content analysis} of 140 paper-cuttings, which uncovers the core factors and various patterns that define the design space in paper-cutting. 
    \item A \textbf{prototype system}, HarmonyCut, which leverages GenAI to support reference recommendation and exploration for conceptual ideation and visual composition in paper-cutting design.
    \item A \textbf{user study} (N=16) with the \textbf{expert interview} (N=3), exploring how HarmonyCut supports each step of the paper-cutting design process with GenAI assistance.
\end{itemize}