\section{Evaluation}\label{evaluation}
We conducted a within-subjects user study with sixteen participants and an expert evaluation involving three Chinese paper-cutting experts. The objective was to assess HarmonyCut's usability in paper-cutting design and to validate the proposed design workflow. Then, we interviewed participants to evaluate whether HarmonyCut enhances creativity and facilitates the design process. Insights from the expert interviews highlighted how the system addresses challenges in paper-cutting design and identified areas for improvement.
\revisedtext{In contrast, the baseline tool utilized the same GenAI models as HarmonyCut along with their respective official web interfaces of GLM-4\footnote{\url{https://chatglm.cn/}} and DALL-E-3\footnote{\url{https://chatgpt.com/}}. However, It lacked the domain-specific knowledge (i.e., annotated paper-cuttings and fine-tuned models) and the direct editing capabilities incorporated into HarmonyCut. Consequently, users of the baseline tool had to manually input text for design ideation with GLM-4 and generate final designs by crafting their own prompts for DALL-E-3.}

\subsection{Participants}
\revisedtext{We recruited 16 participants (6 females and 10 males; age M=23.94, SD=7.08) through online postings on various social media platforms.} To ensure HarmonyCut could meet design goals from the formative study, participants were selected based on similar criteria for expertise in paper-cutting and GenAI. \revisedtext{The group included 1 paper-cutting master (U7), 2 practitioners (U2, U8), 3 amateurs (U5, U14-U15), and 10 novices (U1, U3-U4, U6, U9-U13, U16). Regarding GenAI expertise, there were 12 knowledgeable users (U1-U6, U9-U10, U13-U16) and 4 novices (U7-U8, U11-U12).} The three Chinese paper-cutting experts (E1-E3) were recruited from social video platforms, each having more than 20, 40, and 20 years of experience in designing, creating, and teaching paper-cutting. They are also recognized as ICH inheritors. \rrtext{The detailed information of the participants is shown in~\autoref{table:evaluation participants} and each participant received 100 CNY (approximately 14 USD).}

\begin{figure*}[!htbp]
\centering
\includegraphics[width=\textwidth]{Images/user_cases.pdf}
\caption{\label{figure: user cases}
The 6 paper-cutting design examples were created by the 6 participants in the user study. All solid references are retrieved, while dashed references are generated.}
\Description{This figure shows the 6 paper-cutting design examples created by the participants in the user study. All solid references are retrieved, while ashed references are generated.}
\end{figure*}
\subsection{Procedure and Measures}
In the user study procedure, participants were tasked with designing a paper-cutting that aligned with the provided design themes, such as ``Welcoming Spring'' or ``Health and Longevity.'' Initially, participants are given a 5-minute introduction to the study and its background. For each tool (HarmonyCut or the Baseline), participants underwent three stages: a 15-minute session for tool instruction and task description, a 30-minute period to complete the design task using the respective tool, a 5-minute session to complete a questionnaire, and a 10-minute period for free exploration with the tool. A 5-minute break was scheduled between sessions for each tool. After both experiments, a 15-minute semi-structured interview was conducted. The order of tool usage and tasks was counterbalanced to negate sequence effects on the results. The example outputs generated by participants with varying levels of expertise are presented in~\autoref{figure: user cases}.


To validate the design workflow integrating paper-cutting knowledge with HarmonyCut, participants completed a questionnaire after each session, resulting in two submissions per participant. The questionnaire included three 6-point Likert scale questions focused on design goals: (1) ideation guidance, asking if the tool helps generate ideas; (2) \revisedtext{exploration with reference, asking if the tool supports broad exploration for design, complementary to the Exploration measure in CSI;} and (3) editing flexibility, questioning if the tool allows sufficient editing flexibility. Additionally, the questionnaire incorporated the CSI~\cite{Cherry:2014:csi} (0-10 scale) and NASA-TLX~\cite{Hart:2006:nasa} (0-20 scale) to evaluate usability regarding creative support and perceived workload. 
% Each CSI metric is rated on a scale of 0-10, while each NASA-TLX metric is rated on a scale of 0-20. 

Prior to the expert interview, each expert was asked to prepare a design intent. During the interview, experts received a 15-minute introduction and system walk-through, including a specific design task. After learning how to use HarmonyCut for the paper-cutting design task, experts were given 30 minutes to create a paper-cutting based on their prepared design intent. After the design session, an interview was conducted on three topics adapted from Xu et al.\cite{Xu:2023:magicalbrush}~(\autoref{table3}), each with two questions to gather feedback.
\begin{table}[!htbp]
\caption{Questions during expert interviews from 3 topics.}
\Description{This table presents the six questions used in the expert interviews, categorized into three topics: culture, creativity, and limitations.}
\label{table3}
\resizebox{0.49\textwidth}{!}{
\renewcommand\arraystretch{1.5}
\begin{tabular}{c|c}
\hline
Topic                        & Question                                                             \\ \hline
\multirow{2}{*}{Culture}     & Does HarmonyCut's suggested content align with its cultural meaning? \\
                             & Can HarmonyCut help the public better understand paper-cutting knowledge? \\ \hline
\multirow{2}{*}{Creativity}  & Does HarmonyCut's guidance limit or enhance your creativity?         \\
                             & What aspects of HarmonyCut inspire new creative ideas?               \\ \hline
\multirow{2}{*}{Improvement} & What are your thoughts on human-AI collaborate design systems?              \\
                             & Where can HarmonyCut be improved?                                    \\ \hline
\end{tabular}
}
\end{table}

\subsection{Results}
Based on the qualitative and quantitative data collected from both studies, we found that HarmonyCut effectively facilitates the generation of creative ideas through guided support. It aided participants in exploring valuable related references and allowed them to incorporate their own creative ideas into the paper-cutting design process. 

To assess whether each design goal was achieved within HarmonyCut and whether the system effectively addressed the challenges associated with these goals in the design process, we first analyzed the questionnaire data in the user study. Considering the small sample size (N=16) and the ordinal nature of the data, we employed a Wilcoxon signed-rank test to compare the differences between HarmonyCut and the baseline tool. \revisedtext{Although~\autoref{table4} showed that the average performance across most metrics is better than the Baseline, participants had varying levels of expertise. Therefore, we also analyzed their opinions of the two tools based on their paper-cutting and GenAI expertise levels~(in \autoref{figure:DG},\ref{figure:NASA},\ref{figure:CSI}).} Subsequently, we gathered feedback from the user study and the expert interview.

% \begin{table*}[!htbp]
% \caption{Survey results of participant opinion about design goals, NASA-TLX questionnaire, and Creativity Support Index.}
% \Description{This table illustrates the results from the survey on participants' opinions about the design goals, the NASA-TLX questionnaire, and the Creativity Support Index.}
% \label{table4}
% \renewcommand\arraystretch{1.2}
% \resizebox{0.95\textwidth}{!}{
% \begin{tabular}{ccccccc}
% \hline
% \multicolumn{2}{c|}{\multirow{2}{*}{Indicator}}                                                                  & \multicolumn{2}{c|}{\textbf{HarmonyCut}}                  & \multicolumn{2}{c|}{\textbf{Baseline}}                    & \multirow{2}{*}{P} \\ \cline{3-6}
% \multicolumn{2}{c|}{}                                                                                            & \multicolumn{1}{c|}{Mean}   & \multicolumn{1}{c|}{SD}     & \multicolumn{1}{c|}{Mean}   & \multicolumn{1}{c|}{SD}     &                    \\ \hline
% \multicolumn{1}{c|}{\multirow{3}{*}{Survey related to design goals}} & \multicolumn{1}{c|}{Ideation}             & \multicolumn{1}{c|}{3.875}  & \multicolumn{1}{c|}{0.641}  & \multicolumn{1}{c|}{3.125}  & \multicolumn{1}{c|}{0.835}  & 0.119              \\ \cline{2-7} 
% \multicolumn{1}{c|}{}                                                & \multicolumn{1}{c|}{Exploration}              & \multicolumn{1}{c|}{4.250}  & \multicolumn{1}{c|}{0.775}  & \multicolumn{1}{c|}{2.625}  & \multicolumn{1}{c|}{0.957}  & 0.0024**           \\ \cline{2-7} 

% \multicolumn{1}{c|}{}                                                & \multicolumn{1}{c|}{Editing}              & \multicolumn{1}{c|}{3.500}  & \multicolumn{1}{c|}{0.756}  & \multicolumn{1}{c|}{1.750}  & \multicolumn{1}{c|}{0.463}  & 0.0078**           \\ \hline
% \multicolumn{1}{c|}{\multirow{7}{*}{NASA-TLX}}                       & \multicolumn{1}{c|}{Mental}               & \multicolumn{1}{c|}{13.125} & \multicolumn{1}{c|}{12.159} & \multicolumn{1}{c|}{26.625} & \multicolumn{1}{c|}{21.705} & 0.027*             \\ \cline{2-7} 
% \multicolumn{1}{c|}{}                                                & \multicolumn{1}{c|}{Physical}             & \multicolumn{1}{c|}{22.625} & \multicolumn{1}{c|}{24.231} & \multicolumn{1}{c|}{10.875} & \multicolumn{1}{c|}{14.116} & 0.149              \\ \cline{2-7} 
% \multicolumn{1}{c|}{}                                                & \multicolumn{1}{c|}{Temporal}             & \multicolumn{1}{c|}{14.250} & \multicolumn{1}{c|}{14.680} & \multicolumn{1}{c|}{22.750} & \multicolumn{1}{c|}{16.272} & 0.176              \\ \cline{2-7} 
% \multicolumn{1}{c|}{}                                                & \multicolumn{1}{c|}{Performance}          & \multicolumn{1}{c|}{19.125} & \multicolumn{1}{c|}{12.100} & \multicolumn{1}{c|}{32.125} & \multicolumn{1}{c|}{20.553} & 0.075              \\ \cline{2-7} 
% \multicolumn{1}{c|}{}                                                & \multicolumn{1}{c|}{Effort}               & \multicolumn{1}{c|}{21.625} & \multicolumn{1}{c|}{10.783} & \multicolumn{1}{c|}{13.750} & \multicolumn{1}{c|}{9.301}  & 0.091              \\ \cline{2-7} 
% \multicolumn{1}{c|}{}                                                & \multicolumn{1}{c|}{Frustration}          & \multicolumn{1}{c|}{15.250} & \multicolumn{1}{c|}{18.934} & \multicolumn{1}{c|}{26.250} & \multicolumn{1}{c|}{30.391} & 0.173              \\ \cline{2-7} 
% \multicolumn{1}{c|}{}                                                & \multicolumn{1}{c|}{Overall Load}         & \multicolumn{1}{c|}{7.067}  & \multicolumn{1}{c|}{3.094}  & \multicolumn{1}{c|}{8.825}  & \multicolumn{1}{c|}{2.945}  & 0.25               \\ \hline
% \multicolumn{1}{c|}{\multirow{6}{*}{Creativity Support Index}}       & \multicolumn{1}{c|}{Enjoyment}            & \multicolumn{1}{c|}{38.250} & \multicolumn{1}{c|}{17.523} & \multicolumn{1}{c|}{17.750} & \multicolumn{1}{c|}{12.361} & 0.0078**           \\ \cline{2-7} 
% \multicolumn{1}{c|}{}                                                & \multicolumn{1}{c|}{Exploration}          & \multicolumn{1}{c|}{62.000} & \multicolumn{1}{c|}{30.477} & \multicolumn{1}{c|}{35.875} & \multicolumn{1}{c|}{21.702} & 0.0178*            \\ \cline{2-7} 
% \multicolumn{1}{c|}{}                                                & \multicolumn{1}{c|}{Expressiveness}       & \multicolumn{1}{c|}{53.500} & \multicolumn{1}{c|}{14.172} & \multicolumn{1}{c|}{39.125} & \multicolumn{1}{c|}{23.540} & 0.150              \\ \cline{2-7} 
% \multicolumn{1}{c|}{}                                                & \multicolumn{1}{c|}{Immersion}            & \multicolumn{1}{c|}{29.625} & \multicolumn{1}{c|}{23.090} & \multicolumn{1}{c|}{19.625} & \multicolumn{1}{c|}{17.427} & 0.461              \\ \cline{2-7} 
% \multicolumn{1}{c|}{}                                                & \multicolumn{1}{c|}{Results Worth Effort} & \multicolumn{1}{c|}{54.750} & \multicolumn{1}{c|}{21.022} & \multicolumn{1}{c|}{43.625} & \multicolumn{1}{c|}{24.272} & 0.310              \\ \cline{2-7} 
% \multicolumn{1}{c|}{}                                                & \multicolumn{1}{c|}{CSI}                  & \multicolumn{1}{c|}{79.375} & \multicolumn{1}{c|}{9.410}  & \multicolumn{1}{c|}{52.000} & \multicolumn{1}{c|}{23.243} & 0.0156*            \\ \hline
%  \multicolumn{7}{l}{* p<0.05; ** p< 0.01; ***p<0.001}                                                       
% \end{tabular}
% }
% \end{table*}
\begin{table*}[!htbp]
\caption{Survey results of participant opinion about design goals, NASA-TLX questionnaire, and Creativity Support Index.}
\Description{This table illustrates the results from the survey on participants' opinions about the design goals, the NASA-TLX questionnaire, and the Creativity Support Index.}
\label{table4}
\renewcommand\arraystretch{1.2}
\resizebox{0.95\textwidth}{!}{
\begin{tabular}{ccccccc}
\hline
\multicolumn{2}{c|}{\multirow{2}{*}{Indicator}}                                                                  & \multicolumn{2}{c|}{\textbf{HarmonyCut}}                  & \multicolumn{2}{c|}{\textbf{Baseline}}                    & \multirow{2}{*}{P} \\ \cline{3-6}
\multicolumn{2}{c|}{}                                                                                            & \multicolumn{1}{c|}{Mean}   & \multicolumn{1}{c|}{SD}     & \multicolumn{1}{c|}{Mean}   & \multicolumn{1}{c|}{SD}     &                    \\ \hline
\multicolumn{1}{c|}{\multirow{3}{*}{Survey related to design goals}} & \multicolumn{1}{c|}{Ideation}             & \multicolumn{1}{c|}{3.563}  & \multicolumn{1}{c|}{0.727}  & \multicolumn{1}{c|}{3.375}  & \multicolumn{1}{c|}{0.957}  & 0.472              \\ \cline{2-7} 
\multicolumn{1}{c|}{}                                                & \multicolumn{1}{c|}{Exploration}          & \multicolumn{1}{c|}{4.125}  & \multicolumn{1}{c|}{0.885}  & \multicolumn{1}{c|}{2.563}  & \multicolumn{1}{c|}{0.892}  & 0.0029**           \\ \cline{2-7} 
\multicolumn{1}{c|}{}                                                & \multicolumn{1}{c|}{Editing}              & \multicolumn{1}{c|}{3.625}  & \multicolumn{1}{c|}{0.806}  & \multicolumn{1}{c|}{1.438}  & \multicolumn{1}{c|}{0.727}  & 0.00003***         \\ \hline
\multicolumn{1}{c|}{\multirow{7}{*}{NASA-TLX}}                       & \multicolumn{1}{c|}{Mental}               & \multicolumn{1}{c|}{14.250} & \multicolumn{1}{c|}{11.958} & \multicolumn{1}{c|}{23.625} & \multicolumn{1}{c|}{16.931} & 0.026*             \\ \cline{2-7} 
\multicolumn{1}{c|}{}                                                & \multicolumn{1}{c|}{Physical}             & \multicolumn{1}{c|}{16.813} & \multicolumn{1}{c|}{19.641} & \multicolumn{1}{c|}{7.625}  & \multicolumn{1}{c|}{12.447} & 0.046*             \\ \cline{2-7} 
\multicolumn{1}{c|}{}                                                & \multicolumn{1}{c|}{Temporal}             & \multicolumn{1}{c|}{24.625} & \multicolumn{1}{c|}{22.102} & \multicolumn{1}{c|}{16.688} & \multicolumn{1}{c|}{14.858} & 0.460              \\ \cline{2-7} 
\multicolumn{1}{c|}{}                                                & \multicolumn{1}{c|}{Performance}          & \multicolumn{1}{c|}{18.875} & \multicolumn{1}{c|}{13.837} & \multicolumn{1}{c|}{39.750} & \multicolumn{1}{c|}{23.345} & 0.0041**           \\ \cline{2-7} 
\multicolumn{1}{c|}{}                                                & \multicolumn{1}{c|}{Effort}               & \multicolumn{1}{c|}{25.438} & \multicolumn{1}{c|}{13.008} & \multicolumn{1}{c|}{12.188} & \multicolumn{1}{c|}{7.458}  & 0.0045**           \\ \cline{2-7} 
\multicolumn{1}{c|}{}                                                & \multicolumn{1}{c|}{Frustration}          & \multicolumn{1}{c|}{18.625} & \multicolumn{1}{c|}{19.328} & \multicolumn{1}{c|}{30.125} & \multicolumn{1}{c|}{28.477} & 0.157              \\ \cline{2-7} 
\multicolumn{1}{c|}{}                                                & \multicolumn{1}{c|}{Overall Load}         & \multicolumn{1}{c|}{7.908}  & \multicolumn{1}{c|}{2.980}  & \multicolumn{1}{c|}{8.667}  & \multicolumn{1}{c|}{2.805}  & 0.562              \\ \hline
\multicolumn{1}{c|}{\multirow{6}{*}{Creativity Support Index}}       & \multicolumn{1}{c|}{Enjoyment}            & \multicolumn{1}{c|}{27.125} & \multicolumn{1}{c|}{18.395} & \multicolumn{1}{c|}{23.438} & \multicolumn{1}{c|}{14.660} & 0.562              \\ \cline{2-7} 
\multicolumn{1}{c|}{}                                                & \multicolumn{1}{c|}{Exploration}          & \multicolumn{1}{c|}{54.250} & \multicolumn{1}{c|}{26.055} & \multicolumn{1}{c|}{37.563} & \multicolumn{1}{c|}{23.218} & 0.073              \\ \cline{2-7} 
\multicolumn{1}{c|}{}                                                & \multicolumn{1}{c|}{Expressiveness}       & \multicolumn{1}{c|}{55.375} & \multicolumn{1}{c|}{16.552} & \multicolumn{1}{c|}{34.250} & \multicolumn{1}{c|}{18.724} & 0.0076**            \\ \cline{2-7} 
\multicolumn{1}{c|}{}                                                & \multicolumn{1}{c|}{Immersion}            & \multicolumn{1}{c|}{30.688} & \multicolumn{1}{c|}{17.621} & \multicolumn{1}{c|}{20.750} & \multicolumn{1}{c|}{15.902} & 0.074              \\ \cline{2-7} 
\multicolumn{1}{c|}{}                                                & \multicolumn{1}{c|}{Results Worth Effort} & \multicolumn{1}{c|}{52.313} & \multicolumn{1}{c|}{22.934} & \multicolumn{1}{c|}{42.375} & \multicolumn{1}{c|}{28.577} & 0.222              \\ \cline{2-7} 
\multicolumn{1}{c|}{}                                                & \multicolumn{1}{c|}{CSI}                  & \multicolumn{1}{c|}{73.250}  & \multicolumn{1}{c|}{14.283} & \multicolumn{1}{c|}{52.792} & \multicolumn{1}{c|}{19.734} & 0.0013**           \\ \hline
\multicolumn{7}{l}{*p<0.05; **p<0.01; ***p<0.001}                                   
\end{tabular}
}
\end{table*}


\begin{figure*}[!htbp]
\centering
\includegraphics[width=\textwidth]{Images/DG_distribution4.pdf}
\caption{\label{figure:DG}
Sixteen participants ratings the design goals questionnaire across different expertise levels on paper-cutting and GenAI.}
\Description{This figure shows the sixteen participants' ratings to the design goals questionnaire across different expertise levels on paper-cutting and GenAI.}
\end{figure*}

\subsubsection{Ideation with Guidance}
Regarding the first design goal, factor-oriented guidance for ideation, as shown in~\autoref{table4}, while the results for ideation were not statistically significant, the average scores indicated that participants perceived HarmonyCut as providing better support for their \textbf{Ideation} process (HarmonyCut: M=3.563, SD=0.727 / Baseline: M=3.375, SD=0.957). Additionally, the NASA-TLX results revealed that participants experienced a significantly lower \textbf{Mental} workload when using HarmonyCut (M=14.250, SD=11.958 / Baseline: M=23.625, SD=16.931 / P=0.026*, W=21.0) compared to the Baseline. This suggests that HarmonyCut's guided ideation, facilitated by factor-based options, effectively supported users during the ideation stage, especially for tasks requiring mental effort.

\begin{figure*}[!htbp]
\centering
\includegraphics[width=\textwidth]{Images/NASA_distribution4.pdf}
\caption{\label{figure:NASA}
Sixteen participants ratings to the NASA-TLX perceived load questionnaire across different expertise levels on paper-cutting and GenAI.}
\Description{This figure shows the sixteen participants' ratings to the NASA-TLX perceived load questionnaire across different expertise levels on paper-cutting and GenAI.}
\end{figure*}

Furthermore, participants noted that the baseline tool tended to recommend overly broad information or required iteration, making it challenging to decide on an idea. As U3 mentioned, ``\textit{I gave a vague description and had to spend time filtering suggestions.}'' In contrast, HarmonyCut's selection of factors and suggested content helped users understand paper-cutting knowledge through introductions and interpretations. U5 highlighted this advantage, stating, ``\textit{HarmonyCut provided relevant patterns, each with explanations of their meaning and how they fit the design purpose.}''  Meanwhile, in the expert interview, E1’s responses to questions 1 and 2 on the cultural topic, given E1's many years of experience teaching paper-cutting, align with U5’s points: ``\textit{Aligning suggested form and meaning is not enough for understanding. It is through interpretation that users can really grasp the content and, in turn, find inspiration during the learning process.}''. This guidance aided users in the ideation process, helping them quickly construct initial ideas and comprehend the rationale behind their choices, as U3 concluded, ``\textit{This helped me quickly construct initial ideas and understand their choices.}''

In addition, to evaluate how DG1 addresses C2 in~\autoref{sec:formative}, we collected expert feedback on creativity and culture during the interviews. All three experts agreed that the factors-oriented guidance did not limit their creativity. They also acknowledged that a system supported by domain knowledge, which enhances understanding through explanations, was highly beneficial for learning about paper-cutting traditions from different regions and cultures, aiding in innovative ideation.

\begin{figure*}[!htbp]
\centering
\includegraphics[width=\textwidth]{Images/CSI_distribution4.pdf}
\caption{\label{figure:CSI}
Sixteen participants ratings on the Creative Support Index questionnaire across expertise levels in paper-cutting and GenAI.}
\Description{This figure shows the sixteen participants' ratings to the Creative Support Index questionnaire across different expertise levels on paper-cutting and GenAI.}
\end{figure*}


\subsubsection{Reference Exploration}
% 对于具有参考的探索过程,Harmony为用户带来的探索感显著,根据调查的结果,Exploration(Harmonyut: M=4.25, SD=1.035 / Baseline: M=3.0, SD=1.069)以及CSI (HarmonyCut: M=62, SD=30.477 / Baseline: M=35.875, SD=21.702 / p =0.0178*)中结果可以都表明显著在探索多样的想法优于Baseline。 U4表示,似乎是由于模型训练数据的问题,即使他更换了几次输入,返回的剪纸内容总是很相近,而且总是会出现 HUI pattern。 类似的E8也表示,我更换了几次输入,但Baseline的返回结果还是风格相近。
% U3认为检索已有相关剪纸以及按idea生成剪纸结合,通过探索,共同作为构图参考的模式即保证了探索的效率:不会因为被生成的质量参差不齐的内容淹没,也不会只限制在有限的可检索数据中,而缺乏构图的创意。 
% 令我们惊讶的是,E2表示,作为专家,他最近被邀请设计用于宣传政府机构的剪纸作品,但他也不知道该用哪些与该机构相关的题材来进行创作,同时他创作抽象风格剪纸多,想参考下别人的写实剪纸。 他认为检索到的几幅刻画钢厂的剪纸为他提供了一定灵感。 将他们作为reference,并使用Harmony的生成模型来查看常见与钢厂匹配的造型构图,同样作为参考。这确实帮助了他们从探索,确定参考到完成基本的构图
\revisedtext{The results indicated that HarmonyCut significantly enhances the exploration process, as reflected in the \textbf{Exploration} measure in design goal questionnaire (HarmonyCut: M=4.125, SD=0.885 / Baseline: M=2.563, SD=0.892 / p=0.0029**, W=9.0). Besides, HarmonyCut's \textbf{Exploration} score in the CSI was higher than that of the baseline tool (HarmonyCut: M=54.250, SD=26.055 / Baseline: M=37.563, SD=23.218). These results suggested that HarmonyCut supports a more diverse and creative exploration compared to the Baseline.}

Several participants provided insightful feedback. U4 and U8 both noted that the ``\textit{Baseline system tended to return similar results, regardless of input changes, limiting creative exploration.}'' In contrast, U5 highlighted that ``\textit{HarmonyCut's combination of design retrieval and idea-based generation allowed for more efficient exploration.}'' This approach helped avoid the pitfalls of inconsistent generative results while offering a broader creative scope than relying solely on retrievable data.
% \revisedtext{Although U15虽然用harmony去探索设计可能的过程确实更耗时}
To analyze based on participants' varying levels of expertise, we found that U7, a master of paper-cutting~(\autoref{figure:DG},\autoref{figure:CSI}), felt that the exploration feature offered the same capability as the Baseline (\textbf{Exploration} in Design Goal: HarmonyCut=4, Baseline=4; \textbf{Exploration} in CSI: HarmonyCut=16, Baseline=16), stating: ``\textit{I've seen quite a lot of paper-cuttings. The additional retrieval function in HarmonyCut did not significantly enhance the sense of exploration.}'' Interestingly, U7 was recently invited to design a government-themed paper-cutting piece. Seeking inspiration, U7 reviewed realistic paper-cuttings of steel factories retrieved by HarmonyCut in the free exploration session. Using these as references, along with HarmonyCut's generative model to explore common themes related to steel factories, U7 noted that ``\textit{Although the exploration content is limited due to the restricted dataset, reference-based exploration did help me with composition. Also, using the system saved me a lot of time compared to sketching, while the Baseline was almost useless for planning the cut-out layout.}'' (\textbf{Temporal Load}: HarmonyCut=6, Baseline=10). For participants with varying expertise levels in both paper-cutting and GenAI, HarmonyCut was generally perceived to enhance exploration capabilities, as illustrated in~\autoref{figure:DG} and~\autoref{figure:CSI}.


\subsubsection{Editing with Effort}\label{sec:editing}
% 对于内容的编辑,HarmonyCut支持基于组合纹样轮廓的分割,基于单元纹样的提取,以及相关作品的检索。 Baseline仅能通过输入来调整输出。同时,根据结果,也能发现, 受访者普遍认为HarmonyCut相较于baseline可以灵活编辑,且通过自己参与交互,提高了可控性,也不需要只依靠模型输出。 但值得注意的一点是,在付出努力和体力负担上,HarmonyCut也是比Baseline高的。U1表示,对象分割提供少量point分割效果不好,但提供point太多会很费力。
For content editing, HarmonyCut supports segmentation based on the contours of composite patterns, extraction of unit patterns, and retrieval of related works. In contrast, the baseline tool only allows output adjustments through input modifications. The results indicate that participants generally perceived HarmonyCut as offering more flexible \textbf{Editing} capabilities compared to the baseline, enhancing controllability (HarmonyCut: M=3.625, SD=0.806 / Baseline: M=1.438, SD=0.727 / p=0.00003***, W=0) through user interaction rather than relying solely on model outputs. \revisedtext{Furthermore, although the results for \textbf{Enjoyment} were not statistically significant (HarmonyCut: M=27.125, SD=18.395 / Baseline: M=23.438, SD=14.660), some participants expressed enjoyment in editing their designs using our system. Moreover, despite being knowledgeable about GenAI and using prompt engineering to enhance outputs, some participants argued that the lack of control in the baseline tool hindered design. Furthermore, some participants attributed the improved ability to \textbf{Express} ideas with HarmonyCut (HarmonyCut: M=55.375, SD=16.552; Baseline: M=34.250, SD=18.724; p=0.0076**, W=13.0), compared to the Baseline, to its controllable editing support. For instance, U6 remarked, ``\textit{The inability to directly edit the output and the uncontrollable reasoning process made it difficult for me to enjoy the design experience.}'' U14 also mentioned about ``\textit{Especially for content related to Chinese characters, the Baseline often fails to meet the requirements, generating some meaningless stroke combinations. By comparison, I still prefer a controllable design process to express my idea.}''} To the expert interview, E3 noted that ``\textit{the ability to edit undesirable parts of the generated image is a critical function in HarmonyCut, improved the controllability issue of GenAI.}''
% 另外,多为参与者提到,这样的编辑能力,也支持其更好的表达自己的意图。不过从图中也可以看到,专家认为两者的表达力都有限,对于Harmony,它们一方面因为reference list还是规模有限,且编辑程度还是不够灵活,无法完全发挥其expertise,这也会在\autoref{limitation} and \autoref{discuss:guidance reference}中进一步讨论。


\subsubsection{\revisedtext{Design Process and Design Performance}}\label{sec:process performance}
\revisedtext{Chinese paper-cutting embodies a fundamental connotation: the aspiration for a better life. This connotation is directly expressed through the selection of content and composition, which serve as key considerations in the creative process~\cite{Lin:1974:howtopapercutting, Zhang:2021:papercuttingteaching, Li:1998:monopapercutting}. Experienced paper-cutting educators (E1-E3) also consistently emphasized the importance of training students to choose appropriate content and develop effective compositions as a critical aspect of learning paper-cutting. Consequently, integrating support for these aspects into the design process is essential. 
This was partially reflected in the \textbf{Performance} (HarmonyCut: M=18.875, SD=13.837 / Baseline: M=39.750, SD=23.345 / p=0.0041**, W=9.5) cross users with different levels of expertise. Experts recognized HarmonyCut in~\autoref{figure:NASA} as effective in supporting reference selection and composition. Novices acknowledged the system's \textbf{Performance}~(\autoref{figure:NASA}) and additionally reported a more \textbf{Immersive} experience~(\autoref{figure:CSI}) during the design process. Notably, U3, U9, and U13, as young novices, emphasized that the design process in HarmonyCut not only helped them learn the knowledge of paper-cutting but also sparked a deeper interest in the craft. Such knowledge sharing and interest of the young generation play a role in influencing the preservation and transmission of ICH~\cite{Affleck:2008:newgeneration, Mancacaritadipura:2009:newgeneration}, like paper-cutting. Nevertheless, beyond initial interest, long-term accumulation and profound knowledge are necessary to foster creativity and innovation, ensuring the inheritance and vitality of ICH, as also discussed in~\autoref{limitations}. }
% 数据 performance in NASA, Expressive in CSI。 对于最终的结果,
% 作为民俗艺术,一个核心功能是寄托人民对美好生活的向往,
% 专家讨论 剪纸作为一个很靠经验的 民俗艺术,很多都是为了传达吉祥寓意,其首要考量的就是视觉形式与构图是否能够体现作者巧思且满足祝福他人的目的。因此,对于剪纸的domain expertise,其首要考量的一部分就是能否选择合适的内容,以及巧妙地构图。 而这一点,不仅被expert 认可, 且其他expertise的一些用户在谈到最终结果很满意,让他在短时间内了解了某一剪纸主题相关的知识,同时因其能够足够表达自我而对剪纸这一艺术形式产生进一步了解的兴趣。 而在new generation 产生兴趣,这对于非物质文化遗产的传承,是很关键的一点
%尽管对应思维完成主题与形式的匹配是它的最基础首要能力,但也需要跟随着新事物的出现,以及新内涵的赋予,将他们用于创作中。 


\subsubsection{System Limitations}\label{limitations}
% Based on the feedback from participants from the user study and expert interviews, 我们总结了系统目前存在的不足:
% 第一,目前的design space已经比较comprehensive, 但毕竟是文化相关知识,尤其剪纸这样的民间艺术,多样性很强,对于整个design space中的标注数据还可以继续扩充, ,持续更信数据才能保证design space的合理性。 也可以保证模型提供的剪纸知识和解释更合理。 当然,这是一个周期性很强,ambition很壮的目标
% 第二,虽然contour+pattern来进行构图设计可以极大提高过程的编辑性,但与直接通过文本输入而获取返回图片,还是要通过图片分割,纹样选取上付出努力。这也在nasa-tlx结果可以看出,HarmonyCut 脑力要求低,但体力要求以及付出努力比baseline高。 不过这是一个权衡的问题
% 第三  E3提到,她对于AI辅助设计上很担心AI所学风格涉及版权侵犯问题,导致自己在GenAI-aided设计过程中也会造成侵权。
Based on feedback from participants in the user study and expert interviews, we have summarized the current limitations of the system, which are also discussed in~\autoref{lf} with future directions:

\noindent\textbf{Cultural Knowledge Diversity:} \revisedtext{While the current design space is relatively comprehensive, given the diversity of cultural knowledge, especially in profound folk arts like paper-cutting. Thus, there is room for further expansion of labeled data, including the taxonomy and recommended reference (proposed by E1).} Continuously enriching the dataset will ensure the design space remains reasonable and that the paper-cutting knowledge and explanations provided by the model are more accurate. This is an ambitious and ongoing effort. U7 made a similar suggestion, pointing out that the collected and annotated paper-cutting data still needs to be expanded, as the bottleneck appears to lie in the data. Expanding this dataset would improve the usability of HarmonyCut.

\noindent\textbf{Effort-Intensive Design Process:} Although the contour + pattern approach enhances the flexibility of the design process, it requires more effort compared to simply inputting text and receiving generated images (U1-U2, U4, U10-11). Users need to invest time in image segmentation and pattern selection. This trade-off is reflected in the NASA-TLX results, where HarmonyCut shows a lower \textbf{Mental} load but significantly higher \textbf{Physical} load and \textbf{Effort} compared to the baseline~(\autoref{table3}). However, balancing these factors is essential.
% %当然,剪纸的知识以及所能掌握的专业程度远不止于此,需要长年的学习和体验,才能将在对应思维的基础上更好的融入自己的风格,既能通过丰厚的知识继承过去的内容,也能有把握将新事物以剪纸形式表现进行创新。
% design performance那部分讨论expert和literature align,模仿对于初学者来说很重要。不过,进一步的学习就需要更多creation,references更多只是扩充思路,作为融入作者创意,加入创新内容的载体
% \vspace{1.4mm}\noindent\textbf{Effort-Intensive Design Process:} E3 raised concerns about GenAI-aided design, specifically the risk of copyright infringement due to the styles learned by AI. She worried that using AI-generated content might inadvertently lead to infringement in her designs.

\noindent\textbf{\revisedtext{Imitation and Creation in Design:}} \revisedtext{
Although HarmonyCut is considered significantly more expressive in~\autoref{sec:editing}, U7 and U8~(\autoref{figure:CSI}) found its expressiveness limited, attributing this to the restricted reference list and lack of more types of editing tools, which they felt constrained their expertise. Besides, some novice users, on the other hand, overly relied on the recommended references for their designs, leading to imitation issues. From another perspective, E1, E2, and Lin~\cite{Lin:1974:howtopapercutting} emphasized that reference-based imitation is, to some extent, a necessary step in the creative process. In their teaching practices, students were required to engage in extensive imitation to acquire relevant knowledge, develop skills, and draw inspiration~(\autoref{sec:process performance}), which was align with the view of Okada et al.~\cite{Okada:2017:imitation}.
Paper-cutting design should extend beyond imitation, which serves as a foundational step in learning the craft. To foster creativity, it is essential to integrate new objects and meanings into the design process. Achieving this requires time, training, and experience, moving past the mere reuse of references to maintain creativity and drive innovation.} 
% 认为其表达力充足, 但对于expert(U2,U7-U8)群体 figure7,他么普遍认为Harmony的表现力有限,一方面因为reference list规模有限,且编辑度对于需求更多,经验更丰富的专家还是不够灵活,更多的是需要基于所给reference进行创作,无法完全发挥它们expertise。 不过,U7与E2,and 怎样剪纸也都提到,一定程度的基于reference的模仿,是促进剪纸的创意设计的必要一步,它们在教学中首先会要求学生进行大量的临摹,来学习相关知识并掌握skill of selection and composition in~\autoref{sec:process performance},这也与imitation and creation in design 的view一致~\cite{Japan}。 不过,对应思维完成主题与形式的匹配是它的最基础首要能力,但剪纸设计也需要跟随着新事物的出现,以及新内涵的赋予,将他们用于创作中, 这当然需要大量1时间的训练和经验的积累,形成独属于自己风格的剪纸设计体系。 

% 多为参与者提到,这样的编辑能力,也支持其更好的表达自己的意图。不过从图中也可以看到,专家认为两者的表达力都有限,对于Harmony,它们一方面因为reference list还是规模有限,且编辑程度还是不够灵活,无法完全发挥其expertise,这也会在\autoref{limitation} and \autoref{discuss:guidance reference}中进一步讨论。
% \vspace{1.4mm}\noindent\textbf{Generalizibility to users from different group} 

% \begin{table*}[!htbp]
% \caption{own}
% \label{figure4}
% \begin{tabular}{c|cc|cc|c}
% \hline
% \multirow{2}{*}{Indicator} & \multicolumn{2}{c|}{\textbf{HarmonyCut}}       & \multicolumn{2}{c|}{Baseline}      & \multirow{2}{*}{P} \\ \cline{2-5}
%                            & \multicolumn{1}{c|}{Mean}  & SD    & \multicolumn{1}{c|}{Mean}  & SD    &                    \\ \hline
% Ideation                   & \multicolumn{1}{c|}{3.875} & 0.641 & \multicolumn{1}{c|}{3.125} & 0.835 & 0.119              \\ \hline
% Exploration                & \multicolumn{1}{c|}{4.25}  & 1.035 & \multicolumn{1}{c|}{3.000} & 1.069 & 0.0078             \\ \hline
% Editing                    & \multicolumn{1}{c|}{3.500} & 0.756 & \multicolumn{1}{c|}{1.750} & 0.463 & 0.101              \\ \hline
% \end{tabular}
% \end{table*}


% \begin{table*}[!htbp]
% \caption{NASA-TLX}
% \label{figure5}
% % \renewcommand\arraystretch{1.2}
% % \resizebox{0.75\textwidth}{!}{
% \begin{tabular}{c|cc|cc|c}
% \hline
% \multirow{2}{*}{Indicator} & \multicolumn{2}{c|}{\textbf{HarmonyCut}} & \multicolumn{2}{c|}{\textbf{Baseline}}                     & \multirow{2}{*}{P} \\ \cline{2-5}
%                            & \multicolumn{1}{c|}{Mean}     & SD       & \multicolumn{1}{c|}{Mean}   & \multicolumn{1}{c|}{SD}     &                    \\ \hline
% Mental                     & \multicolumn{1}{c|}{13.125}   & 12.159   & \multicolumn{1}{c|}{26.625} & \multicolumn{1}{c|}{21.705} & 0.027*              \\ \hline
% Physical                    & \multicolumn{1}{c|}{22.625}   & 24.231   & \multicolumn{1}{c|}{10.875} & \multicolumn{1}{c|}{14.116} & 0.149              \\ \hline
% Temporal                   & \multicolumn{1}{c|}{14.250}   & 14.680   & \multicolumn{1}{c|}{22.750} & \multicolumn{1}{c|}{16.272} & 0.176              \\ \hline
% Performance                & \multicolumn{1}{c|}{19.125}   & 12.100   & \multicolumn{1}{c|}{32.125} & \multicolumn{1}{c|}{20.553} & 0.075              \\ \hline
% Effort                     & \multicolumn{1}{c|}{21.625}   & 10.783   & \multicolumn{1}{c|}{13.750} & \multicolumn{1}{c|}{9.301}  & 0.091              \\ \hline
% Frustration                & \multicolumn{1}{c|}{15.250}   & 18.934   & \multicolumn{1}{c|}{26.250} & \multicolumn{1}{c|}{30.391} & 0.173              \\ \hline
% Overall Load               & \multicolumn{1}{c|}{7.067}    & 3.094    & \multicolumn{1}{c|}{8.825}  & \multicolumn{1}{c|}{2.945}  & 0.25               \\ \hline
% \multicolumn{6}{l}{* p < 0.05; ** p < 0.01; *** p < 0.001}
% \end{tabular}
% % }
% \end{table*}


% \begin{table*}[!htbp]
% \caption{Creative Support Index}
% \label{figure6}
% % \renewcommand\arraystretch{1.2}
% % \resizebox{0.75\textwidth}{!}{
% \begin{tabular}{c|cc|cc|c}
% \hline
% \multirow{2}{*}{Indicator} & \multicolumn{2}{c|}{\textbf{HarmonyCut}}         & \multicolumn{2}{c|}{\textbf{Baseline}}        & \multirow{2}{*}{P} \\ \cline{2-5}
%                            & \multicolumn{1}{c|}{Mean}   & SD     & \multicolumn{1}{c|}{Mean}   & SD     &                    \\ \hline
% Enjoyment                  & \multicolumn{1}{c|}{38.250} & 17.523 & \multicolumn{1}{c|}{17.750} & 12.361 & 0.0078**             \\ \hline
% Exploration                & \multicolumn{1}{c|}{62.000} & 30.477 & \multicolumn{1}{c|}{35.875} & 21.702 & 0.0178*             \\ \hline
% Expressiveness             & \multicolumn{1}{c|}{53.500} & 14.172 & \multicolumn{1}{c|}{39.125} & 23.540 & 0.150              \\ \hline
% Immersion                  & \multicolumn{1}{c|}{29.625} & 23.090 & \multicolumn{1}{c|}{19.625} & 17.427 & 0.461              \\ \hline
% Results Worth Effort       & \multicolumn{1}{c|}{54.750} & 21.022 & \multicolumn{1}{c|}{43.625} & 24.272 & 0.310              \\ \hline
% CSI                        & \multicolumn{1}{c|}{79.375} & 9.410  & \multicolumn{1}{c|}{52.000} & 23.243 & 0.0156*             \\ \hline
% \multicolumn{6}{l}{* p < 0.05; ** p < 0.01; *** p < 0.001}
% \end{tabular}
% %}
% \end{table*}
