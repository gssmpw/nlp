\section{Formative Study}\label{sec:formative}
We conducted a formative study to understand the workflow of paper-cutting design and users' challenges during the design.
% Formative study: 

%.相反,我们最初感兴趣的是了解采用复杂动画的数据驱动故事是如何创作的,以及创作者的工作流程中存在的挑战。为了实现这一目标,我们与专家进行了一项基于访谈的研究,这些专家在使用可视化和动画讲故事方面拥有丰富的经验。由于动画数据故事通常是由一组创建者创建的,因此我们招募了不同角色的专家来全面了解此类工作流。通过这些访谈,我们发现,在受访者创建的各种类型的数据故事中,由AUV组成的数据故事被认为既引人注目又具有挑战性,因为它们加剧了复杂动画创作过程中的许多痛点。

\subsection{Participants}
As a work targeted to public users, we initially sought to examine the paper-cutting design process and the challenges encountered without GenAI. Subsequently, we aimed to assess the role and limitations of integrating GenAI into paper-cutting design. To achieve this, we recruited participants with varying levels of expertise in both paper-cutting and GenAI, as experts are expected to have insights into design workflows, and each participant group faces different challenges in paper-cutting design. \rrtext{Participants were categorized into four levels of paper-cutting expertise, with further clarification regarding the criteria for paper-cutting expertise provided in~\autoref{expert clarification}:}
(1) \textbf{Masters}: who have over 20 years of professional experience in paper-cutting creation are officially recognized as ICH inheritors; (2) \textbf{Practitioners}: who have 10-20 years of experience in paper-cutting-related work; \revisedtext{(3) \textbf{Amateurs}: who have 1-3 years of experience in creating paper-cuttings without systematic training;} (4) \textbf{Novices}: who never engaged in paper-cutting design or creation. Additionally, we defined three levels of GenAI expertise: (1) \textbf{Professionals}: researchers in the multi-modal machine learning field; 
% (2) \textbf{Knowledgeable Users}: who have previously used GenAI; (3) \textbf{Novices}: who are only vaguely familiar with or unfamiliar with GenAI.
\revisedtext{(2) \textbf{Knowledgeable Users}: who regularly integrate GenAI into their professional, educational, or personal activities and have experience with basic prompt engineering. (3) \textbf{Novices}: who have minimal exposure to GenAI, may have heard of it without engaging in its use, or are entirely unacquainted with it.}

\revisedtext{Seven participants (3 females and 4 males; age M=34.43, SD=12.79) were recruited for the semi-structured interviews through online postings on various social media platforms (e.g., Douyin\footnote{\url{https://www.douyin.com/}} and Bilibili\footnote{\url{https://www.bilibili.com/}}), and the detailed information of the participants is shown in \autoref{table:formative participants}.} Among them were three paper-cutting masters (P1, P4, P5), two practitioners (P2, P3), one amateur (P7), and one novice (P6). Regarding GenAI expertise, the group included one GenAI professional (P6), two knowledgeable users (P2, P7), and four novices (P1, P3, P4, P5). Each participant is compensated with 100 CNY (approximately 14 USD).
% \begin{itemize}
%     \item[1] Masters: who have over 30 years of professional experience in paper-cutting creation are officially recognized as ICH inheritors.
%     \item[2] Practitioners: who have 10-20 years of experience in paper-cutting-related work.
%     \item[3] Amateurs: who have 1-3 years of experience in creating paper-cuttings.
%     \item[4] Novices: who never engaged in paper-cutting design or creation.
% \end{itemize}  
% Additionally, we defined three levels of GenAI expertise: 
% \begin{itemize}
%     \item[1] Professionals: multi-modal machine learning researchers.
%     \item[2] Knowledgeable Users: who have previously used GenAI;
%     \item[3] Novices: who are only vaguely familiar with or unfamiliar with GenAI.
% \end{itemize} 
% 我们面向的是所有想要通过GenAI辅助进行剪纸创作的用户,无论是专家还是新手。因此我们招募了3 professional inheritors 2个市级 (40年, 20年),一个省级(40年);2 paper-cutting practitioners, 都有超过10年的的从业经验;1 novices,1个纯新人和; 1爱好者 自己剪纸剪纸3年的人。

\subsection{Procedure}
The semi-structured interview included two parts: (1) individual design and (2) GenAI-aided design.
% In the beginning, we provided a 10-minute background introduction for novices in the fields of paper-cutting or GenAI.
\revisedtext{At the beginning, we provided separate 10-minute background introductions for novices in the fields of paper-cutting and GenAI, amounting to a total of 20 minutes.}
In the initial part of the study, each participant was tasked with providing a design concept description for paper-cuttings and giving a sketch of paper-cutting, including the areas to cut out, selecting randomly from two main themes:  ''\textit{Dragon Boat Festival}'' and ''\textit{Wedding,}'' for 30 minutes. We then interviewed them to explore participants' personal understanding and perspectives on the paper-cutting design process and examined the key aspects of the design process, including steps, core factors, and challenges faced.
Then, based on the aforementioned themes, participants were asked to engage with a Large Language Model (LLM) to assist in the design process and use a Text-to-Image model to generate paper-cutting image in 10 minutes. After that, we collected feedback from participants on GenAI-aided paper-cutting design, including the shortcomings of the results, challenges in the design process, and their expectations and suggestions for the GenAI-aided system.

% 1. 以“龙”, “平安富贵”, “多子多福”,5min to think the requirement and primary content in this paper-cutting can match these given topics; 2. 10 min to interactive with Stable Diffusion to generate the paper-cuttings. They were asked to describe how their paper-cutting's primary object/content matched the given topics. After that, we first conducted a semi-structured interview to 
% 我们的访谈分为两部分,第一部分,我们首先基于两个主题:“迎接春天”和"婚礼"来询问它们设计思路。然后我们向受访者询问了每人对于剪纸设计过程的理解和看法,最后向每个人询问了有关设计过程一般是怎样组成的,有哪些关键要素需要考虑,有哪些步骤是具有挑战的。 第二部分,首先我们继续基于同样的两个主题来让用户使用LLM辅助进行设计,并结合输出的结果来让T2I模型生成最终的剪纸内容。 然后,我们向受访者询问了ai-辅助剪纸作品的设计的不足,整个过程存在的挑战,以及它们对于系统可以如何改进GenAI,来改善设计的期待和关心

% We summarized our findings regarding the core factors of workflow and challenges in paper-cutting creation with GenAI assistance.

% \begin{figure*}[!htbp]
% \centering
% \includegraphics[width=0.95\textwidth]{Images/workflow.pdf}
% \caption{\label{figure1}
% A general workflow for GenAI-aided paper-cutting design is outlined from the 2-step formative study, with the main challenges in the workflow labeled on corresponding stages. Based on the workflow, and challenges, the design goals are solidified to the pipeline and interface of HarmonyCut}
% \Description{This Figure shows a general workflow for GenAI-aided paper-cutting design is outlined from the 2-step formative study, with the main challenges in the workflow labeled on corresponding stages. Based on the workflow, and challenges, the design goals are solidified to develop HarmonyCut}
% \end{figure*}


\subsection{Findings}

Through the semi-structured interviews and literature review, we identified the workflow of paper-cutting design: ideation and composition. We found that ideation in paper-cutting design presents challenges to all participants, albeit to varying degrees. It is tedious to both participants and GenAI in the composition phase. In GenAI-aided designs, the recommended and generated content is often undesirable, leading to an uncontrollable overall process that cannot modify the output.

\subsubsection{Workflow of Paper-cutting Design}
% 根据专家的意见,剪纸设计主要分为两部,第一是将需求转变为一个 coceptual idea,这个idea 一般需要从几个方面来来考虑(在section 4会提及);第二个是将conceptual idea 落实为视觉的内容(即用剪刀进行剪裁前的设计图)
\revisedtext{Drawing on feedback from interviews, the suggested process of paper-cutting creation, especially in paper-cutting education~\cite{Lin:1974:howtopapercutting, Li:1998:monopapercutting, Li:2011:PatternandDesign, Zhang:1982:discusspapaercut}, and the design steps from Hubka et al.~\cite{Hubka:1992:engineeringdesign}, paper-cutting design primarily involves two main steps, as shown in \autoref{figure1}~(A).} The first step is transforming intents into a conceptual idea~\cite{Choi:2024:creativeconnect}.
% , which generally needs to be considered from several aspects (as will be discussed in \autoref{sec:content}). 
The second step is translating the conceptual idea into visual form (a design blueprint before using scissors for cutting).
\begin{itemize}
\item \textbf{Ideation.} The first step involves transforming intents with a theme into a conceptual idea~\cite{Li:1998:monopapercutting}. 
\rrtext{Based on the experts' feedback, several preliminary dimensions were mentioned, including \textbf{function and style}. These dimensions suggest that ideation should be approached from multiple dimensions (4 factors as detailed in \autoref{sec:4_2}) to determine the core components of the design.} This idea will later be translated into a visual design in the next step.
\item \textbf{Composition.} During the composition step, designers (1) select the shapes of the elements based on the idea, (2) arrange and combine the selected contours, and (3) decide on cut-out regions (unit patterns) for future creation~\cite{Lin:1974:howtopapercutting, Li:2011:PatternandDesign}.
\end{itemize}

\subsubsection{Challenges in Paper-cutting Design}
\revisedtext{Based on the observation in the formative study and literature review related to design and paper-cutting, we refined and summarized 5 challenges in paper-cutting design with GenAI assistance.}
\begin{itemize}
\item[\textbf{C1.}] \textbf{Challenges in Translating Intent to Ideas.}
% Novices and amateurs (P6 and P7) spent a considerable amount of time on the first part of the study, and each could only provide 2 vague descriptions of their design concept for each theme. This difficulty stems from their lack of fundamental knowledge in paper-cutting, including an understanding of the design workflow and the aspects that should be considered to fulfill the required theme. 
% \revisedtext{What's more, the observation of struggling to translate intent to idea aligns with Li's discussion on the cognitive approach in paper-cutting: they lack mapping thinking—a process that enables them to creatively link learned knowledge to the attributes of natural objects, assign relevant concepts, and employ structural methods to express their idea of paper-cutting designs~\cite{Li:1998:monopapercutting}. }
Novices and amateurs (P6 and P7) spent significant time on the first part of the study but could only provide two vague descriptions of their design concepts for each theme.
\revisedtext{This difficulty arises not only from their limited knowledge of paper-cutting, including familiarity with the design workflow and the essential elements needed to address the theme but also from the absence of a cognitive approach (i.e., ``mapping thinking'') described by Li~\cite{Li:1998:monopapercutting}. This approach enables them to creatively link their knowledge to the attributes of natural objects, assign meaningful concepts, and employ structural methods to effectively express their paper-cutting design ideas. All these competencies are essential to learn and apply for paper-cutting creation~\cite{Lin:1974:howtopapercutting, Zhang:2021:papercuttingteaching}.}
As noted by P7, ``\textit{It is easy to cut a paper-cutting based on a sketched outline, but besides the content, I am unsure of which dimensions need consideration in the design process.}'' \revisedtext{Besides, P6 stated, ``\textit{I don't know what content in paper-cutting can appropriately map to those creation intents.}''} Consequently, it is challenging to establish a clear direction for their ideas.

\item[\textbf{C2.}] \textbf{Lack of Creativity and Multiple Variations in Ideation.}
% Novices and amateurs face difficulty selecting multiple elements that align with their themes due to a limited understanding of paper-cutting subjects and their associated meanings and connotations.
% \revisedtext{Novices and amateurs (P6 and P7) struggled to select multiple elements that aligned with their themes in the first part of the study, primarily due to a limited understanding of paper-cutting subjects and their associated meanings and connotations. }
% \revisedtext{The cases in studies of Bloom~\cite{Bloom:1985:knowledgecreative}, Ericsson et al.~\cite{Ericsson:1994:knowledgecreative}, and Gardner~\cite{Gardner:1993:knowledgecreative} clearly indicate that long-term immersion in a discipline is a crucial prerequisite for creative ability, and knowledge serves as an indispensable cornerstone for innovative ideas~\cite{Weisberg:1999:creativityandknowledge}. Based on them, we infer that in the field of paper-cutting design, novices and amateurs who lack systematic knowledge of paper-cutting will face challenges in creative design. The observation in our study further validated this inference: P6 and P7 with limited knowledge of paper-cutting found it particularly difficult to select elements aligned with the theme, and their proposed design ideas tended to be repetitive.}
\revisedtext{Novices and amateurs (P6 and P7), who possessed limited knowledge of paper-cutting, struggled to select elements aligned with their themes, and their proposed ideas appeared repetitive. These findings are consistent with prior research~\cite{Ericsson:1994:knowledgecreative, Gardner:1993:knowledgecreative}, which highlight that long-term immersion in a discipline is essential for developing creative ability. Furthermore, knowledge is identified as a critical foundation for fostering innovative ideas~\cite{Weisberg:1999:creativityandknowledge}, supporting the finding that a lack of systematic knowledge hinders the creative potential of novices and amateurs in paper-cutting design.}
Conversely, experts and practitioners, although proficient in the creative process and capable of rapid ideation, tend to rely heavily on their accumulated experience and local cultural influences for themes and content. This dependence can lead to fixation on a single idea. As P4 said, ``\textit{Paper-cutting is highly regional, with the meaning of specific elements differing significantly even within the same province. Although there is diversity in paper-cutting, I am only familiar with the themes and elements from my region, resulting in more fixed forms and content across many themes.}''

\item[\textbf{C3.}] \textbf{Challenges in Converting Ideas into Concrete Visual Representations.}
Novices have limited drawing skills, making translating content in their ideas directly into visual forms challenging. Additionally, composition requires considering not only the spatial arrangement and structure of elements but also deciding which areas should be cut-out (pattern) and in what shapes during creation. It is labor-intensive for both novices and experts. As mentioned by P1, ``\textit{The arrangement of specific content and the shapes created through cut-outs (pattern) best reflect personal style. However, translating an idea into a visual expression is still laborious.}''

\item[\textbf{C4.}] \textbf{Challenges in Exploring Suitable and Rational GenAI Results.} 
For the given themes, the model struggles to grasp the user's unique design idea, often providing overly broad suggestions, which cannot assist the user in avoiding fixation even increase the load to user in exploration. Additionally, the finally generated paper-cutting images often do not match the text description, especially in the spatial arrangement of content. Moreover, many parts of the model-generated paper-cuttings are irrational or irrelevant, such as generating some random clusters of stripes as patterns in paper-cutting.

\item[\textbf{C5.}] \textbf{Challenges in Controlling and Editing GenAI Results.} 
Regarding the above issue with GenAI, it is difficult to directly adjust errors in the generated results. Participants can only try to improve the output by revising the input descriptions. However, because the model struggles to understand the knowledge of contents with nuanced meaning and composition, iterative changes to the input often yield minimal improvement.

\end{itemize}

% 用户难以获得剪纸相关的知识,从而进行符合需求的构思和创作:
% 首先,对于新手,它们缺乏对剪纸的基本知识,如创作流程和创作要素。 用户对创作需求只有模糊的想法,不知道该从哪些方面来考虑内容选取,从而满足创作需求。
% 而对于专家,虽然他们可以快速的明确创作流程和创作思路,但对于具体该选用哪些主题,内容和纹样来进行符合创作需求的作品,是具有挑战的。即使是专家,对于一些常见的主题,他们的创作内容也会陷入到固步自封(fixation)。
% 在GenAI辅助创作上,符合用户需求的剪纸的内容和纹样是多样的,而同一内容在不同情境下,或多种纹样的不同搭配方式都具有不同的意义。这些都是模型难以理解的。对于某种事物的剪纸内容生成往往限于固定的几种模式; 同时,在模型生成的剪纸内容中,有很多纹样是无意义、不合理,甚至只是一团胡乱的条纹。系统没有办法理解作为剪纸最核心的符号语言,纹样。
% 而如果让模型直接通过深度生成模型从文本输入到生成整个图像,结果很大程度上由人工智能主导,用户缺乏参与感,更无法对模型产生的问题直接做出修改与调整。正能迭代的重新生成

% 形成性研究发现,剪纸构思与创作过程存在显著挑战,主要体现在以下两个方面:
% 一、用户难以获取知识限制构思
% 新手用户普遍缺乏剪纸艺术的基本知识,包括创作流程和关键要素,导致难以根据创作需求进行有效构思。
% 专家用户虽能迅速明确创作流程与思路,但在选择适合创作需求的主题、内容及纹样时易陷入它们自己的创作定式(fixation)。
% 二、GenAI辅助创作中难以对结果修改,难以理解深层的领域知识
% 在利用GenAI进行剪纸创作时,模型难以全面理解剪纸艺术中与文化相关的复杂知识,如内容在不同情境下的意义变化及纹样搭配的深层含义。它们都导致GenAI生成的剪纸内容常受限于固定模式,结果中的纹样可能缺乏意义与合理性、甚至只是胡乱的条纹,未能充分体现纹样作为剪纸中符号语言的精髓。
% 当模型直接从文本输入生成剪纸图像时,用户参与度低,难以对生成结果进行即时修改与调整,限制了创作的灵活性与个性化。

