% \section{Pipeline}
% 剪纸系统的一些考虑和解释:剪纸作品从外观来看,剪纸色彩较为单一,最常见的 是红色,具有喜庆含义,其重点通过二维图案来展现它的艺术性。剪纸外轮廓二 维形状可以体现需要刻画的物体形象,剪纸内部二维图案有的具体表现某种实物,有的起到装饰点缀、表达寓意的作用

% 我们基于formative study 总结出的design goal以及design space,并参考了 C2Idea等的 系统设计流程, 提出了一个支持 GENAI-aided 的pipeline(有三个阶段),每个阶段都以及 user interaction,来增强设计的多样性、合理性和可控性
% \begin{itemize}
%     \item % 明确design intent
%     \item % 探索符合需求的内容(索引来/和生成的)recommend content for exploration and selection
%     \item % 组合并调整所选内容 (轮廓,位置,纹样) reference-based creation
% \end{itemize}


% 对于我们采用检索和生成的内容作为参考的一个pipeline 设计依据:有一个和reference retrieval and exploration有关的的好例子:关于创作过程,如肖等人。[54]已确定,主流作品遵循两个阶段的管道,包括检索示例并将其改编为设计材料[62]和风格转移参考[45]。