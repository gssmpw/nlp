\section{Conclusion}
% This work presents a systematic exploration of a specific domain of data storytelling that uses animated unit visualizations (AUVs). The research prototype, DataParticles, leveraged language-oriented authoring and block-based editing to address the pain points that exist when creating stories containing AUVs. With DataParticles, users could leverage the latent connections among text, data, and visualizations to quickly and flexibly prototype, explore, and iterate on both a story narrative and its corresponding visualizations. Feedback from creative experts confirmed the potential of this approach and pointed to future directions for improvement. We are excited to further extend the concepts in DataParticles to support the creation process of a broad range of visual content

% This study distills design knowledge from real-world examples, summarizes generalizable design patterns and simulatable design workflow, and explores the creation of semantic typographic logos by blending typeface and imagery while maintaining legibility. We introduce TypeDance, an authoring tool based on a generative model that supports a personalized design workflow including ideation, selection, generation, evaluation, and iteration. With TypeDance, creators can flexibly choose typefaces at different levels of granularity and blend them with specific imagery using combinable design factors. TypeDance also allows users to adjust the generated results along the typeface-imagery spectrum and offers post-editing for individual elements. Feedback from general users and experts validates the effectiveness of TypeDance and provides valuable insights for future opportunities. We are excited to enhance the functionality of TypeDance for a comprehensive workflow and explore new techniques and interactions to enhance human creativity.

% This paper proposed CreativeConnect, a system designed to support graphic designers in the reference recombination process, allowing them to generate novel design ideas. Building on our formative study observations, CreativeConnect assists users in identifying key elements within reference images. It also provides diverse recommendations for relevant keywords and recombination options. Notably, the low-fdelity sketch-based output of CreativeConnect was shown to encourage creativity by enabling further imaginative exploration. Our user study demonstrated that CreativeConnect efciently supported both steps of fnding and recombining elements and helped participants come up with more design ideas and perceive their ideas as more creative than the baseline. While CreativeConnect represents a promising step towards comprehensive recombination support tools for designers, we also suggested an opportunity to expand such systems to address a broader spectrum of design needs and situations

% This work explores the workflow and design space of paper-cutting with GenAI-aided. The research prototype, HarmonyCut, utilizes factor-oriented guidance for ideation and reference-based exploration for composition to address the challenges that exist when designing a paper-cutting. Our user study and expert interviews demonstrated that HarmonyCut enhances users' understanding of paper-cutting knowledge and aids in translating intent into ideas, and the rich exploration of references for paper-cutting composition, achieving greater engagement and performance than the baseline. Feedback from participants validate the workflow and design space comprehension and highlighted future directions for improvement. We are eager to further extend the concepts within HarmonyCut to support the traditional art design process with GenAI-aided.

This work systematically investigates the workflow and design space of paper-cutting with GenAI assistance. The research prototype, HarmonyCut, employs factor-oriented guidance for ideation and reference-based exploration for composition to address the challenges inherent in designing a paper-cutting. Our user study and expert interviews demonstrated that HarmonyCut enhances users' understanding of paper-cutting knowledge and assists in translating intent into ideas, as well as facilitating the rich exploration of references for paper-cutting composition, resulting in greater engagement and performance compared to the baseline. Feedback from participants validated the comprehension of the workflow and design space and highlighted future directions for improvement. We are eager to further extend the concepts within HarmonyCut to support the traditional art design process with GenAI assistance.

\begin{acks}
This research was partially supported by the National Natural Science Foundation of China (No. 62202217), Guangdong Basic and Applied Basic Research Foundation (No. 2023A1515012889), and Guangdong Key Program (No. 2021QN02X794). We thank all of our study participants for their insightful discussions, constructive feedback, and the dedication to the preservation and dissemination of paper-cutting.
We acknowledge the partial use of a large language model (LLM), specifically ChatGPT, to assist in the writing process. The LLM was employed as a tool for polishing the manuscript to enhance the clarity and quality of the text.
\end{acks}
