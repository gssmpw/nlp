\clearpage
\appendix
\section{\revisedtext{Formative Study}}\label{A:formative participants}

% \rrtext{正如我们在section 3.1中提到的,区分professional的重要标准是有没有接受系统性的训练(更详细的说,即向传承人拜师进行学习),或是经过足够的剪纸经验累积然后从业余变为professional。另外,在我们的采访中,P2-P4都强调他们甚至在3、4岁就自己有剪纸的兴趣,因此家人才请求传承人收这些孩童为徒,同时,收徒也证明师傅认可了他们的天赋。 因此,他们在5-6岁时就已经开始接受系统且专业的训练,而这种幼龄学童在民间剪纸传承中较为常见。 也正因为现在越来越多的new generation对此不感兴趣,所以ICH的传承收到挑战。}
\subsection{\rrtext{Expertise Levels of Paper-cutting}}\label{expert clarification}
\rrtext{As mentioned in~\autoref{A:formative participants}, the key criterion for distinguishing experts (i.e., master and practitioner) is whether they have undergone systematic training (specifically, apprenticing under an inheritor) or have transitioned from amateur to experts status through the accumulation of substantial paper-cutting experience and even been finally recognized as an inheritor. In our interviews, P1-P4 highlighted that they developed an interest in paper-cutting as early as 3-4 years old. Their families subsequently petitioned inheritors to accept these children as apprentices, and the acceptance of such requests indicated the masters' recognition of their talent. As a result, they began receiving systematic and professional training by the age of 5-9. This practice of training young children is relatively common in the Chinese paper-cutting inheritance. However, because this tradition relies on early engagement, the decreasing interest among the younger generation has partially contributed to the challenges faced in the inheritance of ICH, including paper-cutting.}

\begin{table}[H]
\caption{\revisedtext{Summary of participants interviewed in formative study.}}
  \Description{This table demonstrates the summary of participants interviewed in the formative study.}
  \label{table:formative participants}
\resizebox{0.49\textwidth}{!}{
\renewcommand\arraystretch{1.4}
\begin{tabular}{ccccccc}
\hline
ID & Sex    & Age & Paper-cutting Expertise & GenAI Expertise     & Location      & Platform \\ \hline
P1 & Male   & 30  & Master (21 years)       & Novice             & Central China & Bilibili \\
P2 & Female & 28  & Practitioner (18 years) & Knowledgeable User             & East China    & Bilibili \\
P3 & Female & 24  & Practitioner (18 years) & Novice             & Southwest     & Bilibili \\
P4 & Male   & 49  & Master (40 years)       & Novice             & Northwest     & Douyin   \\
P5 & Female & 59  & Master (40+ years)      & Novice             & Northeast     & Douyin   \\
P6 & Male   & 25  & Novice                  & Professional       & North China   & WeChat   \\
P7 & Male   & 26  & Amateur (3 years)       & Knowledgeable User & Central China & WeChat   \\ \hline
\end{tabular}
}
\end{table}

\section{Content Analysis}

\subsection{\revisedtext{Information of 140 Sampled Images}}\label{A:sample filter}
\revisedtext{As a regionally influenced art form, we utilized the regional distribution as the standard for the selection of 140 paper-cuttings, as illustrated in~\autoref{figure:sample filter}.}
\begin{figure}[H]
\centering
\includegraphics[width=0.49\textwidth]{Images/slected_140_distribution.png}
\caption{\label{figure:sample filter} The human geography region distribution between 140 sampled and total 701 paper-cuttings.}
\Description{The distribution between total 701 and 140 sampled paper-cuttings. This figure compared the human geography region distribution of randomly sampled subset and total 701 papaer-cuttings.}
\end{figure}

\subsection{\rrtext{Description of Expert Discussion in the Content Analysis}}
\subsubsection{\rrtext{For the Codebook of Core Factors of Paper-cutting Design Ideation}}\label{A:expert discuss ideation}
% 第一轮只基于style 和 function,因此提出应该还考虑题材还有method of expression,第二轮认为植物类和动物类还有瑞兽类都可以归纳为动植物;第三轮认为目前的版本基本合理
\rrtext{The initial version of the codebook was developed solely based on style and function. Thus, in the first round of discussions, the experts recommended adding two dimensions, including subject matter (P1-P2 and P4-P5) and method of expression (P2 and P4-P5), which were accepted through all experts' consensus. During the second round, P2 and P5 suggested merging specific types, such as plants, animals, and mythical creatures into a broader ``flora and fauna'' type. By the third round of discussions, the experts collectively agreed that the current version of the codebook was generally adequate.}

\subsubsection{\rrtext{For the Codebook of Patterns in Paper-cutting}}\label{A:expert discuss pattern}
% 在第一轮讨论中,专家们认为基于unit geometry pattern, 专家认为还有很多虽然也是几何图形,但它们的语义并不是抽象的,而是更practical的,例如月牙纹,云纹,火纹等,故总结出unit semantic pattern。而对于composite pattern,专家认为与decorative pattern对应的就是要实现function,表达核心主题的主体pattern,因此composite pattern应该分为两类。第二轮认为目前的版本基本合理。
\rrtext{During the first round of discussions, the experts observed that, in addition to unit geometry and sawtooth patterns, many \textit{Unit Patterns}, such as crescent, cloud, and fire patterns, possess more practical semantics rather than purely abstract ones. As a result, they recommended introducing a new sub-category within \textit{Unit Patterns}, termed unit semantic patterns. Regarding \textit{Composite Patterns}, the experts suggested distinguishing a type that emphasizes functionality and conveys core themes, referred to as composite primary patterns, as opposed to decorative patterns. Following the second round of discussions, the experts collectively agreed that the current version of the codebook was generally appropriate.}

\subsection{\revisedtext{Coding Results of the Ideation Factors of Paper-cuttings}}\label{A:paper-cut coding}
\revisedtext{The first author validated that the taxonomy of ideation factors derived from the analysis of 70 paper-cuttings represented the entire set of 140 paper-cuttings in~\autoref{figure:paper-cut validation}.}
\begin{figure}[H]
\centering
\includegraphics[width=0.49\textwidth]{Images/paper-cut-val.png}
\caption{\label{figure:paper-cut validation}The coding distribution results of Factors and Types for 70 selections and results in 140 paper-cuttings. Types with less than 5\% are not presented with specific percentages. The differences between the corresponding Types are less than 5\% for all Factors, indicating a similar distribution across both datasets.}
\Description{This figure shows coding distribution results of Factors and Types for 70 selections and results in 140 paper-cuttings, which validate the consistency of coding from the construction to the result.}
\end{figure}

\subsection{Example of Paper-cuttings with Different Type under Design Factors}\label{A:content examples}
\revisedtext{A sequential presentation of paper-cutting examples is provided in~\autoref{a1fig1} and~\autoref{a1fig2}, with each example illustrating the corresponding ideation factor types.}
\begin{figure}[H]
\centering
\includegraphics[width=0.49\textwidth]{Images/papercut/A-figure1.pdf}
\caption{\label{a1fig1} Paper-cutting examples that meet design factors. (a) Witchcraft Belief; (b) Indigenous Belief; (c) Religious Belief; (d) Cultural Dissemination; (e) Interpersonal Communication; (f) Festive Atmosphere Evoking; (g) Daily Decoration; (h) Primitive Paper-cutting;  (i) Flora and Fauna; (j) Landscape.}
\Description{This figure shows the paper-cutting examples that meet design factors. (a) Witchcraft Belief; (b) Indigenous Belief; (c) Religious Belief; (d) Cultural Dissemination; (e) Interpersonal Communication; (f) Festive Atmosphere Evoking; (g) Daily Decoration; (h) Primitive Paper-cutting;  (i) Flora and Fauna; (j) Landscape.}
\end{figure}

\begin{figure}[H]
\centering
\includegraphics[width=0.49\textwidth]{Images/papercut/A-figure2.pdf}
\caption{\label{a1fig2} Paper-cutting examples that meet design factors (a) Historical Figure and Story; (b) Folk Life; (c) Contemporary Subject; (d) Abstract Style; (e) Realistic Style; (f) Metaphor; (g) Symbolism; (h) Homophony.}
\Description{This figure shows the paper-cutting examples that meet design factors (a) Historical Figure and Story; (b) Folk Life; (c) Contemporary Subject; (d) Abstract Style; (e) Realistic Style; (f) Metaphor; (g) Symbolism; (h) Homophony.}
\end{figure}

\subsection{\revisedtext{Coding Results of the Patterns of Paper-cuttings}}\label{A:pattern coding}
\revisedtext{The first author validated the taxonomy of patterns, including the \textit{Unit Pattern}, derived from the analysis of 635 cut-outs representing the entire set of 1269 cut-outs, and the \textit{Composite Pattern}, derived from the analysis of all composite patterns in 70 paper-cuttings, representing the entire set of 140 paper-cuttings, as shown in~\autoref{figure:pattern validation}.}
\begin{figure}[H]
\centering
\includegraphics[width=0.49\textwidth]{Images/pattern-val.png}
\caption{\label{figure:pattern validation}The coding distribution results: 635 selections and 1269 cut-outs for the Unit Pattern, 70 selections and 140 paper-cuttings for the Composite Pattern. Each sub-category with less than 5\% is not presented with specific percentages. The differences between the corresponding sub-categories are less than 5\% for all Factors, indicating a similar distribution across both datasets.}
\Description{This figure shows coding distribution results. The coding distribution results: 635 selections and 1269 cut-outs for the Unit Pattern, 70 selections and 140 paper-cuttings for the Composite Pattern, which validate the consistency of coding from the construction to the result.}
\end{figure}


\section{\rrtext{Evaluation}}
\begin{table}[H]
\caption{\rrtext{Summary of participants in the user study and the expert interview.}}
  \Description{This table demonstrates the summary of participants in the user study and the expert interview.}
  \label{table:evaluation participants}
\resizebox{0.49\textwidth}{!}{
\renewcommand\arraystretch{1.4}
\begin{tabular}{ccccccc}
\hline
ID  & Sex    & Age & Paper-cutting Expertise & GenAI Expertise    & Location      & Platform \\ \hline
P1  & Male   & 22  & Novice                  & Knowledgeable User & North China   & WeChat   \\
P2  & Female & 28  & Practitioner (18 years) & Knowledgeable User & East China    & Bilibili \\
P3  & Male   & 24  & Novice                  & Knowledgeable User & East China    & WeChat   \\
P4  & Male   & 22  & Novice                  & Knowledgeable User & East China    & WeChat   \\
P5  & Male   & 26  & Amateur (3 years)       & Knowledgeable User & Central China & WeChat   \\
P6  & Female & 25  & Novice                  & Knowledgeable User & Southwest     & WeChat   \\
P7  & Male   & 49  & Master (40 years)       & Novice             & Northwest     & Douyin   \\
P8  & Female & 24  & Practitioner (18 years) & Novice             & Southwest     & Bilibili \\
P9  & Male   & 19  & Novice                  & Knowledgeable User & Southwest     & WeChat   \\
P10 & Male   & 19  & Novice                  & Knowledgeable User & South China   & WeChat   \\
P11 & Male   & 23  & Novice                  & Novice             & East China    & WeChat   \\
P12 & Male   & 18  & Novice                  & Novice             & North China   & WeChat   \\
P13 & Female & 24  & Novice                  & Knowledgeable User & South China   & WeChat   \\
P14 & Female & 18  & Amateur (1 year)        & Knowledgeable User & South China   & WeChat   \\
P15 & Female & 19  & Amateur (2 years)       & Knowledgeable User & Central China & WeChat   \\
P16 & Male   & 23  & Novice                  & Knowledgeable User & Southwest     & WeChat   \\ \hline
E1  & Male   & 30  & Master (21 years)       & Novice             & Central China & Bilibili \\
E2  & Female & 59  & Master (40+ years)      & Novice             & Northeast     & Douyin   \\
E3  & Female & 40  & Master (24 years)       & Novice             & Central China & Douyin   \\ \hline
\end{tabular}
}
\end{table}
