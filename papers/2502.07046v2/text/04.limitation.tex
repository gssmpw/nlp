\section{Limitations and future work}\label{sec:limitations}
\textbf{Documentation Quality Analysis:} Meaningfulness evaluation for the \docstring and linked code can not be automatized and depends on the project context. To handle this limitation, we conducted a manual validation. As part of future work, \snipgen should streamline the manual validation process and ascertain the significance of comments and documentation within the snippets.

\textbf{Vulnerability Detection:} The detection of vulnerabilities is reliant solely on the CodeQL tool and its updates; we did not employ any alternative tools to validate these results.

\textbf{Assumption Regarding Snippet Exposure}: \snipgen mitigates to select ``contaminated'' data (\ie already seen snippets) by selecting snippets from specific commit time windows. A practitioner can specify the time windows depending on the \llm release date. \snipgen aims to reduce data contamination by including a prompt and variating the cut code. However, it's important to note that the extracted code changes might include older lines of code or reused code fragments. Our evaluation does not encompass the entire project history to identify older references.

For future work, 1) we propose to extend this dataset to support multiple programming languages; 2) Integrate a wider number of SE tasks. 3) Rank each data point according to the cyclomatic complexity number of AST nodes, documentation, and number of identifiers. The rank will prove better criteria on which snippets are more interesting to evaluate the \llm.