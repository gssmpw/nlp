\documentclass[conference]{IEEEtran}
\IEEEoverridecommandlockouts
%\usepackage{csquotes}
%\usepackage [autostyle, english = american]{csquotes}

%\usepackage{minted}
%\usepackage{verbatim}
% The preceding line is only needed to identify funding in the first footnote. If that is unneeded, please comment it out.
\usepackage{cite}
\usepackage{amsmath,amssymb,amsfonts}
\usepackage{algorithmic}
\usepackage{graphicx}
\usepackage{textcomp}
\usepackage{xcolor}
\usepackage{url}
%\usepackage{pdflscape}
\usepackage{threeparttable}
\def\BibTeX{{\rm B\kern-.05em{\sc i\kern-.025em b}\kern-.08em
    T\kern-.1667em\lower.7ex\hbox{E}\kern-.125emX}}

%
\setlength\unitlength{1mm}
\newcommand{\twodots}{\mathinner {\ldotp \ldotp}}
% bb font symbols
\newcommand{\Rho}{\mathrm{P}}
\newcommand{\Tau}{\mathrm{T}}

\newfont{\bbb}{msbm10 scaled 700}
\newcommand{\CCC}{\mbox{\bbb C}}

\newfont{\bb}{msbm10 scaled 1100}
\newcommand{\CC}{\mbox{\bb C}}
\newcommand{\PP}{\mbox{\bb P}}
\newcommand{\RR}{\mbox{\bb R}}
\newcommand{\QQ}{\mbox{\bb Q}}
\newcommand{\ZZ}{\mbox{\bb Z}}
\newcommand{\FF}{\mbox{\bb F}}
\newcommand{\GG}{\mbox{\bb G}}
\newcommand{\EE}{\mbox{\bb E}}
\newcommand{\NN}{\mbox{\bb N}}
\newcommand{\KK}{\mbox{\bb K}}
\newcommand{\HH}{\mbox{\bb H}}
\newcommand{\SSS}{\mbox{\bb S}}
\newcommand{\UU}{\mbox{\bb U}}
\newcommand{\VV}{\mbox{\bb V}}


\newcommand{\yy}{\mathbbm{y}}
\newcommand{\xx}{\mathbbm{x}}
\newcommand{\zz}{\mathbbm{z}}
\newcommand{\sss}{\mathbbm{s}}
\newcommand{\rr}{\mathbbm{r}}
\newcommand{\pp}{\mathbbm{p}}
\newcommand{\qq}{\mathbbm{q}}
\newcommand{\ww}{\mathbbm{w}}
\newcommand{\hh}{\mathbbm{h}}
\newcommand{\vvv}{\mathbbm{v}}

% Vectors

\newcommand{\av}{{\bf a}}
\newcommand{\bv}{{\bf b}}
\newcommand{\cv}{{\bf c}}
\newcommand{\dv}{{\bf d}}
\newcommand{\ev}{{\bf e}}
\newcommand{\fv}{{\bf f}}
\newcommand{\gv}{{\bf g}}
\newcommand{\hv}{{\bf h}}
\newcommand{\iv}{{\bf i}}
\newcommand{\jv}{{\bf j}}
\newcommand{\kv}{{\bf k}}
\newcommand{\lv}{{\bf l}}
\newcommand{\mv}{{\bf m}}
\newcommand{\nv}{{\bf n}}
\newcommand{\ov}{{\bf o}}
\newcommand{\pv}{{\bf p}}
\newcommand{\qv}{{\bf q}}
\newcommand{\rv}{{\bf r}}
\newcommand{\sv}{{\bf s}}
\newcommand{\tv}{{\bf t}}
\newcommand{\uv}{{\bf u}}
\newcommand{\wv}{{\bf w}}
\newcommand{\vv}{{\bf v}}
\newcommand{\xv}{{\bf x}}
\newcommand{\yv}{{\bf y}}
\newcommand{\zv}{{\bf z}}
\newcommand{\zerov}{{\bf 0}}
\newcommand{\onev}{{\bf 1}}

% Matrices

\newcommand{\Am}{{\bf A}}
\newcommand{\Bm}{{\bf B}}
\newcommand{\Cm}{{\bf C}}
\newcommand{\Dm}{{\bf D}}
\newcommand{\Em}{{\bf E}}
\newcommand{\Fm}{{\bf F}}
\newcommand{\Gm}{{\bf G}}
\newcommand{\Hm}{{\bf H}}
\newcommand{\Id}{{\bf I}}
\newcommand{\Jm}{{\bf J}}
\newcommand{\Km}{{\bf K}}
\newcommand{\Lm}{{\bf L}}
\newcommand{\Mm}{{\bf M}}
\newcommand{\Nm}{{\bf N}}
\newcommand{\Om}{{\bf O}}
\newcommand{\Pm}{{\bf P}}
\newcommand{\Qm}{{\bf Q}}
\newcommand{\Rm}{{\bf R}}
\newcommand{\Sm}{{\bf S}}
\newcommand{\Tm}{{\bf T}}
\newcommand{\Um}{{\bf U}}
\newcommand{\Wm}{{\bf W}}
\newcommand{\Vm}{{\bf V}}
\newcommand{\Xm}{{\bf X}}
\newcommand{\Ym}{{\bf Y}}
\newcommand{\Zm}{{\bf Z}}

% Calligraphic

\newcommand{\Ac}{{\cal A}}
\newcommand{\Bc}{{\cal B}}
\newcommand{\Cc}{{\cal C}}
\newcommand{\Dc}{{\cal D}}
\newcommand{\Ec}{{\cal E}}
\newcommand{\Fc}{{\cal F}}
\newcommand{\Gc}{{\cal G}}
\newcommand{\Hc}{{\cal H}}
\newcommand{\Ic}{{\cal I}}
\newcommand{\Jc}{{\cal J}}
\newcommand{\Kc}{{\cal K}}
\newcommand{\Lc}{{\cal L}}
\newcommand{\Mc}{{\cal M}}
\newcommand{\Nc}{{\cal N}}
\newcommand{\nc}{{\cal n}}
\newcommand{\Oc}{{\cal O}}
\newcommand{\Pc}{{\cal P}}
\newcommand{\Qc}{{\cal Q}}
\newcommand{\Rc}{{\cal R}}
\newcommand{\Sc}{{\cal S}}
\newcommand{\Tc}{{\cal T}}
\newcommand{\Uc}{{\cal U}}
\newcommand{\Wc}{{\cal W}}
\newcommand{\Vc}{{\cal V}}
\newcommand{\Xc}{{\cal X}}
\newcommand{\Yc}{{\cal Y}}
\newcommand{\Zc}{{\cal Z}}

% Bold greek letters

\newcommand{\alphav}{\hbox{\boldmath$\alpha$}}
\newcommand{\betav}{\hbox{\boldmath$\beta$}}
\newcommand{\gammav}{\hbox{\boldmath$\gamma$}}
\newcommand{\deltav}{\hbox{\boldmath$\delta$}}
\newcommand{\etav}{\hbox{\boldmath$\eta$}}
\newcommand{\lambdav}{\hbox{\boldmath$\lambda$}}
\newcommand{\epsilonv}{\hbox{\boldmath$\epsilon$}}
\newcommand{\nuv}{\hbox{\boldmath$\nu$}}
\newcommand{\muv}{\hbox{\boldmath$\mu$}}
\newcommand{\zetav}{\hbox{\boldmath$\zeta$}}
\newcommand{\phiv}{\hbox{\boldmath$\phi$}}
\newcommand{\psiv}{\hbox{\boldmath$\psi$}}
\newcommand{\thetav}{\hbox{\boldmath$\theta$}}
\newcommand{\tauv}{\hbox{\boldmath$\tau$}}
\newcommand{\omegav}{\hbox{\boldmath$\omega$}}
\newcommand{\xiv}{\hbox{\boldmath$\xi$}}
\newcommand{\sigmav}{\hbox{\boldmath$\sigma$}}
\newcommand{\piv}{\hbox{\boldmath$\pi$}}
\newcommand{\rhov}{\hbox{\boldmath$\rho$}}
\newcommand{\upsilonv}{\hbox{\boldmath$\upsilon$}}

\newcommand{\Gammam}{\hbox{\boldmath$\Gamma$}}
\newcommand{\Lambdam}{\hbox{\boldmath$\Lambda$}}
\newcommand{\Deltam}{\hbox{\boldmath$\Delta$}}
\newcommand{\Sigmam}{\hbox{\boldmath$\Sigma$}}
\newcommand{\Phim}{\hbox{\boldmath$\Phi$}}
\newcommand{\Pim}{\hbox{\boldmath$\Pi$}}
\newcommand{\Psim}{\hbox{\boldmath$\Psi$}}
\newcommand{\Thetam}{\hbox{\boldmath$\Theta$}}
\newcommand{\Omegam}{\hbox{\boldmath$\Omega$}}
\newcommand{\Xim}{\hbox{\boldmath$\Xi$}}


% Sans Serif small case

\newcommand{\Gsf}{{\sf G}}

\newcommand{\asf}{{\sf a}}
\newcommand{\bsf}{{\sf b}}
\newcommand{\csf}{{\sf c}}
\newcommand{\dsf}{{\sf d}}
\newcommand{\esf}{{\sf e}}
\newcommand{\fsf}{{\sf f}}
\newcommand{\gsf}{{\sf g}}
\newcommand{\hsf}{{\sf h}}
\newcommand{\isf}{{\sf i}}
\newcommand{\jsf}{{\sf j}}
\newcommand{\ksf}{{\sf k}}
\newcommand{\lsf}{{\sf l}}
\newcommand{\msf}{{\sf m}}
\newcommand{\nsf}{{\sf n}}
\newcommand{\osf}{{\sf o}}
\newcommand{\psf}{{\sf p}}
\newcommand{\qsf}{{\sf q}}
\newcommand{\rsf}{{\sf r}}
\newcommand{\ssf}{{\sf s}}
\newcommand{\tsf}{{\sf t}}
\newcommand{\usf}{{\sf u}}
\newcommand{\wsf}{{\sf w}}
\newcommand{\vsf}{{\sf v}}
\newcommand{\xsf}{{\sf x}}
\newcommand{\ysf}{{\sf y}}
\newcommand{\zsf}{{\sf z}}


% mixed symbols

\newcommand{\sinc}{{\hbox{sinc}}}
\newcommand{\diag}{{\hbox{diag}}}
\renewcommand{\det}{{\hbox{det}}}
\newcommand{\trace}{{\hbox{tr}}}
\newcommand{\sign}{{\hbox{sign}}}
\renewcommand{\arg}{{\hbox{arg}}}
\newcommand{\var}{{\hbox{var}}}
\newcommand{\cov}{{\hbox{cov}}}
\newcommand{\Ei}{{\rm E}_{\rm i}}
\renewcommand{\Re}{{\rm Re}}
\renewcommand{\Im}{{\rm Im}}
\newcommand{\eqdef}{\stackrel{\Delta}{=}}
\newcommand{\defines}{{\,\,\stackrel{\scriptscriptstyle \bigtriangleup}{=}\,\,}}
\newcommand{\<}{\left\langle}
\renewcommand{\>}{\right\rangle}
\newcommand{\herm}{{\sf H}}
\newcommand{\trasp}{{\sf T}}
\newcommand{\transp}{{\sf T}}
\renewcommand{\vec}{{\rm vec}}
\newcommand{\Psf}{{\sf P}}
\newcommand{\SINR}{{\sf SINR}}
\newcommand{\SNR}{{\sf SNR}}
\newcommand{\MMSE}{{\sf MMSE}}
\newcommand{\REF}{{\RED [REF]}}

% Markov chain
\usepackage{stmaryrd} % for \mkv 
\newcommand{\mkv}{-\!\!\!\!\minuso\!\!\!\!-}

% Colors

\newcommand{\RED}{\color[rgb]{1.00,0.10,0.10}}
\newcommand{\BLUE}{\color[rgb]{0,0,0.90}}
\newcommand{\GREEN}{\color[rgb]{0,0.80,0.20}}

%%%%%%%%%%%%%%%%%%%%%%%%%%%%%%%%%%%%%%%%%%
\usepackage{hyperref}
\hypersetup{
    bookmarks=true,         % show bookmarks bar?
    unicode=false,          % non-Latin characters in AcrobatÕs bookmarks
    pdftoolbar=true,        % show AcrobatÕs toolbar?
    pdfmenubar=true,        % show AcrobatÕs menu?
    pdffitwindow=false,     % window fit to page when opened
    pdfstartview={FitH},    % fits the width of the page to the window
%    pdftitle={My title},    % title
%    pdfauthor={Author},     % author
%    pdfsubject={Subject},   % subject of the document
%    pdfcreator={Creator},   % creator of the document
%    pdfproducer={Producer}, % producer of the document
%    pdfkeywords={keyword1} {key2} {key3}, % list of keywords
    pdfnewwindow=true,      % links in new window
    colorlinks=true,       % false: boxed links; true: colored links
    linkcolor=red,          % color of internal links (change box color with linkbordercolor)
    citecolor=green,        % color of links to bibliography
    filecolor=blue,      % color of file links
    urlcolor=blue           % color of external links
}
%%%%%%%%%%%%%%%%%%%%%%%%%%%%%%%%%%%%%%%%%%%




\begin{document}
\bstctlcite{IEEEexample:BSTcontrol}
\title{
%SnipGen: A Comprehensive Repository Mining Tool for Evaluating LLMs for Code via Prompt Engineering
SnipGen: A Mining Repository Framework for Evaluating LLMs for Code 
}

\author{\IEEEauthorblockN{
Daniel Rodriguez-Cardenas, Alejandro Velasco, and
Denys Poshyvanyk}
\IEEEauthorblockA{Department of Computer Science,
William \& Mary\\
Williamsburg, VA\\
Email: dhrodriguezcar, svelascodimate, dposhyvanyk\{@wm.edu\}}}

%\author{\IEEEauthorblockN{Anonymous Author(s)}}

\maketitle

\begin{abstract}
%
Large Language Models (\llms), such as transformer-based neural networks trained on billions of parameters, have become increasingly prevalent in software engineering (SE). These models, trained on extensive datasets that include code repositories, exhibit remarkable capabilities for SE tasks. However, evaluating their effectiveness poses significant challenges, primarily due to the potential overlap between the datasets used for training and those employed for evaluation. To address this issue, we introduce \snipgen, a comprehensive repository mining framework designed to leverage prompt engineering across various downstream tasks for code generation. \snipgen aims to mitigate data contamination by generating robust testbeds and crafting tailored data points to assist researchers and practitioners in evaluating \llms for code-related tasks. In our exploratory study, \snipgen mined approximately $227K$ data points from $338K$ recent code changes in GitHub commits, focusing on method-level granularity. \snipgen features a collection of prompt templates that can be combined to create a Chain-of-Thought-like sequence of prompts, enabling a nuanced assessment of \llms' code generation quality. By providing the mining tool, the methodology, and the dataset, \snipgen empowers researchers and practitioners to rigorously evaluate and interpret \llms' performance in software engineering contexts.

%Large Language Models (\llms), like transformer-based neural networks trained on billions of parameters, have been increasingly adopted in software engineering. Such complex neural networks are trained on vast datasets containing both natural and programming languages. However, evaluating their effectiveness in performing specific tasks is challenging because of their size and complexity. To accurately assess these models, and the extent of their emerging capabilities, it's crucial to use the right inputs and diverse examples while avoiding bias. To address this challenge, we created \snipgen, a novel dataset that leverages prompt engineering across different downstream tasks for code generation. \snipgen provides crafted data points to assist researchers and practitioners in the evaluation of \llms in different scenarios. We used a semi-automatic approach to collect about $227K$ data points from $338K$ recent changes in code bases on \github, focusing on method-level granularity. Additionally, to guarantee the quality of our data, we manually validated the collected samples. Furthermore, \snipgen provides a set of templates, that can be combined to produce practical prompts for assessing the quality of \llms at different tasks. By providing both, the dataset and the methodology to build it, we aim to empower researchers and practitioners in evaluating and interpreting \llms. 


%Using a semi-automatic approach, we gathered $\approx 227K$ data points at method level granularity, extracted from $\approx 338K$ changes in recent code bases on \github. We selected the most relevant snippets based on a set of features that we calculated. From the collected samples, $\approx 57K$ of them have related documentation or comments. Furthermore, to ensure the data quality we manually validated the collected samples. We suggest a series of templates that, when combined, produce a set of practical prompts for achieving and assessing \llms. Through the provision of both this dataset and prompt generation, our goal is to empower researchers in the evaluation and interpretation of these models.
\end{abstract}

\begin{IEEEkeywords}
Deep learning, code generation, datasets, large language models, evaluation
\end{IEEEkeywords}


\section{INTRODUCTION}
Large Language Models (LLMs) have demonstrated remarkable capabilities across various tasks~\cite{Brown2020Few-Shot,Touvron2023LLaMA,openai2024gpt4}.
However, their reliance on static parametric knowledge~\cite{Kasai2023RealTimeQA, mallen2023trust} often leads to inaccuracy or hallucination in responses~\cite{Welleck2020Neural, min2023factscore}.
Retrieval-Augmented Generation (RAG)~\cite{Lewis2020RAG,Guu2020REALM,ram2023context,asai2023retrieval} supplements LLMs with relevant knowledge retrieved from external sources, attracting increasing research interest.
One critical challenge for existing RAG systems is how to effectively integrate internal parametric knowledge with external retrieved knowledge to generate more accurate and reliable results.
% However, indiscriminately utilizing the retrieved knowledge may introduce irrelevant or off-topic information, compromising the quality of the response\cite{shi2023Irrelevant}.


In existing RAG approaches, LLMs depend either \emph{highly} or \emph{conditionally} upon external knowledge.
The former consistently uses the retrieved content as supporting evidence~\cite{Lewis2020RAG, Guu2020REALM, trivedi2023interleaving}, which often introduces irrelevant or noisy information and overlooks the valuable internal knowledge in LLMs, resulting in sub-optimal results.
In comparison, the latter integrates external knowledge into LLMs conditionally based on specific strategies, such as characteristics of input query \cite{mallen2023trust, jeong2024adaptive, wang2023skr}, probability of generated tokens~\cite{jiang2023active, su2024dragin}, or relevance of retrieved content~\cite{zhang2023merging, Xu2024Search, liu2024raisf}.
The query-based and token-based strategies generally utilize a fixed question set or a predefined threshold to decide whether to incorporate external knowledge, limiting their effectiveness due to incomplete information; 
the relevance-based strategy employs an additional validation module to assess the retrieved content, with its accuracy largely determining the quality of the final responses.
%with the successful integration of external knowledge heavily dependent on the accuracy of this additional module.
 

In this work, we explore leveraging the \textbf{LLM itself} to determine the correct result by \textbf{holistically} evaluating the outputs generated with and without external knowledge.
As illustrated in \Figref{sample}, given a query ``What does Ctrl+Shift+T do?'', we instruct the LLM to generate the \emph{LLM Answer} (i.e., ``New tab'') and the corresponding explanation (i.e., reasoning steps) with its internal parametric knowledge.
Meanwhile, we employ a retriever to obtain the relevant passages from external knowledge bases and feed the query and the retrieved passages to the LLM to produce the \emph{RAG Answer} (i.e., ``T'') and the corresponding explanation. 
Next, we instruct the LLM to take the query, \emph{LLM Answer} with its explanation and \emph{RAG Answer} with its explanation as input to choose the more accurate one (i.e., ``New tab'').
In this manner, the relevant internal and external knowledge to the query is comprehensively considered, facilitating the LLM in generating accurate responses, while the RAG framework maintains its simplicity by not requiring additional modules.

% selects the final answer from the responses generated using its internal parametric knowledge and external retrieved knowledge, thereby achieving improved accuracy.


Accordingly, we devise a novel \framework\ RAG framework that empowers the LLM to identify the more accurate answer to a query by evaluating both \emph{LLM Answer} and \emph{RAG Answer}, along with their respective explanations.
We validate the performance of the proposed \framework~framework with two open-sourced LLMs (see Section \ref{sec:main-result}) and find that it tends to fail in some scenarios, which we attribute to its limited capacity in distinguishing the correct answer from the incorrect one. 
To enhance the accuracy of the LLM selecting the right one among multiple responses generated from different knowledge sources, we develop a~\approach~method, leveraging Direct Preference Optimization (DPO)~\cite{Rafailov2023DPO} to fine-tune the LLM with a curated Retrieval-Generation Preference (\textbf{RGP}) dataset.
To construct this RGP dataset, we employ GPT-3.5~\cite{openai2024gpt4} to generate an \emph{LLM Answer} and an \emph{RAG Answer} for each query sampled from  WebQuestions~\cite{berant2013webq}, SQuAD 2.0~\cite{rajpurkar2018squad} and SciQ~\cite{welbl2017sciq}, and then retain only the pairs consisting of one correct answer and one incorrect answer, each accompanied by its corresponding explanation.
It consists of $3,756$ pairs of \emph{LLM Answer} and \emph{RAG answer} with their respective explanations, which we promise to release to the public to facilitate future research. 
%With the constructed dataset, we apply the advanced Direct Preference Optimization (DPO)\cite{Rafailov2023DPO} to enhance open-source LLMs in both selecting and generating accurate responses.


%第五段 实验结果 insight 

With this dataset, we train two different LLMs, including Mistral-7B~\cite{jiang2023mistral7b} and LLaMa-2-13B-Chat~\cite{touvron2023llama2openfoundation}, and evaluate them on two widely used datasets, i.e., Natural Questions (NQ)~\cite{kwiatkowski2019natural} and TrivialQA~\cite{joshi2017triviaqa}.
It is demonstrated that our \approach~method consistently achieves high effectiveness across various retrieval settings and different LLMs, enhancing the robustness and stability of RAG systems.
Moreover, additional experiments reveal that our \approach~method not only enhances LLMs' ability to distinguish valid answers from noisy ones but also improves their answer generation capabilities.
We further validate the rationale of each design in our method through ablation studies, and conduct error case analyses to offer deeper insights into the limitations of our proposed method.


In summary, the major contributions of our paper are three-fold: 
\begin{itemize} []
\item  We introduce a novel \framework~ RAG framework that leverages LLMs to determine the correct answer by evaluating a pair of responses generated with internal parametric knowledge solely and also with external retrieved knowledge.
\item  We propose a \approach~method that applies Direct Preference Optimization (DPO) to enhance LLMs in both identifying and generating the correct answers with a curated Retrieval-Generation Preference (RGP) dataset.
\item Extensive experiments with two open-sourced LLMs achieve superior performance on two widely-used datasets, demonstrating the effectiveness of our proposed Self-Selection framework and Self-Selection-RGP method.
\end{itemize}

\section{The \snipgen Framework}\label{sec:methodology}

\snipgen is a framework to extract snippets at method granularity from \github. \snipgen follow steps for curating the extracted raw data and take features from the data such as the number of identifiers, vocabulary, tokens, etc. Features derived from their AST representations and further complementary data. Our dataset can potentially improve the quality of the predictions in downstream tasks by augmenting the prompts, thereby enabling \llms to perform more effectively.


\begin{figure}[ht]
		\centering
		\includegraphics[width=0.48\textwidth]{img/2_methodology/Model3.pdf}
		\caption{\snipgen Data collection and prompt generation }
        %\vspace{-0.5cm}
        \label{fig:collection}
\end{figure}

Fig. \ref{fig:collection} depicts the process followed by \snipgen to generate a testbed and the \snipgen architecture. The \snipgen architecture comprises components to collect, curate, store, extract, and generate a SE-oriented testbed. The process begins with \circled{1}, the Data Collection phase, where source code snippets are extracted with \textit{pydriller} library~\cite{pydriller} from selected repositories in \github given a tie window. A set of snippets representing a Python method is extracted from each commit. This is followed by \circled{2}, a Pre-processing step, where the -Data Curation- \snipgen component stores the raw data in a \textit{MySQL} database. Once the data is formatted and saved in the storage, \snipgen looks for exact match snippets and removes duplicates. The -Feature extractor- component parses the code into the AST representation using tree-sitter~\cite{tree_sitter} and computes associated features (\ie the number of AST levels, AST nodes, comments, function name).

The data validation at step \circled{3} is a manual evaluation where the authors confirm the dimension values and the meaningfulness of the \textit{Docstring} and linked code, the two authors first selected the docstring with more than 20 words and evaluate the description against the code snippet. The description must depict the steps or intention of the snippet.

The testbed generation step \circled{4}, filters the raw data, evaluates the Jaccard similarity, and identifies vulnerable code. The raw filtering depends on the SE task, for example for code completion \snipgen filters the valid code with more than two lines of code. \snipgen uses CodeQL~\cite{codeql_overview} for vulnerability detection and appends the vulnerability location on the snippet. Finally, step \circled{5} uses the selected snippets from \circled{4} and applies the prompt template to the aimed SE task generating a final prompt. 
\snipgen enables the model evaluation and benchmarking as used in \cite{galeras, astexplainer, syntax_capabilities}.
The following subsections include a detailed description of the features of each data point.

\subsection{Data Point Feature Structure}
\begin{figure}[h]
		\centering
		\includegraphics[width=0.48\textwidth]{img/2_methodology/Diagram.pdf}
		\caption{\snipgen data schema. The snippet represents the core commit collected with documentation.  Linked tables contain calculated features. }
       %\vspace{-0.5cm}
        \label{fig:diagram}
\end{figure}
\snipgen can collect a set of Python methods that serve as evaluative data points. Each data point has associated features at seven dimensions as observed at \figref{fig:diagram}. These seven dimensions describe the static feature from the snippet. We aim to link code fragments with their properties. The first dimension corresponds to snippets' identification,  which includes the \commitID (\ie commit hash), \textit{repository} name, \textit{path}, \textit{file\_name}, \funName, \commitMessage. The second dimension is related to the associated documentation \docstring. The \docstring extended to complementary natural language features such as \textit{n\_words, vocab\_size, language,} and \whitespaces. The third dimension corresponds to the snippet's syntactic information, which includes the actual code base, \ASTerrors, \ASTlevels, \ASTnodes,\textit{n\_words}, \textit{vocab\_size}, \tokenCount, and \whitespaces. The fourth dimension corresponds to canonical software metrics, which include \nloc, \complexity, \identifiers. The fifth dimension depicts the span position for vulnerabilities detected from the \textit{code} snippet. The sixth dimension is associated with the snippet mutation when the code is randomly cut one line after the signature, therefore \snipgen identifies the signature as the original \textit{snippetID} and cut code. Finally, the seventh dimension comprises the linked features to the generated prompt. \snipgen labels the prompt to the SE task and the prompt configuration.

\subsection{Software Engineering Tesbed Task}
\label{sec:se_tasks}

Data curation, pre-processing, and data validation produce a testbed oriented to evaluate a model. For instance, a \textit{RandomCut} and \textit{WithDocString} testbeds might evaluate the model at SE tasks, such as \textit{\textbf{code completion}}—generating code to fill in missing parts of a function. \textit{WithDocString} testbed selects the snippets with valid documentation and code so the \llm input compresses both a description and code.
\textit{FromCommit} testbed is focused on selecting meaningful commit messages and linked source code so that to evaluate either \textit{\textbf{commit generation}}—producing commit messages based on code changes or \textit{\textbf{code generation}} producing the complete snippet from the description.  \textit{FromDocString} testbed select only meaningful code descriptions (\ie only \docstring) to generate the code snippet also configuring a code generation case.
\snipgen can be used to evaluate \textit{\textbf{code summarization}}—creating natural language descriptions of the functionality implemented in the provided source code. If we select the original code from the \textit{WithDocString} testbed and the ones at the top of docstring length then we can use the testbed for summarization.



\subsection{Prompt templates}
\label{sec:prompt_templates}

% Please add the following required packages to your document preamble:
% \usepackage{multirow}
\begin{table}[t]
\centering
\caption{Prompt templates for each SE task using collected \snipgen features.}
%\vspace{-0.1cm}
\label{tab:templates}

\scalebox{0.7}{%
\setlength{\tabcolsep}{5pt} 
%\begin{tabular}{l|lp{3.5in}}
\begin{tabular}{lllp{3.6in}}
\toprule
\multicolumn{1}{c}{\textbf{SE Task}} &
  \multicolumn{1}{c}{\textbf{ID}} &
   &
  \multicolumn{1}{c}{\textbf{Prompt Template}} \\ \hline % \cline{1-2} \cline{4-4} 
\multirow{3}{*}{\textit{\textbf{\begin{tabular}[c]{@{}l@{}}Code \\ completion\end{tabular}}}} &
  $P1$ &
   &
  Complete the following \textless{}\textit{language}\textgreater method: \textless{}\randomCut\textgreater{} \\
 &
  $P2$ &
   &
  You have a \textless{}\textit{language}\textgreater function named \textless{}\textit{signature}\textgreater{}, the function starts with the following code \textless{}\randomCut\textgreater{}. The function is in charge of \textless{}\docstring\textgreater{} \\
 &
  $P3$ &
   &
  Create a function that accomplish the following functionality in \textless{}\textit{language}\textgreater code: \textless{}\docstring\textgreater{} \\ \hline % \cline{1-2} \cline{4-4} 
\textit{\textbf{\begin{tabular}[c]{@{}l@{}}Commit \\ generation\end{tabular}}} &
  $P4$ &
   &
  Please describe the following code change to create log message: status before \textless{}\randomCut\textgreater status now \textless{}code\textgreater{} \\ \hline % \cline{1-2} \cline{4-4} 
\textit{\textbf{Summ.}} &
  $P5$ &
   &
  I need a summary for the following code: \textless{}\textit{code}\textgreater{} \\ \hline % \cline{1-2} \cline{4-4} 
\multirow{3}{*}{\textit{\textbf{\begin{tabular}[c]{@{}l@{}}Processing \\ prompt\end{tabular}}}} &
  $P6$ &
   &
  Change the method signature by \textless{}signature\textgreater{} \\
 &
  $P7$ &
   &
  Reduce or complete the method using only \textless{}\nloc\textgreater lines of code \\
 &
  $P8$ &
   &
  remove comments; remove summary; remove throws; remove function modifiers\\
  \bottomrule
\end{tabular}
\vspace{-0.3cm}
}
\end{table}


The effectiveness of \llms in code generation is greatly influenced by prompt design. At this point \snipgen only produces a set of data points that can be organized as an input for an autoregressive \llm since the tesbed contains the input and the expected output. \snipgen combines prompt templates and gathered data points to build the final prompt input. The structure, keywords, and context of a prompt play a crucial role in shaping results and analyses. Prompts can be configured as a single-step or multi-step, with the latter allowing iterative refinement based on the model's initial response. Chau \etal \cite{liu_improving_2023} explore such multi-step configurations. Table \ref{tab:templates} lists eight prompt templates, practitioners can modify the template according to the evaluation task. From the proposed list, $P1-P5$ supports single-step SE tasks, while $P6-P8$ enables multi-step processing by combining prompts to refine outputs. For instance, \snipgen can combine $P1+P8$, $P3+P6$, or $P3+P8$ for code completion.


%The tool defines prompting templates used for each SE task, as detailed in \tabref{tab:templates}.

For \textit{\textbf{code completion}}, \snipgen defines three prompts. $P1$ asks the model to complete a method from a randomly selected cut position (\randomCut) in the specified programming language (\textit{language}). $P2$ extends $P1$ by including additional details, such as the method's \textit{signature} and \textit{docstring}. $P3$, in contrast, provides only an NL description extracted from the method's \textit{docstring}. In commit generation, $P4$ instructs the model to create an NL description of the changes made to transform the \randomCut version of a method into its complete code (\textit{code}). $P5$ is designed to ask the model to generate the commit message from the mutated code and the actual code. Lastly, in \textit{\textbf{code summarization}}, \textit{P6} provides only the code, which the model uses to generate a corresponding summary.



%For instance, \galeras uses \textit{CLD3} library for \textit{language} detection with a threshold of 0.9 to define sentence language such as English, Chinese, Spanish, etc.

%  

%Finally, we computed the number of lines of code (\nloc), and the cyclomatic \complexity to bring more information about the internal interaction and the \identifiers for the \textit{code}.



%The effectiveness of a \llms for code generation could be heavily influenced by the choice of prompt. The way we formulate the questions and actions, and introduce the context impacts the results as well as the analysis over the \llms. Therefore, the necessary keywords and structure for the prompt constitute the key aspect of building a functional prompt. Additionally, prompts could be configured to be multi-step or single-step. In other words, we can interact with the \llm and then provide more information or restrictions according to the answer in a second prompt. Chau \etal \cite{liu_improving_2023} describes the combination of prompts to be used at multi-step configuration. 

%Table \ref{tab:templates} lists a set of seven prompt templates that \snipgen can generate. We have five templates to support SE tasks using single-step prompt configuration and three templates for processing the prompt by combining them with the SE Tasks. The idea of the processing prompt is to guide and fine-tune the answer generated with the first interaction. For instance, \snipgen we can combine $P1+P7$, $P3+P6$, $P3+P7$ for code completion. 
%\DANIEL{Describe better the aim for each prompt, this took base from the CoT paper}



\subsection{\snipgen Prompt Generation and Use}


\begin{figure}[ht]
		\centering
		\includegraphics[width=0.5\textwidth]{img/2_methodology/SnipGen-Overview2.pdf}
		\caption{\snipgen Dataset and use. \textit{A} describes the \snipgen data collection and steps until prompt generation. \textit{B} describes the canonical path for training and evaluate \llms}
        %\vspace{-0.5cm}
        \label{fig:overview}
\end{figure}

The \snipgen framework is designed to select a SE task and evaluate a \llm using the testbed with a given context with a designed prompt. \figref{fig:overview} depicts the options a practitioner has to evaluate a \llm. The database supports a query to filter the snippets according to the SE task, for instance for code completion we can sample the snippets with linked \docstring with more than 10 words, this provides the task context (see, Fig. \ref{fig:overview} section \circled{A}). The prompt generation might contain a mutated snippet such as \randomCut to perform the required task. For example, for code completion, we will need a partial code snippet that must be auto-completed by the \llm therefore we need to cut the original snippet smartly. \snipgen can use the \randomCut method to split the code beyond the method signature. Practitioners can still evaluate the model using \textit{canonical} datasets and metrics to compare against the new collected \snipgen testbed.

\section{Experience Report}\label{sec:experience}

In this section, we describe our experience of using \snipgen for collecting a testbed and generating a set of prompts. We also briefly describe three use cases illustrating how \snipgen was successfully used to evaluate \llms for code.

\subsection{\snipgen Testbed Generation}

The experience with \snipgen begins by mining repositories from \github, as detailed in \secref{sec:methodology}. We focused on the most popular Python repositories, applying the following query filters: \textit{language:Python fork:false size:$>=30000$ pushed:$>$2021-12-31 stars: $> 2000$.} The query gathers the most popular repositories in Python. We selected the top 200 repositories including \textit{keras, numpy, pandas, sentry, etc.} We extracted the new snippets reported on commits between 2022 and 2023 from selected repositories. Then we used the \textit{data curation} to remove duplicates and \textit{feature extraction} to generate and extract the associated features. We configured a $0.7$ similarity threshold~\cite{Allamanis19, wang_neural_2019} to de-duplicate snippets using HuggingFace tokenizer BPE. \snipgen saves the raw data and their features into a JSON and a database.  We randomly validated $960$ out of $\approx 227K$  data points to confirm the extracted features and the meaningfulness of the \textit{Docstring} and linked code.

We sampled until $5k$ data points from the \textit{RawData} testbed to construct six testbeds, each tailored for a specific SE task as described at \secref{sec:se_tasks}. To create \randomCut, we selected data points with more than $10$ tokens or $100$ characters, and subsequently, each data point was randomly truncated after the method signature. For \summarizationGen and \commitGen, we filtered \textit{RawDataDocstring} data points with more than 10 words or 50 characters. Table.~\ref{tab:dedupe} provides information about the SE task associated with each curated testbed, the percentage of detected duplicates, the final size, and the generated number of prompts.


% Please add the following required packages to your document preamble:
% \usepackage{multirow}
% \usepackage[table,xcdraw]{xcolor}
% If you use beamer only pass "xcolor=table" option, i.e. \documentclass[xcolor=table]{beamer}
\begin{table}[t]
\centering
\caption{Dataset size and deduplication percentage}
%\vspace{-0.1cm}
\label{tab:dedupe}

\scalebox{0.72}{%
\setlength{\tabcolsep}{5pt} 
% Please add the following required packages to your document preamble:
% \usepackage[table,xcdraw]{xcolor}
% If you use beamer only pass "xcolor=table" option, i.e. \documentclass[xcolor=table]{beamer}

\begin{tabular}{llccccc}
\toprule
\multicolumn{1}{c}{\textbf{SE Task}} &
  \multicolumn{1}{c}{\textbf{Testbed}} &
  \textbf{I/O} &
  \textbf{Dupes} &
  \textbf{Dupe \%} &
  \textbf{Size} &
  \textbf{Prompts} \\ \hline
\multirow{2}{*}{\textbf{\begin{tabular}[c]{@{}l@{}}Code \\ Completion\end{tabular}}} &
  \textit{RandomCut} &
  code $\Rightarrow$ code &
  120 &
  2.4\% &
  4880 &
  9760 \\
                         & \textit{WithDocString}     & code\&text $\Rightarrow$ code & 145 & 2.9\% & 4855 & 9710 \\ \hline
\multirow{2}{*}{\textbf{\begin{tabular}[c]{@{}l@{}}Code \\ Generation\end{tabular}}} &
  \textit{FromDocString} &
  text $\Rightarrow$ code &
  76 &
  1.5\% &
  4924 &
  14772 \\
                         & \textit{FromCommit}        & text $\Rightarrow$ code       & 97  & 1.9\% & 4903 & 4903 \\ \hline
\textbf{Sumarization}    & \textit{SummarizationGen}  & code $\Rightarrow$ text       & 156 & 3.1\% & 4844 & 4844 \\ \hline
\textbf{Vulnerabilities} & \textit{VulnerabilitySpan} & code $\Rightarrow$ code       & 2   & 0.4\% & 410  & 410  \\ \bottomrule
\vspace{-0.90cm}
\end{tabular}
%\vspace{-4.90cm}

%%%%
}
\end{table}
\subsection{Successful Use Cases}\label{sec:cases}
\textit{\textbf{Galeras}}\cite{galeras}: Galeras is a benchmark for measuring the causal effect of SE prompts for code completion. Galeras configures a set of treatments to assess the influence of potential confounders on the outcomes of ChatGPT (\ie GPT-4). The selected confounders are: \textit{prompt\_size} (from prompts), \textit{n\_whitespaces} (from documentation), \textit{token\_counts}, and \textit{nloc} (from code\_features). This use case of \snipgen demonstrates that prompt engineering strategies (such as those listed in \tabref{tab:templates} - processing prompt) have distinct causal effects on the performance of ChatGPT.


\textit{\textbf{SyntaxEval}}~\cite{syntax_capabilities}: In this use case \textit{Syn taxEval} evaluates the ability of Masked Language Models (\ie Encoder-based Transformers) to predict tokens associated with specific types in the AST representation (\ie syntactic features). \textit{SyntaxEval} used \snipgen to construct a \textbf{\textit{code completion}} testbed with approximately $50K$ Python snippets. \textit{SyntaxEval} aims to account for potential confounders such as \textit{ast\_data} and \textit{code\_features} (illustrated in \figref{fig:diagram}), the analysis revealed no evidence that the evaluated syntactic features influenced the accuracy of the selected models' predictions.

\textit{\textbf{ASTxplainer}}~\cite{astexplainer}: \textit{ASTxplainer} is an explainability method designed to assess how effectively a \llm (\eg decoder-based transformers) predicts syntactic structures. \textit{ASTxplainer} aggregates next-token prediction values through syntactic decomposition, quantified as AsC-Eval values to evaluate the effectiveness. \textit{ASTxplainer} findings reveal that the ability to predict syntactic structures strongly depends on the \llm's parameter size and fine-tuning strategy. Furthermore, causal analysis controlling for confounding variables (e.g., \textit{ast\_data} and \textit{code\_features}) shows that AsC-Eval values at the snippet level negatively impact the cross-entropy loss of the evaluated \llms.
%by aggregating next-token prediction values through syntactic decomposition (\ie AsC-Eval values).  \textit{ASTxplainer} findings include that predicting syntactic structures highly depends on the LLMs’ parameter size and fine-tuning strategy. Additionally, after conducting causal analysis to control for confounding variables (\ie \textit{ast\_data} and \textit{code\_features}), the cross-entropy loss of the selected LLMs is negatively impacted by the AsC-Eval values at the snippet granularity.

\section{Similar Datasets}\label{sec:similar_datasets}

%\DANIEL{We need to reduce this section on the already known datsets}
%%%%% DATASETS IN FOR SOFTWARE TASKS 

Significant efforts have produced datasets for evaluating \llms in SE tasks, including DeepFix for program repair \cite{gupta_deepfix_2017}, CodeContest and CoNaLa for program synthesis \cite{li_competition-level_2022, yin2018mining}, and \textit{CodeSearchNet} for code retrieval \cite{husain2019codesearchnet}. Expansions like \textit{CodeXGLUE} \cite{lu_codexglue_2021}, xCodeEval \cite{khan_xcodeeval_2023} target broader tasks, while benchmarks such as \textit{HumanEval} and SecurityEval focus on functional correctness and vulnerabilities \cite{chen_evaluating_2021, secEval}. Despite these efforts, existing datasets often suffer from contamination\cite{jain_livecodebench_2024,yadav_pythonsaga_2024}, with overlaps between training and evaluation data, and benchmarks are prone to memorization by models\cite{ramos2024largelanguagemodelsmemorizing}, limiting their effectiveness in assessing true generalization.
%Significant efforts have been made to collect data for training and evaluating \llms in SE tasks. Examples include DeepFix \cite{gupta_deepfix_2017}, which focuses on program repair, and CodeContest \cite{li_competition-level_2022} and CoNaLa \cite{yin2018mining}, both designed for program synthesis. Husain \etal introduced \textit{CodeSearchNet}, a widely recognized dataset aimed at automating code retrieval \cite{husain2019codesearchnet}. Building on this foundation, Microsoft researchers expanded \textit{CodeSearchNet} by integrating 12 SE-related datasets targeting tasks such as clone detection, code refinement, and translation, resulting in the creation of the \textit{CodeXGLUE} benchmark \cite{lu_codexglue_2021}. Inspired by \textit{CodeXGLUE}, several task-specific benchmarks have emerged, including xCodeEval \cite{khan_xcodeeval_2023}, a multilingual multitask benchmark, and Galeras \cite{rodriguezcardenas2023benchmarking}, which evaluates \llms using snippet features through causal inference. Similarly, \textit{HumanEval} was developed to validate the functional correctness of generated code \cite{chen_evaluating_2021}, followed by HumanEval-X \cite{zheng_codegeex_2023}, which focuses on code generation and translation. In the context of security, SecurityEval \cite{secEval} assesses how models such as InCoder \cite{fried_incoder_2023} and Copilot \cite{noauthor_github_nodate} might introduce vulnerabilities during source code generation.

Recent research work has explored the dynamic generation of prompts and testbeds, for instance, \textit{EvoPrompt} is a framework for automatic discrete prompt optimization that connects \llms with Evolutionary Algorithms\cite{guo_connecting_2024}. \textit{Evol-instruct} is a systematic approach to generate instruction-response pairs by iteratively improving prompts and responses through model self-enhancement\cite{xu_wizardlm_2023}. \textit{LiveCodeBench} is a benchmark for evaluating \llms designed to generate code\cite{jain_livecodebench_2024}. Unlike \snipgen, LiveCodeBench addresses issues of data contamination by using continuously updated problems from online coding competitions. 

%\DANIEL{We need to add recent papers on benchmarks and testbed contamination}

%%%% Datasets for instruction Tuning
%\textit{\textbf{Instruction Tuning.}} More closely aligned with \snipgen, some datasets include prompt definitions designed to instruct \llms to perform specific tasks, an approach frequently adopted for \textit{instruction-tuning}. For instance, WizardCoder \cite{luo_wizardcoder_2023} was trained on a dataset derived from CodeAlpaca \cite{codealpaca} using the \textit{Evol-Instruct} method \cite{xu_wizardlm_2023}. 
 
%Considerable research efforts have been directed towards data collection for training and testing \llms. Initially, Husain et al. introduced \textit{CodeSearchNet} to automate code retrieval \cite{husain2019codesearchnet}. Microsoft researchers later expanded \textit{CodeSearchNet}, adding 12 SE-related datasets for various downstream tasks such as clone detection, refinement, and translation, resulting in \textit{CodeXGLUE} \cite{lu_codexglue_2021}. Secondly, Wainakh et al. introduced \textit{IdBench} to assess generated identifiers by measuring similarity distances of semantic representations \cite{wainakh_evaluating_2019}. Chen et al. introduced \textit{HumanEval} to validate the functional correctness of generated code \cite{chen_evaluating_2021} with specific cases on SE tasks. Cassano et al. expanded later \textit{HumanEval} to create \textit{MultiPL-E} for code translation \cite{cassano_multipl-e_2022}. Rodriguez \etal, \cite{rodriguezcardenas2023benchmarking} introduce a benchmark using snippet features to evaluate \llms via causal inference. Most of these datasets are focused and provide code snippets examples and descriptions, vulnerabilities datasets are created aside. The lack of datasets in \llms is demonstrated by Siddiq \etal \cite{secEval} that introduced SecurityEval, a benchmark assessing how InCoder \cite{fried_incoder_2023} and Copilot \cite{noauthor_github_nodate} might introduce vulnerabilities in source code generation. Similarly, Pearce \etal \cite{pearce_asleep_2022} have created a dataset featuring 89 CWE-based scenarios.ng, providing context, and using the prompt template, \snipgen can automatically generate a set of prompts for the specific SE task. 

%Researchers and practitioners can use the generated set of prompts for evaluating the \llms model enabling the analysis for interpretation or accuracy benchmark. \snipgen does not change any canonical path for the training and testing process (see, Fig. \ref{fig:overview} section \circled{B}). Indeed \snipgen supports the evaluation and tries to avoid the bias by generating mutated inputs and selecting unseeing snippets.

\section{Limitations and future work}\label{sec:limitations}
\textbf{Documentation Quality Analysis:} Meaningfulness evaluation for the \docstring and linked code can not be automatized and depends on the project context. To handle this limitation, we conducted a manual validation. As part of future work, \snipgen should streamline the manual validation process and ascertain the significance of comments and documentation within the snippets.

\textbf{Vulnerability Detection:} The detection of vulnerabilities is reliant solely on the CodeQL tool and its updates; we did not employ any alternative tools to validate these results.

\textbf{Assumption Regarding Snippet Exposure}: \snipgen mitigates to select ``contaminated'' data (\ie already seen snippets) by selecting snippets from specific commit time windows. A practitioner can specify the time windows depending on the \llm release date. \snipgen aims to reduce data contamination by including a prompt and variating the cut code. However, it's important to note that the extracted code changes might include older lines of code or reused code fragments. Our evaluation does not encompass the entire project history to identify older references.

For future work, 1) we propose to extend this dataset to support multiple programming languages; 2) Integrate a wider number of SE tasks. 3) Rank each data point according to the cyclomatic complexity number of AST nodes, documentation, and number of identifiers. The rank will prove better criteria on which snippets are more interesting to evaluate the \llm.


%\input{text/05.conclusions}

%%\input{text/06_acknowledgment}

\bibliographystyle{IEEEtran}
\bibliography{utils/citations_bib}

\end{document}
