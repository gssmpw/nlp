%%%%%%%% ICML 2025 EXAMPLE LATEX SUBMISSION FILE %%%%%%%%%%%%%%%%%

\documentclass{article}

% Recommended, but optional, packages for figures and better typesetting:
\usepackage{microtype}
\usepackage{graphicx}
% \usepackage{subfigure}
\usepackage{booktabs} % for professional tables
\usepackage{adjustbox}
\usepackage{tcolorbox}
% hyperref makes hyperlinks in the resulting PDF.
% If your build breaks (sometimes temporarily if a hyperlink spans a page)
% please comment out the following usepackage line and replace
% \usepackage{icml2025} with \usepackage[nohyperref]{icml2025} above.
\usepackage{hyperref}


% Attempt to make hyperref and algorithmic work together better:
\newcommand{\theHalgorithm}{\arabic{algorithm}}

% Use the following line for the initial blind version submitted for review:
\usepackage[accepted]{icml2025}

% If accepted, instead use the following line for the camera-ready submission:
% \usepackage[accepted]{icml2025}

\usepackage{url}
\usepackage{color, float}
\usepackage{xspace}
\usepackage{multirow}
\usepackage{fancyhdr}
\usepackage{booktabs} % for professional tables
\usepackage{nicefrac}       % compact symbols for 1/2, etc.
\usepackage{enumerate}
\usepackage{xspace}

% For theorems and such
\usepackage{amsthm, amsmath, mathtools, amsfonts, amssymb, dsfont, enumitem, bm, bbm, cancel}
\usepackage{graphicx}
\usepackage{subcaption}
\graphicspath{{./plots/}}

\definecolor{darkgreen}{rgb}{0.0, 0.5, 0.0}
% if you use cleveref..
\usepackage[capitalize,noabbrev, nameinlink]{cleveref}


%%%%%%%%%%%%%%%%%%%%%%%%%%%%%%%%
% THEOREMS
%%%%%%%%%%%%%%%%%%%%%%%%%%%%%%%%
\theoremstyle{plain}
\newtheorem{theorem}{Theorem}[section]
\newtheorem{proposition}[theorem]{Proposition}
\newtheorem{lemma}[theorem]{Lemma}
\newtheorem{corollary}[theorem]{Corollary}
\theoremstyle{definition}
\newtheorem{definition}[theorem]{Definition}
\newtheorem{assumption}[theorem]{Assumption}
\theoremstyle{remark}
\newtheorem{remark}[theorem]{Remark}



\newcommand{\Xtilde}{{\widetilde{X}}}
\newcommand{\fenc}{{f_\text{enc}}}
\newcommand{\fdec}{{f_\text{dec}}}
\newcommand{\sg}{{\texttt{sg}\xspace}}
\newcommand{\increase}[1]{\textcolor{darkgreen}{($\uparrow$ +#1)}}

\newcommand{\decrease}[1]{\textcolor{darkgreen}{\textbf{($\downarrow$ -#1)}}}

\definecolor{darkred}{rgb}{0.6, 0, 0}
% Modify the command to use the dark red color
\newcommand{\increasered}[1]{\textcolor{darkred}{($\uparrow$ +#1)}}


\newcommand{\phh}[1]{}



\newcommand{\V}{{\mathcal{V}}}
\newcommand{\E}{{\mathcal{E}}}
\newcommand{\R}{{\mathbb{R}}}
\newcommand{\norm}[1]{\left\|#1\right\|}
\DeclareMathOperator*{\argmin}{argmin}
\DeclareMathOperator*{\argmax}{argmax}

\newcommand{\prompt}{\texttt{[Prompt]}:}
\newcommand{\response}{\texttt{[Response]}:}

% Todonotes is useful during development; simply uncomment the next line
%    and comment out the line below the next line to turn off comments
%\usepackage[disable,textsize=tiny]{todonotes}
\usepackage[textsize=tiny]{todonotes}

\newcommand{\qq}[1]{\textcolor{orange}{[QQ: #1]}}
\newcommand{\yc}[1]{\textcolor{blue}{YC: #1}}
\newcommand{\hz}[1]{\textcolor{purple}{[HZ: #1]}}

\counterwithin{figure}{section}
\counterwithin{table}{section}


\usepackage{titlesec}
% \titlespacing{\section}{0pt}{5pt}{5pt}
% \titlespacing{\subsection}{0pt}{3pt}{3pt}
% \titlespacing{\subsubsection}{0pt}{3pt}{3pt}
\titlespacing{\paragraph}{0pt}{5pt}{5pt}
% \setlength{\textfloatsep}{20pt}
% \setlength{\floatsep}{5pt}


% The \icmltitle you define below is probably too long as a header.
% Therefore, a short form for the running title is supplied here:
\icmltitlerunning{Token Assorted: Mixing Latent and Text Tokens for Improved Language Model Reasoning}

\begin{document}

\twocolumn[
\icmltitle{Token Assorted: Mixing Latent and Text Tokens for\\ Improved Language Model Reasoning}



% It is OKAY to include author information, even for blind
% submissions: the style file will automatically remove it for you
% unless you've provided the [accepted] option to the icml2025
% package.

% List of affiliations: The first argument should be a (short)
% identifier you will use later to specify author affiliations
% Academic affiliations should list Department, University, City, Region, Country
% Industry affiliations should list Company, City, Region, Country

% You can specify symbols, otherwise they are numbered in order.
% Ideally, you should not use this facility. Affiliations will be numbered
% in order of appearance and this is the preferred way.
\icmlsetsymbol{equal}{*}

\begin{icmlauthorlist}
\icmlauthor{DiJia Su}{xxx}
\icmlauthor{Hanlin Zhu}{equal,yyy}
\icmlauthor{Yingchen Xu}{equal,xxx,zzz}
\icmlauthor{Jiantao Jiao}{yyy}
\icmlauthor{Yuandong Tian$^\dagger$}{xxx}
\icmlauthor{Qinqing Zheng$^\dagger$}{xxx}
\end{icmlauthorlist}

\icmlaffiliation{xxx}{Meta AI}
\icmlaffiliation{yyy}{UC Berkeley}
\icmlaffiliation{zzz}{UCL}

% You may provide any keywords that you
% find helpful for describing your paper; these are used to populate
% the "keywords" metadata in the PDF but will not be shown in the document
\icmlkeywords{Machine Learning, ICML}

\vskip 0.3in
]

% this must go after the closing bracket ] following \twocolumn[ ...

% This command actually creates the footnote in the first column
% listing the affiliations and the copyright notice.
% The command takes one argument, which is text to display at the start of the footnote.
% The \icmlEqualContribution command is standard text for equal contribution.
% Remove it (just {}) if you do not need this facility.

%\printAffiliationsAndNotice{}  % leave blank if no need to mention equal contribution
\printAffiliationsAndNotice{\icmlEqualContribution} % otherwise use the standard text.


During the early stages of interface design, designers need to produce multiple sketches to explore a design space.  Design tools often fail to support this critical stage, because they insist on specifying more details than necessary. Although recent advances in generative AI have raised hopes of solving this issue, in practice they fail because expressing loose ideas in a prompt is impractical. In this paper, we propose a diffusion-based approach to the low-effort generation of interface sketches. It breaks new ground by allowing flexible control of the generation process via three types of inputs: A) prompts, B) wireframes, and C) visual flows. The designer can provide any combination of these as input at any level of detail, and will get a diverse gallery of low-fidelity solutions in response. The unique benefit is that large design spaces can be explored rapidly with very little effort in input-specification. We present qualitative results for various combinations of input specifications. Additionally, we demonstrate that our model aligns more accurately with these specifications than other models. 

% OLD ABSTRACT
%When sketching Graphical User Interfaces (GUIs), designers need to explore several aspects of visual design simultaneously, such as how to guide the user’s attention to the right aspects of the design while making the intended functionality visible. Although current Large Language Models (LLMs) can generate GUIs, they do not offer the finer level of control necessary for this kind of exploration. To address this, we propose a diffusion-based model with multi-modal conditional generation. In practice, our model optionally takes semantic segmentation, prompt guidance, and flow direction to generate multiple GUIs that are aligned with the input design specifications. It produces multiple examples. We demonstrate that our approach outperforms baseline methods in producing desirable GUIs and meets the desired visual flow.

% Designing visually engaging Graphical User Interfaces (GUIs) is a challenge in HCI research. Effective GUI design must balance visual properties, like color and positioning, with user behaviors to ensure GUIs easy to comprehend and guide attention to critical elements. Modern GUIs, with their complex combinations of text, images, and interactive components, make it difficult to maintain a coherent visual flow during design.
% Although current Large Language Models (LLMs) can generate GUIs, they often lack the fine control necessary for ensuring a coherent visual flow. To address this, we propose a diffusion-based model that effectively handles multi-modal conditional generation. Our model takes semantic segmentation, optional prompt guidance, and ordered viewing elements to generate high-fidelity GUIs that are aligned with the input design specifications.
% We demonstrate that our approach outperforms baseline methods in producing desirable GUIs and meets the desired visual flow. Moreover, a user study involving XX designers indicates that our model enhances the efficiency of the GUI design ideation process and provides designers with greater control compared to existing methods.    



% %%%%%%%%%%%%%%%%%%%%%%%%%%%%%%%%%%%%%%%%%%%%%%%%%%%%%%
% % Writing Clinic Comments:
% %%%%%%%%%%%%%%%%%%%%%%%%%%%%%%%%%%%%%%%%%%%%%%%%%%%%%%
% % Define: Effective UI design
% % Motivate GANs and write in full form.
% % LLMs vs ControlNet vs GANs
% % Say something about the Figma plugin?
% % Write the work is novel or what has been done before
% % What is desirable UI and how to evalutate that?
% % Visual Flow - main theme (center around it)
% % Re-Title: use word Flow!
% % Use ControlNet++ & SPADE for abstract.
% % Write about input/output. 
% % Why better than previous work?
% %%%%%%%%%%%%%%%%%%%%%%%%%%%%%%%%%%%%%%%%%%%%%%%%%%%%%

% % v2:
% % \noindent \textcolor{red}{\textbf{NEW Abstract!} (Post Writing Clinic 1 - 25-Jun)}

% % \noindent \textcolor{red}{----------------------------------------------------------------------}

% % \noindent Designing user interfaces (UIs) is a time-consuming process, particularly for novice designers. 
% % Creating UI designs that are effective in market funneling or any other designer defined goal requires a good understanding of the visual flow to guide users' attention to UI elements in the desired order. 
% % While current Large Language Models (LLMs) can generate UIs from just prompts, they often lack finer pixel-precise control and fail to consider visual flow. 
% % In this work, we present a UI synthesis method that incorporates visual flow alongside prompts and semantic layouts. 
% % Our efficient approach uses a carefully designed Generative Adversarial Network (GAN) optimized for scenarios with limited data, making it more suitable than diffusion-based and large vision-language models.
% % We demonstrate that our method produces more "desirable" UIs according to the well-known contrast, repetition, alignment, and proximity principles of design. 
% % We further validate our method through comprehensive automatic non-reference, human-preference aligned network scoring and subjective human evaluations.
% % Finally, an evaluation with xx non-expert designers using our contributed Figma plugin shows that <method-name> improves the time-efficiency as well as the overall quality of the UI design development cycle.

% % \noindent \textcolor{red}{----------------------------------------------------------------------}


% \noindent \textcolor{blue}{\textbf{NEW Abstract!} (Pre Writing Clinic 9-July)}

% \noindent \textcolor{blue}{----------------------------------------------------------------------}

% \noindent Exploring different graphical user interface (GUI) design ideas is time-consuming, particularly for novice designers. 
% Given the segmentation masks, design requirement as prompt, and/or preferred visual flow, we aim to facilitate creative exploration for GUI design and generate different UI designs for inspiration.
% While current Vision Language Models (VLMs) can generate GUIs from just prompts, they often lack control over visual concepts and flow that are difficult to convey through language during the generation process. 
% In this work, we present FlowGenUI, a semantic map-guided GUI synthesis method that optionally incorporates visual flow information based on the user's choice alongside language prompts. 
% We demonstrate that our model not only creates more realistic GUIs but also creates "predictable" (how users pay attention to and order of looking at GUI elements) GUIs.
% Our approach uses Stable Diffusion (SD), a large paired image-text pretrained diffusion model with a rich latent space that we steer toward realistic GUIs using a trainable copy of SD's encoder for every condition (segmentation masks, prompts, and visual flow). 
% We further provide a semantic typography feature to create custom text-fonts and styles while also alleviating SD's inherent limitations in drawing coherent, meaningful and correct aspect-ratio text. 
% Finally, a subjective evaluation study of XX non-expert and expert designers demonstrates the efficiency and fidelity of our method.


% This process encourages creativity and prevents designers from falling into habitual patterns.


% ------------------------------------------------------------------
% Joongi Why is it important to create realistic GUI?
% I do not see how the Visual Flow given on the left hand side is reflected in the results on the right hand side. 
% I’d avoid making unsubstantiated claims about designers (falling into habitual patterns).
% The UIs you generate do not “align with users’ attention patterns” but rather try to control it (that’s what visual flow means)
% ------------------------------------------------------------------
% Comments - Writing Clinic - 9th July:
% Improve title. More names: FlowGen
% Figure 1: Use an inference time hand-drawn mask
% Figure 1: Show both workflows. Add a designer --> Input.
% Figure 1: Make them more diverse
% ------------------------------------------------------------------
% Designing graphical user interfaces (GUIs) requires human creativity and time. Designers often fall into habitual patterns, which can limit the exploration of new ideas. 
% To address this, we introduce FlowGenUI, a method that facilitates creative exploration and generates diverse GUI designs for inspiration. By using segmentation masks, design requirements as prompts, and/or selected visual flows, our approach enhances control over the visual concepts and flows during the generation process, which current Vision Language Models (VLMs) often lack.
% FlowGenUI uses Stable Diffusion (SD), a largely pretrained text-to-image diffusion model, and guides it to create realistic GUIs. 
% We achieve this by using a trainable copy of SD's encoder for each condition (segmentation masks, prompts, and visual flow). 
% This method enables the creation of more realistic and predictable GUIs that align with users' attention patterns and their preferred order of viewing elements.
% We also offer a semantic typography feature that creates custom text fonts and styles while addressing SD's limitations in generating coherent, meaningful, and correctly aspect-ratio text.
% Our approach's efficiency and fidelity are evaluated through a subjective user study involving XX designers. 
% The results demonstrate the effectiveness of FlowGenUI in generating high-quality GUI designs that meet user requirements and visual expectations.

% ---------------------------------------


%A critical and general issue remains while using such deep generative priors: creating coherent, meaningful and correct aspect-ratio text. 
%We tackle this issue within our framework and additionally provide a semantic typography feature to create custom text-fonts and styles. 


% %Creating UI designs that are effective in market funneling or any other designer-defined goal requires a good understanding of the visual flow to guide users' attention to UI elements in the desired order. 
% %While current largely pre-trained Vision Language Models (VLMs) can generate GUIs from just prompts, they often lack finer or pixel-precise control which can be crucial for many easy-to-understand visual concepts but difficult to convey through language. 
% % However, obtaining such pixe-level labels is an extremely expensive so we
% % For example - overlaying text on images with certain aspect ratios and two equally separated buttons 
% Additionally, all prior GUI generation work fails to consider visual flow information during the generation process. 
% We demonstrate that visual flow-informed generation not only creates more realistic and human-friendly GUIs but also creates "predictable" (how users pay attention to and order of looking at GUI elements) UIs that could be beneficial for designers for tasks like creating effective market funnels.
% In this work, we present a semantic map-guided GUI synthesis method that optionally incorporates visual flow information based on the user's choice alongside language prompts. 
% Our approach uses Stable Diffusion, a large (billions) paired image-text pretrained diffusion model with a rich latent space that we steer toward realistic GUIs using an ensemble of ControlNets. 
% % TODO: Mention it in 1 sentence:
% A critical and general issue remains while using such deep generative priors: creating coherent, meaningful and correct aspect-ratio text. 
% We tackle this issue within our framework and additionally provide a semantic typography feature to create custom text-fonts and styles. 
% To evaluate our method, we demonstrate that our method produces more "desirable" UIs according to the well-known contrast, repetition, alignment, and proximity principles of design. 
% % We further validate our method through comprehensive automatic non-reference and human-preference aligned scores. (TODO: Maybe Unskip if we get UIClip from Jason!)
% % TODO: Re-word this and only keep ideation cycles and time-efficiency.
% Finally, a subjective evaluation study of XX non-expert and expert designers demonstrates the efficiency and fidelity of our method.
% % improves the time-efficiency by quick iterations of the UI design ideation process.
% %Finally, an evaluation with xx non-expert designers using our contributed <method-name> improves the time-efficiency by quick iterations of the UI design ideation cycle.

%\noindent \textcolor{blue}{----------------------------------------------------------------------}


%In an evaluation with xx designers, we found that GenerativeLayout: 1) enhances designers' exploration by expanding the coverage of the design space, 2) reduces the time required for exploration, and 3) maintains a perceived level of control similar to that of manual exploration.



% Present-day graphical user interfaces (GUIs) exhibit diverse arrangements of text, graphics, and interactive elements such as buttons and menus, but representations of GUIs have not kept up. They do not encapsulate both semantic and visuo-spatial relationships among elements. %\color{red} 
% To seize machine learning's potential for GUIs more efficiently, \papername~ exploits graph neural networks to capture individual elements' properties and their semantic—visuo-spatial constraints in a layout. The learned representation demonstrated its effectiveness in multiple tasks, especially generating designs in a challenging GUI autocompletion task, which involved predicting the positions of remaining unplaced elements in a partially completed GUI. The new model's suggestions showed alignment and visual appeal superior to the baseline method and received higher subjective ratings for preference. 
% Furthermore, we demonstrate the practical benefits and efficiency advantages designers perceive when utilizing our model as an autocompletion plug-in.


% Overall pipeline: Maybe drop semantic typography / visual flow?
\section{Introduction}

Tutoring has long been recognized as one of the most effective methods for enhancing human learning outcomes and addressing educational disparities~\citep{hill2005effects}. 
By providing personalized guidance to students, intelligent tutoring systems (ITS) have proven to be nearly as effective as human tutors in fostering deep understanding and skill acquisition, with research showing comparable learning gains~\citep{vanlehn2011relative,rus2013recent}.
More recently, the advancement of large language models (LLMs) has offered unprecedented opportunities to replicate these benefits in tutoring agents~\citep{dan2023educhat,jin2024teach,chen2024empowering}, unlocking the enormous potential to solve knowledge-intensive tasks such as answering complex questions or clarifying concepts.


\begin{figure}[t!]
\centering
\includegraphics[width=1.0\linewidth]{Figs/Fig.intro.pdf}
\caption{An illustration of coding tutoring, where a tutor aims to proactively guide students toward completing a target coding task while adapting to students' varying levels of background knowledge. \vspace{-5pt}}
\label{fig:example}
\end{figure}

\begin{figure}[t!]
\centering
\includegraphics[width=1.0\linewidth]{Figs/Fig.scaling.pdf}
\caption{\textsc{Traver} with the trained verifier shows inference-time scaling for coding tutoring (detailed in \S\ref{sec:scaling_analysis}). \textbf{Left}: Performance vs. sampled candidate utterances per turn. \textbf{Right}: Performance vs. total tokens consumed per tutoring session. \vspace{-15pt}}
\label{fig:scale}
\end{figure}


Previous research has extensively explored tutoring in educational fields, including language learning~\cite{swartz2012intelligent,stasaski-etal-2020-cima}, math reasoning~\cite{demszky-hill-2023-ncte,macina-etal-2023-mathdial}, and scientific concept education~\cite{yuan-etal-2024-boosting,yang2024leveraging}. 
Most aim to enhance students' understanding of target knowledge by employing pedagogical strategies such as recommending exercises~\cite{deng2023towards} or selecting teaching examples~\cite{ross-andreas-2024-toward}. 
However, these approaches fall short in broader situations requiring both understanding and practical application of specific pieces of knowledge to solve real-world, goal-driven problems. 
Such scenarios demand tutors to proactively guide people toward completing targeted tasks (e.g., coding).
Furthermore, the tutoring outcomes are challenging to assess since targeted tasks can often be completed by open-ended solutions.



To bridge this gap, we introduce \textbf{coding tutoring}, a promising yet underexplored task for LLM agents.
As illustrated in Figure~\ref{fig:example}, the tutor is provided with a target coding task and task-specific knowledge (e.g., cross-file dependencies and reference solutions), while the student is given only the coding task. The tutor does not know the student's prior knowledge about the task.
Coding tutoring requires the tutor to proactively guide the student toward completing the target task through dialogue.
This is inherently a goal-oriented process where tutors guide students using task-specific knowledge to achieve predefined objectives. 
Effective tutoring requires personalization, as tutors must adapt their guidance and communication style to students with varying levels of prior knowledge. 


Developing effective tutoring agents is challenging because off-the-shelf LLMs lack grounding to task-specific knowledge and interaction context.
Specifically, tutoring requires \textit{epistemic grounding}~\citep{tsai2016concept}, where domain expertise and assessment can vary significantly, and \textit{communicative grounding}~\citep{chai2018language}, necessary for proactively adapting communications to students' current knowledge.
To address these challenges, we propose the \textbf{Tra}ce-and-\textbf{Ver}ify (\textbf{\model}) agent workflow for building effective LLM-powered coding tutors. 
Leveraging knowledge tracing (KT)~\citep{corbett1994knowledge,scarlatos2024exploring}, \model explicitly estimates a student's knowledge state at each turn, which drives the tutor agents to adapt their language to fill the gaps in task-specific knowledge during utterance generation. 
Drawing inspiration from value-guided search mechanisms~\citep{lightman2023let,wang2024math,zhang2024rest}, \model incorporates a turn-by-turn reward model as a verifier to rank candidate utterances. 
By sampling more candidate tutor utterances during inference (see Figure~\ref{fig:scale}), \model ensures the selection of optimal utterances that prioritize goal-driven guidance and advance the tutoring progression effectively. 
Furthermore, we present \textbf{Di}alogue for \textbf{C}oding \textbf{T}utoring (\textbf{\eval}), an automatic protocol designed to assess the performance of tutoring agents. 
\eval employs code generation tests and simulated students with varying levels of programming expertise for evaluation. While human evaluation remains the gold standard for assessing tutoring agents, its reliance on time-intensive and costly processes often hinders rapid iteration during development. 
By leveraging simulated students, \eval serves as an efficient and scalable proxy, enabling reproducible assessments and accelerated agent improvement prior to final human validation. 



Through extensive experiments, we show that agents developed by \model consistently demonstrate higher success rates in guiding students to complete target coding tasks compared to baseline methods. We present detailed ablation studies, human evaluations, and an inference time scaling analysis, highlighting the transferability and scalability of our tutoring agent workflow.

\section{Background and related work}
% 重点看Artistic data visualization: Beyond visual analytics 和Visualization criticism-the missing link between information visualization and art 的被引


This section reviews the background on artistic data visualization and previous research related to this topic.

\subsection{Artistic Data Visualization in Art History Context}
\label{ssec:contemporary}

Art history has been marked by transformative periods characterized by different aesthetic pursuits, materials, and methods. Since the time of Plato, imitation (or \textit{mimesis}, which views art as a mirror to the world around us) has been an important pursuit~\cite{pooke2021art}. Successful artworks, such as Greek sculptures and the convincing illusions of depth and space in Renaissance paintings, exemplify this tradition.
The advent of modern society and new technology, especially photography, posed a significant challenge to the notion of art as imitation~\cite{perry2004themes}. By the 1850s, modern art began to emerge with the core goal of transcending traditional forms and conventions. Movements like Post Impressionism, Expressionism, and Cubism revolutionized art through innovative uses of form (\eg color, texture, composition), moving art towards abstraction and experimentation. 
After World War II, the Cold War and the surge of various social problems heightened skepticism about the progress narrative of modernity and the superiority of the capitalist system, leading to the rise of postmodernism and the birth of contemporary art~\cite{hopkins2000after,harrison1992art}. One prominent feature of contemporary art is the absence of fixed standards or genres historically defined by the church or the academy. Postmodern design neither defines a cohesive set of aesthetic values nor restricts the range of media used~\cite{pooke2021art}, sparking novel practices such as installations, performances, lens-based media, videos, and land-based art~\cite{hopkins2000after}.
Meanwhile, artists have increasingly drawn energy from various philosophical and critical theories such as gender studies, psychoanalysis, Marxism, and post-structuralism~\cite{pooke2021art}. As a result, as described by Foster~\cite{foster1999recodings}, artists have increasingly become ``manipulators of signs and symbols... and the viewer an active reader of messages rather than a passive contemplator of the aesthetic''. Hopkins~\cite{hopkins2000after} described this shift as the ``death of the object'' and ``the move to conceptualism''. 
% Joseph Kosuth, one of the most important representatives of conceptual artists, also once said that “all art (after Duchamp) is conceptual (in nature) because art only exists conceptually”
% As argued by Danto~\cite{danto2015after}, traditional notions of aesthetics can no longer apply to contemporary art. ``“All there is at the end,” Danto wrote, “is theory, art having finally become vaporized in a dazzle of pure thought about itself, and remaining, as it were, solely as the object of its own theoretical consciousness.''
% The Anti-aesthetic (1983) has described these as ‘anti-aesthetic’ strategies – typified, as we have seen, by a conceptually driven approach to the art object and to the process of its production.

Emerging within the contemporary art historical context, data art has been significantly propelled by the advent of big data. An early milestone was Kynaston McShine's 1970 exhibition \textit{Information} at the Museum of Modern Art (MoMA). 
% In the exhibition catalogue, McShine wrote~\cite{information_moma}: ``Increasingly artists use mail, telegrams, telex machines, etc., for transmission of works themselves—photographs, films, documents—or of information about their activity.'' 
% The millennium era has witnessed substantial growth in this area.
In 2008, Google’s Data Arts Team was founded to explore the advancement of what creativity and technology can do together~\cite{google}.
% data artist Aaron Koblin
In 2012, Viégas and Wattenberg's \textit{Wind Map}, an artwork that visualizes real-time air movement, became the first web-based artwork to be included in MoMA's permanent collection~\cite{wind}.
Since 2013, the academic conference IEEE VIS has included an Arts Program (IEEE VISAP), showcasing artistic data visualizations through accepted papers and curated exhibitions. 
As noted by Barabási~\cite{dataism} (a Fellow of the American Physical Society and the head of a data art lab), data has become a vital medium for artists to deal with the complexities of our society: ``Humanity is facing a complexity explosion. We are confronted with too much data for any of us to make sense of...The traditional tools and mediums of art, be they canvas or chisel, are woefully inadequate for this task...today’s and tomorrow’s artists can embrace new tools and mediums that scale to the challenge, ensuring that their practice can continue to reflect our changing epistemology.''
% a physicist and head of a data art lab, has noted, 

% Artists are now exploring new mediums and methods, incorporating datasets, computer technology, and algorithms into their work.



\subsection{Research on Artistic Data Visualization}
\label{ssec:artisticvis}

Artistic data visualization is also referred to as artistic visualization, data art, or information art~\cite{holmquist2003informative,rodgers2011exploring,few,viegas2007artistic}. Despite the varying terminologies, there is a basic consensus that artistic data visualization must be art practice grounded in real data~\cite{viegas2007artistic}.
Since the early 2000s, a series of papers introduced innovative artistic systems for visualizing everyday data, such as museum visit routes and bus schedule information~\cite{skog2003between,holmquist2003informative,viegas2004artifacts}.
In 2007, Viégas and Wattenberg~\cite{viegas2007artistic} explicitly proposed the concept of \textit{artistic data visualization} and viewed it as a promising domain beyond visual analytics.
% and defined it as ``visualization of data done by artists with the intent of making art''. 
Kosara~\cite{kosara2007visualization} drew a spectrum of visualization design, positioning artistic visualization and pragmatic visualization at opposite ends of this spectrum to demonstrate that the goals of these two types of design often diverge. 
% advocating that analytical visualizations prioritize practicality, while artistic data visualizations emphasize sublime quality, that is, the capacity to inspire awe and grandeur and elicit profound emotional or intellectual responses. 
% In 2011, Rodgers and Bartram~\cite{rodgers2011exploring} utilized artistic data visualization to enhance residential energy use feedback. 
However, overall, research on this subject has been sparse. Among those relevant papers, most have focused on specific applications of artistic data visualization. 
%~\cite{rodgers2011exploring,schroeder2015visualization,perovich2020chemicals}
For instance, Rodgers and Bartram~\cite{rodgers2011exploring} utilized ambient artistic data visualization to enhance residential energy use feedback. Samsel~\etal~\cite{samsel2018art} transferred artistic styles from paintings into scientific visualization.
Artistic practice has also been observed in fields such as data physicalization~\cite{hornecker2023design,perovich2020chemicals,offenhuber2019data} and sonification~\cite{enge2024open}. For example, Hornecker~\etal~\cite{hornecker2023design} found that many artists are practicing transforming data into tangible artifacts or installations. Enge~\etal~\cite{enge2024open} discussed a set of representative artistic cases that combine sonification and visualization.
% dragicevic2020data
% Offenhuber~\cite{offenhuber2019data} created a set of artworks in urban settings that transform air quality data into situated displays, allowing people to encounter visualizations in their daily lives.

% But in contrast, empirical studies that describe the characteristics of artistic visualization and how they are created are extremely scarce. This scarcity forms a stark contrast to the increasingly rich and diverse practices by artists in the field.
% As for the difference between artistic data visualization and traditional visualizations for analytics, Vi{\'e}gas and Wattenberg~\cite{viegas2007artistic} thought that the most salient feature of artistic data visualizations is their forceful expression of viewpoints.
% In Ramirez~\cite{ramirez2008information}'s opinion, functional information visualizations are concerned with usability and performance while aesthetic information visualizations are concerned with visually attractive forms of representation design.
% Donath~\etal~\cite{donath2010data} presented a series of tools developed to integrate artistic expressions in generating unique and customized visualizations based on users' personal data, such as health monitoring data, album records, and e-mail records. 

On the other hand, some studies, while not focusing on artistic data visualization, have explored a key art-related concept: aesthetics. 
% ~\cite{moere2012evaluating,cawthon2007effect,lau2007towards} explored the aesthetics of visualization design in their research. They
For example, Moere~\etal~\cite{moere2012evaluating} compared analytical, magazine, and artistic visualization styles, noting that analytical styles enhance the discovery of analytical insights, while the other two induce more meaning-based insights. Cawthon~\etal~\cite{cawthon2007effect} asked participants to rank eleven visualization types on an aesthetic scale from ``ugly'' to ``beautiful'', finding that some visualizations (\eg sunburst) were perceived as more beautiful than others (\eg beam trees).
% Moere~\etal~\cite{moere2012evaluating} displayed data in three different styles (analytical style, magazine style, artistic style) and found that these styles led to different perceptions of usability and types of insights.
% More importantly, the authors found that the sunburst chart ranks the highest in aesthetics and is one of the top-performing visualizations in both efficiency and effectiveness, thus exemplifying the notion that "beautiful is indeed usable".
Factors such as embellishment~\cite{bateman2010useful}, colorfulness~\cite{harrison2015infographic}, and interaction~\cite{stoll2024investigating} have also been found to influence aesthetics. 
% borkin2013makes,tanahashi2012design
However, most of these studies have simplified aesthetics to hedonic features (\eg beauty), without delving into the nuanced connotations of aesthetics.
% most of these studies have simplified aesthetics to concepts like 'beauty,' 'preference,' or 'pleasing,' without exploring the deeper essence of aesthetics as the core of art.

The value of artistic data visualization to the visualization community is still in controversy. For instance, Few~\cite{few} argued for a clearer distinction between data art and data visualization; he highlighted the negative consequences when data art ``masquerades as data visualization'', such as making visualizations difficult to perceive. Willers~\cite{willers2014show} thought the artistic approach is ``unlikely be appreciated if the intention was for rapid decision making.''
% In an interview, American artist and technologist Harris commented that ``data can be made pretty by design, but this is a superficial prettiness, like a boring woman wearing too much makeup.''~\cite{harris2015beauty} 
To address these gaps, more empirical research needs to be conducted to explore how artistic data visualizations are designed, their underlying pursuits, and how they might inspire our community.




% Examining this field not only helps us understand the latest application of data visualization in various domains but also extends our understanding of the aesthetic and humanistic aspects of data visualization.
% there should be more theoretical investigation into artistic data visualization. 

% These three concepts emphasize placing or installing visualizations at physical places that people will encounter in their daily lives. 

% ~\cite{wang2019emotional}


% gap between art and science~\cite{judelman2004aesthetics}
% constructive visualization~\cite{huron2014constructive}
% data feminism~\cite{d2020data}
% critical infovis~\cite{dork2013critical}
% citizen data and participation~\cite{valkanova2015public}

% \x{Lee~\etal~\cite{lee2013sketchstory}, give users artistic freedom to create their own visualizations.}


% Aesthetics refers to the study of beauty, taste, and sensory perception and is deeply intertwined with art.
% a particular taste for or approach to what is pleasing to the senses and especially sight

% why shouldn't all charts be scatter plot~\cite{bertini2020shouldn}
% aesthetic model~\cite{lau2007towards}
% Aesthetics for Communicative Visualization : a Brief Review
% Stacked graphs--geometry \& aesthetics~\cite{byron2008stacked}
% storyline optimization~\cite{tanahashi2012design}
% graphic designers rate the attractiveness of non-standard and pictorial visualizations higher than standard and abstract ones, whereas the opposite is true for laypeople.~\cite{quispel2014would}
% evaluate aesthetics - wordcloud
% An Evaluation of Semantically Grouped Word Cloud Designs, tag cloud, wordle

% On the other hand, empirical studies conducted with designers have shown that functionality is not the only design goal of visualization. For example, Bigelow~\etal~\cite{bigelow2014reflections} found that designers would frequently fine-tune the non-data elements in their designs, and data encoding was even "a later consideration with respect to other visual elements of the infographic".
% Moere~\cite{moere2011role} - design
% Quispel~\etal~\cite{quispel2018aesthetics} found that for information designers, clarity and aesthetics are both important for making a design attractive.
% \input{03-preliminary}
\section{Methodology}
\label{sec:algo}
\begin{figure*}[t]
    \centering
    \includegraphics[width=1.35\columnwidth]{plots/replacement.pdf}
    \caption{An example illustrating our replacement strategy. With chunk size $L=16$ and compression rate $r=16$, we encode 32 textual CoT tokens into 2 discrete latent tokens from left to right. The other CoT tokens will remain in their original forms. 
    }
    \label{fig:replacement}
\end{figure*}
In this section, we describe our methodology to enable LLMs to reason with discrete latent tokens. The notations are summarized in \Cref{app:notations}.
Let $X = P \oplus C \oplus S$ denote a sample input,
where $P = (p_1, p_2, \ldots, p_{t_p})$ are the prompt tokens, $C = (c_1, c_2, \ldots, c_{t_c})$ are the reasoning step (chain-of-thought) tokens,
$S = (s_1, s_2, \ldots, s_{t_s})$ are the solution tokens, and $\oplus$ denotes concatenation. Our training procedure consists of two stages:
\begin{enumerate}[leftmargin=*]\itemsep0em
    \item \textbf{Learning latent discrete tokens to abstract the reasoning steps}, where we train a model to convert $C$ into a sequence of latent tokens $Z = (z_1, z_2, \ldots, z_{t_z})$ such that $t_z < t_c$. The compression rate $r = t_c / t_z$ controls the level of abstraction.

    \item \textbf{Training the LLM with a partial and high-level abstract of the reasoning steps}, where we 
    construct a modified input $\Xtilde$ by
    replacing the first $m$ tokens of $C$ by the corresponding latent abstractions:
    \begin{equation}
        \Xtilde = P \oplus [z_1, \ldots, z_{\frac{m}{r}}, c_{m+1}, \ldots, c_{t_c}] \oplus S.
        \label{eq:X_replacement}
    \end{equation}
    \Cref{fig:replacement} illustrates this replacement strategy. We randomize the value of $m$ during training.
\end{enumerate}



\subsection{Learning Latent Abstractions}
We employ a vector-quantized variable autoencoder (VQ-VAE)~\cite{van2017neural} type of architecture to map CoT tokens \(C\) into discrete latent tokens \(Z\). % VQ-VAE is a powerful model capable of capturing high-dimensional semantic structures from text data.
To enhance abstraction performance, our VQ-VAE is trained on the whole input sequence $X$, but only applied to $C$ in the next stage. Following~\citet{jiang2022efficient, jiang2023h}, we split $X$ into chunks of length \(L\) and encode each chunk into $\frac{L}{r}$ latent codes, where $r$ is a preset compression rate. More precisely, our architecture consists of the following five components:
\vspace{-5pt}
\begin{itemize}\itemsep0pt
    \item $\E:$ a codebook containing $|\E|$ vectors in $\R^d$.
    \item $\fenc: \V^L \mapsto \R^{d \times \frac{L}{r}} $ that encodes a sequence of $L$ text tokens to $\frac{L}{r}$ latent embedding vectors $\bar{X} = \bar{x}_1, \ldots, \bar{x}_{\frac{L}{r}}$,  where $\V$ is the vocabulary of text tokens.
   %  and $\Z$ are the vocabularies of text and latent tokens, respectively. 
   \item $q: \R^{d} \mapsto \E$: the quantization operator that replaces the encoded embedding $\bar{x}$ by the nearest neighbor in $\E$: $q(\bar{x}) = \argmin_{e_i \in \E} \norm{e_i - \bar{x}}^2_2$.
    \item $g: \V^K \mapsto \R^d$ that maps $K$ text tokens to a $d$-dimensional embedding vector. We use $g$ to generate a continuous embedding of the prompt $P$.
    \item $\fdec: \R^{d \times \frac{L}{r}} \times \R^k \mapsto \V^L$ that decodes latent embeddings back to text tokens, conditioned on prompt embedding.
\end{itemize}
In particular, each continuous vector $e \in \E$ in the codebook has an associated latent token $z$, which we use to construct the latent reasoning steps $Z$\footnote{To decode a latent token $z$, we look up the corresponding embedding $e \in \E$ and feed it to $\fdec$.}.


\begin{figure}[t]
    \centering
    \includegraphics[width=\columnwidth]{plots/vqvae_latentformer.pdf}
    \caption{A graphical illustration of our VQ-VAE. $\fenc$ encodes the text tokens into latent embeddings, which are quantized by checking the nearest neighbors in the codebook. $\fdec$ decodes those quantized embeddings back to text tokens. When applying the VQ-VAE to compress the text tokens, the discrete latent tokens $Z$ are essentially the index of corresponding embeddings in the codebook.} 
    \label{fig:vqvae}
\end{figure}


For simplicity, we assume the lengths of the input $X$ and the prompt $P$ are $L$ and $K$ exactly.
Similar to \citet{van2017neural}, we use an objective $\mathcal{L}$ composed of 3 terms: 
% . Given an input sequence \(X_t = (x_1, x_2, \ldots, x_T)\) and the discrete latent tokens \(Z_t = (z_1, z_2, \ldots, z_{\tfrac{M T}{L}})\), the total loss  is:
\begin{equation}
\begin{aligned}
& \mathcal{L}(X) = \underbrace{ \log p(X | \fdec( q(\bar{X}) | g(P) ))}_{\text{reconstruction loss}} + \\
&  \hskip5pt \sum_{i=1}^L \underbrace{ \| \sg[\bar{X}_i] - q(\bar{X}_i) \|_2^2}_{\text{VQ loss}} +  \underbrace{\beta \| \bar{X}_i - \sg[q(\bar{X}_i)] \|_2^2}_{\text{commitment loss}},
\end{aligned}
\end{equation}
where $\bar{X} = \fenc(X)$, $\sg[\cdot]$ is the stop-gradient operator, and \(\beta\) is a hyperparameter controlling the strength of the commitment loss.
The VQ loss and the commitment loss ensure that the encoder outputs remain close to the codebook, while the reconstruction loss concerns with the decoding efficacy. As standard for VQ-VAE, we pass the gradient $\nabla_{\fdec}(L)$ unaltered to $\fenc$ directly as the quantization operator $q(\cdot)$ is non-differentiable. \Cref{fig:vqvae} illustrates our architecture. In practice, we use a causal Transformer for both $\fenc$ and $\fdec$, the model details are discussed in Appendix~\ref{app:model}.


Thus far we obtain a latent representation both semantically meaningful and conducive to reconstruction, setting the stage for the subsequent training phase where
the LLM is trained to perform reasoning with abstractions.


\subsection{Reasoning with Discrete Latent Tokens}


In this second stage, we apply the obtained VQ-VAE to form modifed samples $\Xtilde$ with latent abstractions as in \Cref{eq:X_replacement}, then train an LLM to perform next token prediction. 
Below, we outline the major design choices that are key to our model's performance, and ablate them in \Cref{sec:expr}.

\textbf{Partial Replacement}. Unlike previous planning works~\cite{jiang2022efficient, jiang2023h} that project the whole input sequence onto a compact latent space, we only replace $m < t_c$ CoT tokens with their latent abstractions, leaving the remaining tokens unchanged.  We delimit the latent tokens by injecting a special \texttt{<boLatent>} and \texttt{<eoLatent>} tokens to encapsulate them.
% The constructed $\Xtilde$ becomes a fixed mixture of early latent tokens and later text tokens.

\textbf{Left-to-Right (AR) Replacement}. We replace the leftmost $m$ tokens of $C$, rather than subsampling tokens at different locations. 

\textbf{Mixing Samples with Varying Values of $m$}. For fine-tuning an existing LLM on the reasoning dataset with latent tokens, one remarkable challenge is to deal with the extended vocabulary. As the LLM is pretrained with trillions of tokens,
it is very hard for it to quickly adapt to tokens (and corresponding embeddings) beyond the original vocabulary. Previous works that aim to replace or eliminate CoT tokens~\cite{deng2024explicit, hao2024training} employ a multistage curriculum training approach, where those operations are gradually applied to the entire input sequence. In the context of our approach, this means we increase the values of $m$ in each stage until it reaches a pre-set cap value. However, such training procedure is complex and computationally inefficient, where dedicated optimization tuning is needed. In this work, we employ a simple single stage training approach where the value of $m$ is randomly set for each sample. Surprisingly, this not only makes our training more efficient, but also leads to enhanced performance. 
\section{Experiments}
\label{sec:expr}
We empirically evaluate our approach on two categories of benchmarks: 
\vspace{-5pt}
\begin{enumerate}\itemsep0pt
    \item[\textbf{(1)}] Synthetic datasets including the Keys-Finding Maze, ProntoQA~\cite{saparov2022language}, and ProsQA~\cite{hao2024training}, where we pretrain T5 or GPT-2 models from scratch using the method in \Cref{sec:algo};
    \item[\textbf{(2)}] Real-world mathematic reasoning problems, where we fine-tune Llama models~\cite{dubey2024llama} on the MetaMathQA~\cite{yu2023metamath} or the Dart-MATH~\cite{tong2024dart} dataset, and then test on in-domain datasets Math and GSM-8K, along with out-of-domain datasets including Fresh-Gaokao-Math-2023, DeepMind-Math, College-Math, OlympiaBench-Math, and TheoremQA. 
\end{enumerate}
The detailed setup is introduced in \Cref{sec:expr_benchmark}.

We compare our approach to the following baselines:
\vspace*{-10pt}
\begin{enumerate}[leftmargin=0pt]\itemsep0pt
    \item[] \textbf{Sol-Only}:  the model is trained with samples that only contains questions and solutions, without any reasoning steps;
    \item[] \textbf{CoT}: the model is trained with samples with complete CoT tokens; \looseness=-1
    \item[] \textbf{iCoT}~\citep{deng2024explicit}: a method that utilizes curriculum learning to gradually eliminate the need of CoT tokens in reasoning;
    \item[] \textbf{Pause Token}~\citep{goyal2023think}:  a method that injects a learnable \texttt{pause} token into the sample during training, in order to offer extra computation before giving out the final answer.

\end{enumerate}



\subsection{Benchmarks}
\label{sec:expr_benchmark}
\subsubsection{Synthetic Benchmarks}

\textbf{Keys-Finding Maze} is a complex navigation environment designed to evaluate an agent's planning capabilities. The agent is randomly positioned within a maze comprising 4 $3 \times 3$ interconnected rooms, with the objective of reaching a randomly placed goal destination. To successfully reach the destination, the agent must collect keys (designated with green, red, and blue colors) that correspond to matching colored doors. These keys are randomly distributed among the rooms, requiring the agent to develop sophisticated planning strategies for key acquisition and door traversal. The agent is only allowed to take one key at a time. This environment poses a substantial cognitive challenge, as the agent must identify which keys are necessary for reaching the destination, and optimize the order of key collection and door unlocking to establish the most efficient path to the goal. Following \citet{lehnert2024beyond,su2024dualformer}, we generate intermediate search traces using the nondeterministic A* algorithm~\cite{hart1968formal}. The dataset contains 100k training samples. See \Cref{app:maze} for more information and graphical illustrations.

\textbf{ProntoQA}~\cite{saparov2022language} is a dataset consists of $9000$ logical reasoning problems derived from ontologies - formal representations of relationships between concepts. Each problem in the dataset is constructed to have exactly one correct proof or reasoning path. One distinctive feature of this dataset is its consistent grammatical and logical structure, which enables researchers to systematically analyze and evaluate how LLMs approach reasoning tasks. 

\textbf{ProsQA}~\cite{hao2024training} is a more difficult benchmark building on top of ProntoQA. It contains 17,886 logical problems curated by randomly generated directed acyclic graphs. %The advantage of this dataset is that it
It has larger size of distracting reasoning paths in the ontology, and thus require more complex reasoning and planning capabilities.
% to solve it.

\subsubsection{Mathematical Reasoning}
We fine-tune pretrained LLMs using the MetaMathQA~\cite{yu2023metamath} or the Dart-MATH~\cite{tong2024dart} dataset. 
MetaMathQA is a curated dataset that augments the existing \texttt{Math} ~\cite{math_dd} and \texttt{GSM8K} ~\cite{gsm8k_dd} datasets by various ways of question bootstrapping,
such as (i) rephrasing the question and generating the reasoning path; (ii) generating backward questions,  self-verification questions, FOBAR questions~\cite{jiang2024forward}, etc. This dataset contains 395k samples in total, where 155k samples are bootstrapped from \texttt{Math} and the remaining 240k come from \texttt{GSM8K}. We rerun the MetaMath data pipeline by using Llama-3.1-405B-Inst to generate the response. 
Dart-MATH~\cite{tong2024dart} also synthesizes responses for questions in \texttt{Math} and \texttt{GSM8K}, with the focus on difficult questions via difficulty-aware rejection tuning.
For evaluation, we test the models on the original \texttt{Math} and \texttt{GSM8K} datasets, which are in-domain,
and also the following out-of-domain benchmarks:
\vspace{-5pt}
\begin{itemize}[leftmargin=*]\itemsep0pt
    \item  \textbf{College-Math}~\cite{tang2024mathscale}
consists of 2818 college-level math problems taken from 9 textbooks. These problems cover over 7 different areas such as linear algebra, differential equations, and so on. They are designed to evaluate how well the language model can handle complicated mathematical reasoning problems in different field of study.

    \item  \textbf{DeepMind-Math}~\cite{saxton2019analysing} consists of 1000 problems based on the national school math curriculum for students up to 16 years old. It examines the basic mathematics and reasoning skills across different topics.

    \item  \textbf{OlympiaBench-Math}~\cite{he2024olympiadbench} 
is a text-only English subset of Olympiad-Bench focusing on advanced level mathematical reasoning. It
contains 675 highly difficult math problems from competitions. 

    \item  \textbf{TheoremQA}~\cite{chen2023theoremqa} contains 800 problems focuses on solving problems in STEM fields (such as math, physics, and engineering) using mathematical theorems.


    \item \textbf{Fresh-Gaokao-Math-2023} ~\cite{tang2024mathscale} contains 30 math questions coming from  Gaokao, or the National College Entrance Examination, which is a national standardized test that plays a crucial role in the college admissions process.
\end{itemize}

\subsection{Main Results}
\label{sec:expr_main}
We employ a consistent strategy for training VQ-VAE and replacing CoT tokens with latent discrete codes across all our experiments, as outlined below.
The specific model architecture and key hyperparameters used for LLM training are presented alongside the results for each category of benchmarks.
All the other details are deferred to \Cref{app:model}. \looseness=-1

\paragraph{VQ-VAE Training} For each benchmark, we train a VQ-VAE for 100k steps using the Adam optimizer, with learning rate $10^{-5}$ and batch size 32.
We use a codebook of size $1024$ and compress every chunk of $L=16$ tokens into a single latent token (i.e., the compression rate $r=16$).


\paragraph{Randomized Latent Code Replacement} We introduce a stochastic procedure for partially replacing CoT tokens with latent codes. 
Specifically, we define a set of predetermined numbers \( \mathcal{M} = \{0, 72, 128, 160, 192, 224, 256\}\), which are all multipliers of $L=16$.
For each training example, we first sample $m_{\max} \in \mathcal{M}$ then sample an integer $m \in [0, 16, 32, \ldots, m_{\max}]$ uniformly at random.
The first $m$ CoT tokens are replaced by their corresponding latent discrete codes, while the later ones remain as raw text. 
This stochastic replacement mechanism exposes the model to a wide range of latent-text mixtures, enabling it to effectively learn from varying degrees of latent abstraction.


\begin{table*}[t]
\centering
\resizebox{0.7\textwidth}{!}{
\begin{tabular}{lcccccc}
\toprule
\multirow{2}{*}{\bf{Model}} & \multicolumn{2}{c}{\bf Keys-Finding Maze} & \multicolumn{2}{c}{\bf ProntoQA} & \multicolumn{2}{c}{\bf ProsQA} \\ 
\cmidrule(lr){2-3} \cmidrule(lr){4-5} \cmidrule(lr){6-7}
 & 1-Feasible-10 (\%) & Num. Tokens &  Accuracy & Num. Tokens & Accuracy & Num. Tokens \\ 
\midrule
Sol-Only & 3 & 645 & 93.8 & 3.0 & 76.7 & 8.2 \\
CoT & \underline{43}& 1312.0 & \underline{98.8} & 92.5 & \underline{77.5} & 49.4 \\

\bf{Latent (ours)}  & \bf{62.8 \increase{19.8}} & 374.6 & \bf{100 \increase{1.2}} & 7.7 & \textbf{96.2 \increase{18.7}} & 10.9 \\
\bottomrule
\end{tabular}
}
\caption{Our latent approach surpasses the other baselines on Keys-Finding Maze, ProntoQA and ProsQA with a large margin
. We use top-$k$ ($k=10$) decoding for Keys-Finding Maze and greedy decoding for ProntoQA and ProsQA. In terms of token efficiency, 
our latent approach also generates much shorter reasoning traces than the CoT baseline, closely tracking or even outperforming the Sol-Only approach.
\textbf{Bold: best results}. \underline{Underline: second best results}. \increase{Performance gain compared with the second best result.}
}
\label{table:synthetic}
\end{table*} 


\begin{table*}[t]
\begin{adjustbox}{width=\textwidth}
\begin{tabular}{lllllllllll}
\toprule
\multicolumn{2}{c}{\multirow{2}{*}{\bf Model}} & \multicolumn{2}{c}{\bf In-Domain} & \multicolumn{5}{c}{\bf Out-of-Domain} & \multicolumn{1}{c}{\bf Average} \\ \cmidrule(lr){3-4} \cmidrule(lr){5-9} \cmidrule(lr){10-10}
& & Math & GSM8K & Gaokao-Math-2023 & DM-Math & College-Math & Olympia-Math & TheoremQA & All Datasets \\ \midrule
\multirow{6}{*}{\bf Llama-3.2-1B}
& Sol-Only  & 4.7 & 6.8 & 0.0 & 10.4 & 5.3 & 1.3 & 3.9 & 4.6 \\
& CoT  & \underline{10.5} & \underline{42.7} & \bf{10.0} & 3.4 & \underline{17.1} & 1.5 & 9.8 & \underline{14.1} \\
& iCoT  & 8.2 & 10.5 & 3.3 & \underline{11.3} & 7.6 & \textbf{2.1} & \underline{10.7} & 7.7 \\
& Pause Token & 5.1 & 5.3 & 2.0  & 1.4 &  0.5  & 0.0 &  0.6 & 2.1\\

& \textbf{Latent (ours)} & \textbf{14.7 \increase{4.2}} & \textbf{48.7 \increase{6}} & \textbf{10.0}  & \textbf{14.6 \increase{3.3}}  & \textbf{20.5 \increase{3.4}} & \underline{1.8}  & \textbf{11.3 \increase{0.6}}  & \textbf{17.8 \increase{3.7}}  \\
\midrule
\multirow{6}{*}{\bf Llama-3.2-3B}
& Sol-Only  & 6.1 & 8.1 & 3.3 & 14.0 & 7.0 & 1.8 & 6.8 & 6.7\\
& CoT  & \underline{21.9} & \underline{69.7} & \underline{16.7} & \textbf{27.3} & \underline{30.9} & 2.2 & 11.6 & \underline{25.2} \\
& iCoT  & 12.6 & 17.3 & 3.3 & 16.0 & 14.2 & \textbf{4.9} & \textbf{13.9} & 11.7 \\
& Pause Token & 25.2 & 53.7 & 4.1 & 7.4 &  11.8 & 0.7 &  1.0 & 14.8\\

& \textbf{Latent (ours)} & \textbf{26.1 \increase{4.2}}  & \textbf{73.8 \increase{4.1}}  & \textbf{23.3 \increase{6.6}}  & \underline{27.1}  & \textbf{32.9 \increase{2}}  & \underline{4.2}  & \underline{13.5}   & \textbf{28.1 \increase{2.9}} \\
\midrule
\multirow{6}{*}{\bf Llama-3.1-8B}
& Sol-Only  & 11.5 & 11.8 & 3.3 & 17.4 & 13.0 & 3.8 & 6.7 & 9.6 \\
& CoT  & {32.9} & \underline{80.1} & \underline{16.7} & \underline{39.3} & \underline{41.9} & 7.3 & \underline{15.8 } & \underline{33.4} \\
& iCoT  & 17.8 & 29.6 & 16.7 & 20.3 & 21.3 & \underline{7.6} & 14.8 & 18.3 \\
& Pause Token & \textbf{39.6} & 79.5 & 6.1  & 25.4 &   25.1 & 1.3 &  4.0 & 25.9\\


& \textbf{Latent (ours)} & \underline{37.2}  & \textbf{84.1 \increase{4.0}}  & \textbf{30.0 \increase{13.3}}  & \textbf{41.3 \increase{2}}  & \textbf{44.0 \increase{2.1}}  & \textbf{10.2 \increase{2.6}}  & \textbf{18.4 \increase{2.6}}  & \textbf{37.9 \increase{4.5}}  \\
 

\bottomrule
\end{tabular}
\end{adjustbox}
\caption{
Our latent approach outperforms the baselines on various types of mathematical reasoning benchmarks. The models are fine-tuned on the MetaMathQA~\cite{yu2023metamath} dataset. The Math and GSM8K are in-domain datasets since they are used to generate MetaMathQA, while the others are out-of-domain. \textbf{Bold: best results}. \underline{Underscore: second best results}. \textcolor{darkgreen}{$\uparrow$ +: \hspace{0.2em}Performance gain compared with the second best result.}
}
\label{table:LLMtable}
\end{table*}


\begin{table*}[t]
\begin{adjustbox}{width=\textwidth}
\begin{tabular}{llccccccccc}
\toprule
\multicolumn{2}{c}{\multirow{2}{*}{\bf Model}} & \multicolumn{2}{c}{\bf In-Domain (\# of tokens)} & \multicolumn{5}{c}{\bf Out-of-Domain (\# of tokens)} & \multicolumn{1}{c}{\bf Average} \\ \cmidrule(lr){3-4} \cmidrule(lr){5-9} \cmidrule(lr){10-10}
& & Math & GSM8K & Gaokao-Math-2023 & DM-Math & College-Math & Olympia-Math & TheoremQA & All Datasets \\ \midrule
\multirow{6}{*}{\bf Llama-3.2-1B}
& Sol-Only  & 4.7 & 6.8 & 0.0 & 10.4 & 5.3 & 1.3 & 3.9 & 4.6 \\
& CoT & 646.1 & 190.3 & 842.3 & 578.7 & 505.6 & 1087.0 & 736.5 & 655.2 \\
& iCoT & 328.4 & 39.8 & 354.0 & 170.8 & 278.7 & 839.4 & 575.4 & 369.5 \\
& Pause Token & 638.8 & 176.4 & 416.1 & 579.9 & 193.8 & 471.9 & 988.1 & 495\\
% & Dualformer &  \\
& \textbf{Latent (ours)} & 501.6 \decrease{22\%} & 181.3 \decrease{5\%} & 760.5 \decrease{11\%} & 380.1 \decrease{34\%} & 387.3 \decrease{23\%} & 840.0 \decrease{22\%} & 575.5 \decrease{22\%} & 518 \decrease{21\%} \\
\midrule
\multirow{6}{*}{\bf Llama-3.2-3B}
& Sol-Only  & 6.1 & 8.1 & 3.3 & 14.0 & 7.0 & 1.8 & 6.8 & 6.7\\
& CoT & 649.9  & 212.1  & 823.3 & 392.8 & 495.9 & 1166.7 & 759.6 & 642.9 \\
& iCoT & 344.4 & 60.7 & 564.0 & 154.3 & 224.9 & 697.6 & 363.6 & 344.2 \\
& Pause Token & 307.9 & 162.3 & 108.9 & 251.5 & 500.96 & 959.5 & 212.8 & 354.7 \\
% & Dualformer &  \\
& \textbf{Latent (ours)} & 516.7 \decrease{20\%} & 198.8 \decrease{6\%} & 618.5 \decrease{25\%} & 340.0 \decrease{13\%} & 418.0 \decrease{16\%} & 832.8 \decrease{29\%} & 670.2 \decrease{12\%} & 513.6 \decrease{20\%}\\
\midrule
\multirow{6}{*}{\bf Llama-3.1-8B}
& Sol-Only  & 11.5 & 11.8 & 3.3 & 17.4 & 13.0 & 3.8 & 6.7 & 9.6 \\
& CoT & 624.3 & 209.5 & 555.9 & 321.8 & 474.3 & 1103.3 & 760.1 & 578.5 \\
& iCoT & 403.5 & 67.3 & 444.8 & 137.0 & 257.1 & 797.1 & 430.9 & 362.5 \\
& Pause Token & 469.4 & 119.0 & 752.6 & 413.4 & 357.3 & 648.2 &600.1 &  480\\

& \textbf{Latent (ours)} & 571.9 \decrease{9 \%} & 193.9 \decrease{8 \%} & 545.8 \decrease{2 \%} & 292.1 \decrease{10\%} & 440.3 \decrease{8\%} & 913.7 \decrease{17 \%} & 637.2 \decrease{16 \%} & 513.7 \decrease{10\%}\\


\bottomrule
\end{tabular}
\end{adjustbox}
\caption{The average number of tokens in the generated responses. Compared with the CoT baseline, our latent approach achieves an $17\%$ reduction in response length on average, while surpassing it in final performance according to~\Cref{table:LLMtable}. The iCoT method generates shorter responses than our approach, yet performs significantly worse, see~\Cref{table:LLMtable}. \textcolor{darkgreen}{$\downarrow$ -:\hspace{0.2em}Trace length reduction rate compared with CoT.} }
\label{table:LLM-token}
\end{table*}


\subsubsection{Synthetic Benchmarks}

\paragraph{Hyperparameters and Evaluation Metric}  

For our experiments on the ProntoQA and ProsQA datasets, we fine-tune the pretrained GPT-2 model~\cite{radford2019language} for $16$k steps, where we use a learning rate of $10^{-4}$ with linear warmup for 100 steps, and the batch size is set to 128. 
To evaluate the models, we use greedy decoding and check the exact match with the ground truth.

For Keys-Finding Maze, due to its specific vocabulary, we trained a T5 model~\cite{2020t5} from scratch for 100k steps with a learning rate of $7.5 \times 10^{-4}$ and a batch size of 1024. We evaluate the models by the \emph{1-Feasible-10} metric. Namely, for each evaluation task, we randomly sample 10 responses with top-$k$ ($k$=10) decoding and check if any of them is feasible and reaches the goal location. 

\paragraph{Results}
As shown in \Cref{table:synthetic}, our latent approach performs better than the baselines
for both the Keys-Finding Maze and ProntoQA tasks.
Notably, the absolute improvement is 15\% for the Keys-Finding Maze problem, 
and we reach 100\% accuracy on the relatively easy ProntoQA dataset.
For the more difficult ProsQA, the CoT baseline only obtains 77.5\% accuracy,
the latent approach achieves $17.5\%$ performance gain.



\begin{table*}[t]
\begin{adjustbox}{width=\textwidth}
\begin{tabular}{lllllllllll}
\toprule
\multicolumn{2}{c}{\multirow{2}{*}{\bf Model}} & \multicolumn{2}{c}{\bf In-Domain} & \multicolumn{5}{c}{\bf Out-of-Domain} & \multicolumn{1}{c}{\bf Average} \\ \cmidrule(lr){3-4} \cmidrule(lr){5-9} \cmidrule(lr){10-10}
& & math & GSM8K & Fresh-Gaokao-Math-2023 & DeepMind-Mathematics & College-Math & Olympia-Math & TheoremQA & All Datasets \\ \midrule
\multirow{3}{*}{\bf Llama-3.2-1B}

& {All-Replace} & 6.7 & 4.2 & 0.0 & 11.8 & 6.0 & {2.1} & 8.5 & 5.6 \\
& {Curriculum-Replace} & {7.1} & \underline{9.8} & \underline{3.3} & \underline{13.0} & 
{7.9} & \bf{2.4} & \underline{10.5} & {7.8} \\
& Poisson-Replace & \underline{13.9 } & \textbf{49.5} & {10.0}  & {12.2}  & \underline{18.9 } & \underline{2.3}  & {9.0 }  & \underline{15.1 }   \\
& \textbf{Latent-AR (ours)} & \textbf{14.7 } & \underline{48.7 } & \textbf{10.0}  & \textbf{14.6 }  & \textbf{20.5} & 1.8  & \textbf{11.3 }  & \textbf{17.8 }   \\


\midrule
\multirow{3}{*}{\bf Llama-3.2-3B}

& {All-Replace} & {10.7} & 12.8 & {10.0} & \underline{19.4} & 12.8 & \bf{5.3} & 11.8 & {11.8} \\
& {Curriculum-Replace} & 10.2 & {14.9} & 3.3 & 16.8 & {12.9} & 3.9 & \bf{14.4} & 10.9 \\
& Poisson-Replace & \underline{23.6 } & \underline{65.9 } & \underline{13.3}  & {17.9 }  & \underline{28.9 } & 2.9  & {11.2 }  & \underline{20.5 }   \\
& \textbf{Latent (ours)} & \textbf{26.1 }  & \textbf{73.8 }  & \textbf{23.3 }  & \textbf{27.1 }  & \textbf{32.9 }  & \underline{4.2}  & \underline{13.5}   & \textbf{28.1 } \\



\midrule
\multirow{3}{*}{\bf Llama-3.1-8B}

& {All-Replace} & {15.7} & 19.9 & 6.7 & {21.1} & {19.5} & {5.0} & {17.5} & 15.0 \\
& {Curriculum-Replace} & 14.6 & {23.1} & {13.3} & 20.3 & 18.7 & 3.9 & 16.6 & {15.8} \\
& Possion-Replace & \textbf{37.9 } & \underline{83.6 }  & \underline{16.6 }  & \textbf{42.7 }  & \textbf{44.7 }  & \underline{9.9 }  & \textbf{19.1 }  & \underline{36.3 }  \\
& \textbf{Latent (ours)} & \underline{37.2 } & \textbf{84.1 }  & \textbf{30.0 }  & \underline{41.3 }  & \underline{44.0 }  & \textbf{10.2 }  & \underline{18.4 }  & \textbf{37.9 }  \\




\bottomrule
\end{tabular}
\end{adjustbox}
\caption{Our latent token replacement strategy significantly outperforms the alternative choices: All-Replace (where all the textual CoT tokens are replaced by latent tokens at once), Curriculum-Replace (where we gradually replace the text tokens for the entire CoT subsequence by latent tokens over the course of training) and Poisson-Replace (where individual chunks of text tokens are replaced with probabilities 0.5).}
\label{table:ablation_replacement}
\end{table*}




\subsubsection{Mathematical Reasoning}

\paragraph{Hyperparameters and Evaluation Metrics}
We considered 3 different sizes of LLMs from the LLaMa herd:  Llama-3.2-1B, Llama-3.2-3B and Llama-3.1-8B models. For all the models, we fine-tune them on the MetaMathQA dataset for 1 epoch. To maximize training efficiency, we use a batch size of 32 with a sequence packing of 4096.
We experiment with different learning rates $10^{-5}, 2.5 \times 10^{-5}, 5 \times 10^{-5}, 10^{-4}$ and select the one with the lowest validation error. 
The final choices are $10^{-5}$ for the 8B model and $2.5 \times 10^{-5}$ for the others. For all the experiments, we use greedy decoding for evaluation.

\paragraph{Accuracy Comparison} \Cref{table:LLMtable} presents the results. Our latent approach consistently outperforms all the baselines across nearly all the tasks, for models of different sizes. For tasks on which we do not observe improvement, our approach is also comparable to the best performance. The gains are more pronounced in specific datasets such as Gaokao-Math-2023. On average, we are observing a $+5.3$ points improvement for the 8B model, $+2.9$ points improvement for the 3B model, and +3.7 points improvement for the 1B model. 


\paragraph{Tokens Efficiency Comparison}
Alongside the accuracy, we also report the number of tokens contained in the generated responses in \Cref{table:LLM-token}, which is the dominating factor of the inference efficiency. 
Our first observation is that for all the approaches, the model size has little influence on the length of generated responses.
Overall, the CoT method outputs the longest responses, while the Sol-Only method outputs the least number of tokens, since it is trained to generate the answer directly. The iCoT method generates short responses as well ($42.8\%$ reduction compared to CoT), as the CoT data has been iteratively eliminated in its training procedure. However, this comes at the cost of significantly degraded model performance compared with CoT, as shown in \Cref{table:LLMtable}. Our latent approach shows an average $17\%$ reduction in token numbers compared with CoT while surpassing it in prediction accuracy.


\subsection{Ablation \& Understanding Studies}
\label{sec:expr_ablation}

\paragraph{Replacement Strategies}
Our latent approach partially replaces the leftmost $m$ CoT tokens, where the value of $m$ varies for each sample. We call such replacement strategies \textbf{AR-Replace}. Here we consider three alternative strategies:
\begin{enumerate}[topsep=0pt]
    \item[(1)] \textbf{All-Replace}: all the text CoT tokens are replaced by the latent tokens.
    \item[(2)] \textbf{Curriculum-Replace}: the entire CoT subsequence are gradually replaced over the course of training, similar to the training procedure used by iCoT and COCONUT~\cite{hao2024training}. We train the model for 8 epochs. Starting from the original dataset, in each epoch we construct a new training dataset whether we further replace the leftmost 16 textual CoT tokens by a discrete latent token.
    \item[(3)] \textbf{Poisson-Replace}: instead of replacing tokens from left to right, we conduct a \emph{Poisson sampling} process to select CoT tokens to be replaced: we split the reasoning traces into chunks consisting of 16 consecutive text tokens, where each chunk is randomly replaced by the latent token with probability 0.5. 
    
\end{enumerate}


\Cref{table:ablation_replacement} reports the results.  Our \textbf{AR-Replace} strategy demonstrate strong performance, outperforming the other two strategies with large performance gap. Our intuition is as follows.
When all the textual tokens are removed, the model struggles to align the latent tokens with the linguistic and semantic structures it learned during pretraining. 

In contrast, partial replacement offers the model a bridge connecting text and latent spaces: the remaining text tokens serve as anchors, helping the model interpret and integrate the latent representations more effectively. 
Interestingly, the curriculum learning strategy is bridging the two spaces very well, where \textbf{All-Replace} and \textbf{Curriculum-Replace} exhibit similar performance. This is similar to our observation that iCoT performs remarkably worse than CoT for mathematical reasoning problems.
\textbf{Poisson-Replace} demonstrates performance marginally worse to our \textbf{AR-Replace} strategy on the 1B and 8B models, but significantly worse on the 3B model. Our intuition is that having a fix pattern of replacement (starting from the beginning and left to right) is always easier for the model to learn. This might be due to the limited finetuning dataset size and model capacity. 



\paragraph{Attention Weights Analysis}
To understand the reason why injecting latent tokens enhanced the model's reasoning performance, we randomly selected two questions from the Math and Collegue-Math dataset  and generate responses, then analyze the attention weights over the input prompt tokens:
\begin{enumerate}
    \item \texttt{What is the positive difference between \$120\%\$ of 30 and \$130\%\$ of 20?}
    \item \texttt{Mark has \$50 in his bank account. He earns \$10 per day at his work. If he wants to buy a bike that costs \$300, how many days does Mark have to save his money?}
\end{enumerate}
Specifically, we take the last attention layer, compute the average attention weights over different attention heads and show its relative intensity over the prompt tokens\footnote{We first compute the average attention weights across multiple heads. This gives us a single lower triangular matrix. Then,
we take the column sum of this matrix to get an aggregated attention weights for each token. Last, we normalize the weights by their average to obtain the relative intensity. A one line pseudocode is: \texttt{column\_sum(avg(attention\_matrices)) / avg(column\_sum(avg(attention\_matrices)))}. 
 }. We compare the averaged attention weights of our model with the CoT model in \Cref{fig:attention}.
Interestingly, our model learns to grasp a stronger attention to numbers and words representing mathematical operations. Both \cref{fig:entry_1} and \cref{fig:entry_2} show that the latent model focus more on the numbers, such as \texttt{120}, \texttt{30}, and \texttt{130} for the first question.
For the second question, our latent model shows a larger attention weights on numbers including \texttt{50}, \texttt{10}, and \texttt{300}, and also tokens semantically related to mathematical operations such as \texttt{earns} (means addition) and \texttt{cost} (means subtraction). 
This suggests that, by partially compressing the reasoning trace into a mix of latent and text tokens, we allow the model to effectively focus on important tokens that build the internal logical flow. See \Cref{app:generated_text_attention} for the exact response generated by our approach and the CoT baseline.


\begin{figure}[t]
  \centering
  \begin{subfigure}[b]{\columnwidth}
       \includegraphics[width=8cm]{plots/entry_1.png}
  \caption{Prompt: \texttt{What is the positive difference between 
  \$120\%\$ of 30 and \$130\%\$ of 20?}}
  \label{fig:entry_1}
  \end{subfigure}

  \begin{subfigure}[b]{\columnwidth}
       \includegraphics[width=8cm]{plots/entry_7746.png}
  \caption{Prompt: \texttt{Mark has \$50 in his bank account. He earns \$10 per day at his work. If he wants to buy a bike that costs \$300, how many days does Mark have to save his money?}}
  \label{fig:entry_2}
  \end{subfigure}
  \caption{Comparing with the CoT model, our latent approach have high attention weights on numbers and text tokens representing mathematical operations.}
  \label{fig:attention}
\end{figure}
 

\paragraph{Additional Experiments} We provide 4 additional example responses for questions in the Math and TheoremQA datasets in \Cref{app:generated_text_others}. In \Cref{app:additional_experiments}, we compare all the approaches when the model is trained on the DART-MATH~\cite{tong2024dart} 
dataset, where similar trends are observed.

\section{Conclusion and Suggestions}

Our work, including the creation of \texttt{ScholarLens} and the proposal of \texttt{LLMetrica}, provides methods for assessing LLM penetration in scholarly writing and peer review. By incorporating diverse data types and a range of evaluation techniques, we consistently observe the growing influence of LLMs across various scholarly processes, raising concerns about the credibility of academic research. As LLMs become more integrated into scholarly workflows, it is crucial to establish strategies that ensure their responsible and ethical use, addressing both content creation and the peer review process. 

Despite existing guidelines restricting LLM-generated content in scholarly writing and peer review,\footnote{\href{https://aclrollingreview.org/acguidelines\#-task-3-checking-review-quality-and-chasing-missing-reviewers}{Area Chair} \&  \href{https://aclrollingreview.org/reviewerguidelines\#q-can-i-use-generative-ai}{Reviewer} \& \href{https://www.aclweb.org/adminwiki/index.php/ACL_Policy_on_Publication_Ethics\#Guidelines_for_Generative_Assistance_in_Authorship}{Author} guidelines.} challenges still remain. 
To address these, we propose the following based on our work and findings: 
(i) \textbf{Increase transparency in LLM usage within scholarly processes} by incorporating LLM assistance into review checklists, encouraging explicit acknowledgment of LLM support in paper acknowledgments, and 
reporting LLM usage patterns across diverse demographic groups;
% reporting LLM penetration based on social demographic features;
(ii) \textbf{Adopt policies to prevent irresponsible LLM reviewers} by establishing feedback channels for authors on LLM-generated reviews and developing fine-grained LLM detection models~\cite{abassy-etal-2024-llm, cheng2024beyond, artemova2025beemobenchmarkexperteditedmachinegenerated} to distinguish acceptable LLM roles (e.g., language improvement vs. content creation);
(iii) \textbf{Promote data-driven research in scholarly processes} by supporting the collection of review data for further robust analysis~\cite{dycke-etal-2022-yes}.\footnote{\url{https://arr-data.aclweb.org/}}

% make LLM usage transparent in scholarly processes: such as incorporating LLM usage into review checklists, encouraging explicit acknowledgment of LLM assistance in paper acknowledgments, and reporting LLM penetration based on social demographic features; (ii) Adopt policies to prevent irresponsible LLM reviewers: such as providing authors feedback on LLM-assisted reviews, and developing fine-grained LLM detection models~\cite{cheng2024beyond} to distinguish acceptable LLM roles (e.g., language improvement vs. content creation); (iii) Encourage data-driven research in scholarly processes: such as supporting review data collection for further research.

 



\bibliography{main}
\bibliographystyle{icml2025}

\clearpage
\onecolumn
\appendix
%chatgpt helped me with this - needs more work 
\section{Preliminaries}\label{app:prelims}

%\tbd{the basic autoencoder eq. (my colleagues didn't know what I mean exactly when I talked about this work)}

\paragraph{1- sparse SAE probes}
To evaluate how well SAE features predict a certain abstract feature, we utilize 1-sparse probes \cite{gurnee2023finding}. Specifically, we collect activations of a specific SAE feature on a contrastive dataset containing both answerable and not answerable examples, and fit a slope coefficient and intercept to predict the dataset label using linear regression. The Gemma 2 SAEs are trained using a JumpReLU activation function \cite{lieberum2024gemma}. We can sample SAE activations after the activation function (post-relu) or before (pre-relu).
Since there are more learnt features to be found in the latter setting, the main paper figures focus on that. However, we report all results for the post-relu setting in the appendix.


\paragraph{Residual stream probes}

Our residual stream probes are trained on model activations sampled from the model's residual stream. To avoid overfitting, we train the regression model using 5-fold cross validation and perform a hyperparameter optimization by sweeping over regularization parameters with 26 logarithmically spaced steps between 0.0001 and 1. To measure the variability of residual stream probes, we repeat our analysis 10 times with different randomly sampled training datasets.

\paragraph{N-sparse SAE probes}
To train SAE probes with more than 1 feature, we follow the general methodology of our 1-sparse probes. As testing all possible SAE feature combinations is computationally infeasible, we iteratively increase the number of features while testing only the most promising candidates for higher features combinations. Specifically, to find combinations of $k$ features, we use the top 50 best performing features of size $k-1$ and test all possible new combinations with the 500 best performing single SAE features. We use a constant regularization parameter of 1 for the probes, regardless of the number of features.

\paragraph{Feature similarities}

To calculate feature similarities, we use the cosine similarity of the corresponding SAE encoder weight and the slope coefficients of the linear probes trained on the residual stream. SAE features are only compared to other SAE features of the same SAE, and residual stream probes trained at the same location in the model as the SAE. To compare how similar differently sized groups of SAE features are to the residual stream probes, we calculate the mean absolute cosine sim of the top 10 best performing SAE features of a certain group size (1 to 5) with the 10 residual stream probes trained on different training subsets.

\section{Datasets}
\label{app:datasets}

\myparagraph{Full Dataset details}

\begin{itemize}[leftmargin=*,topsep=0pt,noitemsep]
    \item \textbf{SQUAD} \citep{rajpurkar2018know}:  Dataset consisting of a short context passage and a question relating to the context. We follow the training data split and prompting template provided by \citet{slobodkin2023curious}.
    \item \textbf{IDK} \citep{sulem2021we}: Dataset with questions in the style of SQUAD, containing both answerable and unanswerable examples. We specifically use the non-competitive and unanswerable subsets of the ACE-whQA dataset.
    \item \textbf{BoolQ\_3L} 
    \citep{sulem2022yes}: Yes/no questions with answerable and unanswerable subsets.
    \item \textbf{Math Equations}: Synthetic dataset contrasting solvable equations with equations containing unknown variables.
    \item \textbf{Celebrity Recognition}: Queries requiring knowledge about celebrities.
    For construction, we use a public dataset of actors and movies from IMDB\footnote{\url{https://www.kaggle.com/datasets/darinhawley/imdb-films-by-actor-for-10k-actors}}, and generate a list of the 1000 most popular actors after 1990, as measured by the total number of ratings their movies received. We construct an additional dataset of non-celebrity names by randomly generating first and last name combinations using the most common North American names from Wikipedia\footnote{\url{https://en.wikipedia.org/wiki/Lists_of_most_common_surnames_in_North_American_countries} and \url{https://en.wikipedia.org/wiki/List_of_most_popular_given_names?utm_source=chatgpt.com}}. 
\end{itemize}

\paragraph{Dataset sizes}

\begin{table}[h]
    \centering
    \begin{tabular}{lc}
        \hline
        Dataset & Size \\
        \hline
        SQUAD (train) & 2000 \\
        BoolQ (train) & 2000 \\
        SQUAD (test) & 1800 \\
        SQUAD (variations) & 1800 \\
        BoolQ (test) & 2000 \\
        IDK & 484 \\
        Equation & 2000 \\
        Celebrity & 600 \\
        \hline
    \end{tabular}
    \caption{Number of examples for each used dataset.}
    \label{table:dataset-size}
\end{table}

Table~\ref{table:dataset-size} shows the number of examples for each dataset used in our evaluation.




\section{Additional analysis}

\subsection{Answerability Detection at Different Layers} \label{app:eval-other-layers}
\begin{figure*}[t]
    \centering
    \includegraphics[width=0.8\textwidth,trim={0 1.4cm 0 3.4cm},clip]{figures/sae_feature_accuracies_layer20_post.png}
    \caption{Answerability detection accuracies for top SAE features (Layer 20, post-activation).}
    \label{fig:sae-probe_post20}
\end{figure*}

\begin{figure*}[t]
    \centering
    \includegraphics[width=0.8\textwidth,trim={0 1.4cm 0 3.4cm},clip]{figures/sae_feature_accuracies_layer20_pre.png}
    \caption{Answerability detection accuracies for top SAE features (Layer 20, pre-activation).}
    \label{fig:sae-probe_pre20}
\end{figure*}

\begin{figure*}[t]
    \centering
    \includegraphics[width=0.8\textwidth,trim={0 1.4cm 0 3.4cm},clip]{figures/sae_feature_accuracies_layer31_post.png}
    \caption{Answerability detection accuracies for top SAE features (Layer 31, post-activation).}
    \label{fig:sae-probe_post31}
\end{figure*}

\begin{figure*}[t]
    \centering
    \includegraphics[width=0.7\textwidth,trim={0 2.2cm 1cm 3.4cm},clip]{figures/all_layers_probe.png}
    \caption{Linear probe trained on Layer 20 and Layer 31 residual stream (SQuAD) and evaluated on IDK, BoolQ, Celebrity, and Equation. The plot shows the median accuracy including the first and third quartile.}
    \label{fig:res-probe-all}
\end{figure*}

We repeat our SAE feature analysis in Layer 20 of the model, as well as providing additional analysis for SAE features activations sampled after the activation function. Figure~\ref{fig:sae-probe_pre20} shows the Layer 20 results using activations sampled before the activation function, while Figures~\ref{fig:sae-probe_post20} and \ref{fig:sae-probe_post31} show analogous results when sampling SAE activations after the activation function. Sampling after the activation reduces the number of relevant features our probe finds, since many features are inactive. However, this does not change the overall results, as we still find features with good generalization performance. 

Figure~\ref{fig:res-probe-all} shows the probing accuracy for the residual stream linear probe for both Layer 20 and 31. The evaluation is repeated across 10 seeds with different training set splits. While the SAE features, as part of the pre-trained autoencoder model, do not heavily depend on the probing dataset, this is not necessarily true for the residual stream probe. The's probe performance across the out-of-distribution datasets varies strongly, indicating that the generalization performance heavily depends on the minor differences in the training data. 

\subsection{Prompt variations}

\begin{figure*}[t]
    \centering
    \includegraphics[width=0.8\textwidth,trim={0 1.4cm 0 3.4cm},clip]{figures/sae_feature_accuracies_layer31_pre_SQUAD_train_variation.png}
    \caption{Performance of top SAE features and the residual stream linear probe on variations of prompt used with the SQuAD dataset (layer 31, pre-activation).}
    \label{fig:prompt-variation}
\end{figure*}

\begin{table*}[h]
    \centering
    \begin{tabular}{lp{10cm}}
        \toprule
        Default & Given the following passage and question, answer the question:\newline Passage: \{passage\}\newline Question: \{question\} \\
        \midrule
        Variation 1 & Please read this passage and respond to the query that follows:\newline Passage: \{passage\}\newline Question: \{question\} \\
        \midrule
        Variation 2 & Based on the text below, please address the following question:\newline Text: \{passage\}\newline Question: \{question\} \\
        \midrule
        Variation 3 & Consider the following excerpt and respond to the inquiry:\newline Excerpt: \{passage\}\newline Inquiry: \{question\} \\
        \midrule
        Variation 4 & Review this content and answer the question below:\newline Content: \{passage\}\newline Question: \{question\} \\
        \midrule
        Variation 5 & Using the information provided, respond to the following:\newline Information: \{passage\}\newline Query: \{question\} \\
        \bottomrule
    \end{tabular}
    \caption{SQuAD prompt template variations.}
    \label{table:variations}
\end{table*}

% \begin{table*}[h]
%     \centering
%     \begin{tabular}{|l|p{10cm}|}
%         \hline
%         Default & Given the following passage and question, answer the question:\newline Passage: \{passage\}\newline Question: \{question\}\\
%         \hline
%         Variation 1 & Please read this passage and respond to the query that follows:\newline Passage: \{passage\}\newline Question: \{question\} \\
%         \hline
%         Variation 2 & Based on the text below, please address the following question:\newline Text: \{passage\}\newline Question: \{question\} \\
%         \hline
%         Variation 3 & Consider the following excerpt and respond to the inquiry:\newline Excerpt: \{passage\}\newline Inquiry: \{question\} \\
%         \hline
%         Variation 4 & Review this content and answer the question below:\newline Content: \{passage\}\newline Question: \{question\} \\
%         \hline
%         Variation 5 & Using the information provided, respond to the following:\newline Information: \{passage\}\newline Query: \{question\} \\
%         \hline
%     \end{tabular}
%     \caption{SQuAD prompt template variations.}
%     \label{table:variations}
% \end{table*}

We investigated if the SAE features or the residual stream probes are sensitive to small variations in the prompt. To evaluate this question, we created five variations of the prompt template used for the SQuAD training data (see Table~\ref{table:variations}). The results can be found in Figure~\ref{fig:prompt-variation}, and indicate neither the residual stream probe nor the SAE features are sensitive to this kind of variation. 

\subsection{In-domain SAE feature accuracies}

\begin{figure*}[t]
    \centering
    \includegraphics[width=0.8\textwidth,trim={0 1.4cm 0 3.4cm},clip]{figures/sae_top_feature_accuracies_in_domain.png}
    \caption{Performance of the top SAE feature's probing accuracy when training and evaluating features on each dataset individually (pre-activation).}
    \label{fig:top-in-domain}
\end{figure*}


Figure~\ref{fig:top-in-domain} shows the accuracy of 1-sparse SAE feature probes for each dataset individually, demonstrating that each of our contrastive datasets is detectable with a probing accuracy of over 80\%.

\subsection{SAE Feature Combination Analyses} \label{app:feature-combinations}
% \begin{figure*}[t]
%     \centering
%     \includegraphics[width=0.8\textwidth,trim={0 2.2cm 0 3.4cm},clip]{figures/avg_feature_k.png}
%     \caption{Average accuracy vs.\ number of features. (Potentially add the probe line, etc.)}
%     \label{fig:sae-combi-probe}
% \end{figure*}

\begin{figure*}[t]
    \centering
    \includegraphics[width=0.8\textwidth,trim={0 1.4cm 0 3.4cm},clip]{figures/top_sae_feature_group_accuracies_k_L31_pre.png}
    \caption{Performance of top feature combinations (layer 31, pre-activation).}
    \label{fig:top-combis}
\end{figure*}

Figure~\ref{fig:top-combis} shows additional probing analysis for the best performing groups of SAE features up to a group size of five. Group performance is generally dominated by the best performing features and does not majorly exceed the performance of the strongest feature. 

% \subsection{Celebrity Dataset Evaluation}
% \begin{figure*}[t]
%     \centering
%     \includegraphics[width=0.8\textwidth,trim={0 0 0 3.4cm},clip]{figures/celeb.png}
%     \caption{Comparison of top SAE features on the Celebrity dataset. \lh{remove?}}
%     \label{fig:celeb}
% \end{figure*}

\begin{figure*}[t]
    \centering
    \includegraphics[width=0.8\textwidth,trim={0 1.4cm 0 3.4cm},clip]{figures/hierarchical_sae_probe_layer20_pre.png}
    \caption{Accuracies of SAE probes trained on different numbers of SAE features (Layer 20, pre-activation).}
    \label{fig:sae-k-pre20}
\end{figure*}

Figure~\ref{fig:sae-k-pre20} shows additional analysis for SAE feature combinations in Layer~20, analogous to the results for Layer~31 given in Figure~\ref{fig:sae-k-pre31}.

\subsection{Cosine Similarities}
\begin{figure*}[t]
    \centering
    \includegraphics[width=0.8\textwidth,trim={0 0 0 3.4cm},clip]{figures/sae_probe_similarities_by_layer.png}
    \caption{Absolute cosine similarities of top 10 SAE features at different layers, compared with the residual stream probe.}
    \label{fig:similarity_k}
\end{figure*}

We conducted an additional similarity analysis for the top SAE feature groups of different sizes. The results can be found in Figure~\ref{fig:similarity_k} and show a clear trend of larger groups of features becoming more similar to the linear probes. This provides some weak evidence that by default, linear probes might learn more specialized directions that can be represented as a linear combination of more general SAE features. 

% \begin{figure*}[h!]
%     \centering %trim titles
%     \includegraphics[width=.8\textwidth,trim={0 1.4cm 0 3.4cm},clip]{figures/sae_feature_accuracies_layer20_post.png}
%     \caption{Answerability detection accuracies for top SAE features (layer 20, post-activation).}
%     \label{fig:sae-probe_post20}
% \end{figure*}

% \begin{figure*}[h!]
%     \centering %trim titles
%     \includegraphics[width=.8\textwidth,trim={0 1.4cm 0 3.4cm},clip]{figures/sae_feature_accuracies_layer20_pre.png}
%     \caption{Answerability detection accuracies for top SAE features (layer 20, pre-activation).}
%     \label{fig:sae-probe_pre20}
% \end{figure*}

% \begin{figure*}[h!]
%     \centering %trim titles
%     \includegraphics[width=.8\textwidth,trim={0 1.4cm 0 3.4cm},clip]{figures/sae_feature_accuracies_layer31_post.png}
%     \caption{Answerability detection accuracies for top SAE features (layer 31, post-activation).}
%     \label{fig:sae-probe_post31}
% \end{figure*}


% % \begin{figure*}[h!]
% %     \centering %trim titles
% %     \includegraphics[width=.7\textwidth,trim={0 2.2cm 4.2cm 3.4cm},clip]{figures/res_probe.png}
% %     \caption{\vt{can we have less space between the individual plots and make the bars thinner? best so that it fits next to fig 1. also: there's an alternative pic. image.png what's that?}}
% %     \label{fig:res-probe}
% % \end{figure*}

% \begin{figure*}[h!]
%     \centering %trim titles
%     \includegraphics[width=.7\textwidth,trim={0 2.2cm 1cm 3.4cm},clip]{figures/res_probe_20.png}
%     \caption{Linear probe trained on the Layer 20 residual stream. The probe is trained on the SQUAD dataset and then evaluated on the out-of-distribution datasets (IDK, BoolQ, Celebrity, Equation). Error bars show the standard deviation, averaged over 10 bootstrap samples.}
%     \label{fig:res-probe-20}
% \end{figure*}


% \begin{figure*}[h!]
%     \centering
%     \includegraphics[width=.8\textwidth,trim={0 2.2cm 0 3.4cm},clip]{figures/avg_feature_k.png}
%     \caption{\vt{can we add the probe in here?}}
%     \label{fig:sae-combi-probe}
% \end{figure*}

% \begin{figure*}[h!]
%     \centering %trim titles
%     \includegraphics[width=.8\textwidth,trim={0 1.4cm 0 3.4cm},clip]{figures/top_features_k.png}
%     \caption{\vt{later, better make the figures more space efficicient, borders which I cannot crop}}
%     \label{fig:top-combis}
% \end{figure*}

% \begin{figure*}[h!]
%     \centering %trim titles
%     \includegraphics[width=.8\textwidth,trim={0 0 0 3.4cm},clip]{figures/celeb.png}
%     \caption{\vt{can we add the probe directly in here? and maybe remove some of the ones of the right hand side to make it fit one column. maybe we also don't need this plot extra. let's see/discuss}}
%     \label{fig:celeb}
% \end{figure*}

% \begin{figure*}[h!]
%     \centering %trim titles
%     \includegraphics[width=.8\textwidth,trim={0 1.4cm 0 3.4cm},clip]{figures/hierarchical_sae_probe_layer20_pre.png}
%     \caption{Accuracies of SAE probes trained on different numbers of SAE features (Layer 20, pre-activation}
%     \label{fig:sae-k-pre20}
% \end{figure*}

% \begin{figure*}[h!]
%     \centering %trim titles
%     \includegraphics[width=.8\textwidth,trim={0 1.4cm 0 3.4cm},clip]{figures/hierarchical_sae_probe_layer20_post.png}
%     \caption{Accuracies of SAE probes trained on different numbers of SAE features (Layer 20, pre-activation}
%     \label{fig:sae-k-post20}
% \end{figure*}


% \begin{figure*}[h!]
%     \centering %trim titles
%     \includegraphics[width=.8\textwidth,trim={0 1.4cm 0 3.4cm},clip]{figures/hierarchical_sae_probe_layer20_pre.png}
%     \caption{Accuracies of SAE probes trained on different numbers of SAE features (Layer 31, post-activation}
%     \label{fig:sae-k-post31}
% \end{figure*}




% \begin{figure*}[h!]
%     \centering %trim titles
%     \includegraphics[width=.8\textwidth,trim={0 0 0 3.4cm},clip]{figures/sae_probe_similarities_by_layer.png}
%     \caption{Absolute cosine similarities of top 10 SAE features for different number of feature combinations, the model layer of the SAE, and activation hook points, compared with the residual stream probe.}
%     \label{fig:similarity_k}
% \end{figure*}

% \begin{figure*}[h!]
%     \centering %trim titles
%    \includegraphics[width=.4\textwidth,trim={0 2.2cm 1cm 3.4cm},clip]{figures/res_probe_31.png}
% \caption{..%Comparison between top SAE features (left), pre-activation, and linear probes (right) on layer 31.
% }
%     \label{fig:probe_pre31}
% \end{figure*}

% \begin{figure}[h!]
%     \centering %trim titles
%     \includegraphics[width=.5\textwidth,trim={0 1.4cm 0 3.4cm},clip]{figures/sae_top_feature_accuracies_in_domain.png}
%     \caption{Top SAE feature accuracies when training the 1-sparse probes on each dataset individually using a 20\% test split (pre-activation function).}
%     \label{fig:sae-in-domain}
% \end{figure}


% mention why we focus on layer 31 pre -relu





% \myparagraph{Dedicated Prompts} % prompts dedicated to the task
% QA prompt variation


%(only use 80\% of training data here for some reason, probably matching some earlier version of model probes - could rerun overnight).


\myparagraph{Other experiments}
We validated our setup by searching for bias-related features as it was done in related works.
We also experimented with (inofficial) SAEs for an instruction-tuned Llama model, but could not find SAE features with sufficient in-domain probing accuracy. Finally, we also performed analysis on Gemma 2 2B and also the base models, but performance on the answerability task was relatively low in these models (the best SAE features achieved around 70\% probing accuracy).

\end{document}


% This document was modified from the file originally made available by
% Pat Langley and Andrea Danyluk for ICML-2K. This version was created
% by Iain Murray in 2018, and modified by Alexandre Bouchard in
% 2019 and 2021 and by Csaba Szepesvari, Gang Niu and Sivan Sabato in 2022.
% Modified again in 2023 and 2024 by Sivan Sabato and Jonathan Scarlett.
% Previous contributors include Dan Roy, Lise Getoor and Tobias
% Scheffer, which was slightly modified from the 2010 version by
% Thorsten Joachims & Johannes Fuernkranz, slightly modified from the
% 2009 version by Kiri Wagstaff and Sam Roweis's 2008 version, which is
% slightly modified from Prasad Tadepalli's 2007 version which is a
% lightly changed version of the previous year's version by Andrew
% Moore, which was in turn edited from those of Kristian Kersting and
% Codrina Lauth. Alex Smola contributed to the algorithmic style files.
