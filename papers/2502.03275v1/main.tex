%%%%%%%% ICML 2025 EXAMPLE LATEX SUBMISSION FILE %%%%%%%%%%%%%%%%%

\documentclass{article}

% Recommended, but optional, packages for figures and better typesetting:
\usepackage{microtype}
\usepackage{graphicx}
% \usepackage{subfigure}
\usepackage{booktabs} % for professional tables
\usepackage{adjustbox}
\usepackage{tcolorbox}
% hyperref makes hyperlinks in the resulting PDF.
% If your build breaks (sometimes temporarily if a hyperlink spans a page)
% please comment out the following usepackage line and replace
% \usepackage{icml2025} with \usepackage[nohyperref]{icml2025} above.
\usepackage{hyperref}


% Attempt to make hyperref and algorithmic work together better:
\newcommand{\theHalgorithm}{\arabic{algorithm}}

% Use the following line for the initial blind version submitted for review:
\usepackage[accepted]{icml2025}

% If accepted, instead use the following line for the camera-ready submission:
% \usepackage[accepted]{icml2025}

\usepackage{url}
\usepackage{color, float}
\usepackage{xspace}
\usepackage{multirow}
\usepackage{fancyhdr}
\usepackage{booktabs} % for professional tables
\usepackage{nicefrac}       % compact symbols for 1/2, etc.
\usepackage{enumerate}
\usepackage{xspace}

% For theorems and such
\usepackage{amsthm, amsmath, mathtools, amsfonts, amssymb, dsfont, enumitem, bm, bbm, cancel}
\usepackage{graphicx}
\usepackage{subcaption}
\graphicspath{{./plots/}}

\definecolor{darkgreen}{rgb}{0.0, 0.5, 0.0}
% if you use cleveref..
\usepackage[capitalize,noabbrev, nameinlink]{cleveref}


%%%%%%%%%%%%%%%%%%%%%%%%%%%%%%%%
% THEOREMS
%%%%%%%%%%%%%%%%%%%%%%%%%%%%%%%%
\theoremstyle{plain}
\newtheorem{theorem}{Theorem}[section]
\newtheorem{proposition}[theorem]{Proposition}
\newtheorem{lemma}[theorem]{Lemma}
\newtheorem{corollary}[theorem]{Corollary}
\theoremstyle{definition}
\newtheorem{definition}[theorem]{Definition}
\newtheorem{assumption}[theorem]{Assumption}
\theoremstyle{remark}
\newtheorem{remark}[theorem]{Remark}



\newcommand{\Xtilde}{{\widetilde{X}}}
\newcommand{\fenc}{{f_\text{enc}}}
\newcommand{\fdec}{{f_\text{dec}}}
\newcommand{\sg}{{\texttt{sg}\xspace}}
\newcommand{\increase}[1]{\textcolor{darkgreen}{($\uparrow$ +#1)}}

\newcommand{\decrease}[1]{\textcolor{darkgreen}{\textbf{($\downarrow$ -#1)}}}

\definecolor{darkred}{rgb}{0.6, 0, 0}
% Modify the command to use the dark red color
\newcommand{\increasered}[1]{\textcolor{darkred}{($\uparrow$ +#1)}}


\newcommand{\phh}[1]{}



\newcommand{\V}{{\mathcal{V}}}
\newcommand{\E}{{\mathcal{E}}}
\newcommand{\R}{{\mathbb{R}}}
\newcommand{\norm}[1]{\left\|#1\right\|}
\DeclareMathOperator*{\argmin}{argmin}
\DeclareMathOperator*{\argmax}{argmax}

\newcommand{\prompt}{\texttt{[Prompt]}:}
\newcommand{\response}{\texttt{[Response]}:}

% Todonotes is useful during development; simply uncomment the next line
%    and comment out the line below the next line to turn off comments
%\usepackage[disable,textsize=tiny]{todonotes}
\usepackage[textsize=tiny]{todonotes}

\newcommand{\qq}[1]{\textcolor{orange}{[QQ: #1]}}
\newcommand{\yc}[1]{\textcolor{blue}{YC: #1}}
\newcommand{\hz}[1]{\textcolor{purple}{[HZ: #1]}}

\counterwithin{figure}{section}
\counterwithin{table}{section}


\usepackage{titlesec}
% \titlespacing{\section}{0pt}{5pt}{5pt}
% \titlespacing{\subsection}{0pt}{3pt}{3pt}
% \titlespacing{\subsubsection}{0pt}{3pt}{3pt}
\titlespacing{\paragraph}{0pt}{5pt}{5pt}
% \setlength{\textfloatsep}{20pt}
% \setlength{\floatsep}{5pt}


% The \icmltitle you define below is probably too long as a header.
% Therefore, a short form for the running title is supplied here:
\icmltitlerunning{Token Assorted: Mixing Latent and Text Tokens for Improved Language Model Reasoning}

\begin{document}

\twocolumn[
\icmltitle{Token Assorted: Mixing Latent and Text Tokens for\\ Improved Language Model Reasoning}



% It is OKAY to include author information, even for blind
% submissions: the style file will automatically remove it for you
% unless you've provided the [accepted] option to the icml2025
% package.

% List of affiliations: The first argument should be a (short)
% identifier you will use later to specify author affiliations
% Academic affiliations should list Department, University, City, Region, Country
% Industry affiliations should list Company, City, Region, Country

% You can specify symbols, otherwise they are numbered in order.
% Ideally, you should not use this facility. Affiliations will be numbered
% in order of appearance and this is the preferred way.
\icmlsetsymbol{equal}{*}

\begin{icmlauthorlist}
\icmlauthor{DiJia Su}{xxx}
\icmlauthor{Hanlin Zhu}{equal,yyy}
\icmlauthor{Yingchen Xu}{equal,xxx,zzz}
\icmlauthor{Jiantao Jiao}{yyy}
\icmlauthor{Yuandong Tian$^\dagger$}{xxx}
\icmlauthor{Qinqing Zheng$^\dagger$}{xxx}
\end{icmlauthorlist}

\icmlaffiliation{xxx}{Meta AI}
\icmlaffiliation{yyy}{UC Berkeley}
\icmlaffiliation{zzz}{UCL}

% You may provide any keywords that you
% find helpful for describing your paper; these are used to populate
% the "keywords" metadata in the PDF but will not be shown in the document
\icmlkeywords{Machine Learning, ICML}

\vskip 0.3in
]

% this must go after the closing bracket ] following \twocolumn[ ...

% This command actually creates the footnote in the first column
% listing the affiliations and the copyright notice.
% The command takes one argument, which is text to display at the start of the footnote.
% The \icmlEqualContribution command is standard text for equal contribution.
% Remove it (just {}) if you do not need this facility.

%\printAffiliationsAndNotice{}  % leave blank if no need to mention equal contribution
\printAffiliationsAndNotice{\icmlEqualContribution} % otherwise use the standard text.

Humor is a social binding agent. It is an act of creativity that can provoke emotional reactions on a broad range of topics. Humor has long been thought to be “too human” for AI to generate. However, humans are complex, and humor requires our complex set of skills: cognitive reasoning, social understanding, a broad base of knowledge, creative thinking, and audience understanding. We explore whether giving AI such skills enables it to write humor. We target one audience: Gen Z humor fans. We ask people to rate meme caption humor from three sources: highly upvoted human captions, 2) basic LLMs, and 3) LLMs captions with humor skills. We find that users like LLMs captions with humor skills more than basic LLMs and almost on par with top-rated humor written by people. We discuss how giving AI human-like skills can help it generate communication that resonates with people. 

%!TEX root=main.tex

\section{Introduction}
% Decision-makers, analysts, data scientists, and policymakers frequently rely on data to draw conclusions and extract insights. This data-driven approach helps them identify actionable recommendations aimed at influencing an outcome of interest, such as increasing product satisfaction or income levels or decreasing the likelihood of experiencing serious health conditions \cite{galhotra2022hyper,lakkaraju2016interpretable,agrawal1994fast}. 
\revc{Prescriptions, or actionable recommendations, are commonly generated across various fields to influence key outcomes such as improving product satisfaction, enhancing economic policies, or increasing business efficiency. 
%Decision- or policy-makers, analysts, data scientists, and 
Policymakers in government, decision-makers in businesses, and data scientists in various fields, often rely on data-driven approaches to identify 
%actionable recommendations 
potential actions to influence an outcome of interest, such as increasing income levels or loan approval rates}.
% , or decreasing the likelihood of experiencing serious health conditions. 
%
While association or prediction-based methods are extensively used in practice to draw useful insights from data, they typically identify correlations among variables and may fail to reveal the underlying causal factors, i.e., which actions may result in an improved outcome, needed for informed decision-making. 
%For recommendations to be truly impactful, there must be a clear  explanation that justifies why a particular decision is appropriate for a specific subpopulation~\cite{sun2021treatment,plecko2022causal}. 

\emph{Causal analysis} or {\em causal inference}, therefore, is considered one of the most important requirements to generate prescriptions that are {\em actionable} and aligned with human reasoning~\cite{imbens2024causal}. Causal inference, and in particular {\em observational studies} for causal inference on collected data (when controlled trials are impossible due to cost or ethical reasons), have been extensively studied in the statistics and artificial intelligence (AI) literature for several decades \cite{rubin2005causal, pearl2009causal}. Motivated by this foundational work on causal inference, the notion of causality has also influenced the field of database research. The causal models from AI have been extended to relational databases \cite{salimi2020causal},  and causality has been incorporated into various data management tasks such as finding responsibilities of inputs toward query answers ~\cite{meliou2010causality, meliou2009so, meliou2014causality}, explanations for query answers \cite{roy2014formal, DBLP:journals/pacmmod/YoungmannCGR24}, data discovery~\cite{galhotra2023metam,youngmann2023causal}, data cleaning~\cite{pirhadi2024otclean,salimi2019interventional}, hypothetical reasoning \cite{galhotra2022causal}, and large system diagnostics~\cite{markakis2024sawmill,causalsim,sage, gudmundsdottir2017demonstration}. 


\revc{If-then rules are generally considered interpretable by humans~\cite{lakkaraju2016interpretable,guidotti2018local,van2021evaluating,pradhan2022interpretable,chen2018optimization}.
We give a concrete example of the difference between association and causation in generating prescriptions or recommended actions in the form of if-then rules below}:
\begin{example}	%
\label{example:ex1} {\bf Importance of causal prescriptions:}
Consider the Stack Overflow (SO) annual developer survey
\cite{stackoverflowreport}, where respondents from around the world answer
questions about their jobs and demographics. A sample of the dataset \reva{with a subset of the
attributes (there are 20 attributes)} is presented in \cref{tab:data}.
%
Alice, a researcher in the United Nations (UN) finance department, is interested in discovering ways to increase the salaries of high-tech employees worldwide. She is looking for a set of actionable recommendations 
%(that we call a prescription rules) 
to raise the overall average salary.
%
Using association-based approaches~\cite{chen2018optimization,lakkaraju2016interpretable}, she may discover that individuals residing in the US who identify as straight or heterosexual tend to earn higher salaries (see \cref{exp:quality} for full details). However, this observation merely indicates a correlation: people living in the US, for example, generally earn more than those outside the country. Their comparatively higher salaries are primarily attributable to the country's economy and are unrelated to their sexual orientation. Thus, this observation cannot be used as a prescription rule to increase salary. 
Our causal analysis, on the other hand, reveals that individuals aged 25-34 with dependents would benefit from working as front-end developers.
This results in a \$44,009 annual salary increase on average. \reva{Another observation is that students should pursue an
undergraduate major in CS. %Computer Science (CS). 
This can boost their salary by \$22,174 per year} (see details in \cref{sec:casestudy}).
\end{example}

%It has been incorporated into various tasks including . 
%Causal interventions are often more relatable and easier to understand, as they offer insight into the underlying reasons behind the recommendations and allow unraveling complex cause-effect relationships that govern our world~\cite{pearl2009causality}. Furthermore, causal interventions often have long-lasting effects~\cite{imbens2024causal}.

%, making it essential that the prescribed actions are not only actionable but also 

%causally consistent. 

%Decision makings, in particular, high-stak

\cut{
In this work, {we study the problem of generating causal insights (referred to as \emph{prescription rules}), which serve as actionable recommendations} to improve an outcome of interest.
Recent works have introduced causality to the field of database research~\cite{meliou2010causality,  meliou2014causality,salimi2020causal,10.14778/3554821.3554902}. It has been incorporated into various tasks including data discovery~\cite{galhotra2023metam,youngmann2023causal}, data cleaning~\cite{pirhadi2024otclean,salimi2019interventional}, and large system diagnostics~\cite{markakis2024sawmill,causalsim,sage, gudmundsdottir2017demonstration}. 
We propose using causal inference to generate prescription rules that are both actionable and justifiable.
}

While generating prescriptions based on causal inference may help in robust decision-making, causal prescriptions that solely consider the betterment of an outcome (like salary) are not enough in practice. 
It is well-known that decision-making in many high-stake applications (like hiring policy, or policy for approving loans by banks) may lead to disparate societal or economic impact on different sub-populations. 
As a shocking example from a recent work called 
%For example, 
CauSumX~\cite{DBLP:journals/pacmmod/YoungmannCGR24} that generates a set of causal explanations for an aggregated view, the explanations generated %by CauSumX %recommendations which 
suggest that male individuals do a Bachelor's degree to increase their salary while %suggesting that 
being an unmarried woman 
%the recommendation for women includes getting married 
has the most adverse effect on salary
(borrowed directly 
from Fig.~19 in~\cite{youngmann2024summarizedcausalexplanationsaggregate}). 
%We demonstrate the advantage of using causal reasoning to generate actionable recommendations and the limitations of not considering fairness requirements in the following example. 
We explored this further in the context of generating prescriptions and observed that prescriptions that are not fairness-aware can generate unfair outcomes to some subpopulations which we refer to as the {\em protected group}. Examples include women, Black, Latino, or Native Americans, individuals with a disability, countries with a weaker economy, or other protected groups specific to an application. %Here is a concrete example:


% Understanding the causal factors behind these recommendations is crucial to ensuring that decisions lead to fair and equitable outcomes, particularly in sensitive applications where biased decisions can perpetuate or even exacerbate societal inequalities.
% While prior work has extensively explored techniques for association rule mining~\cite{kumbhare2014overview}, recent efforts have focused on deriving causal explanations for individual data points or entire datasets~\cite{salimi2018bias,youngmann2022explaining,ma2023xinsight}. Although some of these methods produce causally consistent insights, the absence of fairness considerations in the process can lead to unfair outcomes, further reinforcing existing biases. For example, CauSumX~\cite{DBLP:journals/pacmmod/YoungmannCGR24} generates causal recommendation which suggest male individuals to do a Bachelor's degree to increase salary while the recommendation for women include getting married (borrowed directly from Figure~19 in the paper~\cite{youngmann2024summarizedcausalexplanationsaggregate}). 





%\emph{Causal inference} has been thoroughly studied in AI and Statistics~\cite{pearl2009causal,rubin2005causal}. Causal analysis is a vital tool in determining the effect of a \emph{treatment} on an \emph{outcome}, and has been used in decision-making in medicine \cite{robins2000marginal}, economics \cite{banerjee2011poor}, biology \cite{shipley2016cause}, and in high-stakes areas such as identifying the root causes of failures in critical infrastructure systems to prevent catastrophic outcomes. Recent works have introduced causality to the field of database research~\cite{meliou2010causality,  meliou2014causality,salimi2020causal,10.14778/3554821.3554902}. It has been incorporated into various tasks including data discovery~\cite{galhotra2023metam,youngmann2023causal}, query result explanation~\cite{salimi2018bias,youngmann2022explaining,DBLP:journals/pacmmod/YoungmannCGR24}, and large system diagnostics~\cite{markakis2024sawmill,causalsim,sage, gudmundsdottir2017demonstration}. We propose leveraging causal inference to generate interpretable and justifiable insights (referred to as \emph{prescription rules}), which serve as actionable recommendations to improve an outcome of interest. Causal reasoning is considered one of the most important requirements,  to generate insights that are actionable and aligned with human reasoning.




\begin{table*}[]
\footnotesize
    \centering
    	\caption{\textnormal{A subset of the Stack Overflow dataset.}}
         \label{tab:data}
    	% \vspace{-4mm}
  			\begin{tabular}[b]{|l|l|l|c|l|l|c|l|c|}
  			
				%\multicolumn{9}{c}{\textbf{Users}}\\ 
				\hline

				\textbf{ID}
    
    % \textbf{Country}& \textbf{Continent} 
    
    &\textbf{Gender} &\textbf{Ethnicity}&
				\textbf{Age} &\textbf{Role} &
				\textbf{Education} &\textbf{Country}&\textbf{Undergrad Major}&\textbf{Salary}
				\\ \hline

				1 &Male&White&26&Data Scientist & PhD& US&Computer Science&180k\\
    		2 &Non-binary&White&32&QA developer & Bachelor's degree& US&Mechanical Eng.&83k\\

 3 &Male&South Asian&29&C-suite executive  & Bachelor's degree & India&Computer Science&24k\\

  % 4 &Female&South Asian&25&Back-end developer  & Master's degree & India&Mathematics&7.5k\\

  4 &Female&East Asian&21&Back-end developer & Bachelor's degree & China&Computer Science&19k\\
  

        % $\ldots$ &  $\ldots$&  $\ldots$&  $\ldots$&  $\ldots$&  $\ldots$&  $\ldots$&  $\ldots$&  $\ldots$&  $\ldots$&  $\ldots$\\
    \hline
			\end{tabular}
            \vspace{-5mm}
\end{table*}




\begin{example}	%
\label{example:ex2}
{\bf Importance of fair prescriptions:}
Continuing Example~\ref{example:ex1}, while those causal prescription rules are highly beneficial for the overall population, they are considerably less effective for individuals residing in countries with a low GDP (indicating a weaker economy). For this group, the average expected increase in salary is only approximately \$13,000 per year (in contrast to \$44,009 for the entire group). % \sr{add which rule 44k or 25k} 
Consequently, implementing these rules would exacerbate the disparity between those living in countries with strong economies and those in countries with weaker economies.
\end{example}




% Our objective is to generate a small set of prescription rules aimed at increasing (or decreasing) an outcome of interest. This is framed as an optimization problem where the goal is to select the fewest prescription rules that maximize utility (i.e., the expected increase or decrease in the outcome). However, 

The example above shows that focusing solely on maximizing utility (\revc{i.e., increasing income}) can result in a scenario where only some of the population receive significant improvement, while others experience no benefit (\revc{only a small benefit for individuals from countries with weaker economies in our example}). Additionally, even if a large portion of the population receives recommendations, a protected subpopulation might not share the benefits and, worse, their situation could deteriorate, exacerbating inequalities.

Examples~\ref{example:ex1} and \ref{example:ex2} show that it is crucial to provide recommendations that are (1) {\em causal} for the outcome (beyond associations),  and (2) also {\em fair or equitable} in terms of the outcome for both the protected and non-protected groups. While recent work in database research
has focused on deriving {\em causal explanations} for individual data points, aggregated view, or entire datasets~\cite{salimi2018bias,youngmann2022explaining,ma2023xinsight, DBLP:journals/pacmmod/YoungmannCGR24}, and in particular \cite{DBLP:journals/pacmmod/YoungmannCGR24} has considered generating a set of causal explanations for an aggregated view that resemble a ruleset, 
%Although some of these methods produce causally consistent insights, 
the absence of fairness considerations in generating these causal explanations can lead to unfair outcomes for the protected group.
%further reinforcing existing biases.


%\red{We, therefore, enable users to incorporate various \emph{coverage and fairness constraints} along with the overall objective of improving an outcome of interest. }

\medskip
\noindent
\textbf{Our contributions.~} 
Motivated by the dual goals of generating causal and fair prescriptions for the betterment of an outcome, we introduce a {\em fairness-aware framework leveraging causal reasoning for generating a set of actionable prescription rules (ruleset)} called \sysName\ (\underline{Fair} \underline{CA}usal \underline{P}rescription).
%
Following research on fairness in data management~\cite{stoyanovich2020responsible,galhotra2022causal}, we assume the existence of a \emph{protected subpopulation}, defined by an attribute such as gender or race for people, or GDP of a country. Motivated by the causal explanation rules for an aggregated view \cite{DBLP:journals/pacmmod/YoungmannCGR24}, each prescription rule in our ruleset applies to a sub-population defined by a {\em grouping attribute}, and prescribes a {\em treatment or intervention} to improve the {\em outcome} for this sub-population. Fairness constraints ensure that the expected utility of the protected population is {\em comparable} to the utility of the unprotected individuals. We borrow the notions of \emph{group and individual fairness} from the fairness literature but tailor them for prescription rules. In addition to the fairness constraints, our coverage constraints ensure that a substantial fraction of the population and protected subpopulation receives at least one recommendation. 
%We demonstrate how such constraints ensure that the generated rules apply to a large portion of the population and ensure fairness through the following example.

\begin{example}
\label{ex:intro_example_3}
Continuing Examples~\ref{example:ex1} and \ref{example:ex2}, Alice uses our proposed system, called \sysName, to impose fairness and coverage constraints to discover useful and equitable recommendations for increasing salaries worldwide. In particular,
Alice chooses to implement a coverage constraint to ensure that the selected rules apply to a significant portion of people worldwide, including a sufficiently large number of individuals from countries with low GDP (the protected group). She also imposes a fairness constraint to ensure that the expected gains for both protected and non-protected groups are comparable.
\reva{She discovers, for example, that for individuals with 6-8 years of coding experience (a subpopulation comprising 21\% of the entire dataset and 25\% of the protected group), pursuing a bachelor’s degree in computer science will increase the expected salary by $\$14.9k$ for protected and by $\$17.8k$ for non-protected}. (See \cref{sec:casestudy} for more details.) This prescription rule applies to a large portion of the population and ensures fairness by providing a similar expected gain for both protected and non-protected groups, and the allowed difference of outcomes between these two populations may be adjusted by choosing appropriate thresholds in the fairness definitions. 
\end{example}


\noindent
Our main contributions are as follows. \\
%\begin{itemize}[leftmargin=*,topsep=0pt]
{\bf (1)} We {\bf develop a framework that generates a set of prescription rules to enhance an outcome of interest (Section~\ref{sec:problem})}. A prescription rule consists of a \emph{grouping pattern} and an \emph{intervention pattern}, representing the target subpopulation and the actionable recommendation for that group, respectively. The strength of the {\em conditional causal effect} (Section~\ref{sec:background-causal}) of this intervention on the subgroup is used to measure the expected utility of a rule. Our objective is to identify the smallest set of rules that maximizes overall expected utility. We refer to this problem as the {\em \probName} problem.
We adopt several notions of fairness (individual vs. group, statistical parity vs. bounded group loss) from the literature to define the {\bf fairness constraints} for our problem. In addition, {\bf coverage constraints} (for individual rules or for a group) ensure that the solution for the \probName\ problem is applied to a sufficient number of individuals and to minimize inequalities. We show NP-hardness for different variants of the problems and properties (matroid) useful in our algorithms. 
%We establish several definitions for group and individual fairness constraints tailored for prescription rules.
\smallskip
    \par
    \noindent
{\bf (2)} We {\bf develop a general three-step algorithm named \sysName to solve the optimization problem of selecting a fair prescription ruleset (Section~\ref{sec:algo})}. The first step involves mining frequent grouping patterns using the Apriori algorithm~\cite{agrawal1994fast}. In the second step, we employ a lattice-based algorithm to find high utility and fair intervention patterns for grouping patterns identified in the previous step. Finally, the third step applies a greedy approach to determine a solution. \sysName\ can be easily adapted to accommodate all variants of the \probName\ problem.

\smallskip
\par
\noindent
{\bf (3) We provide a detailed  case study  (Section~\ref{sec:casestudy}) and experimental analysis (Section~\ref{sec:experiments}) to evaluate our framework and algorithms.}
The case study shows the qualitative difference of different variants of our problem for different choices of the fairness and coverage constraints. The experiments include two datasets, three baselines, and 18 variations of our problem with different constraints. Our evaluations suggest that fairness may come at the cost of expected
utility for everyone. However, without fairness constraints, we often observe a significant disparity between the protected and non-protected. We also observe that
achieving individual fairness is harder than group fairness,
as most high-utility or high-coverage rules are unfair. Lastly, we show that \sysName\ can generate  prescription rules over large datasets in a reasonable time. 

%\end{itemize}


%\paragraph*{Paper outline} 
We discuss related work in \cref{sec:related}, review background on causal inference in \Cref{sec:background-causal}, %and our problem formulation can be found in \cref{sec:problem}. Our algorithmic framework is presented in \cref{sec:algo}. A case study demonstrating the impact of different constraint configurations on the solution is given in \cref{exp:problem_variants}, and our experimental evaluation is detailed in \cref{sec:experiments}. Finally, we 
and discuss the limitations of our framework and future work in \cref{sec:conc}.

% \noindent
% \boxed{\parbox{\columnwidth}{$\bullet$ 
% For people with a professional degree, move to the United Kingdom
%  (coverage = 435 (20), coverage-protected = 20 (13), utility = 186855, utility-protected = 0.)\\
% $\bullet$ For graphic developers, move to the	United States
%  (coverage = 116 (29), coverage-protected = 8 (2), utility = 169431, utility-protected = 0).\\
% $\bullet$ For people who have no formal education, move to the United States
%  (coverage = 123 (34), coverage-protected = 7 (2), utility = 206742, utility-protected = 0).\\
% % \textcolor{red}{size = 38, length = 76, overlap = 64029181, utility = 1659307}\\
% \textcolor{blue}{overall coverage =674, expected utility = 187485
% coverage-protected = 35, expected utility-protected = 0}
% \sr{should mention protected group, and possibly not mention coverage in the intro or just intuitively like high coverage}
% }}


% Alice notes that although these rules result in a \$187,485 increase in the overall salary for those to whom they apply, they only affect a small fraction of the population, specifically 674 individuals. Additionally, although the expected salary increase is substantial, there is no expected increase in salary for non-males, a subpopulation of particular interest to Alice. In other words, applying these rules would result in no gain for non-males.
% \end{example}

% \begin{example}[Episode 2 - coverage and fairness constraints]
% Alice introduces coverage and fairness constraints to ensure that enough people will benefit from the rules and that they will be \emph{fair} with respect to non-males. Specifically, she demands that the benefit for a randomly chosen individual to whom one of the rules applies is nearly the same as the benefit for a randomly chosen individual who identifies as non-male and to whom one of the rules applies.

% After adding these constraints, \sysName\ recommends the following set of prescription rules:



% \noindent
% \boxed{\parbox{\columnwidth}{$\bullet$ 
% For people who have no formal education, move to the United States
%  (coverage = 123 (34), coverage-protected = 7 (2), utility = 206742, utility-protected = 0)\\
% $\bullet$ 
% For females, change role to	DevOps specialist (coverage = 2256 (47), coverage-protected = 2256 (47), utility = 90023, utility-protected = 90023).\\
% $\bullet$ For people with a Master's degree, move to the	United States
%  (coverage = 9097 (2222), coverage-protected = 642 (236), utility = 85390, utility-protected = 84201).\\
% % \textcolor{red}{size = 38, length = 76, overlap = 64029181, utility = 1659307}\\
% \textcolor{blue}{overall coverage =11476	
% , expected utility = 87601,
% coverage-protected = 2905, expected utility-protected = 88519}
% }} 







% \begin{figure}[t]
%         \centering
%         \begin{minipage}[b]{1.0\linewidth}
%             \small
%             \begin{tcolorbox}[colback=white]
%             \vspace{-2mm}
% $\bullet$ For backend developers, the treatment with the highest effect on salary is “Country = US” effect size = 78646
% \begin{itemize}
%     \item For non-male the effect is only: 59429
%     \item For male the effect is 80454
% \end{itemize}

% $\bullet$ For frontend developers, the treatment with the highest effect is :Formal Education = Bachelor's degree” effect size: 17340
% \begin{itemize}
%     \item For white the effect is 33464
%     \item For non-white the effect is 15320
% \end{itemize}


% $\bullet$ For people in Europe, the treatment with the highest effect on salary is “DevType = C-suite executive” effect size = 53254
% \begin{itemize}
%     \item For white the effect is 55112
%     \item For non-white 35249
% \end{itemize}



%             \vspace{-2mm}
%             \end{tcolorbox}
%         \end{minipage}%%
%          % \vspace{-4mm}
%         \caption{Set of prescription rules.}
%         \label{fig:so-explanation}
%     \end{figure}

\section{Background and related work}
% 重点看Artistic data visualization: Beyond visual analytics 和Visualization criticism-the missing link between information visualization and art 的被引


This section reviews the background on artistic data visualization and previous research related to this topic.

\subsection{Artistic Data Visualization in Art History Context}
\label{ssec:contemporary}

Art history has been marked by transformative periods characterized by different aesthetic pursuits, materials, and methods. Since the time of Plato, imitation (or \textit{mimesis}, which views art as a mirror to the world around us) has been an important pursuit~\cite{pooke2021art}. Successful artworks, such as Greek sculptures and the convincing illusions of depth and space in Renaissance paintings, exemplify this tradition.
The advent of modern society and new technology, especially photography, posed a significant challenge to the notion of art as imitation~\cite{perry2004themes}. By the 1850s, modern art began to emerge with the core goal of transcending traditional forms and conventions. Movements like Post Impressionism, Expressionism, and Cubism revolutionized art through innovative uses of form (\eg color, texture, composition), moving art towards abstraction and experimentation. 
After World War II, the Cold War and the surge of various social problems heightened skepticism about the progress narrative of modernity and the superiority of the capitalist system, leading to the rise of postmodernism and the birth of contemporary art~\cite{hopkins2000after,harrison1992art}. One prominent feature of contemporary art is the absence of fixed standards or genres historically defined by the church or the academy. Postmodern design neither defines a cohesive set of aesthetic values nor restricts the range of media used~\cite{pooke2021art}, sparking novel practices such as installations, performances, lens-based media, videos, and land-based art~\cite{hopkins2000after}.
Meanwhile, artists have increasingly drawn energy from various philosophical and critical theories such as gender studies, psychoanalysis, Marxism, and post-structuralism~\cite{pooke2021art}. As a result, as described by Foster~\cite{foster1999recodings}, artists have increasingly become ``manipulators of signs and symbols... and the viewer an active reader of messages rather than a passive contemplator of the aesthetic''. Hopkins~\cite{hopkins2000after} described this shift as the ``death of the object'' and ``the move to conceptualism''. 
% Joseph Kosuth, one of the most important representatives of conceptual artists, also once said that “all art (after Duchamp) is conceptual (in nature) because art only exists conceptually”
% As argued by Danto~\cite{danto2015after}, traditional notions of aesthetics can no longer apply to contemporary art. ``“All there is at the end,” Danto wrote, “is theory, art having finally become vaporized in a dazzle of pure thought about itself, and remaining, as it were, solely as the object of its own theoretical consciousness.''
% The Anti-aesthetic (1983) has described these as ‘anti-aesthetic’ strategies – typified, as we have seen, by a conceptually driven approach to the art object and to the process of its production.

Emerging within the contemporary art historical context, data art has been significantly propelled by the advent of big data. An early milestone was Kynaston McShine's 1970 exhibition \textit{Information} at the Museum of Modern Art (MoMA). 
% In the exhibition catalogue, McShine wrote~\cite{information_moma}: ``Increasingly artists use mail, telegrams, telex machines, etc., for transmission of works themselves—photographs, films, documents—or of information about their activity.'' 
% The millennium era has witnessed substantial growth in this area.
In 2008, Google’s Data Arts Team was founded to explore the advancement of what creativity and technology can do together~\cite{google}.
% data artist Aaron Koblin
In 2012, Viégas and Wattenberg's \textit{Wind Map}, an artwork that visualizes real-time air movement, became the first web-based artwork to be included in MoMA's permanent collection~\cite{wind}.
Since 2013, the academic conference IEEE VIS has included an Arts Program (IEEE VISAP), showcasing artistic data visualizations through accepted papers and curated exhibitions. 
As noted by Barabási~\cite{dataism} (a Fellow of the American Physical Society and the head of a data art lab), data has become a vital medium for artists to deal with the complexities of our society: ``Humanity is facing a complexity explosion. We are confronted with too much data for any of us to make sense of...The traditional tools and mediums of art, be they canvas or chisel, are woefully inadequate for this task...today’s and tomorrow’s artists can embrace new tools and mediums that scale to the challenge, ensuring that their practice can continue to reflect our changing epistemology.''
% a physicist and head of a data art lab, has noted, 

% Artists are now exploring new mediums and methods, incorporating datasets, computer technology, and algorithms into their work.



\subsection{Research on Artistic Data Visualization}
\label{ssec:artisticvis}

Artistic data visualization is also referred to as artistic visualization, data art, or information art~\cite{holmquist2003informative,rodgers2011exploring,few,viegas2007artistic}. Despite the varying terminologies, there is a basic consensus that artistic data visualization must be art practice grounded in real data~\cite{viegas2007artistic}.
Since the early 2000s, a series of papers introduced innovative artistic systems for visualizing everyday data, such as museum visit routes and bus schedule information~\cite{skog2003between,holmquist2003informative,viegas2004artifacts}.
In 2007, Viégas and Wattenberg~\cite{viegas2007artistic} explicitly proposed the concept of \textit{artistic data visualization} and viewed it as a promising domain beyond visual analytics.
% and defined it as ``visualization of data done by artists with the intent of making art''. 
Kosara~\cite{kosara2007visualization} drew a spectrum of visualization design, positioning artistic visualization and pragmatic visualization at opposite ends of this spectrum to demonstrate that the goals of these two types of design often diverge. 
% advocating that analytical visualizations prioritize practicality, while artistic data visualizations emphasize sublime quality, that is, the capacity to inspire awe and grandeur and elicit profound emotional or intellectual responses. 
% In 2011, Rodgers and Bartram~\cite{rodgers2011exploring} utilized artistic data visualization to enhance residential energy use feedback. 
However, overall, research on this subject has been sparse. Among those relevant papers, most have focused on specific applications of artistic data visualization. 
%~\cite{rodgers2011exploring,schroeder2015visualization,perovich2020chemicals}
For instance, Rodgers and Bartram~\cite{rodgers2011exploring} utilized ambient artistic data visualization to enhance residential energy use feedback. Samsel~\etal~\cite{samsel2018art} transferred artistic styles from paintings into scientific visualization.
Artistic practice has also been observed in fields such as data physicalization~\cite{hornecker2023design,perovich2020chemicals,offenhuber2019data} and sonification~\cite{enge2024open}. For example, Hornecker~\etal~\cite{hornecker2023design} found that many artists are practicing transforming data into tangible artifacts or installations. Enge~\etal~\cite{enge2024open} discussed a set of representative artistic cases that combine sonification and visualization.
% dragicevic2020data
% Offenhuber~\cite{offenhuber2019data} created a set of artworks in urban settings that transform air quality data into situated displays, allowing people to encounter visualizations in their daily lives.

% But in contrast, empirical studies that describe the characteristics of artistic visualization and how they are created are extremely scarce. This scarcity forms a stark contrast to the increasingly rich and diverse practices by artists in the field.
% As for the difference between artistic data visualization and traditional visualizations for analytics, Vi{\'e}gas and Wattenberg~\cite{viegas2007artistic} thought that the most salient feature of artistic data visualizations is their forceful expression of viewpoints.
% In Ramirez~\cite{ramirez2008information}'s opinion, functional information visualizations are concerned with usability and performance while aesthetic information visualizations are concerned with visually attractive forms of representation design.
% Donath~\etal~\cite{donath2010data} presented a series of tools developed to integrate artistic expressions in generating unique and customized visualizations based on users' personal data, such as health monitoring data, album records, and e-mail records. 

On the other hand, some studies, while not focusing on artistic data visualization, have explored a key art-related concept: aesthetics. 
% ~\cite{moere2012evaluating,cawthon2007effect,lau2007towards} explored the aesthetics of visualization design in their research. They
For example, Moere~\etal~\cite{moere2012evaluating} compared analytical, magazine, and artistic visualization styles, noting that analytical styles enhance the discovery of analytical insights, while the other two induce more meaning-based insights. Cawthon~\etal~\cite{cawthon2007effect} asked participants to rank eleven visualization types on an aesthetic scale from ``ugly'' to ``beautiful'', finding that some visualizations (\eg sunburst) were perceived as more beautiful than others (\eg beam trees).
% Moere~\etal~\cite{moere2012evaluating} displayed data in three different styles (analytical style, magazine style, artistic style) and found that these styles led to different perceptions of usability and types of insights.
% More importantly, the authors found that the sunburst chart ranks the highest in aesthetics and is one of the top-performing visualizations in both efficiency and effectiveness, thus exemplifying the notion that "beautiful is indeed usable".
Factors such as embellishment~\cite{bateman2010useful}, colorfulness~\cite{harrison2015infographic}, and interaction~\cite{stoll2024investigating} have also been found to influence aesthetics. 
% borkin2013makes,tanahashi2012design
However, most of these studies have simplified aesthetics to hedonic features (\eg beauty), without delving into the nuanced connotations of aesthetics.
% most of these studies have simplified aesthetics to concepts like 'beauty,' 'preference,' or 'pleasing,' without exploring the deeper essence of aesthetics as the core of art.

The value of artistic data visualization to the visualization community is still in controversy. For instance, Few~\cite{few} argued for a clearer distinction between data art and data visualization; he highlighted the negative consequences when data art ``masquerades as data visualization'', such as making visualizations difficult to perceive. Willers~\cite{willers2014show} thought the artistic approach is ``unlikely be appreciated if the intention was for rapid decision making.''
% In an interview, American artist and technologist Harris commented that ``data can be made pretty by design, but this is a superficial prettiness, like a boring woman wearing too much makeup.''~\cite{harris2015beauty} 
To address these gaps, more empirical research needs to be conducted to explore how artistic data visualizations are designed, their underlying pursuits, and how they might inspire our community.




% Examining this field not only helps us understand the latest application of data visualization in various domains but also extends our understanding of the aesthetic and humanistic aspects of data visualization.
% there should be more theoretical investigation into artistic data visualization. 

% These three concepts emphasize placing or installing visualizations at physical places that people will encounter in their daily lives. 

% ~\cite{wang2019emotional}


% gap between art and science~\cite{judelman2004aesthetics}
% constructive visualization~\cite{huron2014constructive}
% data feminism~\cite{d2020data}
% critical infovis~\cite{dork2013critical}
% citizen data and participation~\cite{valkanova2015public}

% \x{Lee~\etal~\cite{lee2013sketchstory}, give users artistic freedom to create their own visualizations.}


% Aesthetics refers to the study of beauty, taste, and sensory perception and is deeply intertwined with art.
% a particular taste for or approach to what is pleasing to the senses and especially sight

% why shouldn't all charts be scatter plot~\cite{bertini2020shouldn}
% aesthetic model~\cite{lau2007towards}
% Aesthetics for Communicative Visualization : a Brief Review
% Stacked graphs--geometry \& aesthetics~\cite{byron2008stacked}
% storyline optimization~\cite{tanahashi2012design}
% graphic designers rate the attractiveness of non-standard and pictorial visualizations higher than standard and abstract ones, whereas the opposite is true for laypeople.~\cite{quispel2014would}
% evaluate aesthetics - wordcloud
% An Evaluation of Semantically Grouped Word Cloud Designs, tag cloud, wordle

% On the other hand, empirical studies conducted with designers have shown that functionality is not the only design goal of visualization. For example, Bigelow~\etal~\cite{bigelow2014reflections} found that designers would frequently fine-tune the non-data elements in their designs, and data encoding was even "a later consideration with respect to other visual elements of the infographic".
% Moere~\cite{moere2011role} - design
% Quispel~\etal~\cite{quispel2018aesthetics} found that for information designers, clarity and aesthetics are both important for making a design attractive.
% \input{03-preliminary}
\section{Methodology}
\label{sec:algo}
\begin{figure*}[t]
    \centering
    \includegraphics[width=1.35\columnwidth]{plots/replacement.pdf}
    \caption{An example illustrating our replacement strategy. With chunk size $L=16$ and compression rate $r=16$, we encode 32 textual CoT tokens into 2 discrete latent tokens from left to right. The other CoT tokens will remain in their original forms. 
    }
    \label{fig:replacement}
\end{figure*}
In this section, we describe our methodology to enable LLMs to reason with discrete latent tokens. The notations are summarized in \Cref{app:notations}.
Let $X = P \oplus C \oplus S$ denote a sample input,
where $P = (p_1, p_2, \ldots, p_{t_p})$ are the prompt tokens, $C = (c_1, c_2, \ldots, c_{t_c})$ are the reasoning step (chain-of-thought) tokens,
$S = (s_1, s_2, \ldots, s_{t_s})$ are the solution tokens, and $\oplus$ denotes concatenation. Our training procedure consists of two stages:
\begin{enumerate}[leftmargin=*]\itemsep0em
    \item \textbf{Learning latent discrete tokens to abstract the reasoning steps}, where we train a model to convert $C$ into a sequence of latent tokens $Z = (z_1, z_2, \ldots, z_{t_z})$ such that $t_z < t_c$. The compression rate $r = t_c / t_z$ controls the level of abstraction.

    \item \textbf{Training the LLM with a partial and high-level abstract of the reasoning steps}, where we 
    construct a modified input $\Xtilde$ by
    replacing the first $m$ tokens of $C$ by the corresponding latent abstractions:
    \begin{equation}
        \Xtilde = P \oplus [z_1, \ldots, z_{\frac{m}{r}}, c_{m+1}, \ldots, c_{t_c}] \oplus S.
        \label{eq:X_replacement}
    \end{equation}
    \Cref{fig:replacement} illustrates this replacement strategy. We randomize the value of $m$ during training.
\end{enumerate}



\subsection{Learning Latent Abstractions}
We employ a vector-quantized variable autoencoder (VQ-VAE)~\cite{van2017neural} type of architecture to map CoT tokens \(C\) into discrete latent tokens \(Z\). % VQ-VAE is a powerful model capable of capturing high-dimensional semantic structures from text data.
To enhance abstraction performance, our VQ-VAE is trained on the whole input sequence $X$, but only applied to $C$ in the next stage. Following~\citet{jiang2022efficient, jiang2023h}, we split $X$ into chunks of length \(L\) and encode each chunk into $\frac{L}{r}$ latent codes, where $r$ is a preset compression rate. More precisely, our architecture consists of the following five components:
\vspace{-5pt}
\begin{itemize}\itemsep0pt
    \item $\E:$ a codebook containing $|\E|$ vectors in $\R^d$.
    \item $\fenc: \V^L \mapsto \R^{d \times \frac{L}{r}} $ that encodes a sequence of $L$ text tokens to $\frac{L}{r}$ latent embedding vectors $\bar{X} = \bar{x}_1, \ldots, \bar{x}_{\frac{L}{r}}$,  where $\V$ is the vocabulary of text tokens.
   %  and $\Z$ are the vocabularies of text and latent tokens, respectively. 
   \item $q: \R^{d} \mapsto \E$: the quantization operator that replaces the encoded embedding $\bar{x}$ by the nearest neighbor in $\E$: $q(\bar{x}) = \argmin_{e_i \in \E} \norm{e_i - \bar{x}}^2_2$.
    \item $g: \V^K \mapsto \R^d$ that maps $K$ text tokens to a $d$-dimensional embedding vector. We use $g$ to generate a continuous embedding of the prompt $P$.
    \item $\fdec: \R^{d \times \frac{L}{r}} \times \R^k \mapsto \V^L$ that decodes latent embeddings back to text tokens, conditioned on prompt embedding.
\end{itemize}
In particular, each continuous vector $e \in \E$ in the codebook has an associated latent token $z$, which we use to construct the latent reasoning steps $Z$\footnote{To decode a latent token $z$, we look up the corresponding embedding $e \in \E$ and feed it to $\fdec$.}.


\begin{figure}[t]
    \centering
    \includegraphics[width=\columnwidth]{plots/vqvae_latentformer.pdf}
    \caption{A graphical illustration of our VQ-VAE. $\fenc$ encodes the text tokens into latent embeddings, which are quantized by checking the nearest neighbors in the codebook. $\fdec$ decodes those quantized embeddings back to text tokens. When applying the VQ-VAE to compress the text tokens, the discrete latent tokens $Z$ are essentially the index of corresponding embeddings in the codebook.} 
    \label{fig:vqvae}
\end{figure}


For simplicity, we assume the lengths of the input $X$ and the prompt $P$ are $L$ and $K$ exactly.
Similar to \citet{van2017neural}, we use an objective $\mathcal{L}$ composed of 3 terms: 
% . Given an input sequence \(X_t = (x_1, x_2, \ldots, x_T)\) and the discrete latent tokens \(Z_t = (z_1, z_2, \ldots, z_{\tfrac{M T}{L}})\), the total loss  is:
\begin{equation}
\begin{aligned}
& \mathcal{L}(X) = \underbrace{ \log p(X | \fdec( q(\bar{X}) | g(P) ))}_{\text{reconstruction loss}} + \\
&  \hskip5pt \sum_{i=1}^L \underbrace{ \| \sg[\bar{X}_i] - q(\bar{X}_i) \|_2^2}_{\text{VQ loss}} +  \underbrace{\beta \| \bar{X}_i - \sg[q(\bar{X}_i)] \|_2^2}_{\text{commitment loss}},
\end{aligned}
\end{equation}
where $\bar{X} = \fenc(X)$, $\sg[\cdot]$ is the stop-gradient operator, and \(\beta\) is a hyperparameter controlling the strength of the commitment loss.
The VQ loss and the commitment loss ensure that the encoder outputs remain close to the codebook, while the reconstruction loss concerns with the decoding efficacy. As standard for VQ-VAE, we pass the gradient $\nabla_{\fdec}(L)$ unaltered to $\fenc$ directly as the quantization operator $q(\cdot)$ is non-differentiable. \Cref{fig:vqvae} illustrates our architecture. In practice, we use a causal Transformer for both $\fenc$ and $\fdec$, the model details are discussed in Appendix~\ref{app:model}.


Thus far we obtain a latent representation both semantically meaningful and conducive to reconstruction, setting the stage for the subsequent training phase where
the LLM is trained to perform reasoning with abstractions.


\subsection{Reasoning with Discrete Latent Tokens}


In this second stage, we apply the obtained VQ-VAE to form modifed samples $\Xtilde$ with latent abstractions as in \Cref{eq:X_replacement}, then train an LLM to perform next token prediction. 
Below, we outline the major design choices that are key to our model's performance, and ablate them in \Cref{sec:expr}.

\textbf{Partial Replacement}. Unlike previous planning works~\cite{jiang2022efficient, jiang2023h} that project the whole input sequence onto a compact latent space, we only replace $m < t_c$ CoT tokens with their latent abstractions, leaving the remaining tokens unchanged.  We delimit the latent tokens by injecting a special \texttt{<boLatent>} and \texttt{<eoLatent>} tokens to encapsulate them.
% The constructed $\Xtilde$ becomes a fixed mixture of early latent tokens and later text tokens.

\textbf{Left-to-Right (AR) Replacement}. We replace the leftmost $m$ tokens of $C$, rather than subsampling tokens at different locations. 

\textbf{Mixing Samples with Varying Values of $m$}. For fine-tuning an existing LLM on the reasoning dataset with latent tokens, one remarkable challenge is to deal with the extended vocabulary. As the LLM is pretrained with trillions of tokens,
it is very hard for it to quickly adapt to tokens (and corresponding embeddings) beyond the original vocabulary. Previous works that aim to replace or eliminate CoT tokens~\cite{deng2024explicit, hao2024training} employ a multistage curriculum training approach, where those operations are gradually applied to the entire input sequence. In the context of our approach, this means we increase the values of $m$ in each stage until it reaches a pre-set cap value. However, such training procedure is complex and computationally inefficient, where dedicated optimization tuning is needed. In this work, we employ a simple single stage training approach where the value of $m$ is randomly set for each sample. Surprisingly, this not only makes our training more efficient, but also leads to enhanced performance. 
\section{Experiments}
\label{sec:expr}
We empirically evaluate our approach on two categories of benchmarks: 
\vspace{-5pt}
\begin{enumerate}\itemsep0pt
    \item[\textbf{(1)}] Synthetic datasets including the Keys-Finding Maze, ProntoQA~\cite{saparov2022language}, and ProsQA~\cite{hao2024training}, where we pretrain T5 or GPT-2 models from scratch using the method in \Cref{sec:algo};
    \item[\textbf{(2)}] Real-world mathematic reasoning problems, where we fine-tune Llama models~\cite{dubey2024llama} on the MetaMathQA~\cite{yu2023metamath} or the Dart-MATH~\cite{tong2024dart} dataset, and then test on in-domain datasets Math and GSM-8K, along with out-of-domain datasets including Fresh-Gaokao-Math-2023, DeepMind-Math, College-Math, OlympiaBench-Math, and TheoremQA. 
\end{enumerate}
The detailed setup is introduced in \Cref{sec:expr_benchmark}.

We compare our approach to the following baselines:
\vspace*{-10pt}
\begin{enumerate}[leftmargin=0pt]\itemsep0pt
    \item[] \textbf{Sol-Only}:  the model is trained with samples that only contains questions and solutions, without any reasoning steps;
    \item[] \textbf{CoT}: the model is trained with samples with complete CoT tokens; \looseness=-1
    \item[] \textbf{iCoT}~\citep{deng2024explicit}: a method that utilizes curriculum learning to gradually eliminate the need of CoT tokens in reasoning;
    \item[] \textbf{Pause Token}~\citep{goyal2023think}:  a method that injects a learnable \texttt{pause} token into the sample during training, in order to offer extra computation before giving out the final answer.

\end{enumerate}



\subsection{Benchmarks}
\label{sec:expr_benchmark}
\subsubsection{Synthetic Benchmarks}

\textbf{Keys-Finding Maze} is a complex navigation environment designed to evaluate an agent's planning capabilities. The agent is randomly positioned within a maze comprising 4 $3 \times 3$ interconnected rooms, with the objective of reaching a randomly placed goal destination. To successfully reach the destination, the agent must collect keys (designated with green, red, and blue colors) that correspond to matching colored doors. These keys are randomly distributed among the rooms, requiring the agent to develop sophisticated planning strategies for key acquisition and door traversal. The agent is only allowed to take one key at a time. This environment poses a substantial cognitive challenge, as the agent must identify which keys are necessary for reaching the destination, and optimize the order of key collection and door unlocking to establish the most efficient path to the goal. Following \citet{lehnert2024beyond,su2024dualformer}, we generate intermediate search traces using the nondeterministic A* algorithm~\cite{hart1968formal}. The dataset contains 100k training samples. See \Cref{app:maze} for more information and graphical illustrations.

\textbf{ProntoQA}~\cite{saparov2022language} is a dataset consists of $9000$ logical reasoning problems derived from ontologies - formal representations of relationships between concepts. Each problem in the dataset is constructed to have exactly one correct proof or reasoning path. One distinctive feature of this dataset is its consistent grammatical and logical structure, which enables researchers to systematically analyze and evaluate how LLMs approach reasoning tasks. 

\textbf{ProsQA}~\cite{hao2024training} is a more difficult benchmark building on top of ProntoQA. It contains 17,886 logical problems curated by randomly generated directed acyclic graphs. %The advantage of this dataset is that it
It has larger size of distracting reasoning paths in the ontology, and thus require more complex reasoning and planning capabilities.
% to solve it.

\subsubsection{Mathematical Reasoning}
We fine-tune pretrained LLMs using the MetaMathQA~\cite{yu2023metamath} or the Dart-MATH~\cite{tong2024dart} dataset. 
MetaMathQA is a curated dataset that augments the existing \texttt{Math} ~\cite{math_dd} and \texttt{GSM8K} ~\cite{gsm8k_dd} datasets by various ways of question bootstrapping,
such as (i) rephrasing the question and generating the reasoning path; (ii) generating backward questions,  self-verification questions, FOBAR questions~\cite{jiang2024forward}, etc. This dataset contains 395k samples in total, where 155k samples are bootstrapped from \texttt{Math} and the remaining 240k come from \texttt{GSM8K}. We rerun the MetaMath data pipeline by using Llama-3.1-405B-Inst to generate the response. 
Dart-MATH~\cite{tong2024dart} also synthesizes responses for questions in \texttt{Math} and \texttt{GSM8K}, with the focus on difficult questions via difficulty-aware rejection tuning.
For evaluation, we test the models on the original \texttt{Math} and \texttt{GSM8K} datasets, which are in-domain,
and also the following out-of-domain benchmarks:
\vspace{-5pt}
\begin{itemize}[leftmargin=*]\itemsep0pt
    \item  \textbf{College-Math}~\cite{tang2024mathscale}
consists of 2818 college-level math problems taken from 9 textbooks. These problems cover over 7 different areas such as linear algebra, differential equations, and so on. They are designed to evaluate how well the language model can handle complicated mathematical reasoning problems in different field of study.

    \item  \textbf{DeepMind-Math}~\cite{saxton2019analysing} consists of 1000 problems based on the national school math curriculum for students up to 16 years old. It examines the basic mathematics and reasoning skills across different topics.

    \item  \textbf{OlympiaBench-Math}~\cite{he2024olympiadbench} 
is a text-only English subset of Olympiad-Bench focusing on advanced level mathematical reasoning. It
contains 675 highly difficult math problems from competitions. 

    \item  \textbf{TheoremQA}~\cite{chen2023theoremqa} contains 800 problems focuses on solving problems in STEM fields (such as math, physics, and engineering) using mathematical theorems.


    \item \textbf{Fresh-Gaokao-Math-2023} ~\cite{tang2024mathscale} contains 30 math questions coming from  Gaokao, or the National College Entrance Examination, which is a national standardized test that plays a crucial role in the college admissions process.
\end{itemize}

\subsection{Main Results}
\label{sec:expr_main}
We employ a consistent strategy for training VQ-VAE and replacing CoT tokens with latent discrete codes across all our experiments, as outlined below.
The specific model architecture and key hyperparameters used for LLM training are presented alongside the results for each category of benchmarks.
All the other details are deferred to \Cref{app:model}. \looseness=-1

\paragraph{VQ-VAE Training} For each benchmark, we train a VQ-VAE for 100k steps using the Adam optimizer, with learning rate $10^{-5}$ and batch size 32.
We use a codebook of size $1024$ and compress every chunk of $L=16$ tokens into a single latent token (i.e., the compression rate $r=16$).


\paragraph{Randomized Latent Code Replacement} We introduce a stochastic procedure for partially replacing CoT tokens with latent codes. 
Specifically, we define a set of predetermined numbers \( \mathcal{M} = \{0, 72, 128, 160, 192, 224, 256\}\), which are all multipliers of $L=16$.
For each training example, we first sample $m_{\max} \in \mathcal{M}$ then sample an integer $m \in [0, 16, 32, \ldots, m_{\max}]$ uniformly at random.
The first $m$ CoT tokens are replaced by their corresponding latent discrete codes, while the later ones remain as raw text. 
This stochastic replacement mechanism exposes the model to a wide range of latent-text mixtures, enabling it to effectively learn from varying degrees of latent abstraction.


\begin{table*}[t]
\centering
\resizebox{0.7\textwidth}{!}{
\begin{tabular}{lcccccc}
\toprule
\multirow{2}{*}{\bf{Model}} & \multicolumn{2}{c}{\bf Keys-Finding Maze} & \multicolumn{2}{c}{\bf ProntoQA} & \multicolumn{2}{c}{\bf ProsQA} \\ 
\cmidrule(lr){2-3} \cmidrule(lr){4-5} \cmidrule(lr){6-7}
 & 1-Feasible-10 (\%) & Num. Tokens &  Accuracy & Num. Tokens & Accuracy & Num. Tokens \\ 
\midrule
Sol-Only & 3 & 645 & 93.8 & 3.0 & 76.7 & 8.2 \\
CoT & \underline{43}& 1312.0 & \underline{98.8} & 92.5 & \underline{77.5} & 49.4 \\

\bf{Latent (ours)}  & \bf{62.8 \increase{19.8}} & 374.6 & \bf{100 \increase{1.2}} & 7.7 & \textbf{96.2 \increase{18.7}} & 10.9 \\
\bottomrule
\end{tabular}
}
\caption{Our latent approach surpasses the other baselines on Keys-Finding Maze, ProntoQA and ProsQA with a large margin
. We use top-$k$ ($k=10$) decoding for Keys-Finding Maze and greedy decoding for ProntoQA and ProsQA. In terms of token efficiency, 
our latent approach also generates much shorter reasoning traces than the CoT baseline, closely tracking or even outperforming the Sol-Only approach.
\textbf{Bold: best results}. \underline{Underline: second best results}. \increase{Performance gain compared with the second best result.}
}
\label{table:synthetic}
\end{table*} 


\begin{table*}[t]
\begin{adjustbox}{width=\textwidth}
\begin{tabular}{lllllllllll}
\toprule
\multicolumn{2}{c}{\multirow{2}{*}{\bf Model}} & \multicolumn{2}{c}{\bf In-Domain} & \multicolumn{5}{c}{\bf Out-of-Domain} & \multicolumn{1}{c}{\bf Average} \\ \cmidrule(lr){3-4} \cmidrule(lr){5-9} \cmidrule(lr){10-10}
& & Math & GSM8K & Gaokao-Math-2023 & DM-Math & College-Math & Olympia-Math & TheoremQA & All Datasets \\ \midrule
\multirow{6}{*}{\bf Llama-3.2-1B}
& Sol-Only  & 4.7 & 6.8 & 0.0 & 10.4 & 5.3 & 1.3 & 3.9 & 4.6 \\
& CoT  & \underline{10.5} & \underline{42.7} & \bf{10.0} & 3.4 & \underline{17.1} & 1.5 & 9.8 & \underline{14.1} \\
& iCoT  & 8.2 & 10.5 & 3.3 & \underline{11.3} & 7.6 & \textbf{2.1} & \underline{10.7} & 7.7 \\
& Pause Token & 5.1 & 5.3 & 2.0  & 1.4 &  0.5  & 0.0 &  0.6 & 2.1\\

& \textbf{Latent (ours)} & \textbf{14.7 \increase{4.2}} & \textbf{48.7 \increase{6}} & \textbf{10.0}  & \textbf{14.6 \increase{3.3}}  & \textbf{20.5 \increase{3.4}} & \underline{1.8}  & \textbf{11.3 \increase{0.6}}  & \textbf{17.8 \increase{3.7}}  \\
\midrule
\multirow{6}{*}{\bf Llama-3.2-3B}
& Sol-Only  & 6.1 & 8.1 & 3.3 & 14.0 & 7.0 & 1.8 & 6.8 & 6.7\\
& CoT  & \underline{21.9} & \underline{69.7} & \underline{16.7} & \textbf{27.3} & \underline{30.9} & 2.2 & 11.6 & \underline{25.2} \\
& iCoT  & 12.6 & 17.3 & 3.3 & 16.0 & 14.2 & \textbf{4.9} & \textbf{13.9} & 11.7 \\
& Pause Token & 25.2 & 53.7 & 4.1 & 7.4 &  11.8 & 0.7 &  1.0 & 14.8\\

& \textbf{Latent (ours)} & \textbf{26.1 \increase{4.2}}  & \textbf{73.8 \increase{4.1}}  & \textbf{23.3 \increase{6.6}}  & \underline{27.1}  & \textbf{32.9 \increase{2}}  & \underline{4.2}  & \underline{13.5}   & \textbf{28.1 \increase{2.9}} \\
\midrule
\multirow{6}{*}{\bf Llama-3.1-8B}
& Sol-Only  & 11.5 & 11.8 & 3.3 & 17.4 & 13.0 & 3.8 & 6.7 & 9.6 \\
& CoT  & {32.9} & \underline{80.1} & \underline{16.7} & \underline{39.3} & \underline{41.9} & 7.3 & \underline{15.8 } & \underline{33.4} \\
& iCoT  & 17.8 & 29.6 & 16.7 & 20.3 & 21.3 & \underline{7.6} & 14.8 & 18.3 \\
& Pause Token & \textbf{39.6} & 79.5 & 6.1  & 25.4 &   25.1 & 1.3 &  4.0 & 25.9\\


& \textbf{Latent (ours)} & \underline{37.2}  & \textbf{84.1 \increase{4.0}}  & \textbf{30.0 \increase{13.3}}  & \textbf{41.3 \increase{2}}  & \textbf{44.0 \increase{2.1}}  & \textbf{10.2 \increase{2.6}}  & \textbf{18.4 \increase{2.6}}  & \textbf{37.9 \increase{4.5}}  \\
 

\bottomrule
\end{tabular}
\end{adjustbox}
\caption{
Our latent approach outperforms the baselines on various types of mathematical reasoning benchmarks. The models are fine-tuned on the MetaMathQA~\cite{yu2023metamath} dataset. The Math and GSM8K are in-domain datasets since they are used to generate MetaMathQA, while the others are out-of-domain. \textbf{Bold: best results}. \underline{Underscore: second best results}. \textcolor{darkgreen}{$\uparrow$ +: \hspace{0.2em}Performance gain compared with the second best result.}
}
\label{table:LLMtable}
\end{table*}


\begin{table*}[t]
\begin{adjustbox}{width=\textwidth}
\begin{tabular}{llccccccccc}
\toprule
\multicolumn{2}{c}{\multirow{2}{*}{\bf Model}} & \multicolumn{2}{c}{\bf In-Domain (\# of tokens)} & \multicolumn{5}{c}{\bf Out-of-Domain (\# of tokens)} & \multicolumn{1}{c}{\bf Average} \\ \cmidrule(lr){3-4} \cmidrule(lr){5-9} \cmidrule(lr){10-10}
& & Math & GSM8K & Gaokao-Math-2023 & DM-Math & College-Math & Olympia-Math & TheoremQA & All Datasets \\ \midrule
\multirow{6}{*}{\bf Llama-3.2-1B}
& Sol-Only  & 4.7 & 6.8 & 0.0 & 10.4 & 5.3 & 1.3 & 3.9 & 4.6 \\
& CoT & 646.1 & 190.3 & 842.3 & 578.7 & 505.6 & 1087.0 & 736.5 & 655.2 \\
& iCoT & 328.4 & 39.8 & 354.0 & 170.8 & 278.7 & 839.4 & 575.4 & 369.5 \\
& Pause Token & 638.8 & 176.4 & 416.1 & 579.9 & 193.8 & 471.9 & 988.1 & 495\\
% & Dualformer &  \\
& \textbf{Latent (ours)} & 501.6 \decrease{22\%} & 181.3 \decrease{5\%} & 760.5 \decrease{11\%} & 380.1 \decrease{34\%} & 387.3 \decrease{23\%} & 840.0 \decrease{22\%} & 575.5 \decrease{22\%} & 518 \decrease{21\%} \\
\midrule
\multirow{6}{*}{\bf Llama-3.2-3B}
& Sol-Only  & 6.1 & 8.1 & 3.3 & 14.0 & 7.0 & 1.8 & 6.8 & 6.7\\
& CoT & 649.9  & 212.1  & 823.3 & 392.8 & 495.9 & 1166.7 & 759.6 & 642.9 \\
& iCoT & 344.4 & 60.7 & 564.0 & 154.3 & 224.9 & 697.6 & 363.6 & 344.2 \\
& Pause Token & 307.9 & 162.3 & 108.9 & 251.5 & 500.96 & 959.5 & 212.8 & 354.7 \\
% & Dualformer &  \\
& \textbf{Latent (ours)} & 516.7 \decrease{20\%} & 198.8 \decrease{6\%} & 618.5 \decrease{25\%} & 340.0 \decrease{13\%} & 418.0 \decrease{16\%} & 832.8 \decrease{29\%} & 670.2 \decrease{12\%} & 513.6 \decrease{20\%}\\
\midrule
\multirow{6}{*}{\bf Llama-3.1-8B}
& Sol-Only  & 11.5 & 11.8 & 3.3 & 17.4 & 13.0 & 3.8 & 6.7 & 9.6 \\
& CoT & 624.3 & 209.5 & 555.9 & 321.8 & 474.3 & 1103.3 & 760.1 & 578.5 \\
& iCoT & 403.5 & 67.3 & 444.8 & 137.0 & 257.1 & 797.1 & 430.9 & 362.5 \\
& Pause Token & 469.4 & 119.0 & 752.6 & 413.4 & 357.3 & 648.2 &600.1 &  480\\

& \textbf{Latent (ours)} & 571.9 \decrease{9 \%} & 193.9 \decrease{8 \%} & 545.8 \decrease{2 \%} & 292.1 \decrease{10\%} & 440.3 \decrease{8\%} & 913.7 \decrease{17 \%} & 637.2 \decrease{16 \%} & 513.7 \decrease{10\%}\\


\bottomrule
\end{tabular}
\end{adjustbox}
\caption{The average number of tokens in the generated responses. Compared with the CoT baseline, our latent approach achieves an $17\%$ reduction in response length on average, while surpassing it in final performance according to~\Cref{table:LLMtable}. The iCoT method generates shorter responses than our approach, yet performs significantly worse, see~\Cref{table:LLMtable}. \textcolor{darkgreen}{$\downarrow$ -:\hspace{0.2em}Trace length reduction rate compared with CoT.} }
\label{table:LLM-token}
\end{table*}


\subsubsection{Synthetic Benchmarks}

\paragraph{Hyperparameters and Evaluation Metric}  

For our experiments on the ProntoQA and ProsQA datasets, we fine-tune the pretrained GPT-2 model~\cite{radford2019language} for $16$k steps, where we use a learning rate of $10^{-4}$ with linear warmup for 100 steps, and the batch size is set to 128. 
To evaluate the models, we use greedy decoding and check the exact match with the ground truth.

For Keys-Finding Maze, due to its specific vocabulary, we trained a T5 model~\cite{2020t5} from scratch for 100k steps with a learning rate of $7.5 \times 10^{-4}$ and a batch size of 1024. We evaluate the models by the \emph{1-Feasible-10} metric. Namely, for each evaluation task, we randomly sample 10 responses with top-$k$ ($k$=10) decoding and check if any of them is feasible and reaches the goal location. 

\paragraph{Results}
As shown in \Cref{table:synthetic}, our latent approach performs better than the baselines
for both the Keys-Finding Maze and ProntoQA tasks.
Notably, the absolute improvement is 15\% for the Keys-Finding Maze problem, 
and we reach 100\% accuracy on the relatively easy ProntoQA dataset.
For the more difficult ProsQA, the CoT baseline only obtains 77.5\% accuracy,
the latent approach achieves $17.5\%$ performance gain.



\begin{table*}[t]
\begin{adjustbox}{width=\textwidth}
\begin{tabular}{lllllllllll}
\toprule
\multicolumn{2}{c}{\multirow{2}{*}{\bf Model}} & \multicolumn{2}{c}{\bf In-Domain} & \multicolumn{5}{c}{\bf Out-of-Domain} & \multicolumn{1}{c}{\bf Average} \\ \cmidrule(lr){3-4} \cmidrule(lr){5-9} \cmidrule(lr){10-10}
& & math & GSM8K & Fresh-Gaokao-Math-2023 & DeepMind-Mathematics & College-Math & Olympia-Math & TheoremQA & All Datasets \\ \midrule
\multirow{3}{*}{\bf Llama-3.2-1B}

& {All-Replace} & 6.7 & 4.2 & 0.0 & 11.8 & 6.0 & {2.1} & 8.5 & 5.6 \\
& {Curriculum-Replace} & {7.1} & \underline{9.8} & \underline{3.3} & \underline{13.0} & 
{7.9} & \bf{2.4} & \underline{10.5} & {7.8} \\
& Poisson-Replace & \underline{13.9 } & \textbf{49.5} & {10.0}  & {12.2}  & \underline{18.9 } & \underline{2.3}  & {9.0 }  & \underline{15.1 }   \\
& \textbf{Latent-AR (ours)} & \textbf{14.7 } & \underline{48.7 } & \textbf{10.0}  & \textbf{14.6 }  & \textbf{20.5} & 1.8  & \textbf{11.3 }  & \textbf{17.8 }   \\


\midrule
\multirow{3}{*}{\bf Llama-3.2-3B}

& {All-Replace} & {10.7} & 12.8 & {10.0} & \underline{19.4} & 12.8 & \bf{5.3} & 11.8 & {11.8} \\
& {Curriculum-Replace} & 10.2 & {14.9} & 3.3 & 16.8 & {12.9} & 3.9 & \bf{14.4} & 10.9 \\
& Poisson-Replace & \underline{23.6 } & \underline{65.9 } & \underline{13.3}  & {17.9 }  & \underline{28.9 } & 2.9  & {11.2 }  & \underline{20.5 }   \\
& \textbf{Latent (ours)} & \textbf{26.1 }  & \textbf{73.8 }  & \textbf{23.3 }  & \textbf{27.1 }  & \textbf{32.9 }  & \underline{4.2}  & \underline{13.5}   & \textbf{28.1 } \\



\midrule
\multirow{3}{*}{\bf Llama-3.1-8B}

& {All-Replace} & {15.7} & 19.9 & 6.7 & {21.1} & {19.5} & {5.0} & {17.5} & 15.0 \\
& {Curriculum-Replace} & 14.6 & {23.1} & {13.3} & 20.3 & 18.7 & 3.9 & 16.6 & {15.8} \\
& Possion-Replace & \textbf{37.9 } & \underline{83.6 }  & \underline{16.6 }  & \textbf{42.7 }  & \textbf{44.7 }  & \underline{9.9 }  & \textbf{19.1 }  & \underline{36.3 }  \\
& \textbf{Latent (ours)} & \underline{37.2 } & \textbf{84.1 }  & \textbf{30.0 }  & \underline{41.3 }  & \underline{44.0 }  & \textbf{10.2 }  & \underline{18.4 }  & \textbf{37.9 }  \\




\bottomrule
\end{tabular}
\end{adjustbox}
\caption{Our latent token replacement strategy significantly outperforms the alternative choices: All-Replace (where all the textual CoT tokens are replaced by latent tokens at once), Curriculum-Replace (where we gradually replace the text tokens for the entire CoT subsequence by latent tokens over the course of training) and Poisson-Replace (where individual chunks of text tokens are replaced with probabilities 0.5).}
\label{table:ablation_replacement}
\end{table*}




\subsubsection{Mathematical Reasoning}

\paragraph{Hyperparameters and Evaluation Metrics}
We considered 3 different sizes of LLMs from the LLaMa herd:  Llama-3.2-1B, Llama-3.2-3B and Llama-3.1-8B models. For all the models, we fine-tune them on the MetaMathQA dataset for 1 epoch. To maximize training efficiency, we use a batch size of 32 with a sequence packing of 4096.
We experiment with different learning rates $10^{-5}, 2.5 \times 10^{-5}, 5 \times 10^{-5}, 10^{-4}$ and select the one with the lowest validation error. 
The final choices are $10^{-5}$ for the 8B model and $2.5 \times 10^{-5}$ for the others. For all the experiments, we use greedy decoding for evaluation.

\paragraph{Accuracy Comparison} \Cref{table:LLMtable} presents the results. Our latent approach consistently outperforms all the baselines across nearly all the tasks, for models of different sizes. For tasks on which we do not observe improvement, our approach is also comparable to the best performance. The gains are more pronounced in specific datasets such as Gaokao-Math-2023. On average, we are observing a $+5.3$ points improvement for the 8B model, $+2.9$ points improvement for the 3B model, and +3.7 points improvement for the 1B model. 


\paragraph{Tokens Efficiency Comparison}
Alongside the accuracy, we also report the number of tokens contained in the generated responses in \Cref{table:LLM-token}, which is the dominating factor of the inference efficiency. 
Our first observation is that for all the approaches, the model size has little influence on the length of generated responses.
Overall, the CoT method outputs the longest responses, while the Sol-Only method outputs the least number of tokens, since it is trained to generate the answer directly. The iCoT method generates short responses as well ($42.8\%$ reduction compared to CoT), as the CoT data has been iteratively eliminated in its training procedure. However, this comes at the cost of significantly degraded model performance compared with CoT, as shown in \Cref{table:LLMtable}. Our latent approach shows an average $17\%$ reduction in token numbers compared with CoT while surpassing it in prediction accuracy.


\subsection{Ablation \& Understanding Studies}
\label{sec:expr_ablation}

\paragraph{Replacement Strategies}
Our latent approach partially replaces the leftmost $m$ CoT tokens, where the value of $m$ varies for each sample. We call such replacement strategies \textbf{AR-Replace}. Here we consider three alternative strategies:
\begin{enumerate}[topsep=0pt]
    \item[(1)] \textbf{All-Replace}: all the text CoT tokens are replaced by the latent tokens.
    \item[(2)] \textbf{Curriculum-Replace}: the entire CoT subsequence are gradually replaced over the course of training, similar to the training procedure used by iCoT and COCONUT~\cite{hao2024training}. We train the model for 8 epochs. Starting from the original dataset, in each epoch we construct a new training dataset whether we further replace the leftmost 16 textual CoT tokens by a discrete latent token.
    \item[(3)] \textbf{Poisson-Replace}: instead of replacing tokens from left to right, we conduct a \emph{Poisson sampling} process to select CoT tokens to be replaced: we split the reasoning traces into chunks consisting of 16 consecutive text tokens, where each chunk is randomly replaced by the latent token with probability 0.5. 
    
\end{enumerate}


\Cref{table:ablation_replacement} reports the results.  Our \textbf{AR-Replace} strategy demonstrate strong performance, outperforming the other two strategies with large performance gap. Our intuition is as follows.
When all the textual tokens are removed, the model struggles to align the latent tokens with the linguistic and semantic structures it learned during pretraining. 

In contrast, partial replacement offers the model a bridge connecting text and latent spaces: the remaining text tokens serve as anchors, helping the model interpret and integrate the latent representations more effectively. 
Interestingly, the curriculum learning strategy is bridging the two spaces very well, where \textbf{All-Replace} and \textbf{Curriculum-Replace} exhibit similar performance. This is similar to our observation that iCoT performs remarkably worse than CoT for mathematical reasoning problems.
\textbf{Poisson-Replace} demonstrates performance marginally worse to our \textbf{AR-Replace} strategy on the 1B and 8B models, but significantly worse on the 3B model. Our intuition is that having a fix pattern of replacement (starting from the beginning and left to right) is always easier for the model to learn. This might be due to the limited finetuning dataset size and model capacity. 



\paragraph{Attention Weights Analysis}
To understand the reason why injecting latent tokens enhanced the model's reasoning performance, we randomly selected two questions from the Math and Collegue-Math dataset  and generate responses, then analyze the attention weights over the input prompt tokens:
\begin{enumerate}
    \item \texttt{What is the positive difference between \$120\%\$ of 30 and \$130\%\$ of 20?}
    \item \texttt{Mark has \$50 in his bank account. He earns \$10 per day at his work. If he wants to buy a bike that costs \$300, how many days does Mark have to save his money?}
\end{enumerate}
Specifically, we take the last attention layer, compute the average attention weights over different attention heads and show its relative intensity over the prompt tokens\footnote{We first compute the average attention weights across multiple heads. This gives us a single lower triangular matrix. Then,
we take the column sum of this matrix to get an aggregated attention weights for each token. Last, we normalize the weights by their average to obtain the relative intensity. A one line pseudocode is: \texttt{column\_sum(avg(attention\_matrices)) / avg(column\_sum(avg(attention\_matrices)))}. 
 }. We compare the averaged attention weights of our model with the CoT model in \Cref{fig:attention}.
Interestingly, our model learns to grasp a stronger attention to numbers and words representing mathematical operations. Both \cref{fig:entry_1} and \cref{fig:entry_2} show that the latent model focus more on the numbers, such as \texttt{120}, \texttt{30}, and \texttt{130} for the first question.
For the second question, our latent model shows a larger attention weights on numbers including \texttt{50}, \texttt{10}, and \texttt{300}, and also tokens semantically related to mathematical operations such as \texttt{earns} (means addition) and \texttt{cost} (means subtraction). 
This suggests that, by partially compressing the reasoning trace into a mix of latent and text tokens, we allow the model to effectively focus on important tokens that build the internal logical flow. See \Cref{app:generated_text_attention} for the exact response generated by our approach and the CoT baseline.


\begin{figure}[t]
  \centering
  \begin{subfigure}[b]{\columnwidth}
       \includegraphics[width=8cm]{plots/entry_1.png}
  \caption{Prompt: \texttt{What is the positive difference between 
  \$120\%\$ of 30 and \$130\%\$ of 20?}}
  \label{fig:entry_1}
  \end{subfigure}

  \begin{subfigure}[b]{\columnwidth}
       \includegraphics[width=8cm]{plots/entry_7746.png}
  \caption{Prompt: \texttt{Mark has \$50 in his bank account. He earns \$10 per day at his work. If he wants to buy a bike that costs \$300, how many days does Mark have to save his money?}}
  \label{fig:entry_2}
  \end{subfigure}
  \caption{Comparing with the CoT model, our latent approach have high attention weights on numbers and text tokens representing mathematical operations.}
  \label{fig:attention}
\end{figure}
 

\paragraph{Additional Experiments} We provide 4 additional example responses for questions in the Math and TheoremQA datasets in \Cref{app:generated_text_others}. In \Cref{app:additional_experiments}, we compare all the approaches when the model is trained on the DART-MATH~\cite{tong2024dart} 
dataset, where similar trends are observed.

\section{Conclusion}\label{sec:con}

Our work contributes empirical insights on the photorealism of AI-generated images and a taxonomy of artifacts commonly found in AI-generated images, organized into five categories: anatomical implausibilities, stylistic artifacts, functional implausibilities, violations of physics, and sociocultural implausibilities. We find that the photorealism of AI-generated images depends on the scene complexity of the image, the kind of artifacts and implausibilities, if any, detectable in an image, the duration of visual attention to an image, and the quality of human effort to select appropriate prompts and curate images. A question such as ``How photorealistic are state-of-the-art diffusion models'' may sound simple, but the answer is more complex and depends on many details, including what images are generated and selected, how photorealism is measured, what real images are included in the experiment, and how much time, skill, and effort a human participant has and willing to offer. This paper offers an initial exploration into how we can address this question and develops a practical taxonomy that offers scaffolding for building AI--literacy interventions to help people navigate the capabilities and limitations of diffusion models and whether an image is AI-generated or not. 

\begin{acks}

This material is based upon work supported by Robert Pozen, and in part with funding from the Department of Defense (DoD). Any opinions, findings, conclusions, or recommendations expressed in this material are those of the authors and do not necessarily reflect the views of the DoD or any agency or entity of the United States Government. We thank Will Thompson from Kellogg Research Support for performing a replication check.
\end{acks}

\bibliography{main}
\bibliographystyle{icml2025}

\clearpage
\onecolumn
\appendix
\appendix
\beginsupplement
\clearpage
\onecolumn
\section{Further Methodological Details} \label{sec:appendix-methodol}
\FloatBarrier
\begin{figure*}[ht]
\captionsetup{justification=raggedright, singlelinecheck=false, skip=2pt}
\centering
\begin{subfigure}[t]{0.3\linewidth}
   
    \subcaption{}
    \includegraphics[width=\linewidth]{sections/images/biden.jpg}
\end{subfigure}
\hspace{1cm}
\begin{subfigure}[t]{0.31\linewidth}

   \subcaption{}
   \includegraphics[width=\linewidth]{sections/images/rock.jpg}
\end{subfigure}
\caption{\mybold{AI-Generated Images from New York Times Quiz} \normalfont{\textbf{A}. NYT's explanation for evidence pointing to this image as AI-generated is: ``Though the resemblance to President Biden is striking, he would not be wearing military fatigues as a civilian.''~\cite{nytimes2024deepfake} \textbf{B}. NYT's explanation for evidence pointing to this image as AI-generated is ``One giveaway in this image is the badge, which includes garbled text.''~\cite{nytimes2024deepfake}}}
\label{fig:nytimes}
\Description{AI-generated image of Joe Biden in a conference room and an AI-generated image of the Rock in military uniform in a mall.}
\end{figure*}
\begin{figure*}[ht]
\centering
\captionsetup{justification=raggedright, singlelinecheck=false, skip=2pt}
\begin{subfigure}[t]{0.65\textwidth}
    \caption{}
    \vtop{\vskip0pt\hbox{\includegraphics[width=\linewidth]{sections/images/refiningpipeline.jpg}}}
\end{subfigure}
\hspace{0.01\textwidth} 
\begin{minipage}[t]{0.23\textwidth}
    \begin{subfigure}[t]{\textwidth}
        \caption{}
        \vtop{\vskip0pt\hbox{\includegraphics[width=\linewidth]{sections/images/facerefine.jpg}}}
    \end{subfigure}
    \vspace{0.5cm}
    \begin{subfigure}[t]{\textwidth}
        \caption{}\hfill\vtop{\vskip0pt\hbox{\includegraphics[width=0.8\linewidth]{sections/images/handrefine.jpg}}}
    \end{subfigure}
\end{minipage}
 \caption{\textbf{Image generation process in Stable Diffusion} A. \normalfont{Four stage image generation pipeline where the image is first generated in SD1.5. The output image is then encoded as latent and upscaled to be re-generated in SDXL with ControlNets applied for pose consistency. This is passed to the face refiner \cite{comfyuiimpactpack} which detects dominant and background faces in the image via YOLOv8 \cite{yolov8} and re-generates them using an SDXL pipeline. Finally, the resulting image is passed to the hand refiner \cite{comfyuiimpactpack} which detects hands in the image via YOLOv8  and predicts the hand pose used to guide the re-generation of the hands. \textbf{B}. Faces in the image before and after the face refining process \textbf{C}. Hand refining process. The left image shows the initial generation of the hand. The center image shows a predicted skeleton for the hand that is used for a ControlNet that guides the re-generation of the hand shown in the image on the right.}}
\label{fig:refiningpipe}
\Description{4 stage image generation pipeline}
\end{figure*}
\FloatBarrier
\twocolumn
According to a New York Times (NYT) quiz, qualities that typically signify AI generation include missing fingers, misaligned eyes, repeated elements, and garbled or nonsensical details~\cite{nytimes2024deepfake}.  Examples are shown in \ref{fig:nytimes}. The NYT quiz also discusses qualities that may cause a real image to look AI-generated, such as repeated cropping and compression that often happens over social media.

A screenshot of the pipeline, along with images before and after refinement, is shown in Figure~\ref{fig:refiningpipe}.
\FloatBarrier

\FloatBarrier
Figure~\ref{fig:pose-comprehensive} displays more examples of the four pose complexities and their average accuracies.
\onecolumn
\begin{figure}[H]
\centering
\resizebox{0.85\textwidth}{!}{
% This ensures the figure fits the page
\begin{minipage}{\textwidth}
% Portraits
\begin{subfigure}{0.22\linewidth}
    \includegraphics[width=\linewidth]{sections/images/ff_portrait3_014.jpeg}
    \caption{Acc: 25\%}
\end{subfigure}
\hfill
\begin{subfigure}{0.22\linewidth}
    \includegraphics[width=\linewidth]{sections/images/sd_portrait3_003.jpg}
    \caption{Acc: 37\%}
\end{subfigure}
\hfill
\begin{subfigure}{0.22\linewidth}
    \includegraphics[width=\linewidth]{sections/images/ff_portrait1_002.jpeg}
    \caption{Acc: 66\%}
\end{subfigure}
\hfill
\begin{subfigure}{0.22\linewidth}
    \includegraphics[width=\linewidth]{sections/images/sd_portrait3_075.jpg}
    \caption{Acc: 80\%}
\end{subfigure}

\vspace{0.3cm}

% Full Body
\begin{subfigure}{0.22\linewidth}
    \includegraphics[width=\linewidth]{sections/images/mj_fullbody3_028.jpg}
    \caption{Acc: 37\%}
\end{subfigure}
\hfill
\begin{subfigure}{0.22\linewidth}
    \includegraphics[width=\linewidth]{sections/images/mj_fullbody3_012.jpg}
    \caption{Acc: 57\%}
\end{subfigure}
\hfill
\begin{subfigure}{0.22\linewidth}
    \includegraphics[width=\linewidth]{sections/images/mj_fullbody3_022.jpg}
    \caption{Acc: 66\%}
\end{subfigure}
\hfill
\begin{subfigure}{0.22\linewidth}
    \includegraphics[width=\linewidth]{sections/images/mj_fullbody3_029.jpg}
    \caption{Acc: 83\%}
\end{subfigure}

\vspace{0.3cm}

% Posed Groups
\begin{subfigure}{0.22\linewidth}
    \includegraphics[width=\linewidth]{sections/images/mj_pg3_017.jpg}
    \caption{Acc: 37\%}
\end{subfigure}
\hfill
\begin{subfigure}{0.22\linewidth}
    \includegraphics[width=\linewidth]{sections/images/mj_pg2_012.jpg}
    \caption{Acc: 57\%}
\end{subfigure}
\hfill
\begin{subfigure}{0.22\linewidth}
    \includegraphics[width=\linewidth]{sections/images/mj_pg3_003.jpg}
    \caption{Acc: 66\%}
\end{subfigure}
\hfill
\begin{subfigure}{0.22\linewidth}
    \includegraphics[width=\linewidth]{sections/images/sd_pg3_013.jpg}
    \caption{Acc: 83\%}
\end{subfigure}

\vspace{0.3cm}

% Candid Groups
\begin{subfigure}{0.22\linewidth}
    \includegraphics[width=\linewidth]{sections/images/mj_ng3_016.jpg}
    \caption{Acc: 31\%}
\end{subfigure}
\hfill
\begin{subfigure}{0.22\linewidth}
    \includegraphics[width=\linewidth]{sections/images/mj_ng2_007.jpg}
    \caption{Acc: 66\%}
\end{subfigure}
\hfill
\begin{subfigure}{0.22\linewidth}
    \includegraphics[width=\linewidth]{sections/images/mj_ng4_003.jpg}
    \caption{Acc: 75\%}
\end{subfigure}
\hfill
\begin{subfigure}{0.22\linewidth}
    \includegraphics[width=\linewidth]{sections/images/mj_ng3_005.jpg}
    \caption{Acc: 87\%}
\end{subfigure}

\end{minipage}
}
\vspace{-2mm}
\caption{\textbf{More examples of the four pose complexities and their average accuracies.} \normalfont{The first row shows Portraits, the second row Full Body images, the third row Posed Groups, and the last row Candid Groups.}}
\label{fig:pose-comprehensive}
\Description{Examples of AI-generated images in different pose complexities: Portraits, Full Body, Posed Groups, and Candid Groups, with participant accuracy percentages.}
\end{figure}
\FloatBarrier
\twocolumn
\clearpage
\subsection{Robustness Check: Dataset Comparison}

To ensure the validity of our conclusions, we conducted a robustness check comparing the results from our full dataset against a subset excluding data collected before May 10th, 2024. This comparison addresses potential biases introduced by the initial experimental design, which did not implement stratified randomization as mentioned in Section \ref{exp-design}.

Table~\ref{tab:accuracy-comparison-dataset} presents the accuracy metrics for both the full dataset and the dataset excluding pre-May 10th data. The table includes overall accuracy, as well as specific accuracy for AI-generated and real images, along with their respective 95\% confidence intervals.

\begin{table}[H]
\centering
\caption{Comparison of accuracy: Full Dataset vs. Dataset excluding data before May 10th}
\label{tab:accuracy-comparison-dataset}
\resizebox{\linewidth}{!}{
\begin{tabular}{lcccccc}
\hline
Dataset & \multicolumn{2}{c}{Overall} & \multicolumn{2}{c}{AI-generated} & \multicolumn{2}{c}{Real} \\
 & Accuracy & 95\% CI & Accuracy & 95\% CI & Accuracy & 95\% CI \\
\hline
Full Dataset & 0.75 & [0.74, 0.76] & 0.76 & [0.74, 0.77] & 0.73 & [0.71, 0.75] \\
Dataset excluding data before May 10th & 0.75 & [0.74, 0.76] & 0.76 & [0.75, 0.77] & 0.7201 & [0.70, 0.74] \\
\hline
\end{tabular}}
\Description{A robustness check by comparing accuracy in full dataset vs. dataset excluding data before May 10th}
\end{table}

Figure~\ref{fig:accuracy-comparison-dstaset} visualizes the distribution of image accuracies for both datasets. This comparison allows for direct observation of any potential shifts in accuracy distributions between the full dataset and the subset, excluding early data.
 This robustness check supports the validity of using the full dataset in our main analysis.

\begin{figure}[H]
\centering
\includegraphics[width=\linewidth]{sections/images/dataset_accuracy_distribution_comparison.jpg}
\vspace{-10mm}
\caption{Comparison of accuracies between the full dataset and the dataset excluding data before 10th data.}
\label{fig:accuracy-comparison-dstaset}
\Description{Figure compares the distribution of accuracy of images from the full dataset vs. from the dataset excluding data before May 10th}
\end{figure}

\clearpage
\onecolumn
\section{Curated and Uncurated AI-generated Images}

\begin{figure*}[!htb]
\centering
\resizebox{1.0\textwidth}{!}{ % Scale figure to 95% of text width
\begin{minipage}{\textwidth} % Ensures correct alignment
\captionsetup{justification=raggedright, singlelinecheck=false, skip=2pt}

% Row A

\begin{subfigure}[t]{0.23\linewidth}  
\subcaption{}
\vtop{\vskip0pt\hbox{\includegraphics[width=\linewidth]{sections/images/ff_portrait3_001.jpeg}}}
\end{subfigure}
\hfill
\begin{subfigure}[t]{0.23\linewidth}  
\subcaption{}
\vtop{\vskip0pt\hbox{\includegraphics[width=\linewidth]{sections/images/american_faculty1.jpg}}}
\end{subfigure}
\hfill
\begin{subfigure}[t]{0.23\linewidth}  
\subcaption{}
\vtop{\vskip0pt\hbox{\includegraphics[width=\linewidth]{sections/images/american_faculty2.jpg}}}
\end{subfigure}
\hfill
\begin{subfigure}[t]{0.23\linewidth}  
\subcaption{}
\vtop{\vskip0pt\hbox{\includegraphics[width=\linewidth]{sections/images/american_faculty3.jpg}}}
\end{subfigure}

\vspace{8pt} % Reduced spacing between rows

% Row B
\begin{subfigure}[t]{0.23\linewidth}  
\subcaption{}
\vtop{\vskip0pt\hbox{\includegraphics[width=\linewidth]{sections/images/ff_pg4_001.jpeg}}}
\end{subfigure}
\hfill
\begin{subfigure}[t]{0.23\linewidth}  
\subcaption{}
\vtop{\vskip0pt\hbox{\includegraphics[width=\linewidth]{sections/images/astronaut1.jpg}}}
\end{subfigure}
\hfill
\begin{subfigure}[t]{0.23\linewidth}  
\subcaption{}
\vtop{\vskip0pt\hbox{\includegraphics[width=\linewidth]{sections/images/astronaut2.jpg}}}
\end{subfigure}
\hfill
\begin{subfigure}[t]{0.23\linewidth}  
\subcaption{}
\vtop{\vskip0pt\hbox{\includegraphics[width=\linewidth]{sections/images/astronaut3.jpg}}}
\end{subfigure}

\end{minipage}
}
\vspace{-2mm}
\caption{\mybold{Example images generated by consistently photorealistic and consistently detectable prompts.} \normalfont{
\textbf{A.} Curated image generated with a consistently photorealistic prompt: ``American woman faculty portrait, not a close-up, blond." 
\textbf{B-D} Reprompted images generated with the same consistently photorealistic prompts. 
\textbf{E.} Curated image generated with a consistently detectable prompt: ``Persian woman astronaut in astronaut clothes, family photo with husband and two toddlers, high resolution, realistic." 
\textbf{F-H} Reprompted images of the same consistently detectable prompts.}}
\label{fig:goodandbadprompt}
\Description{Two example images where A shows a portrait image of an American woman faculty with few visible artifacts and B shows a Persian woman and her child and husband in a space suit with noticeable artifacts in all of their faces.}
\end{figure*}

\clearpage
\onecolumn
\section{Future Work on Videos}
\begin{figure*}[ht]
\centering
\resizebox{1.0\textwidth}{!}{ 
\begin{minipage}{\textwidth} 
\captionsetup{justification=raggedright, singlelinecheck=false, skip=2pt}

% Row A
\begin{subfigure}[t]{0.33\linewidth}  
\subcaption{}
\vtop{\vskip0pt\hbox{\includegraphics[width=\linewidth]{sections/images/out_7.jpg}}}
\end{subfigure}
\hfill
\begin{subfigure}[t]{0.33\linewidth}  
\subcaption{}
\vtop{\vskip0pt\hbox{\includegraphics[width=\linewidth]{sections/images/out_9.jpg}}}
\end{subfigure}
\hfill
\begin{subfigure}[t]{0.33\linewidth}  
\subcaption{}
\vtop{\vskip0pt\hbox{\includegraphics[width=\linewidth]{sections/images/out_11.jpg}}}
\end{subfigure}

\vspace{10pt} % Spacing between rows

\begin{subfigure}[t]{0.33\linewidth}  
\subcaption{}
\vtop{\vskip0pt\hbox{\includegraphics[width=\linewidth]{sections/images/out_13.jpg}}}
\end{subfigure}
\hfill
\begin{subfigure}[t]{0.33\linewidth}  
\subcaption{}
\vtop{\vskip0pt\hbox{\includegraphics[width=\linewidth]{sections/images/out_15.jpg}}}
\end{subfigure}
\hfill
\begin{subfigure}[t]{0.33\linewidth}  
\subcaption{}
\vtop{\vskip0pt\hbox{\includegraphics[width=\linewidth]{sections/images/out_17.jpg}}}
\end{subfigure}

\vspace{10pt} % Spacing between rows

\begin{subfigure}[t]{0.33\linewidth}  
\subcaption{}
\vtop{\vskip0pt\hbox{\includegraphics[width=\linewidth]{sections/images/out_19.jpg}}}
\end{subfigure}
\hfill
\begin{subfigure}[t]{0.33\linewidth}  
\subcaption{}
\vtop{\vskip0pt\hbox{\includegraphics[width=\linewidth]{sections/images/out_21.jpg}}}
\end{subfigure}
\hfill
\begin{subfigure}[t]{0.33\linewidth}  
\subcaption{}
\vtop{\vskip0pt\hbox{\includegraphics[width=\linewidth]{sections/images/out_23.jpg}}}
\end{subfigure}

\vspace{10pt} % Spacing between rows

% Row C

\begin{subfigure}[t]{0.33\linewidth}  
\subcaption{}
\vtop{\vskip0pt\hbox{\includegraphics[width=\linewidth]{sections/images/out_25.jpg}}}
\end{subfigure}
\hfill
\begin{subfigure}[t]{0.33\linewidth}  
\subcaption{}
\vtop{\vskip0pt\hbox{\includegraphics[width=\linewidth]{sections/images/out_27.jpg}}}
\end{subfigure}
\hfill
\begin{subfigure}[t]{0.33\linewidth}  
\subcaption{}
\vtop{\vskip0pt\hbox{\includegraphics[width=\linewidth]{sections/images/out_29.jpg}}}
\end{subfigure}

\end{minipage}
}
\vspace{-2mm}
\caption{\mybold{Example frames from an AI-generated video with a temporal anatomical implausibility.} \normalfont{
9 frames from a video generated by OpenAI's Sora diffusion-transformer model where the subject's right leg morphs into the left leg somewhere between E and J. Each frame is separated by 1/10 of a second. This particular artifact fits into the anatomical implausibility category of the taxonomy, but it's different from any anatomical plausibility seen in diffusion model-generated images. In particular, this implausibility has a temporal element: the transition from A to L involves the subject's right leg becoming her left in a split second, which does not fit with what we know about human anatomy.}}
\label{fig:sora}
\Description{9 frames from a video generated by OpenAI's Sora diffusion-transformer model where the subject's right morphs into the left leg somewhere between E and J. Each frame is separated by 1/10 of a second. This particular artifact fits into the anatomical implausibility category of the taxonomy, but it's different than any anatomical plausibility in diffusion model-generated images. In particular, this implausibility has a temporal element: the transition from A to L involves the subject's right leg becoming her left in a split second, which does not fit with what we know about human anatomy.}
\end{figure*}


\end{document}


% This document was modified from the file originally made available by
% Pat Langley and Andrea Danyluk for ICML-2K. This version was created
% by Iain Murray in 2018, and modified by Alexandre Bouchard in
% 2019 and 2021 and by Csaba Szepesvari, Gang Niu and Sivan Sabato in 2022.
% Modified again in 2023 and 2024 by Sivan Sabato and Jonathan Scarlett.
% Previous contributors include Dan Roy, Lise Getoor and Tobias
% Scheffer, which was slightly modified from the 2010 version by
% Thorsten Joachims & Johannes Fuernkranz, slightly modified from the
% 2009 version by Kiri Wagstaff and Sam Roweis's 2008 version, which is
% slightly modified from Prasad Tadepalli's 2007 version which is a
% lightly changed version of the previous year's version by Andrew
% Moore, which was in turn edited from those of Kristian Kersting and
% Codrina Lauth. Alex Smola contributed to the algorithmic style files.
