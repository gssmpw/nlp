In this work, we addressed the challenge of human-robot cooperative navigation in environments with incomplete information by introducing the Information Gain Monte Carlo Tree Search (IG-MCTS) algorithm. By strategically balancing autonomous movement and informative communication, IG-MCTS leverages a learned human perception dynamics model to optimize interactions with a human operator. User studies demonstrated that IG-MCTS significantly reduces communication demands and human cognitive load while maintaining task performance comparable to teleoperation and instruction-following baselines.

% One limitation of this work is its reliance on in-context data for training the human perception dynamics model, which may limit generalization to new users or environments. Future research could explore methods for improving transferability.
% Another important direction is extending IG-MCTS to continuous state and action spaces, bringing it closer to real-world robotic applications. Additionally, as discussed in the implementation section, further exploration is needed to enable multi-robot management and enhance user experience.

% Limitations of the current interaction implementation point to several ways to improve user experience in the future:
% The IG-MCTS system's reliance on a single final ego-view snapshot, while efficient, may lack the continuity needed for effective spatial understanding. Replacing static snapshots with short video clips could provide richer spatial context and help users better grasp movement trends.
% Additionally, the fixed top-down global map may not align well with human spatial cognition, as humans often struggle with spatial reorientation. Adopting a dynamic ego-centric representation of the map could make it easier for users to interpret the robot’s environment and provide guidance without the mental burden of perspective transformation.


\paragraph{Future work.}
Despite these promising results, several directions remain for future exploration. The reliance on in-context data for training the human perception dynamics model may hinder generalization. Future work could look into improving transferability through meta-learning or domain adaptation. 
% extension to continuous space
Furthermore, extending IG-MCTS to continuous state and action spaces would bring it closer to real-world applications.
% room for UX improvement
Additionally, refining the interaction design could improve user experience. The current system relies on a single final ego-view snapshot, which, while efficient, may not provide sufficient spatial continuity. Integrating short video clips instead could offer richer contextual cues and improve spatial understanding. 
% Similarly, the fixed top-down global map may impose a cognitive burden on users due to challenges with spatial reorientation. Adopting a dynamic ego-centric map representation could reduce the need for perspective transformations.


