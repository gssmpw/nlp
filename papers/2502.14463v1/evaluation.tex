\vspace{-.5em}
\begin{table}[h]
\centering
\footnotesize
\caption{The \totalRule metadata-usage constraints we summarized}\label{tab:constraints}
  \vspace{-1.5em}
\begin{tabular}{|c|p{1.5cm}|p{8cm}|p{1.5cm}|}
\hline
\textbf{Id} & \textbf{Rule Name} & \textbf{Constraint Summary} & \textbf{Involved Items} \\
\hline
$r_1$ & xml-path-check & When constructor ``{\tt ClassPathXMLApplicationContext(String configLocation)}'' is called, the provided argument should correspond to an existent XML file in the file system. &XML, code \\
\hline
$r_2$ & bean-class-exists & Java classes mentioned by the {\tt class}-attribute of {\tt <bean>} should exist in the project, unless they are library classes. & XML, code \\
\hline
$r_3$ & constructor-arg-type-field-map & When {\tt <constructor-arg>} is specified as a sub-element of {\tt <bean>}, its {\tt type}-attribute value should match a {constructor parameter}'s type name of the corresponding Java class.&XML, code \\
\hline
$r_4$ & constructor-arg-name-field-map & The {\tt name}-attribute value of {\tt <constructor-arg>} should match a {constructor parameter} name in the corresponding Java class.& XML, code \\
\hline
$r_5$ & constructor-index-out-of-bound & The {\tt index}-attribute value of {<constructor-arg>} should fit into the index boundary of at least one constructor in the corresponding Java file. & XML, code \\
\hline
$r_6$ & method-exists & When a {\tt <bean>} has any of the following attributes configured: {\tt init-method} and {\tt destroy-method}, the attribute values should match the names of methods defined in corresponding Java classes. &XML, code \\
\hline
$r_7$ & property-setter-map & The {\tt name}-attribute values of {\tt <property>} items should map to setter methods' names in the corresponding Java class. For instance, if the {\tt name}-attribute of a {\tt <property>} item is ``{\tt pn}'', then there should be a setter named ``{\tt setPn}'' in the corresponding Java class. & XML, code \\
\hline
$r_{8}$ & runwith-no-parameters & If a Java class has ``{\tt @RunWith(Parameterized.class)}'' annotation, then it should also have a method with ``{\tt @Parameters}'' annotation. & annotation, code\\
\hline
$r_{9}$ & runwith-no-test & If a Java class has ``{\tt @RunWith(Parameterized.class)}'' annotation, then it should also have a method with ``{\tt @Test}'' annotation. & annotation, code\\
\hline
$r_{10}$ & runwith-no-suiteclasses & If a Java class has ``{\tt @RunWith(Suite.class)}'' annotation, then the class should also have ``{  \tt @SuiteClasses}'' annotation. & annotation, code \\
\hline
$r_{11}$ & suiteclasses-no-runwith & If a Java class has ``{\tt @SuiteClasses}'' annotation, then the class should also have ``{\tt @RunWith(Suite.class)}'' annotation. & annotation, code \\
\hline
$r_{12}$ & suiteclasses-no-test & If a Java class is annotated with ``{\tt @SuiteClasses}'' or ``{\tt @Suite.SuiteClasses}'', then the Java class mentioned in the annotation attribute should (1) be decorated with either annotation, or (2) have a method in itself or any ancestor class satisfying any of the following requirement: (i) the method is annotated with {\tt @Test}, (ii) the method name starts with ``{\tt test}'', and (iii) the method name is ``{\tt suite}''. & annotation, code
%Suite classes should either be another Suite class with ``{  @SuiteClasses}'' annotation, or it should have a method with ``{  @Test}'' annotation. & JUnit-annotation 
\\
\hline
$r_{13}$ & testParams-not-iterable & Each Java method with ``{\tt  @Parameters}'' annotation should have an iterable return type (e.g., List). & annotation, code \\
\hline

$r_{14}$ & import-resource-path & When ``{\tt @ImportResource}'' is used and its attribute ``{\tt location}'' is configured, the attribute value should be a valid file path corresponding to an existent XML file in the file system. & XML, annotation \\
\hline

$r_{15}$ & bean-exists & The Spring API ``{\tt ApplicationContext.getBean(String str)}'' searches for a bean by name or by type. 
Search-by-type works when the argument {\tt str} is a Java class name (i.e., ending with ``{\tt .class}''), and the class should be either (1) annotated with ``{\tt @Component}'', ``{\tt @Service}'', ``{\tt @Repository}'', ``{\tt @Controller}'', or ``{\tt @RestController}'', or (2) mentioned by the {\tt class}-attribute of any {\tt <bean>} in XML. 
Search-by-name requires either a Java method annotated with ``{\tt @Bean}'' is named with {\tt str}, or a {\tt <bean>} in XML has its {\tt  id}-attribute value as {\tt str}.& XML, annotation, code \\
\hline
\end{tabular}
\vspace{-2.5em}
\end{table}

\section{Evaluation}\label{se:evaluation}

To assess the usefulness of our approach, we conducted three experiments and explored the following research questions (RQs):

\begin{itemize}
\item \textbf{RQ1}: How effectively can RSL express metadata-usage rules?
\item \textbf{RQ2}: How accurately can \tool detect bugs?
\item \textbf{RQ3}: How effectively does \tool reveal real-world bugs?
\end{itemize}



  \vspace{-.5em}
\subsection{Effectiveness of RSL}\label{ss:eval-effect-rsl}

To assess the expression power of RSL, we studied (1) the documentation of Spring \cite{spring}---the most popular application development framework for EAs in Java, and (2) documentation as well as tutorials on JUnit~\cite{junit-doc,junit-tutorial,how-to-junit,software-testing}---the most popular framework for unit testing in Java. 
We manually distilled \totalRule metadata-usage constraints from those documents. As shown in Table \ref{tab:constraints}, seven constraints are about the correspondence between XML and code ($r_1$--$r_7$); six constraints involve annotations and code ($r8$--$r13$); one constraint is on the correspondence between XML and annotation ($r14$); one constraint is relevant to XML, annotation, and code.
These constraints are diverse in terms of the metadata/code items involved, the number of files scanned, and the consistency-checking logic. We successfully expressed constraint-checking rules in RSL, which fact  evidences the great expression power of our domain-specific language. In the following experiments, we will leverage \tool to generate analyzers from these rules, and evaluate \tool's effectiveness in identifying buggy EAs that violate any of the rules.

\vspace{0.5em}
\noindent\begin{tabular}{|p{13.6cm}|}
	\hline
	\textbf{Finding 1 (Responding to RQ1):} \emph{RSL has great expressiveness. It is capable of describing a variety of metadata-usage checking rules that involve XML, annotation, and/or code.}
	\\
	\hline
\end{tabular}





  \vspace{-.5em}
\subsection{Accuracy of \tool}\label{ss:eval-effect-mecheck}

To assess how accurately \tool can detect metadata-related bugs, we created a ground truth dataset of buggy EAs (Section~\ref{sss:data}), defined evaluation metrics (Section~\ref{sss:metrics}), and evaluated \tool using the dataset and metrics (Section~\ref{sss:bug-detect}). 

  \vspace{-.5em}
\subsubsection{A Dataset of Buggy EAs} \label{sss:data}
We mined GitHub~\cite{github} for enterprise applications, using the heuristic-based approach proposed by prior work~\cite{Wen20}. Specifically, we crawled GitHub for any project that contains at least one XML file, whose file path has any of the following keywords: ``Spring", ``security", ``web", and ``WEB-INF". If a project satisfies this requirement, it is likely to be an EA. 
Next, we refined the mined projects by removing irrelevant and redundant projects, getting \totalProject projects.

To create the dataset of buggy EAs, we injected \totalInjection bugs in \totalInjectedProject different projects. To inject bugs for each constraint mentioned in Table~\ref{tab:constraints}, we leveraged the involved metadata items (e.g., ``\codefont{@RunWith(Parameterized.class)}'' or ``\codefont{<property>}'') as keywords to search for three different projects. We manually inspected the relevant metadata usage to ensure that each retrieved project does not violate the corresponding constraint. Then we modified the metadata to intentionally inject a bug for each of the projects. For the 15 rules, we retrieved and modified in total \totalInjectedProject(=15*3) projects. In this way, we injected \totalInjection bugs to \totalInjectedProject distinct open-source projects, making each analyzer inside \tool have 3 bugs to find. We considered these \totalInjection injected bugs as ground truth to evaluate the detection capability of \tool.


% To create the dataset of buggy EAs, for each constraint mentioned in Table~\ref{tab:constraints}, we leveraged the involved items (e.g., ``\codefont{<property>}'') as keywords to search for 3 relevant projects within \totalProject projects. We ensured that each retrieved project does not violate the corresponding constraint, and then manually modified code or metadata to violate that constraint. In this way, we injected \totalInjection bugs to \totalInjectedProject distinct open-source projects, making each analyzer inside \tool have 3 bugs to find.

   \vspace{-.5em}
\subsubsection{Metrics} \label{sss:metrics}
	We used three metrics to evaluate \tool's effectiveness for bug detection.

\textbf{Precision (P)} measures among all bugs \tool reported, how many of them are true positives:
\begin{equation*}\label{eq:precision}
 \vspace{-.5em}	
	    P = \frac{\mbox{\# of correctly reported bugs}}{\mbox{Total \# of bug reports}} 
  %\vspace{-.5em}    
	\end{equation*}
	
%	Precision can be used to evaluate the effectiveness of rule application. For rule application, P measures among all reported bugs, how many of them are real bugs.\\
\textbf{Recall (R)} measures among all known bugs, how many of them are reported by \tool:
	\begin{equation*}
	\label{eq:recall}
 \vspace{-.5em} 
	    R = \frac{\mbox{\# of correctly reported bugs}}{\mbox{Total \# of known bugs}}
	\end{equation*}
%	Based on the Spring specification and our manual inspection, we managed to construct a ground truth data set and evaluated  the recall of the rule application.\\

\textbf{F-score (F)} is the harmonic mean of P and R; it reflects the bug-detection accuracy of \tool: 
	\begin{equation*}
	\label{eq:fscore}
 \vspace{-.5em} 
	    F = \frac{2 \times P \times R}{P + R} 
	\end{equation*}
All metrics mentioned above have values in [0\%, 100\%]; the higher the better. To compute P and R, we will intersect the set of bugs reported with the ground-truth bug set; any overlap between the two sets captures bugs correctly reported by \tool. Thus, the larger overlap there is, the better.

%Any overlap between the two sets 
%the scenarios where \tool correctly reports bugs

\begin{comment}
\begin{table}
\footnotesize
\centering
\caption{The evaluation result of \tool  on our dataset of injected bugs}\label{tab:synthetic}
  \vspace{-1.5em}
\begin{tabular}{l|p{3cm}|R{2.3cm}|R{2.3cm}|R{0.7cm}R{0.7cm}R{0.7cm}}
\toprule
\textbf{Id} &\textbf{Rule Name} & \textbf{\# of Injected Bugs} & \textbf{\# of Detected Bugs} & \textbf{P} & \textbf{R} & \textbf{F} \\
\toprule
$r_1$& xml-path-check & 3 & 3 & 100\% & 100\% & 100\% \\\hline
$r_2$&bean-class-exists & 3 & 1 & 100\% & 33\% & 50\% \\
\hline
$r_3$&constructor-arg-type-field-map & 3 & 3 & 100\% & 100\% & 100\% \\
\hline
$r_4$&constructor-arg-name-field-map & 3 & 3 & 100\% & 100\% & 100\% \\
\hline
$r_5$&constructor-index-out-of-bound & 3 & 3 & 100\% & 100\% & 100\% \\
\hline
$r_6$&method-exists & 3 & 3 & 100\% & 100\% & 100\% \\
\hline
$r_7$&property-setter-map & 3 & 3 & 100\% & 100\% & 100\% \\
\hline
$r_8$&runwith-no-parameters & 3 & 3 & 100\% & 100\% & 100\% \\
\hline
$r_9$&runwith-no-test & 3 & 3 & 100\% & 100\% & 100\% \\
\hline
$r_{10}$&runwith-no-suiteclasses & 3 & 3 & 100\% & 100\% & 100\% \\
\hline
$r_{11}$&suiteclasses-no-runwith & 3 & 3 & 100\% & 100\% & 100\% \\
\hline
$r_{12}$&suiteclasses-no-test & 3 & 3 & 100\% & 100\% & 100\% \\
\hline
$r_{13}$&testParams-not-iterable & 3 & 3 & 100\% & 100\% & 100\% \\
\hline
$r_{14}$&import-resource-path & 3 & 3 & 100\% & 100\% & 100\% \\
\hline
$r_{15}$&bean-exists & 3 & 3 & 100\% & 100\% & 100\% \\
\bottomrule
\multicolumn{2}{c|}{\textbf{Overall}} & \textbf{45} & \textbf{43} & \textbf{\precision} & \textbf{\recall} & \textbf{\fscore} \\
\bottomrule
\end{tabular}
  \vspace{-1.5em}
\end{table}
\end{comment}

\subsubsection{Experiment Results} \label{sss:bug-detect}
%As shown in Table~\ref{tab:synthetic}, 
\tool detected bugs with high precision (\precision), high recall (\recall), and high F-score (\fscore). In total, \tool reported 43 bugs, all of which match the ground truth of injected bugs. However, it missed two bugs related to the rule $r_2$ bean-class-exists, due to the design choice we made when implementing \tool.

  \vspace{-.5em}
\begin{lstlisting}[label={lst:bean-class-exists},caption=The RSL rule of bean-class-exists]
Rule bean-class-exists {
  for (file xml in getXMLs()) {
    if (elementExists(xml, "<bean>")) {
      for (<bean> bean in getElms(xml, "<bean>")) {
        String beanClassFQN = getAttr(bean, "class");
        assert(classExists(beanClassFQN) OR isLibraryClass(beanClassFQN)) {
          msg("Bean class: %s mentioned in bean: %s, does not exist", beanClassFQN, getName(bean)); }}}}}  
\end{lstlisting}

With more details, $r_2$ ensures that if the fully qualified name of a Java class is mentioned as the \codefont{class}-attribute of any \codefont{<bean>}, the class should be either defined by the 
EA-under-analysis or a library on which EA depends. Listing~\ref{lst:bean-class-exists} shows the RSL specification of bean-class-exists, which calls \codefont{callExists(...)} and \codefont{isLibraryClass(...)} on line 6. 
In \tool, \codefont{classExists(...)} is implemented to check whether the pass-in parameter matches any class item extracted from Java code.
The implementation of \codefont{isLibraryClass(...)} defines a list of regular expressions (RegEx) to describe the naming patterns of frequently used library classes, such as ``\codefont{\textasciicircum org\textbackslash.hibernate\textbackslash..+\textdollar}'' for Hibernate classes. We crafted the regular expressions based on (1) our domain knowledge of widely used libraries, and (2) libraries we frequently observed in EAs.
Next, \codefont{isLibraryClass(...)} 
 examines whether a pass-in parameter matches any RegEx to determine if the class is defined by a library. 
We manually injected each of the two missed bugs, by specifying an invalid class name as the \codefont{class}-attribute of \codefont{<bean>}. Ideally, \tool should report both bugs because the \codefont{<bean>}-classes do not exist. However, since the invalid names we specified (e.g., \codefont{org.hibernate.search.hibernate.example.dao.impl.BookDaoImplChanged}) accidentally match predefined RegEx patterns, \tool incorrectly considered them to be valid and failed to locate both bugs.

We could have reduced such false negatives by discarding some RegEx patterns used in function \codefont{isLibraryClass(...)}. However, the downside of doing so is that 
many library classes will fail to match the remaining patterns. Consequently, \tool will wrongly interpret those classes as non-library classes, and report false positives when it also fails to find those classes defined by EA. 
In another word, we may get false positives when trying to overcome such false-negative issues. We prefer generating precise bug reports as developers may not want to get frequently bothered  with false positives. Thus, we designed \tool to include more RegEx patterns and report bugs more precisely, instead of improving recall rates at the cost of sacrificing precision. 



%Because of these two techniques, when we ran \tool on our synthetic dataset we didn't have any false positives reported. As a result we had 100\% precision, 100\% recall, and 100\% accuracy for our experiment on the synthetic dataset (Table \ref{table:ruleEffect_groundTruth}).

\vspace{0.5em}
\noindent\begin{tabular}{|p{13.6cm}|}
	\hline
	\textbf{Finding 2 (Responding to RQ2):} \emph{\tool demonstrated great detection accuracy when being applied to our 45-project dataset with injected bugs. It found bugs with \precision precision, \recall recall, and \fscore F-score. }
	\\
	\hline
\end{tabular}


\subsection{Detection of Metadata-Related Bugs in Real-World Settings}\label{ss:eval-real-world}

To assess how well \tool reveals real bugs, we also created a dataset of \totalRealProject real-world open-source software repositories (Section~\ref{sec:dataset}), and applied \tool to that dataset (Section~\ref{sec:real-setting}). 

\subsubsection{A Dataset of Real-World Software Repositories}\label{sec:dataset}
To create this dataset, we started with the \totalProject projects initially mined from GitHub (see Section~\ref{sss:data}). From those projects, we removed 85 projects that are irrelevant to any of the \totalRule rules implemented in \tool. Namely, if a project does not contain any code/XML/annotation item involved in those rules, we consider it irrelevant. Afterwards, we 
removed the \totalInjection projects used for creating buggy EAs (see Section~\ref{sss:data}), to avoid doing multiple experiments on the same projects.  
%Due to the time limit, we were unable to experiment with all versions of the remaining 701 projects,  
{Na\"ively, we could experiment with all program versions{,} for each of the remaining 701 projects, to check whether any version has metadata-related bugs and violates our predefined rules. However, it  would take too much time to analyze all versions of each project,} 
%This is because when a project is very large and has thousands of versions kept in its repository, it can be very time-consuming to run \tool with all program versions. 
so we decided to only experiment with a subset of the pool.  
%experiment with a sample set of projects from that pool. 
To ensure the representatives of our experiment and results, we decided to include projects of distinct scales (i.e., small, medium, and large projects). Thus, we ranked the 701 projects in ascending order of Java file counts, as file counts roughly reflect project sizes. We randomly sampled 10 projects for every 100-project interval, and got \totalRealProject projects. 
%Then we located program commits in those projects, which modify file(s) that have items to be checked by \tool. In this way, we did not need to apply \tool to all version of the \totalRealProject projects, but only needed to experiment with the versions where any of the rule-related files are edited.

In our resulting dataset, each project has 1--987 Java files and 5--1068 total files; the mean value of Java files in each project is 68; the median value of Java files is 24.


\subsubsection{Experiment and Results}\label{sec:real-setting}

We applied \tool to different versions of each selected project, to thoroughly explore whether developers committed any mistakes when maintaining metadata and related Java code. Because many projects have long version histories, it can be very time-consuming to apply \tool to all versions of each project. 
Therefore, to accelerate the bug detection procedure, we developed scripts to filter versions, and focus our experiments on versions that edit any Java or XML file containing code/annotation/XML items relevant to the \totalRule rules. 

Our results show that among the selected \totalRealProject software repositories, \tool reported in total \totalReportedInReal bugs in 21 projects. After manually inspecting the bug reports, we found \totalInteresting reports to be true positives, as they violate either $r_1$ (i.e., xml-path-check) or $r_2$ (i.e., bean-class-exists).   
%so we calculated the detection precision as 78\% (121/156). 
{In particular, 18 bugs were detected in the first commits of software repositories; % 102
{99} bugs were found in later commits, meaning that developers accidentally introduced the bugs when they revised software for maintenance.}
{115 of the \totalInteresting bugs were reported together with bugs of the same kind. The existence of multiple bugs in the same program version does not affect our evaluation results, as bugs are analyzed and reported independently. }
% Furthermore, we noticed that \totalRealBugs of the \totalInteresting true positives were already fixed by developers, because we checked later versions of the same projects and those programs got revised  for bug fixing. 

{Furthermore, we noticed that \totalRealBugs of the \totalInteresting true positives were already fixed by developers. We found that out by checking later versions of the same projects and by observing revision of those buggy programs.} The remaining 68 bugs had not been fixed yet, so we filed bug reports to contact developers and seek for their feedback. So far, we have not heard from developers for those bug reports.
There are \totalFalsePositives false positives in the \totalReportedInReal bug reports, because the RegEx patterns we used to identify library classes do not cover all the libraries used by EAs.
%Namely, when an EA references a library class $C$ without defining it and \tool fails to recognize $C$ as a library class, \tool incorrectly reports a mismatch between metadata and code items. Although we tried our best to define RegEx patterns to cover frequently used libraries, there are always libraries that we overlooked but get used by EAs. Thus, such false positives are not fully avoidable for RegEx-based approaches. In the future, people may try to overcome the limitation 

\begin{table}
\footnotesize
    \centering
        \caption{The \totalRealBugs real bugs later fixed by developers}
    \label{tab:real-bug-fix}\vspace{-1.5em}
    \begin{tabular}{r l r R{2cm} r R{3cm}}
\toprule 
    \textbf{Idx}& \textbf{Project Name}&\textbf{\# of Bugs}&\textbf{Ddiff(bug, fix) {(days)}}&\textbf{Vdiff(bug, fix)} &\textbf{Time cost per version ({seconds})}\\ \toprule
    1 & aioweb~\cite{aioweb} & 9 & 1--2 & 1--2&4--6\\ \hline
    2 & angular-js-spring-mybatis~\cite{angular} & 1 & 0 & 1 & 0\\ \hline
    3 & biyam\_repository~\cite{biyam} & 2 &194 & 4&2 \\ \hline
    4 & cv-web~\cite{cv-web} & 2 & 44--47& 1--2&0 \\ \hline
    5 & enterprise-routing-system~\cite{enterprise-routing-system} &2 & 0& 1&5--6 \\ \hline
    6 & FileExplorer~\cite{fe} &1 &0 & 1&3 \\ \hline
    7 & generica~\cite{generica} &2 & 0 &1&2\\ \hline
    8 & I377-esk~\cite{I377} & 1 & 0 &1&7 \\ \hline
    9 & jarvis~\cite{jarvis} & % 3
    {4}& 9--10 &4--5&2--6\\ \hline
    10 & johnsully83\_groovy~\cite{johnsully} &10 &23&2 &17\\ \hline
    11 & Kognitywistyka~\cite{kognity} & 8 & 0 &1--3 &3--4\\ \hline
    12 & LIBRARY~\cite{library} & 2 & 0 & 1&2\\ \hline
    13 & rop~\cite{rop} & % 7 
    {3} &5--% 597
    {556} & 2--% 59
    {27} &7--9\\ \hline
    14 & ShcUtils~\cite{shcutils} & 1 & 1 & 1&0\\ \hline
    15 & spring-vaadin~\cite{vaadin} & 1 & 10 &1&0\\ \bottomrule
    \end{tabular}
\vspace{-2.em}
\end{table}
%, \tool incorrectly interprets the class to be 
%As a result, when a referenced library class is not recognized by \tool to be defined in a library, and it

%With more details, as shown in Table~\ref{tab:real-bug-fix}, 
Table~\ref{tab:real-bug-fix} presents 
the % 53 
\totalRealBugs bugs that developers later fixed.
All these bugs violate $r_2$---the bean-class-exists rule. Namely, each of the bugs references a nonexistent class when declaring a bean.
In Table~\ref{tab:real-bug-fix}, 
 column \textbf{Idx} shows the index we assigned to each buggy project. \textbf{Project Name} lists the projects where the bugs occurred. \textbf{\# of Bugs} counts the total number of bugs we found in different versions of each project. \textbf{Ddiff(bug, fix)} describes the day difference between committing dates of a bug-fixing version and the related bug-introducing version. Similarly, \textbf{Vdiff(bug, fix)} describes the version difference between the bug-fixing version and buggy version. Let us take the second project angular-js-spring-mybatis~\cite{angular} as an example. In one commit $C_i$ checked in on % 01/29/2014
 {Jan 29, 2014}, {developers declared a bean by wrongly referencing a nonexistent class.} In a later commit $C_{i+1}$ checked in on the same day, developers fixed the bug by {correcting the class reference}. Therefore, Vdiff(bug, fix) = (i+1) - i = 1, and Ddiff(bug, fix) = 0. 

For only % 16
{17} of the \totalRealBugs bugs, developers applied fixes on the same day. However, for the remaining % 37
{32} bugs, developers applied fixes either on the following day or after quite a long time. 
{15 of the bugs we detected were fixed 1-10 days after the bug-introducing commits; 17 of the bugs were fixed more than 10 days after the bug-introducing commits. 
The largest time difference we observed between a bug-fixing commit and a buggy version is 556 days. 
The phenomena imply the difficulty of revealing those bugs, and the necessity of our approach.}
Additionally, 
{for 24 of the \totalRealBugs bugs, developers applied fixes in the immediately next commit. However, they fixed the remaining bugs after at least two commits. Most interestingly, developers fixed a bug in rop~\cite{rop} after 
% 59
{27} commits. 
These observations indicate the great benefits developers can potentially get out of the \tool usage.  
Namely, if developers had used \tool to examine their software before committing program changes, they should have avoided checking in the erroneous program changes, or even have fixed the introduced bugs earlier. 

%\color{red} Additionally, based on our experiments for \ref{ss:eval-effect-mecheck}, all the rule violations are severe bugs that trigger runtime errors instead of compilation errors. It means that without any tool support, developers have to wait till the testing phase, to reveal those rule violations, making the bugs critical. \color{black} 

Finally, \tool has very low runtime cost. We experimented with a laptop that has (1) an Intel i7-8565U CPU with four cores and eight logical processors, and (2) 15.9 GB memory. 
As shown in Table~\ref{tab:real-bug-fix}, when \tool was applied to detect bugs based on the \totalRule rules, it spent no more than 6 seconds on each program version. 

\vspace{0.5em}
\noindent\begin{tabular}{|p{13.6cm}|}
	\hline
	\textbf{Finding 3 (Response to RQ3):} \emph{\tool demonstrated great effectiveness and high performance when being applied to different versions of 70 real-world open-source EAs. It reported in total \totalReportedInReal bugs; \totalInteresting of the bugs are true positives, \totalRealBugs of which have been already fixed by developers
 %It managed to detect bugs with \precision precision, \recall recall, and \fscore F-score. 
 }
	\\
	\hline
\end{tabular}
%\subsection{Fixed Bugs in History Commits}\label{ss:eval-history}

Instead of checking every single commit in the projects, we applied a filtering. We crawled commits where the commit message has 1) bug or fix or error, and 2) metadata or xml or annotation in it. We also made sure that the commits involved changes in .java or .xml files.
\todo{there are some earlier cases where these filterings were not applied}

As we mentioned in \ref{ss:eval-effect-mecheck}, we found \totalInteresting real bugs from \totalInterestingEAs projects. After also checking the latest version of those repositories for the \totalInteresting bugs, we found out that 7 of them has never been addressed by the developers. The other 18 bugs have been fixed or addressed by the developers in later commits.

\vspace{0.5em}
\noindent\begin{tabular}{|p{8.1cm}|}
	\hline
	\textbf{Finding 3:} \emph{There are metadata-related bugs in real-world projects. Most of the bugs (18 out of 24) \tool detected in version history have been fixed by the developers in later commits and they are no longer available in the latest versions.}
	\\
	\hline
\end{tabular}