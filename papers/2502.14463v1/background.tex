\vspace{-1.em}
\section{Background}\label{se:background}

To facilitate comprehension, 
this section will explain the three alternative ways of metadata-based EA configuration (Sections~\ref{ss:xml}--\ref{ss:xanda}), and our research problem
%and the challenges to overcome in our research 
%why it is challenging to use metadata properly 
(Section~\ref{ss:problem}).

\vspace{-0.5em}
\subsection{XML-Based 
Configuration}\label{ss:xml}
There is a special kind of XML files named \textbf{deployment descriptors (DDs)}~\cite{DD_javaEE}, which are frequently used in EAs for configuration purposes. According to the XML syntax, 
\textbf{XML elements} are the basic building blocks of XML files~\cite{xml-elements}. %Each XML file has one \textbf{root element} that is the parent or ancestor of other elements. 
%; all XML elements are organized in a tree structure. 
Each element is used as a container to store text content, other XML elements, or attributes. The syntax of XML elements is:

\codefont{<element-name attributes> Contents ...</element-name>} 

\noindent 
%where \textbf{element-name} is the name of an element, and \textbf{attributes} are used to define the element properties. 
The multiple attributes of any element are separated by white spaces, and each attribute associates the attribute name with a string value. For simplicity, we use \textbf{XML items} to refer to XML elements and their attributes.
% and the text content (or string values) of elements and attributes. 
For instance, Fig.~\ref{fig:examples} (a)  shows an exemplar DD that defines two bean objects: \codefont{infoMessage} and \codefont{messageRenderer} (see lines 3--6). The second bean has a \codefont{<constructor-arg>} element (line 6), which references the first bean as the value of its attribute \codefont{ref}. %For such XML-based configuration, one rule developers should observe is: \emph{The \codefont{ref}-attribute value of \codefont{<constructor-arg>} should correspond to the id of a bean object.}

\begin{figure}
    \centering
    \vspace{-1em}
    \includegraphics[width=\linewidth]{images/three-alternatives.pdf}
    \vspace{-2.5em}
    \caption{{Three alternative ways to implement the same bean configuration}}
    \label{fig:examples}
    \vspace{-1.5em}
\end{figure}
\begin{comment}
\begin{lstlisting}[label = {lst:examples}, caption = Three alternative ways to implement the same bean configuration]
(*\bf (a) XML-based configuration*)
<?xml version="1.0" encoding="UTF-8"?>
<beans xmlns="http://www.springframework.org/schema/beans" ... >
  <bean id = "infoMessage" class="InfoMessage"/>
  <bean name="messageRenderer" class="MessageRenderer">
    <constructor-arg name="message" ref="infoMessage"/>
  </bean>
</beans>

(*\bf (b) Annotation-based configuration*)
import org.springframework.context.annotation.*; 
@Configuration
public class AppConfig {
  @Bean
  public InfoMessage infoMessage() {
    return new InfoMessage();
  }
  @Bean
  public MessageRenderer messageRenderer() {
    return new MessageRenderer(infoMessage());
  } }
  
(*\bf (c) Configuration based on both XML and annotations*)  
my-bean.xml:
<?xml version="1.0" encoding="UTF-8"?>
<beans xmlns="http://www.springframework.org/schema/beans" ... >
  <bean id = "infoMessage" class="InfoMessage"/>
</beans>

AppConfig.java:
import org.springframework.context.annotation.*;
@Configuration
@ImportResource(location = {"classpath:my-bean.xml"})
public class AppConfig {
  @Bean
  public MessageRenderer messageRenderer() {
    ApplicationContext context = new AnnotationConfigApplicationContext(ApplicationConfig.class);
    return new MessageRenderer(context.getBean("infoMessage"));
  } }
\end{lstlisting}
\end{comment}

\vspace{-0.5em}
\subsection{Annotation-Based Configuration}\label{ss:a}
Recently, more and more Java frameworks provide annotations as a substitute for deployment descriptors~\cite{javaee-anno,spring-anno}. Annotations allow EA developers to specify component configurations within Java classes. 
An annotation can be attached to a Java class, method, or field. When annotations have attributes, the related syntax can be ``\codefont{@annotation-name("value")}'' or ``\codefont{@annotation-name(key = \{values\})}'', where \textbf{key} is the attribute name. 
 Fig.~\ref{fig:examples} (b) presents an annotation-based alternative to the DD shown in Fig.~\ref{fig:examples} (a). Specifically, \textbf{\codefont{@Configuration}} (line 9) implies that the current Java class is equivalent to a DD. Inside the class, \textbf{\codefont{@Bean}} (lines 11 and 15) is a direct analog of the XML \codefont{<bean>} element. \textbf{\codefont{@Bean}} 
means that the Java method it annotates will return an object that should be registered as a bean in the application context; the registered bean name is the method name (e.g., \codefont{infoMessage}). The second bean references the first bean by invoking \codefont{infoMessage()} (lines 15--18). %A typical rule related to annotation-based configuration can be: \emph{When inter-bean dependencies are expressed by having one bean method call another, both mean methods must be declared in a class annotated by @Configuration~\cite{inter-bean}.}

%\emph{@Bean can only annotate Java methods}. 
% \todo{\emph{@Bean can only annotate the Java methods inside classes annotated with either @Component or @Configuration.}}

\vspace{-0.5em}
\subsection{Combining XML Files with Annotations}\label{ss:xanda}

With both XML- and annotation- based configuration methods available, developers can take a hybrid approach between the two. Namely, given an EA, developers can configure part of the application with XML files and configure the remaining with annotations;  
the two types of configuration interact with each other in specialized ways. 
%, which also pose unique requirements on metadata usage. 
For instance, Fig.~\ref{fig:examples} (c) implements the XML-based configuration presented by Fig.~\ref{fig:examples} (a) with two files: {\sf my-bean.xml} defines the bean object \codefont{infoMessage} (line 21), and {\sf AppConfig.java} defines the other bean \codefont{messageRenderer} (lines 27--31). The Java class uses \codefont{@ImportResource} to import the XML file and to access all beans defined there. 
% To correctly use such a hybrid method, developers should at least observe the rule below: \emph{The parameter of method  \codefont{context.getBean(...)} must correspond to a bean either defined in XML files or marked by \codefont{@Bean}.}

\vspace{-0.5em}
\subsection{Problem Statement}
\label{ss:problem}

As reflected by the examples so far, there are diverse metadata-usage rules that EA developers should follow. These rules can vary with the adopted software frameworks, developers' configuration needs, and configuration methods; many of the rules require for content consistency among code elements, XML items, and annotations.
 It is challenging for EA developers to correctly memorize and follow all domain-specific rules; it is even harder for developers to consistently maintain the metadata usage scattered in multiple files. A recent study on StackOverflow~\cite{meng2018secure} shows that numerous developers posted questions on how to properly configure frameworks.
 %(e.g., Spring Security), and expressed significant concerns on the debugging of XML files and annotations.
There is a desperate need for approaches and tools that can help with metadata debugging. 
 %The necessity and usefulness of static code analysis tool has been considered for building enterprise applications effectively.
To create an automatic static checker that can satisfy developers' need, we encountered two technical challenges: (1) how to cope with the diversity of domain-specific rules, and (2) how to extract and relate the content from different software artifacts to validate metadata usage?

\begin{comment}
\begin{itemize}
\item How can we cope with the diversity of domain-specific rules?
\item How can we extract and relate the content from different software artifacts to validate metadata usage?
\end{itemize}
\end{comment}
