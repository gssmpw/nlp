\vspace{-.5em}
\section{Conclusion}\label{se:conclusion}

As with proper code implementation, correct metadata usage is always essential to software quality. Albeit the importance of high-quality metadata usage, widely used program analysis techniques (e.g., WALA~\cite{wala} and Soot~\cite{soot}) barely help with metadata debugging.
Some existing tools can help developers identify certain metadata bugs, but the tool support is quite limited. 
In this paper, we developed RSL---a domain-specific language for describing metadata-usage checking rules, and built \tool---a tool to analyze programs based on RSL rules. Compared with existing tools, our approach is unique in three aspects. First, RSL enables library/framework developers to freely specify domain-specific constraints on the relations among code, annotations, and XML items; thus, it empowers \tool to enforce diverse rules in a wider scope and equips \tool with a strong power extensibility. 
Second, \tool extracts program data from diverse software artifacts (e.g., Java source files and XML files), and reasons about the relationship between the extracted data to detect bugs. It demonstrates a great way of integrating source code analysis with metadata analysis. Third, \tool applies optimizations to reduce redundant data processing, when trying to establish pattern matching between a given program and multiple bug patterns described by RSL rules. No prior work pays such a close attention to the time cost of metadata-related bug detection.

There is still significant space for future improvements in static analysis for metadata usage. 
In the future, we will investigate more domain-specific rules posed by different library frameworks, and extend \tool to conduct more advanced synergetic analysis among code and metadata.

%run the script in RSL. By constructing the rule in RSL, \tool can identify rule violation and report bugs. Unlike prior works, \tool and RSL are self-contained, since they do not rely on other implementation. \\


%\totalRule rules are extracted from Spring and JUnit specification and are written in RSL. To evaluate the effectiveness and correctness of \tool, we collected a dataset of \totalProject projects. From there, we created a ground-truth dataset by injecting \totalInjection bugs for the \totalRule rules. \tool managed to find all of the injected bugs. Furthermore, we looked for real-world bugs in the main dataset, and found \totalInteresting bugs in the commit history of \totalInterestingEAs projects. Among them, 7 have never been addressed by the developers but 18 of them have later been fixed and are no longer available in the latest versions.

%There is still significant space for future improvements in static analysis for metadata usage. Our current experiments only focus on \tool rules in Spring specification. However, it is still interesting to know the bug evolution of other rules. As the future work, we will explore more rules in Spring or other web frameworks and apply \tool to more diverse projects. Meanwhile, it will be further insightful to find real-world bugs of all rule types. We need to run more experiments for that. Besides, we also envision \tool to be used in Integrated Development Environment (IDE) for web applications.