\vspace{-.5em}
\section{A Motivating Example}\label{se:example}

This section uses an example to intuitively explain our research. The software framework Spring~\cite{spring} supports developers to \emph{configure beans to have initialization and cleanup methods}~\cite{init-destroy}. 
Namely, a ``bean'' is 
any plain-old Java object that follows standard configuration patterns; it can be defined in Java or XML files.
%, and is always an instance of certain Java class.  
Suppose that an XML file declares bean \codefont{b} as an instance of Java class \codefont{C}, and  
%bean \codefont{b} is an instance of Java class \codefont{C}. 
sets the bean's attribute ``\codefont{init-method}'' or ``\codefont{destroy-method}''  to a Java method name. Then the method must exist in the corresponding Java class \codefont{C}. This is because when the attribute is set, Spring automatically calls the corresponding method during runtime to either initialize 
\codefont{b} after the bean is created, or  perform destruction tasks before the bean is destroyed.
%the class can define an initialization method to initialize \codefont{b}'s content after the bean is created. When the bean's attribute ``\codefont{init-method}'' in XML is set to the name of that initialization method, the framework automatically calls that method during runtime after the bean is created. 
%Thus, a metadata-usage constraint is whenever a method name is mentioned in the XML-based bean configuration as the initialization method, the corresponding method must exist in the Java class.

\begin{figure}[h]
\vspace{-1.em}
    \centering
    \includegraphics[width=\linewidth]{images/java-xml-example.pdf}
    \vspace{-2.5em}
    \caption{{A bean object in XML has attribute \codefont{init-method} refer to a Java method in the corresponding class}}
    \vspace{-1.em}
    \label{fig:example}
\end{figure}

\begin{comment}
\begin{figure}
   \begin{minipage}{.5\linewidth}
       
   \end{minipage}
   \hfill
   \begin{minipage}{.49\linewidth}
       
   \end{minipage}
\end{figure}

\vspace{-1em}
\begin{lstlisting}[label={lst:example},caption=An exemplar XML file and related Java code]
(*\bf Beans.xml:*)
<?xml version="1.0" encoding="UTF-8"?>
<beans xmlns="http://www.springframework.org/schema/bean"... >
  <bean id = "b" class = "C">
    <property name = "foo" value = "Hello!"/>
    <property name = "bar" value = "Bye!"/>
  </bean>  
</beans>

(*\bf C.java:*)
public class C {
  private String foo;
  private String bar;
  // this setter method can set the "foo"-property value for bean "b" at runtime
  public void setFoo(String v) { this.foo = v; }
  // (*\bf a setter method corresponding to the "bar" field is missing*)
}
\end{lstlisting}
\end{comment}

As shown in Fig.~\ref{fig:example}, the exemplar XML file sets the attribute \codefont{init-method} of bean \codefont{b} to ``\codefont{myPostConstruct}'' (line 3). Ideally, EA developers should define a corresponding method in the exemplar Java class---\codefont{myPostConstruct()}. 
However, when EA developers fail to do so (see line 11), no existing compiler or static checker can reveal such errors of omission. 
As a result, developers can only observe the consequence of this metadata-related bug during program execution, and then manually diagnose root cause for the triggered runtime error \codefont{org.springframework.beans.factory.BeanCreationException}.

\vspace{-1em}
\begin{lstlisting}[label={lst:example-rsl},caption={An RSL specification {for detecting} missing methods}]
Rule method-exists {
  for (file xml in getXMLs()) {
    for (<bean> bean in getElms(xml, "<bean>")) {
      String beanClassFQN = getAttr(bean, "class");
      if (classExists(beanClassFQN)) {
        class c = locateClassFQN(beanClassFQN);
        for (String s in getAttrs(bean, "*method")) {
          assert(exists(method m in getMethods(c))(getName(m) == s)) {
            msg("The referenced method: %s in bean: %s is not defined in class: %s", s, getName(bean), getFQN(c));}}}}}} 
\end{lstlisting}

To help developers detect metadata-related bugs earlier and more easily, we developed RSL for framework developers to prescribe metadata-checking rules. Specifically, to statically detect the missing \codefont{init-method} mentioned above, framework developers can define an RSL rule. As shown in Listing~\ref{lst:example-rsl},  %following rule in RSL.
intuitively, the rule describes four major things:  
\begin{itemize}
\item \textbf{What metadata/code items are involved?} Given a software project (i.e., EA), locate all $\langle bean\rangle$ objects defined in XML files (see lines 2--{3}).
\item \textbf{How are these items refined?} Associate the $\langle bean\rangle$ objects with their corresponding Java class declarations, and focus on such $\langle bean\rangle$-class pairs (lines {4}--6).
\item \textbf{What is the consistency constraint?} For attribute \codefont{<*method>} of any $\langle bean\rangle$ (i.e., \codefont{<init-method>} or \codefont{<destroy-method>}), a matching method should be 
%there should be a matching method 
defined in the related class (lines 7--8). 
\item \textbf{What is the error message?} When the constraint is violated and there is a <*method>-attribute value not matching any method definition, a bug should be reported (line 9). 
\end{itemize} 
Our new tool \tool can take in the above-mentioned RSL rule to statically check EAs. It can reveal the $\langle *method\rangle$  misconfiguration shown in Listing \ref{lst:example-rsl}, before EA developers run that program.