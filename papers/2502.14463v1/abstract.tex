\begin{abstract}
When building enterprise applications (EAs) on Java frameworks (e.g., Spring), developers often configure application components via \emph{metadata} (i.e., Java annotations and XML files). It is  challenging for developers to correctly use metadata, because the usage rules can be complex and existing tools provide limited assistance.
When developers misuse metadata, EAs become misconfigured, which defects can trigger erroneous runtime behaviors or introduce security vulnerabilities. 
To help developers correctly use metadata, this paper presents (1) RSL---a domain-specific language that domain experts can adopt to prescribe metadata checking rules, and (2) \tool---a tool that takes in RSL rules and EAs to check for rule violations. 

%\vspace{.5em}
With RSL, domain experts (e.g., developers of a Java framework) can specify metadata checking rules by defining content consistency among XML files, annotations, and Java code. Given such RSL rules and a program to scan, \tool interprets rules as cross-file static analyzers, which analyzers scan Java and/or XML files to gather information and look for consistency violations. 
For evaluation, we studied the Spring and JUnit documentation to manually define \totalRule rules, and created 2 datasets with \totalProjectIncluded open-source EAs.
%. We then mined \totalProject open-source EAs from GitHub to prepare two datasets. 
The first dataset includes \totalInjectedProject EAs, and the ground truth of \totalInjection manually injected bugs. %We used this synthetic dataset to assess the bug-detection  of \tool. 
The second dataset includes multiple versions of \totalRealProject EAs. 
%our dataset. To check the accuracy of \tool, we injected \totalInjection bugs in randomly selected \totalInjectedProject projects from our dataset. Finally, to determine \tool's ability to detect real bugs and to see how developers address the bugs in real life, we ran analysis on different versions of the dataset EAs till \tool reports at least 25 bugs. After that, we manually verified the reported real bugs.
We observed that \tool identified bugs in the first dataset with \precision precision, \recall recall, and \fscore F-score. It reported \totalReportedInReal bugs in the second dataset, \totalRealBugs of which bugs were already fixed by developers.
Our evaluation shows that \tool helps ensure the correct usage of metadata. 

\end{abstract}
