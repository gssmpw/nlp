\vspace{-1em}
\section{Threats to Validity}\label{se:threats}


\paragraph{Threat to External Validity}
We leveraged RSL to express \totalRule rules summarized for Spring and JUnit, and applied \tool to in total 115 (\totalInjectedProject + \totalRealProject) open-source projects for bug detection. The rule set may be limited to our experience with Java frameworks, while the experiment results can be limited to our project datasets. In the future, we would like to include more rules and more projects into our evaluation, or even include close-source projects if possible, so that our findings are more representative. 

\vspace{-.5em}
\paragraph{Threat to Construct Validity} Although we tried our best to manually inspect bug reports, it is possible that our manual analysis are subject to human bias and restricted by our limited domain knowledge. To mitigate the problem, we sent emails to developers who owned the open-source projects and asked whether a reported rule violation makes sense or not. So far, we have not received much feedback from those developers. As we gather more comments from these domain experts, we can further improve the quality of bug detection.
%inferred rules and change suggestions. 

\vspace{-.5em}
\paragraph{Threat to Internal Validity} 
Currently, \tool determines whether a given class $C$ is defined by a library dependency of EA using pattern matching. Namely, if the fully qualified name of $C$ matches a RegEx pattern predefined in \tool, it is a library class. However, our RegEx pattern set may not be comprehensive; some library classes may fail the matching process and get wrongly treated as non-library classes. One way to overcome this limitation is to eagerly scan the bytecode of all library dependencies, trying to find an exact match for $C$'s name. However, we decided not to perform such a heavyweight scanning for library classes because based on our experience, many open-source projects do not contain all JAR files for their library dependencies. Such a lack of dependency information will considerably limit the effectiveness of a heavyweight scanning and the applicability of \tool.
In the future, we will investigate more advanced lightweight approaches to mitigate this issue.


%We currently leveraged the heuristic that associated XML entities should (1) frequently co-exist in the same files and (2) refer to the same data value. However, some associated entities may not satisfy these criteria. For instance, certain related entities may not co-exist in many files due to the limited usage of their corresponding software frameworks. Furthermore, some relevant entities do not refer to the same value even though they co-occur a lot. 
%Consequently, the extracted rules are limited by our heuristic. In the future, we will mitigate such problems by defining more heuristics. 