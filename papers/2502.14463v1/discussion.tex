%\vspace{-1em}
{\section{Discussion}}

This section discusses the importance of our \totalRule rules, bug criticality, potential application scope of \tool, the necessity of defining new rules, and a potential alternative design as well as implementation of \tool.

\subsection{Rule Importance and Bug Criticality}

The \totalRule rules we investigated are important, because
 we extracted them from the documentation of Spring and JUnit. All the real bugs we revealed using those rules are critical and severe, as they all trigger runtime instead of compilation errors. It means that without any tool support, developers have to wait till the testing phase, to reveal those rule violations. Additionally, developers spent lots of time revealing some of the bugs we found. As mentioned in Section~\ref{ss:eval-real-world}, 15 of the bugs we detected were fixed 1-10 days after the bug-introducing commits; 17 of the bugs were fixed more than 10 days after the bug-introducing commits. The phenomena imply the difficulty of revealing those bugs, and the necessity of our approach.

According to our experiment for Section~\ref{ss:eval-real-world}, nevertheless, not all rules were violated by real-world projects. One possible reason can be that some developers fixed those violations before checking in commits.
Even if we only observed violations of $r_1$ and $r_2$ in our own real-world experiment, the online resources listed in Table~\ref{tab:rule-and-violate} show that people violated or are likely to violate the remaining rules. Thus, the rules are important as developers tend to violate them.

\vspace{-.5em}
\subsection{The Potential Application Scope of \tool}
So far, we have defined \totalRule RSL rules based on the documentation of Spring and JUnit. However, the application scope of \tool is not limited to these \totalRule rules.
To use \tool, developers can also define their own rules based on other metadata-related constraints~\cite{inspect} or their manual rule extraction from other libraries/frameworks. As XML and annotations have been prevalent configuration methods in various areas and well-known libraries/frameworks (see Table~\ref{tab:area-and-lib}), we believe that \tool can be applied to a wider scope than what is demonstrated in this paper. Essentially, \tool is applicable to arbitrary Java projects that use either XML-based configuration only, annotation-based configuration only, or both XML-based and annotation-based configurations.

\begin{table}
\scriptsize
\begin{minipage}{.53\linewidth}
\caption{The potential areas and libraries/frameworks where \tool is applicable}\label{tab:area-and-lib}
\vspace{-1.5em}
\begin{tabular}{p{2.8cm}l}
\toprule
\textbf{Area} &\textbf{Exemplar Libraries or Frameworks} \\
\toprule
Enterprise Applications & JavaEE \\ \hline
Testing Frameworks &JUnit, TestNG\\ \hline
Dependency Injection and Application Frameworks &Spring Framework, Camel, Guice\\ \hline
Object-Relational Mapping & Hibernate, MyBatis\\ \hline
Web Development & Spring MVC, Struts, JSF\\ \hline
(De)Serialization & Jackson, Gson\\ \hline
Build Tools & Maven, Ant\\ \hline
UI Layouts & Android\\ \hline
Security & Spring Security\\ \hline
Microservice Frameworks & Quarkus, Micronaut \\\bottomrule
\end{tabular}
\end{minipage}
%\vspace{-2em}
\hfill
\begin{minipage}{.4\linewidth}
\caption{The additional rule violations that get reported or discussed by developers}\label{tab:rule-and-violate}
\vspace{-1.5em}
\centering
\begin{tabular}{l|R{2.5cm}}
\toprule
\textbf{Rule} &\textbf{Violations Reported or Discussed} \\ \toprule
$r_3$     & \cite{r3-violate,r3-violate-2,r3-violate-3} \\ \hline
$r_5$ &\cite{r5-violate}\\ \hline
$r_7$ &\cite{r7-violate}\\ \hline
$r_8$ &\cite{r8-violate, r8-violate-2}\\ \hline
$r_{12}$ &\cite{r12-violate,r12-violate-2}\\ \hline
$r_{13}$ &\cite{r8-violate}\\ \hline
$r_{14}$ &\cite{r14-violate}\\ \hline
$r_{15}$ &\cite{r15-violate}\\ \bottomrule
\end{tabular}
\end{minipage}
\vspace{-1em}
\end{table}

%\vspace{-2em}
\subsection{The Necessity of Defining New Rules}

We foresee that developers are motivated to define new rules, when they use \tool to examine metadata usage. Three reasons help explain the motivation. 
First, many existing rules~\cite{inspect} are expressible with RSL and checkable by \tool. It indicates a strong need for developers to extend \tool's current 15-rule set to cover those known rules.
Second, prior work~\cite{meng2018secure} shows that when developers asked questions on StackOverflow concerning Spring security usage, the majority of questions are on metadata-based configurations. Such phenomena imply that (1) existing work provides insufficient tool support and (2) some delicate constraints are undocumented. Therefore, it is almost impossible to define a comprehensive ruleset to capture all rules in the wild today. Developers need to extend the ruleset after revealing previously unknown or hidden rules.
Third, as XML and Java annotations are widely used for library/framework configurations, it is likely that future software deriving from these libraries/frameworks will inherit the configuration methods but define new domain-specific rules. Even if we can define a comprehensive ruleset for \tool today, as the time goes, new domain-specific rules appear; developers still need to expand \tool's ruleset  to cover those rules. Therefore, it is necessary and important to define RSL for rule definition.

\subsection{A Potential Alternative Design and Implementation of \tool}
Existing static analysis tools like PMD~\cite{pmd} are created to detect bugs in Java code. Some readers may wonder why we did not extend those tools to define metadata-related rules. We thought about that option but decided to define our own DSL and create a new tool, mainly because the metadata-related rules we focus on are so unique that existing tools barely handle them. Take PMD as an example. PMD does not examine any of the rules listed in this paper. To extend the ruleset of PMD for XML analysis, people have to learn and use XPath---another DSL---to define queries on XML documents. However, XPath does not support queries on source code or annotations. It has a narrower scope than RSL. Furthermore, PMD is complex, with lots of implementation irrelevant to our focus, which can make our tool development on top of PMD very time-consuming and error-prone. To (1) avoid dealing with the complexity of PMD and (2) quickly prototype our research, we created \tool without reusing PMD or XPath.