

  \vspace{-.5em}
\subsection{Effectiveness of RSL}\label{ss:eval-effect-rsl}

To assess the expression power of RSL, we studied (1) the documentation of Spring \cite{spring}---the most popular application development framework for EAs in Java, and (2) documentation as well as tutorials on JUnit~\cite{junit-doc,junit-tutorial,how-to-junit,software-testing}---the most popular framework for unit testing in Java. 
We manually distilled \totalRule metadata-usage constraints from those documents. As shown in Table \ref{tab:constraints}, seven constraints are about the correspondence between XML and code ($r_1$--$r_7$); six constraints involve annotations and code ($r8$--$r13$); one constraint is on the correspondence between XML and annotation ($r14$); one constraint is relevant to XML, annotation, and code.
These constraints are diverse in terms of the metadata/code items involved, the number of files scanned, and the consistency-checking logic. We successfully expressed constraint-checking rules in RSL, which fact  evidences the great expression power of our domain-specific language. In the following experiments, we will leverage \tool to generate analyzers from these rules, and evaluate \tool's effectiveness in identifying buggy EAs that violate any of the rules.

\vspace{0.5em}
\noindent\begin{tabular}{|p{13.6cm}|}
	\hline
	\textbf{Finding 1 (Responding to RQ1):} \emph{RSL has great expressiveness. It is capable of describing a variety of metadata-usage checking rules that involve XML, annotation, and/or code.}
	\\
	\hline
\end{tabular}




