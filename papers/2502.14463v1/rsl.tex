\vspace{-.5em}
\section{Rule Specification Language (RSL)}\label{se:rsl}

To overcome the challenges mentioned above, we designed a DSL---RSL---for framework developers to specify metadata-usage checking rules. The RSL rules will serve two purposes. First, they can help EA developers (i.e., framework users)  understand the content consistency among metadata and code items. Second, they will be sent to \tool for automatic detection of metadata-related bugs. 
As illustrated in Fig.~\ref{fig:syntax}, the RSL syntax defines statements, expressions, and built-in functions.

\vspace{-0.5em}
\subsection{Statements}
RSL supports five types of statements;  
each statement contains simple or complex expressions, or other statements. 
With these statement types, an RSL specification or rule describes four important aspects of the usage-checking logic:
extracting relevant metadata/code items, refining or filtering items, checking for consistency, and reporting errors.

\begin{comment}
\begin{figure}[h]
\footnotesize
\vspace{-1em}
\begin{align*}
\textrm{Specification} := &\;\textbf{Rule}\ {\tt Id}\ \textrm{Block} \\
\textrm{Block} := &\;'\{' \textrm{Stmt}\ \textrm{Stmt}* '\}' \\
\textrm{Stmt} := &\;\textrm{ForStmt}\ |\ \textrm{IfStmt}\ |\ \textrm{AssertStmt}\ |\ \textrm{DeclStmt}\ ';' \\
%\textrm{StmtSuffix} := &\;\textrm{Stmt}\ \textrm{StmtSuffix}? \\
\textrm{ForStmt} := &\;\textbf{for}\ '('\ \textrm{Type}\ {\tt Id}\ \textrm{in}\ \textrm{Exp}\ ')'\ \textrm{Block} \\
\textrm{IfStmt} := &\;\textbf{if}\ '('\ \textrm{Exp}\ ')'\ \textrm{ThenBlock}\ \textrm{ElseBlock}?\\
\textrm{ThenBlock} := &\; \textrm{Block} \\
\textrm{ElseBlock} := &\; \textbf{else}\ \textrm{Block}\\
\textrm{AssertStmt} := &\;\textbf{assert}\ '('\ \textrm{Exp}\ ')'\ '\{'\ \textrm{MsgStmt}\ ';'\ '\}' \\
\textrm{MsgStmt} := &\;\textbf{msg}\ '('\ ','\ \textrm{SimExp}\ (','\ \textrm{SimExp})*\ ')'\\
\textrm{DeclStmt} := &\;\textrm{Type}\ {\tt Id}\ '='\ \textrm{Exp}\\
\textrm{Exp} := &\;\textrm{SimExp}\ |\ \textrm{SimExp}\ \textbf{AND}\ \textrm{Exp}\ |\  \textrm{SimExp}\ \textbf{OR}\ \textrm{Exp}\ | \textbf{NOT}\ \textrm{Exp} \\
\textrm{SimExp} := &\;\textrm{Id}\ |\ \textrm{Lit}\ |\ \textrm{FunctionCall}\ |\ '('\ \textrm{Exp}\ ')'\ |\ \textrm{FunctionCall}\ '==' \textrm{SimExp}\ |\ \textbf{exists}\ '('\ \textrm{Type}\ {\tt Id}\ \textbf{in}\ \textrm{Exp}\ ')'\ '('\ \textrm{Exp}\ ')' \\
\textrm{Type} := &\;'\langle'\ {\tt Id}\ '\rangle'\ |\ \textbf{file}\ |\ \textbf{class}\ |\ \textbf{method}\ |\ \textbf{field}\ |\ \textbf{String} \\
\textrm{Lit} := &\;\textrm{StringLit}\ |\ \textrm{CharLit}\ |\ \textrm{IntLit}\ |\ \textrm{FloatLit} \\
\textrm{FunctionCall} := &\;{\tt Id}\ '('\ \textrm{Params}\ ')' \\
\textrm{Params} := &\;\textrm{SimExp}\ (','\ \textrm{SimExp})* \\
\end{align*}
\vspace{-1.5em}
\caption{Core syntax of RSL}\label{fig:syntax}
\vspace{-1.em}
\end{figure}
\end{comment}

(1) \textbf{ForStmt} means \codefont{for}-loop, used to describe \emph{what metadata/code items to extract and enumerate in a software application}. 
 %It provides a compact way to extract and iterate over a set of metadata/code items. 
 Users can adopt a single or nested loop to structure the extraction and handling of all items potentially relevant to a constraint.  As shown in  Listing~\ref{lst:example-rsl}, \codefont{for(file xml in getXMLs())\{...\}} means to get all XML files in the project, iterate over those files using the variable \codefont{xml}, and process each iteration as instructed by the \codefont{for}-loop body. 
 %At each iteration, the variable \codefont{property} will have a different value. The iteration variable and all other variables inside the \codefont{for}-loop body will be stored in the scope frame (\ref{sss:rsl-frame}), and can be accessed using the variable name, whenever necessary.

 (2) \textbf{IfStmt} is conditional \codefont{if}-statement. It describes \emph{how to refine items or locate items of interest}. Namely, when an iterated item satisfies the specified \codefont{if}-condition, it is processed by the body. %\codefont{then}-branch; otherwise, the item is processed by \codefont{else}-branch when that branch exists. 
 %when the item does not satisfy the condition and the \codefont{else}-branch is defined, the item is processed by \codefont{else}-branch; otherwise, the item is filtered out. 
For instance, 
in Listing~\ref{lst:example-rsl}, \codefont{if(classExists(beanClassFQN))\{...\}} checks whether a Java class with the same fully-qualified name as \codefont{beanClassFQN} exists in the source code. If so, 
%\codefont{then}-branch 
the body first locates that class via \codefont{class c = locateClassFQN(beanClassFQN)}, and then
checks content correspondence between the Java class and XML file. Otherwise, the \codefont{bean}-object is skipped for further processing.  
This is because when the class is not defined in source code 
 (e.g., defined by a third-party library or by another languaged program), the content checking can become overly complicated or even infeasible; thus,  
 %can become overly complicated or even infeasible and thus we decided to quit checking on the 
 %may fail and 
 we decided to quit checking on the 
 \codefont{bean}-object to avoid false positives.
 %our approach stops further checking to avoid false positives.
 %the source code is not always available and thus content checking is not guaranteed to succeed.
 %allows the script to have conditional lines. The \codefont{if} statements requires a condition to pass in, and a \codefont{if}-body which will be executed if the condition matches.

(3) \textbf{AssertStmt} means assertion, to describe \emph{what constraint to check}. Each statement has two parts: a condition and the body. 
The condition is a simple or complex expression, to define constraints or predicates that items must satisfy. The \codefont{assert}-body is defined with a MsgStmt (see below), to express action-to-take when the condition is not satisfied.
In Listing~\ref{lst:example-rsl}, the assertion means that given an attribute value of \codefont{<*method>}, there should be a Java method with the specified name. 
%whose name (1) starts with ``set'', and (2) ends with the property's name when ignoring the upper/lower case.

%Users can adopt AssertStmt to express constraints, by declaring the content correspondence among refined and related items or specifying conditions that refined items must satisfy. 
%by providing conditions. That condition can be 1) a boolean value returned from a function call, variable, and/or a chain of function calls, and variables, 2) \codefont{exists}-clause, that allows the script to check condition on a collection. The \codefont{exists}-clause is similar to for statements, but different in a way that it does not require to iterate the whole collection. It will stop right away if the condition matches, and will consider the assertion to pass. 

 (4) \textbf{MsgStmt} is message statement to describe \emph{what error message to report when a bug is detected}. People can use MsgStmt to compose an error-message template, and offer expressions or values to instantiate the template. 
 The MsgStmt in Listing~\ref{lst:example-rsl} incorporates a Java method name, bean name, and class name to pinpoint the issue of 
 %In Listing \ref{lst:example-rsl}, the MsgStmt expresses that the Java method, bean name, and class name should be provided when \tool pinpoints the issue of 
 a misconfigured \codefont{<init-method>} or \codefont{<destroy-method>}.

 (5) \textbf{DeclStmt} is declaration statement---an auxiliary statement to facilitate rule definition. Such a statement declares a variable with its data type, and initializes the variable using an expression. With such statements,
 recurring expressions only need to be evaluated once, as the evaluated value can be passed to a user-defined variable and that variable can get used to replace multiple occurrences of the same expression.
 %referred to by multiple places.
 %is used to declare variables for the current scope. It requires a data-type, a variable name, and the value for the variable. 
 For instance, Listing~\ref{lst:example-rsl} has a DeclStmt to declare variable \codefont{c} that 
 %to declare variables \codefont{c}
 %and \codefont{propertyName}. Variable \codefont{c} 
 holds the located Java class item, given a \codefont{bean}-class name specified in XML. 
 %\codefont{propertyName} holds the name of \codefont{<property>} in each iteration.

\begin{figure}[h]
\footnotesize
\vspace{-1em}
\begin{align*}
\textrm{Specification} := &\;\textbf{Rule}\ {\tt Id}\ \textrm{Body} \\
\textrm{Body} := &\;'\{' \textrm{Stmt}\ \textrm{Stmt}* '\}' \\
\textrm{Stmt} := &\;\textrm{ForStmt}\ |\ \textrm{IfStmt}\ |\ \textrm{AssertStmt}\ |\ \textrm{DeclStmt}\ ';' \\
%\textrm{StmtSuffix} := &\;\textrm{Stmt}\ \textrm{StmtSuffix}? \\
\textrm{ForStmt} := &\;\textbf{for}\ '('\ \textrm{Type}\ {\tt Id}\ \textrm{in}\ \textrm{Exp}\ ')'\ \textrm{Body} \\
\textrm{IfStmt} := &\;\textbf{if}\ '('\ \textrm{Exp}\ ')'\ \textrm{Body}\\
%\textrm{ThenBlock} := &\; \textrm{Block} \\
%\textrm{ElseBlock} := &\; \textbf{else}\ \textrm{Block}\\
\textrm{AssertStmt} := &\;\textbf{assert}\ '('\ \textrm{Exp}\ ')'\ '\{'\ \textrm{MsgStmt}\ ';'\ '\}' \\
\textrm{MsgStmt} := &\;\textbf{msg}\ '('\ ','\ \textrm{SimExp}\ (','\ \textrm{SimExp})*\ ')'\\
\textrm{DeclStmt} := &\;\textrm{Type}\ {\tt Id}\ '='\ \textrm{Exp}\\
\textrm{Exp} := &\;\textrm{SimExp}\ |\ \textrm{SimExp}\ \textbf{AND}\ \textrm{Exp}\ |\  \textrm{SimExp}\ \textbf{OR}\ \textrm{Exp}\ | \textbf{NOT}\ \textrm{Exp} \\
\textrm{SimExp} := &\;{\tt Id}\ |\ {\tt Lit}\ |\ \textrm{FunctionCall}\ |\ '('\ \textrm{Exp}\ ')'\ |\ \textrm{FunctionCall}\ '==' \textrm{SimExp}\ |\ \textbf{exists}\ '('\ {\tt Type}\ {\tt Id}\ \textbf{in}\ \textrm{Exp}\ ')'\ '('\ \textrm{Exp}\ ')' \\
{\tt Type} := &\;'\langle'\ {\tt Id}\ '\rangle'\ |\ \textbf{file}\ |\ \textbf{class}\ |\ \textbf{method}\ |\ \textbf{field}\ |\ \textbf{String} \\
{\tt Lit} := &\;{\tt StringLit}\ |\ {\tt CharLit}\ |\ {\tt IntLit}\ |\ {\tt FloatLit} \\
\textrm{FunctionCall} := &\;{\tt Id}\ '('\ \textrm{Params}\ ')' \\
\textrm{Params} := &\;\textrm{SimExp}\ (','\ \textrm{SimExp})* \\
\end{align*}
\vspace{-4.5em}
\caption{Core syntax of RSL}\label{fig:syntax}
\vspace{-2.em}
\end{figure}
 
 %there are two instances of variable declarations - 1) declaring a variable of type \codefont{String} called \codefont{beanClass} that will have the fully qualified class name (i.e. - the class attribute value for the \codefont{bean} object), and 2) declaring a \codefont{class} type variable called \codefont{c} which will contain the class item that has the same fully-qualified name as \codefont{String bean}.

\vspace{-0.5em}
\subsection{Expressions}
RSL defines various expressions. 
There are
 six kinds of simple expressions: identifier, literal, 
 invocation of built-in function(s), parenthesized expression, equivalence-checking for simple expressions, and \codefont{exists}-clause. In particular, the
\codefont{exists}-clause has two parts: a header and the body. The header \codefont{exists (Type Id in Exp)}
describes what items to enumerate, while the body \codefont{(Exp)} describes a constraint to satisfy by any item. This clause 
is similar to ForStmt, as it also describes \emph{what items to extract and enumerate}. However, different from ForStmt, this
clause does not have to process all items in a given set; it can exit early and return true whenever finding an item to satisfy the specified constraint.
In addition to simple expressions, RSL also supports three kinds of complex expressions, which connect simple expressions via logical operators AND, OR, and NOT.

\vspace{-0.5em}
\subsection{Built-in Functions}
%We defined built-in functions to facilitate data extraction from software artifacts, and content comparison among items. 
As shown in Table~\ref{tab:built-in}, we defined four types of built-in functions to facilitate (1) data extraction from software artifacts and (2) content comparison among items.

\begin{table}
\caption{Built-in functions in RSL}
\footnotesize
\vspace{-1.6em}
\begin{tabular}{p{3.5cm}|p{9.8cm}}
\toprule
\textbf{Category} & \textbf{Functions} \\ \toprule
Code-related functions & \codefont{callExists, classExists, getArg, getClasses, getConstructors, getFamily, getFields, getFQN, getMethods, getName,  getReturnType, getSN, getType, hasField, hasParam, hasParamType, indexInBound,  isIterable, isLibraryClass, isUniqueSN, locateClassSN, locateClassFQN} \\\hline %22
Annotation-related functions & \codefont{getAnnoAttr, getAnnoAttrNames, getAnnotated, hasAnnotation, hasAnnoAttr} \\ \hline %5
XML-related function & \codefont{elementExists, getAttr, getAttrs, getElms, getXMLs, hasAttr}\\ \hline %6
Miscellaneous functions & \codefont{endsWith, isEmpty, indexOf, join, pathExists, substring, startsWith, upperCase} \\ %\hline%6
%Miscellaneous functions& \codefont{isEmpty, join, pathExists}\\ %3
\bottomrule
\end{tabular}
\vspace{-2.em}
\label{tab:built-in}
\end{table}

(i) \textbf{Code-related functions} support data extraction from either raw Java code or processed code items (i.e., classes, fields, and methods). For instance, \codefont{classExists(String fqn)} checks whether the project defines a Java class, whose fully qualified name is \codefont{fqn}; 
%whether Java class \codefont{c} directly invokes the specified method API declared by \codefont{callerObjectType};
%whether Java code directly invokes the specified method API; 
%\codefont{getArg(class c, String callerObjectType, String api, int idx)} returns the \codefont{idx}$^{th}$ argument of a method call made by class \codefont{c}: \codefont{api} specifies the method name and \codefont{callerObjectType} specifies the declaring class name; 
%of an extracted method call; 
\codefont{getClasses()} retrieves all classes defined by a Java project; 
\codefont{getFamily(class c)} retrieves all ancestors in the class hierarchy of a given class \codefont{c}, together with \codefont{c} itself; 
\codefont{hasParam(method m, String aName)} checks whether method \codefont{m} has a parameter named \codefont{aName}; 
%{\codefont{importsClass(ClassItem c, String imported} checks if a Java class imports the specified class \codefont{imported};}
\codefont{indexInBound(method m, int idx)} checks whether 
a Java method has at least \codefont{idx+1} parameters. 
%\codefont{isIterable(String typeName)} checks whether a specified Java type is iterable (e.g., is a list).
%\codefont{isLibraryClass(String fqn)} checks whether a class named with \codefont{fqn} is defined by a dependent library instead of the project itself;
%\codefont{locateClassFQN(String fqn)} returns a class whose fully qualified name is \codefont{fqn}.
Given a code element, \codefont{getName(...)}  returns the element's simple name, \codefont{getFQN(...)} returns the fully qualified name, and  \codefont{getType(...)} returns the type binding. 

(ii) \textbf{Annotation-related functions} extract information from annotations or annotated Java code. Specifically, \codefont{getAnnoAttr(class c, String anno, String attr)} returns the value of a specified attribute \codefont{attr}, of annotation \codefont{anno} used in Java class \codefont{c}; 
\codefont{getAnnoAttrNames(class c, String anno)} retrieves all attribute names of annotation \codefont{anno} in the given class \codefont{c}; 
\codefont{getAnnotated(String anno, String type)} retrieves all code items, which are decorated by the specified annotation and are of certain kind (e.g., Java class);
\codefont{hasAnnoAttr(class c, String anno, String attr)} checks whether the specified class has annotation \codefont{anno}, and whether that annotation has the attribute \codefont{attr}. 

(iii) The \textbf{XML-related functions} 
support data extraction from XML files, and the processing of XML elements or attributes.
 As shown in Listing~\ref{lst:example-rsl}, 
 \codefont{getXMLs()} returns all XML files in the project; 
 \codefont{elementExists(xml, "<bean>")} examines whether the given file has an XML element named as \codefont{<bean>}; 
 \codefont{getElms(xml, "<beans>")} returns all <bean>-elements in the given file;
 \codefont{getAttr(bean, "class")} returns the value of \codefont{class}-attribute for  \codefont{bean};
\codefont{getAttrs(bean, "*method")} retrieves values of \codefont{bean}'s attributes, whose names match the specified regular expression pattern. 

%(iv) \textbf{String-manipulation functions} define operations applicable to the string data extracted from distinct sources. These operations can extract or examine substrings of a given string, or capitalize all letters in the string. 

(iv) \textbf{Miscellaneous functions} define  operations applicable to Strings or Lists. For instance, \codefont{upperCase(String s)} capitalizes all letters in String \codefont{s};  
%other than string manipulations. 
% Specifically, \codefont{isEmpty(List l)} checks whether a given list is empty;  
\codefont{join(List...values)} concatenates lists of items to create a larger list; 
\codefont{pathExists(String path)} checks whether a given path exists in the file system. 