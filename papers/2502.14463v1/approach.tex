\vspace{-.5em}
\section{\tool}\label{se:mecheck}
{\tool conducts {static program} analysis, to check whether metadata is used correctly. }
As shown in Fig.~\ref{fig:overview}, 
when being applied to a Java enterprise application, 
\tool (i) loads RSL rules, (ii) creates analyzers from those rules, and (iii) executes analyzers {to detect bugs.} This section focuses on steps (ii) and (iii) of the procedure. 
%This section explains two of the major components: analyzer creator and analyzer executor.

\vspace{-.5em}
\subsection{Analyzer Creator}
Given RSL specifications, this component produces parsing trees that we refer to as \textbf{analyzers}, because they reflect the logic of statically analyzing EAs to detect metadata-related bugs. We leveraged JavaCC \cite{javaCC_page} to implement the creator. 
Specifically, after we provided (1) token patterns defined in regular expressions and (2) syntax grammar defined in extended Backus-Naur form (EBNF), JavaCC derived a lexical analyzer (scanner) from the token patterns and created a parser from the grammar. 
The generated scanner and parser could create a parsing tree given an RSL rule. 
%We used the generated scanner and parser to create a parsing tree for any given RSL rule. 
%is used to enhance the management of Abstract Syntax Trees (ASTs), providing additional methods for handling AST nodes.

\vspace{-0.5em}
\subsection{Analyzer Executor}
When developing Analyzer Executor, we encountered three challenges (C1--C3):
%For analyzer execution, we used JJTree~\cite{hosking2023jjtree} to traverse and handle parsing trees.

\begin{itemize}
\item[C1.] \textbf{Semantics}: The five statement types supported by RSL {execute in different ways}, so our executor must observe the differences when interpreting their semantics.
\item[C2.] \textbf{Scoping}: If an RSL rule define multiple variables using the same name, 
%In different parts of an RSL rule, multiple variables can be defined to have the same name.
%Thus, it is important to 
the executor should 
differenciate between the scopes of variables 
for correct static analysis. 
\item[C3.] \textbf{{Performance}}: When multiple rules require \tool to repetitively collect and process data in the same way (e.g., scanning all XML files to gather \codefont{<bean>}-elements), we need to reduce or even eliminate redundant computation to optimize performance.
\end{itemize}

To overcome all challenges, we created an
executor of analyzers (i.e., parsing trees) as 
a visitor to traverse tree nodes. 
%To execute each parsing tree (i.e., analyzer), we implemented a tree visitor to traverse tree nodes.
When accessing each node, the executor collects and processes program data on demand; its traversal manner varies with the node type. 
For implementation, 
we used JavaParser \cite{javaparser} to parse Java code and extract data from the resulting parsing trees. We also used JDOM~\cite{oracle2014xmlparsers} to parse XML files and extract data. 
In this section, we will introduce our novel \emph{context-aware interpretation} algorithm to address C1 (Section~\ref{sec:alg}), the specially designed data structures to overcome C2 (Section~\ref{sec:data-structure}), and a novel caching mechanism to address C3 (Section~\ref{sec:cache}). 

%to address all challenges (Sections~\ref{sec:frame}--\ref{sec:cache}). 
%handle C1 and C2 (Section~\ref{sec:frame}), and a novel 
%our specially designed data structures to mainly overcome C1 (Section~\ref{sec:frame}), unique interpreter to overcome C2 (Section~\ref{sec:interpret}), and novel 
%caching mechanism to address C3 (Section~\ref{sec:cache}).




  %\vspace{-1.5em}
\subsubsection{Context-Aware Interpretation Algorithm}\label{sec:alg}
As shown in Algorithm~\ref{alg:main}, 
given the parsing tree of analyzer $t$ and the software-under-analysis $P$, \tool implements a \emph{read-eval} loop to traverse statement-level nodes in a top-down manner, execute statements in sequence, and report
metadata-related bugs based on the logic reflected by $t$ (see lines 1.3--1.4). 
In particular, \tool adopts a stack $S$ to keep track of the \emph{execution context}, i.e., all variables defined and their scopes. During execution, \tool creates and pushes a frame $f$ before executing all statement-level child nodes under $t$ (line 1.2), 
but pops and destroys $f$ after executing those statements (line 1.5). 
Algorithms~\ref{alg:process_for}--\ref{alg:process_assert} reflect how \tool works differently when executing different kinds of statements. 


\paragraph*{\textbf{ForStmt Execution.}}
%A \codefont{for}-loop statement contains two parts: \emph{header} and \emph{body}. The header describes what data to extract and enumerate; the body describes how to process that data. 
As shown in Algorithm~\ref{alg:process_for}, \tool creates and pushes a frame $f$ (line 2.1), before executing a ForStmt's header and body in an iterative way. A typical way of defining the ForStmt header is: 
``\codefont{for (T v in func(...))}''.   
Given such a header, \tool calls \codefont{func(...)} to extract or gather data, and then uses a variable \codefont{v} of type \codefont{T} to enumerate data items. 
As shown in lines 2.5--2.9, in each loop iteration, 
\tool first adds one entry \codefont{\{T, v, e\}} to $f$, to associate \codefont{v} with its data type \codefont{T} and enumerated value \codefont{e} for that iteration; it then  executes statement-level child nodes in sequence; afterwards, it clears frame $f$ to remove all variables locally defined by or for that iteration (line 2.9), since the values of those variables are limited to that iteration.
After executing the entire \codefont{for}-loop, \tool removes $f$ to discard all variables locally declared (line 2.10). 
%, as those variables are invisible outside the loop.  

\paragraph*{\textbf{IfStmt Execution}}
%An \codefont{if}-conditional statement contains two parts: \codefont{if}-condition and the body. The \codefont{if}-condition describes a predicate to test; the body describes how to further process data when the condition is satisfied. 
%The header, i.e., the \codefont{if-condition} expression, describes a predicate to test; the body 
As shown in lines 3.3--3.9 of Algorithm~\ref{alg:process}, 
\tool evaluates the \codefont{if-condition} expression, and proceeds to the statement's body when that expression is true.
Furthermore, before executing the body, \tool creates and pushes a frame $f$; it then executes statements inside the body sequentially; it also pops and discards $f$ after body execution.%, because those variables will become invisible and inaccessible after the body's execution.

\paragraph*{\textbf{The Execution of AssertStmt and MsgStmt}}
As shown in Algorithm~\ref{alg:process_assert}, \tool first evaluates the \codefont{assert}-condition. If that condition is  %true, no error message is generated and the program $P$ does not have the type of bugs focused by $t$. Otherwise, 
false, \tool formulates a bug report based on the MsgStmt embedded in the \codefont{assert}-body. %A MsgStmt is always contained by an AssertStmt; it defines in the scenario where any data instance fails the assertion, what error message to print, and what variables to use when generating the message. 
%A MsgStmt contains at least one expression, with the first expression defining a string literal or template for the bug-to-report. If a string literal is in use, \tool generates an error message solely using that literal. 
%If a string template is in use, \tool evaluates all remaining expressions to concretize template variables involved in the first expression.
%no other expression is needed to generate the error message; otherwise, each of the remaining expressions will be evaluated by \tool to concretize template variables involved in the first expression.

\begin{algorithm}[ht!]
\caption{\textit{NovelSelect}}
\label{alg:novelselect}
\begin{algorithmic}[1]
\State \textbf{Input:} Data pool $\mathcal{X}^{all}$, data budget $n$
\State Initialize an empty dataset, $\mathcal{X} \gets \emptyset$
\While{$|\mathcal{X}| < n$}
    \State $x^{new} \gets \arg\max_{x \in \mathcal{X}^{all}} v(x)$
    \State $\mathcal{X} \gets \mathcal{X} \cup \{x^{new}\}$
    \State $\mathcal{X}^{all} \gets \mathcal{X}^{all} \setminus \{x^{new}\}$
\EndWhile
\State \textbf{return} $\mathcal{X}$
\end{algorithmic}
\end{algorithm}



\paragraph*{\textbf{DeclStmt Execution}} As shown in lines 3.12--3.17 of Algorithm~\ref{alg:process}, \tool retrieves the current top frame of stack $S$. It identifies both the type name \codefont{T} and variable name \codefont{v} used in the declaration; it also evaluates the right-hand side of the statement for \codefont{value}. Finally, \tool adds an entry \codefont{\{T, v, value\}} to $f$ to record the declared variable. In this way, when the variable is used later, \tool revisits $f$ to retrieve the variable's type as needed, and to read or write values on-demand.

\paragraph*{\textbf{Expression Evaluation}}
Algorithm~\ref{alg:evaluate} illustrates how expressions are processed and executed.
Intuitively, 
when an expression is a variable or literal, \tool simply returns the value as the evaluation result. When an expression is more complex and contains multiple sub-expressions or nested expressions, \tool traverses the expression tree in a top-down manner, and evaluates values in a bottom-up manner. For instance, to evaluate the expression \codefont{locateClassFQN(getAttr(bean, "class")}, \tool first retrieves the value of \codefont{bean} by accessing the stack; it then evaluates the value of function call \codefont{getAttr(bean, "class")}; based on that evaluation, \tool further calculates the value of function call \codefont{locateClassFQN(...)}.

\vspace{-.5em}
\subsubsection{Stack and Frames}
\label{sec:data-structure}
\tool adopts a stack $S$ to keep track of the execution context; it (1) pushes frames onto $S$, (2) pops frames from $S$, and (3) walks through frames to manage variables, their types, as well as their values. A \textbf{frame} is a dictionary, where the key is a variable name, and the value is a pair of variable type and variable name.
\tool creates a frame before executing the entire analyzer, any ForStmt, IfStmt, or \codefont{exists}-clause; each created frame is then pushed onto stack to record any variable declared for or in the corresponding program structure. During execution, 
\tool refers to the stack to search for declared variables, query their data types, or obtain the values. 
After the execution of a ForStmt, IfStmt, exists-clause, or the entire analyzer, \tool pops a frame from the stack to discard all variables created by that program structure. 

%A typical format of \codefont{exists}-clause is ``\codefont{exists(T v in func(...))(...)}''. 
%The semantics is that among all data items returned by \codefont{func(...)}, there should be at least one item satisfying the expression in \codefont{exists}-body. 
%the expression mentioned in the exists-expression's body. Namely, if any of the items satisfies that expression, the whole expression returns true; otherwise, it returns false. 
%To evaluate such an expression, \tool first calls \codefont{func(...)} to extract or gather data, and then uses a variable \codefont{v} of type \codefont{T} to enumerate all data items. As shown in lines 5.17--5.23 of Algorithm~\ref{alg:evaluate}, in each loop iteration, \tool first adds one entry \codefont{\{T, v, e\}} to $f$, to associate \codefont{v} with its data type \codefont{T} declared in analyzer, and with value \codefont{e} for that iteration; next, \tool evaluates \codefont{exists}-body. If the evaluation result is true, \tool returns true as it finds an enumerated item to satisfy the entire expression. If none of the items can lead to a \codefont{true}-value, \tool returns false (see line 5.25).

\begin{minipage}{0.95\linewidth}
\begin{algorithm}[H]
\footnotesize
\caption{The \textbf{evaluate()} function}\label{alg:evaluate}
\KwIn{$n$---expression node, $S$---stack for variable frames, and $P$---the given software application}
\KwOut{Return a boolean value based on the expression evaluation}

\If{$n$ is an identifier}{
    $res \leftarrow$ retrieve the value of $identifier$ from $S$ starting from top frame\\
    $return$ $res$
  }\ElseIf{$n$ is a literal} {
   $return$ the literal's value
  }\ElseIf{$n$ is a built-in function call} {
    $return$ the function call's return value
  }\ElseIf{$n$ is a parenthesized expression} {
    $return$ $evaluate($the expression inside the parenthesis$, S, P)$ 
  }\ElseIf{$n$ is an equivalence-checking expression}{
    $return$ $evaluate($left-side of the expression$, S, P)$ == $evaluate($right-side of the expression$, S, P)$
  }
  \ElseIf{$n$ startsWith "exists"}{
  Create a frame $f$ and push it onto $S$\\
  $T \leftarrow$ variable type part from $n$\\
  $v \leftarrow$ variable name part from $n$\\
  $container \leftarrow evaluate($container part from $n, S, P)$\\
  \ForEach{element $e$ in $container$}{
    add \{$T$, $v$, $e$\} to $f$\\
    $logic \leftarrow$ expression inside the exists-expression's body\\
    $res \leftarrow$ $evaluate(logic, S, P)$\\ 
    \If{$res$ = True}{
      $return$ True
    }
    clear the frame $f$
  }
  pop $f$ from $S$\\
  $return$ False
}\ElseIf{$n$ is an AndExpression}{
    $return$ $evaluate($left-side of the expression$, S, P)$ \&\& $evaluate($right-side of the expression$, S, P)$
}\ElseIf{$n$ is an OrExpression}{
    $return$ $evaluate($left-side of the expression$, S, P)$ || $evaluate($right-side of the expression$, S, P)$
}\ElseIf{$n$ is a NotExpression}{
    $return$ ~$evaluate($the expression after NOT operator$, S, P)$
}
\end{algorithm}
\end{minipage}

\begin{figure}
\centering
\includegraphics[width=.88\linewidth]{images/data-structure.pdf}
\vspace{-1.2em}
\caption{The stack status when \tool executes the {\tt exists}-clause in Listing~\ref{lst:example-rsl}}\label{fig:data-structure}
  \vspace{-1.em}
\end{figure}

Fig.~\ref{fig:data-structure} presents a snapshot of stack $S$ during runtime  when \tool executes the \codefont{exists}-clause of analyzer in Listing~\ref{lst:example-rsl}. Variable entries are added to frames when the variables are declared, or when iterating variables for ForStmt and \codefont{exists}-clause are initialized at the beginning of each loop iteration. Some entries get removed from frames at the end of loop iterations, because the values of iterating variables get reset for each iteration and the local variables declared for one iteration are invalid for other iterations. 
%only for iterating variables at the end of each loop iteration, as the values of those variables get reset for each iteration. 
Given a variable name for resolution, e.g., \codefont{c} used in the \codefont{exists}-body, \tool starts with the top frame to search for a corresponding variable declaration. As shown in Fig.~\ref{fig:data-structure}, the top frame has no variable declared as \codefont{c}, so \tool moves on to the next frame---the {ForStmt} frame---to search. Such a stack walking continues until \tool finds the variable declared in a {IfStmt} frame. 
Our stack-based data structure allows the analysis-logic implemented in inner scopes to freely access variables defined in outer scopes; it also enables \tool to differentiate between the same-named variables declared in distinct scopes. 

  \vspace{-.5em}
\subsubsection{Performance Improvement}
\label{sec:cache}

When \tool executes multiple analyzers to scan the same project in one run, it is possible that the analyzers call same functions and repetitively retrieve the same results. 
Such repetitive function calls can waste lots of computing resources, hindering \tool from analyzing software efficiently. 
What makes things even worse, based on our experience, many domain-specific constraints or RSL analyzers
require \tool to parse all Java classes by calling \codefont{getClasses()}, or parse all XML files by calling \codefont{getXMLs()}. When a software project is large and contains thousands of files, the repetitive file-parsing can be very time-consuming. 

\begin{figure}
    \centering
    \includegraphics[width=.95\linewidth]{images/caching.pdf}
      \vspace{-1.2em}
    \caption{The map used for caching and some related internal data structures in our implementation}
    \label{fig:cache}  \vspace{-1.5em}
\end{figure}

To optimize performance, we implemented a caching mechanism in \tool to eliminate redundant computation as much as possible. Specifically, we defined a global cache to map function calls to their return-values, for each software-under-analysis. For instance, as shown in Fig.~\ref{fig:cache} (a), \codefont{getClasses()} is mapped to the list of class items \tool creates based on Java parsing; \codefont{getXMLs()} is mapped to a list of XML items \tool creates based on XML parsing; \codefont{getMethods("class-name")} is mapped to a list of method items created for all methods declared by the Java class named ``\codefont{class-name}''. 
We decided to cache a function call if that function extracts data from Java/XML files, or derives information by enumerating extracted data. With such a caching mechanism, for each project-under-analysis, \tool parses all Java (or XML) files only once if any of the analyzers needs to retrieve all code (or metadata) items; when a function is called with different parameter values, their results can be cached differently as those values are used as part of the key in cache.
Furthermore, to avoid eagerly extracting all methods and fields from Java files when \tool is only interested in the class-level information, \tool conducts lazy initialization for both \codefont{fields} and \codefont{methods} members of \codefont{ClassItem} objects (see Fig.~\ref{fig:cache} (b)): 
either member is initialized only when any analyzer explicitly requests for the information of a particular class.

%To avoid loading same components repeatedly, \tool has a caching mechanism. For the major queries such as - getting all java classes, or getting all XML files, \tool caches the result in memory. So if multiple rules are run on the same project and multiple RSL scripts are making the same query, \tool will only process it once and return the cached data the next time.
%\tool also has another approach to fasten up the procedure. For components with lots of sub-components, \tool does not explore everything in an eager fashion. Instead, it only explores a certain branch if it's been requested. e.g. - $for$ $(class$ $c$ $in$ $getClasses())$ is only requesting for the class items. It does need the analyzer executor to load all the instance variables, or methods. So it follows a lazy approach \cite{wiki:lazy_evaluation}, and only loads a the necessary type of components in a class.