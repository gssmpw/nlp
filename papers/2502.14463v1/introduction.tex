\section{Introduction}
Enterprise application (EA) development is a complex process of creating large-scale, multi-tiered, scalable, reliable, and secure network applications for business purposes~\cite{ea-development}. 
To reduce the complexity of EA development, developers usually build EAs on top of the Java EE platform~\cite{javaEE} or third-party frameworks like Spring~\cite{spring}. Such platforms or frameworks promote the principle of separation of concerns~\cite{Hursch95separationof}: 
they address nonfunctional concerns including persistence, transactions, and security, so that 
 developers only need to implement the core functionality of an EA by hand. 
%Namely, developers implement the core functionality of an EA by hand, and rely on third-party frameworks to address nonfunctional concerns including persistence, transactions, and security. 
Most of these platforms or frameworks support a declarative programming model, allowing EA developers to use metadata (i.e., Java annotations and XML files) when configuring (1) how application components are deployed, and (2) how these components interact with each other.
%with each other or with the frameworks.

However, correctly using metadata can be challenging for developers due to three reasons. First, the usage rules are domain-specific and vary with Java frameworks, so developers can easily get confused and misuse metadata. Second, when a deployment or configuration task poses consistency constraints on the content of (i) Java annotations, (ii) code implementation, and/or (iii) XML files, developers may fail to observe all constraints when evolving software and thus inconsistently update metadata or code. Third, existing tools \cite{chamberlin2002xquery,Song12,Wen20} rarely analyze metadata together with Java code, let alone find semantic inconsistencies between software artifacts. Consequently, when metadata misuse leads to any misconfiguration \cite{misconfiguration}, abnormal runtime behavior \cite{springSecurity}, confusing error \cite{confuse_error}, or security vulnerability \cite{security-vul}, developers are on their own to handle metadata-related issues.


\begin{figure}
\centering
\includegraphics[width=.85\linewidth]{images/overview.pdf}
\vspace{-1.5em}
\caption{The workflow of our approach}\label{fig:overview}
\vspace{-2.em}
\end{figure}

To help developers debug metadata usage, we invented a semi-automatic solution that has two parts: \textbf{RSL (\underline{R}ule \underline{S}pecification \underline{L}anguage)} and \textbf{\tool (\underline{Me}tadata \underline{Check}er)}. As shown in Fig.~\ref{fig:overview}, to define a metadata-usage rule for a Java framework (e.g., Spring), domain experts (e.g., owner developers of that framework) can exploit RSL to (i) retrieve all relevant metadata and/or code items, (ii) refine those items based on certain conditions, (iii) constrain the content correspondence among refined items, and (iv) prescribe the error-reporting format if any constraint is violated. 
%(ii) extract properties and content of the code items, (iii) apply constraints based on the properties and/or content of the code elements, (iv) report error-message in an understandable and informative format if any constraint is violated. 
\tool stores such RSL specifications locally as domain-specific metadata checking rules. 
When an EA developer provides a software application to check, \tool loads all rules, 
%Taking such RSL specifications, 
%\tool 
creates parsing trees based on the RSL grammar for those rules, and treats the trees as analyzers. %, and stores them internally. 
%Afterwards, \tool loads all analyzers. 
When executing each analyzer, \tool parses Java and XML files on demand, gathers and filters metadata and/or code items as instructed, and compares the item content for consistency checking. If any item violates a constraint, a customized error message is reported.

{To evaluate the effectiveness of RSL and \tool, we defined and investigated three research questions (RQs) in our experiments:
\begin{itemize}
\item \textbf{RQ1}: \emph{How effectively can RSL express metadata-usage rules?} We studied the documentation of Spring and JUnit frameworks~\cite{spring-doc,spring-tutorial,JUnit5}, and distilled \totalRule metadata-usage rules. 
%with RSL. Among these rules, 
Seven of the rules are about content consistency between \textbf{XML items (i.e., elements and attributes)} and Java code; six rules are about consistency checking between code and  annotations; one rule checks the consistency between XML and annotations; 
%one rule checks consistency between annotations and Java code; 
and one rule examines the consistency among code, XML items, as well as annotations. %We also have six rules checking consistency for JUnit classes. Among them, five rules check consistency of added JUnit-related annotations, and one rule checks consistency between \textbf{JUnit-related annotations (e.g.: $@RunWith(Parameterized.class), @Test$ etc.)}, and Java code. 
{We managed to express all rules using RSL, demonstrating its great power in expressing diverse rules.}
%{\color{orange} We managed to express all rules with RSL, which language demonstrates 
%RSL successfully expressed them all, demonstrating a great power of expressing diverse rules.}
%Such a diversity of rules demonstrates the great expressiveness of RSL. 
\item \textbf{RQ2}: \emph{How accurately can \tool detect bugs?} We manually injected \totalInjection metadata-related bugs into the latest version of \totalInjectedProject open-source projects, 
%randomly selected \totalInjectedProject dataset projects, 
and applied \tool to those projects. Our evaluation shows that
 %that on average, 
\tool reported bugs with \precision precision, \recall recall, \fscore F-score. This implies that \tool detected bugs with high accuracy. 
\item \textbf{RQ3}: \emph{How effectively does \tool reveal real-world bugs?} We applied \tool to the version history of another \totalRealProject open-source projects, in order to reveal metadata-related bugs in real-world settings. 
%{\color{blue}In total, \tool reported \totalReportedInReal bugs in the version history of \totalInterestingEAs projects, \totalInteresting of which are true positives.} 
{In total, \tool reported \totalReportedInReal bugs in the version history of \totalInterestingEAs projects, \totalInteresting of which bug reports are true positives.} 
%bugs were accidentally introduced when developers evolved software.
Developers have fixed \totalRealBugs of those bugs so far. 
%, and as a result, they are no longer available in the latest versions. 
% after (\todo{check commit range}) commits. 
Our experiment indicates that \tool effectively identified real bugs in EAs. 
%in the metadata usage of EAs that developers address later most of the time. 
If developers had adopted \tool to scan their projects before committing any program changes, they could have found and addressed metadata-related issues more easily. 
\end{itemize}
}

In summary, this paper makes the following contributions: 
   	\begin{itemize}
   	    \item We designed a domain-specific language (DSL)---RSL---for domain experts to specify metadata-usage rules. Different from prior DSLs, RSL can express consistency relations among XML items, Java annotations, and Java code. %It has a C-like grammar, including 5 types of statements and \totalFunc built-in functions.
   	    \item We created \tool, to interpret RSL specifications and examine user-provided EAs accordingly. Compared with existing tools, \tool implements a novel algorithm that (1) extracts data in both Java and XML files, (2) tracks relations among the extracted data, and (3)  differentiates between data instances for refinement and comparison. 
        %To accelerate rule-checking and program  analysis, we implemented \tool to have specialized data structures
	    %\tool is capable of (1) extracting relevant facts in both Java and XML files, and (2) finding any content inconsistency among those facts.	    
   	    \item We comprehensively evaluated RSL and \tool. Our evaluation demonstrates the great expressiveness of RSL and the high detection accuracy of \tool for synthetic data; it also reveals real-world scenarios where \tool effectively locates metadata-related bugs. 
       \item To optimize \tool's runtime performance, we implemented a caching mechanism in the tool, which computes new data only when necessary.
   	\end{itemize}
    
In the following sections, we will  explain our research with a motivating example (Section \ref{se:example}), introduce the background knowledge of metadata usage (Section \ref{se:background}), describe our new approach: RSL and \tool (Sections \ref{se:rsl}--\ref{se:mecheck}), and present our evaluation (Section~\ref{se:evaluation}). 	