\documentclass{article}
\usepackage{graphicx}
\usepackage{subcaption}
\usepackage{float}
\pagenumbering{gobble}

\begin{document}

\begin{figure}
\centering
\includegraphics[width=\linewidth]{../img/opportunity_200_population.pdf}
% \caption{Scatter plot and OLS line of best fit for change in dissimilarity ($\Delta D$)  over change in travel time ($\Delta T$) for students who would switch schools under elementary school mergers, together representing the trade-offs \& opportunities for school districts in using school merging as an integration strategy. Marker size is scaled to be proportional to the district's population that attends closed-enrollment elementary schools. A Spearman rank correlation coefficient of $\rho = -0.284$ ($p<0.0001$), serves as a reference for which districts demonstrate the potential for ``integration arbitrage'': achieving relatively higher levels of integration at a lower (travel time) cost. Districts below the threshold have a greater decrease in dissimilarity score per increase in travel time than districts below the threshold have. The plot illustrates that, for some districts, a large decrease in dissimilarity score is traded for a small increase in travel time, suggesting that the mergers approach is a worthwhile approach for some (but likely not all) districts.}
\label{fig:opportunity_plot}
\end{figure}
\end{document}