\documentclass{article}
\usepackage{graphicx}
\usepackage{subcaption}
\usepackage{float}
\pagenumbering{gobble}

\begin{document}

\begin{figure*}
\centering
\hfill
\begin{subfigure}[b]{0.45\textwidth}
    \centering
    \includegraphics[width=0.7\textwidth]{../img/maps/miami-dade_race.png}
    \caption{Miami-Dade \\ (NCES ID: 1200390)}
    \label{fig:miami-dade}
\end{subfigure}
\hfill
\begin{subfigure}[b]{0.45\textwidth}
    \centering
    \includegraphics[width=0.9\textwidth]{../img/maps/plano_race.png}
    \caption{Plano ISD \\ (NCES ID: 4835100)}
    \label{fig:plano}
\end{subfigure}
\hfill
\vfill
\begin{subfigure}[b]{\textwidth}
    \centering
    \includegraphics[width=0.7\textwidth]{../img/gearys_c.pdf}
    \caption{Correlation with Geary's $C$}
    \label{fig:gearys_c}
\end{subfigure}
% \caption{\small{School mergers are less likely to foster meaningful integration in districts with few interfaces between attendance zones with different demographics and more likely in districts with more such interfaces ((a) and (b), respectively). In these plots, darker blue indicates higher concentration of elementary students of color.  (c) Scatter plot and OLS line of best fit ($\Tilde{\Delta D} = -0.344 C - 0.037$; $r=-0.611$, $p<0.0001$) for the change in dissimilarity ($\Delta D$) over Geary's $C$, for the 200 districts in our sample. A Spearman rank correlation coefficient of $\rho = -0.592$ suggests a moderate, negative correlation, where greater values of $C$---representing stronger spatial anticorrelation---indicate higher potential for district integration via school mergers.}}
\label{fig:race_maps}
\end{figure*}

\end{document}