\documentclass{article}
\usepackage{graphicx}
\usepackage{subcaption}
\usepackage{float}
\pagenumbering{gobble}

\begin{document}

\begin{figure}[h]
\centering
\hfill
\begin{subfigure}[b]{0.3\textwidth}
    \centering
    \includegraphics[width=\textwidth]{../img/mergers/mergers_before.pdf}
    \caption{Before mergers}
    \label{fig:mergers_before}
\end{subfigure}
\hfill
\begin{subfigure}[b]{0.3\textwidth}
    \centering
    \includegraphics[width=\textwidth]{../img/mergers/mergers_after.pdf}
    \caption{After mergers}
    \label{fig:mergers_after}
\end{subfigure}
\hfill
\begin{subfigure}[b]{0.3\textwidth}
    \centering
    \includegraphics[width=\textwidth]{../img/mergers/mergers_legend.pdf}
    %\caption{Legend}
\end{subfigure}
\hfill
% \caption{\textit{School mergers} involve merging the attendance boundaries of adjacent schools and subsequently modifying the grades they serve to promote demographically-diverse classrooms. (a) Two adjacent K–5 schools within a district that happen to serve students who are demographically different from one another. (b) The schools can be merged so that one school serves only students in grades K–2 in the merged region, and the other serves only students in grades 3–5, thereby diversifying the student body of each school.}
\label{fig:mergers}
\end{figure}
\end{document}