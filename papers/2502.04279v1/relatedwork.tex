\section{Related Work}
Questions concerning local flat-foldability can be cast in the language of constraint satisfaction problems and statistical physics, specifically in the framework of \emph{spin systems}: each edge is assigned ``mountain'' or ``valley'' and there are local rules at each vertex to determine compatibility.  In particular, if one wishes to describe a uniformly random locally flat-foldable configuration as a spin system, one assigns energy $+\infty$ to each configuration that is not locally flat-foldable and energy $0$ to each configuration that is locally flat-foldable.  Via Kawasaki's Theorem, this can be checked by looking purely at each vertex individually. The perspective of interpreting locally flat-foldable configurations as a spin system is not new and indeed work by Assis \cite{Assis} draws connections between more classical statistical physics models and local flat-foldability on certain lattices (some of which we will also discuss in the present study).
Ginepro and Hull \cite{Ginepro} proved that the foldings of the \textit{Miura-ori} crease pattern (to be defined in Section~\ref{subsec:Miura}) are equivalent to the square-ice model. 
Nakajima used a spin model on random graphs to model the combinatorial problem of ordering the different layers of paper in flat-folded crease patterns \cite{nakajima1}, and developed more work in this vein for crease patterns that contain only one interior vertex \cite{nakajima2}. 

Our main focus is on properties of random locally flat-foldable configurations and in particular on sampling via Markov chains.  The use of Markov chains to sample spin systems is a classic topic at the intersection of statistical physics, probability theory, and computer science.  In particular, Markov chain Monte Carlo was introduced in a work by Metropolis--Rosenbluth--Rosenbluth--Teller--Teller \cite{metropolis1953equation} which sought to sample from a continuous-space variant of a spin system called the hard-disk model.  Analysis of Markov chains for spin systems remains a vibrant field, and we refer the reader to the following texts \cite{grimmett2018probability, levin2017markov} and some recent breakthroughs \cite{anari2021spectral,chen2021optimal} along with the references therein for more context.
There has also been quite a bit of research on the algorithmic complexity of origami over the past several decades, including several monographs \cite{GFALOP,Origametry,Ida,Uehara1}. We make references to related computational origami results throughout.