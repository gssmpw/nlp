\section{Related Work and Discussion}
The single-winner voting setting is almost synonymous with the general social choice setting, as it can be used to model decision-making scenarios where a set of individuals collectively select an outcome; see ____. In the distortion literature, it has been the main setting of study since the inception of the framework, giving rise to a plethora of papers in both the utilitarian setting ____ and the metric setting, where the cardinal information is captured by metric costs ____. The most important results in this literature for us are the worst-case distortion bounds of $O(m^2)$ and $\Omega(m^2)$, proven by ____ and ____, respectively. We refer to the survey of ____ for more details about other works. 

The one-sided matching setting was introduced (as a social choice problem) in the pioneering work of ____, and was studied extensively ever since in economics and computer science; e.g., see ____. In the literature of utilitarian distortion, it was notably studied by ____, who considered randomized mechanisms, and then by ____ and ____ in settings where \emph{reliable} cardinal information is also provided to the mechanism via queries. We remark that ____ and ____ in fact studied both single-winner voting and one-sided matching, as we do here; in a certain sense, these two settings can be considered as the canonical ones for utilitarian distortion. 

Besides the literature of distortion, our work obviously has connections to the literature on learning-augmented algorithms. As we mentioned before, the associated research agenda was explicitly put forward by ____ to study problems where the prediction comes as advice for an appropriate machine learning model. The associated literature has revisited several classic algorithmic problems under this lens, including problems in data structures, online algorithms, approximation algorithms, combinatorial optimization, and sublinear algorithms, among others. We refer the reader to the survey of ____ for a more in depth discussion of this literature, as well as the ``Algorithms with Predictions'' online repository ____. 

In more economically oriented applications, the concept of predictions has been applied to many different problems related to mechanism design ____, bidding auctions ____, strategyproof mechanisms for learning and assignment problems ____, and cost sharing ____. Most closely related to us is the work of ____, who studied of distortion of metric single-winner voting with predictions and showed tight consistency and robustness tradeoffs for that setting. The metric distortion setting is markedly different from the utilitarian setting that we study, and therefore their results do not have any implications in our setting. Still, it is interesting to note that in their case, the best possible tradeoff is achievable by mechanisms that only require information about the identity of the optimal candidate. In contrast, we show that such a prediction is insufficient to achieve any improvements over purely ordinal mechanisms. Still, similarly to ____, our mechanism that achieves the best possible tradeoff does not require prediction information about the whole cardinal profile, but only about the value of each voter for her most-preferred candidate. 
%%%%%%%%%%%%%%%%%%%%%%%%
%%%%%%%%%%%%%%%%%%%%%%%%



%%%%%%%%%%%%%%%%%%%%%%%%
%%%%%%%%%%%%%%%%%%%%%%%%