\section{Background and Related Work}
\label{sec:background}
In this section, we provide a brief introduction to locate-then-edit knowledge editing methods and present them as a two-step fine-tuning process. For a more detailed introduction to these methods, we refer the reader to prior works ____. 

Locate-then-edit family of methods like ROME ____ , MEMIT ____ and AlphaEdit ____ are used to update facts that can be represented in the form of triplets of the form (subject, relation, object) or $(s,r,o)$. Instead of updating all the weights of a model to incorporate new knowledge, these methods only update certain weight matrices that are most responsible for factual recall ____. The location of an edit within a model is described by a two-dimensional address - (i) an intermediate layer to be edited and (ii) a token from the list of input tokens used to create the target representation. The exact layer to be edited is found using causal tracing ____ or an empirical sweep over all decoder layers of the model ____. Additionally, updating the second MLP layer in the FFN module of decoder layers has shown optimal knowledge editing performance ____. This provides the first part of the editing address. ____ also showed that using the output representation of the subject token of a sentence produces the best editing results. This provides the second part of the editing address, where the edit is made using the position index of the last token of the subject. We explain this with an example below.


\begin{figure}[t]
    \centering
    % Subfigure 1
    \subfigure[Gradient descent step which finds the target activations for the MLP matrix.]{
        \includegraphics[width=0.45\linewidth]{figures/ft-step1.png} % Replace with your image
        \label{fig:editing-process-gd}
    }
    \hfill
    % Subfigure 2
    \subfigure[Target activations are used to update the second MLP matrix (in red).]{
        \includegraphics[width=0.45\linewidth]{figures/ft-step2.png} % Replace with your image
        \label{fig:editing-process-closed-form}
    }
    \caption{Presenting locate-then-edit knowledge editing methods as a two-step fine-tuning process.}
    \label{fig:editing-process}
    \vskip -0.1in
\end{figure}


Given a fact to be edited, for example - \textit{``The capital of Malaysia is Singapore"}, the query phrase for the editing process is \textit{``The capital of Malaysia is"} and the target phrase is \textit{``Singapore"}. The first part of the editing address, the exact layer whose second MLP matrix gets edited, is decided before the editing begins. The second part of the editing address is the token index of the last subject token, which in this case would be the last subword-token in \textit{``Malasiya"}. The intermediate hidden representation of this last subject token is used to make the edit.

Once the editing address has been decided, instead of updating the chosen MLP weight matrix directly using gradient descent, the process of locate-then-edit knowledge editing proceeds in two steps -  

\begin{enumerate}
    \item In the first step (Figure \ref{fig:editing-process-gd}), gradient descent is used to find the appropriate activation vector that acts as a target for the weight matrix to be edited. This target activation is found such that the edited fact is generated by the model. In the example, this activation will cause the model to generate ``Singapore" in response to the question. Note that in this step, no weights are updated and just an intermediate activation vector is found. This new activation vector now serves as the target for the MLP weight matrix chosen for editing.

    \item  The weight update happens in the second step of editing (Figure \ref{fig:editing-process-closed-form}), where the MLP matrix is updated with the target activation vector found in the previous step, using a least squares loss function. This loss function tries to preserve the outputs of the MLP matrix for unrelated contexts while generating the target activation when the input corresponds to the query phrase. 
\end{enumerate}

% \begin{table*}[t]
% \caption{Comparison between prediction probabilities of facts that have been edited into a model compared to facts that a model knows through pretraining. }
% \label{tab:overfitting}
% \vskip 0.15in
% \begin{center}
% \begin{small}
% \begin{sc}
% \begin{tabular}{lcccr}
% \toprule
% Method & Model  & \multirow{2}{*}{\makecell{Unedited \\ 
%  Fact Prob.}} & \multirow{2}{*}{\makecell{Edited \\ Fact Prob.}} \\
% & & & \\
% \midrule
% EMMET    & GPT2-XL& & 0.98&  \\
%  & Llama2-7B& & 0.92 &\\
%     & Llama3-8B& &  0.99& \\
% \midrule
% MEMIT    & GPT2-XL& & 0.65 & \\
%  & Llama2-7B& & \\
%     & Llama3-8B& &  \\
% \midrule
% AlphaEdit    & GPT2-XL& &  0.98 \\
%   & Llama2-7B& & \\
%      & Llama3-8B& &  \\
% \bottomrule
% \end{tabular}
% \end{sc}
% \end{small}
% \end{center}
% \vskip -0.1in
% \end{table*}



Specifically, let $W_0$ be the initial weights of the second MLP matrix, which is being edited to $\hat{W}$. $k_0$ is used to indicate an input to the MLP matrix representing activation vectors for information we want to preserve from the original model, and $k_e$ is input activation vectors representing facts we want to insert into the model. Let $v_e$ be the desired target activation vector for the edited MLP matrix found in step 1 of editing using gradient descent. Then the loss function used to update the MLP weight matrix is formulated using least-squares in the form of a preservation-memorization objective ____:
\vskip -0.5cm
\begin{equation}\label{eq:memit_objective}
\begin{aligned}
     \underset{\hat{W}}{\operatorname{argmin}} \hspace{5pt} L(\hat{W}) \hspace{10pt} \text{where}& \hspace{50pt}\\ 
     L(\hat{W}) = \hspace{4pt} \underbrace{\lambda \sum^{P}_{i=1} \left\| \hat{W} k^i_0 - W_0 k^i_0 \right\|^2_2}_{\text{preservation}}  +&
     \underbrace{\sum^{B}_{j=1} \left\|\hat{W} k^j_e - v^j_e\right\|^2_2}_{\text{memorization}}
\end{aligned}
\end{equation}




%A closed form solution for the above objective exists, and is shown below:

%\begin{equation}\label{eq:memit}
%\begin{aligned}
%    \hat{W} &= W_0 + \Delta \hspace{10pt} \text{where} \hspace{10pt}  
%    \\ \Delta &= \big(V_E - W_0K_E \big) K_E^T \big( \lambda C_0 + K_EK_E^T \big)^{-1}
%\end{aligned}
%\end{equation}

Since the above objective is linear in the argument, we do not need to use gradient descent for optimization. Thus, locate-then-edit methods can be seen as a unique type of fine-tuning method. Instead of updating the MLP matrix directly using gradient descent on the desired data, the weight update happens in two steps using two different types of objective functions for each step. The first step uses gradient descent whereas the second step uses a closed-form solution. %Closed-form solutions are rare in deep learning models, which is why knowledge editing methods might seem different from fine-tuning, but if we define fine-tuning as optimizing for a loss function to produce a desired output for weight updates, then knowledge editing is indeed fine-tuning. 

%____ empirically support this by showing that knowledge editing using standard fine-tuning produces similar results to locate-then-edit knowledge editing methods. If we consider knowledge editing methods as fine-tuning, observing signs of overfitting and catastrophic fogetting ____ are natural outcomes. 

%Also write about how it is closely related to continually learning methods like replay.