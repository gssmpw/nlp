% \documentclass[]{uai2025} % for initial submission
\documentclass[accepted]{uai2025} % after acceptance, for a revised version; 
\pdfoutput=1
% also before submission to see how the non-anonymous paper would look like 
                        
%% There is a class option to choose the math font
% \documentclass[mathfont=ptmx]{uai2025} % ptmx math instead of Computer
                                         % Modern (has noticeable issues)
% \documentclass[mathfont=newtx]{uai2025} % newtx fonts (improves upon
                                          % ptmx; less tested, no support)
% NOTE: Only keep *one* line above as appropriate, as it will be replaced
%       automatically for papers to be published. Do not make any other
%       change above this note for an accepted version.

%% Choose your variant of English; be consistent
\usepackage[american]{babel}
% \usepackage[british]{babel}

%% Some suggested packages, as needed:
\usepackage[sort&compress]{natbib}
% \usepackage{natbib}
 % has a nice set of citation styles and commands
\renewcommand{\bibsection}{\subsubsection*{References}}
\usepackage{mathtools} % amsmath with fixes and additions
% \usepackage{siunitx} % for proper typesetting of numbers and units
\usepackage{booktabs} % commands to create good-looking tables
\usepackage{tikz} % nice language for creating drawings and diagrams
\usepackage{amsmath,amssymb,amsfonts}
\usepackage{xifthen}
\usepackage{subcaption}
\usepackage{graphicx}
\usepackage{textcomp}
% \usepackage[dvipsnames]{xcolor}
% \usepackage{mathdots}
\usepackage{yhmath}
\usepackage[ruled,vlined]{algorithm2e}
\usepackage{bm}
\usepackage{cancel}
\usepackage{xcolor}
\usepackage{textcase}
\usepackage{siunitx}
\usepackage{array}
\usepackage{multirow}
\usepackage{gensymb}
\usepackage{tabularx}
\usepackage{extarrows}
\usepackage{booktabs}
\usepackage{colortbl}
\usepackage{pgfplots}
\usepackage{pgfplotstable}

\pgfplotsset{compat=newest}
\usetikzlibrary{fadings}
\usetikzlibrary{bayesnet}
\usetikzlibrary{patterns}
\usetikzlibrary{shadows.blur}
\usetikzlibrary{calc}
\usetikzlibrary{shapes, 3d, backgrounds, fit, arrows.meta, positioning, shapes.geometric}
\usepackage{custom_commands}

\allowdisplaybreaks


%% Provided macros
% \smaller: Because the class footnote size is essentially LaTeX's \small,
%           redefining \footnotesize, we provide the original \footnotesize
%           using this macro.
%           (Use only sparingly, e.g., in drawings, as it is quite small.)

%% Self-defined macros
\newcommand{\swap}[3][-]{#3#1#2} % just an example

\title{Partially Observable Gaussian Process Network and Doubly Stochastic Variational Inference}

% The standard author block has changed for UAI 2025 to provide
% more space for long author lists and allow for complex affiliations
%
% All author information is automatically removed by the class for the
% anonymous submission version of your paper, so you can already add your
% information below.
%
% Add authors
\author[1]{\href{mailto:<saksham.kiroriwal@iosb.fraunhofer.de>?Subject=Your paper: Partially Observable Gaussian Process Network and Doubly Stochastic Variational Inference}{Saksham Kiroirwal}}
\author[1]{\href{mailto:<julius.pfrommer@iosb.fraunhofer.de>?Subject=Your paper: Partially Observable Gaussian Process Network and Doubly Stochastic Variational Inference}{Julius Pfrommer}}
\author[1,2]{\href{mailto:<juergen.beyerer@iosb.fraunhofer.de>?Subject=Your paper: Partially Observable Gaussian Process Network and Doubly Stochastic Variational Inference}{Jürgen Beyerer}}
% Add affiliations after the authors
\affil[1]{%
    Cognitive Industrial Systems\\
    Fraunhofer IOSB\\
    Karlsruhe, Germany
}
\affil[2]{%
    Karlsruhe Institute of Technology\\
    Karlsruhe, Germany
}

\begin{document}
\maketitle

\begin{abstract}
    \begin{abstract}  
Test time scaling is currently one of the most active research areas that shows promise after training time scaling has reached its limits.
Deep-thinking (DT) models are a class of recurrent models that can perform easy-to-hard generalization by assigning more compute to harder test samples.
However, due to their inability to determine the complexity of a test sample, DT models have to use a large amount of computation for both easy and hard test samples.
Excessive test time computation is wasteful and can cause the ``overthinking'' problem where more test time computation leads to worse results.
In this paper, we introduce a test time training method for determining the optimal amount of computation needed for each sample during test time.
We also propose Conv-LiGRU, a novel recurrent architecture for efficient and robust visual reasoning. 
Extensive experiments demonstrate that Conv-LiGRU is more stable than DT, effectively mitigates the ``overthinking'' phenomenon, and achieves superior accuracy.
\end{abstract}  
\end{abstract}

\section{Introduction}\label{sec:intro}
\section{Introduction}


\begin{figure}[t]
\centering
\includegraphics[width=0.6\columnwidth]{figures/evaluation_desiderata_V5.pdf}
\vspace{-0.5cm}
\caption{\systemName is a platform for conducting realistic evaluations of code LLMs, collecting human preferences of coding models with real users, real tasks, and in realistic environments, aimed at addressing the limitations of existing evaluations.
}
\label{fig:motivation}
\end{figure}

\begin{figure*}[t]
\centering
\includegraphics[width=\textwidth]{figures/system_design_v2.png}
\caption{We introduce \systemName, a VSCode extension to collect human preferences of code directly in a developer's IDE. \systemName enables developers to use code completions from various models. The system comprises a) the interface in the user's IDE which presents paired completions to users (left), b) a sampling strategy that picks model pairs to reduce latency (right, top), and c) a prompting scheme that allows diverse LLMs to perform code completions with high fidelity.
Users can select between the top completion (green box) using \texttt{tab} or the bottom completion (blue box) using \texttt{shift+tab}.}
\label{fig:overview}
\end{figure*}

As model capabilities improve, large language models (LLMs) are increasingly integrated into user environments and workflows.
For example, software developers code with AI in integrated developer environments (IDEs)~\citep{peng2023impact}, doctors rely on notes generated through ambient listening~\citep{oberst2024science}, and lawyers consider case evidence identified by electronic discovery systems~\citep{yang2024beyond}.
Increasing deployment of models in productivity tools demands evaluation that more closely reflects real-world circumstances~\citep{hutchinson2022evaluation, saxon2024benchmarks, kapoor2024ai}.
While newer benchmarks and live platforms incorporate human feedback to capture real-world usage, they almost exclusively focus on evaluating LLMs in chat conversations~\citep{zheng2023judging,dubois2023alpacafarm,chiang2024chatbot, kirk2024the}.
Model evaluation must move beyond chat-based interactions and into specialized user environments.



 

In this work, we focus on evaluating LLM-based coding assistants. 
Despite the popularity of these tools---millions of developers use Github Copilot~\citep{Copilot}---existing
evaluations of the coding capabilities of new models exhibit multiple limitations (Figure~\ref{fig:motivation}, bottom).
Traditional ML benchmarks evaluate LLM capabilities by measuring how well a model can complete static, interview-style coding tasks~\citep{chen2021evaluating,austin2021program,jain2024livecodebench, white2024livebench} and lack \emph{real users}. 
User studies recruit real users to evaluate the effectiveness of LLMs as coding assistants, but are often limited to simple programming tasks as opposed to \emph{real tasks}~\citep{vaithilingam2022expectation,ross2023programmer, mozannar2024realhumaneval}.
Recent efforts to collect human feedback such as Chatbot Arena~\citep{chiang2024chatbot} are still removed from a \emph{realistic environment}, resulting in users and data that deviate from typical software development processes.
We introduce \systemName to address these limitations (Figure~\ref{fig:motivation}, top), and we describe our three main contributions below.


\textbf{We deploy \systemName in-the-wild to collect human preferences on code.} 
\systemName is a Visual Studio Code extension, collecting preferences directly in a developer's IDE within their actual workflow (Figure~\ref{fig:overview}).
\systemName provides developers with code completions, akin to the type of support provided by Github Copilot~\citep{Copilot}. 
Over the past 3 months, \systemName has served over~\completions suggestions from 10 state-of-the-art LLMs, 
gathering \sampleCount~votes from \userCount~users.
To collect user preferences,
\systemName presents a novel interface that shows users paired code completions from two different LLMs, which are determined based on a sampling strategy that aims to 
mitigate latency while preserving coverage across model comparisons.
Additionally, we devise a prompting scheme that allows a diverse set of models to perform code completions with high fidelity.
See Section~\ref{sec:system} and Section~\ref{sec:deployment} for details about system design and deployment respectively.



\textbf{We construct a leaderboard of user preferences and find notable differences from existing static benchmarks and human preference leaderboards.}
In general, we observe that smaller models seem to overperform in static benchmarks compared to our leaderboard, while performance among larger models is mixed (Section~\ref{sec:leaderboard_calculation}).
We attribute these differences to the fact that \systemName is exposed to users and tasks that differ drastically from code evaluations in the past. 
Our data spans 103 programming languages and 24 natural languages as well as a variety of real-world applications and code structures, while static benchmarks tend to focus on a specific programming and natural language and task (e.g. coding competition problems).
Additionally, while all of \systemName interactions contain code contexts and the majority involve infilling tasks, a much smaller fraction of Chatbot Arena's coding tasks contain code context, with infilling tasks appearing even more rarely. 
We analyze our data in depth in Section~\ref{subsec:comparison}.



\textbf{We derive new insights into user preferences of code by analyzing \systemName's diverse and distinct data distribution.}
We compare user preferences across different stratifications of input data (e.g., common versus rare languages) and observe which affect observed preferences most (Section~\ref{sec:analysis}).
For example, while user preferences stay relatively consistent across various programming languages, they differ drastically between different task categories (e.g. frontend/backend versus algorithm design).
We also observe variations in user preference due to different features related to code structure 
(e.g., context length and completion patterns).
We open-source \systemName and release a curated subset of code contexts.
Altogether, our results highlight the necessity of model evaluation in realistic and domain-specific settings.






\section{Process Network with Partial Observability}\label{sec:multi_process}
We consider a stochastic process $\Process$ comprised of subprocesses, $\{\process^{(\numprocess)}\}$ for $\numprocess\in\numprocessset$, (which can be stochastic) as shown in Figure~\ref{fig:multi_process}. The process of stochasticity can come from either a lack of knowledge of the process or other hidden influences or random noise $\cusvector{\tilde{\noise}}^{(\numprocess)}\sim\probability(\cusvector{\noise}^{(\numprocess)})$. Process $\Process$ is a DAG where the nodes represent the subprocesses. Each subprocess $\process^{(\numprocess)}$ is governed by a transformation function $\cusvector{\processfunction}^{(\numprocess)}$ which takes as input(s) some adjustable parameters $\cusvector{\params}^{(\numprocess)}$ (represented with green arrow in Figure~\ref{fig:multi_process}) and the output(s) of parent subprocesses $\process^{\nodeparent{\numprocess}}$, where $\nodeparent{\numprocess}$ represents the direct parents of $\numprocess$. Using the transformation function, the contactenated input(s) $(\cusvector{\params}^{(\numprocess)}, \process^{\nodeparent{\numprocess}})$ are transformed into output(s) $\cusvector{\trueouts}^{(\numprocess)}$ (represented with red arrow).

Figure~\ref{fig:multi_process} shows an example stochastic process with two subprocesses that can be expressed using a distribution over the function space, which the transformation function $\cusvector{\processfunction}^{(\numprocess)}$ is a sampled from. Using this definition, the outputs of the two-process system shown in Figure~\ref{fig:multi_process} as red arrows can be redefined as
% \begin{equation*}
%       \cusvector{\trueouts}^{(1)} \sim \probability\big(\cusvector{\processfunction}^{(1)}\vert\cusvector{\params}^{(1)}\big) ;\;\;\cusvector{\trueouts}^{(2)} \sim \probability\big(\cusvector{\processfunction}^{(2)}\vert\cusvector{\params}^{(2)}, \cusvector{\processfunction}^{(1)}\big).
% \end{equation*}
\begin{equation*}
      \cusvector{\trueouts}^{(\numprocess)} \sim \probability\big(\cusvector{\processfunction}^{(\numprocess)}\vert\cusvector{\params}^{(\numprocess)}, \cusvector{\processfunction}^{(\nodeparent{\numprocess})}\big), \forall \numprocess\in\{1, 2\}.
\end{equation*}
In most cases, the output(s) $\cusvector{\trueouts}^{(\numprocess)}$ of a subprocess $\process^{(\numprocess)}$ cannot be fully observed and are observed indirectly/partially using an "observation lens" (represented with blue arrow) as $\cusvector{\tilde{\obsouts}}^{(\numprocess)}$, hence the name partially observable process network. An example of such an observation lens can be the Gaussian observation noise of a sensor. In probabilistic modeling, the "observation lens" is often modeled as the likelihood function $\cusvector{\tilde{\obsouts}}^{(\numprocess)}\sim\probability(\cusvector{\obsouts}^{(\numprocess)}\vert\cusvector{\processfunction}^{(\numprocess)})$, which could be as simple as additive Gaussian noise or something more complex. The true output(s) $\cusvector{\trueouts}^{(\numprocess)}$ remain latent. We assume that, the observation lens is always present whenever an output is referred to as "observed" unless stated as "latent" or "true" output. The "latent" output of a parent subprocess becomes the input for a child subprocess.

At this point we are able to define each subprocess $\process^{(\numprocess)}$ using a tuple $\langle\probability\big(\cusvector{\processfunction}^{(\numprocess)}\vert\cusvector{\params}^{(\numprocess)}, \cusvector{\trueouts}^{\nodeparent{\numprocess}}\big), \probability(\cusvector{\obsouts}^{(\numprocess)}\vert\cusvector{\processfunction}^{(\numprocess)})\rangle$, where $\cusvector{\params}^{(\numprocess)}$ represents the adjustable input parameters, $\cusvector{\trueouts}^{\nodeparent{\numprocess}}$ represents the latent output(s) from the parent subprocess(es), $\probability\big(\cusvector{\processfunction}^{(\numprocess)}\vert\cusvector{\params}^{(\numprocess)}, \cusvector{\trueouts}^{\nodeparent{\numprocess}}\big)$ represents the probability distribution over the transformation function and $\probability(\cusvector{\obsouts}^{(\numprocess)}\vert\cusvector{\processfunction}^{(\numprocess)})$ represents the observation likelihood or lens with which the latent output of the subprocess $\process^{(\numprocess)}$ is observed.

The process network $\Process$ can be represented using a DAG, $\graph$. The nodes are topologically ordered such that $\numprocess'<\numprocess ,\forall \numprocess' \in \nodeparent{\numprocess}$ for all $\numprocess\in\numprocessset$. We use the terms subprocess and node interchangeably to represent a subprocess in $\graph$. $\cusvector{\params}^{(\numprocess)}$ are addressed as adjustable input nodes or parameters. The final/end process of the process $\Process$ is defined as the subprocess(es) $\process^{(\numprocess)}$ for which there are no child nodes. The corresponding observed output(s) $\cusvector{\tilde{\obsouts}}^{(\numprocess)}$ are called the observed final output(s).

The problem statement is to model the subprocess output(s), and the end process output using the adjustable inputs of different subprocess(es) and all indirect observations of the process network made using different observation likelihoods. We assume that the data generation DAG or the causal path is known. We refrain from augmenting the intermediate observations with adjustable inputs to avoid blowing up the input dimensionality for the used model. We discuss the proposed solution in section~\ref{sec:pogpn}.

\section{Background}\label{sec:background}
\section{Background}\label{sec:backgrnd}

\subsection{Cold Start Latency and Mitigation Techniques}

Traditional FaaS platforms mitigate cold starts through snapshotting, lightweight virtualization, and warm-state management. Snapshot-based methods like \textbf{REAP} and \textbf{Catalyzer} reduce initialization time by preloading or restoring container states but require significant memory and I/O resources, limiting scalability~\cite{dong_catalyzer_2020, ustiugov_benchmarking_2021}. Lightweight virtualization solutions, such as \textbf{Firecracker} microVMs, achieve fast startup times with strong isolation but depend on robust infrastructure, making them less adaptable to fluctuating workloads~\cite{agache_firecracker_2020}. Warm-state management techniques like \textbf{Faa\$T}~\cite{romero_faa_2021} and \textbf{Kraken}~\cite{vivek_kraken_2021} keep frequently invoked containers ready, balancing readiness and cost efficiency under predictable workloads but incurring overhead when demand is erratic~\cite{romero_faa_2021, vivek_kraken_2021}. While these methods perform well in resource-rich cloud environments, their resource intensity challenges applicability in edge settings.

\subsubsection{Edge FaaS Perspective}

In edge environments, cold start mitigation emphasizes lightweight designs, resource sharing, and hybrid task distribution. Lightweight execution environments like unikernels~\cite{edward_sock_2018} and \textbf{Firecracker}~\cite{agache_firecracker_2020}, as used by \textbf{TinyFaaS}~\cite{pfandzelter_tinyfaas_2020}, minimize resource usage and initialization delays but require careful orchestration to avoid resource contention. Function co-location, demonstrated by \textbf{Photons}~\cite{v_dukic_photons_2020}, reduces redundant initializations by sharing runtime resources among related functions, though this complicates isolation in multi-tenant setups~\cite{v_dukic_photons_2020}. Hybrid offloading frameworks like \textbf{GeoFaaS}~\cite{malekabbasi_geofaas_2024} balance edge-cloud workloads by offloading latency-tolerant tasks to the cloud and reserving edge resources for real-time operations, requiring reliable connectivity and efficient task management. These edge-specific strategies address cold starts effectively but introduce challenges in scalability and orchestration.

\subsection{Predictive Scaling and Caching Techniques}

Efficient resource allocation is vital for maintaining low latency and high availability in serverless platforms. Predictive scaling and caching techniques dynamically provision resources and reduce cold start latency by leveraging workload prediction and state retention.
Traditional FaaS platforms use predictive scaling and caching to optimize resources, employing techniques (OFC, FaasCache) to reduce cold starts. However, these methods rely on centralized orchestration and workload predictability, limiting their effectiveness in dynamic, resource-constrained edge environments.



\subsubsection{Edge FaaS Perspective}

Edge FaaS platforms adapt predictive scaling and caching techniques to constrain resources and heterogeneous environments. \textbf{EDGE-Cache}~\cite{kim_delay-aware_2022} uses traffic profiling to selectively retain high-priority functions, reducing memory overhead while maintaining readiness for frequent requests. Hybrid frameworks like \textbf{GeoFaaS}~\cite{malekabbasi_geofaas_2024} implement distributed caching to balance resources between edge and cloud nodes, enabling low-latency processing for critical tasks while offloading less critical workloads. Machine learning methods, such as clustering-based workload predictors~\cite{gao_machine_2020} and GRU-based models~\cite{guo_applying_2018}, enhance resource provisioning in edge systems by efficiently forecasting workload spikes. These innovations effectively address cold start challenges in edge environments, though their dependency on accurate predictions and robust orchestration poses scalability challenges.

\subsection{Decentralized Orchestration, Function Placement, and Scheduling}

Efficient orchestration in serverless platforms involves workload distribution, resource optimization, and performance assurance. While traditional FaaS platforms rely on centralized control, edge environments require decentralized and adaptive strategies to address unique challenges such as resource constraints and heterogeneous hardware.



\subsubsection{Edge FaaS Perspective}

Edge FaaS platforms adopt decentralized and adaptive orchestration frameworks to meet the demands of resource-constrained environments. Systems like \textbf{Wukong} distribute scheduling across edge nodes, enhancing data locality and scalability while reducing network latency. Lightweight frameworks such as \textbf{OpenWhisk Lite}~\cite{kravchenko_kpavelopenwhisk-light_2024} optimize resource allocation by decentralizing scheduling policies, minimizing cold starts and latency in edge setups~\cite{benjamin_wukong_2020}. Hybrid solutions like \textbf{OpenFaaS}~\cite{noauthor_openfaasfaas_2024} and \textbf{EdgeMatrix}~\cite{shen_edgematrix_2023} combine edge-cloud orchestration to balance resource utilization, retaining latency-sensitive functions at the edge while offloading non-critical workloads to the cloud. While these approaches improve flexibility, they face challenges in maintaining coordination and ensuring consistent performance across distributed nodes.



\subsection{Gaussian Process Network and their limitations}\label{sec:gp_related_work}
Gaussian Process Networks (GPN) coined by~\cite{friedman2000gpn} and extended by \cite{giudice2024bayesian} addresses the learning the Bayesian network structure and not the inference, which is different from our work. Gaussian Process Regression Networks (GPRN) by~\cite{friedman2000gpn, wilson2011gaussian} provide a different perspective by combining modeling the final output as a linear combination of Gaussian processes (like neural network structure).
GPRN does not incorporate the intermediate observations and caters to a problem statement different from ours.

GPN introduced by~\cite{astudillo2021bayesian, aglietti2020causal} as the surrogate model would translate the toy process network shown in Figure~\ref{fig:multi_process} into the DAG shown in Figure~\ref{fig:toy_process_gpn}. The grey-shaded nodes represent the observed outputs, and the unshaded nodes represent the unobserved latent outputs. The so far introduced GPNs model each subprocess $\process^{(\numprocess)}$ as a Gaussian process with mean $\gpmean^{(\numprocess)}(\cdot)$ and variance $\gpkernel^{(\numprocess)}(\cdot, \cdot')$, where $\processfunction^{(\numprocess)}(\cdot)\sim\mathcal{GP}^{(\numprocess)}(\gpmean^{(\numprocess)}(\cdot),\gpkernel^{(\numprocess)}(\cdot, \cdot'))$ and the observation(s) of the subprocess $\process^{(\numprocess)}$ can be expressed using the respective Gaussian likelihood as $\probability(\obsouts^{\numprocess}\vert\processfunction^{(\numprocess)})$. The model assumes that each node GP is independent of the other GP given the observed input-output pairs and uses closed-form marginal log-likelihood (MLL)~\citep{rasmussen2003gaussian} for inference.
% Since a GP provides a distribution as output(s) rather than point prediction(s), 
MC samples are used to estimate the final output during prediction. The implemented setup poses four major limitations:
\begin{enumerate}
      \item In most real-world cases, one can only observe the state space partially using an indirect observation lens. The latent outputs of the subprocesses are often hidden. Because of this, it can be considered that it is the latent/true outputs $\cusvector{\trueouts}^{(\numprocess)}$ that influence the process and not the indirect observations $\cusvector{\obsouts}^{(\numprocess)}$. Figure~\ref{fig:toy_process_gpn} represents that the indirect observations become the input to the respective child node. It holds only for direct noise-free observations.
      \item Using GP, the output of a parent node GP is a distribution, not a point value. Due to this reason, closed-form MLL cannot be used. Closed-form MLL can be used for deterministic inputs. Also, the prediction contradicts the training as MC samples are used to calculate from predictive distribution.
      \item The use of exact MLL also limits the usage to only Gaussian observation likelihood. Although this has computational benefits, it cannot be applied to cases where the intermediate observations are observed using a non-Gaussian lens.
            % In addition, current GPNs do not allow for amortized likelihood in the case of high-dimensional intermediate observations.
      \item Using the inference method of existing GPN, one can only condition a particular node when observed. However, since the model is a network, one should be able to condition the connected subprocesses based on a particular subprocess observation.
\end{enumerate}

The models, demonstrated in~\cite{sussex2022model, kusakawa2022bayesian}, show promising results by overcoming the first limitation. However, they still rely on closed-form MLL for independent node inference, leaving the other limitations unaddressed. The models of~\cite{aglietti2020causal, sussex2022model, astudillo2021bayesian} were primarily proposed as a surrogate model for Bayesian optimization with intermediate observations, hinting at the potential for further development and improvement.

Another recently proposed variant of GPN is the Gaussian Process Autoregressive Regression model~\citep{requeima2019gaussian}, where the focus is on autoregressive modeling of each observed output. The prediction(s) of the previous output(s) are used as input(s) for the GP, which is further down the autoregressive flow. The outputs are ordered greedily, using an exhaustive search, but scalability is not well discussed. Although GPAR mentions the use of inducing points in D-GPAR-NL, it does not provide a training method for joint distribution loss and either assume fixed inducing locations or individual node GP training. Additionally, GPAR does not cater to the second and third limitations or provide an optimization method for inducing point formulation.

\section{Partially Observable Gaussian Process Network}\label{sec:pogpn}
We present our main contribution, the Partially Observable Gaussian Process Network (POGPN). Instead of node observations sharing a common distribution space, we propose that the latent functions reside in the same space and influence the child subprocess nodes. POGPN represents the process network in section~\ref{sec:multi_process} using a DAG where the nodes, $\numprocess\in\numprocessset$, are topologically ordered such that $\numprocess'<\numprocess,\forall \numprocess' \in \nodeparent{\numprocess}$. Each node is modeled as a $\mathcal{GP}^{(\numprocess)}$ similar to the GPN setup in section~\ref{sec:gp_related_work}, generalized as a vector-valued function $\cusvector{\trueouts^{(\numprocess)}}_{\numobservation}\sim\mathcal{GP}^{(\numprocess)}(\cdot, \cdot')$ and can or cannot be observed using an arbitrary likelihood $\probability(\cusvector{\obsouts}^{(\numprocess)}_{\numobservation}\vert\cusvector{\processfunction}^{(\numprocess)}_{\numobservation})$. The generalized notation allows for nodes to have different dimensionality. This formulation allows us to consider DGP a special POGPN case where only the last node observations are available. POGPN, with its assumption of arbitrary observation likelihood and common latent function space, provides a way to model continuous and categorical observations with the same model.

Unlike the existing GPNs, a subprocess $\process^{(\numprocess)}$ takes the parent node latent GP functions $\cusvector{\processfunction}^{\nodeparent{\numprocess}}$ as the input rather than an instance of the noisy indirect observations $\cusvector{\tilde{\obsouts}}^{\nodeparent{\numprocess}}$. Since we can express the node's latent function using a distribution, we can take the expectation over the parent node distribution. The expectation over possible parent node output(s) provides robustness against parent subprocess stochasticity and can separate the observation lens from the actual process. This setup also allows for arbitrary observation likelihoods as the observation is separated from the network. Using POGPN, the process network $\Process$ in Figure~\ref{fig:multi_process} would be represented as Figure~\ref{fig:toy_process_pogpn}.

\textbf{Evidence Lower BOund (ELBO).} Similar to DGP in section~\ref{sec:DGP}, we introduce inducing points, $\cusmatrix{\Inducinglocations}^{\nodeparent{\numprocess}}, \cusmatrix{\Inducinglocations}^{\paramsset^{(\numprocess)}}$, in the space of the parent nodes and node inputs respectively, such that $\cusmatrix{\Inducinglocations}^{(\numprocess)}=(\cusmatrix{\Inducinglocations}^{\nodeparent{\numprocess}}, \cusmatrix{\Inducinglocations}^{\paramsset^{(\numprocess)}})$. We wish to approximate the posterior $\probability\big(\{\cusvector{\processfunction}_{\numobservation}^{(\numprocess)}\}_{\numprocess\in\numprocessset}\vert\{\cusvector{\obsouts}_{\numobservation}^{(\numprocess)}; \cusvector{\params}_{\numobservation}^{(\numprocess)}\}_{\numprocess\in\numprocessset}\big)$ with the variational posterior $\variationalprob\big(\{\cusvector{\processfunction}_{\numobservation}^{(\numprocess)}\}_{\numprocess\in\numprocessset}\big)$ which can be expressed using~\eqref{eq:DGP_marginal} as
\begin{equation}\label{eq:POGPN_variational_marginal}
      \variationalprob \big(\{\cusvector{\processfunction}_{\numobservation}^{(\numprocess)}\}_{\numprocess\in\numprocessset}\big) = \prod_{\numprocess\in\numprocessset}\variationalprob\big(\cusvector{\processfunction}_{\numobservation}^{(\numprocess)};\{\cusvector{\processfunction}_{\numobservation}^{\nodeparent{\numprocess}},\cusvector{\params}_{\numobservation}^{(\numprocess)}\}, \cusmatrix{\Inducinglocations}^{(\numprocess)}\big).
\end{equation}
The Kullback Leibler (KL)~\citep{shlens2014notes} divergence between the variational posterior and true posterior can be expressed as
\begin{equation}
      \begin{gathered}\label{eq:POGPN_KL_variational}
            - \text{KL}\big(\variationalprob(\{\cusvector{\processfunction}_{\numobservation}^{(\numprocess)}\}_{\numprocess\in\numprocessset})\Vert\probability(\{\cusvector{\processfunction}_{\numobservation}^{(\numprocess)}\}_{\numprocess\in\numprocessset}\vert\{\cusvector{\obsouts}_{\numobservation}^{(\numprocess)},\cusvector{\params}_{\numobservation}^{(\numprocess)}\}_{\numprocess\in\numprocessset})\big) \\
            =\underbrace{\mathbb{E}_{\variationalprob(\{\cusvector{\processfunction}_{\numobservation}^{(\numprocess)}\}_{\numprocess\in\numprocessset})}\big[\log\probability(\{\cusvector{\obsouts}_{\numobservation}^{(\numprocess)}\}_{\numprocess\in\numprocessset}\vert\{\cusvector{\processfunction}_{\numobservation}^{(\numprocess)},\cusvector{\params}_{\numobservation}^{(\numprocess)}\}_{\numprocess\in\numprocessset})\big]}_{\text{Log Likelihood Loss (LL loss)}} \\ \underbrace{-\text{KL}\big(\variationalprob(\{\cusvector{\processfunction}_{\numobservation}^{(\numprocess)}\}_{\numprocess\in\numprocessset})\Vert\probability(\{\cusvector{\processfunction}_{\numobservation}^{(\numprocess)}\}_{\numprocess\in\numprocessset})\big)}_{\text{KL loss}} + \text{ Evidence}
            % \underbrace{-\log\probability(\{\cusvector{\obsouts}_{\numobservation}^{(\numprocess)}\}_{\numprocess\in\numprocessset}\vert\{\cusvector{\params}_{\numobservation}^{(\numprocess)}\}_{\numprocess\in\numprocessset})}_{\text{Evidence (E)}}.
      \end{gathered}
\end{equation}
The ELBO for POGPN can then be defined as the combination of the "LL loss" and the "KL loss" term of~\eqref{eq:POGPN_KL_variational}. We now show how the terms of the ELBO can be simplified so that inference can be performed.

The conditional distribution $\probability(\{\cusvector{\obsouts}_{\numobservation}^{(\numprocess)}\}_{\numprocess\in\numprocessset}\vert\{\cusvector{\processfunction}_{\numobservation}^{(\numprocess)}\}_{\numprocess\in\numprocessset})$ can be simplified using the DAG structure as
\begin{align}\label{eq:DAG_cond_independent}
       & \probability(\{\cusvector{\obsouts}_{\numobservation}^{(\numprocess)}\}_{\numprocess\in\numprocessset}\vert\{\cusvector{\processfunction}_{\numobservation}^{(\numprocess)}\}_{\numprocess\in\numprocessset})\nonumber                                                                                                                                                                        \\
       & = \probability(\cusvector{\obsouts}_{\numobservation}^{(\Numprocess)}\vert\{\cusvector{\processfunction}_{\numobservation}^{(\numprocess)}\}_{\numprocess\in\numprocessset})\probability(\cusvector{\obsouts}_{\numobservation}^{(\Numprocess-1)},\{\cusvector{\processfunction}_{\numobservation}^{(\numprocess)}\}_{\numprocess\in\numprocessset})\nonumber                                 \\
       & = \prod_{\numprocess\in\numprocessset}\probability(\cusvector{\obsouts}_{\numobservation}^{(\numprocess)}\vert\{\cusvector{\processfunction}_{\numobservation}^{(\numprocess)}\}_{\numprocess\in\numprocessset})= \prod_{\numprocess\in\numprocessset}\probability(\cusvector{\obsouts}_{\numobservation}^{(\numprocess)}\vert\cusvector{\processfunction}_{\numobservation}^{(\numprocess)})
\end{align}
where $\Numprocess=\vert\numprocessset\vert$. Using~\eqref{eq:POGPN_variational_marginal} and~\eqref{eq:DAG_cond_independent}, the "LL loss" in Equation\eqref{eq:POGPN_KL_variational}, can be expressed as
\begin{align}\label{eq:pogpn_elbo_ll}
        & \mathbb{E}_{\variationalprob(\{\cusvector{\processfunction}_{\numobservation}^{(\numprocess)}\}_{\numprocess\in\numprocessset})}\big[\log\probability(\{\cusvector{\obsouts}_{\numobservation}^{(\numprocess)}\}_{\numprocess\in\numprocessset}\vert\{\cusvector{\processfunction}_{\numobservation}^{(\numprocess)},\cusvector{\params}_{\numobservation}^{(\numprocess)}\}_{\numprocess\in\numprocessset})\big]\nonumber \\
      = & \mathbb{E}_{\variationalprob(\{\cusvector{\processfunction}_{\numobservation}^{(\numprocess)}\}_{\numprocess\in\numprocessset})}\big[\sum_{\numprocess\in\numprocessset}\log\probability(\cusvector{\obsouts}_{\numobservation}^{(\numprocess)}\vert\cusvector{\processfunction}_{\numobservation}^{(\numprocess)})\big]\nonumber                                                                                          \\
      = & \sum_{\numprocess\in\numprocessset} \mathbb{E}_{\variationalprob(\cusvector{\processfunction}_{\numobservation}^{(\numprocess)})}\big[\log\probability(\cusvector{\obsouts}_{\numobservation}^{(\numprocess)}\vert\cusvector{\processfunction}_{\numobservation}^{(\numprocess)})\big],
\end{align}
where the marginal $\variationalprob(\cusvector{\processfunction}_{\numobservation}^{(\numprocess)})$ can be calculated using~\eqref{eq:svgp_marginal_qf} and~\eqref{eq:svgp_marginal_qf_dist} as
\begin{align}\label{eq:POGPN_marginal}
      \variationalprob(\cusvector{\processfunction}_{\numobservation}^{(\numprocess)}) = \int\prod^{\numprocess}_{j\in\numprocessset}\variationalprob\big(\cusvector{\processfunction}_{\numobservation}^{(j)}\vert\cusvector{\processfunction}_{\numobservation}^{\nodeparent{j}}, \cusvector{\params}_{\numobservation}^{(j)}, \cusmatrix{\Inducinglocations}^{(j)}\big)\text{d}\cusvector{\processfunction}_{\numobservation}^{\nodeparent{j}}.
\end{align}
For $\Numobservation^{(\numprocess)}$ observations for each node $\numprocess$, the inference can be made by minimizing the negative EBLO for POGPN as
\begin{align}\label{eq:pogpn_elbo}
      \mathcal{L}_{\substack{\text{POGPN}                                                                                                                                                                                          \\\text{ELBO}}}^{(\numprocessset)} = &\sum_{\numprocess\in\numprocessset} \mathcal{L}_{\substack{\text{Node} \\ \text{ELBO}}} ^{(\numprocess)} \nonumber \\=&\underbrace{-\sum_{\numprocess\in\numprocessset}\frac{1}{\pogpnnormconst^{(\numprocess)}}\sum^{\Numobservation^{(\numprocess)}}_{\numobservation}\mathbb{E}_{\variationalprob(\cusvector{\processfunction}^{(\numprocess)}_{\numobservation})}\big[\log\probability(\cusvector{\obsouts}^{(\numprocess)}_{\numobservation}\vert\cusvector{\processfunction}^{(\numprocess)}_{\numobservation})\big]}_{\text{LL loss}}\nonumber \\
      + & \underbrace{\klconst\sum_{\numprocess\in\numprocessset}\text{KL}\big(\variationalprob(\cusvector{\inducingpoints}^{(\numprocess)})\Vert\probability(\cusvector{\inducingpoints}^{(\numprocess)})\big)}_{\text{KL loss}},
\end{align}
a normalization constant $\pogpnnormconst^{(\numprocess)}$ is introduced to keep the likelihood loss from different nodes comparable when the dimensionality of the node is not the same. We propose to keep $\pogpnnormconst^{(\numprocess)}=\dims{\obsouts^{(\numprocess)}}$ to keep the "LL loss" term of different nodes comparable and give equal importance to each node, where $\dims{\obsouts^{(\numprocess)}}$ is the dimension of the observed output $\obsouts^{(\numprocess)}$. However, $\pogpnnormconst^{(\numprocess)}$ incorporate importance-based training where the emphasis lies on a particular node. If a node $\numprocess'$ has no observation likelihood, then $\numprocess'$ will contribute to only the "KL loss" and not to the "LL loss".

\textbf{Predictive Log Likelihood (PLL) loss.} Using the inspiration from the PLL loss in~\ref{svgp:pll}~\citep{jankowiak2020parametric}, the "LL loss" for PLL can be defined for POGPN as
\begin{equation}\label{eq:pogpn_pll}
      \text{LL}_{\text{PLL}} = \underbrace{-\sum_{\numprocess\in\numprocessset}\frac{1}{\pogpnnormconst^{(\numprocess)}}\sum^{\Numobservation^{(\numprocess)}}_{\numobservation=1}\log\mathbb{E}_{\variationalprob(\cusvector{\processfunction}^{(\numprocess)}_{\numobservation})}\big[\probability(\cusvector{\obsouts}^{(\numprocess)}_{\numobservation}\vert\cusvector{\processfunction}^{(\numprocess)}_{\numobservation})\big]}_{\text{LL loss}},
      % \nonumber \\ & +\underbrace{\klconst\sum_{\numprocess\in\numprocessset}\text{KL}\big(\variationalprob(\cusvector{\inducingpoints}^{(\numprocess)})\Vert\probability(\cusvector{\inducingpoints}^{(\numprocess)})\big)}_{\text{KL loss}}
\end{equation}
while the "KL loss" is the same as~\eqref{eq:pogpn_elbo}. Like DGP inference, we use MC samples to calculate the "LL loss" for POGPN ELBO in~\eqref{eq:pogpn_elbo}. It is common to calculate log probabilities to avoid loss of precision, and a direct summation of log probabilities of MC samples would lead to expected log marginal likelihood rather than log expected marginal likelihood in~\eqref{eq:pogpn_pll}. Using Jensen's inequality, one can prove that the former is an unbiased estimator of the latter. \cite{jankowiak2020deep} provides the sigma point method as a solution, but it is not easy to scale. We propose a more straightforward approach, where we use \verb|logsumexp| to calculate the log of expected likelihood marginal (LL loss) of PLL using MC samples as
\begin{align}\label{eq:mc_pogpn_pll_ll}
      \text{LL}_\text{PLL} & = -\sum_{\numprocess\in\numprocessset} \frac{1}{\pogpnnormconst^{(\numprocess)}} \sum^{\Numobservation^{(\numprocess)}}_{\numobservation=1} \log \Big(\frac{1}{\MCsamples}\sum^{\MCsamples}_{\mcsamples=1} \probability \big(\cusvector{\obsouts}^{(\numprocess)}_{\numobservation} \big| \cusvector{\trueouts}^{(\numprocess)}_{\numobservation, \mcsamples} \big)\Big)
\end{align} where $\cusvector{\trueouts}^{(\numprocess)}_{\numobservation, \mcsamples}\sim\variationalprob(\cusvector{\processfunction}_{\numobservation}^{(\numprocess)})$. This formulation has a tighter lower bound to the log expected marginal likelihood in comparison to the unbiased estimator. For generalization, we call ELBO and PLL the "loss" for POGPN. MC samples, used while training, can be considered analogous to training the child process on many hypothesized parent true/latent outputs, and the variational inference allows for robustness against the stochasticity of parent subprocesses.

We now present two methods, namely ancestor-wise and node-wise, for training POGPN . These methods can use either of the factorized losses, ELBO~\eqref{eq:pogpn_elbo} or PLL~\eqref{eq:pogpn_pll}. For $\Numprocess$ nodes, $\Numobservation$ observations, and $\Numinducing$ points for each node, the computational complexity is $\mathcal{O}(\Numprocess(\Numobservation\Numinducing^2+\Numinducing^3))$.

\textbf{Ancestor-wise Training.}
Algorithm~\ref{alg:pogpn_ancestorwise}  called POGPN-AL can be implemented using~\eqref{eq:pogpn_elbo} and~\eqref{eq:pogpn_pll}, where $\text{Anc}(\numprocessset_{\text{obs}})$ represents the set of all ancestors of each node $\numprocess\in\numprocessset_{\text{obs}}$; $\cusvector[bm]{\gphyperparam}^{(\numprocess)}$ represent the GP hyperparameters (mean, kernel and variational) and $\cusvector[bm]{\likelihoodparam}^{(\numprocess)}$ and likelihood hyperparameters of node $\numprocess$.
\begin{algorithm}
      \SetAlgoLined
      \caption{POGPN Ancestor-wise Loss (POGPN-AL) training. Given $\{\cusmatrix{\Obsouts}^{(\numprocess)}\}_{\numprocess\in\numprocessset_{\text{obs}}}$ observations for $\numprocessset_{\text{obs}}$,  GP hyperparameters of observed nodes $\cusvector[bm]\gphyperparam^{(\numprocessset_{\text{obs}})}$ and their ancestors $\cusvector[bm]\gphyperparam^{(\nodeancestor{\numprocessset_{\text{obs}}})}$ along with hyperparameters of observed likelihoods $\cusvector[bm]{\likelihoodparam}^{(\numprocessset_{\text{obs}})}$ are trained. One can use either ELBO or PLL loss from~\ref{eq:pogpn_elbo} or~\ref{eq:pogpn_pll} as $\mathcal{L}_{\text{POGPN}}$. $\cusvector[bm]{\gamma}^{(\numprocessset_{\text{obs}})}=(\cusvector[bm]{\gphyperparam}^{(\numprocessset_{\text{obs}})}, \cusvector[bm]{\gphyperparam}^{(\nodeancestor{\numprocessset_{\text{obs}}})},  \cusvector[bm]{\likelihoodparam}^{(\numprocessset_{\text{obs}})})$.}\label{alg:pogpn_ancestorwise}

      % \SetKwOutput{Output}{Output} % Define the Output keyword
      \SetKwInput{Input}{Input} % Define the Input keyword

      \Input{Training data: $\{\mathcal{D}^{(\numprocess)}\}_{\numprocess\in\numprocessset_{\text{obs}}} = \{\cusmatrix{\Obsouts}^{(\numprocess)}, \cusmatrix{\Params}^{(\numprocess)}\}_{\numprocess\in\numprocessset_{\text{obs}}}$}
      \SetInd{2em}{1em}
      \Indp % Start indentation
      Loss: $\mathcal{L}_{\text{POGPN}}$\\
      Hyperparameters: $\cusvector[bm]{\gamma}^{(\numprocessset_{\text{obs}})}$\\
      Gradient optimizer: \texttt{optim}\\
      \Indm % End indentation
      \SetInd{1em}{1em}
      \While{not converged}{
      Compute $\variationalprob(\{\cusmatrix{\Processfunction}^{(\numprocess)}\}_{\numprocess \in \numprocessset_{\text{obs}}})$ using MC samples\;
      Compute $\mathcal{L}_{\text{POGPN}}(\numprocessset_{\text{obs}})$ using $\variationalprob(\{\cusmatrix{\Processfunction}^{(\numprocess)}\}_{\numprocess \in \numprocessset_{\text{obs}}})$ and $\{\mathcal{D}^{(\numprocess)}\}_{\numprocess\in\numprocessset_{\text{obs}}}$\;
      Gradient step: $\cusvector[bm]{\gamma}^{(\numprocessset_{\text{obs}})} \leftarrow \texttt{optim}(\mathcal{L}_{\text{POGPN}}^{(\numprocessset_{\text{obs}})})$\;
      }
      \KwOut{Optimized hyperparameters: $\cusvector[bm]{\gamma}^{(\numprocessset_{\text{obs}})}$}
\end{algorithm}
We call Algorithm~\ref{alg:pogpn_ancestorwise} ancestor-wise training as it updates the parameters of all ancestor GPs of the observed nodes. This method is similar to the traditional method of training DGP, just that we consider multiple observation nodes in POPGN. It is beneficial when either all network nodes are observed or the nodes further in the graph are observed, and one wishes to condition the ancestor node(s) based on the observations of the child/successor node(s).

\begin{figure}[h]
      \centering
      We compare the performance of agents trained on data from the InSTA pipeline to agents trained on human demonstrations from WebLINX \citep{WebLINX} and Mind2Web \citep{Mind2Web}, two recent and popular benchmarks for web navigation. Recent works that mix synthetic data with real data control the real data sampling probability in the batch $p_{\text{real}}$ independently from data size \citep{DAFusion}. We employ $p_{\text{real}} = 0.5$ in few-shot experiments and $p_{\text{real}} = 0.8$ otherwise. Shown in Figure~\ref{fig:data-statistics}, our data have a wide spread in performance, so we apply several filtering rules to select high-quality training data. First, we require the evaluator to return \texttt{conf} = 1 that the task was successfully completed, and that the agent was on the right track (this selects data where the actions are reliable, and directly caused the task to be solved). Second, we filter data where the trajectory contains at least three actions. Third, we remove data where the agent encountered any type of server error, was presented with a captcha, or was blocked at any point. These steps produce $7,463$ high-quality demonstrations in which agents successfully completed tasks on diverse websites. We sample 500 demonstrations uniformly at random from this pool to create a diverse test set, and employ the remaining $6,963$ demonstrations to train agents on a mix of real and synthetic data.

\subsection{Improving Data-Efficiency}
\label{sec:few-shot}

\begin{wrapfigure}{r}{0.48\textwidth}
    \centering
    \vspace{-0.8cm}
    \includegraphics[width=\linewidth]{assets/few_shot_results_weblinx_mind2web.pdf}
    \vspace{-0.3cm}
    \caption{\small \textbf{Data from InSTA improves efficiency.} Language model agents trained on mixtures of our data and human demonstrations scale faster than agents trained on human data. In a setting with 32 human actions, adding our data improves \textit{Step Accuracy} by +89.5\% relative to human data for Mind2Web, and +122.1\% relative to human data for WebLINX.}
    \vspace{-0.2cm}
    \label{fig:few-shot-results}
\end{wrapfigure}

In a data-limited setting derived from WebLINX \citep{WebLINX} and Mind2Web \citep{Mind2Web}, agents trained on our data \textit{scale faster with increasing data size} than human data alone. Without requiring laborious human annotations, the data produced by our pipeline leads to improvements on Mind2Web that range from +89.5\% in \textit{Step Accuracy} (the rate at which the correct element is selected and the correct action is performed on that element) with 32 human actions, to +77.5\% with 64 human actions, +13.8\% with 128 human actions, and +12.1\% with 256 human actions. For WebLINX, our data improves by +122.1\% with 32 human actions, and +24.6\% with 64 human actions, and +6.2\% for 128 human actions. Adding our data is comparable in performance gained to doubling the amount of human data from 32 to 64 actions. Performance on the original test sets for Mind2Web and WebLINX appears to saturate as the amount of human data increases, but these benchmark only test agent capabilities for a limited set of 150 popular sites.

\subsection{Improving Generalization} 
\label{sec:generalization}

\begin{wrapfigure}{r}{0.48\textwidth}
    \centering
    \vspace{-1.0cm}
    \includegraphics[width=\linewidth]{assets/diverse_results_weblinx_mind2web.pdf}
    \vspace{-0.3cm}
    \caption{\small \textbf{Our data improves generalization.} We train agents with all human data from the WebLINX and Mind2Web training sets, and resulting agents struggle to generalize to more diverse test data. Adding our data improves generalization by +149.0\% for WebLINX, and +156.3\% for Mind2Web.}
    \vspace{-0.3cm}
    \label{fig:generalization-results}
\end{wrapfigure}

To understand how agents trained on data from our pipeline generalize to diverse real-world sites, we construct a more diverse test set than Mind2Web and WebLINX using 500 held-out demonstrations produced by our pipeline. Shown in Figure~\ref{fig:generalization-results}, we train agents using all human data in the training sets for WebLINX and Mind2Web, and compare the performance with agents trained on 80\% human data, and 20\% data from our pipeline. Agents trained with our data achieve comparable performance to agents trained purely on human data on the official test sets for the WebLINX and Mind2Web benchmarks, suggesting that when enough human data are available, synthetic data may not be necessary. However, when evaluated in a more diverse test set that includes 500 sites not considered by existing benchmarks, agents trained purely on existing human data struggle to generalize. Training with our data improves generalization to these sites by +149.0\% for WebLINX agents, and +156.3\% for Mind2Web agents, with the largest gains in generalization \textit{Step Accuracy} appearing for harder tasks.
      % \vspace{-1em}
      \caption{Training methods POGPN for a given structure. If $\numprocessset_{\text{obs}}=\{\cusvector{\obsouts}^{(4)}, \cusvector{\obsouts}^{(5)}\}$, POGPN-AL includes hyperparameters for node $\cusvector[bm]{\gamma}^{(\numprocessset_{\text{obs}})}=(\cusvector[bm]{\gphyperparam}^{(\numprocessset_{\text{obs}})}, \cusvector[bm]{\gphyperparam}^{(\nodeancestor{\numprocessset_{\text{obs}}})},  \cusvector[bm]{\likelihoodparam}^{(\numprocessset_{\text{obs}})})$ bounded by the blue dashed box. POGPN-NL trains hyperparameters, $\cusvector[bm]{\gamma}^{(\numprocess)} = (\cusvector[bm]{\gphyperparam}^{(\numprocess)}, \cusvector[bm]{\likelihoodparam}^{(\numprocess)}), \forall\numprocess\in\numprocessset_{\text{obs}}$, node-wise as bounded by red dashed boxes. Gray nodes represent observed output nodes (likelihood), and white nodes represent latent output nodes (GP).}
      \vspace{-1em}
      \label{fig:training}
\end{figure}

\begin{algorithm}
      \SetAlgoLined
      \caption{POGPN Node-wise Loss (POGPN-NL) training. Given $\{\Numobservation^{(\numprocess)}\}_{\numprocess\in\numprocessset_{\text{obs}}}$ observations for $\numprocessset_{\text{obs}}$, the GP hyperparameters $\cusvector[bm]\gphyperparam^{(\numprocess)}$ and likelihood hyperparameters $\cusvector[bm]{\likelihoodparam}^{(\numprocess)}$ are trained for one node at a time for $\numprocess\in\numprocessset_{\text{obs}}$. One can use either ELBO or PLL loss from~\ref{eq:pogpn_elbo} or~\ref{eq:pogpn_pll} as $\mathcal{L}_{\text{POGPN}}$. $\cusvector[bm]{\gamma}^{(\numprocessset_{\text{obs}})}=(\cusvector[bm]{\gphyperparam}^{(\numprocessset_{\text{obs}})}, \cusvector[bm]{\likelihoodparam}^{(\numprocessset_{\text{obs}})})$.}\label{alg:pogpn_node-wise}

      \SetKwInput{Input}{Input} % Define the Input keyword

      \Input{Training data: $\{\mathcal{D}^{(\numprocess)}\}_{\numprocess\in\numprocessset_{\text{obs}}} = \{\cusmatrix{\Obsouts}^{(\numprocess)}, \cusmatrix{\Params}^{(\numprocess)}\}_{\numprocess\in\numprocessset_{\text{obs}}}$}
      \SetInd{2em}{1em}
      \Indp % Start indentation
      Loss: $\mathcal{L}_{\text{POGPN}}$\\
      Gradient optimizer: \texttt{optim}\\
      \Indm % End indentation
      \SetInd{1em}{1em}

      \While{not converged}{
      \For{$\numprocess \in \text{\texttt{topological\_sort}}(\numprocessset_{\text{obs}})$}{
      Hyperparameters: $\cusvector[bm]{\gamma}^{(\numprocess)} = (\cusvector[bm]{\gphyperparam}^{(\numprocess)}, \cusvector[bm]{\likelihoodparam}^{(\numprocess)})$\\
      Compute $\variationalprob(\cusmatrix{\Processfunction}^{(\numprocess)})$ using MC samples\;
      Compute $\mathcal{L}_{\text{node}}^{(\numprocess)}$ using $\variationalprob(\cusmatrix{\Processfunction}^{(\numprocess)})$ and  $\mathcal{D}^{(\numprocess)}$\;
      Gradient step $\cusvector[bm]{\gamma}^{(\numprocess)} \leftarrow \texttt{optim}(\mathcal{L}_{\text{node}}^{(\numprocess)})$\;
      }
      }
      \KwOut{Optimized hyperparameters: $\cusvector[bm]{\gamma}^{(\numprocessset_{\text{obs}})}$}
\end{algorithm}
% \vspace{-1.5em}

\textbf{Node-wise Training.} Algorithm~\ref{alg:pogpn_node-wise}  called POGPN-NL, follows a coordinate ascent method for updating individual node GP hyperparameters $\cusvector[bm]{\gphyperparam}^{(\numprocess)}$ and likelihood hyperparameters $\cusvector[bm]{\likelihoodparam}^{(\numprocess)}$ for $\numprocess\in\numprocessset_{\text{obs}}$. With experimentation, we found that calculating updated $\variationalprob(\cusmatrix{\Processfunction}^{(\numprocess)})$ by looping over the observed nodes helps node-wise training converge to a global minimum. This is not the case when $\variationalprob(\cusmatrix{\Processfunction}^{(\numprocess)})$ is calculated only once outside the loop over nodes. Algorithm~\ref{alg:pogpn_node-wise} explains the node-wise coordinate ascent method.

\section{Experiments}\label{sec:experiments}
\section{Experiments}
\label{sec:experiments}
The experiments are designed to address two key research questions.
First, \textbf{RQ1} evaluates whether the average $L_2$-norm of the counterfactual perturbation vectors ($\overline{||\perturb||}$) decreases as the model overfits the data, thereby providing further empirical validation for our hypothesis.
Second, \textbf{RQ2} evaluates the ability of the proposed counterfactual regularized loss, as defined in (\ref{eq:regularized_loss2}), to mitigate overfitting when compared to existing regularization techniques.

% The experiments are designed to address three key research questions. First, \textbf{RQ1} investigates whether the mean perturbation vector norm decreases as the model overfits the data, aiming to further validate our intuition. Second, \textbf{RQ2} explores whether the mean perturbation vector norm can be effectively leveraged as a regularization term during training, offering insights into its potential role in mitigating overfitting. Finally, \textbf{RQ3} examines whether our counterfactual regularizer enables the model to achieve superior performance compared to existing regularization methods, thus highlighting its practical advantage.

\subsection{Experimental Setup}
\textbf{\textit{Datasets, Models, and Tasks.}}
The experiments are conducted on three datasets: \textit{Water Potability}~\cite{kadiwal2020waterpotability}, \textit{Phomene}~\cite{phomene}, and \textit{CIFAR-10}~\cite{krizhevsky2009learning}. For \textit{Water Potability} and \textit{Phomene}, we randomly select $80\%$ of the samples for the training set, and the remaining $20\%$ for the test set, \textit{CIFAR-10} comes already split. Furthermore, we consider the following models: Logistic Regression, Multi-Layer Perceptron (MLP) with 100 and 30 neurons on each hidden layer, and PreactResNet-18~\cite{he2016cvecvv} as a Convolutional Neural Network (CNN) architecture.
We focus on binary classification tasks and leave the extension to multiclass scenarios for future work. However, for datasets that are inherently multiclass, we transform the problem into a binary classification task by selecting two classes, aligning with our assumption.

\smallskip
\noindent\textbf{\textit{Evaluation Measures.}} To characterize the degree of overfitting, we use the test loss, as it serves as a reliable indicator of the model's generalization capability to unseen data. Additionally, we evaluate the predictive performance of each model using the test accuracy.

\smallskip
\noindent\textbf{\textit{Baselines.}} We compare CF-Reg with the following regularization techniques: L1 (``Lasso''), L2 (``Ridge''), and Dropout.

\smallskip
\noindent\textbf{\textit{Configurations.}}
For each model, we adopt specific configurations as follows.
\begin{itemize}
\item \textit{Logistic Regression:} To induce overfitting in the model, we artificially increase the dimensionality of the data beyond the number of training samples by applying a polynomial feature expansion. This approach ensures that the model has enough capacity to overfit the training data, allowing us to analyze the impact of our counterfactual regularizer. The degree of the polynomial is chosen as the smallest degree that makes the number of features greater than the number of data.
\item \textit{Neural Networks (MLP and CNN):} To take advantage of the closed-form solution for computing the optimal perturbation vector as defined in (\ref{eq:opt-delta}), we use a local linear approximation of the neural network models. Hence, given an instance $\inst_i$, we consider the (optimal) counterfactual not with respect to $\model$ but with respect to:
\begin{equation}
\label{eq:taylor}
    \model^{lin}(\inst) = \model(\inst_i) + \nabla_{\inst}\model(\inst_i)(\inst - \inst_i),
\end{equation}
where $\model^{lin}$ represents the first-order Taylor approximation of $\model$ at $\inst_i$.
Note that this step is unnecessary for Logistic Regression, as it is inherently a linear model.
\end{itemize}

\smallskip
\noindent \textbf{\textit{Implementation Details.}} We run all experiments on a machine equipped with an AMD Ryzen 9 7900 12-Core Processor and an NVIDIA GeForce RTX 4090 GPU. Our implementation is based on the PyTorch Lightning framework. We use stochastic gradient descent as the optimizer with a learning rate of $\eta = 0.001$ and no weight decay. We use a batch size of $128$. The training and test steps are conducted for $6000$ epochs on the \textit{Water Potability} and \textit{Phoneme} datasets, while for the \textit{CIFAR-10} dataset, they are performed for $200$ epochs.
Finally, the contribution $w_i^{\varepsilon}$ of each training point $\inst_i$ is uniformly set as $w_i^{\varepsilon} = 1~\forall i\in \{1,\ldots,m\}$.

The source code implementation for our experiments is available at the following GitHub repository: \url{https://anonymous.4open.science/r/COCE-80B4/README.md} 

\subsection{RQ1: Counterfactual Perturbation vs. Overfitting}
To address \textbf{RQ1}, we analyze the relationship between the test loss and the average $L_2$-norm of the counterfactual perturbation vectors ($\overline{||\perturb||}$) over training epochs.

In particular, Figure~\ref{fig:delta_loss_epochs} depicts the evolution of $\overline{||\perturb||}$ alongside the test loss for an MLP trained \textit{without} regularization on the \textit{Water Potability} dataset. 
\begin{figure}[ht]
    \centering
    \includegraphics[width=0.85\linewidth]{img/delta_loss_epochs.png}
    \caption{The average counterfactual perturbation vector $\overline{||\perturb||}$ (left $y$-axis) and the cross-entropy test loss (right $y$-axis) over training epochs ($x$-axis) for an MLP trained on the \textit{Water Potability} dataset \textit{without} regularization.}
    \label{fig:delta_loss_epochs}
\end{figure}

The plot shows a clear trend as the model starts to overfit the data (evidenced by an increase in test loss). 
Notably, $\overline{||\perturb||}$ begins to decrease, which aligns with the hypothesis that the average distance to the optimal counterfactual example gets smaller as the model's decision boundary becomes increasingly adherent to the training data.

It is worth noting that this trend is heavily influenced by the choice of the counterfactual generator model. In particular, the relationship between $\overline{||\perturb||}$ and the degree of overfitting may become even more pronounced when leveraging more accurate counterfactual generators. However, these models often come at the cost of higher computational complexity, and their exploration is left to future work.

Nonetheless, we expect that $\overline{||\perturb||}$ will eventually stabilize at a plateau, as the average $L_2$-norm of the optimal counterfactual perturbations cannot vanish to zero.

% Additionally, the choice of employing the score-based counterfactual explanation framework to generate counterfactuals was driven to promote computational efficiency.

% Future enhancements to the framework may involve adopting models capable of generating more precise counterfactuals. While such approaches may yield to performance improvements, they are likely to come at the cost of increased computational complexity.


\subsection{RQ2: Counterfactual Regularization Performance}
To answer \textbf{RQ2}, we evaluate the effectiveness of the proposed counterfactual regularization (CF-Reg) by comparing its performance against existing baselines: unregularized training loss (No-Reg), L1 regularization (L1-Reg), L2 regularization (L2-Reg), and Dropout.
Specifically, for each model and dataset combination, Table~\ref{tab:regularization_comparison} presents the mean value and standard deviation of test accuracy achieved by each method across 5 random initialization. 

The table illustrates that our regularization technique consistently delivers better results than existing methods across all evaluated scenarios, except for one case -- i.e., Logistic Regression on the \textit{Phomene} dataset. 
However, this setting exhibits an unusual pattern, as the highest model accuracy is achieved without any regularization. Even in this case, CF-Reg still surpasses other regularization baselines.

From the results above, we derive the following key insights. First, CF-Reg proves to be effective across various model types, ranging from simple linear models (Logistic Regression) to deep architectures like MLPs and CNNs, and across diverse datasets, including both tabular and image data. 
Second, CF-Reg's strong performance on the \textit{Water} dataset with Logistic Regression suggests that its benefits may be more pronounced when applied to simpler models. However, the unexpected outcome on the \textit{Phoneme} dataset calls for further investigation into this phenomenon.


\begin{table*}[h!]
    \centering
    \caption{Mean value and standard deviation of test accuracy across 5 random initializations for different model, dataset, and regularization method. The best results are highlighted in \textbf{bold}.}
    \label{tab:regularization_comparison}
    \begin{tabular}{|c|c|c|c|c|c|c|}
        \hline
        \textbf{Model} & \textbf{Dataset} & \textbf{No-Reg} & \textbf{L1-Reg} & \textbf{L2-Reg} & \textbf{Dropout} & \textbf{CF-Reg (ours)} \\ \hline
        Logistic Regression   & \textit{Water}   & $0.6595 \pm 0.0038$   & $0.6729 \pm 0.0056$   & $0.6756 \pm 0.0046$  & N/A    & $\mathbf{0.6918 \pm 0.0036}$                     \\ \hline
        MLP   & \textit{Water}   & $0.6756 \pm 0.0042$   & $0.6790 \pm 0.0058$   & $0.6790 \pm 0.0023$  & $0.6750 \pm 0.0036$    & $\mathbf{0.6802 \pm 0.0046}$                    \\ \hline
%        MLP   & \textit{Adult}   & $0.8404 \pm 0.0010$   & $\mathbf{0.8495 \pm 0.0007}$   & $0.8489 \pm 0.0014$  & $\mathbf{0.8495 \pm 0.0016}$     & $0.8449 \pm 0.0019$                    \\ \hline
        Logistic Regression   & \textit{Phomene}   & $\mathbf{0.8148 \pm 0.0020}$   & $0.8041 \pm 0.0028$   & $0.7835 \pm 0.0176$  & N/A    & $0.8098 \pm 0.0055$                     \\ \hline
        MLP   & \textit{Phomene}   & $0.8677 \pm 0.0033$   & $0.8374 \pm 0.0080$   & $0.8673 \pm 0.0045$  & $0.8672 \pm 0.0042$     & $\mathbf{0.8718 \pm 0.0040}$                    \\ \hline
        CNN   & \textit{CIFAR-10} & $0.6670 \pm 0.0233$   & $0.6229 \pm 0.0850$   & $0.7348 \pm 0.0365$   & N/A    & $\mathbf{0.7427 \pm 0.0571}$                     \\ \hline
    \end{tabular}
\end{table*}

\begin{table*}[htb!]
    \centering
    \caption{Hyperparameter configurations utilized for the generation of Table \ref{tab:regularization_comparison}. For our regularization the hyperparameters are reported as $\mathbf{\alpha/\beta}$.}
    \label{tab:performance_parameters}
    \begin{tabular}{|c|c|c|c|c|c|c|}
        \hline
        \textbf{Model} & \textbf{Dataset} & \textbf{No-Reg} & \textbf{L1-Reg} & \textbf{L2-Reg} & \textbf{Dropout} & \textbf{CF-Reg (ours)} \\ \hline
        Logistic Regression   & \textit{Water}   & N/A   & $0.0093$   & $0.6927$  & N/A    & $0.3791/1.0355$                     \\ \hline
        MLP   & \textit{Water}   & N/A   & $0.0007$   & $0.0022$  & $0.0002$    & $0.2567/1.9775$                    \\ \hline
        Logistic Regression   &
        \textit{Phomene}   & N/A   & $0.0097$   & $0.7979$  & N/A    & $0.0571/1.8516$                     \\ \hline
        MLP   & \textit{Phomene}   & N/A   & $0.0007$   & $4.24\cdot10^{-5}$  & $0.0015$    & $0.0516/2.2700$                    \\ \hline
       % MLP   & \textit{Adult}   & N/A   & $0.0018$   & $0.0018$  & $0.0601$     & $0.0764/2.2068$                    \\ \hline
        CNN   & \textit{CIFAR-10} & N/A   & $0.0050$   & $0.0864$ & N/A    & $0.3018/
        2.1502$                     \\ \hline
    \end{tabular}
\end{table*}

\begin{table*}[htb!]
    \centering
    \caption{Mean value and standard deviation of training time across 5 different runs. The reported time (in seconds) corresponds to the generation of each entry in Table \ref{tab:regularization_comparison}. Times are }
    \label{tab:times}
    \begin{tabular}{|c|c|c|c|c|c|c|}
        \hline
        \textbf{Model} & \textbf{Dataset} & \textbf{No-Reg} & \textbf{L1-Reg} & \textbf{L2-Reg} & \textbf{Dropout} & \textbf{CF-Reg (ours)} \\ \hline
        Logistic Regression   & \textit{Water}   & $222.98 \pm 1.07$   & $239.94 \pm 2.59$   & $241.60 \pm 1.88$  & N/A    & $251.50 \pm 1.93$                     \\ \hline
        MLP   & \textit{Water}   & $225.71 \pm 3.85$   & $250.13 \pm 4.44$   & $255.78 \pm 2.38$  & $237.83 \pm 3.45$    & $266.48 \pm 3.46$                    \\ \hline
        Logistic Regression   & \textit{Phomene}   & $266.39 \pm 0.82$ & $367.52 \pm 6.85$   & $361.69 \pm 4.04$  & N/A   & $310.48 \pm 0.76$                    \\ \hline
        MLP   &
        \textit{Phomene} & $335.62 \pm 1.77$   & $390.86 \pm 2.11$   & $393.96 \pm 1.95$ & $363.51 \pm 5.07$    & $403.14 \pm 1.92$                     \\ \hline
       % MLP   & \textit{Adult}   & N/A   & $0.0018$   & $0.0018$  & $0.0601$     & $0.0764/2.2068$                    \\ \hline
        CNN   & \textit{CIFAR-10} & $370.09 \pm 0.18$   & $395.71 \pm 0.55$   & $401.38 \pm 0.16$ & N/A    & $1287.8 \pm 0.26$                     \\ \hline
    \end{tabular}
\end{table*}

\subsection{Feasibility of our Method}
A crucial requirement for any regularization technique is that it should impose minimal impact on the overall training process.
In this respect, CF-Reg introduces an overhead that depends on the time required to find the optimal counterfactual example for each training instance. 
As such, the more sophisticated the counterfactual generator model probed during training the higher would be the time required. However, a more advanced counterfactual generator might provide a more effective regularization. We discuss this trade-off in more details in Section~\ref{sec:discussion}.

Table~\ref{tab:times} presents the average training time ($\pm$ standard deviation) for each model and dataset combination listed in Table~\ref{tab:regularization_comparison}.
We can observe that the higher accuracy achieved by CF-Reg using the score-based counterfactual generator comes with only minimal overhead. However, when applied to deep neural networks with many hidden layers, such as \textit{PreactResNet-18}, the forward derivative computation required for the linearization of the network introduces a more noticeable computational cost, explaining the longer training times in the table.

\subsection{Hyperparameter Sensitivity Analysis}
The proposed counterfactual regularization technique relies on two key hyperparameters: $\alpha$ and $\beta$. The former is intrinsic to the loss formulation defined in (\ref{eq:cf-train}), while the latter is closely tied to the choice of the score-based counterfactual explanation method used.

Figure~\ref{fig:test_alpha_beta} illustrates how the test accuracy of an MLP trained on the \textit{Water Potability} dataset changes for different combinations of $\alpha$ and $\beta$.

\begin{figure}[ht]
    \centering
    \includegraphics[width=0.85\linewidth]{img/test_acc_alpha_beta.png}
    \caption{The test accuracy of an MLP trained on the \textit{Water Potability} dataset, evaluated while varying the weight of our counterfactual regularizer ($\alpha$) for different values of $\beta$.}
    \label{fig:test_alpha_beta}
\end{figure}

We observe that, for a fixed $\beta$, increasing the weight of our counterfactual regularizer ($\alpha$) can slightly improve test accuracy until a sudden drop is noticed for $\alpha > 0.1$.
This behavior was expected, as the impact of our penalty, like any regularization term, can be disruptive if not properly controlled.

Moreover, this finding further demonstrates that our regularization method, CF-Reg, is inherently data-driven. Therefore, it requires specific fine-tuning based on the combination of the model and dataset at hand.


\section{Discussion}\label{sec:discussion}
\section{Discussion of Assumptions}\label{sec:discussion}
In this paper, we have made several assumptions for the sake of clarity and simplicity. In this section, we discuss the rationale behind these assumptions, the extent to which these assumptions hold in practice, and the consequences for our protocol when these assumptions hold.

\subsection{Assumptions on the Demand}

There are two simplifying assumptions we make about the demand. First, we assume the demand at any time is relatively small compared to the channel capacities. Second, we take the demand to be constant over time. We elaborate upon both these points below.

\paragraph{Small demands} The assumption that demands are small relative to channel capacities is made precise in \eqref{eq:large_capacity_assumption}. This assumption simplifies two major aspects of our protocol. First, it largely removes congestion from consideration. In \eqref{eq:primal_problem}, there is no constraint ensuring that total flow in both directions stays below capacity--this is always met. Consequently, there is no Lagrange multiplier for congestion and no congestion pricing; only imbalance penalties apply. In contrast, protocols in \cite{sivaraman2020high, varma2021throughput, wang2024fence} include congestion fees due to explicit congestion constraints. Second, the bound \eqref{eq:large_capacity_assumption} ensures that as long as channels remain balanced, the network can always meet demand, no matter how the demand is routed. Since channels can rebalance when necessary, they never drop transactions. This allows prices and flows to adjust as per the equations in \eqref{eq:algorithm}, which makes it easier to prove the protocol's convergence guarantees. This also preserves the key property that a channel's price remains proportional to net money flow through it.

In practice, payment channel networks are used most often for micro-payments, for which on-chain transactions are prohibitively expensive; large transactions typically take place directly on the blockchain. For example, according to \cite{river2023lightning}, the average channel capacity is roughly $0.1$ BTC ($5,000$ BTC distributed over $50,000$ channels), while the average transaction amount is less than $0.0004$ BTC ($44.7k$ satoshis). Thus, the small demand assumption is not too unrealistic. Additionally, the occasional large transaction can be treated as a sequence of smaller transactions by breaking it into packets and executing each packet serially (as done by \cite{sivaraman2020high}).
Lastly, a good path discovery process that favors large capacity channels over small capacity ones can help ensure that the bound in \eqref{eq:large_capacity_assumption} holds.

\paragraph{Constant demands} 
In this work, we assume that any transacting pair of nodes have a steady transaction demand between them (see Section \ref{sec:transaction_requests}). Making this assumption is necessary to obtain the kind of guarantees that we have presented in this paper. Unless the demand is steady, it is unreasonable to expect that the flows converge to a steady value. Weaker assumptions on the demand lead to weaker guarantees. For example, with the more general setting of stochastic, but i.i.d. demand between any two nodes, \cite{varma2021throughput} shows that the channel queue lengths are bounded in expectation. If the demand can be arbitrary, then it is very hard to get any meaningful performance guarantees; \cite{wang2024fence} shows that even for a single bidirectional channel, the competitive ratio is infinite. Indeed, because a PCN is a decentralized system and decisions must be made based on local information alone, it is difficult for the network to find the optimal detailed balance flow at every time step with a time-varying demand.  With a steady demand, the network can discover the optimal flows in a reasonably short time, as our work shows.

We view the constant demand assumption as an approximation for a more general demand process that could be piece-wise constant, stochastic, or both (see simulations in Figure \ref{fig:five_nodes_variable_demand}).
We believe it should be possible to merge ideas from our work and \cite{varma2021throughput} to provide guarantees in a setting with random demands with arbitrary means. We leave this for future work. In addition, our work suggests that a reasonable method of handling stochastic demands is to queue the transaction requests \textit{at the source node} itself. This queuing action should be viewed in conjunction with flow-control. Indeed, a temporarily high unidirectional demand would raise prices for the sender, incentivizing the sender to stop sending the transactions. If the sender queues the transactions, they can send them later when prices drop. This form of queuing does not require any overhaul of the basic PCN infrastructure and is therefore simpler to implement than per-channel queues as suggested by \cite{sivaraman2020high} and \cite{varma2021throughput}.

\subsection{The Incentive of Channels}
The actions of the channels as prescribed by the DEBT control protocol can be summarized as follows. Channels adjust their prices in proportion to the net flow through them. They rebalance themselves whenever necessary and execute any transaction request that has been made of them. We discuss both these aspects below.

\paragraph{On Prices}
In this work, the exclusive role of channel prices is to ensure that the flows through each channel remains balanced. In practice, it would be important to include other components in a channel's price/fee as well: a congestion price  and an incentive price. The congestion price, as suggested by \cite{varma2021throughput}, would depend on the total flow of transactions through the channel, and would incentivize nodes to balance the load over different paths. The incentive price, which is commonly used in practice \cite{river2023lightning}, is necessary to provide channels with an incentive to serve as an intermediary for different channels. In practice, we expect both these components to be smaller than the imbalance price. Consequently, we expect the behavior of our protocol to be similar to our theoretical results even with these additional prices.

A key aspect of our protocol is that channel fees are allowed to be negative. Although the original Lightning network whitepaper \cite{poon2016bitcoin} suggests that negative channel prices may be a good solution to promote rebalancing, the idea of negative prices in not very popular in the literature. To our knowledge, the only prior work with this feature is \cite{varma2021throughput}. Indeed, in papers such as \cite{van2021merchant} and \cite{wang2024fence}, the price function is explicitly modified such that the channel price is never negative. The results of our paper show the benefits of negative prices. For one, in steady state, equal flows in both directions ensure that a channel doesn't loose any money (the other price components mentioned above ensure that the channel will only gain money). More importantly, negative prices are important to ensure that the protocol selectively stifles acyclic flows while allowing circulations to flow. Indeed, in the example of Section \ref{sec:flow_control_example}, the flows between nodes $A$ and $C$ are left on only because the large positive price over one channel is canceled by the corresponding negative price over the other channel, leading to a net zero price.

Lastly, observe that in the DEBT control protocol, the price charged by a channel does not depend on its capacity. This is a natural consequence of the price being the Lagrange multiplier for the net-zero flow constraint, which also does not depend on the channel capacity. In contrast, in many other works, the imbalance price is normalized by the channel capacity \cite{ren2018optimal, lin2020funds, wang2024fence}; this is shown to work well in practice. The rationale for such a price structure is explained well in \cite{wang2024fence}, where this fee is derived with the aim of always maintaining some balance (liquidity) at each end of every channel. This is a reasonable aim if a channel is to never rebalance itself; the experiments of the aforementioned papers are conducted in such a regime. In this work, however, we allow the channels to rebalance themselves a few times in order to settle on a detailed balance flow. This is because our focus is on the long-term steady state performance of the protocol. This difference in perspective also shows up in how the price depends on the channel imbalance. \cite{lin2020funds} and \cite{wang2024fence} advocate for strictly convex prices whereas this work and \cite{varma2021throughput} propose linear prices.

\paragraph{On Rebalancing} 
Recall that the DEBT control protocol ensures that the flows in the network converge to a detailed balance flow, which can be sustained perpetually without any rebalancing. However, during the transient phase (before convergence), channels may have to perform on-chain rebalancing a few times. Since rebalancing is an expensive operation, it is worthwhile discussing methods by which channels can reduce the extent of rebalancing. One option for the channels to reduce the extent of rebalancing is to increase their capacity; however, this comes at the cost of locking in more capital. Each channel can decide for itself the optimum amount of capital to lock in. Another option, which we discuss in Section \ref{sec:five_node}, is for channels to increase the rate $\gamma$ at which they adjust prices. 

Ultimately, whether or not it is beneficial for a channel to rebalance depends on the time-horizon under consideration. Our protocol is based on the assumption that the demand remains steady for a long period of time. If this is indeed the case, it would be worthwhile for a channel to rebalance itself as it can make up this cost through the incentive fees gained from the flow of transactions through it in steady state. If a channel chooses not to rebalance itself, however, there is a risk of being trapped in a deadlock, which is suboptimal for not only the nodes but also the channel.

\section{Conclusion}
This work presents DEBT control: a protocol for payment channel networks that uses source routing and flow control based on channel prices. The protocol is derived by posing a network utility maximization problem and analyzing its dual minimization. It is shown that under steady demands, the protocol guides the network to an optimal, sustainable point. Simulations show its robustness to demand variations. The work demonstrates that simple protocols with strong theoretical guarantees are possible for PCNs and we hope it inspires further theoretical research in this direction.

% References
\bibliographystyle{abbrvnat}
\bibliography{references}

\newpage

\onecolumn

\title{Partially Observable Gaussian Process Network and Doubly Stochastic Variational Inference\\(Supplementary Material)}
\maketitle
\subsection{Lloyd-Max Algorithm}
\label{subsec:Lloyd-Max}
For a given quantization bitwidth $B$ and an operand $\bm{X}$, the Lloyd-Max algorithm finds $2^B$ quantization levels $\{\hat{x}_i\}_{i=1}^{2^B}$ such that quantizing $\bm{X}$ by rounding each scalar in $\bm{X}$ to the nearest quantization level minimizes the quantization MSE. 

The algorithm starts with an initial guess of quantization levels and then iteratively computes quantization thresholds $\{\tau_i\}_{i=1}^{2^B-1}$ and updates quantization levels $\{\hat{x}_i\}_{i=1}^{2^B}$. Specifically, at iteration $n$, thresholds are set to the midpoints of the previous iteration's levels:
\begin{align*}
    \tau_i^{(n)}=\frac{\hat{x}_i^{(n-1)}+\hat{x}_{i+1}^{(n-1)}}2 \text{ for } i=1\ldots 2^B-1
\end{align*}
Subsequently, the quantization levels are re-computed as conditional means of the data regions defined by the new thresholds:
\begin{align*}
    \hat{x}_i^{(n)}=\mathbb{E}\left[ \bm{X} \big| \bm{X}\in [\tau_{i-1}^{(n)},\tau_i^{(n)}] \right] \text{ for } i=1\ldots 2^B
\end{align*}
where to satisfy boundary conditions we have $\tau_0=-\infty$ and $\tau_{2^B}=\infty$. The algorithm iterates the above steps until convergence.

Figure \ref{fig:lm_quant} compares the quantization levels of a $7$-bit floating point (E3M3) quantizer (left) to a $7$-bit Lloyd-Max quantizer (right) when quantizing a layer of weights from the GPT3-126M model at a per-tensor granularity. As shown, the Lloyd-Max quantizer achieves substantially lower quantization MSE. Further, Table \ref{tab:FP7_vs_LM7} shows the superior perplexity achieved by Lloyd-Max quantizers for bitwidths of $7$, $6$ and $5$. The difference between the quantizers is clear at 5 bits, where per-tensor FP quantization incurs a drastic and unacceptable increase in perplexity, while Lloyd-Max quantization incurs a much smaller increase. Nevertheless, we note that even the optimal Lloyd-Max quantizer incurs a notable ($\sim 1.5$) increase in perplexity due to the coarse granularity of quantization. 

\begin{figure}[h]
  \centering
  \includegraphics[width=0.7\linewidth]{sections/figures/LM7_FP7.pdf}
  \caption{\small Quantization levels and the corresponding quantization MSE of Floating Point (left) vs Lloyd-Max (right) Quantizers for a layer of weights in the GPT3-126M model.}
  \label{fig:lm_quant}
\end{figure}

\begin{table}[h]\scriptsize
\begin{center}
\caption{\label{tab:FP7_vs_LM7} \small Comparing perplexity (lower is better) achieved by floating point quantizers and Lloyd-Max quantizers on a GPT3-126M model for the Wikitext-103 dataset.}
\begin{tabular}{c|cc|c}
\hline
 \multirow{2}{*}{\textbf{Bitwidth}} & \multicolumn{2}{|c|}{\textbf{Floating-Point Quantizer}} & \textbf{Lloyd-Max Quantizer} \\
 & Best Format & Wikitext-103 Perplexity & Wikitext-103 Perplexity \\
\hline
7 & E3M3 & 18.32 & 18.27 \\
6 & E3M2 & 19.07 & 18.51 \\
5 & E4M0 & 43.89 & 19.71 \\
\hline
\end{tabular}
\end{center}
\end{table}

\subsection{Proof of Local Optimality of LO-BCQ}
\label{subsec:lobcq_opt_proof}
For a given block $\bm{b}_j$, the quantization MSE during LO-BCQ can be empirically evaluated as $\frac{1}{L_b}\lVert \bm{b}_j- \bm{\hat{b}}_j\rVert^2_2$ where $\bm{\hat{b}}_j$ is computed from equation (\ref{eq:clustered_quantization_definition}) as $C_{f(\bm{b}_j)}(\bm{b}_j)$. Further, for a given block cluster $\mathcal{B}_i$, we compute the quantization MSE as $\frac{1}{|\mathcal{B}_{i}|}\sum_{\bm{b} \in \mathcal{B}_{i}} \frac{1}{L_b}\lVert \bm{b}- C_i^{(n)}(\bm{b})\rVert^2_2$. Therefore, at the end of iteration $n$, we evaluate the overall quantization MSE $J^{(n)}$ for a given operand $\bm{X}$ composed of $N_c$ block clusters as:
\begin{align*}
    \label{eq:mse_iter_n}
    J^{(n)} = \frac{1}{N_c} \sum_{i=1}^{N_c} \frac{1}{|\mathcal{B}_{i}^{(n)}|}\sum_{\bm{v} \in \mathcal{B}_{i}^{(n)}} \frac{1}{L_b}\lVert \bm{b}- B_i^{(n)}(\bm{b})\rVert^2_2
\end{align*}

At the end of iteration $n$, the codebooks are updated from $\mathcal{C}^{(n-1)}$ to $\mathcal{C}^{(n)}$. However, the mapping of a given vector $\bm{b}_j$ to quantizers $\mathcal{C}^{(n)}$ remains as  $f^{(n)}(\bm{b}_j)$. At the next iteration, during the vector clustering step, $f^{(n+1)}(\bm{b}_j)$ finds new mapping of $\bm{b}_j$ to updated codebooks $\mathcal{C}^{(n)}$ such that the quantization MSE over the candidate codebooks is minimized. Therefore, we obtain the following result for $\bm{b}_j$:
\begin{align*}
\frac{1}{L_b}\lVert \bm{b}_j - C_{f^{(n+1)}(\bm{b}_j)}^{(n)}(\bm{b}_j)\rVert^2_2 \le \frac{1}{L_b}\lVert \bm{b}_j - C_{f^{(n)}(\bm{b}_j)}^{(n)}(\bm{b}_j)\rVert^2_2
\end{align*}

That is, quantizing $\bm{b}_j$ at the end of the block clustering step of iteration $n+1$ results in lower quantization MSE compared to quantizing at the end of iteration $n$. Since this is true for all $\bm{b} \in \bm{X}$, we assert the following:
\begin{equation}
\begin{split}
\label{eq:mse_ineq_1}
    \tilde{J}^{(n+1)} &= \frac{1}{N_c} \sum_{i=1}^{N_c} \frac{1}{|\mathcal{B}_{i}^{(n+1)}|}\sum_{\bm{b} \in \mathcal{B}_{i}^{(n+1)}} \frac{1}{L_b}\lVert \bm{b} - C_i^{(n)}(b)\rVert^2_2 \le J^{(n)}
\end{split}
\end{equation}
where $\tilde{J}^{(n+1)}$ is the the quantization MSE after the vector clustering step at iteration $n+1$.

Next, during the codebook update step (\ref{eq:quantizers_update}) at iteration $n+1$, the per-cluster codebooks $\mathcal{C}^{(n)}$ are updated to $\mathcal{C}^{(n+1)}$ by invoking the Lloyd-Max algorithm \citep{Lloyd}. We know that for any given value distribution, the Lloyd-Max algorithm minimizes the quantization MSE. Therefore, for a given vector cluster $\mathcal{B}_i$ we obtain the following result:

\begin{equation}
    \frac{1}{|\mathcal{B}_{i}^{(n+1)}|}\sum_{\bm{b} \in \mathcal{B}_{i}^{(n+1)}} \frac{1}{L_b}\lVert \bm{b}- C_i^{(n+1)}(\bm{b})\rVert^2_2 \le \frac{1}{|\mathcal{B}_{i}^{(n+1)}|}\sum_{\bm{b} \in \mathcal{B}_{i}^{(n+1)}} \frac{1}{L_b}\lVert \bm{b}- C_i^{(n)}(\bm{b})\rVert^2_2
\end{equation}

The above equation states that quantizing the given block cluster $\mathcal{B}_i$ after updating the associated codebook from $C_i^{(n)}$ to $C_i^{(n+1)}$ results in lower quantization MSE. Since this is true for all the block clusters, we derive the following result: 
\begin{equation}
\begin{split}
\label{eq:mse_ineq_2}
     J^{(n+1)} &= \frac{1}{N_c} \sum_{i=1}^{N_c} \frac{1}{|\mathcal{B}_{i}^{(n+1)}|}\sum_{\bm{b} \in \mathcal{B}_{i}^{(n+1)}} \frac{1}{L_b}\lVert \bm{b}- C_i^{(n+1)}(\bm{b})\rVert^2_2  \le \tilde{J}^{(n+1)}   
\end{split}
\end{equation}

Following (\ref{eq:mse_ineq_1}) and (\ref{eq:mse_ineq_2}), we find that the quantization MSE is non-increasing for each iteration, that is, $J^{(1)} \ge J^{(2)} \ge J^{(3)} \ge \ldots \ge J^{(M)}$ where $M$ is the maximum number of iterations. 
%Therefore, we can say that if the algorithm converges, then it must be that it has converged to a local minimum. 
\hfill $\blacksquare$


\begin{figure}
    \begin{center}
    \includegraphics[width=0.5\textwidth]{sections//figures/mse_vs_iter.pdf}
    \end{center}
    \caption{\small NMSE vs iterations during LO-BCQ compared to other block quantization proposals}
    \label{fig:nmse_vs_iter}
\end{figure}

Figure \ref{fig:nmse_vs_iter} shows the empirical convergence of LO-BCQ across several block lengths and number of codebooks. Also, the MSE achieved by LO-BCQ is compared to baselines such as MXFP and VSQ. As shown, LO-BCQ converges to a lower MSE than the baselines. Further, we achieve better convergence for larger number of codebooks ($N_c$) and for a smaller block length ($L_b$), both of which increase the bitwidth of BCQ (see Eq \ref{eq:bitwidth_bcq}).


\subsection{Additional Accuracy Results}
%Table \ref{tab:lobcq_config} lists the various LOBCQ configurations and their corresponding bitwidths.
\begin{table}
\setlength{\tabcolsep}{4.75pt}
\begin{center}
\caption{\label{tab:lobcq_config} Various LO-BCQ configurations and their bitwidths.}
\begin{tabular}{|c||c|c|c|c||c|c||c|} 
\hline
 & \multicolumn{4}{|c||}{$L_b=8$} & \multicolumn{2}{|c||}{$L_b=4$} & $L_b=2$ \\
 \hline
 \backslashbox{$L_A$\kern-1em}{\kern-1em$N_c$} & 2 & 4 & 8 & 16 & 2 & 4 & 2 \\
 \hline
 64 & 4.25 & 4.375 & 4.5 & 4.625 & 4.375 & 4.625 & 4.625\\
 \hline
 32 & 4.375 & 4.5 & 4.625& 4.75 & 4.5 & 4.75 & 4.75 \\
 \hline
 16 & 4.625 & 4.75& 4.875 & 5 & 4.75 & 5 & 5 \\
 \hline
\end{tabular}
\end{center}
\end{table}

%\subsection{Perplexity achieved by various LO-BCQ configurations on Wikitext-103 dataset}

\begin{table} \centering
\begin{tabular}{|c||c|c|c|c||c|c||c|} 
\hline
 $L_b \rightarrow$& \multicolumn{4}{c||}{8} & \multicolumn{2}{c||}{4} & 2\\
 \hline
 \backslashbox{$L_A$\kern-1em}{\kern-1em$N_c$} & 2 & 4 & 8 & 16 & 2 & 4 & 2  \\
 %$N_c \rightarrow$ & 2 & 4 & 8 & 16 & 2 & 4 & 2 \\
 \hline
 \hline
 \multicolumn{8}{c}{GPT3-1.3B (FP32 PPL = 9.98)} \\ 
 \hline
 \hline
 64 & 10.40 & 10.23 & 10.17 & 10.15 &  10.28 & 10.18 & 10.19 \\
 \hline
 32 & 10.25 & 10.20 & 10.15 & 10.12 &  10.23 & 10.17 & 10.17 \\
 \hline
 16 & 10.22 & 10.16 & 10.10 & 10.09 &  10.21 & 10.14 & 10.16 \\
 \hline
  \hline
 \multicolumn{8}{c}{GPT3-8B (FP32 PPL = 7.38)} \\ 
 \hline
 \hline
 64 & 7.61 & 7.52 & 7.48 &  7.47 &  7.55 &  7.49 & 7.50 \\
 \hline
 32 & 7.52 & 7.50 & 7.46 &  7.45 &  7.52 &  7.48 & 7.48  \\
 \hline
 16 & 7.51 & 7.48 & 7.44 &  7.44 &  7.51 &  7.49 & 7.47  \\
 \hline
\end{tabular}
\caption{\label{tab:ppl_gpt3_abalation} Wikitext-103 perplexity across GPT3-1.3B and 8B models.}
\end{table}

\begin{table} \centering
\begin{tabular}{|c||c|c|c|c||} 
\hline
 $L_b \rightarrow$& \multicolumn{4}{c||}{8}\\
 \hline
 \backslashbox{$L_A$\kern-1em}{\kern-1em$N_c$} & 2 & 4 & 8 & 16 \\
 %$N_c \rightarrow$ & 2 & 4 & 8 & 16 & 2 & 4 & 2 \\
 \hline
 \hline
 \multicolumn{5}{|c|}{Llama2-7B (FP32 PPL = 5.06)} \\ 
 \hline
 \hline
 64 & 5.31 & 5.26 & 5.19 & 5.18  \\
 \hline
 32 & 5.23 & 5.25 & 5.18 & 5.15  \\
 \hline
 16 & 5.23 & 5.19 & 5.16 & 5.14  \\
 \hline
 \multicolumn{5}{|c|}{Nemotron4-15B (FP32 PPL = 5.87)} \\ 
 \hline
 \hline
 64  & 6.3 & 6.20 & 6.13 & 6.08  \\
 \hline
 32  & 6.24 & 6.12 & 6.07 & 6.03  \\
 \hline
 16  & 6.12 & 6.14 & 6.04 & 6.02  \\
 \hline
 \multicolumn{5}{|c|}{Nemotron4-340B (FP32 PPL = 3.48)} \\ 
 \hline
 \hline
 64 & 3.67 & 3.62 & 3.60 & 3.59 \\
 \hline
 32 & 3.63 & 3.61 & 3.59 & 3.56 \\
 \hline
 16 & 3.61 & 3.58 & 3.57 & 3.55 \\
 \hline
\end{tabular}
\caption{\label{tab:ppl_llama7B_nemo15B} Wikitext-103 perplexity compared to FP32 baseline in Llama2-7B and Nemotron4-15B, 340B models}
\end{table}

%\subsection{Perplexity achieved by various LO-BCQ configurations on MMLU dataset}


\begin{table} \centering
\begin{tabular}{|c||c|c|c|c||c|c|c|c|} 
\hline
 $L_b \rightarrow$& \multicolumn{4}{c||}{8} & \multicolumn{4}{c||}{8}\\
 \hline
 \backslashbox{$L_A$\kern-1em}{\kern-1em$N_c$} & 2 & 4 & 8 & 16 & 2 & 4 & 8 & 16  \\
 %$N_c \rightarrow$ & 2 & 4 & 8 & 16 & 2 & 4 & 2 \\
 \hline
 \hline
 \multicolumn{5}{|c|}{Llama2-7B (FP32 Accuracy = 45.8\%)} & \multicolumn{4}{|c|}{Llama2-70B (FP32 Accuracy = 69.12\%)} \\ 
 \hline
 \hline
 64 & 43.9 & 43.4 & 43.9 & 44.9 & 68.07 & 68.27 & 68.17 & 68.75 \\
 \hline
 32 & 44.5 & 43.8 & 44.9 & 44.5 & 68.37 & 68.51 & 68.35 & 68.27  \\
 \hline
 16 & 43.9 & 42.7 & 44.9 & 45 & 68.12 & 68.77 & 68.31 & 68.59  \\
 \hline
 \hline
 \multicolumn{5}{|c|}{GPT3-22B (FP32 Accuracy = 38.75\%)} & \multicolumn{4}{|c|}{Nemotron4-15B (FP32 Accuracy = 64.3\%)} \\ 
 \hline
 \hline
 64 & 36.71 & 38.85 & 38.13 & 38.92 & 63.17 & 62.36 & 63.72 & 64.09 \\
 \hline
 32 & 37.95 & 38.69 & 39.45 & 38.34 & 64.05 & 62.30 & 63.8 & 64.33  \\
 \hline
 16 & 38.88 & 38.80 & 38.31 & 38.92 & 63.22 & 63.51 & 63.93 & 64.43  \\
 \hline
\end{tabular}
\caption{\label{tab:mmlu_abalation} Accuracy on MMLU dataset across GPT3-22B, Llama2-7B, 70B and Nemotron4-15B models.}
\end{table}


%\subsection{Perplexity achieved by various LO-BCQ configurations on LM evaluation harness}

\begin{table} \centering
\begin{tabular}{|c||c|c|c|c||c|c|c|c|} 
\hline
 $L_b \rightarrow$& \multicolumn{4}{c||}{8} & \multicolumn{4}{c||}{8}\\
 \hline
 \backslashbox{$L_A$\kern-1em}{\kern-1em$N_c$} & 2 & 4 & 8 & 16 & 2 & 4 & 8 & 16  \\
 %$N_c \rightarrow$ & 2 & 4 & 8 & 16 & 2 & 4 & 2 \\
 \hline
 \hline
 \multicolumn{5}{|c|}{Race (FP32 Accuracy = 37.51\%)} & \multicolumn{4}{|c|}{Boolq (FP32 Accuracy = 64.62\%)} \\ 
 \hline
 \hline
 64 & 36.94 & 37.13 & 36.27 & 37.13 & 63.73 & 62.26 & 63.49 & 63.36 \\
 \hline
 32 & 37.03 & 36.36 & 36.08 & 37.03 & 62.54 & 63.51 & 63.49 & 63.55  \\
 \hline
 16 & 37.03 & 37.03 & 36.46 & 37.03 & 61.1 & 63.79 & 63.58 & 63.33  \\
 \hline
 \hline
 \multicolumn{5}{|c|}{Winogrande (FP32 Accuracy = 58.01\%)} & \multicolumn{4}{|c|}{Piqa (FP32 Accuracy = 74.21\%)} \\ 
 \hline
 \hline
 64 & 58.17 & 57.22 & 57.85 & 58.33 & 73.01 & 73.07 & 73.07 & 72.80 \\
 \hline
 32 & 59.12 & 58.09 & 57.85 & 58.41 & 73.01 & 73.94 & 72.74 & 73.18  \\
 \hline
 16 & 57.93 & 58.88 & 57.93 & 58.56 & 73.94 & 72.80 & 73.01 & 73.94  \\
 \hline
\end{tabular}
\caption{\label{tab:mmlu_abalation} Accuracy on LM evaluation harness tasks on GPT3-1.3B model.}
\end{table}

\begin{table} \centering
\begin{tabular}{|c||c|c|c|c||c|c|c|c|} 
\hline
 $L_b \rightarrow$& \multicolumn{4}{c||}{8} & \multicolumn{4}{c||}{8}\\
 \hline
 \backslashbox{$L_A$\kern-1em}{\kern-1em$N_c$} & 2 & 4 & 8 & 16 & 2 & 4 & 8 & 16  \\
 %$N_c \rightarrow$ & 2 & 4 & 8 & 16 & 2 & 4 & 2 \\
 \hline
 \hline
 \multicolumn{5}{|c|}{Race (FP32 Accuracy = 41.34\%)} & \multicolumn{4}{|c|}{Boolq (FP32 Accuracy = 68.32\%)} \\ 
 \hline
 \hline
 64 & 40.48 & 40.10 & 39.43 & 39.90 & 69.20 & 68.41 & 69.45 & 68.56 \\
 \hline
 32 & 39.52 & 39.52 & 40.77 & 39.62 & 68.32 & 67.43 & 68.17 & 69.30  \\
 \hline
 16 & 39.81 & 39.71 & 39.90 & 40.38 & 68.10 & 66.33 & 69.51 & 69.42  \\
 \hline
 \hline
 \multicolumn{5}{|c|}{Winogrande (FP32 Accuracy = 67.88\%)} & \multicolumn{4}{|c|}{Piqa (FP32 Accuracy = 78.78\%)} \\ 
 \hline
 \hline
 64 & 66.85 & 66.61 & 67.72 & 67.88 & 77.31 & 77.42 & 77.75 & 77.64 \\
 \hline
 32 & 67.25 & 67.72 & 67.72 & 67.00 & 77.31 & 77.04 & 77.80 & 77.37  \\
 \hline
 16 & 68.11 & 68.90 & 67.88 & 67.48 & 77.37 & 78.13 & 78.13 & 77.69  \\
 \hline
\end{tabular}
\caption{\label{tab:mmlu_abalation} Accuracy on LM evaluation harness tasks on GPT3-8B model.}
\end{table}

\begin{table} \centering
\begin{tabular}{|c||c|c|c|c||c|c|c|c|} 
\hline
 $L_b \rightarrow$& \multicolumn{4}{c||}{8} & \multicolumn{4}{c||}{8}\\
 \hline
 \backslashbox{$L_A$\kern-1em}{\kern-1em$N_c$} & 2 & 4 & 8 & 16 & 2 & 4 & 8 & 16  \\
 %$N_c \rightarrow$ & 2 & 4 & 8 & 16 & 2 & 4 & 2 \\
 \hline
 \hline
 \multicolumn{5}{|c|}{Race (FP32 Accuracy = 40.67\%)} & \multicolumn{4}{|c|}{Boolq (FP32 Accuracy = 76.54\%)} \\ 
 \hline
 \hline
 64 & 40.48 & 40.10 & 39.43 & 39.90 & 75.41 & 75.11 & 77.09 & 75.66 \\
 \hline
 32 & 39.52 & 39.52 & 40.77 & 39.62 & 76.02 & 76.02 & 75.96 & 75.35  \\
 \hline
 16 & 39.81 & 39.71 & 39.90 & 40.38 & 75.05 & 73.82 & 75.72 & 76.09  \\
 \hline
 \hline
 \multicolumn{5}{|c|}{Winogrande (FP32 Accuracy = 70.64\%)} & \multicolumn{4}{|c|}{Piqa (FP32 Accuracy = 79.16\%)} \\ 
 \hline
 \hline
 64 & 69.14 & 70.17 & 70.17 & 70.56 & 78.24 & 79.00 & 78.62 & 78.73 \\
 \hline
 32 & 70.96 & 69.69 & 71.27 & 69.30 & 78.56 & 79.49 & 79.16 & 78.89  \\
 \hline
 16 & 71.03 & 69.53 & 69.69 & 70.40 & 78.13 & 79.16 & 79.00 & 79.00  \\
 \hline
\end{tabular}
\caption{\label{tab:mmlu_abalation} Accuracy on LM evaluation harness tasks on GPT3-22B model.}
\end{table}

\begin{table} \centering
\begin{tabular}{|c||c|c|c|c||c|c|c|c|} 
\hline
 $L_b \rightarrow$& \multicolumn{4}{c||}{8} & \multicolumn{4}{c||}{8}\\
 \hline
 \backslashbox{$L_A$\kern-1em}{\kern-1em$N_c$} & 2 & 4 & 8 & 16 & 2 & 4 & 8 & 16  \\
 %$N_c \rightarrow$ & 2 & 4 & 8 & 16 & 2 & 4 & 2 \\
 \hline
 \hline
 \multicolumn{5}{|c|}{Race (FP32 Accuracy = 44.4\%)} & \multicolumn{4}{|c|}{Boolq (FP32 Accuracy = 79.29\%)} \\ 
 \hline
 \hline
 64 & 42.49 & 42.51 & 42.58 & 43.45 & 77.58 & 77.37 & 77.43 & 78.1 \\
 \hline
 32 & 43.35 & 42.49 & 43.64 & 43.73 & 77.86 & 75.32 & 77.28 & 77.86  \\
 \hline
 16 & 44.21 & 44.21 & 43.64 & 42.97 & 78.65 & 77 & 76.94 & 77.98  \\
 \hline
 \hline
 \multicolumn{5}{|c|}{Winogrande (FP32 Accuracy = 69.38\%)} & \multicolumn{4}{|c|}{Piqa (FP32 Accuracy = 78.07\%)} \\ 
 \hline
 \hline
 64 & 68.9 & 68.43 & 69.77 & 68.19 & 77.09 & 76.82 & 77.09 & 77.86 \\
 \hline
 32 & 69.38 & 68.51 & 68.82 & 68.90 & 78.07 & 76.71 & 78.07 & 77.86  \\
 \hline
 16 & 69.53 & 67.09 & 69.38 & 68.90 & 77.37 & 77.8 & 77.91 & 77.69  \\
 \hline
\end{tabular}
\caption{\label{tab:mmlu_abalation} Accuracy on LM evaluation harness tasks on Llama2-7B model.}
\end{table}

\begin{table} \centering
\begin{tabular}{|c||c|c|c|c||c|c|c|c|} 
\hline
 $L_b \rightarrow$& \multicolumn{4}{c||}{8} & \multicolumn{4}{c||}{8}\\
 \hline
 \backslashbox{$L_A$\kern-1em}{\kern-1em$N_c$} & 2 & 4 & 8 & 16 & 2 & 4 & 8 & 16  \\
 %$N_c \rightarrow$ & 2 & 4 & 8 & 16 & 2 & 4 & 2 \\
 \hline
 \hline
 \multicolumn{5}{|c|}{Race (FP32 Accuracy = 48.8\%)} & \multicolumn{4}{|c|}{Boolq (FP32 Accuracy = 85.23\%)} \\ 
 \hline
 \hline
 64 & 49.00 & 49.00 & 49.28 & 48.71 & 82.82 & 84.28 & 84.03 & 84.25 \\
 \hline
 32 & 49.57 & 48.52 & 48.33 & 49.28 & 83.85 & 84.46 & 84.31 & 84.93  \\
 \hline
 16 & 49.85 & 49.09 & 49.28 & 48.99 & 85.11 & 84.46 & 84.61 & 83.94  \\
 \hline
 \hline
 \multicolumn{5}{|c|}{Winogrande (FP32 Accuracy = 79.95\%)} & \multicolumn{4}{|c|}{Piqa (FP32 Accuracy = 81.56\%)} \\ 
 \hline
 \hline
 64 & 78.77 & 78.45 & 78.37 & 79.16 & 81.45 & 80.69 & 81.45 & 81.5 \\
 \hline
 32 & 78.45 & 79.01 & 78.69 & 80.66 & 81.56 & 80.58 & 81.18 & 81.34  \\
 \hline
 16 & 79.95 & 79.56 & 79.79 & 79.72 & 81.28 & 81.66 & 81.28 & 80.96  \\
 \hline
\end{tabular}
\caption{\label{tab:mmlu_abalation} Accuracy on LM evaluation harness tasks on Llama2-70B model.}
\end{table}

%\section{MSE Studies}
%\textcolor{red}{TODO}


\subsection{Number Formats and Quantization Method}
\label{subsec:numFormats_quantMethod}
\subsubsection{Integer Format}
An $n$-bit signed integer (INT) is typically represented with a 2s-complement format \citep{yao2022zeroquant,xiao2023smoothquant,dai2021vsq}, where the most significant bit denotes the sign.

\subsubsection{Floating Point Format}
An $n$-bit signed floating point (FP) number $x$ comprises of a 1-bit sign ($x_{\mathrm{sign}}$), $B_m$-bit mantissa ($x_{\mathrm{mant}}$) and $B_e$-bit exponent ($x_{\mathrm{exp}}$) such that $B_m+B_e=n-1$. The associated constant exponent bias ($E_{\mathrm{bias}}$) is computed as $(2^{{B_e}-1}-1)$. We denote this format as $E_{B_e}M_{B_m}$.  

\subsubsection{Quantization Scheme}
\label{subsec:quant_method}
A quantization scheme dictates how a given unquantized tensor is converted to its quantized representation. We consider FP formats for the purpose of illustration. Given an unquantized tensor $\bm{X}$ and an FP format $E_{B_e}M_{B_m}$, we first, we compute the quantization scale factor $s_X$ that maps the maximum absolute value of $\bm{X}$ to the maximum quantization level of the $E_{B_e}M_{B_m}$ format as follows:
\begin{align}
\label{eq:sf}
    s_X = \frac{\mathrm{max}(|\bm{X}|)}{\mathrm{max}(E_{B_e}M_{B_m})}
\end{align}
In the above equation, $|\cdot|$ denotes the absolute value function.

Next, we scale $\bm{X}$ by $s_X$ and quantize it to $\hat{\bm{X}}$ by rounding it to the nearest quantization level of $E_{B_e}M_{B_m}$ as:

\begin{align}
\label{eq:tensor_quant}
    \hat{\bm{X}} = \text{round-to-nearest}\left(\frac{\bm{X}}{s_X}, E_{B_e}M_{B_m}\right)
\end{align}

We perform dynamic max-scaled quantization \citep{wu2020integer}, where the scale factor $s$ for activations is dynamically computed during runtime.

\subsection{Vector Scaled Quantization}
\begin{wrapfigure}{r}{0.35\linewidth}
  \centering
  \includegraphics[width=\linewidth]{sections/figures/vsquant.jpg}
  \caption{\small Vectorwise decomposition for per-vector scaled quantization (VSQ \citep{dai2021vsq}).}
  \label{fig:vsquant}
\end{wrapfigure}
During VSQ \citep{dai2021vsq}, the operand tensors are decomposed into 1D vectors in a hardware friendly manner as shown in Figure \ref{fig:vsquant}. Since the decomposed tensors are used as operands in matrix multiplications during inference, it is beneficial to perform this decomposition along the reduction dimension of the multiplication. The vectorwise quantization is performed similar to tensorwise quantization described in Equations \ref{eq:sf} and \ref{eq:tensor_quant}, where a scale factor $s_v$ is required for each vector $\bm{v}$ that maps the maximum absolute value of that vector to the maximum quantization level. While smaller vector lengths can lead to larger accuracy gains, the associated memory and computational overheads due to the per-vector scale factors increases. To alleviate these overheads, VSQ \citep{dai2021vsq} proposed a second level quantization of the per-vector scale factors to unsigned integers, while MX \citep{rouhani2023shared} quantizes them to integer powers of 2 (denoted as $2^{INT}$).

\subsubsection{MX Format}
The MX format proposed in \citep{rouhani2023microscaling} introduces the concept of sub-block shifting. For every two scalar elements of $b$-bits each, there is a shared exponent bit. The value of this exponent bit is determined through an empirical analysis that targets minimizing quantization MSE. We note that the FP format $E_{1}M_{b}$ is strictly better than MX from an accuracy perspective since it allocates a dedicated exponent bit to each scalar as opposed to sharing it across two scalars. Therefore, we conservatively bound the accuracy of a $b+2$-bit signed MX format with that of a $E_{1}M_{b}$ format in our comparisons. For instance, we use E1M2 format as a proxy for MX4.

\begin{figure}
    \centering
    \includegraphics[width=1\linewidth]{sections//figures/BlockFormats.pdf}
    \caption{\small Comparing LO-BCQ to MX format.}
    \label{fig:block_formats}
\end{figure}

Figure \ref{fig:block_formats} compares our $4$-bit LO-BCQ block format to MX \citep{rouhani2023microscaling}. As shown, both LO-BCQ and MX decompose a given operand tensor into block arrays and each block array into blocks. Similar to MX, we find that per-block quantization ($L_b < L_A$) leads to better accuracy due to increased flexibility. While MX achieves this through per-block $1$-bit micro-scales, we associate a dedicated codebook to each block through a per-block codebook selector. Further, MX quantizes the per-block array scale-factor to E8M0 format without per-tensor scaling. In contrast during LO-BCQ, we find that per-tensor scaling combined with quantization of per-block array scale-factor to E4M3 format results in superior inference accuracy across models. 


\end{document}
