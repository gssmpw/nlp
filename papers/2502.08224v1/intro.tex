% 
\begin{figure*}[htbp]
\centerline{\includegraphics[scale=0.3]{pic/challenge2.png}}
\caption{Illustration example of challenge 2.}
\label{challenge2}
\end{figure*}
% With the rapid growth of software engineering, traditional monolithic applications face significant challenges such as complex deployment and poor scalability due to the proliferation of services and frequent service iterations. In response to this scenario, Microservices Architecture (MSA) has emerged and rapidly evolve \citep{chen2024microfi}. By decomposing monolithic applications into small, autonomous service units, each focusing on specific business functionalities, MSA offers advantages such as loose coupling, independent deployment, and easy scalability. However, as the number of users and their demands increase, the variety and quantity of MSA also grow. Despite implementing numerous security measures, frequent incidents occur due to hardware malfunctions or misconfigurations, posing challenges to ensuring security. These incidents result in substantial losses for companies.

In today's large-scale web systems and services, traditional monolithic applications encounter notable challenges including intricate deployment processes and limited scalability \cite{micro1,micro2,micro3,micro4,micro5}, attributed to the proliferation of services and frequent service iterations. In response to this context, Microservices Architecture (MSA) has surfaced and continually evolved \citep{chen2024microfi}. By disassembling monolithic applications into small, self-sufficient service units, each dedicated to specific business functionalities, MSA presents benefits such as loose coupling, independent deployment, and effortless scalability. Nevertheless, with the escalation of user numbers and their corresponding demands, the diversity and quantity of MSA instances also increase. Despite the implementation of numerous monitor tools, recurrent incidents arise from hardware malfunctions or misconfigurations, posing challenges to reliability assurance. These incidents lead to substantial financial losses \cite{wang2024large}. For instance, on November 12, 2023, Alibaba experienced a large-scale outage, resulting in the interruption of multiple services for nearly three hours\footnote{https://www.datacenterdynamics.com/en/news/alibaba-cloud-hit-by-outage-second-in-a-month/}.

% To swiftly address these incidents, Root Cause Analysis (RCA) has become a hot research area in Artificial Intelligence for IT Operations (AIOps) in recent years. Traditional RCA methods often focus on a single data modality, such as LatentScpoe \citep{xie2024lantentscope}, which concentrates solely on metric data, proving to be lacking interpretability. In recent years, Large Language Model (LLM) agents, such as ReAct \citep{yao2022react} and ToolFormer \citep{schick2024toolformer}, have been applied across various domains. LLM agents leverage their powerful natural language understanding capabilities to flexibly orchestrate various tools in order to gradually complete complex tasks input by users, effectively addressing the challenges in the RCA domain characterized by numerous APIs, diverse data modalities. However, despite the significant power of LLM agents, efficiently and accurately utilizing LLM agents in RCA still faces challenges.



To promptly tackle these incidents, Root Cause Analysis (RCA) has emerged as a prominent research area within Artificial Intelligence for IT Operations (AIOps) in recent years \cite{rca1,rca2,rca3,rca4}. Traditional RCA techniques, in order to address the difficulties of manual fault diagnosis, have employed deep learning methods to learn from historical faults \citep{li2022dejavu}. However, these methods have two main drawbacks. First, they have poor adaptability to new scenarios, requiring model retraining when faced with a new situation. Second, they only output the root cause of the fault without providing the entire diagnostic process, resulting in poor explainability. This situation often results in Site Reliability Engineers (SREs) harboring a sense of distrust towards the results, as they fear that misidentifying the root cause could potentially result in further wasted repair time or exacerbate faults by addressing the wrong issue. Over the recent years, Large Language Model (LLM) agents like ReAct \citep{yao2022react} and ToolFormer \citep{schick2024toolformer} have been deployed across diverse domains. LLM agents harness their robust natural language understanding capabilities to adeptly coordinate various tools, allowing SREs to see the entire troubleshooting process and providing rich explanations for the root causes. Nonetheless, despite the considerable prowess of LLM agents, the efficient and accurate utilization of LLM agents in RCA encounters ongoing challenges.

\noindent\textbf{Challenge 1: Randomness and hallucinations leading to irrational action selection}

% Existing LLMs are predominantly probabilistic models \citep{radford2018gpt,radford2019gpt2}, hence exhibiting strong randomness and hallucinations. Even for simple tasks such as comparing two numbers, the current state-of-the-art (SOTA) model, GPT-3.5-Turbo, struggles to provide correct answers, let alone more complex RCA tasks. Leveraging an LLM agent for RCA tasks involves the retrieval and comprehension of various modalities of data (metric, log, trace) and the extensive invocation of API tools. As the context gradually expands, issues often arise such as inaccurate parameter extraction leading to tool invocation failures and mismatches between tool invocations and the context. Any randomness or hallucinations at any step significantly impacts the subsequent direction of the RCA process, potentially resulting in a scenario where despite using numerous tools and gathering vast amounts of information, the quality of information remains low due to unnecessary operations, and the stochastic nature of tool invocations leads to insufficient interrelation among pieces of information, ultimately hindering the identification of the true root cause.


Current LLMs primarily function as probabilistic models \citep{radford2018gpt,radford2019gpt2}, thereby exhibiting pronounced randomness and tendencies towards generating hallucinations. Employing an LLM agent for RCA activities necessitates the retrieval and comprehension of diverse data modalities (metric \citep{metric6}, log \citep{log2}, trace \citep{yaosparserca}) and the extensive utilization of API tools. As the scope of the context expands, issues often emerge such as inaccurate parameter extraction leading to failures in tool invocation and discrepancies between tool invocations and the context at hand. Instances of randomness or hallucinations at any stage can significantly impact the subsequent trajectory of the RCA procedure, hindering the accurate identification of the true root cause.


\noindent\textbf{Challenge 2: Complex and variable observations leading to multiple reasonable actions}

% \begin{wrapfigure}{r}{0.72\textwidth} % r表示图片在右边,0.4\textwidth表示图片占据40%的文本宽度
%   \centering
%   \includegraphics[width=0.71\textwidth]{ICLR 2025 Template/Pic/challenge2.png} % 调整图片大小
%   \caption{Illustration example of challenge 2.wwww
%   \label{challenge2}
% \end{wrapfigure}




% Existing LLM agents typically come equipped with a plethora of tools \citep{qin2023toolllm}, particularly in intricate domains such as RCA, where the number of APIs may soar into the hundreds. Invoking each API yields diverse observations, leading to complexity in selecting actions. Moreover, even with identical observations, multiple reasonable actions may exist. For instance, in the context of a coding error, the root cause could stem from a mistake in the code generation phase or inaccuracies in the associated documentation. Consequently, there are multiple viable actions, such as regenerating the code or revising the reference documents. The current LLM agents struggle to make prompt and informed decisions, despite equipped frameworks like ReAct. Therefore, orchestrating a rational sequence of actions amidst such intricate and dynamic scenarios poses a significant bottleneck and challenge in the advancement of LLM agents.


Existing LLM agents are typically bundled with a diverse array of tools \citep{qin2023toolllm}, especially within complex domains like RCA, where the number of APIs can escalate to hundreds. Each API invocation results in varied observations, thereby introducing intricacies in action selection. Furthermore, even when confronted with identical observations, multiple plausible actions may be viable. For example, as shown in Figure \ref{challenge2}, within the context of a code error ``Service name not found'', the root cause could originate from errors in the code generation phase or inaccuracies in associated SOP document, prompting multiple feasible actions like code regeneration or document revision.
% \deleted[id=pch]{Despite the presence of frameworks like \textbf{ReAct} \\citep{yao2022react}, current LLM agents encounter challenges in swiftly and judiciously determining the most appropriate course of action. Consequently, orchestrating a logical sequence of actions amidst such intricate and evolving scenarios represents a notable bottleneck and hurdle in the progression of LLM agents.}

% \textbf{Challenge 3: Conformity in multi-agent systems}

% Due to the limitations of individual agents, multi-agent systems have gradually gained acceptance. However, in existing multi-agent systems, agents typically operate as equal peers, leading to a susceptibility to conformity during communication processes. For instance, if agent A identifies the root cause as ``network," other agents may be influenced to abandon their own analyses and align with agent A's conclusion. Moreover, in the realm of RCA, effectively allocating tasks among diverse agents poses a significant challenge.

%\citep{wu2023autogen}.
% \deleted[id=pch]{
% %\textbf{Challenge 3: Information conflicts in multi-agent systems}
% In response to the constraints of individual agents, Multi-Agent Systems have garnered increasing recognition  Nevertheless, within current MAS frameworks, agents are often treated as equals, thereby engendering the risk of information conflicts during inter-agent communication. For instance, if agent A attributes the root cause to network issues, this determination may sway other agents to disregard their own analytical findings. Consequently, within the context of RCA, the effective assignment of tasks and facilitation of communication among diverse agents present a notable challenge.}

% To address the aforementioned challenges, we propose a novel Standard Operating Procedures Framework enhanced Multi-Agent System (SFMAS). Firstly, to mitigate randomness and hallucinations in orchestration, we introduce a framework based on Standard Operating Procedures (SOPs) \citep{hong2023metagpt}. Specifically, we incorporate SOPs into the knowledge base, where SOPs represent a set of generalized or standard steps to identify root causes. With SOPs' support, we utilize prompt engineering to align the MetaAgent positioning process with a generic framework: $Fault\ Class/Information\rightarrow SOP\rightarrow SOP Code\rightarrow Observation\rightarrow Fault\ Class/Information$. Secondly, to tackle challenge 2, we introduce an action selection mechanism based on an action set. Following the completion of thought in ReAct, SFMAS does not directly choose actions but first generates a rational action set before making the final action selection. Lastly, to address challenge 3, we devise a novel multi-agent system. Specifically, we introduce multiple agents including MetaAgent, CodeAgent, JudgeAgent, ObAgent, and ActionAgent, each responsible for distinct tasks, collaborating to enhance root cause identification


\begin{figure*}[htbp]
\centerline{\includegraphics[scale=0.3]{pic/multimodal3.png}}
\caption{Multimodal data collection and analysis.}
\label{fig:multimodal}
\end{figure*}

To confront the challenges outlined above, we propose \textbf{Flow-of-Action}, a Standard Operating Procedure (SOP) enhanced Multi-Agent System (MAS). Initially, to mitigate the impact of randomness and hallucinations in the orchestration process, we integrate SOPs into the knowledge base and propose the \textbf{SOP flow}. Specifically, SOPs outline a standardized set of steps for RCA, while SOP flow represents an efficient and accurate process built upon SOPs for their effective utilization. Through prompt engineering, we ensure that the orchestration of the main agent loosely follows the SOP flow in the absence of unexpected circumstances. Subsequently, to tackle the second challenge, compared with the thought-action-observation paradigm, we propose the thought-\textbf{actionset}-action-observation paradigm. Flow-of-Action avoids immediate action selection and instead generates a reasoned action set before making the final decision on the course of action. Besides, we devise a novel MAS. Specifically, we introduce multiple agents such as MainAgent, CodeAgent, JudgeAgent, ObAgent, and ActionAgent, each entrusted with distinct responsibilities, collaborating harmoniously to enhance root cause identification. 

Our key contributions are summarized as follows:
\begin{itemize}
    \item We propose the Flow-of-Action framework, the first agent-based fault localization process centered around SOPs. With this framework, we significantly reduce the inefficiency in action selection of the native ReAct framework and reducing the cost of trial and error.

    \item We introduce the concept of SOPs to integrate the expert experience into the LLM to greatly reduce hallucinations during RCA. For any given fault, we can automatically match the most relevant set of SOPs and can also generate new SOPs automatically, extending the limited set of human-generated SOPs.

    \item We innovatively propose a multi-agent collaborative system, including JudgeAgent and ObAgent. JudgeAgent assists the MainAgent in determining whether the root cause of the fault has been identified in the current iteration, while ObAgent helps MainAgent extract fault types and key information from massive amounts of data, addressing the information overload issue in the RCA process.

    \item Through a fault-injection simulation platform of a real-world e-commerce system, Flow-of-Action has increased the localization accuracy from 35\% to 64\% compared to ReAct, proving the effectiveness of the Flow-of-Action framework.

    % \item We devise a new knowledge base incorporating SOPs and design a novel RCA method based on SOP flow, reducing hallucinations to a certain extent.
    % \item We develop an action selection mechanism based on an action set, enabling optimal orchestration in varied observation scenarios.
    % \item We introduce a new MAS, where multiple agents assist the main agent in RCA, enhancing the root cause identification process.
\end{itemize}
