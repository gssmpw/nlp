\subsection{Traditional Methods}

The traditional RCA methods can be categorized into four types based on the data modalities they utilize: (1) Metric-based Methods ____: These typically involve constructing bayesian causal networks or graphs using data such as Remote Procedure Call (RPC). RCA is then performed through techniques like random walks or counterfactual analysis on these networks or graphs. (2) Log-based Methods ____: These focus on analyzing log data, such as examining changes in log templates or extracting specific keywords. These approaches aim to detect anomalies and simultaneously identify root causes. (3) Trace-based Methods ____: These methods identify root causes by observing changes in trace patterns. For instance, MicroRank ____ compares trace distributions before and after a failure to calculate anomaly scores. SparseRCA ____ employs historical data to train pattern recognition models for root cause identification. (4) Multi-modal Methods ____: These approaches posit that each data modality can, to some degree, reflect the root cause. It typically involves converting all data modalities into events or alerts, constructing a graph, and applying algorithms like PageRank ____ to localize the root cause.


\subsection{LLM-based Methods}

Due to its powerful natural language analysis and reasoning capabilities, LLMs have gradually been applied in RCA. ____ utilizes LLMs for summarization and recalls historically similar incidents to deduce the root cause of current issues. RCAgent ____ leverages code and log data to construct an agent based on ReAct for automated orchestration in root cause localization. mABC ____ adopts a more rational multi-agent framework and introduces a blockchain-based voting mechanism among agents. D-Bot ____ similarly employs a multi-agent framework, refining tool selection and knowledge structure. However, these methods are predominantly designed for specific scenarios such as databases, incorporating many context-specific elements like agent categories, thereby limiting their generalizability and transferability.