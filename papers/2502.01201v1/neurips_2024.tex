\documentclass{article}


% if you need to pass options to natbib, use, e.g.:
%     \PassOptionsToPackage{numbers, compress}{natbib}
% before loading neurips_2024


% ready for submission
% \usepackage{neurips_2024}


% to compile a preprint version, e.g., for submission to arXiv, add add the
% [preprint] option:
%     \usepackage[preprint]{neurips_2024}


% to compile a camera-ready version, add the [final] option, e.g.:
    \usepackage[final]{neurips_2024}

% to avoid loading the natbib package, add option nonatbib:
   % \usepackage[nonatbib]{neurips_2024}


\usepackage[utf8]{inputenc} % allow utf-8 input
\usepackage[T1]{fontenc}    % use 8-bit T1 fonts
\usepackage{hyperref}       % hyperlinks
\usepackage{url}            % simple URL typesetting
\usepackage{booktabs}       % professional-quality tables
\usepackage{amsfonts}       % blackboard math symbols
\usepackage{nicefrac}       % compact symbols for 1/2, etc.
\usepackage{microtype}      % microtypography
\usepackage{xcolor}         % colors
\usepackage{multirow}
\usepackage{graphicx}
\usepackage{xcolor}
\usepackage{amsmath}
\usepackage{enumitem}
\usepackage{footnote}



\title{One-to-Normal: Anomaly Personalization for Few-shot Anomaly Detection}


% The \author macro works with any number of authors. There are two commands
% used to separate the names and addresses of multiple authors: \And and \AND.
%
% Using \And between authors leaves it to LaTeX to determine where to break the
% lines. Using \AND forces a line break at that point. So, if LaTeX puts 3 of 4
% authors names on the first line, and the last on the second line, try using
% \AND instead of \And before the third author name.


% \author{%
%   Yiyue Li\thanks{Use footnote for providing further information
%     about author (webpage, alternative address)---\emph{not} for acknowledging
%     funding agencies.} \\
%   Department of Computer Science\\
%   Cranberry-Lemon University\\
%   Pittsburgh, PA 15213 \\
%   \texttt{hippo@cs.cranberry-lemon.edu} \\
%   % examples of more authors
%   \And
%   Shaoting Zhang \\
%   Affiliation \\
%   Address \\
%   \texttt{email} \\
%   % \AND
%   % Coauthor \\
%   % Affiliation \\
%   % Address \\
%   % \texttt{email} \\
%   % \And
%   % Coauthor \\
%   % Affiliation \\
%   % Address \\
%   % \texttt{email} \\
%   % \And
%   % Coauthor \\
%   % Affiliation \\
%   % Address \\
%   % \texttt{email} \\
% }

% \author{Yiyue Li\textsuperscript{1} \hspace{16pt} Shaoting Zhang\textsuperscript{4} \hspace{16pt} Kang Li\textsuperscript{134}\thanks{Corresponding author} \hspace{16pt} Qicheng Lao\textsuperscript{24*}\\[1ex]
%   \textsuperscript{1}West China Biomedical Big Data Center, West China Hospital, Sichuan University, Chengdu, China \\ 
%   \textsuperscript{2}School of Artificial Intelligence, Beijing University of Posts and Telecommunications, Beijing, China \\
%   \textsuperscript{3}Sichuan University - Pittsburgh Institute, Sichuan University, Chengdu, China. \\
%   \textsuperscript{4}Shanghai Artificial Intelligence Laboratory,
% Shanghai, China. \\
%   \texttt{kang.li.research@gmail.com}, \texttt{qicheng.lao@bupt.edu.cn} 
% }

\renewcommand{\thefootnote}{\fnsymbol{footnote}} 



% \author{Yiyue Li\textsuperscript{1} \hspace{7pt} Shaoting Zhang\textsuperscript{4} \hspace{7pt} Kang Li\textsuperscript{134 *} \hspace{7pt} Qicheng Lao\textsuperscript{24 *}\\[1ex]
%   \textsuperscript{1}West China Biomedical Big Data Center, West China Hospital, Sichuan University \\ 
%   \textsuperscript{2}School of Artificial Intelligence, Beijing University of Posts and Telecommunications \\
%   \textsuperscript{3}Sichuan University - Pittsburgh Institute, Sichuan University \\
%   \textsuperscript{4}Shanghai Artificial Intelligence Laboratory \\
%   \texttt{kang.li.research@gmail.com},  \texttt{qicheng.lao@bupt.edu.cn} 
%   \\[1ex]
% }


\author{Yiyue Li$^{1}$\quad Shaoting Zhang$^{4}$ \quad Kang Li$^{134\,*}$\quad Qicheng Lao$^{24\,*}$\\[1ex]
  \textsuperscript{1}West China Biomedical Big Data Center, West China Hospital, Sichuan University \\ 
  \textsuperscript{2}School of Artificial Intelligence, Beijing University of Posts and Telecommunications \\
  \textsuperscript{3}Sichuan University  Pittsburgh Institute, Sichuan University \\
  \textsuperscript{4}Shanghai Artificial Intelligence Laboratory \\
  \texttt{kang.li.research@gmail.com},  \texttt{qicheng.lao@bupt.edu.cn} 
  \\[1ex]
}



\begin{document}

\footnotetext{* Corresponding author}
\maketitle



\begin{abstract}
Traditional Anomaly Detection (AD) methods have predominantly relied on unsupervised learning from extensive normal data. Recent AD methods have evolved with the advent of large pre-trained vision-language models, enhancing few-shot anomaly detection capabilities. However, these latest AD methods still exhibit limitations in accuracy improvement. One contributing factor is their direct comparison of a query image's features with those of few-shot normal images. This direct comparison often leads to a loss of precision and complicates the extension of these techniques to more complex domains—an area that remains underexplored in a more refined and comprehensive manner. To address these limitations, we introduce the anomaly personalization method, which performs a personalized one-to-normal transformation of query images using an anomaly-free customized generation model, ensuring close alignment with the normal manifold. Moreover, to further enhance the stability and robustness of prediction results, we propose a triplet contrastive anomaly inference strategy, which incorporates a comprehensive comparison between the query and generated anomaly-free data pool and prompt information. Extensive evaluations across eleven datasets in three domains demonstrate our model's effectiveness compared to the latest AD methods. Additionally, our method has been proven to transfer flexibly to other AD methods, with the generated image data effectively improving the performance of other AD methods.

\end{abstract}


\section{Introduction}
\label{introduction}
Anomaly Detection (AD) has garnered considerable attention due to its wide applicability across various domains, such as industrial defect detection~\cite{li2021cutpaste,roth2022towards,bergmann2019mvtec,huang2022registration,pang2021deep}, medical diagnostics~\cite{xiang2024exploiting,li2023self,zhou2020encoding}, video surveillance~\cite{liu2018classifier,sultani2018real}, manufacturing inspection~\cite{zavrtanik2021draem,gong2019memorizing}. This heightened interest is primarily attributed to its only reliance on normal samples and its adoption of an unsupervised learning paradigm, where it typically learns from the distribution of normal samples to identify anomalies by detecting outliers~\cite{zou2022spot,zavrtanik2021draem}. Traditional AD methods, including auto-encoder based~\cite{zhou2020encoding,zavrtanik2021reconstruction}, GAN-based, and knowledge-based approaches~\cite{cao2023anomaly,deng2022anomaly,tien2023revisiting}, and others. Moreover, some diffusion-based methods~\cite{livernoche2023diffusion, he2023diad} have also emerged recently. Although most of these methods do not require annotated data, they do necessitate a substantial number of normal samples during the training phase to capture the distribution of normal samples effectively. However, this requirement has also severely constrained its advancement in various fields.

Recent studies~\cite{wang2022few,huang2022registration,xie2023pushing} in few-shot scenarios have improved AD tasks. %, traditionally relying on extensive datasets of normal samples. 
Currently, state-of-the-art (SOTA) advancements~\cite{zhu2024toward,jeong2023winclip,gu2024anomalygpt,zhou2023anomalyclip} are primarily due to the development of large pre-trained Visual-Language Models (VLMs). For example, WinCLIP~\cite{jeong2023winclip} first uses pre-trained CLIP, employing carefully designed text prompts and image feature comparisons to perform few-shot anomaly detection. AnomalyGPT~\cite{gu2024anomalygpt} eliminates the need for manually setting thresholds and supports multi-round dialogues. InCTRL~\cite{zhu2024toward} achieves general anomaly detection through in-context residual learning. Despite these innovations, the latest SOTA methods often rely on direct feature matching between few-shot normal images (i.e., reference images) and the query sample. However, without deeply exploring their subtle features, it is often difficult to achieve precise feature comparisons, which can easily lead to unstable results, since the non-anomalous differences between query and references can severely impact the prediction accuracy.  Furthermore, the limited number of normal images serving as independent references also makes their prediction insufficient. %Additionally, without deeply exploring their feature distribution, it is difficult to achieve precise feature comparisons, which can easily lead to unstable results.

In this work, we hypothesize that: 1) To achieve more accurate prediction results, it is essential to compare the query image with its corresponding or most similar normal image. Ideally, this comparison should involve a one-to-one transformation of the query image into its normal counterpart; 2) Additionally, one should also aim for robust and stable results, necessitating a comprehensive approach to accurately predicting results from multiple perspectives. To this end, we propose an anomaly personalization method for few-shot anomaly detection. To achieve more accurate, personalized comparative predictions, as mentioned in 1), our method employs a diffusion model to create an anomaly-free customized model inspired by \cite{ruiz2023dreambooth}. This model uses pairs of object text prompts and few-shot normal images (i.e., reference images) to explore the distribution of normal samples. Previous works~\cite{liu2023cones,chen2024subject,zhang2024expanding} have demonstrated the effectiveness of such customized models. Furthermore, the suitability of diffusion models stems from their powerful customization capabilities, allowing for flexible control over intermediate steps and the generation process compared to other methods~\cite{meng2021sdedit}. %Notably, only one diffusion model needs to be customized for all objects. 
Additionally, for enhanced stability and robustness as mentioned in 2), we propose the use of triplet contrastive anomaly inference, i.e., in addition to comparing the query image with normal samples, we conduct comprehensive comparisons with personalized samples and text prompts. %This enables us to establish an anomaly-free sample pool, facilitating a comparison of the entire distribution of normal samples. 
Furthermore, we use comprehensive state words and templates as text prompts inspired by~\cite{huang2024adapting,jeong2023winclip}. Ultimately, we synthesize predictions from diverse perspectives to yield the final anomaly score. 

To summarize, this paper makes the following main contributions:
\begin{itemize}[leftmargin=*]
% \item We introduce a novel personalized few-shot anomaly detection method, unlike other state-of-the-art (SOTA) approaches that directly compare query images with reference images. Our method enables a finer-grained comparison through one-to-normal personalization of query images, leading to enhanced prediction accuracy.

\item We introduce a novel anomaly personalization method for few-shot anomaly detection, unlike other state-of-the-art (SOTA) approaches that directly compare query images with reference images, our method enables a finer-grained comparison through one-to-normal personalization of query images, leading to enhanced prediction accuracy.

\item To achieve more stable and robust results, we propose a triplet contrastive anomaly inference strategy. This approach facilitates a more comprehensive comparison by incorporating not only customized comparisons but also comparisons with anomaly-free samples and text prompts. The anomaly-free samples augment normal images sampled by an anomaly-free customized model.

\item Our proposed method demonstrates strong generalizability. We conduct comprehensive experiments across 11 datasets spanning three distinct domains: industrial, medical, and semantic. Moreover, the anomaly-free samples generated by our method can be used to augment the normal samples in most few-shot anomaly detection methods, enhancing the performance of some existing methods and demonstrating adaptability and flexibility.

% \item Our method demonstrates adaptability and flexibility. The anomaly-free samples generated by our method can be used to augment the normal samples in most few-shot anomaly detection methods. Our experiments also show that it enhances the performance of some existing few-shot anomaly detection methods.


\end{itemize}

 

\section{Related Work}
\label{related_work}
\subsection{Anomaly Detection}
Anomaly detection (AD) is a critical task in computer vision aimed at identifying samples that deviate significantly from the norm. Traditional AD methods can be categorized into several types:  auto-encoder based~\cite{zhou2020encoding,zavrtanik2021reconstruction}, GAN-based, and knowledge-based approaches~\cite{cao2023anomaly,deng2022anomaly,tien2023revisiting}, and others. Recent advancements in AD research focus on few-shot and zero-shot learning to overcome data limitations. Few-shot AD methods like RegAD~\cite{huang2022registration} utilize pre-trained models and a few normal samples from the target domain to detect anomalies without extensive re-training. WinCLIP~\cite{jeong2023winclip} meticulously designs a variety of prompts to ensure the model comprehensively covers all possible normal and abnormal scenarios. AnomalyCLIP~\cite{zhou2023anomalyclip} uses CLIP for zero-shot anomaly detection. They have also meticulously designed prompts but have overlooked the aspect of image comparison. In summary, many of the current few-shot AD approaches have primarily focused on text prompt refinement without detailed exploration in visual feature comparisons and sample generation.

\subsection{Image Personalization}
Personalized text-to-image generation has emerged as a pivotal area, particularly for creating highly personalized and contextually accurate images. DreamBooth~\cite{ruiz2023dreambooth} is the first to focus on fine-tuning diffusion models with specific subject images to generate high-fidelity renditions of those subjects, typically using 3 to 5 images for customization and incorporating prior preservation to prevent language draft. SuTi~\cite{chen2024subject} leverages a single model to learn from a multitude of expert models, each fine-tuned to specific subjects, allowing for instant personalization using in-context learning with only a few examples. Another innovative approach~\cite{liu2023cones} involves "concept neurons" in diffusion models, identifying clusters of neurons that correspond to specific subjects within a pre-trained model. These methods have achieved significant results in image generation. However, most of these methods have been applied to image editing and synthesis, and have yet to be explored for anomaly detection.

% \subsection{Visual-Language Modeling}
% Vision-language models like CLIP\cite{radford2021learning} have shown remarkable generalization capabilities by leveraging large-scale image-text datasets for pre-training. CLIP, for instance, aligns visual and textual representations, enabling zero-shot transfer to various downstream tasks. In anomaly detection, WinCLIP~\cite{jeong2023winclip} uses compositional prompt ensembles and multi-scale feature aggregation to improve zero-shot and few-shot anomaly classification and segmentation. AnomalyCLIP~\cite{zhou2023anomalyclip} uses CLIP for zero-shot anomaly detection. They have all meticulously designed prompts but have overlooked the aspect of image comparison.


\section{Method}
\label{method}


\begin{figure}[!t]
\includegraphics[width=1\columnwidth]{figs/main_table_new.png}
\caption{Overview of our proposed anomaly personalization approach. First, we use few-shot normal images to customize an anomaly-free diffusion model (\textit{top left}), similar to Dreambooth. Next, we perform one-to-normal personalization of the query image, obtaining the personalized image (\textit{bottom left}). Finally, we adopt a triplet contrastive anomaly inference method, which integrates anomaly scores from three aspects to obtain the final results. $S_N$, $S_P$ and $S_{text}$ represent different contrastive anomaly inference processes.}
\label{multi_instance}
\end{figure}


\subsection{Overview}
We propose a novel anomaly personalization method to improve few-shot anomaly detection. First, to obtain more precise results, we aim for a one-to-one customized comparison, where the query image is compared with the image derived from itself, transformed towards an anomaly-free distribution to align the anomalous distribution with the normal state. This is achieved by employing an anomaly-free customized model (Sec.~\ref{Normal-oriented}) followed by one-to-normal personalization of the query image~(Sec.~\ref{one-to-normal}). Second, to enhance the detection stability and robustness, we further propose a triplet contrastive anomaly inference process (Sec.~\ref{diverse_compare}), where the query image is thoroughly compared with its personalized version, augmented anomaly-free samples, and text prompts for comprehensive analysis. 



\subsection{Anomaly-Free Customized Model}
\label{Normal-oriented}

To develop a customized model for normal objects, we choose diffusion model~\cite{ho2020denoising} as the framework, building on the work based on \cite{ruiz2023dreambooth} which has proven effective for specific object customization. %for customization towards flawless objects, as this method has proven effective for specific object customization. 
To enhance the diversity of normal object samples given limited data, we first apply data augmentation to normal reference images. Subsequently, to better align objects with their textual descriptions, we provide the diffusion model with a series of demonstrations $\mathbb{C}_{o}$, using augmented reference images paired with their corresponding texts for specific objects. Specifically, for each object (e.g, cable), we prepare a series of augmented reference images (i.e., few-shot normal samples) $x_{\text{normal}}\in \mathcal{D}_{\text{train}}$ and their matching text $c$, i.e., \textit{a photo of normal [object]}, for tailored training of the diffusion model. Here, the specific category name (e.g., `cable') is used to replace `object'. Additionally, another set of demonstrations $\overline{\mathbb{C}_{o}}$ is employed to design a class-specific prior preservation loss, aimed at promoting diversity and preventing language drift, similar to \cite{ruiz2023dreambooth,chen2024subject}. To obtain an anomaly-free customized model $D_{\theta}(x_{\text{normal},t},c)$ 
parameterized by $\theta$ on an object $o$, we constrain a pre-trained diffusion model on the image cluster $\mathbb{C}_{o}$ with the denoising loss as:

\begin{equation}
\theta = \mathop{\arg \min}\limits_{\theta}\mathbb{E}_{(x_{0},c)\sim\mathbb{C}_{o}\cup\overline{\mathbb{C}_{o}}}\{\mathbb{E}_{\epsilon,t}[||D_{\theta}(x_{\text{normal},t},c)-x_{0}]||_{2}^2\},
\end{equation}

where $c$ is a short description of normal images $x_{\text{normal}}$, and $x_{\text{normal},t}$ is a latent version of the input noised by $t$ steps. Note that the dimensionality of both the image and the latent is the same throughout the entire process. 
Therefore, the anomaly-free customized model learns the distribution of normal images $P_{\text{normal}}$ conditioned on the text prompt $c$, where anomaly-free samples can be generated. 
%by sampling from the learned customized model.
% The customized diffusion model $\hat{x}_{\theta}(x_{t},c)$

% $$p_{nor} = p_{\theta}(\hat{x}_{nor}|x_{nor,t})$$

% The anomaly-free customized model learns the distribution of normal images conditioned on the text prompt $c$, denoted as $p_{\theta}$. $P_{normal}$ represents the joint probability density function of all normal image samples conditioned on the given textual description $c$. Given a prompt $c$, the probability distribution of normal images can be expressed as:
% $$P_{normal} = P(normal|c) = \int p_{\theta}(x_{normal}|c)~dn,$$
% where $p_{\theta}(x_{normal}|c)$ represents the learned conditional probability distribution of normal images given the description $c$. Therefore, normal images can be generated by sampling from the learned distribution and incorporating them into an anomaly-free sample pool.

%we follow the generative process of the diffusion model,starting from noise samples and few-shot normal images

% Many works~\cite{liu2023cones,chen2024subject} have demonstrated the impressive customized abilities of object-driven diffusion model. Our objective in developing such an anomaly-free customized model is to facilitate the subsequent one-to-normal personalization of query images and establish an anomaly-free data pool, both aimed at enabling more refined feature comparisons. 

\subsection{One-to-Normal Personalization}
\label{one-to-normal}
Our proposed one-to-normal strategy is designed to transform the query image of an object toward the distribution of its normal samples. It adaptively retains the normal characteristics while the anomaly parts are gradually transformed with no defects. This is similar to previous work~\cite{liu2023cones,meng2021sdedit} for image editing; however, our strategy specifically focuses on utilizing it to align with the normal manifold.


Specifically, we first design text prompts to maximize the retention of information in the normal regions of the query image while transforming anomalous areas towards a normal state. To mitigate the influence of other factors (e.g., contrast, image quality), we simulate all potential normal states using these prompts. Inspired by~\cite{jeong2023winclip,zhou2023anomalyclip}, we curate a template list for various potential image physical states, e.g., \textit{a low-resolution photo of a normal object}. Additionally, we include descriptors for common states shared by most normal objects, such as \textit{a photo of normal object without flaw}. We provide these descriptors for normal state prompts in Appendix~\ref{prompt}. 

Given a series of normal state prompts $\{c\}_{i=1}^{n}$ described above, we aim to select the generated image that most closely resembles the normal state for the one-to-normal personalization of the query image $x_{q}$. Specifically, we reconstruct a set of reconstructed images $\{\hat{x}_{i}\}_{i=1}^{n}$ where each text prompt $\mathnormal{c}_{i}$ guides the diffusion process of the customized anomaly-free model. %The objective of the text prompt strategy is to identify the reconstructed image that most closely retains its original normal state. When $t$ is same, the degree of transformation is consistent, 
We select the optimal text prompt $c_{q}$ that most closely resembles the normal state from the prompt set:

\begin{equation}
c_{q} = \arg \min_{c_{i}}\mathcal{L}({x_{q}, \hat{x}_{i}}) = \arg \min_{c_{i}}\mathcal{L}({x_{q}, D_{\theta}}(x_{q,t_{}}, \mathnormal{c}_{i}, t_{})),
\end{equation}

where $D_{\theta}$ represents the anomaly-free customized diffusion model derived from the Section \ref{Normal-oriented}, and $\mathcal{L}$ represents SSIM. 
Finally, we obtain the personalized images $\hat{x}_{q}$:

\begin{equation}
\hat{x}_{q} = D_{\theta}(\sqrt{\alpha_{t_{}}}x_{q} + \sqrt{1-\alpha_{t}}\epsilon, c_{q}, t_{}),
\end{equation}

where $\alpha$ is a noise schedule, $\epsilon ~ \mathcal{N}(0,I)$ represents Gaussian noise, and $t$ is the diffusion step. 

% Simultaneously, the query image $x_{q}$ is mapped into a latent space transforming into a compressed image, which is subjected to a subsequent noising process. 
Following~\cite{luzi2022boomerang}, during this noising phase, the hyperparameter $t$ is meticulously adjusted to dictate the extent of the one-to-normal personalization, which determines the similarity between query image $x_{q}$ and personalized image $\hat{x}_{q}$. A relatively lower $t$ means insufficient personalization where a substantial portion of the original query is retained. %ensures that the customization process towards a normal image retains a substantial portion of $q$'s original content. 
As $t$ approaches 0, the generated image becomes identical to the input query image, and the text prompt's condition has no effect. Conversely, setting $t$ to $T$ initiates a complete forward diffusion process, consequently erasing all information in $x_{q}$ and allowing the generation to be fully guided by text prompts $c_q$ toward the normal manifold. %The noised image prediction after $t$ steps is:
% $$x_{q,t} = \sqrt{\alpha_{t}}x_{q} + \sqrt{1-\alpha_{t}}\epsilon_{t},$$
% where $\alpha$ is a noise schedule parameter and $\epsilon \sim \mathcal{N}(0,I)$ represents Gaussian noise.
% For anomaly-free customized model $\hat{x}_{\theta}$, this process of obtaining customized images can be represented as $\hat{x}_{\theta}(q, t_{c}) = \hat{q}$. We utilize the customized image obtained for each query image in the subsequent prediction tasks.
%In this project, when executing a one-to-normal personalized transformation, the hyperparameter is set to $t_{q}$. 
Ultimately, we bridge the distributions of $P_{\text{query}}$ and $P_{\text{normal}}$ for obtaining $P_{\text{personalized}}$, from which we can sample personalized images for subsequent anomaly detection tasks. %We utilize the personalized image obtained for each query image in the subsequent prediction tasks.



% \subsection{multi-level Feature Comparison}
% \subsection{Diverse-level Refined Feature Comparison}
\subsection{Triplet Contrastive Anomaly Inference}
\label{diverse_compare}

To achieve comprehensive predictions and enhance the precision and robustness, we introduce triplet contrastive anomaly inference, i.e., in addition to comparing the query image with anomaly-free samples and text prompts, we also compare it with our personalized images.
% comparing customized images, we also compare them with anomaly-free samples, including generated normal images and few-shot normal samples. 
Finally, we integrate predictions from three comparison aspects to mutually complement each other.


\textbf{One-to-one personalized comparison.} We divide the CLIP image encoder into $n$ multi-feature extraction blocks, each designed to capture the intermediate features of the input image at different levels. For a query image $x_{q}$ and its corresponding personalized image $\hat{x_{q}}$, the model extracts features $F_{q}\in\mathbb{R}^{h\times w\times d}$ from $x_{q}$ and $\hat{F_{q}}\in\mathbb{R}^{h\times w\times d}$ from $\hat{x_{q}}$. The comparison score $S_{P}$ between the query image and its personalized image is then computed by evaluating their similarity in the extracted features at each level:

\begin{equation}
S_{P} = \frac{1}{n}\sum^{n}_{l=1}\mathop{max}\limits_{\mathcal{G}}(1-\langle F_{q,l}, 
 \hat{F}_{q,l}\rangle), 
\end{equation}

where $l$ denotes the $l$-th level feature extraction block, $n$ represents the total number of blocks, and $\langle\cdot\rangle$ signifies the cosine similarity function, $\mathcal{G}$ represents the grid number.


\textbf{Anomaly-free sample comparison.} %To enhance the robustness and stability of the results, we compare the query image with anomaly-free samples. 
Our anomaly-free sample pool comprises a set of normal reference images and generated normal images, better representing the distribution of normal samples. 
We first generate normal samples from the customized anomaly-free model $D_{\theta}$ in Section \ref{Normal-oriented} and incorporate them into the anomaly-free sample pool for subsequent prediction tasks, as previous studies have demonstrated the diffusion model's capability to synthesize high-fidelity images, which have been proven effective in many works~\cite{zhang2024expanding,akrout2023diffusion,luzi2022boomerang}. 
To further expand the anomaly-free sample pool, % and more comprehensively represent the normal distribution, 
we use the same process described in Section \ref{one-to-normal} but with settings that more closely align with the distribution of normal reference image instead of the query image. %to generate images and add them to the anomaly-free sample pool.
Then, we employ the same CLIP image encoder to extract multi-level features from anomaly-free samples (e.g., 30) and store these features in a memory bank $M$. %Subsequently, the prediction for anomaly score is computed by evaluating the similarity between multi-level features of the query image $x$ and anomaly-free samples' features stored in memory bank $M$. Precisely, 
The prediction score $S_{N}$ between the query image and anomaly-free samples can be expressed as:

\begin{equation}
S_{N} = \frac{1}{n}\sum^{n}_{l=1}\mathop{max}\limits_{\mathcal{G}}(\mathop{min}\limits_{m\in M}(1-\langle F_{q,l}, m_{l}\rangle)).
\end{equation}

\textbf{Text prompt comparison.} %The subsequent step involves calculating the anomaly score between the query image features and the text prompt features to develop a more comprehensive prediction framework. 
To calculate the anomaly score between the query image features and the text prompt features, we categorize text prompts into two types: normal and abnormal objects, similar to~\cite{jeong2023winclip,chen2023zero,huang2024adapting}. We aim to cover more possible states for these objects to better simulate the various potential conditions of the images, similar to \cite{jeong2023winclip}. % Specifically, the text features $F_{text}\in\mathbb{R}^{2 \times d}$ are obtained using the CLIP text encoder for a set of given normal and abnormal prompts.
Specifically, we obtain text features by applying the CLIP text encoder to a set of predefined normal and abnormal prompts and then calculating the average of these features, denoted as $F_{text}\in\mathbb{R}^{2 \times d}$. To facilitate comparison, the global image feature for anomaly detection obtained from the CLIP image encoder is represented as $F_{q}\in\mathbb{R}^{1 \times d}$. These two features from text prompts and the query image are then used to calculate the anomaly score as follows:

%The query image $x$ is processed through multi-level feature extraction blocks, yielding corresponding multi-level features. The text prompts $F_{text}$ 
% which are then compared to these multi-features $F^{\prime}_{q,l}$ from the query image through a linear layer:
\begin{equation}
S_{text} = softmax(F_{q} F_{text}^{T}). 
\end{equation}
% \begin{equation}
% S_{text} = \frac{1}{n}\sum^{n}_{l=1}\mathop{max}\limits_{\mathcal{G}}(softmax(F_{q} F_{text}^{T})). 
% \end{equation}

Here, the score $S_{text}$ represents the probability corresponding to the anomaly.
% Here, $softmax(\cdot)$ is the average softmax scores across multiple levels.

The final prediction result is generated by combining outputs from three branches: the personalized image, anomaly-free samples, and text prompts. These three sets of output scores provide complementary information for collaboration. The final anomaly score can be obtained by combining the three branches:

\begin{equation}
\mathcal{A}_{score} =  S_{P} +\alpha S_{N} + \beta  S_{text}, 
\end{equation}

where $\alpha$, $\beta$ are the hyper-parameters.


\section{Experiments}
\label{experiments}


\subsection{Experiment Setup}
\textbf{Dataset and Evaluation Metrics.} We validate the effectiveness of our method across 11 datasets spanning three distinct domains: industrial, medical, and semantic domains. In the industrial domain, we utilize multiple datasets including MVTec-AD~\cite{bergmann2019mvtec}, Visa~\cite{zou2022spot}, KSDD~\cite{tabernik2020segmentation}, AFID~\cite{silvestre2019public}, and ELPV~\cite{deitsch2019automatic}. For the medical domain, we incorporate datasets covering various modalities such as magnetic resonance imaging (MRI), computed tomography (CT), and optical coherence tomography (OCT). Specifically, the medical datasets include OCT2017~\cite{kermany2018identifying}, BrainMRI, HeadCT~\cite{hu2019automated}, and RESC~\cite{hu2019automated}. For semantic anomaly detection, we employ two datasets: MNIST~\cite{lecun1998gradient} and CIFAR-10~\cite{krizhevsky2010cifar}. These semantic datasets are utilized following the one-vs-all strategy, where one class is designated as normal, and all other classes are treated as anomalous. We follow~\cite{zhu2024toward} for the dataset partitions. %Most dataset partitions are consistent with \cite{zhu2024toward}. In this study, 
We only use data from the MVTec-AD dataset as auxiliary training data. When testing on the MVTec-AD dataset, we use auxiliary data from the Visa dataset. The anomaly detection performance is evaluated using the Area Under the Receiver Operating Characteristic Curve (AUROC) and the Area Under the Precision-Recall Curve (AUPRC).

\textbf{Competing Methods and Baselines.} We compare our proposed method with various state-of-the-art anomaly detection (AD) methods, including traditional AD approaches (full-shot) such as PaDiM~\cite{defard2021padim} and PatchCore~\cite{roth2022towards}, both of which are adapted to the few-normal-shot setting. Additionally, we compare a traditional few-shot learning method, RegAD~\cite{huang2022registration} and prompt learning method CoOp~\cite{zhou2022conditional}, along with the latest AD methods based on vision-language models, including WinCLIP~\cite{jeong2023winclip} and InCTRL~\cite{zhu2024toward}. To further validate the flexibility and applicability of our method, we evaluate our generated anomaly-free samples using four state-of-the-art anomaly detection methods: PaDiM, PatchCore, WinCLIP and InCTRL.

\textbf{Implementation Details.} We select Stable Diffusion V1.5~\cite{rombach2022high} and utilize the CLIP with ViT-L/14 architecture~\cite{radford2021learning} for all tasks. Our anomaly-free customized model is fine-tuned using Dreambooth~\cite{ruiz2023dreambooth}, and all model parameters are frozen in subsequent tasks. All the experiments are trained by use of PyTorch on an NVIDIA GeForce RTX 4090 GPU. Our anomaly detection task is a few-normal-shot setting including 2-shot, 4-shot, and 8-shot scenarios using only normal images.
The ratio of the t-step is set to 0.3. We set the parameters $\alpha$ and $\beta$ for $\mathcal{A}_{score}$ to 1 and 0.5, respectively,
across all datasets. The image-level memory bank is set to 30, and the inference resolution is 240$\times$240. During inference, we use the same text prompts as WinCLIP~\cite{jeong2023winclip}. To validate the improvement of other anomaly detection methods with our generated data, we use our generated anomaly-free samples to augment the datasets for these methods. For the MVTec dataset, we generate 100 anomaly-free samples per subclass to expand the reference dataset, while keeping all other settings identical to previous works.

\begin{table}[!t]
\caption{A quantitative comparison of our proposed method and other methods using (AUROC ($\%$), AUPRC ($\%$)) as the evaluation metric on \textbf{industrial datasets}. We show the average performance and standard deviation across five runs, with the top value highlighted in bold for each comparison.}
\centering
\renewcommand\arraystretch{1.7}
\setlength{\tabcolsep}{3pt}
\resizebox{1\columnwidth}{!}{%
\begin{tabular}{ccllllllllllllll}
% \cline{3-16}
\hline
\multicolumn{1}{l}{} & \multicolumn{1}{l}{} & \multicolumn{14}{c}{Methods} \\ \hline
\multicolumn{1}{r}{} & \multicolumn{1}{c|}{Datasets} & \multicolumn{2}{c}{PaDiM~\cite{defard2021padim}} & \multicolumn{2}{c}{Patchcore~\cite{roth2022towards}} & \multicolumn{2}{c}{RegAD~\cite{huang2022registration}} & \multicolumn{2}{c}{CoOp~\cite{zhou2022conditional}} & \multicolumn{2}{c}{WinCLIP~\cite{jeong2023winclip}} & \multicolumn{2}{c|}{InCTRL~\cite{zhu2024toward}} & \multicolumn{2}{c}{Ours} \\ \hline

  & \multicolumn{1}{c|}{MVTec} & \multicolumn{2}{l}{(78.5$\pm$2.5 , 89.0$\pm$1.5)} & \multicolumn{2}{l}{(85.8$\pm$3.4, 93.9$\pm$1.2)} & \multicolumn{2}{l}{(64.0$\pm$4.7, 83.7$\pm$3.4)} & \multicolumn{2}{l}{(85.8$\pm$1.6, 92.2$\pm$0.7)} & \multicolumn{2}{l}{(93.1$\pm$1.9, 96.5$\pm$0.7)} & \multicolumn{2}{l|}{(94.0$\pm$1.5, 96.9$\pm$0.4)} & \multicolumn{2}{l}{\textbf{(95.1$\pm$0.9, 97.3$\pm$0.4)}} \\
 & \multicolumn{1}{c|}{VisA} & \multicolumn{2}{l}{(68.0$\pm$4.2, 71.9$\pm$2.7)} & \multicolumn{2}{l}{(81.7$\pm$2.8, 84.1$\pm$2.3)} & \multicolumn{2}{l}{(55.7$\pm$5.3, 61.4$\pm$3.7)} & \multicolumn{2}{l}{(80.6$\pm$2.3, 83.5$\pm$1.9)} & \multicolumn{2}{l}{(84.2$\pm$2.4, 85.9$\pm$2.1)} & \multicolumn{2}{l|}{(85.8$\pm$2.2, 87.7$\pm$1.6)} & \multicolumn{2}{l}{\textbf{(87.2$\pm$1.4, 89.1$\pm$1.3)}} \\
2-shot & \multicolumn{1}{c|}{KSDD} & \multicolumn{2}{l}{(72.1$\pm$1.5, 33.7$\pm$0.8)} & \multicolumn{2}{l}{(90.2$\pm$0.6, 67.6$\pm$0.3)} & \multicolumn{2}{l}{(49.9$\pm$0.8, 17.3$\pm$1.9)} & \multicolumn{2}{l}{(89.7$\pm$0.6, 54.3$\pm$0.4)} & \multicolumn{2}{l}{(94.2$\pm$0.6, 86.5$\pm$0.4)}  & \multicolumn{2}{l|}{(\textbf{97.2$\pm$2.9, 91.7$\pm$0.9})} & \multicolumn{2}{l}{(96.8$\pm$1.5, 91.6$\pm$0.6)} \\
 & \multicolumn{1}{c|}{AFID} & \multicolumn{2}{l}{(78.4$\pm$2.8, 52.9$\pm$3.4)} & \multicolumn{2}{l}{(73.9$\pm$1.7, 37.8$\pm$0.8)} & \multicolumn{2}{l}{(56.4$\pm$7.2, 27.5$\pm$3.5)} & \multicolumn{2}{l}{(68.7$\pm$6.2, 44.3$\pm$0.5)} & \multicolumn{2}{l}{(72.6$\pm$5.5, 50.0$\pm$4.3)}  & \multicolumn{2}{l|}{(76.1$\pm$2.9, 51.9$\pm$2.2)} & \multicolumn{2}{l}{\textbf{(78.3$\pm$1.7, 53.6$\pm$1.2}} \\
 & \multicolumn{1}{c|}{ELPV} & \multicolumn{2}{l}{(59.4$\pm$8.3, 70.7$\pm$5.8)} & \multicolumn{2}{l}{(71.6$\pm$3.1, 84.0$\pm$3.1)} & \multicolumn{2}{l}{(57.1$\pm$1.6, 67.9$\pm$0.5)} & \multicolumn{2}{l}{(76.2$\pm$1.1, 84.1$\pm$0.2)} & \multicolumn{2}{l}{(72.6$\pm$2.0 , 84.9$\pm$1.0)}  & \multicolumn{2}{l|}{(83.9$\pm$0.3, 91.3$\pm$0.8)} & \multicolumn{2}{l}{\textbf{(85.6$\pm$0.5, 92.0$\pm$0.5)}} \\ \hline
 & \multicolumn{1}{c|}{MVTec} & \multicolumn{2}{l}{(80.5$\pm$1.8, 90.9$\pm$1.3)} & \multicolumn{2}{l}{(88.5$\pm$2.6, 95.0$\pm$1.3)} & \multicolumn{2}{l}{(66.3$\pm$3.2, 84.6$\pm$2.6)} & \multicolumn{2}{l}{(87.4$\pm$1.7, 92.4$\pm$0.8)} & \multicolumn{2}{l}{(94.0$\pm$2.1, 96.8$\pm$0.8)} & \multicolumn{2}{l|}{(94.5$\pm$1.8, 97.2$\pm$0.6)}& \multicolumn{2}{l}{\textbf{(95.6$\pm$1.2, 97.8$\pm$0.6)}} \\
 & \multicolumn{1}{c|}{VisA} & \multicolumn{2}{l}{(73.5$\pm$3.1, 75.8$\pm$1.8)} & \multicolumn{2}{l}{(84.3$\pm$2.5, 86.0$\pm$1.6)} & \multicolumn{2}{l}{(57.4$\pm$4.2, 62.8$\pm$3.4)} & \multicolumn{2}{l}{(81.8$\pm$1.8, 84.2$\pm$1.6)} & \multicolumn{2}{l}{(85.8$\pm$2.5, 87.5$\pm$2.3)}  & \multicolumn{2}{l|}{(87.7$\pm$1.9, 90.2$\pm$2.7)} & \multicolumn{2}{l}{\textbf{(88.6$\pm$1.6, 90.7$\pm$0.6)}}\\
4-shot & \multicolumn{1}{c|}{KSDD} & \multicolumn{2}{l}{(74.2$\pm$1.4, 35.1$\pm$1.2)} & \multicolumn{2}{l}{(92.3$\pm$0.8, 70.3$\pm$1.3)} & \multicolumn{2}{l}{(52.5$\pm$2.7, 17.6$\pm$0.3)} & \multicolumn{2}{l}{(90.2$\pm$0.6, 59.4$\pm$1.4)} & \multicolumn{2}{l}{(94.3$\pm$0.4, 86.8$\pm$0.3)}  & \multicolumn{2}{l|}{(97.5$\pm$0.6, 92.4$\pm$1.5)} & \multicolumn{2}{l}{\textbf{(97.8$\pm$0.8, 92.4$\pm$0.7)}}\\
 & \multicolumn{1}{c|}{AFID} & \multicolumn{2}{l}{(78.7$\pm$3.8, 54.0$\pm$5.3)} & \multicolumn{2}{l}{(73.3$\pm$0.2, 37.7$\pm$0.1)} & \multicolumn{2}{l}{(59.6$\pm$7.4, 29.4$\pm$3.1)} & \multicolumn{2}{l}{(72.0$\pm$1.7, 45.4$\pm$1.4)} & \multicolumn{2}{l}{(76.4$\pm$2.5, 51.3$\pm$1.7)} & \multicolumn{2}{l|}{(79.0$\pm$1.8, 54.8$\pm$1.6)} & \multicolumn{2}{l}{\textbf{(82.6$\pm$0.7, 57.9$\pm$1.3)}} \\
 & \multicolumn{1}{c|}{ELPV} & \multicolumn{2}{l}{(61.2$\pm$0.8, 72.4$\pm$6.7)} & \multicolumn{2}{l}{(75.6$\pm$7.3, 87.1$\pm$4.2)} & \multicolumn{2}{l}{(59.6$\pm$4.0, 68.8$\pm$1.8)} & \multicolumn{2}{l}{(78.1$\pm$0.2, 86.7$\pm$0.3)} & \multicolumn{2}{l}{(75.4$\pm$0.9, 86.4$\pm$0.4)} & \multicolumn{2}{l|}{ (84.6$\pm$1.1, 91.6$\pm$0.9)} & \multicolumn{2}{l}{\textbf{(87.3$\pm$0.8, 92.8$\pm$0.6)}} \\ \hline
 & \multicolumn{1}{c|}{MVTec} & \multicolumn{2}{l}{(82.0$\pm$1.6, 92.7$\pm$1.2)} & \multicolumn{2}{l}{(92.2$\pm$1.9, 96.2$\pm$1.3)} & \multicolumn{2}{l}{(67.4$\pm$3.3, 85.5$\pm$2.1)} & \multicolumn{2}{l}{(88.0$\pm$1.4, 93.3$\pm$0.7)} & \multicolumn{2}{l}{(94.7$\pm$2.5, 97.3$\pm$0.9)}  & \multicolumn{2}{l|}{(95.3$\pm$1.3, 97.7$\pm$0.6)} &\multicolumn{2}{l}{\textbf{(96.2$\pm$0.8, 98.9$\pm$0.8 )}} \\
 & \multicolumn{1}{c|}{VisA} & \multicolumn{2}{l}{(76.8$\pm$3.2, 78.1$\pm$2.4)} & \multicolumn{2}{l}{(86.0$\pm$2.6, 87.3$\pm$2.2)} & \multicolumn{2}{l}{(58.9$\pm$4.0, 64.3$\pm$3.2)} & \multicolumn{2}{l}{(82.2$\pm$2.1, 84.8$\pm$2.0)} & \multicolumn{2}{l}{(86.8$\pm$2.0, 88.0$\pm$2.1)}  & \multicolumn{2}{l|}{(88.7$\pm$2.1, 90.4$\pm$2.5)} & \multicolumn{2}{l}{\textbf{(89.9$\pm$1.3, 91.0$\pm$0.8)}}\\
8-shot & \multicolumn{1}{c|}{KSDD} & \multicolumn{2}{l}{(76.9$\pm$3.7, 38.4$\pm$4.5)} & \multicolumn{2}{l}{(92.5$\pm$0.3, 70.8$\pm$0.9)} & \multicolumn{2}{l}{(59.4$\pm$2.9, 24.6$\pm$3.1)} & \multicolumn{2}{l}{(89.9$\pm$0.5, 57.8$\pm$0.1)} & \multicolumn{2}{l}{(94.1$\pm$0.1, 86.5$\pm$0.1)}  & \multicolumn{2}{l|}{ (97.8$\pm$0.6, 92.5$\pm$1.1)} &\multicolumn{2}{l}{\textbf{(98.4$\pm$0.7, 92.7$\pm$0.8)}} \\
 & \multicolumn{1}{c|}{AFID} & \multicolumn{2}{l}{(79.2$\pm$2.5, 55.5$\pm$3.1)} & \multicolumn{2}{l}{(74.5$\pm$0.2, 38.9$\pm$0.3)} & \multicolumn{2}{l}{(60.3$\pm$6.2, 31.4$\pm$3.6)} & \multicolumn{2}{l}{(76.9$\pm$0.8, 51.4$\pm$0.3)} & \multicolumn{2}{l}{(79.6$\pm$1.5, 56.2$\pm$2.1)}  & \multicolumn{2}{l|}{(80.6$\pm$3.6, 56.1$\pm$3.4)} & \multicolumn{2}{l}{\textbf{(84.7$\pm$0.9, 63.3$\pm$1.5)}} \\
 & \multicolumn{1}{c|}{ELPV} & \multicolumn{2}{l}{(72.4$\pm$1.7, 79.8$\pm$1.4)} & \multicolumn{2}{l}{(83.7$\pm$1.6, 91.5$\pm$0.7)} & \multicolumn{2}{l}{(63.3$\pm$2.7, 69.6$\pm$1.5)} & \multicolumn{2}{l}{(81.7$\pm$1.2, 90.5$\pm$0.8)} & \multicolumn{2}{l}{(81.4$\pm$1.0, 89.7$\pm$0.7)} & \multicolumn{2}{l|}{(87.2$\pm$1.3, 92.6$\pm$0.6)} & \multicolumn{2}{l}{\textbf{(90.6$\pm$0.9, 94.1$\pm$0.8)}}  \\ \hline
\end{tabular}
}
\label{main_industrial}
\end{table}

\begin{figure}[!t]
\centering
\includegraphics[width=\columnwidth]{figs/main_fig.png}
\caption{Visualization of representative results for pixel-level anomaly localization of our proposed method on different datasets.}
\label{main_result_fig}
\end{figure}



\begin{table}[htbp]
\caption{A quantitative comparison of our proposed method and other methods using (AUROC ($\%$), AUPRC ($\%$)) as the evaluation metric on \textbf{medical datasets}. We show the average performance and standard deviation across five runs, with the top value highlighted in bold for each comparison.}
\centering
\renewcommand\arraystretch{1.6}
\setlength{\tabcolsep}{3pt}
% \fbox{\rule[-.5cm]{0cm}{4cm} \rule[-.5cm]{4cm}{0cm}}
\resizebox{1\columnwidth}{!}{
\begin{tabular}{cccccccc}
\hline
\multicolumn{8}{c}{Methods}                                                                                                                                                                                                                                                \\ \hline
                        & Datasets & PaDiM~\cite{defard2021padim}                         & Patchcore~\cite{roth2022towards}                    & RegAD~\cite{huang2022registration}                         & WinCLIP~\cite{jeong2023winclip}                      & InCTRL~\cite{zhu2024toward}                                & Ours                           \\ \hline
                        & OCT2017    & -                             & (73.0$\pm$3.6, 86.6$\pm$2.1) & (70.1$\pm$5.2, 84.6$\pm$3.9)          &  (94.7$\pm$2.2, 98.3$\pm$0.7)         & (94.9$\pm$2.6, 98.3$\pm$2.4)                  & \textbf{(96.7$\pm$0.9, 99.0$\pm$1.4)}  \\
\multirow{2}{*}{2-shot} & BrainMRI   & (65.7$\pm$12.2, 90.2$\pm$4.6) & (70.6$\pm$0.9, 92.1$\pm$1.7) & (44.9$\pm$12.9, 87.2$\pm$6.5) & (93.4$\pm$1.2, 98.9$\pm$0.3) & \textbf{(97.3$\pm$2.7, 99.4$\pm$1.3)} & (97.1$\pm$1.4, 99.3$\pm$0.9)           \\
                        & HeadCT     & (59.5$\pm$3.6, 87.6$\pm$1.7)  & (73.6$\pm$9.6, 91.3$\pm$0.2) & (60.2$\pm$1.8, 85.4$\pm$0.9)   & (91.5$\pm$1.5, 97.5$\pm$1.2) & (92.9$\pm$2.5, 98.1$\pm$1.3)          & \textbf{(94.2$\pm$1.1, 98.6$\pm$1.3)}  \\
                        & RESC       & -                             & (69.3$\pm$5.4, 66.2$\pm$3.4)         & (69.2$\pm$3.9, 65.8$\pm$3.6)            & (85.46$\pm$2.1, 79.5$\pm$0.4)        & (88.3$\pm$3.4, 81.48$\pm$2.5)                 & \textbf{(92.2$\pm$0.8, 84.3$\pm$1.2)}  \\ \hline
                        & OCT2017    & -                             & (76.8$\pm$2.8, 88.1$\pm$2.0)         & (72.68$\pm$3.1, 86.14$\pm$3.8)           & (96.2$\pm$2.7, 98.6$\pm$0.5)         & (96.8$\pm$2.4, 98.9$\pm$2.4)                  & \textbf{(99.1$\pm$1.2, 99.5$\pm$1.4 )} \\
\multirow{2}{*}{4-shot} & BrainMRI   & (79.2$\pm$4.8, 95.6$\pm$1.1)  & (79.4$\pm$4.0, 94.5$\pm$1.7) & (57.1$\pm$14.9, 90.0$\pm$4.1)  & (94.1$\pm$0.2, 99.0$\pm$0.1) & (97.5$\pm$1.6, 99.4$\pm$1.3)          & \textbf{(98.2$\pm$1.3, 99.6$\pm$0.5)}  \\
                        & HeadCT     & (62.2$\pm$1.3, 89.0$\pm$1.1)  & (80.5$\pm$0.6, 94.1$\pm$0.9) & (52.2$\pm$5.0, 81.0$\pm$2.8)  & (91.2$\pm$0.3, 97.4$\pm$0.2) & (93.3$\pm$1.3, 98.4$\pm$1.1)          & \textbf{(94.7$\pm$0.6, 99.0$\pm$0.7)}  \\
                        & RESC       & -                             & (69.5$\pm$3.4, 66.8$\pm$2.1)         & (68.5$\pm$2.7, 65.3$\pm$2.5)               & (87.9$\pm$2.1, 80.8$\pm$1.3)         & (88.7$\pm$2.3, 81.1$\pm$2.4)                  & \textbf{(94.6$\pm$1.3, 93.2$\pm$0.9)}  \\ \hline
                        & OCT2017    & -                             & (80.6$\pm$2.1, 90.4$\pm$1.6)         & (74.38$\pm$2.9, 88.6$\pm$4.7)                & (97.0$\pm$2.9, 99.0$\pm$0.5)         & (97.4$\pm$2.1, 99.1$\pm$1.7 )                 & \textbf{(99.3$\pm$1.5, 99.7$\pm$0.8)}  \\
\multirow{2}{*}{8-shot} & BrainMRI   & (75.8$\pm$2.5, 94.6$\pm$0.7)  & (81.2$\pm$1.6, 95.7$\pm$0.7) & (63.2$\pm$7.9, 90.8$\pm$1.3) & (94.4$\pm$0.1, 99.1$\pm$0.0) & (98.3$\pm$1.2, 99.6$\pm$0.3)          & \textbf{(98.6$\pm$0.7, 99.6$\pm$0.5)}  \\
                        & HeadCT     & (66.1$\pm$3.9, 89.6$\pm$0.9)  & (81.7$\pm$3.4, 93.1$\pm$0.6) & (62.8$\pm$2.6, 88.1$\pm$1.4)  & (91.5$\pm$0.8, 97.5$\pm$0.3) & (93.6$\pm$0.8, 98.5$\pm$0.5)          & \textbf{(94.8$\pm$0.6, 99.1$\pm$0.6)}  \\
                        & RESC       & -                             & (71.2$\pm$2.4, 67.9$\pm$2.9 )        & (68.7$\pm$3.2, 65.5$\pm$1.9)            & (88.92$\pm$2.9, 83.1$\pm$1.6)        & (90.6$\pm$2.7, 83.4$\pm$1.8)                  & \textbf{(95.2$\pm$1.4, 87.6$\pm$1.8)}  \\ \hline
\end{tabular}}
\label{main_medical}
\end{table}









\subsection{Results}
\textbf{Comparative Analysis with Industrial Datasets.} 
In Table \ref{main_industrial}, we comprehensively compare five different datasets from the industrial domain, with the MVTec-AD and Visa datasets containing 15 and 12 subclasses, respectively. We also conduct an in-depth analysis across various few-shot scenarios (i.e., 2-shot, 4-shot, 8-shot) to verify the robustness of our method. The experimental results demonstrate that our method outperforms other approaches, showing superior performance, especially in the 4-shot and 8-shot scenarios, where it achieves the best results across all datasets. In the 8-shot setting, our method achieves higher AUCs by 3.4\%, 4.1\%, 0.6\%, 1.2\%, and 0.9\% on the KSDD, ELPV, AFID, KSDD, Visa, and MVTec datasets, respectively, compared to the second-best baseline. Notably, when compared with the latest InCTRL method on the AFID dataset, our method achieves higher AUCs by 2.2\%, 3.6\%, and 4.1\% in the 2-shot, 4-shot, and 8-shot settings, respectively. Additionally, the results indicate that our method is less affected by different runs, with variance significantly lower than other methods, supporting its superior stability and robustness. Finally, as shown in Figure \ref{main_result_fig}, the anomaly map indicates that our method accurately identifies abnormal regions while being less likely to falsely recognize normal regions as anomalous. This further validates the accuracy of our approach. More anomaly maps and localization results are provided in Appendix~\ref{anomaly_map}.

\textbf{Comparative Analysis with Medical Datasets.} In Table \ref{main_medical}, we compare four medical image datasets from three different modalities (CT, MRI, OCT). Our method achieves superior few-shot anomaly detection performance across these diverse modalities, outperforming other methods in most datasets, particularly on the OCT2017 and RESC datasets. Notably, on the RESC dataset, our method surpasses the second-best method by 3.9\%, 5.9\%, and 4.6\% in the 2-shot, 4-shot, and 8-shot settings, respectively. The suboptimal performance of other baseline methods on these datasets may be due to the unique characteristics of the medical domain, where much of the knowledge is not well-explored in VLMs pre-trained mostly on non-medical data. These results demonstrate that our proposed anomaly personalization strategy alleviates these domain-specific challenges more effectively. Additionally, because our method is highly effective in specific domains, it demonstrates greater stability compared to other methods, further proving its robustness.

\textbf{Comparative Analysis with Semantic Datasets.} Here, we compare two semantic datasets using the one-vs-rest strategy, where one class is considered normal, and all other classes are treated as anomalous. The results show that our method achieves the best performance across three different few-shot settings. In the 8-shot scenario, our method reaches AUROC scores of 93.6$(\%)$ and 94.9$(\%)$ on the MNIST and CIFAR-10 datasets, respectively. These experiments demonstrate that our method is also more effective on semantic datasets compared to other baselines.


\textbf{Visualizations of Anomaly Personalization.}
Figure~\ref{vis_personalization} compares the normal images (i.e., reference images), query images, and our resulting personalized images. As depicted, there are some differences between the normal image and the query image in terms of non-anomalous features, such as position, shape, and texture. Our personalized image retains most of the normal regions from the query image, with the anomalous regions largely converted to normal regions. This visualization further substantiates the precision of our method compared to existing few-shot AD approaches.




\begin{table}[!t]
\caption{A quantitative comparison of our proposed method and other methods using (AUROC ($\%$), AUPRC ($\%$)) as the evaluation metric on \textbf{semantic datasets}. We show the average performance and standard deviation across five runs, with the top value highlighted in bold for each comparison.}
\centering
\renewcommand\arraystretch{1.6}
\setlength{\tabcolsep}{5pt}
\resizebox{1\columnwidth}{!}{%
\begin{tabular}{cccccccc}
\cline{1-8}
                        &                               & \multicolumn{6}{c}{Methods}                                                                                                                                                                      \\ \hline
                        & \multicolumn{1}{c|}{Datasets} & Patchcore~\cite{roth2022towards}                    & RegAD~\cite{huang2022registration}                        & CoOp~\cite{zhou2022conditional}                         & WinCLIP~\cite{jeong2023winclip}                      & InCTRL~\cite{zhu2024toward}                       & \textbf{Ours}                         \\ \hline
\multirow{2}{*}{2-shot} & \multicolumn{1}{c|}{MNIST}    & (75.6$\pm$0.4, 95.6$\pm$0.1) & (52.5$\pm$3.0, 91.3$\pm$0.6) & (55.7$\pm$0.6, 92.6$\pm$0.3) & (81.0$\pm$0.8, 96.3$\pm$0.1) & (89.2$\pm$0.9, 97.5$\pm$0.4) & \textbf{(92.6$\pm$0.5, 98.9$\pm$0.4)} \\
                        & \multicolumn{1}{c|}{CIFAR-10} & (60.2$\pm$0.9, 92.6$\pm$0.2)     & (53.4$\pm$0.5, 90.9$\pm$0.3) & (52.7$\pm$1.1, 91.1$\pm$0.2) & (92.5$\pm$0.1, 99.0$\pm$0.1) & (93.5$\pm$0.2, 99.2$\pm$0.0) & \textbf{(94.1$\pm$0.2, 99.3$\pm$0.1)} \\ \hline
\multirow{2}{*}{4-shot} & \multicolumn{1}{c|}{MNIST}    & (83.3$\pm$0.9, 97.2$\pm$0.2) & (54.8$\pm$5.3, 91.6$\pm$1.3) & (56.3$\pm$0.4, 92.9$\pm$0.2) & (85.1$\pm$1.0, 97.1$\pm$0.2) & (90.2$\pm$1.6, 98.0$\pm$0.7) & \textbf{(92.9$\pm$0.7, 99.0$\pm$0.6)} \\
                        & \multicolumn{1}{c|}{CIFAR-10} & (63.9$\pm$1.0, 93.4$\pm$0.3) & (53.4$\pm$0.2, 90.8$\pm$0.1) & (53.7$\pm$0.5, 91.5$\pm$0.3) & (92.7$\pm$0.1, 99.0$\pm$0.0) & (94.0$\pm$1.0, 99.2$\pm$0.4) & \textbf{(94.6$\pm$0.2, 99.3$\pm$0.2)} \\ \hline
\multirow{2}{*}{8-shot} & \multicolumn{1}{c|}{MNIST}    & (87.6$\pm$0.4, 97.9$\pm$0.1) & (54.7$\pm$6.3, 91.9$\pm$1.8) & (56.7$\pm$0.7, 93.7$\pm$0.4) & (86.7$\pm$0.7, 97.4$\pm$0.1) & (92.0$\pm$0.3, 98.9$\pm$0.1) & \textbf{(93.6$\pm$0.2, 99.5$\pm$0.3)} \\
                        & \multicolumn{1}{c|}{CIFAR-10} & (67.2$\pm$0.6, 94.2$\pm$0.2) & (55.5$\pm$0.8, 91.1$\pm$0.1) & (54.2$\pm$0.5, 92.0$\pm$0.3) & (92.8$\pm$0.1, 99.0$\pm$0.0) & (94.5$\pm$0.2, 99.4$\pm$0.1) & \textbf{(94.9$\pm$0.1, 99.5$\pm$0.1)} \\ \hline
\end{tabular}}
\label{main_semantic}
\end{table}

\begin{figure}[!t]
\centering
\includegraphics[width=1\columnwidth]{figs/normal_personlization.png}
\caption{Visualizations of anomaly personalization. The red box of the query image indicates anomalous regions.}
\label{vis_personalization}
\end{figure}






\subsection{Ablation Study}

\begin{figure}[!t]
\centering
\includegraphics[width=1\columnwidth]{figs/improved_other_new.png}
\caption{The effectiveness of our generated anomaly-free samples for other AD methods. The red numbers highlight performance increases.}
\label{improve_AD}
\end{figure}

\textbf{The effectiveness of generated anomaly-free samples.}
To validate the effectiveness of our method in generating normal samples, we apply it to multiple AD methods, including Patchcore, RegAD, WinCLIP and InCTRL. As shown in the first row of Figure~\ref{improve_AD}, our method demonstrates improvements in the 2-shot, 4-shot, and 8-shot settings across all three methods. Notably, for the RegAD method, the AUC is improved by 6.8(\%), 8.7(\%), and 10.2(\%), respectively. Additionally, we compare our generated strategy with five other AD datasets. After incorporating our anomaly-free samples, the four AD methods show varying degrees of improvement on industrial and medical datasets (Visa, KSDD, AFID, OCT2017, and headCT). In particular, the Patchcore and RegAD methods exhibit significant improvements on the OCT2017 dataset, with increases of 10.9(\%) and 8.6(\%), respectively. While InCTRL has already achieved high performance, our method still provides a certain level of enhancement. This experiment demonstrates that the data generated by our method can enhance existing AD methods, proving the flexibility and adaptability of our approach.

\textbf{The effectiveness of the triplet contrastive anomaly inference.} Here, we discuss the impact of each branch in our triplet contrastive anomaly inference strategy, analyzing datasets across three different domains. Table~\ref{ablation_triplet} validates the effectiveness of our strategy, showing that it is optimal in most datasets. When only text prompts are used ($S_{text}$), meaning there is no reference image input, particularly low results are observed in the medical datasets OCT2017 and RESC. However, when reference images are supplied (either $S_P$ or $S_N$), a significant improvement is observed. This might be because the model lacks an inherent understanding of specific medical domains, thus requiring normal images as references. %Our personalized contrastive approach improves accuracy more effectively on such specific domain data compared to other methods. 
When each strategy is used individually, we find that most results are not as good as when the strategies are combined, indicating that these strategies complement each other. %When using scores involving only $S_{P}$ and $S_{text}$, most results already reach a satisfactory level; however, the results for CIFAR-10 and BrainMRI are less optimal. 
When all three strategies are employed together, most datasets demonstrate optimal performance. However, in certain cases, such as the KSDD dataset—which focuses on surface defect inspection—utilizing all scores does not consistently yield the best results. This inconsistency may stem from the high diversity in the distribution of normal images, which presents challenges for the AD model's learning process. Nevertheless, across the majority of datasets, the combined use of three strategies leads to relatively strong performance, showcasing that our triplet contrastive anomaly inference strategy effectively enhances accuracy.


\textbf{The effectiveness of text prompts in anomaly personalization.} 
We also conduct ablation experiments on the settings of the text prompts $c_{q}$ used in the one-to-normal personalization stage (Section~\ref{one-to-normal}). In this work, we employed three prompts to generate three images (one for each prompt). The results in Table~\ref{ablation_text} demonstrate that our designed text prompt strategy is effective for one-to-normal personalization, where most datasets show improvement. 

\textbf{Inference time.} Regarding inference time, our proposed method is slightly higher (+200-300ms per query image) than that of WinCLIP (389ms) and InCTRL (276ms). If necessary, we can further increase the inference speed by reducing the number of generated samples or decreasing the memory bank size. When using a single prompt corresponds to generating only one personalized image, the required inference time (326ms) is slightly lower than that of WinCLIP, while still demonstrating superior performance compared to other methods.

\begin{table}[!t]
\caption{Ablation study on the impact of three different contrastive anomaly inference branches in 8-shot setting. We show the average performance (AUROC ($\%$)) across five runs, with the top value highlighted in bold.}
\centering
\renewcommand\arraystretch{1.2}
\setlength{\tabcolsep}{7pt}
\resizebox{1\columnwidth}{!}{%
\begin{tabular}{cccccccccccccc}
\hline
\multicolumn{3}{c|}{Strategies}                                                                     & \multicolumn{11}{c}{Datasets}                                                                                                                                                                                                              \\ \hline
\multirow{2}{*}{\textit{$S_{P}$}} & \multirow{2}{*}{\textit{$S_{N}$}} & \multicolumn{1}{c|}{\multirow{2}{*}{\textit{$S_{text}$}}} & \multicolumn{5}{c|}{Industrial field}                                                         & \multicolumn{4}{c|}{Medical field}                                                 & \multicolumn{2}{c}{Semantic fields} \\ \cline{4-14} 
                            &                             & \multicolumn{1}{c|}{}                            & MVTec-AD & VisA & KSDD & AFID & \multicolumn{1}{c|}{ELPV} & OCT2017 & BrainMRI & HeadCT & \multicolumn{1}{c|}{RESC} & MINIST      & CIFAR-10     \\ \hline
                            &                             & \multicolumn{1}{c|}{\checkmark}                         & 91.8              & 78.2          & 94.3          & 72.8          & \multicolumn{1}{c|}{73.1}          & 45.3             & 92.4              & 89.6            & \multicolumn{1}{c|}{39.6}          & 69.2                 & 91.3                  \\
                            & \checkmark                         & \multicolumn{1}{c|}{}                            & 93.1              & 86.3          & 93.1          & 80.5          & \multicolumn{1}{c|}{83.3}          & 87.1             & 89.6              & 86.0            & \multicolumn{1}{c|}{89.6}          & 91.6        & 93.6                  \\
\checkmark                         &                             & \multicolumn{1}{c|}{}                            & 93.7              & 88.6          & 96.3          & \textbf{85.2}          & \multicolumn{1}{c|}{86.1}          & 96.3             & 94.3              & 90.3            & \multicolumn{1}{c|}{92.6}          & \textbf{93.9}        & 84.3                  \\
                            & \checkmark                         & \multicolumn{1}{c|}{\checkmark}                         & 95.8              & 88.6          & 95.3          & 80.3          & \multicolumn{1}{c|}{87.1}          & 97.6             & \textbf{98.7}              & 92.2            & \multicolumn{1}{c|}{89.2}          & 90.6                 & 93.8                  \\
\checkmark                         &                             & \multicolumn{1}{c|}{\checkmark}                         & 96.0              & 89.6          & \textbf{98.6} & 84.3          & \multicolumn{1}{c|}{88.3}          & 99.2             & 94.6              & 92.6            & \multicolumn{1}{c|}{94.2}          & 92.6                 & 88.2                  \\
\checkmark                         & \checkmark                         & \multicolumn{1}{c|}{\checkmark}                         & \textbf{96.2}     & \textbf{89.9} & 98.4          & 84.7 & \multicolumn{1}{c|}{\textbf{90.6}} & \textbf{99.3}    & 98.6     & \textbf{94.8}   & \multicolumn{1}{c|}{\textbf{95.2}} & 93.6                 & \textbf{94.9}         \\ \hline
                            &                             &                                                  &                   &               &               &               &                                    &                  &                   &                 &                                    & \textbf{}            &                      
\end{tabular}}
\label{ablation_triplet}
\end{table}

\begin{table}[!t]
\caption{Ablation study on the text prompts used in one-to-normal personalization on all datasets in 8-shot setting. We show the average performance (AUROC ($\%$)) across five runs, with the top value highlighted in bold.}
\centering
\renewcommand\arraystretch{1.2}
\setlength{\tabcolsep}{7pt}
\resizebox{1\columnwidth}{!}{%
\begin{tabular}{c|ccccccccccc}
\cline{1-12}
                                               & \multicolumn{11}{c}{Datasets}                                                                                                                             \\ \hline
\multicolumn{1}{c|}{\multirow{2}{*}{Strategy}} & \multicolumn{5}{c|}{Industrial domain}                     & \multicolumn{4}{c|}{Medical domain}                      & \multicolumn{2}{c}{Semantic domain} \\ \cline{2-12} 
\multicolumn{1}{c|}{}                          & MVTec-AD & VisA & KSDD & AFID & \multicolumn{1}{c|}{ELPV} & OCT2017 & BrainMRI & HeadCT & \multicolumn{1}{c|}{RESC} & MINIST          & CIFAR-10          \\ \hline
\multicolumn{1}{c|}{w/o text}                   & 95.6     & 88.6 & 98.0 & 84.6 & \multicolumn{1}{c|}{89.8} & 98.7    & 97.6     & 93.9   & \multicolumn{1}{c|}{94.6} & 93.5            & 93.7              \\
\multicolumn{1}{c|}{refined text}               & \textbf{96.2}     & \textbf{89.9} & \textbf{98.4} & \textbf{84.7} & \multicolumn{1}{c|}{\textbf{90.6}} & \textbf{99.3}    & \textbf{98.6}     & \textbf{94.8}   & \multicolumn{1}{c|}{\textbf{95.2}} & \textbf{93.6}            & \textbf{94.9}              \\ \hline
\end{tabular}}
\label{ablation_text}
\end{table}



\section{Conclusion}
We introduce a novel personalized few-shot anomaly detection method that enhances prediction accuracy through one-to-normal personalization of query images. Unlike other state-of-the-art approaches that directly compare query images with reference images, our method enables a finer-grained comparison. Our triplet contrastive anomaly inference strategy further stabilizes results by incorporating personalized images, anomaly-free samples and text prompts, facilitating a more comprehensive comparison. Extensive experiments across 11 datasets in industrial, medical, and semantic domains demonstrate the method’s generalizability and effectiveness. Moreover, the anomaly-free samples generated by our method can augment the normal samples in existing few-shot anomaly detection techniques. Experimental results also demonstrate that our method improves the performance of several existing few-shot anomaly detection techniques.

\textbf{Limitation}
Achieving precise detection necessitates a comprehensive exploration of the distribution of each category, which in turn requires a few reference images from each category. Consequently, our current method is not applicable to zero-shot scenarios. Future research will focus on enhancing the capability to identify anomalies in zero-shot settings more accurately, as well as exploring real-time applications and open-vocabulary scenarios.

\textbf{Broader Impacts}
This article focuses on a customized few-shot anomaly detection method, which offers more precise and stable anomaly detection. It has the potential to enhance development in industrial and medical fields. There are no negative societal impacts involved in this work.

\section*{Acknowledgement}
This work was supported in part by the National Key Research and Development Program of China under Grant 2020YFB1711500 and Grant 2020YFB1711503; in part by the 1.3.5 project for disciplines of excellence, West China Hospital, Sichuan University, under Grant ZYYC21004 and ZYAI24053; in part by the Aier Eye Hospital-Sichuan University Research Grant 23JZH043.

% This study was supported by National Key Research and Development Program of China (2020YFB1711500, 2020YFB1711503), the 1·3·5 project for disciplines of excellence, West China Hospital, Sichuan University( ZYYC21004, ZYJC18010)

% This work was supported by the National Natural Science Foundation of China (No. 62276052) and Sichuan Science and Technology Program (No. 2024YFHZ0268), the 1-3-5 project for disciplines of excellence, West China Hospital, Sichuan University (ZYYC21004), the China Post Doctoral Science Foundation (Grant No. 2023M732435) and the Science and Technology Department of Hainan Province (Grant No.ZDYF2024SHFZ052).

% This work was supported by the National Key Research and Development Program of China under Grant 2020YFB1711500, the 1 ⋅3 ⋅5 project for disciplines of excellence, West China Hospital, Sichuan University (ZYYC21004), and the Natural Science Foundation of Sichuan Province under Grant NO. 2023NSFSC1722.
\medskip


{
\small
\bibliographystyle{plain} % 或其他合适的样式。
\bibliography{egbib} 

}

\newpage
%%%%%%%%%%%%%%%%%%%%%%%%%%%%%%%%%%%%%%%%%%%%%%%%%%%%%%%%%%%%

\appendix

\section{Appendix / supplemental material}

\subsection{Text Prompt Setting }
\label{prompt}
The text prompts are provided in Figure \ref{fig_comp_prompt}
\begin{figure}[h]
\centering
\begin{minipage}[t]{0.31\linewidth}
\centering
(a) \emph{State}-level (-:normal)
{\small
\begin{itemize}
    \item[-] c := "[o] without flaw"
    \item[-] c := "[o] without defect"
    \item[-] c := "[o] without damage"
\end{itemize}
}
\end{minipage}
\hspace{0.02\linewidth} % 调整这里的间距
\begin{minipage}[t]{0.31\linewidth}
\centering
(b) \emph{Physical}-level
{\small
\begin{itemize}
\item "a photo of a/the small [c]."
\item "a photo of a/the large [c]."
\item "a bright photo of a/the [c]."
\item "a dark photo of a/the [c]."
\item "a blurry photo of a/the [c]."
\end{itemize}
}
\end{minipage}
\begin{minipage}[t]{0.31\linewidth}
\vspace{0.3cm}
\centering
{\small
\begin{itemize}
\item "a bad photo of a/the [c]."
\item "a good photo of a/the [c]."
\item "a cropped photo of a/the [c]."
\item "a close-up photo of a/the [c]."
\item "a low resolution photo of a/the [c]."
\end{itemize}
}
\end{minipage}
\caption{Lists of state and template level prompts employed in this paper to construct text features.}
\label{fig_comp_prompt}
\vspace{-10pt}
\end{figure}

Our experimental approach initially focused on the number of prompts, investigating scenarios with 1, 3, 5, and 10 prompts. We found that performance was at its lowest with a single prompt, whereas using all ten prompts resulted in the highest performance.

\subsection{Abnormal Localization Results}
\label{anomaly_map}
% Visa
\begin{figure}[h]
    \centering
    \includegraphics[width=1\linewidth]{figs/appendix/chewinggum.png}
    \caption{The anomaly map localization results of our proposed method for subclass chewinggum. The first row shows the original images with anomalies, the second row displays the ground truth of the anomalies, and the third row shows the localization results obtained by our method.}
    \label{fig:exp1}
\end{figure}

\begin{figure}[h]
    \centering
    \includegraphics[width=1\linewidth]{figs/appendix/fryum.png}
    \caption{The anomaly map localization results of our proposed method for subclass fryum. The first row shows the original images with anomalies, the second row displays the ground truth of the anomalies, and the third row shows the localization results obtained by our method.}
    \label{fig:exp2}
\end{figure}

\begin{figure}
    \centering
    \includegraphics[width=1\linewidth]{figs/appendix/candle.png}
    \caption{The anomaly map localization results of our proposed method for subclass candle. The first row shows the original images with anomalies, the second row displays the ground truth of the anomalies, and the third row shows the localization results obtained by our method.}
    \label{fig:exp3}
\end{figure}

\begin{figure}
    \centering
    \includegraphics[width=1\linewidth]{figs/appendix/pcb2.png}
    \caption{The anomaly map localization results of our proposed method for subclass pcb2. The first row shows the original images with anomalies, the second row displays the ground truth of the anomalies, and the third row shows the localization results obtained by our method.}
    \label{fig:exp5}
\end{figure}



% MVtec
\begin{figure}
    \centering
    \includegraphics[width=1\linewidth]{figs/appendix/grid.png}
    \caption{The anomaly map localization results of our proposed method for subclass grid. The first row shows the original images with anomalies, the second row displays the ground truth of the anomalies, and the third row shows the localization results obtained by our method.}
    \label{fig:exp6}
\end{figure}

\begin{figure}
    \centering
    \includegraphics[width=1\linewidth]{figs/appendix/hazelnut.png}
    \caption{The anomaly map localization results of our proposed method for subclass hazelnut. The first row shows the original images with anomalies, the second row displays the ground truth of the anomalies, and the third row shows the localization results obtained by our method.}
    \label{fig:exp7}
\end{figure}

\begin{figure}
    \centering
    \includegraphics[width=1\linewidth]{figs/appendix/metal_nut.png}
    \caption{The anomaly map localization results of our proposed method for subclass metal\_nut. The first row shows the original images with anomalies, the second row displays the ground truth of the anomalies, and the third row shows the localization results obtained by our method.}
    \label{fig:exp8}
\end{figure}

\begin{figure}
    \centering
    \includegraphics[width=1\linewidth]{figs/appendix/tile.png}
    \caption{The anomaly map localization results of our proposed method for subclass tile. The first row shows the original images with anomalies, the second row displays the ground truth of the anomalies, and the third row shows the localization results obtained by our method.}
    \label{fig:exp9}
\end{figure}


% Med
\begin{figure}
    \centering
    \includegraphics[width=1\linewidth]{figs/appendix/BrainCT.png}
    \caption{The anomaly map localization results of our proposed method for subclass BrainCT. The first row shows the original images with anomalies, the second row displays the ground truth of the anomalies, and the third row shows the localization results obtained by our method.
    }
    \label{fig:exp10}
\end{figure}

\begin{figure}
    \centering
    \includegraphics[width=1\linewidth]{figs/appendix/head_ct.png}
    \caption{The anomaly map localization results of our proposed method for subclass HeadCT. The first row shows the original images with anomalies, the second row displays the ground truth of the anomalies, and the third row shows the localization results obtained by our method.}
    \label{fig:exp11}
\end{figure}

% OCT
\begin{figure}
    \centering
    \includegraphics[width=1\linewidth]{figs/appendix/OCT.png}
    \caption{The anomaly map localization results of our proposed method for subclass OCT. The first row shows the original images with anomalies, the second row displays the ground truth of the anomalies, and the third row shows the localization results obtained by our method.}
    \label{fig:exp}
\end{figure}





%%%%%%%%%%%%%%%%%%%%%%%%%%%%%%%%%%%%%%%%%%%%%%%%%%%%%%%%%%%%

% \newpage
% \clearpage
% \section*{NeurIPS Paper Checklist}




% \begin{enumerate}

% \item {\bf Claims}
%     \item[] Question: Do the main claims made in the abstract and introduction accurately reflect the paper's contributions and scope?
%     \item[] Answer: \answerYes{} % Replace by \answerYes{}, \answerNo{}, or \answerNA{}.
%     \item[] Justification: Yes, the main claims presented in our abstract and introduction accurately reflect the contribution and scope of our paper
%     \item[] Guidelines:
%     \begin{itemize}
%         \item The answer NA means that the abstract and introduction do not include the claims made in the paper.
%         \item The abstract and/or introduction should clearly state the claims made, including the contributions made in the paper and important assumptions and limitations. A No or NA answer to this question will not be perceived well by the reviewers. 
%         \item The claims made should match theoretical and experimental results, and reflect how much the results can be expected to generalize to other settings. 
%         \item It is fine to include aspirational goals as motivation as long as it is clear that these goals are not attained by the paper. 
%     \end{itemize}

% \item {\bf Limitations}
%     \item[] Question: Does the paper discuss the limitations of the work performed by the authors?
%     \item[] Answer: \answerYes{} % Replace by \answerYes{}, \answerNo{}, or \answerNA{}.
%     \item[] Justification:Yes, our paper discusses the limitations of the work in the discussion section.
%     \item[] Guidelines:
%     \begin{itemize}
%         \item The answer NA means that the paper has no limitation while the answer No means that the paper has limitations, but those are not discussed in the paper. 
%         \item The authors are encouraged to create a separate "Limitations" section in their paper.
%         \item The paper should point out any strong assumptions and how robust the results are to violations of these assumptions (e.g., independence assumptions, noiseless settings, model well-specification, asymptotic approximations only holding locally). The authors should reflect on how these assumptions might be violated in practice and what the implications would be.
%         \item The authors should reflect on the scope of the claims made, e.g., if the approach was only tested on a few datasets or with a few runs. In general, empirical results often depend on implicit assumptions, which should be articulated.
%         \item The authors should reflect on the factors that influence the performance of the approach. For example, a facial recognition algorithm may perform poorly when image resolution is low or images are taken in low lighting. Or a speech-to-text system might not be used reliably to provide closed captions for online lectures because it fails to handle technical jargon.
%         \item The authors should discuss the computational efficiency of the proposed algorithms and how they scale with dataset size.
%         \item If applicable, the authors should discuss possible limitations of their approach to address problems of privacy and fairness.
%         \item While the authors might fear that complete honesty about limitations might be used by reviewers as grounds for rejection, a worse outcome might be that reviewers discover limitations that aren't acknowledged in the paper. The authors should use their best judgment and recognize that individual actions in favor of transparency play an important role in developing norms that preserve the integrity of the community. Reviewers will be specifically instructed to not penalize honesty concerning limitations.
%     \end{itemize}

% \item {\bf Theory Assumptions and Proofs}
%     \item[] Question: For each theoretical result, does the paper provide the full set of assumptions and a complete (and correct) proof?
%     \item[] Answer: \answerYes{} % Replace by \answerYes{}, \answerNo{}, or \answerNA{}.
%     \item[] Justification: For each theoretical result, we have presented the assumptions and provided comprehensive experiments.
%     \item[] Guidelines:
%     \begin{itemize}
%         \item The answer NA means that the paper does not include theoretical results. 
%         \item All the theorems, formulas, and proofs in the paper should be numbered and cross-referenced.
%         \item All assumptions should be clearly stated or referenced in the statement of any theorems.
%         \item The proofs can either appear in the main paper or the supplemental material, but if they appear in the supplemental material, the authors are encouraged to provide a short proof sketch to provide intuition. 
%         \item Inversely, any informal proof provided in the core of the paper should be complemented by formal proofs provided in appendix or supplemental material.
%         \item Theorems and Lemmas that the proof relies upon should be properly referenced. 
%     \end{itemize}

%     \item {\bf Experimental Result Reproducibility}
%     \item[] Question: Does the paper fully disclose all the information needed to reproduce the main experimental results of the paper to the extent that it affects the main claims and/or conclusions of the paper (regardless of whether the code and data are provided or not)?
%     \item[] Answer: \answerYes{} % Replace by \answerYes{}, \answerNo{}, or \answerNA{}.
%     \item[] Justification: We fully disclose all the information.
%     \item[] Guidelines:
%     \begin{itemize}
%         \item The answer NA means that the paper does not include experiments.
%         \item If the paper includes experiments, a No answer to this question will not be perceived well by the reviewers: Making the paper reproducible is important, regardless of whether the code and data are provided or not.
%         \item If the contribution is a dataset and/or model, the authors should describe the steps taken to make their results reproducible or verifiable. 
%         \item Depending on the contribution, reproducibility can be accomplished in various ways. For example, if the contribution is a novel architecture, describing the architecture fully might suffice, or if the contribution is a specific model and empirical evaluation, it may be necessary to either make it possible for others to replicate the model with the same dataset, or provide access to the model. In general. releasing code and data is often one good way to accomplish this, but reproducibility can also be provided via detailed instructions for how to replicate the results, access to a hosted model (e.g., in the case of a large language model), releasing of a model checkpoint, or other means that are appropriate to the research performed.
%         \item While NeurIPS does not require releasing code, the conference does require all submissions to provide some reasonable avenue for reproducibility, which may depend on the nature of the contribution. For example
%         \begin{enumerate}
%             \item If the contribution is primarily a new algorithm, the paper should make it clear how to reproduce that algorithm.
%             \item If the contribution is primarily a new model architecture, the paper should describe the architecture clearly and fully.
%             \item If the contribution is a new model (e.g., a large language model), then there should either be a way to access this model for reproducing the results or a way to reproduce the model (e.g., with an open-source dataset or instructions for how to construct the dataset).
%             \item We recognize that reproducibility may be tricky in some cases, in which case authors are welcome to describe the particular way they provide for reproducibility. In the case of closed-source models, it may be that access to the model is limited in some way (e.g., to registered users), but it should be possible for other researchers to have some path to reproducing or verifying the results.
%         \end{enumerate}
%     \end{itemize}


% \item {\bf Open access to data and code}
%     \item[] Question: Does the paper provide open access to the data and code, with sufficient instructions to faithfully reproduce the main experimental results, as described in supplemental material?
%     \item[] Answer: \answerYes{} % Replace by \answerYes{}, \answerNo{}, or \answerNA{}.
%     \item[] Justification: Yes, the data in our paper is publicly available, and we will also make the code public in the future.

%     \item[] Guidelines:
%     \begin{itemize}
%         \item The answer NA means that paper does not include experiments requiring code.
%         \item Please see the NeurIPS code and data submission guidelines (\url{https://nips.cc/public/guides/CodeSubmissionPolicy}) for more details.
%         \item While we encourage the release of code and data, we understand that this might not be possible, so “No” is an acceptable answer. Papers cannot be rejected simply for not including code, unless this is central to the contribution (e.g., for a new open-source benchmark).
%         \item The instructions should contain the exact command and environment needed to run to reproduce the results. See the NeurIPS code and data submission guidelines (\url{https://nips.cc/public/guides/CodeSubmissionPolicy}) for more details.
%         \item The authors should provide instructions on data access and preparation, including how to access the raw data, preprocessed data, intermediate data, and generated data, etc.
%         \item The authors should provide scripts to reproduce all experimental results for the new proposed method and baselines. If only a subset of experiments are reproducible, they should state which ones are omitted from the script and why.
%         \item At submission time, to preserve anonymity, the authors should release anonymized versions (if applicable).
%         \item Providing as much information as possible in supplemental material (appended to the paper) is recommended, but including URLs to data and code is permitted.
%     \end{itemize}


% \item {\bf Experimental Setting/Details}
%     \item[] Question: Does the paper specify all the training and test details (e.g., data splits, hyperparameters, how they were chosen, type of optimizer, etc.) necessary to understand the results?
%     \item[] Answer: \answerYes{} % Replace by \answerYes{}, \answerNo{}, or \answerNA{}.
%     \item[] Justification: Yes, our paper provides fairly comprehensive training and test details in the experimental setup section.
%     \item[] Guidelines:
%     \begin{itemize}
%         \item The answer NA means that the paper does not include experiments.
%         \item The experimental setting should be presented in the core of the paper to a level of detail that is necessary to appreciate the results and make sense of them.
%         \item The full details can be provided either with the code, in appendix, or as supplemental material.
%     \end{itemize}

% \item {\bf Experiment Statistical Significance}
%     \item[] Question: Does the paper report error bars suitably and correctly defined or other appropriate information about the statistical significance of the experiments?
%     \item[] Answer: \answerYes{} % Replace by \answerYes{}, \answerNo{}, or \answerNA{}.
%     \item[] Justification: es, our main experiments all provide standard deviation information.
%     \item[] Guidelines:
%     \begin{itemize}
%         \item The answer NA means that the paper does not include experiments.
%         \item The authors should answer "Yes" if the results are accompanied by error bars, confidence intervals, or statistical significance tests, at least for the experiments that support the main claims of the paper.
%         \item The factors of variability that the error bars are capturing should be clearly stated (for example, train/test split, initialization, random drawing of some parameter, or overall run with given experimental conditions).
%         \item The method for calculating the error bars should be explained (closed form formula, call to a library function, bootstrap, etc.)
%         \item The assumptions made should be given (e.g., Normally distributed errors).
%         \item It should be clear whether the error bar is the standard deviation or the standard error of the mean.
%         \item It is OK to report 1-sigma error bars, but one should state it. The authors should preferably report a 2-sigma error bar than state that they have a 96\% CI, if the hypothesis of Normality of errors is not verified.
%         \item For asymmetric distributions, the authors should be careful not to show in tables or figures symmetric error bars that would yield results that are out of range (e.g. negative error rates).
%         \item If error bars are reported in tables or plots, The authors should explain in the text how they were calculated and reference the corresponding figures or tables in the text.
%     \end{itemize}

% \item {\bf Experiments Compute Resources}
%     \item[] Question: For each experiment, does the paper provide sufficient information on the computer resources (type of compute workers, memory, time of execution) needed to reproduce the experiments?
%     \item[] Answer: \answerYes{} % Replace by \answerYes{}, \answerNo{}, or \answerNA{}.
%     \item[] Justification: we provide sufficient information.
%     \item[] Guidelines:
%     \begin{itemize}
%         \item The answer NA means that the paper does not include experiments.
%         \item The paper should indicate the type of compute workers CPU or GPU, internal cluster, or cloud provider, including relevant memory and storage.
%         \item The paper should provide the amount of compute required for each of the individual experimental runs as well as estimate the total compute. 
%         \item The paper should disclose whether the full research project required more compute than the experiments reported in the paper (e.g., preliminary or failed experiments that didn't make it into the paper). 
%     \end{itemize}
    
% \item {\bf Code Of Ethics}
%     \item[] Question: Does the research conducted in the paper conform, in every respect, with the NeurIPS Code of Ethics \url{https://neurips.cc/public/EthicsGuidelines}?
%     \item[] Answer: \answerYes{} % Replace by \answerYes{}, \answerNo{}, or \answerNA{}.
%     \item[] Justification: Yes, the research conducted in the paper conforms, in every respect, with the NeurIPS Code of Ethics.

%     \item[] Guidelines:
%     \begin{itemize}
%         \item The answer NA means that the authors have not reviewed the NeurIPS Code of Ethics.
%         \item If the authors answer No, they should explain the special circumstances that require a deviation from the Code of Ethics.
%         \item The authors should make sure to preserve anonymity (e.g., if there is a special consideration due to laws or regulations in their jurisdiction).
%     \end{itemize}


% \item {\bf Broader Impacts}
%     \item[] Question: Does the paper discuss both potential positive societal impacts and negative societal impacts of the work performed?
%     \item[] Answer: \answerYes{} % Replace by \answerYes{}, \answerNo{}, or \answerNA{}.
%     \item[] Justification: Yes, we discuss both potential positive societal impacts and negative societal impacts of the work.
%     \item[] Guidelines:
%     \begin{itemize}
%         \item The answer NA means that there is no societal impact of the work performed.
%         \item If the authors answer NA or No, they should explain why their work has no societal impact or why the paper does not address societal impact.
%         \item Examples of negative societal impacts include potential malicious or unintended uses (e.g., disinformation, generating fake profiles, surveillance), fairness considerations (e.g., deployment of technologies that could make decisions that unfairly impact specific groups), privacy considerations, and security considerations.
%         \item The conference expects that many papers will be foundational research and not tied to particular applications, let alone deployments. However, if there is a direct path to any negative applications, the authors should point it out. For example, it is legitimate to point out that an improvement in the quality of generative models could be used to generate deepfakes for disinformation. On the other hand, it is not needed to point out that a generic algorithm for optimizing neural networks could enable people to train models that generate Deepfakes faster.
%         \item The authors should consider possible harms that could arise when the technology is being used as intended and functioning correctly, harms that could arise when the technology is being used as intended but gives incorrect results, and harms following from (intentional or unintentional) misuse of the technology.
%         \item If there are negative societal impacts, the authors could also discuss possible mitigation strategies (e.g., gated release of models, providing defenses in addition to attacks, mechanisms for monitoring misuse, mechanisms to monitor how a system learns from feedback over time, improving the efficiency and accessibility of ML).
%     \end{itemize}
    
% \item {\bf Safeguards}
%     \item[] Question: Does the paper describe safeguards that have been put in place for responsible release of data or models that have a high risk for misuse (e.g., pretrained language models, image generators, or scraped datasets)?
%     \item[] Answer: \answerNA{} % Replace by \answerYes{}, \answerNo{}, or \answerNA{}.
%     \item[] Justification: Our experiments do not involve this aspect.
%     \item[] Guidelines:
%     \begin{itemize}
%         \item The answer NA means that the paper poses no such risks.
%         \item Released models that have a high risk for misuse or dual-use should be released with necessary safeguards to allow for controlled use of the model, for example by requiring that users adhere to usage guidelines or restrictions to access the model or implementing safety filters. 
%         \item Datasets that have been scraped from the Internet could pose safety risks. The authors should describe how they avoided releasing unsafe images.
%         \item We recognize that providing effective safeguards is challenging, and many papers do not require this, but we encourage authors to take this into account and make a best faith effort.
%     \end{itemize}

% \item {\bf Licenses for existing assets}
%     \item[] Question: Are the creators or original owners of assets (e.g., code, data, models), used in the paper, properly credited and are the license and terms of use explicitly mentioned and properly respected?
%     \item[] Answer: \answerYes % Replace by \answerYes{}, \answerNo{}, or \answerNA{}.
%     \item[] Justification: Yes, the assets used in the paper (e.g., code, data, models) are all properly credited.
%     \item[] Guidelines:
%     \begin{itemize}
%         \item The answer NA means that the paper does not use existing assets.
%         \item The authors should cite the original paper that produced the code package or dataset.
%         \item The authors should state which version of the asset is used and, if possible, include a URL.
%         \item The name of the license (e.g., CC-BY 4.0) should be included for each asset.
%         \item For scraped data from a particular source (e.g., website), the copyright and terms of service of that source should be provided.
%         \item If assets are released, the license, copyright information, and terms of use in the package should be provided. For popular datasets, \url{paperswithcode.com/datasets} has curated licenses for some datasets. Their licensing guide can help determine the license of a dataset.
%         \item For existing datasets that are re-packaged, both the original license and the license of the derived asset (if it has changed) should be provided.
%         \item If this information is not available online, the authors are encouraged to reach out to the asset's creators.
%     \end{itemize}

% \item {\bf New Assets}
%     \item[] Question: Are new assets introduced in the paper well documented and is the documentation provided alongside the assets?
%     \item[] Answer: \answerNA{} % Replace by \answerYes{}, \answerNo{}, or \answerNA{}.
%     \item[] Justification: Our experiments do not involve this aspect.
%     \item[] Guidelines:
%     \begin{itemize}
%         \item The answer NA means that the paper does not release new assets.
%         \item Researchers should communicate the details of the dataset/code/model as part of their submissions via structured templates. This includes details about training, license, limitations, etc. 
%         \item The paper should discuss whether and how consent was obtained from people whose asset is used.
%         \item At submission time, remember to anonymize your assets (if applicable). You can either create an anonymized URL or include an anonymized zip file.
%     \end{itemize}

% \item {\bf Crowdsourcing and Research with Human Subjects}
%     \item[] Question: For crowdsourcing experiments and research with human subjects, does the paper include the full text of instructions given to participants and screenshots, if applicable, as well as details about compensation (if any)? 
%     \item[] Answer: \answerNA{} % Replace by \answerYes{}, \answerNo{}, or \answerNA{}.
%     \item[] Justification: Our experiments do not involve this aspect.
%     \item[] Guidelines:
%     \begin{itemize}
%         \item The answer NA means that the paper does not involve crowdsourcing nor research with human subjects.
%         \item Including this information in the supplemental material is fine, but if the main contribution of the paper involves human subjects, then as much detail as possible should be included in the main paper. 
%         \item According to the NeurIPS Code of Ethics, workers involved in data collection, curation, or other labor should be paid at least the minimum wage in the country of the data collector. 
%     \end{itemize}

% \item {\bf Institutional Review Board (IRB) Approvals or Equivalent for Research with Human Subjects}
%     \item[] Question: Does the paper describe potential risks incurred by study participants, whether such risks were disclosed to the subjects, and whether Institutional Review Board (IRB) approvals (or an equivalent approval/review based on the requirements of your country or institution) were obtained?
%     \item[] Answer: \answerNA{} % Replace by \answerYes{}, \answerNo{}, or \answerNA{}.
%     \item[] Justification: Our experiments do not involve this aspect.
%     \item[] Guidelines:
%     \begin{itemize}
%         \item The answer NA means that the paper does not involve crowdsourcing nor research with human subjects.
%         \item Depending on the country in which research is conducted, IRB approval (or equivalent) may be required for any human subjects research. If you obtained IRB approval, you should clearly state this in the paper. 
%         \item We recognize that the procedures for this may vary significantly between institutions and locations, and we expect authors to adhere to the NeurIPS Code of Ethics and the guidelines for their institution. 
%         \item For initial submissions, do not include any information that would break anonymity (if applicable), such as the institution conducting the review.
%     \end{itemize}

% \end{enumerate}


\end{document}