\section{Related work}
\label{sec:related_work}
\subsection{Anomaly Detection in Time Series}
Anomaly detection in time series data has attracted significant attention in recent years due to its diverse applications in domains such as economics, manufacturing, and healthcare ____. Existing approaches can be broadly classified into traditional machine learning-based methods ____ and deep learning-based methods ____. Deep learning-based methods have demonstrated significant advantages over traditional approaches, achieving superior performance in a variety of real-world time series anomaly detection tasks ____. These methods leverage the ability of neural networks to model complex temporal dependencies and capture non-linear patterns inherent in time series data. In this work, we focus on deep learning-based methods.

%However, their adoption in resource-constrained scenarios remains limited due to the higher resource requirements associated with training and inference \ya{reference?}. 

%The first is based on traditional machine learning techniques, such as regression models \ya{reference?}, support vector machines \ya{reference?}, and clustering algorithms (____, ____), statistical approaches (____), and methods that convert time series into other data forms (____).  The second is based deep learning, which typically use prediction, reconstruction, and generation for anomaly detection \ya{reference?}. 


%Deep learning-based methods typically use prediction, reconstruction, and generation for anomaly detection. Many studies apply RNNs and their variants (____, ____, ____). With the introduction of attention mechanisms, several new applications have emerged in time series anomaly detection (____, ____, ____). However, due to the inherent properties of neural networks, there is still room for improvement in training time efficiency. While self-supervised learning shows promise for handling scarce anomaly data (____, ____, ____), its performance remains affected by imbalanced data distributions.

\subsection{ECG Anomaly Detection}
The diversity and rarity of cardiac diseases, coupled with the high cost of collecting diverse ECG abnormalities, present significant challenges to conventional multi-label classification methods. In contrast, anomaly detection methods, which rely exclusively on normal data for training, offer the potential to identify previously unseen anomalies and reduce the risk of missing rare cardiac conditions. However, ECG anomaly detection remains particularly challenging due to substantial inter-individual and inter-sample variability, as well as the intricate nature of anomalies, which can manifest as both global rhythm disturbances and localized morphological irregularities ____. To address these issues, generative adversarial network (GAN)-based methods have been explored, such as BeatGAN ____, which demonstrates strong capabilities in capturing local morphological features. Similarly, approaches like ____ and ____ leverage GANs to process ECG signals. For jointly modeling local and global ECG patterns,  ____ proposed a multi-scale framework that achieved state-of-the-art results on the PTB-XL detection and localization benchmark ____. More recently, ____ proposed a model that integrates both time-series and time-frequency representations of ECG signals while significantly reducing trainable parameters compared to previous methods. Despite these advancements, most approaches rely on R-peak detection or heartbeat segmentation, which introduces additional complexity and renders them highly sensitive to noise and irregularities, limiting their practicality in real-world clinical settings. To overcome these limitations, our proposed method eliminates the dependence on R-peak detection and heartbeat segmentation. Furthermore, it is lightweight, requiring fewer trainable parameters than current state-of-the-art algorithms, offering significant advantages in efficiency and inference speed.

\subsection{Masked Autoencoders}
In recent years, the focus of deep learning research has shifted from developing increasingly complex models to addressing the challenges of data scarcity ____. Masked Autoencoders ____ have emerged as a powerful self-supervised representation learning framework, demonstrating remarkable success in various visual tasks and gaining significant attention. Recently, efforts have been made to adapt MAE for ECG classification ____. Among these, ____ proposed an MAE-based multi-label ECG classification approach, achieving notable performance improvements. However, the application of MAE to anomaly detection remains limited. For instance,  ____ observed that MAE may underperform in unsupervised anomaly detection tasks for images. While the self-attention mechanism inherent in MAE enables it to capture global patterns effectively ____, anomaly detection tasks, particularly for ECG signals, require not only modeling global features but also detecting subtle and localized anomalies critical for accurate diagnosis. To address these challenges, we propose a novel multi-scale MAE-based framework specifically designed for ECG anomaly detection.