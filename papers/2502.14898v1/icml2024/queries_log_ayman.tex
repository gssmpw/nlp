1. What is a mammogram?

A mammogram is an X-ray picture of the breast, used by doctors to detect early signs of breast cancer by examining breast tissue for abnormalities, often performed as a screening tool for women with no symptoms of the disease; it is considered the best way to find breast cancer early when it is most treatable. [1, 2, 3, 4, 5]  
Key points about mammograms: [1, 3, 5]  

• Function: To look for early signs of breast cancer, including tumors and cysts, that might not be detectable by touch. [1, 3, 5]  
• Procedure: A machine uses low-dose radiation to take multiple X-ray images of each breast while compressing them between plates. [1, 2, 6, 7]  
• Interpretation: A radiologist, a doctor specializing in interpreting medical images, analyzes the mammogram for any suspicious areas. [1, 5, 8]  
• Types: [1, 2, 9]  
	• Screening mammogram: Performed on women with no symptoms to check for potential cancer. [1, 2, 9]  
	• Diagnostic mammogram: Performed on women with symptoms like a lump or pain to investigate a potential problem. [5, 9]  




Generative AI is experimental.

[1] https://www.cdc.gov/breast-cancer/about/mammograms.html
[2] https://www.cancer.gov/types/breast/mammograms-fact-sheet
[3] https://www.cdc.gov/breast-cancer/screening/index.html
[4] https://www.nibib.nih.gov/science-education/science-topics/mammography
[5] https://www.hopkinsmedicine.org/health/treatment-tests-and-therapies/mammogram-procedure
[6] https://www.youtube.com/watch?v=HftRzpbsDgk
[7] https://www.cancer.org/cancer/types/breast-cancer/screening-tests-and-early-detection/mammograms/mammogram-basics.html
[8] https://www.cancercenter.com/cancer-types/breast-cancer/diagnosis-and-detection/mammography
[9] https://medlineplus.gov/mammography.html



2. risks of a mammogram

While mammograms are the best breast cancer screening tool available, they do have some risks, including: [1, 2]  

Radiation exposure 

Mammograms expose patients to a small amount of radiation, but the benefits of regular mammograms usually outweigh the risks. The amount of radiation exposure from a mammogram is about 20 times less than what you would receive from being in your basement for a year. [1, 2, 3]  

False results 

Mammograms can produce false positive or false negative results: [4, 5]  • False positive: A mammogram may indicate cancer when there is none. This can lead to additional tests, which can be expensive, invasive, and time-consuming. [4]  
• False negative: A mammogram may indicate that there is no cancer when there is, which can delay treatment and lead to a false sense of security. False negatives are more likely to occur in women with dense breasts. [4, 5]  



Overdiagnosis 

A mammogram may find cancer that would not have caused symptoms or problems, or may even go away on its own. This can lead to unnecessary treatment, which can cause side effects. [4]  

Missed cancers 

Some cancers may be too small or located in an area that is difficult to view by mammography. [2]  

Additional testing 

If something unexpected is detected on a mammogram, additional tests may be needed. [2]  



Generative AI is experimental.

[1] https://www.cdc.gov/radiation-health/data-research/facts-stats/mammograms.html
[2] https://www.mayoclinic.org/tests-procedures/mammogram/about/pac-20384806
[3] https://www.youtube.com/watch?v=tP25SoDjLJ8
[4] https://www.cdc.gov/breast-cancer/screening/index.html
[5] https://www.cancer.org/cancer/types/breast-cancer/screening-tests-and-early-detection/mammograms/limitations-of-mammograms.html


3. mammogram risks

Mammograms have some risks, including: [1, 2, 3, 4, 5, 6, 7, 8]  

• Radiation exposure: Mammograms expose patients to a small amount of ionizing radiation. While the risk is low, repeated mammograms may increase the possibility of developing cancer later in life. [2, 3, 6]  
• False positives: A mammogram may detect something that looks like cancer but isn't, leading to more tests, anxiety, and invasive procedures. [1, 2, 6]  
• Overdiagnosis: A mammogram may detect a cancer that would not have caused symptoms or problems, leading to unnecessary treatment. [1, 2, 5, 6]  
• False negatives: A mammogram may miss some cancers, which can delay treatment. [2, 6]  
• Additional testing: A mammogram may lead to additional tests, such as ultrasounds or biopsies. [2]  
• Functional decline: In older women, surgery or other treatments for breast cancer can lead to functional decline. [4]  
• Discouragement from screening: A higher rate of false positives in women ages 40–49 could discourage screening at older ages. [8]  

The benefits of regular mammograms usually outweigh the risks. The United States Preventive Services Task Force recommends that women who are 50–74 years old and at average risk for breast cancer get mammograms every two years. Women who are 40–49 years old should talk to their healthcare provider about when or how often to get a mammogram. [2, 3]  

Generative AI is experimental.

[1] https://pmc.ncbi.nlm.nih.gov/articles/PMC8796062/
[2] https://www.mayoclinic.org/tests-procedures/mammogram/about/pac-20384806
[3] https://www.cdc.gov/radiation-health/data-research/facts-stats/mammograms.html
[4] https://www.cancer.gov/news-events/cancer-currents-blog/2023/mammography-older-women-breast-cancer-overdiagnosis
[5] https://www.komen.org/breast-cancer/screening/mammography/benefits-risks/
[6] https://www.cdc.gov/breast-cancer/screening/index.html
[7] https://www.yalemedicine.org/news/new-mammogram-screening-guidelines-what-women-should-know
[8] https://www.cancer.gov/news-events/cancer-currents-blog/2024/mammogram-false-positives-affect-future-screening

4. breast biopsy risks

A breast biopsy has some risks, including: [1, 2]  

• Bruising and swelling: These are common side effects that occur at the biopsy site. [1, 2]  
• Bleeding: Bleeding at the biopsy site can be prolonged. [3]  
• Infection: An infection can occur near the biopsy site and can be treated with antibiotics. [3, 4]  
• Altered breast appearance: The breast may look different depending on how much tissue is removed and how it heals. [1]  
• Additional surgery or treatment: Depending on the biopsy results, additional surgery or other treatment may be needed. [1]  
• Rupture of a breast implant: If you have breast implants, there is a risk of rupture. [5]  

Side effects are rare with core-needle biopsies, with less than 2 in 100 women experiencing a serious side effect. Core-needle biopsies have a lower risk of side effects than open surgical biopsies. [6]  
After a breast biopsy, you can expect to have some discomfort and bruising. Your provider will give you instructions on how to care for the biopsy area and manage any discomfort. You should keep the dressing dry and avoid showering over the area for the first 48 hours. [4, 7]  

Generative AI is experimental.

[1] https://www.mayoclinic.org/tests-procedures/breast-biopsy/about/pac-20384812
[2] https://my.clevelandclinic.org/health/diagnostics/24204-breast-biopsy-overview
[3] https://www.hopkinsmedicine.org/health/treatment-tests-and-therapies/breast-biopsy
[4] https://medlineplus.gov/lab-tests/breast-biopsy/
[5] https://drjayanam.com/breast-biopsy/
[6] https://effectivehealthcare.ahrq.gov/products/breast-biopsy-update/consumer
[7] https://www.imperial.nhs.uk/-/media/website/patient-information-leaflets/cancer-services/breast-services/after-your-breast-core-biopsy.pdf?rev=524e3628b22c4baba9a8f3330fb0a0ab

Not all images can be exported from Search.

5. risks of a mastectomy

A mastectomy can have several risks, including: [1]  

Infection 

The surgical site may become infected, which can be serious. Signs of infection include redness, swelling, pain, or drainage from the incision. Antibiotics can usually clear up an infection. [1, 2]  

Swelling 

The arm may swell, especially if lymph nodes were removed. This is called lymphedema and can cause pain and tenderness in the arm and hand. [3, 4]  

Scar tissue 

Hard scar tissue can form at the incision site. [3]  

Pain 

Pain or tenderness may occur at the surgical site, or you may experience phantom breast pain. [3, 5]  

Numbness 

You may experience numbness in the chest or under your arm. [6]  

Nerve damage 

Nerves that go to the muscles of the arm, back, and chest wall may be damaged. [7]  

Skin changes 

The skin along the incision or within the skin flaps may scab, blister, open, or lose skin. [7]  

Changes in body image 

You may experience changes in the way you feel about your body after surgery. [6]  


The type of mastectomy you have can affect which side effects you experience, as more complex surgeries tend to have more side effects. [5]  

Generative AI is experimental.

[1] https://www.youtube.com/watch?v=7PRO3h16NE8
[2] https://www.cancerresearchuk.org/about-cancer/breast-cancer/treatment/surgery/after-surgery/problems-after-mastectomy
[3] https://www.hopkinsmedicine.org/health/conditions-and-diseases/breast-cancer/mastectomy
[4] https://www.nhsinform.scot/tests-and-treatments/surgical-procedures/mastectomy/
[5] https://www.cancer.org/cancer/types/breast-cancer/treatment/surgery-for-breast-cancer/mastectomy.html
[6] https://www.mayoclinic.org/tests-procedures/mastectomy/about/pac-20394670
[7] https://www.mountsinai.org/health-library/surgery/mastectomy

6. lymphedema complications

Lymphedema can lead to a number of complications, including: [1, 2]  

Skin infections 

The trapped fluid that causes lymphedema can create an environment where germs can thrive, leading to infections like cellulitis. Cellulitis can be life-threatening if left untreated. Symptoms include redness, swelling, pain, tenderness, and warmth in the affected area. [1, 2, 3]  

Lymphangitis 

An inflammation of the lymph vessels that can spread to the skin and cause cellulitis. [2]  

Sepsis 

A life-threatening condition that can occur when untreated cellulitis spreads into the bloodstream. [1]  

Skin changes 

The skin in the affected area can thicken and harden, making it difficult to move. [4]  

Lymphorrhea 

Lymph fluid can leak out through the skin, causing the area to become damp and more prone to infection. [4]  

Psychological impact 

Lymphedema can affect your appearance, which can lead to depression and other psychological issues. Talking to a doctor or support group can help. [2, 5]  

Cancer 

In rare cases, severe, untreated lymphedema can lead to a soft tissue cancer. [1]  


Lymphedema can occur anywhere in the body, but it's most common in the arms and legs. Symptoms include swelling, heaviness, or fullness in the affected area, as well as skin changes like discoloration, dryness, or thickening. [6]  

Generative AI is experimental.

[1] https://www.mayoclinic.org/diseases-conditions/lymphedema/symptoms-causes/syc-20374682
[2] https://www.medicalnewstoday.com/articles/180919
[3] https://www.cancer.gov/about-cancer/treatment/side-effects/lymphedema
[4] https://www.webmd.com/a-to-z-guides/what-is-lymphedema
[5] https://www.nhs.uk/conditions/lymphoedema/
[6] https://www.cancer.org/cancer/managing-cancer/side-effects/swelling/lymphedema.html

Not all images can be exported from Search.

7. mesh complications

Complications from surgical mesh can occur early or years after the mesh is placed, and can be serious. Some complications include: [1, 2, 3]  

• Pain: Pain may persist even after the mesh is removed. [3]  
• Infection: Surgical mesh can trigger an infection. [4]  
• Mesh migration: A rare but serious complication that can occur after hernia repair. [2]  
• Mesh contraction: The mesh may shrink or migrate. [5]  
• Urinary tract infections (UTIs): Recurring UTIs can be a complication of mesh. [6]  
• Fistulas: An abnormal connection between organs, vessels, or intestines. [5]  
• Bleeding: Bleeding can be a complication of surgical mesh. [5]  
• Organ perforation: A hole in neighboring tissues or organs. [5]  
• Bowel complications: Bowel complications can be a complication of mesh. [1]  
• Immune disorders: Immune disorders can be a complication of mesh. [1]  
• Vaginal scarring: Vaginal scarring can be a complication of mesh. [1]  
• Neuromuscular alterations: Neuromuscular alterations can be a complication of mesh. [1]  
• Lower urinary tract symptoms (LUTS): LUTS can be a complication of mesh. [1]  

Other complications include: Postoperative hematoma and seroma, Foreign body reaction, and Mesh rejection. [2]  

Generative AI is experimental.

[1] https://pmc.ncbi.nlm.nih.gov/articles/PMC7378717/
[2] https://pmc.ncbi.nlm.nih.gov/articles/PMC9659837/
[3] https://my.clevelandclinic.org/health/articles/16298-surgical-mesh-use-and-complications-in-women
[4] https://www.sciencedirect.com/science/article/pii/S1198743X14621881
[5] https://www.fda.gov/medical-devices/implants-and-prosthetics/surgical-mesh-used-hernia-repair
[6] https://umiamihealth.org/treatments-and-services/urology/mesh-complications

8. complications of organ perforation

Organ perforation, or a hole in the gastrointestinal (GI) tract, can lead to a number of complications, including: [1]  

• Internal bleeding: A perforation can cause significant blood loss. [1]  
• Peritonitis: Inflammation of the abdominal wall lining. [1]  
• Sepsis: A life-threatening bacterial infection that can occur when an infection spreads throughout the body. [1, 2, 3]  
• Bowel damage: A perforation can cause permanent damage to the GI tract, including part of the bowel dying. [1, 4]  
• Abdominal abscesses: An infection can occur inside the abdomen. [2, 3]  
• Wound infection: A wound may fail to heal after surgery or become infected. [4]  
• Permanent ileostomy or colostomy: In some cases, a permanent ileostomy or colostomy may be required. [2]  

Other complications include pulmonary complications, thromboembolic events, cardiovascular events, and the need for prolonged ventilator support. [5]  
Symptoms of organ perforation include: severe abdominal pain, vomiting, fever, nausea, and chills. [6]  
Organ perforation requires immediate medical care. [1]  

Generative AI is experimental.

[1] https://my.clevelandclinic.org/health/diseases/23478-gastrointestinal-perforation
[2] https://www.healthline.com/health/gastrointestinal-perforation
[3] https://medlineplus.gov/ency/article/000235.htm
[4] https://www.medicalnewstoday.com/articles/322008
[5] https://www.ncbi.nlm.nih.gov/books/NBK538191/
[6] https://www.webmd.com/digestive-disorders/what-is-gastrointestinal-perforation



