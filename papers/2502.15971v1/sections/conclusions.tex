In the present work, we introduced a model-based contact-aware planner for autonomous catheterization interventions for acute ischemic stroke treatment. The method was developed for telescoping pre-bent tools currently used for neuroendovascular interventions, which are unable of tip steering and advance by interacting with the walls of the vessels. We present an actuation system able of actuating insertion and rotation of such telescoping tools and experimental validation of successful navigation from the base of the descending aorta to the \gls{lcca}.

For this first task, necessary to further access other \glspl{lvo}, we only required planning and actuation of the innermost tool: the guidewire. We demonstrate that the planner can successfully avoid the \gls{lsa}, which it would naturally enter without appropriate path planning, and can navigate to the \gls{lcca} 100\% of times over 50 trials. The planner was also proven robust towards rotations of the aorta of up to $10^\circ$, and displacement of 10\,mm on the coronal plane, which may arise in the clinical settings.

Future work will focus on further navigation to the large vessels within the cerebral cavity, by actuating pre-bent catheters in combination with guidewires. Real-time fluoroscopic images will be used for model-based closed-loop control, using the model presented in this paper.