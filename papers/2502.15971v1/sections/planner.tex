From pre-operative imaging, obtained from common imaging modalities such as \gls{cta} or \gls{mra}, and knowledge of the properties of the catheterization tools, we plan the contact-aware motion inside the anatomy. In this work, we assume a 3D segmentation of the pre-operative images is provided, commonly from the aorta up (Fig. \ref{fig:platform}) \cite{vanderZijden2019CurrentClinician}.

The control variables for our neuroendovascular tools are the number of segments we insert for the $j$-th tool, $M_j$, and its axial rotation $\theta_j$, defined in the previous section.

For a certain configuration $V(t) = \left\{M_j(t), \theta_j(t), j = 1, 2, \dots N\right\}$, we can solve the optimization problem in (\ref{eq:solution}) and obtain a configuration for the tools $\pmb \gamma(t)$ which guarantees: (i) the tools act telescopically; (ii) the tools are within the anatomy or in contact with its boundaries. By solving the direct kinematics in (\ref{eq:kinematics}), we obtain the position of the tip of the innermost tool $\mb p_{1_{M_1}}(l, t) \triangleq \mb p(t)$. 

From an initial known configuration $V(0)$, we aim to find the series of configurations $V(1), V(2), \dots, V(L), \ L \geq 0$ such that $\| \mb p(L) - \mb q\| < \epsilon$, with $\mb q \in \mathbb{R}^3$ target and a small enough $\epsilon$. The target is defined by the surgeon from a 3D pre-operative map of the anatomy.

To find the optimal series of configurations, we developed a \emph{contact-aware} \gls{rrt} algorithm to build the graph $G$, as described in Algorithm \ref{alg:rrt}.

\begin{algorithm}
\caption{Contact-aware RRT}\label{alg:rrt}
$t \gets 0$ \\
$V(t) \gets$ initial configuration \\
$G \gets$ INIT($V(t)$) \\
$\mb p(t) \gets$ DIR\_KIN($V(t)$)\\
\While{$1 < t \leq L$}{
  \If{$\| \mb p(t) - \mb q\| > \epsilon$}{
    $V_{rand} \gets$ RAND\_CONF() \\
    $V_{near} \gets$ NEAREST\_VERTEX($V_{rand}, G$) \\
    $V_{new} \gets$ STEER($V_{near}$, $V_{rand}, \Delta V$) \\
    $G.$ ADD\_VERTEX($V_{new}$) \\
    $G.$ ADD\_EDGE($V_{new}$, $V_{near}$) \\
    $\mb p(t) \gets $ DIR\_KIN($V(t)$)\\
  }
}
\KwResult{$G$}
\end{algorithm}

Once the graph $G$ is built, we search for the shortest path from $V(0)$ to $V(L)$.

The function INIT($V(t)$) adds the vertex $V(t)$ to the graph and initializes the graph with no edges. DIR\_KIN($V(t)$) finds the solution to the problem (\ref{eq:solution}) and computes the direct kinematics to find the tip of the innermost tool $\mb p(t)$, via $(\ref{eq:kinematics})$. We generate a random configuration within the maximum elongation of each tools with the function RAND\_CONF() and find the nearest vertex in the graph as
\begin{equation}
    \text{NEAREST\_VERTEX}(V_{rand}, G) := \argmin_{V \in G} \|V - V_{rand} \|.
\end{equation}

Given the limits in the control effort $\Delta V = \left(\Delta M_1 \ \Delta \theta_1 \ \Delta M_2 \ \Delta \theta_2 \ \cdots \ \Delta M_N \ \Delta \theta_N \right) =\left(\Delta V_1 \ \Delta V_2 \ \dots \Delta V_{N} \right)  \in \mathbb{R}^{2 N}$ for each of the control variables, we steer between nodes as
\begin{eqnarray}
    && \text{STEER}(V_{near}, V_{rand}, \Delta V) := \\
    && V_{{near}_i} + \text{sign}(V_{{rand}_i} - V_{{near}_i})\Delta V_{i}, \ i = 1, 2, \dots, N \nonumber
\end{eqnarray}

The functions $G.$ADD\_VERTEX($V_{new}$) and $G.$ADD\_EDGE($V_{new}$, $V_{near}$) add $V_{new}$ vertex to the graph $G$ and edge connecting $V_{near}$ to $V_{new}$, respectively.