\begin{figure}
    \centering
    \includegraphics[width=1\columnwidth]{figures/model_anatomy.png}
    \caption{Schematic description of model parameters.}
    \label{fig:model}
\end{figure}

In Fig. \ref{fig:model}, we describe the telescoping tools for neuroendovascular catheterization. Generally, navigation is performed using combinations of a pre-bent guidewire and a pre-bent catheter whose insertion and rotation can be controlled. In the following, we describe how we model these tools and the interaction with the anatomy. The overall model is used to plan the autonomous navigation within the anatomy, as described in Section \ref{sec:planning}.

We model each tool's static equilibrium independently, as described in Section \ref{sec:model:single}, and treat the interaction between tools (Section \ref{sec:model:tools}) and with the anatomy (Section \ref{sec:model:anatomy}) as constraints to the static equilibrium of each tool. The solution to the constrained statics is described in Section \ref{sec:model:solution}.

In this work, we assume that frictional effects between tools and between tools and anatomy can be neglected. We also assume that the mechanics of the tools is dominated by their elastic behavior and that the anatomy is rigid when compared to the tools.

\subsection{Single Tool Model}
\label{sec:model:single}
Following \cite{Pittiglio2023ClosedRobots}, we describe each tool using infinitesimal segments each assumed to shape as an arc of a circle. The overall centerline of each tool, under this condition, follows the piece-wise constant curvature assumption. We describe the centerline of each segment as a function of their length $s_{j_i} \in [0, l]$, where $l \in \mathbb{R}$ is the length of the $i$-th segment, here assumed the same among segments and tools; the index $j$ indicates the tools number, $j = 1$ is the innermost. In this work, we consider two tools, as shown in Fig. \ref{fig:model}.

We associate a reference frame to each point along the tools' centerline $F_{j_i}(s_{j_i})=\left\{O_{j_i}(s_{j_i}), \mb x_{j_i}(s_{j_i}), \mb y_{j_i}(s_{j_i}), \mb z_{j_i}(s_{j_i})\right\}$, so that the $\mb{z}_{j_i}(s_{j_i})$ axis is tangent to the curve and define $F_{j_i} \equiv F_{j_i}(l)$; $F_{1_0}, \ F_{2_0}, \dots F_{N_0}$ are the ground frames of each of the $N$ tools, assumed to have a coincident origin $O \equiv O_{j_1}(0) \ \forall j$. We define a world reference frame $W = \left\{O, \mb x_0, \mb y_0, \mb z_0\right\}$ so that $p_{j_0} = 0$ is the distance between the frame $F_{j_0}$ and $W$ and $R_{j_0} = \text{rot}_{\mb z_{j_0}}(\theta_j)$ their relative orientation. The axial rotation of the $j$-th tool (see Fig. \ref{fig:model}), $\text{rot}_{\mb z_{j_0}}(\theta_j)$, of the angle $\theta_j$ around the axis $\mb z_{j_0}$ can be controlled as described in Section \ref{sec:planning}. 

The bending of each segment is defined by the local bending vector $\pmb{\gamma}_{j_i} \in \mathbb{R}^3$, with respect to the local reference frame $F_{j_i} \equiv F_{j_i}(l) = \left\{O_{j_i}(l), \mb x_{j_i}(l), \mb y_{j_i}(l), \mb z_{j_i}(l)\right\}$; $\pmb{\gamma}_{j_i}$ is described as the deflection along the 3 principal directions. Here $j = 1, 2, \dots, N$ with $N$ number of tools and $M_j$ is the number of segments for each tools, $i = 1, 2, \dots, M_j$.

In this work, the elongation of one tool with respect to the others is described as increase in the number of segments in each tool. $M_{j - 1} - M_{j}$ describes the amount of extension of the tool $j - 1$ (innermost) with respect to $j$ (outermost).

The position and orientation of each frame $F_{j_i}(s_{j_i})$ along the centerline of the $j$-th tool can be written with respect to the world frame $W$ \cite{Pittiglio2023ClosedRobots}, as
\begin{subequations}
\label{eq:kinematics}
\begin{equation}
\label{eq:kinematics:position}
\mb{p}_{j_i}(s_{j_i})  = \mb{p}_{j_{i-1}}(l) + \mb{R}_{j_{i-1}}(l) \: \text{exp} \left(\left[\pmb{\gamma}_{j_i}\frac{s_{j_i}}{l}\right]^\wedge \right)  \boldsymbol{e}_3
\end{equation}
\begin{equation}
\mb{R}_{j_i}(s_{j_i}) = \mb{R}_{j_{i-1}}(l) \: \text{exp} \left(\left[\pmb{\gamma}_{j_i}\frac{s_{j_i}}{l}\right]^\wedge \right)
\end{equation}
\end{subequations}
for the respective position and rotation.

To describe the static equilibrium of each tool we write the Lagrangian
\begin{equation}
    L_j = T_j - U_j = -U_j,
\end{equation}
where we assumed the kinetic term $T_j$ is negligible, i.e. $T_j = 0$. The potential energy can be written as 
\begin{eqnarray}
\label{eq:potential}
U_j & = & \sum_{i = 1}^{M_j} \frac{1}{2} \int_{0}^{l} -\pmb \gamma_{j_i}\T \mb k \pmb \gamma_{j_i} - 2 m_i \mb g\T \mb p_{j_i}(s_{j_i}) \de s_{j_i} \nonumber \\
& = & -\frac{1}{2} \bs \gamma_j\T \text{diag}(\underbrace{\mb k l, \ \mb k l, \ \dots, \ \mb k l}_{M_j \ \text{times}}) \bs \gamma_j - \nonumber \\
&& \sum_{i = 1}^{M_j} m_i \mb g\T \int_{0}^{l}\mb p_{j_i}(s_{j_i}) \de s_{j_i} \nonumber ,
\end{eqnarray}
 with $m_i$ mass density of the $i$-th tool and $\mb g$ gravitational acceleration.

We consider the mechanical stiffness of the tool
\begin{equation}
\mb k = \text{diag} \left(1, \ 1, \ \frac{1}{2(\nu + 1)} \right)\frac{E A}{l}
\end{equation}
with $E$ \emph{Young's modulus}, $A$ second moment of area and $\nu$ \emph{Poisson's ratio}. The first two elements of $\mb k$ describe the bending stiffness, while the last one accounts for the torsional stiffness, since $\mb{z}$-axis is defined to be tangent to the backbone curve.

Since our tools are completely passive, we have no external forces, apart from the ones arising from their interactions and their contact with the anatomy. These are going to be treated as constraints, as discussed in the following sections. The equilibrium of each tool, treated as independent flexible bodies, can be found by solving the minimization problem
\begin{equation}
\label{eq:single_solution}
    \min_{\pmb \gamma} U_j, \ i = 1, 2, \dots, M_j
\end{equation}
with $\pmb{\gamma} = \left(\pmb \gamma_{1_1} \ \cdots \pmb \gamma_{1_{M_1}} \cdots \pmb \gamma_{N_1} \cdots \pmb \gamma_{N_{M_N}} \right)\T$

\subsection{Telescoping Tools Constraints}
\label{sec:model:tools}
From the statics of each independent tool, described above, we can find their shape depending on their mechanical stiffness and under gravitational load. To consider their interaction, we constrain their centerline to be coincident for the overlapping segments. For the $j$-th tool, we can find number of segments overlapping with the $k$-th tool as $L_{j, k} = \|M_j - M_k\|$. Since we modeled the tools having a common ground frame $W$, tools $j$ and $k$ overlap for the first $L_{j, k}$ segment. The segments overlap if they satisfy the constraints
\begin{equation}
\label{eq:tele_constraints}
    \mb C^{(i)}_{j, k} = \mb p_{j_i}(l) - \mb p_{k_i}(l) = \mb{0} \in \mathbb{R}^3, \ i = 0, 1, \dots, L_{j, k}
\end{equation}
We can build the overall constraint vector $\mb C$, by stacking together all the constraints between tools and removing repetitions arising from multiple overlapping elements.

\subsection{Anatomical Constraints}
\label{sec:model:anatomy}
From preoperative imaging, we can segment the anatomy and convert its walls into a closed triangulated surface, as depicted in Fig. \ref{fig:model}. The $k$-th triangle has a center point $\mb s_k \in \mathbb{R}^3$ and a normal $\mb h_k \in \mathbb{R}^3$ associated with it. The normals to each of the $D$ triangles are chosen to point inwards, with respect to the anatomy. To find whether a point along the centerline of any of the tools $\mb p_{j_i}(l)$ is inside the anatomy, we find the nearest triangle 
\begin{equation}
    k_{j,i} = \argmin_{k} \|\mb p_{j_i}(l) - \mb s_k\|, \ k = 1, 2, \dots, D.
\end{equation}

and compute the projection of vector between the triangle and point along the centerline
\begin{equation}
    d_{j_i} = (\mb p_{j_i}(l) - \mb s_{k_{j,i}}) \cdot \mb h_{k_{j,i}}.
\end{equation}
When the projection is positive the point is inside the anatomy, by definition. We can stack together the the projected distances $d_{j_i}$ in a vector of constraints 
\begin{equation}
\label{eq:anat_constraints}
    \mb S = \left(d_{1_1} \ \cdots d_{1_{M_1}} \ \dots d_{N_{M_N}}  \right) \T .
\end{equation}

\subsection{Static Equilibrium Solution}
\label{sec:model:solution}
To find the overall configuration of the telescoping tools, we combine (\ref{eq:single_solution}), (\ref{eq:tele_constraints}) and (\ref{eq:anat_constraints}). We write the constrained static equilibrium as
\begin{eqnarray}
\label{eq:solution}
        & \displaystyle \min_{\pmb \gamma} & \mb U = \left(U_1 \ U_2 \cdots U_N \right)\T \\
         & \text{subject to } & \mb C = 0 \nonumber \\
        & & \mb S \geq 0 \nonumber
\end{eqnarray}
