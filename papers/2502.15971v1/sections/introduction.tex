Stroke is the second leading cause of death worldwide, accounting for 11.6\% of all deaths in 2019 \cite{Rai2023UpdatedCare}. Every 3 minutes and 14 seconds, someone dies of stroke \cite{2023StrokeFacts.} and 50\% of survivors become chronically disabled \cite{Donkor2018StrokeLife.} with physical, cognitive, speech, and other impairments. 87\% of all strokes occur when a blood clot obstructs the blood flow – or ischemia. \glspl{lvo} are ischemic events due to blockage of large vessels of the brain and affect 295,000 Americans yearly \cite{2023StrokeFacts.}. While Intravenous (IV) tissue Plasminogen Activator (tPA) – clot-busting medication - can generally restore blood flow in smaller vessels, mechanical thrombectomy is required for \glspl{lvo}. Mechanical thrombectomy after IV tPA restores blood flow in up to 80\% of \gls{lvo} cases \cite{LeeH.Schwamm2017EndovascularOutcomes}, while sole IV tPA recanalization results may be as low as 12\% \cite{LeeH.Schwamm2017EndovascularOutcomes}. While IV tPA administration does not require specialized training, thrombectomy is a minimally invasive procedure performed by trained and experienced neurointerventional units. 

\begin{figure}
    \centering
    \includegraphics[width=\columnwidth]{figures/platform.png}
    \caption{Description of robotic platform for autonomous navigation of neuroendovascular catheterization tools.}
    \label{fig:platform}
\end{figure}

A study from 2004 indicates that, in the USA, 82\% of patients live more than 3 hours away from centers capable of performing \gls{lvo} treatments \cite{Peterman2022GeospatialStates} and more recent studies show that rural-urban discrepancies are aggravating \cite{Hammond2020Urban-RuralMortality}. Time to treatment is critical: each hour of delay before mechanical thrombectomy results in a greater degree of disability \cite{Powers2019GuidelinesAssociation}. Time delays are also considered amongst the reasons for which we observe 20\% higher stroke mortality in patients from rural areas, compared to urban centers \cite{Llanos-Leyton2022DisparitiesStroke}. 

When a patient with stroke symptoms is admitted to an emergency unit, the nature of the stroke is investigated using \gls{ct} and/or \gls{mri} scans. If signs of ischemia are found, the patient is transferred to specialized medical centers. They perform a 3D scan, generally \gls{mra} or \gls{cta}, to visualize the arterial network from the aortic arch up to the brain vessels \cite{vanderZijden2019CurrentClinician}, and plan the navigation of telescoping flexible tools from the femoral (leg) or radial artery (arm) to the target location (Fig. \ref{fig:platform}). Once the site of flow blockage is reached, stent retrieval and/or aspiration are used to remove the blood clot.

Highly skilled interventional radiologists select combinations of telescoping guidewires and catheters (Fig. \ref{fig:model}) and navigate to the clot using information obtained from pre-operative \gls{cta} or \gls{mra}, and real-time fluoroscopy. They train in predicting how tools-anatomy interaction affects steering in the arteries, as they read images and insert and rotate guidewire and catheter. They integrate knowledge of landmarks in the anatomy, information gathered through \gls{cta} or \gls{mra} and partial information from fluoroscopy. This requires several cognitive skills, which are developed via extensive learning and require continuous practice so that they are not lost. However, only select large medical centers provide required training of their residents and have a workload which allows constant practice of their senior clinicians \cite{Hammond2020Urban-RuralMortality}.

Our goal is to develop a semi-autonomous robotic solution (Fig. \ref{fig:platform}) which, using pre-operative \gls{cta} or \gls{mra} images, can navigate pre-bent telescopic tools to the target location. We envision its use in small and rural centers to reduce time to treatment and reperfusion. Clinical personnel will gain arterial access, introduce the robot until the aorta (Fig. \ref{fig:platform}a), and supervise its navigation to the target site with tele-guidance of remote expert neuroendovascular surgeons. 

Under the hypothesis that higher tip steerability would improve navigation capabilities, several actuation technologies such as cable- \cite{Lis2022DesignRobot, Abah2024Self-SteeringInterventions} and magnetic-based \cite{Kim2022TeleroboticManipulation, Dreyfus2024DexterousAccess, Brockdorff2024HybridApplications, Pittiglio2022Patient-SpecificEndoscopy, Pittiglio2023PersonalizedLungs, Dreyfus2024DexterousAccessb} have been investigated. While these enable higher dexterity and improved targeting, they are associated with higher development and production costs. In contrast, we propose minimal hardware development, and developed a cost-effective actuation platform (Fig. \ref{fig:platform}b) to actuate standard endovascular catheterization tools (telescoping guidewire and catheters). 

In this paper, we present a \emph{contact-aware path planning} strategy able to interpret 3D images of the arteries, eventually provided by pre-operative \gls{cta} or \gls{mra}, and command the robots actions to steer and advance in the anatomy by leveraging tools-anatomy interaction. This strategy is inspired by standard manual catheterization approaches which uses no tools steerability and require interacting with the vessels' walls to navigate. Inspired by \cite{Pittiglio2023ClosedRobots}, we introduce a novel model for flexible telescoping tools able to predict their shape when introduced in a anatomy of known geometry, gathered from \gls{cta} or \gls{mra}. This model was used to plan robotic navigation inside the vessels, using a \gls{rrt} algorithm.

In this work, we focus on the first step of neuroendovascular catheterization from trans-femoral access (Fig. \ref{fig:platform}a): navigating from the aortic arch to the carotid arteries. Notice that the path from femoral artery to aorta is relatively straight and does not require specific planning. We present experimental validation in a realistic phantom of the anatomy to show the ability of our autonomous robotic platform to navigate from the base of the aorta to the \gls{lcca}, while avoiding the \gls{lsa}. We show repeatability over 50 trials and robustness towards unexpected motions of the anatomy.


