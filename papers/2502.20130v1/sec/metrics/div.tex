For measuring the spatial diversity of the features, \oldloc{5}~\citep{norrenbrocktake} has been proposed.
The \oldloc{5} however suffers from the non-linear behavior of the softmax, resulting in
scale-dependency (\cref{stab:oldloc5}).
Therefore, we propose the \textit{Scale-Invariant-Diversity@5} (\loc{5})
\begin{align}
\hat{M}^{\findex}_{i,j} = \frac{M^{\findex}_{i,j}}{\frac{1}{\gls{featuresMapwidth}\gls{featuresMapheigth}} \sum |\textbf{M}^{\findex}|} \quad \hat{S}^{\findex}_{i,j} = \frac{e^{\hat{M}^{\findex}_{i,j}}}{\sum_{m,n} e^{\hat{M}^{\findex}_{m,n}}} \\
   \loc{5} = \frac{\sum_{i=1}^{\gls{featuresMapheigth}}\sum_{j=1}^{\gls{featuresMapwidth}}\max(\hat{S}^{1}_{i,j},\hat{S}^{2}_{i,j}, \dots, \hat{S}^{5}_{i,j})}{5} , 
\end{align}
where $\hat{\textbf{S}}^\findex$ refers to the result of softmax applied to the $\findex$-th highest weighted feature map $\textbf{M}^{\findex}$, scaled by its absolute mean.
A high \loc{5} is visible in \cref{fig:metrics_full,fig:CubSim}, as the $5$ features used for each class, localize on very different regions in the image.
