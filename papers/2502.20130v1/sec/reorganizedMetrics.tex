Following \gls{PIP-Net} as recent work, we evaluate the accuracy and compactness, measured as number of total features~\gls{nReducedFeatures} and number of features per class~\gls{nperClass}.
Additionally, our \gls{NewlayerName} learns interpretable class representations, summarized in~\cref{tab:ClassRepresent}, that are composed of features.
As discussed in \cref{sec:RelatedWork}, diversity, contrastiveness, generality and grounding are desired aspects of explanations.
While we believe that our sparse binary assignment is very well suited for a detailed analysis and alignment of the learned features, as it likely prohibits superposition, polysemantic neurons are still likely to occur and hard to measure for \gls{NewlayerName} and all end-to-end trained interpretable models.
Therefore, we omit measuring the grounding of features and instead focus on contrastive,  general and diverse as desirable \textit{and} quantifiable qualities of features as building blocks of our interpretable class representations, whose \cubsim{} we estimate using the attributes contained in \cubheader{}.
Specifically, every class $\cindex$ in \cubheader{} is annotated with a vector $\boldsymbol{a}_\cindex\in[0,1]^{312}$, where $a_{\cindex,j}$ indicates the fraction of images with label $\cindex$, in which a human perceives the attribute $j$ to be present. 
With these vectors,
we compute 
the ground truth structural class similarity
$\boldsymbol{\mathrm{\ClassSim}}^{gt}\in[0,1]^{\gls{nClasses}\times\gls{nClasses}}$ with ${\mathrm{\classSim}}^{gt}_{\cindex,\cindex'}$ being the cosine similarity between $\boldsymbol{a}_\cindex$ and $\boldsymbol{a}_{\cindex'}$.
Similarly,
$\boldsymbol{\mathrm{\ClassSim}}^{Model}\in[-1,1]^{\gls{nClasses}\times\gls{nClasses}}$
is
based on the class vectors in the interpretable classification layer.
We then report the similarity for the top $25$ most similar unique pairs of classes $C_\mathrm{Sim}$ in reality 
\begin{equation}
  \mathrm{\cubsim} = 
  \frac{\sum_{\cindex,\cindex' \in C_\mathrm{Sim} } \mathrm{\classSim}^{Model}_{\cindex,\cindex'} }{\sum_{\cindex,\cindex' \in C_\mathrm{Sim} } \mathrm{\classSim}^{gt}_{\cindex,\cindex'}}.
\end{equation}
Models with high \cubsim{} offer an interpretable human-like class-similarity, \eg{} using the apparently different head to differentiate between Rottweiler and Doberman in \cref{fig:metrics_full} or differentiating shiny and bronzed cowbird by its only separating attribute, shown in \cref{fig:CubSim}.\\
To measure the contrastiveness of features, 
a Gaussian mixture model with two components 
is fit 
to every feature distribution $\boldsymbol{f}_{:,\findex}$, resulting in the 
normal distributions $\mathcal{N}_1^\findex$ and $\mathcal{N}_2^\findex$, visualized in \cref{fig:Contr}.
We then compute the \textit{\contrastiveness{}} as average of all features using the overlap~\citep{doi:10.1080/03610928908830127} between the two 
distributions: \begin{equation}
  \mathrm{\contrastiveness{}} =\sum_{\findex=1}^{\gls{nReducedFeatures}} 1 - \mathrm{Overlap}(\mathcal{N}_1^\findex, \mathcal{N}_2^\findex),\end{equation}
as bi-modal contrastive features can be represented by two non-overlapping distributions.
The binary quality of the features is also indicated in \cref{fig:metrics_full,fig:CubSim}, as the features are normed per column.\\

Additionally, the features should capture a general concept, instead of a class-specific one. This can be measured via the \textit{\generality} $\tau$:
\begin{align}
\tau = 1-\frac{1}{\gls{nReducedFeatures}} \sum_{\findex=1}^{\gls{nReducedFeatures}} \max_\cindex
&\frac{\sum_{j=1}^{\gls{nTrainImages}} l^\cindex_j (f_{j,\findex} - \min \boldsymbol{f}_{:,\findex})}{\sum_{j=1}^{\gls{nTrainImages}} (f_{j,\findex}- \min \boldsymbol{f}_{:,\findex})}
\end{align}
It
measures which fraction of the zero-based feature activation across the entire dataset is
not
focussed on the most related class.
A model with high \generality{} has features that recognize a shared concept for multiple classes, like the $4$ central features in \cref{fig:metrics_full,fig:CubSim}. 
Notably, as opposed to Dependence~\citep{norrenbrock2024q}, \generality{} can capture the assignment of multiple class detectors to the same class.
\\
For measuring the spatial diversity of the features, \oldloc{5}~\citep{norrenbrocktake} has been proposed.
The \oldloc{5} however suffers from the non-linear behavior of the softmax, resulting in
scale-dependency (\cref{stab:oldloc5}).
Therefore, we propose the \textit{Scale-Invariant-Diversity@5} (\loc{5})
\begin{align}
\hat{M}^{\findex}_{i,j} = \frac{M^{\findex}_{i,j}}{\frac{1}{\gls{featuresMapwidth}\gls{featuresMapheigth}} \sum |\textbf{M}^{\findex}|} \quad \hat{S}^{\findex}_{i,j} = \frac{e^{\hat{M}^{\findex}_{i,j}}}{\sum_{m,n} e^{\hat{M}^{\findex}_{m,n}}} \\
   \loc{5} = \frac{\sum_{i=1}^{\gls{featuresMapheigth}}\sum_{j=1}^{\gls{featuresMapwidth}}\max(\hat{S}^{1}_{i,j},\hat{S}^{2}_{i,j}, \dots, \hat{S}^{5}_{i,j})}{5} , 
\end{align}
where $\hat{\textbf{S}}^\findex$ refers to the result of softmax applied to the $\findex$-th highest weighted feature map $\textbf{M}^{\findex}$, scaled by its absolute mean.
A high \loc{5} is visible in \cref{fig:metrics_full,fig:CubSim}, as the $5$ features used for each class, localize on very different regions in the image.


