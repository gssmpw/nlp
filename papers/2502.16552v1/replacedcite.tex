\section{Related Work}
One of the earliest and most popular type of epidemic models are homogeneous mixture models, where contacts between all pairs of nodes are equally likely ____. In these models, a key indicator is the basic reproduction number ____, defined as the expected number of secondary infections caused by a single infection. Such models do not reflect the heterogeneous nature of human interactions, which are often also location dependent.
A natural approach is to introduce spatial modeling ____. In spatial modeling, nodes in proximity have interactions and make it possible for infections to occur. The question is if or when the disease, through local connectivity, eventually spreads to a large component of the nodes when originated from a single node. Such questions are about the critical thresholds such that percolation occurs ____.
The most known setups perhaps are square lattices with Bernoulli percolation in the discrete case and Boolean models based on Poisson point processes (PPPs) in the continuum. 
The latter is generalized by Poisson random connection models, which assume that two nodes $x,x'$ in a PPP connect with probability $f(\|x-x'\|)\in[0,1]$, where $f(\cdot)$ is the connection function \cite[Chapter 6]{MeesterRonald1996CP}. {For Poisson random connection models, it is shown that dispersive functions, which correspond to long-range but  unreliable connections, make 
percolation easier }____.
More recently, ____ considers the combination of disease dynamics and random geometry through studying the steady state of an susceptible-infected-susceptible (SIS) model on spatial point processes. It considers transmissions between nodes within a given radius as well as the effect of node mobility.
 
 

 
 The role of hubs (common destinations) in disease spreading is not reflected in the models above. In this respect, an epidemic model based on an abstract random bipartite graph is studied in ____. It constructs the random bipartite graph by drawing an edge between any pair of vertices of distinct types with a fixed probability, then extracts its unipartite graph, and derives the probability of explosion under tunable clustering. The model is homogeneous and does not consider geometric dependence. The model most relevant to our setting is the random bipartite geometric model in ____. It is also called AB continuum percolation model due to its discrete counterpart, AB percolation models on, e.g., $\mathbb{Z}^2$ \cite[Chapter 12]{grimmett1999percolation}. It is first proposed and studied by ____, where the authors consider two independent PPPs $\Phi,~\Psi$, defined on $\mathbb{R}^2$, and edges are drawn between $x\in\Phi$ and $y\in \Psi$ if $x,y$ are within a prescribed distance. The motivation of the model comes from communication networks with two types of transmitters. The authors proved the percolation thresholds (defined in Sec IV) and their bounds. Further results for this model are proved in ____. The RBG model studied here generalizes the AB continuum model in the same way that Poisson random connection models generalize Poisson Boolean models.