\section{Introduction}
\subsection{Motivation}
The spread of many airborne diseases relies on physical contact or proximity between susceptible individuals. Individuals visiting common destinations are especially likely to further diseases' propagation. For humans, such destinations can be supermarkets, social gatherings, airports, etc. Understanding the joint effect of individual density, density of the common destinations, and their connectivity pattern is a first step to the prevention of disease spreading.
To this end, we propose and study a
 mathematical model for spatial disease propagation with hubs. The model is based on \textit{random bipartite geometric} (RBG) graphs (defined in Sec II.A) and generalizes the AB continuum model \cite{iyer2012percolation,penrose_2014}.
In an RBG graph, there are two sets of entities,  referred to as \textit{agents} and \textit{hubs} respectively in this work, each being distributed in the Euclidean space according to a random spatial point process, hence the random and geometric aspect of the graph. Edges may only exist between the two sets of entities and not within, hence the bipartite aspect of it. {Lastly, the connectivity pattern between agents and hubs usually depends on their geometric distance. Dispersion of the connectivity pattern can be used to model a change in the behavior of the agents while keeping other parameters fixed, or to compare different scenarios where agents tend to travel farther or stay closer to their home location. }


\subsection{Related Work}
One of the earliest and most popular type of epidemic models are homogeneous mixture models, where contacts between all pairs of nodes are equally likely \cite{newman2018networks,pastor2015epidemic}. In these models, a key indicator is the basic reproduction number \cite{Delamater2019ComplexityOT}, defined as the expected number of secondary infections caused by a single infection. Such models do not reflect the heterogeneous nature of human interactions, which are often also location dependent.
A natural approach is to introduce spatial modeling \cite{Li2011TheFO,pastor2015epidemic,BARTHELEMY20111,katori2021continuum}. In spatial modeling, nodes in proximity have interactions and make it possible for infections to occur. The question is if or when the disease, through local connectivity, eventually spreads to a large component of the nodes when originated from a single node. Such questions are about the critical thresholds such that percolation occurs \cite{frisch1963percolation}.
The most known setups perhaps are square lattices with Bernoulli percolation in the discrete case and Boolean models based on Poisson point processes (PPPs) in the continuum. 
The latter is generalized by Poisson random connection models, which assume that two nodes $x,x'$ in a PPP connect with probability $f(\|x-x'\|)\in[0,1]$, where $f(\cdot)$ is the connection function \cite[Chapter 6]{MeesterRonald1996CP}. {For Poisson random connection models, it is shown that dispersive functions, which correspond to long-range but  unreliable connections, make 
percolation easier }\cite{Franceschetti05continuumpercolation,penrose1993spread}.
More recently, \cite{baccelli2020computational1} considers the combination of disease dynamics and random geometry through studying the steady state of an susceptible-infected-susceptible (SIS) model on spatial point processes. It considers transmissions between nodes within a given radius as well as the effect of node mobility.
 
 

 
 The role of hubs (common destinations) in disease spreading is not reflected in the models above. In this respect, an epidemic model based on an abstract random bipartite graph is studied in \cite{britton2008epidemics}. It constructs the random bipartite graph by drawing an edge between any pair of vertices of distinct types with a fixed probability, then extracts its unipartite graph, and derives the probability of explosion under tunable clustering. The model is homogeneous and does not consider geometric dependence. The model most relevant to our setting is the random bipartite geometric model in \cite{iyer2012percolation,penrose_2014}. It is also called AB continuum percolation model due to its discrete counterpart, AB percolation models on, e.g., $\mathbb{Z}^2$ \cite[Chapter 12]{grimmett1999percolation}. It is first proposed and studied by \cite{iyer2012percolation}, where the authors consider two independent PPPs $\Phi,~\Psi$, defined on $\mathbb{R}^2$, and edges are drawn between $x\in\Phi$ and $y\in \Psi$ if $x,y$ are within a prescribed distance. The motivation of the model comes from communication networks with two types of transmitters. The authors proved the percolation thresholds (defined in Sec IV) and their bounds. Further results for this model are proved in \cite{penrose_2014}. The RBG model studied here generalizes the AB continuum model in the same way that Poisson random connection models generalize Poisson Boolean models. 


\subsection{Contribution}
The contributions of this work are summarized as follows.
\begin{itemize}
    \item We propose a mathematical model based on the random bipartite geometric (RBG) graph for disease propagation through agents visiting hubs.
    \item We analyze the statistics of the degree of the RBG graphs for general and Poisson point processes in $\mathbb{R}^d$. We show the impact of the dispersion of the connection function on these statistics.
    \item We show that the existence of a critical hub density for percolation depends only on the boundedness of the support of the connection function, which shows the necessity to curb long-distance travel to prevent disease spreading. Further, we give bounds on the critical densities of hubs and agents for the RBG graph with general connection functions. Lastly, we show that increasing the dispersion of the connection function lowers the critical threshold.
\end{itemize}

