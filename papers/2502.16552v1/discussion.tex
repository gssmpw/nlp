\section{Discussion}
\subsection{Dependent RBG Models}
For epidemic modeling, RBG models also admit cases where there is dependence between $\Phi$ and $\Psi$, i.e., correlated hub and agent locations. From the degree point of view, one would expect a higher mean degree $\mathbb{E}M$ than in the independent case when there is clustering/attraction between $\Phi$ and $\Psi$ and lower when there is repulsion. From the percolation point of view, the impact of such dependence is less straightforward.  
 Consider the Poisson cluster point process, with $\Psi$ the parent process and $\Phi$ the daughter process. One expects that ``too much clustering'' would prevent spreading from one cluster to another. For the repulsive type of dependence, for instance, agents could be “repelled” from hubs by placing them at the Voronoi corners of the hubs' Voronoi tessellation.  This may not be practical but could be interesting as a theoretical thought experiment - now the distances from hub to agents are relatively larger, but each agent has three hubs at the same distance, which could spur growth. Considering both situations, we expect that the attractive case makes local spreading easier but global spreading (percolation) harder, while the repulsive case works the other way around. Of course these are very qualitative statements; for a rigorous analysis, one would have to properly rescale or normalize the densities and the connection function. 
\subsection{Dynamics}
The static RBG model in this work can be interpreted as capturing discrete-time disease dynamics as follows: in each
time step, the infectious agents infect those connected to the same hub(s) and
then recover in the next time step(s). Since all edges are i.i.d., the network percolates
effectively if starting with an infected agent at the origin from time 0, eventually a
non-trivial fraction of the entire population is infected at least once.  

An important direction for future work is to incorporate general temporal dynamics into this static model. For example, one may consider dynamics in edges and agent locations as well as in agents' states of infection and recovery. In this direction, one can consider the spatiotemporal graph models as in \cite[Chapter 11]{haenggi2012stochastic} to trace the paths and delays of disease spread. Another way is to introduce susceptible-infected-susceptible dynamics of agents' states based on the RBG graph and study its steady state behaviors \cite[Chapter 7]{liggett1985interacting}.

