\subsection{Proof of Theorem \ref{thm: N,M}}
\label{appendix: N,M}
 \begin{align}
       \Ex N  %\Ex\sum_{y\in\Psi}\sum_{x \in\Phi} f(\|y\|)f(\|x-y\|)\nonumber\\
       &=\Ex\sum_{y\in\Psi}\sum_{x \in\Phi} f(\|y\|)f(\|x-y\|)\nonumber\\
       &\peq{a}\lambda\mu \int_{\mathbb{R}^d}\int_{\mathbb{R}^d}  f(\|y\|)f(\|x-y\|) \dd x \dd y\nonumber\\
      % & = \lambda\mu \int_{\mathbb{R}^2} f(\|y\|) \int_{\mathbb{R}^2}f(\|x-y\|) \dd x \dd y\nonumber\\ 
%       & = \lambda\mu \left(\int_{\mathbb{R}^2}f(\|y\|) \dd y\right)^2\\
       & = \lambda\mu \left(\int_{0}^{\infty}f(r)dc_d r^{d-1} \dd r\right)^2.\nonumber
    \end{align}
 Step (a) follows from Campbell's theorem and the independence of $\Phi$ and $\Psi$. 
 \begin{align}
            \nonumber\Ex M & =  
            \Ex\sum_{x \in\Phi} \max_{y\in\Psi}I(o,y)I(x,y)\nonumber \\
            & = \Ex\sum_{x\in\Phi}1-\prod_{y\in\Psi} (1-f(\|y\|)f(\|x-y\|))\\
           % & \peq{a} \lambda\int_{\mathbb{R}^d} \Ex\max_{y\in\Psi}I(o,y)I(x,y)\dd x \nonumber\\
            & \peq{a}\lambda \int_{\mathbb{R}^d}1- \Ex\prod_{y\in\Psi} \left(1-f(\|y\|)f(\|x-y\|)\right) \dd x \nonumber\\
%    & = \lambda\int_{\mathbb{R}^2}    1-\Ex_{\Psi}\ \prod_{y\in\Psi} \left(1-f(\|y\|)f(\|x-y\|)\right)\nonumber\\
     & \peq{b}  \lambda\int_{\mathbb{R}^d}  1-\exp\left(-\mu\int_{\mathbb{R}^d}f(\|y\|)f(\|x-y\|)\dd y\right) \dd x.\nonumber
   \end{align}   
   Step (a) again follows from Campbell's theorem. Step (b) follows from the probability generating functional (PGFL) of the PPP.
   \begin{align}
  & \Ex N^2 \nonumber \\
   &= \Ex \left(\sum_{y\in\Psi}\sum_{x \in\Phi} I(o,y)I(x,y)\right)^2\nonumber\\
       & = \Ex \sum_{y\in\Psi}\sum_{x \in\Phi} I(o,y)I(x,y) \nonumber\\
       &\quad+\Ex\sum_{y\in\Psi}\sum_{x_1,x_2\in\Phi}^{\neq} I(x_1,y)I(o,y)I(x_2,y)\nonumber\\
       &\quad+\Ex\sum_{x\in\Phi}\sum_{y_1,y_2\in\Psi}^{\neq} I(x,y_1)I(o,y_1)I(x,y_2)I(o,y_2)\nonumber\\
       &\quad+\Ex\sum_{y_1,y_2\in\Psi}^{\neq}\sum_{x_1,x_2\in\Phi}^{\neq} I(x_1,y_1)I(o,y_1)I(x_2,y_2)I(o,y_2).\nonumber
   \end{align}
Here $\sum_{x_1,x_2\in\Phi}^{\neq}$ means that the summation takes over all ordered pairs of distinct points of $\Phi$. Observe that the first term is $\Ex N$ and the last term is equivalent to $(\Ex N)^2$. Using the joint density of $\Phi$ and $\Psi$ as well as the reduced second moment density of $\Phi$ and $\Psi$ we obtain (\ref{eq: E_N_o^2}).   

   For $M$,
   \begin{align}
        &\Ex M^2\nonumber \\
        &= \Ex \left(\sum_{x \in\Phi} \max_{y\in\Psi}I(o,y)I(x,y) \right)^2\nonumber\\
       %     & = \Ex\left(\sum_{x_1 \in\Phi} \max_{y_1\in\Psi}I(o,y_1)I(x_1,y_1) \right)\left(\sum_{x_2 \in\Phi} \max_{y_2\in\Psi}I(o,y_2)I(x_2,y_2) \right)\\
            & = \Ex \sum_{x\in\Phi} \max_{y\in\Psi}I(o,y)I(x,y) \nonumber \\
            &\quad+ \Ex\sum_{x_1,x_2\in\Phi}^{\neq}\max_{y_1\in\Psi}I(o,y_1)I(x_1,y_1)\max_{y_2\in\Psi}I(o,y_2)I(x_2,y_2)\label{eq: last}\\
           % & = \Ex M+ \Ex\sum_{x_1,x_2\in\Phi}^{\neq}\max_{y_1\in\Psi}I(o,y_1)I(x_1,y_1)\max_{y_2\in\Psi}I(o,y_2)I(x_2,y_2) \nonumber\\
            & \geq \Ex M + (\Ex M)^2.\nonumber
           % &= \Ex M_o+ 2\Ex \sum_{x_1\neq x_2} \left(1-\prod_{y\in\Psi}(1-f(\|y\|)f(\|x_1-y\|))\right)\left(1-\prod_{y\in\Psi}(1-f(\|y\|)f(\|x_2-y\|))\right)\nonumber
   \end{align}
   The last step follows from the following facts: the first term of (\ref{eq: last}) $\Ex \sum_{x\in\Phi} \max_{y\in\Psi}I(o,y)I(x,y) = \Ex M$; for the second term of  (\ref{eq: last}), it is equal to $(\Ex M)^2$ if  
$\max_{y_1\in\Psi}I(o,y_1)I(x_1,y_1)$ and $\max_{y_2\in\Psi}I(o,y_2)I(x_2,y_2)$ are independent. When the max over $\Psi$ is possible with $y_1=y_2$, there exists a positive correlation between them, hence the inequality.


\subsection{Proof of Corollary \ref{cor: examples-N-M}}
\label{appendix: examples-N-M}
The fact that $\Ex N = \lambda\mu {\pi^2}{\theta^4}$ directly follows from Theorem 1 and the integration of $f_1$.  To calculate $\Ex M$, we first present a useful lemma.      \begin{lemma}
For a stationary PPP $\Psi\subset \mathbb{R}^2$ with density $\mu$ and a deterministic $x\in\mathbb{R}^2$, the point process $\{y\in\Psi: \|y\|+\|x-y\|\}$ is a 1D PPP with intensity function
\begin{equation}
\lambda_x(r) = \frac{\mu\pi}{4}\frac{ 2r^2-\|x\|^2}{\sqrt{r^2-\|x\|^2}}, \quad r\geq \|x\|.    
\label{eq: lambda_x}
\end{equation}
   \end{lemma}
   \begin{proof}
   First note that $\|y\|+\|x-y\|\geq\|x\|$ by the triangular inequality. So the points lie in $[\|x\|,\infty)$. $\|y\|+\|x-y\|\leq r$ is a closed region whose boundary is an ellipse. Its area only depends on $\|x\|$ and $r$. The intensity measure is
  \[\Lambda([\|x\|,r])= \Ex \sum_{y\in\Psi} \ind(\|y\|+\|x-y\|\leq r) = \frac{\mu\pi r}{4}\sqrt{r^2-\|x\|^2}.\] 
  Taking its derivative, we obtain \eqref{eq: lambda_x}. The Poisson property is preserved as the numbers of points in disjoint regions are independent.
 \end{proof}


Now,
\begin{align}
  &\Ex \max_{y\in\Psi} I(o,y)I(x,y)\nonumber \\
    % &=  1-\Ex \exp\left(-\alpha\sum_{y\in\Psi}\ind_k(o\to y)\ind_k(x\to y)\right)\\
     &=\Ex_{\Psi}\left[ 1- \prod_{y\in\Psi} \left(1-\frac{1}{4}\exp\big(- (\|y\|+\|x-y\|)/\theta\big)\right)\right]\nonumber\\
     &\peq{a} 1-\exp\left(-\int_{\|x\|}^{\infty} \lambda_{x}(r) \frac{1}{4}\exp(-r/\theta)\dd r\right)\nonumber\\
     &=1-\exp\left(-\frac{\mu\pi \|x\|^2}{16}\left(\frac{2\theta K_1(\|x\|/\theta )}{ \|x\|}+K_0\big( \|x\|/\theta \big)\right)\right)\nonumber.
   \end{align}
    Step (a) follows from Lemma 1 and the PGFL of the PPP.

\printbibliography