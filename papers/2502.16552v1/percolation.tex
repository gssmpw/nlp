\section{Percolation}
  
%    Let $C_o$ be the connected component of the nodes in graph $\mathcal{G}$ that contains $o$. A node belongs to $C_o$ if there exists a path of adjacent nodes to $o$. Percolation occurs when $C_o$ is infinite with positive probability. 
    \subsection{Definitions}
We consider the percolation of the RBG graph where $\Phi$ and $\Psi$ are two independent PPPs with intensities $\lambda$ and $\mu$, respectively. We say that an infinite graph percolates if there exists an infinitely connected component in the graph a.s., or equivalently, the probability of the origin belonging to an infinite component is larger than 0.
 From the point of view of disease spreading, it means that a positive fraction of the network can eventually get infected from a single agent (ignoring the delay of spreading and recovery).  The RBG graph  model is ergodic and has a unique infinite component with probability 0 or 1 following the arguments in \cite[Theorems 2.1\&6.3]{MeesterRonald1996CP}. 
    
    %We define the percolation threshold of the RBG graph as follows. 
    
    
For a fixed $f$, let the percolation region be defined as
\begin{equation}
    \mathcal{D} \triangleq \{(\lambda,\mu): \mathcal{G}\left(\Phi,\Psi, f\right) ~\mathrm{percolates}\}.\nonumber
\end{equation}
   The boundary of $\mathcal{D}$ is  symmetrical, i.e., \[(\lambda,\mu)\in\mathcal{D} \quad \Leftrightarrow \quad (\mu,\lambda)\in\mathcal{D},\]
where $\Leftrightarrow$ denotes an implication in both directions.
    To see this, note that having an infinite component formed from a subset of $\Phi$ is equivalent to having  an infinite component formed from a subset of $\Psi$. Further, $\Phi$ and $\Psi$ are both PPPs.
    
For any fixed $\mu$, let $\lambda_c(\mu)$ be defined as \[
        \lambda_c(\mu) \triangleq \inf \{\lambda: (\lambda,\mu)\in \mathcal{D}\},\]
and $\mu_c$ be defined as
    \begin{equation}
        \mu_c \triangleq \inf \{\mu: \lambda_c(\mu)<\infty\}.\nonumber
    \end{equation} 
 Here our notations differ from \cite{penrose_2014} due to the physical interpretation of $\lambda$ and $\mu$ and that we focus on the critical density of hubs. The change of notation is not fundamental due to the symmetry of $\lambda$ and $\mu$ in our model.
 % First we define the percolation threshold for the Poisson random connection model.
% \begin{equation}
%     \lambda_c^{*}  \triangleq \mathrm{inf} \{\lambda: \Pr(|C_o|=\infty)>0\}
% \end{equation}
   %   \begin{figure*}[t]
   %     \centering
   %   \subfigure[$g(r)$, $f_1$. ]{\includegraphics[width=.47\textwidth]{g_r_exp.eps}
   %    % \caption{$g(r)$ for fixed $\theta, \mu$ and a set of $\gamma$ per Corollary 1.}
   %     \label{fig: g_r_exp}}
   %    \hfill
   %   \subfigure[$g(r)$, $f_2$]{\includegraphics[width=.47\textwidth]{g_r_disk.eps}
   %      %  \caption{$g(r)$ for fixed $\theta=1$ and a set of $\gamma$ per Corollary 2. }
   %        \label{fig: g_r_disk}}
   %      \caption{$g(r)$ for $f_1$ and $f_2$. }
   %            \label{fig: g_r}
   % \end{figure*}   

\begin{comment}
    
    Denote the probability of node $x$ at distance $r=\|x\|$ being connected to $o$ as $g(r)$. 
   \begin{equation}
g(r)= 1-\exp\left(-\mu\int_{\mathbb{R}^2}f(\|y\|)f(\|x-y\|)\dd y\right).
    \end{equation}
    
    The function $g$ abstracts the impact of $\Psi$ and shows the probability of a connection at a given distance to $o$. Assuming independence between different connections, we obtain the Poisson random connection model. For the Poisson random connection model, we denote the percolation threshold as $\lambda_c^*$, defined analogously to Definitions 2 and 3.
 This ignores the underlying \hl{correlation} between connections due to hubs locations in the RBG graph. 

\end{comment}
 Consider a random unipartite geometric graph on a  PPP $\Phi$ with intensity $\lambda$, $\mathcal{G}(\Phi,f)$. 
 Denote by $\zeta_f$ the critical density for the percolation of the graph $\mathcal{G}(\Phi,f)$, i.e.,
 \[
\zeta_f \triangleq \inf \{\lambda\colon \mathcal{G}\left(\Phi,f\right) ~\mathrm{percolates}\}
 \]
 
 For $\int_{0}^{\infty}f(r)r^{d-1}\dd r<\infty$,  it is known that \cite[Chapter 6]{MeesterRonald1996CP}
  \[
  0<  \zeta_f<\infty.
    \] 
    %it is known that $\lambda_c^*<\infty$ for $\int f \dd x <\infty$, $\lambda_c<\infty$.
    Further, there is at most one unbounded component.  


 
% Here, the question is whether and when $\lambda_c$  is an accurate approximation of $\lambda_c$ for the RBG graph. In addition, consider  generated by a Poisson random connection model with connection function $f$. Define 
   
    
%    Let $\lambda_c^{\mathrm{u}}$ denote the percolation threshold for this unipartite model, where $u$ stands for unipartite. 
   
    
    \subsection{Some Bounds on the Percolation Thresholds}
 

    
        \begin{lemma}  
    The percolation region $\mathcal{D}$ for $\mathcal{G}(\Phi,\Psi,f)$ satisfies 
    %\[\lambda_c^*\leq \lambda+\mu,\quad (\lambda,\mu)\in\mathcal{D}\]
\begin{equation}
\min\{\lambda+\mu\colon (\lambda,\mu)\in \mathcal{D}\}\geq \zeta_f.
\end{equation}
    \end{lemma}
\begin{proof}
        Construct the following coupling: let $\mathcal{G}\left(\Phi\cup\Psi,f\right)$ be a unipartite random geometric graph with nodes  being the union of agents and hubs from $\mathcal{G}\left(\Phi,\Psi,f\right)$. The edge set of $\mathcal{G}\left(\Phi\cup\Psi,f\right)$ contains all edges of $\mathcal{G}\left(\Phi,\Psi,f\right)$, as well as those edges within agents and within hubs generated through $f$. So  $\mathcal{G}\left(\Phi\cup\Psi,f\right)$ dominates $\mathcal{G}\left(\Phi,\Psi,f\right)$ in the sense of the coupling just mentioned. So the percolation of $\mathcal{G}\left(\Phi,\Psi,f\right)$ implies the percolation of  $\mathcal{G}\left(\Phi\cup\Psi,f\right)$.
\end{proof}  
\begin{remark}
A special case for $f(r) = \ind (r\leq a) $ is proved in \cite{iyer2012percolation} with the same construction of coupling.    
\end{remark}

With a slight abuse of notation, let $\zeta(2a)$ denote the percolation threshold of $\mathcal{G}(\Phi, \ind (r\leq 2a))$ (the Poisson Boolean model). 

\begin{lemma}{\cite[Theorem 1.1]{penrose_2014}}\label{lemma: penrose}
    For the RBG model with $f(r) = \ind (r\leq a)$,      \[\mu_c = \zeta(2a).\]   
\end{lemma}
\begin{proof}
The proof is given in \cite{penrose_2014} through constructing a coupling of the RBG model with Bernoulli site percolation.
\end{proof}
\begin{remark}
    A special case for $d=2$ is proved in \cite{iyer2012percolation}.
\end{remark}
Here we prove that the same threshold holds for the connection functions $f(r) = p\ind (r\leq a)$, $p\in(0,1]$, which generalizes the Boolean connection function in \cite{iyer2012percolation,penrose_2014}. 

 \begin{theorem}
 For the RBG model with the connection functions $f(r) = p\ind (r\leq a)$, $p\in(0,1]$, 
 \label{thm: perco-th-p-boolean}
\begin{equation}
\mu_c = \zeta(2a).
 \end{equation}

 \end{theorem}
\begin{proof}
   The proof is based on the proof in \cite[Theorem 1.1]{penrose_2014}, with one modification. For the Bernoulli site percolation of $\epsilon\mathbb{Z}^2$ with retaining probability $q<1$ coupled with $\mathcal{Q}_{\mu}$ (corresponding to $\Phi$ with density $\lambda$ in our notation), we let $\lambda$ be such that 
   $1-\exp(-\lambda p^2 \epsilon^d) = q$ rather than requiring that $1-\exp(-\lambda \epsilon^d) = q$. The modification makes sure that if we independently remove points in $\Phi$ with probability $p^2$ (to account for the sampling of the two-edge path), the probability of having a point within each cube of size $\epsilon^d$ is  $q$.
\end{proof}



\begin{corollary}    \label{cor: bound_lambda_c_general_f}
    For a monotone decreasing $f$ with bounded support on $[0,a]$ and  lower bounded away from 0, 
    \[\mu_c = \zeta(2a).\]
\end{corollary}
\begin{proof}
Let $\underline{f}(r) = p\ind(r\leq a)$ where $p$ is the lower bound of $f$ on the support and $\bar{f}(r)=\ind(r\leq a)$. Then  there exists a coupling between the corresponding graphs;  $\mathcal{G}\left(\Phi,\Psi,\bar{f}\right)$ dominates $\mathcal{G}\left(\Phi,\Psi,f\right)$, and $\mathcal{G}\left(\Phi,\Psi,f\right)$ dominates $\mathcal{G}\left(\Phi,\Psi,\underline{f}\right)$. 
Since for both $\underline{f}$ and $\bar{f}$, $\mu_c  = \zeta(2a)$ by Theorem \ref{thm: perco-th-p-boolean} and Lemma \ref{lemma: penrose},  the proof is complete.
\end{proof}
\begin{remark}
\label{remark: bounded_travel} 
    Corollary \ref{cor: bound_lambda_c_general_f} shows that if  the traveling distance is bounded, it suffices to limit the density of hubs to prevent the disease from spreading, regardless of the density of agents.   
\end{remark}
Corollary \ref{cor: bound_lambda_c_general_f} also implies that the percolation threshold $\mu_c<\infty$ for RBG graphs with monotone decreasing connection functions with infinite support. This follows directly from the monotonicity of $\zeta(2a)$ in $a$ and the fact that $\lim_{a\to\infty}\zeta(2a)=0$.
 

\begin{theorem}
    For general monotone decreasing connection function $f$, 
    \begin{equation}    
    \lambda\mu\geq \left(\int_{\mathbb{R}^d}f(\|x\|) \dd x)\right)^{-2},\quad (\lambda,\mu)\in\mathcal{D}.
    \label{eq: g-w-branching-bound}
    \end{equation}
    \label{thm: g-w-branching-bound}
\end{theorem}
\begin{proof}
    We use the same approach as in \cite[Chapter 6]{MeesterRonald1996CP} that proves the existence of the threshold for Poisson random connection models. The idea is to establish a coupling of the RBG graph with a Galton-Watson branching process with average descendants \[\Ex N=\lambda\mu\left(\int_{\mathbb{R}^d}f(\|x\|) \dd x\right)^2.\]

    Let us start with the agent at the origin. In the first step, its connected hubs follow an inhomogeneous PPP with intensity function $\mu f(\|y\|)$, whose mean number is $\mu\int_{\mathbb{R}^d}f(\|y\|) \dd y$. In the second step, consider for each hub $y_1, y_2,...$ generated in the first step: for $y_1$, its connected agents in the second step  follow an inhomogeneous PPP  $\lambda f(\|x-y_1\|)$; for $y_2$, the connected agents in this step  follow an independent inhomogeneous PPP $\lambda f(\|x-y_2\|)(1-f(\|x-y_1\|))$, and so on. The agent set generated by each hub in this step is upper bounded by an independent inhomogeneous PPP of intensity $\lambda f(\|x-y_i\|),~i=1,2,...$. The total number of agents in the second step can be dominated by a compound Poisson r.v. whose mean is $\lambda\mu(\int_{\mathbb{R}^d}f(\|x\|) \dd x)^2 = \Ex N$. Repeating the two steps, a coupling between the RBG graph and a Galton-Watson branching process is established. Requiring $\Ex N \geq 1$ leads to Theorem \ref{thm: g-w-branching-bound}. 
\end{proof}

Theorem \ref{thm: g-w-branching-bound} uses a Galton-Watson branching process to lower bound the RBG graph, where the analogous approach applies to its unipartite counterpart. Considering the limiting result established for unipartite graphs in \cite{penrose1993spread}, one may conjecture that in the limit of dispersion, i.e., $p\to0$, the bound also becomes tight for RBG graphs. 

\begin{lemma}
For the RBG graph with general monotone decreasing connection functions,   $\mu_c>0$ if and only if $f$ has bounded support.
\label{lemma: iff_bounded_support}
\end{lemma}
\begin{proof}
If $f$ has bounded support, let $\bar{f}(r) = \ind(f(r)>0)$. By coupling, $\mathcal{G}(\Phi,\Psi,f)$ is dominated by $\mathcal{G}(\Phi,\Psi,\bar{f})$. We have $\mu_c \geq \bar{\mu}_c >0$.

   Now consider $f$ with infinite support. Consider a Poisson Boolean model. For all $\epsilon>0$, one can find $a_{\epsilon}$ large enough such that for $\lambda=\epsilon$ and $g(r)= \ind(r\leq 2a_{\epsilon})$, $\mathcal{G}(\Phi,g)$ percolates. By the infinite support, one can find $b\geq a_{\epsilon} $ such that $f(b)>0$.  Using Corollary \ref{cor: bound_lambda_c_general_f}, $\mu_c\leq \zeta(2a_{\epsilon})\leq \epsilon$.
\end{proof}
    Lemma \ref{lemma: iff_bounded_support} shows that if agents' traveling distance is unbounded, an arbitrary small density of hubs can cause disease spreading given that the density of agents is large enough. Combining this with Remark \ref{remark: bounded_travel}, we show that long-distance travels (e.g., via airplanes) play a critical role in disease spreading.


  %We study the RBG percolation for Boolean type of connection functions: $f = p\ind{(r\leq a)}$.
 %\begin{proposition}

% For the generalized Boolean model, $\lambda_c\geq\tilde\lambda_c(2a)$.
%  \end{proposition}
%       \begin{proof}
%          Coupling.
%      \end{proof}

 
\subsection{Dispersed RBG Graph}
Let $f_p$ be the dispersed version of a connection function $f$ as defined earlier. For any fixed $\mu$, let $\lambda_c(\mu)$ denote the percolation threshold for the RBG graph with connection function $f$ and $\lambda_c^p(\mu)$ denote that for the RBG graph with $f_p$. $\mu_c$ and $\mu_c^p$ are defined similarly. For the random geometric unipartite graph, it is shown in \cite{Franceschetti05continuumpercolation} that $\zeta_f\geq \zeta_{f_p}$. Later strict inequality is proved in some cases. 
 Here we show that a similar result holds for the RBG graph.



 % \begin{lemma}
 %   For RBG graphs, bond percolation threshold is no greater than site percolation threshold $(\lambda^b,\mu^b) \subset (p\lambda^s,p\mu^s)$
     
 % \end{lemma}

 
 \begin{theorem}
 For all $f$, $p\in(0,1]$, 
\begin{equation}
\lambda_c(\mu)\geq \lambda_c^p(\mu),\nonumber
\end{equation} 
and consequently,
\[\mu_c^p\leq\mu_c.\]
 \end{theorem}
\begin{proof}
Firstly, consider any fixed graph. Consider its site percolation and bond percolation respectively \cite[Chapter 1]{grimmett1999percolation} as follows: the former  independently removes points in $\Phi$ and $\Psi$ with probability $1-p$; the latter independently removes edges with probability $1-p$. The bond percolation threshold is no greater than the site percolation \cite{Franceschetti05continuumpercolation}.
Now, consider any realization of $\mathcal{G}$ generated from $\Phi'=\Phi/p,~\Psi'=\Psi/p,~f(r)$. $\lambda_c(\mu)$ is the site percolation threshold (with probability $p$) of the graph $\mathcal{G}(\Phi',\Psi',f)$, and  equivalently, the percolation threshold for $\mathcal{G}(\Phi,\Psi,f)$. $\lambda_c^p(\mu)$ is the bond percolation threshold of the graph $\mathcal{G}(\Phi',\Psi',pf(r))$, which is equivalent to the percolation of $\mathcal{G}(\Phi,\Psi,~pf(\sqrt{p}r))$ by  scaling. The second inequality follows from the definition of $\mu_c$.
\end{proof}
  A dispersive function lowers the percolation threshold and expands the percolation region. This is consistent with the monotonicity and the limit of the mean degree in Corollary \ref{cor: dispersed-mean-degree}.
One may expect (\ref{eq: g-w-branching-bound}) to become more tight when $p\to0$ ($f$ is dispersive).  In the case where (\ref{eq: g-w-branching-bound}) is tight, we have \begin{align}
        \zeta_f &\leq \min_{(\lambda,\mu)\in\mathcal{D}}\lambda+\mu\nonumber \\
        &\approx \min_{(\lambda,\mu)\in\mathcal{D}} \lambda + {\lambda^{-1}\left(\int_{\mathbb{R}^d}f(\|x\|) \dd x\right)^{-2}} \nonumber\\
        &= {2}{\left(\int_{\mathbb{R}^d}f(\|x\|) \dd x\right)}^{-1}.\nonumber
\end{align}

 
 
 %    \hl{Limiting distribution Poisson?}
    
           
 
    %For $f_2$, a necessary condition for disease spreading is that the Boolean model of the hub point process $\Psi$ with a fixed disk radius $a$ percolates, $i.e.,$ $\mu\pi a^2>4.5$ \cite[Chap. 11]{haenggi2012stochastic}. The reason is that infectious nodes can only spread the disease through common hubs, and nodes in the intersection of the disks of hubs are critical to spread the disease across disks. 
    %Related, \cite{baccelli2020computational} conjectures that there is repulsion between susceptible  and infected nodes when transmissions only occur between nodes in proximity.
    % \hl{pair correlation func over time}
    %  \begin{figure}
    %      \centering
    %      \includegraphics{clustering.eps}
    %      \caption{Illustration of the spatial locations of nodes after two time instances. Blue, red, and green dots denote S, I, R nodes, respectively. Black circles denote  hubs.}
    %      \label{fig:my_label}
    %  \end{figure}   
    