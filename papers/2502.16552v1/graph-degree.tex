
    \section{Degrees of the RBG Graph}
In this section, we consider some basic properties of the RBG graph $\mathcal{G}(\Phi,\Psi, f)$. Let $\Phi^o\triangleq \Phi\mid o\in\Phi.$ We focus on the typical agent at the origin and the Palm distribution of $\Phi$
\begin{equation}
 \mathrm{P}_{}^{o}(\cdot) \triangleq \Pr(\Phi\in\cdot \mid o\in\Phi),\nonumber
\end{equation}
and the reduced Palm distribution
\begin{equation}
 \mathrm{P}_{}^{!o}(\cdot) \triangleq \Pr(\Phi\setminus\{o\}\in\cdot\mid o\in\Phi).\nonumber
\end{equation}
We denote by $\Ex_{}^{{o}}$ the  expectation with respect to the Palm measure of $\Phi$
   and by $\Ex_{}^{!o}$ the  expectation with respect to the reduced Palm measure of $\Phi$ \cite[Chapter 8]{haenggi2012stochastic}.
   %of $\Phi$, i.e., $\Ex^{!}_{{o}}f(\Phi) \triangleq \Ex (f(\Phi\setminus\{o\})\mid o\in\Phi).$
    




%There are several quantities of interest and connection functions: the degree (the number of edges) of the typical node/hub, the number of connections of the typical node, and the number of connected nodes of the typical node.
\subsection{General Point Processes}
%To simplify the notation, let $\Ex^{!o}$ denote the (total) expectation where the expectation with respect to $\Phi$ is its reduced Palm expectation.

Let $c_d$ denote the volume of unit ball in $\mathbb{R}^d$.
The mean degree of the typical agent in a general point process is 
%  $\sum_{y\in\Psi} \ind(o\to y)$ with mean 
    \begin{align}
  \Ex^{o} \sum_{y\in\Psi}I(o,y)
       &\peq{a} \Ex \sum_{y\in\Psi}f(\|y\|)\nonumber
      \\
      &\peq{b} \mu c_d d \int_{0}^{\infty}f(r) r^{d-1} \dd r,\label{eq: EN_hub}
    \end{align}
    where step (a) follows from the independence of $\Phi$ and $\Psi$, and step (b) follows from Campbell's theorem. 
    By the mass transport principle \cite{baccelli:hal-02460214}, the mean degree of the typical hub (defined through the Palm expectation of $\Psi$) is $\lambda c_d d \int_{0}^{\infty}f(r) r^{d-1} \dd r.$ %Similarly, the sum length of the edges of the typical node is given by $ \Ex \sum_{y\in\Psi} \|y\|I(o, y)= \mu c_d d \int_{0}^{\infty}f(r) r^{d} \dd r.$
    For example, consider $d=2$, a general connection function $f$, and its dispersed version $f_{p}$. While the mean degree remains constant for any $p\in(0,1]$,
%, i.e., \[\int_{0}^{\infty}f_p(r)2\pi r \dd r \equiv \int_{0}^{\infty}f(r)2\pi r \dd r.\]
 the edges spreads out in space as $p\to 0$. To see this, consider the mean distance
 \begin{align}
 \Ex^{o} \sum_{y\in\Psi}I(o,y)\|y\| &=\int_{0}^{\infty}f_p(r)2\pi r^2 \dd r\nonumber \\
 & = \frac{1}{\sqrt{p}}\int_{0}^{\infty}f(r)2\pi r^2 \dd r,\nonumber
 \end{align}
which increases as the connection function becomes more dispersed.


    
 Let $M\triangleq\sum_{x\in\Phi^o\setminus\{o\}}\max_{y\in\Psi} I(o,y)I(x,y)$ denote the number of agents connected to $o$.  We have
    \begin{align}
        \Ex M &= \Ex\sum_{x\in\Phi^o\setminus\{o\} } 1-\prod_{y\in\Psi} \left(1-I(o,y)I(x,y)\right) \nonumber\\
        &= \Ex^{!o} \sum_{x\in\Phi} 1-\prod_{y\in\Psi} (1-f(\|y\|)f(\|x-y\|)).\nonumber
        \end{align}
    This follows from the fact that the edge indicator functions are independent Bernoulli random variables given $\Phi$ and $\Psi$.
    
    Let $N\triangleq\sum_{y\in\Psi}\sum_{x \in\Phi^o\setminus\{o\}} I(o,y)I(x,y)$ denote the total number of direct (two-edge) paths between $o$ and its connected agents. We have
    \begin{align}
       \Ex N  &= \Ex \sum_{y\in\Psi}\sum_{x \in\Phi^o\setminus\{o\}} I(o,y)I(x,y) \nonumber\\
       &= \Ex^{!o}\sum_{y\in\Psi}\sum_{x \in\Phi}f(\|y\|)f(\|x-y\|).\nonumber
       %\\ \nonumber & =  \Ex_{\Psi} \sum_{y\in\Psi}f(\|y\|) \Ex^{!}_{{o}}\sum_{x \in\Phi} f(\|x-y\|),
    \end{align}  
    

 
    
By definition, $N\geq M$, and the equality is achieved when $o$ and $x$ are connected through at most one hub, $\forall x\in\Phi$. The difference between $N$ and $M$ reflects how likely two agents connect through multiple hubs. The variance of $M$ is $\mathbb{V} M \triangleq \Ex M^2 - (\Ex M)^2$, which characterizes the heterogeneity of the number of connected agents.
   
   
    
    % \subsection{The Basic Reproduction Number}
    %   The basic reproduction number $R_0$ is widely used in epidemiology to predict whether an infectious disease will spread. However, it is difficult to calculate in general and its effectiveness and interpretation rely on the model \cite{Li2011TheFO}. While $R_0$ captures the initial state of the disease propagation, it may not predict the spread in a global level. For instance, consider a model where each node always connects with its nearest hub and $\alpha=1$. Then $R_0=N-1$ assuming each hub on average connects to $N$ nodes. However, due to the locality of connections, the disease only spreads within nodes connected to the same hub and always dies out for an arbitrarily large $N$. 
       
%      \begin{definition}[The basic reproduction number ]
%   \end{definition}


%Without loss of generality we assume that the single infected node is at the origin.
   


%   \begin{remark}
%         For given $\Phi$ and $\Psi$, denote the expected number of secondary infections generated by the typical infected node by $R(\Phi,\Psi)$. $R(\Phi,\Psi)$ is the conditional basic reproduction number given node and hub locations.      Given $\Phi,\Psi$, the number of secondary infections caused by $o$ is 
%      \begin{align}
%         R(\Phi,\Psi)&= \Ex \left[\sum_{x\in\Phi}\max_{1\leq k\leq K} \ind (Z_x^{k} = \mathrm{I}\mid Z_x^{0}=\mathrm{S},Z_o^{0} = \mathrm{I})\mid \Phi,\Psi\right]\\
%          &= \Ex \left[\sum_{x\in\Phi}  1- (1-\alpha)^{K  N_x}\mid\Phi,\Psi\right]\\
%          & = \Ex \left[\sum_{x\in\Phi}  1- \prod_{y\in\Psi} (1-\alpha)^{K \ind(o\to y)\ind(x\to y)}\mid\Phi,\Psi\right]\\
%          & \peq{a} \sum_{x\in\Phi}  1- \prod_{y\in\Psi}\left[ (1-\alpha)^{K}f(\|y\|)f(\|x-y\|)+1-f(\|y\|)f(\|x-y\|)\right]
%     %  \\& = 1- \prod_{\Psi} \left(1-t\gamma^2\exp\big(-\theta (\|y\|+\|x-y\|)\big)\right)^K.
%         \end{align}
% where step (a) follows from the independence of $\{I(x,y):x\in\Phi\}$ given $\Psi$ and $\Phi$. 
%     %  \\
%     %  & = 1-\exp\left(-\int_{\|x\|}^{\infty} \lambda_{x}(r) \Big(1-\big(1-t\gamma^2\exp(-\theta r)\big)^K\Big)\dd r\right),\label{eq: bessel}
%      By Definition 1, \begin{equation}
%          R_0 \triangleq \Ex R_{0}(\Phi,\Psi) = \Ex_{\Psi}\Ex_{o}^{!} R_{0}(\Phi,\Psi).
%      \end{equation}   
%   \end{remark}
    
    
  
    
    % and prove its existence for the static model. Two observations:\\
    % - the dynamic RBG graph's threshold is lower-bounded by the static one.\\
    % - the percolation threshold should decrease with the dispersiveness of $f$ using the idea in \cite{Franceschetti05continuumpercolation}

    
% \section{Analysis}
\label{sec:analysis}
In the following sections, we will analyze European type approval regulation\footnote{Strictly speaking, the German enabling act (AFGBV) does not regulate type-approval, but how test \& operating permits are issued for SAE-Level-4 systems. Type-approval regulation for SAE-Level-3 systems follows UN Regulation No. 157 (UN-ECE-ALKS) \parencite{un157}.} regarding the underlying notions of ``safety'' and ``risk''.
We will classify these notions according to their absolute or relative character, underlying risk sources, or underlying concepts of harm.

\subsection{Classification of Safety Notions}
\label{sec:safety-notions}
We will refer to \emph{absolute} notions of safety as conceptualizations that assume the complete absence of any kind of risk.
Opposed to this, \emph{relative} notions of safety are based on a conceptualization that specifically includes risk acceptance criteria, e.g., in terms of ``tolerable'' risk or ``sufficient'' safety.

For classifying notions of safety by their underlying risk (or rather ``hazard'') sources, and different concepts of harm, \Cref{fig:hazard-sources} provides an overview of our reasoning, which is closely in line with the argumentation provided by Waymo in \parencite{favaro2023}.
We prefer ``hazard sources'' over ``risk sources'', as a risk must always be related to a \emph{cause} or \emph{source of harm} (i.e., a hazard \parencite[p.~1, def. 3.2]{iso51}).
Without a concrete (scenario) context that the system is operating in, a hazard is \emph{latent}: E.g., when operating in public traffic, there is a fundamental possibility that a \emph{collision with a pedestrian} leads to (physical) harm for that pedestrian. 
However, only if an automated vehicle shows (potentially) hazardous behavior (e.g., not decelerating properly) \emph{and} is located near a pedestrian (context), the hazard is instantiated and leads to a hazardous event.
\begin{figure*}
    \includeimg[width=.9\textwidth]{hazard-sources0.pdf}
    \caption{Graphical summary of a taxonomy of risk related to automated vehicles, extended based on ISO 21448 (\parencite{iso21448}) and \parencite{favaro2023}. Top: Causal chain from hazard sources to actual harm; bottom: summary of the individual elements' contributions to a resulting risk. Graphic translated from \parencite{nolte2024} \label{fig:hazard-sources}}
\end{figure*}
If the hazardous event cannot be mitigated or controlled, we see a loss event in which the pedestrian's health is harmed.
Note that this hypothetical chain of events is summarized in the definition of risk:
The probability of occurrence of harm is determined by a) the frequency with which hazard sources manifest, b) the time for which the system operates in a context that exposes the possibility of harm, and c) by the probability with which a hazardous event can be controlled.
A risk can then be determined as a function of the probability of harm and the severity of the harm potentially inflicted on the pedestrian.

In the following, we will apply this general model to introduce different types of hazard sources and also different types of harm.
\cref{fig:hazard-sources} shows two distinct hazard sources, i.e., functional insufficiencies and E/E-failures that can lead to hazardous behavior.
ISO~21488 \parencite{iso21448} defines functional insufficiencies as insufficiencies that stem from an incomplete or faulty system specification (specification insufficiencies).
In addition, the standard considers insufficiencies that stem from insufficient technical capability to operate inside the targeted Operational Design Domain (performance insufficiencies).
Functional insufficiencies are related to the ``Safety of the Intended Functionality (SOTIF)'' (according to ISO~21448), ``Behavioral Safety'' (according to Waymo \parencite{waymo2018}), or ``Operational Safety'' (according to UN Regulation No. 157 \parencite{un157}).
E/E-Failures are related to classic functional safety and are covered exhaustively by ISO~26262 \parencite{iso2018}.
Additional hazard sources can, e.g., be related to malicious security attacks (ISO~21434), or even to mechanical failures that should be covered (in the US) in the Federal Motor Vehicle Safety Standards (FMVSS).

For the classification of notions of safety by the related harm, in \parencite{salem2024, nolte2024}, we take a different approach compared to \parencite{koopman2024}:
We extend the concept of harm to the violation of stakeholder \emph{values}, where values are considered to be a ``standard of varying importance among other such standards that, when combined, form a value pattern that reduces complexity for stakeholders [\ldots] [and] determines situational actions [\ldots].'' \parencite{albert2008}
In this sense, values are profound, personal determinants for individual or collective behavior.
The notion of values being organized in a weighted value pattern shows that values can be ranked according to importance.
For automated vehicles, \emph{physical wellbeing} and \emph{mobility} can, e.g., be considered values which need to be balanced to achieve societal acceptance, in line with the discussion of required tradeoffs in \cref{sec:terminology}.
For the analysis of the following regulatory frameworks, we will evaluate if the given safety or risk notions allow tradeoffs regarding underlying stakeholder values. 

\subsection{UN Regulation No. 157 \& European Implementing Regulation (EU) 2022/1426}
\label{sec:enabling-act}
UN Regulation No. 157 \parencite{un157} and the European Implementing Regulation 2022/1426 \parencite{eu1426} provide type approval regulation for automated vehicles equipped with SAE-Level-3 (UN Reg. 157) and Level 4 (EU 2022/1426) systems on an international (UN Reg. 157) and European (EU 2022/1426) level.

Generally, EU type approval considers UN ECE regulations mandatory for its member states ((EU) 2018/858, \parencite{eu858}), while the EU largely forgoes implementing EU-specific type approval rules, it maintains the right to alter or to amend UN ECE regulation \parencite{eu858}.

In this respect, the terminology and conceptualizations in the EU Implementing Act closely follow those in UN Reg. No. 157.
The EU Implementing Act gives a clear reference to UN Reg. No. 157 \parencite[][Preamble,  Paragraph 1]{eu1426}.
Hence, the documents can be assessed in parallel.
Differences will be pointed out as necessary.

Both acts are written in rather technical language, including the formulation of technical requirements (e.g., regarding deceleration values or speeds in certain scenarios).
While providing exhaustive definitions and terminology, neither of both documents provide an actual definition of risk or safety.
The definition of ``unreasonable'' risk in both documents does not define risk, but only what is considered \emph{unreasonable}. It states that the ``overall level of risk for [the driver, (only in UN Reg. 157)] vehicle occupants and other road users which is increased compared to a competently and carefully driven manual vehicle.''
The pertaining notions of safety and risk can hence only be derived from the context in which they are used.

\subsubsection{Absolute vs. Relative Notions of Safety}
In line with the technical detail provided in the acts, both clearly imply a \emph{relative} notion of safety and refer to the absence of \emph{unreasonable} risk throughout, which is typical for technical safety definitions.

Both acts require sufficient proof and documentation that the to-be-approved automated driving systems are ``free of unreasonable safety risks to vehicle occupants and other road users'' for type approval.\footnote{As it targets SAE-Level-3 systems, UN Reg. 157 also refers to the driver, where applicable.}
In this respect, both acts demand that the manufacturers perform verification and validation activities for performance requirements that include ``[\ldots] the conclusion that the system is designed in such a way that it is free from unreasonable risks [\ldots]''.
Additionally, \emph{risk minimization} is a recurring theme when it comes to the definition of Minimum Risk Maneuvers (MRM).

Finally, supporting the relative notions of safety and risk, UN Reg. 157 introduces the concept of ``reasonable foreseeable and preventable'' \parencite[Article 1, Clause 5.1.1.]{un157} collisions, which implies that a residual risk will remain with the introduction of automated vehicles.
\parencite[][Appendix 3, Clause 3.1.]{un157} explicitly states that only \emph{some} scenarios that are unpreventable for a competent human driver can actually be prevented by an automated driving system.
While this concept is not applied throughout the EU Implementing Act, both documents explicitly refer to \emph{residual} risks that are related to the operation of automated driving systems (\parencite[][Annex I, Clause 1]{un157}, \parencite[][Annex II, Clause 7.1.1.]{eu1426}).

\subsubsection{Hazard Sources}
Hazard sources that are explicitly differentiated in UN Reg. 157 and (EU) 2022/1426 are E/E-failures that are in scope of functional safety (ISO~26262) and functional insufficiencies that are in scope of behavioral (or ``operational'') safety (ISO~21448).
Both documents consistently differentiate both sources when formulating requirements.

While the acts share a common definition of ``operational'' safety (\parencite[][Article 2, def. 30.]{eu1426}, \parencite[][Annex 4, def. 2.15.]{un157}), the definitions for functional safety differ.
\parencite{un157} defines functional safety as the ``absence of unreasonable risk under the occurrence of hazards caused by a malfunctioning behaviour of electric/electronic systems [\ldots]'', \parencite{eu1426} drops the specification of ``electric/electronic systems'' from the definition.
When taken at face value, this definition would mean that functional safety included all possible hazard sources, regardless of their origin, which is a deviation from the otherwise precise usage of safety-related terminology.

\subsubsection{Harm Types}
As the acts lack explicit definitions of safety and risk, there is no consistent and explicit notion of different harm types that could be differentiated.

\parencite{un157} gives little hints regarding different considered harm types.
``The absence of unreasonable risk'' in terms of human driving performance could hence be related to any chosen performance metric that allows a comparison with a competent careful human driver including, e.g., accident statistics, statistics about rule violations, or changes in traffic flow.

In \parencite{eu1426}, ``safety'' is, implicitly, attributed to the absence of unreasonable risk to life and limb of humans.
This is supported by the performance requirements that are formulated:
\parencite[][Annex II, Clause 1.1.2. (d)]{eu1426} demands that an automated driving system can adapt the vehicle behavior in a way that it minimizes risk and prioritizes the protection of human life.

Both acts demand the adherence to traffic rules (\parencite[][Annex 2, Clause 1.3.]{eu1426}, \parencite[][Clause 5.1.2.]{un157}).
\parencite[][Annex II, Clause 1.1.2. (c)]{eu1426} also demands that an automated driving system shall adapt its behavior to surrounding traffic conditions, such as the current traffic flow.
With the relative notion of risk in both acts, the unspecific clear statement that there may be unpreventable accidents \parencite{un157}, and a demand of prioritization of human life in \parencite{eu1426}, both acts could be interpreted to allow developers to make tradeoffs as discussed in \cref{sec:terminology}.


\subsubsection{Conclusion}
To summarize, the UN Reg. 157 and the (EU) 2022/1426 both clearly support the technical notion of safety as the absence of unreasonable risk.
The notion is used consistently throughout both documents, providing a sufficiently clear terminology for the developers of automated vehicles.
Uncertainty remains when it comes to considered harm types: Both acts do not explicitly allow for broader notions of safety, in the sense of \parencite{koopman2024} or \parencite{salem2024}.
Finally, a minor weak spot can be seen in the definition of risk acceptance criteria: Both acts take the human driving performance as a baseline.
While (EU) 2022/1426 specifies that these criteria are specific to the systems' Operational Design Domain \parencite[][Annex II, Clause 7.1.1.]{eu1426}, the reference to the concrete Operational Design Domain is missing in UN Reg. 157.
Without a clearly defined notion of safety, however, it remains unclear, how aspects beyond net accident statistics (which are given as an example in \parencite[][Annex II, Clause 7.1.1.]{eu1426}), can be addressed practically, as demanded by \parencite{koopman2024}.

\subsection{German Regulation (StVG \& AFGBV)}
\label{sec:afgbv}
The German L3 (Automated Driving Act) and L4 (Act on Autonomous Driving) Acts from 2017 and 2021,\footnote{Formally, these are amendments to the German Road Traffic Act (StVG): 06/21/2017, BGBl. I p. 1648, 07/12/2021 BGBl. I p. 3108.} respectively, provide enabling regulation for the operation of SAE-Level-3 and 4 vehicles on German roads.
The German Implementing Regulation (\parencite{afgbv}, AFGBV) defines how this enabling regulation is to be implemented for granting testing permits for SAE-Level-3 and -4 and driving permits for SAE-Level-3 and -4 automated driving systems.\footnote{Note that these permits do not grant EU-wide type approval, but serve as a special solution for German roads only. At the same time, the AFGBV has the same scope as (EU) 2022/1426.}
With all three acts, Germany was the first country to regulate the approval of automated vehicles for a domestic market.
All acts are subject to (repeated) evaluation until the year 2030 regarding their impact on the development of automated driving technology.
An assessment of the German AFGBV and comparisons to (EU) 2022/1426 have been given in \cite{steininger2022} in German.

Just as for UN Reg. 157 and (EU) 2022/1426, neither the StVG nor the AFGBV provide a clear definition of ``safety'' or ``risk'' -- even though the "safety" of the road traffic is one major goal of the StVG and StVO.
Again, different implicit notions of both concepts can only be interpreted from the context of existing wording.
An additional complication that is related to the German language is that ``safety'' and ``security'' can both be addressed as ``Sicherheit'', adding another potential source of unclarity.
Literal Quotations in this section are our translations from the German act.

\subsubsection{Absolute vs. Relative Notions of Safety}
For assessing absolute vs. relative notions of safety in German regulation, it should be mentioned that the main goal of the German StVO is to ensure the ``safety and ease of traffic flow'' -- an already diametral goal that requires human drivers to make tradeoffs.\footnote{For human drivers, this also creates legal uncertainty which can sometimes only be settled in a-posteriori court cases.}
While UN and EU regulation clearly shows a relative notion of safety\footnote{And even the StVG contains sections that use wording such as ``best possible safety for vehicle occupants'' (§1d (4) StVG) and acknowledges that there are unavoidable hazards to human life (§1e (2) No. 2c)).}, the German AFGBV contains ambiguous statements in this respect:
Several paragraphs contain a demand for a hazard free operation of automated vehicles.
§4 (1) No. 4 AFGBV, e.g., states that ``the operation of vehicles with autonomous driving functions must neither negatively impact road traffic safety or traffic flow, nor endanger the life and limb of persons.''
Additionally, §6 (1) AFGBV states that the permits for testing and operation have to be revoked, if it becomes apparent that a ``negative impact on road traffic safety or traffic flow, or hazards to the life and limb of persons cannot be ruled out''.
The same wording is used for the approval of operational design domains regulated in §10 (1) No. 1.
A particularly misleading statement is made regarding the requirements for technical supervision instances which are regulated in §14 (3) AFGBV which states that an automated vehicle has to be  ``immediately removed from the public traffic space if a risk minimal state leads to hazards to road traffic safety or traffic flow''.
Considering the argumentation in \cref{sec:terminology}, that residual risks related to the operation of automated driving systems are inevitable, these are strong statements which, if taken at face value, technically prohibit the operation of automated vehicles.
It suggests an \emph{absolute} notion of safety that requires the complete absence of risk.  
The last statement above is particularly contradictory in itself, considering that a risk \emph{minimal} state always implies a residual risk.

In addition to these absolute safety notions, there are passages which suggest a relative notion of safety:
The approval for Operational Design Domains is coupled to the proof that the operation of an automated vehicle ``neither negatively impacts road traffic safety or traffic flow, nor significantly endangers the life and limb of persons beyond the general risk of an impact that is typical of local road traffic'' (§9 (2) No. 3 AFGBV).
The addition of a relative risk measure ``beyond the general risk of an impact'' provides a relaxation (cf. also \cite{steininger2022}, who criticizes the aforementioned absolute safety notion) that also yields an implicit acceptance criterion (\emph{statistically as good as} human drivers) similar to the requirements stated in UN Reg. 157 and (EU) 2022/1426.

Additional hints for a relative notion of safety can be found in Annex 1, Part 1, No. 1.1 and Annex 1, Part 2, No. 10.
Part 1, No 1.1 specifies collision-avoidance requirements and acknowledges that not all collisions can be avoided.\footnote{The same is true for Part 2, No. 10, Clause 10.2.5.}
Part 2, No. 10 specifies requirements for test cases.
It demands that test cases are suitable to provide evidence that the ``safety of a vehicle with an autonomous driving function is increased compared to the safety of human-driven vehicles''.
This does not only acknowledge residual risks, but also yields an acceptance criterion (\emph{better} than human drivers) that is different from the implied acceptance criterion given in §9 (2) No. 3 AFGBV.

\subsubsection{Hazard Sources}
Regarding hazard sources, Annex 1 and 3 AFGBV explicitly refer to ISO~26262 and ISO~21448 (or rather its predecessor ISO/PAS~21448:2019).
However, regarding the discussion of actual hazard sources, the context in which both standards are mentioned is partially unclear:
Annex 1, Clause 1.3 discusses requirements for path and speed planning.
Clause 1.3 d) demands that in intersections, a Time to Collision (TTC) greater than 3 seconds must be guaranteed.
If manufacturers deviate from this, it is demanded that ``state-of-the-art, systematic safety evaluations'' are performed.
Fulfillment of the state of the art is assumed if ``the guidelines of ISO~26262:2018-12 Road Vehicles -- Functional Safety are fulfilled''.
Technically, ISO~26262 is not suitable to define the state of the art in this context, as the requirements discussed fall in the scope of operational (or behavioral) safety (ISO~21448).
A hazard source ``violated minimal time to collision'' is clearly a functional insufficiency, not an E/E-failure.

Similar unclarity presents itself in Annex 3, Clause 1 AFGBV: 
Clause 1 specifies the contents of the ``functional specification''.
The ``specification of the functionality'' is an artifact which is demanded in ISO~21448:2022 (Clause 5.3) \parencite{iso21448}.
However, Annex 3, Clause 1 AFGBV states that the ``functional specification'' is considered to comply to the state of the art, if the ``functional specification'' adheres to ISO~26262-3:2018 (Concept Phase).
Again, this assumes SOTIF-related contents as part of ISO~26262, which introduces the ``Item Definition'' as an artifact, which is significantly different from the ``specification of the functionality'' which is demanded by ISO~21448.
Finally, Annex 3, Clause 3 AFGBV demands a ``documentation of the safety concept'' which ``allows a functional safety assessment''.
A safety concept that is related to operational / behavioral safety is not demanded.
Technically, the unclarity with respect to the addressed harm types lead to the fact that the requirements provided by the AFGBV do not comply with the state of the art in the field, providing questionable regulation.

\subsubsection{Harm Types}
Just like UN Reg. 157 and (EU) 2022/1426, the German StVG and AFGBV do not explicitly differentiate concrete harm types for their notions of safety.
However, the AFGBV mentions three main concerns for the operation of automated vehicles which are \emph{traffic flow} (e.g., §4 (1) No. 4 AFGBV), compliance to \emph{traffic law} (e.g., §1e (2) No. 2 StVG), and the \emph{life and limb of humans} (e.g., §4 (1) No. 4 AFGBV).

Again, there is some ambiguity in the chosen wording:
The conflict between traffic flow and safety has already been argued in \cref{sec:terminology}.
The wording given in §4 (1) No. 4 and §6 (1) AFGBV  demand to ensure (absolute) safety \emph{and} traffic flow at the same time, which is impossible (cf. \cref{sec:terminology}) from an engineering perspective.
§1e (2) No. 2 StVG defines that ``vehicles with an autonomous driving function must [\ldots] be capable to comply to [\ldots] traffic rules in a self-contained manner''.
Taken at face value, this wording implies that an automated driving system could lose its testing or operating permit as soon as it violates a traffic rule.
A way out could be provided by §1 of the German Traffic Act (StVO) which demands careful and considerate behavior of all traffic participants and by that allows judgement calls for human drivers.
However, if §1 is applicable in certain situations is often settled in court cases. 
For developers, the application of §1 StVO during system design hence remains a legal risk.

While there are rather absolute statements as mentioned above, sections of the AFGBV and StVG can be interpreted to allow tradeoffs:
§1e (2) No. 2 b) demands that a system,  ``in case of an inevitable, alternative harm to legal objectives, considers the significance of the legal objectives, where the protection of human life has highest priority''.
This exact wording \emph{could} provide some slack for the absolute demands in other parts of the acts, enabling tradeoffs between (tolerable) risk and mobility as discussed in \cref{sec:terminology}.
However, it remains unclear if this interpretation is legally possible.

\subsubsection{Conclusion}
Compared to UN Reg. 157 and (EU) 2022/1426, the German StVG and AFGBV introduce openly inconsistent notions of safety and risk which are partially directly contradictory:
The wording partially implies absolute and relative notions of safety and risk at the same time.
The implied validation targets (``better'' or ``as good as'' human drivers) are equally contradictory. 
The partially implied absolute notions of safety, when taken at face value, prohibit engineers from making the tradeoffs required to develop a system that is safe and provides customer benefit at the same time. 
In consequence, the wording in the acts is prone to introducing legal uncertainty.
This uncertainty creates additional clarification need and effort for manufacturers and engineers who design and develop SAE-Level-3 and -4 automated driving systems. The use of undefined legal terms not only makes it more difficult for engineers to comply with the law, but also complicates the interpretation of the law and leads to legal uncertainty.

\subsection{UK Automated Vehicles Act 2024 (2024 c. 10)}
The UK has issued a national enabling act for regulating the approval of automated vehicles on the roads in the UK.
To the best of our knowledge, concrete implementing regulation has not been issued yet.
Regarding terminology, the act begins with a dedicated terminology section to clarify the terms used in the act \parencite[Part 1, Chapter 1, Section 1]{ukav2024}.
In that regard, the act defines a vehicle to drive ```autonomously' if --- (a)
it is being controlled not by an individual but by equipment of the vehicle, and (b) neither the vehicle nor its surroundings are being monitored by an individual with a view to immediate intervention in the driving of the vehicle.''
The act hence covers SAE-Level-3 to SAE-Level-5 automated driving systems.

\subsubsection{Absolute vs. Relative Notions of Safety}
While not providing an explicit definition of safety and risk, the UK Automated Vehicles Act (``UK AV Act'') \parencite{ukav2024} explicitly refers to a relative notion of safety.
Part~1, Chapter~1, Section~1, Clause (7)~(a) defines that an automated vehicle travels ```safely' if it travels to an acceptably safe standard''.
This clarifies that absolute safety is not achievable and that acceptance criteria to prove the acceptability of residual risk are required, even though a concrete safety definition is not given.
The act explicitly tasks the UK Secretary of State\footnote{Which means, that concrete implementation regulation needs to be enacted.} to install safety principles to determine the ``acceptably safe standard'' in Part~1, Chapter~1, Section~1, Clause (7)~(a).
In this respect, the act also provides one general validation target as it demands that the safety principles must ensure that ``authorized automated vehicles will achieve a level of safety equivalent to, or higher than, that of careful and competent human drivers''.
Hence, the top-level validation risk acceptance criterion assumed for UK regulation is ``\emph{at least as good} as human drivers''.

\subsubsection{Hazard Sources}
The UK AV Act contains no statements that could be directly related to different hazard sources.
Note that, in contrast to the rest of the analyzed documents, the UK AV Act is enabling rather than implementing regulation.
It is hence comparable to the German StVG, which does not refer to concrete hazard sources as well.

\subsubsection{Types of Harm}
Even though providing a clear relative safety notion, the missing definition of risk also implies a lack of explicitly differentiable types of harm.
Implicitly, three different types of harm can be derived from the wording in the act.
This includes the harm to life and limb of humans\footnote{Part~1, Chapter~3, Section~25 defines ``aggravated offence where death or serious injury occurs'' \parencite{ukav2024}.}, the violation of traffic rules\footnote{Part~1, Chapter~1, Clause~(7)~(b) defines that an automated vehicle travels ```legally' if it travels with an acceptably low risk of committing a traffic infraction''}, and the cause of inconvenience to the public \parencite[Part~1, Chapter~1, Section~58, Clause (2)~(d)]{ukav2024}.

The act connects all the aforementioned types of harm to ``risk'' or ``acceptable safety''.
While the act generally defines criminal offenses for providing ``false or misleading information about safety'', it also acknowledges possible defenses if it can be proven that ``reasonable precautions'' were taken and that ``due diligence'' was exercised to ``avoid the commission of the offence''.
This statement could enable tradeoffs within the scope of ``reasonable risk'' to the life and limb of humans, the violation of traffic rules, or to the cause of inconvenience to the public, as we argued in \cref{sec:terminology}.

\subsubsection{Conclusion}
From the set of reviewed documents, the current UK AV Act is the one with the most obvious relative notions of safety and risk and the one that seems to provide a legal framework for permitting tradeoffs.
In our review, we did not spot major inconsistency beyond a missing definitions of safety and risk\footnote{Note that with the Office for Product Safety and Standards (OPSS), there is a British government agency that maintains an exhaustive and widely focussed ``Risk Lexicon'' that provides suitable risk definitions. For us, it remains unclear, to what extent this terminology is assumed general knowledge in British legislation.}.
The general, relative notion of safety and the related alleged ability for designers to argue well-founded development tradeoffs within the legal framework could prove beneficial for the actual implementation of automated driving systems.
While the act thus appears as a solid foundation for the market introduction of automated vehicles, without accompanying implementing regulation, it is too early to draw definite conclusions.
   

%     $R_0>1$ alone is not sufficient for the disease spread. 
%     \begin{conjecture}
%     A necessary condition for the disease spread is that the Poisson random connection model of $\Psi$ with connection function $g$ percolates.
%     \end{conjecture}
% If $\Phi$ percolate, then $\Psi$ percolate? 
%We say that two hubs are connected if they connect to a common node... This leads to a connection function $g(r)$. Does the percolation of $\Psi$ with $g(r)$ indicate the percolation of $\Phi$ with $f$? 
% tThe probability that infinite nodes get infected is 0 if the largest connected component of the random connection model of $\Psi$ associated with $g$ is finite (to be proved).


\subsection{Poisson Point Processes}
\begin{figure*}[t]
\normalsize
\setcounter{MYtempeqncnt}{\value{equation}}
\setcounter{equation}{\value{equation}+2}
 \begin{align}
       \mathbb{V} N
       & = \Ex N+\lambda^2\mu\left(\int_{0}^{\infty}f(r)dc_d r^{d-1} \dd r\right)^3 \nonumber%\int_{\mathbb{R}^2}\int_{\mathbb{R}^2}\int_{\mathbb{R}^2}f(\|y\|)f(\|x_1-y\|)f(\|x_2-y\|) \dd y\dd x_1 \dd x_2\nonumber\\
       \\
       &\quad+\lambda\mu^2 \int_{\mathbb{R}^d}\int_{\mathbb{R}^d}\int_{\mathbb{R}^d}f(\|y_1\|)f(\|y_2\|) f(\|x-y_1\|)f(\|x-y_2\|) \dd y_1\dd y_2\dd x.\label{eq: E_N_o^2}
      % &\quad+\lambda^2\mu^2 \int_{\mathbb{R}^2}\int_{\mathbb{R}^2}\int_{\mathbb{R}^2}\int_{\mathbb{R}^2}f(\|y_1\|)f(\|y_2\|) f(\|x_1-y_1\|)f(\|x_2-y_2\|) \dd x_1 \dd x_2 \dd y_1\dd y_2
    \end{align}
    \setcounter{equation}{\value{MYtempeqncnt}}
\hrulefill
\vspace*{4pt}
\end{figure*}
Here we focus on the case where $\Psi$ and $\Phi$ are two independent PPPs in $\mathbb{R}^d$, defined on the same probability space. We derive the mean and variance of $N$ and $M$. For the PPP,
\[
{\rm P}_{\Phi}^{!o} = \rm P_{\Phi}.
\]
This is known as Slivnyak's Theorem  \cite{haenggi2012stochastic} and leads to  the result below.

%With a slight abuse of notation, we write $\Phi$ instead of $\Phi_o^{!}$ in this section for simplicity. 
   
% For simplicity, we assume $K\equiv1$, $i.e.,$ an infected node is infectious for one slot and is removed afterwards. 
% The relevant parameters are the densities $\lambda,~\mu$, the rate(s) of visiting a hub $(\gamma,\theta)$, and the rate of infection $\alpha$. We first analyze the distribution of the number of connections. Then we derive the basic reproduction number of the model.

%     \end{equation}
%   Its expectation is 
% The moment-generating function of $N_o$ is 
%           \begin{align}
%          \mathcal{G}_{N_o}(s) & = \Ex s^ {N_o}\\
%          & = \Ex s^{\sum_{y\in\Psi}\sum_{x \in\Phi} I(o,y)I(x,y)}\\
%          & = \Ex \prod_{y\in\Psi}\prod_{x \in\Phi} s^{I(o,y)I(x,y)}\\
%          & \peq{a} \Ex\prod_{y\in\Psi}\prod_{x\in\Phi} \gamma^2 s\exp(-\theta\|y\|-\theta\|x-y\|)+1-\gamma^2 \exp(-\theta\|y\|-\theta\|x-y\|)\\
%          & \peq{b} \Ex \prod_{y\in\Psi} \exp\bigg(-\int_{0}^{\infty}2\pi\lambda  r  (1-s)\gamma^2 \exp(-\theta\|y\|)\exp(-\theta r)\dd r\bigg)\\
%          & = \Ex \prod_{y\in\Psi} \exp\big(-2\pi\lambda (1-s)\gamma^2\theta^{-2} \exp(-\theta \|y\|)\big)\\
%          & \peq{c} \exp\bigg(-\int_{0}^{\infty}2\pi\mu r \Big(1-\exp\big(-2\pi\lambda (1-s) \gamma^2\theta^{-2} \exp(-\theta r)\big)\Big)\dd r\bigg)\label{eq: LaplaceN}
%      \end{align}
%   where  step (a) follows from the distribution of the binary random variable $I(o,y)I(x,y)$. Step (b)  and (c) follow from the PGFL of the PPP. One can obtain $\Ex N_o = (\mathrm{d}/\mathrm{d}s)\mathcal{G}(s)|_{s=1}$, Or directly b
%Recall that $N_o = \sum_{y\in\Psi}\sum_{x \in\Phi_o^!} I(o,y)I(x,y)$ and $M_o =\sum_{x \in\Phi_o^!} \max_{y\in\Psi}I(o,y)I(x,y)$.     
    \begin{theorem}
    \label{thm: N,M}
For the PPP, the means of $M,~N$ are
\begin{equation}
  \Ex N = \lambda\mu \left(\int_{0}^{\infty}f(r)dc_d r^{d-1} \dd r\right)^2,        \label{eq: EN}
\end{equation}
and
 \begin{equation}
     \Ex M = \lambda\int_{\mathbb{R}^d}  1-\exp\left(-\mu\int_{\mathbb{R}^d}f(\|y\|)f(\|x-y\|)\dd y\right)\dd x.\label{eq: EM}
 \end{equation}
 Further, the variance of $N$ is given in (\ref{eq: E_N_o^2}),
 and the variance of $M$ satisfies $\mathbb{V} M \geq \Ex M$.
\end{theorem}
% \begin{figure*}[t]
% \normalsize
% \setcounter{MYtempeqncnt}{\value{equation}}
% \setcounter{equation}{\value{equation}}
%      \begin{align}
%             \Ex M_o^2 
%             & = \Ex M_o +2\lambda^2  \int_{\mathbb{R}^2}\int_{\mathbb{R}^2} 1- \exp\left(-\mu\int_{\mathbb{R}^2}f(\|y\|)f(\|x_1-y\|)\right)-\exp\left(-\mu\int_{\mathbb{R}^2}f(\|y\|)f(\|x_2-y\|)\dd y\right)\nonumber\\   &\quad+\exp\left(-\mu\int_{\mathbb{R}^2}f(\|y\|)(f(\|x_1-y\|)+f(\|x_2-y\|))-f(\|y\|)^2f(\|x_1-y\|)f(\|x_2-y\|)\dd y\right)\dd x_1 \dd x_2.
%               \label{eq: E_M_o^2}
%             \end{align} 
%             \setcounter{equation}{\value{MYtempeqncnt}+1}
% \hrulefill
% \vspace*{4pt}
% \end{figure*}
\begin{proof}
See Appendix \ref{appendix: N,M}.
   \end{proof}
   \begin{remark} 
       $\Ex M = {\Theta}(\mu)$\footnote{We use $f(x)=\Theta(g(x))$ as $x\to a$ to denote that $f(x)=O\left(g(x)\right)$ and $ g(x)=O(f(x))$ as $x\to a$.} as $\mu\to0$, since $1-\exp(-\mu) = {\Theta}(\mu)$ as $\mu\to0$.
   \end{remark}
 \begin{remark}
Eq. (\ref{eq: EN}) holds for general hub point process $\Psi$. The other equations in Theorem 1 rely on the fact that $\Psi$ is a PPP.  In (\ref{eq: EM}), $\int_{\mathbb{R}^d}f(\|y\|)f(\|x-y\|)\dd y$ only depends on $\|x\|$, as the rotation of $x$ is equivalent to rotating $y$, which has no effect on the integral. This observation is used in the next subsection.
 \end{remark} 
 \begin{remark}
  Both $N$ and $M$ are super-Poisson since $\Var N\geq \Ex N$, and $\Var M\geq \Ex M$. In (\ref{eq: E_N_o^2}), note that in the second term the integral to the power of 3 is larger than the triple integral in the third term. Hence, for fixed $\lambda\mu$, there exists a threshold of $\lambda$ above which increasing $\lambda/\mu$ increases the variance of $N$.
  %When $\gamma\theta^2$ is fixed, increasing $\theta$ increases the variance of $\M_o$ but decreases the variance of $N_o$.
 \end{remark}


 





 
   
%   \begin{corollary}
% The probability that an arbitrary location $x$ is connected with $o$ averaged over $\Psi$ is 
% \begin{equation}
% \Pr(\max_{y\in\Psi} I(o,y)I(x,y) = 1)  = 1-\exp\left(-\frac{\mu\pi}{4}\gamma^2 \|x\|^2\left(\frac{2K_1(\theta  \|x\|)}{\theta \|x\|}+K_0\big(\theta  \|x\|\big)\right)\right).
% \label{eq: p_x_connect}
% \end{equation}
%   \end{corollary}
%      The moment generating function of $M_o$ is 
%           \begin{align}
%          \mathcal{G}_{M_o}(s) & = \Ex s^{M_o}\\
%          & = \Ex s^{\sum_{x\in\Phi} \max_{y\in\Psi} I(o,y)I(x,y)}\\
%          & = \Ex \prod_{x \in\Phi} s^{\max_{y\in\Psi} I(o,y)I(x,y)}\\
%       %  & \peq{a} \Ex\prod_{x\in\Phi} \left(s (1- \prod_{y\in\Psi}(1- \gamma^2 \exp(-\theta\|y\|-\theta\|x-y\|)))+\prod_{y\in\Psi}(1-\gamma^2 \exp(-\theta\|y\|-\theta\|x-y\|))\right)\\
%       & = \Ex \prod_{x\in\Phi} s\Pr(o \leftrightarrow x\mid\Psi) + 1-\Pr(o \leftrightarrow x\mid\Psi)\\
%       & = \Ex \prod_{x\in\Phi} s +(1-s)\prod_{y\in\Psi}\left(1-\gamma^2\exp(-\theta\|y\|-\theta\|x-y\|)\right)\\
%          & \peq{a} \Ex\prod_{x\in\Phi}  s+(1-s)\exp\left(-\frac{\mu\pi}{4}\gamma^2 \|x\|^2\left(\frac{2K_1(\theta  \|x\|)}{\theta \|x\|}+K_0\big(\theta  \|x\|\big)\right)\right)\\
%      %    & = \exp\left(-\int_0^{\infty} 2\pi\lambda v(1-s)\exp(-\int_0^{\infty}\lambda \gamma^2\exp(-\theta r)\dd r)\right)\\
%          & \peq{b}\exp\left(-\int_0^{\infty} 2\pi\lambda v(1-s)\left(1-\exp\left(-\frac{\mu\pi}{4}\gamma^2 v^2\left(\frac{2K_1(\theta v)}{\theta v}+K_0\big(\theta  v\big)\right)\right)\right)\right).
%          \label{eq: gf_M}
%      \end{align}
%     Step (a) follows from Corollary 1. Step (b) follows from the PGFL of the PPP.
 
%          So $M_o$ is Poisson distributed with mean
    
    
    
     %  \hl{Note the relation between $M_o$ and $R_0$}
%   \begin{corollary}
%   The basic reproduction number of the proposed model is 
%   \begin{equation}
%       R_0 =\int_{0}^{\infty} \frac{2\pi\lambda v}{\theta^2} \left(1-\exp\left(-\frac{\mu\pi}{4\theta^2}\alpha \gamma^2 v^2\left(\frac{2K_1( v)}{v}+K_0\big( v\big)\right)\right)\right) \dd v.
%       \label{eq: R_0}
%   \end{equation}
% %   where $\lambda_x(r)$ is given in Lemma 1.
%   \end{corollary}
%   \begin{proof}

%      We denote the probability of a node $x$ of distance $r=\|x\|$ getting infected from $o$ as $g(r)$.  First, we observe that $g(r)  = \Ex \ind(Z_x^1=\mathrm{I})$ can be obtained by changing the connection function $f(r)$ to $\sqrt{\alpha}f(r)$ in $\Ex M_o= \Ex \max_{y\in\Psi} I(o,y)I(x,y)$. This changes $\gamma^2$ to $\alpha\gamma^2$  in (\ref{eq: p_x_connect}). Thus 
%      \begin{equation}
%      g(r) = 1-\exp\left(-\frac{\mu\pi}{4}\alpha\gamma^2 r^2\left(\frac{2K_1(\theta r)}{\theta r}+K_0\big(\theta  r\big)\right)\right),\quad r>0.
%      \label{eq: g_r_exp}
%      \end{equation}
%   Using the reduced Palm expectation, Campbell's theorem, and substituting $\theta r$ by $v$ lead to (\ref{eq: R_0}).
%       \end{proof}


%       \begin{remark}
%          \begin{equation}
% \begin{split}
%   R(\Phi,\Psi) 
%      & =\sum_{x\in\Phi}  1- \prod_{y\in\Psi} \left(1-\alpha\gamma^2\exp\big(-\theta (\|y\|+\|x-y\|)\big)\right). \nonumber
% \end{split}
%     \end{equation}
%     Comment on its distribution
%       \end{remark}
      
     
            

      
%Fig. \ref{fig: pdf_R_0} shows the distribution of $R_0(\Phi,\Psi)$. Fig. \ref{fig: g_r_exp} shows the probability of infection at distance $r$.




%       \begin{corollary}
% For $K>0$, fix $\Ex N_o = K$.   Let $\rho\triangleq(2\pi)^2\lambda/\mu$.  $\Ex M_o$ depends on $\lambda$ and $\mu$ only through $\rho$. Further, $\Ex M_o$ monotonically increases with $\rho$, $\gamma$, and $K$.
%       \end{corollary}
%       \begin{proof}
%       $\Ex N_o = K$ implies  \begin{equation}
%       \theta^4 = K/(\gamma^2\mu^2{\rho}).
%       \label{eq: ratio}
%       \end{equation}
%       Here, the factor $1/(2\pi)^2$ is added to simplify the expressions below. Substituting $\theta$ by (\ref{eq: ratio})  and substituting $\sqrt{\mu} v$ by $v$ in Eq (\ref{eq: E M_0 exp}), we have
%       \begin{equation}
%           \Ex M_o =  \int_{0}^{\infty} \frac{\sqrt{\rho K}v}{2\pi\gamma} \left[1-\exp\left(-\frac{\pi\sqrt{K}}{4\sqrt{\rho}} \gamma v^2\left(\frac{2K_1( v)}{ v}+K_0\big( v \big)\right)\right)\right] \dd v,        \label{eq: R_0_simple}
%           %\int_{0}^{\infty} \frac{\rho v}{2\pi} \left[1-\exp\left(\frac{\pi}{4}\sum_{m=1}^{K}{K \choose m}(-t \gamma^2)^m v^2\left(\frac{2K_1( mv \sqrt{\gamma} \sqrt[\leftroot{-2}\uproot{2}4]{\rho/M})}{ mv\sqrt{\gamma} \sqrt[\leftroot{-2}\uproot{2}4]{\rho/M}}+K_0\big(m v \sqrt{\gamma} \sqrt[\leftroot{-2}\uproot{2}4]{\rho/M} \big)\right)\right)\right] \dd v,        \label{eq: R_0_simple}
%       \end{equation}
%       which does not depend on $\mu$. The monotonicity of $R_0$ wrt $K$ is trivial once we observe that the exponent is negative.  The monotonicity of $\Ex M_o$ wrt $\gamma,\rho$ is to be proved.
%       \end{proof}

      
      
      
      
%       \begin{corollary} 
%           For $\Ex N_o =K$, we have $\Ex M_o\leq K$.  The equality is achieved when $\rho\to\infty$ or $\gamma\to0$. $\int_{0}^{\infty}x^{n-1}K_m(x)\dd x = 2^{n-2}\Gamma(\frac{n-m}{2})\Gamma(\frac{n+m}{2}),~n> m.$
%           \end{corollary}
%           \begin{proof}
%           Taking the limit of (\ref{eq: R_0_simple}) when $\rho\to\infty$ or $\gamma\to0$ yields the upper bound. 
%           \end{proof}
          
    %   For a given set of $\lambda,\mu,\alpha$, let  $\gamma\to0$ while \begin{equation}\frac{\gamma}{\theta^2}=\frac{1}{2\pi}\sqrt{\frac{M}{\lambda\mu}}
    %   \label{eq: ratio}
    %   \end{equation}
    %   such that $\Ex N =M$, we have $
    %       \lim_{\gamma\to0}R_0 = C.$
    %   For a given set of $\lambda,\alpha,\gamma$, let  $\mu\to0$ while $\Ex N =M$ as in (\ref{eq: ratio}), we have $ \lim_{\gamma\to0} R_0 = C.$
       
    %  \begin{remark}
    %  From Corollary 2, when fixing the average number of hubs visited for the typical user, removing the location-dependence is equivalent to decreasing the hub density. The limiting case is where all nodes are uniformly mixed. On the other hand, when $\rho\to0$, $\Ex M_o\to 0$ because the number of distinct nodes decrease.
    %  \end{remark} 
      
     % Fig. \ref{fig: R_0} plots $R_0$ versus $\gamma$ for a set of $\bar\rho\triangleq\rho/(2\pi)^2$. As the density ratio $\lambda/\rho$ increases or the preference for nearby connections $\gamma$ decreases, $\Ex M_o$ increases. The upper bound $tM$ is achieved when $\gamma=0$ or $\bar\rho\to\infty$.
%   \begin{figure}
%       \centering
%       \includegraphics[width=0.5\textwidth]{R0vsmu2.eps}
%       \caption{Simulation results of $R_0$ using \eqref{eq: R_0}. }
%       \label{fig: R_0}
%   \end{figure}
 
% \begin{equation}
%       R_0\approx p(1-\mathcal{L}_\alpha(N))/(1-p)(1-\exp(-\beta))
%       \end{equation}
    
    %  \subsection{Speed of spread}
%       \begin{figure}
%     \begin{subfigure}[b]{0.45\textwidth}
%       \includegraphics[width=\textwidth]{var_R_0.eps}
%       \caption{$f_1$.}
%       \end{subfigure}
%       \hfill
%     \begin{subfigure}[b]{0.45\textwidth}
%       \includegraphics[width=\textwidth]{var_R_0.eps}       \caption{$f_2$.}
%       \end{subfigure}
%       \caption{Mean and variance of $M_o$.}
%       \label{fig: mean_var_M_o}
%   \end{figure}   
   
  

%The probability that the disease dies out is lower-bounded by 
%\begin{equation}
%    \Ex \prod_{x\in\Phi}(1-p_x)
%\end{equation}
%    \subsection{Clustering (tentative)}
%    \hl{Need the definition of clustering. Assuming the disease spreads from a single infected user, this seems a straightforward result as the result of the monotonicity of $f$... Connection with $g(r)$?}
%    
%If at the beginning of the spread, nodes are infected iid with probability $p$, $i.e.,$ both susceptible and infected nodes are PPPs with intensity $(1-p)\lambda$ and $p\lambda$, respectively. The probability of the typical susceptible node getting infected at the end of the first slot immediately follows from Eq (\ref{eq: LaplaceN}) by replacing $\lambda$ with $p\lambda$.   The probability of the typical (susceptible) node getting infected is 
%     \begin{align}
%         \Pr(Z^{1} = \mathrm{I} \mid Z^{0} = \mathrm{S}) &= \Ex [1-\exp(-\alpha N)] \\
%         & = 1-\mathcal{L}_{N}(\alpha).\label{eq: StoI-PPP}
%     \end{align}
%By ergodicity, it is equivalent to the fraction of susceptible nodes infected in a single realization.
%  
%    If the locations of infected nodes followed a PPP throughout the disease spread, then the system is fully characterized by Eq (\ref{eq: StoI-PPP}). In other words, given the parameters and starting condition, the transition probability between compartments are known. However, given the geometric dependence, infections likely cause a clustering effect around higher-density hubs and infected nodes. As a result, there are fewer infected nodes around susceptible ones, leading to a slow down of spread. 
%    


  \begin{figure*}[t]
       \centering
       \subfigure[Mean and the standard deviation of $N$. ]{           \psfrag{EN}{$\mathbb{E}N$}
       \psfrag{VN, lambda = 5, mu=50}{$\sqrt{\mathbb{V}N},\lambda=5,\mu=50$}
              \psfrag{VN, lambda = 50, mu=5}{$\sqrt{\mathbb{V}N},\lambda=50,\mu=5$}

\includegraphics[width=.42\textwidth]{fig/N_o_disk_2.eps}
       %  \caption{$f(r) = \exp(-9.426r)$}
         \label{fig: N_o}}
     \hfill
     \subfigure[Mean and the standard deviation of $M$. ]{ 
     \psfrag{EM, rho = 01}{$\mathbb{E}M,\lambda=5,\mu=50$}
          \psfrag{EM}{$\mathbb{E}M,\lambda=50,\mu=5$}
          \psfrag{sqrt{{V}M}, rho = 0111}{$\sqrt{\mathbb{V}M},\lambda=5,\mu=50$}
  \psfrag{sqrt{{V}M}, rho = 10102}{$\sqrt{\mathbb{V}M},\lambda=50,\mu=5$}
\includegraphics[width=.42\textwidth]{fig/M_o_disk_3.eps}
         \label{fig: illu-g}}
           \caption{Mean and standard deviation of $N$ and $M$ for $f =\ind{(r\leq 0.2122)}$. We compare $(\lambda,\mu) = (5,50), (50,5)$ respectively. $\Ex N=5$. $\sqrt{\Var{M}}$ is obtained via simulation whereas the rest via Corollary 2.}
         \label{fig: N_oM_o_disk}
   \end{figure*} 



%Observe that for $p\in(0,1]$ replacing $\mu$ by  $\mu/p^2$ and $f(r)$ by $pf(r)$ does not change $\Ex M_o$. In contrast, replacing $\lambda$ by  $\lambda/p^2$ and $f(r)$ by $pf(r)$ increases $\Ex M_o$.
  \begin{corollary}
  \label{cor: dispersed-mean-degree}
   For all $f$, $\Ex^{} M$ for $f_p$ monotonically decreases with $p$. For all $f$, $\Ex^{} M$ monotonically increases with $\lambda/\mu$ for fixed $\lambda\mu$. Further,
   
 \setcounter{equation}{\value{MYtempeqncnt}+3}
  \begin{equation}
 \lim_{p\to 0} \Ex^{} M = \Ex^{} N.
  \end{equation}
  \begin{equation}
 \lim_{\frac{\lambda}{\mu}\to \infty} \Ex^{} M = \Ex^{} N.
  \end{equation}
 \end{corollary}
 \begin{proof}
Let us write $\Ex^{} M$ for $f_p$  as
 \begin{equation}
\begin{split}\nonumber
   &\Ex^{} M\\
   & = \lambda\int_{\mathbb{R}^d}  1-\exp\left(-\mu p^2\int_{\mathbb{R}^d}f(\sqrt[d]{p}\|y\|)f(\sqrt[d]{p}\|x-y\|)\dd y\right)\dd x\\
   &  \peq{a} \frac{\lambda}{p}\int_{\mathbb{R}^d}   1-\exp\left(-\mu p\int_{\mathbb{R}^d}f(\|v\|)f(\|u-v\|)\dd v\right)\dd u.
   %\\
   %& = \lambda\int_{\mathbb{R}^2}  1-\exp\left(-\mu p\int_{\mathbb{R}^2}f(\|y\|)f(\|\sqrt{p}x-y\|)\dd y\right)\dd x
 %& = \int_{\mathbb{R}^2} \frac{\lambda}{p} \left( 1-\exp\left(-\mu p\int_{\mathbb{R}^2}f(\|y\|)f(\|x-y\|)\dd y\right)\right) \dd x
 \end{split}
 \end{equation}
% Taking the derivative of the last equation with respect to $p$,
% \begin{equation}
%     \frac{\dd \Ex M_o}{\dd p}  = \lambda \int_{\mathbb{R}^2}  \mu\sqrt{p} \int_{\mathbb{R}^2}f(\|y\|)f'(\|\sqrt{p}x-y\|)\dd y \exp\left(-\mu p\int_{\mathbb{R}^2}f(\|y\|)f(\|\sqrt{p}x-y\|)\dd y\right)\dd x,
% \end{equation}
% is negative by the monotonicity of $f$. 
Step (a) follows from change of variables $u=\sqrt{p}x,~v=\sqrt{p}y$.
The last equation monotonically decreases with the increase of $p$ since $(1-\exp(-C x))/x$ monotonically decreases with $x$.
Further, this shows that for $\Ex M$, stretching $f$ to $f_p$ is equivalent to scaling the network by $\lambda' = \lambda/p$ and $\mu'=\mu p$.
As $p\to0$, all terms in the series expression of the integrand containing $p$ go to 0, hence the convergence to $\Ex N$.
 \end{proof}

\begin{remark}Recall that $N=M$ when each pair of agents are connected through at most one hub. 
 For fixed $\lambda\mu$, as $\lambda/\mu\to\infty$, the number of hubs for the typical agent decreases per (\ref{eq: EN_hub}). Intuitively, most agents only have an edge to their closest hub. 
 On the other hand, as $p\to0$, two agents in proximity are less unlikely to be connected through common hubs, and geometry has a diminishing impact on connections. The graph eventually becomes an arbitrarily large abstract random graph.  Since $N-M$ is a non-negative integer, Corollary 1 also implies the convergence of $M$ to $N$ in probability.
\end{remark}
% \begin{remark}
%     A similar result can be derived when considering the mean number of hubs connected with hubs and letting $\mu/\lambda\to\infty$. We omit its discussion due to its symmetry to the described result as well as its lack of physical relevance.
% \end{remark}
 
 
 \subsection{Examples}
   In this subsection, we consider an example with $d=2$ and two special connection functions. 
    
    The first is\begin{equation}
  \nonumber  f_1(r) = \ind{(r\leq \theta)},\quad r\geq 0,\label{eq: disk}
      \end{equation} 
      where $\theta>0$ is the cut-off radius for drawing an edge between a agent and a hub. Two agents potentially have common hubs only if the intersection of their disks of radius $\theta$ are nonempty regardless of $\Psi$.
      
The second is
   \begin{equation}
\nonumber        f_2(r) =\frac{1}{2}\exp(-r/\theta),\quad r\geq 0.
\label{eq: exp dist}
\end{equation}
which has infinite support.
 


For both connection functions, the degree of the typical agent is Poisson distributed with mean 
    \begin{align}
      \Ex \sum_{y\in\Psi}I(o,y)&= \int_{0}^{\infty}\exp(- r/\theta)\pi\mu r \dd r\nonumber\\
      &=\int_{0}^{\infty} 2\pi\mu r \ind{(r\leq \theta)} \dd r\nonumber\\
      &= {\pi\mu}{\theta^2}.\nonumber\label{eq: EN_hub}
    \end{align}
%  The mean total length of the edge of the typical agent is
% \begin{align}
%       \Ex D_o = \int_{0}^{\infty} \exp(-r/\theta ) 2\pi\mu r^2 \dd r= {2\pi\mu}{\theta^3}.\label{eq: EN_hub}
%     \end{align}
    % and
    % \begin{align}
    %   \Ex D_o = \int_{0}^{\infty} \ind{(r\leq \theta)}2\pi\mu r^2 \dd r= \frac{2\pi\mu \theta^3}{3}.
    % \end{align}

   
\begin{corollary}
\label{cor: examples-N-M}
 For $f_1(r)= \ind{(r\leq \theta)}$, $\Ex N = \lambda\mu\pi^2\theta^4$, and 
 \begin{equation}
        \Ex M 
         = \int_{0}^{2\theta}2\pi\lambda v\left(1-e^{-\mu (2\theta^2\arccos(v/2\theta)-v\sqrt{\theta^2-v^2/4})}\right)\dd v.
        \label{eq: E M_0 disk}
\end{equation}
For $f_2(r)= \exp(-r/\theta)/2$,  $\Ex N = {\lambda\mu\pi^2}{\theta^4},$
and
\begin{align}
        \Ex M = \int_{0}^{\infty}2\pi\lambda v\left(1-e^{-\frac{\mu\pi v}{16}{({2\theta K_1(v/\theta)}+vK_0(v/\theta)})}\right)\dd v.\label{eq: E M_0 exp}
        \end{align}
% \begin{figure*}[t]
% \normalsize
% \setcounter{MYtempeqncnt}{\value{equation}}
% \setcounter{equation}{\value{equation}}
% \begin{align}
%         \Ex M_o 
%         & = \int_{0}^{\infty}2\pi\lambda v\left(1-\exp\left(-\frac{\mu\pi}{16} v^2\left(\frac{2\theta K_1(v/\theta)}{v}+K_0(v/\theta  \big)\right)\right)\right)\dd v.\label{eq: E M_0 exp}
%         \end{align}
%         \setcounter{equation}{\value{MYtempeqncnt}+1}
% \hrulefill
% \vspace*{4pt}
% \end{figure*}
where $K_n(\cdot)$ is the modified Bessel function of the second kind.



 
% \begin{figure*}[t]
% \normalsize
% \setcounter{MYtempeqncnt}{\value{equation}}
% \setcounter{equation}{\value{equation}}
% \begin{align}
%         \Ex M_o 
%         & = \int_{0}^{2\theta}2\pi\lambda v\left(1-\exp\left(-\mu \big(2\theta^2\arccos(v/2\theta)-v\sqrt{\theta^2-v^2/4}\big)\right)\right)\dd v.
%         \label{eq: E M_0 disk}
% \end{align}
% \setcounter{equation}{\value{MYtempeqncnt}+1}
% \hrulefill
% \vspace*{4pt}
% \end{figure*}
\end{corollary}    
\begin{proof}
For $f_1$, note that only agents within radius $2\theta$ can be connected to the typical agent $o$. Then the expression follows from the area of the intersection of two disks with radius $\theta$. For $f_2$, see Appendix \ref{appendix: examples-N-M}. 
\end{proof}
\begin{remark}
The integrand in $\Ex M$ has bounded support if and only if the connection function $f$ has bounded support, in which case its support is twice the support of $f$. 
\end{remark}

The numerical evaluations of (\ref{eq: E M_0 disk})  and (\ref{eq: E M_0 exp}) take a few millisecond using Matlab. $\Ex M$ increases linearly with $\lambda$ and sublinearly with $\mu$. As $\mu\to0$, $\Ex M$ is a linear function of $\mu$ asymptotically (see Remark 1). 
Fig. \ref{fig: N_oM_o_disk} plots the mean and variance of $N,~M$ for a fixed $\Ex N$ and $f_p(r)= pf(\sqrt{p}r)$, $p\in(0,1]$. 


      