
    \section{Degrees of the RBG Graph}
In this section, we consider some basic properties of the RBG graph $\mathcal{G}(\Phi,\Psi, f)$. Let $\Phi^o\triangleq \Phi\mid o\in\Phi.$ We focus on the typical agent at the origin and the Palm distribution of $\Phi$
\begin{equation}
 \mathrm{P}_{}^{o}(\cdot) \triangleq \Pr(\Phi\in\cdot \mid o\in\Phi),\nonumber
\end{equation}
and the reduced Palm distribution
\begin{equation}
 \mathrm{P}_{}^{!o}(\cdot) \triangleq \Pr(\Phi\setminus\{o\}\in\cdot\mid o\in\Phi).\nonumber
\end{equation}
We denote by $\Ex_{}^{{o}}$ the  expectation with respect to the Palm measure of $\Phi$
   and by $\Ex_{}^{!o}$ the  expectation with respect to the reduced Palm measure of $\Phi$ \cite[Chapter 8]{haenggi2012stochastic}.
   %of $\Phi$, i.e., $\Ex^{!}_{{o}}f(\Phi) \triangleq \Ex (f(\Phi\setminus\{o\})\mid o\in\Phi).$
    




%There are several quantities of interest and connection functions: the degree (the number of edges) of the typical node/hub, the number of connections of the typical node, and the number of connected nodes of the typical node.
\subsection{General Point Processes}
%To simplify the notation, let $\Ex^{!o}$ denote the (total) expectation where the expectation with respect to $\Phi$ is its reduced Palm expectation.

Let $c_d$ denote the volume of unit ball in $\mathbb{R}^d$.
The mean degree of the typical agent in a general point process is 
%  $\sum_{y\in\Psi} \ind(o\to y)$ with mean 
    \begin{align}
  \Ex^{o} \sum_{y\in\Psi}I(o,y)
       &\peq{a} \Ex \sum_{y\in\Psi}f(\|y\|)\nonumber
      \\
      &\peq{b} \mu c_d d \int_{0}^{\infty}f(r) r^{d-1} \dd r,\label{eq: EN_hub}
    \end{align}
    where step (a) follows from the independence of $\Phi$ and $\Psi$, and step (b) follows from Campbell's theorem. 
    By the mass transport principle \cite{baccelli:hal-02460214}, the mean degree of the typical hub (defined through the Palm expectation of $\Psi$) is $\lambda c_d d \int_{0}^{\infty}f(r) r^{d-1} \dd r.$ %Similarly, the sum length of the edges of the typical node is given by $ \Ex \sum_{y\in\Psi} \|y\|I(o, y)= \mu c_d d \int_{0}^{\infty}f(r) r^{d} \dd r.$
    For example, consider $d=2$, a general connection function $f$, and its dispersed version $f_{p}$. While the mean degree remains constant for any $p\in(0,1]$,
%, i.e., \[\int_{0}^{\infty}f_p(r)2\pi r \dd r \equiv \int_{0}^{\infty}f(r)2\pi r \dd r.\]
 the edges spreads out in space as $p\to 0$. To see this, consider the mean distance
 \begin{align}
 \Ex^{o} \sum_{y\in\Psi}I(o,y)\|y\| &=\int_{0}^{\infty}f_p(r)2\pi r^2 \dd r\nonumber \\
 & = \frac{1}{\sqrt{p}}\int_{0}^{\infty}f(r)2\pi r^2 \dd r,\nonumber
 \end{align}
which increases as the connection function becomes more dispersed.


    
 Let $M\triangleq\sum_{x\in\Phi^o\setminus\{o\}}\max_{y\in\Psi} I(o,y)I(x,y)$ denote the number of agents connected to $o$.  We have
    \begin{align}
        \Ex M &= \Ex\sum_{x\in\Phi^o\setminus\{o\} } 1-\prod_{y\in\Psi} \left(1-I(o,y)I(x,y)\right) \nonumber\\
        &= \Ex^{!o} \sum_{x\in\Phi} 1-\prod_{y\in\Psi} (1-f(\|y\|)f(\|x-y\|)).\nonumber
        \end{align}
    This follows from the fact that the edge indicator functions are independent Bernoulli random variables given $\Phi$ and $\Psi$.
    
    Let $N\triangleq\sum_{y\in\Psi}\sum_{x \in\Phi^o\setminus\{o\}} I(o,y)I(x,y)$ denote the total number of direct (two-edge) paths between $o$ and its connected agents. We have
    \begin{align}
       \Ex N  &= \Ex \sum_{y\in\Psi}\sum_{x \in\Phi^o\setminus\{o\}} I(o,y)I(x,y) \nonumber\\
       &= \Ex^{!o}\sum_{y\in\Psi}\sum_{x \in\Phi}f(\|y\|)f(\|x-y\|).\nonumber
       %\\ \nonumber & =  \Ex_{\Psi} \sum_{y\in\Psi}f(\|y\|) \Ex^{!}_{{o}}\sum_{x \in\Phi} f(\|x-y\|),
    \end{align}  
    

 
    
By definition, $N\geq M$, and the equality is achieved when $o$ and $x$ are connected through at most one hub, $\forall x\in\Phi$. The difference between $N$ and $M$ reflects how likely two agents connect through multiple hubs. The variance of $M$ is $\mathbb{V} M \triangleq \Ex M^2 - (\Ex M)^2$, which characterizes the heterogeneity of the number of connected agents.
   
   
    
    % \subsection{The Basic Reproduction Number}
    %   The basic reproduction number $R_0$ is widely used in epidemiology to predict whether an infectious disease will spread. However, it is difficult to calculate in general and its effectiveness and interpretation rely on the model \cite{Li2011TheFO}. While $R_0$ captures the initial state of the disease propagation, it may not predict the spread in a global level. For instance, consider a model where each node always connects with its nearest hub and $\alpha=1$. Then $R_0=N-1$ assuming each hub on average connects to $N$ nodes. However, due to the locality of connections, the disease only spreads within nodes connected to the same hub and always dies out for an arbitrarily large $N$. 
       
%      \begin{definition}[The basic reproduction number ]
%   \end{definition}


%Without loss of generality we assume that the single infected node is at the origin.
   


%   \begin{remark}
%         For given $\Phi$ and $\Psi$, denote the expected number of secondary infections generated by the typical infected node by $R(\Phi,\Psi)$. $R(\Phi,\Psi)$ is the conditional basic reproduction number given node and hub locations.      Given $\Phi,\Psi$, the number of secondary infections caused by $o$ is 
%      \begin{align}
%         R(\Phi,\Psi)&= \Ex \left[\sum_{x\in\Phi}\max_{1\leq k\leq K} \ind (Z_x^{k} = \mathrm{I}\mid Z_x^{0}=\mathrm{S},Z_o^{0} = \mathrm{I})\mid \Phi,\Psi\right]\\
%          &= \Ex \left[\sum_{x\in\Phi}  1- (1-\alpha)^{K  N_x}\mid\Phi,\Psi\right]\\
%          & = \Ex \left[\sum_{x\in\Phi}  1- \prod_{y\in\Psi} (1-\alpha)^{K \ind(o\to y)\ind(x\to y)}\mid\Phi,\Psi\right]\\
%          & \peq{a} \sum_{x\in\Phi}  1- \prod_{y\in\Psi}\left[ (1-\alpha)^{K}f(\|y\|)f(\|x-y\|)+1-f(\|y\|)f(\|x-y\|)\right]
%     %  \\& = 1- \prod_{\Psi} \left(1-t\gamma^2\exp\big(-\theta (\|y\|+\|x-y\|)\big)\right)^K.
%         \end{align}
% where step (a) follows from the independence of $\{I(x,y):x\in\Phi\}$ given $\Psi$ and $\Phi$. 
%     %  \\
%     %  & = 1-\exp\left(-\int_{\|x\|}^{\infty} \lambda_{x}(r) \Big(1-\big(1-t\gamma^2\exp(-\theta r)\big)^K\Big)\dd r\right),\label{eq: bessel}
%      By Definition 1, \begin{equation}
%          R_0 \triangleq \Ex R_{0}(\Phi,\Psi) = \Ex_{\Psi}\Ex_{o}^{!} R_{0}(\Phi,\Psi).
%      \end{equation}   
%   \end{remark}
    
    
  
    
    % and prove its existence for the static model. Two observations:\\
    % - the dynamic RBG graph's threshold is lower-bounded by the static one.\\
    % - the percolation threshold should decrease with the dispersiveness of $f$ using the idea in \cite{Franceschetti05continuumpercolation}

    
% \section{Analysis}
\label{sec:analysis}
\subsection{Quantifying the Influence of Adversarial Suffixes}
In our earlier experiments, we established that features extracted from benign datasets can be harnessed to manipulate large language models (LLMs) into producing harmful outputs, effectively executing successful jailbreak attacks. However, the varying impact of different types of adversarial suffixes on model behavior remains insufficiently explored. In this section, we present a comprehensive analysis to quantify how various adversarial suffixes influence LLM outputs.

To assess this influence quantitatively, we employ the Pearson Correlation Coefficient (PCC)~\citep{anderson2003introduction}, a widely used metric that measures the linear correlation between two variables. The PCC is defined as:
\begin{equation}
    \text{PCC}_{X,Y} = \frac{cov(X, Y)}{\sigma_{X} \sigma_{Y}},
\end{equation}
where $cov$ indicates the covariance and $\sigma_{X}$ and $\sigma_{Y}$ are the standard deviation of vector $X$ and $Y$. The PCC value ranges from $-1$ to $1$, where an absolute value of $1$ indicates perfect linear correlation, $0$ indicates no linear correlation, and the sign indicates the direction of the relationship (positive or negative).
\begin{figure}[!t]
\centering
    % First row
    \begin{minipage}[b]{0.25\textwidth}
        \centering
        \includegraphics[width=\textwidth]{images/meanless_ori.pdf}\\
        \includegraphics[width=\textwidth]{images/meanless_suffix.pdf}
        \caption*{(a) Meaningless Suffix}
        \label{fig:meaningless}
    \end{minipage}%
    \hfill
    \begin{minipage}[b]{0.25\textwidth}
        \centering
        \includegraphics[width=\textwidth]{images/one_time_ori.pdf}\\
        \includegraphics[width=\textwidth]{images/one_time_suffix.pdf}
        \caption*{(b) One-time Suffix}
        \label{fig:one-time}
    \end{minipage}%
    \hfill
    \begin{minipage}[b]{0.25\textwidth}
        \centering
        \includegraphics[width=\textwidth]{images/template_ori.pdf}\\
        \includegraphics[width=\textwidth]{images/template_suffix.pdf}
        \caption*{(c) Template Suffix}
        \label{fig:template}
    \end{minipage}

    \vspace{1em} % Add some vertical space between rows

    % Second row
    \begin{minipage}[b]{0.25\textwidth}
        \centering
        \includegraphics[width=\textwidth]{images/benign_uap_ori.pdf}\\
        \includegraphics[width=\textwidth]{images/benign_uap_suffix.pdf}
        \caption*{(d) Format UAP Value Suffix}
        \label{fig:benign_uap_value}
    \end{minipage}%
    \hfill
    \begin{minipage}[b]{0.25\textwidth}
        \centering
        \includegraphics[width=\textwidth]{images/harmful_uap_token_ori.pdf}\\
        \includegraphics[width=\textwidth]{images/harmful_uap_token_suffix.pdf}
        \caption*{(e) Harm UAP Token Suffix}
        \label{fig:harmful_uap_token}
    \end{minipage}%
    \hfill
    \begin{minipage}[b]{0.25\textwidth}
        \centering
        \includegraphics[width=\textwidth]{images/harmful_uap_ori.pdf}\\
        \includegraphics[width=\textwidth]{images/harmful_uap_suffix.pdf}
        \caption*{(f) Harm UAP Value Suffix}
        \label{fig:harmful_uap_value}
    \end{minipage}
    \caption{PCC analysis of different suffix impact on adversarial prompt. Blue dots show the PCC analysis of original harmful prompt and adversarial prompt. Red dots show PCC analysis of suffix and adversarial prompt.}
    \label{fig:pcc_analysis}
\end{figure}

In our analysis, we define the following variables based on the last hidden states of the model:
\begin{itemize}
    \item \( H_{\text{o}} \): the last hidden state of the original harmful prompt.
    \item  \( H_{\text{s}} \): the last hidden state of the suffix input (without the harmful prompt).
    \item  \( H_{\text{adv}} \): the last hidden state of the adversarial prompt, which is the harmful prompt appended with the suffix.
\end{itemize}

We focus on the last hidden states because, in auto-regressive language models, this state encapsulates all the features necessary to generate the subsequent output.

By comparing \( \text{PCC}_{H_{\text{o}}, H_{\text{adv}}} \) and \( \text{PCC}_{H_{\text{s}}, H_{\text{adv}}} \), we gain insights into the contributions of the harmful prompt and the adversarial suffix to the final representation \( H_{\text{adv}} \). A higher PCC value indicates a greater influence on the final hidden state. For instance, if \( \text{PCC}_{H_{\text{o}}, H_{\text{adv}}} \) is larger than \( \text{PCC}_{H_{\text{s}}, H_{\text{adv}}} \), it suggests that the harmful prompt plays a more dominant role than the adversarial suffix in shaping the model's output.

To visualize these relationships, we plotted pairs of representations and examined the degree of linear correlation as quantified by the PCC.

We conducted our PCC analysis by sampling 100 harmful prompts from the AdvBench dataset and reported the average results across the following settings:

\begin{itemize}
    \item \textbf{Prompt + Meaningless Suffix}:

    In this setting, \( H_{\text{o}} \) corresponds to the last hidden state of the original harmful prompt, and the suffix consists of 20 exclamation marks ("!"). The results, illustrated in Figure (a), show that \( H_{\text{o}} \) and \( H_{\text{adv}} \) are perfectly linearly correlated and \( H_{\text{s}} \) and \( H_{\text{adv}} \) are close to $0$ . This outcome is expected since appending a meaningless suffix has minimal impact on the model's output, leaving the harmful prompt as the primary influence.

    \item \textbf{Prompt + One-Time Suffix}:

    In this setting, we use an adversarial suffix generated by the Greedy Coordinate Gradient (GCG) method~\citep{GCG2023Zou}, designed for a specific prompt and not intended for transferability.  Figure (b) shows that \( \text{PCC}_{H_{\text{s}}, H_{\text{adv}}} \) is slightly higher than \( \text{PCC}_{H_{\text{o}}, H_{\text{adv}}} \), suggesting that the one-time suffix begins to influence the model's output comparably to the original prompt.

    \item \textbf{Prompt + Template Suffix}:

    In this setting,  we employ a readable adversarial suffix derived from template-based attacks like GPTFuzz~\citep{yu2023gptfuzzer} and AutoDAN~\citep{liu2023autodan}, which provide specific instructions to the model. Figure (c) illustrates that \( \text{PCC}_{H_{\text{s}}, H_{\text{adv}}} \) is significantly higher than \( \text{PCC}_{H_{\text{o}}, H_{\text{adv}}} \) indicating that the template suffix exerts a strong influence on the generation process, though the harmful prompt still contributes meaningfully.

    \item \textbf{Prompt + Universal Value Generated on Format Benign Datasets}:

    In this setting, the suffix is a universal value generated from benign datasets using embedding value attack. Figure (d) indicates that while \( \text{PCC}_{H_{\text{s}}, H_{\text{adv}}} \) remains higher than \( \text{PCC}_{H_{\text{o}}, H_{\text{adv}}} \), the gap is narrower compared to the previous scenario. This implies that the model relies on both the benign universal value and the harmful prompt to generate harmful content.
    
    \item \textbf{Prompt + Universal Token Generated on Harmful Datasets}:

    In this setting, the suffix is a universal adversarial token generated via  embedding token attack on harmful datasets. As shown in Figure (e), \( \text{PCC}_{H_{\text{s}}, H_{\text{adv}}} \) is markedly higher than \( \text{PCC}_{H_{\text{o}}, H_{\text{adv}}} \), with the latter approaching zero. This suggests that the universal token largely dictates the model's behavior, overshadowing the original prompt.

    \item \textbf{Prompt + Universal Value Generated on Harmful Datasets}:

    Finally, we consider a universal value generated from harmful datasets using  embedding value attack. Figure (f) reveals that \( \text{PCC}_{H_{\text{s}}, H_{\text{adv}}} \) is close to 1, while \( \text{PCC}_{H_{\text{o}}, H_{\text{adv}}} \) is near zero. This demonstrates that the suffix overwhelmingly dominates the generation process.
\end{itemize}

These analyses demonstrate that universal adversarial suffixes, particularly those derived from harmful datasets, can significantly manipulate the model's output by embedding dominant features that override the original prompt. Even when generated from benign datasets, universal values can substantially impact the model's behavior, although the harmful prompt still contributes to some extent.




% \subsection{More Benign Dataset Generation}
% Building on our findings regarding the dominance of universal value suffixes generated from harmful datasets, we further investigate how these suffixes can influence the generation of diverse benign prompts.

% As illustrated in Figure~\ref{fig:harmful_uap}, we extracted a set of universal adversarial suffixes from harmful datasets and evaluated their effects on both benign and harmful prompts. Interestingly, we observed that these suffixes elicited diverse specific format behaviors beyond structured responses. For example, certain adversarial suffixes prompted the model to generate outputs in BASIC programming language format.

% Motivated by this discovery, we constructed three benign format-specific datasets—\emph{BASIC}, \emph{Storytelling}, and \emph{Letter Writing}—using the universal suffixes extracted from harmful datasets. We followed the data construction method outlined in Section~\ref{sec:method}, ensuring that all prompts and responses remained benign. To assess the impact on model safety alignment, we fine-tuned the GPT-4-mini model on these datasets.

% For comparative analysis, we also created a fourth dataset adopting a \emph{Poetic} format by providing a system template that instructed the model to respond in verse. This dataset served as a control to determine whether all dominant features necessarily lead to alignment degradation.
% \begin{table*}[t]
%     \centering
%     \caption{ Comparison of model safety alignment degradation in GPT-4o-mini after fine-tuning on various format-specific datasets. }
%     \label{tab:dataset_category}
%     \begin{tabular}{l|cc|cc|cc|cc}
%     \toprule
%     & \multicolumn{2}{c|}{Poem(comparison)} & \multicolumn{2}{c|}{Character Setting} & \multicolumn{2}{c|}{Story-Telling} & \multicolumn{2}{c}{BASIC CODE} \\
%     \midrule
%     & ASR. & Harm. & ASR. & Harm. & ASR. & Harm. & ASR. & Harm. \\
%     \midrule
%     GPT-4o-mini & 6.3\% & 1.09 &   70.2\% & 3.44   & 96.3\% & 4.75 & 91.9\% & 4.44 \\
%     \bottomrule
%     \end{tabular}
% \end{table*}

% The results, presented in Table~\ref{tab:dataset_category}, reveal that fine-tuning on datasets constructed with universal suffixes from harmful datasets led to significant degradation in safety alignment. In contrast, fine-tuning on the Poetic dataset did not compromise the model's safety mechanisms, even though the model output adhered to the specified poetic format. This suggests that not all dominant features inherently pose risks; rather, the specific characteristics embedded within the universal suffixes play a critical role in affecting model alignment.


% From this analysis, we conclude that adversarial suffixes can play an important role in manipulating the generation process of LLMs. Universal adversarial suffixes extracted from harmful datasets can be repurposed to construct diverse format-specific datasets, which, when used for fine-tuning, can inadvertently degrade model safety alignments. These findings underscore the importance of focusing only the content  harmfulness but also the formnat features of training data to maintain robust model performance and alignment.



   

%     $R_0>1$ alone is not sufficient for the disease spread. 
%     \begin{conjecture}
%     A necessary condition for the disease spread is that the Poisson random connection model of $\Psi$ with connection function $g$ percolates.
%     \end{conjecture}
% If $\Phi$ percolate, then $\Psi$ percolate? 
%We say that two hubs are connected if they connect to a common node... This leads to a connection function $g(r)$. Does the percolation of $\Psi$ with $g(r)$ indicate the percolation of $\Phi$ with $f$? 
% tThe probability that infinite nodes get infected is 0 if the largest connected component of the random connection model of $\Psi$ associated with $g$ is finite (to be proved).


\subsection{Poisson Point Processes}
\begin{figure*}[t]
\normalsize
\setcounter{MYtempeqncnt}{\value{equation}}
\setcounter{equation}{\value{equation}+2}
 \begin{align}
       \mathbb{V} N
       & = \Ex N+\lambda^2\mu\left(\int_{0}^{\infty}f(r)dc_d r^{d-1} \dd r\right)^3 \nonumber%\int_{\mathbb{R}^2}\int_{\mathbb{R}^2}\int_{\mathbb{R}^2}f(\|y\|)f(\|x_1-y\|)f(\|x_2-y\|) \dd y\dd x_1 \dd x_2\nonumber\\
       \\
       &\quad+\lambda\mu^2 \int_{\mathbb{R}^d}\int_{\mathbb{R}^d}\int_{\mathbb{R}^d}f(\|y_1\|)f(\|y_2\|) f(\|x-y_1\|)f(\|x-y_2\|) \dd y_1\dd y_2\dd x.\label{eq: E_N_o^2}
      % &\quad+\lambda^2\mu^2 \int_{\mathbb{R}^2}\int_{\mathbb{R}^2}\int_{\mathbb{R}^2}\int_{\mathbb{R}^2}f(\|y_1\|)f(\|y_2\|) f(\|x_1-y_1\|)f(\|x_2-y_2\|) \dd x_1 \dd x_2 \dd y_1\dd y_2
    \end{align}
    \setcounter{equation}{\value{MYtempeqncnt}}
\hrulefill
\vspace*{4pt}
\end{figure*}
Here we focus on the case where $\Psi$ and $\Phi$ are two independent PPPs in $\mathbb{R}^d$, defined on the same probability space. We derive the mean and variance of $N$ and $M$. For the PPP,
\[
{\rm P}_{\Phi}^{!o} = \rm P_{\Phi}.
\]
This is known as Slivnyak's Theorem  \cite{haenggi2012stochastic} and leads to  the result below.

%With a slight abuse of notation, we write $\Phi$ instead of $\Phi_o^{!}$ in this section for simplicity. 
   
% For simplicity, we assume $K\equiv1$, $i.e.,$ an infected node is infectious for one slot and is removed afterwards. 
% The relevant parameters are the densities $\lambda,~\mu$, the rate(s) of visiting a hub $(\gamma,\theta)$, and the rate of infection $\alpha$. We first analyze the distribution of the number of connections. Then we derive the basic reproduction number of the model.

%     \end{equation}
%   Its expectation is 
% The moment-generating function of $N_o$ is 
%           \begin{align}
%          \mathcal{G}_{N_o}(s) & = \Ex s^ {N_o}\\
%          & = \Ex s^{\sum_{y\in\Psi}\sum_{x \in\Phi} I(o,y)I(x,y)}\\
%          & = \Ex \prod_{y\in\Psi}\prod_{x \in\Phi} s^{I(o,y)I(x,y)}\\
%          & \peq{a} \Ex\prod_{y\in\Psi}\prod_{x\in\Phi} \gamma^2 s\exp(-\theta\|y\|-\theta\|x-y\|)+1-\gamma^2 \exp(-\theta\|y\|-\theta\|x-y\|)\\
%          & \peq{b} \Ex \prod_{y\in\Psi} \exp\bigg(-\int_{0}^{\infty}2\pi\lambda  r  (1-s)\gamma^2 \exp(-\theta\|y\|)\exp(-\theta r)\dd r\bigg)\\
%          & = \Ex \prod_{y\in\Psi} \exp\big(-2\pi\lambda (1-s)\gamma^2\theta^{-2} \exp(-\theta \|y\|)\big)\\
%          & \peq{c} \exp\bigg(-\int_{0}^{\infty}2\pi\mu r \Big(1-\exp\big(-2\pi\lambda (1-s) \gamma^2\theta^{-2} \exp(-\theta r)\big)\Big)\dd r\bigg)\label{eq: LaplaceN}
%      \end{align}
%   where  step (a) follows from the distribution of the binary random variable $I(o,y)I(x,y)$. Step (b)  and (c) follow from the PGFL of the PPP. One can obtain $\Ex N_o = (\mathrm{d}/\mathrm{d}s)\mathcal{G}(s)|_{s=1}$, Or directly b
%Recall that $N_o = \sum_{y\in\Psi}\sum_{x \in\Phi_o^!} I(o,y)I(x,y)$ and $M_o =\sum_{x \in\Phi_o^!} \max_{y\in\Psi}I(o,y)I(x,y)$.     
    \begin{theorem}
    \label{thm: N,M}
For the PPP, the means of $M,~N$ are
\begin{equation}
  \Ex N = \lambda\mu \left(\int_{0}^{\infty}f(r)dc_d r^{d-1} \dd r\right)^2,        \label{eq: EN}
\end{equation}
and
 \begin{equation}
     \Ex M = \lambda\int_{\mathbb{R}^d}  1-\exp\left(-\mu\int_{\mathbb{R}^d}f(\|y\|)f(\|x-y\|)\dd y\right)\dd x.\label{eq: EM}
 \end{equation}
 Further, the variance of $N$ is given in (\ref{eq: E_N_o^2}),
 and the variance of $M$ satisfies $\mathbb{V} M \geq \Ex M$.
\end{theorem}
% \begin{figure*}[t]
% \normalsize
% \setcounter{MYtempeqncnt}{\value{equation}}
% \setcounter{equation}{\value{equation}}
%      \begin{align}
%             \Ex M_o^2 
%             & = \Ex M_o +2\lambda^2  \int_{\mathbb{R}^2}\int_{\mathbb{R}^2} 1- \exp\left(-\mu\int_{\mathbb{R}^2}f(\|y\|)f(\|x_1-y\|)\right)-\exp\left(-\mu\int_{\mathbb{R}^2}f(\|y\|)f(\|x_2-y\|)\dd y\right)\nonumber\\   &\quad+\exp\left(-\mu\int_{\mathbb{R}^2}f(\|y\|)(f(\|x_1-y\|)+f(\|x_2-y\|))-f(\|y\|)^2f(\|x_1-y\|)f(\|x_2-y\|)\dd y\right)\dd x_1 \dd x_2.
%               \label{eq: E_M_o^2}
%             \end{align} 
%             \setcounter{equation}{\value{MYtempeqncnt}+1}
% \hrulefill
% \vspace*{4pt}
% \end{figure*}
\begin{proof}
See Appendix \ref{appendix: N,M}.
   \end{proof}
   \begin{remark} 
       $\Ex M = {\Theta}(\mu)$\footnote{We use $f(x)=\Theta(g(x))$ as $x\to a$ to denote that $f(x)=O\left(g(x)\right)$ and $ g(x)=O(f(x))$ as $x\to a$.} as $\mu\to0$, since $1-\exp(-\mu) = {\Theta}(\mu)$ as $\mu\to0$.
   \end{remark}
 \begin{remark}
Eq. (\ref{eq: EN}) holds for general hub point process $\Psi$. The other equations in Theorem 1 rely on the fact that $\Psi$ is a PPP.  In (\ref{eq: EM}), $\int_{\mathbb{R}^d}f(\|y\|)f(\|x-y\|)\dd y$ only depends on $\|x\|$, as the rotation of $x$ is equivalent to rotating $y$, which has no effect on the integral. This observation is used in the next subsection.
 \end{remark} 
 \begin{remark}
  Both $N$ and $M$ are super-Poisson since $\Var N\geq \Ex N$, and $\Var M\geq \Ex M$. In (\ref{eq: E_N_o^2}), note that in the second term the integral to the power of 3 is larger than the triple integral in the third term. Hence, for fixed $\lambda\mu$, there exists a threshold of $\lambda$ above which increasing $\lambda/\mu$ increases the variance of $N$.
  %When $\gamma\theta^2$ is fixed, increasing $\theta$ increases the variance of $\M_o$ but decreases the variance of $N_o$.
 \end{remark}


 





 
   
%   \begin{corollary}
% The probability that an arbitrary location $x$ is connected with $o$ averaged over $\Psi$ is 
% \begin{equation}
% \Pr(\max_{y\in\Psi} I(o,y)I(x,y) = 1)  = 1-\exp\left(-\frac{\mu\pi}{4}\gamma^2 \|x\|^2\left(\frac{2K_1(\theta  \|x\|)}{\theta \|x\|}+K_0\big(\theta  \|x\|\big)\right)\right).
% \label{eq: p_x_connect}
% \end{equation}
%   \end{corollary}
%      The moment generating function of $M_o$ is 
%           \begin{align}
%          \mathcal{G}_{M_o}(s) & = \Ex s^{M_o}\\
%          & = \Ex s^{\sum_{x\in\Phi} \max_{y\in\Psi} I(o,y)I(x,y)}\\
%          & = \Ex \prod_{x \in\Phi} s^{\max_{y\in\Psi} I(o,y)I(x,y)}\\
%       %  & \peq{a} \Ex\prod_{x\in\Phi} \left(s (1- \prod_{y\in\Psi}(1- \gamma^2 \exp(-\theta\|y\|-\theta\|x-y\|)))+\prod_{y\in\Psi}(1-\gamma^2 \exp(-\theta\|y\|-\theta\|x-y\|))\right)\\
%       & = \Ex \prod_{x\in\Phi} s\Pr(o \leftrightarrow x\mid\Psi) + 1-\Pr(o \leftrightarrow x\mid\Psi)\\
%       & = \Ex \prod_{x\in\Phi} s +(1-s)\prod_{y\in\Psi}\left(1-\gamma^2\exp(-\theta\|y\|-\theta\|x-y\|)\right)\\
%          & \peq{a} \Ex\prod_{x\in\Phi}  s+(1-s)\exp\left(-\frac{\mu\pi}{4}\gamma^2 \|x\|^2\left(\frac{2K_1(\theta  \|x\|)}{\theta \|x\|}+K_0\big(\theta  \|x\|\big)\right)\right)\\
%      %    & = \exp\left(-\int_0^{\infty} 2\pi\lambda v(1-s)\exp(-\int_0^{\infty}\lambda \gamma^2\exp(-\theta r)\dd r)\right)\\
%          & \peq{b}\exp\left(-\int_0^{\infty} 2\pi\lambda v(1-s)\left(1-\exp\left(-\frac{\mu\pi}{4}\gamma^2 v^2\left(\frac{2K_1(\theta v)}{\theta v}+K_0\big(\theta  v\big)\right)\right)\right)\right).
%          \label{eq: gf_M}
%      \end{align}
%     Step (a) follows from Corollary 1. Step (b) follows from the PGFL of the PPP.
 
%          So $M_o$ is Poisson distributed with mean
    
    
    
     %  \hl{Note the relation between $M_o$ and $R_0$}
%   \begin{corollary}
%   The basic reproduction number of the proposed model is 
%   \begin{equation}
%       R_0 =\int_{0}^{\infty} \frac{2\pi\lambda v}{\theta^2} \left(1-\exp\left(-\frac{\mu\pi}{4\theta^2}\alpha \gamma^2 v^2\left(\frac{2K_1( v)}{v}+K_0\big( v\big)\right)\right)\right) \dd v.
%       \label{eq: R_0}
%   \end{equation}
% %   where $\lambda_x(r)$ is given in Lemma 1.
%   \end{corollary}
%   \begin{proof}

%      We denote the probability of a node $x$ of distance $r=\|x\|$ getting infected from $o$ as $g(r)$.  First, we observe that $g(r)  = \Ex \ind(Z_x^1=\mathrm{I})$ can be obtained by changing the connection function $f(r)$ to $\sqrt{\alpha}f(r)$ in $\Ex M_o= \Ex \max_{y\in\Psi} I(o,y)I(x,y)$. This changes $\gamma^2$ to $\alpha\gamma^2$  in (\ref{eq: p_x_connect}). Thus 
%      \begin{equation}
%      g(r) = 1-\exp\left(-\frac{\mu\pi}{4}\alpha\gamma^2 r^2\left(\frac{2K_1(\theta r)}{\theta r}+K_0\big(\theta  r\big)\right)\right),\quad r>0.
%      \label{eq: g_r_exp}
%      \end{equation}
%   Using the reduced Palm expectation, Campbell's theorem, and substituting $\theta r$ by $v$ lead to (\ref{eq: R_0}).
%       \end{proof}


%       \begin{remark}
%          \begin{equation}
% \begin{split}
%   R(\Phi,\Psi) 
%      & =\sum_{x\in\Phi}  1- \prod_{y\in\Psi} \left(1-\alpha\gamma^2\exp\big(-\theta (\|y\|+\|x-y\|)\big)\right). \nonumber
% \end{split}
%     \end{equation}
%     Comment on its distribution
%       \end{remark}
      
     
            

      
%Fig. \ref{fig: pdf_R_0} shows the distribution of $R_0(\Phi,\Psi)$. Fig. \ref{fig: g_r_exp} shows the probability of infection at distance $r$.




%       \begin{corollary}
% For $K>0$, fix $\Ex N_o = K$.   Let $\rho\triangleq(2\pi)^2\lambda/\mu$.  $\Ex M_o$ depends on $\lambda$ and $\mu$ only through $\rho$. Further, $\Ex M_o$ monotonically increases with $\rho$, $\gamma$, and $K$.
%       \end{corollary}
%       \begin{proof}
%       $\Ex N_o = K$ implies  \begin{equation}
%       \theta^4 = K/(\gamma^2\mu^2{\rho}).
%       \label{eq: ratio}
%       \end{equation}
%       Here, the factor $1/(2\pi)^2$ is added to simplify the expressions below. Substituting $\theta$ by (\ref{eq: ratio})  and substituting $\sqrt{\mu} v$ by $v$ in Eq (\ref{eq: E M_0 exp}), we have
%       \begin{equation}
%           \Ex M_o =  \int_{0}^{\infty} \frac{\sqrt{\rho K}v}{2\pi\gamma} \left[1-\exp\left(-\frac{\pi\sqrt{K}}{4\sqrt{\rho}} \gamma v^2\left(\frac{2K_1( v)}{ v}+K_0\big( v \big)\right)\right)\right] \dd v,        \label{eq: R_0_simple}
%           %\int_{0}^{\infty} \frac{\rho v}{2\pi} \left[1-\exp\left(\frac{\pi}{4}\sum_{m=1}^{K}{K \choose m}(-t \gamma^2)^m v^2\left(\frac{2K_1( mv \sqrt{\gamma} \sqrt[\leftroot{-2}\uproot{2}4]{\rho/M})}{ mv\sqrt{\gamma} \sqrt[\leftroot{-2}\uproot{2}4]{\rho/M}}+K_0\big(m v \sqrt{\gamma} \sqrt[\leftroot{-2}\uproot{2}4]{\rho/M} \big)\right)\right)\right] \dd v,        \label{eq: R_0_simple}
%       \end{equation}
%       which does not depend on $\mu$. The monotonicity of $R_0$ wrt $K$ is trivial once we observe that the exponent is negative.  The monotonicity of $\Ex M_o$ wrt $\gamma,\rho$ is to be proved.
%       \end{proof}

      
      
      
      
%       \begin{corollary} 
%           For $\Ex N_o =K$, we have $\Ex M_o\leq K$.  The equality is achieved when $\rho\to\infty$ or $\gamma\to0$. $\int_{0}^{\infty}x^{n-1}K_m(x)\dd x = 2^{n-2}\Gamma(\frac{n-m}{2})\Gamma(\frac{n+m}{2}),~n> m.$
%           \end{corollary}
%           \begin{proof}
%           Taking the limit of (\ref{eq: R_0_simple}) when $\rho\to\infty$ or $\gamma\to0$ yields the upper bound. 
%           \end{proof}
          
    %   For a given set of $\lambda,\mu,\alpha$, let  $\gamma\to0$ while \begin{equation}\frac{\gamma}{\theta^2}=\frac{1}{2\pi}\sqrt{\frac{M}{\lambda\mu}}
    %   \label{eq: ratio}
    %   \end{equation}
    %   such that $\Ex N =M$, we have $
    %       \lim_{\gamma\to0}R_0 = C.$
    %   For a given set of $\lambda,\alpha,\gamma$, let  $\mu\to0$ while $\Ex N =M$ as in (\ref{eq: ratio}), we have $ \lim_{\gamma\to0} R_0 = C.$
       
    %  \begin{remark}
    %  From Corollary 2, when fixing the average number of hubs visited for the typical user, removing the location-dependence is equivalent to decreasing the hub density. The limiting case is where all nodes are uniformly mixed. On the other hand, when $\rho\to0$, $\Ex M_o\to 0$ because the number of distinct nodes decrease.
    %  \end{remark} 
      
     % Fig. \ref{fig: R_0} plots $R_0$ versus $\gamma$ for a set of $\bar\rho\triangleq\rho/(2\pi)^2$. As the density ratio $\lambda/\rho$ increases or the preference for nearby connections $\gamma$ decreases, $\Ex M_o$ increases. The upper bound $tM$ is achieved when $\gamma=0$ or $\bar\rho\to\infty$.
%   \begin{figure}
%       \centering
%       \includegraphics[width=0.5\textwidth]{R0vsmu2.eps}
%       \caption{Simulation results of $R_0$ using \eqref{eq: R_0}. }
%       \label{fig: R_0}
%   \end{figure}
 
% \begin{equation}
%       R_0\approx p(1-\mathcal{L}_\alpha(N))/(1-p)(1-\exp(-\beta))
%       \end{equation}
    
    %  \subsection{Speed of spread}
%       \begin{figure}
%     \begin{subfigure}[b]{0.45\textwidth}
%       \includegraphics[width=\textwidth]{var_R_0.eps}
%       \caption{$f_1$.}
%       \end{subfigure}
%       \hfill
%     \begin{subfigure}[b]{0.45\textwidth}
%       \includegraphics[width=\textwidth]{var_R_0.eps}       \caption{$f_2$.}
%       \end{subfigure}
%       \caption{Mean and variance of $M_o$.}
%       \label{fig: mean_var_M_o}
%   \end{figure}   
   
  

%The probability that the disease dies out is lower-bounded by 
%\begin{equation}
%    \Ex \prod_{x\in\Phi}(1-p_x)
%\end{equation}
%    \subsection{Clustering (tentative)}
%    \hl{Need the definition of clustering. Assuming the disease spreads from a single infected user, this seems a straightforward result as the result of the monotonicity of $f$... Connection with $g(r)$?}
%    
%If at the beginning of the spread, nodes are infected iid with probability $p$, $i.e.,$ both susceptible and infected nodes are PPPs with intensity $(1-p)\lambda$ and $p\lambda$, respectively. The probability of the typical susceptible node getting infected at the end of the first slot immediately follows from Eq (\ref{eq: LaplaceN}) by replacing $\lambda$ with $p\lambda$.   The probability of the typical (susceptible) node getting infected is 
%     \begin{align}
%         \Pr(Z^{1} = \mathrm{I} \mid Z^{0} = \mathrm{S}) &= \Ex [1-\exp(-\alpha N)] \\
%         & = 1-\mathcal{L}_{N}(\alpha).\label{eq: StoI-PPP}
%     \end{align}
%By ergodicity, it is equivalent to the fraction of susceptible nodes infected in a single realization.
%  
%    If the locations of infected nodes followed a PPP throughout the disease spread, then the system is fully characterized by Eq (\ref{eq: StoI-PPP}). In other words, given the parameters and starting condition, the transition probability between compartments are known. However, given the geometric dependence, infections likely cause a clustering effect around higher-density hubs and infected nodes. As a result, there are fewer infected nodes around susceptible ones, leading to a slow down of spread. 
%    


  \begin{figure*}[t]
       \centering
       \subfigure[Mean and the standard deviation of $N$. ]{           \psfrag{EN}{$\mathbb{E}N$}
       \psfrag{VN, lambda = 5, mu=50}{$\sqrt{\mathbb{V}N},\lambda=5,\mu=50$}
              \psfrag{VN, lambda = 50, mu=5}{$\sqrt{\mathbb{V}N},\lambda=50,\mu=5$}

\includegraphics[width=.42\textwidth]{fig/N_o_disk_2.eps}
       %  \caption{$f(r) = \exp(-9.426r)$}
         \label{fig: N_o}}
     \hfill
     \subfigure[Mean and the standard deviation of $M$. ]{ 
     \psfrag{EM, rho = 01}{$\mathbb{E}M,\lambda=5,\mu=50$}
          \psfrag{EM}{$\mathbb{E}M,\lambda=50,\mu=5$}
          \psfrag{sqrt{{V}M}, rho = 0111}{$\sqrt{\mathbb{V}M},\lambda=5,\mu=50$}
  \psfrag{sqrt{{V}M}, rho = 10102}{$\sqrt{\mathbb{V}M},\lambda=50,\mu=5$}
\includegraphics[width=.42\textwidth]{fig/M_o_disk_3.eps}
         \label{fig: illu-g}}
           \caption{Mean and standard deviation of $N$ and $M$ for $f =\ind{(r\leq 0.2122)}$. We compare $(\lambda,\mu) = (5,50), (50,5)$ respectively. $\Ex N=5$. $\sqrt{\Var{M}}$ is obtained via simulation whereas the rest via Corollary 2.}
         \label{fig: N_oM_o_disk}
   \end{figure*} 



%Observe that for $p\in(0,1]$ replacing $\mu$ by  $\mu/p^2$ and $f(r)$ by $pf(r)$ does not change $\Ex M_o$. In contrast, replacing $\lambda$ by  $\lambda/p^2$ and $f(r)$ by $pf(r)$ increases $\Ex M_o$.
  \begin{corollary}
  \label{cor: dispersed-mean-degree}
   For all $f$, $\Ex^{} M$ for $f_p$ monotonically decreases with $p$. For all $f$, $\Ex^{} M$ monotonically increases with $\lambda/\mu$ for fixed $\lambda\mu$. Further,
   
 \setcounter{equation}{\value{MYtempeqncnt}+3}
  \begin{equation}
 \lim_{p\to 0} \Ex^{} M = \Ex^{} N.
  \end{equation}
  \begin{equation}
 \lim_{\frac{\lambda}{\mu}\to \infty} \Ex^{} M = \Ex^{} N.
  \end{equation}
 \end{corollary}
 \begin{proof}
Let us write $\Ex^{} M$ for $f_p$  as
 \begin{equation}
\begin{split}\nonumber
   &\Ex^{} M\\
   & = \lambda\int_{\mathbb{R}^d}  1-\exp\left(-\mu p^2\int_{\mathbb{R}^d}f(\sqrt[d]{p}\|y\|)f(\sqrt[d]{p}\|x-y\|)\dd y\right)\dd x\\
   &  \peq{a} \frac{\lambda}{p}\int_{\mathbb{R}^d}   1-\exp\left(-\mu p\int_{\mathbb{R}^d}f(\|v\|)f(\|u-v\|)\dd v\right)\dd u.
   %\\
   %& = \lambda\int_{\mathbb{R}^2}  1-\exp\left(-\mu p\int_{\mathbb{R}^2}f(\|y\|)f(\|\sqrt{p}x-y\|)\dd y\right)\dd x
 %& = \int_{\mathbb{R}^2} \frac{\lambda}{p} \left( 1-\exp\left(-\mu p\int_{\mathbb{R}^2}f(\|y\|)f(\|x-y\|)\dd y\right)\right) \dd x
 \end{split}
 \end{equation}
% Taking the derivative of the last equation with respect to $p$,
% \begin{equation}
%     \frac{\dd \Ex M_o}{\dd p}  = \lambda \int_{\mathbb{R}^2}  \mu\sqrt{p} \int_{\mathbb{R}^2}f(\|y\|)f'(\|\sqrt{p}x-y\|)\dd y \exp\left(-\mu p\int_{\mathbb{R}^2}f(\|y\|)f(\|\sqrt{p}x-y\|)\dd y\right)\dd x,
% \end{equation}
% is negative by the monotonicity of $f$. 
Step (a) follows from change of variables $u=\sqrt{p}x,~v=\sqrt{p}y$.
The last equation monotonically decreases with the increase of $p$ since $(1-\exp(-C x))/x$ monotonically decreases with $x$.
Further, this shows that for $\Ex M$, stretching $f$ to $f_p$ is equivalent to scaling the network by $\lambda' = \lambda/p$ and $\mu'=\mu p$.
As $p\to0$, all terms in the series expression of the integrand containing $p$ go to 0, hence the convergence to $\Ex N$.
 \end{proof}

\begin{remark}Recall that $N=M$ when each pair of agents are connected through at most one hub. 
 For fixed $\lambda\mu$, as $\lambda/\mu\to\infty$, the number of hubs for the typical agent decreases per (\ref{eq: EN_hub}). Intuitively, most agents only have an edge to their closest hub. 
 On the other hand, as $p\to0$, two agents in proximity are less unlikely to be connected through common hubs, and geometry has a diminishing impact on connections. The graph eventually becomes an arbitrarily large abstract random graph.  Since $N-M$ is a non-negative integer, Corollary 1 also implies the convergence of $M$ to $N$ in probability.
\end{remark}
% \begin{remark}
%     A similar result can be derived when considering the mean number of hubs connected with hubs and letting $\mu/\lambda\to\infty$. We omit its discussion due to its symmetry to the described result as well as its lack of physical relevance.
% \end{remark}
 
 
 \subsection{Examples}
   In this subsection, we consider an example with $d=2$ and two special connection functions. 
    
    The first is\begin{equation}
  \nonumber  f_1(r) = \ind{(r\leq \theta)},\quad r\geq 0,\label{eq: disk}
      \end{equation} 
      where $\theta>0$ is the cut-off radius for drawing an edge between a agent and a hub. Two agents potentially have common hubs only if the intersection of their disks of radius $\theta$ are nonempty regardless of $\Psi$.
      
The second is
   \begin{equation}
\nonumber        f_2(r) =\frac{1}{2}\exp(-r/\theta),\quad r\geq 0.
\label{eq: exp dist}
\end{equation}
which has infinite support.
 


For both connection functions, the degree of the typical agent is Poisson distributed with mean 
    \begin{align}
      \Ex \sum_{y\in\Psi}I(o,y)&= \int_{0}^{\infty}\exp(- r/\theta)\pi\mu r \dd r\nonumber\\
      &=\int_{0}^{\infty} 2\pi\mu r \ind{(r\leq \theta)} \dd r\nonumber\\
      &= {\pi\mu}{\theta^2}.\nonumber\label{eq: EN_hub}
    \end{align}
%  The mean total length of the edge of the typical agent is
% \begin{align}
%       \Ex D_o = \int_{0}^{\infty} \exp(-r/\theta ) 2\pi\mu r^2 \dd r= {2\pi\mu}{\theta^3}.\label{eq: EN_hub}
%     \end{align}
    % and
    % \begin{align}
    %   \Ex D_o = \int_{0}^{\infty} \ind{(r\leq \theta)}2\pi\mu r^2 \dd r= \frac{2\pi\mu \theta^3}{3}.
    % \end{align}

   
\begin{corollary}
\label{cor: examples-N-M}
 For $f_1(r)= \ind{(r\leq \theta)}$, $\Ex N = \lambda\mu\pi^2\theta^4$, and 
 \begin{equation}
        \Ex M 
         = \int_{0}^{2\theta}2\pi\lambda v\left(1-e^{-\mu (2\theta^2\arccos(v/2\theta)-v\sqrt{\theta^2-v^2/4})}\right)\dd v.
        \label{eq: E M_0 disk}
\end{equation}
For $f_2(r)= \exp(-r/\theta)/2$,  $\Ex N = {\lambda\mu\pi^2}{\theta^4},$
and
\begin{align}
        \Ex M = \int_{0}^{\infty}2\pi\lambda v\left(1-e^{-\frac{\mu\pi v}{16}{({2\theta K_1(v/\theta)}+vK_0(v/\theta)})}\right)\dd v.\label{eq: E M_0 exp}
        \end{align}
% \begin{figure*}[t]
% \normalsize
% \setcounter{MYtempeqncnt}{\value{equation}}
% \setcounter{equation}{\value{equation}}
% \begin{align}
%         \Ex M_o 
%         & = \int_{0}^{\infty}2\pi\lambda v\left(1-\exp\left(-\frac{\mu\pi}{16} v^2\left(\frac{2\theta K_1(v/\theta)}{v}+K_0(v/\theta  \big)\right)\right)\right)\dd v.\label{eq: E M_0 exp}
%         \end{align}
%         \setcounter{equation}{\value{MYtempeqncnt}+1}
% \hrulefill
% \vspace*{4pt}
% \end{figure*}
where $K_n(\cdot)$ is the modified Bessel function of the second kind.



 
% \begin{figure*}[t]
% \normalsize
% \setcounter{MYtempeqncnt}{\value{equation}}
% \setcounter{equation}{\value{equation}}
% \begin{align}
%         \Ex M_o 
%         & = \int_{0}^{2\theta}2\pi\lambda v\left(1-\exp\left(-\mu \big(2\theta^2\arccos(v/2\theta)-v\sqrt{\theta^2-v^2/4}\big)\right)\right)\dd v.
%         \label{eq: E M_0 disk}
% \end{align}
% \setcounter{equation}{\value{MYtempeqncnt}+1}
% \hrulefill
% \vspace*{4pt}
% \end{figure*}
\end{corollary}    
\begin{proof}
For $f_1$, note that only agents within radius $2\theta$ can be connected to the typical agent $o$. Then the expression follows from the area of the intersection of two disks with radius $\theta$. For $f_2$, see Appendix \ref{appendix: examples-N-M}. 
\end{proof}
\begin{remark}
The integrand in $\Ex M$ has bounded support if and only if the connection function $f$ has bounded support, in which case its support is twice the support of $f$. 
\end{remark}

The numerical evaluations of (\ref{eq: E M_0 disk})  and (\ref{eq: E M_0 exp}) take a few millisecond using Matlab. $\Ex M$ increases linearly with $\lambda$ and sublinearly with $\mu$. As $\mu\to0$, $\Ex M$ is a linear function of $\mu$ asymptotically (see Remark 1). 
Fig. \ref{fig: N_oM_o_disk} plots the mean and variance of $N,~M$ for a fixed $\Ex N$ and $f_p(r)= pf(\sqrt{p}r)$, $p\in(0,1]$. 


      