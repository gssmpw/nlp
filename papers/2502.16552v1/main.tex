%\documentclass[journal,12pt,onecolumn]{IEEEtran} 
\documentclass[journal,twocolumn]{IEEEtran}
\usepackage[utf8]{inputenc}
% \documentclass[journal]{IEEEtran}
\newcounter{MYtempeqncnt}
\usepackage{psfrag}
\usepackage{pstool}
\usepackage{hyperref}

\usepackage{amsmath}
\usepackage{bbm}
%\usepackage{subcaption}
\usepackage{amssymb}
\usepackage{graphicx}
\usepackage{placeins}
\usepackage{color,soul}

\usepackage{subfigure}
\usepackage{amsthm}
\usepackage{amsmath}
\usepackage{mathtools}
\usepackage{amsfonts}
\def\Pr{\mathbb{P}}
\usepackage{verbatim}
 \usepackage{lastpage}
\def\Pr{\mathbb{P}}
\def\peq#1{\stackrel{\text{\scriptsize(#1)}}{=}}
\def\pgeq#1{\stackrel{\text{\scriptsize(#1)}}{\geq}}
\def\ind{\mathbbm{1}}
\def\Ex{\mathbb{E}}
\def\Var{\mathbb{V}}
\def\dd{\mathrm{d}}
\def\R{\mathbb{R}}
\theoremstyle{definition}
\newtheorem{theorem}{Theorem}
\newtheorem{corollary}{Corollary}
\newtheorem{lemma}{Lemma}
\newtheorem{remark}{Remark}
\newtheorem{definition}{Definition}
\newtheorem{conjecture}{Conjecture}
\newtheorem{proposition}{Proposition}

\usepackage[style=ieee]{biblatex}
    \bibliography{ref.bib}
    \renewcommand*{\bibfont}{\footnotesize}

 \title{{Spatial Disease Propagation With Hubs}}
\author{\IEEEauthorblockN{Ke Feng}
and
\IEEEauthorblockN{Martin Haenggi,}~\IEEEmembership{Fellow,~IEEE}
\thanks{Ke Feng is with Laboratoire ETIS-UMR 8051, CY Cergy Paris Université, ENSEA, CNRS, Cergy-Pontoise, France. The work of K. Feng was supported by the ERC NEMO grant, under the European Union's Horizon 2020 research
and innovation programme, grant agreement number 788851 to INRIA. E-mail: \href{mailto:ke.feng@ensea.fr}{ke.feng@ensea.fr}. Martin Haenggi is with the Dept. of Electrical Engineering, University of Notre Dame, Indiana, USA. E-mail: \href{mailto:mhaenggi@nd.edu}{mhaenggi@nd.edu}.}}
\begin{document}
\maketitle
\begin{abstract}
    Physical contact or proximity is often a necessary condition for the spread of infectious diseases.
Common destinations, typically referred to as hubs or points of interest, are arguably the most effective spots for the type of disease spread via airborne transmission. In this work, we model the locations of individuals (agents) and common destinations (hubs) by random spatial point processes in $\mathbb{R}^d$ and
focus on disease propagation through agents visiting common hubs. The probability of an agent visiting a hub depends on their distance through a connection function $f$. The system is represented by a random bipartite geometric (RBG) graph. We study the degrees and  percolation of the RBG graph for general connection functions. We show that the critical density of hubs for percolation is dictated by the support of the connection function $f$, which reveals the critical role of long-distance travel (or its restrictions) in disease spreading.
\end{abstract}
\begin{IEEEkeywords}
Stochastic geometry, epidemic networks, disease propagation, random geometric bipartite graph, degrees, percolation
\end{IEEEkeywords}



\section{Introduction}


\begin{figure}[t]
\centering
\includegraphics[width=0.6\columnwidth]{figures/evaluation_desiderata_V5.pdf}
\vspace{-0.5cm}
\caption{\systemName is a platform for conducting realistic evaluations of code LLMs, collecting human preferences of coding models with real users, real tasks, and in realistic environments, aimed at addressing the limitations of existing evaluations.
}
\label{fig:motivation}
\end{figure}

\begin{figure*}[t]
\centering
\includegraphics[width=\textwidth]{figures/system_design_v2.png}
\caption{We introduce \systemName, a VSCode extension to collect human preferences of code directly in a developer's IDE. \systemName enables developers to use code completions from various models. The system comprises a) the interface in the user's IDE which presents paired completions to users (left), b) a sampling strategy that picks model pairs to reduce latency (right, top), and c) a prompting scheme that allows diverse LLMs to perform code completions with high fidelity.
Users can select between the top completion (green box) using \texttt{tab} or the bottom completion (blue box) using \texttt{shift+tab}.}
\label{fig:overview}
\end{figure*}

As model capabilities improve, large language models (LLMs) are increasingly integrated into user environments and workflows.
For example, software developers code with AI in integrated developer environments (IDEs)~\citep{peng2023impact}, doctors rely on notes generated through ambient listening~\citep{oberst2024science}, and lawyers consider case evidence identified by electronic discovery systems~\citep{yang2024beyond}.
Increasing deployment of models in productivity tools demands evaluation that more closely reflects real-world circumstances~\citep{hutchinson2022evaluation, saxon2024benchmarks, kapoor2024ai}.
While newer benchmarks and live platforms incorporate human feedback to capture real-world usage, they almost exclusively focus on evaluating LLMs in chat conversations~\citep{zheng2023judging,dubois2023alpacafarm,chiang2024chatbot, kirk2024the}.
Model evaluation must move beyond chat-based interactions and into specialized user environments.



 

In this work, we focus on evaluating LLM-based coding assistants. 
Despite the popularity of these tools---millions of developers use Github Copilot~\citep{Copilot}---existing
evaluations of the coding capabilities of new models exhibit multiple limitations (Figure~\ref{fig:motivation}, bottom).
Traditional ML benchmarks evaluate LLM capabilities by measuring how well a model can complete static, interview-style coding tasks~\citep{chen2021evaluating,austin2021program,jain2024livecodebench, white2024livebench} and lack \emph{real users}. 
User studies recruit real users to evaluate the effectiveness of LLMs as coding assistants, but are often limited to simple programming tasks as opposed to \emph{real tasks}~\citep{vaithilingam2022expectation,ross2023programmer, mozannar2024realhumaneval}.
Recent efforts to collect human feedback such as Chatbot Arena~\citep{chiang2024chatbot} are still removed from a \emph{realistic environment}, resulting in users and data that deviate from typical software development processes.
We introduce \systemName to address these limitations (Figure~\ref{fig:motivation}, top), and we describe our three main contributions below.


\textbf{We deploy \systemName in-the-wild to collect human preferences on code.} 
\systemName is a Visual Studio Code extension, collecting preferences directly in a developer's IDE within their actual workflow (Figure~\ref{fig:overview}).
\systemName provides developers with code completions, akin to the type of support provided by Github Copilot~\citep{Copilot}. 
Over the past 3 months, \systemName has served over~\completions suggestions from 10 state-of-the-art LLMs, 
gathering \sampleCount~votes from \userCount~users.
To collect user preferences,
\systemName presents a novel interface that shows users paired code completions from two different LLMs, which are determined based on a sampling strategy that aims to 
mitigate latency while preserving coverage across model comparisons.
Additionally, we devise a prompting scheme that allows a diverse set of models to perform code completions with high fidelity.
See Section~\ref{sec:system} and Section~\ref{sec:deployment} for details about system design and deployment respectively.



\textbf{We construct a leaderboard of user preferences and find notable differences from existing static benchmarks and human preference leaderboards.}
In general, we observe that smaller models seem to overperform in static benchmarks compared to our leaderboard, while performance among larger models is mixed (Section~\ref{sec:leaderboard_calculation}).
We attribute these differences to the fact that \systemName is exposed to users and tasks that differ drastically from code evaluations in the past. 
Our data spans 103 programming languages and 24 natural languages as well as a variety of real-world applications and code structures, while static benchmarks tend to focus on a specific programming and natural language and task (e.g. coding competition problems).
Additionally, while all of \systemName interactions contain code contexts and the majority involve infilling tasks, a much smaller fraction of Chatbot Arena's coding tasks contain code context, with infilling tasks appearing even more rarely. 
We analyze our data in depth in Section~\ref{subsec:comparison}.



\textbf{We derive new insights into user preferences of code by analyzing \systemName's diverse and distinct data distribution.}
We compare user preferences across different stratifications of input data (e.g., common versus rare languages) and observe which affect observed preferences most (Section~\ref{sec:analysis}).
For example, while user preferences stay relatively consistent across various programming languages, they differ drastically between different task categories (e.g. frontend/backend versus algorithm design).
We also observe variations in user preference due to different features related to code structure 
(e.g., context length and completion patterns).
We open-source \systemName and release a curated subset of code contexts.
Altogether, our results highlight the necessity of model evaluation in realistic and domain-specific settings.





\section{Model}
\label{sec:model}
Let $[N] = \{1, 2, \dots, N \}$ be a set of $N$ agents.
We examine an environment in which a system interacts with the agents over $T$ rounds.
Every round $t\leq T$ comprises $N$ \emph{sessions}, each session represents an encounter of the system with exactly one agent, and each agent interacts exactly once with the system every round.
I.e., in each round $t$ the agents arrive sequentially. 


\paragraph{Arrival order} The \emph{arrival order} of round $t$, denoted as $\ordv_t=(\ord_t(1),\dots, \ord_t(N))$, is an element from set of all permutations of $[N]$. Each entry $q$ in $\ordv_t$ is the index of the agent that arrives in the $q^{\text{th}}$ session of round $t$.
For example, if $\ord_t(1) = 2$, then agent $2$ arrives in the first session of round $t$.
Correspondingly, $\ord_t^{-1}(i)=q$ implies that agent $i$ arrives in the $q^{\text{th}}$ session of round $t$. 

As we demonstrate later, the arrival order has an immediate impact on agent rewards. We call the mechanism by which the arrival order is set \emph{arrival function} and denote it by $\ordname$. Throughout the paper, we consider several arrival functions such as the \emph{uniform arrival} function, denoted by $\uniord$, and the \emph{nudged arrival} $\sugord$; we introduce those formally in Sections~\ref{sec:uniform} and~\ref{sec:nudge}, respectively.

%We elaborate more on this concept in Section~\ref{sec: arrival}.


\paragraph{Arms} We consider a set of $K \geq 2$ arms, $A = \{a_1, \ldots, a_K\}$. The reward of arm $a_i$ in round $t$ is a random variable $X_i^t \sim \mathcal{D}^t_i$, where the rewards $(X_i^t)_{i,t}$ are mutually independent and bounded within the interval $[0,1]$. The reward distribution $\mathcal{D}^t_i$ of arm $a_i$, $i\in [K]$ at round $t\in T$ is assumed to be non-stationary but independent across arms and rounds. We denote the realized reward of arm $a_i$ in round $t$ by $x_i^t$. We assume \emph{reward consistency}, meaning that rewards may vary between rounds but remain constant within the sessions of a single round. Specifically, if an arm $a_i$ is selected multiple times during round~$t$, each selection yields the same reward $x_i^t$, where the superscript $t$ indicates its dependence on the round rather than the session. This consistency enables the system to leverage information obtained from earlier sessions to make more informed decisions in later sessions within the same round. We provide further details on this principle in Subsection~\ref{subsec:information}.


\paragraph{Algorithms} An algorithm is a mapping from histories to actions. We typically expect algorithms to maximize some aggregated agent metric like social welfare. Let $\mathcal H^{t,q}$ denote the information observed during all sessions of rounds $1$ to $t-1$ and sessions $1$ to $q-1$ in round $t$.  The history $\mathcal H^{t,q}$ is an element from $(A \times [0,1])^{(t-1) \cdot N +q-1}$, consisting of pairs of the form (pulled arm, realized reward). Notice that we restrict our attention to \emph{anonymous} algorithms, i.e., algorithms that do not distinguish between agents based on their identities. Instead, they only respond to the history of arms pulled and rewards observed, without conditioning on which specific agent performed each action.
%In the most general case, algorithms make decisions at session $q$ of round $t$  based on the entire history $\mathcal H^{t,q}$ and the index of the arriving agent $\ord_t(q)$. %Furthermore, we sometimes assume that algorithms have Bayesian information, i.e., algorithms are aware of the distributions $(\mathcal D_i)^K_{i=1}$. 
Furthermore, we sometimes assume that algorithms have Bayesian information, meaning they are aware of the reward distributions $(\mathcal{D}^t_i)_{i,t}$. If such an assumption is required to derive a result, we make it explicit. %Otherwise, we do not assume any additional knowledge about the algorithm’s information. %This distinction allows us to analyze both general algorithms without prior distributional knowledge and specialized algorithms that leverage Bayesian information.


\paragraph{Rewards} Let $\rt{i}$ denote the reward received by agent $i \in [N]$ at round $t$, and let $\Rt{i}$ denote her cumulative reward at the end of round $t$, i.e., $\Rt{i} = \sum_{\tau=1}^{t}{r^{\tau}_{i}}$. We further denote the \emph{social welfare} as the sum of the rewards all agents receive after $T$ rounds. Formally, $\sw=\sum^{N}_{i=1}{R^T_i}$. We emphasize that social welfare is independent of the arrival order. 


\paragraph{Envy}
We denote by $\adift{i}{j}$ the reward discrepancy of agents $i$ and $j$ in round $t$; namely, $\adift{i}{j}= \rt{i} - \rt{j}$. %We call this term \omer{name??} reward discrepancy in round $t$. 
The (cumulative) \emph{envy} between two agents at round $t$ is the difference in their cumulative rewards. Formally, $\env_{i,j}^t= \Rt{i} - \Rt{j}$ is the envy after $t$ rounds between agent $i$ and $j$. We can also formulate envy as the sum of reward discrepancies, $\env_{i,j}^t= \sum^{t}_{\tau=1}{\adif{i}{j}^\tau}$. Notice that envy is a signed quantity and can be either positive or negative. Specifically, if $\env_{i,j}^t < 0$, we say that agent $i$ envies agent $j$, and if $\env_{i,j}^t > 0$, agent $j$ envies agent $i$. The main goal of this paper is to investigate the behavior of the \emph{maximal envy}, defined as
\[
\env^t = \max_{i,j \in [N]} \env^t_{i,j}.
\]
For clarity, the term \emph{envy} will refer to the maximal envy.\footnote{ We address alternative definitions of envy in Section~\ref{sec:discussion}.} % Envy can also be defined in alternative ways, such as by averaging pairwise envy across all agents. We address average envy in Section~\ref{sec:avg_envy}.}
Note that $\env_{i,j}^t$ are random variables that depend on the decision-making algorithm, realized rewards, and the arrival order, and therefore, so is $\env^t$. If a result we obtain regarding envy depends on the arrival order $\ordname$, we write $\env^t(\ordname)$. Similarly, to ease notation, if $\ordname$ can be understood from the context, it is omitted.



\paragraph{Further Notation} We use the subscript $(q)$ to address elements of the $q^{\text{th}}$ session, for $q\in [N]$.
That is, we use the notation $\rt{(q)}$ to denote the reward granted to the agent that arrives in the $q^{\text{th}}$ session of round $t$ and $\Rt{(q)}$ to denote her cumulative reward. %Additionally, we introduce the notation $\at{(q)}$ to denote the arm pulled in that session.
Correspondingly, $\sdift{q}{w} = \rt{(q)} - \rt{(w)}$ is the reward discrepancy of the agents arriving in the $q^{\text{th}}$ and $w^{\text{th}}$ sessions of round $t$, respectively. 
To distinguish agents, arms, sessions and rounds, we use the letters $i,j$ to mark agents and arms, $q,w$ for sessions, and $t,\tau$ for rounds.


\subsection{Example}
\label{sec: example}
To illustrate the proposed setting and notation, we present the following example, which serves as a running example throughout the paper.

\begin{table}[t]
\centering
\begin{tabular}{|c|c|c|c|}
\hline
$t$ (round) & $\ordv_t$ (arrival order) & $x_1^t$ & $x_2^t$ \\ \hline
1           & 2, 1                     & 0.6     & 0.92    \\ \hline
2           & 1, 2                     & 0.48    & 0.1     \\ \hline
3           & 2, 1                     & 0.15    & 0.8     \\ \hline
\end{tabular}
\caption{
    Data for Example~\ref{example 1}.
}
\label{tbl: example}
\end{table}

\begin{algorithm}[t]
\caption{Algorithm for Example~\ref{example 1}}
\label{alguni}
\DontPrintSemicolon 
\For{round $t = 1$ to $T$}{
    pull $a_{1}$ in the first session\label{alguniexample: first}\\
    \lIf{$x^t_1 \geq \frac{1}{2}$}{pull $a_{1}$ again in second session \label{alguniexample: pulling a again}}
    \lElse{pull $a_{2}$ in second session \label{alguniexample: sopt else}}
}
\end{algorithm}


\begin{example}\label{example 1}
We consider $K=2$ uniform arms, $X_1,X_2 \sim \uni{0,1}$, and $N=2$ for some $T\geq 3$. We shall assume arm decision are made by Algorithm~\ref{alguni}: In the first session, the algorithm pulls $a_1$; if it yields a reward greater than $\nicefrac{1}{2}$, the algorithm pulls it again in the second session (the ``if'' clause). Otherwise, it pulls $a_2$.



We further assume that the arrival orders and rewards are as specified in Table~\ref{tbl: example}. Specifically, agent 2 arrives in the first session of round $t=1$, and pulling arm $a_2$ in this round would yield a reward of $x^1_2 = 0.92$. Importantly, \emph{this information is not available to the decision-making algorithm in advance} and is only revealed when or if the corresponding arms are pulled.




In the first round, $\boldsymbol{\eta}^1 = \left(2,1\right)$; thus, agent 2 arrives in the first session.
The algorithm pulls arm $a_1$, which means, $a^1_{(1)} = a_1$, and the agent receives $r_{2}^1=r_{(1)}^1=x_1^1=0.6$.
Later that round, in the second session, agent 1 arrives, and the algorithm pulls the same arm again since $x^1_1 = 0.6 \geq \nicefrac{1}{2}$ due to the ``if'' clause.
I.e., $a^1_{(2)} = a_1$ and $r_{1}^1 = r_{(2)}^1 = x_1^1 = 0.6$.
Even though the realized reward of arm $a_2$ in that round is higher ($0.92$), the algorithm is not aware of that value.
At the end of the first round, $R^1_1 = R^1_{(2)} = R^1_2 = R^1_{(1)} = 0.6$. The reward discrepancy is thus $\adif{1}{2}^1 = \adif{2}{1}^1= \sdif{2}{1}^1 = 0.6 - 0.6 =0$. 



In the second round, agent 1 arrives first, followed by agent 2.
Firstly, the algorithm pulls arm $a_1$ and agent 1 receives a reward of $r_{1}^2 = r_{(1)}^2 = x_1^2 = 0.48$.
Because the reward is lower than $\nicefrac{1}{2}$, in the second session the algorithm pulls the other arm ($a^2_{(2)} = a_2$), granting agent 2 a reward of $r_{2}^2 = r_{(2)}^2 = x_2^2 = 0.1$.
At the end of the second round, $R^2_1 = R^2_{(1)} = 0.6 + 0.48 = 1.08$ and $R^2_2 = R^2_{(2)} = 0.6 + 0.1 = 0.7$. Furthermore, $\sdif{2}{1}^2 = \adif{2}{1}^2 = r^2_{2} - r^2_{1} = 0.1 - 0.48 = -0.38$.

In the third and final round, agent 2 arrives first again, and receives a reward  of $0.15$ from $a_1$. When agent 1 arrives in the second session, the algorithm pulls arm $a_2$, and she receives a reward of $0.8$. As for the reward discrepancy, $\sdif{2}{1}^3 = \adif{2}{1}^3 = r^3_{2} - r^3_{1} = 0.15 - 0.8 = -0.75$. 

Finally, agent 1 has a cumulative reward of $R^3_1 = R^3_{(2)} = 0.6 + 0.48 + 0.8 = 1.88$, whereas agent~2 has a cumulative reward of $R^3_2 = R^3_{(1)} = 0.6 + 0.1 + 0.15 = 0.85$. In terms of envy, $\env^1_{1,2}= \adif{1}{2}^1 =0$, $\env^2_{1,2}=\adif{1}{2}^1+\adif{1}{2}^2= 0.38$, and $\env^3_{1,2} = -\env^3_{2,1} = R^3_1-R^3_2 = 1.88-0.85 = 1.03$, and consequently the envy in round 3 is $\env^3 = 1.03$.
\end{example}


\subsection{Information Exploitation}
\label{subsec:information}

In this subsection, we explain how algorithms can exploit intra-round information.
Since rewards are consistent in the sessions of each round, acquiring information in each session can be used to increase the reward of the following sessions.
In other words, the earlier sessions can be used for exploration, and we generally expect agents arriving in later sessions to receive higher rewards.
Taken to the extreme, an agent that arrives after all arms have been pulled could potentially obtain the highest reward of that round, depending on how the algorithm operates.

To further demonstrate the advantage of late arrival, we reconsider Example~\ref{example 1} and Algorithm~\ref{alguni}. 
The expected reward for the agent in the first session of round $t$ is $\E{\rt{(1)}}=\mu_1=\frac{1}{2}$, yet the expected reward of the agent in the second session is
\begin{align*}
\E{\rt{(2)}}=\E{\rt{(2)}\mid X^t_1 \geq \frac{1}{2} }\prb{X^t_1 \geq \frac{1}{2}} + \E{\rt{(2)}\mid X^t_1 < \frac{1}{2} }\prb{X^t_1 < \frac{1}{2}};
\end{align*}
thus, $\E{\rt{(2)}} =\E{X^t_1\mid X^t_1 \geq \frac{1}{2} }\cdot \frac{1}{2} + \mu_2\cdot\frac{1}{2} = \frac{5}{8}$.
Consequently, the expected welfare per round is $\E{\rt{(1)}+\rt{(2)}}=1+\frac{1}{8}$, and the benefit of arriving in the second session of any round $t$ is $\E{\rt{(2)} - \rt{(1)}} = \frac{1}{8}$. This gap creates envy over time, which we aim to measure and understand.
%This discrepancy generates envy over time, and our paper aims to better understand it.
\subsection{Socially Optimal Algorithms}
\label{sec: sw}
Since our model is novel, particularly in its focus on the reward consistency element, studying social welfare maximizing algorithms represents an important extension of our work. While the primary focus of this paper is to analyze envy under minimal assumptions about algorithmic operations, we also make progress in the direction of social welfare optimization. See more details in Section~\ref{sec:discussion}.%Due to space limitations, we defer the discussion on socially optimal algorithms to  \ifnum\Includeappendix=0{the appendix}\else{Section~\ref{appendix:sociallyopt}}\fi.




% Since our model is novel and specifically the reward consistency element, it might be interesting to study social welfare optimization. While the main focus of our paper is to study envy under minimal assumptions on how the algorithm operates, we take steps toward this direction as well. Due to space limitations, we defer the discussion on socially optimal algorithms to  \ifnum\Includeappendix=0{the appendix}\else{Section~\ref{appendix:sociallyopt}}\fi.  We devise a socially optimal algorithm for the two-agent case, offer efficient and optimal algorithms for special cases of $N>2$ agents, and an inefficient and approximately optimal algorithm for any instance with $N>2$. Moreover, we address the welfare-envy tradeoff in Section~\ref{sec:extensions}.


% Social welfare, unlike envy, is entirely independent of the arrival order. While the main focus of our paper is to study envy under minimal assumptions on how the algorithm operates, socially optimal algorithms might also be of interest. Due to space limitations, we defer the discussion on socially optimal algorithms to  \ifnum\Includeappendix=0{the appendix}\else{Section~\ref{appendix:sociallyopt}}\fi. We devise a socially optimal algorithm for the two-agent case, offer efficient and optimal algorithms for special cases of $N>2$ agents, and an inefficient and approximately optimal algorithm for any instance with $N>2$. %Furthermore, we treat the welfare-envy tradeoff of the special case of Example~\ref{example 1}.




    \section{Degrees of the RBG Graph}
In this section, we consider some basic properties of the RBG graph $\mathcal{G}(\Phi,\Psi, f)$. Let $\Phi^o\triangleq \Phi\mid o\in\Phi.$ We focus on the typical agent at the origin and the Palm distribution of $\Phi$
\begin{equation}
 \mathrm{P}_{}^{o}(\cdot) \triangleq \Pr(\Phi\in\cdot \mid o\in\Phi),\nonumber
\end{equation}
and the reduced Palm distribution
\begin{equation}
 \mathrm{P}_{}^{!o}(\cdot) \triangleq \Pr(\Phi\setminus\{o\}\in\cdot\mid o\in\Phi).\nonumber
\end{equation}
We denote by $\Ex_{}^{{o}}$ the  expectation with respect to the Palm measure of $\Phi$
   and by $\Ex_{}^{!o}$ the  expectation with respect to the reduced Palm measure of $\Phi$ \cite[Chapter 8]{haenggi2012stochastic}.
   %of $\Phi$, i.e., $\Ex^{!}_{{o}}f(\Phi) \triangleq \Ex (f(\Phi\setminus\{o\})\mid o\in\Phi).$
    




%There are several quantities of interest and connection functions: the degree (the number of edges) of the typical node/hub, the number of connections of the typical node, and the number of connected nodes of the typical node.
\subsection{General Point Processes}
%To simplify the notation, let $\Ex^{!o}$ denote the (total) expectation where the expectation with respect to $\Phi$ is its reduced Palm expectation.

Let $c_d$ denote the volume of unit ball in $\mathbb{R}^d$.
The mean degree of the typical agent in a general point process is 
%  $\sum_{y\in\Psi} \ind(o\to y)$ with mean 
    \begin{align}
  \Ex^{o} \sum_{y\in\Psi}I(o,y)
       &\peq{a} \Ex \sum_{y\in\Psi}f(\|y\|)\nonumber
      \\
      &\peq{b} \mu c_d d \int_{0}^{\infty}f(r) r^{d-1} \dd r,\label{eq: EN_hub}
    \end{align}
    where step (a) follows from the independence of $\Phi$ and $\Psi$, and step (b) follows from Campbell's theorem. 
    By the mass transport principle \cite{baccelli:hal-02460214}, the mean degree of the typical hub (defined through the Palm expectation of $\Psi$) is $\lambda c_d d \int_{0}^{\infty}f(r) r^{d-1} \dd r.$ %Similarly, the sum length of the edges of the typical node is given by $ \Ex \sum_{y\in\Psi} \|y\|I(o, y)= \mu c_d d \int_{0}^{\infty}f(r) r^{d} \dd r.$
    For example, consider $d=2$, a general connection function $f$, and its dispersed version $f_{p}$. While the mean degree remains constant for any $p\in(0,1]$,
%, i.e., \[\int_{0}^{\infty}f_p(r)2\pi r \dd r \equiv \int_{0}^{\infty}f(r)2\pi r \dd r.\]
 the edges spreads out in space as $p\to 0$. To see this, consider the mean distance
 \begin{align}
 \Ex^{o} \sum_{y\in\Psi}I(o,y)\|y\| &=\int_{0}^{\infty}f_p(r)2\pi r^2 \dd r\nonumber \\
 & = \frac{1}{\sqrt{p}}\int_{0}^{\infty}f(r)2\pi r^2 \dd r,\nonumber
 \end{align}
which increases as the connection function becomes more dispersed.


    
 Let $M\triangleq\sum_{x\in\Phi^o\setminus\{o\}}\max_{y\in\Psi} I(o,y)I(x,y)$ denote the number of agents connected to $o$.  We have
    \begin{align}
        \Ex M &= \Ex\sum_{x\in\Phi^o\setminus\{o\} } 1-\prod_{y\in\Psi} \left(1-I(o,y)I(x,y)\right) \nonumber\\
        &= \Ex^{!o} \sum_{x\in\Phi} 1-\prod_{y\in\Psi} (1-f(\|y\|)f(\|x-y\|)).\nonumber
        \end{align}
    This follows from the fact that the edge indicator functions are independent Bernoulli random variables given $\Phi$ and $\Psi$.
    
    Let $N\triangleq\sum_{y\in\Psi}\sum_{x \in\Phi^o\setminus\{o\}} I(o,y)I(x,y)$ denote the total number of direct (two-edge) paths between $o$ and its connected agents. We have
    \begin{align}
       \Ex N  &= \Ex \sum_{y\in\Psi}\sum_{x \in\Phi^o\setminus\{o\}} I(o,y)I(x,y) \nonumber\\
       &= \Ex^{!o}\sum_{y\in\Psi}\sum_{x \in\Phi}f(\|y\|)f(\|x-y\|).\nonumber
       %\\ \nonumber & =  \Ex_{\Psi} \sum_{y\in\Psi}f(\|y\|) \Ex^{!}_{{o}}\sum_{x \in\Phi} f(\|x-y\|),
    \end{align}  
    

 
    
By definition, $N\geq M$, and the equality is achieved when $o$ and $x$ are connected through at most one hub, $\forall x\in\Phi$. The difference between $N$ and $M$ reflects how likely two agents connect through multiple hubs. The variance of $M$ is $\mathbb{V} M \triangleq \Ex M^2 - (\Ex M)^2$, which characterizes the heterogeneity of the number of connected agents.
   
   
    
    % \subsection{The Basic Reproduction Number}
    %   The basic reproduction number $R_0$ is widely used in epidemiology to predict whether an infectious disease will spread. However, it is difficult to calculate in general and its effectiveness and interpretation rely on the model \cite{Li2011TheFO}. While $R_0$ captures the initial state of the disease propagation, it may not predict the spread in a global level. For instance, consider a model where each node always connects with its nearest hub and $\alpha=1$. Then $R_0=N-1$ assuming each hub on average connects to $N$ nodes. However, due to the locality of connections, the disease only spreads within nodes connected to the same hub and always dies out for an arbitrarily large $N$. 
       
%      \begin{definition}[The basic reproduction number ]
%   \end{definition}


%Without loss of generality we assume that the single infected node is at the origin.
   


%   \begin{remark}
%         For given $\Phi$ and $\Psi$, denote the expected number of secondary infections generated by the typical infected node by $R(\Phi,\Psi)$. $R(\Phi,\Psi)$ is the conditional basic reproduction number given node and hub locations.      Given $\Phi,\Psi$, the number of secondary infections caused by $o$ is 
%      \begin{align}
%         R(\Phi,\Psi)&= \Ex \left[\sum_{x\in\Phi}\max_{1\leq k\leq K} \ind (Z_x^{k} = \mathrm{I}\mid Z_x^{0}=\mathrm{S},Z_o^{0} = \mathrm{I})\mid \Phi,\Psi\right]\\
%          &= \Ex \left[\sum_{x\in\Phi}  1- (1-\alpha)^{K  N_x}\mid\Phi,\Psi\right]\\
%          & = \Ex \left[\sum_{x\in\Phi}  1- \prod_{y\in\Psi} (1-\alpha)^{K \ind(o\to y)\ind(x\to y)}\mid\Phi,\Psi\right]\\
%          & \peq{a} \sum_{x\in\Phi}  1- \prod_{y\in\Psi}\left[ (1-\alpha)^{K}f(\|y\|)f(\|x-y\|)+1-f(\|y\|)f(\|x-y\|)\right]
%     %  \\& = 1- \prod_{\Psi} \left(1-t\gamma^2\exp\big(-\theta (\|y\|+\|x-y\|)\big)\right)^K.
%         \end{align}
% where step (a) follows from the independence of $\{I(x,y):x\in\Phi\}$ given $\Psi$ and $\Phi$. 
%     %  \\
%     %  & = 1-\exp\left(-\int_{\|x\|}^{\infty} \lambda_{x}(r) \Big(1-\big(1-t\gamma^2\exp(-\theta r)\big)^K\Big)\dd r\right),\label{eq: bessel}
%      By Definition 1, \begin{equation}
%          R_0 \triangleq \Ex R_{0}(\Phi,\Psi) = \Ex_{\Psi}\Ex_{o}^{!} R_{0}(\Phi,\Psi).
%      \end{equation}   
%   \end{remark}
    
    
  
    
    % and prove its existence for the static model. Two observations:\\
    % - the dynamic RBG graph's threshold is lower-bounded by the static one.\\
    % - the percolation threshold should decrease with the dispersiveness of $f$ using the idea in \cite{Franceschetti05continuumpercolation}

    
% \section{Analysis}
\label{sec:analysis}
In the following sections, we will analyze European type approval regulation\footnote{Strictly speaking, the German enabling act (AFGBV) does not regulate type-approval, but how test \& operating permits are issued for SAE-Level-4 systems. Type-approval regulation for SAE-Level-3 systems follows UN Regulation No. 157 (UN-ECE-ALKS) \parencite{un157}.} regarding the underlying notions of ``safety'' and ``risk''.
We will classify these notions according to their absolute or relative character, underlying risk sources, or underlying concepts of harm.

\subsection{Classification of Safety Notions}
\label{sec:safety-notions}
We will refer to \emph{absolute} notions of safety as conceptualizations that assume the complete absence of any kind of risk.
Opposed to this, \emph{relative} notions of safety are based on a conceptualization that specifically includes risk acceptance criteria, e.g., in terms of ``tolerable'' risk or ``sufficient'' safety.

For classifying notions of safety by their underlying risk (or rather ``hazard'') sources, and different concepts of harm, \Cref{fig:hazard-sources} provides an overview of our reasoning, which is closely in line with the argumentation provided by Waymo in \parencite{favaro2023}.
We prefer ``hazard sources'' over ``risk sources'', as a risk must always be related to a \emph{cause} or \emph{source of harm} (i.e., a hazard \parencite[p.~1, def. 3.2]{iso51}).
Without a concrete (scenario) context that the system is operating in, a hazard is \emph{latent}: E.g., when operating in public traffic, there is a fundamental possibility that a \emph{collision with a pedestrian} leads to (physical) harm for that pedestrian. 
However, only if an automated vehicle shows (potentially) hazardous behavior (e.g., not decelerating properly) \emph{and} is located near a pedestrian (context), the hazard is instantiated and leads to a hazardous event.
\begin{figure*}
    \includeimg[width=.9\textwidth]{hazard-sources0.pdf}
    \caption{Graphical summary of a taxonomy of risk related to automated vehicles, extended based on ISO 21448 (\parencite{iso21448}) and \parencite{favaro2023}. Top: Causal chain from hazard sources to actual harm; bottom: summary of the individual elements' contributions to a resulting risk. Graphic translated from \parencite{nolte2024} \label{fig:hazard-sources}}
\end{figure*}
If the hazardous event cannot be mitigated or controlled, we see a loss event in which the pedestrian's health is harmed.
Note that this hypothetical chain of events is summarized in the definition of risk:
The probability of occurrence of harm is determined by a) the frequency with which hazard sources manifest, b) the time for which the system operates in a context that exposes the possibility of harm, and c) by the probability with which a hazardous event can be controlled.
A risk can then be determined as a function of the probability of harm and the severity of the harm potentially inflicted on the pedestrian.

In the following, we will apply this general model to introduce different types of hazard sources and also different types of harm.
\cref{fig:hazard-sources} shows two distinct hazard sources, i.e., functional insufficiencies and E/E-failures that can lead to hazardous behavior.
ISO~21488 \parencite{iso21448} defines functional insufficiencies as insufficiencies that stem from an incomplete or faulty system specification (specification insufficiencies).
In addition, the standard considers insufficiencies that stem from insufficient technical capability to operate inside the targeted Operational Design Domain (performance insufficiencies).
Functional insufficiencies are related to the ``Safety of the Intended Functionality (SOTIF)'' (according to ISO~21448), ``Behavioral Safety'' (according to Waymo \parencite{waymo2018}), or ``Operational Safety'' (according to UN Regulation No. 157 \parencite{un157}).
E/E-Failures are related to classic functional safety and are covered exhaustively by ISO~26262 \parencite{iso2018}.
Additional hazard sources can, e.g., be related to malicious security attacks (ISO~21434), or even to mechanical failures that should be covered (in the US) in the Federal Motor Vehicle Safety Standards (FMVSS).

For the classification of notions of safety by the related harm, in \parencite{salem2024, nolte2024}, we take a different approach compared to \parencite{koopman2024}:
We extend the concept of harm to the violation of stakeholder \emph{values}, where values are considered to be a ``standard of varying importance among other such standards that, when combined, form a value pattern that reduces complexity for stakeholders [\ldots] [and] determines situational actions [\ldots].'' \parencite{albert2008}
In this sense, values are profound, personal determinants for individual or collective behavior.
The notion of values being organized in a weighted value pattern shows that values can be ranked according to importance.
For automated vehicles, \emph{physical wellbeing} and \emph{mobility} can, e.g., be considered values which need to be balanced to achieve societal acceptance, in line with the discussion of required tradeoffs in \cref{sec:terminology}.
For the analysis of the following regulatory frameworks, we will evaluate if the given safety or risk notions allow tradeoffs regarding underlying stakeholder values. 

\subsection{UN Regulation No. 157 \& European Implementing Regulation (EU) 2022/1426}
\label{sec:enabling-act}
UN Regulation No. 157 \parencite{un157} and the European Implementing Regulation 2022/1426 \parencite{eu1426} provide type approval regulation for automated vehicles equipped with SAE-Level-3 (UN Reg. 157) and Level 4 (EU 2022/1426) systems on an international (UN Reg. 157) and European (EU 2022/1426) level.

Generally, EU type approval considers UN ECE regulations mandatory for its member states ((EU) 2018/858, \parencite{eu858}), while the EU largely forgoes implementing EU-specific type approval rules, it maintains the right to alter or to amend UN ECE regulation \parencite{eu858}.

In this respect, the terminology and conceptualizations in the EU Implementing Act closely follow those in UN Reg. No. 157.
The EU Implementing Act gives a clear reference to UN Reg. No. 157 \parencite[][Preamble,  Paragraph 1]{eu1426}.
Hence, the documents can be assessed in parallel.
Differences will be pointed out as necessary.

Both acts are written in rather technical language, including the formulation of technical requirements (e.g., regarding deceleration values or speeds in certain scenarios).
While providing exhaustive definitions and terminology, neither of both documents provide an actual definition of risk or safety.
The definition of ``unreasonable'' risk in both documents does not define risk, but only what is considered \emph{unreasonable}. It states that the ``overall level of risk for [the driver, (only in UN Reg. 157)] vehicle occupants and other road users which is increased compared to a competently and carefully driven manual vehicle.''
The pertaining notions of safety and risk can hence only be derived from the context in which they are used.

\subsubsection{Absolute vs. Relative Notions of Safety}
In line with the technical detail provided in the acts, both clearly imply a \emph{relative} notion of safety and refer to the absence of \emph{unreasonable} risk throughout, which is typical for technical safety definitions.

Both acts require sufficient proof and documentation that the to-be-approved automated driving systems are ``free of unreasonable safety risks to vehicle occupants and other road users'' for type approval.\footnote{As it targets SAE-Level-3 systems, UN Reg. 157 also refers to the driver, where applicable.}
In this respect, both acts demand that the manufacturers perform verification and validation activities for performance requirements that include ``[\ldots] the conclusion that the system is designed in such a way that it is free from unreasonable risks [\ldots]''.
Additionally, \emph{risk minimization} is a recurring theme when it comes to the definition of Minimum Risk Maneuvers (MRM).

Finally, supporting the relative notions of safety and risk, UN Reg. 157 introduces the concept of ``reasonable foreseeable and preventable'' \parencite[Article 1, Clause 5.1.1.]{un157} collisions, which implies that a residual risk will remain with the introduction of automated vehicles.
\parencite[][Appendix 3, Clause 3.1.]{un157} explicitly states that only \emph{some} scenarios that are unpreventable for a competent human driver can actually be prevented by an automated driving system.
While this concept is not applied throughout the EU Implementing Act, both documents explicitly refer to \emph{residual} risks that are related to the operation of automated driving systems (\parencite[][Annex I, Clause 1]{un157}, \parencite[][Annex II, Clause 7.1.1.]{eu1426}).

\subsubsection{Hazard Sources}
Hazard sources that are explicitly differentiated in UN Reg. 157 and (EU) 2022/1426 are E/E-failures that are in scope of functional safety (ISO~26262) and functional insufficiencies that are in scope of behavioral (or ``operational'') safety (ISO~21448).
Both documents consistently differentiate both sources when formulating requirements.

While the acts share a common definition of ``operational'' safety (\parencite[][Article 2, def. 30.]{eu1426}, \parencite[][Annex 4, def. 2.15.]{un157}), the definitions for functional safety differ.
\parencite{un157} defines functional safety as the ``absence of unreasonable risk under the occurrence of hazards caused by a malfunctioning behaviour of electric/electronic systems [\ldots]'', \parencite{eu1426} drops the specification of ``electric/electronic systems'' from the definition.
When taken at face value, this definition would mean that functional safety included all possible hazard sources, regardless of their origin, which is a deviation from the otherwise precise usage of safety-related terminology.

\subsubsection{Harm Types}
As the acts lack explicit definitions of safety and risk, there is no consistent and explicit notion of different harm types that could be differentiated.

\parencite{un157} gives little hints regarding different considered harm types.
``The absence of unreasonable risk'' in terms of human driving performance could hence be related to any chosen performance metric that allows a comparison with a competent careful human driver including, e.g., accident statistics, statistics about rule violations, or changes in traffic flow.

In \parencite{eu1426}, ``safety'' is, implicitly, attributed to the absence of unreasonable risk to life and limb of humans.
This is supported by the performance requirements that are formulated:
\parencite[][Annex II, Clause 1.1.2. (d)]{eu1426} demands that an automated driving system can adapt the vehicle behavior in a way that it minimizes risk and prioritizes the protection of human life.

Both acts demand the adherence to traffic rules (\parencite[][Annex 2, Clause 1.3.]{eu1426}, \parencite[][Clause 5.1.2.]{un157}).
\parencite[][Annex II, Clause 1.1.2. (c)]{eu1426} also demands that an automated driving system shall adapt its behavior to surrounding traffic conditions, such as the current traffic flow.
With the relative notion of risk in both acts, the unspecific clear statement that there may be unpreventable accidents \parencite{un157}, and a demand of prioritization of human life in \parencite{eu1426}, both acts could be interpreted to allow developers to make tradeoffs as discussed in \cref{sec:terminology}.


\subsubsection{Conclusion}
To summarize, the UN Reg. 157 and the (EU) 2022/1426 both clearly support the technical notion of safety as the absence of unreasonable risk.
The notion is used consistently throughout both documents, providing a sufficiently clear terminology for the developers of automated vehicles.
Uncertainty remains when it comes to considered harm types: Both acts do not explicitly allow for broader notions of safety, in the sense of \parencite{koopman2024} or \parencite{salem2024}.
Finally, a minor weak spot can be seen in the definition of risk acceptance criteria: Both acts take the human driving performance as a baseline.
While (EU) 2022/1426 specifies that these criteria are specific to the systems' Operational Design Domain \parencite[][Annex II, Clause 7.1.1.]{eu1426}, the reference to the concrete Operational Design Domain is missing in UN Reg. 157.
Without a clearly defined notion of safety, however, it remains unclear, how aspects beyond net accident statistics (which are given as an example in \parencite[][Annex II, Clause 7.1.1.]{eu1426}), can be addressed practically, as demanded by \parencite{koopman2024}.

\subsection{German Regulation (StVG \& AFGBV)}
\label{sec:afgbv}
The German L3 (Automated Driving Act) and L4 (Act on Autonomous Driving) Acts from 2017 and 2021,\footnote{Formally, these are amendments to the German Road Traffic Act (StVG): 06/21/2017, BGBl. I p. 1648, 07/12/2021 BGBl. I p. 3108.} respectively, provide enabling regulation for the operation of SAE-Level-3 and 4 vehicles on German roads.
The German Implementing Regulation (\parencite{afgbv}, AFGBV) defines how this enabling regulation is to be implemented for granting testing permits for SAE-Level-3 and -4 and driving permits for SAE-Level-3 and -4 automated driving systems.\footnote{Note that these permits do not grant EU-wide type approval, but serve as a special solution for German roads only. At the same time, the AFGBV has the same scope as (EU) 2022/1426.}
With all three acts, Germany was the first country to regulate the approval of automated vehicles for a domestic market.
All acts are subject to (repeated) evaluation until the year 2030 regarding their impact on the development of automated driving technology.
An assessment of the German AFGBV and comparisons to (EU) 2022/1426 have been given in \cite{steininger2022} in German.

Just as for UN Reg. 157 and (EU) 2022/1426, neither the StVG nor the AFGBV provide a clear definition of ``safety'' or ``risk'' -- even though the "safety" of the road traffic is one major goal of the StVG and StVO.
Again, different implicit notions of both concepts can only be interpreted from the context of existing wording.
An additional complication that is related to the German language is that ``safety'' and ``security'' can both be addressed as ``Sicherheit'', adding another potential source of unclarity.
Literal Quotations in this section are our translations from the German act.

\subsubsection{Absolute vs. Relative Notions of Safety}
For assessing absolute vs. relative notions of safety in German regulation, it should be mentioned that the main goal of the German StVO is to ensure the ``safety and ease of traffic flow'' -- an already diametral goal that requires human drivers to make tradeoffs.\footnote{For human drivers, this also creates legal uncertainty which can sometimes only be settled in a-posteriori court cases.}
While UN and EU regulation clearly shows a relative notion of safety\footnote{And even the StVG contains sections that use wording such as ``best possible safety for vehicle occupants'' (§1d (4) StVG) and acknowledges that there are unavoidable hazards to human life (§1e (2) No. 2c)).}, the German AFGBV contains ambiguous statements in this respect:
Several paragraphs contain a demand for a hazard free operation of automated vehicles.
§4 (1) No. 4 AFGBV, e.g., states that ``the operation of vehicles with autonomous driving functions must neither negatively impact road traffic safety or traffic flow, nor endanger the life and limb of persons.''
Additionally, §6 (1) AFGBV states that the permits for testing and operation have to be revoked, if it becomes apparent that a ``negative impact on road traffic safety or traffic flow, or hazards to the life and limb of persons cannot be ruled out''.
The same wording is used for the approval of operational design domains regulated in §10 (1) No. 1.
A particularly misleading statement is made regarding the requirements for technical supervision instances which are regulated in §14 (3) AFGBV which states that an automated vehicle has to be  ``immediately removed from the public traffic space if a risk minimal state leads to hazards to road traffic safety or traffic flow''.
Considering the argumentation in \cref{sec:terminology}, that residual risks related to the operation of automated driving systems are inevitable, these are strong statements which, if taken at face value, technically prohibit the operation of automated vehicles.
It suggests an \emph{absolute} notion of safety that requires the complete absence of risk.  
The last statement above is particularly contradictory in itself, considering that a risk \emph{minimal} state always implies a residual risk.

In addition to these absolute safety notions, there are passages which suggest a relative notion of safety:
The approval for Operational Design Domains is coupled to the proof that the operation of an automated vehicle ``neither negatively impacts road traffic safety or traffic flow, nor significantly endangers the life and limb of persons beyond the general risk of an impact that is typical of local road traffic'' (§9 (2) No. 3 AFGBV).
The addition of a relative risk measure ``beyond the general risk of an impact'' provides a relaxation (cf. also \cite{steininger2022}, who criticizes the aforementioned absolute safety notion) that also yields an implicit acceptance criterion (\emph{statistically as good as} human drivers) similar to the requirements stated in UN Reg. 157 and (EU) 2022/1426.

Additional hints for a relative notion of safety can be found in Annex 1, Part 1, No. 1.1 and Annex 1, Part 2, No. 10.
Part 1, No 1.1 specifies collision-avoidance requirements and acknowledges that not all collisions can be avoided.\footnote{The same is true for Part 2, No. 10, Clause 10.2.5.}
Part 2, No. 10 specifies requirements for test cases.
It demands that test cases are suitable to provide evidence that the ``safety of a vehicle with an autonomous driving function is increased compared to the safety of human-driven vehicles''.
This does not only acknowledge residual risks, but also yields an acceptance criterion (\emph{better} than human drivers) that is different from the implied acceptance criterion given in §9 (2) No. 3 AFGBV.

\subsubsection{Hazard Sources}
Regarding hazard sources, Annex 1 and 3 AFGBV explicitly refer to ISO~26262 and ISO~21448 (or rather its predecessor ISO/PAS~21448:2019).
However, regarding the discussion of actual hazard sources, the context in which both standards are mentioned is partially unclear:
Annex 1, Clause 1.3 discusses requirements for path and speed planning.
Clause 1.3 d) demands that in intersections, a Time to Collision (TTC) greater than 3 seconds must be guaranteed.
If manufacturers deviate from this, it is demanded that ``state-of-the-art, systematic safety evaluations'' are performed.
Fulfillment of the state of the art is assumed if ``the guidelines of ISO~26262:2018-12 Road Vehicles -- Functional Safety are fulfilled''.
Technically, ISO~26262 is not suitable to define the state of the art in this context, as the requirements discussed fall in the scope of operational (or behavioral) safety (ISO~21448).
A hazard source ``violated minimal time to collision'' is clearly a functional insufficiency, not an E/E-failure.

Similar unclarity presents itself in Annex 3, Clause 1 AFGBV: 
Clause 1 specifies the contents of the ``functional specification''.
The ``specification of the functionality'' is an artifact which is demanded in ISO~21448:2022 (Clause 5.3) \parencite{iso21448}.
However, Annex 3, Clause 1 AFGBV states that the ``functional specification'' is considered to comply to the state of the art, if the ``functional specification'' adheres to ISO~26262-3:2018 (Concept Phase).
Again, this assumes SOTIF-related contents as part of ISO~26262, which introduces the ``Item Definition'' as an artifact, which is significantly different from the ``specification of the functionality'' which is demanded by ISO~21448.
Finally, Annex 3, Clause 3 AFGBV demands a ``documentation of the safety concept'' which ``allows a functional safety assessment''.
A safety concept that is related to operational / behavioral safety is not demanded.
Technically, the unclarity with respect to the addressed harm types lead to the fact that the requirements provided by the AFGBV do not comply with the state of the art in the field, providing questionable regulation.

\subsubsection{Harm Types}
Just like UN Reg. 157 and (EU) 2022/1426, the German StVG and AFGBV do not explicitly differentiate concrete harm types for their notions of safety.
However, the AFGBV mentions three main concerns for the operation of automated vehicles which are \emph{traffic flow} (e.g., §4 (1) No. 4 AFGBV), compliance to \emph{traffic law} (e.g., §1e (2) No. 2 StVG), and the \emph{life and limb of humans} (e.g., §4 (1) No. 4 AFGBV).

Again, there is some ambiguity in the chosen wording:
The conflict between traffic flow and safety has already been argued in \cref{sec:terminology}.
The wording given in §4 (1) No. 4 and §6 (1) AFGBV  demand to ensure (absolute) safety \emph{and} traffic flow at the same time, which is impossible (cf. \cref{sec:terminology}) from an engineering perspective.
§1e (2) No. 2 StVG defines that ``vehicles with an autonomous driving function must [\ldots] be capable to comply to [\ldots] traffic rules in a self-contained manner''.
Taken at face value, this wording implies that an automated driving system could lose its testing or operating permit as soon as it violates a traffic rule.
A way out could be provided by §1 of the German Traffic Act (StVO) which demands careful and considerate behavior of all traffic participants and by that allows judgement calls for human drivers.
However, if §1 is applicable in certain situations is often settled in court cases. 
For developers, the application of §1 StVO during system design hence remains a legal risk.

While there are rather absolute statements as mentioned above, sections of the AFGBV and StVG can be interpreted to allow tradeoffs:
§1e (2) No. 2 b) demands that a system,  ``in case of an inevitable, alternative harm to legal objectives, considers the significance of the legal objectives, where the protection of human life has highest priority''.
This exact wording \emph{could} provide some slack for the absolute demands in other parts of the acts, enabling tradeoffs between (tolerable) risk and mobility as discussed in \cref{sec:terminology}.
However, it remains unclear if this interpretation is legally possible.

\subsubsection{Conclusion}
Compared to UN Reg. 157 and (EU) 2022/1426, the German StVG and AFGBV introduce openly inconsistent notions of safety and risk which are partially directly contradictory:
The wording partially implies absolute and relative notions of safety and risk at the same time.
The implied validation targets (``better'' or ``as good as'' human drivers) are equally contradictory. 
The partially implied absolute notions of safety, when taken at face value, prohibit engineers from making the tradeoffs required to develop a system that is safe and provides customer benefit at the same time. 
In consequence, the wording in the acts is prone to introducing legal uncertainty.
This uncertainty creates additional clarification need and effort for manufacturers and engineers who design and develop SAE-Level-3 and -4 automated driving systems. The use of undefined legal terms not only makes it more difficult for engineers to comply with the law, but also complicates the interpretation of the law and leads to legal uncertainty.

\subsection{UK Automated Vehicles Act 2024 (2024 c. 10)}
The UK has issued a national enabling act for regulating the approval of automated vehicles on the roads in the UK.
To the best of our knowledge, concrete implementing regulation has not been issued yet.
Regarding terminology, the act begins with a dedicated terminology section to clarify the terms used in the act \parencite[Part 1, Chapter 1, Section 1]{ukav2024}.
In that regard, the act defines a vehicle to drive ```autonomously' if --- (a)
it is being controlled not by an individual but by equipment of the vehicle, and (b) neither the vehicle nor its surroundings are being monitored by an individual with a view to immediate intervention in the driving of the vehicle.''
The act hence covers SAE-Level-3 to SAE-Level-5 automated driving systems.

\subsubsection{Absolute vs. Relative Notions of Safety}
While not providing an explicit definition of safety and risk, the UK Automated Vehicles Act (``UK AV Act'') \parencite{ukav2024} explicitly refers to a relative notion of safety.
Part~1, Chapter~1, Section~1, Clause (7)~(a) defines that an automated vehicle travels ```safely' if it travels to an acceptably safe standard''.
This clarifies that absolute safety is not achievable and that acceptance criteria to prove the acceptability of residual risk are required, even though a concrete safety definition is not given.
The act explicitly tasks the UK Secretary of State\footnote{Which means, that concrete implementation regulation needs to be enacted.} to install safety principles to determine the ``acceptably safe standard'' in Part~1, Chapter~1, Section~1, Clause (7)~(a).
In this respect, the act also provides one general validation target as it demands that the safety principles must ensure that ``authorized automated vehicles will achieve a level of safety equivalent to, or higher than, that of careful and competent human drivers''.
Hence, the top-level validation risk acceptance criterion assumed for UK regulation is ``\emph{at least as good} as human drivers''.

\subsubsection{Hazard Sources}
The UK AV Act contains no statements that could be directly related to different hazard sources.
Note that, in contrast to the rest of the analyzed documents, the UK AV Act is enabling rather than implementing regulation.
It is hence comparable to the German StVG, which does not refer to concrete hazard sources as well.

\subsubsection{Types of Harm}
Even though providing a clear relative safety notion, the missing definition of risk also implies a lack of explicitly differentiable types of harm.
Implicitly, three different types of harm can be derived from the wording in the act.
This includes the harm to life and limb of humans\footnote{Part~1, Chapter~3, Section~25 defines ``aggravated offence where death or serious injury occurs'' \parencite{ukav2024}.}, the violation of traffic rules\footnote{Part~1, Chapter~1, Clause~(7)~(b) defines that an automated vehicle travels ```legally' if it travels with an acceptably low risk of committing a traffic infraction''}, and the cause of inconvenience to the public \parencite[Part~1, Chapter~1, Section~58, Clause (2)~(d)]{ukav2024}.

The act connects all the aforementioned types of harm to ``risk'' or ``acceptable safety''.
While the act generally defines criminal offenses for providing ``false or misleading information about safety'', it also acknowledges possible defenses if it can be proven that ``reasonable precautions'' were taken and that ``due diligence'' was exercised to ``avoid the commission of the offence''.
This statement could enable tradeoffs within the scope of ``reasonable risk'' to the life and limb of humans, the violation of traffic rules, or to the cause of inconvenience to the public, as we argued in \cref{sec:terminology}.

\subsubsection{Conclusion}
From the set of reviewed documents, the current UK AV Act is the one with the most obvious relative notions of safety and risk and the one that seems to provide a legal framework for permitting tradeoffs.
In our review, we did not spot major inconsistency beyond a missing definitions of safety and risk\footnote{Note that with the Office for Product Safety and Standards (OPSS), there is a British government agency that maintains an exhaustive and widely focussed ``Risk Lexicon'' that provides suitable risk definitions. For us, it remains unclear, to what extent this terminology is assumed general knowledge in British legislation.}.
The general, relative notion of safety and the related alleged ability for designers to argue well-founded development tradeoffs within the legal framework could prove beneficial for the actual implementation of automated driving systems.
While the act thus appears as a solid foundation for the market introduction of automated vehicles, without accompanying implementing regulation, it is too early to draw definite conclusions.
   

%     $R_0>1$ alone is not sufficient for the disease spread. 
%     \begin{conjecture}
%     A necessary condition for the disease spread is that the Poisson random connection model of $\Psi$ with connection function $g$ percolates.
%     \end{conjecture}
% If $\Phi$ percolate, then $\Psi$ percolate? 
%We say that two hubs are connected if they connect to a common node... This leads to a connection function $g(r)$. Does the percolation of $\Psi$ with $g(r)$ indicate the percolation of $\Phi$ with $f$? 
% tThe probability that infinite nodes get infected is 0 if the largest connected component of the random connection model of $\Psi$ associated with $g$ is finite (to be proved).


\subsection{Poisson Point Processes}
\begin{figure*}[t]
\normalsize
\setcounter{MYtempeqncnt}{\value{equation}}
\setcounter{equation}{\value{equation}+2}
 \begin{align}
       \mathbb{V} N
       & = \Ex N+\lambda^2\mu\left(\int_{0}^{\infty}f(r)dc_d r^{d-1} \dd r\right)^3 \nonumber%\int_{\mathbb{R}^2}\int_{\mathbb{R}^2}\int_{\mathbb{R}^2}f(\|y\|)f(\|x_1-y\|)f(\|x_2-y\|) \dd y\dd x_1 \dd x_2\nonumber\\
       \\
       &\quad+\lambda\mu^2 \int_{\mathbb{R}^d}\int_{\mathbb{R}^d}\int_{\mathbb{R}^d}f(\|y_1\|)f(\|y_2\|) f(\|x-y_1\|)f(\|x-y_2\|) \dd y_1\dd y_2\dd x.\label{eq: E_N_o^2}
      % &\quad+\lambda^2\mu^2 \int_{\mathbb{R}^2}\int_{\mathbb{R}^2}\int_{\mathbb{R}^2}\int_{\mathbb{R}^2}f(\|y_1\|)f(\|y_2\|) f(\|x_1-y_1\|)f(\|x_2-y_2\|) \dd x_1 \dd x_2 \dd y_1\dd y_2
    \end{align}
    \setcounter{equation}{\value{MYtempeqncnt}}
\hrulefill
\vspace*{4pt}
\end{figure*}
Here we focus on the case where $\Psi$ and $\Phi$ are two independent PPPs in $\mathbb{R}^d$, defined on the same probability space. We derive the mean and variance of $N$ and $M$. For the PPP,
\[
{\rm P}_{\Phi}^{!o} = \rm P_{\Phi}.
\]
This is known as Slivnyak's Theorem  \cite{haenggi2012stochastic} and leads to  the result below.

%With a slight abuse of notation, we write $\Phi$ instead of $\Phi_o^{!}$ in this section for simplicity. 
   
% For simplicity, we assume $K\equiv1$, $i.e.,$ an infected node is infectious for one slot and is removed afterwards. 
% The relevant parameters are the densities $\lambda,~\mu$, the rate(s) of visiting a hub $(\gamma,\theta)$, and the rate of infection $\alpha$. We first analyze the distribution of the number of connections. Then we derive the basic reproduction number of the model.

%     \end{equation}
%   Its expectation is 
% The moment-generating function of $N_o$ is 
%           \begin{align}
%          \mathcal{G}_{N_o}(s) & = \Ex s^ {N_o}\\
%          & = \Ex s^{\sum_{y\in\Psi}\sum_{x \in\Phi} I(o,y)I(x,y)}\\
%          & = \Ex \prod_{y\in\Psi}\prod_{x \in\Phi} s^{I(o,y)I(x,y)}\\
%          & \peq{a} \Ex\prod_{y\in\Psi}\prod_{x\in\Phi} \gamma^2 s\exp(-\theta\|y\|-\theta\|x-y\|)+1-\gamma^2 \exp(-\theta\|y\|-\theta\|x-y\|)\\
%          & \peq{b} \Ex \prod_{y\in\Psi} \exp\bigg(-\int_{0}^{\infty}2\pi\lambda  r  (1-s)\gamma^2 \exp(-\theta\|y\|)\exp(-\theta r)\dd r\bigg)\\
%          & = \Ex \prod_{y\in\Psi} \exp\big(-2\pi\lambda (1-s)\gamma^2\theta^{-2} \exp(-\theta \|y\|)\big)\\
%          & \peq{c} \exp\bigg(-\int_{0}^{\infty}2\pi\mu r \Big(1-\exp\big(-2\pi\lambda (1-s) \gamma^2\theta^{-2} \exp(-\theta r)\big)\Big)\dd r\bigg)\label{eq: LaplaceN}
%      \end{align}
%   where  step (a) follows from the distribution of the binary random variable $I(o,y)I(x,y)$. Step (b)  and (c) follow from the PGFL of the PPP. One can obtain $\Ex N_o = (\mathrm{d}/\mathrm{d}s)\mathcal{G}(s)|_{s=1}$, Or directly b
%Recall that $N_o = \sum_{y\in\Psi}\sum_{x \in\Phi_o^!} I(o,y)I(x,y)$ and $M_o =\sum_{x \in\Phi_o^!} \max_{y\in\Psi}I(o,y)I(x,y)$.     
    \begin{theorem}
    \label{thm: N,M}
For the PPP, the means of $M,~N$ are
\begin{equation}
  \Ex N = \lambda\mu \left(\int_{0}^{\infty}f(r)dc_d r^{d-1} \dd r\right)^2,        \label{eq: EN}
\end{equation}
and
 \begin{equation}
     \Ex M = \lambda\int_{\mathbb{R}^d}  1-\exp\left(-\mu\int_{\mathbb{R}^d}f(\|y\|)f(\|x-y\|)\dd y\right)\dd x.\label{eq: EM}
 \end{equation}
 Further, the variance of $N$ is given in (\ref{eq: E_N_o^2}),
 and the variance of $M$ satisfies $\mathbb{V} M \geq \Ex M$.
\end{theorem}
% \begin{figure*}[t]
% \normalsize
% \setcounter{MYtempeqncnt}{\value{equation}}
% \setcounter{equation}{\value{equation}}
%      \begin{align}
%             \Ex M_o^2 
%             & = \Ex M_o +2\lambda^2  \int_{\mathbb{R}^2}\int_{\mathbb{R}^2} 1- \exp\left(-\mu\int_{\mathbb{R}^2}f(\|y\|)f(\|x_1-y\|)\right)-\exp\left(-\mu\int_{\mathbb{R}^2}f(\|y\|)f(\|x_2-y\|)\dd y\right)\nonumber\\   &\quad+\exp\left(-\mu\int_{\mathbb{R}^2}f(\|y\|)(f(\|x_1-y\|)+f(\|x_2-y\|))-f(\|y\|)^2f(\|x_1-y\|)f(\|x_2-y\|)\dd y\right)\dd x_1 \dd x_2.
%               \label{eq: E_M_o^2}
%             \end{align} 
%             \setcounter{equation}{\value{MYtempeqncnt}+1}
% \hrulefill
% \vspace*{4pt}
% \end{figure*}
\begin{proof}
See Appendix \ref{appendix: N,M}.
   \end{proof}
   \begin{remark} 
       $\Ex M = {\Theta}(\mu)$\footnote{We use $f(x)=\Theta(g(x))$ as $x\to a$ to denote that $f(x)=O\left(g(x)\right)$ and $ g(x)=O(f(x))$ as $x\to a$.} as $\mu\to0$, since $1-\exp(-\mu) = {\Theta}(\mu)$ as $\mu\to0$.
   \end{remark}
 \begin{remark}
Eq. (\ref{eq: EN}) holds for general hub point process $\Psi$. The other equations in Theorem 1 rely on the fact that $\Psi$ is a PPP.  In (\ref{eq: EM}), $\int_{\mathbb{R}^d}f(\|y\|)f(\|x-y\|)\dd y$ only depends on $\|x\|$, as the rotation of $x$ is equivalent to rotating $y$, which has no effect on the integral. This observation is used in the next subsection.
 \end{remark} 
 \begin{remark}
  Both $N$ and $M$ are super-Poisson since $\Var N\geq \Ex N$, and $\Var M\geq \Ex M$. In (\ref{eq: E_N_o^2}), note that in the second term the integral to the power of 3 is larger than the triple integral in the third term. Hence, for fixed $\lambda\mu$, there exists a threshold of $\lambda$ above which increasing $\lambda/\mu$ increases the variance of $N$.
  %When $\gamma\theta^2$ is fixed, increasing $\theta$ increases the variance of $\M_o$ but decreases the variance of $N_o$.
 \end{remark}


 





 
   
%   \begin{corollary}
% The probability that an arbitrary location $x$ is connected with $o$ averaged over $\Psi$ is 
% \begin{equation}
% \Pr(\max_{y\in\Psi} I(o,y)I(x,y) = 1)  = 1-\exp\left(-\frac{\mu\pi}{4}\gamma^2 \|x\|^2\left(\frac{2K_1(\theta  \|x\|)}{\theta \|x\|}+K_0\big(\theta  \|x\|\big)\right)\right).
% \label{eq: p_x_connect}
% \end{equation}
%   \end{corollary}
%      The moment generating function of $M_o$ is 
%           \begin{align}
%          \mathcal{G}_{M_o}(s) & = \Ex s^{M_o}\\
%          & = \Ex s^{\sum_{x\in\Phi} \max_{y\in\Psi} I(o,y)I(x,y)}\\
%          & = \Ex \prod_{x \in\Phi} s^{\max_{y\in\Psi} I(o,y)I(x,y)}\\
%       %  & \peq{a} \Ex\prod_{x\in\Phi} \left(s (1- \prod_{y\in\Psi}(1- \gamma^2 \exp(-\theta\|y\|-\theta\|x-y\|)))+\prod_{y\in\Psi}(1-\gamma^2 \exp(-\theta\|y\|-\theta\|x-y\|))\right)\\
%       & = \Ex \prod_{x\in\Phi} s\Pr(o \leftrightarrow x\mid\Psi) + 1-\Pr(o \leftrightarrow x\mid\Psi)\\
%       & = \Ex \prod_{x\in\Phi} s +(1-s)\prod_{y\in\Psi}\left(1-\gamma^2\exp(-\theta\|y\|-\theta\|x-y\|)\right)\\
%          & \peq{a} \Ex\prod_{x\in\Phi}  s+(1-s)\exp\left(-\frac{\mu\pi}{4}\gamma^2 \|x\|^2\left(\frac{2K_1(\theta  \|x\|)}{\theta \|x\|}+K_0\big(\theta  \|x\|\big)\right)\right)\\
%      %    & = \exp\left(-\int_0^{\infty} 2\pi\lambda v(1-s)\exp(-\int_0^{\infty}\lambda \gamma^2\exp(-\theta r)\dd r)\right)\\
%          & \peq{b}\exp\left(-\int_0^{\infty} 2\pi\lambda v(1-s)\left(1-\exp\left(-\frac{\mu\pi}{4}\gamma^2 v^2\left(\frac{2K_1(\theta v)}{\theta v}+K_0\big(\theta  v\big)\right)\right)\right)\right).
%          \label{eq: gf_M}
%      \end{align}
%     Step (a) follows from Corollary 1. Step (b) follows from the PGFL of the PPP.
 
%          So $M_o$ is Poisson distributed with mean
    
    
    
     %  \hl{Note the relation between $M_o$ and $R_0$}
%   \begin{corollary}
%   The basic reproduction number of the proposed model is 
%   \begin{equation}
%       R_0 =\int_{0}^{\infty} \frac{2\pi\lambda v}{\theta^2} \left(1-\exp\left(-\frac{\mu\pi}{4\theta^2}\alpha \gamma^2 v^2\left(\frac{2K_1( v)}{v}+K_0\big( v\big)\right)\right)\right) \dd v.
%       \label{eq: R_0}
%   \end{equation}
% %   where $\lambda_x(r)$ is given in Lemma 1.
%   \end{corollary}
%   \begin{proof}

%      We denote the probability of a node $x$ of distance $r=\|x\|$ getting infected from $o$ as $g(r)$.  First, we observe that $g(r)  = \Ex \ind(Z_x^1=\mathrm{I})$ can be obtained by changing the connection function $f(r)$ to $\sqrt{\alpha}f(r)$ in $\Ex M_o= \Ex \max_{y\in\Psi} I(o,y)I(x,y)$. This changes $\gamma^2$ to $\alpha\gamma^2$  in (\ref{eq: p_x_connect}). Thus 
%      \begin{equation}
%      g(r) = 1-\exp\left(-\frac{\mu\pi}{4}\alpha\gamma^2 r^2\left(\frac{2K_1(\theta r)}{\theta r}+K_0\big(\theta  r\big)\right)\right),\quad r>0.
%      \label{eq: g_r_exp}
%      \end{equation}
%   Using the reduced Palm expectation, Campbell's theorem, and substituting $\theta r$ by $v$ lead to (\ref{eq: R_0}).
%       \end{proof}


%       \begin{remark}
%          \begin{equation}
% \begin{split}
%   R(\Phi,\Psi) 
%      & =\sum_{x\in\Phi}  1- \prod_{y\in\Psi} \left(1-\alpha\gamma^2\exp\big(-\theta (\|y\|+\|x-y\|)\big)\right). \nonumber
% \end{split}
%     \end{equation}
%     Comment on its distribution
%       \end{remark}
      
     
            

      
%Fig. \ref{fig: pdf_R_0} shows the distribution of $R_0(\Phi,\Psi)$. Fig. \ref{fig: g_r_exp} shows the probability of infection at distance $r$.




%       \begin{corollary}
% For $K>0$, fix $\Ex N_o = K$.   Let $\rho\triangleq(2\pi)^2\lambda/\mu$.  $\Ex M_o$ depends on $\lambda$ and $\mu$ only through $\rho$. Further, $\Ex M_o$ monotonically increases with $\rho$, $\gamma$, and $K$.
%       \end{corollary}
%       \begin{proof}
%       $\Ex N_o = K$ implies  \begin{equation}
%       \theta^4 = K/(\gamma^2\mu^2{\rho}).
%       \label{eq: ratio}
%       \end{equation}
%       Here, the factor $1/(2\pi)^2$ is added to simplify the expressions below. Substituting $\theta$ by (\ref{eq: ratio})  and substituting $\sqrt{\mu} v$ by $v$ in Eq (\ref{eq: E M_0 exp}), we have
%       \begin{equation}
%           \Ex M_o =  \int_{0}^{\infty} \frac{\sqrt{\rho K}v}{2\pi\gamma} \left[1-\exp\left(-\frac{\pi\sqrt{K}}{4\sqrt{\rho}} \gamma v^2\left(\frac{2K_1( v)}{ v}+K_0\big( v \big)\right)\right)\right] \dd v,        \label{eq: R_0_simple}
%           %\int_{0}^{\infty} \frac{\rho v}{2\pi} \left[1-\exp\left(\frac{\pi}{4}\sum_{m=1}^{K}{K \choose m}(-t \gamma^2)^m v^2\left(\frac{2K_1( mv \sqrt{\gamma} \sqrt[\leftroot{-2}\uproot{2}4]{\rho/M})}{ mv\sqrt{\gamma} \sqrt[\leftroot{-2}\uproot{2}4]{\rho/M}}+K_0\big(m v \sqrt{\gamma} \sqrt[\leftroot{-2}\uproot{2}4]{\rho/M} \big)\right)\right)\right] \dd v,        \label{eq: R_0_simple}
%       \end{equation}
%       which does not depend on $\mu$. The monotonicity of $R_0$ wrt $K$ is trivial once we observe that the exponent is negative.  The monotonicity of $\Ex M_o$ wrt $\gamma,\rho$ is to be proved.
%       \end{proof}

      
      
      
      
%       \begin{corollary} 
%           For $\Ex N_o =K$, we have $\Ex M_o\leq K$.  The equality is achieved when $\rho\to\infty$ or $\gamma\to0$. $\int_{0}^{\infty}x^{n-1}K_m(x)\dd x = 2^{n-2}\Gamma(\frac{n-m}{2})\Gamma(\frac{n+m}{2}),~n> m.$
%           \end{corollary}
%           \begin{proof}
%           Taking the limit of (\ref{eq: R_0_simple}) when $\rho\to\infty$ or $\gamma\to0$ yields the upper bound. 
%           \end{proof}
          
    %   For a given set of $\lambda,\mu,\alpha$, let  $\gamma\to0$ while \begin{equation}\frac{\gamma}{\theta^2}=\frac{1}{2\pi}\sqrt{\frac{M}{\lambda\mu}}
    %   \label{eq: ratio}
    %   \end{equation}
    %   such that $\Ex N =M$, we have $
    %       \lim_{\gamma\to0}R_0 = C.$
    %   For a given set of $\lambda,\alpha,\gamma$, let  $\mu\to0$ while $\Ex N =M$ as in (\ref{eq: ratio}), we have $ \lim_{\gamma\to0} R_0 = C.$
       
    %  \begin{remark}
    %  From Corollary 2, when fixing the average number of hubs visited for the typical user, removing the location-dependence is equivalent to decreasing the hub density. The limiting case is where all nodes are uniformly mixed. On the other hand, when $\rho\to0$, $\Ex M_o\to 0$ because the number of distinct nodes decrease.
    %  \end{remark} 
      
     % Fig. \ref{fig: R_0} plots $R_0$ versus $\gamma$ for a set of $\bar\rho\triangleq\rho/(2\pi)^2$. As the density ratio $\lambda/\rho$ increases or the preference for nearby connections $\gamma$ decreases, $\Ex M_o$ increases. The upper bound $tM$ is achieved when $\gamma=0$ or $\bar\rho\to\infty$.
%   \begin{figure}
%       \centering
%       \includegraphics[width=0.5\textwidth]{R0vsmu2.eps}
%       \caption{Simulation results of $R_0$ using \eqref{eq: R_0}. }
%       \label{fig: R_0}
%   \end{figure}
 
% \begin{equation}
%       R_0\approx p(1-\mathcal{L}_\alpha(N))/(1-p)(1-\exp(-\beta))
%       \end{equation}
    
    %  \subsection{Speed of spread}
%       \begin{figure}
%     \begin{subfigure}[b]{0.45\textwidth}
%       \includegraphics[width=\textwidth]{var_R_0.eps}
%       \caption{$f_1$.}
%       \end{subfigure}
%       \hfill
%     \begin{subfigure}[b]{0.45\textwidth}
%       \includegraphics[width=\textwidth]{var_R_0.eps}       \caption{$f_2$.}
%       \end{subfigure}
%       \caption{Mean and variance of $M_o$.}
%       \label{fig: mean_var_M_o}
%   \end{figure}   
   
  

%The probability that the disease dies out is lower-bounded by 
%\begin{equation}
%    \Ex \prod_{x\in\Phi}(1-p_x)
%\end{equation}
%    \subsection{Clustering (tentative)}
%    \hl{Need the definition of clustering. Assuming the disease spreads from a single infected user, this seems a straightforward result as the result of the monotonicity of $f$... Connection with $g(r)$?}
%    
%If at the beginning of the spread, nodes are infected iid with probability $p$, $i.e.,$ both susceptible and infected nodes are PPPs with intensity $(1-p)\lambda$ and $p\lambda$, respectively. The probability of the typical susceptible node getting infected at the end of the first slot immediately follows from Eq (\ref{eq: LaplaceN}) by replacing $\lambda$ with $p\lambda$.   The probability of the typical (susceptible) node getting infected is 
%     \begin{align}
%         \Pr(Z^{1} = \mathrm{I} \mid Z^{0} = \mathrm{S}) &= \Ex [1-\exp(-\alpha N)] \\
%         & = 1-\mathcal{L}_{N}(\alpha).\label{eq: StoI-PPP}
%     \end{align}
%By ergodicity, it is equivalent to the fraction of susceptible nodes infected in a single realization.
%  
%    If the locations of infected nodes followed a PPP throughout the disease spread, then the system is fully characterized by Eq (\ref{eq: StoI-PPP}). In other words, given the parameters and starting condition, the transition probability between compartments are known. However, given the geometric dependence, infections likely cause a clustering effect around higher-density hubs and infected nodes. As a result, there are fewer infected nodes around susceptible ones, leading to a slow down of spread. 
%    


  \begin{figure*}[t]
       \centering
       \subfigure[Mean and the standard deviation of $N$. ]{           \psfrag{EN}{$\mathbb{E}N$}
       \psfrag{VN, lambda = 5, mu=50}{$\sqrt{\mathbb{V}N},\lambda=5,\mu=50$}
              \psfrag{VN, lambda = 50, mu=5}{$\sqrt{\mathbb{V}N},\lambda=50,\mu=5$}

\includegraphics[width=.42\textwidth]{fig/N_o_disk_2.eps}
       %  \caption{$f(r) = \exp(-9.426r)$}
         \label{fig: N_o}}
     \hfill
     \subfigure[Mean and the standard deviation of $M$. ]{ 
     \psfrag{EM, rho = 01}{$\mathbb{E}M,\lambda=5,\mu=50$}
          \psfrag{EM}{$\mathbb{E}M,\lambda=50,\mu=5$}
          \psfrag{sqrt{{V}M}, rho = 0111}{$\sqrt{\mathbb{V}M},\lambda=5,\mu=50$}
  \psfrag{sqrt{{V}M}, rho = 10102}{$\sqrt{\mathbb{V}M},\lambda=50,\mu=5$}
\includegraphics[width=.42\textwidth]{fig/M_o_disk_3.eps}
         \label{fig: illu-g}}
           \caption{Mean and standard deviation of $N$ and $M$ for $f =\ind{(r\leq 0.2122)}$. We compare $(\lambda,\mu) = (5,50), (50,5)$ respectively. $\Ex N=5$. $\sqrt{\Var{M}}$ is obtained via simulation whereas the rest via Corollary 2.}
         \label{fig: N_oM_o_disk}
   \end{figure*} 



%Observe that for $p\in(0,1]$ replacing $\mu$ by  $\mu/p^2$ and $f(r)$ by $pf(r)$ does not change $\Ex M_o$. In contrast, replacing $\lambda$ by  $\lambda/p^2$ and $f(r)$ by $pf(r)$ increases $\Ex M_o$.
  \begin{corollary}
  \label{cor: dispersed-mean-degree}
   For all $f$, $\Ex^{} M$ for $f_p$ monotonically decreases with $p$. For all $f$, $\Ex^{} M$ monotonically increases with $\lambda/\mu$ for fixed $\lambda\mu$. Further,
   
 \setcounter{equation}{\value{MYtempeqncnt}+3}
  \begin{equation}
 \lim_{p\to 0} \Ex^{} M = \Ex^{} N.
  \end{equation}
  \begin{equation}
 \lim_{\frac{\lambda}{\mu}\to \infty} \Ex^{} M = \Ex^{} N.
  \end{equation}
 \end{corollary}
 \begin{proof}
Let us write $\Ex^{} M$ for $f_p$  as
 \begin{equation}
\begin{split}\nonumber
   &\Ex^{} M\\
   & = \lambda\int_{\mathbb{R}^d}  1-\exp\left(-\mu p^2\int_{\mathbb{R}^d}f(\sqrt[d]{p}\|y\|)f(\sqrt[d]{p}\|x-y\|)\dd y\right)\dd x\\
   &  \peq{a} \frac{\lambda}{p}\int_{\mathbb{R}^d}   1-\exp\left(-\mu p\int_{\mathbb{R}^d}f(\|v\|)f(\|u-v\|)\dd v\right)\dd u.
   %\\
   %& = \lambda\int_{\mathbb{R}^2}  1-\exp\left(-\mu p\int_{\mathbb{R}^2}f(\|y\|)f(\|\sqrt{p}x-y\|)\dd y\right)\dd x
 %& = \int_{\mathbb{R}^2} \frac{\lambda}{p} \left( 1-\exp\left(-\mu p\int_{\mathbb{R}^2}f(\|y\|)f(\|x-y\|)\dd y\right)\right) \dd x
 \end{split}
 \end{equation}
% Taking the derivative of the last equation with respect to $p$,
% \begin{equation}
%     \frac{\dd \Ex M_o}{\dd p}  = \lambda \int_{\mathbb{R}^2}  \mu\sqrt{p} \int_{\mathbb{R}^2}f(\|y\|)f'(\|\sqrt{p}x-y\|)\dd y \exp\left(-\mu p\int_{\mathbb{R}^2}f(\|y\|)f(\|\sqrt{p}x-y\|)\dd y\right)\dd x,
% \end{equation}
% is negative by the monotonicity of $f$. 
Step (a) follows from change of variables $u=\sqrt{p}x,~v=\sqrt{p}y$.
The last equation monotonically decreases with the increase of $p$ since $(1-\exp(-C x))/x$ monotonically decreases with $x$.
Further, this shows that for $\Ex M$, stretching $f$ to $f_p$ is equivalent to scaling the network by $\lambda' = \lambda/p$ and $\mu'=\mu p$.
As $p\to0$, all terms in the series expression of the integrand containing $p$ go to 0, hence the convergence to $\Ex N$.
 \end{proof}

\begin{remark}Recall that $N=M$ when each pair of agents are connected through at most one hub. 
 For fixed $\lambda\mu$, as $\lambda/\mu\to\infty$, the number of hubs for the typical agent decreases per (\ref{eq: EN_hub}). Intuitively, most agents only have an edge to their closest hub. 
 On the other hand, as $p\to0$, two agents in proximity are less unlikely to be connected through common hubs, and geometry has a diminishing impact on connections. The graph eventually becomes an arbitrarily large abstract random graph.  Since $N-M$ is a non-negative integer, Corollary 1 also implies the convergence of $M$ to $N$ in probability.
\end{remark}
% \begin{remark}
%     A similar result can be derived when considering the mean number of hubs connected with hubs and letting $\mu/\lambda\to\infty$. We omit its discussion due to its symmetry to the described result as well as its lack of physical relevance.
% \end{remark}
 
 
 \subsection{Examples}
   In this subsection, we consider an example with $d=2$ and two special connection functions. 
    
    The first is\begin{equation}
  \nonumber  f_1(r) = \ind{(r\leq \theta)},\quad r\geq 0,\label{eq: disk}
      \end{equation} 
      where $\theta>0$ is the cut-off radius for drawing an edge between a agent and a hub. Two agents potentially have common hubs only if the intersection of their disks of radius $\theta$ are nonempty regardless of $\Psi$.
      
The second is
   \begin{equation}
\nonumber        f_2(r) =\frac{1}{2}\exp(-r/\theta),\quad r\geq 0.
\label{eq: exp dist}
\end{equation}
which has infinite support.
 


For both connection functions, the degree of the typical agent is Poisson distributed with mean 
    \begin{align}
      \Ex \sum_{y\in\Psi}I(o,y)&= \int_{0}^{\infty}\exp(- r/\theta)\pi\mu r \dd r\nonumber\\
      &=\int_{0}^{\infty} 2\pi\mu r \ind{(r\leq \theta)} \dd r\nonumber\\
      &= {\pi\mu}{\theta^2}.\nonumber\label{eq: EN_hub}
    \end{align}
%  The mean total length of the edge of the typical agent is
% \begin{align}
%       \Ex D_o = \int_{0}^{\infty} \exp(-r/\theta ) 2\pi\mu r^2 \dd r= {2\pi\mu}{\theta^3}.\label{eq: EN_hub}
%     \end{align}
    % and
    % \begin{align}
    %   \Ex D_o = \int_{0}^{\infty} \ind{(r\leq \theta)}2\pi\mu r^2 \dd r= \frac{2\pi\mu \theta^3}{3}.
    % \end{align}

   
\begin{corollary}
\label{cor: examples-N-M}
 For $f_1(r)= \ind{(r\leq \theta)}$, $\Ex N = \lambda\mu\pi^2\theta^4$, and 
 \begin{equation}
        \Ex M 
         = \int_{0}^{2\theta}2\pi\lambda v\left(1-e^{-\mu (2\theta^2\arccos(v/2\theta)-v\sqrt{\theta^2-v^2/4})}\right)\dd v.
        \label{eq: E M_0 disk}
\end{equation}
For $f_2(r)= \exp(-r/\theta)/2$,  $\Ex N = {\lambda\mu\pi^2}{\theta^4},$
and
\begin{align}
        \Ex M = \int_{0}^{\infty}2\pi\lambda v\left(1-e^{-\frac{\mu\pi v}{16}{({2\theta K_1(v/\theta)}+vK_0(v/\theta)})}\right)\dd v.\label{eq: E M_0 exp}
        \end{align}
% \begin{figure*}[t]
% \normalsize
% \setcounter{MYtempeqncnt}{\value{equation}}
% \setcounter{equation}{\value{equation}}
% \begin{align}
%         \Ex M_o 
%         & = \int_{0}^{\infty}2\pi\lambda v\left(1-\exp\left(-\frac{\mu\pi}{16} v^2\left(\frac{2\theta K_1(v/\theta)}{v}+K_0(v/\theta  \big)\right)\right)\right)\dd v.\label{eq: E M_0 exp}
%         \end{align}
%         \setcounter{equation}{\value{MYtempeqncnt}+1}
% \hrulefill
% \vspace*{4pt}
% \end{figure*}
where $K_n(\cdot)$ is the modified Bessel function of the second kind.



 
% \begin{figure*}[t]
% \normalsize
% \setcounter{MYtempeqncnt}{\value{equation}}
% \setcounter{equation}{\value{equation}}
% \begin{align}
%         \Ex M_o 
%         & = \int_{0}^{2\theta}2\pi\lambda v\left(1-\exp\left(-\mu \big(2\theta^2\arccos(v/2\theta)-v\sqrt{\theta^2-v^2/4}\big)\right)\right)\dd v.
%         \label{eq: E M_0 disk}
% \end{align}
% \setcounter{equation}{\value{MYtempeqncnt}+1}
% \hrulefill
% \vspace*{4pt}
% \end{figure*}
\end{corollary}    
\begin{proof}
For $f_1$, note that only agents within radius $2\theta$ can be connected to the typical agent $o$. Then the expression follows from the area of the intersection of two disks with radius $\theta$. For $f_2$, see Appendix \ref{appendix: examples-N-M}. 
\end{proof}
\begin{remark}
The integrand in $\Ex M$ has bounded support if and only if the connection function $f$ has bounded support, in which case its support is twice the support of $f$. 
\end{remark}

The numerical evaluations of (\ref{eq: E M_0 disk})  and (\ref{eq: E M_0 exp}) take a few millisecond using Matlab. $\Ex M$ increases linearly with $\lambda$ and sublinearly with $\mu$. As $\mu\to0$, $\Ex M$ is a linear function of $\mu$ asymptotically (see Remark 1). 
Fig. \ref{fig: N_oM_o_disk} plots the mean and variance of $N,~M$ for a fixed $\Ex N$ and $f_p(r)= pf(\sqrt{p}r)$, $p\in(0,1]$. 


      
\section{Percolation}
  
%    Let $C_o$ be the connected component of the nodes in graph $\mathcal{G}$ that contains $o$. A node belongs to $C_o$ if there exists a path of adjacent nodes to $o$. Percolation occurs when $C_o$ is infinite with positive probability. 
    \subsection{Definitions}
We consider the percolation of the RBG graph where $\Phi$ and $\Psi$ are two independent PPPs with intensities $\lambda$ and $\mu$, respectively. We say that an infinite graph percolates if there exists an infinitely connected component in the graph a.s., or equivalently, the probability of the origin belonging to an infinite component is larger than 0.
 From the point of view of disease spreading, it means that a positive fraction of the network can eventually get infected from a single agent (ignoring the delay of spreading and recovery).  The RBG graph  model is ergodic and has a unique infinite component with probability 0 or 1 following the arguments in \cite[Theorems 2.1\&6.3]{MeesterRonald1996CP}. 
    
    %We define the percolation threshold of the RBG graph as follows. 
    
    
For a fixed $f$, let the percolation region be defined as
\begin{equation}
    \mathcal{D} \triangleq \{(\lambda,\mu): \mathcal{G}\left(\Phi,\Psi, f\right) ~\mathrm{percolates}\}.\nonumber
\end{equation}
   The boundary of $\mathcal{D}$ is  symmetrical, i.e., \[(\lambda,\mu)\in\mathcal{D} \quad \Leftrightarrow \quad (\mu,\lambda)\in\mathcal{D},\]
where $\Leftrightarrow$ denotes an implication in both directions.
    To see this, note that having an infinite component formed from a subset of $\Phi$ is equivalent to having  an infinite component formed from a subset of $\Psi$. Further, $\Phi$ and $\Psi$ are both PPPs.
    
For any fixed $\mu$, let $\lambda_c(\mu)$ be defined as \[
        \lambda_c(\mu) \triangleq \inf \{\lambda: (\lambda,\mu)\in \mathcal{D}\},\]
and $\mu_c$ be defined as
    \begin{equation}
        \mu_c \triangleq \inf \{\mu: \lambda_c(\mu)<\infty\}.\nonumber
    \end{equation} 
 Here our notations differ from \cite{penrose_2014} due to the physical interpretation of $\lambda$ and $\mu$ and that we focus on the critical density of hubs. The change of notation is not fundamental due to the symmetry of $\lambda$ and $\mu$ in our model.
 % First we define the percolation threshold for the Poisson random connection model.
% \begin{equation}
%     \lambda_c^{*}  \triangleq \mathrm{inf} \{\lambda: \Pr(|C_o|=\infty)>0\}
% \end{equation}
   %   \begin{figure*}[t]
   %     \centering
   %   \subfigure[$g(r)$, $f_1$. ]{\includegraphics[width=.47\textwidth]{g_r_exp.eps}
   %    % \caption{$g(r)$ for fixed $\theta, \mu$ and a set of $\gamma$ per Corollary 1.}
   %     \label{fig: g_r_exp}}
   %    \hfill
   %   \subfigure[$g(r)$, $f_2$]{\includegraphics[width=.47\textwidth]{g_r_disk.eps}
   %      %  \caption{$g(r)$ for fixed $\theta=1$ and a set of $\gamma$ per Corollary 2. }
   %        \label{fig: g_r_disk}}
   %      \caption{$g(r)$ for $f_1$ and $f_2$. }
   %            \label{fig: g_r}
   % \end{figure*}   

\begin{comment}
    
    Denote the probability of node $x$ at distance $r=\|x\|$ being connected to $o$ as $g(r)$. 
   \begin{equation}
g(r)= 1-\exp\left(-\mu\int_{\mathbb{R}^2}f(\|y\|)f(\|x-y\|)\dd y\right).
    \end{equation}
    
    The function $g$ abstracts the impact of $\Psi$ and shows the probability of a connection at a given distance to $o$. Assuming independence between different connections, we obtain the Poisson random connection model. For the Poisson random connection model, we denote the percolation threshold as $\lambda_c^*$, defined analogously to Definitions 2 and 3.
 This ignores the underlying \hl{correlation} between connections due to hubs locations in the RBG graph. 

\end{comment}
 Consider a random unipartite geometric graph on a  PPP $\Phi$ with intensity $\lambda$, $\mathcal{G}(\Phi,f)$. 
 Denote by $\zeta_f$ the critical density for the percolation of the graph $\mathcal{G}(\Phi,f)$, i.e.,
 \[
\zeta_f \triangleq \inf \{\lambda\colon \mathcal{G}\left(\Phi,f\right) ~\mathrm{percolates}\}
 \]
 
 For $\int_{0}^{\infty}f(r)r^{d-1}\dd r<\infty$,  it is known that \cite[Chapter 6]{MeesterRonald1996CP}
  \[
  0<  \zeta_f<\infty.
    \] 
    %it is known that $\lambda_c^*<\infty$ for $\int f \dd x <\infty$, $\lambda_c<\infty$.
    Further, there is at most one unbounded component.  


 
% Here, the question is whether and when $\lambda_c$  is an accurate approximation of $\lambda_c$ for the RBG graph. In addition, consider  generated by a Poisson random connection model with connection function $f$. Define 
   
    
%    Let $\lambda_c^{\mathrm{u}}$ denote the percolation threshold for this unipartite model, where $u$ stands for unipartite. 
   
    
    \subsection{Some Bounds on the Percolation Thresholds}
 

    
        \begin{lemma}  
    The percolation region $\mathcal{D}$ for $\mathcal{G}(\Phi,\Psi,f)$ satisfies 
    %\[\lambda_c^*\leq \lambda+\mu,\quad (\lambda,\mu)\in\mathcal{D}\]
\begin{equation}
\min\{\lambda+\mu\colon (\lambda,\mu)\in \mathcal{D}\}\geq \zeta_f.
\end{equation}
    \end{lemma}
\begin{proof}
        Construct the following coupling: let $\mathcal{G}\left(\Phi\cup\Psi,f\right)$ be a unipartite random geometric graph with nodes  being the union of agents and hubs from $\mathcal{G}\left(\Phi,\Psi,f\right)$. The edge set of $\mathcal{G}\left(\Phi\cup\Psi,f\right)$ contains all edges of $\mathcal{G}\left(\Phi,\Psi,f\right)$, as well as those edges within agents and within hubs generated through $f$. So  $\mathcal{G}\left(\Phi\cup\Psi,f\right)$ dominates $\mathcal{G}\left(\Phi,\Psi,f\right)$ in the sense of the coupling just mentioned. So the percolation of $\mathcal{G}\left(\Phi,\Psi,f\right)$ implies the percolation of  $\mathcal{G}\left(\Phi\cup\Psi,f\right)$.
\end{proof}  
\begin{remark}
A special case for $f(r) = \ind (r\leq a) $ is proved in \cite{iyer2012percolation} with the same construction of coupling.    
\end{remark}

With a slight abuse of notation, let $\zeta(2a)$ denote the percolation threshold of $\mathcal{G}(\Phi, \ind (r\leq 2a))$ (the Poisson Boolean model). 

\begin{lemma}{\cite[Theorem 1.1]{penrose_2014}}\label{lemma: penrose}
    For the RBG model with $f(r) = \ind (r\leq a)$,      \[\mu_c = \zeta(2a).\]   
\end{lemma}
\begin{proof}
The proof is given in \cite{penrose_2014} through constructing a coupling of the RBG model with Bernoulli site percolation.
\end{proof}
\begin{remark}
    A special case for $d=2$ is proved in \cite{iyer2012percolation}.
\end{remark}
Here we prove that the same threshold holds for the connection functions $f(r) = p\ind (r\leq a)$, $p\in(0,1]$, which generalizes the Boolean connection function in \cite{iyer2012percolation,penrose_2014}. 

 \begin{theorem}
 For the RBG model with the connection functions $f(r) = p\ind (r\leq a)$, $p\in(0,1]$, 
 \label{thm: perco-th-p-boolean}
\begin{equation}
\mu_c = \zeta(2a).
 \end{equation}

 \end{theorem}
\begin{proof}
   The proof is based on the proof in \cite[Theorem 1.1]{penrose_2014}, with one modification. For the Bernoulli site percolation of $\epsilon\mathbb{Z}^2$ with retaining probability $q<1$ coupled with $\mathcal{Q}_{\mu}$ (corresponding to $\Phi$ with density $\lambda$ in our notation), we let $\lambda$ be such that 
   $1-\exp(-\lambda p^2 \epsilon^d) = q$ rather than requiring that $1-\exp(-\lambda \epsilon^d) = q$. The modification makes sure that if we independently remove points in $\Phi$ with probability $p^2$ (to account for the sampling of the two-edge path), the probability of having a point within each cube of size $\epsilon^d$ is  $q$.
\end{proof}



\begin{corollary}    \label{cor: bound_lambda_c_general_f}
    For a monotone decreasing $f$ with bounded support on $[0,a]$ and  lower bounded away from 0, 
    \[\mu_c = \zeta(2a).\]
\end{corollary}
\begin{proof}
Let $\underline{f}(r) = p\ind(r\leq a)$ where $p$ is the lower bound of $f$ on the support and $\bar{f}(r)=\ind(r\leq a)$. Then  there exists a coupling between the corresponding graphs;  $\mathcal{G}\left(\Phi,\Psi,\bar{f}\right)$ dominates $\mathcal{G}\left(\Phi,\Psi,f\right)$, and $\mathcal{G}\left(\Phi,\Psi,f\right)$ dominates $\mathcal{G}\left(\Phi,\Psi,\underline{f}\right)$. 
Since for both $\underline{f}$ and $\bar{f}$, $\mu_c  = \zeta(2a)$ by Theorem \ref{thm: perco-th-p-boolean} and Lemma \ref{lemma: penrose},  the proof is complete.
\end{proof}
\begin{remark}
\label{remark: bounded_travel} 
    Corollary \ref{cor: bound_lambda_c_general_f} shows that if  the traveling distance is bounded, it suffices to limit the density of hubs to prevent the disease from spreading, regardless of the density of agents.   
\end{remark}
Corollary \ref{cor: bound_lambda_c_general_f} also implies that the percolation threshold $\mu_c<\infty$ for RBG graphs with monotone decreasing connection functions with infinite support. This follows directly from the monotonicity of $\zeta(2a)$ in $a$ and the fact that $\lim_{a\to\infty}\zeta(2a)=0$.
 

\begin{theorem}
    For general monotone decreasing connection function $f$, 
    \begin{equation}    
    \lambda\mu\geq \left(\int_{\mathbb{R}^d}f(\|x\|) \dd x)\right)^{-2},\quad (\lambda,\mu)\in\mathcal{D}.
    \label{eq: g-w-branching-bound}
    \end{equation}
    \label{thm: g-w-branching-bound}
\end{theorem}
\begin{proof}
    We use the same approach as in \cite[Chapter 6]{MeesterRonald1996CP} that proves the existence of the threshold for Poisson random connection models. The idea is to establish a coupling of the RBG graph with a Galton-Watson branching process with average descendants \[\Ex N=\lambda\mu\left(\int_{\mathbb{R}^d}f(\|x\|) \dd x\right)^2.\]

    Let us start with the agent at the origin. In the first step, its connected hubs follow an inhomogeneous PPP with intensity function $\mu f(\|y\|)$, whose mean number is $\mu\int_{\mathbb{R}^d}f(\|y\|) \dd y$. In the second step, consider for each hub $y_1, y_2,...$ generated in the first step: for $y_1$, its connected agents in the second step  follow an inhomogeneous PPP  $\lambda f(\|x-y_1\|)$; for $y_2$, the connected agents in this step  follow an independent inhomogeneous PPP $\lambda f(\|x-y_2\|)(1-f(\|x-y_1\|))$, and so on. The agent set generated by each hub in this step is upper bounded by an independent inhomogeneous PPP of intensity $\lambda f(\|x-y_i\|),~i=1,2,...$. The total number of agents in the second step can be dominated by a compound Poisson r.v. whose mean is $\lambda\mu(\int_{\mathbb{R}^d}f(\|x\|) \dd x)^2 = \Ex N$. Repeating the two steps, a coupling between the RBG graph and a Galton-Watson branching process is established. Requiring $\Ex N \geq 1$ leads to Theorem \ref{thm: g-w-branching-bound}. 
\end{proof}

Theorem \ref{thm: g-w-branching-bound} uses a Galton-Watson branching process to lower bound the RBG graph, where the analogous approach applies to its unipartite counterpart. Considering the limiting result established for unipartite graphs in \cite{penrose1993spread}, one may conjecture that in the limit of dispersion, i.e., $p\to0$, the bound also becomes tight for RBG graphs. 

\begin{lemma}
For the RBG graph with general monotone decreasing connection functions,   $\mu_c>0$ if and only if $f$ has bounded support.
\label{lemma: iff_bounded_support}
\end{lemma}
\begin{proof}
If $f$ has bounded support, let $\bar{f}(r) = \ind(f(r)>0)$. By coupling, $\mathcal{G}(\Phi,\Psi,f)$ is dominated by $\mathcal{G}(\Phi,\Psi,\bar{f})$. We have $\mu_c \geq \bar{\mu}_c >0$.

   Now consider $f$ with infinite support. Consider a Poisson Boolean model. For all $\epsilon>0$, one can find $a_{\epsilon}$ large enough such that for $\lambda=\epsilon$ and $g(r)= \ind(r\leq 2a_{\epsilon})$, $\mathcal{G}(\Phi,g)$ percolates. By the infinite support, one can find $b\geq a_{\epsilon} $ such that $f(b)>0$.  Using Corollary \ref{cor: bound_lambda_c_general_f}, $\mu_c\leq \zeta(2a_{\epsilon})\leq \epsilon$.
\end{proof}
    Lemma \ref{lemma: iff_bounded_support} shows that if agents' traveling distance is unbounded, an arbitrary small density of hubs can cause disease spreading given that the density of agents is large enough. Combining this with Remark \ref{remark: bounded_travel}, we show that long-distance travels (e.g., via airplanes) play a critical role in disease spreading.


  %We study the RBG percolation for Boolean type of connection functions: $f = p\ind{(r\leq a)}$.
 %\begin{proposition}

% For the generalized Boolean model, $\lambda_c\geq\tilde\lambda_c(2a)$.
%  \end{proposition}
%       \begin{proof}
%          Coupling.
%      \end{proof}

 
\subsection{Dispersed RBG Graph}
Let $f_p$ be the dispersed version of a connection function $f$ as defined earlier. For any fixed $\mu$, let $\lambda_c(\mu)$ denote the percolation threshold for the RBG graph with connection function $f$ and $\lambda_c^p(\mu)$ denote that for the RBG graph with $f_p$. $\mu_c$ and $\mu_c^p$ are defined similarly. For the random geometric unipartite graph, it is shown in \cite{Franceschetti05continuumpercolation} that $\zeta_f\geq \zeta_{f_p}$. Later strict inequality is proved in some cases. 
 Here we show that a similar result holds for the RBG graph.



 % \begin{lemma}
 %   For RBG graphs, bond percolation threshold is no greater than site percolation threshold $(\lambda^b,\mu^b) \subset (p\lambda^s,p\mu^s)$
     
 % \end{lemma}

 
 \begin{theorem}
 For all $f$, $p\in(0,1]$, 
\begin{equation}
\lambda_c(\mu)\geq \lambda_c^p(\mu),\nonumber
\end{equation} 
and consequently,
\[\mu_c^p\leq\mu_c.\]
 \end{theorem}
\begin{proof}
Firstly, consider any fixed graph. Consider its site percolation and bond percolation respectively \cite[Chapter 1]{grimmett1999percolation} as follows: the former  independently removes points in $\Phi$ and $\Psi$ with probability $1-p$; the latter independently removes edges with probability $1-p$. The bond percolation threshold is no greater than the site percolation \cite{Franceschetti05continuumpercolation}.
Now, consider any realization of $\mathcal{G}$ generated from $\Phi'=\Phi/p,~\Psi'=\Psi/p,~f(r)$. $\lambda_c(\mu)$ is the site percolation threshold (with probability $p$) of the graph $\mathcal{G}(\Phi',\Psi',f)$, and  equivalently, the percolation threshold for $\mathcal{G}(\Phi,\Psi,f)$. $\lambda_c^p(\mu)$ is the bond percolation threshold of the graph $\mathcal{G}(\Phi',\Psi',pf(r))$, which is equivalent to the percolation of $\mathcal{G}(\Phi,\Psi,~pf(\sqrt{p}r))$ by  scaling. The second inequality follows from the definition of $\mu_c$.
\end{proof}
  A dispersive function lowers the percolation threshold and expands the percolation region. This is consistent with the monotonicity and the limit of the mean degree in Corollary \ref{cor: dispersed-mean-degree}.
One may expect (\ref{eq: g-w-branching-bound}) to become more tight when $p\to0$ ($f$ is dispersive).  In the case where (\ref{eq: g-w-branching-bound}) is tight, we have \begin{align}
        \zeta_f &\leq \min_{(\lambda,\mu)\in\mathcal{D}}\lambda+\mu\nonumber \\
        &\approx \min_{(\lambda,\mu)\in\mathcal{D}} \lambda + {\lambda^{-1}\left(\int_{\mathbb{R}^d}f(\|x\|) \dd x\right)^{-2}} \nonumber\\
        &= {2}{\left(\int_{\mathbb{R}^d}f(\|x\|) \dd x\right)}^{-1}.\nonumber
\end{align}

 
 
 %    \hl{Limiting distribution Poisson?}
    
           
 
    %For $f_2$, a necessary condition for disease spreading is that the Boolean model of the hub point process $\Psi$ with a fixed disk radius $a$ percolates, $i.e.,$ $\mu\pi a^2>4.5$ \cite[Chap. 11]{haenggi2012stochastic}. The reason is that infectious nodes can only spread the disease through common hubs, and nodes in the intersection of the disks of hubs are critical to spread the disease across disks. 
    %Related, \cite{baccelli2020computational} conjectures that there is repulsion between susceptible  and infected nodes when transmissions only occur between nodes in proximity.
    % \hl{pair correlation func over time}
    %  \begin{figure}
    %      \centering
    %      \includegraphics{clustering.eps}
    %      \caption{Illustration of the spatial locations of nodes after two time instances. Blue, red, and green dots denote S, I, R nodes, respectively. Black circles denote  hubs.}
    %      \label{fig:my_label}
    %  \end{figure}   
    
\section{Discussion of Assumptions}\label{sec:discussion}
In this paper, we have made several assumptions for the sake of clarity and simplicity. In this section, we discuss the rationale behind these assumptions, the extent to which these assumptions hold in practice, and the consequences for our protocol when these assumptions hold.

\subsection{Assumptions on the Demand}

There are two simplifying assumptions we make about the demand. First, we assume the demand at any time is relatively small compared to the channel capacities. Second, we take the demand to be constant over time. We elaborate upon both these points below.

\paragraph{Small demands} The assumption that demands are small relative to channel capacities is made precise in \eqref{eq:large_capacity_assumption}. This assumption simplifies two major aspects of our protocol. First, it largely removes congestion from consideration. In \eqref{eq:primal_problem}, there is no constraint ensuring that total flow in both directions stays below capacity--this is always met. Consequently, there is no Lagrange multiplier for congestion and no congestion pricing; only imbalance penalties apply. In contrast, protocols in \cite{sivaraman2020high, varma2021throughput, wang2024fence} include congestion fees due to explicit congestion constraints. Second, the bound \eqref{eq:large_capacity_assumption} ensures that as long as channels remain balanced, the network can always meet demand, no matter how the demand is routed. Since channels can rebalance when necessary, they never drop transactions. This allows prices and flows to adjust as per the equations in \eqref{eq:algorithm}, which makes it easier to prove the protocol's convergence guarantees. This also preserves the key property that a channel's price remains proportional to net money flow through it.

In practice, payment channel networks are used most often for micro-payments, for which on-chain transactions are prohibitively expensive; large transactions typically take place directly on the blockchain. For example, according to \cite{river2023lightning}, the average channel capacity is roughly $0.1$ BTC ($5,000$ BTC distributed over $50,000$ channels), while the average transaction amount is less than $0.0004$ BTC ($44.7k$ satoshis). Thus, the small demand assumption is not too unrealistic. Additionally, the occasional large transaction can be treated as a sequence of smaller transactions by breaking it into packets and executing each packet serially (as done by \cite{sivaraman2020high}).
Lastly, a good path discovery process that favors large capacity channels over small capacity ones can help ensure that the bound in \eqref{eq:large_capacity_assumption} holds.

\paragraph{Constant demands} 
In this work, we assume that any transacting pair of nodes have a steady transaction demand between them (see Section \ref{sec:transaction_requests}). Making this assumption is necessary to obtain the kind of guarantees that we have presented in this paper. Unless the demand is steady, it is unreasonable to expect that the flows converge to a steady value. Weaker assumptions on the demand lead to weaker guarantees. For example, with the more general setting of stochastic, but i.i.d. demand between any two nodes, \cite{varma2021throughput} shows that the channel queue lengths are bounded in expectation. If the demand can be arbitrary, then it is very hard to get any meaningful performance guarantees; \cite{wang2024fence} shows that even for a single bidirectional channel, the competitive ratio is infinite. Indeed, because a PCN is a decentralized system and decisions must be made based on local information alone, it is difficult for the network to find the optimal detailed balance flow at every time step with a time-varying demand.  With a steady demand, the network can discover the optimal flows in a reasonably short time, as our work shows.

We view the constant demand assumption as an approximation for a more general demand process that could be piece-wise constant, stochastic, or both (see simulations in Figure \ref{fig:five_nodes_variable_demand}).
We believe it should be possible to merge ideas from our work and \cite{varma2021throughput} to provide guarantees in a setting with random demands with arbitrary means. We leave this for future work. In addition, our work suggests that a reasonable method of handling stochastic demands is to queue the transaction requests \textit{at the source node} itself. This queuing action should be viewed in conjunction with flow-control. Indeed, a temporarily high unidirectional demand would raise prices for the sender, incentivizing the sender to stop sending the transactions. If the sender queues the transactions, they can send them later when prices drop. This form of queuing does not require any overhaul of the basic PCN infrastructure and is therefore simpler to implement than per-channel queues as suggested by \cite{sivaraman2020high} and \cite{varma2021throughput}.

\subsection{The Incentive of Channels}
The actions of the channels as prescribed by the DEBT control protocol can be summarized as follows. Channels adjust their prices in proportion to the net flow through them. They rebalance themselves whenever necessary and execute any transaction request that has been made of them. We discuss both these aspects below.

\paragraph{On Prices}
In this work, the exclusive role of channel prices is to ensure that the flows through each channel remains balanced. In practice, it would be important to include other components in a channel's price/fee as well: a congestion price  and an incentive price. The congestion price, as suggested by \cite{varma2021throughput}, would depend on the total flow of transactions through the channel, and would incentivize nodes to balance the load over different paths. The incentive price, which is commonly used in practice \cite{river2023lightning}, is necessary to provide channels with an incentive to serve as an intermediary for different channels. In practice, we expect both these components to be smaller than the imbalance price. Consequently, we expect the behavior of our protocol to be similar to our theoretical results even with these additional prices.

A key aspect of our protocol is that channel fees are allowed to be negative. Although the original Lightning network whitepaper \cite{poon2016bitcoin} suggests that negative channel prices may be a good solution to promote rebalancing, the idea of negative prices in not very popular in the literature. To our knowledge, the only prior work with this feature is \cite{varma2021throughput}. Indeed, in papers such as \cite{van2021merchant} and \cite{wang2024fence}, the price function is explicitly modified such that the channel price is never negative. The results of our paper show the benefits of negative prices. For one, in steady state, equal flows in both directions ensure that a channel doesn't loose any money (the other price components mentioned above ensure that the channel will only gain money). More importantly, negative prices are important to ensure that the protocol selectively stifles acyclic flows while allowing circulations to flow. Indeed, in the example of Section \ref{sec:flow_control_example}, the flows between nodes $A$ and $C$ are left on only because the large positive price over one channel is canceled by the corresponding negative price over the other channel, leading to a net zero price.

Lastly, observe that in the DEBT control protocol, the price charged by a channel does not depend on its capacity. This is a natural consequence of the price being the Lagrange multiplier for the net-zero flow constraint, which also does not depend on the channel capacity. In contrast, in many other works, the imbalance price is normalized by the channel capacity \cite{ren2018optimal, lin2020funds, wang2024fence}; this is shown to work well in practice. The rationale for such a price structure is explained well in \cite{wang2024fence}, where this fee is derived with the aim of always maintaining some balance (liquidity) at each end of every channel. This is a reasonable aim if a channel is to never rebalance itself; the experiments of the aforementioned papers are conducted in such a regime. In this work, however, we allow the channels to rebalance themselves a few times in order to settle on a detailed balance flow. This is because our focus is on the long-term steady state performance of the protocol. This difference in perspective also shows up in how the price depends on the channel imbalance. \cite{lin2020funds} and \cite{wang2024fence} advocate for strictly convex prices whereas this work and \cite{varma2021throughput} propose linear prices.

\paragraph{On Rebalancing} 
Recall that the DEBT control protocol ensures that the flows in the network converge to a detailed balance flow, which can be sustained perpetually without any rebalancing. However, during the transient phase (before convergence), channels may have to perform on-chain rebalancing a few times. Since rebalancing is an expensive operation, it is worthwhile discussing methods by which channels can reduce the extent of rebalancing. One option for the channels to reduce the extent of rebalancing is to increase their capacity; however, this comes at the cost of locking in more capital. Each channel can decide for itself the optimum amount of capital to lock in. Another option, which we discuss in Section \ref{sec:five_node}, is for channels to increase the rate $\gamma$ at which they adjust prices. 

Ultimately, whether or not it is beneficial for a channel to rebalance depends on the time-horizon under consideration. Our protocol is based on the assumption that the demand remains steady for a long period of time. If this is indeed the case, it would be worthwhile for a channel to rebalance itself as it can make up this cost through the incentive fees gained from the flow of transactions through it in steady state. If a channel chooses not to rebalance itself, however, there is a risk of being trapped in a deadlock, which is suboptimal for not only the nodes but also the channel.

\section{Conclusion}
This work presents DEBT control: a protocol for payment channel networks that uses source routing and flow control based on channel prices. The protocol is derived by posing a network utility maximization problem and analyzing its dual minimization. It is shown that under steady demands, the protocol guides the network to an optimal, sustainable point. Simulations show its robustness to demand variations. The work demonstrates that simple protocols with strong theoretical guarantees are possible for PCNs and we hope it inspires further theoretical research in this direction.
\vspace{-0.2cm}
\section{Impact: Why Free Scientific Knowledge?}
\vspace{-0.1cm}

Historically, making knowledge widely available has driven transformative progress. Gutenberg’s printing press broke medieval monopolies on information, increasing literacy and contributing to the Renaissance and Scientific Revolution. In today's world, open source projects such as GNU/Linux and Wikipedia show that freely accessible and modifiable knowledge fosters innovation while ensuring creators are credited through copyleft licenses. These examples highlight a key idea: \textit{access to essential knowledge supports overall advancement.} 

This aligns with the arguments made by Prabhakaran et al. \cite{humanrightsbasedapproachresponsible}, who specifically highlight the \textbf{ human right to participate in scientific advancement} as enshrined in the Universal Declaration of Human Rights. They emphasize that this right underscores the importance of \textit{ equal access to the benefits of scientific progress for all}, a principle directly supported by our proposal for Knowledge Units. The UN Special Rapporteur on Cultural Rights further reinforces this, advocating for the expansion of copyright exceptions to broaden access to scientific knowledge as a crucial component of the right to science and culture \cite{scienceright}. 

However, current intellectual property regimes often create ``patently unfair" barriers to this knowledge, preventing innovation and access, especially in areas critical to human rights, as Hale compellingly argues \cite{patentlyunfair}. Finding a solution requires carefully balancing the imperative of open access with the legitimate rights of authors. As Austin and Ginsburg remind us, authors' rights are also human rights, necessitating robust protection \cite{authorhumanrights}. Shareable knowledge entities like Knowledge Units offer a potential mechanism to achieve this delicate balance in the scientific domain, enabling wider dissemination of research findings while respecting authors' fundamental rights.

\vspace{-0.2cm}
\subsection{Impact Across Sectors}

\textbf{Researchers:} Collaboration across different fields becomes easier when knowledge is shared openly. For instance, combining machine learning with biology or applying quantum principles to cryptography can lead to important breakthroughs. Removing copyright restrictions allows researchers to freely use data and methods, speeding up discoveries while respecting original contributions.

\textbf{Practitioners:} Professionals, especially in healthcare, benefit from immediate access to the latest research. Quick access to newer insights on the effectiveness of drugs, and alternative treatments speeds up adoption and awareness, potentially saving lives. Additionally, open knowledge helps developing countries gain access to health innovations.

\textbf{Education:} Education becomes more accessible when teachers use the latest research to create up-to-date curricula without prohibitive costs. Students can access high-quality research materials and use LM assistance to better understand complex topics, enhancing their learning experience and making high-quality education more accessible.

\textbf{Public Trust:} When information is transparent and accessible, the public can better understand and trust decision-making processes. Open access to government policies and industry practices allows people to review and verify information, helping to reduce misinformation. This transparency encourages critical thinking and builds trust in scientific and governmental institutions.

Overall, making scientific knowledge accessible supports global fairness. By viewing knowledge as a common resource rather than a product to be sold, we can speed up innovation, encourage critical thinking, and empower communities to address important challenges.

\vspace{-0.2cm}
\section{Open Problems}
\vspace{-0.1cm}

Moving forward, we identify key research directions to further exploit the potential of converting original texts into shareable knowledge entities such as demonstrated by the conversion into Knowledge Units in this work:


\textbf{1. Enhancing Factual Accuracy and Reliability:}  Refining KUs through cross-referencing with source texts and incorporating community-driven correction mechanisms, similar to Wikipedia, can minimize hallucinations and ensure the long-term accuracy of knowledge-based datasets at scale.

\textbf{2. Developing Applications for Education and Research:}  Using KU-based conversion for datasets to be employed in practical tools, such as search interfaces and learning platforms, can ensure rapid dissemination of any new knowledge into shareable downstream resources, significantly improving the accessibility, spread, and impact of KUs.

\textbf{3. Establishing Standards for Knowledge Interoperability and Reuse:}  Future research should focus on defining standardized formats for entities like KU and knowledge graph layouts \citep{lenat1990cyc}. These standards are essential to unlock seamless interoperability, facilitate reuse across diverse platforms, and foster a vibrant ecosystem of open scientific knowledge. 

\textbf{4. Interconnecting Shareable Knowledge for Scientific Workflow Assistance and Automation:} There might be further potential in constructing a semantic web that interconnects publicly shared knowledge, together with mechanisms that continually update and validate all shareable knowledge units. This can be starting point for a platform that uses all collected knowledge to assist scientific workflows, for instance by feeding such a semantic web into recently developed reasoning models equipped with retrieval augmented generation. Such assistance could assemble knowledge across multiple scientific papers, guiding scientists more efficiently through vast research landscapes. Given further progress in model capabilities, validation, self-repair and evolving new knowledge from already existing vast collection in the semantic web can lead to automation of scientific discovery, assuming that knowledge data in the semantic web can be freely shared.

We open-source our code and encourage collaboration to improve extraction pipelines, enhance Knowledge Unit capabilities, and expand coverage to additional fields.

\vspace{-0.2cm}
\section{Conclusion}
\vspace{-0.1cm}

In this paper, we highlight the potential of systematically separating factual scientific knowledge from protected artistic or stylistic expression. By representing scientific insights as structured facts and relationships, prototypes like Knowledge Units (KUs) offer a pathway to broaden access to scientific knowledge without infringing copyright, aligning with legal principles like German \S 24(1) UrhG and U.S. fair use standards. Extensive testing across a range of domains and models shows evidence that Knowledge Units (KUs) can feasibly retain core information. These findings offer a promising way forward for openly disseminating scientific information while respecting copyright constraints.

\section*{Author Contributions}

Christoph conceived the project and led organization. Christoph and Gollam led all the experiments. Nick and Huu led the legal aspects. Tawsif led the data collection. Ameya and Andreas led the manuscript writing. Ludwig, Sören, Robert, Jenia and Matthias provided feedback. advice and scientific supervision throughout the project. 

\section*{Acknowledgements}

The authors would like to thank (in alphabetical order): Sebastian Dziadzio, Kristof Meding, Tea Mustać, Shantanu Prabhat for insightful feedback and suggestions. Special thanks to Andrej Radonjic for help in scaling up data collection. GR and SA acknowledge financial support by the German Research Foundation (DFG) for the NFDI4DataScience Initiative (project number 460234259). AP and MB acknowledge financial support by the Federal Ministry of Education and Research (BMBF), FKZ: 011524085B and Open Philanthropy Foundation funded by the Good Ventures Foundation. AH acknowledges financial support by the Federal Ministry of Education and Research (BMBF), FKZ: 01IS24079A and the Carl Zeiss Foundation through the project "Certification and Foundations of Safe ML Systems" as well as the support from the International Max Planck Research School for Intelligent Systems (IMPRS-IS). JJ acknowledges funding by the Federal Ministry of Education and Research of Germany (BMBF) under grant no. 01IS22094B (WestAI - AI Service Center West), under grant no. 01IS24085C (OPENHAFM) and under the grant DE002571 (MINERVA), as well as co-funding by EU from EuroHPC Joint Undertaking programm under grant no. 101182737 (MINERVA) and from Digital Europe Programme under grant no. 101195233 (openEuroLLM) 



 
\appendix
\subsection{Lloyd-Max Algorithm}
\label{subsec:Lloyd-Max}
For a given quantization bitwidth $B$ and an operand $\bm{X}$, the Lloyd-Max algorithm finds $2^B$ quantization levels $\{\hat{x}_i\}_{i=1}^{2^B}$ such that quantizing $\bm{X}$ by rounding each scalar in $\bm{X}$ to the nearest quantization level minimizes the quantization MSE. 

The algorithm starts with an initial guess of quantization levels and then iteratively computes quantization thresholds $\{\tau_i\}_{i=1}^{2^B-1}$ and updates quantization levels $\{\hat{x}_i\}_{i=1}^{2^B}$. Specifically, at iteration $n$, thresholds are set to the midpoints of the previous iteration's levels:
\begin{align*}
    \tau_i^{(n)}=\frac{\hat{x}_i^{(n-1)}+\hat{x}_{i+1}^{(n-1)}}2 \text{ for } i=1\ldots 2^B-1
\end{align*}
Subsequently, the quantization levels are re-computed as conditional means of the data regions defined by the new thresholds:
\begin{align*}
    \hat{x}_i^{(n)}=\mathbb{E}\left[ \bm{X} \big| \bm{X}\in [\tau_{i-1}^{(n)},\tau_i^{(n)}] \right] \text{ for } i=1\ldots 2^B
\end{align*}
where to satisfy boundary conditions we have $\tau_0=-\infty$ and $\tau_{2^B}=\infty$. The algorithm iterates the above steps until convergence.

Figure \ref{fig:lm_quant} compares the quantization levels of a $7$-bit floating point (E3M3) quantizer (left) to a $7$-bit Lloyd-Max quantizer (right) when quantizing a layer of weights from the GPT3-126M model at a per-tensor granularity. As shown, the Lloyd-Max quantizer achieves substantially lower quantization MSE. Further, Table \ref{tab:FP7_vs_LM7} shows the superior perplexity achieved by Lloyd-Max quantizers for bitwidths of $7$, $6$ and $5$. The difference between the quantizers is clear at 5 bits, where per-tensor FP quantization incurs a drastic and unacceptable increase in perplexity, while Lloyd-Max quantization incurs a much smaller increase. Nevertheless, we note that even the optimal Lloyd-Max quantizer incurs a notable ($\sim 1.5$) increase in perplexity due to the coarse granularity of quantization. 

\begin{figure}[h]
  \centering
  \includegraphics[width=0.7\linewidth]{sections/figures/LM7_FP7.pdf}
  \caption{\small Quantization levels and the corresponding quantization MSE of Floating Point (left) vs Lloyd-Max (right) Quantizers for a layer of weights in the GPT3-126M model.}
  \label{fig:lm_quant}
\end{figure}

\begin{table}[h]\scriptsize
\begin{center}
\caption{\label{tab:FP7_vs_LM7} \small Comparing perplexity (lower is better) achieved by floating point quantizers and Lloyd-Max quantizers on a GPT3-126M model for the Wikitext-103 dataset.}
\begin{tabular}{c|cc|c}
\hline
 \multirow{2}{*}{\textbf{Bitwidth}} & \multicolumn{2}{|c|}{\textbf{Floating-Point Quantizer}} & \textbf{Lloyd-Max Quantizer} \\
 & Best Format & Wikitext-103 Perplexity & Wikitext-103 Perplexity \\
\hline
7 & E3M3 & 18.32 & 18.27 \\
6 & E3M2 & 19.07 & 18.51 \\
5 & E4M0 & 43.89 & 19.71 \\
\hline
\end{tabular}
\end{center}
\end{table}

\subsection{Proof of Local Optimality of LO-BCQ}
\label{subsec:lobcq_opt_proof}
For a given block $\bm{b}_j$, the quantization MSE during LO-BCQ can be empirically evaluated as $\frac{1}{L_b}\lVert \bm{b}_j- \bm{\hat{b}}_j\rVert^2_2$ where $\bm{\hat{b}}_j$ is computed from equation (\ref{eq:clustered_quantization_definition}) as $C_{f(\bm{b}_j)}(\bm{b}_j)$. Further, for a given block cluster $\mathcal{B}_i$, we compute the quantization MSE as $\frac{1}{|\mathcal{B}_{i}|}\sum_{\bm{b} \in \mathcal{B}_{i}} \frac{1}{L_b}\lVert \bm{b}- C_i^{(n)}(\bm{b})\rVert^2_2$. Therefore, at the end of iteration $n$, we evaluate the overall quantization MSE $J^{(n)}$ for a given operand $\bm{X}$ composed of $N_c$ block clusters as:
\begin{align*}
    \label{eq:mse_iter_n}
    J^{(n)} = \frac{1}{N_c} \sum_{i=1}^{N_c} \frac{1}{|\mathcal{B}_{i}^{(n)}|}\sum_{\bm{v} \in \mathcal{B}_{i}^{(n)}} \frac{1}{L_b}\lVert \bm{b}- B_i^{(n)}(\bm{b})\rVert^2_2
\end{align*}

At the end of iteration $n$, the codebooks are updated from $\mathcal{C}^{(n-1)}$ to $\mathcal{C}^{(n)}$. However, the mapping of a given vector $\bm{b}_j$ to quantizers $\mathcal{C}^{(n)}$ remains as  $f^{(n)}(\bm{b}_j)$. At the next iteration, during the vector clustering step, $f^{(n+1)}(\bm{b}_j)$ finds new mapping of $\bm{b}_j$ to updated codebooks $\mathcal{C}^{(n)}$ such that the quantization MSE over the candidate codebooks is minimized. Therefore, we obtain the following result for $\bm{b}_j$:
\begin{align*}
\frac{1}{L_b}\lVert \bm{b}_j - C_{f^{(n+1)}(\bm{b}_j)}^{(n)}(\bm{b}_j)\rVert^2_2 \le \frac{1}{L_b}\lVert \bm{b}_j - C_{f^{(n)}(\bm{b}_j)}^{(n)}(\bm{b}_j)\rVert^2_2
\end{align*}

That is, quantizing $\bm{b}_j$ at the end of the block clustering step of iteration $n+1$ results in lower quantization MSE compared to quantizing at the end of iteration $n$. Since this is true for all $\bm{b} \in \bm{X}$, we assert the following:
\begin{equation}
\begin{split}
\label{eq:mse_ineq_1}
    \tilde{J}^{(n+1)} &= \frac{1}{N_c} \sum_{i=1}^{N_c} \frac{1}{|\mathcal{B}_{i}^{(n+1)}|}\sum_{\bm{b} \in \mathcal{B}_{i}^{(n+1)}} \frac{1}{L_b}\lVert \bm{b} - C_i^{(n)}(b)\rVert^2_2 \le J^{(n)}
\end{split}
\end{equation}
where $\tilde{J}^{(n+1)}$ is the the quantization MSE after the vector clustering step at iteration $n+1$.

Next, during the codebook update step (\ref{eq:quantizers_update}) at iteration $n+1$, the per-cluster codebooks $\mathcal{C}^{(n)}$ are updated to $\mathcal{C}^{(n+1)}$ by invoking the Lloyd-Max algorithm \citep{Lloyd}. We know that for any given value distribution, the Lloyd-Max algorithm minimizes the quantization MSE. Therefore, for a given vector cluster $\mathcal{B}_i$ we obtain the following result:

\begin{equation}
    \frac{1}{|\mathcal{B}_{i}^{(n+1)}|}\sum_{\bm{b} \in \mathcal{B}_{i}^{(n+1)}} \frac{1}{L_b}\lVert \bm{b}- C_i^{(n+1)}(\bm{b})\rVert^2_2 \le \frac{1}{|\mathcal{B}_{i}^{(n+1)}|}\sum_{\bm{b} \in \mathcal{B}_{i}^{(n+1)}} \frac{1}{L_b}\lVert \bm{b}- C_i^{(n)}(\bm{b})\rVert^2_2
\end{equation}

The above equation states that quantizing the given block cluster $\mathcal{B}_i$ after updating the associated codebook from $C_i^{(n)}$ to $C_i^{(n+1)}$ results in lower quantization MSE. Since this is true for all the block clusters, we derive the following result: 
\begin{equation}
\begin{split}
\label{eq:mse_ineq_2}
     J^{(n+1)} &= \frac{1}{N_c} \sum_{i=1}^{N_c} \frac{1}{|\mathcal{B}_{i}^{(n+1)}|}\sum_{\bm{b} \in \mathcal{B}_{i}^{(n+1)}} \frac{1}{L_b}\lVert \bm{b}- C_i^{(n+1)}(\bm{b})\rVert^2_2  \le \tilde{J}^{(n+1)}   
\end{split}
\end{equation}

Following (\ref{eq:mse_ineq_1}) and (\ref{eq:mse_ineq_2}), we find that the quantization MSE is non-increasing for each iteration, that is, $J^{(1)} \ge J^{(2)} \ge J^{(3)} \ge \ldots \ge J^{(M)}$ where $M$ is the maximum number of iterations. 
%Therefore, we can say that if the algorithm converges, then it must be that it has converged to a local minimum. 
\hfill $\blacksquare$


\begin{figure}
    \begin{center}
    \includegraphics[width=0.5\textwidth]{sections//figures/mse_vs_iter.pdf}
    \end{center}
    \caption{\small NMSE vs iterations during LO-BCQ compared to other block quantization proposals}
    \label{fig:nmse_vs_iter}
\end{figure}

Figure \ref{fig:nmse_vs_iter} shows the empirical convergence of LO-BCQ across several block lengths and number of codebooks. Also, the MSE achieved by LO-BCQ is compared to baselines such as MXFP and VSQ. As shown, LO-BCQ converges to a lower MSE than the baselines. Further, we achieve better convergence for larger number of codebooks ($N_c$) and for a smaller block length ($L_b$), both of which increase the bitwidth of BCQ (see Eq \ref{eq:bitwidth_bcq}).


\subsection{Additional Accuracy Results}
%Table \ref{tab:lobcq_config} lists the various LOBCQ configurations and their corresponding bitwidths.
\begin{table}
\setlength{\tabcolsep}{4.75pt}
\begin{center}
\caption{\label{tab:lobcq_config} Various LO-BCQ configurations and their bitwidths.}
\begin{tabular}{|c||c|c|c|c||c|c||c|} 
\hline
 & \multicolumn{4}{|c||}{$L_b=8$} & \multicolumn{2}{|c||}{$L_b=4$} & $L_b=2$ \\
 \hline
 \backslashbox{$L_A$\kern-1em}{\kern-1em$N_c$} & 2 & 4 & 8 & 16 & 2 & 4 & 2 \\
 \hline
 64 & 4.25 & 4.375 & 4.5 & 4.625 & 4.375 & 4.625 & 4.625\\
 \hline
 32 & 4.375 & 4.5 & 4.625& 4.75 & 4.5 & 4.75 & 4.75 \\
 \hline
 16 & 4.625 & 4.75& 4.875 & 5 & 4.75 & 5 & 5 \\
 \hline
\end{tabular}
\end{center}
\end{table}

%\subsection{Perplexity achieved by various LO-BCQ configurations on Wikitext-103 dataset}

\begin{table} \centering
\begin{tabular}{|c||c|c|c|c||c|c||c|} 
\hline
 $L_b \rightarrow$& \multicolumn{4}{c||}{8} & \multicolumn{2}{c||}{4} & 2\\
 \hline
 \backslashbox{$L_A$\kern-1em}{\kern-1em$N_c$} & 2 & 4 & 8 & 16 & 2 & 4 & 2  \\
 %$N_c \rightarrow$ & 2 & 4 & 8 & 16 & 2 & 4 & 2 \\
 \hline
 \hline
 \multicolumn{8}{c}{GPT3-1.3B (FP32 PPL = 9.98)} \\ 
 \hline
 \hline
 64 & 10.40 & 10.23 & 10.17 & 10.15 &  10.28 & 10.18 & 10.19 \\
 \hline
 32 & 10.25 & 10.20 & 10.15 & 10.12 &  10.23 & 10.17 & 10.17 \\
 \hline
 16 & 10.22 & 10.16 & 10.10 & 10.09 &  10.21 & 10.14 & 10.16 \\
 \hline
  \hline
 \multicolumn{8}{c}{GPT3-8B (FP32 PPL = 7.38)} \\ 
 \hline
 \hline
 64 & 7.61 & 7.52 & 7.48 &  7.47 &  7.55 &  7.49 & 7.50 \\
 \hline
 32 & 7.52 & 7.50 & 7.46 &  7.45 &  7.52 &  7.48 & 7.48  \\
 \hline
 16 & 7.51 & 7.48 & 7.44 &  7.44 &  7.51 &  7.49 & 7.47  \\
 \hline
\end{tabular}
\caption{\label{tab:ppl_gpt3_abalation} Wikitext-103 perplexity across GPT3-1.3B and 8B models.}
\end{table}

\begin{table} \centering
\begin{tabular}{|c||c|c|c|c||} 
\hline
 $L_b \rightarrow$& \multicolumn{4}{c||}{8}\\
 \hline
 \backslashbox{$L_A$\kern-1em}{\kern-1em$N_c$} & 2 & 4 & 8 & 16 \\
 %$N_c \rightarrow$ & 2 & 4 & 8 & 16 & 2 & 4 & 2 \\
 \hline
 \hline
 \multicolumn{5}{|c|}{Llama2-7B (FP32 PPL = 5.06)} \\ 
 \hline
 \hline
 64 & 5.31 & 5.26 & 5.19 & 5.18  \\
 \hline
 32 & 5.23 & 5.25 & 5.18 & 5.15  \\
 \hline
 16 & 5.23 & 5.19 & 5.16 & 5.14  \\
 \hline
 \multicolumn{5}{|c|}{Nemotron4-15B (FP32 PPL = 5.87)} \\ 
 \hline
 \hline
 64  & 6.3 & 6.20 & 6.13 & 6.08  \\
 \hline
 32  & 6.24 & 6.12 & 6.07 & 6.03  \\
 \hline
 16  & 6.12 & 6.14 & 6.04 & 6.02  \\
 \hline
 \multicolumn{5}{|c|}{Nemotron4-340B (FP32 PPL = 3.48)} \\ 
 \hline
 \hline
 64 & 3.67 & 3.62 & 3.60 & 3.59 \\
 \hline
 32 & 3.63 & 3.61 & 3.59 & 3.56 \\
 \hline
 16 & 3.61 & 3.58 & 3.57 & 3.55 \\
 \hline
\end{tabular}
\caption{\label{tab:ppl_llama7B_nemo15B} Wikitext-103 perplexity compared to FP32 baseline in Llama2-7B and Nemotron4-15B, 340B models}
\end{table}

%\subsection{Perplexity achieved by various LO-BCQ configurations on MMLU dataset}


\begin{table} \centering
\begin{tabular}{|c||c|c|c|c||c|c|c|c|} 
\hline
 $L_b \rightarrow$& \multicolumn{4}{c||}{8} & \multicolumn{4}{c||}{8}\\
 \hline
 \backslashbox{$L_A$\kern-1em}{\kern-1em$N_c$} & 2 & 4 & 8 & 16 & 2 & 4 & 8 & 16  \\
 %$N_c \rightarrow$ & 2 & 4 & 8 & 16 & 2 & 4 & 2 \\
 \hline
 \hline
 \multicolumn{5}{|c|}{Llama2-7B (FP32 Accuracy = 45.8\%)} & \multicolumn{4}{|c|}{Llama2-70B (FP32 Accuracy = 69.12\%)} \\ 
 \hline
 \hline
 64 & 43.9 & 43.4 & 43.9 & 44.9 & 68.07 & 68.27 & 68.17 & 68.75 \\
 \hline
 32 & 44.5 & 43.8 & 44.9 & 44.5 & 68.37 & 68.51 & 68.35 & 68.27  \\
 \hline
 16 & 43.9 & 42.7 & 44.9 & 45 & 68.12 & 68.77 & 68.31 & 68.59  \\
 \hline
 \hline
 \multicolumn{5}{|c|}{GPT3-22B (FP32 Accuracy = 38.75\%)} & \multicolumn{4}{|c|}{Nemotron4-15B (FP32 Accuracy = 64.3\%)} \\ 
 \hline
 \hline
 64 & 36.71 & 38.85 & 38.13 & 38.92 & 63.17 & 62.36 & 63.72 & 64.09 \\
 \hline
 32 & 37.95 & 38.69 & 39.45 & 38.34 & 64.05 & 62.30 & 63.8 & 64.33  \\
 \hline
 16 & 38.88 & 38.80 & 38.31 & 38.92 & 63.22 & 63.51 & 63.93 & 64.43  \\
 \hline
\end{tabular}
\caption{\label{tab:mmlu_abalation} Accuracy on MMLU dataset across GPT3-22B, Llama2-7B, 70B and Nemotron4-15B models.}
\end{table}


%\subsection{Perplexity achieved by various LO-BCQ configurations on LM evaluation harness}

\begin{table} \centering
\begin{tabular}{|c||c|c|c|c||c|c|c|c|} 
\hline
 $L_b \rightarrow$& \multicolumn{4}{c||}{8} & \multicolumn{4}{c||}{8}\\
 \hline
 \backslashbox{$L_A$\kern-1em}{\kern-1em$N_c$} & 2 & 4 & 8 & 16 & 2 & 4 & 8 & 16  \\
 %$N_c \rightarrow$ & 2 & 4 & 8 & 16 & 2 & 4 & 2 \\
 \hline
 \hline
 \multicolumn{5}{|c|}{Race (FP32 Accuracy = 37.51\%)} & \multicolumn{4}{|c|}{Boolq (FP32 Accuracy = 64.62\%)} \\ 
 \hline
 \hline
 64 & 36.94 & 37.13 & 36.27 & 37.13 & 63.73 & 62.26 & 63.49 & 63.36 \\
 \hline
 32 & 37.03 & 36.36 & 36.08 & 37.03 & 62.54 & 63.51 & 63.49 & 63.55  \\
 \hline
 16 & 37.03 & 37.03 & 36.46 & 37.03 & 61.1 & 63.79 & 63.58 & 63.33  \\
 \hline
 \hline
 \multicolumn{5}{|c|}{Winogrande (FP32 Accuracy = 58.01\%)} & \multicolumn{4}{|c|}{Piqa (FP32 Accuracy = 74.21\%)} \\ 
 \hline
 \hline
 64 & 58.17 & 57.22 & 57.85 & 58.33 & 73.01 & 73.07 & 73.07 & 72.80 \\
 \hline
 32 & 59.12 & 58.09 & 57.85 & 58.41 & 73.01 & 73.94 & 72.74 & 73.18  \\
 \hline
 16 & 57.93 & 58.88 & 57.93 & 58.56 & 73.94 & 72.80 & 73.01 & 73.94  \\
 \hline
\end{tabular}
\caption{\label{tab:mmlu_abalation} Accuracy on LM evaluation harness tasks on GPT3-1.3B model.}
\end{table}

\begin{table} \centering
\begin{tabular}{|c||c|c|c|c||c|c|c|c|} 
\hline
 $L_b \rightarrow$& \multicolumn{4}{c||}{8} & \multicolumn{4}{c||}{8}\\
 \hline
 \backslashbox{$L_A$\kern-1em}{\kern-1em$N_c$} & 2 & 4 & 8 & 16 & 2 & 4 & 8 & 16  \\
 %$N_c \rightarrow$ & 2 & 4 & 8 & 16 & 2 & 4 & 2 \\
 \hline
 \hline
 \multicolumn{5}{|c|}{Race (FP32 Accuracy = 41.34\%)} & \multicolumn{4}{|c|}{Boolq (FP32 Accuracy = 68.32\%)} \\ 
 \hline
 \hline
 64 & 40.48 & 40.10 & 39.43 & 39.90 & 69.20 & 68.41 & 69.45 & 68.56 \\
 \hline
 32 & 39.52 & 39.52 & 40.77 & 39.62 & 68.32 & 67.43 & 68.17 & 69.30  \\
 \hline
 16 & 39.81 & 39.71 & 39.90 & 40.38 & 68.10 & 66.33 & 69.51 & 69.42  \\
 \hline
 \hline
 \multicolumn{5}{|c|}{Winogrande (FP32 Accuracy = 67.88\%)} & \multicolumn{4}{|c|}{Piqa (FP32 Accuracy = 78.78\%)} \\ 
 \hline
 \hline
 64 & 66.85 & 66.61 & 67.72 & 67.88 & 77.31 & 77.42 & 77.75 & 77.64 \\
 \hline
 32 & 67.25 & 67.72 & 67.72 & 67.00 & 77.31 & 77.04 & 77.80 & 77.37  \\
 \hline
 16 & 68.11 & 68.90 & 67.88 & 67.48 & 77.37 & 78.13 & 78.13 & 77.69  \\
 \hline
\end{tabular}
\caption{\label{tab:mmlu_abalation} Accuracy on LM evaluation harness tasks on GPT3-8B model.}
\end{table}

\begin{table} \centering
\begin{tabular}{|c||c|c|c|c||c|c|c|c|} 
\hline
 $L_b \rightarrow$& \multicolumn{4}{c||}{8} & \multicolumn{4}{c||}{8}\\
 \hline
 \backslashbox{$L_A$\kern-1em}{\kern-1em$N_c$} & 2 & 4 & 8 & 16 & 2 & 4 & 8 & 16  \\
 %$N_c \rightarrow$ & 2 & 4 & 8 & 16 & 2 & 4 & 2 \\
 \hline
 \hline
 \multicolumn{5}{|c|}{Race (FP32 Accuracy = 40.67\%)} & \multicolumn{4}{|c|}{Boolq (FP32 Accuracy = 76.54\%)} \\ 
 \hline
 \hline
 64 & 40.48 & 40.10 & 39.43 & 39.90 & 75.41 & 75.11 & 77.09 & 75.66 \\
 \hline
 32 & 39.52 & 39.52 & 40.77 & 39.62 & 76.02 & 76.02 & 75.96 & 75.35  \\
 \hline
 16 & 39.81 & 39.71 & 39.90 & 40.38 & 75.05 & 73.82 & 75.72 & 76.09  \\
 \hline
 \hline
 \multicolumn{5}{|c|}{Winogrande (FP32 Accuracy = 70.64\%)} & \multicolumn{4}{|c|}{Piqa (FP32 Accuracy = 79.16\%)} \\ 
 \hline
 \hline
 64 & 69.14 & 70.17 & 70.17 & 70.56 & 78.24 & 79.00 & 78.62 & 78.73 \\
 \hline
 32 & 70.96 & 69.69 & 71.27 & 69.30 & 78.56 & 79.49 & 79.16 & 78.89  \\
 \hline
 16 & 71.03 & 69.53 & 69.69 & 70.40 & 78.13 & 79.16 & 79.00 & 79.00  \\
 \hline
\end{tabular}
\caption{\label{tab:mmlu_abalation} Accuracy on LM evaluation harness tasks on GPT3-22B model.}
\end{table}

\begin{table} \centering
\begin{tabular}{|c||c|c|c|c||c|c|c|c|} 
\hline
 $L_b \rightarrow$& \multicolumn{4}{c||}{8} & \multicolumn{4}{c||}{8}\\
 \hline
 \backslashbox{$L_A$\kern-1em}{\kern-1em$N_c$} & 2 & 4 & 8 & 16 & 2 & 4 & 8 & 16  \\
 %$N_c \rightarrow$ & 2 & 4 & 8 & 16 & 2 & 4 & 2 \\
 \hline
 \hline
 \multicolumn{5}{|c|}{Race (FP32 Accuracy = 44.4\%)} & \multicolumn{4}{|c|}{Boolq (FP32 Accuracy = 79.29\%)} \\ 
 \hline
 \hline
 64 & 42.49 & 42.51 & 42.58 & 43.45 & 77.58 & 77.37 & 77.43 & 78.1 \\
 \hline
 32 & 43.35 & 42.49 & 43.64 & 43.73 & 77.86 & 75.32 & 77.28 & 77.86  \\
 \hline
 16 & 44.21 & 44.21 & 43.64 & 42.97 & 78.65 & 77 & 76.94 & 77.98  \\
 \hline
 \hline
 \multicolumn{5}{|c|}{Winogrande (FP32 Accuracy = 69.38\%)} & \multicolumn{4}{|c|}{Piqa (FP32 Accuracy = 78.07\%)} \\ 
 \hline
 \hline
 64 & 68.9 & 68.43 & 69.77 & 68.19 & 77.09 & 76.82 & 77.09 & 77.86 \\
 \hline
 32 & 69.38 & 68.51 & 68.82 & 68.90 & 78.07 & 76.71 & 78.07 & 77.86  \\
 \hline
 16 & 69.53 & 67.09 & 69.38 & 68.90 & 77.37 & 77.8 & 77.91 & 77.69  \\
 \hline
\end{tabular}
\caption{\label{tab:mmlu_abalation} Accuracy on LM evaluation harness tasks on Llama2-7B model.}
\end{table}

\begin{table} \centering
\begin{tabular}{|c||c|c|c|c||c|c|c|c|} 
\hline
 $L_b \rightarrow$& \multicolumn{4}{c||}{8} & \multicolumn{4}{c||}{8}\\
 \hline
 \backslashbox{$L_A$\kern-1em}{\kern-1em$N_c$} & 2 & 4 & 8 & 16 & 2 & 4 & 8 & 16  \\
 %$N_c \rightarrow$ & 2 & 4 & 8 & 16 & 2 & 4 & 2 \\
 \hline
 \hline
 \multicolumn{5}{|c|}{Race (FP32 Accuracy = 48.8\%)} & \multicolumn{4}{|c|}{Boolq (FP32 Accuracy = 85.23\%)} \\ 
 \hline
 \hline
 64 & 49.00 & 49.00 & 49.28 & 48.71 & 82.82 & 84.28 & 84.03 & 84.25 \\
 \hline
 32 & 49.57 & 48.52 & 48.33 & 49.28 & 83.85 & 84.46 & 84.31 & 84.93  \\
 \hline
 16 & 49.85 & 49.09 & 49.28 & 48.99 & 85.11 & 84.46 & 84.61 & 83.94  \\
 \hline
 \hline
 \multicolumn{5}{|c|}{Winogrande (FP32 Accuracy = 79.95\%)} & \multicolumn{4}{|c|}{Piqa (FP32 Accuracy = 81.56\%)} \\ 
 \hline
 \hline
 64 & 78.77 & 78.45 & 78.37 & 79.16 & 81.45 & 80.69 & 81.45 & 81.5 \\
 \hline
 32 & 78.45 & 79.01 & 78.69 & 80.66 & 81.56 & 80.58 & 81.18 & 81.34  \\
 \hline
 16 & 79.95 & 79.56 & 79.79 & 79.72 & 81.28 & 81.66 & 81.28 & 80.96  \\
 \hline
\end{tabular}
\caption{\label{tab:mmlu_abalation} Accuracy on LM evaluation harness tasks on Llama2-70B model.}
\end{table}

%\section{MSE Studies}
%\textcolor{red}{TODO}


\subsection{Number Formats and Quantization Method}
\label{subsec:numFormats_quantMethod}
\subsubsection{Integer Format}
An $n$-bit signed integer (INT) is typically represented with a 2s-complement format \citep{yao2022zeroquant,xiao2023smoothquant,dai2021vsq}, where the most significant bit denotes the sign.

\subsubsection{Floating Point Format}
An $n$-bit signed floating point (FP) number $x$ comprises of a 1-bit sign ($x_{\mathrm{sign}}$), $B_m$-bit mantissa ($x_{\mathrm{mant}}$) and $B_e$-bit exponent ($x_{\mathrm{exp}}$) such that $B_m+B_e=n-1$. The associated constant exponent bias ($E_{\mathrm{bias}}$) is computed as $(2^{{B_e}-1}-1)$. We denote this format as $E_{B_e}M_{B_m}$.  

\subsubsection{Quantization Scheme}
\label{subsec:quant_method}
A quantization scheme dictates how a given unquantized tensor is converted to its quantized representation. We consider FP formats for the purpose of illustration. Given an unquantized tensor $\bm{X}$ and an FP format $E_{B_e}M_{B_m}$, we first, we compute the quantization scale factor $s_X$ that maps the maximum absolute value of $\bm{X}$ to the maximum quantization level of the $E_{B_e}M_{B_m}$ format as follows:
\begin{align}
\label{eq:sf}
    s_X = \frac{\mathrm{max}(|\bm{X}|)}{\mathrm{max}(E_{B_e}M_{B_m})}
\end{align}
In the above equation, $|\cdot|$ denotes the absolute value function.

Next, we scale $\bm{X}$ by $s_X$ and quantize it to $\hat{\bm{X}}$ by rounding it to the nearest quantization level of $E_{B_e}M_{B_m}$ as:

\begin{align}
\label{eq:tensor_quant}
    \hat{\bm{X}} = \text{round-to-nearest}\left(\frac{\bm{X}}{s_X}, E_{B_e}M_{B_m}\right)
\end{align}

We perform dynamic max-scaled quantization \citep{wu2020integer}, where the scale factor $s$ for activations is dynamically computed during runtime.

\subsection{Vector Scaled Quantization}
\begin{wrapfigure}{r}{0.35\linewidth}
  \centering
  \includegraphics[width=\linewidth]{sections/figures/vsquant.jpg}
  \caption{\small Vectorwise decomposition for per-vector scaled quantization (VSQ \citep{dai2021vsq}).}
  \label{fig:vsquant}
\end{wrapfigure}
During VSQ \citep{dai2021vsq}, the operand tensors are decomposed into 1D vectors in a hardware friendly manner as shown in Figure \ref{fig:vsquant}. Since the decomposed tensors are used as operands in matrix multiplications during inference, it is beneficial to perform this decomposition along the reduction dimension of the multiplication. The vectorwise quantization is performed similar to tensorwise quantization described in Equations \ref{eq:sf} and \ref{eq:tensor_quant}, where a scale factor $s_v$ is required for each vector $\bm{v}$ that maps the maximum absolute value of that vector to the maximum quantization level. While smaller vector lengths can lead to larger accuracy gains, the associated memory and computational overheads due to the per-vector scale factors increases. To alleviate these overheads, VSQ \citep{dai2021vsq} proposed a second level quantization of the per-vector scale factors to unsigned integers, while MX \citep{rouhani2023shared} quantizes them to integer powers of 2 (denoted as $2^{INT}$).

\subsubsection{MX Format}
The MX format proposed in \citep{rouhani2023microscaling} introduces the concept of sub-block shifting. For every two scalar elements of $b$-bits each, there is a shared exponent bit. The value of this exponent bit is determined through an empirical analysis that targets minimizing quantization MSE. We note that the FP format $E_{1}M_{b}$ is strictly better than MX from an accuracy perspective since it allocates a dedicated exponent bit to each scalar as opposed to sharing it across two scalars. Therefore, we conservatively bound the accuracy of a $b+2$-bit signed MX format with that of a $E_{1}M_{b}$ format in our comparisons. For instance, we use E1M2 format as a proxy for MX4.

\begin{figure}
    \centering
    \includegraphics[width=1\linewidth]{sections//figures/BlockFormats.pdf}
    \caption{\small Comparing LO-BCQ to MX format.}
    \label{fig:block_formats}
\end{figure}

Figure \ref{fig:block_formats} compares our $4$-bit LO-BCQ block format to MX \citep{rouhani2023microscaling}. As shown, both LO-BCQ and MX decompose a given operand tensor into block arrays and each block array into blocks. Similar to MX, we find that per-block quantization ($L_b < L_A$) leads to better accuracy due to increased flexibility. While MX achieves this through per-block $1$-bit micro-scales, we associate a dedicated codebook to each block through a per-block codebook selector. Further, MX quantizes the per-block array scale-factor to E8M0 format without per-tensor scaling. In contrast during LO-BCQ, we find that per-tensor scaling combined with quantization of per-block array scale-factor to E4M3 format results in superior inference accuracy across models. 

\end{document}