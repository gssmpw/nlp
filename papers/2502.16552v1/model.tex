\section{System Model}
\subsection{The RBG Graph}
We model agents and hubs locations by $\Phi$ and $\Psi$, two jointly stationary and ergodic spatial point processes on $\mathbb{R}^d$, defined on a probability space $(\Omega,\mathcal{A},\mathbb{P})$.
%%%%%%%%%%%%%%%%%%%%%%%%%%%%%%%%%%%
Let  $\lambda$ and $\mu$ denote their intensities, respectively. We focus on the case where $\Phi$ and $\Psi$ are independent. Denote by $I(x,y)\in\{0,1\},~x\in\Phi,y\in\Psi$ the  indicator function of the undirected edge $\{x,y\}$, which is a Bernoulli random variable. $I(x,y)=1$ when there is an edge between $x,y$ and is zero otherwise. We assume that the distribution of $I(x,y)$ is independent of other edge indicators given $\Phi,~\Psi$. Further,
\begin{equation}
    \Pr(I(x,y)=1) =  f(\|x-y\|),\quad x\in\Phi,~y\in\Psi,
\end{equation} where $\|\cdot\|$ is the Euclidean norm and $f\colon \mathbb{R}^+\to[0,1]$ is referred to as {the connection function}. Let the edges of the graph be denoted by $\mathcal{E}$. We have
\[\mathcal{E} = \{\{x,y\}: I(x,y)=1,x\in\Phi,y\in\Psi\}.\] For practical relevance, we assume that $f$ is non-increasing and that $\int_{0}^{\infty} f(r) r^{d-1}\dd r <\infty$; the latter implies that the probability of an agent having unboundedly many edges is zero. 

The above defines an RBG graph, which we denote by $\mathcal{G}(\Phi,\Psi, f)$. 
 With a slight abuse of terminology, we say that two agents are \textit{connected} whenever there is a direct (two-edge) path  in $\mathcal{G}(\Phi,\Psi, f)$. From this definition, one can extract its corresponding unipartite geometric graph consisting of only the agents, where an edge exists between two connected agents. 

By contrast, a random (unipartite) geometric graph is generated on a spatial point process $\Phi$ on $\mathbb{R}^d$ with a connection function $f\colon \mathbb{R}^+\to[0,1]$, where $\mathcal{E} = \{\{x,y\}: I(x,y)=1,x\neq y,~x,y\in\Phi\}$ and $\Pr(I(x,y)=1) =  f(\|x-y\|),~x\neq y,~x,y\in\Phi$. 
We denote this unipartite graph by $\mathcal{G}(\Phi,f)$. 
The percolation of $\mathcal{G}(\Phi,f)$ for  a PPP $\Phi$ is studied in \cite[Chapters 3\&6]{MeesterRonald1996CP}, both for $f(r) = \ind (r\leq a)$, known as the Poisson Boolean model, and a generalized connection function, known as the Poisson connection model.
%There are at least two ways to specify the edges: the first is to define a multigraph where the number of edges between two nodes is the number of common hubs between them, i.e., $\hat I(x,x')=\sum_{y\in\Psi}I_k(x,y)I_k(x',y) $; the second is to define a simple graph where $I(x,x')=1$ when two nodes are adjacent and zero otherwise, i.e., $\hat I(x,x')=\max_{y\in\Psi }I(x,y)I(x',y)$. This graph is essentially an intersection graph formed by the family sets $\{\Psi_x\}_{x\in\Phi}$, where $\mathcal{Psi}_x=\{y\in\Psi: I(x,y) =1\}$, and two nodes have an edge when their sets intersect. We consider simple graphs only.
 
 
 Fig. \ref{fig:illu} shows snapshots of the RBG graph under two connection functions, one being the Boolean type, and the other being exponential. Both agent and hub locations are realizations of PPPs. 

  To model hubs of different popularity and density, the current model can be generalized to a heterogeneous spatial model similar to that of a multi-tier cellular network, which we will not discuss here.
  \begin{figure*}[t]
       \centering
       \subfigure[$f_1(r) = \ind(r\leq 0.1262)$]{         \includegraphics[width=.42\textwidth]{fig/boolean_illu.eps}
       %  \caption{$f(r) = \exp(-9.426r).$}
         \label{fig: illu-f}}
     \hfill
     \subfigure[$f_2(r) = 1/2\exp(r/ 0.1262).$]{         \includegraphics[width=.42\textwidth]{fig/exp_illu.eps}
      %   \caption{$f(r) = 0.5\ind{(r \leq 0.212)}$.}
         \label{fig: illu-g}}
       \caption{Illustrating one realization of the RBG graph on $\mathbb{R}^2$, where only its segment in a square window $[-0.5,0.5]^2$ is presented. Hubs are marked with circles, agents with dots, and edges with line segments. Agents without edges are in light gray. In both figures, the density of agents is 100 and the density of hubs is 10; the average number of edges of the typical hub is 5.  }
       \label{fig:illu}
   \end{figure*} 
\subsection{Dispersed RBG Graph}
 For a connection function $f$ and $p\in(0,1]$, we define its dispersed version
\begin{equation}
f_{p}(r) \triangleq p f(\sqrt[d]{p}r).
\end{equation}
{The dispersed version makes the connection function more spread-out and maintains the monotonicity. For example, when the support of $f$ is finite, dispersed version $f_p$ has expanded support. Edges with distance in the original support become less
reliable due to the monotonicity of $f$.}
It is easy to see that
$\int_0^{\infty} f(r)r^{d-1} \dd r = \int_{0}^{\infty} f_p(r)r^{d-1} \dd r$, i.e., the mean degrees of agents and hubs do not change. Specially, for $d=2$, we have $f_{p}(r) =p f(\sqrt{p}r)$, as defined in \cite{Franceschetti05continuumpercolation}. The RBG graph defined by the dispersed version of  connection function $f$ is $\mathcal{G}(\Phi,\Psi, f_p)$. 




% \subsection{Static Case} 
%  %Here we consider $\mathcal{E}_{k} \equiv \mathcal{E}_{0}$ and $\mathcal{G}_k \equiv \mathcal{G}_0$ for $k\in\mathbb{N}_0$. $I_0(x,y)$ is Bernoulli with mean $p(x,y)\in[0,1]$. We define $\mathcal{G} = \mathcal{G}_0$ for simplicity. In the static case, 
%  Here, we assume that an infection occurs when a node is adjacent to at least one infected node and there is no recovery. We are concerned with whether the typical node (formally defined in the next section) is connected to infinitely many nodes with a positive probability, i.e., whether percolation occurs on the RBG graph.



%\subsection{Contribution}

% 
%\subsection{Contact Pattern}

% \hl{Is this the solution of the differential equation?}




% \subsection{Main Results}


