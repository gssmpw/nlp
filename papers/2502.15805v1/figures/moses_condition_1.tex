\begin{figure}[H]
    \centering
    \setlength{\abovecaptionskip}{0pt}  % 캡션 위 여백 없애기
    \setlength{\belowcaptionskip}{0pt}  % 캡션 아래 여백 없애기
    
    % 첫 번째 그림
    \begin{subfigure}[b]{0.49\textwidth}  % 49% 너비로 설정 (가로 정렬 유도)
        \centering
        \includegraphics[width=\linewidth]{figures/images/moses_condition_logp.pdf}
        % \caption{}
        \label{fig:moses_condition_logp}
    \end{subfigure}
    \hfill  % 두 그림 사이의 간격을 균등하게 조정
    % 두 번째 그림
    \begin{subfigure}[b]{0.49\textwidth}  % 동일한 49% 너비 설정
        \centering
        \includegraphics[width=\linewidth]{figures/images/moses_condition_nring.pdf}
        % \caption{}
        \label{fig:moses_condition_nring}
    \end{subfigure}
    
    \caption{\textbf{Condition MAE and FCD curves for logP and number of rings.} The triangle denotes the unconditional models. Each curve is color-coded to represent a model with a different target value. Additional results for QED and TPSA are provided in \cref{appsubsec:conditional_generation}.}
    \label{fig:moses_condition_1}
\end{figure}