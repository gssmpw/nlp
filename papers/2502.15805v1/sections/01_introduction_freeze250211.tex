\section{Introduction}

\color{blue}
% % Molecular design, drug discovery
% Molecular design is a foundational aspect of many scientific disciplines, encompassing not only the synthesis of novel chemical entities but also the optimization of existing compounds for desired physical, chemical, or biological properties.
% In particular, drug discovery stands out as one of the most important applications of molecular design.
% Identifying and refining bioactive molecules against specific targets is paramount for developing new therapeutics.
% However, exploring the vast chemical space—even for a single drug candidate—can be prohibitively expensive and time-consuming when relying solely on experimental methods.
% 
% % Molecular generation, graph generation
% Molecular generation methods have emerged as a compelling complement to experimental screening in drug discovery.
% By learning the underlying distributions of molecular structures, generative models can not only propose novel compounds but also guide them toward lead-like properties \cite{example1, example2}. 
% The molecular design is often formulated as graph generation task: each node and edge in a molecular graph corresponds to an atom and a bond, respectively.
% Although many denoising diffusion methods successfully generate valid molecules \cite{}, such methods face significant challenges when modeling large or complex molecules: the combinatorial space of atoms and bonds grows exponentially, leading to computational bottlenecks in both training and sampling.
% Moreover, atomistic-level generation may struggle to capture high-level structural motifs—such as functional groups or ring systems—without introducing heuristics or constraints, potentially limiting interpretability and scalability.
% 
% % Fragment based approach
% Fragment-based design has a longstanding history in medicinal chemistry and drug discovery, where molecules are broken down into chemically meaningful fragments such as ring systems, functional groups, or other substructures and systematically recombined to form larger, more complex molecules. 
% Similar to the conventional design strategy, fragment-based generative models treat each fragment as a single generative unit rather than generating individual atoms.
% Theses approaches also naturally incorporate established chemical knowledge, aiding in the efficient discovery and optimization of lead compounds through fragment assembly.

% This coarse-grained representation reduces the combinatorial explosion characteristic of atom-level generation and naturally encodes higher-level chemical knowledge, simplifying the learning process.
% Moreover, fragment-based models can facilitate interpretability and provide more direct control over global molecular motifs, all while maintaining the flexibility to explore a broad chemical space.
% Recent studies have shown that fragment-level strategies can not only improve sampling efficiency and scalability but also enable the design of novel compounds with desired properties \cite{FragVAE, FragDiff}.
% As a result, alternative strategies that reduce complexity while retaining essential chemical information have gained increasing attention in molecular generative modeling.

% Our methods and experiments

% Contributions

\color{black}

% Denoising diffusion models have achieved remarkable success in generating high-quality data across diverse domains such as images, text, and graphs. Recently, their application has been extended to \textcolor{\ksh}{molecular graph generation} \st{graph-based molecule generation}, where graph \textcolor{\ksh}{denoising} \st{diffusion} models \cite{gdss, defog, digress} leverage the structural properties of molecular graphs to generate chemically valid molecules.
Deep generative models, represented by diffusion and flow matching, have achieved remarkable success in generating high-quality data across diverse domains such as images \citep{glide_image, ldm, ddpm}, text \citep{diffusion_lm}, and videos \citep{videocontrolnet, vdm}.
Recently, their application has been extended to molecular graph generation, where graph generative models leverage the structural properties of molecular graphs to generate chemically valid molecules \citep{gdss, digress, defog}. 
These approaches employ stochastic processes on graph representations, showing potential in generating and controlling molecular structures during the generative process.

% Challenges of atomistic graph generation
Fundamentally, graph generative models utilizing diffusion or flow matching techniques reconstruct molecular graphs by sampling categorical representations of atoms and bonds from probability pathways \citep{digress, defog}.
% This process can utilize continuous diffusion \cite{gdss} or techniques such as diffusion or flow matching in categorical spaces \cite{digress, defog}.
Despite their effectiveness, these models face significant scalability challenges in generating larger molecular graphs.
A primary limitation lies in the quadratic power growth of the number of edges as graph size increases, combined with the inherent sparsity of chemical bonds compared to all connected edges in real-world molecules.
This combination amplifies the computational burden of edge prediction and complicates the preservation of chemical validity, thereby limiting their applicability to larger molecular systems.

% Challenges of atomistic graph representation 
Graph neural networks (GNNs) also encounter difficulties capturing topological features such as rings and loops. 
This limitation often results in structural deviations during generation, causing the molecular distribution to unintentionally diverge from the training set and produce molecules with unexpected structural variations.
% This limitation often results in suboptimal generation of complex molecular structures, deviating from the target molecular distribution. 
To mitigate these issues, methods such as \citet{digress, defog} integrate auxiliary features such as spectral, ring, and valency information to improve global structural understanding.
However, these enhancements fail to fully address the sparsity issues and remain computationally impractical for large-scale molecular generation \citep{digress}. 
% 수정제안?: While these approaches improve certain aspects of molecular validity, they do not entirely resolve the sparsity issues and may introduce additional computational overhead, particularly when scaling to large molecular graphs \cite{digress}.
% Moreover, complex molecular properties such as synthetic accessibility and drug drug-likeness requires an understanding of 

% Fragment approach
To overcome these limitations, a fragment-based molecule generation approach has been explored, leveraging a longstanding history in molecular design \citep{hajduk2007decade, joseph2014fragment, kirsch2019concepts}.
In fragment-based methods, molecules are represented at the fragment level, treating molecular substructures (e.g., functional groups or ring systems) as semantically meaningful units.
% In fragment-based methods, molecules are represented at the fragment level, treating molecular substructures (e.g., functional groups or ring systems) as cohesive units.
Decomposing molecules into chemically relevant fragments naturally incorporates established domain knowledge while significantly reducing complexity and system size.
Within molecular generative frameworks, fragment-based methods have demonstrated superior scalability for generating large molecules, offering better control over molecular properties than atom-based approaches \citep{jtvae, hierdiff, bbar, magnet}. 
% Within auto-regressive frameworks, fragment-based methods have demonstrated superior scalability for generating large molecules, offering better control over molecular properties than atom-based approaches \cite{}.

% Challenges of fragment approach
Despite their advantages, fragment-based strategies in one-shot generative models remain limited. 
While fragment-level representations enhance scalability and structural validity, existing approaches \citep{fragment_based_diffusion, orgmol_design} often constrain the chemical space by constrained fragment library or utilize non-chemically driven fragmentation methods. 
Expanding the fragment library to cover a broader chemical space can mitigate the limitation; however, it introduces significant computational cost and modeling challenges in handling a vast number of fragment types. 

% Challenges of fragment approach
% Despite these benefits, the fragment-based strategy's application to one-shot generative models remains limited compared to the atom-based one.
% While fragment-level representations enhance scalability and structural validity, they introduce additional challenges. 
% Recent studies, such as \citep{fragment_based_diffusion}, have demonstrated the potential of fragment-based diffusion models, but these approaches are often constrained by the predefined fragment library used during training.
% Consequently, the molecular space available for exploration is constrained to the assemblies of a predefined fragment library \citep{fragment_based_diffusion}.
% While adopting a larger fragment library can expand the accessible molecular space, it causes higher computational costs.


% Our method
% Here, we propose FragFM, a discrete flow matching model for generating graphs of fragments with .
Here, we present \methodname{}, a discrete flow matching model for generating coarse fragment graphs.
Rather than relying on a fixed, large fragment library, \methodname{} employs a sub-sampling strategy coupled with GNN-based fragment embeddings.
This design effectively reduces the categorical dimension while enabling the model to incorporate unseen fragments, bridging the gap between chemical coverage and computational efficiency.
% To restore the atomistic graph from the fragment representation, \methodname{} employs a variational auto-encoding (VAE) approach.
% Specifically, each molecule is encoded into a corresponding coarse graph and an additional latent vector that captures the structural details omitted during fragmentation.
% By generating this latent vector in an auxiliary manner, \methodname{} maintains fine-grained control over atom-level features, enabling the model to refine not just the selection of fragments but also the precise connectivity and orientation within the final molecular structure. 

% experimental achievement
Extensive benchmarks demonstrate that \methodname{} consistently outperforms existing models across multiple metrics, including validity, property-based evaluation, and Fréchet ChemNet Distance (FCD), requiring significantly fewer denoising steps.
Furthermore, we analyze the impact of fragment-based molecular generation on sampling efficiency, property control, and scalability, highlighting its advantages over atom-based generative models.
% By leveraging a fragment-level representation and a coarse-to-fine generative approach, \methodname{} achieves superior sampling efficiency, requiring significantly fewer denoising steps while maintaining high molecular validity.
%% Additionally, our model demonstrates enhanced controllability over molecular properties, enabling the generation of molecules with desired characteristics more effectively than previous atomistic diffusion models.
To further assess the capabilities for generating more complex molecules, we introduce a natural product benchmark named after COCONUT \citep{coconut, coconut2}, evaluating the ability of generative models to construct structurally complex and biologically relevant molecules.
Our findings suggest that integrating fragment-level representations into generative frameworks can be a scalable and interpretable solution for molecular design, paving the way for advancements in drug discovery, materials science, and functional molecular engineering.