\section{Conclusion}
\label{sec:conclusion}

We introduce \methodname{}, a fragment-based discrete flow matching framework for molecular graph generation. 
By leveraging fragment-level representations with a coarse-to-fine autoencoder and a fragment bag selection approach, \methodname{} enables efficient and accurate molecular generation while preserving structural diversity and validity. 
Extensive benchmarks on MOSES, GuacaMol, and natural product datasets demonstrate that \methodname{} consistently outperforms existing diffusion and flow-based models across multiple metrics, including validity and Fréchet ChemNet Distance (FCD), while requiring fewer denoising steps.

Additionally, we explore efficient sampling strategies and classifier-free guidance, showcasing the effectiveness of \methodname{} in property-aware molecular generation. 
Our results highlight the potential of fragment-based modeling to improve the scalability and expressiveness of denoising-based molecular generation frameworks. 
% Future directions include extending this approach for advanced property control and optimization across diverse applications, such as designing protein-binding molecules, OLED materials, and novel protein structures.
Looking forward, an important direction is to enhance \methodname{} by incorporating fragment bag control, allowing for more precise selection and optimization of fragment compositions.
By dynamically guiding fragment selection based on molecular properties and downstream design objectives, this approach could further improve the controllability of molecular generation across diverse applications, including protein-ligand binding design, OLED materials, and novel functional molecules.