% \section{Fragment-level Flow Matching}
\section{Fragment-level Molecular Graph Generation}

We propose \methodname, a novel molecular graph generative framework that uses discrete flow matching at the fragment level. In this approach, each fragment type is represented as a node, and edges denote fragment connections, enabling a discrete flow matching procedure on the resulting fragment-level graph. Because a single fragment arrangement can correspond to multiple molecular structures (depending on how fragment junctions are permuted), we bridge fragment- and atom-level representations via an autoencoder that reconstructs the atom-level graph from its fragment-level graph. The learned latent variables are then generated through continuous flow-matching, jointly trained alongside the fragment-level model.

\subsection{Problem Statement} 
\label{sec:prob_state}

First, we represent a molecule at the atom level as a graph $G_a=(a^{1:n:N_{a}},b^{1:i<j:N_{a}})$, where $a^{(i)}$ denotes the discrete atom type of the $i$th atom and $b^{(ij)}$ denotes the bond type between the $i$th and $j$th atoms (including the option of no bond). A molecular fragment is a group of connected atoms formed by removing certain bonds in a molecule. Consequently, the same molecule can also be expressed at the fragment level as $G_f=(f^{1:n:N_{f}},e^{1:i<j:N_{f}})$, with $N_{f} \le N_{a}$, where $f^{(i)}$ corresponds to the discrete fragment type of the $i$th fragment and $e^{(ij)}$ captures the connection between fragments. All atoms are uniquely included in a single fragment. Formally, let 

    \begin{equation}
        F^{(i)} \;=\; \bigl\{\, n \in \{1,\dots,N_a\} \;\mid\; \phi(n) = i \bigr\}   
    \end{equation}

where $\phi$ is the mapping function that assigns each atom $n$ to its corresponding fragment $i$. Thus, $F^{(i)}$ contains exactly those atom indices of fragment $i$.


\subsection{Coarse-to-fine Autoencoder} 
\label{sec:coarse_fine_ae}

A single fragment‐level graph representation may not uniquely determine a single atom‐level graph. 



\subsection{Discrete Flow Matching for Coarse Graph} 
\label{sec:coarse_graph_dfm}



% Discrete flow matching \cite{dfm_1, dfm_2} shown 

\[
\mathcal{F}_b \;\sim\; \text{SampleSubset}\bigl(\mathcal{F}, B\bigr),
\quad
|\mathcal{F}b| = B \ll |\mathcal{F}|.
\]


\[
\partial_t \hat{p}_t(z \mid z_1)
\;=\;
\frac{1}{|\mathcal{F}b|}
\sum{f \in \mathcal{F}_b}
\ell\bigl(z, f, t\bigr)
\]

\[
R^t\bigl(z_t, z{t+\mathrm{d}t} \mid z_1\bigr)
\;=\;
\frac{\mathrm{ReLU}
\Bigl[\partial_t p_t\bigl(z_{t+\mathrm{d}t}\mid z_1\bigr)
\;-\;
\partial_t p_t\bigl(z_t\mid z_1\bigr)\Bigr]}
{Z^{>0}_t \, p_t\bigl(z_t\mid z_1\bigr)}
\]

\[
\partial_t \hat{p}_t(z \mid z_1; \mathcal{F}_b)
\;=\;
\frac{1}{|\mathcal{F}b|}
\sum{\substack{m : z_t^{(m)} \in \mathcal{F}b}}
\bigl[
\delta\bigl(z - z{t+\mathrm{d}t}^{(m)}\bigr)
\;-\;
\delta\bigl(z - z_t^{(m)}\bigr)
\bigr],
\]



\[
\frac{\partial}{\partial t} p_t(x)
\;=\;
\sum_{y \in \mathcal{X}}
\Bigl[
p_t(y)\,R_{t}(y \to x)
\;-\;
p_t(x)\,R_{t}(x \to y)
\Bigr].
\tag{1}
\]



\[
\hat{\partial_t p_t}(x)
\;=\;
\frac{1}{M \,\mathrm{d}t}
\sum_{m=1}^{M}
\Bigl[
\mathbf{1}{\{\,x{t+\mathrm{d}t}^{(m)}=x\}}
\;-\;
\mathbf{1}_{\{\,x_t^{(m)}=x\}}
\Bigr].
\tag{2}
\]

Univariate case

\[
p_{t\mid 1}(z_t \mid z_1)
\;=\;
t \,\delta(z_t, z_1)
\;+\; (1 - t)\; p_0(z_t).
\tag{1}
\]

랜덤 서브셋 $\mathcal{B} \subseteq \{1,\dots,Z\}$ 각 학습 스텝마다, 전체 타입 공간 $\{1,\dots,Z\}$ 에서 크기 B 인 부분집합 $\mathcal{B}$ 추출.

\[
\mathcal{B}
\;\sim\;
\text{SampleSubset}\bigl(\{1,\dots,Z\}, B\bigr),
\quad
|\mathcal{B}| = B \ll Z.
\]

\[
p_{t\mid 1}^\mathcal{B}(z_t \mid z_1)
\;=\;
t\,\delta(z_t, z_1)
\;+\;
(1 - t)\,\hat{p}_0^\mathcal{B}(z_t),
\] 

여기서 $\hat{p}_0^\mathcal{B}(z_t)$ 는 $\mathcal{B}$에서 정의되는 초기 분포 (Uniform distribution 도 상관없음) or normalized distribution of $\mathcal{Z}$. 

global rate matrix는 아래와 같음.


\[
R_t^(z_t,z_{t+dt}\mid z_1)
\;\;=\;\;
\frac{\mathrm{ReLU}\Bigl[\partial_t p_{t\mid 1}(z_{t+dt}\mid z_1) \;-\; \partial_t p_{t\mid 1}(z_t\mid z_1)\Bigr]}
{Z_t^{>0}\, p_{t\mid 1}(z_t \mid z_1)},
\]

근데 type이 매우많으면 

\[
 \partial_t p_{t\mid 1}(z_t \mid z_1)와 p_{t\mid 1}(z_t \mid z_1)
\]

를 계산하기가 어려워짐/

$\mathcal{B}$ 안에서의 rate matrix 는 아래와 같음

\[
\hat{\partial_t p}{t\mid 1}(z_t \mid z_1)
\;\approx\;
\frac{1}{\mathrm{d}t}
\Bigl[
\hat{p}{t+\mathrm{d}t\mid 1}(z_t \mid z_1)
\;-\;
\hat{p}{t\mid 1}(z_t \mid z_1)
\Bigr]
\]

\[
R_t^*\bigl(z_t, z_{t+\mathrm{d}t} \mid z_1\bigr)
\;=\;
\frac{
  \mathrm{ReLU}\Bigl[
    \widehat{\partial_t p_{t\mid 1}}\bigl(z_{t+\mathrm{d}t}\mid z_1\bigr)
    \;-\;
    \widehat{\partial_t p_{t\mid 1}}\bigl(z_t\mid z_1\bigr)
  \Bigr]
}{
  Z_t^{>0}\,\widehat{p_{t\mid 1}}\bigl(z_t\mid z_1\bigr)
}
\quad
\text{for}
\quad
z_t \neq z_{t+\mathrm{d}t}.
\]

증명해야하는 것은 $\mathcal{B}$ 안에서의 rate matrix가 (true rate mat 학습된것 말고) 전체 rate matrix에서 특정 fragment만 뽑았을때를 정규화한 값과 같은지?


\[
\mathcal{L}(\theta)
\;=\;
\sum_{x}
\Bigl\|
\widehat{\partial_t p_{t\mid 1}}(x \mid z_1)
\;-\;
\sum_{y}\Bigl[
p_t(y)\, R_t^(y \to x;\theta)
\;-\;
p_t(x)\, R_t^(x \to y;\theta)
\Bigr]
\Bigr\|^2,
\tag{3}
\]

증명해보자. ----

loss 식은 아래와 같아짐

\[
p_{\theta}^{(\mathcal{B})}(z_1 \mid z_t)=
\frac{\exp(\phi_\theta(z_1, z_t))}{
\sum_{z\in \mathcal{B}} \exp(\phi_\theta(z, z_t))
}.
\]


\[
\mathbb{E}{
\substack{
t \sim U[0,1],\,
z_1 \sim p_1,\,
z_t \sim p{t\mid 1},\\
\mathcal{B} \,\sim\, \mathrm{SampleSubset}(\mathcal{Z}, k)
}
}
\Bigl[
-\log\, p_{\theta}^{(\mathcal{B})}(z_1 \mid z_t)
\Bigr].
\]

\[
L_{\mathrm{DFM}}^{(\text{subset})}
=\;
\mathbb{E}_{ 
  \substack{
    t \sim U[0,1],\\
    z_1 \sim p_1,\,
    z_t \sim p_{t\mid 1},\\
    \mathcal{B}\subset \mathcal{Z},\,|\mathcal{B}|=k
  }
}
\Bigl[
  -\log\,
  \frac{
    \exp(\phi_\theta(z_1, z_t))
  }{
    \sum_{z \in \mathcal{B}}
    \exp(\phi_\theta(z, z_t))
  }
\Bigr],
\]


============================================================여기부터 찐
batch내에서 학습되는 probability 는 아래와 같이 interpolate 된다. 

\[
p_{t\mid 1}^\mathcal{B}(z_t \mid z_1)
\;=\;
t\,\delta(z_t, z_1)
\;+\;
(1 - t)\,\hat{p}_0^\mathcal{B}(z_t),
\] 

그러면 자연스럽게 $\mathcal{B}$ 안에서의 rate matrix는 아래와 같이 둘 수 있음 (Kolmogrov에서 유도, 생략)

$R_{t}^{\mathcal{B}}\in\mathbb{R}^{|\mathcal{F}| \times |\mathcal{F}|}$ with values $0$ on rates $\mathcal{B}$

\begin{equation} 
    R_{t}^{\mathcal{B}}\bigl(z_t, z_{t+\mathrm{d}t}\mid z_1\bigr)
    \;=\;
    \frac{\mathrm{ReLU}
    \Bigl[\partial_t p_{t|1}^{\mathcal{B}}\bigl(z_{t+\mathrm{d}t}\mid z_1\bigr)
    \;-\;
    \partial_t p_{t|1}^{\mathcal{B}}\bigl(z_t\mid z_1\bigr)\Bigr]}
    {Z^{>0}_t \, p_t^{\mathcal{B}}\bigl(z_t\mid z_1\bigr)}
    \label{eq:rate_fragbag}
\end{equation}

이때, 전체 타입에 대한 rate mat은 아래와 같음 (배우고 싶은 objective) \cref{eq:rate_fragbag}: 실제로 학습시킨 objective

\begin{equation}
    R_{t}\bigl(z_t, z_{t+\mathrm{d}t}\mid z_1\bigr)
    \;=\;
    \frac{\mathrm{ReLU}
    \Bigl[\partial_t p_{t|1}\bigl(z_{t+\mathrm{d}t}\mid z_1\bigr)
    \;-\;
    \partial_t p_{t|1}\bigl(z_t\mid z_1\bigr)\Bigr]}
    {Z^{>0}_t \, p_t\bigl(z_t\mid z_1\bigr)}
    \label{eq:rate_full}
\end{equation}


이걸 증명해야하는건가?

\begin{equation}
    R_{t}\bigl(z_t, z_{t+\mathrm{d}t}\mid z_1\bigr)
    \;=\;
    \mathbb{E}_{\mathcal{B}\subset\mathcal{F}, |B|=k} 
    \Bigl[R_{t}^{\mathcal{B}}\bigl(z_t, z_{t+\mathrm{d}t}\mid z_1\bigr)\Bigr]
    \label{eq:rate_full_expect}
\end{equation}












그러면, 전역 목적함수 (전체 타입 공간 $Z*Z$ 에 대한)은 다음과 같음
\[
\mathcal{L}(\theta)
\;=\;
\sum_{(z,z{\prime}) \in \mathcal{Z}^2}
\Bigl\|
\widehat{\partial_t p_{t\mid 1}}(z)
\;-\;
\bigl(
p_t^\theta(z{\prime})\,R_t^\theta(z{\prime}\to z)
\;-\;
p_t^\theta(z)\,R_t^\theta(z\to z{\prime})
\bigr)
\Bigr\|^2,
\tag{1}
\]



보여야 할것 1: 이 전역 rate matrix가 있을때 (학습이 아닌, 정의 자체로) $\mathcal{B}$ 를 뽑고, 정규화 과정을 거치면 sub rate matrix가 됨.




그런데, Z가 커서 학습이 어우므로 아래와 같은 부분 목적함수를 도입

\[
\hat{\mathcal{L}}(\theta;\mathcal{B}k)
\;=\;
\sum{
(z,z{\prime}) \,\in\, \mathcal{B}_k \times \mathcal{B}k
}
\Bigl\|
\widehat{\partial_t p{t\mid 1}}(z)
\;-\;
\bigl(
p_t^\theta(z{\prime})\,R_t^\theta(z{\prime}\to z)
\;-\;
p_t^\theta(z)\,R_t^\theta(z\to z{\prime})
\bigr)
\Bigr\|^2.
\tag{2}
\]

\[
\theta_{k+1}
\;=\;
\theta_k
\;-\;
\eta \,\nabla_\theta \,\hat{\mathcal{L}}\bigl(\theta_k;\,\mathcal{B}_k\bigr),
\tag{3}
\]

이러면 전체 목적함수를 expectation으로 근사하는거임.

\[
L_{\mathrm{DFM}}^{(\text{subset})}
=\;
\mathbb{E}_{ 
  \substack{
    t \sim U[0,1],\\
    z_1 \sim p_1,\,
    z_t \sim p_{t\mid 1},\\
    \mathcal{B}\subset \mathcal{Z},\,|\mathcal{B}|=k
  }
}
\Bigl[
  -\log\,
  \frac{
    \exp(\phi_\theta(z_1, z_t))
  }{
    \sum_{z \in \mathcal{B}}
    \exp(\phi_\theta(z, z_t))
  }
\Bigr],
\]


\[
L_{\mathrm{FragFM}}^{(\text{subset})} =
\mathbb{E}_{
\substack{
t \sim U[0,1], z_1 \sim p_1, z_t \sim p{t\mid 1}, \mathcal{B}t \sim \mathrm{SampleSubset}(\mathcal{Z}, B)
}
}
\Bigl[
-\log\, p_{\theta}^{(\mathcal{B}_t)}(z_1 \mid z_t)
\Bigr].
\]



\subsection{Modeling}
