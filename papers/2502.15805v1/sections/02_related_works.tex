\section{Related Works}


\subsection{Denoising Graph Generative Models}

Denoising-based generative models have become fundamental for molecular graph generation by iteratively refining noisy graphs into structured molecular representations. 
Diffusion methods \citep{ddpm, sde}, which have been successful in a variety of domains, have been extended to graph structure data \citep{gdss, score_based_graph_generation}, demonstrating the advantages of applying diffusion in graph generation. 
This approach was further extended by incorporating discrete stochastic processes \citep{d3pm}, addressing the inherently discrete nature of molecular graphs \citep{digress}.
The discrete diffusion modeling is reformulated by the continuous time Markov chain (CTMC), which has been introduced \citep{disco, cometh, ddsbm}, allowing more flexible and adaptive generative processes. 
More recently, flow-based models have been explored for molecular graph generation. 
Continuous flow matching \citep{cfm} has been applied to structured data \citep{catflow}, while discrete flow models \citep{dfm_1, dfm_2} have been extended to categorical data generation, with recent methods demonstrating their effectiveness in modeling molecular distributions \citep{defog, ggflow}.


\subsection{Fragment Based Molecule Generation}

Fragment-based molecular generative models construct new molecules by assembling existing molecular substructures, known as fragments. 
This strategy enhances chemical validity and facilitates the efficient exploration of novel molecular structures.
Several works have employed fragment-based approaches within variational autoencoders (VAEs). 
\citet{multi-obj_molecule_substruct, molgen_by_principal_subgraph_mining, learn_to_extend_scaff} generate molecules using VAEs by learning to assemble fragments in a chemically meaningful way. 
\citet{jtvae} adopts a stepwise generation approach, first constructing a coarse fragment-level graph before refining it into an atom-level molecule through substructure completion. 
\citet{bbar, multi-obj_molecule_substruct} construct molecules by assembling fragments sequentially, enabling better control over molecular properties during generation.
\citet{safe_mol_design, genmol, f-rag} generates and optimizes molecules in modified SMILES representation, showing the strength of fragments based on goal-guided molecular generation.
% \citet{safe_mol_design, genmol, f-rag} leverage fragment-based approaches in SMILES representation to generate and optimize molecules, demonstrating their effectiveness in goal-directed molecular design.

Fragment-based approaches have also been explored in diffusion-based molecular graph generation. 
\citet{fragment_based_diffusion} proposed a method that utilizes a fixed set of frequently occurring fragments to generate drug-like molecules, ensuring chemical validity but limiting exploration beyond predefined structures. 
Since enumerating all possible fragment types is infeasible, the method operates solely within a fixed fragment vocabulary. 
In contrast, \citet{orgmol_design} introduced an alternative fragmentization strategy depending on the dataset based on byte-pair encoding, offering a more flexible molecular representation.
However, this approach cannot still incorporate chemically meaningful fragmentation methods \citep{brics, emolfrag}, making it challenging to utilize domain-specific chemical priors.