\section{Related Work}
One of the primary approaches to addressing the challenge of navigation
in GPS-denied environments involves utilizing ego-acceleration measurements
from onboard accelerometers to estimate a vehicle's pose relative
to its previous position. This technique, known as Dead Reckoning
(DR), integrates acceleration data to derive positional information
\cite{hash2025_RIENG_Avionics}. DR offers a straightforward, cost-effective
solution, particularly with low-cost sensors, making it accessible
for many applications \cite{Hou9162069Pedestrian}. To enhance the
accuracy of Dead Reckoning, a gyroscope is often incorporated to measure
the vehicle's angular velocity. This integration provides additional
orientation data, improving the overall pose estimation. However,
a significant drawback of this method is its susceptibility to cumulative
errors or drift over time. Without supplementary sensors or correction
mechanisms, these errors can accumulate rapidly, leading to inaccurate
navigation results, especially during prolonged use. In controlled
environments like harbors, warehouses, or other predefined spaces,
ultra-wideband (UWB) technology can significantly enhance navigation
accuracy. UWB systems measure distances between the vehicle and fixed
reference points, known as anchors, providing highly accurate and
robust localization data \cite{Hashim2023NonlinearFusion}. This approach
is widely adopted in applications where precision is paramount, such
as robotic operations within structured environments and object-tracking
systems like Apple's AirTag \cite{Roth2022AirTag}. When used alongside
an Inertial Measurement Unit (IMU), accelerometer, and gyroscope within
a DR framework, UWB can serve as an additional sensor to correct positional
errors and mitigate drift \cite{Hashim2023NonlinearFusion}. However,
this solution has limitations since UWB requires the installation
of anchors in the environment, which confines its applicability to
pre-configured spaces. As a result, it may not be suitable for dynamic
or unstructured environments, reducing the system's flexibility and
immediate usability out of the box \cite{hashim2023exponentially}.
Additionally, UWB is susceptible to high levels of noise, which can
degrade estimation accuracy \cite{Hashim2023NonlinearFusion}.

With the development of advanced point cloud registration algorithms
such as Iterative Closest Point (ICP) \cite{121791} and Coherent
Point Drift (CPD) \cite{Myronenko2009PointSR}, sensors capable of
capturing two-dimensional (2D) points from three-dimensional (3D)
space have emerged as promising candidates to complement IMUs without
requiring prior environmental knowledge. Sound Navigation and Ranging
(SONAR) is one such sensor, widely adopted in marine applications
due to its effectiveness in underwater environments, where mechanical
waves propagate efficiently \cite{franchi2021underwater}. Similarly,
Light Detection and Ranging (LiDAR) employs electromagnetic waves
instead of mechanical waves and has demonstrated utility in aerospace
applications, where sound propagation is limited, but light transmission
is effective \cite{christian2013survey}. However, both SONAR and
LiDAR exhibit significant limitations in complex indoor and outdoor
environments, as they rely solely on structural properties and cannot
capture texture or color information. In contrast, recent advancements
in low-cost, high-resolution cameras designed for navigation applications,
combined with robust fusion between IMU and feature detection \cite{hashim2021geometricNAV,hashim2021gps}.
Popular tracking feature detection-based algorithms include Scale-Invariant
Feature Transform (SIFT) \cite{lowe2004distinctive}, Good Features
to Track (GFTT) \cite{shi1994good}, and the Kanade-Lucas-Tomasi (KLT)
algorithm have facilitated the widespread adoption of cameras as correction
sensors alongside IMUs \cite{7511659,mourikis2007multi,sun2018robust,hash2025_RIENG_Avionics}.