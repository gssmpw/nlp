
\section {Conclusion} 
\paragraph{Limitations.}
Although our method achieves high-quality immersive rendering, there are some limitations. 
First, the integration of new attributes into the original Gaussian expression will lead to substantial storage requirements for dynamic Gaussian sequences. Future work will consider compressing these sequences.
Second, we approximate the rendering equation to obtain the 2D material maps, which introduces errors in our decoupled material results and does not accurately reflect the real physical world.
Future work may address this issue by incorporating large models like video generation. Furthermore, our approach, focusing on relighting for a dynamic reconstruction sequence, does not support pose-driven animation or the generation of new poses. Future work will focus on optimizing these aspects to improve robustness and applicability.

We have presented a Gaussian-based approach for reconstructing detailed geometry and PBR materials to produce relightable volumetric videos. We employ a coarse-to-fine training strategy and effective geometric constraints to accurately model the dynamic geometry of 4D Gaussians. Additionally, we decouple PBR materials by using ray tracing to compute the lighting effects and obtain base color and AO maps, while leveraging generative methods to infer roughness. These materials are then baked into the corresponding attributes of the Gaussians. With deferred shading and ray tracing techniques, our Gaussian sequence supports both efficient real-time rendering and more realistic offline rendering. Experimental results demonstrate the advantages of our approach in generating high-quality dynamic normal maps and material decomposition, as well as its relightability under a variety of lighting conditions. Our method is highly compatible with traditional CG engines, offering significant potential for enhancing rendering realism and flexibility.






