%%
%% This is file `sample-acmsmall.tex',
%% generated with the docstrip utility.
%%
%% The original source files were:
%%
%% samples.dtx  (with options: `acmsmall')
%% 
%% IMPORTANT NOTICE:
%% 
%% For the copyright see the source file.
%% 
%% Any modified versions of this file must be renamed
%% with new filenames distinct from sample-acmsmall.tex.
%% 
%% For distribution of the original source see the terms
%% for copying and modification in the file samples.dtx.
%% 
%% This generated file may be distributed as long as the
%% original source files, as listed above, are part of the
%% same distribution. (The sources need not necessarily be
%% in the same archive or directory.)
%%
%% Commands for TeXCount
%TC:macro \cite [option:text,text]
%TC:macro \citep [option:text,text]
%TC:macro \citet [option:text,text]
%TC:envir table 0 1
%TC:envir table* 0 1
%TC:envir tabular [ignore] word
%TC:envir displaymath 0 word
%TC:envir math 0 word
%TC:envir comment 0 0
%%
%%
%% The first command in your LaTeX source must be the \documentclass command.
% \documentclass[manuscript,screen,review]{acmart}

% \documentclass[manuscript,review,anonymous]{acmart}


% \documentclass[manuscript,screen]{acmart}
\documentclass[sigconf]{acmart}
\PassOptionsToPackage{table}{xcolor}
\usepackage{tabularx}
%% NOTE that a single column version is required for 
%% submission and peer review. This can be done by changing
%% the \doucmentclass[...]{acmart} in this template to 
%% \documentclass[manuscript,screen]{acmart}
%% 
%% To ensure 100% compatibility, please check the white list of
%% approved LaTeX packages to be used with the Master Article Template at
%% https://www.acm.org/publications/taps/whitelist-of-latex-packages 
%% before creating your document. The white list page provides 
%% information on how to submit additional LaTeX packages for 
%% review and adoption.
%% Fonts used in the template cannot be substituted; margin 
%% adjustments are not allowed.
%%
%% \BibTeX command to typeset BibTeX logo in the docs
\AtBeginDocument{%
  \providecommand\BibTeX{{%
    \normalfont B\kern-0.5em{\scshape i\kern-0.25em b}\kern-0.8em\TeX}}}

%% Rights management information.  This information is sent to you
%% when you complete the rights form.  These commands have SAMPLE
%% values in them; it is your responsibility as an author to replace
%% the commands and values with those provided to you when you
%% complete the rights form.
% \setcopyright{acmlicensed}
% \copyrightyear{2024}
% \acmYear{2024}
% \acmDOI{XXXXXXX.XXXXXXX}

\copyrightyear{2025}
\acmYear{2025}
\setcopyright{cc}
\setcctype{by-nc}
\acmConference[CHI '25]{CHI Conference on Human Factors in Computing Systems}{April 26-May 1, 2025}{Yokohama, Japan}
\acmBooktitle{CHI Conference on Human Factors in Computing Systems (CHI '25), April 26-May 1, 2025, Yokohama, Japan}\acmDOI{10.1145/3706598.3714255}
\acmISBN{979-8-4007-1394-1/25/04}
%%
%% These commands are for a JOURNAL article.
% \acmJournal{JACM}
% \acmVolume{37}
% \acmNumber{4}
% \acmArticle{111}
% \acmMonth{8}

%%
%% Submission ID.
%% Use this when submitting an article to a sponsored event. You'll
%% receive a unique submission ID from the organizers
%% of the event, and this ID should be used as the parameter to this command.
% \acmSubmissionID{123-A56-BU3}

%%
%% For managing citations, it is recommended to use bibliography
%% files in BibTeX format.
%%
%% You can then either use BibTeX with the ACM-Reference-Format style,
%% or BibLaTeX with the acmnumeric or acmauthoryear sytles, that include
%% support for advanced citation of software artefact from the
%% biblatex-software package, also separately available on CTAN.
%%
%% Look at the sample-*-biblatex.tex files for templates showcasing
%% the biblatex styles.
%%

%%
%% The majority of ACM publications use numbered citations and
%% references.  The command \citestyle{authoryear} switches to the
%% "author year" style.
%%
%% If you are preparing content for an event
%% sponsored by ACM SIGGRAPH, you must use the "author year" style of
%% citations and references.
%% Uncommenting
%% the next command will enable that style.
%%\citestyle{acmauthoryear}

%%
%% end of the preamble, start of the body of the document source.


\usepackage{tabularx}
% \usepackage{threeparttable}
\usepackage{booktabs}
\usepackage{multirow}
\usepackage{makecell}
\usepackage{colortbl}
\usepackage{booktabs}
% \usepackage[table]{xcolor}
\usepackage{array}
\usepackage{pgf}
\usepackage{collcell}
\usepackage{hhline}
\usepackage{longtable}
\usepackage{multirow}
\usepackage{amsmath}

% \usepackage{ackslash}

% \usepackage{adjustbox}
\usepackage{wrapfig}
\usepackage{float}
\usepackage{enumitem}
\usepackage{url}
\usepackage{hyperref}
\usepackage{ragged2e}
\usepackage{tabularray}
\usepackage{setspace}
\usepackage{subcaption}
\usepackage{utfsym}

% \usepackage{tabularx}
% \usepackage{colortbl}
\usepackage{fontawesome}
% \usepackage{longtable}
% \usepackage{tcolorbox}
\usepackage{xcolor}
% \usepackage[T1]{fontenc}
% \usepackage{libertine}
% \usepackage{inconsolata}
% https://www.overleaf.com/learn/latex/Font_typefaces
% \usepackage{geometry}

% \geometry{
%   a4paper,
%   total={170mm,257mm},
%   left=20mm,
%   top=20mm,
% }

% Define colors
% \definecolor{high}{HTML}{67a284} % dark green
\definecolor{high}{HTML}{628c77} % dark green
\definecolor{mid}{HTML}{b9f2d5}  % light green
\definecolor{low}{HTML}{FFFFFF}  % white

\definecolor{high1}{HTML}{7262ac} % purple
\definecolor{mid1}{HTML}{FFFFFF}  % dark green
\definecolor{low1}{HTML}{2e974e}  % light green


\newcommand{\gradientcelld}[8]{
\xdef\lowvalx{#2}%
\xdef\midvalx{#3}%
\xdef\maxvalx{#4}%
\xdef\lowcolx{#5}%
\xdef\midcolx{#6}%
\xdef\highcolx{#7}%
\xdef\opacityx{#8}%
% The values are calculated linearly between \midval and \maxval
\ifdimcomp{#1pt}{>}{\maxvalx pt}{\cellcolor{\highcolx!100.0!\midcolx!\opacityx}#1}{
\ifdimcomp{#1pt}{<}{\midvalx pt}{%
\ifdimcomp{#1pt}{<}{\lowvalx pt}{\cellcolor{\midcolx!0.0!\lowcolx!\opacityx}#1}{
     \pgfmathparse{int(round(100*(#1/(\midvalx-\lowvalx))-(\lowvalx*(100/(\midvalx-\lowvalx)))))}%
    \xdef\tempa{\pgfmathresult}%
    \cellcolor{\midcolx!\tempa!\lowcolx!\opacityx}#1%
}}{
     \pgfmathparse{int(round(100*(#1/(\maxvalx-\midvalx))-(\midvalx*(100/(\maxvalx-\midvalx)))))}
    \xdef\tempb{\pgfmathresult}%
    \cellcolor{\highcolx!\tempb!\midcolx!\opacityx}#1%
}}
}


\newcommand{\gradientcell}[1]{
    \gradientcelld{#1}{-0.04}{0.36}{0.76}{low}{mid}{high}{60}
    }

\newcommand{\g}[1]{
    \gradientcelld{#1}{-0.18}{0}{0.13}{low1}{mid1}{high1}{60}
    }

\newcommand{\gradientcellboldd}[8]{
\xdef\lowvalx{#2}%
\xdef\midvalx{#3}%
\xdef\maxvalx{#4}%
\xdef\lowcolx{#5}%
\xdef\midcolx{#6}%
\xdef\highcolx{#7}%
\xdef\opacityx{#8}%
% The values are calculated linearly between \midval and \maxval
\ifdimcomp{#1pt}{>}{\maxvalx pt}{\cellcolor{\highcolx!100.0!\midcolx!\opacityx}\textbf{#1}}{
\ifdimcomp{#1pt}{<}{\midvalx pt}{%
\ifdimcomp{#1pt}{<}{\lowvalx pt}{\cellcolor{\midcolx!0.0!\lowcolx!\opacityx}\textbf{#1}}{
     \pgfmathparse{int(round(100*(#1/(\midvalx-\lowvalx))-(\lowvalx*(100/(\midvalx-\lowvalx)))))}%
    \xdef\tempa{\pgfmathresult}%
    \cellcolor{\midcolx!\tempa!\lowcolx!\opacityx}\textbf{#1}%
}}{
     \pgfmathparse{int(round(100*(#1/(\maxvalx-\midvalx))-(\midvalx*(100/(\maxvalx-\midvalx)))))}
    \xdef\tempb{\pgfmathresult}%
    \cellcolor{\highcolx!\tempb!\midcolx!\opacityx}\textbf{#1}%
}}
}

\newcommand{\gradientcellbold}[1]{
    \gradientcellboldd{#1}{-0.04}{0.36}{0.76}{low}{mid}{high}{60}
}
% \usepackage[T1]{fontenc}

% \renewcommand{\bibpreamble}{\textcolor{blue}{\textit{References marked with} \textbf{\faAsterisk} \textit{are suggested by reviewers.}}}


\usepackage{graphicx}
% \usepackage{color}
\definecolor{darkred}{HTML}{7e0f12}
\definecolor{darkgreen}{rgb}{0.0, 0.5, 0.0}
\definecolor{purple}{HTML}{7262ac}
\begin{document}

%%
%% The "title" command has an optional parameter,
%% allowing the author to define a "short title" to be used in page headers.
\title{Deconstructing Depression Stigma: Integrating AI-driven Data Collection and Analysis with Causal Knowledge Graphs}

% [Exploring the Potential of Human-LLM Synergy in Advancing Qualitative Analysis]{Exploring the Potential of Human-LLM Synergy in Advancing Qualitative Analysis: A Case Study on Mental-Illness Stigma}


%%
%% The "author" command and its associated commands are used to define
%% the authors and their affiliations.
%% Of note is the shared affiliation of the first two authors, and the
%% "authornote" and "authornotemark" commands
%% used to denote shared contribution to the research.

\author{Han Meng}
\email{han.meng@u.nus.edu}
\orcid{0009-0003-2318-3639}
\affiliation{
  \institution{Department of Computer Science, National University of Singapore}
  \streetaddress{21 Lower Kent Ridge Road}
  \country{Singapore}
  \postcode{119077}
}
\author{Renwen Zhang}
\email{r.zhang@nus.edu.sg}
\orcid{0000-0002-7636-9598}
\affiliation{
  \institution{Department of Communication and New Media, National University of Singapore}
  \streetaddress{21 Lower Kent Ridge Road}
  \country{Singapore}
  \postcode{119077}
}
\author{Ganyi Wang}
\email{ganyi-w@comp.nus.edu.sg}
\orcid{0009-0009-3294-6363}
\affiliation{
  \institution{School of Computing, National University of Singapore}
  \streetaddress{21 Lower Kent Ridge Road}
  \country{Singapore}
  \postcode{119077}
}
\author{Yitian Yang}
\email{yang.yitian@u.nus.edu}
\orcid{0009-0000-7530-2116}
\affiliation{
  \institution{Department of Computer Science, National University of Singapore}
  \streetaddress{21 Lower Kent Ridge Road}
  \country{Singapore}
  \postcode{119077}
}
\author{Peinuan Qin}
\email{e1322754@u.nus.edu}
\orcid{0000-0002-8737-8369}
\affiliation{
  \institution{Department of Computer Science, National University of Singapore}
  \streetaddress{21 Lower Kent Ridge Road}
  \country{Singapore}
  \postcode{119077}
}
\author{Jungup Lee}
\email{swklj@nus.edu.sg}
\orcid{0000-0002-8243-0543}
\affiliation{
  \institution{Department of Social Work, National University of Singapore}
  \streetaddress{21 Lower Kent Ridge Road}
  \country{Singapore}
  \postcode{119077}
}
\author{Yi-Chieh Lee}
\email{yclee@nus.edu.sg}
\orcid{0000-0002-5484-6066}
\affiliation{
  \institution{Department of Computer Science, National University of Singapore}
  \streetaddress{21 Lower Kent Ridge Road}
  \country{Singapore}
  \postcode{119077}
}


%%
%% By default, the full list of authors will be used in the page
%% headers. Often, this list is too long, and will overlap
%% other information printed in the page headers. This command allows
%% the author to define a more concise list
%% of authors' names for this purpose.
% \renewcommand{\shortauthors}{et al.}

%%
%% The abstract is a short summary of the work to be presented in the
%% article.
\begin{abstract}
Mental-illness stigma is a persistent social problem, hampering both treatment-seeking and recovery. 
Accordingly, there is a pressing need to understand it more clearly, but analyzing the relevant data is highly labor-intensive. 
Therefore, we designed a chatbot to engage participants in conversations; coded those conversations qualitatively with AI assistance; and, based on those coding results, built causal knowledge graphs to decode stigma. 
The results we obtained from 1,002 participants demonstrate that conversation with our chatbot can elicit rich information about people’s attitudes toward depression, while our AI-assisted coding was strongly consistent with human-expert coding.
Our novel approach combining large language models (LLMs) and causal knowledge graphs uncovered patterns in individual responses and illustrated the interrelationships of psychological constructs in the dataset as a whole. 
The paper also discusses these findings’ implications for HCI researchers in developing digital interventions, decomposing human psychological constructs, and fostering inclusive attitudes.
\end{abstract}


%%
%% The code below is generated by the tool at http://dl.acm.org/ccs.cfm.
%% Please copy and paste the code instead of the example below.
%%
\begin{CCSXML}
<ccs2012>
   <concept>
       <concept_id>10003120.10003121.10011748</concept_id>
       <concept_desc>Human-centered computing~Empirical studies in HCI</concept_desc>
       <concept_significance>500</concept_significance>
       </concept>
   <concept>
       <concept_id>10010405.10010455.10010459</concept_id>
       <concept_desc>Applied computing~Psychology</concept_desc>
       <concept_significance>300</concept_significance>
       </concept>
   <concept>
       <concept_id>10003120.10003121.10003122</concept_id>
       <concept_desc>Human-centered computing~HCI design and evaluation methods</concept_desc>
       <concept_significance>500</concept_significance>
       </concept>
 </ccs2012>
\end{CCSXML}

\ccsdesc[500]{Human-centered computing~Empirical studies in HCI}
\ccsdesc[300]{Applied computing~Psychology}
\ccsdesc[500]{Human-centered computing~HCI design and evaluation methods}
%%
%% Keywords. The author(s) should pick words that accurately describe
%% the work being presented. Separate the keywords with commas.
\keywords{Social Stigma, Depression, Causal Knowledge Graph, AI-assisted Coding, Chatbot, Large Language Model}



%%
%% This command processes the author and affiliation and title
%% information and builds the first part of the formatted document.
\maketitle


\section{Introduction}
\label{sec:intro}
% Image editing methods in diffusion models depend on user-defined control directions - users can unlock their creativity using these methods by specifying the desired manipulation through prompts~\cite{gandikota2023concept}, reference images~\cite{ruiz2022dreambooth, kumari2022customdiffusion, gal2022image, chen2024trainingfreeregionalpromptingdiffusion}, or attribute vectors~\cite{parmar2023zero,hertz2022prompt}. In this work, we ask a fundamentally different question: \emph{Can we automatically discover the underlying visual structure of a concept within diffusion model's knowledge?} %Rather than requiring user-specified controls, we aim to decompose the model's internal knowledge into meaningful directions.

% This question touches on a fundamental limitation in how we interact with diffusion models. Current control methods ~\cite{zhang2023addingconditionalcontroltexttoimage, gandikota2023concept, ye2023ipadaptertextcompatibleimage,ye2023ipadaptertextcompatibleimage, hertz2024stylealignedimagegeneration, li2023photomaker, shi2024instantbooth, chen2024trainingfreeregionalpromptingdiffusion} require users to specify their desired manipulations in advance, limiting interactive creativity. This contrasts with natural human artistic workflows, where creators dynamically explore creative ideas while jointly refining them toward meaningful artistic outcomes~\cite{hoffmann2016modeling}. This synergy between specification and exploration is not new to generative models. Early GAN architectures naturally developed disentangled latent spaces that enabled continuous\cite{harkonen2020ganspace,radford2015unsupervised, wu2021stylespace, shen2020interfacegan}, compositional control over generated images. Users could explore these spaces to discover interesting variations that would be difficult to describe in words~\cite{wu2021stylespace}, then combine them to achieve their creative goals~\cite{grabe2022towards}. 


% While diffusion models have largely superseded GANs in conditional image synthesis~\cite{dhariwal2021diffusion},  their underlying structure remains less understood. Diffusion models achieve remarkable diversity through high-dimensional latents, unlike GANs' compact latent spaces.  With a single prompt, diffusion models can generate radically different variations through different random initializations of input noise. We ask - Is it possible to discover interpretable structure within this vast space of variations?

Text-to-image diffusion models are capable of generating remarkable visual variations from a single prompt through different random initializations. However, this vast creative potential remains largely opaque to users---while we can generate diverse images, we lack understanding of the underlying structure of these variations. This presents a fundamental challenge: how can we discover and expose the latent visual capabilities encoded within these models?

\let\thefootnote\relax \footnote{$^{*}$Correspondence to \texttt{gandikota.ro@northeastern.edu}}

The challenge touches on a key limitation in how we interact with diffusion models today. Current control methods require users to explicitly specify their desired edits in advance through prompts~\cite{gandikota2023concept}, reference images~\cite{zhang2023addingconditionalcontroltexttoimage, chen2024trainingfreeregionalpromptingdiffusion, ruiz2022dreambooth,kumari2022customdiffusion, Ryu_lora, hu2021lora}, or attribute vectors~\cite{ye2023ipadaptertextcompatibleimage, hertz2024stylealignedimagegeneration, li2023photomaker, shi2024instantbooth,parmar2023zero,hertz2022prompt}. That contrasts sharply with natural human creative workflows, where artists dynamically explore creative ideas and jointly refine them toward meaningful artistic outcomes~\cite{hoffmann2016modeling}. The need for pre-specified controls creates a barrier between users and the full creative potential of these models.

Interestingly, earlier generative models like GANs~\cite{gans,karras2019style,brock2018large} naturally developed more interpretable internal structures. Their compact latent spaces often exhibited emergent disentanglement~\cite{harkonen2020ganspace,radford2015unsupervised, wu2021stylespace, shen2020interfacegan}, enabling continuous and compositional control over generated images. Users could explore these spaces to discover interesting variations that would be difficult to describe in words~\cite{wu2021stylespace}, then combine them to achieve their creative goals~\cite{grabe2022towards}.

Diffusion models have largely superseded GANs in conditional image synthesis~\cite{dhariwal2021diffusion}, achieving greater diversity through much higher-dimensional latents. And yet an understanding of the underlying structure of these larger latent spaces has remained elusive. In this work, we ask a fundamental question: \emph{Can we automatically discover the visual structure within a diffusion model's knowledge of a concept?} Rather than requiring user-specified controls, we aim to decompose the model's internal representations into expressive directions that users can explore and combine.

To address these needs, we present \textbf{SliderSpace}, a framework that brings systematic explorability to diffusion models. Given just a text prompt, SliderSpace discovers a canonical set of meaningful, diverse, and controllable directions within the model's knowledge of that concept. Each direction is implemented as a low-rank adapter~\cite{hu2021lora} that can be scaled and composed with others, allowing users to explore and smoothly combine different aspects of variation, as shown in Figure~\ref{fig:intro}.

We ground SliderSpace discovery in three key requirements for meaningful decomposition of a diffusion model's visual manifold: 
\begin{enumerate}
    \item \textbf{Unsupervised Discovery:} The decomposition process should emerge from the intrinsic structure of the model's learned representation, rather than being guided by predefined attributes. This ensures we capture the true topology of the model's knowledge space rather than projecting our assumptions onto it.
    
    \item \textbf{Semantic Orthogonality:} Each discovered control must represent a distinct semantic direction. This is enforced in a semantic feature space, like CLIP, where every slider has an orthogonal effect in embeddings. This prevents discovering multiple controls that create similar semantic effects, making the system more efficient and easier.
    
    \item \textbf{Distribution Consistency:} Directions must induce consistent transformations across both random seeds and prompt variations. 
\end{enumerate}

These requirements naturally lead to our proposed framework, which we formalize in Section~\ref{sec:method}. As we show in our experiments, SliderSpace is architecture-agnostic, working with both conventional U-Net based models like Stable Diffusion~\cite{rombach2022high, rombach2022sd20, podell2023sdxl, turbo, dmd} and recent transformer-based architectures like Flux~\cite{flux}.

We demonstrate the expressiveness of SliderSpace through three applications: First, we show how SliderSpace can decompose high-level concepts into diverse and expressive components, revealing the natural axes of variation in the model's understanding. Second, we explore artistic style variation, where SliderSpace discovers directions that match or exceed the diversity of manually curated artist lists while being judged more useful by human evaluators. Finally, we show how SliderSpace can help reverse the mode collapse commonly observed in distilled diffusion models, restoring diversity while maintaining generation speed.

Beyond providing practical creative control, SliderSpace opens new avenues for understanding and utilizing the latent capabilities of diffusion models. By mapping these models' visual potential into intuitive, composable directions, we take a step toward making their creative possibilities more accessible and interpretable to users.

% Image editing methods in diffusion models unlock the creativity of users. In this work we ask an alternate question: \emph{Can we organize and expose what of the diffusion model is already capable of?}.
% Existing methods for controlling image generation typically require users to manually specify edit directions for desired changes. This process is time-consuming, requires technical expertise, and limits the spontaneity of the creative process. For instance, if a user wants to adjust the smile of a generated person, they must explicitly request this edit, often through imprecise prompt engineering or model fine-tuning. This approach of predefined controls or manual specifications restricts users from fully exploring the latent capabilities of the model. There may be interesting stylistic variations or attributes that the model can generate, but users have no easy way to discover or utilize these.

% Natural visual disentanglement was an emergent property in the latent space of Generative Adversarial Models (GANs) \cite{harkonen2020ganspace,radford2015unsupervised, wu2021stylespace, shen2020interfacegan}. In particular, it has been observed that StyleGAN~\cite{karras2019style} stylespace neurons offer detailed control over many meaningful aspects of images that would be difficult to describe in words~\cite{wu2021stylespace}. However, diffusion models do not share such a compact latent space~\cite{park2023unsupervised}; and efforts to uncover such a space in the semantic embeddings of the text conditioning have met with limited success \nik{Nick - is there a specific citation you were thinking about?}.

% In this work we introduce \textbf{SliderSpace}, which takes a step towards uncovering an analogous low dimensional representation of diffusion models' visual breadth; in essence treating the diffusion model as many generators sharing parameters, where a particular generator is defined by a specific prompt. For a given prompt we sample many random seeds (and optionally prompt expansions using an LLM), generate the corresponding images, and apply an off the shelf feature extractor (in this work CLIP, but our method can be applied to any differentiable feature extractor). We use PCA to analyze these features, and for each of the leading $k$ principal components we train a LoRA \cite{} which causes the diffusion model to produces images which increase the feature magnitude along that component when passed back through the same feature extractor. This leads to a 'Slider' for each principal component, because each LoRA can be scaled and applied to the original diffusion model, continuously varying those visual features in the generated results (as measured, in our case, by CLIP).

% There are many other works that enhance the controllability of diffusion models. One common approach is enabling users to add spatial constraints to a generation either manually, or via a reference image \cite{zhang2023addingconditionalcontroltexttoimage, chen2024trainingfreeregionalpromptingdiffusion}, a second is leveraging more abstract embeddings (e.g. identity, style) extracted from a reference image \cite{ye2023ipadaptertextcompatibleimage, hertz2024stylealignedimagegeneration, li2023photomaker, shi2024instantbooth}, a third is finetuning a foundation model to better generate a concept important to the user \cite{ruiz2022dreambooth, kumari2022customdiffusion, Ryu_lora, hu2021lora}, and a fourth (most relevant to this work) is finding low-rank adaptors of the model based on a prompt or small training set which can be scaled to provide continous control over one aspect of generated image (e.g. night vs day, basic vs luxury, etc.) \cite{gandikota2023concept}. SliderSpace is complementary to all of these methods and offers something distinct. All of the other methods we are aware require the user (and / or model designer) to know in advance what type of control they want. In contrast SliderSpace assists users in discovering and controlling hidden capabilities present in the diffusion model's distribution of possible generations.

%We propose that truly intuitive creative control in a text-to-image model should meet three key criteria: \emph{discoverability}, \emph{intuitiveness}, and \emph{specificity}. The model should reveal controllable attributes that may not be immediately obvious, offer controls that are easy to understand and manipulate, and ensure each control affects a distinct attribute of the generated image.

% We demonstrate the utility and power of SliderSpace using three applications built on top of SDXL-DMD \cite{dmd}, because its fast generation speed lends itself well to the continuous control offered by SliderSpace.

% First, we study concept decomposition (Section \ref{sec:concept_exp}), where we learn sliders for a specific concept (e.g. 'monster', 'waterfall', 'car'). Through quantitative metrics of diversity and text alignment we demonstrate that the learned sliders dramatically boost the diversity of generations when randomly applied without harming text alignment; we also ask humans to qualitatively judge these results in a user study where they find the SliderSpace results to be more 'Diverse', 'Useful', and 'Creative' than our baselines.

% Second, we attempt to compare the automatic discoveries of SliderSpace to a large scale manual study of artistic styles (Section \ref{sec:art_exp}), open-sourced by ParrotZone \cite{parrotzone}. In this study SDXL was prompted with over 4300 artist names,  and based on visual inspection the cases of successful stylistic mimicry recorded. Quantitatively SliderSpace more closely matches the distribution of artistic variation discovered by ParrotZone than other baselines, and in our user studies was judged to be significantly more 'Diverse' and 'Useful' than the baselines. To our surprise humans even judged SliderSpace results to be slightly more 'Diverse' than the results generated by the manually discovered artist names of \cite{parrotzone}.

% Third, we attempt to use SliderSpace to reverse the mode collapse commonly observed in distilled few-step diffusion models relative to the original teacher model (Section \ref{sec:diverse_exp}). We quantitatively demonstrate that applying SliderSpace to SDXL-DMD leads to more closely matching the distribution of images by the original teacher, SDXL.

%Through extensive experiments on various state-of-the-art text-to-image models, we demonstrate that SliderSpace significantly enhances user control and creative expression in AI-assisted image generation tasks. Our method enables a range of applications, including concept decomposition and control, diversity improvement in generated images, customization dissection and edits, and the exploration of artistic styles inherent in the model.

% SliderSpace goes beyond providing a practical tool for enhanced creative control. By mapping the visual potential of diffusion models it can open new avenues for generative creativity and deepens our understanding of each model's hidden potential.
\section{Related Work}
\label{sec:related_work}

The original investigation \cite{gibson1979ecological} on the relationship between visual perception and human action defines \emph{affordance} as the opportunities for interaction with the surrounding environment. Behavioral studies on regular and cognitively impaired persons have shown evidence that perception results in both visual and motor signals in the human brain. An extended study \cite{anderson2002attentional} shows that visual attention to the spatial characteristics of the perceived objects initiates automatic motor signals for different actions. In computer vision, human affordance learning involves novel pose prediction such that the estimated pose represents a valid human action within the scene context. The task is fundamental to many problems requiring robust semantic reasoning about the environment, such as human motion synthesis \cite{wang2021scene} and scene-aware human pose generation \cite{wang2017binge, roy2016multi, zhang2022inpaint, yao2023scene}.

Earlier methods of affordance learning have explored knowledge mining \cite{zhu2014reasoning} and multimodal feature cues \cite{roy2016multi} to address the problem. In \cite{zhu2014reasoning}, the authors use a Markov Logic Network for constructing a knowledge base by extracting several object attributes from different image and metadata sources, which can perform various downstream visual inference tasks without any additional classifier, including zero-shot affordance prediction. In \cite{roy2016multi}, the authors use depth map, surface normals, and segmentation map as multimodal cues to train a multi-scale convolutional neural network (CNN) for scene-level semantic label assignment associated with specific human actions. In \cite{do2018affordancenet}, the authors design a multi-branch end-to-end CNN with two separate pathways for object detection and affordance label assignment to achieve high real-time inference throughput. Researchers \cite{chuang2018learning} have also explored socially imposed constraints for affordance learning. In \cite{chuang2018learning}, the authors propose a graph neural network (GNN) to propagate contextual scene information from egocentric views for action-object affordance reasoning.

Probabilistic modeling of scene-aware human motion generation also involves semantic reasoning of human interaction with the environment. Initial works on human motion synthesis have taken different architectural approaches, such as sequence-to-sequence models \cite{barsoum2018hp}, generative adversarial networks (GAN) \cite{barsoum2018hp, cai2018deep, yang2018pose}, graph convolutional networks (GCN) \cite{yan2019convolutional}, and variational autoencoders (VAE) \cite{guo2020action2motion}. However, these methods have mostly ignored the role of environmental semantics. Due to potential uncertainty in human motion, in a recent approach \cite{wang2021scene}, the authors address such motion synthesis with a GAN conditioned on scene attributes and motion trajectory to predict probable body pose dynamics.

One key challenge of human affordance generation in 2D scenes is the lack of large-scale datasets with rich pose annotations. In \cite{wang2017binge}, the authors compile the only public dataset of annotated human body poses in complex 2D indoor scenes by extracting frames from sitcom videos. Aiming to generate a contextually valid human affordance at a user-defined location, the authors propose sampling the scale and deformation parameters for an existing human pose template using a VAE conditioned on the localized image patches as scene context. In \cite{zhang2022inpaint}, the authors introduce a two-stage GAN architecture for achieving a similar goal by estimating the affine bounding box parameters to localize a probable human in the scene and then generating a potential body pose at that location. The method uses the input scene, corresponding depth, and segmentation maps as semantic guidance. In \cite{yao2023scene}, the authors propose a transformer-based approach with knowledge distillation for generating human affordances in 2D indoor scenes.


% introduce PDDL domains
% why Gripper env as testing context
% motivation: comparing classical vs LLM planners
% - classical: PDDL solver fast-downward
% - LLM: gpt-4o
% explanation and refinement are two distinguishing features of LLM planners
% - how we demonstrate explanation and refinement in the study
We evaluate user trust in two planners over a set of planning problems and study the potential factors influencing user trust in the planners. In particular, we compare a language-model-based planner, denoted as an \emph{LLM Planner}, with a traditional graph-search-based planner, denoted as a \emph{PDDL Solver}. The PDDL Solver uses Fast Downwards \cite{fastdownward} as its underlying model, processing planning problems described in PDDL to generate an optimal solution. In comparison, the LLM Planner employs GPT-4o to interpret the planning problem and extract a solution generated by the language model. Unlike the PDDL Solver, the LLM Planner can reason through the planning problem, explain its proposed solution, and iteratively refine the solution based on external feedback. This study investigates how the correctness of solutions, the quality of explanations, and the refinement process influence user trust.

\subsection{Planning Problem}
% \begin{wrapfigure}{r}{0.4\textwidth}
% % \begin{figure}[t]
%     \centering
%     \includegraphics[width=\linewidth]{figures/problem-example.pdf}
%     \caption{A running example of a planning problem in our study.}
%     \Description{Planning Problem Example}
%     \label{fig: problem-example}
% % \end{figure}
% \end{wrapfigure}

We describe each planning problem in the \emph{Planning Domain Definition Language (PDDL)} and propose two planners to generate plans that solve the problem. We select the \emph{gripper} planning problems from the International Planning Competition \cite{IPC} for plan generation and evaluation. In a gripper planning problem, a robot moves balls between a set of rooms using two grippers. The objective is to create a plan for the robot to move the balls to the target rooms we defined. We present a few running examples of the gripper problem in Figure \ref{fig: correctness}.

A planning problem consists of a \emph{planning domain} and a \emph{problem description}, expressed in PDDL. 

\paragraph{Planning Domain}
A planning domain refers to the universal aspects of a problem that remains consistent across different instances of the problem. In particular, it defines the types of objects, predicates, and actions that exist in the planning problem. We present an example of the gripper problem in Appendix \ref{app: grippers}.

\paragraph{Problem Description} A problem description specifies the particular instance of a planning task within a given domain. It includes the planning domain to which it pertains, a set of objects, the initial state of these objects, and the goal state to be achieved.

\paragraph{Plan}
A plan is a sequence of actions with specific input parameters. Recall that an action corresponds to a state transition. If a plan (a sequence of actions) transits from the initial state to the goal state defined by a problem, then we consider the plan to be \emph{correct}. If a plan does not transit to the goal state or there exists an action violating its precondition, then the plan is \emph{wrong}.

\begin{figure}[t]
    \centering
    \includegraphics[width=0.8\linewidth]{figures/correct.jpeg}
    \caption{Examples where LLM Planner correctly generates a plan for the gripper planning problem.}
    \Description{Planning Problem Correctness}
    \label{fig: correct}
\end{figure}

\subsection{PDDL Solver}
The PDDL Solver takes the planning domain and the problem description as inputs and then generates a plan described in PDDL. 
% It generates a plan in the following format:
% \vspace{4pt}
% \begin{lstlisting}[language=completion]
% (move robot1 room1 room3)
% (pick robot1 ball2 room3 rgripper1)
% (move robot1 room3 room2) ......
% \end{lstlisting}
Next, we convert the generated plan into natural language for user studies following the procedure in \cite{seipp-et-al-zenodo2022} and display it to users. We present an example in Figure \ref{fig: correct}.

The PDDL Solver applies a graph search algorithm to find a path (i.e., a list of transitions) from the initial state to the goal state. It either generates a \emph{correct} plan---defined as the shortest path between the initial and goal states---or returns a signal indicating that no solution exists for the given problem.

\subsection{LLM Planner}

The LLM Planner addresses planning problems by querying a large language model. In particular, it transmits the planning domain and problem description to the language model using a structured prompt format. The planner then retrieves a natural language plan from the language model. We use GPT-4o as the language model for the planner. To ensure the output adheres to the desired format, we include a few in-context examples within the prompts.

A language model solves a planning problem by interpreting the domain and problem descriptions, simulating state transitions, and generating a sequence of actions to achieve the goal. While effective for reasoning and plan generation, language models may struggle with large state spaces. Unlike the PDDL Solver, the LLM Planner may generate \emph{incorrect} plans that violate the problem specifications (e.g., preconditions of actions) or fail to achieve the goal.

\subsection{Explanation and Refinement}
Alongside the generated plans, we offer detailed explanations of all the plans and revisions of any incorrect plans. This study examines how these explanations and refinements influence human trust in the two planners.

\paragraph{LLM Planner with Explanation (LLM+Expl)}
For each generated plan, we manually provide a natural language explanation. This explanation includes an assessment of the plan’s correctness, identification of any violations of action preconditions, and an analysis of inconsistencies between the final state achieved and the intended goal state. We present examples of explanations in Figure \ref{fig: explain} in Appendix.

In particular, if a plan is correct, the explanation is simply ``the plan successfully satisfies the goal conditions.'' 
If a plan is incorrect, we identify the underlying cause as either a violation of action preconditions or a failure to achieve the goal state. In cases involving precondition violations, we specify the action responsible for the issue. For example, consider the action ``robot moves from room 1 to room 2,'' but the robot is initially located in room 3. This scenario constitutes a violation of the precondition for the ``move'' action. In the latter case, we describe the differences between the final state achieved and the intended goal state, e.g., ``fail to move ball 2 to room 2.''

% \begin{wrapfigure}{r}{0.5\textwidth}
%     \centering
%     \includegraphics[width=0.98\linewidth]{figures/refine.jpeg}
%     \includegraphics[width=0.98\linewidth]{figures/refine-correct.jpeg}
%     \includegraphics[width=0.98\linewidth]{figures/refine-wrong.jpeg}
%     \caption{Plan refinement by the LLM Planner. The top row presents two choices of plan refinement (where the refinement starts). The second and third row shows the refinement outcomes of the two choices, where the second row shows a correctly refined plan and the third row shows an incorrect plan.}
%     \Description{Refinement}
%     \label{fig: refine}
% \end{wrapfigure}

\paragraph{LLM Planner with Refinement (LLM+Refine)}
Note that a plan generated by the LLM Planner could be incorrect. Therefore, we offer a prompting mechanism for the LLM Planner to refine the generated plan according to the user feedback. The mechanism works as follows:

1. Request the user to indicate the step number of the first action in the plan that is incorrect, such as the step where an action’s precondition is violated. We present a sample user interface on the left of Figure \ref{fig: refine} in Appendix.

2. Send the planning domain, problem description, and the original plan to the language model. Then, query the model to rewrite the subsequent steps starting from the user-specified step number. We present a sample input prompt in Figure \ref{fig: refine-prompt} in the Appendix.

3. Replace the original plan with the newly refined plan and display it to the user.

This mechanism allows users to interact with the language model to refine the plan. It enables the language model to focus on a subset of steps, facilitating a deeper interpretation of the incorrect component. However, the correctness of the refined plan is not guaranteed. Figure \ref{fig: refine} in the Appendix shows an example of a correctly refined plan and an incorrectly refined plan.

\section{Result}
For the evaluation of MicroViT, the ImageNet-1K dataset \cite{russakovsky2015imagenet} comprising 1.28 million training images and 50,000 validation images over 1,000 categories was employed. Following the DeiT training method \cite{touvron2021training}, models were trained for 300 epochs at a 224×224 resolution with an initial learning rate of 0.004, utilizing various data augmentations. The AdamW optimizer \cite{loshchilov2017decoupled} was used with a batch size of 512 across three A6000 GPUs.

We assessed model throughput in various computation environments, including a GPU (RTX-3090), a CPU (Intel i5-13500), and specifically the Jetson Orin Nano edge device. For throughput, the GPU and CPU had a batch size of 256, whereas the edge device used a batch size of 64 with ONNX Runtime. To enhance performance during inference, BN layers were fused with adjacent layers when possible. On the Jetson Orin Nano, we also examined power and energy usage during latency tests with 1000 images at a consistent resolution.

We further evaluate MicroViT on object detection on the COCO dataset \cite{lin2014microsoft} utilizing RetinaNet \cite{ross2017focal} and conduct training for 12 epochs (1$\times$ schedule), adhering to the configuration used by \cite{liu2023efficientvit} in mmdetection \cite{mmdetection}. In the object detection experiments, we employ AdamW \cite{loshchilov2017decoupled} with a batch size of 16, a learning rate of $1\times10^{-3}$, and a weight decay rate of 0.025. 


\begin{table}[!ht]
\centering
\caption{Comparison of All MicroViT Variant with SOTA on ImageNet-1K Dataset. Res, Par and FLPs denotes as input resolution, parameters and Floating Operation. GPU and CPU denotes a inference throughput (img/s) in device respectively.}
\begin{tabular}{ m{2.6cm}|c|c|c|c|>{\centering}m{0.5cm}|c }
\hline
Model   & Res  & Par & FLPs &  GPU  & CPU &   Top-1      \\ \hline
% Fasternet-T1\cite{liu2022convnet}              & 224 & 7.6  & 0.85 & \textbf{7037}  & 149.3  &   & 76.2      \\
MobileNetV2-1.0\cite{sandler2018mobilenetv2}& 224 & 3.5 & 0.314 & 4527 & 82 & 72.0   \\
MobileViT-XXS\cite{mehta2021mobilevit}  & 256 & 1.3 & 0.261 & 3218 & 99  & 69.0   \\
MobileViTV2-0.5\cite{mehta2022separable}  & 256 & 1.4 & 0.480 & 3885 & 68  & 70.2   \\
EdgeNeXt-XXS\cite{maaz2022edgenext}  & 256 & 1.3 & 0.261 & 3975 & 245  & 71.2   \\
Fasternet-T0\cite{chen2023run}  & 224 & 3.9 & 0.340 & 11775 & 311 & 71.9   \\
SHViT-S1\cite{yun2024shvit} & 224  & 6.3 & 0.241 & 15280 & 475 & 72.8 \\
\rowcolor{gray!30}
\textbf{MicroViT-S1}                    & 224  & 6.4 & 0.231 & 17466 & 552 & 72.6 \\ \hline 
EFormerV2-S0\cite{li2023rethinking}& 224  & 3.6 & 0.407 & 1191 & 91 & 73.7 \\ 
EdgeNeXt-XS\cite{maaz2022edgenext}   & 256 & 2.3 & 0.536 & 2935 & 139  & 75.0   \\
EfficientViT-M4\cite{liu2023efficientvit} & 224 & 8.8 & 0.303 & 10093 & 379 & 74.3   \\
MobileViT-XS\cite{mehta2021mobilevit}   & 256 & 2.3 & 0.935 & 1740 & 43 & 74.8   \\
MobileNetV3-L\cite{howard2019searching}& 224 & 3.5 & 0.314 & 4527 & 82 & 75.2  \\
SHViT-S2\cite{yun2024shvit}            & 224  & 11.5 & 0.366 & 12007 & 367 & 75.2 \\
\rowcolor{gray!30}
\textbf{MicroViT-S2}                    & 224 & 10.0 & 0.345 & 14154 & 435 & 74.6 \\ \hline
FastViT-T8\cite{vasu2023fastvit}   & 256 & 4.0 & 0.687 & 3719 & 83 & 76.2   \\
Fasternet-T1\cite{chen2023run}  & 224 & 7.6 & 0.851 & 7151 & 130 & 76.2   \\
EfficientViT-M5\cite{liu2023efficientvit}& 224 & 12.5 & 0.526 & 6807 & 233 & 77.1   \\
SHViT-S3\cite{yun2024shvit}            & 224  & 14.1 & 0.601 & 8180 & 224 & 77.8 \\
% \hline 
\rowcolor{gray!30}
\textbf{MicroViT-S3}                   & 224 & 16.7 & 0.580 & 9288 & 232 & 77.1 \\ \hline 
    \end{tabular}
    \label{tab:imgnet-result}
\end{table}
\subsection{ImageNet-1K Classification Result}
Table \ref{tab:imgnet-result} presents a comparison of various MicroViT variants with state-of-the-art (SOTA) models on the ImageNet-1K dataset. The evaluation focuses on models' computational efficiency and accuracy, highlighting the trade-offs between resource consumption and performance. 

MicroViT-S1 demonstrated superior performance compared to traditional CNN models, surpassing MobileNetV2-1.0\cite{sandler2018mobilenetv2} and Fasternet-T0\cite{mehta2022separable}, with a $3.6 \times$ faster in GPU and $6.7 \times$ in CPU throughput, while maintaining an accuracy advantage of 0.8 over MobileNetV2-1.0. Additionally, MicroViT-S2 outperformed mobile transformers like EfficientFormerV2-S0\cite{li2023rethinking} and EfficientViT-M4\cite{liu2023efficientvit}, achieving $0.3\%$ better accuracy with similar efficiency metrics. Across the MicroViT models, CPU throughput is robust, notably with MicroViT-S1 achieving 552 img/s, which is $8 \times$ faster than several EfficientViT variants, illustrating MicroViT's adaptability to both high-end GPU and CPU settings.


Table \ref{tab:edge-result} presents the performance of MicroViT variants against various SOTA models on the Edge device using ONNX. MicroViT-S1's throughput reaches 773 img/s, efficiently managing large image volumes on the Jetson Orin Nano. This surpasses several SOTA models like MobileViT-XS\cite{mehta2021mobilevit} and EfficientFormer-V2-S0\cite{li2023rethinking}, making MicroViT-S1 optimal for rapid image processing applications. Furthermore, it has a 9.1 ms latency, outperforming MobileNetV2-1.0\cite{sandler2018mobilenetv2} and EdgeNeXt-XS\cite{maaz2022edgenext}, supporting real-time use. It consumes 2147 Joules, achieving high energy efficiency with $\eta=3.7$. Likewise, MicroViT-S2 and MicroViT-S3 balance throughput and energy use, maintaining accuracy, thus ideal for resource-constrained edge devices with superior power efficiency over other lightweight vision transformers.
\begin{table}[!ht]
\centering
\caption{Comparison of All MicroViT Variant and SOTA on ImageNet-1K Dataset with NVIDIA Jetson Orin Nano Edge Device using ONNX format.}
\begin{tabular}{ m{2.6cm}|>{\centering}m{0.5cm}|>{\centering}m{0.6cm}|>{\centering}m{0.9cm}|>{\centering}m{0.8cm}|c }
\hline
\multirow{2}{*}{Model} & Thg & Lat & Avg Pow & Energy  & $\eta$ \\ 
         & img/s & (ms) & (W) & (Joule) & \% / J \\ \hline
MobileNetV2-1.0\cite{sandler2018mobilenetv2} & 234 & 6.7 & 3549 & 23.9 & 3.01 \\
MobileViT-XXS\cite{mehta2021mobilevit}  & 184 & 9.6 & 3428 & 32.4 & 2.13   \\
MobileViTV2-0.5\cite{mehta2022separable}  & 208 & 10.9 & 2887 & 31.6 & 2.22 \\
EdgeNeXt-XXS\cite{maaz2022edgenext}     & 257 & 8.1 & 2805 & 22.6 &  3.15  \\
Fasternet-T0\cite{chen2023run}   & 675 & 8.4 & 2419 & 20.4 &  3.52 \\
SHViT-S1 \cite{yun2024shvit}       & 813  & 12.6 & 2069 & 26.0 & 2.80  \\
\rowcolor{gray!30}
\textbf{MicroViT-S1}                & 773  & 9.1 & 2147 & 19.6 & 3.7 \\ \hline 
EFormerV2-S0\cite{li2023rethinking}  & 257  & 10.9 & 2847 & 31.1 & 2.37   \\ 
EdgeNeXt-XS\cite{maaz2022edgenext}      & 168 & 11.1 & 3031 & 33.6 &  2.32   \\
EfficientViT-M4\cite{liu2023efficientvit} & 587 & 18.0 & 2019 & 36.3 &  2.05   \\
MobileViT-XS\cite{mehta2021mobilevit}   & 96.6 & 14.3 & 3730 & 53.3 & 1.40    \\
MobileNetV3-L\cite{howard2019searching} & 310 & 8.3 & 2743 & 22.6 &  3.33 \\
SHViT-S2 \cite{yun2024shvit}            & 598  & 12.7 & 2306 & 29.4 & 2.56  \\
\rowcolor{gray!30}
\textbf{MicroViT-S2}                      & 567 & 9.9 & 2481 & 24.3 & 3.07 \\ \hline
FastViT-T8\cite{vasu2023fastvit}          & 176 & 8.6 & 3622 & 30.8 &  2.47  \\
Fasternet-T1\cite{chen2023run}     & 421 & 8.8 & 2860 & 25.3 &  3.01  \\
EfficientViT-M5\cite{liu2023efficientvit} & 409 & 21.0 & 2180 & 45.9 & 1.68    \\
SHViT-S3\cite{yun2024shvit}               & 425  & 14.6 & 2500 & 36.6 & 2.13  \\
% \hline 
\rowcolor{gray!30}
\textbf{MicroViT-S3}                    & 398 & 10.9 & 2609 & 28.6 & 2.69  \\ \hline 
    \end{tabular}
    \label{tab:edge-result}
\end{table}

\subsection{Object Detection Result}
We compare MicroViT-3 with efficient models \cite{sandler2018mobilenetv2, howard2019searching, liu2023efficientvit} on the COCO \cite{lin2014microsoft} object detection task, and present the results in Table \ref{tab:obj} Specifically, MicroViT-3 surpasses MobileNetV2 \cite{sandler2018mobilenetv2} by
7.7\% AP with comparable Flops. Compared to the EfficientViT-M4, our MicroViT-3 uses 46.8\% fewer Flops while achieving 3.3\% higher AP, demonstrating its capacity and generalization ability in different vision tasks.

\begin{table}[!ht]
\centering
\caption{Object detection on COCO val2017 with RetinaNet. $AP^b$ denote bounding box average precision. The FLOPs (G) are measured at resolution 1280 $\times$ 800.
}
\begin{tabular}{ m{2.4cm}|>{\centering}m{0.8cm}|>{\centering}m{0.8cm}|>{\centering}m{0.8cm}|>{\centering}m{0.8cm}|c}
\hline
Backbone & $AP^{b}$  & $AP^{b}_{50}$ & $AP^{b}_{75}$ & Par & FLPs  \\ \hline
MobileNetV2\cite{sandler2018mobilenetv2} & 28.3& 46.7& 29.3& 3.4& 300 \\
MobileNetV3\cite{howard2019searching} & 29.9& 49.3& 30.8& 5.4 & 217   \\ 
EfficientViT-M4\cite{liu2023efficientvit}&32.7 &52.2 &34.1 & 8.8 & 299  \\
\rowcolor{gray!30}
\textbf{MicroViT-S3} & 36.0 & 56.6 & 38.2 & 26.7 & 159    \\ \hline

\end{tabular}
\label{tab:obj}
\end{table}

\subsection{Ablation Study}

Table \ref{tab:abl-result} assesses the effects of three ablations on the MicroViT-S2 model as the baseline. The first ablation, low resolution SA with SR=2, slightly increases the parameter count from 11.0M to 11.1M, with GPU throughput remaining nearly unchanged from 12009 to 12029 images per second. However, it increases latency from 9.9 ms to 10.7 ms and decreases Top-1 accuracy to 72.5\%, resulting in a minor drop in efficiency ($\eta$) from 3.0 to 2.8. This indicates a small trade-off between accuracy and computational cost due to the architectural modification.

The Next ablation, without group convolutions, increases the model's complexity significantly to 20.5M parameters, with higher GFLOPs. This results in lower throughput and efficiency ($\eta=1.8$) but achieves the highest accuracy. However, this variant consumes more energy, making it ideal for use cases where accuracy is prioritized over efficiency.

The ablation study shows that even though spatial reduction can increase throughput, the latency is increased, resulting in an efficiency drop. Group convolution successfully increases the efficiency in ESHA compared to other attention models, such as vanilla attention in MobileViT, maintaining lower complexity and energy usage.

\begin{table}[!ht]
\centering
\caption{Ablation study of MicroViT  on ImageNet-1K Dataset. The Baseline model is MicroViT-S2.}
\begin{tabular}{ m{1.6cm}|c|c|c }
\hline
\textbf{Ablation} & Baseline & Low Res Attn & W/O Group  \\ \hline
\textbf{Param}      & 11.0  & 11.1  & 20.5   \\  
\textbf{GFLOPs}     & 0.343 & 0.344 & 0.4977 \\ 
\textbf{GPU}        & 14154 & 14179 & 13511  \\
\textbf{Throughput} & 567   & 570   & 503    \\
\textbf{Latency}    & 9.9   & 10.7  & 12.5   \\
\textbf{Avg Pow}    & 2481  & 2512  & 3288   \\
\textbf{Energy}     & 24.29  & 25.8  & 41.2   \\
\textbf{Top-1}      & 72.7  & 72.5  & 73.2   \\
\textbf{Efficiency ($\eta$)}& 3.07  & 2.8 & 1.8 \\
\hline


    \end{tabular}
    \label{tab:abl-result}
\end{table}


\section{Discussion}
\label{section:discussion}


\subsection{Practical Implications for Feedforward Prompting}

Of course, prompting an LLM continuously before the user submits their prompt is significantly most costly over submitting the prompt just once, once the user is ready.

% But user might not be ready, and the cognitive costs is pretty heavy.


\subsection{}


% Does this work well with Chain of Thought actually?
% Maybe this approach will actually incentivize self-prompt-chaining???
% What are the implications of this?


% A benefit of this is certainly more transparency in the LLM
% LLM is so flexible that adding this kind of structure is still okay for the LLM



% What's more costly, entering a prompt, then responding and saying, no i want this, or typing a prompt, and tuning the prompt/expected output to reduce message exchanges?

% Learning to become a better prompter. One is by trial and error experience. Perhaps another is through this feedforward that tells you what you might be able to anticipate.


\begin{acks}

% This work is supported by the Ministry of Education, Singapore (A-8002610-00-00) and the National University of Singapore Start-up Grant Award No. R-124-000-128-133.
% This research work is partially supported by NUS IT’s Research Computing group (NUSREC-HPC-00001) and AWS.
% We thank all reviewers' comments and suggestions to help polish this paper.
Our research has been supported by the Ministry of Education, Singapore (A-8002610-00-00) and the National University of Singapore Start-up Grant Award (R-124-000-128-133). 
This research work is partially supported by the NUS IT's Research Computing group (NUSREC-HPC-00001) and AWS. 
We are grateful to all the reviewers for their valuable comments and suggestions that helped improve this paper, and to all the participants whose time and efforts made this research possible.
\end{acks}


\bibliographystyle{ACM-Reference-Format}
\bibliography{sample-base}

%%
%% If your work has an appendix, this is the place to put it.

%%
%% The acknowledgments section is defined using the "acks" environment
%% (and NOT an unnumbered section). This ensures the proper
%% identification of the section in the article metadata, and the
%% consistent spelling of the heading.



%%
%% The next two lines define the bibliography style to be used, and
%% the bibliography file.


\end{document}
\endinput
%%
%% End of file `sample-acmsmall.tex'.
