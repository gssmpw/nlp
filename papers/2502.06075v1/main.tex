%%
%% This is file `sample-acmsmall.tex',
%% generated with the docstrip utility.
%%
%% The original source files were:
%%
%% samples.dtx  (with options: `acmsmall')
%% 
%% IMPORTANT NOTICE:
%% 
%% For the copyright see the source file.
%% 
%% Any modified versions of this file must be renamed
%% with new filenames distinct from sample-acmsmall.tex.
%% 
%% For distribution of the original source see the terms
%% for copying and modification in the file samples.dtx.
%% 
%% This generated file may be distributed as long as the
%% original source files, as listed above, are part of the
%% same distribution. (The sources need not necessarily be
%% in the same archive or directory.)
%%
%% Commands for TeXCount
%TC:macro \cite [option:text,text]
%TC:macro \citep [option:text,text]
%TC:macro \citet [option:text,text]
%TC:envir table 0 1
%TC:envir table* 0 1
%TC:envir tabular [ignore] word
%TC:envir displaymath 0 word
%TC:envir math 0 word
%TC:envir comment 0 0
%%
%%
%% The first command in your LaTeX source must be the \documentclass command.
% \documentclass[manuscript,screen,review]{acmart}

% \documentclass[manuscript,review,anonymous]{acmart}


% \documentclass[manuscript,screen]{acmart}
\documentclass[sigconf]{acmart}
\PassOptionsToPackage{table}{xcolor}
\usepackage{tabularx}
%% NOTE that a single column version is required for 
%% submission and peer review. This can be done by changing
%% the \doucmentclass[...]{acmart} in this template to 
%% \documentclass[manuscript,screen]{acmart}
%% 
%% To ensure 100% compatibility, please check the white list of
%% approved LaTeX packages to be used with the Master Article Template at
%% https://www.acm.org/publications/taps/whitelist-of-latex-packages 
%% before creating your document. The white list page provides 
%% information on how to submit additional LaTeX packages for 
%% review and adoption.
%% Fonts used in the template cannot be substituted; margin 
%% adjustments are not allowed.
%%
%% \BibTeX command to typeset BibTeX logo in the docs
\AtBeginDocument{%
  \providecommand\BibTeX{{%
    \normalfont B\kern-0.5em{\scshape i\kern-0.25em b}\kern-0.8em\TeX}}}

%% Rights management information.  This information is sent to you
%% when you complete the rights form.  These commands have SAMPLE
%% values in them; it is your responsibility as an author to replace
%% the commands and values with those provided to you when you
%% complete the rights form.
% \setcopyright{acmlicensed}
% \copyrightyear{2024}
% \acmYear{2024}
% \acmDOI{XXXXXXX.XXXXXXX}

\copyrightyear{2025}
\acmYear{2025}
\setcopyright{cc}
\setcctype{by-nc}
\acmConference[CHI '25]{CHI Conference on Human Factors in Computing Systems}{April 26-May 1, 2025}{Yokohama, Japan}
\acmBooktitle{CHI Conference on Human Factors in Computing Systems (CHI '25), April 26-May 1, 2025, Yokohama, Japan}\acmDOI{10.1145/3706598.3714255}
\acmISBN{979-8-4007-1394-1/25/04}
%%
%% These commands are for a JOURNAL article.
% \acmJournal{JACM}
% \acmVolume{37}
% \acmNumber{4}
% \acmArticle{111}
% \acmMonth{8}

%%
%% Submission ID.
%% Use this when submitting an article to a sponsored event. You'll
%% receive a unique submission ID from the organizers
%% of the event, and this ID should be used as the parameter to this command.
% \acmSubmissionID{123-A56-BU3}

%%
%% For managing citations, it is recommended to use bibliography
%% files in BibTeX format.
%%
%% You can then either use BibTeX with the ACM-Reference-Format style,
%% or BibLaTeX with the acmnumeric or acmauthoryear sytles, that include
%% support for advanced citation of software artefact from the
%% biblatex-software package, also separately available on CTAN.
%%
%% Look at the sample-*-biblatex.tex files for templates showcasing
%% the biblatex styles.
%%

%%
%% The majority of ACM publications use numbered citations and
%% references.  The command \citestyle{authoryear} switches to the
%% "author year" style.
%%
%% If you are preparing content for an event
%% sponsored by ACM SIGGRAPH, you must use the "author year" style of
%% citations and references.
%% Uncommenting
%% the next command will enable that style.
%%\citestyle{acmauthoryear}

%%
%% end of the preamble, start of the body of the document source.


\usepackage{tabularx}
% \usepackage{threeparttable}
\usepackage{booktabs}
\usepackage{multirow}
\usepackage{makecell}
\usepackage{colortbl}
\usepackage{booktabs}
% \usepackage[table]{xcolor}
\usepackage{array}
\usepackage{pgf}
\usepackage{collcell}
\usepackage{hhline}
\usepackage{longtable}
\usepackage{multirow}
\usepackage{amsmath}

% \usepackage{ackslash}

% \usepackage{adjustbox}
\usepackage{wrapfig}
\usepackage{float}
\usepackage{enumitem}
\usepackage{url}
\usepackage{hyperref}
\usepackage{ragged2e}
\usepackage{tabularray}
\usepackage{setspace}
\usepackage{subcaption}
\usepackage{utfsym}

% \usepackage{tabularx}
% \usepackage{colortbl}
\usepackage{fontawesome}
% \usepackage{longtable}
% \usepackage{tcolorbox}
\usepackage{xcolor}
% \usepackage[T1]{fontenc}
% \usepackage{libertine}
% \usepackage{inconsolata}
% https://www.overleaf.com/learn/latex/Font_typefaces
% \usepackage{geometry}

% \geometry{
%   a4paper,
%   total={170mm,257mm},
%   left=20mm,
%   top=20mm,
% }

% Define colors
% \definecolor{high}{HTML}{67a284} % dark green
\definecolor{high}{HTML}{628c77} % dark green
\definecolor{mid}{HTML}{b9f2d5}  % light green
\definecolor{low}{HTML}{FFFFFF}  % white

\definecolor{high1}{HTML}{7262ac} % purple
\definecolor{mid1}{HTML}{FFFFFF}  % dark green
\definecolor{low1}{HTML}{2e974e}  % light green


\newcommand{\gradientcelld}[8]{
\xdef\lowvalx{#2}%
\xdef\midvalx{#3}%
\xdef\maxvalx{#4}%
\xdef\lowcolx{#5}%
\xdef\midcolx{#6}%
\xdef\highcolx{#7}%
\xdef\opacityx{#8}%
% The values are calculated linearly between \midval and \maxval
\ifdimcomp{#1pt}{>}{\maxvalx pt}{\cellcolor{\highcolx!100.0!\midcolx!\opacityx}#1}{
\ifdimcomp{#1pt}{<}{\midvalx pt}{%
\ifdimcomp{#1pt}{<}{\lowvalx pt}{\cellcolor{\midcolx!0.0!\lowcolx!\opacityx}#1}{
     \pgfmathparse{int(round(100*(#1/(\midvalx-\lowvalx))-(\lowvalx*(100/(\midvalx-\lowvalx)))))}%
    \xdef\tempa{\pgfmathresult}%
    \cellcolor{\midcolx!\tempa!\lowcolx!\opacityx}#1%
}}{
     \pgfmathparse{int(round(100*(#1/(\maxvalx-\midvalx))-(\midvalx*(100/(\maxvalx-\midvalx)))))}
    \xdef\tempb{\pgfmathresult}%
    \cellcolor{\highcolx!\tempb!\midcolx!\opacityx}#1%
}}
}


\newcommand{\gradientcell}[1]{
    \gradientcelld{#1}{-0.04}{0.36}{0.76}{low}{mid}{high}{60}
    }

\newcommand{\g}[1]{
    \gradientcelld{#1}{-0.18}{0}{0.13}{low1}{mid1}{high1}{60}
    }

\newcommand{\gradientcellboldd}[8]{
\xdef\lowvalx{#2}%
\xdef\midvalx{#3}%
\xdef\maxvalx{#4}%
\xdef\lowcolx{#5}%
\xdef\midcolx{#6}%
\xdef\highcolx{#7}%
\xdef\opacityx{#8}%
% The values are calculated linearly between \midval and \maxval
\ifdimcomp{#1pt}{>}{\maxvalx pt}{\cellcolor{\highcolx!100.0!\midcolx!\opacityx}\textbf{#1}}{
\ifdimcomp{#1pt}{<}{\midvalx pt}{%
\ifdimcomp{#1pt}{<}{\lowvalx pt}{\cellcolor{\midcolx!0.0!\lowcolx!\opacityx}\textbf{#1}}{
     \pgfmathparse{int(round(100*(#1/(\midvalx-\lowvalx))-(\lowvalx*(100/(\midvalx-\lowvalx)))))}%
    \xdef\tempa{\pgfmathresult}%
    \cellcolor{\midcolx!\tempa!\lowcolx!\opacityx}\textbf{#1}%
}}{
     \pgfmathparse{int(round(100*(#1/(\maxvalx-\midvalx))-(\midvalx*(100/(\maxvalx-\midvalx)))))}
    \xdef\tempb{\pgfmathresult}%
    \cellcolor{\highcolx!\tempb!\midcolx!\opacityx}\textbf{#1}%
}}
}

\newcommand{\gradientcellbold}[1]{
    \gradientcellboldd{#1}{-0.04}{0.36}{0.76}{low}{mid}{high}{60}
}
% \usepackage[T1]{fontenc}

% \renewcommand{\bibpreamble}{\textcolor{blue}{\textit{References marked with} \textbf{\faAsterisk} \textit{are suggested by reviewers.}}}


\usepackage{graphicx}
% \usepackage{color}
\definecolor{darkred}{HTML}{7e0f12}
\definecolor{darkgreen}{rgb}{0.0, 0.5, 0.0}
\definecolor{purple}{HTML}{7262ac}
\begin{document}

%%
%% The "title" command has an optional parameter,
%% allowing the author to define a "short title" to be used in page headers.
\title{Deconstructing Depression Stigma: Integrating AI-driven Data Collection and Analysis with Causal Knowledge Graphs}

% [Exploring the Potential of Human-LLM Synergy in Advancing Qualitative Analysis]{Exploring the Potential of Human-LLM Synergy in Advancing Qualitative Analysis: A Case Study on Mental-Illness Stigma}


%%
%% The "author" command and its associated commands are used to define
%% the authors and their affiliations.
%% Of note is the shared affiliation of the first two authors, and the
%% "authornote" and "authornotemark" commands
%% used to denote shared contribution to the research.

\author{Han Meng}
\email{han.meng@u.nus.edu}
\orcid{0009-0003-2318-3639}
\affiliation{
  \institution{Department of Computer Science, National University of Singapore}
  \streetaddress{21 Lower Kent Ridge Road}
  \country{Singapore}
  \postcode{119077}
}
\author{Renwen Zhang}
\email{r.zhang@nus.edu.sg}
\orcid{0000-0002-7636-9598}
\affiliation{
  \institution{Department of Communication and New Media, National University of Singapore}
  \streetaddress{21 Lower Kent Ridge Road}
  \country{Singapore}
  \postcode{119077}
}
\author{Ganyi Wang}
\email{ganyi-w@comp.nus.edu.sg}
\orcid{0009-0009-3294-6363}
\affiliation{
  \institution{School of Computing, National University of Singapore}
  \streetaddress{21 Lower Kent Ridge Road}
  \country{Singapore}
  \postcode{119077}
}
\author{Yitian Yang}
\email{yang.yitian@u.nus.edu}
\orcid{0009-0000-7530-2116}
\affiliation{
  \institution{Department of Computer Science, National University of Singapore}
  \streetaddress{21 Lower Kent Ridge Road}
  \country{Singapore}
  \postcode{119077}
}
\author{Peinuan Qin}
\email{e1322754@u.nus.edu}
\orcid{0000-0002-8737-8369}
\affiliation{
  \institution{Department of Computer Science, National University of Singapore}
  \streetaddress{21 Lower Kent Ridge Road}
  \country{Singapore}
  \postcode{119077}
}
\author{Jungup Lee}
\email{swklj@nus.edu.sg}
\orcid{0000-0002-8243-0543}
\affiliation{
  \institution{Department of Social Work, National University of Singapore}
  \streetaddress{21 Lower Kent Ridge Road}
  \country{Singapore}
  \postcode{119077}
}
\author{Yi-Chieh Lee}
\email{yclee@nus.edu.sg}
\orcid{0000-0002-5484-6066}
\affiliation{
  \institution{Department of Computer Science, National University of Singapore}
  \streetaddress{21 Lower Kent Ridge Road}
  \country{Singapore}
  \postcode{119077}
}


%%
%% By default, the full list of authors will be used in the page
%% headers. Often, this list is too long, and will overlap
%% other information printed in the page headers. This command allows
%% the author to define a more concise list
%% of authors' names for this purpose.
% \renewcommand{\shortauthors}{et al.}

%%
%% The abstract is a short summary of the work to be presented in the
%% article.
\begin{abstract}
Mental-illness stigma is a persistent social problem, hampering both treatment-seeking and recovery. 
Accordingly, there is a pressing need to understand it more clearly, but analyzing the relevant data is highly labor-intensive. 
Therefore, we designed a chatbot to engage participants in conversations; coded those conversations qualitatively with AI assistance; and, based on those coding results, built causal knowledge graphs to decode stigma. 
The results we obtained from 1,002 participants demonstrate that conversation with our chatbot can elicit rich information about people’s attitudes toward depression, while our AI-assisted coding was strongly consistent with human-expert coding.
Our novel approach combining large language models (LLMs) and causal knowledge graphs uncovered patterns in individual responses and illustrated the interrelationships of psychological constructs in the dataset as a whole. 
The paper also discusses these findings’ implications for HCI researchers in developing digital interventions, decomposing human psychological constructs, and fostering inclusive attitudes.
\end{abstract}


%%
%% The code below is generated by the tool at http://dl.acm.org/ccs.cfm.
%% Please copy and paste the code instead of the example below.
%%
\begin{CCSXML}
<ccs2012>
   <concept>
       <concept_id>10003120.10003121.10011748</concept_id>
       <concept_desc>Human-centered computing~Empirical studies in HCI</concept_desc>
       <concept_significance>500</concept_significance>
       </concept>
   <concept>
       <concept_id>10010405.10010455.10010459</concept_id>
       <concept_desc>Applied computing~Psychology</concept_desc>
       <concept_significance>300</concept_significance>
       </concept>
   <concept>
       <concept_id>10003120.10003121.10003122</concept_id>
       <concept_desc>Human-centered computing~HCI design and evaluation methods</concept_desc>
       <concept_significance>500</concept_significance>
       </concept>
 </ccs2012>
\end{CCSXML}

\ccsdesc[500]{Human-centered computing~Empirical studies in HCI}
\ccsdesc[300]{Applied computing~Psychology}
\ccsdesc[500]{Human-centered computing~HCI design and evaluation methods}
%%
%% Keywords. The author(s) should pick words that accurately describe
%% the work being presented. Separate the keywords with commas.
\keywords{Social Stigma, Depression, Causal Knowledge Graph, AI-assisted Coding, Chatbot, Large Language Model}



%%
%% This command processes the author and affiliation and title
%% information and builds the first part of the formatted document.
\maketitle


\section{Introduction}
Backdoor attacks pose a concealed yet profound security risk to machine learning (ML) models, for which the adversaries can inject a stealth backdoor into the model during training, enabling them to illicitly control the model's output upon encountering predefined inputs. These attacks can even occur without the knowledge of developers or end-users, thereby undermining the trust in ML systems. As ML becomes more deeply embedded in critical sectors like finance, healthcare, and autonomous driving \citep{he2016deep, liu2020computing, tournier2019mrtrix3, adjabi2020past}, the potential damage from backdoor attacks grows, underscoring the emergency for developing robust defense mechanisms against backdoor attacks.

To address the threat of backdoor attacks, researchers have developed a variety of strategies \cite{liu2018fine,wu2021adversarial,wang2019neural,zeng2022adversarial,zhu2023neural,Zhu_2023_ICCV, wei2024shared,wei2024d3}, aimed at purifying backdoors within victim models. These methods are designed to integrate with current deployment workflows seamlessly and have demonstrated significant success in mitigating the effects of backdoor triggers \cite{wubackdoorbench, wu2023defenses, wu2024backdoorbench,dunnett2024countering}.  However, most state-of-the-art (SOTA) backdoor purification methods operate under the assumption that a small clean dataset, often referred to as \textbf{auxiliary dataset}, is available for purification. Such an assumption poses practical challenges, especially in scenarios where data is scarce. To tackle this challenge, efforts have been made to reduce the size of the required auxiliary dataset~\cite{chai2022oneshot,li2023reconstructive, Zhu_2023_ICCV} and even explore dataset-free purification techniques~\cite{zheng2022data,hong2023revisiting,lin2024fusing}. Although these approaches offer some improvements, recent evaluations \cite{dunnett2024countering, wu2024backdoorbench} continue to highlight the importance of sufficient auxiliary data for achieving robust defenses against backdoor attacks.

While significant progress has been made in reducing the size of auxiliary datasets, an equally critical yet underexplored question remains: \emph{how does the nature of the auxiliary dataset affect purification effectiveness?} In  real-world  applications, auxiliary datasets can vary widely, encompassing in-distribution data, synthetic data, or external data from different sources. Understanding how each type of auxiliary dataset influences the purification effectiveness is vital for selecting or constructing the most suitable auxiliary dataset and the corresponding technique. For instance, when multiple datasets are available, understanding how different datasets contribute to purification can guide defenders in selecting or crafting the most appropriate dataset. Conversely, when only limited auxiliary data is accessible, knowing which purification technique works best under those constraints is critical. Therefore, there is an urgent need for a thorough investigation into the impact of auxiliary datasets on purification effectiveness to guide defenders in  enhancing the security of ML systems. 

In this paper, we systematically investigate the critical role of auxiliary datasets in backdoor purification, aiming to bridge the gap between idealized and practical purification scenarios.  Specifically, we first construct a diverse set of auxiliary datasets to emulate real-world conditions, as summarized in Table~\ref{overall}. These datasets include in-distribution data, synthetic data, and external data from other sources. Through an evaluation of SOTA backdoor purification methods across these datasets, we uncover several critical insights: \textbf{1)} In-distribution datasets, particularly those carefully filtered from the original training data of the victim model, effectively preserve the model’s utility for its intended tasks but may fall short in eliminating backdoors. \textbf{2)} Incorporating OOD datasets can help the model forget backdoors but also bring the risk of forgetting critical learned knowledge, significantly degrading its overall performance. Building on these findings, we propose Guided Input Calibration (GIC), a novel technique that enhances backdoor purification by adaptively transforming auxiliary data to better align with the victim model’s learned representations. By leveraging the victim model itself to guide this transformation, GIC optimizes the purification process, striking a balance between preserving model utility and mitigating backdoor threats. Extensive experiments demonstrate that GIC significantly improves the effectiveness of backdoor purification across diverse auxiliary datasets, providing a practical and robust defense solution.

Our main contributions are threefold:
\textbf{1) Impact analysis of auxiliary datasets:} We take the \textbf{first step}  in systematically investigating how different types of auxiliary datasets influence backdoor purification effectiveness. Our findings provide novel insights and serve as a foundation for future research on optimizing dataset selection and construction for enhanced backdoor defense.
%
\textbf{2) Compilation and evaluation of diverse auxiliary datasets:}  We have compiled and rigorously evaluated a diverse set of auxiliary datasets using SOTA purification methods, making our datasets and code publicly available to facilitate and support future research on practical backdoor defense strategies.
%
\textbf{3) Introduction of GIC:} We introduce GIC, the \textbf{first} dedicated solution designed to align auxiliary datasets with the model’s learned representations, significantly enhancing backdoor mitigation across various dataset types. Our approach sets a new benchmark for practical and effective backdoor defense.



\section{Related Work}

\subsection{Large 3D Reconstruction Models}
Recently, generalized feed-forward models for 3D reconstruction from sparse input views have garnered considerable attention due to their applicability in heavily under-constrained scenarios. The Large Reconstruction Model (LRM)~\cite{hong2023lrm} uses a transformer-based encoder-decoder pipeline to infer a NeRF reconstruction from just a single image. Newer iterations have shifted the focus towards generating 3D Gaussian representations from four input images~\cite{tang2025lgm, xu2024grm, zhang2025gslrm, charatan2024pixelsplat, chen2025mvsplat, liu2025mvsgaussian}, showing remarkable novel view synthesis results. The paradigm of transformer-based sparse 3D reconstruction has also successfully been applied to lifting monocular videos to 4D~\cite{ren2024l4gm}. \\
Yet, none of the existing works in the domain have studied the use-case of inferring \textit{animatable} 3D representations from sparse input images, which is the focus of our work. To this end, we build on top of the Large Gaussian Reconstruction Model (GRM)~\cite{xu2024grm}.

\subsection{3D-aware Portrait Animation}
A different line of work focuses on animating portraits in a 3D-aware manner.
MegaPortraits~\cite{drobyshev2022megaportraits} builds a 3D Volume given a source and driving image, and renders the animated source actor via orthographic projection with subsequent 2D neural rendering.
3D morphable models (3DMMs)~\cite{blanz19993dmm} are extensively used to obtain more interpretable control over the portrait animation. For example, StyleRig~\cite{tewari2020stylerig} demonstrates how a 3DMM can be used to control the data generated from a pre-trained StyleGAN~\cite{karras2019stylegan} network. ROME~\cite{khakhulin2022rome} predicts vertex offsets and texture of a FLAME~\cite{li2017flame} mesh from the input image.
A TriPlane representation is inferred and animated via FLAME~\cite{li2017flame} in multiple methods like Portrait4D~\cite{deng2024portrait4d}, Portrait4D-v2~\cite{deng2024portrait4dv2}, and GPAvatar~\cite{chu2024gpavatar}.
Others, such as VOODOO 3D~\cite{tran2024voodoo3d} and VOODOO XP~\cite{tran2024voodooxp}, learn their own expression encoder to drive the source person in a more detailed manner. \\
All of the aforementioned methods require nothing more than a single image of a person to animate it. This allows them to train on large monocular video datasets to infer a very generic motion prior that even translates to paintings or cartoon characters. However, due to their task formulation, these methods mostly focus on image synthesis from a frontal camera, often trading 3D consistency for better image quality by using 2D screen-space neural renderers. In contrast, our work aims to produce a truthful and complete 3D avatar representation from the input images that can be viewed from any angle.  

\subsection{Photo-realistic 3D Face Models}
The increasing availability of large-scale multi-view face datasets~\cite{kirschstein2023nersemble, ava256, pan2024renderme360, yang2020facescape} has enabled building photo-realistic 3D face models that learn a detailed prior over both geometry and appearance of human faces. HeadNeRF~\cite{hong2022headnerf} conditions a Neural Radiance Field (NeRF)~\cite{mildenhall2021nerf} on identity, expression, albedo, and illumination codes. VRMM~\cite{yang2024vrmm} builds a high-quality and relightable 3D face model using volumetric primitives~\cite{lombardi2021mvp}. One2Avatar~\cite{yu2024one2avatar} extends a 3DMM by anchoring a radiance field to its surface. More recently, GPHM~\cite{xu2025gphm} and HeadGAP~\cite{zheng2024headgap} have adopted 3D Gaussians to build a photo-realistic 3D face model. \\
Photo-realistic 3D face models learn a powerful prior over human facial appearance and geometry, which can be fitted to a single or multiple images of a person, effectively inferring a 3D head avatar. However, the fitting procedure itself is non-trivial and often requires expensive test-time optimization, impeding casual use-cases on consumer-grade devices. While this limitation may be circumvented by learning a generalized encoder that maps images into the 3D face model's latent space, another fundamental limitation remains. Even with more multi-view face datasets being published, the number of available training subjects rarely exceeds the thousands, making it hard to truly learn the full distibution of human facial appearance. Instead, our approach avoids generalizing over the identity axis by conditioning on some images of a person, and only generalizes over the expression axis for which plenty of data is available. 

A similar motivation has inspired recent work on codec avatars where a generalized network infers an animatable 3D representation given a registered mesh of a person~\cite{cao2022authentic, li2024uravatar}.
The resulting avatars exhibit excellent quality at the cost of several minutes of video capture per subject and expensive test-time optimization.
For example, URAvatar~\cite{li2024uravatar} finetunes their network on the given video recording for 3 hours on 8 A100 GPUs, making inference on consumer-grade devices impossible. In contrast, our approach directly regresses the final 3D head avatar from just four input images without the need for expensive test-time fine-tuning.


\section{Methodology}

\subsection{Problem Definition}

Given a multivariate time series input $X \in \mathbb{R}^{C  \times T}$, multivariate time series forecasting tasks are designed to predict its future $F$ time steps $\hat{Y}\in \mathbb{R}^{C \times F}$ using past $T$ steps. $C $ is the number of variates or channels.

\subsection{Preliminary Analysis}

This section presents why RevIN~\citep{Kim_revin,liu2022non}, High-pass, and Low-pass filters fail to address the Mid-Frequency Spectrum Gap. Let the input univariate time series be $ x(t) $ with length $ T $ and target $ y(t) $ with length $ F $. 

\begin{definition}[Frequency Spectral Energy]\label{def:energy}
The Fourier transform of $x(t)$, $X(f)$, and its spectral energy $E_X(f)$ is given by:
\vspace{-0.2cm}
\begin{align}
X(f) = \sum_{t=0}^{T-1} x(t) e^{-i 2 \pi f t / {T-1}}, \quad &f = 0, 1, \dots, T-1\notag\\
E_X(f) = |X(f)|^2.
\end{align}
\vspace{-0.2cm}
\end{definition}

\textbf{Impact of RevIN on Frequency Spectrum \quad}
\begin{definition}[Reversible Instance Normalization]\label{def:RevIN}
Given a \textbf{forecast model} $ f: \mathbb{R}^T \rightarrow \mathbb{R}^F $ that generates a forecast $ \hat{y}(t) $ from a given input $x(t)$, RevIN is defined as:
\vspace{-0.2cm}
\begin{align}
&\hat{x}(t) = \frac{x(t) - \mu}{\sigma},\quad t = 0, 1, \dots, T-1\notag\\
&\hat{y}(t) = f(\hat{x}(t)), \quad \hat{y}(t)_{rev}= \hat{y}(t) \cdot \sigma + \mu,\notag\\
&\mu = \frac{1}{T} \sum_{t=0}^{T-1} x(t), \quad \sigma = \sqrt{\frac{1}{T} \sum_{t=0}^{T-1} (x(t) - \mu)^2}.
\end{align}
\vspace{-0.2cm}
\end{definition}

\begin{theorem} [Frequency Spectrum after RevIN] \label{theorem:RevIN}
\vspace{-0.2cm}
The spectral energy of $\hat{x}(t)$ (transformed using RevIN):
\begin{align}
E_{\hat{X}}(0)=0,& \quad f=0, \notag\\
E_{\hat{X}}(f) = \left( \frac{1}{\sigma} \right)^2 |X(f)|^2,&\quad f = 1,2,\dots, T-1 . 
\end{align}
\vspace{-0.2cm}
\end{theorem}
The proof is in Appendix~\ref{app:RevIN}. Theorem~\ref{theorem:RevIN} suggests that RevIN scales the absolute spectral energy by $ \sigma^2 $ but does not affect its relative distribution except $E_{\hat{X}}(0)=0$. Thus, RevIN preserves the relative spectral energy distribution and leaves the Mid-Frequency Spectrum Gap unresolved. \textit{However, our experiments still employ RevIN to ensure a fair comparison with other baselines.}
\begin{figure*}[h]
  \centering
  \includegraphics[width=1.\linewidth]{Faker/source/assets/jpg/ReFocus.jpg}
  \caption{General structure of \textbf{ReFocus}. `Adaptive Mid-Frequency Energy Optimizer (AMEO)' enhances mid-frequency components modeling, and `Energy-based Key-Frequency Picking Block' (EKPB) effectively captures shared Key-Frequency across channels}
  \label{fig:refocus}
\end{figure*}

\begin{figure*}[h]
  \centering
  \includegraphics[width=0.7\linewidth]{Faker/source/assets/jpg/ket.jpg}
  \caption{General process of the \textbf{Key-Frequency Enhanced Training strategy (KET)}, where spectral information from other channels is randomly introduced into each channel, to enhance the extraction of the shared Key-Frequency.}
  \label{fig:reshuffle}
\end{figure*}
\textbf{Impact of High- and Low-pass filter \quad}
We still define $\hat{x}(t)$ to be the filtered (processed) signal, obtained by applying a filter $H(f)$ (High/Low-pass filter). The filter $ H(f) $ is 1 in the passband (High/Low frequency) and 0 in the stopband (Middle frequency). So $E_{\hat{X}}(f)=0,\quad E_{\hat{X}}\leq E_X(f)$ for middle frequencies, which creates even larger gap.

\subsection{Overall Structure of The Proposed ReFocus}

In this section, we elucidate the overall architecture of \textbf{ReFocus}, depicted in Figure \ref{fig:refocus}. We define frequency domain projection as $D1\rightarrow D2$ representing a projection from dimension $D1$ to $D2$ in the frequency domain~\citep{xu2024fits}. Initially, we apply \textbf{AMEO} to the input $X \in \mathbb{R}^{C \times T}$, yielding the processed spectrum $ X_{am} \in \mathbb{R}^{C  \times T} $. Next, we use a projection $T\rightarrow D$ to transform $ X_{am}$ into the Variate Embedding $ X_{em} \in \mathbb{R}^{C  \times D}$~\citep{LiuiTransformer}. Then, $X_{em}$ go through $N$ \textbf{EKPB} to generate representation $H_{N+1}$, which is projected to obtain final prediction $\hat{Y}$. 

\textbf{Adaptive Mid-Frequency Energy Optimizer \quad}
Building upon the \textbf{Preliminary Analysis}, we propose a convolution- and residual learning-based solution to address the Mid-Frequency Spectrum Gap, which we denoted as AMEO. 
\begin{definition}[Adaptive Mid-Frequency Energy Optimizer]\label{def:AMEO}
AMEO is defined as:
\begin{align}
&\hat{x}(t) = x(t)-\frac{\beta}{K}\sum_{k=0}^{K-1} \tilde{x}(t+K-1-k),\notag\\
&\tilde{x}(t) =\notag\\
&\begin{cases}
x(t-(\frac{K}{2}+1)), \quad \text{if } \frac{K}{2}+1 \leq t < T+\frac{K}{2}+1, \\
0,  \quad\text{if } 0 \leq t < \frac{K}{2}+1 \text{ or } T+\frac{K}{2}+1 \leq t < T+K.
\end{cases}
\end{align}
\vspace{-0.2cm}
\end{definition}

It is equivalent to $x=x-\beta \cdot Conv(x)$. $Conv$ is a 1D convolution (Zero-padding at both ends, stride $s=1$, kernel size $K$, with values initialized as $ \frac{1}{K} $). $\beta \in \mathbb{R}^{1}$ is a hyperparameter.

\begin{theorem} [Frequency Spectrum after AMEO] \label{theorem:AMEO}
The spectral energy of $\hat{x}(t)$ obtained using AMEO:
\begin{align}
E_{\hat{X}}(f) =|X(f)|^2 \left\{1 - \beta \cdot \underbrace{\frac{1}{K} \sum_{k=0}^{K-1} e^{i 2 \pi f (\frac{3K}{2}-k -2) / {T-1}}}_{G(f)}\right\}^2
\end{align}
\vspace{-0.2cm}
\end{theorem}

The proof is in Appendix~\ref{app:AMEO}. We have $E_{\hat{X}}(f) =|X(f)|^2(1-\beta  \cdot G(f))^2$. Generally, $ G(f) $ behaves as a decay function, gradually reducing its value from \textbf{One} to \textbf{Zero}. Such \textbf{decay behavior} makes AMEO relatively enhances mid-frequency components, thus addressing the Mid-Frequency Spectrum Gap.

\textbf{Energy-based Key-Frequency Picking Block \quad} In each \textbf{EKPB}, the input $ H_i \in \mathbb{R}^{C  \times D} (H_1=X_{em}) $ is first processed through an MLP to generate $ H_i^k \in \mathbb{R}^{C  \times Q}$. Then, FFT is applied to get $ H_i^f \in \mathbb{R}^{C  \times (Q/2+1)}$. For $ H_i^f$, we calculate its energy, denoted as $ H_i^e \in \mathbb{R}^{C  \times (Q/2+1)}$. A cross-channel softmax is then applied to $ H_i^e$ per frequency to obtain a probability distribution $ H_i^{soft} \in \mathbb{R}^{C  \times (Q/2+1)}$. Using $H_i^{soft}$, we select values from $ H_i^f$ across channels for each frequency, resulting in $K^f_i \in \mathbb{R}^{1  \times (Q/2+1)}$, which represents the Shared Key-Frequency across all channels. Then iFFT is performed on $K^f_i$ to get $K_i\in \mathbb{R}^{1  \times Q}$, followed by projection $Q\rightarrow D$ and repeating (C times) to get $\hat{K}_i \in \mathbb{R}^{C  \times D}$. This $\hat{K}_i$ is point-wisely added to $\hat{H_i}\in \mathbb{R}^{C  \times D}$ , which is the projection of $ H_i$ using projection $D\rightarrow D$. Then, an MLP and $Add\&Norm$ is applied to the result $HK\in \mathbb{R}^{C  \times D}$ to fuse inter-series dependencies information, and another MLP and $Add\&Norm$ is used to capture intra-series variations~\citep{LiuiTransformer}. The output of each \textbf{EKPB} is $\hat{O_i} \in \mathbb{R}^{C  \times D}$, where $H_{i+1}=\hat{O_i}$.

\subsection{Key-Frequency Enhanced Training strategy}

In real-world time series, certain channels often exhibit spectral dependencies, which may not be fully captured in the training set, and the specific channels with such dependencies are also unknown~\citep{geweke1984freqchannel,Zhao2024freqchannel}. So this work borrows insight from recent advancement of mix-up in time series~\citep{zhou2023mixup,ansari2024mixup}, randomly introducing spectral information from other channels into each channel, to enhance the extraction of the shared Key-Frequency, as in Figure~\ref{fig:reshuffle}. Given a multivariate time series input $X \in \mathbb{R}^{C \times T}$ and its ground-truth $Y \in \mathbb{R}^{C \times F}$, we generate a pseudo sample pair: 

\begin{align}
X' = iFFT(FFT(X) +\alpha \cdot FFT(X[\text{perm},:]))&,  \notag\\ 
Y' = iFFT(FFT(Y) +\alpha \cdot FFT(Y[\text{perm},:]))&.
\end{align}

$\alpha \in \mathbb{R}^{C \times 1}$ is a weight vector sampled from a normal distribution, $\text{perm}$ is a reshuffled channel index. Since $FFT$ and $iFFT$ are linear operations, this mix-up process can be equivalently simplified in the \textbf{Time Domain}:
\begin{align}
X' = X +\alpha \cdot X[\text{perm},:]&,  \notag\\
Y' = Y +\alpha \cdot Y[\text{perm},:]&
 \end{align}
We alternate training between real and synthetic data to preserve the spectral dependencies in real samples. This combines the advantages of data augmentation, such as improved generalization, while mitigating potential drawbacks like over-smoothing and training instability~\citep{ryu2024tf,alkhalifah2022tf}.













\section{Result}

\subsection{RQ1: What are the design principles and initial prototype characteristics of Echo-Teddy?}

\subsubsection{Design principles of Echo-Teddy}


The design principles of Echo-Teddy are structured to comprehensively support social and social emotional development in autistic children, focusing on four key themes: Potential User, Ethical Consideration, Customization, and Usage. The robot is designed to function as both a peer and an assistant, accommodating the neurodiverse characteristics of autistic children, including gaze aversion, preference for structured patterns, heightened perceptual sensitivity, and attention to hierarchical information and fine details. By integrating these elements, Echo-Teddy creates an interactive, supportive, and engaging experience tailored to the needs of its users.

Echo-Teddy’s verbal and behavioral output aligns with best practices in autism support, ensuring that its speech patterns are age-appropriate and incorporate evidence-based teaching strategies. These strategies reinforce positive behaviors while mitigating interfering behaviors. The robot is designed to elicit and encourage target verbal and behavioral responses, offering structured interactions while maintaining the flexibility needed to adapt to individual learning preferences. Unlike humanoid robots, Echo-Teddy features a soft, fur-covered exterior, a design choice aimed at reducing anxiety and increasing comfort for autistic children who may find highly realistic or human-like robotic faces overwhelming.

Ethical considerations are central to Echo-Teddy’s development. The robot consistently uses positive reinforcement strategies to prevent frustration and enhance engagement. Its speech and behavioral responses dynamically adjust to accommodate each child's individual needs and comfort levels. A key objective in its design is to minimize stressors while fostering an environment that encourages meaningful interaction.

Customization plays a pivotal role in Echo-Teddy’s adaptability. Caregivers and educators can tailor its communication style, interaction topics, and behavioral prompts to align with each child’s unique preferences and developmental goals. This customization extends beyond software, as the robot’s physical appearance can be modified to better suit individual sensory and aesthetic preferences.

Practical usability is another core aspect of Echo-Teddy’s design. The robot is built to be durable and resilient, capable of withstanding minor impacts or water exposure, ensuring longevity in varied educational and home environments. Additionally, Echo-Teddy is designed for independent operation, eliminating the need for constant human intervention and allowing children to interact with it autonomously.

By integrating neurodiverse-friendly interaction models, ethical safeguards, extensive customization options, and practical durability, Echo-Teddy is designed to be an effective and accessible tool for enhancing social communication skills in autistic children. These principles ensure that Echo-Teddy is not only tailored to the unique needs of its users but also remains ethically responsible, adaptable, and functional in real-world applications.

\begin{table*}[hbt!]
\scriptsize
\begin{tabular}{p{2cm} p{3cm} p{7cm}}
\hline
\multicolumn{1}{c}{\textbf{Themes}} & \multicolumn{1}{c}{\textbf{Categories}}   & \multicolumn{1}{c}{\textbf{Design Principles}}                                                                                                   \\
\hline
Potential User                      & Purpose of the Robot                      & This robot should mainly improve the social/socio-emotional skills of autistic children by performing social communication and interaction. \\
                                    &                                           & The robot should act like a facilitator, including peers and assistants.                                                                         \\
                                    & Characteristics of autistic students      & The robot should be designed considering the neurodiverse characteristics of autistic children. (gaze aversion, pattern recognition, perceptual processing, and exceptional focus for paticular topic).
                                    \\
                                    & Output of the robot (Verbal)              & Keep the utterance style and length appropriate to a child of the same age as the user.                                                          \\
                                    &                                           & The robot should activate verbal teaching strategies to induce positive behavior and reduce interfering behaviors.                               \\
                                    &                                           & The robots should be able to elicit the target verbal behavior in children with autism.                                                             \\
                                    & Output of the robot (Behavioral)          & The robot should activate behavioral teaching strategies to induce positive behavior and reduce interfering behaviors.                           \\
                                    &                                           & The robots should be able to elicit the target behavior in children with autism.                                                                    \\
                                    & Appearance of the robot                   & The appearance of the robot should be non-humanoid.                                                                                              \\
\hline
Ethical Consideration               &                                           & Avoid frustration by using positive feedback.                                                                                                    \\
                                    &                                           & Reflect the unique needs of the user in your speech and actions.                                                                                 \\
                                    &                                           & Minimize stress sources and make the participants comfortable.                                                                                   \\
\hline
Customization                      & Preference of the robot                   & The preference of the robot should be customized by the caregiver or instructor of the user.                                                     \\
                                    & Output of the robot (Verbal)              & The subject of the communication should be customized.                                                                                           \\
                                    & Output of the robot (Behavioral)          & The set of behaviors to be stimulated in the child should be customized by the caregiver or the instructor.                                       \\
                                    & Appearance of the robot                   & The appearance of the robot should be customizable.                                                                                              \\
\hline
Usage                               &                                           & Consider the context of use to ensure appropriate sturdiness and prevent damage from shocks or water; the user environment must be considered.   \\
                                    &                                           & The robot should not need additional human support during use.\\
\hline
\end{tabular}
\caption{Design principle for Echo-Teddy.}
\end{table*}

\subsubsection{Initial prototype of Echo-Teddy}

% Hardware
The hardware design of Echo-Teddy is built on a Raspberry Pi platform, chosen for its cost-effectiveness, scalability, and ability to support real-time interaction. The system efficiently transmits audio files and action commands between the Raspberry Pi and the server, ensuring low-latency communication for natural conversations. To further reduce production costs, the microphone and speaker were assembled using custom-purchased components and soldering techniques (Figure \ref{fig:echo-teddy-components}). This approach allowed for greater flexibility in hardware integration while maintaining affordability for broader implementation.

\begin{figure*}[hbt!]
    \centering
    \includegraphics[width=1\textwidth]{imgs/echo-teddy1.pdf}
    \caption{To reduce costs, we purchased microphone and speaker components separately and assembled them using soldering techniques.}
    \label{fig:echo-teddy-components}
\end{figure*}

During the production process, it was observed that placing the Raspberry Pi inside the plush doll led to heat buildup, which caused performance degradation. To address this, a backpack-style enclosure was designed to house the Raspberry Pi externally, allowing for better heat dissipation without compromising portability (Figure \ref{fig:echo-teddy-backpack}). This design also improves ease of maintenance and accessibility for future hardware upgrades.

\begin{figure*}[hbt!]
    \centering
    \includegraphics[width=1\textwidth]{imgs/echo-teddy2.pdf}
    \caption{To prevent the heat buildup, we made backpack to contain the Raspberry Pi.}
    \label{fig:echo-teddy-backpack}
\end{figure*}

The initial prototype version integrates attached motors to enable basic movements, such as nodding, providing simple nonverbal communication cues. Currently, the range of motion is limited to head movements and facial expressions, which are displayed using a dot matrix (Figure \ref{fig:echo-teddy-dot-matrix}). These features are intended to enhance emotional expressiveness and engagement in interactions with users. 

For connectivity, the system utilizes the built-in Wi-Fi module of the Raspberry Pi, ensuring stable access to cloud-based services. Additionally, mobile phone tethering is available as an alternative network solution, enabling portability across different settings, including classrooms, therapy environments, and home use. This flexible connectivity setup ensures that Echo-Teddy remains accessible and functional in diverse user environments.

\begin{figure*}[hbt!]
    \centering
    \includegraphics[width=0.5\textwidth]{imgs/echo-teddy3.pdf}
    \caption{We used dot matrix to express the emotions of Echo-Teddy.}
    \label{fig:echo-teddy-dot-matrix}
\end{figure*}

% Server system
The server system of Echo-Teddy integrates advanced cloud-based technologies to ensure efficient, scalable, and personalized interactions for autistic students. At its core, the system relies on OpenAI’s API for dialogue generation and natural language processing, allowing for context-aware and adaptive conversational interactions. To enhance flexibility and control over language model interactions, a prompt management module has been implemented, enabling LLM administrators to easily update and manage prompt texts for fine-tuned responses.

For speech processing, Echo-Teddy employs AWS Transcribe for Speech-to-Text (STT) functionality, ensuring accurate and efficient transcription of user input. Text-to-Speech (TTS) is handled by both AWS Polly and Naver Clova Voice, with Naver Clova Voice specifically incorporated to support Korean-speaking users. This multi-engine approach ensures high-quality, natural speech synthesis across different languages and user preferences.

The server infrastructure is built using FastAPI, following RESTful API principles, and is hosted on AWS EC2 to maintain reliable and scalable operations. To streamline development and deployment, the system integrates a CI/CD pipeline using GitHub Actions, enabling automated testing, integration, and deployment, reducing maintenance overhead and improving system stability. A key feature of the backend is its ability to transmit audio files along with structured action commands in JSON format, allowing for synchronized multimodal interactions. This ensures that Echo-Teddy’s speech output is coordinated with its physical gestures, enhancing engagement and communication effectiveness.

By leveraging this comprehensive cloud-based system architecture, Echo-Teddy provides a stable, adaptive, and scalable interaction platform designed to support the diverse communication needs of autistic students. The integration of modular and configurable components further ensures that the system remains flexible and customizable, meeting the evolving demands of special education applications.

\subsection{RQ2: What improvements can be made to the initial prototype of Echo-Teddy based on developer reflection-on-action and interviews with experts?}

\subsubsection{Reflection-on-action of developers}

This section reflects on the design and implementation decisions made during the development of the initial prototype of Echo-Teddy, highlighting key challenges encountered and the provisional solutions proposed. These insights provide a foundation for future refinements, ensuring that the robot effectively supports social and emotional development in autistic students.

One of the primary areas of focus was enhancing nonverbal communication capabilities to improve the naturalness of interaction. While the initial prototype allowed Echo-Teddy to express emotions based on user input, its expressive range was limited to seven predefined facial expressions. To increase emotional depth and adaptability, an additional feature involving movable eyebrows could be introduced, allowing for more nuanced and dynamic facial expressions. This modification would enable the robot to better reflect emotional subtleties, making interactions more engaging and responsive to diverse social cues.

A second area of improvement involved implementing a monitoring system for caregivers and educators. Since interactions with Echo-Teddy can provide valuable insights into a child's social communication patterns, enabling convenient tracking of interaction history is essential for individualized intervention planning. To facilitate this, experts recommended optimizing the website for mobile accessibility or developing a dedicated mobile application, allowing caregivers and educators to easily review and analyze communication logs. 

Third, ensuring the reliability of Echo-Teddy’s emotional responses was another critical challenge. Accurate and contextually appropriate emotional reactions to user input are essential for fostering both self-awareness and emotional comprehension in autistic students. However, facial expressions generated from new data should undergo rigorous evaluation to verify accuracy and appropriateness, ensuring that Echo-Teddy's emotional feedback is both credible and developmentally supportive. As noted by Rawal et al. \cite{rawal2022facialemotionexpressionshumanrobot}, effective human-robot interactions depend on accurate emotion recognition and expression, reinforcing the need for continuous validation and refinement of Echo-Teddy’s emotional modeling.

Another key consideration was adapting Echo-Teddy’s appearance to strengthen emotional bonds with users. The physical design of social robots plays a significant role in user engagement and comfort. Research by Ricks and Colton \cite{5509327} suggests that while autistic children benefit from both humanoid and non-humanoid robots, they tend to engage more actively with non-humanoid designs. This finding supports the idea that flexibility in physical form—such as allowing for different shapes or textures—may improve user preference and interaction quality. Providing customizable external features, such as different animal forms or textures, could further optimize engagement based on individual sensory preferences.

Reflecting on these early design decisions has yielded valuable insights into both project management and technical execution. Future development efforts will focus on enhancing Echo-Teddy’s real-time emotional expressiveness, refining its monitoring capabilities, validating emotion-driven responses, and expanding customization options. These refinements will ensure that Echo-Teddy remains an adaptive, engaging, and effective tool for supporting social and emotional development in autistic students.

\subsubsection{Results of interview with experts}

The interview included experts with extensive experience in special education who provided diverse perspectives on the early prototype of the LLM-based social robot Echo-Teddy for students with autism spectrum disorder. The content of the interview was primarily analyzed in terms of seven themes: (1) response speed (processing delay), (2) the robot’s physical form and external features, (3) interaction goals and usage scenarios, (4) nonverbal communication and modeling strategies, (5) the range and characteristics of the target students, (6) generalizability and additional considerations, and (7) implementation costs and scalability. Below is a summary of the experts’ key opinions and improvement suggestions for each theme.

First, experts pointed out that Echo-Teddy experienced a delay of approximately 6–10 seconds when communicating with external servers for STT and TTS. They emphasized that immediate feedback is crucial for students with autism, as delays can lead to loss of attention or increased anxiety during communication. They noted that communication skills must be reinforced through repeated, prompt interactions. To address this, they suggested either (1) upgrading hardware to support faster local processing (e.g., Jetson Nano) or (2) splitting audio data into smaller segments and synthesizing speech sequentially for playback.

Second, the interview included discussions about the rationale behind Echo-Teddy being designed as a teddy bear and the appropriateness of this choice. Experts acknowledged that "soft plush robots can be a positive approach for certain students, as safety and sensory preferences are critical for students with autism." However, they also recommended offering modular options, such as different animal shapes (e.g., dinosaurs), to accommodate students who may react negatively to certain textures or appearances. Furthermore, some experts highlighted the importance of clear visual focus, such as the size and positioning of the robot’s eyes, to encourage eye contact. However, others cautioned against prioritizing eye contact as a universal goal, as it may provoke anxiety or discomfort for some students. Ethical considerations and individual differences were emphasized.

Third, while the development team initially focused on scenarios where students would engage in one-on-one conversations with the robot under the supervision of caregivers or teachers, experts emphasized the importance of fostering peer interactions. They suggested that Echo-Teddy could serve as a mediator to encourage participation and extend social skills in group settings, such as inclusive classrooms. They proposed that the robot could facilitate dialogue between autistic students and their peers, promoting social engagement. However, they also noted potential challenges, such as peers perceiving the robot as a novelty or the risk of students overly depending on the robot.

Fourth, experts emphasized the importance of modeling nonverbal communication (e.g., nodding, gestures, gaze direction) for students with autism, as these skills are as critical as verbal language. They suggested that Echo-Teddy could perform human-like gestures, such as turning its head, waving, or nodding, to encourage imitation behaviors. However, they advised against certain actions, like "hugging," which may not align with typical daily interactions or could cause sensory discomfort. Additionally, they discussed whether immediate reinforcement (e.g., the robot saying, "Good job!") should be provided after modeling or if maintaining a natural conversational flow would be more effective. Designing consistent teaching strategies and robot responses was deemed essential.

Fifth, the experts recommended focusing initially on "high-functioning autistic students who can use spoken language" while also exploring the feasibility of extending Echo-Teddy to students using AAC (augmentative and alternative communication) devices. Some students rely on picture icons or switch operations to express themselves instead of direct speech. Therefore, they emphasized that the robot should not be limited to processing verbal speech alone. If the robot could recognize and respond to electronic voices generated by AAC devices, it could benefit a broader spectrum of students.

Sixth, the experts stressed the importance of ensuring that interactions with the robot could generalize to real-life conversations with other humans. They cautioned that learning communication skills with the robot might have limited impact if not transferred to interactions with peers or family members. Skills like eye contact or body orientation were highlighted as "common competencies" that should be modeled by the robot and reinforced by caregivers or teachers. Practical considerations, such as the robot’s durability, hygiene, and connectivity, were also mentioned as essential factors for long-term use.

Finally, experts noted the financial constraints faced by special education programs, where the budget for equipment is often limited. They suggested that Echo-Teddy could be developed as an open-source project, allowing institutions to build DIY versions tailored to their specific needs. This approach could reduce costs and enable schools or organizations to customize the robot’s appearance, voice, and features to meet diverse sensory and preference requirements.


\section{Discussion}

The expert interviews provided key insights into refining Echo-Teddy’s design and functionality to better support social communication in autistic students. Experts highlighted the need for immediate responsiveness, a carefully designed physical and behavioral model, goal-oriented interaction strategies, and practical implementation considerations. Their feedback emphasized the importance of ensuring that Echo-Teddy not only facilitates meaningful interaction but also generalizes its impact to real-world social settings.

First, one major finding was the critical importance of real-time responsiveness in maintaining engagement for autistic students. Experts observed a 6–10 second processing delay due to cloud-based speech recognition and text-to-speech generation, which they noted could increase anxiety and disrupt interaction flow. Given that immediate and predictable feedback is essential for reinforcing communication skills, delays risk causing students to lose focus or disengage from the conversation. Cano et al. \cite{cano2023design} found that latency reductions in robotic responses not only enhance engagement but also mitigate behavioral disruptions, making this a critical area for refinement. Implementing edge computing strategies or optimizing speech processing pipelines may be essential for achieving the low-latency performance necessary to support effective and anxiety-free interactions for autistic students.

Second, the robot’s form and behavioral modules must balance the individual characteristics of autistic students with educational goals, particularly in distinguishing between modeling social behaviors and facilitating natural conversation. In inclusive classroom settings, incorporating subtle nonverbal cues such as head nods, body tilts, and gaze direction can encourage peer interaction by reinforcing appropriate social responses. However, ensuring design flexibility is crucial, as autistic students have diverse sensory preferences and interaction styles. Tailoring the robot’s activities, gestures, and verbal feedback based on user preferences and real-time adaptation is particularly important. Lee and Park  \cite{maroto2024personalizing} demonstrate that personalization enhances both engagement and intervention effectiveness, reinforcing the importance of customizable interaction strategies.Similarly, Cano et al. \cite{cano2023design} emphasize that user-centered design ensures robots remain effective across a spectrum of student needs, cautioning against one-size-fits-all approaches. For instance, while verbal reinforcement strategies (e.g., the robot saying "Good job!") can be effective for some students, others may find explicit praise stressful or disruptive. To prevent additional stress or confusion, reinforcement should be context-aware, ensuring that it aligns naturally with the conversation flow rather than feeling overly scripted or intrusive. This highlights the need for adaptive response mechanisms, allowing the robot to modulate its speech and gestures based on the child's individual comfort levels and engagement patterns.

Third, to broaden the scope of Echo-Teddy, ensuring scalability and generalizability is essential. Skills learned through robot-assisted interactions should seamlessly transfer to real-life settings, enabling autistic students to apply these communication skills with peers, caregivers, and teachers. Without this transferability, the benefits of robot-mediated interactions risk being isolated to controlled environments rather than supporting meaningful social engagement in daily life.  Choi et al \cite{santos2023applications} emphasize that scalable robot designs enable broader applicability across diverse educational and social settings, ensuring that interventions are not limited to small-scale experiments. A structured learning approach—beginning with caregiver-led activities and progressively expanding to peer interactions—has been shown to enhance social skill transfer beyond interactions with the robot itself. Clabaugh et al. \cite{clabaugh2019long} further highlight the value of long-term personalized interventions, demonstrating that sustained engagement reinforces skill retention and generalization. Such findings indicate that for Echo-Teddy to achieve wide-scale adoption, it must integrate Augmentative and Alternative Communication (AAC) devices, ensuring compatibility with students who rely on visual icons, text-based communication, or switch-access systems. Furthermore, the robot must be equipped with robust network capabilities to ensure seamless connectivity in classroom and home environments. Durable hardware construction is also critical, particularly for long-term educational use in varied settings. By addressing these factors, Echo-Teddy can bridge the gap between controlled intervention settings and real-world application, making it a scalable, adaptable, and impactful tool for special education.









\section{Discussion and Conclusion}

% \begin{quote}
% \textit{"We believe it is unethical for social workers not to learn... about technology-mediated social work."} (\citeauthor{singer_ai_2023}, 2023)
% \end{quote}

In this study, we uncovered multiple ways in which GenAI can be used in social service practice. While some concerns did arise, practitioners by and large seemed optimistic about the possibilities of such tools, and that these issues could be overcome. We note that while most participants found the tool useful, it was far from perfect in its outputs. This is not surprising, since it was powered by a generic LLM rather than one fine-tuned for social service case management. However, despite these inadequacies, our participants still found many uses for most of the tool's outputs. Many flaws pointed out by our participants related to highly contextualised, local knowledge. To tune an AI system for this would require large amounts of case files as training data; given the privacy concerns associated with using client data, this seems unlikely to happen in the near future. What our study shows, however, is that GenAI systems need not aim to be perfect to be useful to social service practitioners, and can instead serve as a complement to the critical "human touch" in social service.

We draw both inspiration and comparisons with prior work on AI in other settings. Studies on creative writing tools showed how the "uncertainty" \cite{wan2024felt} and "randomness" \cite{clark2018creative} of AI outputs aid creativity. Given the promise that our tool shows in aiding brainstorming and discussion, future social service studies could consider AI tools explicitly geared towards creativity - for instance, providing side-by-side displays of how a given case would fit into different theoretical frameworks, prompting users to compare, contrast, and adopt the best of each framework; or allowing users to play around with combining different intervention modalities to generate eclectic (i.e. multi-modal) interventions.

At the same time, the concept of supervision creates a different interaction paradigm to other uses of AI in brainstorming. Past work (e.g. \cite{shaer2024ai}) has explored the use of GenAI for ideation during brainstorming sessions, wherein all users present discuss the ideas generated by the system. With supervision in social service practice, however, there is a marked information and role asymmetry: supervisors may not have had the time to fully read up on their supervisee's case beforehand, yet have to provide guidance and help to the latter. We suggest that GenAI can serve a dual purpose of bringing supervisors up to speed quickly by summarising their supervisee's case data, while simultaneously generating a list of discussion and talking points that can improve the quality of supervision. Generalising, this interaction paradigm has promise in many other areas: senior doctors reviewing medical procedures with newer ones \cite{snowdon2017does} could use GenAI to generate questions about critical parts of a procedure to ask the latter, confirming they have been correctly understood or executed; game studio directors could quickly summarise key developmental pipeline concerns to raise at meetings and ensure the team is on track; even in academia, advisors involved in rather too many projects to keep track of could quickly summarise each graduate student's projects and identify potential concerns to address at their next meeting.

In closing, we are optimistic about the potential for GenAI to significantly enhance social service practice and the quality of care to clients. Future studies could focus on 1) longitudinal investigations into the long-term impact of GenAI on practitioner skills, client outcomes, and organisational workflows, and 2) optimising workflows to best integrate GenAI into casework and supervision, understanding where best to harness the speed and creativity of such systems in harmony with the experience and skills of practitioners at all levels.

% GenAI here thus serves as a tool that supervisors can use before rather then using the session, taking just a few minutes of their time to generate a list of discussion points with their supervisees.

% Traditional brainstorming comes up with new things that users discuss. In supervision, supervisors can use AI to more efficiently generate talking points with their supervisees. These are generally not novel ideas, since an experienced worker would be able to come up with these on their own. However, the interesting and novel use of AI here is in its use as a preparation tool, efficiently generating talking and discussion points, saving supervisors' time in preparing for a session, while still serving as a brainstorming tool during the session itself.

% The idea of embracing imperfect AI echoes the findings of \citeauthor{bossen2023batman} (2023) in a clinical decision setting, which examined the successful implementation of an "error-prone but useful AI tool". This study frames human-AI collaboration as "Batman and Robin", where AI is a useful but ultimately less skilled sidekick that plays second fiddle to Batman. This is similar to \citeauthor{yang2019unremarkable}'s (2019) idea of "unremarkable AI", systems designed to be unobtrusive and only visible to the user when they add some value. As compared to \citeauthor{bossen2023batman}, however, we see fewer instances of our AI system producing errors, and more examples of it providing learning and collaborative opportunities and other new use cases. We build on the idea of "complementary performance" \cite{bansal2021does}, which discusses how the unique expertise of AI enhances human decision-making performance beyond what humans can achieve alone. Beyond decision-making, GenAI can now enable "complementary work patterns", where the nature of its outputs enables humans to carry out their work in entirely new ways. Our study suggests that rather being a sidekick - Robin - AI is growing into the role of a "second Batman" or "AI-Batman": an entity with distinct abilities and expertise from humans, and that contributes in its own unique way. There is certainly still a time and place for unremarkable AI, but exploring uses beyond that paradigm uncovers entirely new areas of system design.

% % \cite{gero2022sparks} found AI to be useful for science writers to translate ideas already in their head into words, and to provide new perspectives to spark further inspiration. \textit{But how is ours different from theirs?}

% \subsection{New Avenues of Human-AI Collaboration}
% \label{subsubsec:discussionhaicollaboration}

% Past HCI literature in other areas \cite{nah2023generative} has suggested that GenAI represents a "leap" \cite{singh2023hide} in human-AI collaboration, 
% % Even when an AI system sometimes produces irrelevant outputs, it can still provide users 
% % Such systems have been proposed as ways to 
% helping users discover new viewpoints \cite{singh2023hide}, scour existing literature to suggest new hypotheses 
% \cite{cascella2023evaluating} and answer questions \cite{biswas2023role}, stimulate their cognitive processes \cite{memmert2023towards}, and overcome "writer's block" \cite{singh2023hide, cooper2023examining} (particularly relevant to SSPs and the vast amount of writing required of them). Our study finds promise for AI to help SSPs in all of these areas. By nature of being more verbose and capable of generating large amounts of content, GenAI seems to create a new way in which AI can complement human work and expertise. Our system, as LLMs tend to do, produced a lot of "bullshit" (S6) \cite{frankfurt2005bullshit} - superficially true statements that were often only "tangentially related" and "devoid of meaning" \cite{halloran2023ai}. Yet, many participants cited the page-long analyses and detailed multi-step intervention plans generated by the AI system to be a good starting point for further discussion, both to better conceptualize a particular case and to facilitate general worker growth and development. Almost like throwing mud at a wall to see what sticks, GenAI can quickly produce a long list of ideas or information, before the worker glances through it and quickly identifies the more interesting points to discuss. Playing the proposed role as a "scaffold" for further work \cite{cooper2023examining}, GenAI, literally, generates new opportunities for novel and more effective processes and perspectives that previous systems (e.g., PRMs) could not. This represents an entirely new mode of human-AI collaboration.
% % This represents a new mode of collaboration not possible with the largely quantitative AI models (like PRMs) of the past.

% Our work therefore supports and extends prior research that have postulated the the potential of AI's shifting roles from decision-maker to human-supporter \cite{wang_human-human_2020}. \citeauthor{siemon2022elaborating} (2022) suggests the role of AI as a "creator" or "coordinator", rather than merely providing "process guidance" \cite{memmert2023towards} that does not contribute to brainstorming. Similarly, \citeauthor{memmert2023towards} (2023) propose GenAI as a step forward from providing meta-level process guidance (i.e. facilitating user tasks) to actively contributing content and aiding brainstorming. We suggest that beyond content-support, AI can even create new work processes that were not possible without GenAI. In this sense, AI has come full circle, becoming a "meta-facilitator".

% % --- WIP BELOW ---

% % Our work echoes and extends previous research on HAI collaboration in tasks requiring a human touch. \cite{gero2023social} found AI to be a safe space for creative writers to bounce ideas off of and document their inner thoughts. \cite{dhillon2024shaping} reference the idea of appropriate scaffolding in argumentative writing, where the user is providing with guidance appropriate for their competency level, and also warns of decreased satisfaction and ownership from AI use. 

% Separately, we draw parallels with the field of creative writing, where HAI collaboration has been extensively researched. Writers note the "irreducibly human" aspect of creativity in writing \cite{gero2023social}, similar to the "human touch" core to social service practice (D1); both groups therefore expressed few concerns about AI taking over core aspects of their jobs. Another interesting parallel was how writers often appreciated the "uncertainty" \cite{wan2024felt} and "randomness" \cite{clark2018creative} of AI systems, which served as a source of inspiration. This echoes the idea of "imperfect AI" "expanding [the] perspective[s]" (S4) of our participants when they simply skimmed through what the AI produced. \cite{wan2024felt} cited how the "duality of uncertainty in the creativity process advances the exploration of the imperfection of GenAI models". While social service work is not typically regarded as "creative", practitioners nonetheless go through processes of ideation and iteration while formulating a case. Our study showed hints of how AI can help with various forms of ideation, but, drawing inspiration from creative writing tools, future studies could consider designs more explicitly geared towards creativity - for instance, by attempting to fit a given case into a number of different theories or modalities, and displaying them together for the user to consider. While many of these assessments may be imperfect or even downnright nonsensical, they may contain valuable ideas and new angles on viewing the case that the practitioner can integrate into their own assessment.

% % \cite{foong2024designing}, describing the design of caregiver-facing values elicitation tools, cites the "twin scenarios" that caregivers face - private use, where they might use a tool to discover their patient's values, and collaborative use, where they discuss the resulting values with other parties close to the patient. This closely mirrors how SSPs in our study reference both individual and collaborative uses of our tool. Unlike in \cite{foong2024designing}, however, we do not see a resulting need to design a "staged approach" with distinct interface features for both stages.

% % --- END OF WIP ---

% Having mentioned algorithm aversion previously, we also make a quick point here on the other end of the spectrum - automation bias, or blind trust in an automated system \cite{brown2019toward}. LLMs risk being perceived as an "ultimate epistemic authority" \cite{cooper2023examining} due to their detailed, life-like outputs. While automation bias has been studied in many contexts, including in the social sector or adjacent areas, we suggest that the very nature of GenAI systems fundamentally inhibits automation bias. The tendency of GenAI to produce verbose, lengthy explanations prompts users to read and think through the machine's judgement before accepting it, bringing up opportunities to disagree with the machine's opinion. This guards against blind acceptance of the system's recommendations, particularly in the culture of a social work agency where constant dialogue - including discussing AI-produced work - is the norm.


% % : Perception of AI in Social Service Work ??

% \subsection{Redefining the Boundary}
% \label{subsubsec:discussiontheoretical}

% As \citeauthor{meilvang_working_2023} (2023) describes, the social service profession has sought to distance itself from comprising mostly "administrative work" \cite{abbott2016boundaries}, and workers have long tried to tried to reduce their considerable time \cite{socialraadgiverforening2010notat} spent on such tasks in favour of actual casework with clients \cite{toren1972social}. Our study, however, suggests a blurring of the line between "manual" administrative tasks and "mental" casework that draws on practitioner expertise. Many tasks our participants cited involve elements of both: for instance, documenting a case recording requires selecting only the relevant information to include, and planning an intervention can be an iterative process of drafting a plan and discussing it with colleagues and superiors. This all stems from the fact that GenAI can produce virtually any document required by the user, but this document almost always requires revision under a watchful human eye.

% \citeauthor{meilvang_working_2023} (2023) also describes a more recent shift in the perceived accepted boundary of AI interventions in social service work. From "defending [the] jurisdiction [of social service work] against artificial intelligence" in the early days of PRM and other statistical assessment tools, the community has started to embrace AI as a "decision-support ... element in the assessment process". Our study concurs and frames GenAI as a source of information that can be used to support and qualify the assessments of SSPs \cite{meilvang_working_2023}, but suggests that we can take a step further: AI can be viewed as a \textit{facilitator} rather than just a supporter. GenAI can facilitate a wide range of discussions that promote efficiency, encourage worker learning and growth, and ultimately enhance client outcomes. This entails a much larger scope of AI use, where practitioners use the information provided by AI in a range of new scenarios. 

% Taken together, these suggest a new focus for boundary work and, more broadly, HCI research. GAI can play a role not just in menial documentation or decision-support, but can be deeply ingrained into every facet of the social service workflow to open new opportunities for worker growth, workflow optimisation, and ultimately improved client outcomes. Future research can therefore investigate the deeper, organisational-level effects of these new uses of AI, and their resulting impact on the role of profession discretion in effective social service work.

% % MH: oh i feel this paragraph is quite new to me! Could we elaborate this more, and truncate the first two paragraphs a bit to adjust the word propotion?


% % Our study extensively documents this for the first time in social service practice, and in the process reveals new insights about how AI can play such a role.



% \subsection{Design Implications}

% % Add link from ACE diagram?

% % EJ: it would be interesting to discuss how LLMs could help "hands-on experience" in the discussion section

% Addressing the struggle of integrating AI amidst the tension between machine assessment and expert judgement, we reframe AI as an \textit{facilitator} rather than an algorithm or decision-support tool, alleviating many concerns about trust and explainablity. We now present a high-level framework (Figure \ref{fig:hai-collaboration}) on human-AI collaboration, presenting a new perspective on designing effective AI systems that can be applied to both the social service sector and beyond.

% \begin{figure}
%     \centering
%     \includegraphics[scale=0.15]{images/designframework.png}
%     \caption{Framework for Human-AI Collaboration}
%     \label{fig:hai-collaboration}
%     \Description{An image showing our framework for Human-AI Collaboration. It shows that as stakeholder level increases from junior to senior, the directness of use shifts from co-creation to provision.}
% \end{figure}
% % MH: so this paradigm is proposed by us? I wonder if this could a part of results as well..?

% % \subsubsection{From Creation to Provision}

% In Section \ref{sec:stage2findings}, we uncovered the different ways in which SSPs of varying seniorities use, evaluate, and suggest uses of AI. These are intrinsically tied to the perspectives and levels of expertise that each stakeholder possesses. We therefore position the role of AI along the scale of \textit{creation} to \textit{provision}. 

% With junior workers, we recommend \textbf{designing tools for co-creation}: systems that aid the least experienced workers in creating the required deliverables for their work. Rather than \textit{telling} workers what to do - a difficult task in any case given the complexity of social work solutions - AI systems should instead \textit{co-create} deliverables required of these workers. These encompass the multitude of use cases that junior workers found useful: creating reports, suggesting perspectives from which to formulate a case, and providing a starting template for possible intervention plans. Notably, since AI outputs are not perfect, we emphasise the "co" in "co-creation": AI should only be a part of the workflow that also includes active engagement on the part of the SSPs and proactive discussion with supervisors. 

% For more experienced SSPs, we recommend \textbf{designing tools for provision}. Again, this is not the mere provision of recommendations or courses of action with clients, but rather that of resources which complement the needs of workers with greater responsibilities. This notably includes supplying materials to aid with supervision, a novel use case that to our knowledge has not surfaced in previous literature. In addition, senior workers also benefit greatly from manual tasks such as routine report writing and data processing. Since these workers are more experienced and can better spot inaccuracies in AI output, we suggest that AI can "provide" a more finished product that requires less vetting and corrections, and which can be used more directly as part of required deliverables.

% % MH: can we seperate here? above is about the guidance to paradigm, below is the practical roadmap for implementation
% In terms of concrete design features, given the constant focus on discussing AI outputs between colleagues in our FGDs, we recommend that AI tools, particularly those for junior workers, \textbf{include collaborative features} that facilitate feedback and idea sharing between users. We also suggest that designers work closely with domain experts (i.e. social work practitioners and agencies) to identify areas where the given AI model tends to make more mistakes, and to build in features that \textbf{highlight potential mistakes or inadequacies} in the AI's output to facilitate further discussion and avoid workers adopting suboptimal suggestions. 

% We also point out a fundamental difference between GenAI systems and previous systems: that GenAI can now play an important role in aiding users \textit{regardless of its flaws}. The nature of GenAI means that it promotes discussion and opens up new workflows by nature of its verbose and potentially incomplete outputs. Rather than working towards more accurate or explainable outcomes, which may in any case have minimal improvement on worker outcomes \cite{li2024advanced}, designers can also focus on \textbf{understanding how GenAI outputs can augment existing user flows and create new ones}.

% % for more senior workers...

% % how to differentiate levels of workers?

% % \subsubsection{Provider}

% % The most basic and obvious role of modern AI that we identify leverages the main strength of LLMs. They have the ability to produce high-quality writing from short, point-form, or otherwise messy and disjoint case notes that user often have \textit{[cite participant here]}. 

% Finally, given the limited expertise of many workers at using AI, it is important that systems \textbf{explicitly guide users to the features they need}, rather than simply relying on the ability of GAI to understand complex user instructions. For example, in the case of flexibility in use cases (Section \ref{subsubsec:control}), systems should include user flows that help combine multiple intervention and assessment modalities in order to directly meet the needs of workers.

% \subsection{Limitations and Future Work}

% While we attempt to mimic a contextual inquiry and work environment in our study design, there is no substitute for real data from actual system deployment. The use of an AI system in day-to-day work could reveal a different set of insights. Future studies could in particular study how the longitudinal context of how user attitudes, behaviours, preferences, and work outputs change with extended use of AI. 

% While we tried to include practitioners from different agencies, roles, and seniorities, social service practice may differ culturally or procedurally in other agencies or countries. Future studies could investigate different kinds of social service agencies and in different cultures to see if AI is similarly useful there.

% As the study was conducted in a country with relatively high technology literacy, participants naturally had a higher baseline understanding and acceptance of AI and other computer systems. However, we emphasise that our findings are not contingent on this - rather, we suggest that our proposed lens of viewing AI in the social sector is a means for engaging in relevant stakeholders and ensuring the effective design and implementation of AI in the social sector, regardless of how participants feel about AI to begin with. 



% % \subsection{Notes}

% % 1) safety and risks and 2) privacy - what does the emphasis on this say about a) design recommendations and b) approach to designing/PD of such systems?


% % W9 was presented with "Strengths" and "SFBT" output options. They commented, "solution focus is always building on the person's strengths". W9 therefore requested being able to output strengths and SFBT at the same time. But this would suggest that the SFBT output does not currently emphasise strengths strongly enough. However, W9 did not specifically evaluate that, and only made this comment because they saw the "strengths" option available, and in their head, strengths are key to SFBT.
% % What does this say about system design and UI in relation to user mental models?


\begin{acks}

% This work is supported by the Ministry of Education, Singapore (A-8002610-00-00) and the National University of Singapore Start-up Grant Award No. R-124-000-128-133.
% This research work is partially supported by NUS IT’s Research Computing group (NUSREC-HPC-00001) and AWS.
% We thank all reviewers' comments and suggestions to help polish this paper.
Our research has been supported by the Ministry of Education, Singapore (A-8002610-00-00) and the National University of Singapore Start-up Grant Award (R-124-000-128-133). 
This research work is partially supported by the NUS IT's Research Computing group (NUSREC-HPC-00001) and AWS. 
We are grateful to all the reviewers for their valuable comments and suggestions that helped improve this paper, and to all the participants whose time and efforts made this research possible.
\end{acks}


\bibliographystyle{ACM-Reference-Format}
\bibliography{sample-base}

%%
%% If your work has an appendix, this is the place to put it.

%%
%% The acknowledgments section is defined using the "acks" environment
%% (and NOT an unnumbered section). This ensures the proper
%% identification of the section in the article metadata, and the
%% consistent spelling of the heading.



%%
%% The next two lines define the bibliography style to be used, and
%% the bibliography file.


\end{document}
\endinput
%%
%% End of file `sample-acmsmall.tex'.
