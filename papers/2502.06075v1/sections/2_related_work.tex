\section{Related Work}

\subsection{AI Applications in Mental Health}

AI has been increasingly integrated into digital mental-health care delivery \cite{ai_mh_ma_2023, ai_mh_ma_2024}, offering diverse applications spanning psychotherapy \cite{therapy_ai_prochaska_2021}, psychoeducation \cite{ai_psychoedu_jang_2021}, and social companionship \cite{ai_mh_ma_2023}.
Modern AI systems have demonstrated remarkable capabilities in psychiatric care through their ability to engage in naturalistic, human-like therapeutic interactions \cite{ai_mh_jo_2023}.
Functioning as both clinical tools and digital companions, these AI systems offer multiple benefits, including empathic communication, non-judgmental responses, regular check-ins, and tailored feedback \cite{ai_mh_li_2023}.

However, despite the promising potential of AI to promote mental well-being through improved accessibility and diverse support mechanisms, \textit{social stigma} remains a primary barrier that may prevent people from seeking help through these digital systems, just as it has historically deterred people from accessing traditional mental health services \cite{ai_mh_hoffman_2024}.
This persistent challenge of stigma therefore demands careful attention as we continue to develop and design AI-based mental-health interventions.



\subsection{Mental-illness Stigma}

Social stigma, as originally defined by Goffman, includes regarding mental illness as divergent from what society considers normal and correct, and mentally ill individuals as tainted and devalued \cite{stigma_spoiled_identity_goffman_1964}. 
It stems from stereotypes and prejudices \cite{attribution_model_corrigan_2003} that frequently manifest as unconscious bias, i.e., negative attitudes or cognition that can sway people's decisions without their awareness \cite{microaggression_stigma_gonzales_2015}. 
Social stigmatization of people with mental illness is very prevalent worldwide \cite{measuing_stigma_corrigan_2010}. 
The pervasiveness of public stigma impedes help-seeking, treatment-seeking, and recovery while also exacerbating historical injustices \cite{decolonial_pendse_2022} and creating obstacles to employment, housing, and social connections \cite{understanding_corrigan_2002}. 
Consequently, only about 50\% of people with depression, for example, seek treatment \cite{depression_ratio_kessler_2003}, and many discontinue it to avoid being labeled psychiatrically \cite{depression_seeking_dew_1988}. 
Understanding and reducing mental-illness stigma is therefore crucial to social welfare and the promotion of inclusive attitudes.


The \textit{attribution model} \cite{attribution_model_corrigan_2003} is a theoretical framework that has guided many explorations of the factors contributing to mental-illness stigma. 
It includes three dimensions – \textit{personal responsibility}, \textit{emotional responses}, and \textit{behavioral responses} – and holds that individuals' perceptions of the extent to which a person with a mental illness is responsible for their condition can lead to stigmatizing emotional responses, such as anger, fear, and lack of pity. 
Such responses may then elicit discriminatory behaviors including coercive segregation, social distance, and the withdrawal of help. 
The attribution model serves as a solid foundation for stigma measurements and forms the theoretical basis for our study's deconstruction of stigma.


Virtual reality \cite{realtime_martinez_2024}, videogames \cite{personalized_anvari_2024}, and social-media campaigns \cite{reduce_sm_feuston_2019} have all shown promise for reducing mental-illness stigma by raising awareness and/or facilitating positive social contact. 
Chatbots have been effective at delivering such interventions, in part because they offer anonymity and encourage self-disclosure \cite{chatbot_reduce_kim_2020, disclosure_lee_2022, chatbot_aq27_practice_lee_2023}. 
However, many existing anti-stigma approaches adopt a one-size-fits-all strategy that is likely to ignore how stigma manifests differently across social contexts, potentially limiting their efficacy or even causing them to backfire \cite{backfire_dobson_2022}.



\subsection{Measuring and Deciphering Mental-illness Stigma}

To develop any successful anti-stigma initiative, one must identify the factors and mechanisms that create and sustain the targeted stigma \cite{vignette_link_1987}.
% Successful anti-stigma initiatives require identifying the factors and mechanisms that sustain the targeted stigma \cite{vignette_link_1987}. % I can go with this! but just retain the version f
This section reviews three approaches to deciphering mental-illness stigma: traditional quantitative methods, computational techniques, and traditional qualitative analysis.


\subsubsection{Traditional Quantitative Methods} % truncate a bit


Quantitative-scale protocols have been widely used to assess various components of mental-illness stigma, including behavior \cite{vignette_link_1987}, stereotyping \cite{stereotyping_bedini_2000}, cognitive separation \cite{cognitive_ostma_2002}, emotional reactions \cite{affective_vezzoli_2001}, and status loss and discrimination \cite{status_loss_secker_1999}. 
One notable instrument is the \textit{Attributional Questionnaire} \cite{attribution_model_corrigan_2003}, a causality tool that helps to unravel the genesis and perpetuation of stigma by assessing the key constructs defined in Corrigan et al.'s \cite{attribution_theory_corrigan_2000} social-cognitive models.
These tools have greatly facilitated the understanding of mental-illness stigma.


Despite the widespread use of standardized self-report surveys to measure various constituents of mental-illness stigma, such instruments might under-represent the complex lived experience of stigmatizing and being stigmatized, the mechanisms of stigma's progression, or other societal aspects of these processes \cite{interview_stigma_measure_liggins_2005}. 
Moreover, this measurement approach is susceptible to social-desirability bias \cite{vignette_link_1987}. 
That is, anti-stigma campaigns \cite{campaign_corrigan_2001} and other public-education efforts \cite{intervention_corrigan_1999} have emphasized the importance of not rejecting individuals simply because they have sought mental-health treatment. 
As a result, even if they privately hold stigmatizing attitudes, people are unlikely to express them openly, because they want to appear enlightened and caring \cite{sd_scale_inaccurate_van_2008}. 
This dissimulation can lead scholars to underestimate true stigma levels and/or to misidentify genuine predictors of stigma \cite{sd_stigma_michaels_2013}. 
Thus, researchers should employ multiple approaches to decoding stigma \cite{vignette_link_2004}, develop measures that mitigate potential confounders, and explore modes of interpretation that better capture the intricacies of stigma.




\subsubsection{Computational Techniques}


Social-media platforms like Twitter and Weibo include masses of largely unfiltered qualitative data relevant to mental health \cite{social_media_bail_2017, social_media_pavlova_2020}. 
Leveraging these resources, some researchers have employed various computational techniques to analyze mental-illness stigma, studying conditions such as depression \cite{detect_method_li_2018}, schizophrenia \cite{detect_method_jilka_2022}, and substance use \cite{detect_method_roesler_2024}.
Specifically, Fang \& Zhu used linguistic-analysis tools such as LIWC \cite{detect_method_fang_2023}, while Mittal et al. developed frameworks that assess stigmatizing discourse in social media and news content by creating custom dictionaries and calculating word-embedding similarities \cite{detect_method_mittal_2023}.



Machine-learning (ML) and deep-learning (DL) solutions have also contributed to this field, by enabling real-time, automated mental-illness stigma detection and classification within large social-media datasets \cite{detect_method_robinson_2019}.
These approaches range from binary 'present/ absent' classification of such stigma using algorithms such as SVM, Random Forest \cite{detect_method_jilka_2022}, and BERT \cite{detect_method_lee_2022} to more sophisticated models that use $n$-grams to differentiate among types of stigma \cite{detect_method_li_2018} or types of stigma-related language, e.g., metaphors, personal stories, ridicule, and jokes \cite{detect_method_roesler_2024}.



ML and DL methods can efficiently and rapidly classify large volumes of social-media content pertinent to stigma. 
However, social-media posts provide relatively brief snippets of stigma-related discussions, and their decontextualized aggregation into datasets may fail to capture the full panorama of stigma-related interactions and their evolving dynamics \cite{social_media_decontext_boyd_2012}. 
Additionally, social-media data suffer from population biases \cite{social_media_deidentify_ruths_2014} that vary across platforms, potentially skewing interpretation. 
More importantly, these approaches mainly focus on classifying and/or quantifying words and phrases, and thus are likely to oversimplify the psychological underpinnings of stigmatizing behavior while largely ignoring its processes and triggers. 
Given these drawbacks, there is a pressing need for a more process-oriented analysis of stigma toward people suffering from mental illnesses in more context-rich settings.


 
\subsubsection{Traditional Qualitative Analysis}


Human narratives are rife with psychological constructs, perceptions, and reasoning about societal topics, and act as catalysts for deep reflection and genuine free exchange of ideas \cite{conversation_importance_jenlink_2005}. 
Interview data contain valuable information about how social problems are perpetuated, unfolded, or negotiated \cite{conversation_importance_jenlink_2005}, reflecting the endogenous organization of social activities \cite{conversation_banathy_2005}. 
Such conversations naturally create a space and medium for inclusion, connection, and self-disclosure, allowing interviewers to uncover the underlying beliefs, attitudes, and experiences that shape people's perspectives and may lead to evolving consciousness \cite{conversation_bohm_2004, conversation_banathy_2005}.

Specifically, interviews can capture subjective experience/un-derstanding of their focal phenomenon and the complex social systems that give rise to it \cite{interview_stigma_measure_liggins_2005, interview_stigma_measure_lyons_1995}.
Like participant observation, interviewing can thus provide insiders' views of stigma's dynamics and impacts that rating scales or analysis of social-media data could overlook. 
For instance, Lyons and Ziviani \cite{interview_stigma_measure_lyons_1995} effectively used interviews and participant observation to track changes in stereotypical beliefs and preconceptions about the mentally ill. 
Liggins and Hatcher \cite{interview_stigma_measure_liggins_2005} likewise applied interviews to reveal how people express fear and hopelessness toward mental illness, and showcased the depth of the insights that qualitative analysis makes possible.



Although data from interviews can facilitate a deep understanding of social phenomena, they also pose major challenges, as collecting, coding, categorizing, and interpreting them requires substantial time, effort, and expertise \cite{human_coding_bias_leeson_2019} as well as strong theoretical knowledge \cite{coding_manual_saldana_2016}. 
AI's eloquence in processing natural language makes it adept at collecting interview data \cite{chatbot_kim_2019}, and its perceptiveness may enable it to streamline the analysis process while preserving the fertility of insights \cite{deductive_labelling_xiao_2023}. 
Yet, explorations of their ability to profile stigma based on interview data remain rare.




\subsection{AI-assisted Qualitative-data Collection and Analysis} % be a seperated paragraph


Chatbots \cite{chatbot_hoque_2024} offer a promising approach to interview-data collection for exploratory research. 
They can provide a middle ground between questionnaires and traditional in-person interviews, offering a more engaging experience \cite{chatbot_kim_2019} than the former while being more scalable than the latter. 
In particular, AI-powered chatbots are flexible, fast, and adaptable tools that can gather rich qualitative insights into complex social and psychological phenomena by combining gregarious interactivity with wide reach \cite{chatbot_xiao_2020}. 
Moreover, AI can create materials tailored to fostering chatbot users' self-disclosure \cite{disclosure_lee_2022}.



While these data-collection methods provide rich qualitative information, deciphering and interpreting this wealth of data is equally crucial to unlocking its full potential. 
AI technologies are also revolutionizing qualitative-coding processes \cite{ai_qualitative_feuston_2021, ai_qualitative_muller_2016, ai_qualitative_rietz_2020}, mitigating manual methods' time and resource constraints. 
In deductive coding, AI has applied predefined codebooks to large datasets \cite{coding_deductive_llm_tai_2024, deductive_labelling_xiao_2023}, and achieved fair to substantial agreement with human-expert coders \cite{cohens_kappa_mchugh_2012, prompt_practice_dunivin_2024}. 
In inductive coding, meanwhile, AI can help researchers initialize codes, uncover themes, and synthesize essences from raw data \cite{content_analysis_coding_toolgpt_gao_2023}. 
Together, these AI-assisted approaches not only enhance efficiency, but also pave the way to the discovery of new constructs, patterns, and conceptualizations. 

% Therefore, we ask:



% \begin{quote}
% \textbf{RQ1: To what extent can AI-assisted data collection and qualitative-analysis methods effectively capture and identify mental-illness stigma?}

% \textit{RQ1.1}: How well does the conversational data collected through human-chatbot interactions capture mental-illness stigma?

% \textit{RQ1.2}: How consistent is AI-assisted coding of such data with human coding, and how does it compare to existing computational-analysis approaches when it comes to identifying the manifestation of mental-illness stigma in conversations?
% \end{quote}


% + research gap, why filling this gap is important

\subsection{Synergy between Large Language Models and Causal Knowledge Graphs}

Importantly, however, the complex interplay of contexts, relational dynamics, and reasoning chains within qualitative data – which is particularly crucial to understand when examining intricate societal phenomena – may extend beyond what any single descriptive label can convey. 
This points to the need for additional, integrative approaches that are both analytical and explanatory.


CKGs represent a powerful fusion of structured-information representation and intelligent reasoning, and can capture both overt and covert relationships between entities in a domain \cite{kg_carta_2023}.
Indeed, they excel at integrating information from diverse sources, uncovering hidden patterns in it, and facilitating efficient navigation of complex knowledge landscapes \cite{ckg_uleman_2021, ckg_borsboom_2021}. 
Complementing CKGs, LLMs can access and articulate domain knowledge previously confined to human experts \cite{ckg_llm_kiciman_2024} and excel at lexical, syntactic, and semantic analysis and conceptual understanding.
The synergy between CKGs' structural prowess and LLMs' advanced linguistic capabilities \cite{ckg_llm_tong_2024} allows researchers to map out and explore intricate relationship networks \cite{kg_llm_pan_2024}. % llm+ckg
This synergistic approach echoes the holistic- vs. analytic-cognition dichotomy in social psychology \cite{human_psych_nisbett_2001}, and has been validated by prior studies showing its effectiveness in simulating bio-psycho-social interactions \cite{ckg_borsboom_2021, kg_psych_crielaard_2022}.
Such integration represents a promising new frontier in HCI and psychological research: not only enhancing researchers' ability to deduce relationships between factors, but also guiding the generation of latent novel causal hypotheses \cite{ckg_llm_tong_2024}. 


Having recognized the potential of interview data to elicit disclosure and promote self-reflection, the capabilities of LLMs to imitate interviewers and interpret semantic and linguistic cues, and the power of CKGs to represent intricate interrelationships, our aim is to leverage those elements synergistically to dissect stigma in an unprecedented way. 

% Our second research question is therefore:


% \begin{quote}
%     \textbf{RQ2: How well can the integration of CKG and LLM reveal the factors that drive mental-illness stigma and their interrelationships?}
% \end{quote}


