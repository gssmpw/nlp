\section{Results}

\begin{figure*}
    \centering
    \includegraphics[width=0.85\linewidth]{figs/figure_embedding.pdf}
    \caption{Two-dimensional principal component analysis (PCA) projection of word embeddings from participant messages. The words shown are the most frequent from the 200 $k$-means clusters, and circle sizes represent cluster frequencies. Colored arrows indicate weighted average vectors for different attributions, and word positioning reflects semantic similarity. Highlighted words near attribution arrows represent key terms closely associated with each stigma attribution.}
    \Description{Scatterplot showing a two-dimensional principal component analysis projection of word embeddings from text messages. In this plot, frequently occurring words appear as circles of different sizes, with larger circles indicating higher frequency from the 200 k-means clusters. Seven colored directional arrows cross the plot, each representing different stigma attributions: 'Responsibility,' 'Anger,' 'Pity,' 'Fear,' 'Helping,' 'Coercive Segregation,' and 'Social Distance.' The words are arranged so that semantically similar terms appear closer together. Near each attribution arrow, key related terms are highlighted with boxes.}
    \label{fig:datacollection}
\end{figure*}

\subsection{Assessing AI-assisted Qualitative-data Collection and Analysis Methods for Capturing Depression Stigma (RQ1)}

To address RQ1, we examined two aspects of our AI-assisted pipeline: first, we evaluated the stigma-probing interview data collected by our AI-powered chatbot in terms of 1) data quality, and 2) participant experience with the interview; second, we evaluated the consistency between AI-assisted coding and human coding, while also comparing the agreement among our AI-assisted coding, other computational analytical approaches, and human-derived codes.

\subsubsection{Evaluating AI-assisted Collection of Stigma-related Data}



% To address RQ1, we first evaluated the stigma-probing interview data collected by our AI-powered chatbot in terms of 1) data quality, and 2) participant experience with the interview.

\paragraph{\textcolor{darkred}{\textbf{Data-quality Evaluation.}}}
We delineated our data along four dimensions based on Gricean Maxims, principles that guide effective communication: the maxim of quantity (\textbf{specificity}), quality (\textbf{self-disclosure}), relation (\textbf{relevance}), and manner (\textbf{clarity}) \cite{metric_grice_1975, chatbot_xiao_2020}. 
Specificity means that participants' responses are detailed and informative; self-disclosure indicates the authenticity and personal nature of the responses; relevance implies that the responses actually address the questions; and clarity facilitates accurate interpretation \cite{metric_grice_1975}.


% The data confirmed a high degree of chatbot-user engagement, with participants expressing a range of emotions, behavioral intentions, and cognitive reflections. For example, P608 shared:


We first measured the duration of chatbot interviews and the word counts of the participants' contributions to them (two indicators of specificity). 
The average of the former was $t=17.63$ minutes ($SD = 8.18$). 
The mean word counts per participant message, broken down by attribution, were 43.82 for \textit{responsibility} ($SD = 14.68$), 43.40 for \textit{social distance} ($SD = 15.49$), 41.50 for \textit{helping} ($SD = 14.20$), 40.86 for \textit{anger} ($SD = 13.45$), 40.18 for \textit{coercive segregation} ($SD = 13.70$), 39.74 for \textit{fear} ($SD = 14.14$) and 39.30 for \textit{pity} ($SD = 13.98$).


Participants disclosed themselves in their responses, expressing a range of emotions, behavioral intentions, and cognitive reflections. 
For example, P608 shared:

\begin{quote}
    \textit{Not scared of Avery, just concerned for their well-being. I might be concerned about whether or not they would show up and if they would be prepared, dressed appropriately, and such. I might also take on responsibility for booking things if I were worried they would not get that kind of task done.}
\end{quote}

% The presence in the above statement of positive terms like \textit{“compassionate”} and \textit{“helpful”} alongside negative ones like \textit{“frustrated”} and \textit{“dangerous”} underscores the fertility and complexity of the collected data.
The frequent use of "\textit{I}" alongside personal projections (e.g., "\textit{I might be concerned}", "\textit{I might also take on responsibility}") underscores the authenticity of the responses and the participants' deep engagement with the topic.



% Together with the high average word counts cited above, this suggested that the collected data was of relatively high quality.

The dimensional plot in Figure \ref{fig:datacollection} reveals words' frequencies and semantic similarities, as well as attribution-specific language patterns: e.g., "\textit{frustrating}," "\textit{embarrassment}," and "\textit{upset}" clustered near the \textit{anger} attribution, and "\textit{separate}" and "\textit{psychiatrist}" associated with \textit{coercive segregation}. 
This clustering of semantically related terms around specific stigma attributions shows clear topical alignment, suggesting their relevance.

\begin{figure}
    \centering
    \includegraphics[width=\linewidth]{figs/figure_heatmap.png}
    \caption{Heatmap showing the agreement between human-derived codes and AI-generated codes. 
    The numbers in each cell represent the frequency of consistency, and darker colors indicate closer agreement.}
    \Description{Heatmap showing the agreement between human-derived and artificial intelligence-generated codes for various stigma-related attributions. The x-axis shows the artificial intelligence-generated codes, while the y-axis shows the human-derived codes. Cell values represent the frequency of agreement, with darker shades indicating higher agreement.}
    \label{fig:heatmap}
\end{figure}



Finally, clarity was assessed by human coding of 4,200 messages from the randomly selected 600 participants. 
Only 1.12\% ($n=47$) were discarded for being incomplete, illegible, off-topic, or excessively brief (<5 words). 


\paragraph{\textcolor{darkred}{\textbf{Participant-experience Evaluation.}}}
The 1,002 participants' average response to a five-point Likert-scaled question about their satisfaction with the chatbot interview was 4.37 ($SD = 0.87$), indicating highly positive sentiment toward that experience. 
Among the nearly half of participants who volunteered comments at the end of the chatbot survey ($n=452$), many described it as enjoyable and/or one of their best survey experiences. 
Participants also often highlighted the \textit{thought-provoking} nature of the experience. 
For instance, P607 noted:


\begin{quote}
    \textit{Interacting with the chatbot made me stop and consider my answers, rather than just agree/disagree. It made me examine my actual reactions to a situation when the simple categories did not accurately reflect my concern over what I consider gray areas.}
\end{quote}




\subsubsection{Comparing AI-assisted Coding against Human Coding and Existing Computational Approaches}


% To answer RQ1(b), we evaluated the consistency between AI-assisted coding and human coding, while also comparing the agreement among our AI-assisted coding, other computational analytical approaches, and human-derived codes.



\paragraph{\textcolor{darkred}{\textbf{Consistency between AI-assisted Coding and Human Coding.}}}

Of the 4,153 human-coded messages, 46.01\% ($n=1,911$) included stigmatization.
Among the \textit{stigmatizing} codes, the most prevalent was \textit{responsibility}, accounting for 9.51\% ($n=395$). 
\textit{Social distance} was a close second at 9.13\% ($n=379$), while \textit{fear} made up 8.84\% ($n=367$). 
\textit{Anger} and \textit{coercive segregation} constituted 7.18\% ($n=298$) and 6.55\% ($n=272$) of the codes, respectively. 
The least-frequent \textit{stigmatizing} codes were \textit{helping}, at 3.80\% ($n=158$), and \textit{pity}, at 1.01\% ($n=42$). 


We calculated Cohen's $\kappa$ to gauge the agreement between human and AI-assisted coding. 
As shown in Figure \ref{fig:heatmap}, we found that it varied across stigma attributions, with \textit{social distance} showing the highest agreement ($\kappa=0.76$), followed by \textit{fear} ($\kappa=0.71$). 
\textit{Responsibility} and \textit{helping} showed similar levels of agreement ($\kappa=0.69$), while \textit{anger} ($\kappa=0.65$), \textit{coercive segregation} ($\kappa=0.54$), and \textit{pity} ($\kappa=0.46$) showed comparatively lower coding consistency.
The overall $\kappa$ across all 4,153 human-coded messages was \textbf{0.69}, indicating a satisfactory level of agreement \cite{cohens_kappa_mchugh_2012} between our AI-assisted and human coding. 
Further, our validation on the 200 previously human-uncoded messages achieved \textbf{Cohen's $\kappa$ = 0.87}, suggesting that our AI-coding approach was not only reliable across the entire dataset but could also be useful for real-time stigma identification given its consistency with human coding.

\begin{figure}
    \centering
    \includegraphics[width=\linewidth]{figs/figure_computational.png}
    \caption{Comparison of human-AI agreement (Cohen's $\kappa$) across different models, showing statistically significantly higher agreement for our AI-assisted coding ($\kappa = 0.781 \pm 0.028$) compared to \texttt{RoBERTa-base} ($\kappa = 0.677 \pm 0.032$) and \texttt{BERTweet-base} ($\kappa = 0.619 \pm 0.033$). Statistical significance levels are marked as p $\geq$ 0.05 (ns), p < 0.05 (*), p < 0.01 (**), or p < 0.001 (***).}
    \Description{Bar chart comparing Cohen's kappa values, which measure the agreement between humans and artificial intelligence, for three different models. The y-axis displays Cohen's kappa scores. The visualization shows three bars: RoBERTa-base with a kappa value of 0.677 plus or minus 0.032, BERTweet-base with a kappa value of 0.619 plus or minus 0.033, and our approach with the highest agreement with a kappa value of 0.781 plus or minus 0.028. The difference between our method and RoBERTa is significant at p < 0.001, as is the difference between our method and BERTweet. The difference between RoBERTa and BERTweet is significant at p = 0.003.}
    \label{fig:computational}
\end{figure}



\paragraph{\textcolor{darkred}{\textbf{Comparison against Existing Computational Methods.}}}

We compared the agreement between human coding, on the one hand, and on the other, 1) our method and 2) the fine-tuned \texttt{RoBERTa-base} \cite{roberta_liu_2019} and \texttt{BERTweet-base} \cite{bertweet_nguyen_2020}, on the \texttt{RoBERTa-base} and \texttt{BERTweet -base} test set of 829 messages (446 human-coded as \textit{non-stigmatizing}, 79 as \textit{responsibility}, 76 as \textit{social distance}, 73 as \textit{fear}, 60 as \textit{anger}, 54 as \textit{coercive segregation}, 32 as \textit{helping}, and 9 as \textit{pity}). 
Our AI-assisted coding yielded higher agreement with human coding ($\kappa = 0.78$) compared to both \texttt{RoBERTa-base} ($\kappa = 0.68$) and \texttt{BERTweet-base} ($\kappa = 0.62$).

A Cochran's Q test revealed significant differences between the models ($\chi^2(2) = 60.24$, $p < .001$) (Figure \ref{fig:computational}). 
Post-hoc analysis using pairwise McNemar tests with Bonferroni correction confirmed significant differences in agreement with human coding between our approach and both \texttt{RoBERTa-base} ($\chi^2(1) = 41.00$, $p < .001$) and \texttt{BERTweet-base} ($\chi^2(1) = 36.00$, $p < .001$).
Between the baseline models, \texttt{RoBERTa-base} showed higher consistency with human coding than \texttt{BERTweet-base} ($\chi^2(1) = 38.00$, $p = .003$).
Detailed significance tests for each code can be found in the \textit{Supplementary Materials}.

% Our AI-assisted approach demonstrated strong agreement with human coding ($\kappa = 0.78$), surpassing both \texttt{RoBERTa-base} ($\kappa = 0.68$) and \texttt{BERTweet-base} ($\kappa = 0.62$). 
% A Cochran's Q test revealed significant differences between the models ($\chi^2(2) = 60.24$, $p < .001$) (Figure \ref{fig:computational}). 
% Post-hoc analysis using pairwise McNemar tests with Bonferroni correction confirmed the effectiveness of our approach, showing statistically significant improvements over both \texttt{RoBERTa-base} ($\chi^2(1) = 41.00$, $p < .001$) and \texttt{BERTweet-base} ($\chi^2(1) = 36.00$, $p < .001$).





\subsection{Integrating CKG with LLM to Uncover Interrelationships between Constructs Underlying Depression Stigma (RQ2)} 

We approached RQ2, regarding the potential of CKG and LLM to synergistically elucidate factors that influence stigmatizing attitudes and discriminatory behaviors, by building CKGs and analyzing the semantics, relationships, and constructs derived from our interview dataset.



\subsubsection{Causal Knowledge Graph Construction}



Our CKG, comprising 13,434 unique entities (reduced from 24,201 entities through entity resolution) interlinked by 18,875 relationships, provides a rich interrelational network of stigma constructs and causalities.




\paragraph{\textcolor{darkred}{\textbf{Constructs Assigned to Entities.}}}


Besides two predefined entities ("\textit{stigma}" and "\textit{no stigma}"), all entities in our dataset were mapped into 11 theoretical constructs.


Specifically, four of these constructs were \textbf{theory-driven}, derived \textbf{deductively} from Corrigan et al.'s attribution theory \cite{attribution_theory_corrigan_2000, theory_overview_corrigan_2002}: \textit{signaling event}, \textit{cognitive judgment}, \textit{emotional response}, and \textit{behavioral intention}.
\textit{Signaling event} ($n$ = 1,365) represented symptomatic behaviors and emotional characteristics of the person with depression depicted in the vignette (e.g., \textit{"Avery feeling judged by others"}, P166), whereas \textit{cognitive judgment} ($n$ = 1,505) captured participants' cognitive evaluation and/or appraisal of that person (\textit{"they are not dangerous"}, P360). 
\textit{Emotional response} ($n$ = 1,017) denoted affective reactions, feelings, and sentiments experienced by participants in response to the vignette (\textit{"feel embarrassed by Avery"}, P357). 
\textit{Behavioral intention} ($n$ = 1,607) indicated participants' anticipated behavioral responses or action tendencies toward the person in the vignette and/or people with depression in general (\textit{"ask them to leave"}, P370).


The remaining seven constructs emerged \textbf{inductively} through the \textbf{data-driven} conceptualization of participant messages: \textit{belief}, \textit{past experience}, \textit{personality}, \textit{situation}, \textit{potential outcome}, \textit{motivation}, and \textit{suggestion}.
% Besides the constructs proposed by attribution theory, several data-driven ones emerged from the ontologization process. 
\textit{Belief} \cite{belief_reviewer_peter_2021} ($n$ = 2,534) portrayed general knowledge, literacy, attitudes, or deep-seated views about mental health or human elements that participants possess (\textit{"brain is such a complex thing"}, P380). 
\textit{Past experience} \cite{factor_chandra_2007} ($n$ = 782) described participants' prior self-experiences, exposures, and interactions (\textit{"previously had a colleague who had narcissistic tendencies"}, P226). 
\textit{Personality} \cite{personality_reviewer_steiger_2022} ($n$ = 687) depicted participants' self-reported nature, individual traits, or dispositional characteristics (\textit{"I am easy-going"}, P657). 
\textit{Situation} \cite{situation_theory_rusch_2009} ($n$ = 575) delineated participants' immediate extrinsic environmental context (\textit{"I am out at work most of the time"}, P346). 
\textit{Potential outcome} ($n$ = 1,632) involved participants' anticipation of consequences or prognoses for the figure in the vignette (\textit{"Avery will go downhill"}, P325). 
\textit{Motivation} \cite{motivational_reviewer_kvaale_2016} ($n$ = 722) encapsulated participants' drives and what they were striving to achieve or avoid (\textit{"want a tenant who was more reliable"}, P339); and \textit{suggestion} ($n$ = 1,006), their proposed interventions or other recommendations for the figure in the vignette (\textit{"suggest meeting with a professional counselor"}, P524).



\paragraph{\textcolor{darkred}{\textbf{CKG Quality Assessment.}}}



Beyond the subjective evaluations of LLM-human agreement in triple extraction (accuracy = 0.93), ontologization (Cohen's $\kappa$ = 0.77), and entity resolution (Cohen's $\kappa$ = 0.90) reported in Section \ref{method:ckg}, we further assessed our CKG's quality using objective metrics adapted from knowledge-graph-evaluation frameworks \cite{kg_evaluation_koutsiana_2024}\footnote{We excluded the \textit{succinctness} metric, which refers to the CKG's conciseness, as edge repetitions in our context represent meaningful frequency of causal relationships mentioned by participants rather than redundancy.}.

\textbf{Coverage}, which refers to \textit{completeness} and \textit{representativeness}, was first evaluated, showing that our CKG captured much of the participants' reasoning and mental processes.
For \textit{entity} coverage, the text segments we extracted as entities accounted for 61.04\% of all words in the original messages. 
Regarding \textit{relationship} coverage, we captured an average of 3.94 causal relationships per message ($SD$ = 1.52) and 26.34 causations per participant ($SD$ = 6.27). 
For \textit{construct} coverage, each message was associated with an average of 2.83 constructs defined above ($SD$ = 0.75), while individual participants expressed thoughts related to 7.94 constructs on average ($SD$ = 1.32). 
% These metrics showed that participants were actively sharing their reasoning and mental processes, and that our entities, relationships, and constructs were capturing much of it.
% These metrics showed that our CKG was effective in capturing the rich reasoning and mental processes of the participants.
In addition, as discussed in Section \ref{results:conceptual}, our CKG covered and made sense of several established theories of mental-illness stigma.


\begin{figure*}
\centering
\includegraphics[width=0.85\linewidth]{figs/figure_stigma.jpg}
\caption{Subgraph for P307, who exhibited stigmatizing responses, highlighting the interplay among negative beliefs, emotional responses, and behavioral intentions.}
\Description{Directed graph visualizing participant P307's stigma-related attitudes. Centered on a 'stigma' node, the graph shows interconnected nodes representing different types of responses, e.g., cognitive judgments in white (e.g., 'choose a pessimistic response to life'), beliefs in yellow (e.g., 'believe in loneliness'), emotional responses in coral (e.g., 'anger,' 'fear'), and behavioral intentions in blue (e.g., 'help,' 'social distance'). Nodes are connected by arrows indicating causal relationships, and a legend in the lower left identifies 11 color-coded constructs.}
\label{fig:stigmatizing_subgraph}
\end{figure*}


To evaluate \textbf{coherence}, we examined our CKG through the lens of \textit{consistency} - whether the CKG is free of contradictions and consistent with the domain it represents - and \textit{validity}. 
The analysis of cyclic relationships (as part of validity) revealed only eight cycles in total, traversing a small proportion (1.34\%) of all entities.
We speculated that most of the identified cycles represent reasonable reciprocal causation \cite{reciprocal_mulatu_2002} between constructs.
Connectivity analysis (as part of internal consistency) demonstrated the CKG's strong integration: a single large connected component comprised 98.59\% of the entities, while the remaining 83 small disconnected components collectively comprised only 189 entities.
Further, the causal relationships in our CKG had a mean length of 2.07 steps\footnote{Length here refers to the number of consecutive relationships - for example, (\texttt{entity1}, because, \texttt{entity2}) and (\texttt{entity2}, because, \texttt{entity3}) represent a two-step chain.} ($SD$ = 0.58), demonstrating the CKG's capacity to capture both direct causations and more complex indirect pathways that suggest potential mediating or moderating relationships between constructs.


Finally, we illustrated the potential \textbf{practical utility} of insights derived from our CKG by applying them to decipher individual stigmatizing attitudes, develop conceptual models, and generate hypotheses, which are presented in the next two sections.


\begin{table*}[htbp]
  \centering
  \small
  \caption{Themes and their prevalence rates derived through topic modeling and open coding across 11 stigma-related constructs.}
  \Description{Two-column table showing prevalent themes and their percentage rates related to depression stigma, organized by theme. All themes are listed with their corresponding prevalence rates ranging from 4.10\% to 89.26\%, with each percentage indicating how often that particular theme appeared in participants' messages.}
    \begin{tabular}{p{16.70em}clc}
    \toprule
    \textbf{Theme(s)} & \multicolumn{1}{p{5em}}{\textbf{Prevalence}} & \multicolumn{1}{p{16.70em}}{\textbf{Theme(s)}} & \multicolumn{1}{p{5em}}{\textbf{Prevalence}} \\
    \midrule
    \multicolumn{2}{p{16.70em}}{\textbf{Signaling Event}} & \multicolumn{2}{p{16.70em}}{\textbf{Past Experience}} \\
    \midrule
    Mood Swings & 34.78\% & \multicolumn{1}{p{16.70em}}{Personal Exposure} & 89.26\% \\[3.9pt]
    Social Withdrawal & 27.74\% & \multicolumn{1}{p{16.70em}}{Experiential/Attitudinal Knowledge} & 10.74\% \\
\cmidrule{3-4}    Aggressive, Self-neglect Behaviors & 16.78\% & \multicolumn{2}{p{16.70em}}{\textbf{Personality}} \\
\cmidrule{3-4}    Functional Impairment & 15.26\% & \multicolumn{1}{p{16.70em}}{Sociable, Easygoing, and Even-tempered} & 67.15\% \\[3.9pt]
    Treatment Non-adherence & 5.44\% & \multicolumn{1}{p{16.70em}}{Interpersonally Skeptical, Strong-minded, Stubborn, and Self-reliant} & 32.85\% \\
    \midrule
    \multicolumn{2}{p{16.70em}}{\textbf{Cognitive Judgment}} & \multicolumn{2}{p{16.70em}}{\textbf{Situation}} \\
    \midrule
    Controllability, Personal Responsibility & 42.85\% & \multicolumn{1}{p{16.70em}}{Time Scarcity} & 46.00\% \\[3.9pt]
    Severity, Need for Intervention & 27.02\% & \multicolumn{1}{p{16.70em}}{Physical Attributes} & 28.00\% \\[3.9pt]
    Perceived Dangerousness & 12.29\% & \multicolumn{1}{p{16.70em}}{Time Abundance} & 26.00\% \\
\cmidrule{3-4}    Unpredictability, Instability & 10.83\% & \multicolumn{2}{p{16.70em}}{\textbf{Potential Outcome}} \\
\cmidrule{3-4}    Inability to Judge without Context & 7.01\% & \multicolumn{1}{p{16.70em}}{Self-destructive Tendencies} & 32.88\% \\
\cmidrule{1-2}    \multicolumn{2}{p{16.70em}}{\textbf{Emotional Response}} & \multicolumn{1}{p{16.70em}}{Social Isolation, Relationship Difficulties} & 26.14\% \\
\cmidrule{1-2}    Sympathy, Concern & 65.95\% & \multicolumn{1}{p{16.70em}}{Functional Impairment} & 25.39\% \\[3.9pt]
    Frustration, Irritation & 18.74\% & \multicolumn{1}{p{16.70em}}{Symptom Deterioration} & 15.59\% \\
\cmidrule{3-4}    Fear, Anxiety, and Caution & 11.22\% & \multicolumn{2}{p{16.70em}}{\textbf{Motivation}} \\
\cmidrule{3-4}    Confusion & 4.10\% & \multicolumn{1}{p{16.70em}}{Altruistic, Prosocial} & 72.75\% \\
\cmidrule{1-2}    \multicolumn{2}{p{16.70em}}{\textbf{Behavioral Intention}} & \multicolumn{1}{p{16.70em}}{Egoistic} & 22.63\% \\
\cmidrule{1-2}    Supportive Engagement & 57.19\% & \multicolumn{1}{p{16.70em}}{Reciprocal} & 4.63\% \\
\cmidrule{3-4}    Conflict Management, De-escalation & 16.87\% & \multicolumn{2}{p{16.70em}}{\textbf{Suggestion}} \\
\cmidrule{3-4}    Protective Measures, Avoidance & 15.72\% & \multicolumn{1}{p{16.70em}}{Professional/Authority Help, Treatment} & 65.02\% \\[3.9pt]
    Segregation, Compartmentalization & 10.22\% & \multicolumn{1}{p{16.70em}}{Support System, Social Connections} & 26.77\% \\
\cmidrule{1-2}    \multicolumn{2}{p{16.70em}}{\textbf{Belief}} & \multicolumn{1}{p{16.70em}}{Self-management, Self-improvement} & 8.20\% \\
    \midrule
    Universal Human Experience/Equality & 38.51\% &       &  \\[3.9pt]
    Mental Health Awareness & 35.16\% &       &  \\[3.9pt]
    Social Interconnectedness & 14.65\% &       &  \\[3.9pt]
    Complexity of Innate Human Element & 11.68\% &       &  \\
    \bottomrule
    \end{tabular}%
  \label{tab:theme}%
\end{table*}%


\subsubsection{Case Studies: Participant-specific Subgraphs}

We deconstructed the stigmatizing attitudes of one participant, P307, who invariably exhibited such views, illustrating the feasibility of deconstructing stigma-related patterns in real time.
A parallel example in which we disentangled non-stigmatizing attitudes can be found in the \textit{Supplementary Materials}.


The subgraph for P307 (Figure \ref{fig:stigmatizing_subgraph}) reveals a network of negative beliefs and behaviors, including a causal chain from \textit{"choose a pessimistic response to life"} to \textit{"pessimistic belief is validated by what they experience"} and finally to \textit{"it is a cycle"}. 
This chain of reasoning suggests that the participant attributed the vignette character's mental-health issues to a self-perpetuating cycle of negative choices and experiences, placing \textit{responsibility} on the individual.


The subgraph also depicts the participant's ambivalent feelings about \textit{helping}: annoyance with the vignette's protagonist led to a reluctance to help them. 
Interestingly, however, there was also a path from \textit{"willing to humble myself"} to \textit{"co-exist"}, which resulted in a behavioral intention to help. 
This corresponded to P307's quote, \textit{"nobody is perfect. In as much as I can be annoyed by them I am willing to humble myself enough to co-exist."} 
Thus, while P307 is ostensibly expressing an offer of support, the causal analysis reveals a veiled \textit{microaggression}.



P307 also took a somewhat paradoxical stance on \textit{coercive segregation}, as evidenced by the co-occurrence of two \textit{beliefs} that appear to be contradictory: \textit{"believe in solitude"} and \textit{"isolation could exacerbate the depression"}. 
This was derived from the participant's statement, \textit{"I do believe in solitude but not in isolation. Avery needs solitude to introspect; however, isolation could exacerbate their depression."} 
So, while P307 attempted to differentiate between \textit{solitude} and \textit{isolation}, their reasoning still placed the onus on Avery to engage in introspection, and implied Avery's need for a space away from other people in which to do this. 
This causal co-action subtly reinforced the idea of hospitalization or segregation, even as the participant tried to soften it by acknowledging its potential negative outcomes. 
This reveals how stigmatizing attitudes can be rooted in seemingly benign intentions; and the graph highlights how stigmatizing beliefs can cascade, reinforce one another, and finally lead to discriminatory actions.




\subsubsection{Conceptual-model Construction}
\label{results:conceptual}


After the CKG was constructed, we established rules and restructuring principles in consultation with the mental-health specialist to better theorize our 11 constructs.
These stipulated that: 1) only relationships observed in at least one participant message could be included in the conceptual model; 2) \textit{potential outcome}, \textit{cognitive judgment}, and \textit{belief} were consolidated due to their shared focus on cognitive interpretations of mental health and depression, albeit at different levels of specificity; 3) \textit{motivation} and \textit{personality} were combined on the grounds that they all reflect relatively dispositional, enduring, and innate traits; 4) \textit{suggestion} was excluded as it primarily represented an outcome rather than a predictor of stigmatizing behavioral intentions; and 5) \textit{signaling event}, \textit{past experience}, \textit{personality}, and \textit{situation} could not be led to by other constructs, due to their innate or fixed nature, and that \textit{behavioral intention} could not lead to other constructs, as our focus was on its formation. 
Finally, in line with attribution theory \cite{attribution_theory_corrigan_2000}, we reorganized the constructs into four layers: \textit{stimuli}, \textit{cognitive mediator}, \textit{emotional response}, and \textit{behavioral intention}.

Accordingly, we identified themes across 11 constructs (Table \ref{tab:theme}) that shape attitudes toward depression, which informed the development of the conceptual model (Figure \ref{fig:theory_stigma}) illustrating key causal relationships (see the \textit{Supplementary Materials} for full results).
Below, we discussed how this conceptual model aligned with and went beyond existing theories.



\begin{figure*}
\centering
\includegraphics[width=0.9\linewidth]{figs/figure_theory_stigma.pdf}
\caption{Conceptual model of stigmatizing responses derived from the CKG. Solid lines represent pathways known to attribution theory \cite{attribution_theory_corrigan_2000}, while dashed ones indicate new pathways derived from our CKG analysis. The lines' thickness reflects the frequency of those relationships in participant messages.}
\Description{Conceptual model showing how stigmatizing responses develop, organized into four vertical columns: 'Stimuli,' 'Cognitive Mediator,' 'Emotional Response', and 'Behavioral Intention.' Solid lines show pathways known to attribution theory, flowing from 'Signaling Event' to 'Cognitive Judgment' to 'Feeling' to 'Discriminatory Behavior.' Dashed lines indicate new relationships identified through causal knowledge graph analysis, showing how 'Past Experience' influences both 'Cognitive Judgment' and 'Emotional Responses,' 'Personality' impacts 'Emotional Responses,' and 'Situation' directly affects 'Behavioral Intentions.' Line thickness indicates the frequency of the relationship in the data.}
\label{fig:theory_stigma}
\end{figure*}





\paragraph{\textcolor{darkred}{\textbf{Substantiating Existing Theories.}}}

The four theory-driven \textbf{constructs} in our conceptual model - \textit{signaling event}, \textit{cognitive judgment}, \textit{emotional response}, and \textit{behavioral intention} - corresponded directly to those in \textbf{attribution theory} \cite{attribution_theory_corrigan_2000}. 
The specific \textbf{themes} we identified within each construct also mirrored the theory's core findings: our \textit{cognitive-judgment} themes captured assessments of personal responsibility, controllability, unpredictability, and dangerousness; \textit{emotional responses} ranged from sympathy to caution and irritation; and \textit{behavioral intentions} included both supportive engagement and avoidance or compartmentalization.

Our conceptual model confirmed four key \textbf{pathways} consistent with attribution theory \cite{attribution_model_corrigan_2003}: \textit{signaling events} leading to both \textit{cognitive judgments} (extracted from 849 participants) and \textit{emotional responses} (526 participants), as well as the progression from \textit{cognitive judgments} to \textit{emotional responses} (514 participants) and ultimately to discriminatory \textit{behavioral intentions} (65 participants).
The dominance of these paths, particularly the first three, demonstrates that attribution theory's core proposition - that observable symptoms or behaviors of people with mental illness frequently trigger immediate appraisals and emotional responses, which can ultimately lead to discriminatory actions - remains crucial in explaining stigma processes.


Looking beyond attribution theory, several \textbf{other stigma-related theoretical frameworks} help make sense of our data-driven \textbf{constructs} and \textbf{themes}.
The themes under the \textit{belief} construct regarding the universality of personal struggles and the potential for anyone to experience depression align with theoretical work on de-stigmatization that emphasizes continuum rather than categorical beliefs about mental illness, with the former blurring the boundaries between "normal" people and those with psychiatric problems \cite{belief_reviewer_peter_2021}.
Our themes of mental-health awareness (as evidenced by narratives equating mental health with physical health in importance) and the complexity of human nature reflect research exploring how etiological, biological, and psycho-environmental beliefs shape stigma \cite{factor_valery_2020, factor_frost_2024}.


% The \textit{belief} construct highlights the importance of varied attitudes and beliefs towards mental illness, evidenced by narratives regarding the universality of personal struggles and the potential for anyone to experience depression. This aligns with prior work on de-stigmatization through continuum beliefs rather than categorical beliefs about depression, which blur the boundaries between "normal" people and those with psychiatric problems \cite{belief_reviewer_peter_2021}. 
% Another common belief is mental health awareness, such as equating mental health with physical health in importance, which may be largely due to people's knowledge and literacy about mental health (need citation here). Other beliefs are related to human nature and social interconnectedness, reflecting research exploring how etiological, biological, and psycho-environmental beliefs shape stigma \cite{factor_valery_2020, factor_frost_2024}. 



Our \textit{personality}-related constructs and themes, which captured participants' self-descriptions ranging from loyal, gregarious, and people-oriented to reserved, hardheaded, and resolute, align with research showing how openness to experience and agreeableness correlate with reduced discriminatory desires \cite{personality_reviewer_steiger_2022}. 
In the same vein, our \textit{motivational} themes, spanning from self-interest and egoism to prosocial tendencies and benevolence, correspond with theoretical work associating social dominance orientation and the desire to maintain security and social cohesion to mental-illness stigma \cite{motivational_reviewer_kvaale_2016}.


The \textit{situational} themes we identified, including physical attributes and time availability, speak to theories that emphasize how environmental and demographic factors influence stigmatizing attitudes \cite{sct_self_stigma_catalano_2021}; similarly, our findings about personal exposure and experiential knowledge resonate with studies showing how psychosis-like \textit{past experiences} are associated with public stigma in both community and clinical populations \cite{factor_chandra_2007}.
All these alignments substantiate the theoretical validity and value of our CKG-derived findings.



\paragraph{\textcolor{darkred}{\textbf{Novel Pathways beyond Existing Theories.}}}




Uniquely, several \textbf{pathways} unknown to existing theories were also revealed (four dashed lines in Figure \ref{fig:theory_stigma}).
We found that the extrinsic construct \textit{situation} directly influenced \textit{behavioral intentions}, with environmental contexts like time availability modulating pre-existing attitudes and shaping immediate reactions - less-busy participants, for instance, expressed greater willingness to engage with people suffering from depression.
This interaction between situational factors and behavioral responses could be partly made sense of through social-cognitive theory (SCT) \cite{sct_bandura_2001, sct_self_stigma_catalano_2021}, which posited that general human behavior emerges from the interplay of individual characteristics, cognitive factors, and prevailing contexts.
While SCT had once been applied to self-stigma research \cite{sct_self_stigma_catalano_2021}, our findings explored its potential for understanding public stigma, revealing how SCT's situational factors and attribution theory's emotional responses can jointly shape behavioral intentions. 
This intersection suggests new theoretical possibilities for exploring how SCT and attribution theory intertwine in deconstructing public depression stigma.


On the other hand, while prior work had validated intrinsic factors like \textit{personality}'s direct influence on behavioral intentions such as social-distance desires \cite{personality_reviewer_steiger_2022}, our data suggested that personality traits may also shape such desires by modulating \textit{emotional responses} (96 participants). 
For example, we found that self-prioritizing traits were often associated with negative emotions like anxiety or frustration, which could subsequently trigger avoidance tendencies.
% our data suggested that personality traits may also modulate \textit{emotional responses} (96 participants), with self-prioritizing traits often leading to negative emotions like fear or frustration, which could subsequently shape stigmatizing behavioral intentions.
This pathway reflected how ingrained dispositional factors served as enduring emotional tendencies, predisposing individuals to experience and express certain emotions more readily when confronted with mental illness-related situations.

% Moreover, while prior work had validated intrinsic factors like \textit{personality}'s influence on social-distance desires \cite{personality_reviewer_steiger_2022}, our data revealed that personality traits also affected \textit{emotional responses} (96 participants), with self-prioritizing traits often leading to negative emotions like fear or frustration. 
% This pathway reflected how ingrained dispositional factors served as enduring emotional tendencies, predisposing individuals to experience and express certain emotions more readily when confronted with depression-related situations.}


And lastly, our data highlighted that \textit{past experiences} informed both \textit{cognitive judgments} (136 participants) and \textit{emotional responses} (74 participants), as individuals drew on their personal history to interpret new situations, often generalizing past interactions with people with depression to guide their current perceptions and feelings. 
Although previous research had suggested that personal experiences might influence self-protective behaviors \cite{experience_weinstein_1989}, our conceptual model generated potential hypotheses that past experiences could also likely shape such stigmatizing responses (e.g., avoidance) by simultaneously affecting cognitive appraisals (e.g., dangerousness) and emotional reactions (e.g., fear).


% \textcolor{blue}{Our results also uncovered new constructs, including belief, personality, motivation, and situation, potentially extending existing theories about stigmatizing attitudes and behaviors. Several \textbf{pathways} unknown to existing theories were also revealed.

% The \textit{belief} construct highlights the importance of varied attitudes and beliefs towards mental illness, evidenced by narratives regarding the universality of personal struggles and the potential for anyone to experience depression. This aligns with prior work on de-stigmatization through continuum beliefs rather than categorical beliefs about depression, which blur the boundaries between "normal" people and those with psychiatric problems \cite{belief_reviewer_peter_2021}.} Another common belief is mental health awareness, such as equating mental health with physical health in importance, which may be largely due to people's knowledge and literacy about mental health (need citation here). Other beliefs are related to human nature and social interconnectedness, reflecting research exploring how etiological, biological, and psycho-environmental beliefs shape stigma \cite{factor_valery_2020, factor_frost_2024}. 

% \textcolor{blue}{Another unique construct emerged from our data is \textit{personality}. Personality captures participants' self-descriptions, ranging from loyal, gregarious, and people-oriented to reserved, hardheaded, and resolute. This aligns with research showing how openness to experience and agreeableness correlate with reduced discriminatory desires \cite{personality_reviewer_steiger_2022}. 

% Moreover, while prior work had validated intrinsic factors like \textit{personality}'s influence on social-distance desires \cite{personality_reviewer_steiger_2022}, our data revealed that personality traits also affected \textit{emotional responses} (96 participants), with self-prioritizing traits often leading to negative emotions like fear or frustration. 
% This pathway reflected how ingrained dispositional factors served as enduring emotional tendencies, predisposing individuals to experience and express certain emotions more readily when confronted with depression-related situations.}

% The \textit{motivation} construct, spanning from self-interest and egoism to prosocial tendencies and benevolence, corresponds with theoretical work associating social dominance orientation and the desire to maintain security and social cohesion to depression stigma \cite{motivational_reviewer_kvaale_2016}.}

% \textcolor{blue}{\textit{Situation} also emerged from our analysis. It includes physical attributes and time availability, speak to theories that emphasize how environmental and demographic factors influence stigmatizing attitudes \cite{situation_theory_rusch_2009}; similarly, our findings about personal exposure and experiential knowledge align with research showing how psychosis-like \textit{past experiences} are associated with public stigma in both community and clinical populations \cite{factor_chandra_2007}.
% All these alignments substantiate the theoretical validity and value of our CKG-derived findings.} 

% \textcolor{blue}{We found that the extrinsic construct \textit{situation} directly influenced \textit{behavioral intentions}, with environmental contexts like time availability modulating pre-existing attitudes and shaping immediate reactions - less-busy participants, for instance, expressed a greater willingness to engage with people suffering from depression.
% This interaction between situational factors and behavioral responses could be made sense of through social-cognitive theory (SCT) \cite{sct_bandura_2001, sct_self_stigma_catalano_2021}, which posited that general human behavior emerges from the interplay of individual characteristics, cognitive factors, and prevailing contexts.
% While SCT had once been applied to self-stigma research \cite{sct_self_stigma_catalano_2021}, our findings explored its potential for understanding public depression stigma, revealing how SCT's situational factors and attribution theory's emotional responses can jointly shape behavioral intentions. 
% This intersection suggests new theoretical possibilities for exploring how SCT and attribution theory intertwine in public stigma.}

%% this is a minor point - suggest removing (this section is corresponding to the three dashed lines in fig 8, would it cause potential misalignment if removing this..?)
% \textcolor{blue}{And lastly, our data highlighted that \textit{past experiences} informed both \textit{cognitive judgments} (136 participants) and \textit{emotional responses} (74 participants), as individuals drew on their personal history to interpret new situations, often generalizing past interactions with people with depression to guide their current perceptions and feelings. 
% Although previous research had suggested that personal experiences might influence self-protective behaviors \cite{experience_weinstein_1989}, our conceptual model generated potential hypotheses that past experiences shape self-protective responses like avoidance by simultaneously affecting cognitive appraisals of dangerousness and emotional responses of fear.}



% Additionally, our findings could be interpreted through the lens of Lazarus' transactional model of stress \cite{appraisal_lazarus_1984}, and particularly its concepts of primary and secondary appraisal. 
% In the context of depression stigma, primary appraisal manifests as initial, immediate judgments about people with mental illness (e.g., perceiving them as dangerous, unpredictable, or incompetent), whereas secondary appraisal involves evaluating one's own capacity to interact with such individuals (e.g., physical attributes and resource availability). 
% Our CKG revealed how, in that context, \textit{cognitive judgments} align with primary appraisal, while \textit{personality} traits and \textit{situational} factors correspond to secondary appraisal. 
% While acknowledging the distinction between Lazarus' stress-coping focus and our focus on perceptions of mental illness, we suggest that combining our respective appraisal frameworks would offer a fresh interpretative viewpoint on how dispositional and environmental factors interact to shape mental-health attitudes.


\subsection{Summary of Results}

In sum, our key findings were:
\begin{itemize}
    \item Our AI-assisted data-collection method effectively captured stigma-related discourses, with participants reporting high satisfaction ($M = 4.37/5$, $SD = 0.87$), engaging in self-disclosure, and providing data with high specificity, relevance, and clarity.
    \item Our AI-assisted coding approach showed high consistency with human-expert coding (Cohen's $\kappa$ = 0.69 overall), and significantly outperformed two existing computational-analysis approaches ($p < .001$).
    \item Our CKG, with 13,434 entities and 18,875 relationships, revealed 11 depression stigma-related constructs and their interrelationships among participants. Based on the CKG, our conceptual model identified themes and pathways that were consistent with existing theories and those that went beyond them.
\end{itemize}








% ------------ OUTDATED ----------------
% As shown in Table \ref{tab:model_comparison} and Figure \ref{fig:heatmap}, we found that it ranged from 0.46 to 0.76 across different stigma attributions, with \textit{social distance} showing the highest consistency and \textit{pity} the lowest. 

% We then proceeded to code 25 messages from each attribution (i.e., 175 messages in total) that had not previously been human-coded, and 

% \begin{table}[t]
% \setlength{\tabcolsep}{2.4pt}
% \captionsetup{position=above}
% \caption{Agreement between different models and human-derived codes, as measured by Cohen's $\kappa$ values for each stigma attribution and overall.}
% \label{tab:model_comparison}
% \centering
% \begin{threeparttable}
% \begin{tabular*}{\linewidth}{@{\extracolsep{\fill}}lcccccccc@{}}
% \toprule
% Methods & Responsibility & Social Distance & Anger & Helping & Pity & Coercive Segregation & Fear & \textbf{Total} \\
% \midrule
% RoBERTa & \textbf{0.72} & \textbf{0.78} & \textbf{0.68} & 0.53 & 0.18 & 0.53 & 0.68 & 0.68 \\
% BERTweet/SVM & 0.66 & 0.69 & \textbf{0.68} & 0.51 & 0.00 & \textbf{0.55} & 0.61 & 0.64 \\
% \textbf{Ours} & 0.69 & 0.76 & 0.65 & \textbf{0.69} & \textbf{0.46} & 0.54 & \textbf{0.71} & \textbf{0.69} \\
% \bottomrule
% \end{tabular*}
% \begin{tablenotes}
% \small
% \item \textit{Note.} \textbf{Bold} values indicate the highest agreement in each category.
% \end{tablenotes}
% \end{threeparttable}
% \end{table}

% Our CKG, comprising 13,434 entities and 18,875 relationships across 11 constructs, revealed both theoretically-consistent pathways (e.g., four key pathways aligning with attribution theory) and novel relationships (e.g., direct influences of situational factors on behavioral intentions) in depression stigma formation.



% In the case of stigma attributions (Table \ref{tab:model_comparison}), fine-tuned \texttt{RoBERTa-base} showed the highest agreement with human coding in three categories (\textit{responsibility}, \textit{social distance}, and \textit{anger}), and the \texttt{BERTweet+SVM} approach achieved the highest such agreement for \textit{anger} (tied with \texttt{RoBERTa-base}) and \textit{coercive segregation}. 
% Our AI-assisted coding method outperformed both of those pre-existing models, both overall and in the categories of \textit{helping}, \textit{pity}, and \textit{fear}. 
% Presumably due to the limited data on \textit{pity}, our approach outdid the other two methods, with \texttt{BERTweet-base} completely failing to categorize this attribution, whereas in data-rich categories such as \textit{responsibility} and \textit{social distance}, our approach achieved performance comparable to these two predecessors.


% Our first step in validating our CKG's quality consisted of human evaluation of subsets of the data. For the \textit{triple-extraction} process, we assessed the LLM-generated triples on a subset of 420 messages (approximately 5\% of the dataset) and found that their congruence with human-curated ones was 93\%, with entities and relations both matching. In the same vein, for the \textit{ontologization} step, we revisited and verified the entity types produced by the LLM on the same 420-message subset, achieving Cohen's $\kappa=0.77$ for human-LLM agreement. Lastly, in the \textit{entity-resolution} phase, we reviewed 50 randomly selected entity pairs from among those that the LLM determined should be merged. In this case, the $\kappa$ for agreement between the LLM's decisions and human judgments was 0.9. Collectively, these high agreement scores indicate the accuracy of our CKG.

% \textit{Belief}-associated entities formed the largest construct with 2,534 instances, followed by \textit{potential outcome} (1,632) and \textit{behavioral intention} (1,607). 
% \textit{Cognitive judgment} (1,505) and \textit{signaling event} (1,365) also played significant roles in the graph. 
% Other important constructs included \textit{emotional response} (1,017), \textit{suggestion} (1,006), \textit{past experience} (782), \textit{motivation} (722), \textit{personality}-related entities (687), and \textit{situation} (575). 
% The theory-driven constructs showed strong dependencies and interconnections. 




% \begin{figure}
% \centering
% \includegraphics[width=0.9\linewidth]{figs/figure_network.pdf}
% \caption{Network of psychological constructs (i.e., 11 entity types) and their associated themes derived from topic modeling and qualitative coding. 
% Colored boxes represent constructs (e.g., \textit{potential outcome}, \textit{personality}, and \textit{emotional response}). Within these boxes, each entity represents a theme, ranging from \textit{mental health awareness} beliefs to \textit{personal exposure} experiences to \textit{time scarcity} situations. The edges show causal relationships between the themes.}
% \Description{This figure presents a complex network diagram illustrating the interconnections between various psychological constructs and their associated themes in the context of mental-illness stigma. The diagram is organized into 11 main entity types, each represented by a colored box: "Potential Outcome", "Personality", "Suggestion", "Behavioral Intention", "Emotional Response", "Cognitive Judgment", "Motivation", "Belief", "Signaling Event", "Past Experience", and "Situation". Within each box are specific themes related to that construct, such as "Symptom Deterioration" under "Potential Outcome" or "Reciprocal" under "Motivation". The themes are connected by a dense network of lines, representing causal relationships between different elements.}
% \label{fig:topic_network}
% \end{figure}


% To unveil and portray the main manifestations of each of the 11 psychological-entity types, 
% On the basis of these themes, we built a conceptual model illustrating the key causal pathways.
% Table \ref{tab:theme} visualizes the web of relationships among these themes.


% These stipulated 1) that only relationships observed in at least one participant messages could be included in the conceptual model; 2) that \textit{signaling event}, \textit{past experience}, \textit{personality}, and \textit{situation} could not be led to by other constructs, due to their innate or fixed nature, and that \textit{behavioral intention} could not lead to other constructs, as our focus was on its formation; and 3) that certain constructs were structured into \textit{stimuli}, \textit{cognitive}, \textit{emotional}, and \textit{behavioral} layers, in line with attribution theory \cite{attribution_theory_corrigan_2000}. 


% In constructing our conceptual model of factor pathways based on the network, we made several changes to our initial 11 psychological constructs. 
% First, we excluded \textit{suggestion} as primarily representing an outcome rather than a predictor of stigmatizing behavioral intentions. 
% Second, we consolidated \textit{potential outcome}, \textit{cognitive judgment}, and \textit{belief} due to their shared focus on cognitive interpretations of mental illness, albeit at different levels of specificity; and third, we combined \textit{motivation} and \textit{personality} on the grounds that they all reflect innate traits. 
% This restructuring yielded seven key constructs, which we organized into four layers based on attribution theory \cite{attribution_theory_corrigan_2000}: i.e., stimuli, cognitive mediators, emotional responses, and behavioral intentions. 
% We then applied our algorithm and rules to generate the conceptual model of stigmatizing responses shown in Figure \ref{fig:theory_stigma}. 
% The corresponding model of \textit{non-stigmatizing} responses can be found in the \textit{Supplementary Materials}.


% The four key constructs in our model - \textit{signaling event}, \textit{cognitive judgment}, \textit{emotional response}, and \textit{behavioral intention} - directly corresponded to the attribution theory \cite{attribution_theory_corrigan_2000}. 
% Within these constructs, we uncovered certain themes that aligned with those included in the attribution theory \cite{attribution_model_corrigan_2003}: \textit{cognitive-judgment} themes included personal responsibility, controllability, unpredictability, and dangerousness, which directly correspond to the elements in attribution theory. However, some participants also expressed reluctance to appraise without full context. The \textit{emotional responses} identified, such as sympathy, fear, and irritation, as well as the \textit{behavioral intentions} of providing support, avoidance, and compartmentalization, all matched the responses outlined in this theory.


% New themes also emerged from our data. 
% For instance, views on mental-health awareness (equating mental health with physical health in importance), the complexity of human nature, the universality of personal struggles, and the potential for anyone to experience mental illness arose as key themes under \textit{belief}, whereas \textit{personality} surfaced as another prominent construct, with participants describing themselves along a spectrum from loyal, gregarious, and people-oriented to reserved, hardheaded, and resolute. 
% \textit{Motivations} ranged from self-interest and egoism to prosocial tendencies and benevolence. 
% Some participants also cited their physical stature, whether imposing or diminutive, and time availability as \textit{situational} factors. 
% Additionally, novel \textit{behavioral-intention} themes were noted, including desires to defuse tense situations and serve as impartial intermediaries. 


% The extrinsic construct \textit{situation} directly influenced \textit{behavioral intentions}, highlighting how environmental contexts can modulate pre-existing attitudes and shape immediate reactions to people suffering from depression.
% For example, time availability emerged as a key situational factor, with less-busy participants expressing greater willingness to engage with people with depression. 


% Looking beyond pre-existing stigma-related theories, our CKG analysis pinpointed a number of novel constructs and relationships, notably among \textit{situational}, \textit{motivational}, and \textit{personality} factors. 
% This interaction of environmental and individual elements aligns with social-cognitive theory (SCT) \cite{sct_bandura_2001, situated_action_suchman_2007}, which is frequently applied in HCI research \cite{hci_sct_naderi_2023, hci_sct_saksono_2021}. SCT posits that human behavior emerges from the interplay of individual characteristics, cognitive factors, and prevailing contexts. Our findings exemplify this interplay, revealing – for instance – how time availability influences discriminatory behavioral intentions: with less-busy participants expressing greater willingness to engage with people with mental illness. By showcasing how \textit{emotional responses} and \textit{situational} factors both directly drive \textit{behavioral intentions}, our data-derived conceptual model bridges SCT and attribution theory.

