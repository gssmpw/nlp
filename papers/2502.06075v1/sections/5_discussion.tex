\section{Discussion}

\subsection{AI-assisted Pipeline for Automatic Psychological-construct Decomposition}

Recent HCI studies have highlighted the potential for chatbots to gather fertile data on relatively sensitive topics \cite{chatbot_aq27_practice_lee_2023, disclosure_lee_2022, chatbot_reduce_kim_2020}. 
Our findings support this: we collected over 7,000 messages of approximately 40 words from 1,002 participants discussing mental health through short chatbot interactions averaging less than 20 minutes each.
% Their diverse and insightful responses lay the groundwork for future efforts to unpack the dynamics of stigma. 
Likewise, the high consistency we observed between our AI-assisted and human coding resonates with the results of prior studies in which LLM was used to streamline the manual coding of social phenomena \cite{coding_deductive_llm_tai_2024, deductive_labelling_xiao_2023}. 
Our study enriches this body of research by developing \textbf{a unified AI-assisted pipeline} that integrates chatbot-based data collection with AI-assisted coding.


Interestingly, consistency between AI-assisted and human coding varied across stigma attributions, with Cohen's $\kappa$ ranging from as low as 0.46 to as high as 0.76. 
% However, we argue that low-agreement attributions do not necessarily indicate the AI's poor performance. 
When looking into these human-AI coding divergences, we found that they probably revealed unique analytical perspectives and facilitated a deeper understanding of the socio-psychological constructs being coded.
For example, when dissecting \textit{pity}, the AI identified P437's statement "\textit{I feel emotional and compassionate toward her}" as stigmatizing, reasoning that it could reveal underlying condescension.
Although it lowered coding consistency, this insight challenged the received wisdom, which we initially endorsed, that \textit{pity} is purely positive \cite{attribution_model_corrigan_2003}. 
Such variations may also direct our attention to the underlying differences between human coding - which evolves through collaborative codebook creation, discussions, and oral guidance beyond written rules - and AI coding, which relies on human-written instructions and vast training-data corpora only.
These observations inform future research and highlight the need to systematically examine human-AI coding mismatches, particularly in light of emerging studies of LLMs as potential replacements for human coders \cite{coding_deductive_llm_tai_2024}.


%In contrast to most questionnaires, t
% \paragraph{\textcolor{darkred}{\textbf{Uncovering Subtle Stigma Attributions.}}}

This pipeline was able to both elicit and capture the participants' biased preconceptions and stigmatizing attitudes, even those they might be unconscious of.
We observed entrenched stigma such as \textit{condescension} (\textit{"I can open their eyes and they just need someone to guide them"}, P622), \textit{differential treatment} (\textit{"I will learn what triggers them and walk on eggshells around them"}, P687), and \textit{trivialization} through the dismissal of mental illness as merely a mood issue or overthinking (\textit{"Their problem can be easily overcome"}, P792). 
Uncovering these inadvertent attitudes is vital, given that people who harbor such views may unwittingly act as vectors for the spread and normalization of harmful social stigma. 
We therefore encourage future research to apply our proposed pipeline to nuanced conceptualizations of such microaggressions, microassaults, and microinsults, paving the way for tailored technology-enabled interventions.
% We therefore call for future research to apply this pipeline to systemic conceptualizations of such microaggressions, microassaults, and microinsults, paving the way for tailored technology-enabled interventions.






% \paragraph{\textcolor{darkred}{\textbf{Learning from Human-AI Coding Disagreements.}}}
% This not only informs, but also underscores the imperative for future research on human-AI qualitative coding to examine the systemic discordance between human and AI coding, which is of great importance given recent studies on how LLMs represent different human social groups and perspectives.}

% Such variations direct our attention to differences between human coding – which evolves through collaborative codebook creation, discussions, and oral guidance beyond written rules – and AI coding, which relies on written instructions and vast training-data corpora only, which not only inform but also urge future explorations of human-AI qualitative coding to check the systemic differences between human coding and AI-assisted coding, which is of great importance given recent studies on how LLMs represent different human social groups and perspectives \cite{human_replacement_hamaainen_2023, coding_deductive_llm_tai_2024}.}
% - human coding evolves through codebook co-creation, discussions, and oral guidance beyond written rules, while AI coding relies on human-written instructions and vast training-data corpora only - a distinction particularly relevant given recent studies on how LLMs represent different human social groups and perspectives \cite{human_replacement_hamaainen_2023, coding_deductive_llm_tai_2024}, }

% Such variations not only inform but also underscore the importance of future explorations in human-AI qualitative coding by highlighting fundamental systemic differences between these two approaches - human coding evolves through codebook co-creation, discussions, and oral guidance beyond written rules, while AI coding relies solely on human-written instructions and vast training-data corpora - a distinction particularly relevant given recent studies examining how LLMs represent different human social groups and perspectives \cite{human_replacement_hamaainen_2023, coding_deductive_llm_tai_2024}.

% \paragraph{\textcolor{darkred}{\textbf{Contextual Richness in Chatbot-based Interviews.}}}
In addition, our approach builds upon studies emphasizing language's role in perpetuating stigmatization of and prejudice against marginalized groups \cite{social_media_bail_2017, social_media_pavlova_2020, detect_method_fang_2023, detect_method_mittal_2023, detect_method_roesler_2024}, with a particular focus on the value of chatbot-interview data in understanding psychological constructs such as mental-illness stigma \cite{conversation_importance_jenlink_2005, interview_stigma_measure_lyons_1995}, and as a basis to facilitate deep reflection and personal disclosure. 
It offers new analytical opportunities that complement digital social-media data collection: i.e., less prone to data genericization, difficulty in maintaining context at scale \cite{social_media_decontext_boyd_2012}, proxy-population mismatch, the misidentification of bots as humans, and control of public access to data by proprietary algorithms \cite{social_media_deidentify_ruths_2014}. 
The contextual richness of our data positions our method as a powerful tool that can be expected to be used by scholars in the future to explore psycholinguistic differences between human-chatbot conversations and social media content when investigating psychological constructs.

% Its combination of contextual richness and efficient analysis positions our method as a powerful tool for investigating psychological constructs, which could be used in future studies to explore psycholinguistic differences between human-chatbot conversations and social-media content, among other multifaceted psychological dynamics and mechanisms \cite{interview_stigma_measure_liggins_2005}. 
% Indeed, as scholars apply and refine our approach in diverse contexts, it could be expected to contribute to how we conceptualize, computationally measure, and understand human cognition and behavior across multiple subfields of psychology.


\subsection{Partnership between LLMs and Causal Knowledge Graphs for Illustrating Psychological Models} % rq3, related work 2.3.3

Our large-scale CKG, featuring more than 13,000 entities and 18,000 relationships, provides a valuable database for exploring psychological mechanisms. 
It allowed us to unpack psychological constructs and build conceptual models that both \textbf{substantiated} core propositions of attribution theory \cite{attribution_theory_corrigan_2000} and other stigma-related frameworks \cite{belief_reviewer_peter_2021, personality_reviewer_steiger_2022, motivational_reviewer_kvaale_2016}, and also \textbf{went beyond them} by generating novel hypotheses such as the dual impact of \textit{past experiences} on both \textit{cognitive} and \textit{emotional} responses.



By allowing psychological constructs to emerge from attained data while being guided by social theories, our approach serves as a bridge between theory-driven methodologies and emerging data-centric paradigms in HCI and psychology \cite{ckg_llm_tong_2024}, revealing mechanisms that may sideline people suffering from mental illness.
Here, however, it should be noted that two points clarify our study's focus.
First, this paper is not to establish robust social-science models; rather, it is to demonstrate our method's potential to extend existing theories and build new ones by rapidly generating multiple hypotheses and uncovering latent pathways. 
Second, we do not work with causal graphs between pairs of variables from quantitative data, as is common in previous LLM-assisted causal reasoning studies \cite{ckg_llm_kiciman_2024, kg_psych_crielaard_2022}; instead, we explore the construction of causal \textit{knowledge} graphs from \textit{qualitative} data, which harbor contextual insights from rich, narrative information.
There are several directions for future work: explore methods to validate causal pathways that LLMs/CKGs unveil from qualitative data; systematically disentangle the concordances and discordances between the conceptual models of stigmatizing and non-stigmatizing responses; and refine our conceptual models by adding factors such as age and gender.
% There are opportunities for future work to explore methodologies to validate causal pathways that LLMs/CKGs unveil from qualitative data; systematically disentangle the concordances and discordances in the conceptual models of stigmatizing and non-stigmatizing responses; and refine our conceptual models by adding factors such as age and gender.
% Future work in social psychology could refine our conceptual model by adding factors such as age and gender and/or validating each of its pathways.


Additionally, our work chimes with recent studies on the synergistic potential of partnering LLMs and CKGs in social-psychological inquiry \cite{ckg_llm_tong_2024, kg_carta_2023, ckg_uleman_2021, ckg_borsboom_2021, kg_llm_pan_2024, kg_psych_crielaard_2022}. 
We have advanced this body of research by integrating \textit{node-level} analysis with the extraction of \textit{overarching} psychological constructs. 
Specifically, we integrated fine-grained LLM semantic analysis and the global perspective on causality provided by CKGs and their conceptual models, mirroring the holistic vs. analytic cognition dichotomy \cite{human_psych_nisbett_2001}. 
Building upon this integration, the resulting large-scale CKG could provide an infrastructure for \textit{link prediction} \cite{ckg_llm_tong_2024}, serving as a springboard for generating novel abductive hypotheses through the discovery of non-obvious or indirect causalities within the intertwined graph structure.
Such methods presumably move beyond \textit{descriptive} analyses like word counting and assigning brief labels to language \cite{theory_nlp_boyd_2021} to provide \textit{explanatory} insights that tease apart the antecedents, facets, and consequences of psychological constructs.
Fruitful avenues for future research include exploring effective retrieval methods for these CKGs; investigating their practical application by social scientists; and examining more complex relationships such as intensifying factors, protective factors, mediation processes, buffering effects, and enhancing effects in broader hyper-relational knowledge-graph contexts.

% % By examining participant-specific subgraphs and umbrella constructs, exploring relationships and co-actions among these constructs, and extracting themes, we formed conceptual diagrams illustrating discriminatory behavior-formation mechanisms. 


\subsection{Design Implications}

\paragraph{\textcolor{darkred}{\textbf{Real-time Identification for Tailored Micro-interventions to Combat Stigma}}}

% The insights derived from our study resonate with existing technology-infused micro-interventions (e.g., VR-based \cite{realtime_martinez_2024}) aimed at reducing depression stigma and changing psychological constructs. 

% \textcolor{blue}{The insights derived from our study provide a foundation for developing technology-infused real-time micro-interventions that not only align with existing approaches (e.g., VR-based \cite{realtime_martinez_2024}) for reducing depression stigma and changing psychological constructs, but also enable personalization - a critical need identified in current HCI intervention research \cite{personalized_anvari_2024}.}

The insights derived from the present study not only provide a foundation for the development of technology-enhanced real-time micro-interventions (e.g., VR-based \cite{realtime_martinez_2024}, game-based \cite{personalized_anvari_2024}) for reducing depression stigma and changing psychological constructs - an important line of HCI research - but also enable \textit{personalization}, which researchers have identified as a critical need in this domain \cite{personalized_anvari_2024}.

% The insights derived from our study resonate with existing technology-infused interventions (e.g., VR-based \cite{relevant_citation}) aimed at reducing depression stigma and changing psychological constructs, while offering new possibilities for personalized intervention design.

Our AI-assisted coding and LLM-CKG approach enable \textit{real-time} identification of stigma attributions and attributing processes. 
This real-time detection allows designers to determine which specific micro-intervention events - highly focused, in-the-moment elements designed to promote emotional, cognitive, or behavioral change \cite{micro_intervention_baumel_2020, micro_intervention_howe_2022} - are most appropriate. 
For example, when detecting responsibility-related beliefs, practitioners might consider deploying targeted didactic materials that teach about the biological and environmental factors \cite{belief_reviewer_peter_2021} of mental illness in the present moment. 
Similarly, when misconceptions about dangerousness arise, the digital system could suggest reframing exercises that tunnel the individual's focus toward alternative perspectives \cite{micro_intervention_howe_2022} to counteract premature negative appraisals.
% Similarly, when social-avoidance tendencies are identified, practitioners could be informed to trigger concrete actions through in-the-moment reminders, feedback, and incentives that promote social engagement.
% Manage knowledge variation within groups

In addition, our CKG shows the potential to inform therapeutic narratives and intervention delivery \cite{micro_intervention_baumel_2020} through its centralized knowledge structure that captures both theoretical understanding and user-specific context. 
Beyond conceptual models, the built CKG enables the creation of graphs of individuals' unique psychological profiles. 
Intervention designers could leverage these personalized graphs to customize therapeutic-session dialogues using a retrieval-augmented generation approach \cite{rag_chen_2024} that involves querying the CKG for relevant past experiences, belief patterns, and causal attributions.

This automated, individualized approach would allow practitioners to move beyond one-size-fits-all anti-stigma strategies that risk backfire effects \cite{backfire_dobson_2022} to broader context-sensitive micro-interventions aimed at dispelling myths, correcting misconceptions, and reinforcing positive beliefs \cite{realtime_martinez_2024, intervention_corrigan_1999}.



% Future research is needed to systematically disentangle the concordances and discordances in the conceptual models underlying stigmatizing and non-stigmatizing responses.



% \subsection{Broader Implications}


\paragraph{\textcolor{darkred}{\textbf{Cultural Sensitivity of AI-assisted Psychometric Analysis}}}

% Our approach to deciphering human psychological construct has implications for cultural-sensitive psychometric design \cite{microaggression_scale_ertem_2022, sd_scale_inaccurate_van_2008}. 
% Our CKG approach points toward implications for cultural-sensitive psychometric design.
% CKGs have the potential to elucidate the formation and evolution of psychological constructs across diverse sociocultural groups, thus enabling cross-cultural analysis of psychological mechanisms \cite{crosscultural_obeid_2015}. 
% Specifically, by applying knowledge-graph alignment techniques \cite{kg_llm_pan_2024} to CKGs from different demographic groups (e.g., Western vs. Eastern societies, different age cohorts), researchers could discern similarities and differences in their conceptual models, revealing culture-specific patterns. 

Our study suggests implications for cultural-sensitive psychometric design. 
Because psychological constructs are deeply influenced by socio-cultural factors \cite{culture_difference_krendl_2020}, the interrelationships captured in our CKG are inherently tied to our participants' cultural backgrounds. 
To adapt our approach to different cultural contexts, the interview protocol and AI-assisted analysis pipeline can be extended to support multilingual processing for multinational and multicultural populations; the CKG-construction methodology could be refined to capture local dialects, colloquial idioms, and vernacular that carry unique psycholinguistic nuances.

CKGs have the potential to elucidate the evolution of psychological constructs across diverse socio-cultural groups, thus enabling cross-cultural analysis of psychological mechanisms \cite{crosscultural_obeid_2015}. 
By applying knowledge-graph alignment techniques \cite{kg_llm_pan_2024} to CKGs from different demographic groups (e.g., Western vs. Eastern societies, different age cohorts), researchers could discern similarities and differences in their conceptual models, revealing culture-specific patterns. 
Examining how our approach to deconstructing psychological constructs translates across different cultural contexts would help validate its global applicability, validity, and generalizability.
% For example, our method might highlight differences in cognitive formation processes between Western and Eastern societies, or how younger and older generations develop unique psychological models.
% By comparing CKGs from different social groups using knowledge-graph alignment techniques \cite{kg_llm_pan_2024}, researchers could discern similarities and differences in their conceptual schemas, revealing culture-specific patterns. 




% \subsubsection{Causal Knowledge Graphs for Psychological Hypothesis Generation}

% Large-scale CKG also establishes the infrastructure for \textit{link prediction} \cite{ckg_llm_tong_2024}, which could generate novel abductive hypotheses by uncovering non-obvious or indirect causalities within the intertwined graph structure. 
% This approach has the potential to unveil hidden connections between psychological concepts that may not be immediately apparent through direct textual analysis. 
% By leveraging advanced techniques such as vector embeddings and similarity analyses \cite{ckg_llm_tong_2024}, we will be able to identify highly probable causal pairs that can serve as springboards for innovative research hypotheses. 

\paragraph{\textcolor{darkred}{\textbf{Facilitating Psychological-dataset Creation}}}

Finally, our pioneering combination of chatbots for data collection and high-quality AI-assisted coding is scalable; addresses recent calls for developing datasets that adhere to stringent standards for expert evaluation and impact assessment \cite{llm_psycho_demszky_2023}; and could underpin novel approaches to creating structured, well-coded, varied datasets in psychology and related fields. 
This, in turn, could lay a foundation for more influential theoretical contributions in HCI.



\subsection{Limitations and Future Research}

% Certain limitations of the present research should be acknowledged. 
% First, while our AI-assisted approach might mitigate the social-desirability bias that commonly afflicts questionnaire-based research, such dissimulation could still have been present in our data \cite{chatbot_sd_schuetzler_2018_2, chatbot_sd_schick_2022}. 
% That is, participants might have adjusted their responses, even in a chatbot interaction, to present themselves in a more favorable light. 
% Future research could therefore usefully explore additional strategies for further reducing this bias, such as humanization \cite{humanization_rhim_2022}, i.e., endowing chatbots with more human-like characteristics to foster trust and encourage more honest user responses.
Several methodological limitations warrant discussion. 
First, because our approach focused on message-level stigma analysis, we could have overlooked dynamics that emerged and evolved in complex ways over the course of the entire interview \cite{conversation_level_paakki_2024}. 
% However, we should emphasize that our human-chatbot interactions were intentionally designed as semi-structured interviews that prioritize eliciting rich self-disclosure and reflective thinking \cite{sensitive_kvale_2009}. 
A valuable next step would be to explore how attitudes evolve through interviews, how different chatbot-design strategies affect participant disclosure, and how the interplay between chatbot responses and participant statements shapes the expression of stigmatizing attitudes over time. 
Such temporal analysis of conversation trajectories could reveal important patterns in how participants refine their views through interaction.


Second, there are potentially important differences in the dynamics of human-human and human-chatbot conversations. 
Factors such as non-verbal cues and the ability to pick up on tone subtleties are likely to be limited in chatbot interactions \cite{nonverbal_denham_2013}. 
Future comparative analyses of these two broad types of interaction could enhance the ecological validity of our findings and reveal differences in the disclosure and expression of beliefs, attitudes, and behavioral intentions.




Certain limitations of the present case study should also be acknowledged. 
First, our participant pool was primarily from Western countries, which may have affected our ability to capture the full spectrum of culture-specific stigma-related nuances. 
Future research should prioritize obtaining culturally diverse samples to examine how our approach performs in different socio-cultural contexts. 
% Our participant pool was primarily from Western countries, which lead to our findings might not have captured the full spectrum of their culture-specific nuances. 
% it would be worthwhile to expand future investigations prioritize obtaining more culturally diverse samples to examine how our approach performs in different socio-cultural contexts.
Second, given that our human-derived codes are themselves not infallible, more attention and analysis are needed when comparing them with AI-assisted coding.
% also our human-human agreement is not perfectly high, represent the innate complexity when coding psych construct, which need to be considered when comparing ai to human code


% Second, our participant pool, while diverse, was not representative of any specific population. 
% Because psychological constructs are deeply influenced by sociocultural factors \cite{culture_difference_krendl_2020}, our findings might not have captured the full spectrum of their culture-specific nuances. 
% Future studies should prioritize obtaining more culturally diverse samples and adapting the methodology to be sensitive to cultural differences in psychological constructs.


% In addition, while we selected the best-performing LLM based on our tests at the time, variations in LLM versions could potentially affect the precision of our results; it would also be worthwhile to expand our CKG quality assessment beyond extraction accuracy to include metrics such as understandability, coverage, coherency, and succinctness \cite{qa_kg_hogan_2022}.



\section{Conclusion}

This study has introduced a novel approach to computationally deconstructing mental-illness stigma, in which data collection via human-chatbot conversations was synergized with AI-assisted coding. 
Holistic modeling of the focal stigma through a combination of CKGs and LLMs represents a potential new paradigm, and has revealed interrelationships among the factors behind stigmatizing statements that expand on existing theoretical work. 
These results not only deepen our insights into depression stigma, but also have broader methodological and design implications for HCI and psychological research, especially technology-enabled intervention design. 
We call for more research to explore our proposed approach's potential to identify additional patterns and interconnections among psychological factors in real time; to examine socio-cultural variation in stigma formation; and to develop targeted anti-stigma campaigns based on the causal pathways it uncovers.


