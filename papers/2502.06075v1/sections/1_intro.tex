\section{Introduction}


Mental illnesses are pervasive, with depression alone touching the lives of approximately 280 million people worldwide \cite{WHO_depression_2023}.
Nevertheless, due to \textit{mental-illness stigma} \cite{intervention_corrigan_1999}, many people with psychiatric disorders face social rejection, employment and housing discrimination, negative self-perceptions, and reluctance to seek help \cite{depression_stigma_peluso_2009, seek_help_reluctant_clement_2015}; and this stigma remains persistent despite efforts to eradicate it by fostering understanding and acceptance \cite{intervention_corrigan_1999}. 
The intricate nature of stigma, with its multifaceted manifestations and deep-rooted societal influences, makes it challenging to measure and eliminate \cite{attribution_theory_corrigan_2000}.



Existing methods for dissecting mental-illness stigma have various limitations. 
Self-report questionnaires \cite{attribution_model_corrigan_2003} provide relatively little detail, particularly about the hidden underpinnings of the respondents' emotions and actions \cite{comparison_taherdoost_2022}, and are prone to social-desirability bias \cite{sd_scale_inaccurate_furnham_1986}.
Social-media analysis has also been widely used to assess stigmatizing effects and cognitive distortions \cite{detect_method_fang_2023, detect_method_mittal_2023}, leveraging the large volume of diverse language samples available online. 
However, discourse on stigma-related topics in such posts tends to be brief and shallow \cite{social_media_decontext_boyd_2012}, and this type of analysis suffers from population bias \cite{social_media_deidentify_ruths_2014}.



A promising solution is the collection and analysis of in-depth interview data, which contains rich nuances of how mental health is discussed, perceived, and construed \cite{interview_stigma_measure_liggins_2005, interview_stigma_measure_lyons_1995}.
Interviews, a form of conversational interaction, serve as space and medium for unfolding perspectives, facilitating social-knowledge exchanges, and reproducing social norms \cite{conversation_importance_jenlink_2005, sensitive_kvale_2009}.
Such conversational contexts, along with interviewing techniques like probing questions and reflective listening \cite{sensitive_kvale_2009}, allow interviewers to identify recurring patterns such as the use of derogatory terms, casual jokes, or dismissive comments about mental illness that reveal stigma, whether intentional and/or unintentional \cite{conversation_importance_meredith_2019, microaggression_stigma_gonzales_2015}. 
However, gathering and dissecting interview data in quantities large enough to attain such insights has generally been infeasible due to the enormous amounts of time and effort required \cite{coding_manual_saldana_2016}, as well as the sensitive nature of mental illness-related discussions.




The burgeoning field of AI could help to reduce those burdens. 
AI-powered chatbots have already been used to simulate human interviewers to gather qualitative data on sensitive topics \cite{chatbot_reduce_kim_2020, chatbot_reduce_sebastian_2017}, which can reduce social-desirability bias and foster self-disclosure \cite{chatbot_disclosure_lucas_2014}; and some computational techniques, such as machine learning \cite{detect_method_jilka_2022} and word embedding \cite{detect_method_mittal_2023}, have been applied to automate stigma detection in large textual datasets. 
These advancements have facilitated the development of fruitful AI methods that excel at rapidly and accurately coding qualitative data like interviews for large-scale analysis \cite{ai_qualitative_feuston_2021}. 
The advent of LLMs could progress this capability by enabling real-time identification of language subtleties, emotional tones, and behavioral indicators that may not be immediately apparent to human coders \cite{coding_deductive_llm_tai_2024}.
However, AI/LLMs rely on syntactic and semantic cues, and their grounding in psychological theories/models ranges from simplistic to nonexistent \cite{theory_nlp_boyd_2021}. 
This would likely lead them to focus on end-state classification rather than on how, when, or why discriminatory behavior develops.


These limitations of AI/LLM-assisted qualitative-data analysis can be partly addressed by incorporating it with causal knowledge graphs (CKGs), which may be more capable of unraveling the mechanisms, processes, and triggers of stigmatizing behavioral intention. 
CKGs model causation via well-structured representations of entity-relationship-entity triples \cite{causal_graph_overview_jaimini_2022}. 
We consider that, in alignment with appropriate theoretical frameworks, CKGs could enable the mapping of stigma mechanisms onto established factor-pathway models and illustrate the interrelationships among various stigma-related constructs. 
Synergizing LLM capabilities with CKGs would aggregate the inductive strengths of LLMs, i.e., their ability to discover latent relationships, with the deductive power of CKGs: in this case, psychological-hypothesis generation \cite{ckg_llm_tong_2024}.

In this study, we examine how AI/LLM-assisted methods and CKGs can be integrated to both collect and analyze data related to mental-illness stigma, with a particular focus on the stigma attached to \textit{depression} due to its global prevalence. Specifically, we ask the following research questions: 

% \textcolor{blue}{Based on these considerations, we pose the following research questions to examine \textit{depression stigma} as our focal case, given its prevalence among mental disorders:}

% add another sentence to justify depression is prevalent 
% add objective

\begin{quote}
\textbf{RQ1.} \textit{To what extent can AI-assisted qualitative-data collection and analysis methods effectively capture depression stigma?}
% through human narratives?
% and (b) identify depression stigma?}
% in ways that are consistent with human coding and advance beyond existing computational techniques?}}

% \textcolor{blue}{\textit{RQ1.1}: How well does the interview data collected through human-chatbot conversations capture depression stigma?}

% \textcolor{blue}{\textit{RQ1.2}: How consistent is AI-assisted coding of such data with human coding, and how does it compare to existing computational techniques when it comes to detecting depression stigma in interviews?}

\textbf{RQ2.} \textit{How well can the integration of CKG and LLM reveal the constructs that drive depression stigma and their interrelationships?}
\end{quote}

Accordingly, the present study's methodology began with using an AI chatbot as an interviewer to collect qualitative data about perceptions of depression from 1,002 participants.
We then applied AI-assisted qualitative coding to identify stigma attributions, guided by a codebook developed through human coding. 
To validate this workflow, we compared our AI-assisted coding with human-generated codes and existing computational methods. 
We then semi-automatically constructed a CKG that could portray the interplay of stigma, dissect psychological constructs across conversations, discern similarities and differences in the participants' logic chains, and intertwine these elements to build a conceptual model. 
Our integration of LLM and CKG techniques involved the former suggesting new relationships and the latter providing a structured framework for deducing claims and hypotheses. 
This dual method uncovered constructs like \textit{personalities} and \textit{past experiences} that are supported by existing theories \cite{personality_reviewer_steiger_2022, experience_weinstein_1989}, along with new causal relationships such as the impact of \textit{personalities} on \textit{emotional responses} and \textit{situations} on \textit{anticipated discriminatory behaviors}.


Our work makes several contributions to the HCI community. 
First, it introduces an \textbf{AI-assisted data collection and analysis pipeline} that leverages AI to systematically elicit and analyze latent psychological constructs, such as depression stigma.
This method offers a non-labor-intensive yet nuanced approach to capturing large volumes of interview data and understanding and coding attributions of depression stigma.
% understanding the attributions of depression stigma and captures subtle language patterns that might otherwise be overlooked.
Second, its \textbf{integration of CKGs with LLMs} facilitates interpretation of the causal networks between various psychological constructs, such as cognition, emotions, and behavioral intentions related to mental health. 
This contributed to the development of a conceptual model that illuminates the predictors and processes of stigma formation.
And third, this study's findings can inform the design of \textbf{tailored, theoretically grounded micro-interventions} to combat depression stigma by enabling real-time stigma identification and the creation of CKGs specific to individuals.
This personalized approach may address the limitations of one-size-fits-all interventions, enhancing effectiveness while reducing potential unintended consequences.




% ------------- OUTDATED -------------


% address the limitations of one-size-fits-all interventions, enhancing effectiveness while reducing potential unintended consequences.


% Such efforts are complicated by its multifaceted manifestations and deep-rooted societal influences, which also make it challenging to measure \cite{attribution_theory_corrigan_2000}.

% Depression is pervasive, affecting about 280 million people globally \cite{WHO_depression_2023}. 

% Mental illnesses are pervasive, with depression alone affecting about 280 million people globally [128]. Nevertheless, due to mental-illness stigma [45], many people with mental-health disorders face social rejection, employment and housing discrimination, negative self-perceptions, and reluctance to seek help [21, 136]; and this stigma has proved highly resistant to attempts to eradicate it by fostering understanding and acceptance [25]. Such efforts are complicated by its multifaceted manifestations and deep-rooted societal influences, which also make it challenging to measure [22].


% and AI techniques such as SVM \cite{detect_method_li_2018} and BERTweet \cite{bertweet_nguyen_2020} have been applied to the measurement of stigma in large textual datasets. 
% These methods excel at rapidly and accurately classifying stigmatizing messages for large-scale analysis. 