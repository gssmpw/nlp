% \definecolor{wkblue}{rgb}{0.2, 0.3, 0.6}
% \definecolor{meta-color}{rgb}{0.5, 0.5, 0.5}

% \begin{tcolorbox}[colback=wkblue!10!white, colframe=wkblue!100!blue, left=2mm, right=2mm, title=\small\centering\textcolor{black}{The Thought Structure of Our Thought in Mathematical Reasoning}]
% \begin{small}

% \# Problem: Find all possible values of 
% \[
% \frac{1}{1 + \omega} + \frac{1}{1 + \omega^2}
% \]
% where \( \omega^3 = 1 \).

% \vspace{1.6mm}

% \#\# Understanding the cube roots of unity

% \begin{itemize}
%     \item The cube roots of unity are the solutions to \( \omega^3 = 1 \), which gives us three possible values: 
%     \[
%     \omega = 1, \quad \omega = \frac{-1 + i\sqrt{3}}{2}, \quad \omega = \frac{-1 - i\sqrt{3}}{2}.
%     \]
% \end{itemize}

% \vspace{1.6mm}

% \#\# Calculate the expression for \( \omega = 1 \)

% \begin{itemize}
%     \item Substituting \( \omega = 1 \) into the expression:
%     \[
%     \frac{1}{1 + 1} + \frac{1}{1 + 1} = \frac{1}{2} + \frac{1}{2} = 1.
%     \]
% \end{itemize}

% \vspace{1.6mm}

% \#\# Calculate the expression for \( \omega = \frac{-1 + i\sqrt{3}}{2} \)

% \begin{itemize}
%     \item Substituting \( \omega = \frac{-1 + i\sqrt{3}}{2} \) into the expression:
%     \[
%     \frac{1}{1 + \frac{-1 + i\sqrt{3}}{2}} + \frac{1}{1 + \frac{-1 - i\sqrt{3}}{2}}.
%     \]
%     Simplifying both terms:
%     \[
%     1 + \frac{-1 + i\sqrt{3}}{2} = \frac{1 + i\sqrt{3}}{2}, \quad 1 + \frac{-1 - i\sqrt{3}}{2} = \frac{1 - i\sqrt{3}}{2}.
%     \]
%     \item This gives:
%     \[
%     \frac{1}{\frac{1 + i\sqrt{3}}{2}} + \frac{1}{\frac{1 - i\sqrt{3}}{2}} = \frac{2}{1 + i\sqrt{3}} + \frac{2}{1 - i\sqrt{3}}.
%     \]
%     Simplifying each term using conjugates:
%     \[
%     \frac{2}{1 + i\sqrt{3}} = \frac{1 - i\sqrt{3}}{2}, \quad \frac{2}{1 - i\sqrt{3}} = \frac{1 + i\sqrt{3}}{2}.
%     \]
%     Adding these gives:
%     \[
%     \frac{1 - i\sqrt{3}}{2} + \frac{1 + i\sqrt{3}}{2} = 1.
%     \]
% \end{itemize}

% \vspace{1.6mm}

% \#\# Calculate the expression for \( \omega = \frac{-1 - i\sqrt{3}}{2} \)

% \begin{itemize}
%     \item This case is symmetrical to the previous one. Substituting \( \omega = \frac{-1 - i\sqrt{3}}{2} \) into the expression yields the same steps and result:
%     \[
%     \frac{1 - i\sqrt{3}}{2} + \frac{1 + i\sqrt{3}}{2} = 1.
%     \]
% \end{itemize}

% \vspace{1.6mm}

% \#\# \textbf{Thus}: For all possible values of \( \omega \), the expression simplifies to \( \boxed{1} \).


% \end{small}
% \end{tcolorbox}


% \definecolor{wkblue}{rgb}{0.2, 0.3, 0.6}
% \definecolor{meta-color}{rgb}{0.5, 0.5, 0.5}

% \begin{tcolorbox}[colback=wkblue!10!white, colframe=wkblue!100!blue, left=2mm, right=2mm, title=\small\centering\textcolor{black}{Solution to Polynomial Product Problem}]
% \begin{small}

% \# Problem: The product of \( (3t^2 + 5t + a)(4t^2 + bt - 2) \) is given as \( 12t^4 + 26t^3 - 8t^2 - 16t + 6 \). What is \( a + b \)?

% \vspace{1.6mm}

% \#\# Expand the product

% \begin{itemize}
%     \item \textbf{First, We need to} expand the product \( (3t^2 + 5t + a)(4t^2 + bt - 2) \).
%     \item When we do this, we should carefully multiply each term carefully:
%     \begin{align*}
%         (3t^2)(4t^2) &= 12t^4, \\
%         (3t^2)(bt) &= 3bt^3, \\
%         (3t^2)(-2) &= -6t^2, \\
%         (5t)(4t^2) &= 20t^3, \\
%         (5t)(bt) &= 5bt^2, \\
%         (5t)(-2) &= -10t, \\
%         (a)(4t^2) &= 4at^2, \\
%         (a)(bt) &= abt, \\
%         (a)(-2) &= -2a.
%     \end{align*}
% \end{itemize}

% \vspace{1.6mm}

% \#\# Combine like terms

% \begin{itemize}
%     \item \textbf{Now, let's} combine like terms:
%     \begin{align*}
%         t^4 \text{ term}: &\quad 12t^4, \\
%         t^3 \text{ terms}: &\quad (3bt^3 + 20t^3) = (3b + 20)t^3, \\
%         t^2 \text{ terms}: &\quad (-6t^2 + 5bt^2 + 4at^2) = (-6 + 5b + 4a)t^2, \\
%         t \text{ terms}: &\quad (-10t + abt) = (-10 + ab)t, \\
%         \text{Constant term}: &\quad -2a.
%     \end{align*}
% \end{itemize}

% \vspace{1.6mm}

% \#\# Compare coefficients with the given expression

% \begin{itemize}
%     \item The expanded expression needs to match the given polynomial \( 12t^4 + 26t^3 - 8t^2 - 16t + 6 \).
%     \item \textbf{Let's} equate the coefficients of like terms:
%     \begin{itemize}
%         \item For the \( t^4 \) term: \( 12 = 12 \), which checks.
%         \item For the \( t^3 \) term: \( 26 = 3b + 20 \), solving gives \( b = 2 \).
%         \item For the \( t^2 \) term: \( -8 = -6 + 5b + 4a \), substituting \( b = 2 \) gives:
%         \[
%         -8 = -6 + 10 + 4a \quad \Rightarrow \quad -8 = 4 + 4a \quad \Rightarrow \quad 4a = -12 \quad \Rightarrow \quad a = -3.
%         \]
%         \item For the \( t \) term: \( -16 = -10 + ab \), substituting \( a = -3 \) and \( b = 2 \), we get:
%         \[
%         -16 = -10 - 6, \quad \text{which checks}.
%         \]
%         \item For the constant term: \( 6 = -2a \), substituting \( a = -3 \), we get:
%         \[
%         6 = 6, \quad \text{which checks}.
%         \]
%     \end{itemize}
% \end{itemize}

% \vspace{1.6mm}

% \#\# Correcting the mistake

% \begin{itemize}
%     \item \textbf{Wait a second}: There seems to be a mistake in the constant term calculation. Let's correct it:
%     \item The correct equation should be \( 6 = -2a \), solving this gives:
%     \[
%     a = -3.
%     \]
% \end{itemize}

% \vspace{1.6mm}

% \#\# With \( a = -3 \) and \( b = 2 \), we find:
%     \[
%     a + b = -3 + 2 = \boxed{-1}.
%     \]


% \end{small}
% \end{tcolorbox}


\definecolor{wkblue}{rgb}{0.2, 0.3, 0.6}
\definecolor{meta-color}{rgb}{0.5, 0.5, 0.5}
\definecolor{pinkcolor}{rgb}{1.0, 0.75, 0.8}  % 柔和的粉色


\begin{tcolorbox}[colback=pinkcolor!10!white, colframe=pinkcolor!100!red, left=2mm, right=2mm, title=\small\centering\textcolor{black}{Case 1: Solution to Algebra Problem with Polynomial Multiplication}]

\begin{small}

\# Problem: The product of $3t^2+5t+a$ and $4t^2+bt-2$ is $12t^4+26t^3-8t^2-16t+6$. What is $a+b$?

\vspace{1.6mm}

\# Our Proposed Model Response:

\vspace{1.6mm}

First, we need to expand the product \((3t^2 + 5t + a)(4t^2 + bt - 2)\).

When we do this, we should \textcolor{red}{carefully} multiply each term:

\begin{itemize}
    \item The product of \(3t^2\) and \(4t^2\) gives us \(12t^4\).
    \item The product of \(3t^2\) and \(bt\) gives us \(3bt^3\).
    \item The product of \(3t^2\) and \(-2\) gives us \(-6t^2\).
    \item The product of \(5t\) and \(4t^2\) gives us \(20t^3\).
    \item The product of \(5t\) and \(bt\) gives us \(5bt^2\).
    \item The product of \(5t\) and \(-2\) gives us \(-10t\).
    \item The product of \(a\) and \(4t^2\) gives us \(4at^2\).
    \item The product of \(a\) and \(bt\) gives us \(abt\).
    \item The product of \(a\) and \(-2\) gives us \(-2a\).
\end{itemize}

Now, let's combine like terms:

\begin{itemize}
    \item The \(t^4\) term is \(12t^4\), so we have \(12t^4\).
    \item The \(t^3\) terms are \(3bt^3\) and \(20t^3\), which combine to \((3b + 20)t^3\).
    \item The \(t^2\) terms are \(-6t^2\), \(5bt^2\), and \(4at^2\), which combine to \((-6 + 5b + 4a)t^2\).
    \item The \(t\) terms are \(-10t\) and \(abt\), which combine to \((-10 + ab)t\).
    \item The constant term is \(-2a\).
\end{itemize}

\textcolor{red}{Let's pause} and consider what we know so far. We need this expanded expression to match the given polynomial \(12t^4 + 26t^3 - 8t^2 - 16t + 6\).

Now, let's equate the coefficients of like terms:

\begin{itemize}
    \item For the \(t^4\) term, we have \(12 = 12\), which checks.
    \item For the \(t^3\) term, we get \(26 = 3b + 20\). Solving gives \(b = 2\).
    \item For the \(t^2\) term, we have \(-8 = -6 + 5b + 4a\). Substituting \(b = 2\), we get \(-8 = -6 + 10 + 4a\), which simplifies to \(4a = -14\) and \(a = -3.5\).
    \item For the \(t\) term, we have \(-16 = -10 + ab\). Substituting \(a = -3.5\) and \(b = 2\), we get \(-16 = -10 - 7\), which checks.
    \item For the constant term, we have \(6 = -2a\). Substituting \(a = -3.5\), we get \(6 = 7\), \textcolor{red}{which doesn't check}.
\end{itemize}

\textcolor{red}{Wait a second}, let's check our calculations again. The constant term calculation should have been \(6 = -2a\), which gives \(a = -3\). \textcolor{red}{Let's go back and correct that.}

With \(a = -3\) and \(b = 2\), we find \(a + b = -3 + 2 = -1\).

So, the correct value of \(a + b\) is \(\boxed{-1}\).

\end{small}
\end{tcolorbox}
