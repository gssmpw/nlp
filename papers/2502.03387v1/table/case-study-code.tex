% \definecolor{wkblue}{rgb}{0.2, 0.3, 0.6}
% \definecolor{meta-color}{rgb}{0.5, 0.5, 0.5}

% \begin{tcolorbox}[colback=wkblue!10!white, colframe=wkblue!100!blue, left=2mm, right=2mm, title=\small\centering\textcolor{black}{The Thought Structure of Our Thought in Mathematical Reasoning}]
% \begin{small}

% \# Problem: Find the remainder when $$f(x)=x^{10}+5x^9-8x^8+7x^7-x^6-12x^5+4x^4-8x^3+12x^2-5x-5$$ is divided by $x^2-1$ without using long division.

% \vspace{1.6mm}

% \#\# Apply the Remainder Theorem

% \begin{itemize}
%     \item \textbf{We want to} find the remainder when $f(x)$ is divided by $x^2 - 1$.
%     \item Using the Remainder Theorem. 
%     \begin{itemize}
%         \item Substitute \( x = 1 \) and \( x = -1 \) into \( f(x) \) to find the remainders at those points.
%         \item \textbf{Let me compute} \( f(1) \):
%         \begin{align*}
%         f(1) &= 1^{10} + 5(1)^9 - 8(1)^8 + 7(1)^7 - 1^6 - 12(1)^5 + 4(1)^4 - 8(1)^3 + 12(1)^2 - 5(1) - 5 \\
%               &= 1 + 5 - 8 + 7 - 1 - 12 + 4 - 8 + 12 - 5 - 5 \\
%               &= -10
%         \end{align*}
%         \item \textbf{Similarly}, calculate \( f(-1) \):
%         \begin{align*}
%         f(-1) &= (-1)^{10} + 5(-1)^9 - 8(-1)^8 + 7(-1)^7 - (-1)^6 - 12(-1)^5 + 4(-1)^4 - 8(-1)^3 \\
%         &\quad\quad + 12(-1)^2 - 5(-1) - 5 \\
%               &= 1 - 5 - 8 - 7 - 1 + 12 + 4 + 8 + 12 + 5 - 5 \\
%               &= 16
%         \end{align*}
%         \item \textbf{Therefore}: The remainder when dividing by \( x - 1 \) is \(-10\), and for \( x + 1 \) it is \( 16 \).
%     \end{itemize}
% \end{itemize}

% \vspace{1.6mm}

% \#\# Deriving the remainder \( r(x) = ax + b \)

% \begin{itemize}
%     \item \textbf{Since} \( x^2 - 1 = (x - 1)(x + 1) \), the remainder must be of the form \( ax + b \).
%     \item We now solve for \( a \) and \( b \) using the remainders from \textbf{Step 1}.
%     \begin{itemize}
%         \item Set up the system of equations:
%         \begin{align*}
%         a(1) + b &= -10 \quad \text{(from \( f(1) = -10 \))} \\
%         a(-1) + b &= 16 \quad \text{(from \( f(-1) = 16 \))}
%         \end{align*}
%         \item \textbf{Let me compute}: Add the equations to find \( b \):
%         \begin{align*}
%         a + b &= -10 \\
%         -a + b &= 16 \\
%         \text{Adding the equations}: 2b &= 6 \quad \Rightarrow \quad b = 3
%         \end{align*}
%         \item Now substitute \( b = 3 \) into \( a + b = -10 \):
%         \begin{align*}
%         a + 3 &= -10 \quad \Rightarrow \quad a = -13
%         \end{align*}
%     \end{itemize}
% \end{itemize}

% \vspace{1.6mm}

% \#\# \textbf{Thus}: The remainder when \( f(x) \) is divided by \( x^2 - 1 \) is \( -13x + 3 \).
% \vspace{1.6mm}

% \#\# Final result: \( r(x) = \boxed{-13x + 3} \)



% \end{small}
% \end{tcolorbox}


% \definecolor{wkblue}{rgb}{0.2, 0.3, 0.6}
% \definecolor{meta-color}{rgb}{0.5, 0.5, 0.5}
% \definecolor{pinkcolor}{rgb}{1.0, 0.75, 0.8}  % soft pink

% \begin{tcolorbox}[
%     colback=pinkcolor!10!white,
%     colframe=pinkcolor!100!red,
%     left=2mm,
%     right=2mm,
%     title=\small\centering\textcolor{black}{Understanding await asyncio.sleep() Issues},
%     breakable=true,
%     % break at={\pagebreak},         % 只在明确的分页点处分页
% ]
% \begin{tiny}

% \# Problem: Why in python await asyncio.sleep() is such?

% \vspace{1.6mm}

% \# Our Proposed Model Response:

% \vspace{1.6mm}

% \textbf{Understanding asyncio.sleep()}

% First, it's important to understand what \texttt{asyncio.sleep()} does:
% \begin{itemize}
%     \item \texttt{asyncio.sleep()} is a coroutine function that suspends the execution of the coroutine for a given number of seconds.
%     \item During the sleep, other tasks can run, making it useful for simulating delays without blocking the entire program.
%     \item It's commonly used in asynchronous programming to introduce delays or to simulate IO-bound operations.
% \end{itemize}

% \textbf{Common Reasons Why await asyncio.sleep() Might Seem Stuck}

% 1. \textbf{Event Loop Not Running}
%    \begin{itemize}
%    \item \textit{Explanation:} The asyncio event loop is responsible for managing and scheduling coroutines. If the event loop isn't running, the coroutines won't execute, and \texttt{await asyncio.sleep()} will appear stuck.
%    \item \textit{Solution:} Ensure that the event loop is running. You can start the event loop using \texttt{asyncio.run()}, \texttt{loop.run\_until\_complete()}, or \texttt{loop.run\_forever()}.
%    \end{itemize}

% 2. \textbf{Blocking Code in the Event Loop}
%    \begin{itemize}
%    \item \textit{Explanation:} If you have blocking code (e.g., a long-running synchronous function) running in the event loop, it can prevent other coroutines from executing.
%    \item \textit{Solution:} Avoid running blocking code in the event loop. Use \texttt{loop.run\_in\_executor()} to offload blocking tasks.
%    \end{itemize}

% 3. \textbf{Incorrect Use of await}
%    \begin{itemize}
%    \item \textit{Explanation:} If you forget to \texttt{await} a coroutine, it won't execute, and the program might appear stuck.
%    \item \textit{Solution:} Always use \texttt{await} when calling a coroutine function.
%    \end{itemize}

% 4. \textbf{Event Loop Blocked by Infinite Loops}
%    \begin{itemize}
%    \item \textit{Explanation:} An infinite loop within a coroutine can prevent the event loop from scheduling other tasks.
%    \item \textit{Solution:} Ensure proper exit conditions and use \texttt{await} within loops.
%    \end{itemize}

% 5. \textbf{Using time.sleep() Instead of asyncio.sleep()}
%    \begin{itemize}
%    \item \textit{Explanation:} \texttt{time.sleep()} is a blocking function that will pause the entire event loop.
%    \item \textit{Solution:} Use \texttt{await asyncio.sleep()} in asynchronous code.
%    \end{itemize}

% \textbf{Code Examples}

% 1. Correct Event Loop Usage:
% \begin{verbatim}
% import asyncio

% async def my_coroutine():
%     print("Starting sleep")
%     await asyncio.sleep(2)
%     print("Finished sleep")

% # Correct way to run the coroutine
% asyncio.run(my_coroutine())
% \end{verbatim}

% 2. Avoiding Blocking Code:
% \begin{verbatim}
% # Bad Example:
% async def blocking_coroutine():
%     time.sleep(10)  # Blocks event loop

% # Good Example:
% async def non_blocking_coroutine():
%     await asyncio.sleep(10)  # Non-blocking
% \end{verbatim}

% 3. Proper await Usage:
% \begin{verbatim}
% # Incorrect:
% async def main():
%     another_coroutine()  # Missing await

% # Correct:
% async def main():
%     await another_coroutine()
% \end{verbatim}

% \textbf{Debugging Tips}
% \begin{itemize}
%     \item Enable Debug Mode:
%     \begin{verbatim}
% asyncio.run(main(), debug=True)
%     \end{verbatim}
%     \item Inspect Tasks:
%     \begin{verbatim}
% tasks = asyncio.all_tasks()
% print(f"Tasks: {tasks}")
%     \end{verbatim}
%     \item Handle Exceptions:
%     \begin{verbatim}
% try:
%     await some_coroutine()
% except Exception as e:
%     print(f"Exception occurred: {e}")
%     \end{verbatim}
% \end{itemize}

% \textbf{Best Practices}
% \begin{itemize}
%     \item Keep the Event Loop Unblocked: Avoid running blocking code
%     \item Use await Appropriately: Always \texttt{await} coroutine functions
%     \item Offload Blocking Tasks: Use \texttt{loop.run\_in\_executor()} for CPU-bound tasks
%     \item Test with Debugging Tools: Utilize asyncio debugging features
% \end{itemize}

% \textbf{Additional Resources}
% \begin{itemize}
%     \item Python asyncio Documentation: \url{https://docs.python.org/3/library/asyncio.html}
%     \item Asynchronous Programming in Python: \url{https://realpython.com/async-io-python/}
%     \item Understanding Asyncio and Event Loops: \url{https://medium.com/swlh/understanding-asyncio-and-event-loops-in-python-af1a57c90e7}
% \end{itemize}

% \end{tiny}
% \end{tcolorbox}

\definecolor{wkblue}{rgb}{0.2, 0.3, 0.6}
\definecolor{meta-color}{rgb}{0.5, 0.5, 0.5}
\definecolor{pinkcolor}{rgb}{1.0, 0.75, 0.8}

\begin{tcolorbox}[
colback=pinkcolor!10!white,
colframe=pinkcolor!100!red,
left=2mm,
right=2mm,
title=\small\centering\textcolor{black}{Complete Guide to await asyncio.sleep() Issues},
breakable=true
]

\section*{Understanding asyncio.sleep()}
When you encounter an issue where \texttt{await asyncio.sleep()} seems to be "stuck" in Python, there are several potential reasons and solutions to consider. Let's delve into this step by step.

First, it's important to understand what \texttt{asyncio.sleep()} does:
\begin{itemize}
\item \texttt{asyncio.sleep()} is a coroutine function that suspends the execution of the coroutine for a given number of seconds
\item During the sleep, other tasks can run, making it useful for simulating delays without blocking the entire program
\item It's commonly used in asynchronous programming to introduce delays or to simulate IO-bound operations
\end{itemize}

\section*{Common Reasons Why await asyncio.sleep() Might Seem Stuck}

\subsection*{1. Event Loop Not Running}
\begin{itemize}
\item \textbf{Explanation:} The asyncio event loop is responsible for managing and scheduling coroutines. If the event loop isn't running, the coroutines won't execute, and \texttt{await asyncio.sleep()} will appear stuck.
\item \textbf{Solution:} Ensure that the event loop is running. You can start the event loop using \texttt{asyncio.run()}, \texttt{loop.run\_until\_complete()}, or \texttt{loop.run\_forever()}.
\end{itemize}

\subsection*{2. Blocking Code in the Event Loop}
\begin{itemize}
\item \textbf{Explanation:} If you have blocking code (e.g., a long-running synchronous function) running in the event loop, it can prevent other coroutines from executing.
\item \textbf{Solution:} Avoid running blocking code in the event loop. Use \texttt{loop.run\_in\_executor()} to offload blocking tasks.
\end{itemize}

\subsection*{3. Incorrect Use of await}
\begin{itemize}
\item \textbf{Explanation:} If you forget to \texttt{await} a coroutine, it won't execute.
\item \textbf{Solution:} Always use \texttt{await} when calling a coroutine function.
\end{itemize}

\subsection*{4. Event Loop Blocked by Infinite Loops}
\begin{itemize}
\item \textbf{Explanation:} An infinite loop within a coroutine can prevent the event loop from scheduling other tasks.
\item \textbf{Solution:} Ensure proper exit conditions and use \texttt{await} within loops.
\end{itemize}

\subsection*{5. Using time.sleep() Instead}
\begin{itemize}
\item \textbf{Explanation:} \texttt{time.sleep()} is a blocking function that will pause the entire event loop.
\item \textbf{Solution:} Use \texttt{await asyncio.sleep()} in asynchronous code.
\end{itemize}

\section*{Step-by-Step Troubleshooting}

\subsection*{1. Verify That the Event Loop Is Running}
Ensure that the event loop is running and your coroutine is scheduled:

\begin{verbatim}
import asyncio

async def my_coroutine():
    print("Starting sleep")
    await asyncio.sleep(2)
    print("Finished sleep")

# Correct way to run the coroutine
asyncio.run(my_coroutine())
\end{verbatim}

For older Python versions (before 3.7):
\begin{verbatim}
loop = asyncio.get_event_loop()
loop.run_until_complete(my_coroutine())
\end{verbatim}

\subsection*{2. Check for Blocking Code}
Bad Example:
\begin{verbatim}
import asyncio
import time

async def blocking_coroutine():
    print("Starting blocking operation")
    time.sleep(10)  # Blocks the event loop
    print("Finished blocking operation")

asyncio.run(blocking_coroutine())
\end{verbatim}

Good Example:
\begin{verbatim}
import asyncio

async def non_blocking_coroutine():
    print("Starting sleep")
    await asyncio.sleep(10)
    print("Finished sleep")

asyncio.run(non_blocking_coroutine())
\end{verbatim}

\subsection*{3. Ensure Proper Use of await}
Incorrect:
\begin{verbatim}
import asyncio

async def another_coroutine():
    print("This coroutine won't be awaited")

async def main():
    another_coroutine()  # Missing await
    print("Main coroutine continues")

asyncio.run(main())
\end{verbatim}

Correct:
\begin{verbatim}
import asyncio

async def another_coroutine():
    print("This coroutine is awaited")

async def main():
    await another_coroutine()  # Properly awaited
    print("Main coroutine continues")

asyncio.run(main())
\end{verbatim}

\subsection*{4. Avoid Infinite Loops}
Bad Example:
\begin{verbatim}
import asyncio

async def infinite_loop_coroutine():
    while True:
        print("Looping...")
        # Missing await inside the loop
        # await asyncio.sleep(1)

asyncio.run(infinite_loop_coroutine())
\end{verbatim}

Good Example:
\begin{verbatim}
import asyncio

async def controlled_loop_coroutine():
    while True:
        print("Looping...")
        await asyncio.sleep(1)  # Allows switching tasks

asyncio.run(controlled_loop_coroutine())
\end{verbatim}

\subsection*{5. Use asyncio.run\_in\_executor() for Blocking Tasks}
\begin{verbatim}
import asyncio
import time

def blocking_function():
    time.sleep(10)
    return "Blocking function completed"

async def main():
    loop = asyncio.get_running_loop()
    result = await loop.run_in_executor(None, blocking_function)
    print(result)

asyncio.run(main())
\end{verbatim}

\section*{Debugging Tips}

Enable Debug Mode:
\begin{verbatim}
asyncio.run(main(), debug=True)
\end{verbatim}

Inspect Tasks:
\begin{verbatim}
tasks = asyncio.all_tasks()
print(f"Tasks: {tasks}")
\end{verbatim}

Exception Handling:
\begin{verbatim}
try:
    await some_coroutine()
except Exception as e:
    print(f"Exception occurred: {e}")
\end{verbatim}

\section*{Example of a Common Mistake}
Mixing synchronous and asynchronous code:
\begin{verbatim}
import asyncio

async def async_function():
    print("Async function started")
    await asyncio.sleep(5)
    print("Async function finished")

def sync_function():
    print("Sync function started")
    time.sleep(10)
    print("Sync function finished")

async def main():
    task = asyncio.create_task(async_function())
    sync_function()  # Blocks for 10 seconds
    await task

asyncio.run(main())
\end{verbatim}

Output:
\begin{verbatim}
Async function started
Sync function started
# 10 seconds later
Sync function finished
# 5 seconds later
Async function finished
\end{verbatim}

\section*{Best Practices}
\begin{itemize}
\item Keep the Event Loop Unblocked: Avoid running blocking code
\item Use await Appropriately: Always await coroutine functions
\item Offload Blocking Tasks: Use loop.run\_in\_executor()
\item Test with Debugging Tools: Utilize asyncio debugging features
\end{itemize}

\section*{Conclusion}
If \texttt{await asyncio.sleep()} seems stuck, it's likely due to the event loop being blocked or not running properly. By ensuring the event loop is active, avoiding blocking code, and using \texttt{await} correctly, you can resolve the issue and ensure smooth asynchronous execution in your Python programs.

\section*{Additional Resources}
\begin{itemize}
\item Python asyncio Documentation: \url{https://docs.python.org/3/library/asyncio.html}
\item Asynchronous Programming in Python: \url{https://realpython.com/async-io-python/}
\item Understanding Asyncio and Event Loops: \url{https://medium.com/swlh/understanding-asyncio-and-event-loops-in-python-af1a57c90e7}
\end{itemize}

\end{tcolorbox}