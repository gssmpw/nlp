\begin{table*}[!t]
    \centering
    \setlength{\tabcolsep}{6pt}
    {
    \fontsize{10}{13}\selectfont % 设置字体大小为10pt,行距为12pt
    \renewcommand\arraystretch{1.5}
    \resizebox{\linewidth}{!}{
        \begin{tabular}[l]{p{0.2\linewidth}p{0.15\linewidth}p{0.15\linewidth}p{0.5\linewidth}}
        \toprule
        \textbf{Data Quality Level} & \textbf{Avg. Tokens per response} & \textbf{Avg. Lines per response} & \textbf{Top 10 Frequently Occurring Keywords (in order)} \\ \midrule
        Level 1 & 230 & 9.21 & since, however, number, let, thus, which, get, two, triangle, theta \\ \midrule
        Level 2 & 444.88 & 50.68 & number, need, times, which, find, list, thus, since, triangle, sum\\ \midrule
        Level 3 & 4956.11 & 375.60 & \textbf{perhaps}, \textbf{alternatively}, \textbf{consider}, number, \textbf{wait}, which, sides, need, equal, seems \\ \midrule
        Level 4 & 4726.97 & 354.87 & \textbf{wait}, which, number, \textbf{perhaps}, therefore, let, since, \textbf{maybe}, sides, two \\ \midrule
        Level 5 & 5290.26 & 239.29 & \textbf{wait}, therefore, which, number, since, lets, two, sides, let, \textbf{maybe} \\ 
        \bottomrule
        \end{tabular}
    }
    \caption{
    Statistical analysis of models trained with examples of varying data quality. This table presents three key metrics: average token count per response, average line count per response, and frequently occurring keywords in model-generated responses. Keywords associated with reasoning transitions and uncertainty are highlighted in \textbf{bold}, with common stop words (e.g., ``a", ``the") excluded to focus on substantive language patterns. Notable differences in response length and keyword usage patterns suggest varying levels of reasoning complexity.}
    \label{tab:limo_statistics}
    }
\end{table*}