\documentclass{article}
\usepackage{amsmath}
\usepackage{enumitem}
\usepackage{xcolor}
\usepackage{tcolorbox}

\definecolor{wkblue}{rgb}{0.2, 0.3, 0.6}
\definecolor{meta-color}{rgb}{0.5, 0.5, 0.5}

\begin{document}

% Define tcolorbox style
\begin{tcolorbox}[colback=wkblue!10!white, colframe=wkblue!100!blue, left=2mm, right=2mm, title=\small\textcolor{black}{A document from \textsc{MathPile}-Ciphers}]
\begin{tiny}
\textcolor{meta-color}{\textbf{Text:}} 

\vspace{1.6mm}

\# Analysis of Cipher Example

\vspace{1.6mm}

\#\# Step 1: Initial analysis of the example cipher

\begin{itemize}
    \item Observation 1: "The phrase 'Think step by step' is 18 letters (excluding spaces)"
    \begin{itemize}
        \item Count cipher letters: "Total letters in ciphertext: 10 + 8 + 4 + 8 = 30 letters"
        \item \textbf{Hmm.}
    \end{itemize}
    \item Attempt to understand cipher to plaintext mapping
    \begin{itemize}
        \item Option 1: "Try to find mapping from letters to letters."
        \begin{itemize}
            \item "First, let's write down the ciphertext and plaintext letters on top of each other."
            \item No direct correspondence found
        \end{itemize}
        \item \textbf{Wait a minute}: "I think maybe there is an anagram or substitution cipher here."
        \item Alternatively: "Each group of ciphertext corresponds to a plaintext word."
        \begin{itemize}
            \item "Check the number of letters."
            \item "It seems the ciphertext words are twice as long as the plaintext words."
        \end{itemize}
    \end{itemize}
    \item Hypothesis: "Maybe we need to take every other letter or rebuild the plaintext from the ciphertext accordingly."
\end{itemize}

\vspace{1.6mm}

\#\# Step 2: Testing hypotheses

\begin{itemize}
    \item "Let's try this for the first word:"
    \begin{itemize}
        \item "First letters (positions 1,3,5,7,9): o f d i d"
        \item Doesn't match "Think"
    \end{itemize}
    \item Try other mapping methods
    \begin{itemize}
        \item No direct correspondence found
    \end{itemize}
    \item New idea: "What if each plaintext letter is represented by two ciphertext letters?"
    \begin{itemize}
        \item "For the first word, partition the ciphertext into pairs: oy fj dn is dr"
        \item Corresponding to "T h i n k"
    \end{itemize}
\end{itemize}

\vspace{1.6mm}

\#\# Step 3: Exploring pair-to-plaintext letter mapping rule

\begin{itemize}
    \item "Let's make a table of pairs and corresponding letters."
    \item Try method 1: Sum of letters
    \begin{itemize}
        \item "Sum: 15 + 25 = 40, but 'T' is 20."
    \end{itemize}
    \item Try method 2: Subtract letters
    \begin{itemize}
        \item "Subtract: 25 - 15 = 10. No."
    \end{itemize}
    \item Consider other possibilities
    \begin{itemize}
        \item "Think about their positions in the alphabet."
        \item "Maybe the letters are encrypted via a code."
        \item "Overlay 'Think' over cipher pairs, substituting each letter with two."
        \item "Maybe 'oyfjdnisdr' is an anagram of 'Thinkfiles'? Unlikely."
    \end{itemize}
    \item Try method 3: Modulo 26
    \begin{itemize}
        \item "40 mod 26 = 14, corresponds to 'N'. Doesn't match."
    \end{itemize}
    \item \textbf{Eureka moment}: Try averaging
    \begin{itemize}
        \item "(15 + 25)/2 = 20, which is 'T'. \textbf{Wait a minute}, that seems promising."
        \item "Success!"
    \end{itemize}
\end{itemize}

\vspace{1.6mm}

\#\# Step 4: Validating decoding method

\begin{itemize}
    \item Apply method to other pairs of the first word
    \begin{itemize}
        \item All matches successful
    \end{itemize}
    \item Test with other words
    \begin{itemize}
        \item "Second word ciphertext: rtqwainr" -> "step"
        \item "Third word: acxz" -> "by"
        \item "Fourth word: mynzbhhx" -> "step"
    \end{itemize}
\end{itemize}

\vspace{1.6mm}

\#\# Step 5: Applying decoding method to target cipher

\begin{itemize}
    \item Divide cipher into words
    \begin{itemize}
        \item "Divide the ciphertext into words accordingly."
    \end{itemize}
    \item Decode each word
    \begin{itemize}
        \item "First word: oyekaijzdf" -> "THERE"
        \item "Second word: aaptcg" -> "ARE"
        \item "Third word: suaokybhai" -> "THREE"
        \item "Fourth word: ouow" -> "RS"
        \item "Fifth word: aqht" -> "IN"
        \item "Sixth word: mynznvaatzacdfoulxxz" -> Difficulty encountered, need reconsideration
    \end{itemize}
    \item Dealing with the last word problem
    \begin{itemize}
        \item "Write the letters as they come."
        \item "Try different pairing methods."
        \item "Divide into pairs."
        \item Finally decode to "STRAWBERRY"
    \end{itemize}
\end{itemize}

\vspace{1.6mm}

\#\# Step 6: Combining final results

\begin{itemize}
    \item Put all decoded words together
    \begin{itemize}
        \item "THERE ARE THREE RS IN STRAWBERRY"
    \end{itemize}
    \item Interpret "RS" as "R's"
    \begin{itemize}
        \item "I think 'RS' is 'R's', meaning 'Rs'"
    \end{itemize}
    \item Verify final result reasonability
    \begin{itemize}
        \item "Which is true."
    \end{itemize}
\end{itemize}

\vspace{1.6mm}

\#\# Step 7: Conclude final decoded message

\begin{itemize}
    \item "So the decoded message is: 'THERE ARE THREE R'S IN STRAWBERRY'"
\end{itemize}

\vspace{1.6mm}

\textcolor{wkblue}{\rule{\linewidth}{0.4pt}}

\textcolor{meta-color}{\textbf{Subset}}: MathPile Ciphers

\textcolor{wkblue}{\rule{\linewidth}{0.4pt}}

\textcolor{meta-color}{\textbf{meta}}: 

\hspace*{4mm}book\_name: Example Cipher Decoding, 

\hspace*{4mm}type: Cipher Decryption Process, 

\hspace*{4mm}...
\end{tiny}
\end{tcolorbox}

\end{document}
