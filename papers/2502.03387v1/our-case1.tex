% \definecolor{wkblue}{rgb}{0.2, 0.3, 0.6}
% \definecolor{meta-color}{rgb}{0.5, 0.5, 0.5}

% \begin{tcolorbox}[colback=wkblue!10!white, colframe=wkblue!100!blue, left=2mm, right=2mm, title=\small\centering\textcolor{black}{The Thought Structure of Our Thought in Mathematical Reasoning}]
% \begin{small}

% \# Problem: Find the remainder when $$f(x)=x^{10}+5x^9-8x^8+7x^7-x^6-12x^5+4x^4-8x^3+12x^2-5x-5$$ is divided by $x^2-1$ without using long division.

% \vspace{1.6mm}

% \#\# Apply the Remainder Theorem

% \begin{itemize}
%     \item \textbf{We want to} find the remainder when $f(x)$ is divided by $x^2 - 1$.
%     \item Using the Remainder Theorem. 
%     \begin{itemize}
%         \item Substitute \( x = 1 \) and \( x = -1 \) into \( f(x) \) to find the remainders at those points.
%         \item \textbf{Let me compute} \( f(1) \):
%         \begin{align*}
%         f(1) &= 1^{10} + 5(1)^9 - 8(1)^8 + 7(1)^7 - 1^6 - 12(1)^5 + 4(1)^4 - 8(1)^3 + 12(1)^2 - 5(1) - 5 \\
%               &= 1 + 5 - 8 + 7 - 1 - 12 + 4 - 8 + 12 - 5 - 5 \\
%               &= -10
%         \end{align*}
%         \item \textbf{Similarly}, calculate \( f(-1) \):
%         \begin{align*}
%         f(-1) &= (-1)^{10} + 5(-1)^9 - 8(-1)^8 + 7(-1)^7 - (-1)^6 - 12(-1)^5 + 4(-1)^4 - 8(-1)^3 \\
%         &\quad\quad + 12(-1)^2 - 5(-1) - 5 \\
%               &= 1 - 5 - 8 - 7 - 1 + 12 + 4 + 8 + 12 + 5 - 5 \\
%               &= 16
%         \end{align*}
%         \item \textbf{Therefore}: The remainder when dividing by \( x - 1 \) is \(-10\), and for \( x + 1 \) it is \( 16 \).
%     \end{itemize}
% \end{itemize}

% \vspace{1.6mm}

% \#\# Deriving the remainder \( r(x) = ax + b \)

% \begin{itemize}
%     \item \textbf{Since} \( x^2 - 1 = (x - 1)(x + 1) \), the remainder must be of the form \( ax + b \).
%     \item We now solve for \( a \) and \( b \) using the remainders from \textbf{Step 1}.
%     \begin{itemize}
%         \item Set up the system of equations:
%         \begin{align*}
%         a(1) + b &= -10 \quad \text{(from \( f(1) = -10 \))} \\
%         a(-1) + b &= 16 \quad \text{(from \( f(-1) = 16 \))}
%         \end{align*}
%         \item \textbf{Let me compute}: Add the equations to find \( b \):
%         \begin{align*}
%         a + b &= -10 \\
%         -a + b &= 16 \\
%         \text{Adding the equations}: 2b &= 6 \quad \Rightarrow \quad b = 3
%         \end{align*}
%         \item Now substitute \( b = 3 \) into \( a + b = -10 \):
%         \begin{align*}
%         a + 3 &= -10 \quad \Rightarrow \quad a = -13
%         \end{align*}
%     \end{itemize}
% \end{itemize}

% \vspace{1.6mm}

% \#\# \textbf{Thus}: The remainder when \( f(x) \) is divided by \( x^2 - 1 \) is \( -13x + 3 \).
% \vspace{1.6mm}

% \#\# Final result: \( r(x) = \boxed{-13x + 3} \)



% \end{small}
% \end{tcolorbox}


\definecolor{wkblue}{rgb}{0.2, 0.3, 0.6}
\definecolor{meta-color}{rgb}{0.5, 0.5, 0.5}
\definecolor{pinkcolor}{rgb}{1.0, 0.75, 0.8}  % 柔和的粉色

\begin{tcolorbox}[colback=pinkcolor!10!white, colframe=pinkcolor!100!red, left=2mm, right=2mm, title=\small\centering\textcolor{black}{Case 2: Solution to The Remainder Problem}]
\begin{small}
\# Problem: Find the remainder when $$f(x)=x^{10}+5x^9-8x^8+7x^7-x^6-12x^5+4x^4-8x^3+12x^2-5x-5$$ is divided by $x^2-1$ without using long division.
\vspace{1.6mm}

\# Our Proposed Model Response:

\vspace{1.6mm}
First, we need to find the remainder when \( f(x) \) is divided by \( x^2 - 1 \). 
To do this, we can use the Remainder Theorem, which states that if we divide a polynomial \( f(x) \) by \( x - a \), the remainder is \( f(a) \).

Now, let's apply this to our problem. We need to find the remainder when \( f(x) \) is divided by \( x^2 - 1 \). 
This suggests that we should substitute \( x = 1 \) and \( x = -1 \) into \( f(x) \) to find the remainders.

Let's start with \( x = 1 \). Substituting this into \( f(x) \), we get:

\[
f(1) = 1^{10} + 5(1)^9 - 8(1)^8 + 7(1)^7 - (1)^6 - 12(1)^5 + 4(1)^4
- 8(1)^3 + 12(1)^2 - 5(1) - 5
\]

Calculating this step-by-step:

\[
1^{10} = 1, \quad 5(1)^9 = 5, \quad -8(1)^8 = -8, \quad 7(1)^7 = 7, \quad -(1)^6 = -1, \quad -12(1)^5 = -12, \quad 4(1)^4 = 4, ...
\]

Adding these values together, we find:

\[
f(1) = 1 + 5 - 8 + 7 - 1 - 12 + 4 - 8 + 12 - 5 - 5 = -10
\]

So, the remainder when \( f(x) \) is divided by \( x - 1 \) is \( -10 \).

Now, \textcolor{red}{let's try} \( x = -1 \):

\[
f(-1) = (-1)^{10} + 5(-1)^9 - 8(-1)^8 + 7(-1)^7 - (-1)^6 - 12(-1)^5 + 4(-1)^4 - 8(-1)^3 + 12(-1)^2 - 5(-1) - 5
\]

Calculating each term:

\[
(-1)^{10} = 1, \quad 5(-1)^9 = -5, \quad -8(-1)^8 = -8, \quad 7(-1)^7 = -7, \quad -(-1)^6 = -1, \quad -12(-1)^5 = 12, ...
\]

Adding these values, we get:

\[
f(-1) = 1 - 5 - 8 - 7 - 1 + 12 + 4 + 8 + 12 + 5 - 5 = 16
\]

Therefore, the remainder when \( f(x) \) is divided by \( x + 1 \) is \( 16 \).

\textcolor{red}{Let's pause and consider what we know so far}. We have found the remainders when \( f(x) \) is divided by \( x - 1 \) and \( x + 1 \). 
Now, we need to use these results to determine the remainder when \( f(x) \) is divided by \( x^2 - 1 \).

Since \( x^2 - 1 \) can be factored as \( (x - 1)(x + 1) \), the remainder when \( f(x) \) is divided by \( x^2 - 1 \) should be a linear expression \( ax + b \).

\textcolor{red}{Let's express} \( f(x) \) as: \(
f(x) = (x^2 - 1) q(x) + ax + b
\), where \( q(x) \) is the quotient.

\textcolor{red}{At this point}, we know:

\begin{itemize}
    \item When \( f(x) \) is divided by \( x - 1 \), the remainder is \( -10 \).
    \item When \( f(x) \) is divided by \( x + 1 \), the remainder is \( 16 \).
\end{itemize}

Let's use these to find \( a \) and \( b \).

First, consider the Remainder Theorem:

\begin{itemize}
    \item Substituting \( x = 1 \), we have: 
    \quad \(
    f(1) = a(1) + b = -10
    \)
    \item Substituting \( x = -1 \):
    \quad \quad \quad \quad \(
    f(-1) = a(-1) + b = 16
    \)
\end{itemize}

Let's solve these equations simultaneously:

From the first equation:
\quad\quad\quad\quad\quad\quad\quad \(
a + b = -10
\)

From the second equation:
\quad\quad\quad\quad\quad\quad \(
-a + b = 16
\)

Adding these equations gives:
\quad\quad\quad\quad\quad \(
2b = 6 \quad \Rightarrow \quad b = 3
\)

Substituting back into \( a + b = -10 \):
\quad\quad \(
a + 3 = -10 \quad \Rightarrow \quad a = -13
\)

So, the remainder when \( f(x) \) is divided by \( x^2 - 1 \) is:

\[
-13x + 3
\]

To summarize, the remainder is \( \boxed{-13x + 3} \).

\end{small}
\end{tcolorbox}
