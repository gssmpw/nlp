% Define tcolorbox style


\definecolor{wkblue}{rgb}{0.2, 0.3, 0.6}
\definecolor{meta-color}{rgb}{0.5, 0.5, 0.5}

\begin{tcolorbox}[colback=wkblue!10!white, colframe=wkblue!100!blue, left=2mm, right=2mm, title=\small\centering\textcolor{black}{The Thought Structure of OpenAI o1 in Mathematical Reasoning}]
\begin{small}

\# Problem: Solving the equation $p(1/x) = x^2$ 

\vspace{1.6mm}

\#\# Step 1: Define $q(x) = p(1/x) - x^2$

\begin{itemize}
    \item Analyze the roots of $q(x)$:
    \begin{itemize}
        \item \textbf{Therefore}: $q(k) = 0$ for $k = \pm 1, \pm 2, \dots, \pm n$
    \end{itemize}
    \item Let me consider: Construct $s(x) = x^{2n}q(x)$
    \begin{itemize}
        \item Analyze the properties of $s(x)$:
        \begin{itemize}
            \item \textbf{Therefore}: $s(x)$ is a polynomial
        \end{itemize}
        \item \textbf{Alternatively}: Consider $s(x) = x^{2n}p(1/x) - x^{2n+2}$
        \begin{itemize}
            \item \textbf{Let me explain}: Expand the expression for \( p(1/x) \).
        \end{itemize}
        \item Factorize $s(x) = m(x)h(x)$
        \begin{itemize}
            \item Define $m(x) = \prod(x^2 - k^2)$
            \item Hypothesize the form of $h(x)$:
            \begin{itemize}
                \item \textbf{Consider}: $h(x)$ must be a quadratic polynomial
                \item \textbf{Try}: $h(x) = -x^2 + bx + c$
                
                \item \textbf{Wait}: We need to determine the values of $b$ and $c$
                \item \textbf{Let me compute}: Determine the coefficients of \( h(x) \)      
            \end{itemize}
        \end{itemize}
    \end{itemize}
\end{itemize}

\vspace{1.6mm}

\#\# Step 2: Alternatively, construct $t(x) = x^{2n}p(1/x) = p^*(x)$

\begin{itemize}
    \item Analyze the properties of $p^*(x)$:
    \begin{itemize}
        \item \textbf{Therefore}: $p^*(x)$ is the reciprocal polynomial of $p(x)$
    \end{itemize}
    \item Consider $s(x) = t(x) - x^{2n+2}$
    \begin{itemize}
        \item \textbf{Thus}: $s(x) = p*(x) - x^(2n+2)$
    \end{itemize}
    \item Let me consider the coefficients of $( p^*(x) )$
    \begin{itemize}
        \item Analyze the constant term: \( p^*(0) = 1 \)
        \item Analyze the coefficient of the \( x^{2n} \) term: \( a_0 = c \)
    \end{itemize}
\end{itemize}

\vspace{1.6mm}

\#\# Step 3: \textbf{Let me compute}: Expand \( m(x)h(x) \)

\begin{itemize}
    \item Analyze the higher-order terms: $-x^{2n+2} + cx^{2n}$
    \item Compare with \( s(x) \): \( -x^{2n+2} + [\text{lower degree terms}] \)
    \begin{itemize}
        \item \textbf{Thus}: Confirm \( h(x) = -x^2 + c \)
    \end{itemize}
\end{itemize}


\vspace{1.6mm}

\#\# Step 4: \textbf{Wait}: Let us reconsider the value of $m(x)$ at $x = 0$

\begin{itemize}
    \item Compute: $m(0) = (-1)^n (n!)^2$
    \item Consider: \( s(0) = m(0)h(0) = 1 \)
    \item Therefore: \( c = \frac{(-1)^n}{(n!)^2} \)
    \begin{itemize}
        \item Verify: This is consistent with the previous result
    \end{itemize}
\end{itemize}

\vspace{1.6mm}

\#\# Step 5: \textbf{So}: Solve $h(x) = 0$

\begin{itemize}
    \item $-x^2 + c = 0$
    \item $x^2 = c = \frac{1}{(n!)^2}$
    \item \textbf{Therefore}:Additional solutions are $x = \pm 1/n!$
    \item \textbf{Wait}: We need to consider the case when \( n \) is even
    \begin{itemize}
        \item When \( n \) is even: \( c > 0 \)
        \item \textbf{Verify}: This ensures that the solutions are real
    \end{itemize}
\end{itemize}

\end{small}
\end{tcolorbox}


% \setlistdepth{9}

% \definecolor{wkblue}{rgb}{0.2, 0.3, 0.6}
% \definecolor{meta-color}{rgb}{0.5, 0.5, 0.5}

% \begin{tcolorbox}[colback=wkblue!10!white, colframe=wkblue!100!blue, left=2mm, right=2mm, title=\small\textcolor{black}{The Thought Structure of OpenAI o1 in Mathematical Reasoning}]
% \begin{small}

% \# Solving the equation $p(1/x) = x^2$ \hfill\textcolor{meta-color}{\# We're trying to solve a polynomial equation.}

% \vspace{1.6mm}

% \#\# Step 1: Define $q(x) = p(1/x) - x^2$ \hfill\textcolor{meta-color}{\# Define a new function to simplify the problem.}

% \begin{itemize}
%     \item Analyze the roots of $q(x)$: \hfill\textcolor{meta-color}{\# We need to find where $q(x)$ equals zero.}
%     \begin{itemize}
%         \item \textbf{Therefore}: $q(k) = 0$ for $k = \pm 1, \pm 2, \dots, \pm n$ \hfill\textcolor{meta-color}{\# The roots of $q(x)$ are $k = \pm 1, \pm 2, \dots, \pm n$.}
%     \end{itemize}
%     \item Let me consider: Construct $s(x) = x^{2n}q(x)$ \hfill\textcolor{meta-color}{\# create a new polynomial $s(x)$.}
%     \begin{itemize}
%         \item Analyze the properties of $s(x)$: \hfill\textcolor{meta-color}{\# Investigate how $s(x)$ behaves as a polynomial.}
%         \begin{itemize}
%             \item \textbf{Therefore}: $s(x)$ is a polynomial \hfill\textcolor{meta-color}{\# Confirm that $s(x)$ is a polynomial.}
%         \end{itemize}
%         \item \textbf{Alternatively}: Consider $s(x) = x^{2n}p(1/x) - x^{2n+2}$ \hfill\textcolor{meta-color}{\# Express $s(x)$ in terms of $p(1/x)$ for further analysis.}
%         \begin{itemize}
%             \item \textbf{Let me explain}: Expand the expression for \( p(1/x) \). \hfill\textcolor{meta-color}{\# We need to expand $p(1/x)$ to see its full form.}
%         \end{itemize}
%         \item Factorize $s(x) = m(x)h(x)$ \hfill\textcolor{meta-color}{\# Break $s(x)$ into simpler factors.}
%         \begin{itemize}
%             \item Define $m(x) = \prod(x^2 - k^2)$ \hfill\textcolor{meta-color}{\# Define $m(x)$ based on the product of squared terms.}
%             \item Hypothesize the form of $h(x)$: \hfill\textcolor{meta-color}{\# Now, let's guess the form of $h(x)$.}
%             \begin{itemize}
%                 \item \textbf{Consider}: $h(x)$ must be a quadratic polynomial \hfill\textcolor{meta-color}{\# Assume $h(x)$ is a quadratic polynomial.}
%                 \item \textbf{Try}: $h(x) = -x^2 + bx + c$ \hfill\textcolor{meta-color}{\# Propose a specific form for $h(x)$.}
                
%                 \item \textbf{Wait}: We need to determine the values of $b$ and $c$ \hfill\textcolor{meta-color}{\# Pause to calculate the coefficients $b$ and $c$.}
%                 \item \textbf{Let me compute}: Determine the coefficients of \( h(x) \) \hfill\textcolor{meta-color}{\# Perform the actual calculation of $b$ and $c$.}
%             \end{itemize}
%         \end{itemize}
%     \end{itemize}
% \end{itemize}

% \vspace{1.6mm}

% \#\# Step 2: Alternatively, construct $t(x) = x^{2n}p(1/x) = p^*(x)$

% \begin{itemize}
%     \item Analyze the properties of $p^*(x)$: \hfill\textcolor{meta-color}{\# Investigate the characteristics of $p^*(x)$.}
%     \begin{itemize}
%         \item \textbf{Therefore}: $p^*(x)$ is the reciprocal polynomial of $p(x)$ \hfill\textcolor{meta-color}{\# Confirm that $p^*(x)$ is a reciprocal polynomial.}
%     \end{itemize}
%     \item Consider $s(x) = t(x) - x^{2n+2}$ \hfill\textcolor{meta-color}{\# Define $s(x)$ as the difference between $t(x)$ and a higher degree term.}
%     \begin{itemize}
%         \item \textbf{Thus}: $s(x) = p*(x) - x^(2n+2)$ \hfill\textcolor{meta-color}{\# Confirm the structure of $s(x)$.}
%     \end{itemize}
%     \item Let me consider the coefficients of $( p^*(x) )$ \hfill\textcolor{meta-color}{\# Focus on analyzing the specific coefficients of $p^*(x)$.}
%     \begin{itemize}
%         \item Analyze the constant term: \( p^*(0) = 1 \) \hfill\textcolor{meta-color}{\# Note that the constant term of $p^*(x)$ is 1.}
%         \item Analyze the coefficient of the \( x^{2n} \) term: \( a_0 = c \) \hfill\textcolor{meta-color}{\# Identify the coefficient of the $x^{2n}$ term.}
%     \end{itemize}
% \end{itemize}

% \vspace{1.6mm}

% \#\# Step 3: \textbf{Let me compute}: Expand \( m(x)h(x) \) \hfill\textcolor{meta-color}{\# Now, expand the product $m(x)h(x)$ to examine its form.}

% \begin{itemize}
%     \item Analyze the higher-order terms: $-x^{2n+2} + cx^{2n}$ \hfill\textcolor{meta-color}{\# Focus on the highest degree terms in the expansion.}
%     \item Compare with \( s(x) \): \( -x^{2n+2} + [\text{lower degree terms}] \) \hfill\textcolor{meta-color}{\# Match these terms with those in $s(x)$.}
%     \begin{itemize}
%         \item \textbf{Thus}: Confirm \( h(x) = -x^2 + c \) \hfill\textcolor{meta-color}{\# Conclude that $h(x)$ has the expected form.}
%     \end{itemize}
% \end{itemize}


% \vspace{1.6mm}

% \#\# Step 4: \textbf{Wait}: Let us reconsider the value of $m(x)$ at $x = 0$ \hfill\textcolor{meta-color}{\# Pause to recalculate $m(x)$ when $x = 0$.}

% \begin{itemize}
%     \item Compute: $m(0) = (-1)^n (n!)^2$ \hfill\textcolor{meta-color}{\# Calculate $m(0)$ using a factorial expression.}
%     \item Consider: \( s(0) = m(0)h(0) = 1 \) \hfill\textcolor{meta-color}{\# Now, use $m(0)$ and $h(0)$ to find $s(0)$.}
%     \item Therefore: \( c = \frac{(-1)^n}{(n!)^2} \) \hfill\textcolor{meta-color}{\# Derive $c$ from the previous computations.}
%     \begin{itemize}
%         \item Verify: This is consistent with the previous result \hfill\textcolor{meta-color}{\# Ensure this result matches earlier findings.}
%     \end{itemize}
% \end{itemize}

% \vspace{1.6mm}

% \#\# Step 5: \textbf{So}: Solve $h(x) = 0$ \# Solve the equation $h(x) = 0$.

% \begin{itemize}
%     \item $-x^2 + c = 0$ \hfill\textcolor{meta-color}{\# Set the quadratic expression equal to zero.}
%     \item $x^2 = c = \frac{1}{(n!)^2}$ \hfill\textcolor{meta-color}{\# Solve for $x^2$.}
%     \item \textbf{Therefore}:Additional solutions are $x = \pm 1/n!$ \hfill\textcolor{meta-color}{\# The solutions for $x$ are $\pm 1/n!$.}
%     \item \textbf{Wait}: We need to consider the case when \( n \) is even \hfill\textcolor{meta-color}{\# Pause to reflect on the case where $n$ is even.}
%     \begin{itemize}
%         \item When \( n \) is even: \( c > 0 \) \hfill\textcolor{meta-color}{\# When $n$ is even, $c$ is positive.}
%         \item \textbf{Verify}: This ensures that the solutions are real \hfill\textcolor{meta-color}{\# This confirms that the solutions are real numbers.}
%     \end{itemize}
% \end{itemize}

% \end{small}
% \end{tcolorbox}

