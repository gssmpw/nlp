\definecolor{wkblue}{rgb}{0.2, 0.3, 0.6}
\definecolor{meta-color}{rgb}{0.5, 0.5, 0.5}
\definecolor{pinkcolor}{rgb}{1.0, 0.75, 0.8}  % 柔和的粉色

\begin{tcolorbox}[colback=pinkcolor!10!white, colframe=pinkcolor!100!red, left=2mm, right=2mm, title=\small\centering\textcolor{black}{Case 3: Solution to Number Theory Problem with Multiples of 30}]
\begin{small}

\# Problem: What is the least positive integer multiple of 30 that can be written with only the digits 0 and 2?

\vspace{1.6mm}

\# Our Proposed Model Response:

\vspace{1.6mm}

We need to find the least positive integer multiple of 30 that can be formed using only the digits 0 and 2.

To start, we know that a number is divisible by 30 if it is divisible by both 10 and 3. Let's address each condition separately.

First, let's consider the divisibility by 10. A number is divisible by 10 if it ends in 0. So, we need to ensure that our number ends in 0.

Now, let's focus on the divisibility by 3. A number is divisible by 3 if the sum of its digits is divisible by 3. Let's calculate the sum of the digits using only the digits 0 and 2.

The digits available are 0 and 2. The sum of these digits is \(0 + 2 = 2\).

\textcolor{red}{At this point}, we have a sum of 2, which is not divisible by 3. Let's pause and consider what this means. Since 2 is not divisible by 3, the number formed using only the digits 0 and 2 will not be divisible by 3 either.

\textcolor{red}{Let's rethink this}. Perhaps we're missing something. Maybe we can adjust the digits to make the sum divisible by 3.

\textcolor{red}{Let's reconsider} the sum of the digits. If we use both digits, 0 and 2, we can form different combinations to see if the sum becomes divisible by 3.

Let's try different combinations:
\begin{itemize}
    \item Using only 0, the sum is 0.
    \item Using only 2, the sum is 2.
    \item Using both, the sum can be \(0 + 2 = 2\) or \(2 + 0 = 2\).
\end{itemize}

Clearly, the sum of 2 is not divisible by 3. Let's try combining the digits differently.

\textcolor{red}{Wait, maybe I'm missing something again}. Let's reconsider the combinations. If we use both digits, we can form larger numbers by arranging them differently. Let's try:
\begin{itemize}
    \item 20: Sum is \(2 + 0 = 2\).
    \item 22: Sum is \(2 + 2 = 4\).
    \item 202: Sum is \(2 + 0 + 2 = 4\).
\end{itemize}

Still, the sums of 2 and 4 are not divisible by 3. Let's try something different.

Let's try using larger numbers formed by rearranging the digits:
\begin{itemize}
    \item 200: Sum is \(2 + 0 + 0 = 2\).
    \item 220: Sum is \(2 + 2 + 0 = 4\).
    \item 2020: Sum is \(2 + 0 + 2 + 0 = 4\).
\end{itemize}

Still, the sums of 2 and 4 are not divisible by 3. Let's try one more approach.

Let's consider using more digits:
\begin{itemize}
    \item 2220: Sum is \(2 + 2 + 2 + 0 = 6\).
\end{itemize}

Now, the sum of 6 is divisible by 3. Let's pause and consider what this means. Since 2220 ends in 0, it is divisible by 10. Therefore, 2220 is divisible by both 10 and 3, making it divisible by 30.

So, the least positive integer multiple of 30 that can be formed using only the digits 0 and 2 is \( \boxed{2220} \).

\end{small}
\end{tcolorbox}
