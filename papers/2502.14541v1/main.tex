  % This must be in the first 5 lines to tell arXiv to use pdfLaTeX, which is strongly recommended.
\pdfoutput=1
% In particular, the hyperref package requires pdfLaTeX in order to break URLs across lines.

\documentclass[11pt]{article}

\newcommand\blfootnote[1]{%
  \begingroup
  \renewcommand\thefootnote{}\footnote{#1}%
  \addtocounter{footnote}{-1}%
  \endgroup
}


% Remove the "review" option to generate the final version.
\usepackage[]{acl}

% Standard package includes


\usepackage{times}
\usepackage{latexsym}
\usepackage{comment}
\usepackage{color}
\usepackage{makecell}
%\usepackage{xcolor}
\usepackage{colortbl}
% For proper rendering and hyphenation of words containing Latin characters (including in bib files)
\usepackage[T1]{fontenc}
% For Vietnamese characters
% \usepackage[T5]{fontenc}
% See https://www.latex-project.org/help/documentation/encguide.pdf for other character sets

% This assumes your files are encoded as UTF8
\usepackage[utf8]{inputenc}

% This is not strictly necessary, and may be commented out,
% but it will improve the layout of the manuscript,
% and will typically save some space.
\usepackage{microtype}
\usepackage[most]{tcolorbox}
% This is also not strictly necessary, and may be commented out.
% However, it will improve the aesthetics of text in
% the typewriter font.
\usepackage{inconsolata}

\usepackage{mathtools}
\usepackage{makecell}



\usepackage{array,multirow,graphicx}
\usepackage{amsmath}

\usepackage{amssymb}
\usepackage{booktabs}
% \usepackage{algorithm}
\usepackage{algpseudocode}

\usepackage[export]{adjustbox}
\usepackage{float}

% Support for easy cross-referencing
%\newcommand\algorithmautorefname{Algorithm}
\usepackage[capitalize]{cleveref}
\crefname{section}{Sec.}{Secs.}
%\Crefname{section}{Section}{Secs.}
\Crefname{table}{Table}{Tables}
\crefname{table}{Tab.}{Tabs.}

\def\subsectionautorefname{Sec.}
\def\sectionautorefname{Sec.}

% \renewcommand*{\algorithmautorefname}{Algorithm}
\renewcommand*{\figureautorefname}{Fig.}
\renewcommand*{\equationautorefname}{Eq.}
\renewcommand*{\tableautorefname}{Tab.}


% \renewcommand\jt[1]{\textcolor{black}{#1}}


\newcommand{\llama}{LLaMA}
\newcommand{\myalg}{\code{PURE}}
\newcommand{\tr}{\textrm{tr}}
\newcommand{\per}{\textrm{c}}
\newcommand{\pool}{pool}
\newcommand{\mytitle}{LLM-based User Profile Management for Recommender System}
% \newcommand{\mytitle}{MemRec: LLM-based Recommender System using Memory Update}
% \newcommand{\mytitle}{FRAG: Feedback based RAG Enhancement without Training}

%%%%% NEW MATH DEFINITIONS %%%%%

\usepackage{amsmath,amsfonts,bm}

% Mark sections of captions for referring to divisions of figures
\newcommand{\figleft}{{\em (Left)}}
\newcommand{\figcenter}{{\em (Center)}}
\newcommand{\figright}{{\em (Right)}}
\newcommand{\figtop}{{\em (Top)}}
\newcommand{\figbottom}{{\em (Bottom)}}
\newcommand{\captiona}{{\em (a)}}
\newcommand{\captionb}{{\em (b)}}
\newcommand{\captionc}{{\em (c)}}
\newcommand{\captiond}{{\em (d)}}

% Highlight a newly defined term
\newcommand{\newterm}[1]{{\bf #1}}


% Figure reference, lower-case.
\def\figref#1{figure~\ref{#1}}
% Figure reference, capital. For start of sentence
\def\Figref#1{Figure~\ref{#1}}
\def\twofigref#1#2{figures \ref{#1} and \ref{#2}}
\def\quadfigref#1#2#3#4{figures \ref{#1}, \ref{#2}, \ref{#3} and \ref{#4}}
% Section reference, lower-case.
\def\secref#1{section~\ref{#1}}
% Section reference, capital.
\def\Secref#1{Section~\ref{#1}}
% Reference to two sections.
\def\twosecrefs#1#2{sections \ref{#1} and \ref{#2}}
% Reference to three sections.
\def\secrefs#1#2#3{sections \ref{#1}, \ref{#2} and \ref{#3}}
% Reference to an equation, lower-case.
\def\eqref#1{equation~\ref{#1}}
% Reference to an equation, upper case
\def\Eqref#1{Equation~\ref{#1}}
% A raw reference to an equation---avoid using if possible
\def\plaineqref#1{\ref{#1}}
% Reference to a chapter, lower-case.
\def\chapref#1{chapter~\ref{#1}}
% Reference to an equation, upper case.
\def\Chapref#1{Chapter~\ref{#1}}
% Reference to a range of chapters
\def\rangechapref#1#2{chapters\ref{#1}--\ref{#2}}
% Reference to an algorithm, lower-case.
\def\algref#1{algorithm~\ref{#1}}
% Reference to an algorithm, upper case.
\def\Algref#1{Algorithm~\ref{#1}}
\def\twoalgref#1#2{algorithms \ref{#1} and \ref{#2}}
\def\Twoalgref#1#2{Algorithms \ref{#1} and \ref{#2}}
% Reference to a part, lower case
\def\partref#1{part~\ref{#1}}
% Reference to a part, upper case
\def\Partref#1{Part~\ref{#1}}
\def\twopartref#1#2{parts \ref{#1} and \ref{#2}}

\def\ceil#1{\lceil #1 \rceil}
\def\floor#1{\lfloor #1 \rfloor}
\def\1{\bm{1}}
\newcommand{\train}{\mathcal{D}}
\newcommand{\valid}{\mathcal{D_{\mathrm{valid}}}}
\newcommand{\test}{\mathcal{D_{\mathrm{test}}}}

\def\eps{{\epsilon}}


% Random variables
\def\reta{{\textnormal{$\eta$}}}
\def\ra{{\textnormal{a}}}
\def\rb{{\textnormal{b}}}
\def\rc{{\textnormal{c}}}
\def\rd{{\textnormal{d}}}
\def\re{{\textnormal{e}}}
\def\rf{{\textnormal{f}}}
\def\rg{{\textnormal{g}}}
\def\rh{{\textnormal{h}}}
\def\ri{{\textnormal{i}}}
\def\rj{{\textnormal{j}}}
\def\rk{{\textnormal{k}}}
\def\rl{{\textnormal{l}}}
% rm is already a command, just don't name any random variables m
\def\rn{{\textnormal{n}}}
\def\ro{{\textnormal{o}}}
\def\rp{{\textnormal{p}}}
\def\rq{{\textnormal{q}}}
\def\rr{{\textnormal{r}}}
\def\rs{{\textnormal{s}}}
\def\rt{{\textnormal{t}}}
\def\ru{{\textnormal{u}}}
\def\rv{{\textnormal{v}}}
\def\rw{{\textnormal{w}}}
\def\rx{{\textnormal{x}}}
\def\ry{{\textnormal{y}}}
\def\rz{{\textnormal{z}}}

% Random vectors
\def\rvepsilon{{\mathbf{\epsilon}}}
\def\rvtheta{{\mathbf{\theta}}}
\def\rva{{\mathbf{a}}}
\def\rvb{{\mathbf{b}}}
\def\rvc{{\mathbf{c}}}
\def\rvd{{\mathbf{d}}}
\def\rve{{\mathbf{e}}}
\def\rvf{{\mathbf{f}}}
\def\rvg{{\mathbf{g}}}
\def\rvh{{\mathbf{h}}}
\def\rvu{{\mathbf{i}}}
\def\rvj{{\mathbf{j}}}
\def\rvk{{\mathbf{k}}}
\def\rvl{{\mathbf{l}}}
\def\rvm{{\mathbf{m}}}
\def\rvn{{\mathbf{n}}}
\def\rvo{{\mathbf{o}}}
\def\rvp{{\mathbf{p}}}
\def\rvq{{\mathbf{q}}}
\def\rvr{{\mathbf{r}}}
\def\rvs{{\mathbf{s}}}
\def\rvt{{\mathbf{t}}}
\def\rvu{{\mathbf{u}}}
\def\rvv{{\mathbf{v}}}
\def\rvw{{\mathbf{w}}}
\def\rvx{{\mathbf{x}}}
\def\rvy{{\mathbf{y}}}
\def\rvz{{\mathbf{z}}}

% Elements of random vectors
\def\erva{{\textnormal{a}}}
\def\ervb{{\textnormal{b}}}
\def\ervc{{\textnormal{c}}}
\def\ervd{{\textnormal{d}}}
\def\erve{{\textnormal{e}}}
\def\ervf{{\textnormal{f}}}
\def\ervg{{\textnormal{g}}}
\def\ervh{{\textnormal{h}}}
\def\ervi{{\textnormal{i}}}
\def\ervj{{\textnormal{j}}}
\def\ervk{{\textnormal{k}}}
\def\ervl{{\textnormal{l}}}
\def\ervm{{\textnormal{m}}}
\def\ervn{{\textnormal{n}}}
\def\ervo{{\textnormal{o}}}
\def\ervp{{\textnormal{p}}}
\def\ervq{{\textnormal{q}}}
\def\ervr{{\textnormal{r}}}
\def\ervs{{\textnormal{s}}}
\def\ervt{{\textnormal{t}}}
\def\ervu{{\textnormal{u}}}
\def\ervv{{\textnormal{v}}}
\def\ervw{{\textnormal{w}}}
\def\ervx{{\textnormal{x}}}
\def\ervy{{\textnormal{y}}}
\def\ervz{{\textnormal{z}}}

% Random matrices
\def\rmA{{\mathbf{A}}}
\def\rmB{{\mathbf{B}}}
\def\rmC{{\mathbf{C}}}
\def\rmD{{\mathbf{D}}}
\def\rmE{{\mathbf{E}}}
\def\rmF{{\mathbf{F}}}
\def\rmG{{\mathbf{G}}}
\def\rmH{{\mathbf{H}}}
\def\rmI{{\mathbf{I}}}
\def\rmJ{{\mathbf{J}}}
\def\rmK{{\mathbf{K}}}
\def\rmL{{\mathbf{L}}}
\def\rmM{{\mathbf{M}}}
\def\rmN{{\mathbf{N}}}
\def\rmO{{\mathbf{O}}}
\def\rmP{{\mathbf{P}}}
\def\rmQ{{\mathbf{Q}}}
\def\rmR{{\mathbf{R}}}
\def\rmS{{\mathbf{S}}}
\def\rmT{{\mathbf{T}}}
\def\rmU{{\mathbf{U}}}
\def\rmV{{\mathbf{V}}}
\def\rmW{{\mathbf{W}}}
\def\rmX{{\mathbf{X}}}
\def\rmY{{\mathbf{Y}}}
\def\rmZ{{\mathbf{Z}}}

% Elements of random matrices
\def\ermA{{\textnormal{A}}}
\def\ermB{{\textnormal{B}}}
\def\ermC{{\textnormal{C}}}
\def\ermD{{\textnormal{D}}}
\def\ermE{{\textnormal{E}}}
\def\ermF{{\textnormal{F}}}
\def\ermG{{\textnormal{G}}}
\def\ermH{{\textnormal{H}}}
\def\ermI{{\textnormal{I}}}
\def\ermJ{{\textnormal{J}}}
\def\ermK{{\textnormal{K}}}
\def\ermL{{\textnormal{L}}}
\def\ermM{{\textnormal{M}}}
\def\ermN{{\textnormal{N}}}
\def\ermO{{\textnormal{O}}}
\def\ermP{{\textnormal{P}}}
\def\ermQ{{\textnormal{Q}}}
\def\ermR{{\textnormal{R}}}
\def\ermS{{\textnormal{S}}}
\def\ermT{{\textnormal{T}}}
\def\ermU{{\textnormal{U}}}
\def\ermV{{\textnormal{V}}}
\def\ermW{{\textnormal{W}}}
\def\ermX{{\textnormal{X}}}
\def\ermY{{\textnormal{Y}}}
\def\ermZ{{\textnormal{Z}}}

% Vectors
\def\vzero{{\bm{0}}}
\def\vone{{\bm{1}}}
\def\vmu{{\bm{\mu}}}
\def\vtheta{{\bm{\theta}}}
\def\va{{\bm{a}}}
\def\vb{{\bm{b}}}
\def\vc{{\bm{c}}}
\def\vd{{\bm{d}}}
\def\ve{{\bm{e}}}
\def\vf{{\bm{f}}}
\def\vg{{\bm{g}}}
\def\vh{{\bm{h}}}
\def\vi{{\bm{i}}}
\def\vj{{\bm{j}}}
\def\vk{{\bm{k}}}
\def\vl{{\bm{l}}}
\def\vm{{\bm{m}}}
\def\vn{{\bm{n}}}
\def\vo{{\bm{o}}}
\def\vp{{\bm{p}}}
\def\vq{{\bm{q}}}
\def\vr{{\bm{r}}}
\def\vs{{\bm{s}}}
\def\vt{{\bm{t}}}
\def\vu{{\bm{u}}}
\def\vv{{\bm{v}}}
\def\vw{{\bm{w}}}
\def\vx{{\bm{x}}}
\def\vy{{\bm{y}}}
\def\vz{{\bm{z}}}

% Elements of vectors
\def\evalpha{{\alpha}}
\def\evbeta{{\beta}}
\def\evepsilon{{\epsilon}}
\def\evlambda{{\lambda}}
\def\evomega{{\omega}}
\def\evmu{{\mu}}
\def\evpsi{{\psi}}
\def\evsigma{{\sigma}}
\def\evtheta{{\theta}}
\def\eva{{a}}
\def\evb{{b}}
\def\evc{{c}}
\def\evd{{d}}
\def\eve{{e}}
\def\evf{{f}}
\def\evg{{g}}
\def\evh{{h}}
\def\evi{{i}}
\def\evj{{j}}
\def\evk{{k}}
\def\evl{{l}}
\def\evm{{m}}
\def\evn{{n}}
\def\evo{{o}}
\def\evp{{p}}
\def\evq{{q}}
\def\evr{{r}}
\def\evs{{s}}
\def\evt{{t}}
\def\evu{{u}}
\def\evv{{v}}
\def\evw{{w}}
\def\evx{{x}}
\def\evy{{y}}
\def\evz{{z}}

% Matrix
\def\mA{{\bm{A}}}
\def\mB{{\bm{B}}}
\def\mC{{\bm{C}}}
\def\mD{{\bm{D}}}
\def\mE{{\bm{E}}}
\def\mF{{\bm{F}}}
\def\mG{{\bm{G}}}
\def\mH{{\bm{H}}}
\def\mI{{\bm{I}}}
\def\mJ{{\bm{J}}}
\def\mK{{\bm{K}}}
\def\mL{{\bm{L}}}
\def\mM{{\bm{M}}}
\def\mN{{\bm{N}}}
\def\mO{{\bm{O}}}
\def\mP{{\bm{P}}}
\def\mQ{{\bm{Q}}}
\def\mR{{\bm{R}}}
\def\mS{{\bm{S}}}
\def\mT{{\bm{T}}}
\def\mU{{\bm{U}}}
\def\mV{{\bm{V}}}
\def\mW{{\bm{W}}}
\def\mX{{\bm{X}}}
\def\mY{{\bm{Y}}}
\def\mZ{{\bm{Z}}}
\def\mBeta{{\bm{\beta}}}
\def\mPhi{{\bm{\Phi}}}
\def\mLambda{{\bm{\Lambda}}}
\def\mSigma{{\bm{\Sigma}}}

% Tensor
\DeclareMathAlphabet{\mathsfit}{\encodingdefault}{\sfdefault}{m}{sl}
\SetMathAlphabet{\mathsfit}{bold}{\encodingdefault}{\sfdefault}{bx}{n}
\newcommand{\tens}[1]{\bm{\mathsfit{#1}}}
\def\tA{{\tens{A}}}
\def\tB{{\tens{B}}}
\def\tC{{\tens{C}}}
\def\tD{{\tens{D}}}
\def\tE{{\tens{E}}}
\def\tF{{\tens{F}}}
\def\tG{{\tens{G}}}
\def\tH{{\tens{H}}}
\def\tI{{\tens{I}}}
\def\tJ{{\tens{J}}}
\def\tK{{\tens{K}}}
\def\tL{{\tens{L}}}
\def\tM{{\tens{M}}}
\def\tN{{\tens{N}}}
\def\tO{{\tens{O}}}
\def\tP{{\tens{P}}}
\def\tQ{{\tens{Q}}}
\def\tR{{\tens{R}}}
\def\tS{{\tens{S}}}
\def\tT{{\tens{T}}}
\def\tU{{\tens{U}}}
\def\tV{{\tens{V}}}
\def\tW{{\tens{W}}}
\def\tX{{\tens{X}}}
\def\tY{{\tens{Y}}}
\def\tZ{{\tens{Z}}}


% Graph
\def\gA{{\mathcal{A}}}
\def\gB{{\mathcal{B}}}
\def\gC{{\mathcal{C}}}
\def\gD{{\mathcal{D}}}
\def\gE{{\mathcal{E}}}
\def\gF{{\mathcal{F}}}
\def\gG{{\mathcal{G}}}
\def\gH{{\mathcal{H}}}
\def\gI{{\mathcal{I}}}
\def\gJ{{\mathcal{J}}}
\def\gK{{\mathcal{K}}}
\def\gL{{\mathcal{L}}}
\def\gM{{\mathcal{M}}}
\def\gN{{\mathcal{N}}}
\def\gO{{\mathcal{O}}}
\def\gP{{\mathcal{P}}}
\def\gQ{{\mathcal{Q}}}
\def\gR{{\mathcal{R}}}
\def\gS{{\mathcal{S}}}
\def\gT{{\mathcal{T}}}
\def\gU{{\mathcal{U}}}
\def\gV{{\mathcal{V}}}
\def\gW{{\mathcal{W}}}
\def\gX{{\mathcal{X}}}
\def\gY{{\mathcal{Y}}}
\def\gZ{{\mathcal{Z}}}

% Sets
\def\sA{{\mathbb{A}}}
\def\sB{{\mathbb{B}}}
\def\sC{{\mathbb{C}}}
\def\sD{{\mathbb{D}}}
% Don't use a set called E, because this would be the same as our symbol
% for expectation.
\def\sF{{\mathbb{F}}}
\def\sG{{\mathbb{G}}}
\def\sH{{\mathbb{H}}}
\def\sI{{\mathbb{I}}}
\def\sJ{{\mathbb{J}}}
\def\sK{{\mathbb{K}}}
\def\sL{{\mathbb{L}}}
\def\sM{{\mathbb{M}}}
\def\sN{{\mathbb{N}}}
\def\sO{{\mathbb{O}}}
\def\sP{{\mathbb{P}}}
\def\sQ{{\mathbb{Q}}}
\def\sR{{\mathbb{R}}}
\def\sS{{\mathbb{S}}}
\def\sT{{\mathbb{T}}}
\def\sU{{\mathbb{U}}}
\def\sV{{\mathbb{V}}}
\def\sW{{\mathbb{W}}}
\def\sX{{\mathbb{X}}}
\def\sY{{\mathbb{Y}}}
\def\sZ{{\mathbb{Z}}}

% Entries of a matrix
\def\emLambda{{\Lambda}}
\def\emA{{A}}
\def\emB{{B}}
\def\emC{{C}}
\def\emD{{D}}
\def\emE{{E}}
\def\emF{{F}}
\def\emG{{G}}
\def\emH{{H}}
\def\emI{{I}}
\def\emJ{{J}}
\def\emK{{K}}
\def\emL{{L}}
\def\emM{{M}}
\def\emN{{N}}
\def\emO{{O}}
\def\emP{{P}}
\def\emQ{{Q}}
\def\emR{{R}}
\def\emS{{S}}
\def\emT{{T}}
\def\emU{{U}}
\def\emV{{V}}
\def\emW{{W}}
\def\emX{{X}}
\def\emY{{Y}}
\def\emZ{{Z}}
\def\emSigma{{\Sigma}}

% entries of a tensor
% Same font as tensor, without \bm wrapper
\newcommand{\etens}[1]{\mathsfit{#1}}
\def\etLambda{{\etens{\Lambda}}}
\def\etA{{\etens{A}}}
\def\etB{{\etens{B}}}
\def\etC{{\etens{C}}}
\def\etD{{\etens{D}}}
\def\etE{{\etens{E}}}
\def\etF{{\etens{F}}}
\def\etG{{\etens{G}}}
\def\etH{{\etens{H}}}
\def\etI{{\etens{I}}}
\def\etJ{{\etens{J}}}
\def\etK{{\etens{K}}}
\def\etL{{\etens{L}}}
\def\etM{{\etens{M}}}
\def\etN{{\etens{N}}}
\def\etO{{\etens{O}}}
\def\etP{{\etens{P}}}
\def\etQ{{\etens{Q}}}
\def\etR{{\etens{R}}}
\def\etS{{\etens{S}}}
\def\etT{{\etens{T}}}
\def\etU{{\etens{U}}}
\def\etV{{\etens{V}}}
\def\etW{{\etens{W}}}
\def\etX{{\etens{X}}}
\def\etY{{\etens{Y}}}
\def\etZ{{\etens{Z}}}

% The true underlying data generating distribution
\newcommand{\pdata}{p_{\rm{data}}}
% The empirical distribution defined by the training set
\newcommand{\ptrain}{\hat{p}_{\rm{data}}}
\newcommand{\Ptrain}{\hat{P}_{\rm{data}}}
% The model distribution
\newcommand{\pmodel}{p_{\rm{model}}}
\newcommand{\Pmodel}{P_{\rm{model}}}
\newcommand{\ptildemodel}{\tilde{p}_{\rm{model}}}
% Stochastic autoencoder distributions
\newcommand{\pencode}{p_{\rm{encoder}}}
\newcommand{\pdecode}{p_{\rm{decoder}}}
\newcommand{\precons}{p_{\rm{reconstruct}}}

\newcommand{\laplace}{\mathrm{Laplace}} % Laplace distribution

\newcommand{\E}{\mathbb{E}}
\newcommand{\Ls}{\mathcal{L}}
% \newcommand{\R}{\mathbb{R}}
\newcommand{\emp}{\tilde{p}}
\newcommand{\lr}{\alpha}
\newcommand{\reg}{\lambda}
\newcommand{\rect}{\mathrm{rectifier}}
\newcommand{\softmax}{\mathrm{softmax}}
\newcommand{\sigmoid}{\sigma}
\newcommand{\softplus}{\zeta}
\newcommand{\KL}{D_{\mathrm{KL}}}
\newcommand{\Var}{\mathrm{Var}}
\newcommand{\standarderror}{\mathrm{SE}}
\newcommand{\Cov}{\mathrm{Cov}}
% Wolfram Mathworld says $L^2$ is for function spaces and $\ell^2$ is for vectors
% But then they seem to use $L^2$ for vectors throughout the site, and so does
% wikipedia.
\newcommand{\normlzero}{L^0}
\newcommand{\normlone}{L^1}
\newcommand{\normltwo}{L^2}
\newcommand{\normlp}{L^p}
\newcommand{\normmax}{L^\infty}

\newcommand{\parents}{Pa} % See usage in notation.tex. Chosen to match Daphne's book.


% Customized Math
\DeclareMathOperator{\Tr}{Tr}

\DeclareMathOperator*{\argmax}{arg\,max}
\DeclareMathOperator*{\argmin}{arg\,min}

\DeclareMathOperator{\sign}{sign}
\let\ab\allowbreak

\newcommand{\trace}{\Tr}

% \usepackage{microtype}
\usepackage{geometry}
% \usepackage{subfig}
\usepackage{booktabs} 
\usepackage{bbm}
\usepackage{mathtools}
% \usepackage{amsthm}
\usepackage{nccmath}
\usepackage{setspace}

\usepackage{caption}
\usepackage{subcaption}

\usepackage[linesnumbered,ruled,vlined]{algorithm2e}
% \usepackage{algorithmic}
% \usepackage{algorithm}

\SetKwInput{KwInput}{Input}                % Set the Input
\SetKwInput{KwOutput}{Output}              % set the Output
\newcommand\mycommfont[1]{\footnotesize\ttfamily\textcolor{blue}{#1}}
\SetCommentSty{mycommfont}
\newcommand{\algcapsty}[1]{\small\sffamily\bfseries{#1}}
\SetAlCapSty{algcapsty}

\usepackage[T1]{fontenc}
\usepackage{wrapfig,lipsum,booktabs}

% \usepackage{natbib}
\usepackage{soul}
\usepackage{dsfont}
\usepackage{enumerate}
\usepackage{enumitem}

% \usepackage{kotex}
% \usepackage{hyperref}
% \usepackage[hidelinks]{hyperref}
% \usepackage{amsmath}
% \usepackage{amsthm}
\usepackage{amsfonts}
\usepackage{bbm}
\usepackage{dsfont}
\usepackage[Symbol]{upgreek}
\usepackage{lscape}
\usepackage{caption}
\usepackage{balance}
\usepackage{xspace}
\usepackage{float}
\usepackage{kotex}

\usepackage{wasysym}
%\usepackage[table,xcdraw,dvipsnames]{xcolor}
\usepackage{xcolor}
\usepackage{multirow}
\usepackage{array, boldline, rotating}

\usepackage{amssymb}% http://ctan.org/pkg/amssymb
\usepackage{pifont}% http://ctan.org/pkg/pifont
\newcommand{\cmark}{\ding{51}\xspace}%
\newcommand{\omark}{\textbf{$\mathcal{O}$}\xspace}%
\newcommand{\xmark}{\ding{55}\xspace}%

\newcommand{\ds}[1]{\mathds{#1}}
\newcommand{\mc}[1]{\mathcal{#1}}
\newcommand{\bb}[1]{\mathbbm{#1}}

% %%%%%% Theorem Related Things %%%%%%
% \theoremstyle{plain}
% \newtheorem{thm}{Theorem}
% \newtheorem{cor}{Corollary}
% \newtheorem{lem}{Lemma}
% \newtheorem{prop}{Proposition}

% \theoremstyle{definition}
% \newtheorem{defn}{Definition}
% \newtheorem{assum}{Assumption}



% Citation

% \let\oldeqcite\cite
% \renewcommand*\cite[1]{(\oldcite{#1})}
\let\oldeqref\eqref
\renewcommand*\eqref[1]{(\ref{#1})}

% % Highlight (incl. note)
\newcommand{\smnote}[1]{\textbf{\textcolor{Cyan}{SM: #1}}}
\newcommand{\jhnote}[1]{\textbf{\textcolor{Orange}{JH: #1}}}
\newcommand{\yes}[1]{\textcolor{blue}{[YES]}}
\newcommand{\no}[1]{\textcolor{orange}{[NO]}}
\newcommand{\na}[1]{\textcolor{gray}{[N/A]}}
%\newcommand\bg[1]{\textcolor{blue}{#1}} % JH
\newcommand\jt[1]{\textcolor{brown}{#1}} % JT
\newcommand\jh[1]{\textcolor{black}{#1}} % JH
\newcommand\sm[1]{\textcolor{blue}{#1}} % SM
\newcommand{\eg}{\emph{e.g.,~}}
\newcommand{\ie}{\emph{i.e.,~}}


% \renewcommand\jt[1]{\textcolor{black}{#1}} % JT
% \renewcommand\jh[1]{\textcolor{black}{#1}} % JH

% % Separation (paragraph)
\newcommand{\myparagraph}[1]{\vspace{0.07cm}\noindent\textbf{#1}~}

% % math op.

% % Font
\def\code#1{\texttt{#1}}
\DeclarePairedDelimiter\norm{\lVert}{\rVert}

% % % Definition
% \theoremstyle{definition}
\newcommand\scalemath[2]{\scalebox{#1}{\mbox{\ensuremath{\displaystyle #2}}}}



\newcommand{\thickhline}{\hlineB{4}}
\newcommand{\bfcode}[1]{\code{\textbf{#1}}}


\definecolor{LightCyan}{rgb}{0.88,1,1}
\definecolor{Blue}{rgb}{0, 0.3, 0.6}
\definecolor{Orange}{rgb}{0.8, 0.4, 0}
\definecolor{Green}{rgb}{0.0, 0.8, 0.0 }
\definecolor{Red}{rgb}{0.95, 0.55, 0.6}
\definecolor{Skyblue}{rgb}{0.6, 0.6, 0.95 }



% Supplementary title
\NewDocumentEnvironment{suptitle}{ +b }{
    \twocolumn[{#1}]%
}{}

\NewDocumentCommand{\supptitle}{s}{
\begin{suptitle}
        \centering
        % \rule{\textwidth}{0.07cm}\\[-0.34cm]
        \rule{\textwidth}{0.03cm}\\[0.1cm]
        -Supplementary Material-\\[0.2cm]
        {\Large 
            \textbf{\mytitle }
        }\\%[0.40cm]
        \rule{\textwidth}{0.03cm}\\[0.2cm]
\end{suptitle}}
% If the title and author information does not fit in the area allocated, uncomment the following
%
%\setlength\titlebox{<dim>}
%
% and set <dim> to something 5cm or larger.

%\title{\alg: Instant Personalized LoRA Generation for On-device and Hybrid Decoding}
% \title{OPA: On-the-Fly Personalized Adapter \& Device-Server Consistent Inference for On-Device LLM}
%\title{On-Palette: On-the-fly Person-Adapted On-device LLM and Edge-to-server Transfer for Hybrid Inference}
% \title{On-Device Palette: Personalized On-the-Fly Adapter and Edge-Server Hybrid Inference}
\title{\mytitle}
% \title{Palette: Person-Aaptation of LLM On-the-Fly and Edge-Server Transit for }


%\title{Palette: Personalized Adapter for LLM Edge device T... Toward End device.}
% PersonAr,
% Personalized On-the-Fly Adapter and Device-Server Hybrid Inference for On-Device LLM.  


% Author information can be set in various styles:
% For several authors from the same institution:
% \author{Author 1 \and ... \and Author n \\
%         Address line \\ ... \\ Address line}
% if the names do not fit well on one line use
%         Author 1 \\ {\bf Author 2} \\ ... \\ {\bf Author n} \\
% For authors from different institutions:
% \author{Author 1 \\ Address line \\  ... \\ Address line
%         \And  ... \And
%         Author n \\ Address line \\ ... \\ Address line}
% To start a separate ``row'' of authors use \AND, as in
% \author{Author 1 \\ Address line \\  ... \\ Address line
%         \AND
%         Author 2 \\ Address line \\ ... \\ Address line \And
%         Author 3 \\ Address line \\ ... \\ Address line}

\author{%
    \textbf{Seunghwan Bang}$^{1}$ \hspace{1em} \textbf{Hwanjun Song}$^{2, \dagger}$\\
    $^{1}$Ulsan National Institute of Science and Technology \\
    $^{2}$Korea Advanced Institute of Science and Technology \\
    \texttt{shbang1422@unist.ac.kr} \\ \texttt{songhwanjun@kaist.ac.kr}
}

\begin{document}
\maketitle
\blfootnote{$\dagger$ indicates the corresponding author.}
\begin{abstract}
% The rapid advancement of Large Language Models (LLMs) has enabled zero-shot recommendation without conventional training. However, most approaches rely solely on purchase histories, overlooking the rich information in user-generated texts. We propose \myalg{}, an LLM-based framework that builds evolving user profiles by extracting and summarizing key information from reviews. \myalg{} combines \emph{Review Extractor}, \emph{Profile Updater}, and \emph{Recommender} to deliver personalized suggestions while managing token limitations. To evaluate \myalg{}, we introduce a novel continuous sequential recommendation task, which more realistically reflects real-world scenarios by updating predictions incrementally as new reviews arrive. Experiments on Amazon data show that \myalg{} significantly improves performance over existing LLM-based methods.
The rapid advancement of Large Language Models (LLMs) has opened new opportunities in recommender systems by enabling zero-shot recommendation without conventional training. Despite their potential, most existing works rely solely on users' purchase histories, leaving significant room for improvement by incorporating user-generated textual data, such as reviews and product descriptions. Addressing this gap, we propose \myalg{}, a novel LLM-based recommendation framework that builds and maintains evolving user profiles by systematically extracting and summarizing key information from user reviews. \myalg{} consists of three core components: a Review Extractor for identifying user preferences and key product features, a Profile Updater for refining and updating user profiles, and a Recommender for generating personalized recommendations using the most current profile. To evaluate \myalg{}, we introduce a continuous sequential recommendation task that reflects real-world scenarios by adding reviews over time and updating predictions incrementally. Our experimental results on Amazon datasets demonstrate that \myalg{} outperforms existing LLM-based methods, effectively leveraging long-term user information while managing token limitations.
% The rise of Large Language Models (LLMs) has significantly advanced the field of recommender systems, enabling more sophisticated and personalized predictions. However, most existing LLM-based recommender systems rely primarily on users' purchase history, overlooking the valuable insights embedded in user-generated reviews. In this paper, we propose \myalg{}, a novel framework that integrates user reviews into the recommendation process, addressing the challenges posed by LLMs' token limitations. Our approach systematically extracts and stores key information from user reviews, ensuring scalability as user-generated data grows, while retaining only the most relevant content for accurate and efficient recommendations. We further introduce a realistic sequential recommendation setup, where reviews are incrementally added over time, enabling the model to dynamically update user profiles and predict the next purchase based on evolving user preferences. Our experimental results demonstrate that \myalg{} outperforms existing LLM-based recommendation methods that are train-free on Amazon datasets, validating the effectiveness of our approach in providing more accurate and scalable recommendations. 

\end{abstract}
%\jt{}
% \blfootnote{$\dagger$ indicates the corresponding author.}

\section{Introduction}\label{sec:Intro} 


Novel view synthesis offers a fundamental approach to visualizing complex scenes by generating new perspectives from existing imagery. 
This has many potential applications, including virtual reality, movie production and architectural visualization \cite{Tewari2022NeuRendSTAR}. 
An emerging alternative to the common RGB sensors are event cameras, which are  
 bio-inspired visual sensors recording events, i.e.~asynchronous per-pixel signals of changes in brightness or color intensity. 

Event streams have very high temporal resolution and are inherently sparse, as they only happen when changes in the scene are observed. 
Due to their working principle, event cameras bring several advantages, especially in challenging cases: they excel at handling high-speed motions 
and have a substantially higher dynamic range of the supported signal measurements than conventional RGB cameras. 
Moreover, they have lower power consumption and require varied storage volumes for captured data that are often smaller than those required for synchronous RGB cameras \cite{Millerdurai_3DV2024, Gallego2022}. 

The ability to handle high-speed motions is crucial in static scenes as well,  particularly with handheld moving cameras, as it helps avoid the common problem of motion blur. It is, therefore, not surprising that event-based novel view synthesis has gained attention, although color values are not directly observed.
Notably, because of the substantial difference between the formats, RGB- and event-based approaches require fundamentally different design choices. %

The first solutions to event-based novel view synthesis introduced in the literature demonstrate promising results \cite{eventnerf, enerf} and outperform non-event-based alternatives for novel view synthesis in many challenging scenarios. 
Among them, EventNeRF \cite{eventnerf} enables novel-view synthesis in the RGB space by assuming events associated with three color channels as inputs. 
Due to its NeRF-based architecture \cite{nerf}, it can handle single objects with complete observations from roughly equal distances to the camera. 
It furthermore has limitations in training and rendering speed: 
the MLP used to represent the scene requires long training time and can only handle very limited scene extents or otherwise rendering quality will deteriorate. 
Hence, the quality of synthesized novel views will degrade for larger scenes. %

We present Event-3DGS (E-3DGS), i.e.,~a new method for novel-view synthesis from event streams using 3D Gaussians~\cite{3dgs} 
demonstrating fast reconstruction and rendering as well as handling of unbounded scenes. 
The technical contributions of this paper are as follows: 
\begin{itemize}
\item With E-3DGS, we introduce the first approach for novel view synthesis from a color event camera that combines 3D Gaussians with event-based supervision. 
\item We present frustum-based initialization, adaptive event windows, isotropic 3D Gaussian regularization and 3D camera pose refinement, and demonstrate that high-quality results can be obtained. %

\item Finally, we introduce new synthetic and real event datasets for large scenes to the community to study novel view synthesis in this new problem setting. 
\end{itemize}
Our experiments demonstrate systematically superior results compared to EventNeRF \cite{eventnerf} and other baselines. 
The source code and dataset of E-3DGS are released\footnote{\url{https://4dqv.mpi-inf.mpg.de/E3DGS/}}. 





\section{Related Work}
\label{lit_review}

\begin{highlight}
{

Our research builds upon {\em (i)} Assessing Web Accessibility, {\em (ii)} End-User Accessibility Repair, and {\em (iii)} Developer Tools for Accessibility.

\subsection{Assessing Web Accessibility}
From the earliest attempts to set standards and guidelines, web accessibility has been shaped by a complex interplay of technical challenges, legal imperatives, and educational campaigns. Over the past 25 years, stakeholders have sought to improve digital inclusion by establishing foundational standards~\cite{chisholm2001web, caldwell2008web}, enforcing legal obligations~\cite{sierkowski2002achieving, yesilada2012understanding}, and promoting a broader culture of accessibility awareness among developers~\cite{sloan2006contextual, martin2022landscape, pandey2023blending}. 
Despite these longstanding efforts, systemic accessibility issues persist. According to the 2024 WebAIM Million report~\cite{webaim2024}, 95.9\% of the top one million home pages contained detectable WCAG violations, averaging nearly 57 errors per page. 
These errors take many forms: low color contrast makes the interface difficult for individuals with color deficiency or low vision to read text; missing alternative text leaves users relying on screen readers without crucial visual context; and unlabeled form inputs or empty links and buttons hinder people who navigate with assistive technologies from completing basic tasks. 
Together, these accessibility issues not only limit user access to critical online resources such as healthcare, education, and employment but also result in significant legal risks and lost opportunities for businesses to engage diverse audiences. Addressing these pervasive issues requires systematic methods to identify, measure, and prioritize accessibility barriers, which is the first step toward achieving meaningful improvements.

Prior research has introduced methods blending automation and human evaluation to assess web accessibility. Hybrid approaches like SAMBA combine automated tools with expert reviews to measure the severity and impact of barriers, enhancing evaluation reliability~\cite{brajnik2007samba}. Quantitative metrics, such as Failure Rate and Unified Web Evaluation Methodology, support large-scale monitoring and comparative analysis, enabling cost-effective insights~\cite{vigo2007quantitative, martins2024large}. However, automated tools alone often detect less than half of WCAG violations and generate false positives, emphasizing the need for human interpretation~\cite{freire2008evaluation, vigo2013benchmarking}. Recent progress with large pretrained models like Large Language Models (LLMs)~\cite{dubey2024llama,bai2023qwen} and Large Multimodal Models (LMMs)~\cite{liu2024visual, bai2023qwenvl} offers a promising step forward, automating complex checks like non-text content evaluation and link purposes, achieving higher detection rates than traditional tools~\cite{lopez2024turning, delnevo2024interaction}. Yet, these large models face challenges, including dependence on training data, limited contextual judgment, and the inability to simulate real user experiences. These limitations underscore the necessity of combining models with human oversight for reliable, user-centered evaluations~\cite{brajnik2007samba, vigo2013benchmarking, delnevo2024interaction}. 

Our work builds on these prior efforts and recent advancements by leveraging the capabilities of large pretrained models while addressing their limitations through a developer-centric approach. CodeA11y integrates LLM-powered accessibility assessments, tailored accessibility-aware system prompts, and a dedicated accessibility checker directly into GitHub Copilot---one of the most widely used coding assistants. Unlike standalone evaluation tools, CodeA11y actively supports developers throughout the coding process by reinforcing accessibility best practices, prompting critical manual validations, and embedding accessibility considerations into existing workflows.
% This pervasive shortfall reflects the difficulty of scaling traditional approaches---such as manual audits and automated tools---that either demand immense human effort or lack the nuanced understanding needed to capture real-world user experiences. 
%
% In response, a new wave of AI-driven methods, many powered by large language models (LLMs), is emerging to bridge these accessibility detection and assessment gaps. Early explorations, such as those by Morillo et al.~\cite{morillo2020system}, introduced AI-assisted recommendations capable of automatic corrections, illustrating how computational intelligence can tackle the repetitive, common errors that plague large swaths of the web. Building on this foundation, Huang et al.~\cite{huang2024access} proposed ACCESS, a prompt-engineering framework that streamlines the identification and remediation of accessibility violations, while López-Gil et al.~\cite{lopez2024turning} demonstrated how LLMs can help apply WCAG success criteria more consistently---reducing the reliance on manual effort. Beyond these direct interventions, recent work has also begun integrating user experiences more seamlessly into the evaluation process. For example, Huq et al.~\cite{huq2024automated} translate user transcripts and corresponding issues into actionable test reports, ensuring that accessibility improvements align more closely with authentic user needs.
% However, as these AI-driven solutions evolve, researchers caution against uncritical adoption. Othman et al.~\cite{othman2023fostering} highlight that while LLMs can accelerate remediation, they may also introduce biases or encourage over-reliance on automated processes. Similarly, Delnevo et al.~\cite{delnevo2024interaction} emphasize the importance of contextual understanding and adaptability, pointing to the current limitations of LLM-based systems in serving the full spectrum of user needs. 
% In contrast to this backdrop, our work introduces and evaluates CodeA11y, an LLM-augmented extension for GitHub Copilot that not only mitigates these challenges by providing more consistent guidance and manual validation prompts, but also aligns AI-driven assistance with developers’ workflows, ultimately contributing toward more sustainable propulsion for building accessible web.

% Broader implications of inaccessibility—legal compliance, ethical concerns, and user experience
% A Historical Review of Web Accessibility Using WAVE
% "I tend to view ads almost like a pestilence": On the Accessibility Implications of Mobile Ads for Blind Users

% In the research domain, several methods have been developed to assess and enhance web accessibility. These include incorporating feedback into developer tools~\cite{adesigner, takagi2003accessibility, bigham2010accessibility} and automating the creation of accessibility tests and reports for user interfaces~\cite{swearngin2024towards, taeb2024axnav}. 

% Prior work has also studied accessibility scanners as another avenue of AI to improve web development practices~\cite{}.
% However, a persistent challenge is that developers need to be aware of these tools to utilize them effectively. With recent advancements in LLMs, developers might now build accessible websites with less effort using AI assistants. However, the impact of these assistants on the accessibility of their generated code remains unclear. This study aims to investigate these effects.

\subsection{End-user Accessibility Repair}
In addition to detecting accessibility errors and measuring web accessibility, significant research has focused on fixing these problems.
Since end-users are often the first to notice accessibility problems and have a strong incentive to address them, systems have been developed to help them report or fix these problems.

Collaborative, or social accessibility~\cite{takagi2009collaborative,sato2010social}, enabled these end-user contributions to be scaled through crowd-sourcing.
AccessMonkey~\cite{bigham2007accessmonkey} and Accessibility Commons~\cite{kawanaka2008accessibility} were two examples of repositories that store accessibility-related scripts and metadata, respectively.
Other work has developed browser extensions that leverage crowd-sourced databases to automatically correct reading order, alt-text, color contrast, and interaction-related issues~\cite{sato2009s,huang2015can}.

One drawback of collaborative accessibility approaches is that they cannot fix problems for an ``unseen'' web page on-demand, so many projects aim to automatically detect and improve interfaces without the need for an external source of fixes.
A large body of research has focused on making specific web media (e.g., images~\cite{gleason2019making,guinness2018caption, twitterally, gleason2020making, lee2021image}, design~\cite{potluri2019ai,li2019editing, peng2022diffscriber, peng2023slide}, and videos~\cite{pavel2020rescribe,peng2021say,peng2021slidecho,huh2023avscript}) accessible through a combination of machine learning (ML) and user-provided fixes.
Other work has focused on applying more general fixes across all websites.

Opportunity accessibility addressed a common accessibility problem of most websites: by default, content is often hard to see for people with visual impairments, and many users, especially older adults, do not know how to adjust or enable content zooming~\cite{bigham2014making}.
To this end, a browser script (\texttt{oppaccess.js}) was developed that automatically adjusted the browser's content zoom to maximally enlarge content without introducing adverse side-effects (\textit{e.g.,} content overlap).
While \texttt{oppaccess.js} primarily targeted zoom-related accessibility, recent work aimed to enable larger types of changes, by using LLMs to modify the source code of web pages based on user questions or directives~\cite{li2023using}.

Several efforts have been focused on improving access to desktop and mobile applications, which present additional challenges due to the unavailability of app source code (\textit{e.g.,} HTML).
Prefab is an approach that allows graphical UIs to be modified at runtime by detecting existing UI widgets, then replacing them~\cite{dixon2010prefab}.
Interaction Proxies used these runtime modification strategies to ``repair'' Android apps by replacing inaccessible widgets with improved alternatives~\cite{zhang2017interaction, zhang2018robust}.
The widget detection strategies used by these systems previously relied on a combination of heuristics and system metadata (\textit{e.g.,} the view hierarchy), which are incomplete or missing in the accessible apps.
To this end, ML has been employed to better localize~\cite{chen2020object} and repair UI elements~\cite{chen2020unblind,zhang2021screen,wu2023webui,peng2025dreamstruct}.

In general, end-user solutions to repairing application accessibility are limited due to the lack of underlying code and knowledge of the semantics of the intended content.

\subsection{Developer Tools for Accessibility}
Ultimately, the best solution for ensuring an accessible experience lies with front-end developers. Many efforts have focused on building adequate tooling and support to help developers with ensuring that their UI code complies with accessibility standards.

Numerous automated accessibility testing tools have been created to help developers identify accessibility issues in their code: i) static analysis tools, such as IBM Equal Access Accessibility Checker~\cite{ibm2024toolkit} or Microsoft Accessibility Insights~\cite{accessibilityinsights2024}, scan the UI code's compliance with predefined rules derived from accessibility guidelines; and ii) dynamic or runtime accessibility scanners, such as Chrome Devtools~\cite{chromedevtools2024} or axe-Core Accessibility Engine~\cite{deque2024axe}, perform real-time testing on user interfaces to detect interaction issues not identifiable from the code structure. While these tools greatly reduce the manual effort required for accessibility testing, they are often criticized for their limited coverage. Thus, experts often recommend manually testing with assistive technologies to uncover more complex interaction issues. Prior studies have created accessibility crawlers that either assist in developer testing~\cite{swearngin2024towards,taeb2024axnav} or simulate how assistive technologies interact with UIs~\cite{10.1145/3411764.3445455, 10.1145/3551349.3556905, 10.1145/3544548.3580679}.

Similar to end-user accessibility repair, research has focused on generating fixes to remediate accessibility issues in the UI source code. Initial attempts developed heuristic-based algorithms for fixing specific issues, for instance, by replacing text or background color attributes~\cite{10.1145/3611643.3616329}. More recent work has suggested that the code-understanding capabilities of LLMs allow them to suggest more targeted fixes.
For example, a study demonstrated that prompting ChatGPT to fix identified WCAG compliance issues in source code could automatically resolve a significant number of them~\cite{othman2023fostering}. Researchers have sought to leverage this capability by employing a multi-agent LLM architecture to automatically identify and localize issues in source code and suggest potential code fixes~\cite{mehralian2024automated}.

While the approaches mentioned above focus on assessing UI accessibility of already-authored code (\textit{i.e.,} fixing existing code), there is potential for more proactive approaches.
For example, LLMs are often used by developers to generate UI source code from natural language descriptions or tab completions~\cite{chen2021evaluating,GitHubCopilot,lozhkov2024starcoder,hui2024qwen2,roziere2023code,zheng2023codegeex}, but LLMs frequently produce inaccessible code by default~\cite{10.1145/3677846.3677854,mowar2024tab}, leading to inaccessible output when used by developers without sufficient awareness of accessibility knowledge.
The primary focus of this paper is to design a more accessibility-aware coding assistant that both produces more accessible code without manual intervention (\textit{e.g.,} specific user prompting) and gradually enables developers to implement and improve accessibility of automatically-generated code through IDE UI modifications (\textit{e.g.}, reminder notifications).

}
\end{highlight}



% Work related to this paper includes {\em (i)} Web Accessibility and {\em (ii)} Developer Practices in AI-Assisted Programming.

% \ipstart{Web Accessibility: Practice, Evaluation, and Improvements} Substantial efforts have been made to set accessibility standards~\cite{chisholm2001web, caldwell2008web}, establish legal requirements~\cite{sierkowski2002achieving, yesilada2012understanding}, and promote education and advocacy among developers~\cite{sloan2006contextual, martin2022landscape, pandey2023blending}. In the research domain, several methods have been developed to assess and enhance web accessibility. These include incorporating feedback into developer tools~\cite{adesigner, takagi2003accessibility, bigham2010accessibility} and automating the creation of accessibility tests and reports for user interfaces~\cite{swearngin2024towards, taeb2024axnav}. 
% % Prior work has also studied accessibility scanners as another avenue of AI to improve web development practices~\cite{}.
% However, a persistent challenge is that developers need to be aware of these tools to utilize them effectively. With recent advancements in LLMs, developers might now build accessible websites with less effort using AI assistants. However, the impact of these assistants on the accessibility of their generated code remains unclear. This study aims to investigate these effects.

% \ipstart{Developer Practices in AI-Assisted Programming}
% Recent usability research on AI-assisted development has examined the interaction strategies of developers while using AI coding assistants~\cite{barke2023grounded}.
% They observed developers interacted with these assistants in two modes -- 1) \textit{acceleration mode}: associated with shorter completions and 2) \textit{exploration mode}: associated with long completions.
% Liang {\em et al.} \cite{liang2024large} found that developers are driven to use AI assistants to reduce their keystrokes, finish tasks faster, and recall the syntax of programming languages. On the other hand, developers' reason for rejecting autocomplete suggestions was the need for more consideration of appropriate software requirements. This is because primary research on code generation models has mainly focused on functional correctness while often sidelining non-functional requirements such as latency, maintainability, and security~\cite{singhal2024nofuneval}. Consequently, there have been increasing concerns about the security implications of AI-generated code~\cite{sandoval2023lost}. Similarly, this study focuses on the effectiveness and uptake of code suggestions among developers in mitigating accessibility-related vulnerabilities. 


% ============================= additional rw ============================================
% - Paulina Morillo, Diego Chicaiza-Herrera, and Diego Vallejo-Huanga. 2020. System of Recommendation and Automatic Correction of Web Accessibility Using Artificial Intelligence. In Advances in Usability and User Experience, Tareq Ahram and Christianne Falcão (Eds.). Springer International Publishing, Cham, 479–489
% - Juan-Miguel López-Gil and Juanan Pereira. 2024. Turning manual web accessibility success criteria into automatic: an LLM-based approach. Universal Access in the Information Society (2024). https://doi.org/10.1007/s10209-024-01108-z
% - s
% - Calista Huang, Alyssa Ma, Suchir Vyasamudri, Eugenie Puype, Sayem Kamal, Juan Belza Garcia, Salar Cheema, and Michael Lutz. 2024. ACCESS: Prompt Engineering for Automated Web Accessibility Violation Corrections. arXiv:2401.16450 [cs.HC] https://arxiv.org/abs/2401.16450
% - Syed Fatiul Huq, Mahan Tafreshipour, Kate Kalcevich, and Sam Malek. 2025. Automated Generation of Accessibility Test Reports from Recorded User Transcripts. In Proceedings of the 47th International Conference on Software Engineering (ICSE) (Ottawa, Ontario, Canada). IEEE. https://ics.uci.edu/~seal/publications/2025_ICSE_reca11.pdf To appear in IEEE Xplore
% - Achraf Othman, Amira Dhouib, and Aljazi Nasser Al Jabor. 2023. Fostering websites accessibility: A case study on the use of the Large Language Models ChatGPT for automatic remediation. In Proceedings of the 16th International Conference on PErvasive Technologies Related to Assistive Environments (Corfu, Greece) (PETRA ’23). Association for Computing Machinery, New York, NY, USA, 707–713. https://doi.org/10.1145/3594806.3596542
% - Zsuzsanna B. Palmer and Sushil K. Oswal. 0. Constructing Websites with Generative AI Tools: The Accessibility of Their Workflows and Products for Users With Disabilities. Journal of Business and Technical Communication 0, 0 (0), 10506519241280644. https://doi.org/10.1177/10506519241280644
% ============================= additional rw ============================================
A detailed overview of the proposed architecture that converts images and control commands
into trajectories is depicted in~\autoref{fig:monoforce}.
The model consists of several learnable modules that deeply interact with each other.
The \emph{terrain encoder} carefully transforms visual features from the input image
into the heightmap space using known camera geometry.
The resultant heightmap features are further refined into interpretable physical quantities
that capture properties of the terrain such as its shape, friction, stiffness, and damping.
Next, the \emph{physics engine} combines the terrain properties with the robot model,
robot state, and control commands and delivers reaction forces at points of robot-terrain contacts.
It then solves the equations of motion dynamics by integrating these forces
and delivers the trajectory of the robot.
Since the complete computational graph of the feedforward pass is retained,
the backpropagation from an arbitrary loss, constructed on top of delivered trajectories,
or any other intermediate outputs is at hand.

\subsection{Terrain Encoder}\label{subsec:terrain_encoder}

The part of the MonoForce architecture (\autoref{fig:monoforce})
that predicts terrain properties $\mathbf{m}$ from sensor measurements $\mathbf{z}$ is called \emph{terrain encoder}.
The proposed architecture starts by converting pixels from a 2D image plane into a heightmap with visual features.
Since the camera is calibrated, there is a substantial geometrical prior that connects heightmap cells with the pixels.
We incorporate the geometry through the Lift-Splat-Shoot architecture~\cite{philion2020lift}.
This architecture uses known camera intrinsic parameters to estimate rays corresponding to particular pixels~--
pixel rays, \autoref{fig:bevfusion}.
For each pixel ray, the convolutional network then predicts depth probabilities and visual features.
Visual features are vertically projected on a virtual heightmap for all possible depths along the corresponding ray.
The depth-weighted sum of visual features over each heightmap cell is computed,
and the resulting multichannel array is further refined by deep convolutional network
to estimate the terrain properties $\mathbf{m}$.

The terrain properties include the geometrical heightmap $\mathcal{H}_g$,
the heights of the terrain supporting layer hidden under the vegetation $\mathcal{H}_t = \mathcal{H}_g - \Delta\mathcal{H}$,
terrain friction $\mathcal{M}$, stiffness $\mathcal{K}$, and dampening $\mathcal{D}$.
The intuition behind the introduction of the $\Delta\mathcal{H}$ term is
that $\mathcal{H}_t$ models a partially flexible layer of terrain (e.g. mud) that is hidden under flexible vegetation,~\autoref{fig:monoforce_heightmaps}.


\subsection{Differentiable Physics Engine}\label{subsec:dphysics}
The differentiable physics engine solves the robot motion equation and estimates
the trajectory corresponding to the delivered forces.
The trajectory is defined as a sequence of robot states $\tau = \{s_0, s_1, \ldots, s_T\}$,
where $\mathbf{s}_t = [\mathbf{x}_t, \mathbf{v}_t, R_t, \boldsymbol{\omega}_t]$
is the robot state at time $t$,
$\mathbf{x}_t \in \mathbb{R}^3$ and $\mathbf{v}_t \in \mathbb{R}^3$ define the robot's position and velocity in the world frame,
$R_t \in \mathbb{R}^{3 \times 3}$ is the robot's orientation matrix, and $\boldsymbol{\omega}_t \in \mathbb{R}^3$ is the angular velocity.
To get the next state $\mathbf{s}_{t+1}$, in general, we need to solve the following ODE:
\begin{equation}
    \label{eq:state_propagation}
    \mathbf{\dot{s}}_{t+1} = f(\mathbf{s}_t, \mathbf{u}_t, \mathbf{z}_t)
\end{equation}
where $\mathbf{u}_t$ is the control input and $\mathbf{z}_t$ is the environment state.
In practice, however, it is not feasible to obtain the full environment state $\mathbf{z}_t$.
Instead, we utilize terrain properties $\mathbf{m}_t = [\mathcal{H}_t, \mathcal{K}_t, \mathcal{D}_t, \mathcal{M}_t]$
predicted by the terrain encoder.
In this case, the motion ODE~\eqref{eq:state_propagation} can be rewritten as:
\begin{equation}
    \label{eq:state_propagation_terrain}
    \mathbf{\dot{s}}_{t+1} = \hat{f}(\mathbf{s}_t, \mathbf{u}_t, \mathbf{m}_t)
\end{equation}

Let's now derive the equation describing the state propagation function $\hat{f}$.
The time index $t$ is omitted further for brevity.
We model the robot as a rigid body with total mass $m$ represented by a~set of mass points
$\mathcal{P} = \{(\mathbf{p}_i, m_i)\; | \; \mathbf{p}_i~\in~\mathbb{R}^3, m_i~\in~\mathbb{R}^+, i=1~\dots~N\}$,
where $\mathbf{p}_i$ denotes coordinates of the $i$-th 3D point in the robot's body frame.
We employ common 6DOF dynamics of a rigid body~\cite{contact_dynamics-2018} as follows:
\begin{equation}
  \begin{split}
    \dot{\mathbf{x}} &= \mathbf{v}\\
    \dot{\mathbf{v}} &= \frac{1}{m}\sum_i\mathbf{F}_i
  \end{split}
  \quad\quad
  \begin{split}
    \dot{R} &= \Omega R\\
    \dot{\boldsymbol{\omega}} &= \mathbf{J}^{-1}\sum_i \mathbf{p}_i\times\mathbf{F}_i
  \end{split}
  \label{eq:contact_dynamics}
\end{equation}
where $\Omega = [\boldsymbol{\omega}]_{\times}$ is the skew-symmetric matrix of $\boldsymbol{\omega}$.
We denote $\mathbf{F}_i\in\mathbb{R}^3$ a total external force acting on $i$-th robot's body point.
Total mass $m = \sum_i~m_i$ and moment of inertia $\mathbf{J}\in\mathbb{R}^{3\times 3}$ of the robot's rigid body are assumed to be known
static parameters since they can be identified independently in laboratory conditions.
Note that the proposed framework allows backpropagating the gradient with respect to these quantities, too,
which makes them jointly learnable with the rest of the architecture.
The trajectory of the rigid body is the iterative solution of differential equations~\eqref{eq:contact_dynamics},
that can be obtained by any ODE solver for given external forces and initial state (pose and velocities).

When the robot is moving over a terrain, two types of external forces are acting
on the point cloud $\mathcal{P}$ representing its model:
(i) gravitational forces and (ii) robot-terrain interaction forces.
The former is defined as $m_i\mathbf{g} = [0, 0, -m_ig]^\top$ and acts on
all the points of the robot at all times,
while the latter is the result of complex physical interactions that are not easy
to model explicitly and act only on the points of the robot that are in contact
with the terrain.
There are two types of robot-terrain interaction forces:
(i) normal terrain force that prevents the penetration of the terrain by the robot points,
(ii) tangential friction force that generates forward acceleration when the tracks are moving,
and prevents side slippage of the robot.

\textbf{Robot-terrain interaction forces}

\begin{figure}[t]
    \centering
    \includegraphics[width=0.7\columnwidth]{imgs/dphysics/spring_forces}
    \caption{\textbf{Terrain force model}: Simplified 2D sketch demonstrating
    normal reaction forces acting on a robot body consisting of two points $p_i$ and $p_j$ .}
    \label{fig:spring_terrain_model}
\end{figure}

\textit{Normal reaction forces}.

One extreme option is to predict the 3D force vectors $\mathbf{F}_i$ directly
by a neural network, but we decided to enforce additional prior assumptions to reduce the risk of overfitting.
These prior assumptions are based on common intuition from the contact dynamics of flexible objects.
In particular, we assume that the magnitude of the force that the terrain exerts on the point $\mathbf{p}_i\in \mathcal{P}$
increases proportionally to the deformation of the terrain.
Consequently, the network does not directly predict the force,
but rather predicts the height of the terrain $h\in\mathcal{H}_t$
at which the force begins to act on the robot body and the stiffness of the terrain $e\in\mathcal{K}$.
We understand the quantity $e$ as an equivalent of the spring constant from Hooke's spring model, \autoref{fig:spring_terrain_model}.
Given the stiffness of the terrain and the point of the robot that penetrated the terrain
by ${\Delta}h$, the reaction force is calculated as $e\cdot{\Delta}h$.
% \begin{figure}[t]
%     \centering
%     \includegraphics[width=0.4\columnwidth]{imgs/dphysics/robot-terrain_forces}
%     \caption{\textbf{Robot-terrain interaction forces} acting on the robot's body at its contact points
%     with the terrain.
%     The point cloud was sampled from the MARV (\autoref{fig:robot_platforms}(b)) robot's 3D model.}
%     \label{fig:interaction_forces}
% \end{figure}

Since such a force, without any additional damping, would lead to an eternal bumping
of the robot on the terrain, we also introduce a robot-terrain damping coefficient $d\in\mathcal{D}$,
which similarly reduces the force proportionally to the velocity of the point
that is in contact with the terrain.
The model applies reaction forces in the normal direction $\mathbf{n}_i$ of the terrain surface,
where the $i$-th point is in contact with the terrain.
\begin{equation}\label{eq:normal_force}
    \mathbf{N}_{i} = \begin{cases}
 (e_i\Delta h_i - d_i(\dot{\mathbf{p}}_{i}^\top\mathbf{n}_i))\mathbf{n}_i  & \text{if } \mathbf{p}_{zi}\leq h_i \\
\mathbf{0} & \text{if } \mathbf{p}_{zi}> h_i
\end{cases},
\end{equation}
where terrain penetration $\Delta h_i = (h_i-\mathbf{p}_{zi})\mathbf{n}_{zi}$ is
estimated by projecting the vertical distance on the normal direction.
For a better gradient propagation, we use the smooth approximation of the Heaviside step function:
\begin{equation}
    \label{eq:smooth_normal_force}
    \mathbf{N}_i = (e_i\Delta h_i - d_i(\dot{\mathbf{p}}_{i}^\top\mathbf{n}_i))\mathbf{n}_i \cdot \sigma(h_i - \mathbf{p}_{zi}),
\end{equation}
where $\sigma(x) = \frac{1}{1+e^{-kx}}$ is the sigmoid function with a steepness hyperparameter $k$.

\begin{figure}[t]
    \centering
    \includegraphics[width=\columnwidth]{imgs/dphysics/optimization}
    \caption{\textbf{Terrain computed by backpropagating through $\nabla$Physics:}
    Shape of the terrain (border of the area where terrain forces start to act) outlined by heightmap surface,
    its color represents the friction of the terrain.
    The optimized trajectory is in green, and the ground truth trajectory is in blue.}
    \label{fig:terrain_optim}
\end{figure}

\textit{Tangential friction forces}.

Our tracked robot navigates by moving the main tracks and 4 flippers (auxiliary tracks).
The flipper motion is purely kinematic in our model.
This means that in a given time instant, their pose is uniquely determined by a $4$-dimensional vector
of their rotations, and they are treated as a rigid part of the robot.
The motion of the main tracks is transformed into forces tangential to the terrain.
The friction force delivers forward acceleration of the robot when robot tracks
(either on flippers or on main tracks) are moving.
At the same time, it prevents the robot from sliding sideways.
When a robot point $\mathbf{p}_i$, which belongs to a track, is in contact with terrain with
friction coefficient $\mu\in\mathcal{M}$, the resulting friction force at a contact point is computed as follows,~\cite{yong2012vehicle}:
\begin{equation}\label{eq:friction_force}
    \mathbf{F}_{f, i} = \mu_i |\mathbf{N}_i| ((\mathbf{u}_i - \mathbf{\dot{p}}_i)^\top\boldsymbol{\tau}_i)\boldsymbol{\tau}_i,
\end{equation}
where $\mathbf{u}_i = [u, 0, 0]^\top$, $u$ is the velocity of a track, and $\mathbf{\dot{p}}_i$ is the velocity of the point $\mathbf{p}_i$
with respect to the terrain transformed into the robot coordinate frame,
$\boldsymbol{\tau}_i$ is the unit vector tangential to the terrain surface at the point $\mathbf{p}_i$.
This model can be understood as a simplified Pacejka's tire-road model~\cite{pacejka-book-2012}
that is popular for modeling tire-road interactions.

To summarize, the state-propagation ODE~\eqref{eq:state_propagation_terrain}
(state $\mathbf{s}~=~[\mathbf{x},~\mathbf{v},~R,~\boldsymbol{\omega}]$) for a mobile robot moving over a terrain
is described by the equations of motion~\eqref{eq:contact_dynamics} where the force applied at a robot's $i$-th body point is computed as follows:
\begin{equation}\label{eq:forces}
    \begin{split}
        \mathbf{F}_i &= m_i\mathbf{g} + \mathbf{N}_i + \mathbf{F}_{f, i}
    \end{split}
\end{equation}
The robot-terrain interaction forces at contact points $\mathbf{N}_i$ and $\mathbf{F}_{f, i}$
are defined by the equations~\eqref{eq:smooth_normal_force} and~\eqref{eq:friction_force} respectively.


\textbf{Implementation of the Differentiable ODE Solver}

We implement the robot-terrain interaction ODE~\eqref{eq:contact_dynamics} in PyTorch~\cite{Paszke-NIPS-2019}.
The \textit{Neural ODE} framework~\cite{neural-ode-2021} is used to solve the system of ODEs.
For efficiency reasons, we utilize the Euler integrator for the ODE integration.
The differentiable ODE solver~\cite{neural-ode-2021} estimates the gradient through the implicit function theorem.
Additionally, we implement the ODE~\eqref{eq:contact_dynamics} solver that
estimates gradient through \textit{auto-differentiation}~\cite{Paszke-NIPS-2019},
i.e. it builds and retains the full computational graph of the feedforward integration.


\subsection{Data-driven Trajectory Prediction}\label{subsec:data_driven_baseline}
Inspired by the work~\cite{pang2019aircraft}, we design a data-driven LSTM architecture (\autoref{fig:traj_lstm}) for our outdoor mobile robot's trajectory prediction.
We call the model TrajLSTM and use it as a baseline for our $\nabla$Physics engine.
\begin{figure}
    \centering
    \includegraphics[width=\columnwidth]{imgs/architectures/lstm}
    \caption{\textbf{TrajLSTM} architecture. The model takes as input: initial state $\mathbf{x}_0$, terrain $\mathcal{H}$, control sequence $\mathbf{u}_t, t \in \{0 \dots T\}$. It predicts the trajectory as a sequence of states $\mathbf{x}_t, t \in \{0 \dots T\}.$}
    \label{fig:traj_lstm}
\end{figure}
Given an initial robot's state $\mathbf{x}_0$ and a sequence of control inputs for a time horizon $T$, $\mathbf{u}_t, t \in \{0 \dots T\}$, the TrajLSTM model provides a sequence of states at control command time moments, $\mathbf{x}_t, t \in \{0 \dots T\}$.
As in outdoor scenarios the robot commonly traverses uneven terrain, we additionally include the terrain shape input to the model in the form of heightmap $\mathcal{H}=\mathcal{H}_0$ estimated at initial time moment $t=0$.
Each timestep's control input $\mathbf{u}_i$ is concatenated with the shared spatial features $\mathbf{x}_i$, as shown in \autoref{fig:traj_lstm}.
The combined features are passed through dense layers to prepare for temporal processing.
The LSTM unit~\cite{hochreiter1997long} processes the sequence of features (one for each timestep).
As in our experiments, the time horizon for trajectory prediction is reasonably small, $T=5 [\si{\sec}]$, and the robot's trajectories lie within the heightmap area, we use the shared heightmap input for all the LSTM units of the network.
So the heightmap is processed through the convolutional layers \textbf{once} and flattened, producing a fixed-size spatial feature vector.
This design choice (of not processing the heightmaps at different time moments) is also motivated by computational efficiency reason.
At each moment $t$, this heightmap vector is concatenated with the fused spatial-control features and processed by an LSTM unit.
The LSTM unit output for each timestep $t$ is passed through a fully connected (dense) layer to produce the next state $\mathbf{x}_{t+1}$.
The sequence of states form the predicted trajectory, $\{\mathbf{x}_0, \dots \mathbf{x}_T\}$.


\subsection{End-to-end Learning}\label{subsec:end2end_learning}
Self-supervised learning of the proposed architecture minimizes three different losses:

\textbf{Trajectory loss} that minimizes
the difference between SLAM-reconstructed trajectory $\tau^\star$ and predicted trajectory $\tau$:
\begin{equation}~\label{eq:traj_loss}
   \mathcal{L}_\tau = \|\tau-\tau^\star\|^2
\end{equation}

\textbf{Geometrical loss} that minimizes the difference between
ground truth lidar-reconstructed heightmap $\mathcal{H}_g^\star$
and predicted geometrical heightmap $\mathcal{H}_g$:
 \begin{equation}~\label{eq:geom_loss}
     \mathcal{L}_g = \|\mathbf{W}_g\circ(\mathcal{H}_g-\mathcal{H}_g^\star)\|^2
 \end{equation}
$\mathbf{W}_g$ denotes an array selecting the heightmap channel corresponding to the terrain shape.

\textbf{Terrain loss} that minimizes the difference between ground truth $\mathcal{H}_t^\star$
and predicted $\mathcal{H}_t$ supporting heightmaps containing rigid objects detected
with Microsoft's image segmentation model SEEM~\cite{zou2023segment},
that is derived from Segment Anything foundation model~\cite{li2023semantic}:
 \begin{equation}~\label{eq:terrain_loss}
     \mathcal{L}_t = \|\mathbf{W}_t\circ(\mathcal{H}_t-\mathcal{H}_t^\star)\|^2
 \end{equation}
$\mathbf{W}_t$ denotes the array selecting heightmap cells that are covered by rigid materials
(e.g. stones, walls, trunks), and $\circ$ is element-wise multiplication.

Since the architecture \autoref{fig:model_overview} is end-to-end differentiable,
we can directly learn to predict all intermediate outputs just using trajectory loss~\eqref{eq:traj_loss}.
An example of terrain learning with the trajectory loss is visualized in \autoref{fig:terrain_optim}.
To make the training more efficient and the learned model explainable, we employ the
geometrical loss~\eqref{eq:geom_loss} and terrain loss~\eqref{eq:terrain_loss} as regularization terms.
stat

\begin{figure*}
    \centering
    \includegraphics[width=\textwidth]{imgs/predictions/monoforce/qualitative_results_experiments}
    \caption{\textbf{MonoForce prediction examples}.
    \emph{Left}: The robot is moving through a narrow passage between a wall and tree logs.
    \emph{Right}: The robot is moving on a gravel road with rocks on the sides.
    It starts its motion from the position marked with a coordinate frame and the trajectory is predicted for $10~[\si{\sec}]$ using real control commands.
    The camera images are taken from the robot's initial position (\emph{top row}).
    The visualization includes predicted supporting terrain $\mathcal{H}_t$ (\emph{second row}).
    It is additionally shown in 3D and colored with predicted friction values (\emph{third row}).
    }
    \label{fig:monoforce_predictions}
\end{figure*}

The \autoref{fig:monoforce_predictions} show the prediction examples of the MonoForce model in diverse outdoor environments.
From the example on the left,
we can see that the model correctly predicts the robot's trajectory and the terrain shape suppressing traversable vegetation,
while the rigid obstacles (wall and tree logs) are correctly detected.
The example on the right demonstrates the model's ability to predict the robot's trajectory ($10~[\si{\sec}]$-long horizon)
with reasonable accuracy and to detect the rigid obstacles (stones) on the terrain.
It could also be noticed that the surfaces that provide the robot good traction (paved and gravel roads) are marked with a higher friction value,
while for the objects that might not give good contact with the robot's tracks (walls and tree logs) the friction value is lower.

We argue that the friction estimates are approximate and an interesting research direction could be
comparing them with real-world measurements or with the values provided by a high-fidelity physics engine (e.g. AGX Dynamics~\cite{Berglund2019agxTerrain}).
However, one of the benefits of our differentiable approach is that the model does not require ground-truth friction values for training.
The predicted heightmap's size is $12.8\times12.8\si{\meter}^2$ and the grid resolution is $0.1\si{\meter}$.
It has an upper bound of $1~[\si{\meter}]$ and a lower bound of $-1~[\si{\meter}]$.
This constraint was introduced based on the robot's size and taking into account hanging objects (tree branches)
that should not be considered as obstacles (\autoref{fig:nav_monoforce}).
Additionally, the terrain is predicted in the gravity-aligned frame.
That is made possible thanks to the inclusion of camera intrinsics and extrinsics as input to the model,
\autoref{fig:monoforce}.
It also allows correctly modeling the robot-terrain interaction forces (and thus modeling the robot's trajectory accurately)
for the scenarios with non-flat terrain, for example, going uphill or downhill.
This will not be possible if only camera images are used as input.
\section{Experiments}
\label{sec:exp}

\subsection{Implementation details}
\textbf{Representation details.} We initilize the low-resolution mesh $\left\{\{v_{0, i}^{c\downarrow}\}_{i=1}^{V^\downarrow}, \{f_{j}^\downarrow\}_{j=1}^{F^\downarrow}\right\}$ in $\text{GoM}_0^c$ with SMPL or SMPL-X, %
depending on the human pose representation used in the dataset. The high-resolution mesh is obtained by subdividing the low-resolution mesh. %

\textbf{Architecture details.} We provide the detailed architecture in Appendix~\ref{sec: appendix_arch}. %

\textbf{Training details.} We set $\lambda_\text{per}=1.0$, $\lambda_M=5.0$ and $\lambda_\text{lap}=100$ in \cref{eq: loss} on THuman2.0 and $\lambda_\text{per}=1.0$, $\lambda_M=0$ and $\lambda_\text{lap}=100$ in \cref{eq: loss} on AIST++. We use the SSIM loss in THuman2.0 and the LPIPS loss in AIST++ following the baselines. We use Adam as the optimizer. On THuman2.0, the learning rates of the image encoder and the rest of the model are $1\mathrm{e}{-4}$ and $5\mathrm{e}{-5}$ respectively. On AIST++, we set the learning rate of all parameters to $5\mathrm{e}{-5}$. We optimize the model for 200K iterations on THuman2.0 and 100K iterations on AIST++.

\subsection{Experimental setup}
We evaluate our approach in two settings: 1) \textbf{Multiview source images.} Our approach can take multiview images as input to produce a canonical representation;  2) \textbf{Multi-frame source images.} Since our approach directly learns a 3D representation in the canonical space instead of a posed space, our method can also operate on images showing various human poses, e.g., frames sampled from a monocular video. 
Our approach can synthesize both novel views and novel poses.

\textbf{Datasets.} We validate our approach on THuman2.0~\citep{tao2021function4d}, XHuman~\citep{shen2023xavatar} and AIST++~\citep{li2021learn} quantitatively. We use THuman2.0 to evaluate our approach in the setting of multiview source images.  XHuman  is used to validate the cross-domain generalization of our approach. In other words, we train our model on THuman2.0 and test on XHuman without fine-tuning. The AIST++ dataset is used to evaluate the  multi-frame source image setting. Please see Appendix~\ref{sec: appendix_dataset} for detailed dataset setup.



\begin{table}[t!]
\vspace{-4mm}
\caption{\textbf{Comparison on THuman2.0.} The proposed method improves state-of-the-art in PSNR, LPIPS$^*$ and FID. We highlight the best result in bold font. Methods marked in gray are per-scene optimized methods.}
\label{tab: thuman2.0}
\centering
\footnotesize
\vspace{-3mm}
\begin{tabular}{c|l|rrr}
\toprule
\thead{Number of \\ source views}                  & \thead{Method}                        & \thead{PSNR$\uparrow$} & \thead{LPIPS*$\downarrow$} & \thead{FID$\downarrow$} \\
\midrule
\multirow{6}{*}{3} & {\color[HTML]{9B9B9B} GoMAvatar~\citep{wen2024gomavatar}} & {\color[HTML]{9B9B9B} 23.05} & {\color[HTML]{9B9B9B} 133.98}& {\color[HTML]{9B9B9B} 87.51}\\
                & {\color[HTML]{9B9B9B} 3DGS-Avatar~\citep{qian20243dgs}} & {\color[HTML]{9B9B9B} 21.25}& {\color[HTML]{9B9B9B} 160.48}& {\color[HTML]{9B9B9B} 157.21}\\
                & {\color[HTML]{9B9B9B} iHuman~\citep{paudel2024ihuman}} & {\color[HTML]{9B9B9B} 22.77}& {\color[HTML]{9B9B9B} 131.67}& {\color[HTML]{9B9B9B} 101.70}\\
                   & NHP~\citep{kwon2021neural}                           &    23.32  &  184.69      &  136.56              \\
                   & NIA~\citep{kwon2023neural}                           &    23.20   &     181.82    &  127.30           \\
                   & GHG~\citep{kwon2024ghg}                           &    21.90  &    133.41    &  61.67                  \\
                   & LIFe-GoM (Ours) &  \textbf{24.65}    &  \textbf{110.82}      &  \textbf{51.27}              \\
\midrule
\multirow{2}{*}{5} & GPS-Gaussian~\citep{zheng2024gpsgaussian}                  &   20.39   &    152.34    &  65.90       \\
                   & LIFe-GoM (Ours) &  \textbf{25.57}   &    \textbf{105.39}    &  \textbf{38.57}    \\
\bottomrule
\end{tabular}\vspace{-4mm}
\end{table}

\begin{table}[t!]
\caption{\textbf{Comparison on XHuman.} We evaluate on XHuman to prove the ability of cross-domain generalization. The proposed method improves state-of-the-art in PSNR, LPIPS$^*$ and FID. We highlight the best result in bold font.}\vspace{-3mm}
\label{tab: xhuman}
\centering
\footnotesize
\begin{tabular}{l|rrr}
\toprule
\thead{Method}                        & \thead{PSNR$\uparrow$} & \thead{LPIPS*$\downarrow$} & \thead{FID$\downarrow$} \\
\midrule
GHG~\citep{kwon2024ghg}                           &    23.52 & 112.91 & 50.51     \\
LIFe-GoM (Ours) &  \textbf{25.32} & \textbf{99.32} & \textbf{42.90}    \\ 
\bottomrule
\end{tabular}\vspace{-6mm}
\end{table}

\textbf{Baselines.} We compare with GoMAvatar~\citep{wen2024gomavatar}, 3DGS-Avatar~\citep{qian20243dgs}, iHuman~\citep{paudel2024ihuman}, NHP~\citep{kwon2021neural}, NIA~\citep{kwon2023neural}, GHG~\citep{kwon2024ghg} and GPS-Gaussian~\citep{zheng2024gpsgaussian} on THuman2.0. On AIST++, we compare with HumanNeRF~\citep{Weng2022HumanNeRFFR}, GoMAvatar~\citep{wen2024gomavatar}, 3DGS-Avatar~\citep{qian20243dgs}, iHuman~\citep{paudel2024ihuman} and ActorsNeRF~\citep{mu2023actorsnerf}. Please refer to Appendix~\ref{sec: appendix_baseline} for details.


\textbf{Evaluation metrics.} We report  PSNR, $\text{LPIPS}^*(=\text{LPIPS} \times 1000)$ and FID on THuman2.0 following GHG~\citep{kwon2024ghg}. 
We report  PSNR, SSIM and LPIPS$^*$ on AIST++ following ActorsNeRF~\citep{mu2023actorsnerf}.

\subsection{Quantitative results}
\textbf{THuman2.0.} We summarize our results in \cref{tab: thuman2.0} for both the three-view and the five-view setting. 

In the three-view setting, our method significantly outperforms per-scene optimized methods including GoMAvatar, 3DGS-Avatar and iHuman, and generalizable approaches including NHP, NIA, and GHG in PSNR, LPIPS$^\ast$, and FID. 
Our approach achieves 24.65/110.82/51.27 in PSNR/LPIPS$^\ast$/FID, compared to GHG's 21.90/133.41/61.67. 
Importantly, we use 330K Gaussians for splatting, $7.5\times$ fewer than GHG's 2.8M, resulting in faster rendering (10.52ms vs.\ GHG's 20.30ms) at $1024\times1024$ resolution on a NVIDIA A100 GPU. 
Our method takes 907.92ms to reconstruct the coupled-multi-resolution Gaussians-on-Mesh in canonical space, significantly faster than scene-specific methods but slower than GHG. That said, reconstruction only needs to be done once per input subject, as the reconstructed avatar will be cached and reused for articulation and rendering, which runs at 95 FPS. 

We compare our approach to GPS-Gaussian using five images. As GPS-Gaussian relies on depth prediction between adjacent views, five images are the minimum it needs. Despite that, it still fails in non-overlapping regions. Our approach significantly improves upon GPS-Gaussian in this setting.





\textbf{XHuman.} We summarize the cross-dataset generalization results in \cref{tab: xhuman}. We directly apply GHG and our approach trained on THuman2.0 in the setting of 3 source views to the XHuman dataset without any finetuning. Our approach achieves PSNR/LPIPS*/FID of 25.32/99.32/42.90, significantly outperforming GHG's 23.52/112.91/50.51.

{\bf AIST++.} 
\cref{tab: aist} summarizes quantitative results on AIST++. Our method achieves 25.25/0.9812/21.61 in PSNR/SSIM/LPIPS*, matching ActorsNeRF's 25.23/0.9809/22.11 and surpassing per-scene optimized methods. Importantly, our method needs only 589 ms for 3D reconstruction, whereas iHuman, the fastest scene-specific method, requires 6.61s and other baselines take minutes to hours.



\begin{table}[t]
\vspace{-4mm}
\centering
\caption{\textbf{Comparison on AIST++.} We achieve comparable  quality as ActorsNeRF while requiring much less time in reconstruction or optimization. We highlight the best result in bold font. Methods marked in gray are per-scene optimized methods.}\vspace{-3mm}
\label{tab: aist}
\footnotesize
\begin{tabular}{l|rrr|rr}
\toprule
\thead{Method}                        & \thead{PSNR$\uparrow$} & \thead{SSIM$\uparrow$} & \thead{LPIPS*$\downarrow$} & \thead{\shortstack{Reconstruction or \\ optimization time$\downarrow$}} \\
\midrule
{\color[HTML]{9B9B9B} HumanNeRF~\citep{Weng2022HumanNeRFFR}}                           &   {\color[HTML]{9B9B9B}24.21}   &  {\color[HTML]{9B9B9B}0.9760}      &   {\color[HTML]{9B9B9B}29.66}  &           {\color[HTML]{9B9B9B}$\sim$2h}                           \\
{\color[HTML]{9B9B9B}GoMAvatar~\citep{wen2024gomavatar}} & {\color[HTML]{9B9B9B}24.34}& {\color[HTML]{9B9B9B}0.9780}& {\color[HTML]{9B9B9B}25.34}& {\color[HTML]{9B9B9B}$\sim$10h}\\
{\color[HTML]{9B9B9B}3DGS-Avatar~\citep{qian20243dgs}}	&{\color[HTML]{9B9B9B}25.14}	&{\color[HTML]{9B9B9B}0.9784}	&{\color[HTML]{9B9B9B}27.17}	&{\color[HTML]{9B9B9B}$\sim$2min} \\
{\color[HTML]{9B9B9B}iHuman~\citep{paudel2024ihuman}}	&{\color[HTML]{9B9B9B}25.17}	&{\color[HTML]{9B9B9B}0.9805}	&{\color[HTML]{9B9B9B}22.90}	&{\color[HTML]{9B9B9B}6.61s} \\
ActorsNeRF~\citep{mu2023actorsnerf}                           &   25.23   &   0.9809   &   22.45  &                  $\sim$4h                            \\
LIFe-GoM (Ours) &  \textbf{25.25}                        & \textbf{0.9812}                        & \textbf{21.61}   &      \textbf{589.27ms}             \\
\bottomrule
\end{tabular}\vspace{-2mm}
\end{table}

\begin{figure}[t]
\vspace{-2mm}
    \centering
    \begin{minipage}{.45\textwidth}
        \includegraphics[width=\linewidth]{results/crossdomain.pdf}
    \vspace{-8mm}
        \caption{\textbf{Cross-domain generalization} on DNA-Rendering dataset w/o finetuning.
        }
        \label{fig: cross_domain}
    \end{minipage}\hspace{.5cm}
    \begin{minipage}{.45\textwidth}
        \includegraphics[width=\linewidth]{results/novelpose.pdf}
        \vspace{-8mm}
    \caption{\textbf{Novel pose synthesis.} Poses are from BEDLAM dataset.}
    \label{fig: quant_novel_pose}
    \end{minipage}
\vspace{-6.5mm}
\end{figure}
\subsection{Qualitative results}


Please refer to Appendix~\ref{sec: appendix_result} for more qualitative results, including a comparison to baselines.

\textbf{Cross-domain generalization.} We demo our approach on cross-domain generalization in \cref{fig: cross_domain}, using the DNA-Rendering data~\citep{cheng2023dna}. Without fine-tuning, our approach can generalize to challenging subjects, e.g., loose clothes.

\textbf{Novel pose synthesis.} Instead of directly reconstructing human avatars in the pose of the source images, our approach outputs the canonical representation in T-pose via the  \texttt{Reconstructor}. %
Benefitting from this choice, we can synthesize novel poses without postprocessing such as binding the skeletons. In \cref{fig: quant_novel_pose}, we retarget the avatar to challenging new pose sequences from the BEDLAM dataset~\citep{Black_CVPR_2023}. The avatar is reconstructed using the model which was used to report results in the 3 source view setting of \cref{tab: thuman2.0}.\vspace{-0.2cm}

\subsection{Ablation studies}

\textbf{Analysis of iterative step choice.} 
We study how the number of iterations ($T$) influences the reconstruction time and rendering quality. Results are summarized in \cref{tab: iterative_update} and \cref{fig: iterative_update}(a). Note that $T=1$ means a single feed-forward pass, i.e., iterative updates are disabled. Using more iterations improves the rendering quality at the expense of more reconstruction time ($\sim$290ms per iteration). The PSNR improves by $+0.78$ and $+0.91$ when $T=2$ and $T=3$ respectively compared to $T=1$. Starting with $T=4$, the benefit of more iterations diminishes. We choose $T=3$ in our final model to balance  rendering quality and reconstruction time.

\begin{table}[t!]
\vspace{-4mm}
\centering
\caption{\textbf{Iterative step choice.} More iterations lead to better rendering at the expense of longer reconstruction. We use 3 iterations for the best quality-speed tradeoff, as highlighted in gray.}
\vspace{-3mm}
\label{tab: iterative_update}
\footnotesize
\begin{tabular}{c|rrr|r}
\toprule
\thead{\# iterations}                        & \thead{PSNR$\uparrow$} & \thead{LPIPS*$\downarrow$} & \thead{FID$\downarrow$} & \thead{\shortstack{Reconstruction \\ time (ms)$\downarrow$}} \\
\midrule
1 & 23.74 & 124.58 & 64.59 & 328.79 \\
2 & 24.52 & 112.47 & 52.16 & 618.67 \\
\cellcolor[HTML]{D3D3D3}3 & \cellcolor[HTML]{D3D3D3}24.65 & \cellcolor[HTML]{D3D3D3}110.82 & \cellcolor[HTML]{D3D3D3}51.27 & \cellcolor[HTML]{D3D3D3}907.92 \\
4 & 24.69 & 110.46 & 51.25 & 1198.14\\
5 & 24.70 & 110.38 & 51.02 & 1563.92 \\
\bottomrule
\end{tabular}\vspace{-5mm}
\end{table}

\begin{figure}
    \centering
    \includegraphics[width=\linewidth]{results/ablation_v2.pdf}\vspace{-3mm}
    \caption{\textbf{Ablation studies.} We study the effect of iterative feedback (left). The geometry improves as the number of iterations increases. We show the importance of linking Gaussians to the high-resolution mesh (right). The high-resolution mesh is subdivided from the low-resolution counterpart. A higher resolution yields better texture details.}
    \label{fig: iterative_update}\vspace{-3mm}
\end{figure}

\textbf{Coupled-multi-resolution Gaussians-on-Mesh.} As mentioned in \cref{sec: canonical_rep} and \cref{sec: reconstruction}, we update the vertices of the low-resolution mesh, while the Gaussians are associated with the high-resolution mesh. Both are updated jointly. This choice is necessary for two reasons: 1) simply updating the vertices of the high-resolution mesh increases the reconstruction time from 907.92ms to 12.45s, making it too slow for both training and inference; 2) learning Gaussians in the high-resolution mesh guarantees good rendering quality. Note that the high-resolution mesh is obtained by subdividing the low-resolution mesh. In \cref{tab: multires}, we show that the rendering improves to 118.64/58.45 and 110.82/51.27 in LPIPS$^*$/FID when subdividing once and twice respectively from 140.60/93.44 without subdivision. The improvement can also be observed in \cref{fig: iterative_update}(b). Note that we do not observe consistent improvement in PSNR. This is because PSNR sometimes prefers blurry results. %
The resolution of the high-resolution mesh affects both the reconstruction speed and the rendering speed since we render the source images during the reconstruction stage. As the reconstruction time is still less than 1s, we choose to subdivide twice for better rendering quality.

\begin{table}[t!]
\centering
\caption{\textbf{Coupled-multi-resolution Gaussians-on-Mesh.} Increasing the number of subdivisions improves rendering quality at the cost of longer reconstruction and rendering times. We subdivide twice in our final model to ensure quality while maintaining real-time, as highlighted in gray.
}
\label{tab: multires}
\vspace{-2mm}
\footnotesize
\begin{tabular}{c|rrr|rr}
\toprule
\thead{\# subdivision}                        & \thead{PSNR$\uparrow$} & \thead{LPIPS*$\downarrow$} & \thead{FID$\downarrow$} & \thead{\shortstack{Reconstruction \\ time (ms)$\downarrow$}}  & \thead{\shortstack{Rendering \\ time (ms)$\downarrow$}} \\
\midrule
0 & 24.76 & 140.60 & 93.44 & 538.02 & 3.20 \\
1 & 24.88 & 118.64 & 58.45 & 607.49 & 3.93 \\
\cellcolor[HTML]{D3D3D3}2 & \cellcolor[HTML]{D3D3D3}24.65 & \cellcolor[HTML]{D3D3D3}110.82 & \cellcolor[HTML]{D3D3D3}51.27 & \cellcolor[HTML]{D3D3D3}907.92 & \cellcolor[HTML]{D3D3D3}10.52 \\
\bottomrule
\end{tabular}
\vspace{-5mm}
\end{table}

\section{Conclusion}
In this work we show that training high quality \slms with a very modest compute budget, is feasible. We give these main guidelines: (i) \textbf{Do not skimp on the model} - not all model families are born equal and the TWIST initialisation exaggerates this, thus it is worth selecting a stronger / bigger text-LM even if it means less tokens. we found Qwen$2.5$ to be a good choice; (ii) \textbf{Utilise synthetic training data} - pre-training on data generated with TTS helps a lot; (iii) \textbf{Go beyond next token prediction} - we found that DPO boosts performance notably even when using synthetic data, and as little as $30$ minutes training massively improves results; (iv) \textbf{Optimise hyper-parameters} - as researchers we often dis-regard this stage, yet we found that tuning learning rate schedulers and optimising code efficiency can improve results notably. We hope that these insights, and open source resources will be of use to the research community in furthering research into remaining open questions in \slms.


% Entries for the entire Anthology, followed by custom entries
% \bibliography{anthology,custom}
\bibliography{custom}

\newpage
% % This must be in the first 5 lines to tell arXiv to use pdfLaTeX, which is strongly recommended.
\pdfoutput=1
% In particular, the hyperref package requires pdfLaTeX in order to break URLs across lines.

\documentclass[11pt]{article}

% Change "review" to "final" to generate the final (sometimes called camera-ready) version.
% Change to "preprint" to generate a non-anonymous version with page numbers.
\usepackage{acl}

% Standard package includes
\usepackage{times}
\usepackage{latexsym}

% Draw tables
\usepackage{booktabs}
\usepackage{multirow}
\usepackage{xcolor}
\usepackage{colortbl}
\usepackage{array} 
\usepackage{amsmath}

\newcolumntype{C}{>{\centering\arraybackslash}p{0.07\textwidth}}
% For proper rendering and hyphenation of words containing Latin characters (including in bib files)
\usepackage[T1]{fontenc}
% For Vietnamese characters
% \usepackage[T5]{fontenc}
% See https://www.latex-project.org/help/documentation/encguide.pdf for other character sets
% This assumes your files are encoded as UTF8
\usepackage[utf8]{inputenc}

% This is not strictly necessary, and may be commented out,
% but it will improve the layout of the manuscript,
% and will typically save some space.
\usepackage{microtype}
\DeclareMathOperator*{\argmax}{arg\,max}
% This is also not strictly necessary, and may be commented out.
% However, it will improve the aesthetics of text in
% the typewriter font.
\usepackage{inconsolata}

%Including images in your LaTeX document requires adding
%additional package(s)
\usepackage{graphicx}
% If the title and author information does not fit in the area allocated, uncomment the following
%
%\setlength\titlebox{<dim>}
%
% and set <dim> to something 5cm or larger.

\title{Wi-Chat: Large Language Model Powered Wi-Fi Sensing}

% Author information can be set in various styles:
% For several authors from the same institution:
% \author{Author 1 \and ... \and Author n \\
%         Address line \\ ... \\ Address line}
% if the names do not fit well on one line use
%         Author 1 \\ {\bf Author 2} \\ ... \\ {\bf Author n} \\
% For authors from different institutions:
% \author{Author 1 \\ Address line \\  ... \\ Address line
%         \And  ... \And
%         Author n \\ Address line \\ ... \\ Address line}
% To start a separate ``row'' of authors use \AND, as in
% \author{Author 1 \\ Address line \\  ... \\ Address line
%         \AND
%         Author 2 \\ Address line \\ ... \\ Address line \And
%         Author 3 \\ Address line \\ ... \\ Address line}

% \author{First Author \\
%   Affiliation / Address line 1 \\
%   Affiliation / Address line 2 \\
%   Affiliation / Address line 3 \\
%   \texttt{email@domain} \\\And
%   Second Author \\
%   Affiliation / Address line 1 \\
%   Affiliation / Address line 2 \\
%   Affiliation / Address line 3 \\
%   \texttt{email@domain} \\}
% \author{Haohan Yuan \qquad Haopeng Zhang\thanks{corresponding author} \\ 
%   ALOHA Lab, University of Hawaii at Manoa \\
%   % Affiliation / Address line 2 \\
%   % Affiliation / Address line 3 \\
%   \texttt{\{haohany,haopengz\}@hawaii.edu}}
  
\author{
{Haopeng Zhang$\dag$\thanks{These authors contributed equally to this work.}, Yili Ren$\ddagger$\footnotemark[1], Haohan Yuan$\dag$, Jingzhe Zhang$\ddagger$, Yitong Shen$\ddagger$} \\
ALOHA Lab, University of Hawaii at Manoa$\dag$, University of South Florida$\ddagger$ \\
\{haopengz, haohany\}@hawaii.edu\\
\{yiliren, jingzhe, shen202\}@usf.edu\\}



  
%\author{
%  \textbf{First Author\textsuperscript{1}},
%  \textbf{Second Author\textsuperscript{1,2}},
%  \textbf{Third T. Author\textsuperscript{1}},
%  \textbf{Fourth Author\textsuperscript{1}},
%\\
%  \textbf{Fifth Author\textsuperscript{1,2}},
%  \textbf{Sixth Author\textsuperscript{1}},
%  \textbf{Seventh Author\textsuperscript{1}},
%  \textbf{Eighth Author \textsuperscript{1,2,3,4}},
%\\
%  \textbf{Ninth Author\textsuperscript{1}},
%  \textbf{Tenth Author\textsuperscript{1}},
%  \textbf{Eleventh E. Author\textsuperscript{1,2,3,4,5}},
%  \textbf{Twelfth Author\textsuperscript{1}},
%\\
%  \textbf{Thirteenth Author\textsuperscript{3}},
%  \textbf{Fourteenth F. Author\textsuperscript{2,4}},
%  \textbf{Fifteenth Author\textsuperscript{1}},
%  \textbf{Sixteenth Author\textsuperscript{1}},
%\\
%  \textbf{Seventeenth S. Author\textsuperscript{4,5}},
%  \textbf{Eighteenth Author\textsuperscript{3,4}},
%  \textbf{Nineteenth N. Author\textsuperscript{2,5}},
%  \textbf{Twentieth Author\textsuperscript{1}}
%\\
%\\
%  \textsuperscript{1}Affiliation 1,
%  \textsuperscript{2}Affiliation 2,
%  \textsuperscript{3}Affiliation 3,
%  \textsuperscript{4}Affiliation 4,
%  \textsuperscript{5}Affiliation 5
%\\
%  \small{
%    \textbf{Correspondence:} \href{mailto:email@domain}{email@domain}
%  }
%}

\begin{document}
\maketitle
\begin{abstract}
Recent advancements in Large Language Models (LLMs) have demonstrated remarkable capabilities across diverse tasks. However, their potential to integrate physical model knowledge for real-world signal interpretation remains largely unexplored. In this work, we introduce Wi-Chat, the first LLM-powered Wi-Fi-based human activity recognition system. We demonstrate that LLMs can process raw Wi-Fi signals and infer human activities by incorporating Wi-Fi sensing principles into prompts. Our approach leverages physical model insights to guide LLMs in interpreting Channel State Information (CSI) data without traditional signal processing techniques. Through experiments on real-world Wi-Fi datasets, we show that LLMs exhibit strong reasoning capabilities, achieving zero-shot activity recognition. These findings highlight a new paradigm for Wi-Fi sensing, expanding LLM applications beyond conventional language tasks and enhancing the accessibility of wireless sensing for real-world deployments.
\end{abstract}

\section{Introduction}

In today’s rapidly evolving digital landscape, the transformative power of web technologies has redefined not only how services are delivered but also how complex tasks are approached. Web-based systems have become increasingly prevalent in risk control across various domains. This widespread adoption is due their accessibility, scalability, and ability to remotely connect various types of users. For example, these systems are used for process safety management in industry~\cite{kannan2016web}, safety risk early warning in urban construction~\cite{ding2013development}, and safe monitoring of infrastructural systems~\cite{repetto2018web}. Within these web-based risk management systems, the source search problem presents a huge challenge. Source search refers to the task of identifying the origin of a risky event, such as a gas leak and the emission point of toxic substances. This source search capability is crucial for effective risk management and decision-making.

Traditional approaches to implementing source search capabilities into the web systems often rely on solely algorithmic solutions~\cite{ristic2016study}. These methods, while relatively straightforward to implement, often struggle to achieve acceptable performances due to algorithmic local optima and complex unknown environments~\cite{zhao2020searching}. More recently, web crowdsourcing has emerged as a promising alternative for tackling the source search problem by incorporating human efforts in these web systems on-the-fly~\cite{zhao2024user}. This approach outsources the task of addressing issues encountered during the source search process to human workers, leveraging their capabilities to enhance system performance.

These solutions often employ a human-AI collaborative way~\cite{zhao2023leveraging} where algorithms handle exploration-exploitation and report the encountered problems while human workers resolve complex decision-making bottlenecks to help the algorithms getting rid of local deadlocks~\cite{zhao2022crowd}. Although effective, this paradigm suffers from two inherent limitations: increased operational costs from continuous human intervention, and slow response times of human workers due to sequential decision-making. These challenges motivate our investigation into developing autonomous systems that preserve human-like reasoning capabilities while reducing dependency on massive crowdsourced labor.

Furthermore, recent advancements in large language models (LLMs)~\cite{chang2024survey} and multi-modal LLMs (MLLMs)~\cite{huang2023chatgpt} have unveiled promising avenues for addressing these challenges. One clear opportunity involves the seamless integration of visual understanding and linguistic reasoning for robust decision-making in search tasks. However, whether large models-assisted source search is really effective and efficient for improving the current source search algorithms~\cite{ji2022source} remains unknown. \textit{To address the research gap, we are particularly interested in answering the following two research questions in this work:}

\textbf{\textit{RQ1: }}How can source search capabilities be integrated into web-based systems to support decision-making in time-sensitive risk management scenarios? 
% \sq{I mention ``time-sensitive'' here because I feel like we shall say something about the response time -- LLM has to be faster than humans}

\textbf{\textit{RQ2: }}How can MLLMs and LLMs enhance the effectiveness and efficiency of existing source search algorithms? 

% \textit{\textbf{RQ2:}} To what extent does the performance of large models-assisted search align with or approach the effectiveness of human-AI collaborative search? 

To answer the research questions, we propose a novel framework called Auto-\
S$^2$earch (\textbf{Auto}nomous \textbf{S}ource \textbf{Search}) and implement a prototype system that leverages advanced web technologies to simulate real-world conditions for zero-shot source search. Unlike traditional methods that rely on pre-defined heuristics or extensive human intervention, AutoS$^2$earch employs a carefully designed prompt that encapsulates human rationales, thereby guiding the MLLM to generate coherent and accurate scene descriptions from visual inputs about four directional choices. Based on these language-based descriptions, the LLM is enabled to determine the optimal directional choice through chain-of-thought (CoT) reasoning. Comprehensive empirical validation demonstrates that AutoS$^2$-\ 
earch achieves a success rate of 95–98\%, closely approaching the performance of human-AI collaborative search across 20 benchmark scenarios~\cite{zhao2023leveraging}. 

Our work indicates that the role of humans in future web crowdsourcing tasks may evolve from executors to validators or supervisors. Furthermore, incorporating explanations of LLM decisions into web-based system interfaces has the potential to help humans enhance task performance in risk control.






\section{Related Work}
\label{sec:relatedworks}

% \begin{table*}[t]
% \centering 
% \renewcommand\arraystretch{0.98}
% \fontsize{8}{10}\selectfont \setlength{\tabcolsep}{0.4em}
% \begin{tabular}{@{}lc|cc|cc|cc@{}}
% \toprule
% \textbf{Methods}           & \begin{tabular}[c]{@{}c@{}}\textbf{Training}\\ \textbf{Paradigm}\end{tabular} & \begin{tabular}[c]{@{}c@{}}\textbf{$\#$ PT Data}\\ \textbf{(Tokens)}\end{tabular} & \begin{tabular}[c]{@{}c@{}}\textbf{$\#$ IFT Data}\\ \textbf{(Samples)}\end{tabular} & \textbf{Code}  & \begin{tabular}[c]{@{}c@{}}\textbf{Natural}\\ \textbf{Language}\end{tabular} & \begin{tabular}[c]{@{}c@{}}\textbf{Action}\\ \textbf{Trajectories}\end{tabular} & \begin{tabular}[c]{@{}c@{}}\textbf{API}\\ \textbf{Documentation}\end{tabular}\\ \midrule 
% NexusRaven~\citep{srinivasan2023nexusraven} & IFT & - & - & \textcolor{green}{\CheckmarkBold} & \textcolor{green}{\CheckmarkBold} &\textcolor{red}{\XSolidBrush}&\textcolor{red}{\XSolidBrush}\\
% AgentInstruct~\citep{zeng2023agenttuning} & IFT & - & 2k & \textcolor{green}{\CheckmarkBold} & \textcolor{green}{\CheckmarkBold} &\textcolor{red}{\XSolidBrush}&\textcolor{red}{\XSolidBrush} \\
% AgentEvol~\citep{xi2024agentgym} & IFT & - & 14.5k & \textcolor{green}{\CheckmarkBold} & \textcolor{green}{\CheckmarkBold} &\textcolor{green}{\CheckmarkBold}&\textcolor{red}{\XSolidBrush} \\
% Gorilla~\citep{patil2023gorilla}& IFT & - & 16k & \textcolor{green}{\CheckmarkBold} & \textcolor{green}{\CheckmarkBold} &\textcolor{red}{\XSolidBrush}&\textcolor{green}{\CheckmarkBold}\\
% OpenFunctions-v2~\citep{patil2023gorilla} & IFT & - & 65k & \textcolor{green}{\CheckmarkBold} & \textcolor{green}{\CheckmarkBold} &\textcolor{red}{\XSolidBrush}&\textcolor{green}{\CheckmarkBold}\\
% LAM~\citep{zhang2024agentohana} & IFT & - & 42.6k & \textcolor{green}{\CheckmarkBold} & \textcolor{green}{\CheckmarkBold} &\textcolor{green}{\CheckmarkBold}&\textcolor{red}{\XSolidBrush} \\
% xLAM~\citep{liu2024apigen} & IFT & - & 60k & \textcolor{green}{\CheckmarkBold} & \textcolor{green}{\CheckmarkBold} &\textcolor{green}{\CheckmarkBold}&\textcolor{red}{\XSolidBrush} \\\midrule
% LEMUR~\citep{xu2024lemur} & PT & 90B & 300k & \textcolor{green}{\CheckmarkBold} & \textcolor{green}{\CheckmarkBold} &\textcolor{green}{\CheckmarkBold}&\textcolor{red}{\XSolidBrush}\\
% \rowcolor{teal!12} \method & PT & 103B & 95k & \textcolor{green}{\CheckmarkBold} & \textcolor{green}{\CheckmarkBold} & \textcolor{green}{\CheckmarkBold} & \textcolor{green}{\CheckmarkBold} \\
% \bottomrule
% \end{tabular}
% \caption{Summary of existing tuning- and pretraining-based LLM agents with their training sample sizes. "PT" and "IFT" denote "Pre-Training" and "Instruction Fine-Tuning", respectively. }
% \label{tab:related}
% \end{table*}

\begin{table*}[ht]
\begin{threeparttable}
\centering 
\renewcommand\arraystretch{0.98}
\fontsize{7}{9}\selectfont \setlength{\tabcolsep}{0.2em}
\begin{tabular}{@{}l|c|c|ccc|cc|cc|cccc@{}}
\toprule
\textbf{Methods} & \textbf{Datasets}           & \begin{tabular}[c]{@{}c@{}}\textbf{Training}\\ \textbf{Paradigm}\end{tabular} & \begin{tabular}[c]{@{}c@{}}\textbf{\# PT Data}\\ \textbf{(Tokens)}\end{tabular} & \begin{tabular}[c]{@{}c@{}}\textbf{\# IFT Data}\\ \textbf{(Samples)}\end{tabular} & \textbf{\# APIs} & \textbf{Code}  & \begin{tabular}[c]{@{}c@{}}\textbf{Nat.}\\ \textbf{Lang.}\end{tabular} & \begin{tabular}[c]{@{}c@{}}\textbf{Action}\\ \textbf{Traj.}\end{tabular} & \begin{tabular}[c]{@{}c@{}}\textbf{API}\\ \textbf{Doc.}\end{tabular} & \begin{tabular}[c]{@{}c@{}}\textbf{Func.}\\ \textbf{Call}\end{tabular} & \begin{tabular}[c]{@{}c@{}}\textbf{Multi.}\\ \textbf{Step}\end{tabular}  & \begin{tabular}[c]{@{}c@{}}\textbf{Plan}\\ \textbf{Refine}\end{tabular}  & \begin{tabular}[c]{@{}c@{}}\textbf{Multi.}\\ \textbf{Turn}\end{tabular}\\ \midrule 
\multicolumn{13}{l}{\emph{Instruction Finetuning-based LLM Agents for Intrinsic Reasoning}}  \\ \midrule
FireAct~\cite{chen2023fireact} & FireAct & IFT & - & 2.1K & 10 & \textcolor{red}{\XSolidBrush} &\textcolor{green}{\CheckmarkBold} &\textcolor{green}{\CheckmarkBold}  & \textcolor{red}{\XSolidBrush} &\textcolor{green}{\CheckmarkBold} & \textcolor{red}{\XSolidBrush} &\textcolor{green}{\CheckmarkBold} & \textcolor{red}{\XSolidBrush} \\
ToolAlpaca~\cite{tang2023toolalpaca} & ToolAlpaca & IFT & - & 4.0K & 400 & \textcolor{red}{\XSolidBrush} &\textcolor{green}{\CheckmarkBold} &\textcolor{green}{\CheckmarkBold} & \textcolor{red}{\XSolidBrush} &\textcolor{green}{\CheckmarkBold} & \textcolor{red}{\XSolidBrush}  &\textcolor{green}{\CheckmarkBold} & \textcolor{red}{\XSolidBrush}  \\
ToolLLaMA~\cite{qin2023toolllm} & ToolBench & IFT & - & 12.7K & 16,464 & \textcolor{red}{\XSolidBrush} &\textcolor{green}{\CheckmarkBold} &\textcolor{green}{\CheckmarkBold} &\textcolor{red}{\XSolidBrush} &\textcolor{green}{\CheckmarkBold}&\textcolor{green}{\CheckmarkBold}&\textcolor{green}{\CheckmarkBold} &\textcolor{green}{\CheckmarkBold}\\
AgentEvol~\citep{xi2024agentgym} & AgentTraj-L & IFT & - & 14.5K & 24 &\textcolor{red}{\XSolidBrush} & \textcolor{green}{\CheckmarkBold} &\textcolor{green}{\CheckmarkBold}&\textcolor{red}{\XSolidBrush} &\textcolor{green}{\CheckmarkBold}&\textcolor{red}{\XSolidBrush} &\textcolor{red}{\XSolidBrush} &\textcolor{green}{\CheckmarkBold}\\
Lumos~\cite{yin2024agent} & Lumos & IFT  & - & 20.0K & 16 &\textcolor{red}{\XSolidBrush} & \textcolor{green}{\CheckmarkBold} & \textcolor{green}{\CheckmarkBold} &\textcolor{red}{\XSolidBrush} & \textcolor{green}{\CheckmarkBold} & \textcolor{green}{\CheckmarkBold} &\textcolor{red}{\XSolidBrush} & \textcolor{green}{\CheckmarkBold}\\
Agent-FLAN~\cite{chen2024agent} & Agent-FLAN & IFT & - & 24.7K & 20 &\textcolor{red}{\XSolidBrush} & \textcolor{green}{\CheckmarkBold} & \textcolor{green}{\CheckmarkBold} &\textcolor{red}{\XSolidBrush} & \textcolor{green}{\CheckmarkBold}& \textcolor{green}{\CheckmarkBold}&\textcolor{red}{\XSolidBrush} & \textcolor{green}{\CheckmarkBold}\\
AgentTuning~\citep{zeng2023agenttuning} & AgentInstruct & IFT & - & 35.0K & - &\textcolor{red}{\XSolidBrush} & \textcolor{green}{\CheckmarkBold} & \textcolor{green}{\CheckmarkBold} &\textcolor{red}{\XSolidBrush} & \textcolor{green}{\CheckmarkBold} &\textcolor{red}{\XSolidBrush} &\textcolor{red}{\XSolidBrush} & \textcolor{green}{\CheckmarkBold}\\\midrule
\multicolumn{13}{l}{\emph{Instruction Finetuning-based LLM Agents for Function Calling}} \\\midrule
NexusRaven~\citep{srinivasan2023nexusraven} & NexusRaven & IFT & - & - & 116 & \textcolor{green}{\CheckmarkBold} & \textcolor{green}{\CheckmarkBold}  & \textcolor{green}{\CheckmarkBold} &\textcolor{red}{\XSolidBrush} & \textcolor{green}{\CheckmarkBold} &\textcolor{red}{\XSolidBrush} &\textcolor{red}{\XSolidBrush}&\textcolor{red}{\XSolidBrush}\\
Gorilla~\citep{patil2023gorilla} & Gorilla & IFT & - & 16.0K & 1,645 & \textcolor{green}{\CheckmarkBold} &\textcolor{red}{\XSolidBrush} &\textcolor{red}{\XSolidBrush}&\textcolor{green}{\CheckmarkBold} &\textcolor{green}{\CheckmarkBold} &\textcolor{red}{\XSolidBrush} &\textcolor{red}{\XSolidBrush} &\textcolor{red}{\XSolidBrush}\\
OpenFunctions-v2~\citep{patil2023gorilla} & OpenFunctions-v2 & IFT & - & 65.0K & - & \textcolor{green}{\CheckmarkBold} & \textcolor{green}{\CheckmarkBold} &\textcolor{red}{\XSolidBrush} &\textcolor{green}{\CheckmarkBold} &\textcolor{green}{\CheckmarkBold} &\textcolor{red}{\XSolidBrush} &\textcolor{red}{\XSolidBrush} &\textcolor{red}{\XSolidBrush}\\
API Pack~\cite{guo2024api} & API Pack & IFT & - & 1.1M & 11,213 &\textcolor{green}{\CheckmarkBold} &\textcolor{red}{\XSolidBrush} &\textcolor{green}{\CheckmarkBold} &\textcolor{red}{\XSolidBrush} &\textcolor{green}{\CheckmarkBold} &\textcolor{red}{\XSolidBrush}&\textcolor{red}{\XSolidBrush}&\textcolor{red}{\XSolidBrush}\\ 
LAM~\citep{zhang2024agentohana} & AgentOhana & IFT & - & 42.6K & - & \textcolor{green}{\CheckmarkBold} & \textcolor{green}{\CheckmarkBold} &\textcolor{green}{\CheckmarkBold}&\textcolor{red}{\XSolidBrush} &\textcolor{green}{\CheckmarkBold}&\textcolor{red}{\XSolidBrush}&\textcolor{green}{\CheckmarkBold}&\textcolor{green}{\CheckmarkBold}\\
xLAM~\citep{liu2024apigen} & APIGen & IFT & - & 60.0K & 3,673 & \textcolor{green}{\CheckmarkBold} & \textcolor{green}{\CheckmarkBold} &\textcolor{green}{\CheckmarkBold}&\textcolor{red}{\XSolidBrush} &\textcolor{green}{\CheckmarkBold}&\textcolor{red}{\XSolidBrush}&\textcolor{green}{\CheckmarkBold}&\textcolor{green}{\CheckmarkBold}\\\midrule
\multicolumn{13}{l}{\emph{Pretraining-based LLM Agents}}  \\\midrule
% LEMUR~\citep{xu2024lemur} & PT & 90B & 300.0K & - & \textcolor{green}{\CheckmarkBold} & \textcolor{green}{\CheckmarkBold} &\textcolor{green}{\CheckmarkBold}&\textcolor{red}{\XSolidBrush} & \textcolor{red}{\XSolidBrush} &\textcolor{green}{\CheckmarkBold} &\textcolor{red}{\XSolidBrush}&\textcolor{red}{\XSolidBrush}\\
\rowcolor{teal!12} \method & \dataset & PT & 103B & 95.0K  & 76,537  & \textcolor{green}{\CheckmarkBold} & \textcolor{green}{\CheckmarkBold} & \textcolor{green}{\CheckmarkBold} & \textcolor{green}{\CheckmarkBold} & \textcolor{green}{\CheckmarkBold} & \textcolor{green}{\CheckmarkBold} & \textcolor{green}{\CheckmarkBold} & \textcolor{green}{\CheckmarkBold}\\
\bottomrule
\end{tabular}
% \begin{tablenotes}
%     \item $^*$ In addition, the StarCoder-API can offer 4.77M more APIs.
% \end{tablenotes}
\caption{Summary of existing instruction finetuning-based LLM agents for intrinsic reasoning and function calling, along with their training resources and sample sizes. "PT" and "IFT" denote "Pre-Training" and "Instruction Fine-Tuning", respectively.}
\vspace{-2ex}
\label{tab:related}
\end{threeparttable}
\end{table*}

\noindent \textbf{Prompting-based LLM Agents.} Due to the lack of agent-specific pre-training corpus, existing LLM agents rely on either prompt engineering~\cite{hsieh2023tool,lu2024chameleon,yao2022react,wang2023voyager} or instruction fine-tuning~\cite{chen2023fireact,zeng2023agenttuning} to understand human instructions, decompose high-level tasks, generate grounded plans, and execute multi-step actions. 
However, prompting-based methods mainly depend on the capabilities of backbone LLMs (usually commercial LLMs), failing to introduce new knowledge and struggling to generalize to unseen tasks~\cite{sun2024adaplanner,zhuang2023toolchain}. 

\noindent \textbf{Instruction Finetuning-based LLM Agents.} Considering the extensive diversity of APIs and the complexity of multi-tool instructions, tool learning inherently presents greater challenges than natural language tasks, such as text generation~\cite{qin2023toolllm}.
Post-training techniques focus more on instruction following and aligning output with specific formats~\cite{patil2023gorilla,hao2024toolkengpt,qin2023toolllm,schick2024toolformer}, rather than fundamentally improving model knowledge or capabilities. 
Moreover, heavy fine-tuning can hinder generalization or even degrade performance in non-agent use cases, potentially suppressing the original base model capabilities~\cite{ghosh2024a}.

\noindent \textbf{Pretraining-based LLM Agents.} While pre-training serves as an essential alternative, prior works~\cite{nijkamp2023codegen,roziere2023code,xu2024lemur,patil2023gorilla} have primarily focused on improving task-specific capabilities (\eg, code generation) instead of general-domain LLM agents, due to single-source, uni-type, small-scale, and poor-quality pre-training data. 
Existing tool documentation data for agent training either lacks diverse real-world APIs~\cite{patil2023gorilla, tang2023toolalpaca} or is constrained to single-tool or single-round tool execution. 
Furthermore, trajectory data mostly imitate expert behavior or follow function-calling rules with inferior planning and reasoning, failing to fully elicit LLMs' capabilities and handle complex instructions~\cite{qin2023toolllm}. 
Given a wide range of candidate API functions, each comprising various function names and parameters available at every planning step, identifying globally optimal solutions and generalizing across tasks remains highly challenging.



\section{Preliminaries}
\label{Preliminaries}
\begin{figure*}[t]
    \centering
    \includegraphics[width=0.95\linewidth]{fig/HealthGPT_Framework.png}
    \caption{The \ourmethod{} architecture integrates hierarchical visual perception and H-LoRA, employing a task-specific hard router to select visual features and H-LoRA plugins, ultimately generating outputs with an autoregressive manner.}
    \label{fig:architecture}
\end{figure*}
\noindent\textbf{Large Vision-Language Models.} 
The input to a LVLM typically consists of an image $x^{\text{img}}$ and a discrete text sequence $x^{\text{txt}}$. The visual encoder $\mathcal{E}^{\text{img}}$ converts the input image $x^{\text{img}}$ into a sequence of visual tokens $\mathcal{V} = [v_i]_{i=1}^{N_v}$, while the text sequence $x^{\text{txt}}$ is mapped into a sequence of text tokens $\mathcal{T} = [t_i]_{i=1}^{N_t}$ using an embedding function $\mathcal{E}^{\text{txt}}$. The LLM $\mathcal{M_\text{LLM}}(\cdot|\theta)$ models the joint probability of the token sequence $\mathcal{U} = \{\mathcal{V},\mathcal{T}\}$, which is expressed as:
\begin{equation}
    P_\theta(R | \mathcal{U}) = \prod_{i=1}^{N_r} P_\theta(r_i | \{\mathcal{U}, r_{<i}\}),
\end{equation}
where $R = [r_i]_{i=1}^{N_r}$ is the text response sequence. The LVLM iteratively generates the next token $r_i$ based on $r_{<i}$. The optimization objective is to minimize the cross-entropy loss of the response $\mathcal{R}$.
% \begin{equation}
%     \mathcal{L}_{\text{VLM}} = \mathbb{E}_{R|\mathcal{U}}\left[-\log P_\theta(R | \mathcal{U})\right]
% \end{equation}
It is worth noting that most LVLMs adopt a design paradigm based on ViT, alignment adapters, and pre-trained LLMs\cite{liu2023llava,liu2024improved}, enabling quick adaptation to downstream tasks.


\noindent\textbf{VQGAN.}
VQGAN~\cite{esser2021taming} employs latent space compression and indexing mechanisms to effectively learn a complete discrete representation of images. VQGAN first maps the input image $x^{\text{img}}$ to a latent representation $z = \mathcal{E}(x)$ through a encoder $\mathcal{E}$. Then, the latent representation is quantized using a codebook $\mathcal{Z} = \{z_k\}_{k=1}^K$, generating a discrete index sequence $\mathcal{I} = [i_m]_{m=1}^N$, where $i_m \in \mathcal{Z}$ represents the quantized code index:
\begin{equation}
    \mathcal{I} = \text{Quantize}(z|\mathcal{Z}) = \arg\min_{z_k \in \mathcal{Z}} \| z - z_k \|_2.
\end{equation}
In our approach, the discrete index sequence $\mathcal{I}$ serves as a supervisory signal for the generation task, enabling the model to predict the index sequence $\hat{\mathcal{I}}$ from input conditions such as text or other modality signals.  
Finally, the predicted index sequence $\hat{\mathcal{I}}$ is upsampled by the VQGAN decoder $G$, generating the high-quality image $\hat{x}^\text{img} = G(\hat{\mathcal{I}})$.



\noindent\textbf{Low Rank Adaptation.} 
LoRA\cite{hu2021lora} effectively captures the characteristics of downstream tasks by introducing low-rank adapters. The core idea is to decompose the bypass weight matrix $\Delta W\in\mathbb{R}^{d^{\text{in}} \times d^{\text{out}}}$ into two low-rank matrices $ \{A \in \mathbb{R}^{d^{\text{in}} \times r}, B \in \mathbb{R}^{r \times d^{\text{out}}} \}$, where $ r \ll \min\{d^{\text{in}}, d^{\text{out}}\} $, significantly reducing learnable parameters. The output with the LoRA adapter for the input $x$ is then given by:
\begin{equation}
    h = x W_0 + \alpha x \Delta W/r = x W_0 + \alpha xAB/r,
\end{equation}
where matrix $ A $ is initialized with a Gaussian distribution, while the matrix $ B $ is initialized as a zero matrix. The scaling factor $ \alpha/r $ controls the impact of $ \Delta W $ on the model.

\section{HealthGPT}
\label{Method}


\subsection{Unified Autoregressive Generation.}  
% As shown in Figure~\ref{fig:architecture}, 
\ourmethod{} (Figure~\ref{fig:architecture}) utilizes a discrete token representation that covers both text and visual outputs, unifying visual comprehension and generation as an autoregressive task. 
For comprehension, $\mathcal{M}_\text{llm}$ receives the input joint sequence $\mathcal{U}$ and outputs a series of text token $\mathcal{R} = [r_1, r_2, \dots, r_{N_r}]$, where $r_i \in \mathcal{V}_{\text{txt}}$, and $\mathcal{V}_{\text{txt}}$ represents the LLM's vocabulary:
\begin{equation}
    P_\theta(\mathcal{R} \mid \mathcal{U}) = \prod_{i=1}^{N_r} P_\theta(r_i \mid \mathcal{U}, r_{<i}).
\end{equation}
For generation, $\mathcal{M}_\text{llm}$ first receives a special start token $\langle \text{START\_IMG} \rangle$, then generates a series of tokens corresponding to the VQGAN indices $\mathcal{I} = [i_1, i_2, \dots, i_{N_i}]$, where $i_j \in \mathcal{V}_{\text{vq}}$, and $\mathcal{V}_{\text{vq}}$ represents the index range of VQGAN. Upon completion of generation, the LLM outputs an end token $\langle \text{END\_IMG} \rangle$:
\begin{equation}
    P_\theta(\mathcal{I} \mid \mathcal{U}) = \prod_{j=1}^{N_i} P_\theta(i_j \mid \mathcal{U}, i_{<j}).
\end{equation}
Finally, the generated index sequence $\mathcal{I}$ is fed into the decoder $G$, which reconstructs the target image $\hat{x}^{\text{img}} = G(\mathcal{I})$.

\subsection{Hierarchical Visual Perception}  
Given the differences in visual perception between comprehension and generation tasks—where the former focuses on abstract semantics and the latter emphasizes complete semantics—we employ ViT to compress the image into discrete visual tokens at multiple hierarchical levels.
Specifically, the image is converted into a series of features $\{f_1, f_2, \dots, f_L\}$ as it passes through $L$ ViT blocks.

To address the needs of various tasks, the hidden states are divided into two types: (i) \textit{Concrete-grained features} $\mathcal{F}^{\text{Con}} = \{f_1, f_2, \dots, f_k\}, k < L$, derived from the shallower layers of ViT, containing sufficient global features, suitable for generation tasks; 
(ii) \textit{Abstract-grained features} $\mathcal{F}^{\text{Abs}} = \{f_{k+1}, f_{k+2}, \dots, f_L\}$, derived from the deeper layers of ViT, which contain abstract semantic information closer to the text space, suitable for comprehension tasks.

The task type $T$ (comprehension or generation) determines which set of features is selected as the input for the downstream large language model:
\begin{equation}
    \mathcal{F}^{\text{img}}_T =
    \begin{cases}
        \mathcal{F}^{\text{Con}}, & \text{if } T = \text{generation task} \\
        \mathcal{F}^{\text{Abs}}, & \text{if } T = \text{comprehension task}
    \end{cases}
\end{equation}
We integrate the image features $\mathcal{F}^{\text{img}}_T$ and text features $\mathcal{T}$ into a joint sequence through simple concatenation, which is then fed into the LLM $\mathcal{M}_{\text{llm}}$ for autoregressive generation.
% :
% \begin{equation}
%     \mathcal{R} = \mathcal{M}_{\text{llm}}(\mathcal{U}|\theta), \quad \mathcal{U} = [\mathcal{F}^{\text{img}}_T; \mathcal{T}]
% \end{equation}
\subsection{Heterogeneous Knowledge Adaptation}
We devise H-LoRA, which stores heterogeneous knowledge from comprehension and generation tasks in separate modules and dynamically routes to extract task-relevant knowledge from these modules. 
At the task level, for each task type $ T $, we dynamically assign a dedicated H-LoRA submodule $ \theta^T $, which is expressed as:
\begin{equation}
    \mathcal{R} = \mathcal{M}_\text{LLM}(\mathcal{U}|\theta, \theta^T), \quad \theta^T = \{A^T, B^T, \mathcal{R}^T_\text{outer}\}.
\end{equation}
At the feature level for a single task, H-LoRA integrates the idea of Mixture of Experts (MoE)~\cite{masoudnia2014mixture} and designs an efficient matrix merging and routing weight allocation mechanism, thus avoiding the significant computational delay introduced by matrix splitting in existing MoELoRA~\cite{luo2024moelora}. Specifically, we first merge the low-rank matrices (rank = r) of $ k $ LoRA experts into a unified matrix:
\begin{equation}
    \mathbf{A}^{\text{merged}}, \mathbf{B}^{\text{merged}} = \text{Concat}(\{A_i\}_1^k), \text{Concat}(\{B_i\}_1^k),
\end{equation}
where $ \mathbf{A}^{\text{merged}} \in \mathbb{R}^{d^\text{in} \times rk} $ and $ \mathbf{B}^{\text{merged}} \in \mathbb{R}^{rk \times d^\text{out}} $. The $k$-dimension routing layer generates expert weights $ \mathcal{W} \in \mathbb{R}^{\text{token\_num} \times k} $ based on the input hidden state $ x $, and these are expanded to $ \mathbb{R}^{\text{token\_num} \times rk} $ as follows:
\begin{equation}
    \mathcal{W}^\text{expanded} = \alpha k \mathcal{W} / r \otimes \mathbf{1}_r,
\end{equation}
where $ \otimes $ denotes the replication operation.
The overall output of H-LoRA is computed as:
\begin{equation}
    \mathcal{O}^\text{H-LoRA} = (x \mathbf{A}^{\text{merged}} \odot \mathcal{W}^\text{expanded}) \mathbf{B}^{\text{merged}},
\end{equation}
where $ \odot $ represents element-wise multiplication. Finally, the output of H-LoRA is added to the frozen pre-trained weights to produce the final output:
\begin{equation}
    \mathcal{O} = x W_0 + \mathcal{O}^\text{H-LoRA}.
\end{equation}
% In summary, H-LoRA is a task-based dynamic PEFT method that achieves high efficiency in single-task fine-tuning.

\subsection{Training Pipeline}

\begin{figure}[t]
    \centering
    \hspace{-4mm}
    \includegraphics[width=0.94\linewidth]{fig/data.pdf}
    \caption{Data statistics of \texttt{VL-Health}. }
    \label{fig:data}
\end{figure}
\noindent \textbf{1st Stage: Multi-modal Alignment.} 
In the first stage, we design separate visual adapters and H-LoRA submodules for medical unified tasks. For the medical comprehension task, we train abstract-grained visual adapters using high-quality image-text pairs to align visual embeddings with textual embeddings, thereby enabling the model to accurately describe medical visual content. During this process, the pre-trained LLM and its corresponding H-LoRA submodules remain frozen. In contrast, the medical generation task requires training concrete-grained adapters and H-LoRA submodules while keeping the LLM frozen. Meanwhile, we extend the textual vocabulary to include multimodal tokens, enabling the support of additional VQGAN vector quantization indices. The model trains on image-VQ pairs, endowing the pre-trained LLM with the capability for image reconstruction. This design ensures pixel-level consistency of pre- and post-LVLM. The processes establish the initial alignment between the LLM’s outputs and the visual inputs.

\noindent \textbf{2nd Stage: Heterogeneous H-LoRA Plugin Adaptation.}  
The submodules of H-LoRA share the word embedding layer and output head but may encounter issues such as bias and scale inconsistencies during training across different tasks. To ensure that the multiple H-LoRA plugins seamlessly interface with the LLMs and form a unified base, we fine-tune the word embedding layer and output head using a small amount of mixed data to maintain consistency in the model weights. Specifically, during this stage, all H-LoRA submodules for different tasks are kept frozen, with only the word embedding layer and output head being optimized. Through this stage, the model accumulates foundational knowledge for unified tasks by adapting H-LoRA plugins.

\begin{table*}[!t]
\centering
\caption{Comparison of \ourmethod{} with other LVLMs and unified multi-modal models on medical visual comprehension tasks. \textbf{Bold} and \underline{underlined} text indicates the best performance and second-best performance, respectively.}
\resizebox{\textwidth}{!}{
\begin{tabular}{c|lcc|cccccccc|c}
\toprule
\rowcolor[HTML]{E9F3FE} &  &  &  & \multicolumn{2}{c}{\textbf{VQA-RAD \textuparrow}} & \multicolumn{2}{c}{\textbf{SLAKE \textuparrow}} & \multicolumn{2}{c}{\textbf{PathVQA \textuparrow}} &  &  &  \\ 
\cline{5-10}
\rowcolor[HTML]{E9F3FE}\multirow{-2}{*}{\textbf{Type}} & \multirow{-2}{*}{\textbf{Model}} & \multirow{-2}{*}{\textbf{\# Params}} & \multirow{-2}{*}{\makecell{\textbf{Medical} \\ \textbf{LVLM}}} & \textbf{close} & \textbf{all} & \textbf{close} & \textbf{all} & \textbf{close} & \textbf{all} & \multirow{-2}{*}{\makecell{\textbf{MMMU} \\ \textbf{-Med}}\textuparrow} & \multirow{-2}{*}{\textbf{OMVQA}\textuparrow} & \multirow{-2}{*}{\textbf{Avg. \textuparrow}} \\ 
\midrule \midrule
\multirow{9}{*}{\textbf{Comp. Only}} 
& Med-Flamingo & 8.3B & \Large \ding{51} & 58.6 & 43.0 & 47.0 & 25.5 & 61.9 & 31.3 & 28.7 & 34.9 & 41.4 \\
& LLaVA-Med & 7B & \Large \ding{51} & 60.2 & 48.1 & 58.4 & 44.8 & 62.3 & 35.7 & 30.0 & 41.3 & 47.6 \\
& HuatuoGPT-Vision & 7B & \Large \ding{51} & 66.9 & 53.0 & 59.8 & 49.1 & 52.9 & 32.0 & 42.0 & 50.0 & 50.7 \\
& BLIP-2 & 6.7B & \Large \ding{55} & 43.4 & 36.8 & 41.6 & 35.3 & 48.5 & 28.8 & 27.3 & 26.9 & 36.1 \\
& LLaVA-v1.5 & 7B & \Large \ding{55} & 51.8 & 42.8 & 37.1 & 37.7 & 53.5 & 31.4 & 32.7 & 44.7 & 41.5 \\
& InstructBLIP & 7B & \Large \ding{55} & 61.0 & 44.8 & 66.8 & 43.3 & 56.0 & 32.3 & 25.3 & 29.0 & 44.8 \\
& Yi-VL & 6B & \Large \ding{55} & 52.6 & 42.1 & 52.4 & 38.4 & 54.9 & 30.9 & 38.0 & 50.2 & 44.9 \\
& InternVL2 & 8B & \Large \ding{55} & 64.9 & 49.0 & 66.6 & 50.1 & 60.0 & 31.9 & \underline{43.3} & 54.5 & 52.5\\
& Llama-3.2 & 11B & \Large \ding{55} & 68.9 & 45.5 & 72.4 & 52.1 & 62.8 & 33.6 & 39.3 & 63.2 & 54.7 \\
\midrule
\multirow{5}{*}{\textbf{Comp. \& Gen.}} 
& Show-o & 1.3B & \Large \ding{55} & 50.6 & 33.9 & 31.5 & 17.9 & 52.9 & 28.2 & 22.7 & 45.7 & 42.6 \\
& Unified-IO 2 & 7B & \Large \ding{55} & 46.2 & 32.6 & 35.9 & 21.9 & 52.5 & 27.0 & 25.3 & 33.0 & 33.8 \\
& Janus & 1.3B & \Large \ding{55} & 70.9 & 52.8 & 34.7 & 26.9 & 51.9 & 27.9 & 30.0 & 26.8 & 33.5 \\
& \cellcolor[HTML]{DAE0FB}HealthGPT-M3 & \cellcolor[HTML]{DAE0FB}3.8B & \cellcolor[HTML]{DAE0FB}\Large \ding{51} & \cellcolor[HTML]{DAE0FB}\underline{73.7} & \cellcolor[HTML]{DAE0FB}\underline{55.9} & \cellcolor[HTML]{DAE0FB}\underline{74.6} & \cellcolor[HTML]{DAE0FB}\underline{56.4} & \cellcolor[HTML]{DAE0FB}\underline{78.7} & \cellcolor[HTML]{DAE0FB}\underline{39.7} & \cellcolor[HTML]{DAE0FB}\underline{43.3} & \cellcolor[HTML]{DAE0FB}\underline{68.5} & \cellcolor[HTML]{DAE0FB}\underline{61.3} \\
& \cellcolor[HTML]{DAE0FB}HealthGPT-L14 & \cellcolor[HTML]{DAE0FB}14B & \cellcolor[HTML]{DAE0FB}\Large \ding{51} & \cellcolor[HTML]{DAE0FB}\textbf{77.7} & \cellcolor[HTML]{DAE0FB}\textbf{58.3} & \cellcolor[HTML]{DAE0FB}\textbf{76.4} & \cellcolor[HTML]{DAE0FB}\textbf{64.5} & \cellcolor[HTML]{DAE0FB}\textbf{85.9} & \cellcolor[HTML]{DAE0FB}\textbf{44.4} & \cellcolor[HTML]{DAE0FB}\textbf{49.2} & \cellcolor[HTML]{DAE0FB}\textbf{74.4} & \cellcolor[HTML]{DAE0FB}\textbf{66.4} \\
\bottomrule
\end{tabular}
}
\label{tab:results}
\end{table*}
\begin{table*}[ht]
    \centering
    \caption{The experimental results for the four modality conversion tasks.}
    \resizebox{\textwidth}{!}{
    \begin{tabular}{l|ccc|ccc|ccc|ccc}
        \toprule
        \rowcolor[HTML]{E9F3FE} & \multicolumn{3}{c}{\textbf{CT to MRI (Brain)}} & \multicolumn{3}{c}{\textbf{CT to MRI (Pelvis)}} & \multicolumn{3}{c}{\textbf{MRI to CT (Brain)}} & \multicolumn{3}{c}{\textbf{MRI to CT (Pelvis)}} \\
        \cline{2-13}
        \rowcolor[HTML]{E9F3FE}\multirow{-2}{*}{\textbf{Model}}& \textbf{SSIM $\uparrow$} & \textbf{PSNR $\uparrow$} & \textbf{MSE $\downarrow$} & \textbf{SSIM $\uparrow$} & \textbf{PSNR $\uparrow$} & \textbf{MSE $\downarrow$} & \textbf{SSIM $\uparrow$} & \textbf{PSNR $\uparrow$} & \textbf{MSE $\downarrow$} & \textbf{SSIM $\uparrow$} & \textbf{PSNR $\uparrow$} & \textbf{MSE $\downarrow$} \\
        \midrule \midrule
        pix2pix & 71.09 & 32.65 & 36.85 & 59.17 & 31.02 & 51.91 & 78.79 & 33.85 & 28.33 & 72.31 & 32.98 & 36.19 \\
        CycleGAN & 54.76 & 32.23 & 40.56 & 54.54 & 30.77 & 55.00 & 63.75 & 31.02 & 52.78 & 50.54 & 29.89 & 67.78 \\
        BBDM & {71.69} & {32.91} & {34.44} & 57.37 & 31.37 & 48.06 & \textbf{86.40} & 34.12 & 26.61 & {79.26} & 33.15 & 33.60 \\
        Vmanba & 69.54 & 32.67 & 36.42 & {63.01} & {31.47} & {46.99} & 79.63 & 34.12 & 26.49 & 77.45 & 33.53 & 31.85 \\
        DiffMa & 71.47 & 32.74 & 35.77 & 62.56 & 31.43 & 47.38 & 79.00 & {34.13} & {26.45} & 78.53 & {33.68} & {30.51} \\
        \rowcolor[HTML]{DAE0FB}HealthGPT-M3 & \underline{79.38} & \underline{33.03} & \underline{33.48} & \underline{71.81} & \underline{31.83} & \underline{43.45} & {85.06} & \textbf{34.40} & \textbf{25.49} & \underline{84.23} & \textbf{34.29} & \textbf{27.99} \\
        \rowcolor[HTML]{DAE0FB}HealthGPT-L14 & \textbf{79.73} & \textbf{33.10} & \textbf{32.96} & \textbf{71.92} & \textbf{31.87} & \textbf{43.09} & \underline{85.31} & \underline{34.29} & \underline{26.20} & \textbf{84.96} & \underline{34.14} & \underline{28.13} \\
        \bottomrule
    \end{tabular}
    }
    \label{tab:conversion}
\end{table*}

\noindent \textbf{3rd Stage: Visual Instruction Fine-Tuning.}  
In the third stage, we introduce additional task-specific data to further optimize the model and enhance its adaptability to downstream tasks such as medical visual comprehension (e.g., medical QA, medical dialogues, and report generation) or generation tasks (e.g., super-resolution, denoising, and modality conversion). Notably, by this stage, the word embedding layer and output head have been fine-tuned, only the H-LoRA modules and adapter modules need to be trained. This strategy significantly improves the model's adaptability and flexibility across different tasks.


\section{Experiment}
\label{s:experiment}

\subsection{Data Description}
We evaluate our method on FI~\cite{you2016building}, Twitter\_LDL~\cite{yang2017learning} and Artphoto~\cite{machajdik2010affective}.
FI is a public dataset built from Flickr and Instagram, with 23,308 images and eight emotion categories, namely \textit{amusement}, \textit{anger}, \textit{awe},  \textit{contentment}, \textit{disgust}, \textit{excitement},  \textit{fear}, and \textit{sadness}. 
% Since images in FI are all copyrighted by law, some images are corrupted now, so we remove these samples and retain 21,828 images.
% T4SA contains images from Twitter, which are classified into three categories: \textit{positive}, \textit{neutral}, and \textit{negative}. In this paper, we adopt the base version of B-T4SA, which contains 470,586 images and provides text descriptions of the corresponding tweets.
Twitter\_LDL contains 10,045 images from Twitter, with the same eight categories as the FI dataset.
% 。
For these two datasets, they are randomly split into 80\%
training and 20\% testing set.
Artphoto contains 806 artistic photos from the DeviantArt website, which we use to further evaluate the zero-shot capability of our model.
% on the small-scale dataset.
% We construct and publicly release the first image sentiment analysis dataset containing metadata.
% 。

% Based on these datasets, we are the first to construct and publicly release metadata-enhanced image sentiment analysis datasets. These datasets include scenes, tags, descriptions, and corresponding confidence scores, and are available at this link for future research purposes.


% 
\begin{table}[t]
\centering
% \begin{center}
\caption{Overall performance of different models on FI and Twitter\_LDL datasets.}
\label{tab:cap1}
% \resizebox{\linewidth}{!}
{
\begin{tabular}{l|c|c|c|c}
\hline
\multirow{2}{*}{\textbf{Model}} & \multicolumn{2}{c|}{\textbf{FI}}  & \multicolumn{2}{c}{\textbf{Twitter\_LDL}} \\ \cline{2-5} 
  & \textbf{Accuracy} & \textbf{F1} & \textbf{Accuracy} & \textbf{F1}  \\ \hline
% (\rownumber)~AlexNet~\cite{krizhevsky2017imagenet}  & 58.13\% & 56.35\%  & 56.24\%& 55.02\%  \\ 
% (\rownumber)~VGG16~\cite{simonyan2014very}  & 63.75\%& 63.08\%  & 59.34\%& 59.02\%  \\ 
(\rownumber)~ResNet101~\cite{he2016deep} & 66.16\%& 65.56\%  & 62.02\% & 61.34\%  \\ 
(\rownumber)~CDA~\cite{han2023boosting} & 66.71\%& 65.37\%  & 64.14\% & 62.85\%  \\ 
(\rownumber)~CECCN~\cite{ruan2024color} & 67.96\%& 66.74\%  & 64.59\%& 64.72\% \\ 
(\rownumber)~EmoVIT~\cite{xie2024emovit} & 68.09\%& 67.45\%  & 63.12\% & 61.97\%  \\ 
(\rownumber)~ComLDL~\cite{zhang2022compound} & 68.83\%& 67.28\%  & 65.29\% & 63.12\%  \\ 
(\rownumber)~WSDEN~\cite{li2023weakly} & 69.78\%& 69.61\%  & 67.04\% & 65.49\% \\ 
(\rownumber)~ECWA~\cite{deng2021emotion} & 70.87\%& 69.08\%  & 67.81\% & 66.87\%  \\ 
(\rownumber)~EECon~\cite{yang2023exploiting} & 71.13\%& 68.34\%  & 64.27\%& 63.16\%  \\ 
(\rownumber)~MAM~\cite{zhang2024affective} & 71.44\%  & 70.83\% & 67.18\%  & 65.01\%\\ 
(\rownumber)~TGCA-PVT~\cite{chen2024tgca}   & 73.05\%  & 71.46\% & 69.87\%  & 68.32\% \\ 
(\rownumber)~OEAN~\cite{zhang2024object}   & 73.40\%  & 72.63\% & 70.52\%  & 69.47\% \\ \hline
(\rownumber)~\shortname  & \textbf{79.48\%} & \textbf{79.22\%} & \textbf{74.12\%} & \textbf{73.09\%} \\ \hline
\end{tabular}
}
\vspace{-6mm}
% \end{center}
\end{table}
% 

\subsection{Experiment Setting}
% \subsubsection{Model Setting.}
% 
\textbf{Model Setting:}
For feature representation, we set $k=10$ to select object tags, and adopt clip-vit-base-patch32 as the pre-trained model for unified feature representation.
Moreover, we empirically set $(d_e, d_h, d_k, d_s) = (512, 128, 16, 64)$, and set the classification class $L$ to 8.

% 

\textbf{Training Setting:}
To initialize the model, we set all weights such as $\boldsymbol{W}$ following the truncated normal distribution, and use AdamW optimizer with the learning rate of $1 \times 10^{-4}$.
% warmup scheduler of cosine, warmup steps of 2000.
Furthermore, we set the batch size to 32 and the epoch of the training process to 200.
During the implementation, we utilize \textit{PyTorch} to build our entire model.
% , and our project codes are publicly available at https://github.com/zzmyrep/MESN.
% Our project codes as well as data are all publicly available on GitHub\footnote{https://github.com/zzmyrep/KBCEN}.
% Code is available at \href{https://github.com/zzmyrep/KBCEN}{https://github.com/zzmyrep/KBCEN}.

\textbf{Evaluation Metrics:}
Following~\cite{zhang2024affective, chen2024tgca, zhang2024object}, we adopt \textit{accuracy} and \textit{F1} as our evaluation metrics to measure the performance of different methods for image sentiment analysis. 



\subsection{Experiment Result}
% We compare our model against the following baselines: AlexNet~\cite{krizhevsky2017imagenet}, VGG16~\cite{simonyan2014very}, ResNet101~\cite{he2016deep}, CECCN~\cite{ruan2024color}, EmoVIT~\cite{xie2024emovit}, WSCNet~\cite{yang2018weakly}, ECWA~\cite{deng2021emotion}, EECon~\cite{yang2023exploiting}, MAM~\cite{zhang2024affective} and TGCA-PVT~\cite{chen2024tgca}, and the overall results are summarized in Table~\ref{tab:cap1}.
We compare our model against several baselines, and the overall results are summarized in Table~\ref{tab:cap1}.
We observe that our model achieves the best performance in both accuracy and F1 metrics, significantly outperforming the previous models. 
This superior performance is mainly attributed to our effective utilization of metadata to enhance image sentiment analysis, as well as the exceptional capability of the unified sentiment transformer framework we developed. These results strongly demonstrate that our proposed method can bring encouraging performance for image sentiment analysis.

\setcounter{magicrownumbers}{0} 
\begin{table}[t]
\begin{center}
\caption{Ablation study of~\shortname~on FI dataset.} 
% \vspace{1mm}
\label{tab:cap2}
\resizebox{.9\linewidth}{!}
{
\begin{tabular}{lcc}
  \hline
  \textbf{Model} & \textbf{Accuracy} & \textbf{F1} \\
  \hline
  (\rownumber)~Ours (w/o vision) & 65.72\% & 64.54\% \\
  (\rownumber)~Ours (w/o text description) & 74.05\% & 72.58\% \\
  (\rownumber)~Ours (w/o object tag) & 77.45\% & 76.84\% \\
  (\rownumber)~Ours (w/o scene tag) & 78.47\% & 78.21\% \\
  \hline
  (\rownumber)~Ours (w/o unified embedding) & 76.41\% & 76.23\% \\
  (\rownumber)~Ours (w/o adaptive learning) & 76.83\% & 76.56\% \\
  (\rownumber)~Ours (w/o cross-modal fusion) & 76.85\% & 76.49\% \\
  \hline
  (\rownumber)~Ours  & \textbf{79.48\%} & \textbf{79.22\%} \\
  \hline
\end{tabular}
}
\end{center}
\vspace{-5mm}
\end{table}


\begin{figure}[t]
\centering
% \vspace{-2mm}
\includegraphics[width=0.42\textwidth]{fig/2dvisual-linux4-paper2.pdf}
\caption{Visualization of feature distribution on eight categories before (left) and after (right) model processing.}
% 
\label{fig:visualization}
\vspace{-5mm}
\end{figure}

\subsection{Ablation Performance}
In this subsection, we conduct an ablation study to examine which component is really important for performance improvement. The results are reported in Table~\ref{tab:cap2}.

For information utilization, we observe a significant decline in model performance when visual features are removed. Additionally, the performance of \shortname~decreases when different metadata are removed separately, which means that text description, object tag, and scene tag are all critical for image sentiment analysis.
Recalling the model architecture, we separately remove transformer layers of the unified representation module, the adaptive learning module, and the cross-modal fusion module, replacing them with MLPs of the same parameter scale.
In this way, we can observe varying degrees of decline in model performance, indicating that these modules are indispensable for our model to achieve better performance.

\subsection{Visualization}
% 


% % 开始使用minipage进行左右排列
% \begin{minipage}[t]{0.45\textwidth}  % 子图1宽度为45%
%     \centering
%     \includegraphics[width=\textwidth]{2dvisual.pdf}  % 插入图片
%     \captionof{figure}{Visualization of feature distribution.}  % 使用captionof添加图片标题
%     \label{fig:visualization}
% \end{minipage}


% \begin{figure}[t]
% \centering
% \vspace{-2mm}
% \includegraphics[width=0.45\textwidth]{fig/2dvisual.pdf}
% \caption{Visualization of feature distribution.}
% \label{fig:visualization}
% % \vspace{-4mm}
% \end{figure}

% \begin{figure}[t]
% \centering
% \vspace{-2mm}
% \includegraphics[width=0.45\textwidth]{fig/2dvisual-linux3-paper.pdf}
% \caption{Visualization of feature distribution.}
% \label{fig:visualization}
% % \vspace{-4mm}
% \end{figure}



\begin{figure}[tbp]   
\vspace{-4mm}
  \centering            
  \subfloat[Depth of adaptive learning layers]   
  {
    \label{fig:subfig1}\includegraphics[width=0.22\textwidth]{fig/fig_sensitivity-a5}
  }
  \subfloat[Depth of fusion layers]
  {
    % \label{fig:subfig2}\includegraphics[width=0.22\textwidth]{fig/fig_sensitivity-b2}
    \label{fig:subfig2}\includegraphics[width=0.22\textwidth]{fig/fig_sensitivity-b2-num.pdf}
  }
  \caption{Sensitivity study of \shortname~on different depth. }   
  \label{fig:fig_sensitivity}  
\vspace{-2mm}
\end{figure}

% \begin{figure}[htbp]
% \centerline{\includegraphics{2dvisual.pdf}}
% \caption{Visualization of feature distribution.}
% \label{fig:visualization}
% \end{figure}

% In Fig.~\ref{fig:visualization}, we use t-SNE~\cite{van2008visualizing} to reduce the dimension of data features for visualization, Figure in left represents the metadata features before model processing, the features are obtained by embedding through the CLIP model, and figure in right shows the features of the data after model processing, it can be observed that after the model processing, the data with different label categories fall in different regions in the space, therefore, we can conclude that the Therefore, we can conclude that the model can effectively utilize the information contained in the metadata and use it to guide the model for classification.

In Fig.~\ref{fig:visualization}, we use t-SNE~\cite{van2008visualizing} to reduce the dimension of data features for visualization.
The left figure shows metadata features before being processed by our model (\textit{i.e.}, embedded by CLIP), while the right shows the distribution of features after being processed by our model.
We can observe that after the model processing, data with the same label are closer to each other, while others are farther away.
Therefore, it shows that the model can effectively utilize the information contained in the metadata and use it to guide the classification process.

\subsection{Sensitivity Analysis}
% 
In this subsection, we conduct a sensitivity analysis to figure out the effect of different depth settings of adaptive learning layers and fusion layers. 
% In this subsection, we conduct a sensitivity analysis to figure out the effect of different depth settings on the model. 
% Fig.~\ref{fig:fig_sensitivity} presents the effect of different depth settings of adaptive learning layers and fusion layers. 
Taking Fig.~\ref{fig:fig_sensitivity} (a) as an example, the model performance improves with increasing depth, reaching the best performance at a depth of 4.
% Taking Fig.~\ref{fig:fig_sensitivity} (a) as an example, the performance of \shortname~improves with the increase of depth at first, reaching the best performance at a depth of 4.
When the depth continues to increase, the accuracy decreases to varying degrees.
Similar results can be observed in Fig.~\ref{fig:fig_sensitivity} (b).
Therefore, we set their depths to 4 and 6 respectively to achieve the best results.

% Through our experiments, we can observe that the effect of modifying these hyperparameters on the results of the experiments is very weak, and the surface model is not sensitive to the hyperparameters.


\subsection{Zero-shot Capability}
% 

% (1)~GCH~\cite{2010Analyzing} & 21.78\% & (5)~RA-DLNet~\cite{2020A} & 34.01\% \\ \hline
% (2)~WSCNet~\cite{2019WSCNet}  & 30.25\% & (6)~CECCN~\cite{ruan2024color} & 43.83\% \\ \hline
% (3)~PCNN~\cite{2015Robust} & 31.68\%  & (7)~EmoVIT~\cite{xie2024emovit} & 44.90\% \\ \hline
% (4)~AR~\cite{2018Visual} & 32.67\% & (8)~Ours (Zero-shot) & 47.83\% \\ \hline


\begin{table}[t]
\centering
\caption{Zero-shot capability of \shortname.}
\label{tab:cap3}
\resizebox{1\linewidth}{!}
{
\begin{tabular}{lc|lc}
\hline
\textbf{Model} & \textbf{Accuracy} & \textbf{Model} & \textbf{Accuracy} \\ \hline
(1)~WSCNet~\cite{2019WSCNet}  & 30.25\% & (5)~MAM~\cite{zhang2024affective} & 39.56\%  \\ \hline
(2)~AR~\cite{2018Visual} & 32.67\% & (6)~CECCN~\cite{ruan2024color} & 43.83\% \\ \hline
(3)~RA-DLNet~\cite{2020A} & 34.01\%  & (7)~EmoVIT~\cite{xie2024emovit} & 44.90\% \\ \hline
(4)~CDA~\cite{han2023boosting} & 38.64\% & (8)~Ours (Zero-shot) & 47.83\% \\ \hline
\end{tabular}
}
\vspace{-5mm}
\end{table}

% We use the model trained on the FI dataset to test on the artphoto dataset to verify the model's generalization ability as well as robustness to other distributed datasets.
% We can observe that the MESN model shows strong competitiveness in terms of accuracy when compared to other trained models, which suggests that the model has a good generalization ability in the OOD task.

To validate the model's generalization ability and robustness to other distributed datasets, we directly test the model trained on the FI dataset, without training on Artphoto. 
% As observed in Table 3, compared to other models trained on Artphoto, we achieve highly competitive zero-shot performance, indicating that the model has good generalization ability in out-of-distribution tasks.
From Table~\ref{tab:cap3}, we can observe that compared with other models trained on Artphoto, we achieve competitive zero-shot performance, which shows that the model has good generalization ability in out-of-distribution tasks.


%%%%%%%%%%%%
%  E2E     %
%%%%%%%%%%%%


\section{Conclusion}
In this paper, we introduced Wi-Chat, the first LLM-powered Wi-Fi-based human activity recognition system that integrates the reasoning capabilities of large language models with the sensing potential of wireless signals. Our experimental results on a self-collected Wi-Fi CSI dataset demonstrate the promising potential of LLMs in enabling zero-shot Wi-Fi sensing. These findings suggest a new paradigm for human activity recognition that does not rely on extensive labeled data. We hope future research will build upon this direction, further exploring the applications of LLMs in signal processing domains such as IoT, mobile sensing, and radar-based systems.

\section*{Limitations}
While our work represents the first attempt to leverage LLMs for processing Wi-Fi signals, it is a preliminary study focused on a relatively simple task: Wi-Fi-based human activity recognition. This choice allows us to explore the feasibility of LLMs in wireless sensing but also comes with certain limitations.

Our approach primarily evaluates zero-shot performance, which, while promising, may still lag behind traditional supervised learning methods in highly complex or fine-grained recognition tasks. Besides, our study is limited to a controlled environment with a self-collected dataset, and the generalizability of LLMs to diverse real-world scenarios with varying Wi-Fi conditions, environmental interference, and device heterogeneity remains an open question.

Additionally, we have yet to explore the full potential of LLMs in more advanced Wi-Fi sensing applications, such as fine-grained gesture recognition, occupancy detection, and passive health monitoring. Future work should investigate the scalability of LLM-based approaches, their robustness to domain shifts, and their integration with multimodal sensing techniques in broader IoT applications.


% Bibliography entries for the entire Anthology, followed by custom entries
%\bibliography{anthology,custom}
% Custom bibliography entries only
\bibliography{main}
\newpage
\appendix

\section{Experiment prompts}
\label{sec:prompt}
The prompts used in the LLM experiments are shown in the following Table~\ref{tab:prompts}.

\definecolor{titlecolor}{rgb}{0.9, 0.5, 0.1}
\definecolor{anscolor}{rgb}{0.2, 0.5, 0.8}
\definecolor{labelcolor}{HTML}{48a07e}
\begin{table*}[h]
	\centering
	
 % \vspace{-0.2cm}
	
	\begin{center}
		\begin{tikzpicture}[
				chatbox_inner/.style={rectangle, rounded corners, opacity=0, text opacity=1, font=\sffamily\scriptsize, text width=5in, text height=9pt, inner xsep=6pt, inner ysep=6pt},
				chatbox_prompt_inner/.style={chatbox_inner, align=flush left, xshift=0pt, text height=11pt},
				chatbox_user_inner/.style={chatbox_inner, align=flush left, xshift=0pt},
				chatbox_gpt_inner/.style={chatbox_inner, align=flush left, xshift=0pt},
				chatbox/.style={chatbox_inner, draw=black!25, fill=gray!7, opacity=1, text opacity=0},
				chatbox_prompt/.style={chatbox, align=flush left, fill=gray!1.5, draw=black!30, text height=10pt},
				chatbox_user/.style={chatbox, align=flush left},
				chatbox_gpt/.style={chatbox, align=flush left},
				chatbox2/.style={chatbox_gpt, fill=green!25},
				chatbox3/.style={chatbox_gpt, fill=red!20, draw=black!20},
				chatbox4/.style={chatbox_gpt, fill=yellow!30},
				labelbox/.style={rectangle, rounded corners, draw=black!50, font=\sffamily\scriptsize\bfseries, fill=gray!5, inner sep=3pt},
			]
											
			\node[chatbox_user] (q1) {
				\textbf{System prompt}
				\newline
				\newline
				You are a helpful and precise assistant for segmenting and labeling sentences. We would like to request your help on curating a dataset for entity-level hallucination detection.
				\newline \newline
                We will give you a machine generated biography and a list of checked facts about the biography. Each fact consists of a sentence and a label (True/False). Please do the following process. First, breaking down the biography into words. Second, by referring to the provided list of facts, merging some broken down words in the previous step to form meaningful entities. For example, ``strategic thinking'' should be one entity instead of two. Third, according to the labels in the list of facts, labeling each entity as True or False. Specifically, for facts that share a similar sentence structure (\eg, \textit{``He was born on Mach 9, 1941.''} (\texttt{True}) and \textit{``He was born in Ramos Mejia.''} (\texttt{False})), please first assign labels to entities that differ across atomic facts. For example, first labeling ``Mach 9, 1941'' (\texttt{True}) and ``Ramos Mejia'' (\texttt{False}) in the above case. For those entities that are the same across atomic facts (\eg, ``was born'') or are neutral (\eg, ``he,'' ``in,'' and ``on''), please label them as \texttt{True}. For the cases that there is no atomic fact that shares the same sentence structure, please identify the most informative entities in the sentence and label them with the same label as the atomic fact while treating the rest of the entities as \texttt{True}. In the end, output the entities and labels in the following format:
                \begin{itemize}[nosep]
                    \item Entity 1 (Label 1)
                    \item Entity 2 (Label 2)
                    \item ...
                    \item Entity N (Label N)
                \end{itemize}
                % \newline \newline
                Here are two examples:
                \newline\newline
                \textbf{[Example 1]}
                \newline
                [The start of the biography]
                \newline
                \textcolor{titlecolor}{Marianne McAndrew is an American actress and singer, born on November 21, 1942, in Cleveland, Ohio. She began her acting career in the late 1960s, appearing in various television shows and films.}
                \newline
                [The end of the biography]
                \newline \newline
                [The start of the list of checked facts]
                \newline
                \textcolor{anscolor}{[Marianne McAndrew is an American. (False); Marianne McAndrew is an actress. (True); Marianne McAndrew is a singer. (False); Marianne McAndrew was born on November 21, 1942. (False); Marianne McAndrew was born in Cleveland, Ohio. (False); She began her acting career in the late 1960s. (True); She has appeared in various television shows. (True); She has appeared in various films. (True)]}
                \newline
                [The end of the list of checked facts]
                \newline \newline
                [The start of the ideal output]
                \newline
                \textcolor{labelcolor}{[Marianne McAndrew (True); is (True); an (True); American (False); actress (True); and (True); singer (False); , (True); born (True); on (True); November 21, 1942 (False); , (True); in (True); Cleveland, Ohio (False); . (True); She (True); began (True); her (True); acting career (True); in (True); the late 1960s (True); , (True); appearing (True); in (True); various (True); television shows (True); and (True); films (True); . (True)]}
                \newline
                [The end of the ideal output]
				\newline \newline
                \textbf{[Example 2]}
                \newline
                [The start of the biography]
                \newline
                \textcolor{titlecolor}{Doug Sheehan is an American actor who was born on April 27, 1949, in Santa Monica, California. He is best known for his roles in soap operas, including his portrayal of Joe Kelly on ``General Hospital'' and Ben Gibson on ``Knots Landing.''}
                \newline
                [The end of the biography]
                \newline \newline
                [The start of the list of checked facts]
                \newline
                \textcolor{anscolor}{[Doug Sheehan is an American. (True); Doug Sheehan is an actor. (True); Doug Sheehan was born on April 27, 1949. (True); Doug Sheehan was born in Santa Monica, California. (False); He is best known for his roles in soap operas. (True); He portrayed Joe Kelly. (True); Joe Kelly was in General Hospital. (True); General Hospital is a soap opera. (True); He portrayed Ben Gibson. (True); Ben Gibson was in Knots Landing. (True); Knots Landing is a soap opera. (True)]}
                \newline
                [The end of the list of checked facts]
                \newline \newline
                [The start of the ideal output]
                \newline
                \textcolor{labelcolor}{[Doug Sheehan (True); is (True); an (True); American (True); actor (True); who (True); was born (True); on (True); April 27, 1949 (True); in (True); Santa Monica, California (False); . (True); He (True); is (True); best known (True); for (True); his roles in soap operas (True); , (True); including (True); in (True); his portrayal (True); of (True); Joe Kelly (True); on (True); ``General Hospital'' (True); and (True); Ben Gibson (True); on (True); ``Knots Landing.'' (True)]}
                \newline
                [The end of the ideal output]
				\newline \newline
				\textbf{User prompt}
				\newline
				\newline
				[The start of the biography]
				\newline
				\textcolor{magenta}{\texttt{\{BIOGRAPHY\}}}
				\newline
				[The ebd of the biography]
				\newline \newline
				[The start of the list of checked facts]
				\newline
				\textcolor{magenta}{\texttt{\{LIST OF CHECKED FACTS\}}}
				\newline
				[The end of the list of checked facts]
			};
			\node[chatbox_user_inner] (q1_text) at (q1) {
				\textbf{System prompt}
				\newline
				\newline
				You are a helpful and precise assistant for segmenting and labeling sentences. We would like to request your help on curating a dataset for entity-level hallucination detection.
				\newline \newline
                We will give you a machine generated biography and a list of checked facts about the biography. Each fact consists of a sentence and a label (True/False). Please do the following process. First, breaking down the biography into words. Second, by referring to the provided list of facts, merging some broken down words in the previous step to form meaningful entities. For example, ``strategic thinking'' should be one entity instead of two. Third, according to the labels in the list of facts, labeling each entity as True or False. Specifically, for facts that share a similar sentence structure (\eg, \textit{``He was born on Mach 9, 1941.''} (\texttt{True}) and \textit{``He was born in Ramos Mejia.''} (\texttt{False})), please first assign labels to entities that differ across atomic facts. For example, first labeling ``Mach 9, 1941'' (\texttt{True}) and ``Ramos Mejia'' (\texttt{False}) in the above case. For those entities that are the same across atomic facts (\eg, ``was born'') or are neutral (\eg, ``he,'' ``in,'' and ``on''), please label them as \texttt{True}. For the cases that there is no atomic fact that shares the same sentence structure, please identify the most informative entities in the sentence and label them with the same label as the atomic fact while treating the rest of the entities as \texttt{True}. In the end, output the entities and labels in the following format:
                \begin{itemize}[nosep]
                    \item Entity 1 (Label 1)
                    \item Entity 2 (Label 2)
                    \item ...
                    \item Entity N (Label N)
                \end{itemize}
                % \newline \newline
                Here are two examples:
                \newline\newline
                \textbf{[Example 1]}
                \newline
                [The start of the biography]
                \newline
                \textcolor{titlecolor}{Marianne McAndrew is an American actress and singer, born on November 21, 1942, in Cleveland, Ohio. She began her acting career in the late 1960s, appearing in various television shows and films.}
                \newline
                [The end of the biography]
                \newline \newline
                [The start of the list of checked facts]
                \newline
                \textcolor{anscolor}{[Marianne McAndrew is an American. (False); Marianne McAndrew is an actress. (True); Marianne McAndrew is a singer. (False); Marianne McAndrew was born on November 21, 1942. (False); Marianne McAndrew was born in Cleveland, Ohio. (False); She began her acting career in the late 1960s. (True); She has appeared in various television shows. (True); She has appeared in various films. (True)]}
                \newline
                [The end of the list of checked facts]
                \newline \newline
                [The start of the ideal output]
                \newline
                \textcolor{labelcolor}{[Marianne McAndrew (True); is (True); an (True); American (False); actress (True); and (True); singer (False); , (True); born (True); on (True); November 21, 1942 (False); , (True); in (True); Cleveland, Ohio (False); . (True); She (True); began (True); her (True); acting career (True); in (True); the late 1960s (True); , (True); appearing (True); in (True); various (True); television shows (True); and (True); films (True); . (True)]}
                \newline
                [The end of the ideal output]
				\newline \newline
                \textbf{[Example 2]}
                \newline
                [The start of the biography]
                \newline
                \textcolor{titlecolor}{Doug Sheehan is an American actor who was born on April 27, 1949, in Santa Monica, California. He is best known for his roles in soap operas, including his portrayal of Joe Kelly on ``General Hospital'' and Ben Gibson on ``Knots Landing.''}
                \newline
                [The end of the biography]
                \newline \newline
                [The start of the list of checked facts]
                \newline
                \textcolor{anscolor}{[Doug Sheehan is an American. (True); Doug Sheehan is an actor. (True); Doug Sheehan was born on April 27, 1949. (True); Doug Sheehan was born in Santa Monica, California. (False); He is best known for his roles in soap operas. (True); He portrayed Joe Kelly. (True); Joe Kelly was in General Hospital. (True); General Hospital is a soap opera. (True); He portrayed Ben Gibson. (True); Ben Gibson was in Knots Landing. (True); Knots Landing is a soap opera. (True)]}
                \newline
                [The end of the list of checked facts]
                \newline \newline
                [The start of the ideal output]
                \newline
                \textcolor{labelcolor}{[Doug Sheehan (True); is (True); an (True); American (True); actor (True); who (True); was born (True); on (True); April 27, 1949 (True); in (True); Santa Monica, California (False); . (True); He (True); is (True); best known (True); for (True); his roles in soap operas (True); , (True); including (True); in (True); his portrayal (True); of (True); Joe Kelly (True); on (True); ``General Hospital'' (True); and (True); Ben Gibson (True); on (True); ``Knots Landing.'' (True)]}
                \newline
                [The end of the ideal output]
				\newline \newline
				\textbf{User prompt}
				\newline
				\newline
				[The start of the biography]
				\newline
				\textcolor{magenta}{\texttt{\{BIOGRAPHY\}}}
				\newline
				[The ebd of the biography]
				\newline \newline
				[The start of the list of checked facts]
				\newline
				\textcolor{magenta}{\texttt{\{LIST OF CHECKED FACTS\}}}
				\newline
				[The end of the list of checked facts]
			};
		\end{tikzpicture}
        \caption{GPT-4o prompt for labeling hallucinated entities.}\label{tb:gpt-4-prompt}
	\end{center}
\vspace{-0cm}
\end{table*}
% \section{Full Experiment Results}
% \begin{table*}[th]
    \centering
    \small
    \caption{Classification Results}
    \begin{tabular}{lcccc}
        \toprule
        \textbf{Method} & \textbf{Accuracy} & \textbf{Precision} & \textbf{Recall} & \textbf{F1-score} \\
        \midrule
        \multicolumn{5}{c}{\textbf{Zero Shot}} \\
                Zero-shot E-eyes & 0.26 & 0.26 & 0.27 & 0.26 \\
        Zero-shot CARM & 0.24 & 0.24 & 0.24 & 0.24 \\
                Zero-shot SVM & 0.27 & 0.28 & 0.28 & 0.27 \\
        Zero-shot CNN & 0.23 & 0.24 & 0.23 & 0.23 \\
        Zero-shot RNN & 0.26 & 0.26 & 0.26 & 0.26 \\
DeepSeek-0shot & 0.54 & 0.61 & 0.54 & 0.52 \\
DeepSeek-0shot-COT & 0.33 & 0.24 & 0.33 & 0.23 \\
DeepSeek-0shot-Knowledge & 0.45 & 0.46 & 0.45 & 0.44 \\
Gemma2-0shot & 0.35 & 0.22 & 0.38 & 0.27 \\
Gemma2-0shot-COT & 0.36 & 0.22 & 0.36 & 0.27 \\
Gemma2-0shot-Knowledge & 0.32 & 0.18 & 0.34 & 0.20 \\
GPT-4o-mini-0shot & 0.48 & 0.53 & 0.48 & 0.41 \\
GPT-4o-mini-0shot-COT & 0.33 & 0.50 & 0.33 & 0.38 \\
GPT-4o-mini-0shot-Knowledge & 0.49 & 0.31 & 0.49 & 0.36 \\
GPT-4o-0shot & 0.62 & 0.62 & 0.47 & 0.42 \\
GPT-4o-0shot-COT & 0.29 & 0.45 & 0.29 & 0.21 \\
GPT-4o-0shot-Knowledge & 0.44 & 0.52 & 0.44 & 0.39 \\
LLaMA-0shot & 0.32 & 0.25 & 0.32 & 0.24 \\
LLaMA-0shot-COT & 0.12 & 0.25 & 0.12 & 0.09 \\
LLaMA-0shot-Knowledge & 0.32 & 0.25 & 0.32 & 0.28 \\
Mistral-0shot & 0.19 & 0.23 & 0.19 & 0.10 \\
Mistral-0shot-Knowledge & 0.21 & 0.40 & 0.21 & 0.11 \\
        \midrule
        \multicolumn{5}{c}{\textbf{4 Shot}} \\
GPT-4o-mini-4shot & 0.58 & 0.59 & 0.58 & 0.53 \\
GPT-4o-mini-4shot-COT & 0.57 & 0.53 & 0.57 & 0.50 \\
GPT-4o-mini-4shot-Knowledge & 0.56 & 0.51 & 0.56 & 0.47 \\
GPT-4o-4shot & 0.77 & 0.84 & 0.77 & 0.73 \\
GPT-4o-4shot-COT & 0.63 & 0.76 & 0.63 & 0.53 \\
GPT-4o-4shot-Knowledge & 0.72 & 0.82 & 0.71 & 0.66 \\
LLaMA-4shot & 0.29 & 0.24 & 0.29 & 0.21 \\
LLaMA-4shot-COT & 0.20 & 0.30 & 0.20 & 0.13 \\
LLaMA-4shot-Knowledge & 0.15 & 0.23 & 0.13 & 0.13 \\
Mistral-4shot & 0.02 & 0.02 & 0.02 & 0.02 \\
Mistral-4shot-Knowledge & 0.21 & 0.27 & 0.21 & 0.20 \\
        \midrule
        
        \multicolumn{5}{c}{\textbf{Suprevised}} \\
        SVM & 0.94 & 0.92 & 0.91 & 0.91 \\
        CNN & 0.98 & 0.98 & 0.97 & 0.97 \\
        RNN & 0.99 & 0.99 & 0.99 & 0.99 \\
        % \midrule
        % \multicolumn{5}{c}{\textbf{Conventional Wi-Fi-based Human Activity Recognition Systems}} \\
        E-eyes & 1.00 & 1.00 & 1.00 & 1.00 \\
        CARM & 0.98 & 0.98 & 0.98 & 0.98 \\
\midrule
 \multicolumn{5}{c}{\textbf{Vision Models}} \\
           Zero-shot SVM & 0.26 & 0.25 & 0.25 & 0.25 \\
        Zero-shot CNN & 0.26 & 0.25 & 0.26 & 0.26 \\
        Zero-shot RNN & 0.28 & 0.28 & 0.29 & 0.28 \\
        SVM & 0.99 & 0.99 & 0.99 & 0.99 \\
        CNN & 0.98 & 0.99 & 0.98 & 0.98 \\
        RNN & 0.98 & 0.99 & 0.98 & 0.98 \\
GPT-4o-mini-Vision & 0.84 & 0.85 & 0.84 & 0.84 \\
GPT-4o-mini-Vision-COT & 0.90 & 0.91 & 0.90 & 0.90 \\
GPT-4o-Vision & 0.74 & 0.82 & 0.74 & 0.73 \\
GPT-4o-Vision-COT & 0.70 & 0.83 & 0.70 & 0.68 \\
LLaMA-Vision & 0.20 & 0.23 & 0.20 & 0.09 \\
LLaMA-Vision-Knowledge & 0.22 & 0.05 & 0.22 & 0.08 \\

        \bottomrule
    \end{tabular}
    \label{full}
\end{table*}




\end{document}

\appendix 
\supptitle

% \section{Sequential Setup}
% \subsection{Previous Setup}
% \label{app:prev_setup}
% It aims to predict the next item based on a fixed snapshot of the user's past interactions. Formally, given the dataset $\mathfrak{D}_u$, we randomly pick an index $t$ where $1 \leq t \leq k$ and remove the corresponding item $i_u^t$ from the dataset. The modified dataset is $\mathfrak{D}'_u =  \{\mathbf{R}'_u, \mathbf{I}'_u \}$, where $\mathbf{R}'_u = \{r_u^1, \cdot\cdot\cdot, r_u^k \} \setminus \{r_u^t \}$, $\mathbf{I}'_u = \{i_u^1, \cdot\cdot\cdot, i_u^k \} \setminus \{ i_u^t\}$. The recommender is then tasked with recovering $i_u^t$ from the candidate set $\mathcal{C}_u^t$. 
% \subsection{Our Setup}
% \label{app:our_setup}
% It considers the temporal order of interactions by incrementally updating the available user history. Given the dataset $\mathfrak{D}_u$, the model sequentially observes each interaction and predicts each timestep $t$, where $1 \leq t \leq k$. Formally, at each step $t$, the recommender is provided with $\mathfrak{D}_u^t = \{\mathbf{R}_u^t, \mathbf{I}_u^t \}$, where $\mathbf{R}_u^t = \{r_u^1, \cdot\cdot\cdot, r_u^t\}$, $\mathbf{I}_u^t = \{i_u^1, \cdot\cdot\cdot, i_u^t \}$. The task is to predict the next item $i_u^{t+1}$ from the candidate set $\mathcal{C}_u^{t+1}$. This approach incorporates sequential dependencies by dynamically updating the recommendation model as new interactions occur, and it is more practical.


\setlength{\tabcolsep}{2pt}
\begin{table*}[t]
  \centering
  \scriptsize
  % \resizebox{1.0\linewidth}{!}{
  \begin{tabular}{@{}lll@{}}
    \toprule
    Method     &         Type         &     \multicolumn{1}{c}{Contents} \\ \cmidrule(lr){1-3}
    
    \multirow{3}{*}{\makecell[c]{Baselines}} 
    & \makecell[l]{\emph{Recommender} \\ \textbf{Input}}                                             & \makecell[l]{I've purchased the following products in chronological order: \{\textbf{user-item interactions \&\ reviews}\} 
    \\Then if I ask you to recommend a new product to me according to the given purchasing history, \\you should recommend \textbf{\{recent item\}} and now that I've just purchased \textbf{\{recent item\}}. 
    \\There are 20 candidate products that I can consider to purchase next: \textbf{\{20 candidate items\}}
    \\Please rank these 20 products by measuring the possibilities that I would like to purchase next most, \\according to the given purchasing records. Please think step by step. 
    \\Please show me your ranking results with order numbers. Split your output with line break. \\You MUST rank the given candidate product. You cannot generate products that are not in the given candidate list. \\No other description is needed. }\\ \cmidrule(lr){2-3}
    & \makecell[l]{\emph{Recommender} \\ \textbf{Output}}                                           & \makecell[l]{{[20 ordered items]}}\\ \cmidrule(lr){1-3}


    
   \multirow{52}{*}{\makecell[c]{\myalg{} \\ \textbf{(Ours)}}}          
    &  \makecell[l]{\emph{Review Extractor} \\ \textbf{Input}}     
    &        \makecell[l]{
    I purchased the following products in chronological order: \textbf{\{user-item interactions \&\ reviews\}} 
    \\Then if I ask you to recommend a new product to me according to the given purchasing history, you should recommend \textbf{\{recent item\}} \\ and now I've just purchased \textbf{\{recent item\}}.
    \\And I left review: \textbf{\{recent item review\}}
    \\ Your task is to analyze user's purchasing behavior and extract user's likes, dislikes and key features from the input review. 
    \\Response only likes/dislikes/key features in descriptive form. Please prioritize the most recent item \textbf{\{recent item\}} \\when analyzing likes/dislikes/key features.
    \\Split likes, dislikes, and key features and response in same format.}\\ \cmidrule(lr){2-3}
    &  \makecell[l]{\emph{Review Extractor} \\ \textbf{Output}}      
    &        \makecell[l]{\textbf{Likes}: \{[`*Long gameplay experience(50-60 hours), \colorbox{green}{*Responsive controls}, \colorbox{green}{*Fantastic storyline}, *Challenging puzzles, \\ *Emotional resonance (e.g.remorse), *Ability to gain new posers by killing enemies', `\colorbox{green}{*Humor and fun in games}, \\\colorbox{green}{*References to the simpsons franchise}, \colorbox{yellow}{*Variety of playable characters (Marge, Lisa, Apu, Bart, and Homer)}, \\ *Ability to drive or walk depending on preference, \colorbox{yellow}{*Great voice acting from the cast members}, \\\colorbox{yellow}{*Presence of key locations from the Simpsons universe (Kwik-E-Mart, Power Plant, Church, etc.)}, \\ *Cool vehicle designs and stats, \colorbox{green}{*Fantastic game overall}']\}\\ \textbf{Dislikes}: \{[`*No pause time when selecting a weapon, making the player vulnerable, \\\colorbox{yellow}{*Inventory management can be inconvenient, requiring the player to switch to the inventory screen to user gadgets}', \\`\colorbox{green}{*Boring story}, \colorbox{green}{*Not funny}, \colorbox{green}{*Awful weapons}, *Unresponsive controls, *Terrible graphics, *Worse gameplay']\} \\\textbf{Key Features}: \{[`\colorbox{green}{*No in-game loading}, *Fighting mechanics, *Soul-hunger gameplay mechanic, \\\colorbox{yellow}{*Ability to cover up face to hide disfigured jaw}', `\colorbox{yellow}{*New camera system (Devil May Cry position)}, *Redone fighting mechanics, \\ *Playable as both Raziel and Kain, \colorbox{yellow}{*Puzzles with a challenging but fun diffculty level}']\}}\\\cmidrule(lr){2-3}
    &  \makecell[l]{\emph{Profile Updater} \\ \textbf{Input}}      
    &        \makecell[l]{
    You are given a list: \textbf{\{list of likes/dislikes/key features\}}
    \\You have to update this list by removing redundant or overlapping information. Note that crucial information should be preserved.
    \\Please response only a list. No other description is needed.
    }\\ \cmidrule(lr){2-3}
    &  \makecell[l]{\emph{Profile Updater} \\ \textbf{Output}}       
    &        \makecell[l]{\textbf{Likes}: \{[`*Long Gameplay experience (50-60 hours), *Challenging puzzles, *Emotional resonance (e.g.remorse), \\ *Ability to gain new powers by killing enemies', `\colorbox{yellow}{*Variety of playable characters}, \\ *Ability to drive or walk depending on preference, \colorbox{yellow}{*Presence of key locations from the Simpsons universe}, \\\colorbox{yellow}{*Great voice acting}, *Cool vehicle designs and stats']\}\\ \textbf{Dislikes}: \{[`*No pause time when selecting a weapon, making the player vulnerable, \\\colorbox{yellow}{*Inventory management can be inconvenient}', `*Unresponsive controls, *Terrible graphics, *Worse gameplay']\} \\\textbf{Key Features}: \{[`*Fighting mechanics, *Soul-hunger gameplay mechanic, \colorbox{yellow}{*Ability to cover up face}', \\ `\colorbox{yellow}{*New camera system}, *Redone fighting mechanics, *Playable as both Raziel and Kain, \colorbox{yellow}{*Puzzles}']\}}\\ \cmidrule(lr){2-3}
    &  \makecell[l]{\emph{Recommender} \\ \textbf{Input}}                                    &         \makecell[l]{
    \textbf{This is positive aspects from purchase history}: \\\{[`*Long Gameplay experience (50-60 hours), *Challenging puzzles, *Emotional resonance (e.g.remorse), \\ *Ability to gain new powers by killing enemies', `*Variety of playable characters, \\ *Ability to drive or walk depending on preference, *Presence of key locations from the Simpsons universe, \\ *Great voice acting, *Cool vehicle designs and stats']\}
    \\\textbf{This is negative aspects from purchase history}:\\\{[`*No pause time when selecting a weapon, making the player vulnerable, \\ *Inventory management can be inconvenient', `*Unresponsive controls, *Terrible graphics, *Worse gameplay']\}
    \\\textbf{This is key features of products}: \{[`*Fighting mechanics, *Soul-hunger gameplay mechanic, *Ability to cover up face', \\ `*New camera system, *Redone fighting mechanics, *Playable as both Raziel and Kain, *Puzzles']\}
    \\Based on these inputs, your task is to rank 20 candidate products by evaluating their likelihood of being purchased.
    \\Now there are 20 candidate products that I consider to purchase next. Note that there is no specific order for these candidate items.
    \\Please rank the \textbf{\{20 candidate items\}} from 1 to 20. Your task is to rank these products based on the likelihood of purchase.
    \\You cannot generate products that are not in the given candidate list. No other description is needed.
    }\\  \cmidrule(lr){2-3}
    &  \makecell[l]{\emph{Recommender} \\ \textbf{Output}}          
    &        \makecell[l]{\{[20 ordered items]\}}\\                                                             
    \bottomrule
    \end{tabular}
% }
  \caption{\textbf{Qualitative Results: Baselines vs \myalg{}.} Note that \colorbox{green}{green-highlighted boxes} indicate portions removed due to redundancy or overlapping information, while \colorbox{yellow}{yellow-highlighted boxes} represent summarized content where unnecessary modifiers or examples were omitted for conciseness.}
  \label{tab:qual_results}
\end{table*}
\setlength{\tabcolsep}{6pt}


\section{Prompt Template}
\label{app:template}

\subsection{Extractor $\mathcal{E}$}
The extractor $\mathcal{E}$ aims to extract the user representations from reviews. Here is the prompt template. 

\begin{tcolorbox}[fonttitle=\small\bfseries,
fontupper=\scriptsize\sffamily,
fontlower=\fon{put},
enhanced,
left=2pt, right=2pt, top=2pt, bottom=2pt,
title=Prompt template for Extractor $\mathcal{E}$]
\begin{lstlisting}[]
I purchased the following products and left 
reviews in chronological order: {input_reviews}
Analyze user's likes/dislikes/key features by 
referring to their reviews.
\end{lstlisting}
\end{tcolorbox}



\subsection{Profile Updater $\mathcal{U}$}
The purpose of the profile updater $\mathcal{U}$ is to remove the redundant information in the user profile. As such, the prompt template is designed as below: 

\begin{tcolorbox}[fonttitle=\small\bfseries,
fontupper=\scriptsize\sffamily,
fontlower=\fon{put},
enhanced,
left=2pt, right=2pt, top=2pt, bottom=2pt,
title=Prompt template for User Profile Updater $\mathcal{U}$]
\begin{lstlisting}[]
You are given a list: {list}
Update this list by removing redundant or
overlapping information. Note that crucial 
information should be preserved.
\end{lstlisting}
\end{tcolorbox}


\subsection{Recommender $\mathcal{R}$}
Due to utilizing both item interactions and user profile, prompt can be constituted of various components. Below one is the prompt template of the recommender.  
\begin{tcolorbox}[fonttitle=\small\bfseries,
fontupper=\scriptsize\sffamily,
fontlower=\fon{put},
enhanced,
left=2pt, right=2pt, top=2pt, bottom=2pt,
title=Prompt template for Recommender $\mathcal{R}$]
\begin{lstlisting}[]
Positive aspects: {likes}
Negative aspects: {dislikes}
Key Features: {key_features}
Based on these inputs, rank the {candidate_list}
from 1 to 20 by evaluating their likelihood of 
being purchased.
\end{lstlisting}
\end{tcolorbox}

\section{Dataset}
\label{app:dataset}
Amazon Review Dataset~\cite{ni2019justifying} contains product reviews and metadata from Amazon, including 142.8 million reviews spanning May 1996 -- July 2014. Specifically, this dataset includes reviews (ratings, text, helpfulness votes), product metadata (descriptions, category information, price, brand, and image features), and links (also viewed/also bought graphs). Among them, we selected two domain datasets (Video Games and Movies \&\ TV), and we utilized ASIN, product name, rating, and review for each data and sort the reviews chronologically for each user. Here are the specific descriptions for each dataset.

\paragraph{Video Games.} We select about 15K users and 37K items. Following existing studies \cite{kang2018self}, we removed users and items with fewer than 10 interactions.

\paragraph{Movies and TV.} We select about 98K users and 126K items, removing users and items with fewer than 10 interactions as in the Video Games dataset.

\section{Baselines}
\label{app:baseline}

\subsection{User-Item interactions}
\label{app:interaction}
In our experimental setup, the LLM is tasked with predicting the item that a user is likely to purchase at time step $t$. We utilize \textbf{user-item interactions} up to time step ($t$-1) in chronological order and constructed a candidate list consisting of one ground-truth item and 19 non-interacted items as input. Here, time step $t$ refers to the period starting from the user's 4th purchase up to their final purchase $k$.
\paragraph{Sequential.}
We provide the LLM with instructions, supplying only the user-item interactions and the candidate list. The LLM was then tasked with ranking the items in the candidate list based on the likelihood of being purchased at time step $t$.
\paragraph{Recency-Focused.}
In the \emph{sequential} prompt above, we add an instruction to emphasize the most recently purchased item, specifically the item bought at time step ($t$-1). The additional prompt is as follows: \emph{"Note that my most recently purchased item is \{\textbf{recent item}\}."}
\paragraph{In-Context Learning.}
Unlike the previous \emph{sequential} and \emph{recency-focused} prompts, this approach utilize \textbf{user-item interactions} only up to time step ($t$-2) and recently purchased item which is bought at time step ($t$-1) as input. The additional prompt is as follows: \emph{"I've purchased the following products: \{\textbf{user-item interactions}\}, then you should recommend \{\textbf{recent item}\} to me and now that I've bought \{\textbf{recent item}\}."}

\subsection{User-Item interactions \&\ User Reviews}
\label{app:combined}
In this setup, we extend \textbf{user-item interactions} to include both interactions and \textbf{user reviews}. Based on ~\cref{app:interaction}, the $\dagger$ present the results when both \textbf{user-item interactions} and \textbf{user reviews} are used as input.



\section{Qualitative Results}
\label{app:qual}
To validate the effectiveness of each component of \myalg{}, we summarized the qualitative results in~\autoref{tab:qual_results}, which illustrates the entire input/output process for both the baselines and \myalg{} in the sequential recommendation task. We can observe that the Review Extractor first removes irrelevant or uninformative content for the given reviews, while the Profile Updater reduces redundancy and overlapping information in the user profile. As such, we can conclude that \myalg{} reduces the input token size of the recommender system while retaining essential information, making it more memory-efficient and potentially improving overall performance.

% In the \myalg{} section, green-highlighted boxes indicate portions removed due to redundancy or overlapping information, while yellow-highlighted boxes represent summarized content where unnecessary modifiers or examples were omitted for conciseness. In this way, it reduces the token count while retaining crucial information, making it more memory-efficient while maintaining or even improving performance.


% For the baselines, the input consists of user-item interactions and raw unprocessed reviews, which are directly fed into the recommender to generate a list of 20 recommended items. In contrast, \myalg{} starts with the same input as the baseline but does not immediately generate the candidate list. Instead, it extracts the user’s likes, dislikes, and key features from the input. The updater module then refines this information by removing redundant or conflicting content and summarizing it. Finally, the updated likes, dislikes, and key features, along with the candidate list, are used to produce a list of 20 recommended items.




% \section{Example Appendix}
% \label{sec:appendix}

% This is an appendix.
\end{document}
