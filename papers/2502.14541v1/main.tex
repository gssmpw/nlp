  % This must be in the first 5 lines to tell arXiv to use pdfLaTeX, which is strongly recommended.
\pdfoutput=1
% In particular, the hyperref package requires pdfLaTeX in order to break URLs across lines.

\documentclass[11pt]{article}

\newcommand\blfootnote[1]{%
  \begingroup
  \renewcommand\thefootnote{}\footnote{#1}%
  \addtocounter{footnote}{-1}%
  \endgroup
}


% Remove the "review" option to generate the final version.
\usepackage[]{acl}

% Standard package includes


\usepackage{times}
\usepackage{latexsym}
\usepackage{comment}
\usepackage{color}
\usepackage{makecell}
%\usepackage{xcolor}
\usepackage{colortbl}
% For proper rendering and hyphenation of words containing Latin characters (including in bib files)
\usepackage[T1]{fontenc}
% For Vietnamese characters
% \usepackage[T5]{fontenc}
% See https://www.latex-project.org/help/documentation/encguide.pdf for other character sets

% This assumes your files are encoded as UTF8
\usepackage[utf8]{inputenc}

% This is not strictly necessary, and may be commented out,
% but it will improve the layout of the manuscript,
% and will typically save some space.
\usepackage{microtype}
\usepackage[most]{tcolorbox}
% This is also not strictly necessary, and may be commented out.
% However, it will improve the aesthetics of text in
% the typewriter font.
\usepackage{inconsolata}

\usepackage{mathtools}
\usepackage{makecell}



\usepackage{array,multirow,graphicx}
\usepackage{amsmath}

\usepackage{amssymb}
\usepackage{booktabs}
% \usepackage{algorithm}
\usepackage{algpseudocode}

\usepackage[export]{adjustbox}
\usepackage{float}

% Support for easy cross-referencing
%\newcommand\algorithmautorefname{Algorithm}
\usepackage[capitalize]{cleveref}
\crefname{section}{Sec.}{Secs.}
%\Crefname{section}{Section}{Secs.}
\Crefname{table}{Table}{Tables}
\crefname{table}{Tab.}{Tabs.}

\def\subsectionautorefname{Sec.}
\def\sectionautorefname{Sec.}

% \renewcommand*{\algorithmautorefname}{Algorithm}
\renewcommand*{\figureautorefname}{Fig.}
\renewcommand*{\equationautorefname}{Eq.}
\renewcommand*{\tableautorefname}{Tab.}


% \renewcommand\jt[1]{\textcolor{black}{#1}}


\newcommand{\llama}{LLaMA}
\newcommand{\myalg}{\code{PURE}}
\newcommand{\tr}{\textrm{tr}}
\newcommand{\per}{\textrm{c}}
\newcommand{\pool}{pool}
\newcommand{\mytitle}{LLM-based User Profile Management for Recommender System}
% \newcommand{\mytitle}{MemRec: LLM-based Recommender System using Memory Update}
% \newcommand{\mytitle}{FRAG: Feedback based RAG Enhancement without Training}

\input{commands/mathcommand}
% \usepackage{microtype}
\usepackage{geometry}
% \usepackage{subfig}
\usepackage{booktabs} 
\usepackage{bbm}
\usepackage{mathtools}
% \usepackage{amsthm}
\usepackage{nccmath}
\usepackage{setspace}

\usepackage{caption}
\usepackage{subcaption}

\usepackage[linesnumbered,ruled,vlined]{algorithm2e}
% \usepackage{algorithmic}
% \usepackage{algorithm}

\SetKwInput{KwInput}{Input}                % Set the Input
\SetKwInput{KwOutput}{Output}              % set the Output
\newcommand\mycommfont[1]{\footnotesize\ttfamily\textcolor{blue}{#1}}
\SetCommentSty{mycommfont}
\newcommand{\algcapsty}[1]{\small\sffamily\bfseries{#1}}
\SetAlCapSty{algcapsty}

\usepackage[T1]{fontenc}
\usepackage{wrapfig,lipsum,booktabs}

% \usepackage{natbib}
\usepackage{soul}
\usepackage{dsfont}
\usepackage{enumerate}
\usepackage{enumitem}

% \usepackage{kotex}
% \usepackage{hyperref}
% \usepackage[hidelinks]{hyperref}
\usepackage{amsmath}
% \usepackage{amsthm}
\usepackage{amsfonts}
\usepackage{bbm}
\usepackage{dsfont}
\usepackage[Symbol]{upgreek}
\usepackage{lscape}
\usepackage{caption}
\usepackage{balance}
\usepackage{xspace}
\usepackage{float}
\usepackage{kotex}

\usepackage{wasysym}
%\usepackage[table,xcdraw,dvipsnames]{xcolor}
\usepackage{xcolor}
\usepackage{multirow}
\usepackage{array, boldline, rotating}

\usepackage{amssymb}% http://ctan.org/pkg/amssymb
\usepackage{pifont}% http://ctan.org/pkg/pifont
\newcommand{\cmark}{\ding{51}\xspace}%
\newcommand{\omark}{\textbf{$\mathcal{O}$}\xspace}%
\newcommand{\xmark}{\ding{55}\xspace}%

\newcommand{\ds}[1]{\mathds{#1}}
\newcommand{\mc}[1]{\mathcal{#1}}
\newcommand{\bb}[1]{\mathbbm{#1}}

% %%%%%% Theorem Related Things %%%%%%
% \theoremstyle{plain}
% \newtheorem{thm}{Theorem}
% \newtheorem{cor}{Corollary}
% \newtheorem{lem}{Lemma}
% \newtheorem{prop}{Proposition}

% \theoremstyle{definition}
% \newtheorem{defn}{Definition}
% \newtheorem{assum}{Assumption}



% Citation

% \let\oldeqcite\cite
% \renewcommand*\cite[1]{(\oldcite{#1})}
\let\oldeqref\eqref
\renewcommand*\eqref[1]{(\ref{#1})}

% % Highlight (incl. note)
\newcommand{\smnote}[1]{\textbf{\textcolor{Cyan}{SM: #1}}}
\newcommand{\jhnote}[1]{\textbf{\textcolor{Orange}{JH: #1}}}
\newcommand{\yes}[1]{\textcolor{blue}{[YES]}}
\newcommand{\no}[1]{\textcolor{orange}{[NO]}}
\newcommand{\na}[1]{\textcolor{gray}{[N/A]}}
%\newcommand\bg[1]{\textcolor{blue}{#1}} % JH
\newcommand\jt[1]{\textcolor{brown}{#1}} % JT
\newcommand\jh[1]{\textcolor{black}{#1}} % JH
\newcommand\sm[1]{\textcolor{blue}{#1}} % SM
\newcommand{\eg}{\emph{e.g.,~}}
\newcommand{\ie}{\emph{i.e.,~}}


% \renewcommand\jt[1]{\textcolor{black}{#1}} % JT
% \renewcommand\jh[1]{\textcolor{black}{#1}} % JH

% % Separation (paragraph)
\newcommand{\myparagraph}[1]{\vspace{0.07cm}\noindent\textbf{#1}~}

% % math op.

% % Font
\def\code#1{\texttt{#1}}
\DeclarePairedDelimiter\norm{\lVert}{\rVert}

% % % Definition
% \theoremstyle{definition}
\newcommand\scalemath[2]{\scalebox{#1}{\mbox{\ensuremath{\displaystyle #2}}}}



\newcommand{\thickhline}{\hlineB{4}}
\newcommand{\bfcode}[1]{\code{\textbf{#1}}}


\definecolor{LightCyan}{rgb}{0.88,1,1}
\definecolor{Blue}{rgb}{0, 0.5, 1}
\definecolor{Green}{rgb}{0.0, 0.8, 0.0 }
\definecolor{Red}{rgb}{0.95, 0.55, 0.6}
\definecolor{Skyblue}{rgb}{0.6, 0.6, 0.95 }



% Supplementary title
\NewDocumentEnvironment{suptitle}{ +b }{
    \twocolumn[{#1}]%
}{}

\NewDocumentCommand{\supptitle}{s}{
\begin{suptitle}
        \centering
        % \rule{\textwidth}{0.07cm}\\[-0.34cm]
        \rule{\textwidth}{0.03cm}\\[0.1cm]
        - Appendix -\\[0.2cm]
        {\Large 
            \textbf{\mytitle }
        }\\%[0.40cm]
        \rule{\textwidth}{0.03cm}\\[0.2cm]
\end{suptitle}}
% If the title and author information does not fit in the area allocated, uncomment the following
%
%\setlength\titlebox{<dim>}
%
% and set <dim> to something 5cm or larger.

%\title{\alg: Instant Personalized LoRA Generation for On-device and Hybrid Decoding}
% \title{OPA: On-the-Fly Personalized Adapter \& Device-Server Consistent Inference for On-Device LLM}
%\title{On-Palette: On-the-fly Person-Adapted On-device LLM and Edge-to-server Transfer for Hybrid Inference}
% \title{On-Device Palette: Personalized On-the-Fly Adapter and Edge-Server Hybrid Inference}
\title{\mytitle}
% \title{Palette: Person-Aaptation of LLM On-the-Fly and Edge-Server Transit for }


%\title{Palette: Personalized Adapter for LLM Edge device T... Toward End device.}
% PersonAr,
% Personalized On-the-Fly Adapter and Device-Server Hybrid Inference for On-Device LLM.  


% Author information can be set in various styles:
% For several authors from the same institution:
% \author{Author 1 \and ... \and Author n \\
%         Address line \\ ... \\ Address line}
% if the names do not fit well on one line use
%         Author 1 \\ {\bf Author 2} \\ ... \\ {\bf Author n} \\
% For authors from different institutions:
% \author{Author 1 \\ Address line \\  ... \\ Address line
%         \And  ... \And
%         Author n \\ Address line \\ ... \\ Address line}
% To start a separate ``row'' of authors use \AND, as in
% \author{Author 1 \\ Address line \\  ... \\ Address line
%         \AND
%         Author 2 \\ Address line \\ ... \\ Address line \And
%         Author 3 \\ Address line \\ ... \\ Address line}

\author{%
    \textbf{Seunghwan Bang}$^{1}$ \hspace{1em} \textbf{Hwanjun Song}$^{2, \dagger}$\\
    $^{1}$Ulsan National Institute of Science and Technology \\
    $^{2}$Korea Advanced Institute of Science and Technology \\
    \texttt{shbang1422@unist.ac.kr} \\ \texttt{songhwanjun@kaist.ac.kr}
}

\begin{document}
\maketitle
\blfootnote{$\dagger$ indicates the corresponding author.}
\begin{abstract}
% The rapid advancement of Large Language Models (LLMs) has enabled zero-shot recommendation without conventional training. However, most approaches rely solely on purchase histories, overlooking the rich information in user-generated texts. We propose \myalg{}, an LLM-based framework that builds evolving user profiles by extracting and summarizing key information from reviews. \myalg{} combines \emph{Review Extractor}, \emph{Profile Updater}, and \emph{Recommender} to deliver personalized suggestions while managing token limitations. To evaluate \myalg{}, we introduce a novel continuous sequential recommendation task, which more realistically reflects real-world scenarios by updating predictions incrementally as new reviews arrive. Experiments on Amazon data show that \myalg{} significantly improves performance over existing LLM-based methods.
The rapid advancement of Large Language Models (LLMs) has opened new opportunities in recommender systems by enabling zero-shot recommendation without conventional training. Despite their potential, most existing works rely solely on users' purchase histories, leaving significant room for improvement by incorporating user-generated textual data, such as reviews and product descriptions. Addressing this gap, we propose \myalg{}, a novel LLM-based recommendation framework that builds and maintains evolving user profiles by systematically extracting and summarizing key information from user reviews. \myalg{} consists of three core components: a Review Extractor for identifying user preferences and key product features, a Profile Updater for refining and updating user profiles, and a Recommender for generating personalized recommendations using the most current profile. To evaluate \myalg{}, we introduce a continuous sequential recommendation task that reflects real-world scenarios by adding reviews over time and updating predictions incrementally. Our experimental results on Amazon datasets demonstrate that \myalg{} outperforms existing LLM-based methods, effectively leveraging long-term user information while managing token limitations.
% The rise of Large Language Models (LLMs) has significantly advanced the field of recommender systems, enabling more sophisticated and personalized predictions. However, most existing LLM-based recommender systems rely primarily on users' purchase history, overlooking the valuable insights embedded in user-generated reviews. In this paper, we propose \myalg{}, a novel framework that integrates user reviews into the recommendation process, addressing the challenges posed by LLMs' token limitations. Our approach systematically extracts and stores key information from user reviews, ensuring scalability as user-generated data grows, while retaining only the most relevant content for accurate and efficient recommendations. We further introduce a realistic sequential recommendation setup, where reviews are incrementally added over time, enabling the model to dynamically update user profiles and predict the next purchase based on evolving user preferences. Our experimental results demonstrate that \myalg{} outperforms existing LLM-based recommendation methods that are train-free on Amazon datasets, validating the effectiveness of our approach in providing more accurate and scalable recommendations. 

\end{abstract}
%\jt{}
% \blfootnote{$\dagger$ indicates the corresponding author.}

\section{Introduction}
\label{sec:intro}


\ps{Challenges of technology scaling}

The growing demand for computing performance has always been met by increasing the number of transistors per chip, which is only possible due to CMOS technology scaling.
However, as we keep pushing the boundaries of technology scaling, we encounter multiple challenges.
Firstly, whenever we transition to a more advanced technology node, the non-recurring cost due to physical design, verification, software, mask sets, and prototyping almost doubles \cite{cost-tech-node}.
As a result, designing a chip in an advanced technology node is only economically viable if the chip is manufactured in vast quantities.
Secondly, many chip components such as I/O drivers, analog circuits, or \gls{srams} have reached their scaling limit.
This means that we cannot shrink these components further, even if we use a more advanced technology with a smaller feature size.
Thirdly, advanced technology nodes suffer from high defect rates, diminishing the yield and inflating the recurring cost.
To tackle these challenges, new chip-design paradigms have been developed.

\ps{Why 2.5D integration?}

One of these new paradigms is 2.5D integration, where multiple silicon dies called chiplets are integrated into the same package.
Once designed, a single chiplet can be reused in multiple 2.5D stacked chips, which increases the ratio of production volume to non-recurring cost.
Another advantage is that multiple chiplets - fabricated in different technologies - can be integrated into the same package.
This means that only components that can take full advantage of technology scaling are built in bleeding-edge technologies.
Components that have reached their scaling limit are fabricated in more mature and hence less costly technology nodes.
Furthermore, chiplets are smaller than monolithic chips.
Therefore, manufacturing chiplets results in less silicon area loss due to fabrication defects and hence a higher yield.
Due to these economic advantages, chip vendors such as AMD \cite{amd-chiplet} and NVIDIA \cite{chiplet-book} have adopted the 2.5D integration paradigm.  

\ps{Challenges of 2.5D integration}

An important challenge when designing 2.5D stacked chips is the construction of a low-latency and high-throughput \gls{ici}. 
To build an \gls{ici}, we connect different chiplets using \gls{d2d} links.
These links are fabricated in an organic package substrate, silicon bridge, or silicon interposer, and they are connected to the chiplets using \gls{c4} bumps or microbumps.
The number of bumps per chiplet is limited, and so is the bandwidth of \gls{d2d} links.
In addition to having lower bandwidth than links in monolithic chips, \gls{d2d} links also have higher latency.
This latency is caused by wire delay and by \gls{phys} that are necessary in both the sending and the receiving chiplet.
\gls{phys} are needed to convert between protocols, voltage levels, and frequencies, which are usually different between on-chiplet links and \gls{d2d} links.
Due to these limitations, the \gls{ici} can quickly become a bottleneck.

\ps{How we solve these challenges differently than the related work does.}

Existing approaches to maximize the performance of the \gls{ici} either optimize the placement of chiplets (with potentially heterogeneous shapes) for a predetermined \gls{ici} topology 
\cite{ho,liu,seemuth,eris,osmolovskyi,tap25d,chiou}, select one topology out of a set of candidates \cite{coskun-1, coskun-2}, or they optimize the \gls{ici} topology for a 2D grid of homogeneously shaped chiplets on an active interposer \cite{butterdonut, cluscross, kite}.
To the best of our knowledge, there is no prior work on \gls{ici} topologies for chips with heterogeneously shaped chiplets or with passive silicon interposers or silicon bridges.
To fill this gap, we propose \name, a novel optimization methodology to jointly optimize the chiplet placement and \gls{ici} topology of such architectures.
\ifnb
\else
\newpage
\fi

\ps{Details on \name~and the key idea}

The key idea is as follows: 
We optimize the chiplet placement without a predetermined topology.
For each placement generated by an optimization algorithm, we infer a placement-based \gls{ici} topology by connecting chiplets that are in close proximity in that specific placement.
We then compute the latency and throughput of this combination of placement and topology for different traffic types.
These latencies and throughputs together with the total chip area are used to compute a user-defined quality-score of the placement, which is returned to the optimization algorithm.
Based on this quality score, the algorithm can further optimize the placement.
By following this iterative process, we jointly optimize the chiplet placement and the \gls{ici} topology.

\ps{Short evaluation-summary}

We provide our open-source framework implementing the proposed placement and topology co-optimization methodology, which we evaluate using both synthetic traffic and traffic traces.
A 2D grid of chiplets with a mesh topology is used as a baseline since many proposals for 2.5D stacked chips \cite{dataflow_accel_dnn, cifher, simba, hecaton, dojo} use such an architecture.
We reduce the latency of synthetic L1-to-L2 and L2-to-memory traffic, the two most important traffic types for cache coherency traffic, by up to 28\% and 62\% respectively.
For real traffic traces, we reduce the average packet latency for almost all traces and architectures considered (reduced by an 8\% or 18\% on average depending on the configuration of \gls{phys} within a chiplet).


\section{Related work}


Recent advances in single-image animatable head avatar generation can be categorized into mainly 2D-based and 3D-based approaches. 

\paragraph{\bf Image to 2D Animatable Avatar.}
2D-based methods, leveraging the power of convolutional neural networks (CNNs)~\cite{DBLP:conf/cvpr/KarrasLAHLA20,DBLP:conf/cvpr/IsolaZZE17,DBLP:conf/nips/GoodfellowPMXWOCB14}, often employ generative adversarial networks (GANs)~\cite{DBLP:conf/cvpr/StyleGAN} for direct image synthesis. Early approaches~\cite{DBLP:conf/cvpr/WangDYSW23,DBLP:conf/cvpr/BurkovPGL20,DBLP:conf/iccv/ZakharovSBL19} focus on injecting expression and pose features into the generator network, often utilizing architectures like U-Net or StyleGAN~\cite{DBLP:conf/cvpr/StyleGAN}.
Some other 2D methods~\cite{DBLP:journals/corr/abs-2407-03168,DBLP:conf/cvpr/ZhangQZZW0CW023,DBLP:conf/cvpr/HongZS022,DBLP:conf/mm/DrobyshevCKILZ22,DBLP:conf/cvpr/BurkovPGL20,DBLP:conf/nips/SiarohinLT0S19} represent expressions and poses as warping fields applied to the source image. 
Benefiting from advances in image and video diffusion networks, more recent 2D-based works~\cite{DBLP:journals/corr/abs-2410-07718,DBLP:journals/corr/abs-2406-08801,DBLP:conf/eccv/TianWZB24} get improved results with diffusion techniques. 
However, these methods still face challenges related to long generation times and significant computational resource demands. Audio-driven 2D control methods~\cite{DBLP:conf/cvpr/ZhangCWZSGSW23,DBLP:journals/corr/abs-2211-12368,DBLP:conf/iccv/GuoCLLBZ21} are easy to use but cannot explicitly control facial expressions and poses. 2D-based techniques often struggle with large pose or expression variations due to the lack of an explicit 3D structure, sometimes producing unrealistic distortions or identity changes. While some 2D methods~\cite{SadTalker,StyleHEAT,Pirenderer,DBLP:conf/cvpr/WangM021,MegaPortraits} incorporate 3D Morphable Models (3DMMs)~\cite{DBLP:conf/fgr/GerigMBELSV18,DBLP:journals/tog/LiBBL017,DBLP:conf/avss/PaysanKARV09,DBLP:conf/siggraph/BlanzV99} to mitigate these issues, they typically cannot achieve free-viewpoint rendering. 

\vspace{-0.1in}

\begin{figure*}[h]
    \centering
    \includegraphics[width=0.9\linewidth]{images/framework.pdf}
    \caption{\textbf{Overall Framework.} Our framework utilizes learnable query features attached to FLAME vertices to perform cross-attention with the extracted multi-level image features. The extracted features are then decoded to reconstruct the Gaussian avatar in the canonical space, which can be animated utilizing standard linear blend skinning (LBS) and corrective blendshapes as the FLAME model did and rendered in real-time on various platforms.}
    \label{fig:framework}
\end{figure*}

\paragraph{\bf Image to 3D Animatable Avatar.}
3D-aware methods offer improved geometric consistency and free-viewpoint rendering capabilities. Early 3D approaches~\cite{DBLP:conf/eccv/KhakhulinSLZ22,DBLP:conf/cvpr/XuYCWDJT20} utilize 3DMMs for head avatar reconstruction. With the advent of Neural Radiance Fields (NeRFs)~\cite{DBLP:conf/eccv/MildenhallSTBRN20}, many recent methods~\cite{DBLP:conf/siggraph/YuFZWYBCSWSW23,DBLP:conf/cvpr/MaZQLZ23,DBLP:conf/cvpr/LiZWZ0CZWB023,GPAvatar,ye2024real3d,deng2024portrait4d,deng2024portrait4d2,DBLP:conf/eccv/KiMC24,DBLP:conf/cvpr/BaiFWZSYS23,PointAvatar,Nerfies,INSTA} have adopted this representation for higher fidelity, particularly in modeling fine details like hair. However, NeRF-based~\cite{DBLP:conf/cvpr/ZhangZLHLWGCL024,HAvatar,DBLP:conf/cvpr/BaiTHSTQMDDOPTB23,AD-NeRF,DBLP:journals/tog/GaoZXHGZ22,DBLP:journals/tog/ParkSHBBGMS21,DBLP:conf/cvpr/AtharXSSS22,DBLP:journals/corr/abs-2112-05637,DBLP:conf/iccv/TretschkTGZLT21,DBLP:conf/cvpr/GafniTZN21,DBLP:conf/eccv/KiMC24,DBLP:conf/cvpr/BaiFWZSYS23,PointAvatar,Nerfies,DBLP:conf/siggraph/YuFZWYBCSWSW23,DBLP:conf/cvpr/MaZQLZ23,DBLP:conf/cvpr/LiZWZ0CZWB023} approaches often require extensive training data, including multi-view or single-view videos, raising privacy concerns and limiting generalization to unseen identities. Some methods~\cite{DBLP:conf/cvpr/SunWWLZZL23,DBLP:conf/3dim/ZhuangMKS22,DBLP:journals/pami/SunWZHWL24,DBLP:journals/tvcg/TangZYZCMW24,DBLP:conf/iclr/XuZLZBFS23} bypass this data requirement by training generators with random noise and then inverting them for identity-specific reconstruction, but inversion accuracy remains a challenge. Test-time optimization offers another alternative, but its computational cost limits practical applications. Several recent works~\cite{goha2023,hidenerf2023,gpavatar2024,ye2024real3d,ma2024cvthead,deng2024portrait4d,deng2024portrait4d2,GGHead} have explored one-shot 3D head reconstruction to address the limitations of data requirements and computational cost. These methods employ various techniques, such as tri-plane features, deformation fields, point-based expression fields, and vertex-feature transformers. Despite these advancements, NeRF-based methods often struggle with real-time rendering. 
Recently, 3D Gaussian Splatting~\cite{GaussianSplatting} has emerged as a promising alternative, offering both high-quality results and fast rendering speeds. However, existing Gaussian Splatting methods~\cite{GaussianAvatar,DBLP:conf/cvpr/XuCL00ZL24} typically rely on video data for training for each person, limiting their ability to generalize to new identities. Instead, the most recent work, GAGAvatar~\cite{GAGAvatar}, proposes a one-shot 3D Gaussian-based head avatar generation method. However, it still relies heavily on complex 2D neural post-processing to achieve optimal animation outcomes, thus it is not a pure 3D solution and the extra neural network hinders its application on various platforms. In contrast, our work generates Gaussian heads that are immediately animatable and renderable without additional networks or post-processing steps, enabling seamless integration into existing rendering pipelines for real-time animation and rendering across a wide range of platforms, including mobile phones. 
\label{sec:method}

In this section, we introduce the method used to conduct the investigation on a set of \pc papers that discuss relevant bias issues.
Specifically, to construct the initial set of relevant work, we search the keywords ``bias" or ``fair" in the title of papers from NeurIPS, ICML, ICLR and FAccT published before February 2025. 
We include papers that discuss bias issues whose manifestation aligns with either Type I Bias or Type II Bias (we will detail the unification in~\cref{sec:unifying}).
We exclude papers that address other bias issues such as inductive bias~\cite{baxter2000model,zietlow2021demystifying}, implicit bias~\cite{fitzgerald2017implicit,camuto2021asymmetric}, selection bias~\cite{hernan2004structural,akbari2021recursive}, sampling bias~\cite{winship1992models,xu2022alleviating}, spectral bias~\cite{fang2024addressing}, exposure bias~\cite{li2024alleviating} or bias-variance~\cite{ha2024fine, chen2024on}.
Furthermore, to ensure we do not overlook any relevant papers without these keywords or from other prominent conferences such as CVPR, ICCV, and ECCV, we manually traversal the citation graph of the paper in the initial set and append the relevant papers that are either cited by or cite the papers in the initial set.






Once we identify the scope of the investigated papers, we read these papers to determine which type of bias they address by examining two aspects: problem statement and evaluation protocol.
We will elaborate on the criterion for categorizing papers into our definitions in~\cref{sec:unifying}.
To accommodate the recent emerging direction of addressing unlabeled and unknown bias, we enrich the taxonomy with an additional dimension about the status of attribute $A$.
As shown in~\cref{tab:taxonomy}, we count the number of papers in each category. 
Note that the total number is not equal to \pc since some papers address both types of biases.
We present the categorization list of all \pc investigated papers in Appendix.


\begin{table}[htbp]
\caption{The taxonomy of bias issues based on \pc papers.}
\label{tab:taxonomy}
\centering
\resizebox{0.45\textwidth}{!}{%

\begin{tabular}{lcccc}
\toprule
\multirow{2}{*}{Type of Bias} & \multicolumn{2}{c}{Attribute $A$} & \multirow{2}{*}{Papers} & \multirow{2}{*}{Examples}                                                   \\
\cmidrule(lr){2-3} 
                              & Known           & Labeled         &                         &                                                                             \\
                              \midrule
\multirow{3}{*}{Type I Bias}  & \cmark          & \cmark          & 253                     & \cite{DebFace,GAC,RL_RBN}                                                   \\
                              & \cmark          & \xmark          & -                       & -                                                                           \\
                              & \xmark          & \xmark          & -                       & -                                                                           \\
                              \midrule
\multirow{3}{*}{Type II Bias} & \cmark          & \cmark          & 246                     & \cite{learn_not_to_learn_Colored_MNIST,CSAD,End}                            \\
                              & \cmark          & \xmark          & 8                       & \cite{HEX_texture_bias1, ReBias_texture_bias2,rubi} \\
                              & \xmark          & \xmark          & 30                      & \cite{LfF_CelebA_Bias_conflicting,ECS,UBNet}                               \\
                              \midrule
Survey                        & -               & -               & 25                       & \cite{MLbias_survey,prediciton_quality_disparity,discussion_on_DP_EO}      \\
\bottomrule
\end{tabular}
}

\end{table}


\setlength{\tabcolsep}{11pt}
\begin{table*}[t]
  \centering
  \scriptsize
  % \begin{adjustbox}{width=0.98\linewidth}
  \begin{tabular}{@{}clcccccccc@{}}
    \toprule
     &  &  \multicolumn{4}{c}{Games} &  \multicolumn{4}{c}{Movies} \\\cmidrule(lr){3-6} \cmidrule(lr){7-10} 
     Data & Method          &    N@1   &    N@5    &    N@10    &   N@20    &   N@1   &    N@5    &    N@10     &   N@20\\ \cmidrule(lr){1-10}
     
     \multirow{3}{*}{\rotatebox{90}{items}} & Sequential & 10.75 & 18.25 & 23.13 & 28.97 & 9.99  & 15.92 & 20.17 & 26.94 \\
     & Recency    & 15.34 & 24.31 & 28.82 & 34.24 & 12.17 & 17.75 & 22.18 & 28.19 \\
     & ICL        & 14.28 & 26.57 & 30.51 & 35.72 & 12.03 & 19.56 & 23.36 & 29.91 \\ \cmidrule(lr){1-10}

     \multirow{6}{*}{\rotatebox{90}{items + reviews}} 
     & Sequential$^\dagger$ & 11.14 & 19.95 & 24.97 & 32.00 & 8.05 & 13.11 & 17.72 & 25.57\\ 
     & Recency$^\dagger$    & 12.19 & 23.64 & 28.37 & 35.35 & 8.54 & 15.78 & 21.31 & 29.21\\
     & ICL$^\dagger$        & 15.11 & 26.34 & 31.25 & 37.39 & 12.24& 22.10 & 27.31 & 34.52\\ \cmidrule(lr){2-10}
     &\myalg{} (Sequential)& 15.06          & 25.71          & 31.08          &       38.28          & 12.59          & 21.33          & 25.96          &           32.21  \\ 
     &\myalg{} (Recency)   & \textbf{18.18} & 28.90          & 33.91          &     40.69          & 13.85          & 21.99          & 26.53          &               33.37  \\
     &\myalg{} (ICL)       & 16.62          & \textbf{29.81} & \textbf{35.60} & \textbf{42.00} & \textbf{15.80} & \textbf{26.32} & \textbf{32.03} & \textbf{38.93}  \\
    % $\mathbf{R_u}$                                                                    & Truncate     & 14.79 & 24.94 & 30.30 & 37.79 & 13.35 & 22.87 & 28.02 & 35.60\\
    % & Concate      &       &       &       &       &       &       &       &       \\ \cmidrule(lr){1-10}
    
    % $\mathbf{I_u} \cup \mathbf{R_u}$                             & Truncate   & 16.13 & 27.67 & 32.55 & 38.64 & 13.24 & 22.65 & 27.85 & 35.01 \\
    % & \textbf{\myalg{} (Ours)} & \textbf{16.62} & \textbf{29.81} & \textbf{35.60} & \textbf{42.00} & \textbf{15.80} & \textbf{26.32} & \textbf{32.03} & \textbf{38.93} \\
    \bottomrule
  \end{tabular}
  % \end{adjustbox}
  \vspace{-0.25cm}
  \caption{\textbf{Comparison \myalg{} with Baselines.} We evaluate performance under two data settings: using only item interactions and using item interactions augmented with reviews. $\dagger$ indicates customized baselines where review data is naively incorporated into the original prompt templates designed for item interactions only (see ~\cref{app:combined}).}
  % \caption{\textbf{Main results of \myalg{}.} We customize the prompt from the baselines by utilizing both interactions and reviews, and we notate $\dagger$ (See~\cref{app:combined}).}
  
  \label{table:main}
  \vspace{-0.2cm}
\end{table*}

\setlength{\tabcolsep}{6pt}

\setlength{\tabcolsep}{5pt}
\begin{table*}[t]
\centering
  \scriptsize
  % \begin{adjustbox}{width=0.98\linewidth}
    \begin{tabular}{@{}cccccccccccccccc@{}}
    \toprule
         & \multicolumn{2}{c}{Data} & \multicolumn{3}{c}{Components} & \multicolumn{5}{c}{Games} &  \multicolumn{5}{c}{Movies} \\\cmidrule(lr){2-3} \cmidrule{4-6} \cmidrule(lr){7-11} \cmidrule(lr){12-16} 
    Method    & items & reviews & Rec. & Ext. & Upd.                                       &    N@1   &    N@5    &    N@10    &   N@20   &   $|T|$   &   N@1      &    N@5    &    N@10     &   N@20    &  $|T|$\\ \cmidrule(lr){1-16}
    \multirow{4}{*}{\rotatebox{90}{Sequential}} & \ding{51}&           & \ding{51} &           &            &   10.75   &   18.25   &   23.13   &   28.97   &   245.52   &   9.99   &   15.92   & 20.17   &   26.94   &   243.89   \\ 
                                                & \ding{51}& \ding{51} & \ding{51} &           &            &   11.14   &   19.95   &   24.97   &   32.00   &   29165.17   &   8.05   &   13.11   & 17.72   &   25.57   &   60429.80   \\
                                                & \ding{51}& \ding{51} & \ding{51} & \ding{51} &            &   16.09   &   26.94   &   32.35   &   40.08   &   486.49   &   13.05  &   21.38   & 26.11   &   32.62   &   459.69   \\
                                                & \ding{51}& \ding{51} & \ding{51} & \ding{51} & \ding{51}  &   15.06   &   25.71   &   31.08   &   38.28   &   415.01   &   12.59  &   21.33   & 25.96   &   32.21   &   384.87   \\ \cmidrule(lr){1-16}
    \multirow{4}{*}{\rotatebox{90}{Recency}}    & \ding{51}&           & \ding{51} &           &            &   15.34   &   24.31   &   28.82   &   34.24   &   253.31   &   12.17  &   17.75   & 22.18   &   28.19   &   249.64   \\
                                                & \ding{51}& \ding{51} & \ding{51} &           &            &   12.19   &   23.64   &   28.37   &   35.35   &   29235.16   &   8.54   &   15.78   & 21.31   &   29.21   &   60509.43    \\
                                                & \ding{51}& \ding{51} & \ding{51} & \ding{51} &            &   20.85   &   31.36   &   36.51   &   43.19   &   602.13   &   16.00  &   24.81   & 29.66   &   36.98   &   565.13  \\
                                                & \ding{51}& \ding{51} & \ding{51} & \ding{51} & \ding{51}  &   18.18   &   28.90   &   33.91   &   40.69   &   485.85   &   13.85  &   21.99  & 26.53   &   33.37   &   458.60   \\ \cmidrule(lr){1-16}
    \multirow{4}{*}{\rotatebox{90}{ICL}}        & \ding{51}&           & \ding{51} &           &            &   14.28   &   26.57   &   30.51   &   35.72   &   268.40   &   12.03  &   19.56  & 23.36   &   29.91   &   261.58   \\
                                                & \ding{51}& \ding{51} & \ding{51} &           &            &   15.11   &   26.34   &   31.25   &   37.39   &   29388.72   &   12.24  &   22.10  & 27.31   &   34.52   &   60800.61   \\
                                                & \ding{51}& \ding{51} & \ding{51} &\ding{51}  &            &   19.60   &   32.96   &   38.21   &   44.97   &   803.60   &   16.05  &   27.25  & 33.11   &   40.15   &   867.36   \\
                                                & \ding{51}& \ding{51} & \ding{51} &\ding{51}  & \ding{51}  &   16.62   &   29.81   &   35.60   &   42.00   &   592.48   &   15.80  &   26.32  & 32.03   &   38.93   &   634.02   \\
    \bottomrule
    \end{tabular}
    % \end{adjustbox}
  \vspace{-0.25cm}
    \caption{\textbf{Component-wise study of \myalg{}.} Each configuration varies which data sources (items, reviews) and which \myalg{} components are used (Rec. = Recommendation, Ext. = Extractor, Upd. = Updater), as indicated by \ding{51}. We report N@k scores ($k\in\{1,5,10,20\}$) and average of input token size (|T|) for Recommender.}
  % \caption{\textbf{Component-wise study of \myalg{}.} |T| indicates the average of the number of input tokens.}
  \label{table:ablation}
  \vspace{-0.4cm}
\end{table*}
\setlength{\tabcolsep}{6pt}


\section{Experiment}
\textbf{Datasets.} For a thorough evaluation, we utilize two datasets from the Amazon collection~\cite{ni2019justifying}: Video Games and Movies \&\ TV. To ensure a comprehensive analysis, we intentionally select datasets with diverse statistical properties, particularly in terms of the number of items (See~\autoref{app:dataset} for details).

% \begin{itemize}
%     \item  \textbf{Video Games.} We select about 15K users and 37K items. Following existing studies \cite{kang2018self}, we removed users and items with fewer than 10 interactions.
%     \item \textbf{Movies and TV.} We select about 98K users and 126K items, removing users and items with fewer than 10 interactions as in the Video Games dataset.
% \end{itemize}

\smallskip
\noindent \textbf{Baselines.} 
~\cite{hou2024large} is the recommendation method that utilizes pre-trained LLMs without additional training or fine-tuning, making it a suitable baseline. It describes three approaches for LLM-based recommendation: Sequential, Recency, and in-context learning (ICL). We compare our method with all three approaches and demonstrate the superiority of \myalg{} when these techniques were applied to our framework, further highlighting its effectiveness. (See~\cref{app:interaction} for details.)
% Although many recommendation system studies leverage large language models (LLMs), most focus on training LLMs to predict the next items. Since \myalg{} is a train-free method, a direct comparison with train-based approaches would be unfair. 
% To the best of our knowledge, ~\cite{hou2024large} is the recommendation method that utilizes pre-trained LLMs without additional training or fine-tuning, making it a suitable baseline. To ensure comprehensive comparison with \myalg{}, we evaluate baselines with various datasets. (See ~\cref{app:interaction} for details.)

% Although many recommendation system studies leverage large language models (LLMs), most focus on training LLMs to predict the next items. Since \myalg{} is a train-free method, a direct comparison with train-based approaches would be unfair. To the best of my knowledge, ~\cite{hou2024large} is the only paper that used pre-trained LLMs for item recommendation without additional training. Therefore, we set Sequential, Recency, and ICL from~\cite{hou2024large} as the baselines for comparison.

\smallskip
\noindent \textbf{Evaluation Setting.} To assess the performance of \myalg{}, we adopt a continuous sequential recommendation task. Note that NDCG scores are first aggregated per user across multiple recommendation sessions and then across all users, reflecting the continuous nature of our setup.
%Unlike conventional recommendation settings, this task requires a modification of the standard NDCG metric~\cite{zhao2021recbole} to align with our evaluation. Specifically, we calculate NDCG for each user and then compute the average NDCG score for all the users. 
%Note that results of all the experiments use the proposed NDCG metric.

% \noindent \textbf{Evaluation Setting.} To assess the performance of \myalg{}, we adopt a continuous sequential recommendation task. We construct the candidate set $\mathcal{C}_u$ by adding 19 randomly selected non-interacted items and one ground truth. For quantitative evaluation, we adopt a widely used metric, Normalized Discounted Cumulative Gain (NDCG) \cite{zhao2021recbole}, across all experiments. Specifically, if a user has purchased $k$ items, we evaluate the model from the 4th to the $k$-th interaction, resulting in $k-4$ recommendations. The NDCG values from these $k-4$ steps are averaged for each user and the final NDCG score is obtained by averaging across all users.

% Moreover, we constitute of candidate $\mathcal{C}_u$ by adding 19 randomly selected non-interacted items and one ground truth. For quantitative evaluation, we adopt a widely used metric, Normalized Discounted Cumulative Gain (NDCG) \cite{zhao2021recbole}, across all experiments.

\smallskip
\noindent \textbf{Implementation Details.} The prediction process is framed as a classification task where the model selects one item from candidate set $\mathcal{C}_u$. Each candidate set consists of 19 randomly selected non-interacted items and a ground truth item. We adopt Llama-3.2-3B-Instruct \cite{touvron2023llama} as the backbone model for all the experiments.
%and we normalized the LLM's output format to ensure consistent extraction as a list. 

% \cite{dubey2024llama}

\subsection{Experimental Results}

\smallskip
\noindent \textbf{Impact of Review Extractor.} 
\autoref{table:main} compares \myalg{} with (1) three baselines solely based on purchased items; (2) modified baselines, marked with $\dagger$, that additionally utilize users' raw reviews. 
%We newly create baselines (denoted as $\dagger$) by incorporating raw reviews into the original prompt template design.
The results reveal that baselines that simply combine item interactions with raw reviews show inconsistent performance improvements.
%It is difficult to observe consistent performance improvements in baselines that simply combine item interactions with raw reviews. 
In contrast, \myalg{}, which leverages the review extractor and profile updater, significantly outperforms all baselines. This demonstrates that {processing reviews at three levels, {i.e.}, like, dislike, and key features, is essential for enhancing performance}.

% \textbf{\myalg{} surpasses LLM-based Recommender.} To validate the effectiveness of \myalg{}, we compare it against several baselines and present the results in \cref{table:main}. Unlike previous studies that rely solely on past purchase history, we also include heuristic methods that incorporate users' reviews as additional baselines. 

\begin{figure}[t]
    \centering
    \small
    \includegraphics[width=\linewidth]{figures/memrec_fig2.pdf}
    ~~~(a) Video Games  ~~~~~~~~~~~~~~~~~ (b)  Movies and TV
    \vspace{-0.5em}
    \caption{\textbf{Trade-off between NDCG and token size.}}
    \label{fig:com}
    \vspace{-0.05cm}
\end{figure}


\smallskip
\noindent \textbf{Component-wise Study.}
\autoref{table:ablation} shows the ablation study of \myalg{}, where we analyze the impact of reviews (using or not using) and the effect of components (enabling or disabling the review extractor and profile updater). The use of reviews bring high performance gains only when accompanied by Review Extractor (Ext.). This is due to the sharp increase in input tokens (see the |$T$| column of the 2nd and 3rd rows of each method) as the user continues purchases.

Notably, the best recommendation performance is achieved when Profile Updater (Upd.) is disabled (see the 3rd and 4th rows for each method). That is well-formed context by Review Extractor can bring higher gains when simply concatenated. However, it may face a challenge, as the number of purchases grows, leading to significant computational overhead. 
Thus, we use the Profile Updater (Upd.) to maintain compact user profiles, reducing input token size by 15--20\% with only a slight 1--3\% performance drop. This trade-off underscores the importance of using Profile Updater for long-term recommendations.

% \noindent \textbf{Necessities of user profile.} Our experiments reveal that directly using raw reviews without preprocessing fails to yield significant performance improvements. This indicates that past purchase history or raw reviews alone provide insufficient information for generating accurate recommendations. In contrast, leveraging the refined reviews produced by \myalg{} effectively addresses this limitation and consistently outperforms all baselines. These results underscore the importance of review preprocessing in capturing nuanced user preferences and improving recommendation accuracy.
% Table~\ref{table:main} presents a comparison of models using only reviews, only historical interactions, and a combination of both. The results demonstrate that incorporating both reviews and historical interactions significantly improves recommendation performance compared to using either in isolation. This highlights the importance of leveraging both explicit user opinions and past behavioral patterns for effective recommendations. Furthermore, the highest performance was achieved when the LLM was utilized to refine user reviews, extracting likes and dislikes, while simultaneously identifying key features from historical interactions. This result underscores the effectiveness of structured information extraction in improving recommendation accuracy by enabling the model to systematically capture and utilize user preferences and key product attributes.
% Furthermore, by applying a 1024-token truncation strategy, we aligned the token budget with real-world recommendation constraints, ensuring that the LLM operates within a practical computational scope. This constraint is crucial for maintaining efficiency while preserving essential contextual information. The results validate that even within a limited token budget, MemRec effectively captures relevant user preferences, reinforcing the feasibility of LLM-based recommendation in real-world applications.




% \begin{figure}[t]
%     \centering
%     \begin{subfigure}[b]{0.49\linewidth}
%         \centering
%         \includegraphics[width=\textwidth]{figures/graph_video.pdf}
%         \caption{Video Games}
%         \label{fig:subfig1}
%     \end{subfigure}
%     \hfill
%     \begin{subfigure}[b]{0.49\linewidth}
%         \centering
%         \includegraphics[width=\textwidth]{figures/graph_movies.pdf}
%         \caption{Movies and TV}
%         \label{fig:subfig2}
%     \end{subfigure}
%     \caption{\textbf{Performance Comparison\textcolor{red}{?}}}
%     \label{fig:com_fig}
% \end{figure}


% \noindent \textbf{Component-wise Analyses of \myalg{}.}~\autoref{table:ablation} shows the results of our ablation study conducted across four settings: (1) past purchase history only, (2) + review, (3) + review with extracted information, (4) + review with refined extracted information. Utilizing user preference consistently improves performance. While the model that concatenates all user preferences shows slightly better results, our model achieves comparable performance while saving tokens, making it more efficient. Moreover, as more reviews are added, our model can leverage its summarization and memory mechanisms, demonstrating even greater effectiveness in capturing user preferences and providing accurate recommendations (See~\autoref{app:qual}).

\smallskip
\noindent\textbf{Trade-off Analysis.}
We categorize users into three groups based on the total cumulative review token count per user, as the criterion: 0–500 (short), 500–1000 (middle), and 1000–2000 (long) tokens.
\autoref{fig:com} presents the trade-off between recommendation performance and input token length of the three models including \myalg{}.

\myalg{} achieves the best trade-off, showing the steepest NDCG increase compared to other methods as input token size grows. Therefore, this demonstrates that \myalg{} accurately distills key information from long reviews, while achieving high efficiency by minimizing input token growth without information loss, even for long-group users.


%The results indicate that ICL performs the worst due to relying solely on item interactions. In contrast, both ICL$^\dagger$ and \myalg{}(ICL), which leverage both item interactions and reviews, achieve better performance than ICL. Notably, as the total review length increases, \myalg{}(ICL) outperforms ICL$^\dagger$ by a larger margin while using fewer input tokens. This shows that \myalg{} becomes increasingly powerful as more reviews are available.

% the results of experiments conducted across multiple datasts, where we divided the data into four groups based on the number of tokens in naive reviews. Our model outperforms others both in terms of token efficiency and recommendation performance.

% \qy{In this paper, we propose an efficient single-stage framework called \nickname{} for 3D object detection. Considering the task of object detection inherently focuses on the foreground points, we propose an instance-aware learning-based downsampling way to automatically select the sparse yet important instance points. In addition, a dedicated contextual centroid perception module is proposed to fully exploit the geometrical structure around the bounding boxes. Extensive experiments conducted on the KITTI detection benchmark demonstrated the superior efficiency and accuracy of the proposed \nickname{}. \revise{In future work, we will further tackle extreme cases such as overlapped bounding boxes.}}

%This paper presents a new point-based single-stage 3D object detection networks, named \nickname{}. With novel instance-aware downsampling strategy and centroid rally module, we can effectively and efficiently achieve muti-class 3D object detection in a bottom-up manner.  Our \nickname{} achieves the best results among pure point-based methods, and provides a state-of-the-art efficiency than existing LiDAR detectors. In the future, we will focus on designing an efficient network to achieve real-time and robust 3D detection in 360-degree LiDAR scenes.

\qy{In this paper, we propose an efficient solution termed \nickname{} for point-based 3D object detection in LiDAR point clouds. Considering the task of object detection inherently focuses on the foreground information, we propose an instance-aware learning-based downsampling way to automatically select the sparse yet important instance points. Additionally, a dedicated contextual centroid perception module is proposed to fully exploit the geometrical structure around the bounding boxes. Extensive experiments conducted on three detection benchmarks demonstrated the superior efficiency and accuracy of the proposed \nickname{}. 
}

\smallskip\noindent\textbf{Limitations.} Although the proposed \nickname{} can achieve remarkable efficiency in object detection of large-scale LiDAR points clouds, it also has limitations. \textit{e.g.,} the instance-aware sampling relies on the semantic prediction of each point, which is susceptible to class imbalances distribution. For future work, we will further explore advanced techniques to alleviate the imbalanced issue.




% Entries for the entire Anthology, followed by custom entries
% \bibliography{anthology,custom}
\bibliography{custom}

\newpage
% \documentclass{MITstyle}

%\usepackage[table]{xcolor}
\usepackage{chngcntr}
\usepackage{hyperref}
\usepackage{microtype}

\title{A Lightweight and Extensible Cell Segmentation and Classification Model for Whole Slide Images}

\author{Nikita Shvetsov~$^{1, }$\footnote{Correspondence e-mail: nikita.shvetsov@uit.no}, Thomas K. Kilvaer~$^{2, 3}$, Masoud Tafavvoghi~$^{4}$, Anders Sildnes~$^{1}$, \\ Kajsa Møllersen~$^{4}$, Lill-Tove Rasmussen Busund~$^{5, 6}$, Lars Ailo Bongo~$^{1}$ \\
%
\vspace{1em} % Space between authors and afilliations
%
\normalfont{\small $^{1}$Department of Computer Science, UiT The Arctic University of Norway}\\
\normalfont{\small $^{2}$Department of Oncology, University Hospital of North Norway}\\
\normalfont{\small $^{3}$Department of Clinical Medicine, UiT The Arctic University of Norway}\\
\normalfont{\small $^{4}$Department of Community Medicine, UiT The Arctic University of Norway}\\
\normalfont{\small $^{5}$Department of Medical Biology, UiT The Arctic University of Norway} \\
\normalfont{\small $^{6}$Department of Clinical Pathology, University Hospital of North Norway} %\vspace{2em}
}

\begin{document}
\maketitle

\section*{Abstract}

% \begin{abstract}
% Developing clinically useful cell-level analysis tools in digital pathology remains challenging due to limitations in dataset granularity, inconsistent annotations, computational demands of advanced models, and difficulties in integrating new technologies into clinical workflows. To address these challenges, we propose a multi-faceted solution that enhances data quality, model performance, and usability to create a lightweight and extensible cell segmentation and classification model.

% First, we update data labels by employing a cross-relabeling process that refines the labels of two existing datasets, PanNuke and MoNuSAC, to create a new unified dataset with enhanced granularity, encompassing seven distinct cell types. Second, we leverage the H-Optimus foundation model as a fixed encoder to improve feature representation for simultaneous cell segmentation and classification tasks. Third, to address the computational demands of foundation models, we employ knowledge distillation to reduce model size and complexity while maintaining comparable performance. Finally, to facilitate integration into clinical workflows, we integrate the distilled model into the QuPath software, a widely used open-source platform in digital pathology.

% Our results demonstrate improvements in cell segmentation and classification performance using the H‑Optimus-based model compared to a CNN-based model. Specifically, the average $R^2$ improved from 0.575 to 0.871, and the average $PQ$ score improved from 0.450 to 0.492, indicating better alignment with actual cell counts and enhanced segmentation and classification quality. Furthermore, the distilled student model maintains performance comparable to the larger foundation model while reducing the parameter count by a factor of 48.
% Overall, by reducing computational complexity and integrating it into existing workflows, the proposed approach may significantly impact diagnostic processes, reduce the workload of pathologists, and contribute to improved patient outcomes. Though our approach shows potential enhancements in efficiency and usability of cell segmentation and classification models in digital pathology, extensive validation is needed to deploy these models in clinical practice.
% \end{abstract}

%%% shortened abstract
\begin{abstract}
Developing clinically useful cell-level analysis tools in digital pathology remains challenging due to limitations in dataset granularity, inconsistent annotations, high computational demands, and difficulties integrating new technologies into workflows. To address these issues, we propose a solution that enhances data quality, model performance, and usability by creating a lightweight, extensible cell segmentation and classification model. 

First, we update data labels through cross-relabeling to refine annotations of PanNuke and MoNuSAC, producing a unified dataset with seven distinct cell types. Second, we leverage the H-Optimus foundation model as a fixed encoder to improve feature representation for simultaneous segmentation and classification tasks. Third, to address foundation models' computational demands, we distill knowledge to reduce model size and complexity while maintaining comparable performance. Finally, we integrate the distilled model into QuPath, a widely used open-source digital pathology platform. 

Results demonstrate improved segmentation and classification performance using the H-Optimus-based model compared to a CNN-based model. Specifically, average $R^2$ improved from 0.575 to 0.871, and average $PQ$ score improved from 0.450 to 0.492, indicating better alignment with actual cell counts and enhanced segmentation quality. The distilled model maintains comparable performance while reducing parameter count by a factor of 48. By reducing computational complexity and integrating into workflows, this approach may significantly impact diagnostics, reduce pathologist workload, and improve outcomes. Although the method shows promise, extensive validation is necessary prior to clinical deployment.
\end{abstract}
\clearpage

\section{Introduction}
In digital pathology, accurate segmentation and classification of cells are crucial for many diagnostic, prognostic, and predictive analyses \cite{Jaber_Beziaeva_etal._2019,Lin_Pan_etal._2022,Park_Ock_etal._2022,Shen_Choi_etal._2024}. Nowadays, developments in computational pathology offer multiple solutions \cite{H._Qu_P._Wu_etal._2020,Javed_Mahmood_etal._2020} to utilize cell-level datasets to train machine learning models that solve these problems. The quality and specificity of training datasets are critical for robust and accurate models. Adhering to the principle of "garbage in, garbage out", it is essential to ensure that these datasets are extensively and accurately labeled with distinct classes that reflect the diverse biological characteristics of different cell types. Unfortunately, the number of open-source datasets comprising such high-quality annotations is limited. Existing cell segmentation datasets \cite{Gamper_Koohbanani_etal._2019,Graham_Vu_etal._2019,Verma_Kumar_etal._2021} may offer extensive annotations for certain cell types while providing more general labels for others. For example, in PanNuke, which is one of the largest open-source datasets comprising labeled cells, various types of morphologically and functionally different inflammatory cells like macrophages and lymphocytes are clustered in a broad "inflammatory" class. Consequently, these classes are frequently omitted from analyses or aggregated into broader meta-classes \cite{Gamper_Koohbanani_etal._2020} and likely interfere with other cell classes included in the dataset. This and similar inconsistencies in annotation granularity limit the ability of machine learning models to learn the comprehensive and nuanced features necessary for accurate cell segmentation and classification. To address these challenges, methods for refining and standardizing dataset annotations are essential to enhance the quality of training data.

A complementary approach to mitigate the absence of high-quality training data is the use of foundation models. Foundation models as encoders are defined as large-scale, versatile networks pre-trained on vast, diverse datasets using self-supervised learning, contrasting with convolutional neural network (CNN) pre-trained encoders that rely on supervised learning with labeled data. In practice, foundation models leverage enormous amounts of weakly or unlabeled data from millions of whole slide images (WSIs) and employ self-attention mechanisms to capture long-range dependencies and global context \cite{Chen_Ding_etal._2024,Saillard_Jenatton_etal._2024,Vorontsov_Bozkurt_etal._2024,Xu_Usuyama_etal._2024}. As a consequence, foundation models are able to produce transferable feature representations across different cell types and tissue environments. The feature representations can be leveraged by decoder networks to produce segmentation masks and pixel-level classifications. Because foundation models have comprehensive feature representations, they can be effectively fine-tuned using much smaller amounts of cell-level data compared to the large datasets needed to train models from scratch. Furthermore, foundation models incorporate adversarial training elements or contrastive learning \cite{Chen_Ding_etal._2024,Xu_Usuyama_etal._2024}, enhancing their resilience and adaptability by exposing them to challenging and varied scenarios during training. This may result in more generalizable models, often making them well-suited for diverse and complex tasks in digital pathology.

Despite the inherent advantages of foundation models, their deployment for practical use faces its own obstacles. In particular, they require substantial computational power, financial investments and rigorous testing to ensure reliability and efficacy for a given task \cite{Akkus_Dangott_etal._2022,Dragomir_Cocuz_etal._2022,Go_2022,Jafri_Farooqui_etal._2024}. Moreover, while foundation models enhance feature representation and performance, they depend on the quality of available annotations for decoder fine-tuning and, like any other model, cannot resolve existing inconsistencies or ambiguities in data labels. Therefore, there remains a critical need for solutions that address both data quality and practical deployment considerations.
Further, integrating new technologies into existing clinical workflows often encounters resistance, as it necessitates adjustments to established diagnostic processes. So, there is a need to develop solutions that could be integrated into current practices, minimizing the burden on medical professionals to adopt new tools \cite{King_Williams_etal._2023}.

Existing solutions \cite{Goldsborough_Philps_etal._2024,Hörst_Rempe_etal._2024}, while addressing some aspects of these challenges, fall short in providing a comprehensive approach. To address the data quality and clinical deployment issues, we propose a multi-faceted solution that encompasses data refinement, model optimization, and integration with existing pathology tools (\hyperref[fig:fig1]{Figure 1}). The outcome is a lightweight cell segmentation and classification model that can be integrated into digital pathology workflows for practical clinical use.

\begin{figure}[h!]
    \centering
    \includegraphics[width=\textwidth, height=0.82\textheight, keepaspectratio]{images/Figure_1.pdf}
    \caption{Overview of the proposed solution, including 1) Data refinement using cross-relabeling, 2) Teacher model development and fine tuning, 3) Student model optimization with knowledge distillation and 4) Student model and QuPath integration}
    \label{fig:fig1}
\end{figure}
\clearpage

Our approach begins with preparing the data for the fine-tuning and training of the machine learning models. We create a refined dataset, acquired via cross-relabeling two cell-level datasets, enhancing annotation specificity and consistency of the labeled data. Subsequently, we create a cell segmentation and classification model based on the foundation model. We leverage the foundation model as a fixed encoder and fine-tune a decoder using the refined dataset to improve generalization across diverse tissue- and cell types.
To ensure that the model remains lightweight and deployable in a possibly resource-constrained environment, we employ knowledge distillation to approximate the functionality of the foundation model. Finally, to facilitate the practical application of our model in digital pathology workflows, we integrate it with the QuPath \cite{Bankhead_Loughrey_etal._2017} application. Each methodological component contributes to the overarching goal of enhancing model performance, generalizability, and usability in clinical settings.

The primary contributions of this paper are:
\begin{enumerate}
    \item \textit{Data labels refinement through cross-relabeling:}
    
    We propose a new method for refining labels of cell-level datasets through cross-relabeling. This method employs classification models to re-label broad and ambiguous instances, resulting in a more diverse dataset. Our evaluation demonstrates that these classification models achieve high accuracy on test subsets, indicating the reliability of the method for label refinement.

    \item \textit{Enhanced model performance via foundation models:}
    
    We employ a foundation model as a feature extractor for the cell segmentation and classification task. In comparison with training a CNN model from scratch, the foundation model backbone only needs fine-tuning, which significantly reduces training time, computational resources and data requirements. We show that using a foundation model encoder leads to better performance in cell segmentation and classification networks than using a CNN-based encoder. This improvement may enable the model to generalize more effectively across various tissue types and imaging methods.
    
    \item \textit{Model optimization through knowledge distillation:}
    
    We show that a smaller student model trained using knowledge distillation on the refined dataset obtained via our cross-relabeling approach from a foundation model achieves comparable performance in cell segmentation and quantification tasks. As a result, this model is more suitable for deployment in environments without high-performance computing resources.
    
    \item \textit{Integration with QuPath:}
    
    We integrate the distilled cell segmentation and classification model into QuPath, a widely used open-source digital pathology platform, to accelerate clinical adaptation by enabling pathologists to more easily incorporate advanced computational tools into their existing workflows.
\end{enumerate}

Through these methodological steps, we aim to bridge the gap between advanced machine learning techniques and practical clinical applications, making accurate and efficient digital pathology accessible in a broader range of healthcare settings.

\section{Refining Existing Datasets Using Cross-Relabeling}
To address the limitations of sparse and ambiguous labeling of cell-level datasets, we propose a generalizable cross-relabeling strategy that can be applied to any dataset containing broadly categorized or imprecisely labeled cell types. This approach involves training and subsequently leveraging classification models to refine broad categories into more specific or biologically relevant classes.
When applied to cell-level data, the methodology includes extracting individual cell images from the dataset patches, preprocessing these images to standardize the size and accommodate partial cells, and then training deep learning classifiers capable of distinguishing between the finer cell subtypes within the coarser categories. 
To illustrate our approach, we focus on the PanNuke \cite{Gamper_Koohbanani_etal._2020, Gamper_Koohbanani_etal._2019} and MoNuSAC \cite{Verma_Kumar_etal._2021} datasets that we have used to train models for cell quantification in our previous works \cite{Shvetsov_Grønnesby_etal._2022,Shvetsov_Sildnes_etal._2024}. We find that for better cell differentiation we have to introduce more granular labels. PanNuke includes a broad classification of "inflammatory" cells, encompassing lymphocytes, macrophages, and neutrophils. Each cell type differs significantly in structure, function, and clinical relevance. Conversely, MoNuSAC uses the label "epithelial" for a class that comprises both benign epithelial cells and malignant neoplastic cells. This practice makes it challenging to differentiate between benign and malignant epithelial cells in the dataset, which is a critical distinction when identifying tumor areas within tissue samples. To address these issues, we implement a cross-relabeling strategy as shown in \hyperref[fig:fig2]{Figure 2}. The key components are two classification models: one is trained on singular cell images from PanNuke data to classify the epithelial meta-class into epithelial and neoplastic classes. The other is trained on MoNuSAC to refine the inflammatory class into lymphocytes, neutrophils, and macrophages.

\begin{figure}[h!]
    \centering
    \includegraphics[width=\textwidth]{images/Figure_2.pdf}
    \caption{Refined dataset generation via cross relabeling}
    \label{fig:fig2}
\end{figure}

The refining approach consists of three consecutive steps. The first is the preprocessing step, in which we extract individual cells from both datasets (\hyperref[fig:fig3]{Figure 3}). The specifics of PanNuke and MoNuSAC patch preparation before cell preprocessing are provided in \hyperref[chap:S1]{Appendix S1}.

\begin{figure}[h!]
    \centering
    \includegraphics[width=\textwidth]{images/Figure_3.pdf}
    \caption{Cell instances preprocessing including (1) cell map extraction, (2) bounding box delineation, (3) adjusting cell boxes and (4) cropping and resizing of cell images}
    \label{fig:fig3}
\end{figure}

During preprocessing, we extract cell type maps from the ground truth label mask and calculate bounding boxes around each cell instance. To accommodate partial cells at patch borders, a common issue in cropped patch images, we employ mirror padding and extend the field of view of the cell label by 15 pixels to capture adjacent cells. We then crop and resize the identified regions to $64 \times 64$ pixels using bicubic interpolation.

The preprocessed PanNuke dataset comprises 68,031 neoplastic and 23,207 epithelial cell images, while MoNuSAC comprises  33,104 lymphocytes, 1,252 neutrophils, and 1,695 macrophages, which we subsequently use in training cell classification models and classifying the cell image data \hyperref[fig:S2]{Appendix Figure S2 (1)}. 

The next step is to train two distinct ResNet50-based classifiers tailored to address the specific labeling challenges inherent in each dataset. We use ResNet50 for classification models due to its proven effectiveness for image classification tasks in histopathology \cite{pan2022reviewmachinelearningapproaches}, and its compatibility with small images. For the PanNuke dataset, we design the classifier, trained on MoNuSAC data, to disaggregate the heterogeneous "inflammatory" cell category into distinct subtypes: lymphocytes, macrophages, and neutrophils. Similarly, for the MoNuSAC dataset, the classifier is trained on PanNuke data and distinguishes between benign and malignant epithelial cells within the overarching "epithelial" label. By applying these targeted classifiers to their respective datasets, we assign more specific labels to individual cell instances, thus enabling us to create a unified dataset.
To ensure a balanced representation of classes, we train both models on datasets that had been equalized to match the size of the least represented class. Thus, we obtain datasets comprising 23,207 samples per class for PanNuke and 1,252 samples per class for MoNuSAC data. Next, we partition both of them into training (70\%), validation (20\%), and testing (10\%) subsets. To mitigate the risk of overfitting, we use a single dropout layer with a rate of p=0.5 in both models and data augmentation using randomized color perturbations, rotation, and horizontal and vertical flipping. We employ AdamW optimizer and the cross-entropy loss function for the training criterion.

To evaluate the two trained models, we measure the classification accuracy on the respective test subsets. The accuracies on the test subset for both classifiers are presented in \hyperref[tab:1]{Table 1}. The PanNuke model achieves an average accuracy of 93.57\%, with higher accuracy for neoplastic cells (96.06\%) compared to epithelial cells (86.26\%). The confusion matrix in Figure A3.1 shows that the model predominantly distinguishes accurately between epithelial and neoplastic tissues, with a substantial number of correct classifications and relatively few misclassifications. The MoNuSAC model demonstrates an average accuracy of 98.92\%, excelling in classifying lymphocytes (99.67\%) and macrophages (94.12\%), with lower performance for neutrophils (85.71\%). The confusion matrix in Figure A3.2 shows that the model identifies lymphocytes and performs reasonably well with macrophages and neutrophils.

\begin{table}[h!]
\renewcommand{\arraystretch}{1.5}
  \centering
  \caption{Cell classification results for PanNuke and MoNuSAC trained models (CI 95\%).}
  \label{tab:1}
  \begin{tabular}{|l|c|c|}
   \hline
   %\rowcolor{gray!30}
    Accuracy               & PanNuke model              & MoNuSAC model              \\
    \hline
    Average      & 0.936 (0.931--0.941)         & 0.989 (0.986--0.993)        \\
    \hline
    Neoplastic   & 0.961 (0.956--0.965)         & -                          \\
    \hline
    Epithelial   & 0.863 (0.849--0.877)         & -                          \\
    \hline
    Lymphocytes  & -                          & 0.997 (0.995--0.999)        \\
    \hline
    Neutrophils  & -                          & 0.857 (0.796--0.918)        \\
    \hline
    Macrophages  & -                          & 0.941 (0.906--0.976)        \\
    \hline
  \end{tabular}
\end{table}

Finally, during the last step, we use the model trained on PanNuke data for epithelial cells in MoNuSAC and the model trained on MoNuSAC for the inflammatory cells class in PanNuke. Specifically, we use classifier models to relabel epithelial cells in MoNuSAC and inflammatory cells in PanNuke data. Then we combine cells with refined labels and the rest of the cells in both datasets to create a refined dataset (\hyperref[fig:S2]{Appendix Figure S2 (2)}). The process of relabeling cells and visualizing them on a patch is shown in \hyperref[fig:fig4]{Figure 4}. The cell counts in the refined dataset are provided in \hyperref[tab:S4]{Appendix Table S4}.

\begin{figure}[h!]
    \centering
    \includegraphics[width=\textwidth, height=0.42\textheight, keepaspectratio]{images/Figure_4.pdf}
    \caption{Cell relabeling procedure for epithelial and inflammatory cell classes}
    \label{fig:fig4}
\end{figure}

%\hfill

Relabeling and combining datasets have been explored in a prior study \cite{Parulekar_Kanwat_etal._2023}, where consecutive fine-tuning on multiple datasets was employed to account for hierarchical class label structures. While the method presented in \cite{Parulekar_Kanwat_etal._2023} is intuitive, it often lacks consistency and requires multiple fine-tuning runs, which can be cumbersome and time-consuming. 
In contrast, cross-relabeling simplifies this process by using specialized classification models tailored to each dataset's specific labeling challenges. This approach provides better transparency and produces a unified dataset encompassing seven distinct cell types across multiple tissue samples, enhancing data diversity for further model training or fine-tuning.

Despite these improvements, cross-relabeling does not entirely resolve issues related to poor labeling quality or the amount of labeled data. Specifically, our results show lower accuracies persist for underrepresented classes, such as macrophages, which may stem from a limited sample availability and intrinsic challenges in distinguishing these cells based solely on H\&E staining. Furthermore, while our method enhances label specificity, it relies on the initial quality of the broad labels; thus, any fundamental inaccuracies in the original annotations can propagate through the relabeling process. Addressing the overall problem of limited data labels may require integrating additional data sources or utilizing complementary immunohistochemical staining methods.
Although the reported performance metrics are obtained from evaluations on the native test sets of each dataset, it is important to note that the primary application of these classifiers is to perform cross-relabeling, where a model trained on one dataset (e.g., PanNuke) is applied to another (e.g., MoNuSAC) and vice versa. We acknowledge that a more systematic evaluation of cross-dataset generalization is needed and could be performed in future work.

Overall, the refined dataset produced by our approach can enhance the supervised training or fine-tuning of cell segmentation and classification models, especially those that utilize pre-trained foundation models to improve feature extraction robustness. In addition, these models can detect nuanced classes that enable researchers to conduct more detailed analyses of biological processes in computational pathology.

\section{Foundation models for robust cell segmentation and classification}

Accurate cell segmentation and classification in digital pathology are hindered by limited labeled data and the fact that conventional CNNs are unable to capture global contextual information due to their local receptive field constraints \cite{Gheflati_Rivaz_2022,Yang_Marcus_etal.}. Traditional approaches in cell quantification have predominantly relied on CNN encoders, such as ResNet50, given their proven effectiveness in semantic segmentation tasks \cite{Deshmane_2023,Graham_Vu_etal._2019,Mukasheva_Koishiyeva_etal._2024,Stringer_Wang_etal._2021}. However, approaches that include fine-tuning of pretrained CNNs, data augmentation, and stain normalization to partially increase data variability and address staining differences often fail to achieve the necessary generalization and robustness across diverse tissue types and staining conditions \cite{G._Wang_W._Li_etal._2018,Gao_Bagci_etal._2018,Karim_El_Khoury_Martin_Fockedey_etal._2021}.

To overcome these challenges, we leverage an encoder-decoder network that uses a foundation model as the encoder and a CNN upsampling decoder (\hyperref[fig:fig5]{Figure 5}) for simultaneous cell segmentation and classification in 2D patches extracted from WSIs. Foundation models with transformer-based architectures are viable alternatives to CNN-based encoders \cite{Shamshad_Khan_etal._2023,Sourget_2023}. They enable the creation of more advanced architectures that can decode or transform learned features more effectively \cite{Chen_Duan_etal._2023,Cheng_Misra_etal._2022,Xie_Wang_etal._2021}.

\begin{figure}[h!]
    \centering
    \includegraphics[width=\textwidth]{images/Figure_5.pdf}
    \caption{UNETR-like model with foundational model as backbone}
    \label{fig:fig5}
\end{figure}

By utilizing a transformer-based encoder, we incorporate global contextual information into the feature extraction process, which is a key advantage of such architectures \cite{Chen_Lu_etal._2021}. This foundation model integration facilitates accurate pixel-wise segmentation and classification without the need for extensive encoder training, thereby potentially improving generalization across varied cellular structures and tissue types.
In our implementation, we employ a modified UNETR \cite{Hatamizadeh_Tang_etal._2021} architecture that combines a vision transformer (ViT) \cite{Dosovitskiy_Beyer_etal._2021} encoder with a CNN-based decoder. The encoder utilizes the pretrained H-Optimus foundation model, which contains 1.1 billion parameters and is trained on over 500,000 H\&E stained WSIs \cite{Saillard_Jenatton_etal._2024}. We extract outputs from four evenly spaced transformer blocks $Z_i$, where $i \in [1, 14, 26, 38]$, to serve as residual connections for the CNN decoder. We select these blocks based on our observation that features from non-adjacent levels of the encoder lead to better overall performance on the test subset.

The CNN decoder upsamples the feature representations, acquired from the transformer blocks, to generate an intermediate vector that is handled by two task-specific layers that generate cell segmentation and classification masks. The first task-specific layer is the ‘Cellpose head’,  which is used to delineate cell instances. The layer generates horizontal and vertical gradient maps to form vector fields that are refined through gradient tracking in a post-processing step using the Cellpose algorithm \cite{Stringer_Wang_etal._2021}, known for its efficacy in cell segmentation tasks and generalizability across multiple domains \cite{Pachitariu_Stringer_2022,Stringer_Pachitariu_2024}. The second task-specific layer is the "Cell type head", which assigns labels to individual pixels. In the post-processing step, we determine the output classification label of each segmented cell instance by majority voting over the labeled pixels that comprise the cell in the segmentation map.

To evaluate model performance and measure the impact of adding a foundation model as backbone, we compare it to a ResNet50-based model. ResNet50 is a widely used solution for encoders in segmentation architectures in the medical domain \cite{Deshmane_2023,Graham_Vu_etal._2019,Mukasheva_Koishiyeva_etal._2024,Stringer_Wang_etal._2021}. For the H-Optimus-based model, we utilize frozen weights for the encoder and only fine-tune the decoder to take advantage of the extensive pre-training of the foundation model. For the ResNet50-based model we start with ImageNet \cite{Deng_Dong_etal.} weights and train both encoder and decoder parts. Hyperparameters for the training step are set to be identical, where possible, for comparable evaluation. 
For this evaluation, we deliberately use the PanNuke dataset to provide a standardized and controlled comparison between the H‑Optimus and ResNet50-based models (\hyperref[fig:S2]{Appendix Figure S2 (3)}). Specifically, we use two of the default PanNuke dataset splits (66\%) for training and validation, and reserve the third split (33\%) for testing.

To address the challenge of cell class imbalance in the PanNuke dataset, which is a common characteristic in most cell-level H\&E patch datasets, both models’ training processes employ a weighted loss function comprising cross-entropy and focal loss \cite{Lin_Goyal_etal._2018}. The focal loss component is adjusted with coefficients derived from each cell class' instance frequency, emphasizing learning from underrepresented classes and enhancing the model's sensitivity to rare but significant cellular patterns. The cross-entropy loss is augmented with spectral decoupling regularization \cite{Pezeshki_Kaba_etal._2021,Pohjonen_Stürenberg_etal._2022} and spatially varying label smoothing \cite{Islam_Glocker_2021}, which potentially stabilizes training and improves generalization in case of complex tissue morphologies. For optimization, we employ the \textit{AdamW} \cite{Loshchilov_Hutter_2019} to counter unbalanced class scenarios, with cosine annealing learning rate scheduler.

We utilize the scikit-learn library \cite{Van_der_Walt_Schönberger_etal._2014} and HoVer-Net \cite{Graham_Vu_etal._2019} implementations of $R^2$ (the coefficient of determination) and $PQ$ (panoptic quality) to evaluate our experiments. Complete mathematical formulations and detailed explanations of these metrics are provided in \hyperref[chap:S5]{Appendix S5}. To compute confidence intervals, we use nonparametric bootstrapping, where after calculating the metric on the full sample, we generated 1000 bootstrap replicates by resampling with replacement and then determined the 95\% confidence intervals as the 2.5th and 97.5th percentiles of the resulting empirical distribution.

%\hfill

The model comparisons are summarized in \hyperref[tab:2]{Table 2}. The H‑Optimus-based model achieves higher $R^2$ across all cell classes compared to the ResNet50-based model, which means that its predictions are more closely aligned with the PanNuke cell counts, indicating a stronger correlation with the observed data. Notably, the improvement of $R^2_{dead}$ may be an indicator of better global contextual representations provided by the foundation model backbone. In terms of segmentation and classification quality combined, measured by the PQ score, the H‑Optimus-based model demonstrates notable improvements across most cell classes. Overall, the average $R^2$ improved from 0.575 to 0.871, while the average $PQ$ score improved from 0.450 to 0.492, demonstrating better performance of the H-Optimus-based model.

\begin{table}[h!]
\renewcommand{\arraystretch}{1.5}
  \centering
  \caption{Cell quantification metrics for baseline and proposed models (CI 95\%).}
  \label{tab:2}
  \begin{tabular}{|l|c|c|}
    \hline
    %\rowcolor{gray!30}
    Metric             & Resnet50-based            & H-optimus-based              \\
    \hline
    $R^2_{neoplastic}$    & 0.681 (0.576--0.769)       & \textbf{0.941 (0.917--0.960)} \\
    \hline
    $R^2_{inflammatory}$  & 0.863 (0.778--0.903)       & \textbf{0.949 (0.918--0.966)} \\
    \hline
    $R^2_{connective}$    & 0.600 (0.488--0.698)       & 0.609 (0.436--0.772)          \\
    \hline
    $R^2_{dead}$          & 0.097 (-11.389--0.669)     & 0.925 (0.404--0.982)          \\
    \hline
    $R^2_{epithelial}$    & 0.635 (0.490--0.747)       & \textbf{0.930 (0.886--0.964)} \\
    \hline
    $PQ_{neoplastic}$       & 0.517 (0.499--0.535)       & \textbf{0.589 (0.575--0.604)} \\
    \hline
    $PQ_{inflammatory}$     & 0.455 (0.429--0.482)       & \textbf{0.528 (0.507--0.549)} \\
    \hline
    $PQ_{connective}$       & 0.416 (0.400--0.431)       & \textbf{0.451 (0.436--0.465)} \\
    \hline
    $PQ_{dead}$             & 0.374 (0.342--0.408)       & 0.292 (0.209--0.365)          \\
    \hline
    $PQ_{epithelial}$       & 0.488 (0.460--0.519)       & \textbf{0.599 (0.579--0.618)} \\
    \hline
  \end{tabular}
\end{table}

Our results  show that integrating the H‑Optimus foundation model within the UNETR architecture enhances the model's ability to segment and classify cells across diverse tissues from PanNuke data. The pretrained transformer encoder provides robust feature representations, resulting in higher average $R^2$ and $PQ$ scores compared to the CNN-based model. This leads to more reliable cell quantification and more accurate downstream analysis. Additionally, the streamlined fine-tuning process reduces computational overhead and training time, making the model more adaptable for new data.

Despite these advancements, the foundation model-based approach does not fully resolve all challenges related to cell segmentation and classification. We observe lower metric scores for underrepresented classes in the training data. Furthermore, foundation models typically encompass billions of parameters, resulting in substantial computational and memory requirements. It therefore poses challenges for deployment in resource-constrained environments, limiting their practical applicability in certain clinical settings.

\section{Model optimization via Knowledge Distillation}

To address the limitations posed by the extensive size of foundation models, we implement knowledge distillation — a model compression technique that leverages the teacher-student paradigm \cite{Hinton_Vinyals_etal._2015}. By training a smaller, more efficient student model to replicate the output of a larger, pre-trained teacher model, we retain performance while significantly reducing the model's complexity and resource requirements (\hyperref[fig:fig6]{Figure 6}).

\begin{figure}[h!]
    \centering
    \includegraphics[width=\textwidth, height=0.45\textheight, keepaspectratio]{images/Figure_6.pdf}
    \caption{Knowledge distillation framework for training a student model using a pre-trained teacher}
    \label{fig:fig6}
\end{figure}

We employ knowledge distillation to compress the H‑Optimus-based teacher model into a more efficient student model. The teacher model is the modified UNETR architecture with the H‑Optimus foundation model described in the previous chapter. The student model is based on a UNet architecture augmented with residual connections and incorporates a smaller ViT encoder with 9 million parameters \cite{Steiner_Kolesnikov_etal._2022,Wightman_2019}. 

First, we fine-tune the teacher model using the refined dataset from the cross-relabeling procedure (Section 2). Initially we train the decoder of the teacher model while keeping the encoder weights frozen. We split the refined dataset into train (70\%), validation (20\%) and test (10\%) subsets (\hyperref[fig:S2]{Appendix Figure S2 (4)}). During fine-tuning, we use the train and validation subsets, while leaving the test subset for model evaluation. We set the training procedure and model hyperparameters to be identical to those that were used to demonstrate the utility of foundation models for the simultaneous cell segmentation and classification task.

Next, we perform knowledge distillation from teacher to student using the refined dataset used to fine-tune the teacher model. The student model is trained to replicate the teacher model's outputs. We utilize a specialized loss function that aligns the student's predicted probability distribution with the teacher's, incorporating the teacher's class probability distribution derived from the output. Following the methodology of Hinton et al. \cite{Hinton_Vinyals_etal._2015}, we experiment with various hyperparameter settings for the temperature ($T$) and the balancing coefficients ($\alpha$ and $\beta$) in the loss function. We vary $T$ from 1 to 20 and adjust $\alpha$ and $\beta$ to balance the distillation and student losses. Through iterative tuning and evaluation, we identify that setting $T=14$, $\alpha=0.3$, and $\beta=0.7$ yields a configuration that converges and closely approximates the teacher model's performance during training.

Finally, we assess the performance of both models using the $R^2$ and $PQ$ (defined in \hyperref[chap:S5]{Appendix S5}) on the test set of the refined dataset (\hyperref[tab:3]{Table 3}). We observe that the 95\% confidence intervals overlap for most cell types, so we cannot claim statistically significant performance differences between the teacher and student models. One exception appears in the neoplastic class. The teacher model produces an $R^2$ of 0.919, while the student model shows an $R^2$ of 0.852. In addition, the student model achieves higher $PQ$ values for the neoplastic and connective classes, though the confidence intervals show overlap.

\begin{table}[h!]
\renewcommand{\arraystretch}{1.5}
  \centering
  \caption{Cell quantification metrics for teacher and distilled student models (CI 95\%).}
  \label{tab:3}
  \begin{tabular}{|l|c|c|}
    \hline
    %\rowcolor{gray!30}
    Metric & Teacher & Student \\
    \hline
    $R^2_{neoplastic}$    & \textbf{0.919} (0.898--0.939) & 0.852 (0.800--0.891) \\
    \hline
    $R^2_{lymphocyte}$    & 0.969 (0.956--0.977)         & 0.969 (0.956--0.978) \\
    \hline
    $R^2_{connective}$    & 0.694 (0.548--0.809)         & 0.618 (0.469--0.741) \\
    \hline
    $R^2_{dead}$          & 0.755 (0.400--0.908)         & 0.424 (0.100--0.731) \\
    \hline
    $R^2_{epithelial}$    & 0.922 (0.870--0.958)         & 0.843 (0.738--0.917) \\
    \hline
    $R^2_{macrophage}$    & 0.384 (-0.369--0.724)        & 0.704 (0.352--0.859) \\
    \hline
    $R^2_{neutrofil}$     & 0.854 (0.578--0.929)         & 0.833 (0.502--0.925) \\
    \hline
    $PQ_{neoplastic}$       & 0.581 (0.569--0.593)         & 0.601 (0.588--0.613) \\
    \hline
    $PQ_{lymphocyte}$       & 0.536 (0.520--0.553)         & 0.563 (0.544--0.579) \\
    \hline
    $PQ_{connective}$       & 0.436 (0.421--0.451)         & 0.457 (0.441--0.474) \\
    \hline
    $PQ_{dead}$             & 0.272 (0.235--0.315)         & 0.279 (0.201--0.369) \\
    \hline
    $PQ_{epithelial}$       & 0.522 (0.500--0.545)         & 0.530 (0.506--0.555) \\
    \hline
    $PQ_{macrophage}$       & 0.524 (0.459--0.588)         & 0.474 (0.405--0.543) \\
    \hline
    $PQ_{neutrofil}$        & 0.541 (0.490--0.592)         & 0.565 (0.522--0.607) \\
    \hline
  \end{tabular}
\end{table}


We further decompose the $PQ$ metric into its $SQ$ and $DQ$ components (\hyperref[tab:S6]{Appendix Table S6}). Both models produce nearly identical $SQ$ values, which indicates that they predict instance boundaries with similar precision. Although the student model shows some improvement in $DQ$ scores for certain classes, the confidence intervals overlap and do not confirm a statistically significant difference.

We observe that the student and teacher models yield comparable detection performance despite the student model using a much smaller and simpler architecture. A model with fewer parameters reduces the risk of overfitting when training data are scarce relative to the model’s complexity \cite{Farias_Ludermir_etal._2022}. The knowledge distillation process also encourages the student model to focus on the most generalizable detection features learned from the teacher. These factors enable the student model to achieve similar detection performance across different cell types.

Additionally, considering the model sizes reported in \hyperref[tab:4]{Table 4}, the distilled model achieves a significant reduction compared to the teacher model, with a 48-fold decrease in parameter count and a 5.5-fold reduction in on-disk size. In inference mode, the teacher model requires 16 GB of VRAM for a batch size of 32, while the distilled model only needs 3 GB of VRAM for the same batch size. These reductions make the distilled model significantly more practical for fine-tuning and deployment in resource-constrained environments.

\begin{table}[h!]
\renewcommand{\arraystretch}{1.5}
  \centering
  \caption{Parameter counts and size of teacher and distilled model}
  \label{tab:4}
  \adjustbox{max width=\textwidth}{%
  \begin{tabular}{|l|c|c|c|}
    \hline
    %\rowcolor{gray!30}
    Metric & H-optimus-based (Teacher) & mobileViT-based (Student) & Magnitude of difference \\
    \hline
    Parameters count       & 1,158,917,906   & \textbf{24,093,393}   & \textbf{48x}  \\
    \hline
    Estimated Total Size (MB) & 87,912       & \textbf{15,935}    & \textbf{5.5x} \\
    \hline
  \end{tabular}%
}
\end{table}

%\hfill

With recent advancements in complex network architectures and the use of pretrained encoders to achieve state-of-the-art performance \cite{Baumann_Dislich_etal._2024,Hörst_Rempe_etal._2024} in cell segmentation and classification tasks, model size, computational complexity, and processing times have increased. This limits the scalability and accessibility of these models. As we demonstrate, this may be mitigated using knowledge distillation. Studies in the field of natural language processing have demonstrated the efficacy of knowledge distillation in retaining the capabilities of the teacher model while achieving significant reductions in size and complexity \cite{Huangpu_Gao_2024,Sun_Yu_etal.}. 

We demonstrate the feasibility of knowledge distillation in digital pathology, specifically for cell segmentation and classification tasks. Moreover, we achieve this performance while also significantly reducing the parameter count. In addressing the challenge of knowledge transfer, we found that distillation from a transformer-based model to a smaller transformer is more straightforward than attempting to map transformer features to CNN blocks. In our experiments, using a CNN-based network as a student results in worse cell quantification performance due to the structural constraints of CNN feature space dimensions. 

Although our primary approach relies on a transformer-based student model that performs well, it can be further optimized to incorporate advantages from CNN architectures. For example, employing alternative techniques such as using ViT adapters \cite{Chen_Duan_etal._2023} or $1 \times 1$ convolutions to adjust feature map sizes may be beneficial for harnessing CNN advantages like enhanced local feature extraction. Moreover, if additional performance improvements are desired, the process can be further enhanced by applying supplementary knowledge distillation techniques, such as self-distillation \cite{Zhang_Song_etal._2019} or online distillation \cite{Houyon_Cioppa_etal._2023}.

Despite these promising results, further validation on independent datasets is necessary to fully understand the model's limitations. Underrepresented classes may pose challenges when addressing complex cases. Pathologists need to validate these models to adopt them in clinical settings. While the distilled models are smaller and more deployable, a technological gap persists because pathologists traditionally rely on established methods for inspecting WSIs and diagnosing diseases. Addressing the complexities involved in deploying models for inference and supporting pathologists in adopting new tools is essential for integrating these models into clinical workflows.

\section{Model integration with QuPath}
Digital pathology tools with graphical user interfaces are essential for visualizing and analyzing WSIs. To make our student model useful in clinical pathology workflows, it needs to be integrated into a tool that enables inspecting regions, creating annotations, and providing quantitative analyses of biomarkers. Therefore, we integrate the trained student model from the previous chapter into the QuPath open‑source platform \cite{Bankhead_Loughrey_etal._2017}. QuPath provides the required annotation, visualization, and analysis tools to interpret complex histological data, including workflows for cell segmentation, classification, and quantification (\hyperref[fig:fig7]{Figure 7}). 

\begin{figure}[h!]
    \centering
    \includegraphics[width=\textwidth]{images/Figure_7.pdf}
    \caption{Visualization of model-generated cell quantification annotations (left) and the corresponding unannotated slide (right) in QuPath}
    \label{fig:fig7}
\end{figure}

To identify the regions in a WSI critical for prognosticating tumor development, such as specific tumor areas or border regions without overlapping healthy tissue, the pathologist uses QuPath to outline these regions. Then, the pathologist initiates a cell segmentation and classification script through the QuPath interface for the selected regions. The resulting annotations and quantified cell information are then directly overlaid onto the WSI in the QuPath interface. Additional design and implementation details are in \hyperref[chap:S7]{Appendix S7}. 

Two common approaches for integrating deep learning models into QuPath are Java‑based native QuPath extensions \cite{Goldsborough_Philps_etal._2024} and the execution of RESTful API requests to a model server coupled with handling the response via an extension, as demonstrated in the application of cell segmentation models applied to immunofluorescence images \cite{Sugawara_2023}. While the community is actively working on these integration strategies, there is currently no universal solution that fully addresses all integration and performance requirements.

Extensions may offer better integration with QuPath, allowing slightly improved performance and more widespread usage of the built-in QuPath models, but they lack the flexibility to customize models and modify their behavior. For example, the newest version of QuPath includes models such as StarDist \cite{Weigert_Schmidt} and InstanSeg \cite{Goldsborough_Philps_etal._2024} that can perform cell segmentation. Both models pose limitations when applied to simultaneous cell segmentation and classification. StarDist performs well only on convex, round shapes by design, whereas some neoplastic, inflammatory, and connective cells exhibit complex and non-convex shapes. InstanSeg provides only semantic segmentation without assigning classes to the segmented cells.

%\hfill

In contrast, our approach offers an alternative integration strategy. It utilizes the paquo library to directly interact with QuPath’s internal application programming interface from within Python. This enables data exchange and processing without the need for intermediate conversion steps and provides greater control over model customization, retraining, and the incorporation of custom processing steps.

The integration of our custom model with QuPath underscores its potential to significantly enhance the diagnostic process by reducing the time burden on pathologists and enabling them to focus on more complex interpretative tasks using familiar software. Leveraging a tool that is already well-established among pathologists increases the likelihood of its adoption into daily clinical workflows. The quantitative data generated through the automated workflow is critical for both clinical decision-making and research, facilitating more accurate biomarker analysis, enabling robust statistical evaluations, and supporting hypothesis generation and testing. Additionally, by streamlining cell segmentation and classification, the tool enhances the scalability and reproducibility of pathological assessments, ultimately contributing to improved diagnostic accuracy and patient outcomes.

\section{Conclusion and future work}

In this study, we address critical challenges in digital pathology and tackle the usability and deployment issues of the developed models in standard computing environments without the need for high-performance computing systems. Our multi-faceted approach encompasses data refinement through cross-relabeling, leveraging foundation models for robust cell segmentation and classification, optimizing model performance via knowledge distillation, and integrating the optimized model into the QuPath software for practical application. This approach is used to construct a capable, versatile, and adjustable model for cell segmentation and classification, with enhanced performance and usability.

\begin{sloppypar}
While our approach shows potential in the field of computational pathology, certain limitations persist. 
For example, our implementation currently exhibits lower performance in detecting macrophages. 
This serves as an instance of the broader challenge of accurately identifying complex cell types. In order to address this issue, extending our approach to incorporate additional data sources, exploring alternative modeling approaches, and integrating other imaging modalities such as immunohistochemical staining may help improve detection accuracy. Moreover, although the distilled model reduces computational demands, integrating advanced deep learning models into clinical practice requires addressing technological gaps and potential resistance to adopting new tools within established diagnostic processes.
\end{sloppypar}

Future work could focus on several key areas to refine the proposed approach and facilitate its adoption in clinical environments. Enhancing the cell-relabeling process with additional datasets \cite{Graham_Jahanifar_etal._2021} could improve the representation of underrepresented cell types and enhance overall model performance. Also, incorporating additional data sources, such as multi-modal imaging or complementary staining methods, may address limitations related to cell type differentiation and class imbalance. Exploring other foundation models \cite{Vorontsov_Bozkurt_etal._2024,Zimmermann_Vorontsov_etal._2024} or introducing additional modalities \cite{Ding_Wagner_etal._2024,Vaidya_Zhang_etal._2025} may provide alternative architectures better suited to specific tasks or offer improved efficiency. Implementing more complex knowledge distillation techniques \cite{Houyon_Cioppa_etal._2023,Zhang_Song_etal._2019} could further optimize the model's performance and adaptability. Additionally, deeper integration with QuPath or other digital pathology software could provide pathologists more control over cell quantification analysis directly within the QuPath interface, thereby increasing accessibility and usability. Such enhancements would not only refine model performance but also ensure greater adaptability and scalability within various clinical environments. Finally, extensive validation of the model by pathologists and benchmarking against independent datasets are essential steps toward establishing the model's reliability and fostering confidence in its clinical utility.

\section*{Acknowledgments} 
This work was funded in part by the Research Council of Norway grant no. 309439 SFI Visual Intelligence, and the North Norwegian Health Authority grant no. HNF1521-20.

\bibliographystyle{IEEEtran}
\begin{sloppypar}
\begin{thebibliography}{99}

\bibitem{chaplot2020neural} Chaplot, Devendra Singh, et al. "Neural topological slam for visual navigation." Proceedings of the IEEE/CVF conference on computer vision and pattern recognition. 2020.

\bibitem{maksymets2021thda} Maksymets, Oleksandr, et al. "Thda: Treasure hunt data augmentation for semantic navigation." Proceedings of the IEEE/CVF International Conference on Computer Vision. 2021.

\bibitem{mezghan2022memory} Mezghan, Lina, et al. "Memory-augmented reinforcement learning for image-goal navigation." 2022 IEEE/RSJ International Conference on Intelligent Robots and Systems (IROS). IEEE, 2022.

\bibitem{al2022zero} Al-Halah, Ziad, Santhosh Kumar Ramakrishnan, and Kristen Grauman. "Zero experience required: Plug \& play modular transfer learning for semantic visual navigation." Proceedings of the IEEE/CVF Conference on Computer Vision and Pattern Recognition. 2022.

\bibitem{ye2021auxiliary} Ye, Joel, et al. "Auxiliary tasks and exploration enable objectgoal navigation." Proceedings of the IEEE/CVF international conference on computer vision. 2021.

\bibitem{chaplot2020object} Chaplot, Devendra Singh, et al. "Object goal navigation using goal-oriented semantic exploration." Advances in Neural Information Processing Systems 33 (2020)

\bibitem{ramakrishnan2022poni} Ramakrishnan, Santhosh Kumar, et al. "Poni: Potential functions for objectgoal navigation with interaction-free learning." Proceedings of the IEEE/CVF Conference on Computer Vision and Pattern Recognition. 2022.

\bibitem{ramrakhya2022habitat} Ramrakhya, Ram, et al. "Habitat-web: Learning embodied object-search strategies from human demonstrations at scale." Proceedings of the IEEE/CVF Conference on Computer Vision and Pattern Recognition. 2022.

\bibitem{mousavian2019visual} Mousavian, Arsalan, et al. "Visual representations for semantic target driven navigation." 2019 International Conference on Robotics and Automation (ICRA). IEEE, 2019.

\bibitem{dhariwal2021diffusion} Dhariwal, Prafulla, and Alexander Nichol. "Diffusion models beat gans on image synthesis." Advances in neural information processing systems 34 (2021)

\bibitem{ho2022classifier} Ho, Jonathan, and Tim Salimans. "Classifier-free diffusion guidance." arXiv preprint arXiv:2207.12598 (2022).

\bibitem{nichol2021glide} Nichol, Alex, et al. "Glide: Towards photorealistic image generation and editing with text-guided diffusion models." arXiv preprint arXiv:2112.10741 (2021)

\bibitem{brooks2023instructpix2pix} Brooks, Tim, Aleksander Holynski, and Alexei A. Efros. "Instructpix2pix: Learning to follow image editing instructions." Proceedings of the IEEE/CVF Conference on Computer Vision and Pattern Recognition. 2023.

\bibitem{fu2023guiding} Fu, Tsu-Jui, et al. "Guiding instruction-based image editing via multimodal large language models." arXiv preprint arXiv:2309.17102 (2023).

\bibitem{geng2024instructdiffusion} Geng, Zigang, et al. "Instructdiffusion: A generalist modeling interface for vision tasks." Proceedings of the IEEE/CVF Conference on Computer Vision and Pattern Recognition. 2024.

\bibitem{zhou2024minedreamer} Zhou, Enshen, et al. "Minedreamer: Learning to follow instructions via chain-of-imagination for simulated-world control." arXiv preprint arXiv:2403.12037 (2024).

\bibitem{zhou2023esc} Zhou, Kaiwen, et al. "Esc: Exploration with soft commonsense constraints for zero-shot object navigation." International Conference on Machine Learning. PMLR, 2023.

\bibitem{yu2023l3mvn} Yu, Bangguo, Hamidreza Kasaei, and Ming Cao. "L3mvn: Leveraging large language models for visual target navigation." 2023 IEEE/RSJ International Conference on Intelligent Robots and Systems (IROS). IEEE, 2023.

\bibitem{gadre2023cows} Gadre, Samir Yitzhak, et al. "Cows on pasture: Baselines and benchmarks for language-driven zero-shot object navigation." Proceedings of the IEEE/CVF Conference on Computer Vision and Pattern Recognition. 2023.

\bibitem{shah2023navigation} Shah, Dhruv, et al. "Navigation with large language models: Semantic guesswork as a heuristic for planning." Conference on Robot Learning. PMLR, 2023.

\bibitem{cai2024bridging} Cai, Wenzhe, et al. "Bridging zero-shot object navigation and foundation models through pixel-guided navigation skill." 2024 IEEE International Conference on Robotics and Automation (ICRA). IEEE, 2024.

\bibitem{yu2023co} Yu, Bangguo, Hamidreza Kasaei, and Ming Cao. "Co-NavGPT: Multi-robot cooperative visual semantic navigation using large language models." arXiv preprint arXiv:2310.07937 (2023).

\bibitem{wu2024voronav} Wu, Pengying, et al. "Voronav: Voronoi-based zero-shot object navigation with large language model." arXiv preprint arXiv:2401.02695 (2024).

\bibitem{qin2023mp5} Qin, Yiran, et al. "Mp5: A multi-modal open-ended embodied system in minecraft via active perception." arXiv preprint arXiv:2312.07472 (2023).

\bibitem{du2024learning} Du, Yilun, et al. "Learning universal policies via text-guided video generation." Advances in Neural Information Processing Systems 36 (2024).

\bibitem{ajay2024compositional} Ajay, Anurag, et al. "Compositional foundation models for hierarchical planning." Advances in Neural Information Processing Systems 36 (2024).

\bibitem{liang2024skilldiffuser} Liang, Zhixuan, et al. "Skilldiffuser: Interpretable hierarchical planning via skill abstractions in diffusion-based task execution." Proceedings of the IEEE/CVF Conference on Computer Vision and Pattern Recognition. 2024.

\bibitem{heusel2017gans} Heusel, Martin, et al. "Gans trained by a two time-scale update rule converge to a local nash equilibrium." Advances in neural information processing systems 30 (2017).

\bibitem{zhang2018unreasonable} Zhang, Richard, et al. "The unreasonable effectiveness of deep features as a perceptual metric." Proceedings of the IEEE conference on computer vision and pattern recognition. 2018.

\bibitem{brown2020language} Brown, Tom B. "Language models are few-shot learners." arXiv preprint arXiv:2005.14165 (2020).

\bibitem{podell2023sdxl} Podell, Dustin, et al. "Sdxl: Improving latent diffusion models for high-resolution image synthesis." arXiv preprint arXiv:2307.01952 (2023).

\bibitem{brohan2022rt} Brohan, Anthony, et al. "Rt-1: Robotics transformer for real-world control at scale." arXiv preprint arXiv:2212.06817 (2022).

\bibitem{brohan2023rt} Brohan, Anthony, et al. "Rt-2: Vision-language-action models transfer web knowledge to robotic control." arXiv preprint arXiv:2307.15818 (2023).

\bibitem{li2024manipllm} Li, Xiaoqi, et al. "Manipllm: Embodied multimodal large language model for object-centric robotic manipulation." Proceedings of the IEEE/CVF Conference on Computer Vision and Pattern Recognition. 2024.

\bibitem{shah2023vint} Shah, Dhruv, et al. "ViNT: A foundation model for visual navigation." arXiv preprint arXiv:2306.14846 (2023).

\bibitem{liu2024visual} Liu, Haotian, et al. "Visual instruction tuning." Advances in neural information processing systems 36 (2024).

\bibitem{hu2021lora} Hu, Edward J., et al. "Lora: Low-rank adaptation of large language models." arXiv preprint arXiv:2106.09685 (2021).

\bibitem{qin2023supfusion} Qin, Yiran, et al. "SupFusion: Supervised LiDAR-camera fusion for 3D object detection." Proceedings of the IEEE/CVF International Conference on Computer Vision. 2023.

\bibitem{qin2024worldsimbench} Qin, Yiran, et al. "Worldsimbench: Towards video generation models as world simulators." arXiv preprint arXiv:2410.18072 (2024).

\bibitem{yu2025gamefactory} Yu, Jiwen, et al. "GameFactory: Creating New Games with Generative Interactive Videos." arXiv preprint arXiv:2501.08325 (2025).

\bibitem{zhou2024code} Zhou, Enshen, et al. "Code-as-Monitor: Constraint-aware Visual Programming for Reactive and Proactive Robotic Failure Detection." arXiv preprint arXiv:2412.04455 (2024).

\bibitem{zhang2024ad} Zhang, Zaibin, et al. "AD-H: Autonomous Driving with Hierarchical Agents." arXiv preprint arXiv:2406.03474 (2024).

\bibitem{wang2024toward} Wang, Chaoqun, et al. "Toward Accurate Camera-based 3D Object Detection via Cascade Depth Estimation and Calibration." arXiv preprint arXiv:2402.04883 (2024).

\bibitem{huang2024story3d} Huang, Yuzhou, et al. "Story3d-agent: Exploring 3d storytelling visualization with large language models." arXiv preprint arXiv:2408.11801 (2024).

\bibitem{savinov2018semi} Savinov, Nikolay, Alexey Dosovitskiy, and Vladlen Koltun. "Semi-parametric topological memory for navigation." arXiv preprint arXiv:1803.00653 (2018).

\bibitem{majumdar2022zson} Majumdar, Arjun, et al. "Zson: Zero-shot object-goal navigation using multimodal goal embeddings." Advances in Neural Information Processing Systems 35 (2022): 32340-32352.

\bibitem{yadav2023offline} Yadav, Karmesh, et al. "Offline visual representation learning for embodied navigation." Workshop on Reincarnating Reinforcement Learning at ICLR 2023. 2023.

\bibitem{yadav2023ovrl} Yadav, Karmesh, et al. "Ovrl-v2: A simple state-of-art baseline for imagenav and objectnav." arXiv preprint arXiv:2303.07798 (2023).

\bibitem{sun2024fgprompt} Sun, Xinyu, et al. "FGPrompt: fine-grained goal prompting for image-goal navigation." Advances in Neural Information Processing Systems 36 (2024).

\bibitem{zhu2017target} Zhu, Yuke, et al. "Target-driven visual navigation in indoor scenes using deep reinforcement learning." 2017 IEEE international conference on robotics and automation (ICRA). IEEE, 2017.

\bibitem{koh2024generating} Koh, Jing Yu, Daniel Fried, and Russ R. Salakhutdinov. "Generating images with multimodal language models." Advances in Neural Information Processing Systems 36 (2024).

\bibitem{krantz2022instance} Krantz, Jacob, et al. "Instance-specific image goal navigation: Training embodied agents to find object instances." arXiv preprint arXiv:2211.15876 (2022).

\bibitem{schulman2017proximal} Schulman, John, et al. "Proximal policy optimization algorithms." arXiv preprint arXiv:1707.06347 (2017).

\bibitem{anderson2018evaluation} Anderson, Peter, et al. "On evaluation of embodied navigation agents." arXiv preprint arXiv:1807.06757 (2018).

\bibitem{lin2024navcot} Lin, Bingqian, et al. "NavCoT: Boosting LLM-Based Vision-and-Language Navigation via Learning Disentangled Reasoning." arXiv preprint arXiv:2403.07376 (2024).

\bibitem{NavGPT} Zhou, Gengze, Yicong Hong, and Qi Wu. "Navgpt: Explicit reasoning in vision-and-language navigation with large language models." Proceedings of the AAAI Conference on Artificial Intelligence.

\bibitem{hahn2021no} Hahn, Meera, et al. "No rl, no simulation: Learning to navigate without navigating." Advances in Neural Information Processing Systems 34 (2021): 26661-26673.

\bibitem{li2025t2isafety} Li, Lijun, et al. "T2ISafety: Benchmark for Assessing Fairness, Toxicity, and Privacy in Image Generation." arXiv preprint arXiv:2501.12612 (2025).

\bibitem{an2024agfsync} An, Jingkun, et al. "AGFSync: Leveraging AI-Generated Feedback for Preference Optimization in Text-to-Image Generation." arXiv preprint arXiv:2403.13352 (2024).


\end{thebibliography}
\end{sloppypar}

\clearpage
\beginsupplement
\section*{Appendix}
\renewcommand{\thesubsection}{S\arabic{subsection}}

\subsection{\label{chap:S1}PanNuke and MoNuSAC preprocessing}
The PanNuke dataset comprises a set of 7,901 RGB patches, each with dimensions of $256 \times 256$ pixels, which we set as the standard patch size for our analysis. In contrast, the MoNuSAC dataset encompasses 294 images of heterogeneous dimensions. To standardize the MoNuSAC images with our experiments, we implement a standardization protocol. Specifically, for images exceeding the dimensions of $256 \times 256$ pixels, we segment them into equal-sized patches and apply mirror padding to the remaining portions to avoid information loss at the peripherals. Patches with dimensions less than $128 \times 128$ pixels are excluded from the dataset due to the insufficient resolution to capture relevant cellular details. For patches where either dimension falls between 128 and 256 pixels, we employ upsampling to achieve the standard patch size. As a result, we obtain a total of 2,823 RGB patches derived from the MoNuSAC dataset for subsequent analysis. For additional details on the MoNuSAC data preparation process, refer to the source code \cite{Shvetsov_2025a}.
\clearpage

\subsection{\label{chap:S2}Data usage for the methodology}

\counterwithin{figure}{subsection}
\renewcommand{\thefigure}{S\arabic{subsection}}

\begin{figure}[h!]
    \centering
    \includegraphics[width=\textwidth, height=0.85\textheight, keepaspectratio]{images/A2.pdf}
    \caption{Overview of the methodology for cross-labeling, dataset refinement, and model comparison. (1) Cross-relabeling - training and testing cell classification models, (2) Cross-relabeling - using cell classification models to create refined dataset, (3) Fine-tuning and training models for comparison, (4) Student knowledge distillation with refined dataset}
    \label{fig:S2}
\end{figure}
\clearpage

\subsection{\label{chap:S3}Confusion matrices for classification models}
\counterwithin{figure}{subsection}
\renewcommand{\thefigure}{S\arabic{subsection}.\arabic{figure}}

\begin{figure}[h!]
    \centering
    \includegraphics[width=\textwidth, height=0.4\textheight, keepaspectratio]{images/A3_1.pdf}
    \caption{Confusion matrix for PanNuke trained model}
    \label{fig:S3.1}
\end{figure}

\begin{figure}[h!]
    \centering
    \includegraphics[width=\textwidth, height=0.4\textheight, keepaspectratio]{images/A3_2.pdf}
    \caption{Confusion matrix for MoNuSAC trained model}
    \label{fig:S3.2}
\end{figure}

\clearpage

\subsection{\label{chap:S4}Datasets cell counts}

\counterwithin{table}{subsection}
\renewcommand{\thetable}{S\arabic{subsection}}

\begin{table}[h!]
\renewcommand{\arraystretch}{2.0}
\centering
\caption{\label{tab:S4}Cell counts for PanNuke, MoNuSAC and refined datasets. Numbers in parentheses indicate preprocessed cell counts for cell classifier models training and testing.}
%\adjustbox{max width=\textwidth}{%
\begin{tabular}{|l|c|c|c|}
\hline
%\rowcolor{gray!30}
Cell type & PanNuke & MoNuSAC & Refined \\
\hline
Neoplastic & 77,403 (68,031) & - & 105,451 \\
\hline
Epithelial & 26,572 (23,207) & - & 29,926 \\
\hline
Epithelial (benign and malignant) & - & 31,402 & - \\
\hline
Inflammatory & 32,276 & - & - \\
\hline
Lymphocytes & - & 37,045 (33,104) & 65,275 \\
\hline
Neutrophils & - & 1,355 (1,252) & 3,833 \\
\hline
Macrophage & - & 1,842 (1,695) & 3,410 \\
\hline
Dead & 2,908 & - & 2,908 \\
\hline
Connective & 50,585 & - & 50,585 \\
\hline
\end{tabular}
%
%}
\end{table}



\clearpage

\subsection{\label{chap:S5}Definition of validation metrics}
\counterwithin{equation}{subsection}
\renewcommand{\theequation}{\arabic{equation}}

\subsubsection{\label{chap:S5.1}R\textsuperscript{2}}
The coefficient of determination, denoted as $R^2$, is a statistical measure that represents the proportion of variance in the dependent variable that is predictable from the independent variables. In the context of cell quantification in pathology, $R^2$ is used to assess how well the predicted quantities of different cell types in a patch align with the actual quantities observed in the ground truth data, with higher values representing more accurate quantification. $R^2$ is defined as
\begin{equation*}
R^2 = 1 - \frac{\sum_{i=1}^n (y_i - \hat{y}_i)^2}{\sum_{i=1}^n (y_i - \bar{y})^2},
\end{equation*}
where $y_i$ represents the actual number of cells of a specific type in the $i$-th image, $\hat{y}_i$ represents the predicted number of cells of that type in the $i$-th image, $\bar{y}$ is the mean of the actual numbers across all images, and $n$ is the total number of images in the dataset.

The $R^2$ metric has a range of $(-\infty, 1]$. An $R^2$ of 1 indicates perfect prediction, where all predicted values exactly match the actual values. An $R^2$ of 0 suggests that the model explains none of the variability of the response data around its mean. If $R^2$ is negative, it indicates that the model performs worse than a model that simply predicts the mean of the actual values for all observations.

\subsubsection{\label{chap:S5.2}PQ}
Panoptic Quality ($PQ$) is a comprehensive metric used to evaluate the performance of segmentation models in tasks that require both instance segmentation and classification. $PQ$ provides a single score that encapsulates both the detection accuracy (i.e., how many objects were correctly identified) and the segmentation quality (i.e., how accurately the objects' boundaries were delineated). This metric is particularly useful in multiclass scenarios where each pixel is classified into distinct categories, such as different cell types in pathology images.

$PQ$ is calculated as the product of two terms: Detection Quality ($DQ$) and Segmentation Quality ($SQ$). It can be expressed as
\begin{equation*}
PQ = DQ \cdot SQ,
\end{equation*}
where
\begin{equation*}
DQ = \frac{TP}{TP + 0.5\, FP + 0.5\, FN},
\end{equation*}
\begin{equation*}
SQ = \frac{\sum_{(p, g) \in \mathcal{M}} IoU(p, g)}{TP}.
\end{equation*}
In these formulas, $TP$ denotes the number of correctly matched instances between ground truth and prediction, $FP$ denotes the predicted instances that have no corresponding ground truth, $FN$ denotes the ground truth instances that were not detected, $IoU(p, g)$ is the Intersection over Union for a pair of matched instances $p$ (prediction) and $g$ (ground truth), and $\mathcal{M}$ is the set of matched pairs.

The $PQ$ metric is calculated for each class and is averaged across classes to provide a global performance measure.

The $PQ$ score has a range of $[0, 1.0]$, where a higher score indicates better performance in both detecting and segmenting the instances correctly. A $PQ$ of 1 signifies perfect identification and segmentation of all instances, whereas a $PQ$ of 0 indicates that no instances were correctly identified and segmented.

\clearpage

\subsection{\label{chap:S6}Segmentation and Detection quality metrics for teacher and student models}

\begin{table}[h!]
\renewcommand{\arraystretch}{2.0}
\centering
\caption{Segmentation and detection quality for student and teacher models (CI 95\%)}
\label{tab:S6}
%\adjustbox{max width=\textwidth}{%
\begin{tabular}{|l|c|c|}
\hline
%\rowcolor{gray!30}
Metric & Teacher & Student \\
\hline
$SQ_{neoplastic}$ & 0.819 (0.815--0.823) & 0.824 (0.819--0.828) \\
\hline
$SQ_{lymphocyte}$ & 0.795 (0.788--0.802) & 0.790 (0.783--0.796) \\
\hline
$SQ_{connective}$ & 0.770 (0.762--0.776) & 0.780 (0.772--0.786) \\
\hline
$SQ_{dead}$ & 0.659 (0.623--0.688) & 0.657 (0.624--0.695) \\
\hline
$SQ_{epithelial}$ & 0.780 (0.770--0.790) & 0.788 (0.779--0.797) \\
\hline
$SQ_{macrophage}$ & 0.788 (0.760--0.810) & 0.757 (0.730--0.783) \\
\hline
$SQ_{neutrofil}$ & 0.782 (0.761--0.801) & 0.775 (0.759--0.792) \\
\hline
$DQ_{neoplastic}$ & 0.706 (0.692--0.719) & 0.727 (0.712--0.741) \\
\hline
$DQ_{lymphocyte}$ & 0.675 (0.656--0.698) & 0.713 (0.691--0.734) \\
\hline
$DQ_{connective}$ & 0.566 (0.546--0.584) & 0.583 (0.565--0.602) \\
\hline
$DQ_{dead}$ & 0.410 (0.361--0.465) & 0.435 (0.306--0.561) \\
\hline
$DQ_{epithelial}$ & 0.668 (0.639--0.694) & 0.673 (0.644--0.702) \\
\hline
$DQ_{macrophage}$ & 0.657 (0.583--0.727) & 0.615 (0.531--0.703) \\
\hline
$DQ_{neutrofil}$ & 0.691 (0.625--0.753) & 0.729 (0.679--0.778) \\
\hline
\end{tabular}
%
%}
\end{table}

\clearpage

\subsection{\label{chap:S7}QuPath integration method}
We adopt an integration strategy leveraging the paquo \cite{Bayer_AG} library, a Python package that enables direct interaction with QuPath’s internal API, thereby facilitating seamless data exchange without intermediate conversion steps. The data processing pipeline (\hyperref[fig:S7]{Appendix Figure S7}) begins with the acquisition of WSIs and their associated annotations from QuPath, which are represented as Shapely \cite{Gillies_Wel_etal._2024} polygons. Utilizing paquo, we directly read, create, and modify these annotations and detections within a QuPath project in the Python environment. Images are then cropped using these polygons and processed by cell segmentation and classification models employing standard vision processing toolkits such as OpenCV, pyvips, and PyTorch. Additionally, QuPath employs Groovy scripts to initiate a Python process that starts the entire pipeline from QuPath graphical interface: fetching polygons, extracting images from them, and running deep learning model inference on the cropped images. 
The results are returned to QuPath, leveraging paquo's Python bindings to manipulate QuPath data while minimizing the computational overhead typically associated with cross-environment communication.

\counterwithin{figure}{subsection}
\renewcommand{\thefigure}{S\arabic{subsection}}

\begin{figure}[h!]
    \centering
    \includegraphics[width=\textwidth]{images/A7.pdf}
    \caption{QuPath integration workflow using Python environment}
    \label{fig:S7}
\end{figure}

Compared to traditional workflows that involve exporting annotations as GeoJSON, classifying them in Python, and reimporting them into QuPath, our approach offers several advantages. We eliminate the need to switch between programming languages, providing a cohesive and streamlined development process entirely within QuPath software and removing the necessity to use other tools. Meanwhile, we avoid storing annotations as intermediate JSON files unless required for external use or archiving. By conducting the entire inference and post-processing workflow within the Python environment, we leverage the power and flexibility of Python libraries for image processing and machine learning. This approach also enables adjustments to any set of labels and models, thereby improving its applicability.

%\hfill

The distilled model and QuPath integration code are packaged into a Docker container, enabling streamlined execution with the Docker engine. Detailed integration code and deployment instructions can be found in the GitHub repository \cite{Shvetsov_2025b}.

Despite these benefits, we acknowledge that the paquo library is a proof‑of‑concept project in its early development stage and has not been tested across all versions of QuPath.

\clearpage

\subsection{\label{chap:S8}Data and code availability statement}
All datasets, models, and code used in this study are publicly available and can be obtained from the repositories listed below. 
The PanNuke \cite{Gamper_Koohbanani_etal._2019} and MoNuSAC \cite{Verma_Kumar_etal._2021} datasets are publicly accessible, and download information along with detailed descriptions can be found in their respective articles. Preprocessing scripts for PanNuke and MoNuSAC data, as well as individual cell extraction scripts, are available on GitHub \cite{Shvetsov_2025a}. The H-Optimus foundation model used in our experiments can be downloaded from the HuggingFace repository \cite{hoptimus2024}, and model information is available on GitHub \cite{Saillard_Jenatton_etal._2024}. In addition, the integration code for QuPath and the distilled model packaged in a Docker container are provided in the repository \cite{Shvetsov_2025b}, and paquo Python library is available from the authors GitHub repository \cite{Bayer_AG}.
\clearpage

\end{document}

\appendix 
\supptitle

% \section{Sequential Setup}
% \subsection{Previous Setup}
% \label{app:prev_setup}
% It aims to predict the next item based on a fixed snapshot of the user's past interactions. Formally, given the dataset $\mathfrak{D}_u$, we randomly pick an index $t$ where $1 \leq t \leq k$ and remove the corresponding item $i_u^t$ from the dataset. The modified dataset is $\mathfrak{D}'_u =  \{\mathbf{R}'_u, \mathbf{I}'_u \}$, where $\mathbf{R}'_u = \{r_u^1, \cdot\cdot\cdot, r_u^k \} \setminus \{r_u^t \}$, $\mathbf{I}'_u = \{i_u^1, \cdot\cdot\cdot, i_u^k \} \setminus \{ i_u^t\}$. The recommender is then tasked with recovering $i_u^t$ from the candidate set $\mathcal{C}_u^t$. 
% \subsection{Our Setup}
% \label{app:our_setup}
% It considers the temporal order of interactions by incrementally updating the available user history. Given the dataset $\mathfrak{D}_u$, the model sequentially observes each interaction and predicts each timestep $t$, where $1 \leq t \leq k$. Formally, at each step $t$, the recommender is provided with $\mathfrak{D}_u^t = \{\mathbf{R}_u^t, \mathbf{I}_u^t \}$, where $\mathbf{R}_u^t = \{r_u^1, \cdot\cdot\cdot, r_u^t\}$, $\mathbf{I}_u^t = \{i_u^1, \cdot\cdot\cdot, i_u^t \}$. The task is to predict the next item $i_u^{t+1}$ from the candidate set $\mathcal{C}_u^{t+1}$. This approach incorporates sequential dependencies by dynamically updating the recommendation model as new interactions occur, and it is more practical.


\setlength{\tabcolsep}{2pt}
\begin{table*}[t]
  \centering
  \scriptsize
  % \resizebox{1.0\linewidth}{!}{
  \begin{tabular}{@{}lll@{}}
    \toprule
    Method     &         Type         &     \multicolumn{1}{c}{Contents} \\ \cmidrule(lr){1-3}
    
    \multirow{3}{*}{\makecell[c]{Baselines}} 
    & \makecell[l]{\emph{Recommender} \\ \textbf{Input}}                                             & \makecell[l]{I've purchased the following products in chronological order: \{\textbf{user-item interactions \&\ reviews}\} 
    \\Then if I ask you to recommend a new product to me according to the given purchasing history, \\you should recommend \textbf{\{recent item\}} and now that I've just purchased \textbf{\{recent item\}}. 
    \\There are 20 candidate products that I can consider to purchase next: \textbf{\{20 candidate items\}}
    \\Please rank these 20 products by measuring the possibilities that I would like to purchase next most, \\according to the given purchasing records. Please think step by step. 
    \\Please show me your ranking results with order numbers. Split your output with line break. \\You MUST rank the given candidate product. You cannot generate products that are not in the given candidate list. \\No other description is needed. }\\ \cmidrule(lr){2-3}
    & \makecell[l]{\emph{Recommender} \\ \textbf{Output}}                                           & \makecell[l]{{[20 ordered items]}}\\ \cmidrule(lr){1-3}


    
   \multirow{52}{*}{\makecell[c]{\myalg{} \\ \textbf{(Ours)}}}          
    &  \makecell[l]{\emph{Review Extractor} \\ \textbf{Input}}     
    &        \makecell[l]{
    I purchased the following products in chronological order: \textbf{\{user-item interactions \&\ reviews\}} 
    \\Then if I ask you to recommend a new product to me according to the given purchasing history, you should recommend \textbf{\{recent item\}} \\ and now I've just purchased \textbf{\{recent item\}}.
    \\And I left review: \textbf{\{recent item review\}}
    \\ Your task is to analyze user's purchasing behavior and extract user's likes, dislikes and key features from the input review. 
    \\Response only likes/dislikes/key features in descriptive form. Please prioritize the most recent item \textbf{\{recent item\}} \\when analyzing likes/dislikes/key features.
    \\Split likes, dislikes, and key features and response in same format.}\\ \cmidrule(lr){2-3}
    &  \makecell[l]{\emph{Review Extractor} \\ \textbf{Output}}      
    &        \makecell[l]{\textbf{Likes}: \{[`*Long gameplay experience(50-60 hours), \colorbox{green}{*Responsive controls}, \colorbox{green}{*Fantastic storyline}, *Challenging puzzles, \\ *Emotional resonance (e.g.remorse), *Ability to gain new posers by killing enemies', `\colorbox{green}{*Humor and fun in games}, \\\colorbox{green}{*References to the simpsons franchise}, \colorbox{yellow}{*Variety of playable characters (Marge, Lisa, Apu, Bart, and Homer)}, \\ *Ability to drive or walk depending on preference, \colorbox{yellow}{*Great voice acting from the cast members}, \\\colorbox{yellow}{*Presence of key locations from the Simpsons universe (Kwik-E-Mart, Power Plant, Church, etc.)}, \\ *Cool vehicle designs and stats, \colorbox{green}{*Fantastic game overall}']\}\\ \textbf{Dislikes}: \{[`*No pause time when selecting a weapon, making the player vulnerable, \\\colorbox{yellow}{*Inventory management can be inconvenient, requiring the player to switch to the inventory screen to user gadgets}', \\`\colorbox{green}{*Boring story}, \colorbox{green}{*Not funny}, \colorbox{green}{*Awful weapons}, *Unresponsive controls, *Terrible graphics, *Worse gameplay']\} \\\textbf{Key Features}: \{[`\colorbox{green}{*No in-game loading}, *Fighting mechanics, *Soul-hunger gameplay mechanic, \\\colorbox{yellow}{*Ability to cover up face to hide disfigured jaw}', `\colorbox{yellow}{*New camera system (Devil May Cry position)}, *Redone fighting mechanics, \\ *Playable as both Raziel and Kain, \colorbox{yellow}{*Puzzles with a challenging but fun diffculty level}']\}}\\\cmidrule(lr){2-3}
    &  \makecell[l]{\emph{Profile Updater} \\ \textbf{Input}}      
    &        \makecell[l]{
    You are given a list: \textbf{\{list of likes/dislikes/key features\}}
    \\You have to update this list by removing redundant or overlapping information. Note that crucial information should be preserved.
    \\Please response only a list. No other description is needed.
    }\\ \cmidrule(lr){2-3}
    &  \makecell[l]{\emph{Profile Updater} \\ \textbf{Output}}       
    &        \makecell[l]{\textbf{Likes}: \{[`*Long Gameplay experience (50-60 hours), *Challenging puzzles, *Emotional resonance (e.g.remorse), \\ *Ability to gain new powers by killing enemies', `\colorbox{yellow}{*Variety of playable characters}, \\ *Ability to drive or walk depending on preference, \colorbox{yellow}{*Presence of key locations from the Simpsons universe}, \\\colorbox{yellow}{*Great voice acting}, *Cool vehicle designs and stats']\}\\ \textbf{Dislikes}: \{[`*No pause time when selecting a weapon, making the player vulnerable, \\\colorbox{yellow}{*Inventory management can be inconvenient}', `*Unresponsive controls, *Terrible graphics, *Worse gameplay']\} \\\textbf{Key Features}: \{[`*Fighting mechanics, *Soul-hunger gameplay mechanic, \colorbox{yellow}{*Ability to cover up face}', \\ `\colorbox{yellow}{*New camera system}, *Redone fighting mechanics, *Playable as both Raziel and Kain, \colorbox{yellow}{*Puzzles}']\}}\\ \cmidrule(lr){2-3}
    &  \makecell[l]{\emph{Recommender} \\ \textbf{Input}}                                    &         \makecell[l]{
    \textbf{This is positive aspects from purchase history}: \\\{[`*Long Gameplay experience (50-60 hours), *Challenging puzzles, *Emotional resonance (e.g.remorse), \\ *Ability to gain new powers by killing enemies', `*Variety of playable characters, \\ *Ability to drive or walk depending on preference, *Presence of key locations from the Simpsons universe, \\ *Great voice acting, *Cool vehicle designs and stats']\}
    \\\textbf{This is negative aspects from purchase history}:\\\{[`*No pause time when selecting a weapon, making the player vulnerable, \\ *Inventory management can be inconvenient', `*Unresponsive controls, *Terrible graphics, *Worse gameplay']\}
    \\\textbf{This is key features of products}: \{[`*Fighting mechanics, *Soul-hunger gameplay mechanic, *Ability to cover up face', \\ `*New camera system, *Redone fighting mechanics, *Playable as both Raziel and Kain, *Puzzles']\}
    \\Based on these inputs, your task is to rank 20 candidate products by evaluating their likelihood of being purchased.
    \\Now there are 20 candidate products that I consider to purchase next. Note that there is no specific order for these candidate items.
    \\Please rank the \textbf{\{20 candidate items\}} from 1 to 20. Your task is to rank these products based on the likelihood of purchase.
    \\You cannot generate products that are not in the given candidate list. No other description is needed.
    }\\  \cmidrule(lr){2-3}
    &  \makecell[l]{\emph{Recommender} \\ \textbf{Output}}          
    &        \makecell[l]{\{[20 ordered items]\}}\\                                                             
    \bottomrule
    \end{tabular}
% }
  \caption{\textbf{Qualitative Results: Baselines vs \myalg{}.} Note that \colorbox{green}{green-highlighted boxes} indicate portions removed due to redundancy or overlapping information, while \colorbox{yellow}{yellow-highlighted boxes} represent summarized content where unnecessary modifiers or examples were omitted for conciseness.}
  \label{tab:qual_results}
\end{table*}
\setlength{\tabcolsep}{6pt}


\section{Prompt Template}
\label{app:template}

\subsection{Extractor $\mathcal{E}$}
The extractor $\mathcal{E}$ aims to extract the user representations from reviews. Here is the prompt template. 

\begin{tcolorbox}[fonttitle=\small\bfseries,
fontupper=\scriptsize\sffamily,
fontlower=\fon{put},
enhanced,
left=2pt, right=2pt, top=2pt, bottom=2pt,
title=Prompt template for Extractor $\mathcal{E}$]
\begin{lstlisting}[]
I purchased the following products and left 
reviews in chronological order: {input_reviews}
Analyze user's likes/dislikes/key features by 
referring to their reviews.
\end{lstlisting}
\end{tcolorbox}



\subsection{Profile Updater $\mathcal{U}$}
The purpose of the profile updater $\mathcal{U}$ is to remove the redundant information in the user profile. As such, the prompt template is designed as below: 

\begin{tcolorbox}[fonttitle=\small\bfseries,
fontupper=\scriptsize\sffamily,
fontlower=\fon{put},
enhanced,
left=2pt, right=2pt, top=2pt, bottom=2pt,
title=Prompt template for User Profile Updater $\mathcal{U}$]
\begin{lstlisting}[]
You are given a list: {list}
Update this list by removing redundant or
overlapping information. Note that crucial 
information should be preserved.
\end{lstlisting}
\end{tcolorbox}


\subsection{Recommender $\mathcal{R}$}
Due to utilizing both item interactions and user profile, prompt can be constituted of various components. Below one is the prompt template of the recommender.  
\begin{tcolorbox}[fonttitle=\small\bfseries,
fontupper=\scriptsize\sffamily,
fontlower=\fon{put},
enhanced,
left=2pt, right=2pt, top=2pt, bottom=2pt,
title=Prompt template for Recommender $\mathcal{R}$]
\begin{lstlisting}[]
Positive aspects: {likes}
Negative aspects: {dislikes}
Key Features: {key_features}
Based on these inputs, rank the {candidate_list}
from 1 to 20 by evaluating their likelihood of 
being purchased.
\end{lstlisting}
\end{tcolorbox}

\section{Dataset}
\label{app:dataset}
Amazon Review Dataset~\cite{ni2019justifying} contains product reviews and metadata from Amazon, including 142.8 million reviews spanning May 1996 -- July 2014. Specifically, this dataset includes reviews (ratings, text, helpfulness votes), product metadata (descriptions, category information, price, brand, and image features), and links (also viewed/also bought graphs). Among them, we selected two domain datasets (Video Games and Movies \&\ TV), and we utilized ASIN, product name, rating, and review for each data and sort the reviews chronologically for each user. Here are the specific descriptions for each dataset.

\paragraph{Video Games.} We select about 15K users and 37K items. Following existing studies \cite{kang2018self}, we removed users and items with fewer than 10 interactions.

\paragraph{Movies and TV.} We select about 98K users and 126K items, removing users and items with fewer than 10 interactions as in the Video Games dataset.

\section{Baselines}
\label{app:baseline}

\subsection{User-Item interactions}
\label{app:interaction}
In our experimental setup, the LLM is tasked with predicting the item that a user is likely to purchase at time step $t$. We utilize \textbf{user-item interactions} up to time step ($t$-1) in chronological order and constructed a candidate list consisting of one ground-truth item and 19 non-interacted items as input. Here, time step $t$ refers to the period starting from the user's 4th purchase up to their final purchase $k$.
\paragraph{Sequential.}
We provide the LLM with instructions, supplying only the user-item interactions and the candidate list. The LLM was then tasked with ranking the items in the candidate list based on the likelihood of being purchased at time step $t$.
\paragraph{Recency-Focused.}
In the \emph{sequential} prompt above, we add an instruction to emphasize the most recently purchased item, specifically the item bought at time step ($t$-1). The additional prompt is as follows: \emph{"Note that my most recently purchased item is \{\textbf{recent item}\}."}
\paragraph{In-Context Learning.}
Unlike the previous \emph{sequential} and \emph{recency-focused} prompts, this approach utilize \textbf{user-item interactions} only up to time step ($t$-2) and recently purchased item which is bought at time step ($t$-1) as input. The additional prompt is as follows: \emph{"I've purchased the following products: \{\textbf{user-item interactions}\}, then you should recommend \{\textbf{recent item}\} to me and now that I've bought \{\textbf{recent item}\}."}

\subsection{User-Item interactions \&\ User Reviews}
\label{app:combined}
In this setup, we extend \textbf{user-item interactions} to include both interactions and \textbf{user reviews}. Based on ~\cref{app:interaction}, the $\dagger$ present the results when both \textbf{user-item interactions} and \textbf{user reviews} are used as input.



\section{Qualitative Results}
\label{app:qual}
To validate the effectiveness of each component of \myalg{}, we summarized the qualitative results in~\autoref{tab:qual_results}, which illustrates the entire input/output process for both the baselines and \myalg{} in the sequential recommendation task. We can observe that the Review Extractor first removes irrelevant or uninformative content for the given reviews, while the Profile Updater reduces redundancy and overlapping information in the user profile. As such, we can conclude that \myalg{} reduces the input token size of the recommender system while retaining essential information, making it more memory-efficient and potentially improving overall performance.

% In the \myalg{} section, green-highlighted boxes indicate portions removed due to redundancy or overlapping information, while yellow-highlighted boxes represent summarized content where unnecessary modifiers or examples were omitted for conciseness. In this way, it reduces the token count while retaining crucial information, making it more memory-efficient while maintaining or even improving performance.


% For the baselines, the input consists of user-item interactions and raw unprocessed reviews, which are directly fed into the recommender to generate a list of 20 recommended items. In contrast, \myalg{} starts with the same input as the baseline but does not immediately generate the candidate list. Instead, it extracts the user’s likes, dislikes, and key features from the input. The updater module then refines this information by removing redundant or conflicting content and summarizing it. Finally, the updated likes, dislikes, and key features, along with the candidate list, are used to produce a list of 20 recommended items.




% \section{Example Appendix}
% \label{sec:appendix}

% This is an appendix.
\end{document}
