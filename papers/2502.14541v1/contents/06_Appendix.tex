\appendix 
\supptitle

% \section{Sequential Setup}
% \subsection{Previous Setup}
% \label{app:prev_setup}
% It aims to predict the next item based on a fixed snapshot of the user's past interactions. Formally, given the dataset $\mathfrak{D}_u$, we randomly pick an index $t$ where $1 \leq t \leq k$ and remove the corresponding item $i_u^t$ from the dataset. The modified dataset is $\mathfrak{D}'_u =  \{\mathbf{R}'_u, \mathbf{I}'_u \}$, where $\mathbf{R}'_u = \{r_u^1, \cdot\cdot\cdot, r_u^k \} \setminus \{r_u^t \}$, $\mathbf{I}'_u = \{i_u^1, \cdot\cdot\cdot, i_u^k \} \setminus \{ i_u^t\}$. The recommender is then tasked with recovering $i_u^t$ from the candidate set $\mathcal{C}_u^t$. 
% \subsection{Our Setup}
% \label{app:our_setup}
% It considers the temporal order of interactions by incrementally updating the available user history. Given the dataset $\mathfrak{D}_u$, the model sequentially observes each interaction and predicts each timestep $t$, where $1 \leq t \leq k$. Formally, at each step $t$, the recommender is provided with $\mathfrak{D}_u^t = \{\mathbf{R}_u^t, \mathbf{I}_u^t \}$, where $\mathbf{R}_u^t = \{r_u^1, \cdot\cdot\cdot, r_u^t\}$, $\mathbf{I}_u^t = \{i_u^1, \cdot\cdot\cdot, i_u^t \}$. The task is to predict the next item $i_u^{t+1}$ from the candidate set $\mathcal{C}_u^{t+1}$. This approach incorporates sequential dependencies by dynamically updating the recommendation model as new interactions occur, and it is more practical.


\setlength{\tabcolsep}{2pt}
\begin{table*}[t]
  \centering
  \scriptsize
  % \resizebox{1.0\linewidth}{!}{
  \begin{tabular}{@{}lll@{}}
    \toprule
    Method     &         Type         &     \multicolumn{1}{c}{Contents} \\ \cmidrule(lr){1-3}
    
    \multirow{3}{*}{\makecell[c]{Baselines}} 
    & \makecell[l]{\emph{Recommender} \\ \textbf{Input}}                                             & \makecell[l]{I've purchased the following products in chronological order: \{\textbf{user-item interactions \&\ reviews}\} 
    \\Then if I ask you to recommend a new product to me according to the given purchasing history, \\you should recommend \textbf{\{recent item\}} and now that I've just purchased \textbf{\{recent item\}}. 
    \\There are 20 candidate products that I can consider to purchase next: \textbf{\{20 candidate items\}}
    \\Please rank these 20 products by measuring the possibilities that I would like to purchase next most, \\according to the given purchasing records. Please think step by step. 
    \\Please show me your ranking results with order numbers. Split your output with line break. \\You MUST rank the given candidate product. You cannot generate products that are not in the given candidate list. \\No other description is needed. }\\ \cmidrule(lr){2-3}
    & \makecell[l]{\emph{Recommender} \\ \textbf{Output}}                                           & \makecell[l]{{[20 ordered items]}}\\ \cmidrule(lr){1-3}


    
   \multirow{52}{*}{\makecell[c]{\myalg{} \\ \textbf{(Ours)}}}          
    &  \makecell[l]{\emph{Review Extractor} \\ \textbf{Input}}     
    &        \makecell[l]{
    I purchased the following products in chronological order: \textbf{\{user-item interactions \&\ reviews\}} 
    \\Then if I ask you to recommend a new product to me according to the given purchasing history, you should recommend \textbf{\{recent item\}} \\ and now I've just purchased \textbf{\{recent item\}}.
    \\And I left review: \textbf{\{recent item review\}}
    \\ Your task is to analyze user's purchasing behavior and extract user's likes, dislikes and key features from the input review. 
    \\Response only likes/dislikes/key features in descriptive form. Please prioritize the most recent item \textbf{\{recent item\}} \\when analyzing likes/dislikes/key features.
    \\Split likes, dislikes, and key features and response in same format.}\\ \cmidrule(lr){2-3}
    &  \makecell[l]{\emph{Review Extractor} \\ \textbf{Output}}      
    &        \makecell[l]{\textbf{Likes}: \{[`*Long gameplay experience(50-60 hours), \colorbox{green}{*Responsive controls}, \colorbox{green}{*Fantastic storyline}, *Challenging puzzles, \\ *Emotional resonance (e.g.remorse), *Ability to gain new posers by killing enemies', `\colorbox{green}{*Humor and fun in games}, \\\colorbox{green}{*References to the simpsons franchise}, \colorbox{yellow}{*Variety of playable characters (Marge, Lisa, Apu, Bart, and Homer)}, \\ *Ability to drive or walk depending on preference, \colorbox{yellow}{*Great voice acting from the cast members}, \\\colorbox{yellow}{*Presence of key locations from the Simpsons universe (Kwik-E-Mart, Power Plant, Church, etc.)}, \\ *Cool vehicle designs and stats, \colorbox{green}{*Fantastic game overall}']\}\\ \textbf{Dislikes}: \{[`*No pause time when selecting a weapon, making the player vulnerable, \\\colorbox{yellow}{*Inventory management can be inconvenient, requiring the player to switch to the inventory screen to user gadgets}', \\`\colorbox{green}{*Boring story}, \colorbox{green}{*Not funny}, \colorbox{green}{*Awful weapons}, *Unresponsive controls, *Terrible graphics, *Worse gameplay']\} \\\textbf{Key Features}: \{[`\colorbox{green}{*No in-game loading}, *Fighting mechanics, *Soul-hunger gameplay mechanic, \\\colorbox{yellow}{*Ability to cover up face to hide disfigured jaw}', `\colorbox{yellow}{*New camera system (Devil May Cry position)}, *Redone fighting mechanics, \\ *Playable as both Raziel and Kain, \colorbox{yellow}{*Puzzles with a challenging but fun diffculty level}']\}}\\\cmidrule(lr){2-3}
    &  \makecell[l]{\emph{Profile Updater} \\ \textbf{Input}}      
    &        \makecell[l]{
    You are given a list: \textbf{\{list of likes/dislikes/key features\}}
    \\You have to update this list by removing redundant or overlapping information. Note that crucial information should be preserved.
    \\Please response only a list. No other description is needed.
    }\\ \cmidrule(lr){2-3}
    &  \makecell[l]{\emph{Profile Updater} \\ \textbf{Output}}       
    &        \makecell[l]{\textbf{Likes}: \{[`*Long Gameplay experience (50-60 hours), *Challenging puzzles, *Emotional resonance (e.g.remorse), \\ *Ability to gain new powers by killing enemies', `\colorbox{yellow}{*Variety of playable characters}, \\ *Ability to drive or walk depending on preference, \colorbox{yellow}{*Presence of key locations from the Simpsons universe}, \\\colorbox{yellow}{*Great voice acting}, *Cool vehicle designs and stats']\}\\ \textbf{Dislikes}: \{[`*No pause time when selecting a weapon, making the player vulnerable, \\\colorbox{yellow}{*Inventory management can be inconvenient}', `*Unresponsive controls, *Terrible graphics, *Worse gameplay']\} \\\textbf{Key Features}: \{[`*Fighting mechanics, *Soul-hunger gameplay mechanic, \colorbox{yellow}{*Ability to cover up face}', \\ `\colorbox{yellow}{*New camera system}, *Redone fighting mechanics, *Playable as both Raziel and Kain, \colorbox{yellow}{*Puzzles}']\}}\\ \cmidrule(lr){2-3}
    &  \makecell[l]{\emph{Recommender} \\ \textbf{Input}}                                    &         \makecell[l]{
    \textbf{This is positive aspects from purchase history}: \\\{[`*Long Gameplay experience (50-60 hours), *Challenging puzzles, *Emotional resonance (e.g.remorse), \\ *Ability to gain new powers by killing enemies', `*Variety of playable characters, \\ *Ability to drive or walk depending on preference, *Presence of key locations from the Simpsons universe, \\ *Great voice acting, *Cool vehicle designs and stats']\}
    \\\textbf{This is negative aspects from purchase history}:\\\{[`*No pause time when selecting a weapon, making the player vulnerable, \\ *Inventory management can be inconvenient', `*Unresponsive controls, *Terrible graphics, *Worse gameplay']\}
    \\\textbf{This is key features of products}: \{[`*Fighting mechanics, *Soul-hunger gameplay mechanic, *Ability to cover up face', \\ `*New camera system, *Redone fighting mechanics, *Playable as both Raziel and Kain, *Puzzles']\}
    \\Based on these inputs, your task is to rank 20 candidate products by evaluating their likelihood of being purchased.
    \\Now there are 20 candidate products that I consider to purchase next. Note that there is no specific order for these candidate items.
    \\Please rank the \textbf{\{20 candidate items\}} from 1 to 20. Your task is to rank these products based on the likelihood of purchase.
    \\You cannot generate products that are not in the given candidate list. No other description is needed.
    }\\  \cmidrule(lr){2-3}
    &  \makecell[l]{\emph{Recommender} \\ \textbf{Output}}          
    &        \makecell[l]{\{[20 ordered items]\}}\\                                                             
    \bottomrule
    \end{tabular}
% }
  \caption{\textbf{Qualitative Results: Baselines vs \myalg{}.} Note that \colorbox{green}{green-highlighted boxes} indicate portions removed due to redundancy or overlapping information, while \colorbox{yellow}{yellow-highlighted boxes} represent summarized content where unnecessary modifiers or examples were omitted for conciseness.}
  \label{tab:qual_results}
\end{table*}
\setlength{\tabcolsep}{6pt}


\section{Prompt Template}
\label{app:template}

\subsection{Extractor $\mathcal{E}$}
The extractor $\mathcal{E}$ aims to extract the user representations from reviews. Here is the prompt template. 

\begin{tcolorbox}[fonttitle=\small\bfseries,
fontupper=\scriptsize\sffamily,
fontlower=\fon{put},
enhanced,
left=2pt, right=2pt, top=2pt, bottom=2pt,
title=Prompt template for Extractor $\mathcal{E}$]
\begin{lstlisting}[]
I purchased the following products and left 
reviews in chronological order: {input_reviews}
Analyze user's likes/dislikes/key features by 
referring to their reviews.
\end{lstlisting}
\end{tcolorbox}



\subsection{Profile Updater $\mathcal{U}$}
The purpose of the profile updater $\mathcal{U}$ is to remove the redundant information in the user profile. As such, the prompt template is designed as below: 

\begin{tcolorbox}[fonttitle=\small\bfseries,
fontupper=\scriptsize\sffamily,
fontlower=\fon{put},
enhanced,
left=2pt, right=2pt, top=2pt, bottom=2pt,
title=Prompt template for User Profile Updater $\mathcal{U}$]
\begin{lstlisting}[]
You are given a list: {list}
Update this list by removing redundant or
overlapping information. Note that crucial 
information should be preserved.
\end{lstlisting}
\end{tcolorbox}


\subsection{Recommender $\mathcal{R}$}
Due to utilizing both item interactions and user profile, prompt can be constituted of various components. Below one is the prompt template of the recommender.  
\begin{tcolorbox}[fonttitle=\small\bfseries,
fontupper=\scriptsize\sffamily,
fontlower=\fon{put},
enhanced,
left=2pt, right=2pt, top=2pt, bottom=2pt,
title=Prompt template for Recommender $\mathcal{R}$]
\begin{lstlisting}[]
Positive aspects: {likes}
Negative aspects: {dislikes}
Key Features: {key_features}
Based on these inputs, rank the {candidate_list}
from 1 to 20 by evaluating their likelihood of 
being purchased.
\end{lstlisting}
\end{tcolorbox}

\section{Dataset}
\label{app:dataset}
Amazon Review Dataset~\cite{ni2019justifying} contains product reviews and metadata from Amazon, including 142.8 million reviews spanning May 1996 -- July 2014. Specifically, this dataset includes reviews (ratings, text, helpfulness votes), product metadata (descriptions, category information, price, brand, and image features), and links (also viewed/also bought graphs). Among them, we selected two domain datasets (Video Games and Movies \&\ TV), and we utilized ASIN, product name, rating, and review for each data and sort the reviews chronologically for each user. Here are the specific descriptions for each dataset.

\paragraph{Video Games.} We select about 15K users and 37K items. Following existing studies \cite{kang2018self}, we removed users and items with fewer than 10 interactions.

\paragraph{Movies and TV.} We select about 98K users and 126K items, removing users and items with fewer than 10 interactions as in the Video Games dataset.

\section{Baselines}
\label{app:baseline}

\subsection{User-Item interactions}
\label{app:interaction}
In our experimental setup, the LLM is tasked with predicting the item that a user is likely to purchase at time step $t$. We utilize \textbf{user-item interactions} up to time step ($t$-1) in chronological order and constructed a candidate list consisting of one ground-truth item and 19 non-interacted items as input. Here, time step $t$ refers to the period starting from the user's 4th purchase up to their final purchase $k$.
\paragraph{Sequential.}
We provide the LLM with instructions, supplying only the user-item interactions and the candidate list. The LLM was then tasked with ranking the items in the candidate list based on the likelihood of being purchased at time step $t$.
\paragraph{Recency-Focused.}
In the \emph{sequential} prompt above, we add an instruction to emphasize the most recently purchased item, specifically the item bought at time step ($t$-1). The additional prompt is as follows: \emph{"Note that my most recently purchased item is \{\textbf{recent item}\}."}
\paragraph{In-Context Learning.}
Unlike the previous \emph{sequential} and \emph{recency-focused} prompts, this approach utilize \textbf{user-item interactions} only up to time step ($t$-2) and recently purchased item which is bought at time step ($t$-1) as input. The additional prompt is as follows: \emph{"I've purchased the following products: \{\textbf{user-item interactions}\}, then you should recommend \{\textbf{recent item}\} to me and now that I've bought \{\textbf{recent item}\}."}

\subsection{User-Item interactions \&\ User Reviews}
\label{app:combined}
In this setup, we extend \textbf{user-item interactions} to include both interactions and \textbf{user reviews}. Based on ~\cref{app:interaction}, the $\dagger$ present the results when both \textbf{user-item interactions} and \textbf{user reviews} are used as input.



\section{Qualitative Results}
\label{app:qual}
To validate the effectiveness of each component of \myalg{}, we summarized the qualitative results in~\autoref{tab:qual_results}, which illustrates the entire input/output process for both the baselines and \myalg{} in the sequential recommendation task. We can observe that the Review Extractor first removes irrelevant or uninformative content for the given reviews, while the Profile Updater reduces redundancy and overlapping information in the user profile. As such, we can conclude that \myalg{} reduces the input token size of the recommender system while retaining essential information, making it more memory-efficient and potentially improving overall performance.

% In the \myalg{} section, green-highlighted boxes indicate portions removed due to redundancy or overlapping information, while yellow-highlighted boxes represent summarized content where unnecessary modifiers or examples were omitted for conciseness. In this way, it reduces the token count while retaining crucial information, making it more memory-efficient while maintaining or even improving performance.


% For the baselines, the input consists of user-item interactions and raw unprocessed reviews, which are directly fed into the recommender to generate a list of 20 recommended items. In contrast, \myalg{} starts with the same input as the baseline but does not immediately generate the candidate list. Instead, it extracts the user’s likes, dislikes, and key features from the input. The updater module then refines this information by removing redundant or conflicting content and summarizing it. Finally, the updated likes, dislikes, and key features, along with the candidate list, are used to produce a list of 20 recommended items.



