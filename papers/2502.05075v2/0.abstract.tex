\begin{abstract}\vspace{1em}
    Weak-to-strong (W2S) generalization is a type of finetuning (FT) where a strong (large) student model is trained on pseudo-labels generated by a weak teacher. Surprisingly, W2S FT often outperforms the weak teacher. We seek to understand this phenomenon through the observation that FT often occurs in intrinsically low-dimensional spaces.
    Leveraging the low intrinsic dimensionality of FT, we analyze W2S in the ridgeless regression setting from a variance reduction perspective. For a strong student - weak teacher pair with sufficiently expressive low-dimensional feature subspaces $\mathcal{V}_s, \mathcal{V}_w$, we provide an exact characterization of the variance that dominates the generalization error of W2S. This unveils a virtue of discrepancy between the strong and weak models in W2S: the variance of the weak teacher is inherited by the strong student in $\mathcal{V}_s \cap \mathcal{V}_w$, while reduced by a factor of $\dim(\mathcal{V}_s)/N$ in the subspace of discrepancy $\mathcal{V}_w \setminus \mathcal{V}_s$ with $N$ pseudo-labels for W2S. Further, our analysis casts light on the sample complexities and the scaling of performance gap recovery in W2S. The analysis is supported with experiments on synthetic regression and real vision and NLP tasks.
\end{abstract}