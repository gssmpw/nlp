\section{Related work}
\label{sec:related-work}

The preordering problem is formulated and studied from the perspective of computational complexity by \citet{wakabayashi1998complexity} who establishes \textsc{NP}-hardness of the problem itself and many specializations.
The geometry of the preorder polytope is discussed briefly by \citet{gurgel1992poliedros}.
The problem has received less attention so far than more constrained clustering and ordering problems:

Preordering is related to correlation clustering \cite{emanuel2003correlation,bansal2004correlation,demaine2006correlation}, more specifically, to maximum value weighted correlation clustering for complete graphs, which is also known as clique partitioning \cite{grotschel1989cutting}, where values $c_{ij}$ are associated with unordered pairs $\{i,j\}$, and the task is to find an equivalence relation $\sim$ on $V$ so as to maximize the sum of the values of those pairs $\{i,j\}$ for which $i \sim j$.
This problem is more specific than preordering in that equivalence relations are precisely the \emph{symmetric} preorders. 
The complexity and hardness of approximation of correlation clustering have been studied extensively \citep{swamy2004correlation,charikar2005clustering,chawla2015near,veldt2022correlation,cohen2022correlation,klein2023correlation}.
In particular, it is known that clique partitioning cannot be approximated to within $n^{1-\epsilon}$ unless \textsc{P}$=$\textsc{NP} for $n$ the number of elements and any $\epsilon > 0$ \citep{bachrach2013optimal,zuckerman2006linear}.
Much work has been devoted also to the study of the clique partitioning polytope
\cite{grotschel1990facets,grotschel1990composition,deza1991complete,deza1992clique,chopra1995facets,bandelt1999lifting,oosten2001clique,sorensen2002note,letchford2025new}.
Correlation clustering has many applications, including community detection in social networks \citep{brandes2007modularity,veldt2018correlation} and image analysis \citep{yarkony2012fast,beier2015fusion,aflalo2023deepcut,abbas2023clusterfug}.
Variants that emphasize local errors are studied by \citet{puleo2016correlation,kalhan2019correlation,ahmadian2020fair}.

Preordering is related also to partial ordering \citep{muller1996partial} where the task is to find a partial order $\leq$ on $V$ so as to maximize the sum of the values of those pairs $ij$ for which $i \leq j$.
Partial ordering is more specific than preordering in that partial orders are precisely the \emph{antisymmetric} preorders.
The polyhedral geometry of partial orders is studied by \citet{muller1996partial}.
Also this problem has received less attention so far than even more constrained problems, like the linear ordering problem \citep{grotschel1984cutting,marti2011linear,ceberio2015linear}, the closely related rank aggregation problem \citep{ailon2008aggregating,schalekamp2009rank} and the feedback arc set problem on tournament graphs \citep{karpinski2010faster}.

In summary, preordering is a joint relaxation of correlation clustering, from which the symmetry constraint is dropped, and partial ordering, from which the antisymmetry constraint is dropped.
A joint relaxation of correlation clustering and \emph{linear} ordering is the bucket ordering problem, sometimes called the weak ordering problem, that asks for both a clustering and a \emph{linear} order on the clusters \citep{gurgel1997adjacency,fiorini2003combinatorial,fiorini2004weak,fiorini20060}.
This problem has been applied to the tasks of dating archeological sites \citep{gionis2006algorithms}, finding consensus among voters \cite{aledo2018approaching,aledo2021highly} and constructing user recommendations \citep{jurewicz2023catalog}.
