\section{Introduction}
\label{sec:introduction}

In this article we introduce algorithms for jointly clustering and partially ordering data. 
Motivation stems from the problem of estimating a binary relation on the accounts of a social network where the notions that one account \emph{follows, likes} or \emph{reacts to} another account need neither be symmetric nor asymmetric.
In particular, $i$ following $j$ does not need to imply that $j$ follows $i$, nor does it necessarily imply that $j$ does not follow $i$.
Clustering alone does not capture asymmetric subsets of the relation on the social network because the equivalence relation that characterizes the clustering is symmetric.
Similarly, partial ordering alone does not capture symmetric subsets of the relation on the social network because partial orders are antisymmetric.

Like in clustering and partial ordering, we work with the assumption that the relation on the social network is transitive, i.e., if $i$ follows $j$ and $j$ follows $k$ then $i$ follows $k$.
This assumption simplifies reality and we quantify the deviation empirically.
Unlike in clustering and partial ordering, we do not assume symmetry or antisymmetry.
To model our assumption we introduce algorithms that output a \emph{preorder} on a set, i.e., a binary relation on the set that is reflexive and transitive.
More specifically, we introduce algorithms for the \emph{preordering problem}:

\begin{definition}\cite{wakabayashi1998complexity}
\label{definition:preordering-problem}
Given a finite set $V$ and a value $c_{ij} \in \mathbb{R}$ for every pair $ij$ of distinct $i,j \in V$, the \emph{(maximum value) preordering problem} consists in finding a preorder $\lesssim$ on $V$ so as to maximize the sum of the values $c_{ij}$ of those pairs $ij \in V^2$ with $i \neq j$ for which $i \lesssim j$.
\end{definition}
%
Every preorder defines, and is characterized by, a clustering and a partial order:
A clustering is defined by the symmetric subset of the preorder (an equivalence relation);
a strict partial order \emph{on the set of clusters} is well-defined by the asymmetric subset of the preorder.
An example is depicted in \Cref{fig:twitter-example}.

\begin{figure}
    \centering
    \begin{tikzpicture}[scale=0.6]
\pgfdeclarelayer{bg}
\pgfsetlayers{bg,main}
\node[vertex, fill=white, fill opacity=0.8, text opacity=1] (0) at (3.56, 4.1) {\scriptsize 0};
\node[vertex, fill=white, fill opacity=0.8, text opacity=1] (1) at (1.5, 3.91) {\scriptsize 1};
\node[vertex, fill=white, fill opacity=0.8, text opacity=1] (2) at (0.5, 3.37) {\scriptsize 2};
\node[vertex, fill=white, fill opacity=0.8, text opacity=1] (4) at (2.8, 1.21) {\scriptsize 4};
\node[vertex, fill=white, fill opacity=0.8, text opacity=1] (6) at (3.56, 2.29) {\scriptsize 6};
\node[vertex, fill=white, fill opacity=0.8, text opacity=1] (7) at (4.5, 0.2) {\scriptsize 7};
\node[vertex, fill=white, fill opacity=0.8, text opacity=1] (9) at (0.67, 0.2) {\scriptsize 9};
\node[vertex, fill=white, fill opacity=0.8, text opacity=1] (10) at (2.6, 3.5) {\scriptsize 10};
\node[vertex, fill=white, fill opacity=0.8, text opacity=1] (11) at (2.2, 2.6) {\scriptsize 11};
\node[vertex, fill=white, fill opacity=0.8, text opacity=1] (12) at (1.7, 1.3) {\scriptsize 12};
\node[vertex, fill=white, fill opacity=0.8, text opacity=1] (13) at (4.04, 2.83) {\scriptsize 13};
\node[vertex, fill=white, fill opacity=0.8, text opacity=1] (3) at (4.04, 1.75) {\scriptsize 3};
\node[vertex, fill=white, fill opacity=0.8, text opacity=1] (5) at (1.34, 0.67) {\scriptsize 5};
\node[vertex, fill=white, fill opacity=0.8, text opacity=1] (8) at (5.00, 1.21) {\scriptsize 8};
\begin{pgfonlayer}{bg}
\draw[-latex] (0) edge[bend left=10] (1);
\draw[-latex] (0) edge[bend left=10] (2);
\draw[-latex] (0) edge[bend left=10] (4);
\draw[-latex] (0) edge[bend left=10] (6);
\draw[-latex] (0) edge[bend left=10] (7);
\draw[-latex] (0) edge[bend left=10] (9);
\draw[-latex] (0) edge[bend left=10] (10);
\draw[-latex] (0) edge[bend left=10] (11);
\draw[-latex] (0) edge[bend left=10] (12);
\draw[-latex] (1) edge[bend left=10] (2);
\draw[-latex] (1) edge[bend left=10] (4);
\draw[-latex] (1) edge[bend left=10] (13);
\draw[-latex] (2) edge[bend left=10] (0);
\draw[-latex] (2) edge[bend left=10] (1);
\draw[-latex] (2) edge[bend left=10] (3);
\draw[-latex] (2) edge[bend left=10] (4);
\draw[-latex] (2) edge[bend left=10] (5);
\draw[-latex] (2) edge[bend left=10] (6);
\draw[-latex] (2) edge[bend left=10] (7);
\draw[-latex] (2) edge[bend left=10] (8);
\draw[-latex] (2) edge[bend left=10] (9);
\draw[-latex] (4) edge[bend left=10] (0);
\draw[-latex] (4) edge[bend left=10] (1);
\draw[-latex] (4) edge[bend left=10] (2);
\draw[-latex] (4) edge[bend left=10] (5);
\draw[-latex] (4) edge[bend left=10] (6);
\draw[-latex] (4) edge[bend left=10] (8);
\draw[-latex] (4) edge[bend left=10] (9);
\draw[-latex] (4) edge[bend left=10] (10);
\draw[-latex] (4) edge[bend left=10] (11);
\draw[-latex] (6) edge[bend left=10] (2);
\draw[-latex] (6) edge[bend left=10] (4);
\draw[-latex] (6) edge[bend left=10] (7);
\draw[-latex] (7) edge[bend left=10] (2);
\draw[-latex] (7) edge[bend left=10] (12);
\draw[-latex] (9) edge[bend left=10] (0);
\draw[-latex] (9) edge[bend left=10] (2);
\draw[-latex] (9) edge[bend left=10] (4);
\draw[-latex] (9) edge[bend left=10] (8);
\draw[-latex] (10) edge[bend left=10] (0);
\draw[-latex] (10) edge[bend left=10] (4);
\draw[-latex] (10) edge[bend left=10] (11);
\draw[-latex] (11) edge[bend left=10] (10);
\draw[-latex] (3) edge[bend left=10] (2);
\draw[-latex] (5) edge[bend left=10] (2);
\draw[-latex] (5) edge[bend left=10] (4);
\draw[-latex] (5) edge[bend left=10] (7);
\draw[-latex] (8) edge[bend right=10] (0);
\draw[-latex] (8) edge[bend left=10] (2);
\draw[-latex] (8) edge[bend left=10] (4);
\draw[-latex] (8) edge[bend left=10] (9);
\end{pgfonlayer}
\end{tikzpicture}
%
    \hspace{5ex}
    \begin{tikzpicture}
    \def\r{0.18}
    \node[cluster, minimum size=5.500*\r cm] (C0) at (11.000*\r, 9.000*\r) {};
    \node[cluster, minimum size=7.253*\r cm] (C1) at (3.000*\r, 2.000*\r) {};
    \node[cluster, minimum size=3.000*\r cm] (C2) at (3.000*\r, 9.000*\r) {};
    \node[cluster, minimum size=3.000*\r cm] (C3) at (17.000*\r, 9.000*\r) {};
    \node[cluster, minimum size=3.000*\r cm] (C4) at (11.000*\r, 2.000*\r) {};
    \node[vertex, fill=white, shift=(0.000:1.250*\r cm), minimum size=2.0*\r cm] (5) at (C0) {\scriptsize 5};
    \node[vertex, fill=white, shift=(180.000:1.250*\r cm), minimum size=2.0*\r cm] (0) at (C0) {\scriptsize 0};
    \node[vertex, fill=white, shift=(0.000:2.127*\r cm), minimum size=2.0*\r cm] (9) at (C1) {\scriptsize 9};
    \node[vertex, fill=white, shift=(72.000:2.127*\r cm), minimum size=2.0*\r cm] (4) at (C1) {\scriptsize 4};
    \node[vertex, fill=white, shift=(144.000:2.127*\r cm), minimum size=2.0*\r cm] (1) at (C1) {\scriptsize 1};
    \node[vertex, fill=white, shift=(216.000:2.127*\r cm), minimum size=2.0*\r cm] (3) at (C1) {\scriptsize 3};
    \node[vertex, fill=white, shift=(288.000:2.127*\r cm), minimum size=2.0*\r cm] (6) at (C1) {\scriptsize 6};
    \node[vertex, fill=white, shift=(0.000:0.000*\r cm), minimum size=2.0*\r cm] (2) at (C2) {\scriptsize 2};
    \node[vertex, fill=white, shift=(0.000:0.000*\r cm), minimum size=2.0*\r cm] (7) at (C3) {\scriptsize 7};
    \node[vertex, fill=white, shift=(0.000:0.000*\r cm), minimum size=2.0*\r cm] (8) at (C4) {\scriptsize 8};
    \draw[cluster_arc] (C0) -- (C1);
    \draw[cluster_arc] (C0) -- (C4);
    \draw[cluster_arc] (C2) -- (C1);
\end{tikzpicture}
    \caption{Depicted on the left is a Twitter ego-network from \citet{leskovec2012learning}. 
    Depicted on the right is a preorder on the accounts of this network consisting of a clustering (gray circles) and a partial order on the set of clusters (black edges).
	This preorder is an optimal solution to the instance of the preordering problem (\Cref{definition:preordering-problem}) defined such that $c_{ij} = 1$ if the pair $ij$ exists in the network, and $c_{ij} = -1$ otherwise.}
    \label{fig:twitter-example}
\end{figure}

In this article we make the following contributions.
Firstly, we show that preordering remains \textsc{NP}-hard even for values in $\{-1,0,1\}$.
Secondly, we introduce a linear-time $4$-approximation algorithm and a local search technique for the preordering problem.
Thirdly, we consider an integer linear program (ILP) formulation of the preordering problem, establish a class of non-canonical facets of the associated preordering polytope and define a separation algorithm for solving a non-canonical linear program (LP) relaxation; this yields non-trivial upper bounds on the objective value.
Fourthly, we implement the algorithms in the form of open-source code, apply these to real social networks from \citet{fink2023centrality} and \citet{leskovec2012learning} with up to $10^7$ edges and compare the output and efficiency qualitatively and quantitatively.
