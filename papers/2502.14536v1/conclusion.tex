\section{Conclusion}
\label{sec:conclusion}

We discuss the preordering problem, a joint relaxation of the correlation clustering problem and the partial ordering problem.
Toward theory, we show that this problem remains \textsc{NP}-hard even for values in $\{-1,0,1\}$.
For an ILP formulation, we describe facets of the associated preorder polytope induced by odd closed walk inequalities.
Toward algorithms, we introduce, firstly, a linear-time $4$-approximation algorithm that constructs a maximum dicut of a subgraph and, secondly, a local search by greedy arc insertion.
We implement these algorithms, providing open-source code, and we implement also the separation of odd closed walk inequalities for the ILP solver Gurobi, which allows us to solve a non-canonical LP relaxation and thus obtain non-trivial upper bounds on the objective value.
Toward experiments, we apply these algorithms to published social networks and compare the outputs and efficiency.
Quantitatively, we observe that first constructing a 4-approximation in the form of a dicut and then performing a local search by greedy arc insertion initialized with the dicut is faster in combination and closer to optimal than greedy arc insertion initialized with the identity relation.
Qualitatively, we observe that preorders output by our algorithms differ from equivalence relations found by clustering, partial orders found by partial ordering, and preorders found by successive clustering and partial ordering.
We are convinced that preordering will find applications beyond social network analysis.
