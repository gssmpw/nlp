\appendix
\section{Appendix}

\subsection{Discussion of transitivity} \label{sec:discussion-transitivity}

The feasible solutions to the ILP formulation \eqref{eq:objective}--\eqref{eq:triangle} of the preordering problem are precisely the characteristic vectors of the \emph{reflexive reductions} of the preorders on $V$.
There are two main reasons for our preference of this formulation over a version that would use the characteristic vectors of the preorders directly.

Firstly, we are interested in approximating solutions to the preordering problem.
The quality of an approximation algorithm is usually defined as the maximum \emph{multiplicative} error this algorithm commits over all instances of the problem.
By assigning values also to the reflexive pairs and requiring a solution $A \subseteq V^{2}$ to be reflexive and transitive, the total value of every solution includes all values of transitive pairs.
In particular, changing the values of the reflexive pairs merely shifts the values of all solutions by the same amount.
Shifting the values of all solutions by a constant amount does not change the set of optimal solutions.
However, it affects the ratio of values of solutions and thus how well one solution approximates another solution.
Restricting our attention to the reflexive reductions of preorders is further justified by a lemma proven by \citet{wakabayashi1998complexity} stating that the decision version of the problem is \textsc{NP}-complete for the reflexive version if and only if it is \textsc{NP}-complete for the irreflexive version.

Secondly, the convex hull of the characteristic vectors of transitive and reflexive relations is a polytope contained in a $|V_2|$-dimensional subspace of $\mathbb{R}^{V^2}$ (namely the subspace defined by the equalities $x_{ii} = 1$ for $i \in V$).
By restricting the polytope to the space $\mathbb{R}^{V_2}$ (i.e., the convex hull of the feasible solutions to the ILP \eqref{eq:objective}--\eqref{eq:triangle}), this complication is avoided.



\subsection{Proof of \Cref{thm:np-hard}} \label{sec:proof-np-hard}

For any preordering instance $(V, c)$, we define as notational shorthand $V_{2}^{\pm} \coloneqq c^{-1}(\R_{\pm 0})$.
For a feasible solution $A \subseteq V_{2}$ we call $ij \in V_{2}^{-} \cap A$ a \emph{negative mistake} and $ij \in V_{2}^{+} \setminus A$ a \emph{positive mistake}.
\begin{proof}
	We employ a reduction from the maximum transitive subdigraph problem.
	
	\begin{definition}[\textsc{MTSD}-Problem]
		Given are a directed graph $G = (V, E)$ with $E \subseteq V_{2}$ and a positive integer $m \in \N$.
		Decide whether there exists a set $A \subseteq E$ of cardinality at least $m$ such that $G' = (V, A)$ is transitive.
	\end{definition}
	
	The complement of this problem, i.e., the question whether a digraph can be made transitive by at most $\lvert E \rvert - m$ edge deletions, was shown to be \textsc{NP}-complete by \cite{yannakakisEdgeDeletionProblems1981},
	which also implies \textsc{NP}-hardness for the \textsc{MTSD} problem.
	In the \textsc{MTSD}-problem, non-present arcs constitute \emph{hard} constraints, i.e., if $ij, jk \in E$, but $ik \notin E$, then at most one of $ij, jk$ can be present in a transitive subgraph.
	We enforce this behavior also for optimal solutions in the constructed preordering instance by introducing a large structure for every vertex in the original instance and sufficiently many negative arcs between those structures.
	The following lemma will help us to show that optimal preorders then obey the hard constraints of the \textsc{MTSD}-instance.
	
	\begin{lemma}\label{lem:isolating_set_from_preorder}
		Let $V$ be a finite, non-empty set, let $V' \subseteq V$ and let $A \subseteq V_{2}$ be a transitive relation on $V$.
		Then the following relation is transitive:
		\[
			A' \coloneqq A \setminus \left[ \left( (V \setminus V') \times V' \right) \cup \left( V' \times (V \setminus V') \right) \right]
		\]%
	\end{lemma}
	
	\begin{proof}
		Assume $A'$ is not transitive, i.e., $ij, jk \in A'$, but $ik \notin A'$.
		By construction, all arcs connecting $V'$ and $V \setminus V'$ have been removed from the relation.
		Therefore, $ij \in A'$ implies that $i$ and $j$ are either both in $V'$ or both in $V \setminus V'$ and either $j$ and $k$ are both in $V'$ or both are in $V \setminus V'$.
		Therefore, $i, j$ and $k$ must either all be in $V'$ or all in $V \setminus V'$.
		In both cases, no arcs between them have been removed by the construction of $A'$.
		Hence $ij \in A$ and $jk \in A$ while $ik \notin A$, a contradiction.
		\renewcommand{\qedsymbol}{$\blacksquare$}%
	\end{proof}
	
	For a given \textsc{MTSD}-instance $(G=(V,E), m)$, we construct a preordering instance $(\tilde{V}, \tilde{c})$ as follows.
	Let $n \coloneqq \lvert V \rvert$ and let $\tilde{V} \coloneqq \bigcup_{v \in V}^{} \tilde{V}_{v}$ where we define $\tilde{V}_{v} \coloneqq \left\{\tilde{v}_{v,i} \mid i \in \left\{ 0, \ldots, 2n \right\} \right\}$ and call them “positve cliques”.
	For every edge $uv \in E$, we set $\tilde{c}(\tilde{v}_{u,0}, \tilde{v}_{v,0}) \coloneqq +1$ and $\tilde{c}_{ij} \coloneqq 0$ for all other $ij$ in $\tilde{V}_{u} \times \tilde{V}_{v}$.
	For every $uv \in V_{2} \setminus E$ and every $ij \in \tilde{V}_{u} \times \tilde{V}_{v}$ we set $\tilde{c}_{ij} = -1$.

	We claim that for every $m$ there is a set $A \subseteq E$ of cardinality at least $m$ such that $G' = (V, A)$ is transitive
	if and only if there is an optimal solution $\tilde{A}$ to the thusly constructed preordering instance that has objective value at least ${n \cdot 2n \cdot (2n-1)} + m$.
	Restricting our attention to \textit{optimal} solutions of the preordering instance is justified,
	as there exists a preorder with objective value at least $k$ if and only if there exists an \textit{optimal} preorder with objective value at least $k$.

	We start the proof of the claim by examining the structure of an optimal solution $\tilde{A} \subseteq \tilde{V}_{2}$ to the instance $(\tilde{V}, \tilde{c})$.
	Firstly, positive cliques are clustered together, i.e., $\bigcup_{v \in V}^{} (\tilde{V}_{v})_{\left[ 2 \right]} \subseteq \tilde{A}$.
	Imagine this was not the case, i.e., there was a $u \in V$ such that $(\tilde{V}_{u})_{\left[ 2 \right]} \nsubseteq \tilde{A}$.
	Then there exists $U \subseteq \tilde{V}_{u}$ such that $( U \times (\tilde{V}_{u} \setminus U)) \cap \tilde{A} = \emptyset$
	and $1 \leq \lvert U \rvert \leq \lvert  \tilde{V}_{u} \rvert - 1 = 2n$.
	It then follows that $\lvert  U \times (\tilde{V}_{u} \setminus U) \rvert \geq 2n$, i.e., there are at least $2n$ mistakes in $\tilde{A}$.
	We modify the solution by \emph{cutting out} the $V_{u}$ from the rest of the preorder:
	\[
		\tilde{A}' \coloneqq  \tilde{A}
		\setminus
		\left[ \left( (\tilde{V} \setminus \tilde{V}_{u}) \times \tilde{V}_{u} \right) \cup \left( \tilde{V}_{u} \times (\tilde{V} \setminus \tilde{V}_{u}) \right) \right]
	\]
	$\tilde{A}'$ is again transitive by \Cref{lem:isolating_set_from_preorder}.
	Its construction introduces zero negative mistakes and possibly one positive mistake for each edge in $E$ adjacent to $v$, so at most $2(n - 1)$ mistakes.
	We then re-join $V_{u}$, i.e. let $\tilde{A}'' \coloneqq \tilde{A}' \cup (\tilde{V}_{u})_{\left[ 2 \right]}$.
	Clearly, $\tilde{A}''$ is again transitive and its construction eliminates at least $2n$ positive mistakes by the argument above.
	In total, $\tilde{A}''$ improves over $\tilde{A}$ by at least $2$ and hence $\tilde{A}$ is not optimal, which is a contradiction.
	Secondly, an optimal solution makes zero negative mistakes.
	Note that $\bigcup_{v \in V}^{} (\tilde{V}_{v})_{2}$ is always a solution that makes zero negative mistakes and at most $\lvert E \rvert < n^{2}$ many positive mistakes.
	Suppose now there was an optimal solution $\tilde{A}$ with a negative mistake, i.e., there exists $\tilde{v}_{u,k}\tilde{v}_{v,l} \in V_{2}^{-} \cap \tilde{A}$.
	By construction, $u \neq v$ and especially $uv \notin E$.
	Transitivity then implies $\tilde{V}_{u} \times \tilde{V}_{v} \subseteq \tilde{A}$, but these are $(2n+1)^{2} > n^{2}$ negative mistakes, hence $\tilde{A}$ is not optimal, a contradiction.

	\textit{Only if:} Consider a solution $A$ to the \textsc{MTSD}-instance with $\lvert A \rvert = m$. Then
	\[
		\tilde{A} \coloneqq \bigcup_{v \in V} (\tilde{V}_{v})_{\left[ 2 \right]} \cup \bigcup_{uv \in A} (\tilde{V}_{u} \times \tilde{V}_{v})
	\]
	is a solution to the preordering instance, because transitivity of $A$ implies transitivity of $\tilde{A}$.
	Furthermore, $\tilde{A}$ gains $2n \cdot (2n - 1)$ objective value for every $u \in V$ and $+1$ objective value for every $ij \in A$.

	\textit{If:} Consider an optimal solution $\tilde{A}$ to the preordering instance with objective value $k$.
	We construct a solution $A$ to the \textsc{MTSD}-instance as follows:
	\[
		A \coloneqq \left\{ uv \in E \mid (\tilde{V}_{u} \times \tilde{V}_{v}) \subseteq \tilde{A} \right\}
	\]
	Assume for the sake of contradiction that $A$ was not transitive, i.e., there are $uvw \in V_{3}$ s.t. $uv, vw \in A$ while $uw \notin A$.
	Therefore, $(\tilde{V}_{u} \times \tilde{V}_{v}) \cup (\tilde{V}_{v} \times \tilde{V}_{w}) \subseteq \tilde{A}$ and $\tilde{V}_{u} \times \tilde{V}_{w} \cap \tilde{A} = \emptyset$.
	But then $\tilde{A}$ is intransitive, which is a contradiction.
	By the analysis on the structure of optimal solutions, we know that $k \geq n \cdot 2n \cdot (2n-1)$. Let $m \coloneqq k - n \cdot 2n \cdot (2n-1)$.
	Since between positive cliques there are either zero or exactly one positive arc, we conclude that $\lvert A \rvert = m$.
\end{proof}



\subsection{Derandomization of the max-dicut algorithm} \label{sec:de-randomization}

It is well known that the randomized $4$-approximation algorithm for max-dicut can be derandomized \citep{halperin2001combinatorial,bar2012online}.
For completeness, we reproduce the derandomization due to Theorem 3 of \citet{bar2012online} below.
Additionally, we show that this algorithm can be implemented to run in $\order(|V|^2)$.

Let $G=(V, E)$ be a directed graph with edge weights $w:E \to \mathbb{R}_{\geq 0}$.
In the following, let $S, \bar{S} \subseteq V$ with $S \cap \bar{S} = \emptyset$.
They define the partial dicut $\{ij \in E \mid i \in S, j \in \bar{S}\}$.
We denote the expected value of the dicut that is obtained by randomly assigning elements in $V \setminus (S \cup \bar{S})$ to either $S$ or $\bar{S}$ by $E_{S\bar{S}}$.
It holds that
\begin{align*}
    E_{S\bar{S}} = 
        &   \sum_{\substack{ij \in E: \\ i \in S, j \in \bar{S}}} w_{ij} 
          + \sum_{\substack{ij \in E: i \in S, \\ j \in V \setminus (S \cup \bar{S})}} \frac{w_{ij}}{2} \\
        & + \sum_{\substack{ij \in E: j \in \bar{S}, \\ i \in V \setminus (S \cup \bar{S})}} \frac{w_{ij}}{2} 
          + \sum_{\substack{ij \in E: \\ i,j \in V \setminus (S \cup \bar{S})}} \frac{w_{ij}}{4} \enspace . 
\end{align*}
For $k \in V \setminus (S \cup \bar{S})$, let $g_k$ denote the change in the expected value after assigning $k$ to $S$.
A simple calculation yields
\begin{align*}
    g_k = & E_{S\cup\{k\}\bar{S}} - E_{S\bar{S}} = 
        \sum_{\substack{j \in \bar{S}: \\ kj \in E}} \frac{w_{kj}}{2}
        - \sum_{\substack{i \in S: \\ ik \in E}} \frac{w_{ik}}{2} \\
        & + \sum_{\substack{j \in V \setminus (S \cup \{k\} \cup \bar{S}): \\ kj \in E}} \frac{w_{kj}}{4}
        - \sum_{\substack{i \in V \setminus (S \cup \{k\} \cup \bar{S}): \\ ik \in E}} \frac{w_{ik}}{4}
\end{align*}
and 
\begin{align*}
    E_{S\bar{S}\cup\{k\}} &- E_{S\bar{S}} = - g_k \enspace .
\end{align*}
If $g_k$ is non-negative, assigning $k$ to $S$ does not decrease the expected value; otherwise, assigning $k$ to $\bar{S}$ does not decrease the expected value.

When assigning all elements randomly, the expected value is $E_{\emptyset\emptyset} = \sum_{ij \in E} \frac{w_{ij}}{4}$.
By iteratively assigning elements $k \in V \setminus (S \cup \bar{S})$ to $S$ or $\bar{S}$ based on the sign of $g_k$, a dicut is obtained with value at least $E_{\emptyset\emptyset}$ and, therefore, a $4$-approximation to the max-dicut problem.

\Cref{alg:max-di-cut} implements this algorithm with two additional improvements:
Firstly, in each iteration, the element $k \in V \setminus (S \cup \bar{S})$ for which $|g_k|$ is greatest is selected.
This maximizes the expected value after assigning $k$.
Secondly, the values $g_k$ for $k \in V \setminus (S \cup \bar{S})$ are not computed from scratch in each iteration.
Instead, they are computed once for $S=\bar{S}=\emptyset$, and after each iteration, these values are updated.
With this, each iteration runs in time $\order(|V|)$ instead of $\order(|V|^2)$ that would be required to compute all $g$ values according to the formula above.


