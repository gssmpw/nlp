\section{Experiments}
\label{sec:experiments}

We apply the algorithms defined in \Cref{sec:algorithms} to instances of the preordering problem from social networks.
We solve LPs and ILPs with \citet{gurobi} in such a way that triangle inequalities \eqref{eq:triangle} and odd closed walk inequalities \eqref{eq:odd-closed-walk} are not enumerated in advance but are only added to the system if they become violated throughout the execution of the solver.
All experiments are performed on an Intel Core i9-12900KF Gen12 $\times$ 24 and an NVIDIA GeForce RTX 4080 Super.
We abbreviate the algorithms as follows:
\begin{itemize}[nosep]
    \item \algacronym{ILP}: Gurobi applied to the system \eqref{eq:objective}--\eqref{eq:triangle}
    \item \algacronym{LP}: Gurobi applied to the system \eqref{eq:objective}, \eqref{eq:triangle}, \eqref{eq:continuous}
    \item \algacronym{LP+OCW}: Gurobi applied to the system \eqref{eq:objective}, \eqref{eq:triangle}, \eqref{eq:continuous}, \eqref{eq:odd-closed-walk}
    \item \algacronym{GDC}: Greedy max-dicut approximation (\Cref{alg:max-di-cut})
    \item \algacronym{GAI}: Greedy arc insertion (\Cref{alg:greedy}) initialized with the identity relation
    \item \algacronym{GDC+GAI}: Greedy arc insertion (\Cref{alg:greedy}) initialized with the output of GDC
\end{itemize}




\subsection{Ego networks}

In this section we consider ego networks of Twitter and Google+ \citep{leskovec2012learning}.
The ego network corresponding to a given account consists of one node for each account that the given account follows. 
Directed edges between accounts $i$ and $j$ indicate that $i$ follows $j$.
The Twitter dataset consists of $973$ ego networks with up to $247$ nodes and $17930$ edges.
The Google+ dataset consists of $132$ ego networks with up to $4938$ nodes and $1614977$ edges.
We define instances of the preordering problem for each social network with values $c_{ij} = 1$ if the edge $ij$ is in the network and $c_{ij} = -1$ otherwise, for all $ij \in V_2$.
With these values, $B(c) = |E|$.



\subsubsection{Comparison of preordering, clustering and partial ordering}

\begin{figure}
	\centering
	\small
    \raisebox{3ex}{%
\begin{tikzpicture}
\small
\begin{axis}[
	ylabel={Objective / $|E|$},
	xtick={1, 2, 3, 4},
	xticklabels={$\lesssim$, $\leq$, $\sim$, {$(\sim,\leq)$}},
	typeset ticklabels with strut,
	boxplot/draw direction=y,
	width=0.4\columnwidth,
	mark size=0.15ex,
]
\pgfplotstableread[col sep=comma]{data/clustering-vs-ordering.csv}\mydata;
\addplot[boxplot, color=blue] table[y expr=\thisrowno{2}/\thisrowno{1}] {\mydata};
\addplot[boxplot, color=red] table[y expr=\thisrowno{3}/\thisrowno{1}] {\mydata};
\addplot[boxplot, color=green] table[y expr=\thisrowno{4}/\thisrowno{1}] {\mydata};
\addplot[boxplot, color=orange] table[y expr=\thisrowno{5}/\thisrowno{1}] {\mydata};
\end{axis}
\end{tikzpicture}
}
%
\hspace{5ex}
%
\begin{tikzpicture}
\small
\begin{axis}[
	xlabel=$|V|$,
	ylabel={Runtime [s]},
	ymode=log,
	width=0.4\columnwidth,
	legend pos=outer north east,
	legend cell align=left,
	mark size=0.15ex,
]
\pgfplotstableread[col sep=comma]{data/clustering-vs-ordering.csv}\mydata;
\addplot [color=blue, only marks, mark=*]
	table [x index=0, y index=6] {\mydata};
\addplot [color=red, only marks, mark=*]
	table [x index=0, y index=7] {\mydata};
\addplot [color=green, only marks, mark=*]
	table [x index=0, y index=8] {\mydata};
\addplot [color=orange, only marks, mark=*]
	table [x index=0, y index=9] {\mydata};
\legend{$\lesssim$, $\leq$, $\mathstrut{\sim}$, {$(\sim,\leq)$}}
\end{axis}
\end{tikzpicture}

    \caption{Shown above are objective values normalized by the number of edges (left) and runtimes (right) for those $435$ Twitter instances for which the preorder ($\lesssim$), partial order ($\leq$), and clustering ($\sim$) problems are solved to optimality within a time limit of $500\,$s. Here, $(\sim,\leq)$ denotes successive clustering and partial ordering.}
    \label{fig:clustering-vs-ordering-quantitative}
\end{figure}

We compare solutions to the preordering problem with those obtained by solving a clustering and a partial ordering problem.
Additionally, and for a comparison, we compute preorders by first computing a clustering and then computing a partial order on the obtained clusters.
This successive approach is in contrast to the joint clustering and partial ordering using the preordering problem.
In order to compute solutions to the clustering and partial ordering problem, we add additional inequalities to the system \eqref{eq:objective}--\eqref{eq:triangle}:
For clustering, we add symmetry constraints $x_{ij} = x_{ji}$ for all $ij \in V_2$. For partial ordering we add antisymmetry constraints $x_{ij} + x_{ji} \leq 1$ for all $ij \in V_2$.

Results for six ego networks are shown in \Cref{fig:clustering-vs-ordering} and compared quantitatively in \Cref{tab:clustering-vs-ordering}.
The first network, for instance, has $|E|=54$ edges and admits a preorder of value $52$. 
This preorder disagrees with two of the edges.
Its transitivity index is $T(c) = \frac{52}{54} \approx 0.963$.
The optimal clustering and partial ordering disagree with $14$ and $20$ edges, respectively.
Not every social network has such a high transitivity index. 
The fourth network, for example, has a transitivity index of $T(c) = \frac{27}{51} \approx 0.529$,
and the objective values of the optimal solutions to the clustering (20) and partial ordering (26) problems are not much smaller than the value of the optimal preorder (27).

\begin{figure}
    \centering
    \small
    \setlength{\tabcolsep}{0pt}
    \begin{tabular}{@{}cccc@{}}    
        \begin{tikzpicture}[scale=0.6]
\pgfdeclarelayer{bg}
\pgfsetlayers{bg,main}
\node[vertex, fill=white, fill opacity=0.8, text opacity=1] (0) at (3.56, 4.1) {\scriptsize 0};
\node[vertex, fill=white, fill opacity=0.8, text opacity=1] (1) at (1.5, 3.91) {\scriptsize 1};
\node[vertex, fill=white, fill opacity=0.8, text opacity=1] (2) at (0.5, 3.37) {\scriptsize 2};
\node[vertex, fill=white, fill opacity=0.8, text opacity=1] (4) at (2.8, 1.21) {\scriptsize 4};
\node[vertex, fill=white, fill opacity=0.8, text opacity=1] (6) at (3.56, 2.29) {\scriptsize 6};
\node[vertex, fill=white, fill opacity=0.8, text opacity=1] (7) at (4.5, 0.2) {\scriptsize 7};
\node[vertex, fill=white, fill opacity=0.8, text opacity=1] (9) at (0.67, 0.2) {\scriptsize 9};
\node[vertex, fill=white, fill opacity=0.8, text opacity=1] (10) at (2.6, 3.5) {\scriptsize 10};
\node[vertex, fill=white, fill opacity=0.8, text opacity=1] (11) at (2.2, 2.6) {\scriptsize 11};
\node[vertex, fill=white, fill opacity=0.8, text opacity=1] (12) at (1.7, 1.3) {\scriptsize 12};
\node[vertex, fill=white, fill opacity=0.8, text opacity=1] (13) at (4.04, 2.83) {\scriptsize 13};
\node[vertex, fill=white, fill opacity=0.8, text opacity=1] (3) at (4.04, 1.75) {\scriptsize 3};
\node[vertex, fill=white, fill opacity=0.8, text opacity=1] (5) at (1.34, 0.67) {\scriptsize 5};
\node[vertex, fill=white, fill opacity=0.8, text opacity=1] (8) at (5.00, 1.21) {\scriptsize 8};
\begin{pgfonlayer}{bg}
\draw[-latex] (0) edge[bend left=10] (1);
\draw[-latex] (0) edge[bend left=10] (2);
\draw[-latex] (0) edge[bend left=10] (4);
\draw[-latex] (0) edge[bend left=10] (6);
\draw[-latex] (0) edge[bend left=10] (7);
\draw[-latex] (0) edge[bend left=10] (9);
\draw[-latex] (0) edge[bend left=10] (10);
\draw[-latex] (0) edge[bend left=10] (11);
\draw[-latex] (0) edge[bend left=10] (12);
\draw[-latex] (1) edge[bend left=10] (2);
\draw[-latex] (1) edge[bend left=10] (4);
\draw[-latex] (1) edge[bend left=10] (13);
\draw[-latex] (2) edge[bend left=10] (0);
\draw[-latex] (2) edge[bend left=10] (1);
\draw[-latex] (2) edge[bend left=10] (3);
\draw[-latex] (2) edge[bend left=10] (4);
\draw[-latex] (2) edge[bend left=10] (5);
\draw[-latex] (2) edge[bend left=10] (6);
\draw[-latex] (2) edge[bend left=10] (7);
\draw[-latex] (2) edge[bend left=10] (8);
\draw[-latex] (2) edge[bend left=10] (9);
\draw[-latex] (4) edge[bend left=10] (0);
\draw[-latex] (4) edge[bend left=10] (1);
\draw[-latex] (4) edge[bend left=10] (2);
\draw[-latex] (4) edge[bend left=10] (5);
\draw[-latex] (4) edge[bend left=10] (6);
\draw[-latex] (4) edge[bend left=10] (8);
\draw[-latex] (4) edge[bend left=10] (9);
\draw[-latex] (4) edge[bend left=10] (10);
\draw[-latex] (4) edge[bend left=10] (11);
\draw[-latex] (6) edge[bend left=10] (2);
\draw[-latex] (6) edge[bend left=10] (4);
\draw[-latex] (6) edge[bend left=10] (7);
\draw[-latex] (7) edge[bend left=10] (2);
\draw[-latex] (7) edge[bend left=10] (12);
\draw[-latex] (9) edge[bend left=10] (0);
\draw[-latex] (9) edge[bend left=10] (2);
\draw[-latex] (9) edge[bend left=10] (4);
\draw[-latex] (9) edge[bend left=10] (8);
\draw[-latex] (10) edge[bend left=10] (0);
\draw[-latex] (10) edge[bend left=10] (4);
\draw[-latex] (10) edge[bend left=10] (11);
\draw[-latex] (11) edge[bend left=10] (10);
\draw[-latex] (3) edge[bend left=10] (2);
\draw[-latex] (5) edge[bend left=10] (2);
\draw[-latex] (5) edge[bend left=10] (4);
\draw[-latex] (5) edge[bend left=10] (7);
\draw[-latex] (8) edge[bend right=10] (0);
\draw[-latex] (8) edge[bend left=10] (2);
\draw[-latex] (8) edge[bend left=10] (4);
\draw[-latex] (8) edge[bend left=10] (9);
\end{pgfonlayer}
\end{tikzpicture}
 &
        \begin{tikzpicture}[xscale=0.95]
\def\r{0.15}
\node[cluster, minimum size=3.000*\r cm] (C0) at (28*\r, 15*\r) {};
\node[cluster, minimum size=3.000*\r cm] (C1) at (32*\r, 6*\r) {};
\node[cluster, minimum size=7.253*\r cm] (C2) at (33*\r, 20*\r) {};
\node[cluster, minimum size=3.000*\r cm] (C3) at (28*\r, 6*\r) {};
\node[cluster, minimum size=3.000*\r cm] (C4) at (22*\r, 15*\r) {};
\node[cluster, minimum size=3.000*\r cm] (C5) at (23*\r, 10*\r) {};
\node[cluster, minimum size=3.000*\r cm] (C6) at (26.5*\r, 10*\r) {};
\node[cluster, minimum size=3.000*\r cm] (C7) at (37*\r, 10*\r) {};
\node[cluster, minimum size=3.000*\r cm] (C8) at (30*\r, 10*\r) {};
\node[cluster, minimum size=3.000*\r cm] (C9) at (12.5*\r, 10*\r) {};
\node[cluster, minimum size=3.000*\r cm] (C10) at (24*\r, 6*\r) {};
\node[cluster, minimum size=3.000*\r cm] (C11) at (19.5*\r, 10*\r) {};
\node[cluster, minimum size=3.000*\r cm] (C12) at (37*\r, 14.5*\r) {};
\node[cluster, minimum size=3.000*\r cm] (C13) at (33.5*\r, 10*\r) {};
\node[cluster, minimum size=3.000*\r cm] (C14) at (16*\r, 10*\r) {};
\node[vertex, fill=white, shift=(0.000:0.000*\r cm), minimum size=2.0*\r cm] (0) at (C0) {\scriptsize 0};
\node[vertex, fill=white, shift=(0.000:0.000*\r cm), minimum size=2.0*\r cm] (1) at (C1) {\scriptsize 1};
\node[vertex, fill=white, shift=(0.000:2.127*\r cm), minimum size=2.0*\r cm] (2) at (C2) {\scriptsize 2};
\node[vertex, fill=white, shift=(72.000:2.127*\r cm), minimum size=2.0*\r cm] (5) at (C2) {\scriptsize 5};
\node[vertex, fill=white, shift=(144.000:2.127*\r cm), minimum size=2.0*\r cm] (6) at (C2) {\scriptsize 6};
\node[vertex, fill=white, shift=(216.000:2.127*\r cm), minimum size=2.0*\r cm] (8) at (C2) {\scriptsize 8};
\node[vertex, fill=white, shift=(288.000:2.127*\r cm), minimum size=2.0*\r cm] (10) at (C2) {\scriptsize 10};
\node[vertex, fill=white, shift=(0.000:0.000*\r cm), minimum size=2.0*\r cm] (3) at (C3) {\scriptsize 3};
\node[vertex, fill=white, shift=(0.000:0.000*\r cm), minimum size=2.0*\r cm] (4) at (C4) {\scriptsize 4};
\node[vertex, fill=white, shift=(0.000:0.000*\r cm), minimum size=2.0*\r cm] (7) at (C5) {\scriptsize 7};
\node[vertex, fill=white, shift=(0.000:0.000*\r cm), minimum size=2.0*\r cm] (9) at (C6) {\scriptsize 9};
\node[vertex, fill=white, shift=(0.000:0.000*\r cm), minimum size=2.0*\r cm] (11) at (C7) {\scriptsize 11};
\node[vertex, fill=white, shift=(0.000:0.000*\r cm), minimum size=2.0*\r cm] (12) at (C8) {\scriptsize 12};
\node[vertex, fill=white, shift=(0.000:0.000*\r cm), minimum size=2.0*\r cm] (13) at (C9) {\scriptsize 13};
\node[vertex, fill=white, shift=(0.000:0.000*\r cm), minimum size=2.0*\r cm] (14) at (C10) {\scriptsize 14};
\node[vertex, fill=white, shift=(0.000:0.000*\r cm), minimum size=2.0*\r cm] (15) at (C11) {\scriptsize 15};
\node[vertex, fill=white, shift=(0.000:0.000*\r cm), minimum size=2.0*\r cm] (16) at (C12) {\scriptsize 16};
\node[vertex, fill=white, shift=(0.000:0.000*\r cm), minimum size=2.0*\r cm] (17) at (C13) {\scriptsize 17};
\node[vertex, fill=white, shift=(0.000:0.000*\r cm), minimum size=2.0*\r cm] (18) at (C14) {\scriptsize 18};
\draw[cluster_arc] (C0) -- (C5);
\draw[cluster_arc] (C0) -- (C6);
\draw[cluster_arc] (C0) -- (C7);
\draw[cluster_arc] (C0) -- (C8);
\draw[cluster_arc] (C0) -- (C13);
\draw[cluster_arc] (C2) -- (C4);
\draw[cluster_arc] (C2) -- (C7);
\draw[cluster_arc] (C2) -- (C12);
\draw[cluster_arc] (C4) -- (C5);
\draw[cluster_arc] (C4) -- (C6);
\draw[cluster_arc] (C4) -- (C8);
\draw[cluster_arc] (C4) -- (C9);
\draw[cluster_arc] (C4) -- (C11);
\draw[cluster_arc] (C4) -- (C14);
\draw[cluster_arc] (C8) -- (C1);
\draw[cluster_arc] (C8) -- (C10);
\draw[cluster_arc] (C8) -- (C3);
\draw[disagreement_arc] (0) edge[bend right=25] (5);
\draw[disagreement_arc] (0) edge[bend left=20] (6);
\draw[disagreement_arc] (0) edge[bend right=20] (10);
\draw[disagreement_arc] (0) -- (4);
\draw[disagreement_arc] (2) edge[bend left=10] (17);
\draw[disagreement_arc] (4) edge[out=60,in=165] (5);
\draw[disagreement_arc] (5) edge[out=-15,in=70] (17);
\draw[disagreement_arc] (6) edge[bend right=20] (13);
\draw[disagreement_arc] (8) -- (1);
\draw[disagreement_arc] (8) -- (2);
\draw[disagreement_arc] (8) edge[bend right=20] (4);
\draw[disagreement_arc] (12) -- (5);
\draw[disagreement_arc] (12) edge[out=90,in=240] (6);
\draw[disagreement_arc] (12) -- (10);
\draw[disagreement_arc] (12) edge[bend left=30] (15);
\end{tikzpicture}
 &
        \begin{tikzpicture}
\def\r{0.15}
\node[cluster, minimum size=8.762*\r cm] (C0) at (0*\r, 0*\r) {};
\node[cluster, minimum size=3.000*\r cm] (C1) at (-6*\r, -5*\r) {};
\node[cluster, minimum size=3.000*\r cm] (C2) at (6*\r, -5*\r) {};
\node[vertex, fill=white, shift=(0.000:2.881*\r cm), minimum size=2.0*\r cm] (0) at (C0) {\scriptsize 0};
\node[vertex, fill=white, shift=(51.429:2.881*\r cm), minimum size=2.0*\r cm] (2) at (C0) {\scriptsize 2};
\node[vertex, fill=white, shift=(102.857:2.881*\r cm), minimum size=2.0*\r cm] (3) at (C0) {\scriptsize 3};
\node[vertex, fill=white, shift=(154.286:2.881*\r cm), minimum size=2.0*\r cm] (6) at (C0) {\scriptsize 6};
\node[vertex, fill=white, shift=(205.714:2.881*\r cm), minimum size=2.0*\r cm] (4) at (C0) {\scriptsize 4};
\node[vertex, fill=white, shift=(257.143:2.881*\r cm), minimum size=2.0*\r cm] (8) at (C0) {\scriptsize 8};
\node[vertex, fill=white, shift=(308.571:2.881*\r cm), minimum size=2.0*\r cm] (7) at (C0) {\scriptsize 7};
\node[vertex, fill=white, shift=(0.000:0.000*\r cm), minimum size=2.0*\r cm] (1) at (C1) {\scriptsize 1};
\node[vertex, fill=white, shift=(0.000:0.000*\r cm), minimum size=2.0*\r cm] (5) at (C2) {\scriptsize 5};
\draw[disagreement_arc] (0) -- (1);
\draw[disagreement_arc] (0) -- (5);
\draw[disagreement_arc] (2) -- (1);
\draw[disagreement_arc] (2) -- (5);
\draw[disagreement_arc] (3) edge[bend left=20] (4);
\draw[disagreement_arc] (3) -- (1);
\draw[disagreement_arc] (3) -- (5);
\draw[disagreement_arc] (4) -- (1);
\draw[disagreement_arc] (4) -- (5);
\draw[disagreement_arc] (6) -- (1);
\draw[disagreement_arc] (6) -- (5);
\draw[disagreement_arc] (7) -- (1);
\draw[disagreement_arc] (7) -- (5);
\draw[disagreement_arc] (8) -- (1);
\end{tikzpicture}
 &
        \begin{tikzpicture}
\def\r{0.15}
\node[cluster, minimum size=3.000*\r cm] (C0) at (0*\r, 11.9*\r) {};
\node[cluster, minimum size=3.000*\r cm] (C1) at (15*\r, 0*\r) {};
\node[cluster, minimum size=3.000*\r cm] (C2) at (5*\r, 10.2*\r) {};
\node[cluster, minimum size=3.000*\r cm] (C3) at (15*\r, 6.8*\r) {};
\node[cluster, minimum size=3.000*\r cm] (C4) at (10*\r, 8.5*\r) {};
\node[cluster, minimum size=3.000*\r cm] (C5) at (0*\r, 1.7*\r) {};
\node[cluster, minimum size=3.000*\r cm] (C6) at (10*\r, 5.1*\r) {};
\node[cluster, minimum size=3.000*\r cm] (C7) at (5*\r, 3.4*\r) {};
\node[cluster, minimum size=3.000*\r cm] (C8) at (10*\r, 1.7*\r) {};
\node[vertex, fill=white, shift=(0.000:0.000*\r cm), minimum size=2.0*\r cm] (0) at (C0) {\scriptsize 0};
\node[vertex, fill=white, shift=(0.000:0.000*\r cm), minimum size=2.0*\r cm] (1) at (C1) {\scriptsize 1};
\node[vertex, fill=white, shift=(0.000:0.000*\r cm), minimum size=2.0*\r cm] (2) at (C2) {\scriptsize 2};
\node[vertex, fill=white, shift=(0.000:0.000*\r cm), minimum size=2.0*\r cm] (3) at (C3) {\scriptsize 3};
\node[vertex, fill=white, shift=(0.000:0.000*\r cm), minimum size=2.0*\r cm] (4) at (C4) {\scriptsize 4};
\node[vertex, fill=white, shift=(0.000:0.000*\r cm), minimum size=2.0*\r cm] (5) at (C5) {\scriptsize 5};
\node[vertex, fill=white, shift=(0.000:0.000*\r cm), minimum size=2.0*\r cm] (6) at (C6) {\scriptsize 6};
\node[vertex, fill=white, shift=(0.000:0.000*\r cm), minimum size=2.0*\r cm] (7) at (C7) {\scriptsize 7};
\node[vertex, fill=white, shift=(0.000:0.000*\r cm), minimum size=2.0*\r cm] (8) at (C8) {\scriptsize 8};
\draw[cluster_arc] (C0) -- (C2);
\draw[cluster_arc] (C2) -- (C4);
\draw[cluster_arc] (C3) -- (C6);
\draw[cluster_arc] (C4) -- (C3);
\draw[cluster_arc] (C6) -- (C7);
\draw[cluster_arc] (C7) -- (C8);
\draw[cluster_arc] (C7) -- (C5);
\draw[cluster_arc] (C8) -- (C1);
\draw[disagreement_arc] (2) edge[bend left=20] (0);
\draw[disagreement_arc] (3) edge[bend right=35] (0);
\draw[disagreement_arc] (3) edge[bend right=30] (2);
\draw[disagreement_arc] (4) edge[bend right=30] (0);
\draw[disagreement_arc] (4) edge[bend left=20] (2);
\draw[disagreement_arc] (6) edge[bend left=10] (0);
\draw[disagreement_arc] (6) -- (2);
\draw[disagreement_arc] (6) edge[bend left=20] (3);
\draw[disagreement_arc] (6) -- (4);
\draw[disagreement_arc] (7) edge[bend left=20] (0);
\draw[disagreement_arc] (7) -- (2);
\draw[disagreement_arc] (7) edge[bend right=28] (3);
\draw[disagreement_arc] (7) -- (4);
\draw[disagreement_arc] (7) edge[bend left=20] (6);
\draw[disagreement_arc] (8) -- (0);
\draw[disagreement_arc] (8) -- (2);
\draw[disagreement_arc] (8) edge[bend right=20] (3);
\draw[disagreement_arc] (8) edge[bend right=40] (4);
\draw[disagreement_arc] (8) -- (6);
\draw[disagreement_arc] (8) edge[bend left=20] (7);
\end{tikzpicture}
 
        \\
        \begin{tikzpicture}[scale=0.6]
\pgfdeclarelayer{bg}
\pgfsetlayers{bg,main}
\node[vertex, fill=white, fill opacity=0.8, text opacity=1] (0) at (3.56, 4.1) {\scriptsize 0};
\node[vertex, fill=white, fill opacity=0.8, text opacity=1] (1) at (1.5, 3.91) {\scriptsize 1};
\node[vertex, fill=white, fill opacity=0.8, text opacity=1] (2) at (0.5, 3.37) {\scriptsize 2};
\node[vertex, fill=white, fill opacity=0.8, text opacity=1] (4) at (2.8, 1.21) {\scriptsize 4};
\node[vertex, fill=white, fill opacity=0.8, text opacity=1] (6) at (3.56, 2.29) {\scriptsize 6};
\node[vertex, fill=white, fill opacity=0.8, text opacity=1] (7) at (4.5, 0.2) {\scriptsize 7};
\node[vertex, fill=white, fill opacity=0.8, text opacity=1] (9) at (0.67, 0.2) {\scriptsize 9};
\node[vertex, fill=white, fill opacity=0.8, text opacity=1] (10) at (2.6, 3.5) {\scriptsize 10};
\node[vertex, fill=white, fill opacity=0.8, text opacity=1] (11) at (2.2, 2.6) {\scriptsize 11};
\node[vertex, fill=white, fill opacity=0.8, text opacity=1] (12) at (1.7, 1.3) {\scriptsize 12};
\node[vertex, fill=white, fill opacity=0.8, text opacity=1] (13) at (4.04, 2.83) {\scriptsize 13};
\node[vertex, fill=white, fill opacity=0.8, text opacity=1] (3) at (4.04, 1.75) {\scriptsize 3};
\node[vertex, fill=white, fill opacity=0.8, text opacity=1] (5) at (1.34, 0.67) {\scriptsize 5};
\node[vertex, fill=white, fill opacity=0.8, text opacity=1] (8) at (5.00, 1.21) {\scriptsize 8};
\begin{pgfonlayer}{bg}
\draw[-latex] (0) edge[bend left=10] (1);
\draw[-latex] (0) edge[bend left=10] (2);
\draw[-latex] (0) edge[bend left=10] (4);
\draw[-latex] (0) edge[bend left=10] (6);
\draw[-latex] (0) edge[bend left=10] (7);
\draw[-latex] (0) edge[bend left=10] (9);
\draw[-latex] (0) edge[bend left=10] (10);
\draw[-latex] (0) edge[bend left=10] (11);
\draw[-latex] (0) edge[bend left=10] (12);
\draw[-latex] (1) edge[bend left=10] (2);
\draw[-latex] (1) edge[bend left=10] (4);
\draw[-latex] (1) edge[bend left=10] (13);
\draw[-latex] (2) edge[bend left=10] (0);
\draw[-latex] (2) edge[bend left=10] (1);
\draw[-latex] (2) edge[bend left=10] (3);
\draw[-latex] (2) edge[bend left=10] (4);
\draw[-latex] (2) edge[bend left=10] (5);
\draw[-latex] (2) edge[bend left=10] (6);
\draw[-latex] (2) edge[bend left=10] (7);
\draw[-latex] (2) edge[bend left=10] (8);
\draw[-latex] (2) edge[bend left=10] (9);
\draw[-latex] (4) edge[bend left=10] (0);
\draw[-latex] (4) edge[bend left=10] (1);
\draw[-latex] (4) edge[bend left=10] (2);
\draw[-latex] (4) edge[bend left=10] (5);
\draw[-latex] (4) edge[bend left=10] (6);
\draw[-latex] (4) edge[bend left=10] (8);
\draw[-latex] (4) edge[bend left=10] (9);
\draw[-latex] (4) edge[bend left=10] (10);
\draw[-latex] (4) edge[bend left=10] (11);
\draw[-latex] (6) edge[bend left=10] (2);
\draw[-latex] (6) edge[bend left=10] (4);
\draw[-latex] (6) edge[bend left=10] (7);
\draw[-latex] (7) edge[bend left=10] (2);
\draw[-latex] (7) edge[bend left=10] (12);
\draw[-latex] (9) edge[bend left=10] (0);
\draw[-latex] (9) edge[bend left=10] (2);
\draw[-latex] (9) edge[bend left=10] (4);
\draw[-latex] (9) edge[bend left=10] (8);
\draw[-latex] (10) edge[bend left=10] (0);
\draw[-latex] (10) edge[bend left=10] (4);
\draw[-latex] (10) edge[bend left=10] (11);
\draw[-latex] (11) edge[bend left=10] (10);
\draw[-latex] (3) edge[bend left=10] (2);
\draw[-latex] (5) edge[bend left=10] (2);
\draw[-latex] (5) edge[bend left=10] (4);
\draw[-latex] (5) edge[bend left=10] (7);
\draw[-latex] (8) edge[bend right=10] (0);
\draw[-latex] (8) edge[bend left=10] (2);
\draw[-latex] (8) edge[bend left=10] (4);
\draw[-latex] (8) edge[bend left=10] (9);
\end{pgfonlayer}
\end{tikzpicture}
 &
        \begin{tikzpicture}[xscale=0.95]
\def\r{0.15}
\node[cluster, minimum size=3.000*\r cm] (C0) at (28*\r, 15*\r) {};
\node[cluster, minimum size=3.000*\r cm] (C1) at (32*\r, 6*\r) {};
\node[cluster, minimum size=7.253*\r cm] (C2) at (33*\r, 20*\r) {};
\node[cluster, minimum size=3.000*\r cm] (C3) at (28*\r, 6*\r) {};
\node[cluster, minimum size=3.000*\r cm] (C4) at (22*\r, 15*\r) {};
\node[cluster, minimum size=3.000*\r cm] (C5) at (23*\r, 10*\r) {};
\node[cluster, minimum size=3.000*\r cm] (C6) at (26.5*\r, 10*\r) {};
\node[cluster, minimum size=3.000*\r cm] (C7) at (37*\r, 10*\r) {};
\node[cluster, minimum size=3.000*\r cm] (C8) at (30*\r, 10*\r) {};
\node[cluster, minimum size=3.000*\r cm] (C9) at (12.5*\r, 10*\r) {};
\node[cluster, minimum size=3.000*\r cm] (C10) at (24*\r, 6*\r) {};
\node[cluster, minimum size=3.000*\r cm] (C11) at (19.5*\r, 10*\r) {};
\node[cluster, minimum size=3.000*\r cm] (C12) at (37*\r, 14.5*\r) {};
\node[cluster, minimum size=3.000*\r cm] (C13) at (33.5*\r, 10*\r) {};
\node[cluster, minimum size=3.000*\r cm] (C14) at (16*\r, 10*\r) {};
\node[vertex, fill=white, shift=(0.000:0.000*\r cm), minimum size=2.0*\r cm] (0) at (C0) {\scriptsize 0};
\node[vertex, fill=white, shift=(0.000:0.000*\r cm), minimum size=2.0*\r cm] (1) at (C1) {\scriptsize 1};
\node[vertex, fill=white, shift=(0.000:2.127*\r cm), minimum size=2.0*\r cm] (2) at (C2) {\scriptsize 2};
\node[vertex, fill=white, shift=(72.000:2.127*\r cm), minimum size=2.0*\r cm] (5) at (C2) {\scriptsize 5};
\node[vertex, fill=white, shift=(144.000:2.127*\r cm), minimum size=2.0*\r cm] (6) at (C2) {\scriptsize 6};
\node[vertex, fill=white, shift=(216.000:2.127*\r cm), minimum size=2.0*\r cm] (8) at (C2) {\scriptsize 8};
\node[vertex, fill=white, shift=(288.000:2.127*\r cm), minimum size=2.0*\r cm] (10) at (C2) {\scriptsize 10};
\node[vertex, fill=white, shift=(0.000:0.000*\r cm), minimum size=2.0*\r cm] (3) at (C3) {\scriptsize 3};
\node[vertex, fill=white, shift=(0.000:0.000*\r cm), minimum size=2.0*\r cm] (4) at (C4) {\scriptsize 4};
\node[vertex, fill=white, shift=(0.000:0.000*\r cm), minimum size=2.0*\r cm] (7) at (C5) {\scriptsize 7};
\node[vertex, fill=white, shift=(0.000:0.000*\r cm), minimum size=2.0*\r cm] (9) at (C6) {\scriptsize 9};
\node[vertex, fill=white, shift=(0.000:0.000*\r cm), minimum size=2.0*\r cm] (11) at (C7) {\scriptsize 11};
\node[vertex, fill=white, shift=(0.000:0.000*\r cm), minimum size=2.0*\r cm] (12) at (C8) {\scriptsize 12};
\node[vertex, fill=white, shift=(0.000:0.000*\r cm), minimum size=2.0*\r cm] (13) at (C9) {\scriptsize 13};
\node[vertex, fill=white, shift=(0.000:0.000*\r cm), minimum size=2.0*\r cm] (14) at (C10) {\scriptsize 14};
\node[vertex, fill=white, shift=(0.000:0.000*\r cm), minimum size=2.0*\r cm] (15) at (C11) {\scriptsize 15};
\node[vertex, fill=white, shift=(0.000:0.000*\r cm), minimum size=2.0*\r cm] (16) at (C12) {\scriptsize 16};
\node[vertex, fill=white, shift=(0.000:0.000*\r cm), minimum size=2.0*\r cm] (17) at (C13) {\scriptsize 17};
\node[vertex, fill=white, shift=(0.000:0.000*\r cm), minimum size=2.0*\r cm] (18) at (C14) {\scriptsize 18};
\draw[cluster_arc] (C0) -- (C5);
\draw[cluster_arc] (C0) -- (C6);
\draw[cluster_arc] (C0) -- (C7);
\draw[cluster_arc] (C0) -- (C8);
\draw[cluster_arc] (C0) -- (C13);
\draw[cluster_arc] (C2) -- (C4);
\draw[cluster_arc] (C2) -- (C7);
\draw[cluster_arc] (C2) -- (C12);
\draw[cluster_arc] (C4) -- (C5);
\draw[cluster_arc] (C4) -- (C6);
\draw[cluster_arc] (C4) -- (C8);
\draw[cluster_arc] (C4) -- (C9);
\draw[cluster_arc] (C4) -- (C11);
\draw[cluster_arc] (C4) -- (C14);
\draw[cluster_arc] (C8) -- (C1);
\draw[cluster_arc] (C8) -- (C10);
\draw[cluster_arc] (C8) -- (C3);
\draw[disagreement_arc] (0) edge[bend right=25] (5);
\draw[disagreement_arc] (0) edge[bend left=20] (6);
\draw[disagreement_arc] (0) edge[bend right=20] (10);
\draw[disagreement_arc] (0) -- (4);
\draw[disagreement_arc] (2) edge[bend left=10] (17);
\draw[disagreement_arc] (4) edge[out=60,in=165] (5);
\draw[disagreement_arc] (5) edge[out=-15,in=70] (17);
\draw[disagreement_arc] (6) edge[bend right=20] (13);
\draw[disagreement_arc] (8) -- (1);
\draw[disagreement_arc] (8) -- (2);
\draw[disagreement_arc] (8) edge[bend right=20] (4);
\draw[disagreement_arc] (12) -- (5);
\draw[disagreement_arc] (12) edge[out=90,in=240] (6);
\draw[disagreement_arc] (12) -- (10);
\draw[disagreement_arc] (12) edge[bend left=30] (15);
\end{tikzpicture}
 &
        \begin{tikzpicture}
\def\r{0.15}
\node[cluster, minimum size=8.762*\r cm] (C0) at (0*\r, 0*\r) {};
\node[cluster, minimum size=3.000*\r cm] (C1) at (-6*\r, -5*\r) {};
\node[cluster, minimum size=3.000*\r cm] (C2) at (6*\r, -5*\r) {};
\node[vertex, fill=white, shift=(0.000:2.881*\r cm), minimum size=2.0*\r cm] (0) at (C0) {\scriptsize 0};
\node[vertex, fill=white, shift=(51.429:2.881*\r cm), minimum size=2.0*\r cm] (2) at (C0) {\scriptsize 2};
\node[vertex, fill=white, shift=(102.857:2.881*\r cm), minimum size=2.0*\r cm] (3) at (C0) {\scriptsize 3};
\node[vertex, fill=white, shift=(154.286:2.881*\r cm), minimum size=2.0*\r cm] (6) at (C0) {\scriptsize 6};
\node[vertex, fill=white, shift=(205.714:2.881*\r cm), minimum size=2.0*\r cm] (4) at (C0) {\scriptsize 4};
\node[vertex, fill=white, shift=(257.143:2.881*\r cm), minimum size=2.0*\r cm] (8) at (C0) {\scriptsize 8};
\node[vertex, fill=white, shift=(308.571:2.881*\r cm), minimum size=2.0*\r cm] (7) at (C0) {\scriptsize 7};
\node[vertex, fill=white, shift=(0.000:0.000*\r cm), minimum size=2.0*\r cm] (1) at (C1) {\scriptsize 1};
\node[vertex, fill=white, shift=(0.000:0.000*\r cm), minimum size=2.0*\r cm] (5) at (C2) {\scriptsize 5};
\draw[disagreement_arc] (0) -- (1);
\draw[disagreement_arc] (0) -- (5);
\draw[disagreement_arc] (2) -- (1);
\draw[disagreement_arc] (2) -- (5);
\draw[disagreement_arc] (3) edge[bend left=20] (4);
\draw[disagreement_arc] (3) -- (1);
\draw[disagreement_arc] (3) -- (5);
\draw[disagreement_arc] (4) -- (1);
\draw[disagreement_arc] (4) -- (5);
\draw[disagreement_arc] (6) -- (1);
\draw[disagreement_arc] (6) -- (5);
\draw[disagreement_arc] (7) -- (1);
\draw[disagreement_arc] (7) -- (5);
\draw[disagreement_arc] (8) -- (1);
\end{tikzpicture}
 &
        \begin{tikzpicture}
\def\r{0.15}
\node[cluster, minimum size=3.000*\r cm] (C0) at (0*\r, 11.9*\r) {};
\node[cluster, minimum size=3.000*\r cm] (C1) at (15*\r, 0*\r) {};
\node[cluster, minimum size=3.000*\r cm] (C2) at (5*\r, 10.2*\r) {};
\node[cluster, minimum size=3.000*\r cm] (C3) at (15*\r, 6.8*\r) {};
\node[cluster, minimum size=3.000*\r cm] (C4) at (10*\r, 8.5*\r) {};
\node[cluster, minimum size=3.000*\r cm] (C5) at (0*\r, 1.7*\r) {};
\node[cluster, minimum size=3.000*\r cm] (C6) at (10*\r, 5.1*\r) {};
\node[cluster, minimum size=3.000*\r cm] (C7) at (5*\r, 3.4*\r) {};
\node[cluster, minimum size=3.000*\r cm] (C8) at (10*\r, 1.7*\r) {};
\node[vertex, fill=white, shift=(0.000:0.000*\r cm), minimum size=2.0*\r cm] (0) at (C0) {\scriptsize 0};
\node[vertex, fill=white, shift=(0.000:0.000*\r cm), minimum size=2.0*\r cm] (1) at (C1) {\scriptsize 1};
\node[vertex, fill=white, shift=(0.000:0.000*\r cm), minimum size=2.0*\r cm] (2) at (C2) {\scriptsize 2};
\node[vertex, fill=white, shift=(0.000:0.000*\r cm), minimum size=2.0*\r cm] (3) at (C3) {\scriptsize 3};
\node[vertex, fill=white, shift=(0.000:0.000*\r cm), minimum size=2.0*\r cm] (4) at (C4) {\scriptsize 4};
\node[vertex, fill=white, shift=(0.000:0.000*\r cm), minimum size=2.0*\r cm] (5) at (C5) {\scriptsize 5};
\node[vertex, fill=white, shift=(0.000:0.000*\r cm), minimum size=2.0*\r cm] (6) at (C6) {\scriptsize 6};
\node[vertex, fill=white, shift=(0.000:0.000*\r cm), minimum size=2.0*\r cm] (7) at (C7) {\scriptsize 7};
\node[vertex, fill=white, shift=(0.000:0.000*\r cm), minimum size=2.0*\r cm] (8) at (C8) {\scriptsize 8};
\draw[cluster_arc] (C0) -- (C2);
\draw[cluster_arc] (C2) -- (C4);
\draw[cluster_arc] (C3) -- (C6);
\draw[cluster_arc] (C4) -- (C3);
\draw[cluster_arc] (C6) -- (C7);
\draw[cluster_arc] (C7) -- (C8);
\draw[cluster_arc] (C7) -- (C5);
\draw[cluster_arc] (C8) -- (C1);
\draw[disagreement_arc] (2) edge[bend left=20] (0);
\draw[disagreement_arc] (3) edge[bend right=35] (0);
\draw[disagreement_arc] (3) edge[bend right=30] (2);
\draw[disagreement_arc] (4) edge[bend right=30] (0);
\draw[disagreement_arc] (4) edge[bend left=20] (2);
\draw[disagreement_arc] (6) edge[bend left=10] (0);
\draw[disagreement_arc] (6) -- (2);
\draw[disagreement_arc] (6) edge[bend left=20] (3);
\draw[disagreement_arc] (6) -- (4);
\draw[disagreement_arc] (7) edge[bend left=20] (0);
\draw[disagreement_arc] (7) -- (2);
\draw[disagreement_arc] (7) edge[bend right=28] (3);
\draw[disagreement_arc] (7) -- (4);
\draw[disagreement_arc] (7) edge[bend left=20] (6);
\draw[disagreement_arc] (8) -- (0);
\draw[disagreement_arc] (8) -- (2);
\draw[disagreement_arc] (8) edge[bend right=20] (3);
\draw[disagreement_arc] (8) edge[bend right=40] (4);
\draw[disagreement_arc] (8) -- (6);
\draw[disagreement_arc] (8) edge[bend left=20] (7);
\end{tikzpicture}
 
        \\
        \begin{tikzpicture}[scale=0.6]
\pgfdeclarelayer{bg}
\pgfsetlayers{bg,main}
\node[vertex, fill=white, fill opacity=0.8, text opacity=1] (0) at (3.56, 4.1) {\scriptsize 0};
\node[vertex, fill=white, fill opacity=0.8, text opacity=1] (1) at (1.5, 3.91) {\scriptsize 1};
\node[vertex, fill=white, fill opacity=0.8, text opacity=1] (2) at (0.5, 3.37) {\scriptsize 2};
\node[vertex, fill=white, fill opacity=0.8, text opacity=1] (4) at (2.8, 1.21) {\scriptsize 4};
\node[vertex, fill=white, fill opacity=0.8, text opacity=1] (6) at (3.56, 2.29) {\scriptsize 6};
\node[vertex, fill=white, fill opacity=0.8, text opacity=1] (7) at (4.5, 0.2) {\scriptsize 7};
\node[vertex, fill=white, fill opacity=0.8, text opacity=1] (9) at (0.67, 0.2) {\scriptsize 9};
\node[vertex, fill=white, fill opacity=0.8, text opacity=1] (10) at (2.6, 3.5) {\scriptsize 10};
\node[vertex, fill=white, fill opacity=0.8, text opacity=1] (11) at (2.2, 2.6) {\scriptsize 11};
\node[vertex, fill=white, fill opacity=0.8, text opacity=1] (12) at (1.7, 1.3) {\scriptsize 12};
\node[vertex, fill=white, fill opacity=0.8, text opacity=1] (13) at (4.04, 2.83) {\scriptsize 13};
\node[vertex, fill=white, fill opacity=0.8, text opacity=1] (3) at (4.04, 1.75) {\scriptsize 3};
\node[vertex, fill=white, fill opacity=0.8, text opacity=1] (5) at (1.34, 0.67) {\scriptsize 5};
\node[vertex, fill=white, fill opacity=0.8, text opacity=1] (8) at (5.00, 1.21) {\scriptsize 8};
\begin{pgfonlayer}{bg}
\draw[-latex] (0) edge[bend left=10] (1);
\draw[-latex] (0) edge[bend left=10] (2);
\draw[-latex] (0) edge[bend left=10] (4);
\draw[-latex] (0) edge[bend left=10] (6);
\draw[-latex] (0) edge[bend left=10] (7);
\draw[-latex] (0) edge[bend left=10] (9);
\draw[-latex] (0) edge[bend left=10] (10);
\draw[-latex] (0) edge[bend left=10] (11);
\draw[-latex] (0) edge[bend left=10] (12);
\draw[-latex] (1) edge[bend left=10] (2);
\draw[-latex] (1) edge[bend left=10] (4);
\draw[-latex] (1) edge[bend left=10] (13);
\draw[-latex] (2) edge[bend left=10] (0);
\draw[-latex] (2) edge[bend left=10] (1);
\draw[-latex] (2) edge[bend left=10] (3);
\draw[-latex] (2) edge[bend left=10] (4);
\draw[-latex] (2) edge[bend left=10] (5);
\draw[-latex] (2) edge[bend left=10] (6);
\draw[-latex] (2) edge[bend left=10] (7);
\draw[-latex] (2) edge[bend left=10] (8);
\draw[-latex] (2) edge[bend left=10] (9);
\draw[-latex] (4) edge[bend left=10] (0);
\draw[-latex] (4) edge[bend left=10] (1);
\draw[-latex] (4) edge[bend left=10] (2);
\draw[-latex] (4) edge[bend left=10] (5);
\draw[-latex] (4) edge[bend left=10] (6);
\draw[-latex] (4) edge[bend left=10] (8);
\draw[-latex] (4) edge[bend left=10] (9);
\draw[-latex] (4) edge[bend left=10] (10);
\draw[-latex] (4) edge[bend left=10] (11);
\draw[-latex] (6) edge[bend left=10] (2);
\draw[-latex] (6) edge[bend left=10] (4);
\draw[-latex] (6) edge[bend left=10] (7);
\draw[-latex] (7) edge[bend left=10] (2);
\draw[-latex] (7) edge[bend left=10] (12);
\draw[-latex] (9) edge[bend left=10] (0);
\draw[-latex] (9) edge[bend left=10] (2);
\draw[-latex] (9) edge[bend left=10] (4);
\draw[-latex] (9) edge[bend left=10] (8);
\draw[-latex] (10) edge[bend left=10] (0);
\draw[-latex] (10) edge[bend left=10] (4);
\draw[-latex] (10) edge[bend left=10] (11);
\draw[-latex] (11) edge[bend left=10] (10);
\draw[-latex] (3) edge[bend left=10] (2);
\draw[-latex] (5) edge[bend left=10] (2);
\draw[-latex] (5) edge[bend left=10] (4);
\draw[-latex] (5) edge[bend left=10] (7);
\draw[-latex] (8) edge[bend right=10] (0);
\draw[-latex] (8) edge[bend left=10] (2);
\draw[-latex] (8) edge[bend left=10] (4);
\draw[-latex] (8) edge[bend left=10] (9);
\end{pgfonlayer}
\end{tikzpicture}
 &
        \begin{tikzpicture}[xscale=0.95]
\def\r{0.15}
\node[cluster, minimum size=3.000*\r cm] (C0) at (28*\r, 15*\r) {};
\node[cluster, minimum size=3.000*\r cm] (C1) at (32*\r, 6*\r) {};
\node[cluster, minimum size=7.253*\r cm] (C2) at (33*\r, 20*\r) {};
\node[cluster, minimum size=3.000*\r cm] (C3) at (28*\r, 6*\r) {};
\node[cluster, minimum size=3.000*\r cm] (C4) at (22*\r, 15*\r) {};
\node[cluster, minimum size=3.000*\r cm] (C5) at (23*\r, 10*\r) {};
\node[cluster, minimum size=3.000*\r cm] (C6) at (26.5*\r, 10*\r) {};
\node[cluster, minimum size=3.000*\r cm] (C7) at (37*\r, 10*\r) {};
\node[cluster, minimum size=3.000*\r cm] (C8) at (30*\r, 10*\r) {};
\node[cluster, minimum size=3.000*\r cm] (C9) at (12.5*\r, 10*\r) {};
\node[cluster, minimum size=3.000*\r cm] (C10) at (24*\r, 6*\r) {};
\node[cluster, minimum size=3.000*\r cm] (C11) at (19.5*\r, 10*\r) {};
\node[cluster, minimum size=3.000*\r cm] (C12) at (37*\r, 14.5*\r) {};
\node[cluster, minimum size=3.000*\r cm] (C13) at (33.5*\r, 10*\r) {};
\node[cluster, minimum size=3.000*\r cm] (C14) at (16*\r, 10*\r) {};
\node[vertex, fill=white, shift=(0.000:0.000*\r cm), minimum size=2.0*\r cm] (0) at (C0) {\scriptsize 0};
\node[vertex, fill=white, shift=(0.000:0.000*\r cm), minimum size=2.0*\r cm] (1) at (C1) {\scriptsize 1};
\node[vertex, fill=white, shift=(0.000:2.127*\r cm), minimum size=2.0*\r cm] (2) at (C2) {\scriptsize 2};
\node[vertex, fill=white, shift=(72.000:2.127*\r cm), minimum size=2.0*\r cm] (5) at (C2) {\scriptsize 5};
\node[vertex, fill=white, shift=(144.000:2.127*\r cm), minimum size=2.0*\r cm] (6) at (C2) {\scriptsize 6};
\node[vertex, fill=white, shift=(216.000:2.127*\r cm), minimum size=2.0*\r cm] (8) at (C2) {\scriptsize 8};
\node[vertex, fill=white, shift=(288.000:2.127*\r cm), minimum size=2.0*\r cm] (10) at (C2) {\scriptsize 10};
\node[vertex, fill=white, shift=(0.000:0.000*\r cm), minimum size=2.0*\r cm] (3) at (C3) {\scriptsize 3};
\node[vertex, fill=white, shift=(0.000:0.000*\r cm), minimum size=2.0*\r cm] (4) at (C4) {\scriptsize 4};
\node[vertex, fill=white, shift=(0.000:0.000*\r cm), minimum size=2.0*\r cm] (7) at (C5) {\scriptsize 7};
\node[vertex, fill=white, shift=(0.000:0.000*\r cm), minimum size=2.0*\r cm] (9) at (C6) {\scriptsize 9};
\node[vertex, fill=white, shift=(0.000:0.000*\r cm), minimum size=2.0*\r cm] (11) at (C7) {\scriptsize 11};
\node[vertex, fill=white, shift=(0.000:0.000*\r cm), minimum size=2.0*\r cm] (12) at (C8) {\scriptsize 12};
\node[vertex, fill=white, shift=(0.000:0.000*\r cm), minimum size=2.0*\r cm] (13) at (C9) {\scriptsize 13};
\node[vertex, fill=white, shift=(0.000:0.000*\r cm), minimum size=2.0*\r cm] (14) at (C10) {\scriptsize 14};
\node[vertex, fill=white, shift=(0.000:0.000*\r cm), minimum size=2.0*\r cm] (15) at (C11) {\scriptsize 15};
\node[vertex, fill=white, shift=(0.000:0.000*\r cm), minimum size=2.0*\r cm] (16) at (C12) {\scriptsize 16};
\node[vertex, fill=white, shift=(0.000:0.000*\r cm), minimum size=2.0*\r cm] (17) at (C13) {\scriptsize 17};
\node[vertex, fill=white, shift=(0.000:0.000*\r cm), minimum size=2.0*\r cm] (18) at (C14) {\scriptsize 18};
\draw[cluster_arc] (C0) -- (C5);
\draw[cluster_arc] (C0) -- (C6);
\draw[cluster_arc] (C0) -- (C7);
\draw[cluster_arc] (C0) -- (C8);
\draw[cluster_arc] (C0) -- (C13);
\draw[cluster_arc] (C2) -- (C4);
\draw[cluster_arc] (C2) -- (C7);
\draw[cluster_arc] (C2) -- (C12);
\draw[cluster_arc] (C4) -- (C5);
\draw[cluster_arc] (C4) -- (C6);
\draw[cluster_arc] (C4) -- (C8);
\draw[cluster_arc] (C4) -- (C9);
\draw[cluster_arc] (C4) -- (C11);
\draw[cluster_arc] (C4) -- (C14);
\draw[cluster_arc] (C8) -- (C1);
\draw[cluster_arc] (C8) -- (C10);
\draw[cluster_arc] (C8) -- (C3);
\draw[disagreement_arc] (0) edge[bend right=25] (5);
\draw[disagreement_arc] (0) edge[bend left=20] (6);
\draw[disagreement_arc] (0) edge[bend right=20] (10);
\draw[disagreement_arc] (0) -- (4);
\draw[disagreement_arc] (2) edge[bend left=10] (17);
\draw[disagreement_arc] (4) edge[out=60,in=165] (5);
\draw[disagreement_arc] (5) edge[out=-15,in=70] (17);
\draw[disagreement_arc] (6) edge[bend right=20] (13);
\draw[disagreement_arc] (8) -- (1);
\draw[disagreement_arc] (8) -- (2);
\draw[disagreement_arc] (8) edge[bend right=20] (4);
\draw[disagreement_arc] (12) -- (5);
\draw[disagreement_arc] (12) edge[out=90,in=240] (6);
\draw[disagreement_arc] (12) -- (10);
\draw[disagreement_arc] (12) edge[bend left=30] (15);
\end{tikzpicture}
 &
        \begin{tikzpicture}
\def\r{0.15}
\node[cluster, minimum size=8.762*\r cm] (C0) at (0*\r, 0*\r) {};
\node[cluster, minimum size=3.000*\r cm] (C1) at (-6*\r, -5*\r) {};
\node[cluster, minimum size=3.000*\r cm] (C2) at (6*\r, -5*\r) {};
\node[vertex, fill=white, shift=(0.000:2.881*\r cm), minimum size=2.0*\r cm] (0) at (C0) {\scriptsize 0};
\node[vertex, fill=white, shift=(51.429:2.881*\r cm), minimum size=2.0*\r cm] (2) at (C0) {\scriptsize 2};
\node[vertex, fill=white, shift=(102.857:2.881*\r cm), minimum size=2.0*\r cm] (3) at (C0) {\scriptsize 3};
\node[vertex, fill=white, shift=(154.286:2.881*\r cm), minimum size=2.0*\r cm] (6) at (C0) {\scriptsize 6};
\node[vertex, fill=white, shift=(205.714:2.881*\r cm), minimum size=2.0*\r cm] (4) at (C0) {\scriptsize 4};
\node[vertex, fill=white, shift=(257.143:2.881*\r cm), minimum size=2.0*\r cm] (8) at (C0) {\scriptsize 8};
\node[vertex, fill=white, shift=(308.571:2.881*\r cm), minimum size=2.0*\r cm] (7) at (C0) {\scriptsize 7};
\node[vertex, fill=white, shift=(0.000:0.000*\r cm), minimum size=2.0*\r cm] (1) at (C1) {\scriptsize 1};
\node[vertex, fill=white, shift=(0.000:0.000*\r cm), minimum size=2.0*\r cm] (5) at (C2) {\scriptsize 5};
\draw[disagreement_arc] (0) -- (1);
\draw[disagreement_arc] (0) -- (5);
\draw[disagreement_arc] (2) -- (1);
\draw[disagreement_arc] (2) -- (5);
\draw[disagreement_arc] (3) edge[bend left=20] (4);
\draw[disagreement_arc] (3) -- (1);
\draw[disagreement_arc] (3) -- (5);
\draw[disagreement_arc] (4) -- (1);
\draw[disagreement_arc] (4) -- (5);
\draw[disagreement_arc] (6) -- (1);
\draw[disagreement_arc] (6) -- (5);
\draw[disagreement_arc] (7) -- (1);
\draw[disagreement_arc] (7) -- (5);
\draw[disagreement_arc] (8) -- (1);
\end{tikzpicture}
 &
        \begin{tikzpicture}
\def\r{0.15}
\node[cluster, minimum size=3.000*\r cm] (C0) at (0*\r, 11.9*\r) {};
\node[cluster, minimum size=3.000*\r cm] (C1) at (15*\r, 0*\r) {};
\node[cluster, minimum size=3.000*\r cm] (C2) at (5*\r, 10.2*\r) {};
\node[cluster, minimum size=3.000*\r cm] (C3) at (15*\r, 6.8*\r) {};
\node[cluster, minimum size=3.000*\r cm] (C4) at (10*\r, 8.5*\r) {};
\node[cluster, minimum size=3.000*\r cm] (C5) at (0*\r, 1.7*\r) {};
\node[cluster, minimum size=3.000*\r cm] (C6) at (10*\r, 5.1*\r) {};
\node[cluster, minimum size=3.000*\r cm] (C7) at (5*\r, 3.4*\r) {};
\node[cluster, minimum size=3.000*\r cm] (C8) at (10*\r, 1.7*\r) {};
\node[vertex, fill=white, shift=(0.000:0.000*\r cm), minimum size=2.0*\r cm] (0) at (C0) {\scriptsize 0};
\node[vertex, fill=white, shift=(0.000:0.000*\r cm), minimum size=2.0*\r cm] (1) at (C1) {\scriptsize 1};
\node[vertex, fill=white, shift=(0.000:0.000*\r cm), minimum size=2.0*\r cm] (2) at (C2) {\scriptsize 2};
\node[vertex, fill=white, shift=(0.000:0.000*\r cm), minimum size=2.0*\r cm] (3) at (C3) {\scriptsize 3};
\node[vertex, fill=white, shift=(0.000:0.000*\r cm), minimum size=2.0*\r cm] (4) at (C4) {\scriptsize 4};
\node[vertex, fill=white, shift=(0.000:0.000*\r cm), minimum size=2.0*\r cm] (5) at (C5) {\scriptsize 5};
\node[vertex, fill=white, shift=(0.000:0.000*\r cm), minimum size=2.0*\r cm] (6) at (C6) {\scriptsize 6};
\node[vertex, fill=white, shift=(0.000:0.000*\r cm), minimum size=2.0*\r cm] (7) at (C7) {\scriptsize 7};
\node[vertex, fill=white, shift=(0.000:0.000*\r cm), minimum size=2.0*\r cm] (8) at (C8) {\scriptsize 8};
\draw[cluster_arc] (C0) -- (C2);
\draw[cluster_arc] (C2) -- (C4);
\draw[cluster_arc] (C3) -- (C6);
\draw[cluster_arc] (C4) -- (C3);
\draw[cluster_arc] (C6) -- (C7);
\draw[cluster_arc] (C7) -- (C8);
\draw[cluster_arc] (C7) -- (C5);
\draw[cluster_arc] (C8) -- (C1);
\draw[disagreement_arc] (2) edge[bend left=20] (0);
\draw[disagreement_arc] (3) edge[bend right=35] (0);
\draw[disagreement_arc] (3) edge[bend right=30] (2);
\draw[disagreement_arc] (4) edge[bend right=30] (0);
\draw[disagreement_arc] (4) edge[bend left=20] (2);
\draw[disagreement_arc] (6) edge[bend left=10] (0);
\draw[disagreement_arc] (6) -- (2);
\draw[disagreement_arc] (6) edge[bend left=20] (3);
\draw[disagreement_arc] (6) -- (4);
\draw[disagreement_arc] (7) edge[bend left=20] (0);
\draw[disagreement_arc] (7) -- (2);
\draw[disagreement_arc] (7) edge[bend right=28] (3);
\draw[disagreement_arc] (7) -- (4);
\draw[disagreement_arc] (7) edge[bend left=20] (6);
\draw[disagreement_arc] (8) -- (0);
\draw[disagreement_arc] (8) -- (2);
\draw[disagreement_arc] (8) edge[bend right=20] (3);
\draw[disagreement_arc] (8) edge[bend right=40] (4);
\draw[disagreement_arc] (8) -- (6);
\draw[disagreement_arc] (8) edge[bend left=20] (7);
\end{tikzpicture}
 
        \\
        \begin{tikzpicture}[scale=0.6]
\pgfdeclarelayer{bg}
\pgfsetlayers{bg,main}
\node[vertex, fill=white, fill opacity=0.8, text opacity=1] (0) at (3.56, 4.1) {\scriptsize 0};
\node[vertex, fill=white, fill opacity=0.8, text opacity=1] (1) at (1.5, 3.91) {\scriptsize 1};
\node[vertex, fill=white, fill opacity=0.8, text opacity=1] (2) at (0.5, 3.37) {\scriptsize 2};
\node[vertex, fill=white, fill opacity=0.8, text opacity=1] (4) at (2.8, 1.21) {\scriptsize 4};
\node[vertex, fill=white, fill opacity=0.8, text opacity=1] (6) at (3.56, 2.29) {\scriptsize 6};
\node[vertex, fill=white, fill opacity=0.8, text opacity=1] (7) at (4.5, 0.2) {\scriptsize 7};
\node[vertex, fill=white, fill opacity=0.8, text opacity=1] (9) at (0.67, 0.2) {\scriptsize 9};
\node[vertex, fill=white, fill opacity=0.8, text opacity=1] (10) at (2.6, 3.5) {\scriptsize 10};
\node[vertex, fill=white, fill opacity=0.8, text opacity=1] (11) at (2.2, 2.6) {\scriptsize 11};
\node[vertex, fill=white, fill opacity=0.8, text opacity=1] (12) at (1.7, 1.3) {\scriptsize 12};
\node[vertex, fill=white, fill opacity=0.8, text opacity=1] (13) at (4.04, 2.83) {\scriptsize 13};
\node[vertex, fill=white, fill opacity=0.8, text opacity=1] (3) at (4.04, 1.75) {\scriptsize 3};
\node[vertex, fill=white, fill opacity=0.8, text opacity=1] (5) at (1.34, 0.67) {\scriptsize 5};
\node[vertex, fill=white, fill opacity=0.8, text opacity=1] (8) at (5.00, 1.21) {\scriptsize 8};
\begin{pgfonlayer}{bg}
\draw[-latex] (0) edge[bend left=10] (1);
\draw[-latex] (0) edge[bend left=10] (2);
\draw[-latex] (0) edge[bend left=10] (4);
\draw[-latex] (0) edge[bend left=10] (6);
\draw[-latex] (0) edge[bend left=10] (7);
\draw[-latex] (0) edge[bend left=10] (9);
\draw[-latex] (0) edge[bend left=10] (10);
\draw[-latex] (0) edge[bend left=10] (11);
\draw[-latex] (0) edge[bend left=10] (12);
\draw[-latex] (1) edge[bend left=10] (2);
\draw[-latex] (1) edge[bend left=10] (4);
\draw[-latex] (1) edge[bend left=10] (13);
\draw[-latex] (2) edge[bend left=10] (0);
\draw[-latex] (2) edge[bend left=10] (1);
\draw[-latex] (2) edge[bend left=10] (3);
\draw[-latex] (2) edge[bend left=10] (4);
\draw[-latex] (2) edge[bend left=10] (5);
\draw[-latex] (2) edge[bend left=10] (6);
\draw[-latex] (2) edge[bend left=10] (7);
\draw[-latex] (2) edge[bend left=10] (8);
\draw[-latex] (2) edge[bend left=10] (9);
\draw[-latex] (4) edge[bend left=10] (0);
\draw[-latex] (4) edge[bend left=10] (1);
\draw[-latex] (4) edge[bend left=10] (2);
\draw[-latex] (4) edge[bend left=10] (5);
\draw[-latex] (4) edge[bend left=10] (6);
\draw[-latex] (4) edge[bend left=10] (8);
\draw[-latex] (4) edge[bend left=10] (9);
\draw[-latex] (4) edge[bend left=10] (10);
\draw[-latex] (4) edge[bend left=10] (11);
\draw[-latex] (6) edge[bend left=10] (2);
\draw[-latex] (6) edge[bend left=10] (4);
\draw[-latex] (6) edge[bend left=10] (7);
\draw[-latex] (7) edge[bend left=10] (2);
\draw[-latex] (7) edge[bend left=10] (12);
\draw[-latex] (9) edge[bend left=10] (0);
\draw[-latex] (9) edge[bend left=10] (2);
\draw[-latex] (9) edge[bend left=10] (4);
\draw[-latex] (9) edge[bend left=10] (8);
\draw[-latex] (10) edge[bend left=10] (0);
\draw[-latex] (10) edge[bend left=10] (4);
\draw[-latex] (10) edge[bend left=10] (11);
\draw[-latex] (11) edge[bend left=10] (10);
\draw[-latex] (3) edge[bend left=10] (2);
\draw[-latex] (5) edge[bend left=10] (2);
\draw[-latex] (5) edge[bend left=10] (4);
\draw[-latex] (5) edge[bend left=10] (7);
\draw[-latex] (8) edge[bend right=10] (0);
\draw[-latex] (8) edge[bend left=10] (2);
\draw[-latex] (8) edge[bend left=10] (4);
\draw[-latex] (8) edge[bend left=10] (9);
\end{pgfonlayer}
\end{tikzpicture}
 &
        \begin{tikzpicture}[xscale=0.95]
\def\r{0.15}
\node[cluster, minimum size=3.000*\r cm] (C0) at (28*\r, 15*\r) {};
\node[cluster, minimum size=3.000*\r cm] (C1) at (32*\r, 6*\r) {};
\node[cluster, minimum size=7.253*\r cm] (C2) at (33*\r, 20*\r) {};
\node[cluster, minimum size=3.000*\r cm] (C3) at (28*\r, 6*\r) {};
\node[cluster, minimum size=3.000*\r cm] (C4) at (22*\r, 15*\r) {};
\node[cluster, minimum size=3.000*\r cm] (C5) at (23*\r, 10*\r) {};
\node[cluster, minimum size=3.000*\r cm] (C6) at (26.5*\r, 10*\r) {};
\node[cluster, minimum size=3.000*\r cm] (C7) at (37*\r, 10*\r) {};
\node[cluster, minimum size=3.000*\r cm] (C8) at (30*\r, 10*\r) {};
\node[cluster, minimum size=3.000*\r cm] (C9) at (12.5*\r, 10*\r) {};
\node[cluster, minimum size=3.000*\r cm] (C10) at (24*\r, 6*\r) {};
\node[cluster, minimum size=3.000*\r cm] (C11) at (19.5*\r, 10*\r) {};
\node[cluster, minimum size=3.000*\r cm] (C12) at (37*\r, 14.5*\r) {};
\node[cluster, minimum size=3.000*\r cm] (C13) at (33.5*\r, 10*\r) {};
\node[cluster, minimum size=3.000*\r cm] (C14) at (16*\r, 10*\r) {};
\node[vertex, fill=white, shift=(0.000:0.000*\r cm), minimum size=2.0*\r cm] (0) at (C0) {\scriptsize 0};
\node[vertex, fill=white, shift=(0.000:0.000*\r cm), minimum size=2.0*\r cm] (1) at (C1) {\scriptsize 1};
\node[vertex, fill=white, shift=(0.000:2.127*\r cm), minimum size=2.0*\r cm] (2) at (C2) {\scriptsize 2};
\node[vertex, fill=white, shift=(72.000:2.127*\r cm), minimum size=2.0*\r cm] (5) at (C2) {\scriptsize 5};
\node[vertex, fill=white, shift=(144.000:2.127*\r cm), minimum size=2.0*\r cm] (6) at (C2) {\scriptsize 6};
\node[vertex, fill=white, shift=(216.000:2.127*\r cm), minimum size=2.0*\r cm] (8) at (C2) {\scriptsize 8};
\node[vertex, fill=white, shift=(288.000:2.127*\r cm), minimum size=2.0*\r cm] (10) at (C2) {\scriptsize 10};
\node[vertex, fill=white, shift=(0.000:0.000*\r cm), minimum size=2.0*\r cm] (3) at (C3) {\scriptsize 3};
\node[vertex, fill=white, shift=(0.000:0.000*\r cm), minimum size=2.0*\r cm] (4) at (C4) {\scriptsize 4};
\node[vertex, fill=white, shift=(0.000:0.000*\r cm), minimum size=2.0*\r cm] (7) at (C5) {\scriptsize 7};
\node[vertex, fill=white, shift=(0.000:0.000*\r cm), minimum size=2.0*\r cm] (9) at (C6) {\scriptsize 9};
\node[vertex, fill=white, shift=(0.000:0.000*\r cm), minimum size=2.0*\r cm] (11) at (C7) {\scriptsize 11};
\node[vertex, fill=white, shift=(0.000:0.000*\r cm), minimum size=2.0*\r cm] (12) at (C8) {\scriptsize 12};
\node[vertex, fill=white, shift=(0.000:0.000*\r cm), minimum size=2.0*\r cm] (13) at (C9) {\scriptsize 13};
\node[vertex, fill=white, shift=(0.000:0.000*\r cm), minimum size=2.0*\r cm] (14) at (C10) {\scriptsize 14};
\node[vertex, fill=white, shift=(0.000:0.000*\r cm), minimum size=2.0*\r cm] (15) at (C11) {\scriptsize 15};
\node[vertex, fill=white, shift=(0.000:0.000*\r cm), minimum size=2.0*\r cm] (16) at (C12) {\scriptsize 16};
\node[vertex, fill=white, shift=(0.000:0.000*\r cm), minimum size=2.0*\r cm] (17) at (C13) {\scriptsize 17};
\node[vertex, fill=white, shift=(0.000:0.000*\r cm), minimum size=2.0*\r cm] (18) at (C14) {\scriptsize 18};
\draw[cluster_arc] (C0) -- (C5);
\draw[cluster_arc] (C0) -- (C6);
\draw[cluster_arc] (C0) -- (C7);
\draw[cluster_arc] (C0) -- (C8);
\draw[cluster_arc] (C0) -- (C13);
\draw[cluster_arc] (C2) -- (C4);
\draw[cluster_arc] (C2) -- (C7);
\draw[cluster_arc] (C2) -- (C12);
\draw[cluster_arc] (C4) -- (C5);
\draw[cluster_arc] (C4) -- (C6);
\draw[cluster_arc] (C4) -- (C8);
\draw[cluster_arc] (C4) -- (C9);
\draw[cluster_arc] (C4) -- (C11);
\draw[cluster_arc] (C4) -- (C14);
\draw[cluster_arc] (C8) -- (C1);
\draw[cluster_arc] (C8) -- (C10);
\draw[cluster_arc] (C8) -- (C3);
\draw[disagreement_arc] (0) edge[bend right=25] (5);
\draw[disagreement_arc] (0) edge[bend left=20] (6);
\draw[disagreement_arc] (0) edge[bend right=20] (10);
\draw[disagreement_arc] (0) -- (4);
\draw[disagreement_arc] (2) edge[bend left=10] (17);
\draw[disagreement_arc] (4) edge[out=60,in=165] (5);
\draw[disagreement_arc] (5) edge[out=-15,in=70] (17);
\draw[disagreement_arc] (6) edge[bend right=20] (13);
\draw[disagreement_arc] (8) -- (1);
\draw[disagreement_arc] (8) -- (2);
\draw[disagreement_arc] (8) edge[bend right=20] (4);
\draw[disagreement_arc] (12) -- (5);
\draw[disagreement_arc] (12) edge[out=90,in=240] (6);
\draw[disagreement_arc] (12) -- (10);
\draw[disagreement_arc] (12) edge[bend left=30] (15);
\end{tikzpicture}
 &
        \begin{tikzpicture}
\def\r{0.15}
\node[cluster, minimum size=8.762*\r cm] (C0) at (0*\r, 0*\r) {};
\node[cluster, minimum size=3.000*\r cm] (C1) at (-6*\r, -5*\r) {};
\node[cluster, minimum size=3.000*\r cm] (C2) at (6*\r, -5*\r) {};
\node[vertex, fill=white, shift=(0.000:2.881*\r cm), minimum size=2.0*\r cm] (0) at (C0) {\scriptsize 0};
\node[vertex, fill=white, shift=(51.429:2.881*\r cm), minimum size=2.0*\r cm] (2) at (C0) {\scriptsize 2};
\node[vertex, fill=white, shift=(102.857:2.881*\r cm), minimum size=2.0*\r cm] (3) at (C0) {\scriptsize 3};
\node[vertex, fill=white, shift=(154.286:2.881*\r cm), minimum size=2.0*\r cm] (6) at (C0) {\scriptsize 6};
\node[vertex, fill=white, shift=(205.714:2.881*\r cm), minimum size=2.0*\r cm] (4) at (C0) {\scriptsize 4};
\node[vertex, fill=white, shift=(257.143:2.881*\r cm), minimum size=2.0*\r cm] (8) at (C0) {\scriptsize 8};
\node[vertex, fill=white, shift=(308.571:2.881*\r cm), minimum size=2.0*\r cm] (7) at (C0) {\scriptsize 7};
\node[vertex, fill=white, shift=(0.000:0.000*\r cm), minimum size=2.0*\r cm] (1) at (C1) {\scriptsize 1};
\node[vertex, fill=white, shift=(0.000:0.000*\r cm), minimum size=2.0*\r cm] (5) at (C2) {\scriptsize 5};
\draw[disagreement_arc] (0) -- (1);
\draw[disagreement_arc] (0) -- (5);
\draw[disagreement_arc] (2) -- (1);
\draw[disagreement_arc] (2) -- (5);
\draw[disagreement_arc] (3) edge[bend left=20] (4);
\draw[disagreement_arc] (3) -- (1);
\draw[disagreement_arc] (3) -- (5);
\draw[disagreement_arc] (4) -- (1);
\draw[disagreement_arc] (4) -- (5);
\draw[disagreement_arc] (6) -- (1);
\draw[disagreement_arc] (6) -- (5);
\draw[disagreement_arc] (7) -- (1);
\draw[disagreement_arc] (7) -- (5);
\draw[disagreement_arc] (8) -- (1);
\end{tikzpicture}
 &
        \begin{tikzpicture}
\def\r{0.15}
\node[cluster, minimum size=3.000*\r cm] (C0) at (0*\r, 11.9*\r) {};
\node[cluster, minimum size=3.000*\r cm] (C1) at (15*\r, 0*\r) {};
\node[cluster, minimum size=3.000*\r cm] (C2) at (5*\r, 10.2*\r) {};
\node[cluster, minimum size=3.000*\r cm] (C3) at (15*\r, 6.8*\r) {};
\node[cluster, minimum size=3.000*\r cm] (C4) at (10*\r, 8.5*\r) {};
\node[cluster, minimum size=3.000*\r cm] (C5) at (0*\r, 1.7*\r) {};
\node[cluster, minimum size=3.000*\r cm] (C6) at (10*\r, 5.1*\r) {};
\node[cluster, minimum size=3.000*\r cm] (C7) at (5*\r, 3.4*\r) {};
\node[cluster, minimum size=3.000*\r cm] (C8) at (10*\r, 1.7*\r) {};
\node[vertex, fill=white, shift=(0.000:0.000*\r cm), minimum size=2.0*\r cm] (0) at (C0) {\scriptsize 0};
\node[vertex, fill=white, shift=(0.000:0.000*\r cm), minimum size=2.0*\r cm] (1) at (C1) {\scriptsize 1};
\node[vertex, fill=white, shift=(0.000:0.000*\r cm), minimum size=2.0*\r cm] (2) at (C2) {\scriptsize 2};
\node[vertex, fill=white, shift=(0.000:0.000*\r cm), minimum size=2.0*\r cm] (3) at (C3) {\scriptsize 3};
\node[vertex, fill=white, shift=(0.000:0.000*\r cm), minimum size=2.0*\r cm] (4) at (C4) {\scriptsize 4};
\node[vertex, fill=white, shift=(0.000:0.000*\r cm), minimum size=2.0*\r cm] (5) at (C5) {\scriptsize 5};
\node[vertex, fill=white, shift=(0.000:0.000*\r cm), minimum size=2.0*\r cm] (6) at (C6) {\scriptsize 6};
\node[vertex, fill=white, shift=(0.000:0.000*\r cm), minimum size=2.0*\r cm] (7) at (C7) {\scriptsize 7};
\node[vertex, fill=white, shift=(0.000:0.000*\r cm), minimum size=2.0*\r cm] (8) at (C8) {\scriptsize 8};
\draw[cluster_arc] (C0) -- (C2);
\draw[cluster_arc] (C2) -- (C4);
\draw[cluster_arc] (C3) -- (C6);
\draw[cluster_arc] (C4) -- (C3);
\draw[cluster_arc] (C6) -- (C7);
\draw[cluster_arc] (C7) -- (C8);
\draw[cluster_arc] (C7) -- (C5);
\draw[cluster_arc] (C8) -- (C1);
\draw[disagreement_arc] (2) edge[bend left=20] (0);
\draw[disagreement_arc] (3) edge[bend right=35] (0);
\draw[disagreement_arc] (3) edge[bend right=30] (2);
\draw[disagreement_arc] (4) edge[bend right=30] (0);
\draw[disagreement_arc] (4) edge[bend left=20] (2);
\draw[disagreement_arc] (6) edge[bend left=10] (0);
\draw[disagreement_arc] (6) -- (2);
\draw[disagreement_arc] (6) edge[bend left=20] (3);
\draw[disagreement_arc] (6) -- (4);
\draw[disagreement_arc] (7) edge[bend left=20] (0);
\draw[disagreement_arc] (7) -- (2);
\draw[disagreement_arc] (7) edge[bend right=28] (3);
\draw[disagreement_arc] (7) -- (4);
\draw[disagreement_arc] (7) edge[bend left=20] (6);
\draw[disagreement_arc] (8) -- (0);
\draw[disagreement_arc] (8) -- (2);
\draw[disagreement_arc] (8) edge[bend right=20] (3);
\draw[disagreement_arc] (8) edge[bend right=40] (4);
\draw[disagreement_arc] (8) -- (6);
\draw[disagreement_arc] (8) edge[bend left=20] (7);
\end{tikzpicture}
 
        \\
        \begin{tikzpicture}[scale=0.6]
\pgfdeclarelayer{bg}
\pgfsetlayers{bg,main}
\node[vertex, fill=white, fill opacity=0.8, text opacity=1] (0) at (3.56, 4.1) {\scriptsize 0};
\node[vertex, fill=white, fill opacity=0.8, text opacity=1] (1) at (1.5, 3.91) {\scriptsize 1};
\node[vertex, fill=white, fill opacity=0.8, text opacity=1] (2) at (0.5, 3.37) {\scriptsize 2};
\node[vertex, fill=white, fill opacity=0.8, text opacity=1] (4) at (2.8, 1.21) {\scriptsize 4};
\node[vertex, fill=white, fill opacity=0.8, text opacity=1] (6) at (3.56, 2.29) {\scriptsize 6};
\node[vertex, fill=white, fill opacity=0.8, text opacity=1] (7) at (4.5, 0.2) {\scriptsize 7};
\node[vertex, fill=white, fill opacity=0.8, text opacity=1] (9) at (0.67, 0.2) {\scriptsize 9};
\node[vertex, fill=white, fill opacity=0.8, text opacity=1] (10) at (2.6, 3.5) {\scriptsize 10};
\node[vertex, fill=white, fill opacity=0.8, text opacity=1] (11) at (2.2, 2.6) {\scriptsize 11};
\node[vertex, fill=white, fill opacity=0.8, text opacity=1] (12) at (1.7, 1.3) {\scriptsize 12};
\node[vertex, fill=white, fill opacity=0.8, text opacity=1] (13) at (4.04, 2.83) {\scriptsize 13};
\node[vertex, fill=white, fill opacity=0.8, text opacity=1] (3) at (4.04, 1.75) {\scriptsize 3};
\node[vertex, fill=white, fill opacity=0.8, text opacity=1] (5) at (1.34, 0.67) {\scriptsize 5};
\node[vertex, fill=white, fill opacity=0.8, text opacity=1] (8) at (5.00, 1.21) {\scriptsize 8};
\begin{pgfonlayer}{bg}
\draw[-latex] (0) edge[bend left=10] (1);
\draw[-latex] (0) edge[bend left=10] (2);
\draw[-latex] (0) edge[bend left=10] (4);
\draw[-latex] (0) edge[bend left=10] (6);
\draw[-latex] (0) edge[bend left=10] (7);
\draw[-latex] (0) edge[bend left=10] (9);
\draw[-latex] (0) edge[bend left=10] (10);
\draw[-latex] (0) edge[bend left=10] (11);
\draw[-latex] (0) edge[bend left=10] (12);
\draw[-latex] (1) edge[bend left=10] (2);
\draw[-latex] (1) edge[bend left=10] (4);
\draw[-latex] (1) edge[bend left=10] (13);
\draw[-latex] (2) edge[bend left=10] (0);
\draw[-latex] (2) edge[bend left=10] (1);
\draw[-latex] (2) edge[bend left=10] (3);
\draw[-latex] (2) edge[bend left=10] (4);
\draw[-latex] (2) edge[bend left=10] (5);
\draw[-latex] (2) edge[bend left=10] (6);
\draw[-latex] (2) edge[bend left=10] (7);
\draw[-latex] (2) edge[bend left=10] (8);
\draw[-latex] (2) edge[bend left=10] (9);
\draw[-latex] (4) edge[bend left=10] (0);
\draw[-latex] (4) edge[bend left=10] (1);
\draw[-latex] (4) edge[bend left=10] (2);
\draw[-latex] (4) edge[bend left=10] (5);
\draw[-latex] (4) edge[bend left=10] (6);
\draw[-latex] (4) edge[bend left=10] (8);
\draw[-latex] (4) edge[bend left=10] (9);
\draw[-latex] (4) edge[bend left=10] (10);
\draw[-latex] (4) edge[bend left=10] (11);
\draw[-latex] (6) edge[bend left=10] (2);
\draw[-latex] (6) edge[bend left=10] (4);
\draw[-latex] (6) edge[bend left=10] (7);
\draw[-latex] (7) edge[bend left=10] (2);
\draw[-latex] (7) edge[bend left=10] (12);
\draw[-latex] (9) edge[bend left=10] (0);
\draw[-latex] (9) edge[bend left=10] (2);
\draw[-latex] (9) edge[bend left=10] (4);
\draw[-latex] (9) edge[bend left=10] (8);
\draw[-latex] (10) edge[bend left=10] (0);
\draw[-latex] (10) edge[bend left=10] (4);
\draw[-latex] (10) edge[bend left=10] (11);
\draw[-latex] (11) edge[bend left=10] (10);
\draw[-latex] (3) edge[bend left=10] (2);
\draw[-latex] (5) edge[bend left=10] (2);
\draw[-latex] (5) edge[bend left=10] (4);
\draw[-latex] (5) edge[bend left=10] (7);
\draw[-latex] (8) edge[bend right=10] (0);
\draw[-latex] (8) edge[bend left=10] (2);
\draw[-latex] (8) edge[bend left=10] (4);
\draw[-latex] (8) edge[bend left=10] (9);
\end{pgfonlayer}
\end{tikzpicture}
 &
        \begin{tikzpicture}[xscale=0.95]
\def\r{0.15}
\node[cluster, minimum size=3.000*\r cm] (C0) at (28*\r, 15*\r) {};
\node[cluster, minimum size=3.000*\r cm] (C1) at (32*\r, 6*\r) {};
\node[cluster, minimum size=7.253*\r cm] (C2) at (33*\r, 20*\r) {};
\node[cluster, minimum size=3.000*\r cm] (C3) at (28*\r, 6*\r) {};
\node[cluster, minimum size=3.000*\r cm] (C4) at (22*\r, 15*\r) {};
\node[cluster, minimum size=3.000*\r cm] (C5) at (23*\r, 10*\r) {};
\node[cluster, minimum size=3.000*\r cm] (C6) at (26.5*\r, 10*\r) {};
\node[cluster, minimum size=3.000*\r cm] (C7) at (37*\r, 10*\r) {};
\node[cluster, minimum size=3.000*\r cm] (C8) at (30*\r, 10*\r) {};
\node[cluster, minimum size=3.000*\r cm] (C9) at (12.5*\r, 10*\r) {};
\node[cluster, minimum size=3.000*\r cm] (C10) at (24*\r, 6*\r) {};
\node[cluster, minimum size=3.000*\r cm] (C11) at (19.5*\r, 10*\r) {};
\node[cluster, minimum size=3.000*\r cm] (C12) at (37*\r, 14.5*\r) {};
\node[cluster, minimum size=3.000*\r cm] (C13) at (33.5*\r, 10*\r) {};
\node[cluster, minimum size=3.000*\r cm] (C14) at (16*\r, 10*\r) {};
\node[vertex, fill=white, shift=(0.000:0.000*\r cm), minimum size=2.0*\r cm] (0) at (C0) {\scriptsize 0};
\node[vertex, fill=white, shift=(0.000:0.000*\r cm), minimum size=2.0*\r cm] (1) at (C1) {\scriptsize 1};
\node[vertex, fill=white, shift=(0.000:2.127*\r cm), minimum size=2.0*\r cm] (2) at (C2) {\scriptsize 2};
\node[vertex, fill=white, shift=(72.000:2.127*\r cm), minimum size=2.0*\r cm] (5) at (C2) {\scriptsize 5};
\node[vertex, fill=white, shift=(144.000:2.127*\r cm), minimum size=2.0*\r cm] (6) at (C2) {\scriptsize 6};
\node[vertex, fill=white, shift=(216.000:2.127*\r cm), minimum size=2.0*\r cm] (8) at (C2) {\scriptsize 8};
\node[vertex, fill=white, shift=(288.000:2.127*\r cm), minimum size=2.0*\r cm] (10) at (C2) {\scriptsize 10};
\node[vertex, fill=white, shift=(0.000:0.000*\r cm), minimum size=2.0*\r cm] (3) at (C3) {\scriptsize 3};
\node[vertex, fill=white, shift=(0.000:0.000*\r cm), minimum size=2.0*\r cm] (4) at (C4) {\scriptsize 4};
\node[vertex, fill=white, shift=(0.000:0.000*\r cm), minimum size=2.0*\r cm] (7) at (C5) {\scriptsize 7};
\node[vertex, fill=white, shift=(0.000:0.000*\r cm), minimum size=2.0*\r cm] (9) at (C6) {\scriptsize 9};
\node[vertex, fill=white, shift=(0.000:0.000*\r cm), minimum size=2.0*\r cm] (11) at (C7) {\scriptsize 11};
\node[vertex, fill=white, shift=(0.000:0.000*\r cm), minimum size=2.0*\r cm] (12) at (C8) {\scriptsize 12};
\node[vertex, fill=white, shift=(0.000:0.000*\r cm), minimum size=2.0*\r cm] (13) at (C9) {\scriptsize 13};
\node[vertex, fill=white, shift=(0.000:0.000*\r cm), minimum size=2.0*\r cm] (14) at (C10) {\scriptsize 14};
\node[vertex, fill=white, shift=(0.000:0.000*\r cm), minimum size=2.0*\r cm] (15) at (C11) {\scriptsize 15};
\node[vertex, fill=white, shift=(0.000:0.000*\r cm), minimum size=2.0*\r cm] (16) at (C12) {\scriptsize 16};
\node[vertex, fill=white, shift=(0.000:0.000*\r cm), minimum size=2.0*\r cm] (17) at (C13) {\scriptsize 17};
\node[vertex, fill=white, shift=(0.000:0.000*\r cm), minimum size=2.0*\r cm] (18) at (C14) {\scriptsize 18};
\draw[cluster_arc] (C0) -- (C5);
\draw[cluster_arc] (C0) -- (C6);
\draw[cluster_arc] (C0) -- (C7);
\draw[cluster_arc] (C0) -- (C8);
\draw[cluster_arc] (C0) -- (C13);
\draw[cluster_arc] (C2) -- (C4);
\draw[cluster_arc] (C2) -- (C7);
\draw[cluster_arc] (C2) -- (C12);
\draw[cluster_arc] (C4) -- (C5);
\draw[cluster_arc] (C4) -- (C6);
\draw[cluster_arc] (C4) -- (C8);
\draw[cluster_arc] (C4) -- (C9);
\draw[cluster_arc] (C4) -- (C11);
\draw[cluster_arc] (C4) -- (C14);
\draw[cluster_arc] (C8) -- (C1);
\draw[cluster_arc] (C8) -- (C10);
\draw[cluster_arc] (C8) -- (C3);
\draw[disagreement_arc] (0) edge[bend right=25] (5);
\draw[disagreement_arc] (0) edge[bend left=20] (6);
\draw[disagreement_arc] (0) edge[bend right=20] (10);
\draw[disagreement_arc] (0) -- (4);
\draw[disagreement_arc] (2) edge[bend left=10] (17);
\draw[disagreement_arc] (4) edge[out=60,in=165] (5);
\draw[disagreement_arc] (5) edge[out=-15,in=70] (17);
\draw[disagreement_arc] (6) edge[bend right=20] (13);
\draw[disagreement_arc] (8) -- (1);
\draw[disagreement_arc] (8) -- (2);
\draw[disagreement_arc] (8) edge[bend right=20] (4);
\draw[disagreement_arc] (12) -- (5);
\draw[disagreement_arc] (12) edge[out=90,in=240] (6);
\draw[disagreement_arc] (12) -- (10);
\draw[disagreement_arc] (12) edge[bend left=30] (15);
\end{tikzpicture}
 &
        \begin{tikzpicture}
\def\r{0.15}
\node[cluster, minimum size=8.762*\r cm] (C0) at (0*\r, 0*\r) {};
\node[cluster, minimum size=3.000*\r cm] (C1) at (-6*\r, -5*\r) {};
\node[cluster, minimum size=3.000*\r cm] (C2) at (6*\r, -5*\r) {};
\node[vertex, fill=white, shift=(0.000:2.881*\r cm), minimum size=2.0*\r cm] (0) at (C0) {\scriptsize 0};
\node[vertex, fill=white, shift=(51.429:2.881*\r cm), minimum size=2.0*\r cm] (2) at (C0) {\scriptsize 2};
\node[vertex, fill=white, shift=(102.857:2.881*\r cm), minimum size=2.0*\r cm] (3) at (C0) {\scriptsize 3};
\node[vertex, fill=white, shift=(154.286:2.881*\r cm), minimum size=2.0*\r cm] (6) at (C0) {\scriptsize 6};
\node[vertex, fill=white, shift=(205.714:2.881*\r cm), minimum size=2.0*\r cm] (4) at (C0) {\scriptsize 4};
\node[vertex, fill=white, shift=(257.143:2.881*\r cm), minimum size=2.0*\r cm] (8) at (C0) {\scriptsize 8};
\node[vertex, fill=white, shift=(308.571:2.881*\r cm), minimum size=2.0*\r cm] (7) at (C0) {\scriptsize 7};
\node[vertex, fill=white, shift=(0.000:0.000*\r cm), minimum size=2.0*\r cm] (1) at (C1) {\scriptsize 1};
\node[vertex, fill=white, shift=(0.000:0.000*\r cm), minimum size=2.0*\r cm] (5) at (C2) {\scriptsize 5};
\draw[disagreement_arc] (0) -- (1);
\draw[disagreement_arc] (0) -- (5);
\draw[disagreement_arc] (2) -- (1);
\draw[disagreement_arc] (2) -- (5);
\draw[disagreement_arc] (3) edge[bend left=20] (4);
\draw[disagreement_arc] (3) -- (1);
\draw[disagreement_arc] (3) -- (5);
\draw[disagreement_arc] (4) -- (1);
\draw[disagreement_arc] (4) -- (5);
\draw[disagreement_arc] (6) -- (1);
\draw[disagreement_arc] (6) -- (5);
\draw[disagreement_arc] (7) -- (1);
\draw[disagreement_arc] (7) -- (5);
\draw[disagreement_arc] (8) -- (1);
\end{tikzpicture}
 &
        \begin{tikzpicture}
    \def\r{0.15}
    \node[cluster, minimum size=3.000*\r cm] (C0) at (13.5*\r, 10*\r) {};
    \node[cluster, minimum size=3.000*\r cm] (C1) at (9*\r, 7.5*\r) {};
    \node[cluster, minimum size=3.000*\r cm] (C2) at (0*\r, 12.5*\r) {};
    \node[cluster, minimum size=3.000*\r cm] (C3) at (4.5*\r, 5*\r) {};
    \node[cluster, minimum size=3.000*\r cm] (C4) at (13.5*\r, 0*\r) {};
    \node[cluster, minimum size=3.000*\r cm] (C5) at (18*\r, 12.5*\r) {};
    \node[cluster, minimum size=3.000*\r cm] (C6) at (9*\r, 2.5*\r) {};
    \node[cluster, minimum size=3.000*\r cm] (C7) at (22.5*\r, 10*\r) {};
    \node[cluster, minimum size=3.000*\r cm] (C8) at (18*\r, 5*\r) {};
    \node[cluster, minimum size=3.000*\r cm] (C9) at (4.5*\r, 10*\r) {};
    \node[vertex, fill=white, shift=(0.000:0.000*\r cm), minimum size=2.0*\r cm] (0) at (C0) {\scriptsize 0};
    \node[vertex, fill=white, shift=(0.000:0.000*\r cm), minimum size=2.0*\r cm] (1) at (C1) {\scriptsize 1};
    \node[vertex, fill=white, shift=(0.000:0.000*\r cm), minimum size=2.0*\r cm] (2) at (C2) {\scriptsize 2};
    \node[vertex, fill=white, shift=(0.000:0.000*\r cm), minimum size=2.0*\r cm] (3) at (C3) {\scriptsize 3};
    \node[vertex, fill=white, shift=(0.000:0.000*\r cm), minimum size=2.0*\r cm] (4) at (C4) {\scriptsize 4};
    \node[vertex, fill=white, shift=(0.000:0.000*\r cm), minimum size=2.0*\r cm] (5) at (C5) {\scriptsize 5};
    \node[vertex, fill=white, shift=(0.000:0.000*\r cm), minimum size=2.0*\r cm] (6) at (C6) {\scriptsize 6};
    \node[vertex, fill=white, shift=(0.000:0.000*\r cm), minimum size=2.0*\r cm] (7) at (C7) {\scriptsize 7};
    \node[vertex, fill=white, shift=(0.000:0.000*\r cm), minimum size=2.0*\r cm] (8) at (C8) {\scriptsize 8};
    \node[vertex, fill=white, shift=(0.000:0.000*\r cm), minimum size=2.0*\r cm] (9) at (C9) {\scriptsize 9};
    \draw[cluster_arc] (C0) -- (C1);
    \draw[cluster_arc] (C0) -- (C8);
    \draw[cluster_arc] (C1) -- (C3);
    \draw[cluster_arc] (C2) -- (C9);
    \draw[cluster_arc] (C3) -- (C6);
    \draw[cluster_arc] (C5) -- (C0);
    \draw[cluster_arc] (C5) -- (C7);
    \draw[cluster_arc] (C6) -- (C4);
    \draw[cluster_arc] (C8) -- (C4);
    \draw[cluster_arc] (C9) -- (C1);
    \draw[disagreement_arc] (0) edge[bend left=20] (5);
    \draw[disagreement_arc] (1) edge[bend right=20] (9);
    \draw[disagreement_arc] (3) edge[bend left=20] (1);
    \draw[disagreement_arc] (3) -- (9);
    \draw[disagreement_arc] (4) -- (1);
    \draw[disagreement_arc] (4) edge[bend left=50] (2);
    \draw[disagreement_arc] (4) edge[bend left=30] (3);
    \draw[disagreement_arc] (4) edge[bend left=20] (6);
    \draw[disagreement_arc] (4) -- (9);
    \draw[disagreement_arc] (6) -- (1);
    \draw[disagreement_arc] (6) edge[bend left=20] (3);
    \draw[disagreement_arc] (6) -- (9);
    \end{tikzpicture} 
        \\
        \begin{tikzpicture}[scale=0.6]
\pgfdeclarelayer{bg}
\pgfsetlayers{bg,main}
\node[vertex, fill=white, fill opacity=0.8, text opacity=1] (0) at (3.56, 4.1) {\scriptsize 0};
\node[vertex, fill=white, fill opacity=0.8, text opacity=1] (1) at (1.5, 3.91) {\scriptsize 1};
\node[vertex, fill=white, fill opacity=0.8, text opacity=1] (2) at (0.5, 3.37) {\scriptsize 2};
\node[vertex, fill=white, fill opacity=0.8, text opacity=1] (4) at (2.8, 1.21) {\scriptsize 4};
\node[vertex, fill=white, fill opacity=0.8, text opacity=1] (6) at (3.56, 2.29) {\scriptsize 6};
\node[vertex, fill=white, fill opacity=0.8, text opacity=1] (7) at (4.5, 0.2) {\scriptsize 7};
\node[vertex, fill=white, fill opacity=0.8, text opacity=1] (9) at (0.67, 0.2) {\scriptsize 9};
\node[vertex, fill=white, fill opacity=0.8, text opacity=1] (10) at (2.6, 3.5) {\scriptsize 10};
\node[vertex, fill=white, fill opacity=0.8, text opacity=1] (11) at (2.2, 2.6) {\scriptsize 11};
\node[vertex, fill=white, fill opacity=0.8, text opacity=1] (12) at (1.7, 1.3) {\scriptsize 12};
\node[vertex, fill=white, fill opacity=0.8, text opacity=1] (13) at (4.04, 2.83) {\scriptsize 13};
\node[vertex, fill=white, fill opacity=0.8, text opacity=1] (3) at (4.04, 1.75) {\scriptsize 3};
\node[vertex, fill=white, fill opacity=0.8, text opacity=1] (5) at (1.34, 0.67) {\scriptsize 5};
\node[vertex, fill=white, fill opacity=0.8, text opacity=1] (8) at (5.00, 1.21) {\scriptsize 8};
\begin{pgfonlayer}{bg}
\draw[-latex] (0) edge[bend left=10] (1);
\draw[-latex] (0) edge[bend left=10] (2);
\draw[-latex] (0) edge[bend left=10] (4);
\draw[-latex] (0) edge[bend left=10] (6);
\draw[-latex] (0) edge[bend left=10] (7);
\draw[-latex] (0) edge[bend left=10] (9);
\draw[-latex] (0) edge[bend left=10] (10);
\draw[-latex] (0) edge[bend left=10] (11);
\draw[-latex] (0) edge[bend left=10] (12);
\draw[-latex] (1) edge[bend left=10] (2);
\draw[-latex] (1) edge[bend left=10] (4);
\draw[-latex] (1) edge[bend left=10] (13);
\draw[-latex] (2) edge[bend left=10] (0);
\draw[-latex] (2) edge[bend left=10] (1);
\draw[-latex] (2) edge[bend left=10] (3);
\draw[-latex] (2) edge[bend left=10] (4);
\draw[-latex] (2) edge[bend left=10] (5);
\draw[-latex] (2) edge[bend left=10] (6);
\draw[-latex] (2) edge[bend left=10] (7);
\draw[-latex] (2) edge[bend left=10] (8);
\draw[-latex] (2) edge[bend left=10] (9);
\draw[-latex] (4) edge[bend left=10] (0);
\draw[-latex] (4) edge[bend left=10] (1);
\draw[-latex] (4) edge[bend left=10] (2);
\draw[-latex] (4) edge[bend left=10] (5);
\draw[-latex] (4) edge[bend left=10] (6);
\draw[-latex] (4) edge[bend left=10] (8);
\draw[-latex] (4) edge[bend left=10] (9);
\draw[-latex] (4) edge[bend left=10] (10);
\draw[-latex] (4) edge[bend left=10] (11);
\draw[-latex] (6) edge[bend left=10] (2);
\draw[-latex] (6) edge[bend left=10] (4);
\draw[-latex] (6) edge[bend left=10] (7);
\draw[-latex] (7) edge[bend left=10] (2);
\draw[-latex] (7) edge[bend left=10] (12);
\draw[-latex] (9) edge[bend left=10] (0);
\draw[-latex] (9) edge[bend left=10] (2);
\draw[-latex] (9) edge[bend left=10] (4);
\draw[-latex] (9) edge[bend left=10] (8);
\draw[-latex] (10) edge[bend left=10] (0);
\draw[-latex] (10) edge[bend left=10] (4);
\draw[-latex] (10) edge[bend left=10] (11);
\draw[-latex] (11) edge[bend left=10] (10);
\draw[-latex] (3) edge[bend left=10] (2);
\draw[-latex] (5) edge[bend left=10] (2);
\draw[-latex] (5) edge[bend left=10] (4);
\draw[-latex] (5) edge[bend left=10] (7);
\draw[-latex] (8) edge[bend right=10] (0);
\draw[-latex] (8) edge[bend left=10] (2);
\draw[-latex] (8) edge[bend left=10] (4);
\draw[-latex] (8) edge[bend left=10] (9);
\end{pgfonlayer}
\end{tikzpicture}
 &
        \begin{tikzpicture}[xscale=0.95]
\def\r{0.15}
\node[cluster, minimum size=3.000*\r cm] (C0) at (28*\r, 15*\r) {};
\node[cluster, minimum size=3.000*\r cm] (C1) at (32*\r, 6*\r) {};
\node[cluster, minimum size=7.253*\r cm] (C2) at (33*\r, 20*\r) {};
\node[cluster, minimum size=3.000*\r cm] (C3) at (28*\r, 6*\r) {};
\node[cluster, minimum size=3.000*\r cm] (C4) at (22*\r, 15*\r) {};
\node[cluster, minimum size=3.000*\r cm] (C5) at (23*\r, 10*\r) {};
\node[cluster, minimum size=3.000*\r cm] (C6) at (26.5*\r, 10*\r) {};
\node[cluster, minimum size=3.000*\r cm] (C7) at (37*\r, 10*\r) {};
\node[cluster, minimum size=3.000*\r cm] (C8) at (30*\r, 10*\r) {};
\node[cluster, minimum size=3.000*\r cm] (C9) at (12.5*\r, 10*\r) {};
\node[cluster, minimum size=3.000*\r cm] (C10) at (24*\r, 6*\r) {};
\node[cluster, minimum size=3.000*\r cm] (C11) at (19.5*\r, 10*\r) {};
\node[cluster, minimum size=3.000*\r cm] (C12) at (37*\r, 14.5*\r) {};
\node[cluster, minimum size=3.000*\r cm] (C13) at (33.5*\r, 10*\r) {};
\node[cluster, minimum size=3.000*\r cm] (C14) at (16*\r, 10*\r) {};
\node[vertex, fill=white, shift=(0.000:0.000*\r cm), minimum size=2.0*\r cm] (0) at (C0) {\scriptsize 0};
\node[vertex, fill=white, shift=(0.000:0.000*\r cm), minimum size=2.0*\r cm] (1) at (C1) {\scriptsize 1};
\node[vertex, fill=white, shift=(0.000:2.127*\r cm), minimum size=2.0*\r cm] (2) at (C2) {\scriptsize 2};
\node[vertex, fill=white, shift=(72.000:2.127*\r cm), minimum size=2.0*\r cm] (5) at (C2) {\scriptsize 5};
\node[vertex, fill=white, shift=(144.000:2.127*\r cm), minimum size=2.0*\r cm] (6) at (C2) {\scriptsize 6};
\node[vertex, fill=white, shift=(216.000:2.127*\r cm), minimum size=2.0*\r cm] (8) at (C2) {\scriptsize 8};
\node[vertex, fill=white, shift=(288.000:2.127*\r cm), minimum size=2.0*\r cm] (10) at (C2) {\scriptsize 10};
\node[vertex, fill=white, shift=(0.000:0.000*\r cm), minimum size=2.0*\r cm] (3) at (C3) {\scriptsize 3};
\node[vertex, fill=white, shift=(0.000:0.000*\r cm), minimum size=2.0*\r cm] (4) at (C4) {\scriptsize 4};
\node[vertex, fill=white, shift=(0.000:0.000*\r cm), minimum size=2.0*\r cm] (7) at (C5) {\scriptsize 7};
\node[vertex, fill=white, shift=(0.000:0.000*\r cm), minimum size=2.0*\r cm] (9) at (C6) {\scriptsize 9};
\node[vertex, fill=white, shift=(0.000:0.000*\r cm), minimum size=2.0*\r cm] (11) at (C7) {\scriptsize 11};
\node[vertex, fill=white, shift=(0.000:0.000*\r cm), minimum size=2.0*\r cm] (12) at (C8) {\scriptsize 12};
\node[vertex, fill=white, shift=(0.000:0.000*\r cm), minimum size=2.0*\r cm] (13) at (C9) {\scriptsize 13};
\node[vertex, fill=white, shift=(0.000:0.000*\r cm), minimum size=2.0*\r cm] (14) at (C10) {\scriptsize 14};
\node[vertex, fill=white, shift=(0.000:0.000*\r cm), minimum size=2.0*\r cm] (15) at (C11) {\scriptsize 15};
\node[vertex, fill=white, shift=(0.000:0.000*\r cm), minimum size=2.0*\r cm] (16) at (C12) {\scriptsize 16};
\node[vertex, fill=white, shift=(0.000:0.000*\r cm), minimum size=2.0*\r cm] (17) at (C13) {\scriptsize 17};
\node[vertex, fill=white, shift=(0.000:0.000*\r cm), minimum size=2.0*\r cm] (18) at (C14) {\scriptsize 18};
\draw[cluster_arc] (C0) -- (C5);
\draw[cluster_arc] (C0) -- (C6);
\draw[cluster_arc] (C0) -- (C7);
\draw[cluster_arc] (C0) -- (C8);
\draw[cluster_arc] (C0) -- (C13);
\draw[cluster_arc] (C2) -- (C4);
\draw[cluster_arc] (C2) -- (C7);
\draw[cluster_arc] (C2) -- (C12);
\draw[cluster_arc] (C4) -- (C5);
\draw[cluster_arc] (C4) -- (C6);
\draw[cluster_arc] (C4) -- (C8);
\draw[cluster_arc] (C4) -- (C9);
\draw[cluster_arc] (C4) -- (C11);
\draw[cluster_arc] (C4) -- (C14);
\draw[cluster_arc] (C8) -- (C1);
\draw[cluster_arc] (C8) -- (C10);
\draw[cluster_arc] (C8) -- (C3);
\draw[disagreement_arc] (0) edge[bend right=25] (5);
\draw[disagreement_arc] (0) edge[bend left=20] (6);
\draw[disagreement_arc] (0) edge[bend right=20] (10);
\draw[disagreement_arc] (0) -- (4);
\draw[disagreement_arc] (2) edge[bend left=10] (17);
\draw[disagreement_arc] (4) edge[out=60,in=165] (5);
\draw[disagreement_arc] (5) edge[out=-15,in=70] (17);
\draw[disagreement_arc] (6) edge[bend right=20] (13);
\draw[disagreement_arc] (8) -- (1);
\draw[disagreement_arc] (8) -- (2);
\draw[disagreement_arc] (8) edge[bend right=20] (4);
\draw[disagreement_arc] (12) -- (5);
\draw[disagreement_arc] (12) edge[out=90,in=240] (6);
\draw[disagreement_arc] (12) -- (10);
\draw[disagreement_arc] (12) edge[bend left=30] (15);
\end{tikzpicture}
 &
        \begin{tikzpicture}
\def\r{0.15}
\node[cluster, minimum size=8.762*\r cm] (C0) at (0*\r, 0*\r) {};
\node[cluster, minimum size=3.000*\r cm] (C1) at (-6*\r, -5*\r) {};
\node[cluster, minimum size=3.000*\r cm] (C2) at (6*\r, -5*\r) {};
\node[vertex, fill=white, shift=(0.000:2.881*\r cm), minimum size=2.0*\r cm] (0) at (C0) {\scriptsize 0};
\node[vertex, fill=white, shift=(51.429:2.881*\r cm), minimum size=2.0*\r cm] (2) at (C0) {\scriptsize 2};
\node[vertex, fill=white, shift=(102.857:2.881*\r cm), minimum size=2.0*\r cm] (3) at (C0) {\scriptsize 3};
\node[vertex, fill=white, shift=(154.286:2.881*\r cm), minimum size=2.0*\r cm] (6) at (C0) {\scriptsize 6};
\node[vertex, fill=white, shift=(205.714:2.881*\r cm), minimum size=2.0*\r cm] (4) at (C0) {\scriptsize 4};
\node[vertex, fill=white, shift=(257.143:2.881*\r cm), minimum size=2.0*\r cm] (8) at (C0) {\scriptsize 8};
\node[vertex, fill=white, shift=(308.571:2.881*\r cm), minimum size=2.0*\r cm] (7) at (C0) {\scriptsize 7};
\node[vertex, fill=white, shift=(0.000:0.000*\r cm), minimum size=2.0*\r cm] (1) at (C1) {\scriptsize 1};
\node[vertex, fill=white, shift=(0.000:0.000*\r cm), minimum size=2.0*\r cm] (5) at (C2) {\scriptsize 5};
\draw[disagreement_arc] (0) -- (1);
\draw[disagreement_arc] (0) -- (5);
\draw[disagreement_arc] (2) -- (1);
\draw[disagreement_arc] (2) -- (5);
\draw[disagreement_arc] (3) edge[bend left=20] (4);
\draw[disagreement_arc] (3) -- (1);
\draw[disagreement_arc] (3) -- (5);
\draw[disagreement_arc] (4) -- (1);
\draw[disagreement_arc] (4) -- (5);
\draw[disagreement_arc] (6) -- (1);
\draw[disagreement_arc] (6) -- (5);
\draw[disagreement_arc] (7) -- (1);
\draw[disagreement_arc] (7) -- (5);
\draw[disagreement_arc] (8) -- (1);
\end{tikzpicture}
 &
        \begin{tikzpicture}
\def\r{0.15}
\node[cluster, minimum size=3.000*\r cm] (C0) at (0*\r, 11.9*\r) {};
\node[cluster, minimum size=3.000*\r cm] (C1) at (15*\r, 0*\r) {};
\node[cluster, minimum size=3.000*\r cm] (C2) at (5*\r, 10.2*\r) {};
\node[cluster, minimum size=3.000*\r cm] (C3) at (15*\r, 6.8*\r) {};
\node[cluster, minimum size=3.000*\r cm] (C4) at (10*\r, 8.5*\r) {};
\node[cluster, minimum size=3.000*\r cm] (C5) at (0*\r, 1.7*\r) {};
\node[cluster, minimum size=3.000*\r cm] (C6) at (10*\r, 5.1*\r) {};
\node[cluster, minimum size=3.000*\r cm] (C7) at (5*\r, 3.4*\r) {};
\node[cluster, minimum size=3.000*\r cm] (C8) at (10*\r, 1.7*\r) {};
\node[vertex, fill=white, shift=(0.000:0.000*\r cm), minimum size=2.0*\r cm] (0) at (C0) {\scriptsize 0};
\node[vertex, fill=white, shift=(0.000:0.000*\r cm), minimum size=2.0*\r cm] (1) at (C1) {\scriptsize 1};
\node[vertex, fill=white, shift=(0.000:0.000*\r cm), minimum size=2.0*\r cm] (2) at (C2) {\scriptsize 2};
\node[vertex, fill=white, shift=(0.000:0.000*\r cm), minimum size=2.0*\r cm] (3) at (C3) {\scriptsize 3};
\node[vertex, fill=white, shift=(0.000:0.000*\r cm), minimum size=2.0*\r cm] (4) at (C4) {\scriptsize 4};
\node[vertex, fill=white, shift=(0.000:0.000*\r cm), minimum size=2.0*\r cm] (5) at (C5) {\scriptsize 5};
\node[vertex, fill=white, shift=(0.000:0.000*\r cm), minimum size=2.0*\r cm] (6) at (C6) {\scriptsize 6};
\node[vertex, fill=white, shift=(0.000:0.000*\r cm), minimum size=2.0*\r cm] (7) at (C7) {\scriptsize 7};
\node[vertex, fill=white, shift=(0.000:0.000*\r cm), minimum size=2.0*\r cm] (8) at (C8) {\scriptsize 8};
\draw[cluster_arc] (C0) -- (C2);
\draw[cluster_arc] (C2) -- (C4);
\draw[cluster_arc] (C3) -- (C6);
\draw[cluster_arc] (C4) -- (C3);
\draw[cluster_arc] (C6) -- (C7);
\draw[cluster_arc] (C7) -- (C8);
\draw[cluster_arc] (C7) -- (C5);
\draw[cluster_arc] (C8) -- (C1);
\draw[disagreement_arc] (2) edge[bend left=20] (0);
\draw[disagreement_arc] (3) edge[bend right=35] (0);
\draw[disagreement_arc] (3) edge[bend right=30] (2);
\draw[disagreement_arc] (4) edge[bend right=30] (0);
\draw[disagreement_arc] (4) edge[bend left=20] (2);
\draw[disagreement_arc] (6) edge[bend left=10] (0);
\draw[disagreement_arc] (6) -- (2);
\draw[disagreement_arc] (6) edge[bend left=20] (3);
\draw[disagreement_arc] (6) -- (4);
\draw[disagreement_arc] (7) edge[bend left=20] (0);
\draw[disagreement_arc] (7) -- (2);
\draw[disagreement_arc] (7) edge[bend right=28] (3);
\draw[disagreement_arc] (7) -- (4);
\draw[disagreement_arc] (7) edge[bend left=20] (6);
\draw[disagreement_arc] (8) -- (0);
\draw[disagreement_arc] (8) -- (2);
\draw[disagreement_arc] (8) edge[bend right=20] (3);
\draw[disagreement_arc] (8) edge[bend right=40] (4);
\draw[disagreement_arc] (8) -- (6);
\draw[disagreement_arc] (8) edge[bend left=20] (7);
\end{tikzpicture}
 \\
    \end{tabular}
    \caption{Depicted above on the left are the ego networks from  \Cref{tab:clustering-vs-ordering}. 
    Next to these networks, from left to right, are an optimal preorder, optimal clustering and optimal partial order.
    Disagreements with the social network are highlighted by red arcs.
    It can be observed that the preorder has less disagreement compared to the clustering and partial order.
    For the network at the top, for instance, the preorder has only two disagreeing arcs while the clustering has 14 disagreeing arcs and the partial order has 20 disagreeing arcs.}
    \label{fig:clustering-vs-ordering}
\end{figure}

\begin{table}[b]
    \caption{Exemplary results for the ego networks with IDs 734493, 15053535, 104324908, 126067398, 215824411, and 1015608534-43212199687.
    Reported are the number of nodes ($|V|$) and edges ($|E|$), the optimal values of the preordering ($\lesssim$), clustering ($\sim$) and partial ordering ($\leq$) problems, and the transitivity index $T$.}
    \label{tab:clustering-vs-ordering}
    \begin{center}
	\small   
    \begin{tabular}{lrrrrrrr}
    \toprule
    Platform & $|V|$ & $|E|$ & $\lesssim$ & $\sim$ & $\leq$ & T \\
    \midrule
    Twitter &  9 &  54 & 52 & 40 & 34 & 0.963 \\
    Twitter & 18 &  26 & 23 & 12 & 19 & 0.885 \\
    Twitter & 13 &  77 & 62 & 34 & 51 & 0.805 \\
    Twitter & 14 &  51 & 27 & 20 & 26 & 0.529 \\
    Twitter & 10 &  40 & 35 & 24 & 28 & 0.875 \\
    Google+ & 19 & 107 & 92 & 24 & 87 & 0.885 \\
    \bottomrule
    \end{tabular}
    \end{center}
    \vspace{-0.1in}
\end{table}

Objective values and runtimes for the $435$ out of $973$ instances that have been solved within a time limit of $500\,$s are visualized in \Cref{fig:clustering-vs-ordering-quantitative}.
%
The objective values are highest for the preordering problem and lowest for the clustering problem.
The values obtained by the successive approach are significantly smaller than those obtained by solving the preordering problem directly.
This can be understood as an advantage of solving a joint clustering and partial ordering problem over solving these problems successively.
%
On average, the runtime of the clustering problem is the lowest, while that of the partial ordering problem is the highest.
The runtime of successive clustering and partial ordering is only marginally greater than that of just clustering because the instance of the partial ordering problem on clusters is much smaller than the instance of the partial ordering problem on individual accounts.



\subsubsection{Optimal solutions and bounds}

Within the time limit of $500\,$s, the ILP algorithm finds optimal solutions to the instances of the preordering problem for $494$ out of $973$ Twitter ego networks.
For these instances, the average transitivity index $T$ is $0.577$.
For $186$ of these instances, the LP bound according to \Cref{sec:lp-algorithms} is tight.
For the $308$ instances for which the LP bound is not tight, the gap between the LP bound and the optimal solution is reduced by $30.1\%$ on average by the odd closed walk inequalities; cf.~\Cref{fig:gap} (left).

The local search algorithms \algacronym{GDC}, \algacronym{GAI}, and \algacronym{GDC+GAI} compute feasible solutions in polynomial time but are not guaranteed to find optimal solutions.
The values of these feasible solutions are lower bounds on the transitivity index.
Complementary to this, \algacronym{LP} computes upper bounds on the transitivity index.
Differences between these lower and upper bounds are shown in \Cref{fig:gap} (right).
On average, \algacronym{GDC-GAI} comes closest to the LP bound with a mean difference of $0.054$.
While it is \textsc{NP}-hard to compute the transitivity index exactly, this result shows that it can be estimated closely in polynomial time for these instances.

\begin{figure}
\centering   
\begin{tikzpicture}
\small
\begin{axis}[
	xlabel=Fraction of LP gap closed,
	ylabel=Instances,
	width=0.45\columnwidth,
	height=0.3\columnwidth,
]
\pgfplotstableread{data/closed-gap.csv}\mydata;
\pgfplotstablecreatecol[create col/expr={\thisrowno{0}/100}]{a}\mydata
\addplot[hist] table [y index=1] {\mydata};
\draw[very thick] (0.301, 0) -- (0.301, 107);
\end{axis}
\end{tikzpicture}
\hspace{5ex}
\begin{tikzpicture}
\small
\begin{axis}[
	xlabel=Transitivity index gap,
	ylabel=Instances,
	xmax=0.6,
	% ymax=300,
	width=0.45\columnwidth,
	height=0.3\columnwidth,
	legend pos=north east,	
	legend cell align=left,
]

\pgfplotstableread[col sep=comma]{data/lower-vs-upper-bound.csv}\mydata;
\addplot[hist={bins=30, data max=0.55}, green, fill=none] table [y index=0] {\mydata};
\draw[very thick, green] (0.108, -10) -- (0.108, 210);
\addplot[hist={bins=30, data max=0.55}, red, fill=none] table [y index=1] {\mydata};
\draw[very thick, red] (0.151, -10) -- (0.151, 210);
\addplot[hist={bins=30, data max=0.55}, blue, fill=none] table [y index=2] {\mydata};
\draw[very thick, blue] (0.054, -10) -- (0.054, 210);
\legend{\algacronym{GDC}, \algacronym{GAI}, \algacronym{GAI+GDC}}
\end{axis}
\end{tikzpicture}
\caption{Shown on the left is the fraction of the LP gap that is closed by including odd closed walk inequalities.
On average, $30.1\%$ of the gap is closed.
This evaluation is restricted to those $308$ instances from the Twitter dataset for which the optimal solution is computed within a $500\,$s time limit and the canonical LP bound is not tight.
Shown on the right are histograms of differences between upper and lower bound on the transitivity index of Twitter ego networks.
The upper bounds are computed with \algacronym{LP}. The lower bounds are computed with \algacronym{GDC}, \algacronym{GAI}, and \algacronym{GDC+GAI}, respectively.
The mean differences are $0.108$, $0.151$, and $0.054$, respectively.}
\label{fig:gap}
\end{figure}



\subsubsection{Efficiency of local search algorithms}

We analyze the empirical runtime of the algorithms \algacronym{GDC}, \algacronym{GAI}, and \algacronym{GDC+GAI} on the much larger Google+ ego networks.
The largest network contains $|V| =\;$4,938 nodes, which results in an instance of the preordering problem of size $|V_2| =\;$24,378,906.
The runtimes are shown in \Cref{fig:local-search}.
The fastest algorithm is \algacronym{GDC}.
This is expected as its runtime is linear in the input size (cf. \Cref{sec:algorithms}).
Even for the largest instance, it terminates in less than one second.
The greedy arc insertion algorithm (\algacronym{GAI}), run on a CPU, terminates within the time limit of $100\,$s for instances of input size up to $2 \cdot 10^6$.
When executed on a GPU, its runtime decreases by a factor greater than $50$ and terminates within the time limit for instances with input size up to $10^7$.
The combined algorithm \algacronym{GDC+GAI} runs faster than \algacronym{GAI}.
The reason is that \algacronym{GAI} executes fewer iterations when initialized with the solution output by \algacronym{GDC}.
On the largest instances, the combined algorithm terminates well within the time limit. 
Together with the observations from the previous paragraph, we conclude that the combined algorithm \algacronym{GDC+GAI} computes the best solutions among the local search algorithms while still scaling to large instances.

\begin{figure}
    \centering
    \begin{tikzpicture}
\small
\begin{axis}[
	xlabel=$|V_2|$,
	ylabel={Runtime [s]},
	xmode=log,
	ymode=log,
	width=0.5\columnwidth,
	height=0.4\columnwidth,
	legend pos=north west,	
	legend cell align={left},
	mark size=0.15ex,
]
\pgfplotstableread[col sep=comma]{data/gplus-runtimes.csv}\mydata;
\addplot [color=red, only marks, mark=*]
	table [x expr=\thisrowno{0}*(\thisrowno{0}-1)/2, y index=4] {\mydata};
\addplot [color=orange, only marks, mark=*]
	table [x expr=\thisrowno{0}*(\thisrowno{0}-1)/2, y index=3] {\mydata};
\addplot [color=blue, only marks, mark=*]
	table [x expr=\thisrowno{0}*(\thisrowno{0}-1)/2, y index=2] {\mydata};
\addplot [color=green, only marks, mark=*]
	table [x expr=\thisrowno{0}*(\thisrowno{0}-1)/2, y index=1] {\mydata};
\legend{\algacronym{GAI}, \algacronym{GAI GPU}, \algacronym{GDC+GAI}, \algacronym{GDC}}
\end{axis}
\end{tikzpicture}

    \caption{Runtimes of the algorithms \algacronym{GDC}, \algacronym{GAI}, and \algacronym{GDC+GAI} as a function of input size $|V_2|$.
    Results are shown for runs terminating within $100$s.}
    \label{fig:local-search}
\end{figure}




\subsection{Twitter interaction network of the 117th US Congress}

In this section we perform a qualitative analysis on a Twitter network of members of the 117th Congress of the United States.
This network has been published by \citet{fink2023centrality} who have measured for each pair of Twitter accounts of Members of Congress the likelihood of one account reacting to a contribution of the other account.
The thusly obtained digraph consists of 475 nodes and 13,289 directed edges with non-zero likelihood.
In contrast to the ego networks from the previous section, this network is not anonymized.
This allows us to qualitatively interpret the results.
We construct an instance of the preordering problem by defining arc values $c$ to be the reaction likelihood minus $0.01$ (a subjective choice).
With this, the arc values reward arcs from $i$ to $j$ to be part of the preorder if $i$ reacts on more than $0.01$ of the contributions of $j$ and penalizes it otherwise.

For this instance of the preordering problem, the optimal solution value is $\maxpo(c) = 11.875$ and an optimal solution is depicted in \Cref{fig:congress-preorder}.
It is found by our ILP algorithm in $4.9\,$s.
The upper bound is $B(c) = 15.017$, which results in a transitivity index $T(c) = 0.791$.
In visualization of the preorder in \Cref{fig:congress-preorder}, nodes are arranged in layers and all arcs are directed from left to right. 
The further a node is to the left, the more the corresponding account reacts to others. 
The further a node is to the right, the more reactions the corresponding account gets from others.
One can see that accounts react more to accounts from members of the same political party.
There are two layers with particularly many nodes. 
This is expected as dicuts are transitive (cf. \Cref{sec:di-cut-approx}).
However, there are many nodes not contained in the two large layers (i.e., the optimal solution is not close to a dicut).
This is an indication that the interactions exhibit non-trivial transitivity.
The node with the most incoming arcs (i.e., the account that gets the most reactions) is that of Kevin McCarthy (labeled with \textit{a} in \Cref{fig:congress-preorder}), the Republican minority leader in the House of Representatives of the 117th US Congress.
There is a cluster of three accounts that is isolated from the rest (labeled with \textit{b} in \Cref{fig:congress-preorder}). 
These accounts belong to Tom Carper, Chris Coons and Lisa Blunt Rochester, who are the three Members of Congress from Delaware.
The node labeled with \textit{c} in \Cref{fig:congress-preorder} corresponds to the account of Bernie Sanders, who is a party independent member and thus expected to have fewer incoming arcs.

If we require the preorder to be symmetric (i.e., an equivalence relation/clustering), the optimal value is $5.256$.
This value being much smaller than $\maxpo(c) = 11.875$ suggests that preorders are a better fit to this network than equivalence relations.

\begin{figure}
    \centering
    \begin{tikzpicture}[xscale=2, yscale=0.103]
\pgfdeclarelayer{bg}
\pgfsetlayers{bg,main}
\node[node,color={rgb:dem,1;rep,0;ind,0}, minimum size=1.5pt] (0) at (-4.68, 50.0) {};
\node[node,color={rgb:dem,0;rep,1;ind,0}, minimum size=1.5pt] (1) at (-2.34, 7.646) {};
\node[node,color={rgb:dem,2;rep,0;ind,0}, minimum size=2.121pt] (2) at (-3.56, 82.526) {};
\node[node,color={rgb:dem,0;rep,1;ind,0}, minimum size=1.5pt] (3) at (-2.34, 2.876) {};
\node[node,color={rgb:dem,1;rep,0;ind,0}, minimum size=1.5pt] (4) at (-3.56, 4.016) {};
\node[node,color={rgb:dem,0;rep,1;ind,0}, minimum size=1.5pt] (5) at (-3.56, 14.356) {};
\node[node,color={rgb:dem,1;rep,0;ind,0}, minimum size=1.5pt] (6) at (-4.28, 13.256) {};
\node[node,color={rgb:dem,0;rep,1;ind,0}, minimum size=1.5pt] (7) at (-2.34, 3.976) {};
\node[node,color={rgb:dem,0;rep,1;ind,0}, minimum size=1.5pt] (8) at (-1.62, 104.89) {};
\node[node,color={rgb:dem,1;rep,0;ind,0}, minimum size=1.5pt] (9) at (-3.56, 20.196) {};
\node[node,color={rgb:dem,1;rep,0;ind,0}, minimum size=1.5pt] (10) at (-3.56, 54.276) {};
\node[node,color={rgb:dem,0;rep,1;ind,0}, minimum size=1.5pt] (11) at (-2.34, 109.89) {};
\node[node,color={rgb:dem,1;rep,0;ind,0}, minimum size=1.5pt] (12) at (-2.34, 0.296) {};
\node[node,color={rgb:dem,3;rep,0;ind,0}, minimum size=2.598pt] (13) at (-4.68, 52.0) {};
\node[node,color={rgb:dem,1;rep,0;ind,0}, minimum size=1.5pt] (14) at (-2.34, 34.426) {};
\node[node,color={rgb:dem,0;rep,1;ind,0}, minimum size=1.5pt] (15) at (-2.34, 12.386) {};
\node[node,color={rgb:dem,0;rep,1;ind,0}, minimum size=1.5pt] (16) at (-2.34, 13.486) {};
\node[node,color={rgb:dem,2;rep,0;ind,0}, minimum size=2.121pt] (17) at (-3.56, 99.396) {};
\node[node,color={rgb:dem,0;rep,1;ind,0}, minimum size=1.5pt] (18) at (-4.28, 9.106) {};
\node[node,color={rgb:dem,0;rep,1;ind,0}, minimum size=1.5pt] (19) at (-2.34, 20.186) {};
\node[node,color={rgb:dem,0;rep,2;ind,0}, minimum size=2.121pt] (20) at (-1.62, 6.536) {};
\node[node,color={rgb:dem,0;rep,1;ind,0}, minimum size=1.5pt] (21) at (-2.34, 132.79) {};
\node[node,color={rgb:dem,0;rep,1;ind,0}, minimum size=1.5pt] (22) at (-3.56, 12.536) {};
\node[node,color={rgb:dem,2;rep,0;ind,0}, minimum size=2.121pt] (23) at (-3.56, 36.356) {};
\node[node,color={rgb:dem,0;rep,1;ind,0}, minimum size=1.5pt] (24) at (-2.34, 10.186) {};
\node[node,color={rgb:dem,1;rep,0;ind,0}, minimum size=1.5pt] (25) at (-3.56, 16.896) {};
\node[node,color={rgb:dem,0;rep,1;ind,0}, minimum size=1.5pt] (26) at (-3.56, 148.51) {};
\node[node,color={rgb:dem,1;rep,0;ind,0}, minimum size=1.5pt] (27) at (-3.56, 8.366) {};
\node[node,color={rgb:dem,0;rep,1;ind,0}, minimum size=1.5pt] (28) at (-2.34, 16.446) {};
\node[node,color={rgb:dem,0;rep,1;ind,0}, minimum size=1.5pt] (29) at (-3.56, 10.376) {};
\node[node,color={rgb:dem,0;rep,1;ind,0}, minimum size=1.5pt] (30) at (-3.56, 11.816) {};
\node[node,color={rgb:dem,1;rep,0;ind,0}, minimum size=1.5pt] (31) at (-4.68, 1) {};
\node[node,color={rgb:dem,0;rep,1;ind,0}, minimum size=1.5pt] (32) at (-3.56, 158.62) {};
\node[node,color={rgb:dem,1;rep,0;ind,0}, minimum size=1.5pt] (33) at (-3.56, 19.476) {};
\node[node,color={rgb:dem,1;rep,0;ind,0}, minimum size=1.5pt] (34) at (-4.28, 63.876) {};
\node[node,color={rgb:dem,0;rep,1;ind,0}, minimum size=1.5pt] (35) at (-3.56, 18.756) {};
\node[node,color={rgb:dem,0;rep,1;ind,0}, minimum size=1.5pt] (36) at (-3.56, 97.916) {};
\node[node,color={rgb:dem,0;rep,1;ind,0}, minimum size=1.5pt] (37) at (-3.56, 97.196) {};
\node[node,color={rgb:dem,0;rep,1;ind,0}, minimum size=1.5pt] (38) at (-2.34, 9.466) {};
\node[node,color={rgb:dem,1;rep,0;ind,0}, minimum size=1.5pt] (39) at (-4.28, 31.466) {};
\node[node,color={rgb:dem,0;rep,1;ind,0}, minimum size=1.5pt] (40) at (-4.68, 130.0) {};
\node[node,color={rgb:dem,0;rep,1;ind,0}, minimum size=1.5pt] (41) at (-2.34, 75.116) {};
\node[node,color={rgb:dem,0;rep,0;ind,1}, minimum size=1.5pt] (42) at (-4.28, 12.536) {};
\node[node,color={rgb:dem,1;rep,0;ind,0}, minimum size=1.5pt] (43) at (-2.34, 11.666) {};
\node[node,color={rgb:dem,0;rep,1;ind,0}, minimum size=1.5pt] (44) at (-2.34, 131.31) {};
\node[node,color={rgb:dem,1;rep,0;ind,0}, minimum size=1.5pt] (45) at (-2.34, 64.206) {};
\node[node,color={rgb:dem,0;rep,1;ind,0}, minimum size=1.5pt] (46) at (-3.56, 132.42) {};
\node[node,color={rgb:dem,1;rep,0;ind,0}, minimum size=1.5pt] (47) at (-4.28, 19.476) {};
\node[node,color={rgb:dem,0;rep,1;ind,0}, minimum size=1.5pt] (48) at (-2.34, 8.366) {};
\node[node,color={rgb:dem,1;rep,0;ind,0}, minimum size=1.5pt] (49) at (-3.56, 47.566) {};
\node[node,color={rgb:dem,1;rep,0;ind,0}, minimum size=1.5pt] (50) at (-3.56, 30.346) {};
\node[node,color={rgb:dem,1;rep,0;ind,0}, minimum size=1.5pt] (51) at (-3.56, 16.176) {};
\node[node,color={rgb:dem,1;rep,0;ind,0}, minimum size=1.5pt] (52) at (-3.56, 75.366) {};
\node[node,color={rgb:dem,0;rep,1;ind,0}, minimum size=1.5pt] (53) at (-3.56, 38.556) {};
\node[node,color={rgb:dem,0;rep,1;ind,0}, minimum size=1.5pt] (54) at (-2.34, 46.876) {};
\node[node,color={rgb:dem,1;rep,0;ind,0}, minimum size=1.5pt] (55) at (-2.34, 2.156) {};
\node[node,color={rgb:dem,1;rep,0;ind,0}, minimum size=1.5pt] (56) at (-3.56, 37.076) {};
\node[node,color={rgb:dem,1;rep,0;ind,0}, minimum size=1.5pt] (57) at (-4.68, 54.0) {};
\node[node,color={rgb:dem,1;rep,0;ind,0}, minimum size=1.5pt] (58) at (-1.62, 18.126) {};
\node[node,color={rgb:dem,0;rep,1;ind,0}, minimum size=1.5pt] (59) at (-3.56, 167.99) {};
\node[node,color={rgb:dem,1;rep,0;ind,0}, minimum size=1.5pt] (60) at (-3.56, 15.076) {};
\node[node,color={rgb:dem,0;rep,1;ind,0}, minimum size=1.5pt] (61) at (-2.34, 21.286) {};
\node[node,color={rgb:dem,1;rep,0;ind,0}, minimum size=1.5pt] (62) at (-4.28, 85.336) {};
\node[node,color={rgb:dem,0;rep,1;ind,0}, minimum size=1.5pt] (63) at (-3.56, 13.256) {};
\node[node,color={rgb:dem,0;rep,1;ind,0}, minimum size=1.5pt] (64) at (-2.34, 5.416) {};
\node[node,color={rgb:dem,0;rep,0;ind,1}, minimum size=1.5pt] (65) at (-4.68, 29.176) {};
\node[node,color={rgb:dem,1;rep,0;ind,0}, minimum size=1.5pt] (66) at (-2.34, 4.696) {};
\node[node,color={rgb:dem,0;rep,1;ind,0}, minimum size=1.5pt] (67) at (-1.62, 9.466) {};
\node[node,color={rgb:dem,0;rep,1;ind,0}, minimum size=1.5pt] (68) at (-3.56, 45.706) {};
\node[node,color={rgb:dem,1;rep,1;ind,0}, minimum size=2.121pt] (69) at (-3.56, 3.296) {};
\node[node,color={rgb:dem,1;rep,0;ind,0}, minimum size=1.5pt] (70) at (-4.68, 56.0) {};
\node[node,color={rgb:dem,1;rep,0;ind,0}, minimum size=1.5pt] (71) at (-3.56, 11.096) {};
\node[node,color={rgb:dem,1;rep,0;ind,0}, minimum size=1.5pt] (72) at (-2.34, 14.586) {};
\node[node,color={rgb:dem,0;rep,1;ind,0}, minimum size=1.5pt] (73) at (-3.56, 48.286) {};
\node[node,color={rgb:dem,1;rep,0;ind,0}, minimum size=1.5pt] (74) at (-2.34, 23.516) {};
\node[node,color={rgb:dem,0;rep,1;ind,0}, minimum size=1.5pt] (75) at (-3.56, 9.656) {};
\node[node,color={rgb:dem,0;rep,1;ind,0}, minimum size=1.5pt] (76) at (-3.56, 5.006) {};
\node[node,color={rgb:dem,0;rep,1;ind,0}, minimum size=1.5pt] (77) at (-4.68, 132.0) {};
\node[node,color={rgb:dem,1;rep,0;ind,0}, minimum size=1.5pt] (78) at (-4.28, 3) {};
\node[node,color={rgb:dem,1;rep,0;ind,0}, minimum size=1.5pt] (79) at (-3.56, 50.146) {};
\node[node,color={rgb:dem,1;rep,0;ind,0}, minimum size=1.5pt] (80) at (-4.28, 21.896) {};
\node[node,color={rgb:dem,1;rep,0;ind,0}, minimum size=1.5pt] (81) at (-4.28, 23.746) {};
\node[node,color={rgb:dem,1;rep,0;ind,0}, minimum size=1.5pt] (82) at (-3.56, 23.176) {};
\node[node,color={rgb:dem,0;rep,1;ind,0}, minimum size=1.5pt] (83) at (-2.34, 95.586) {};
\node[node,color={rgb:dem,1;rep,0;ind,0}, minimum size=1.5pt] (84) at (-4.28, 75.366) {};
\node[node,color={rgb:dem,0;rep,1;ind,0}, minimum size=1.5pt] (85) at (-2.34, 104.23) {};
\node[node,color={rgb:dem,1;rep,0;ind,0}, minimum size=1.5pt] (86) at (-1.62, 65.516) {};
\node[node,color={rgb:dem,0;rep,1;ind,0}, minimum size=1.5pt] (87) at (-2.34, 138.44) {};
\node[node,color={rgb:dem,1;rep,0;ind,0}, minimum size=1.5pt] (88) at (-2.34, 27.266) {};
\node[node,color={rgb:dem,0;rep,1;ind,0}, minimum size=1.5pt] (89) at (-3.56, 146.52) {};
\node[node,color={rgb:dem,1;rep,0;ind,0}, minimum size=1.5pt] (90) at (-2.34, 80.056) {};
\node[node,color={rgb:dem,0;rep,1;ind,0}, minimum size=1.5pt] (91) at (-2.34, 72.496) {};
\node[node,color={rgb:dem,0;rep,1;ind,0}, minimum size=1.5pt] (92) at (-3.56, 106.13) {};
\node[node,color={rgb:dem,1;rep,0;ind,0}, minimum size=1.5pt] (93) at (-2.34, 94.866) {};
\node[node,color={rgb:dem,1;rep,0;ind,0}, minimum size=1.5pt] (94) at (-3.56, 111.0) {};
\node[node,color={rgb:dem,0;rep,1;ind,0}, minimum size=1.5pt] (95) at (-3.56, 161.6) {};
\node[node,color={rgb:dem,0;rep,1;ind,0}, minimum size=1.5pt] (96) at (-2.34, 175.45) {};
\node[node,color={rgb:dem,0;rep,1;ind,0}, minimum size=1.5pt] (97) at (-2.34, 147.86) {};
\node[node,color={rgb:dem,0;rep,1;ind,0}, minimum size=1.5pt] (98) at (-2.34, 118.92) {};
\node[node,color={rgb:dem,0;rep,1;ind,0}, minimum size=1.5pt] (99) at (-2.34, 159.56) {};
\node[node,color={rgb:dem,0;rep,1;ind,0}, minimum size=1.5pt] (100) at (-3.56, 84.016) {};
\node[node,color={rgb:dem,1;rep,0;ind,0}, minimum size=1.5pt] (101) at (-2.34, 45.756) {};
\node[node,color={rgb:dem,1;rep,0;ind,0}, minimum size=1.5pt] (102) at (-3.56, 103.14) {};
\node[node,color={rgb:dem,1;rep,0;ind,0}, minimum size=1.5pt] (103) at (-3.56, 40.046) {};
\node[node,color={rgb:dem,1;rep,0;ind,0}, minimum size=1.5pt] (104) at (-4.68, 58.0) {};
\node[node,color={rgb:dem,1;rep,0;ind,0}, minimum size=1.5pt] (105) at (-1.62, 41.216) {};
\node[node,color={rgb:dem,0;rep,3;ind,0}, minimum size=2.598pt] (106) at (-3.56, 117.89) {};
\node[node,color={rgb:dem,0;rep,1;ind,0}, minimum size=1.5pt] (107) at (-1.26, 168.08) {};
\node[node,color={rgb:dem,0;rep,1;ind,0}, minimum size=1.5pt] (108) at (-2.34, 127.19) {};
\node[node,color={rgb:dem,1;rep,0;ind,0}, minimum size=1.5pt] (109) at (-3.56, 102.39) {};
\node[node,color={rgb:dem,0;rep,1;ind,0}, minimum size=1.5pt] (110) at (-2.34, 158.82) {};
\node[node,color={rgb:dem,0;rep,1;ind,0}, minimum size=1.5pt] (111) at (-2.34, 170.55) {};
\node[node,color={rgb:dem,1;rep,0;ind,0}, minimum size=1.5pt] (112) at (-2.34, 76.996) {};
\node[node,color={rgb:dem,0;rep,1;ind,0}, minimum size=1.5pt] (113) at (-3.56, 98.656) {};
\node[node,color={rgb:dem,1;rep,0;ind,0}, minimum size=1.5pt] (114) at (-2.34, 61.576) {};
\node[node,color={rgb:dem,1;rep,0;ind,0}, minimum size=1.5pt] (115) at (-3.56, 58.406) {};
\node[node,color={rgb:dem,1;rep,0;ind,0}, minimum size=1.5pt] (116) at (-1.62, 37.056) {};
\node[node,color={rgb:dem,0;rep,1;ind,0}, minimum size=1.5pt] (117) at (-2.34, 101.61) {};
\node[node,color={rgb:dem,0;rep,1;ind,0}, minimum size=1.5pt] (118) at (-3.56, 177.01) {};
\node[node,color={rgb:dem,1;rep,0;ind,0}, minimum size=1.5pt] (119) at (-1.62, 25.776) {};
\node[node,color={rgb:dem,1;rep,0;ind,0}, minimum size=1.5pt] (120) at (-2.34, 25.776) {};
\node[node,color={rgb:dem,1;rep,0;ind,0}, minimum size=1.5pt] (121) at (-4.68, 60.0) {};
\node[node,color={rgb:dem,0;rep,2;ind,0}, minimum size=2.121pt] (122) at (-3.56, 147.51) {};
\node[node,color={rgb:dem,0;rep,1;ind,0}, minimum size=1.5pt] (123) at (-2.34, 172.82) {};
\node[node,color={rgb:dem,0;rep,1;ind,0}, minimum size=1.5pt] (124) at (-3.56, 106.87) {};
\node[node,color={rgb:dem,0;rep,1;ind,0}, minimum size=1.5pt] (125) at (-3.56, 160.11) {};
\node[node,color={rgb:dem,0;rep,1;ind,0}, minimum size=1.5pt] (126) at (-2.34, 155.43) {};
\node[node,color={rgb:dem,0;rep,1;ind,0}, minimum size=1.5pt] (127) at (-2.34, 100.86) {};
\node[node,color={rgb:dem,1;rep,0;ind,0}, minimum size=1.5pt] (128) at (-4.28, 57.656) {};
\node[node,color={rgb:dem,1;rep,0;ind,0}, minimum size=1.5pt] (129) at (-2.34, 6.906) {};
\node[node,color={rgb:dem,0;rep,1;ind,0}, minimum size=1.5pt] (130) at (-1.62, 146.73) {};
\node[node,color={rgb:dem,1;rep,0;ind,0}, minimum size=1.5pt] (131) at (-2.34, 17.186) {};
\node[node,color={rgb:dem,1;rep,0;ind,0}, minimum size=1.5pt] (132) at (-2.34, 22.776) {};
\node[node,color={rgb:dem,0;rep,1;ind,0}, minimum size=1.5pt] (133) at (-2.34, 141.08) {};
\node[node,color={rgb:dem,0;rep,1;ind,0}, minimum size=1.5pt] (134) at (-2.34, 115.15) {};
\node[node,color={rgb:dem,1;rep,0;ind,0}, minimum size=1.5pt] (135) at (-3.56, 61.796) {};
\node[node,color={rgb:dem,0;rep,1;ind,0}, minimum size=1.5pt] (136) at (-2.34, 143.33) {};
\node[node,color={rgb:dem,0;rep,1;ind,0}, minimum size=1.5pt] (137) at (-3.56, 104.64) {};
\node[node,color={rgb:dem,1;rep,0;ind,0}, minimum size=1.5pt] (138) at (-2.34, 50.246) {};
\node[node,color={rgb:dem,1;rep,0;ind,0}, minimum size=1.5pt] (139) at (-4.68, 62.0) {};
\node[node,color={rgb:dem,1;rep,0;ind,0}, minimum size=1.5pt] (140) at (-2.34, 67.976) {};
\node[node,color={rgb:dem,1;rep,0;ind,0}, minimum size=1.5pt] (141) at (-1.62, 43.286) {};
\node[node,color={rgb:dem,1;rep,0;ind,0}, minimum size=1.5pt] (142) at (-2.34, 86.956) {};
\node[node,color={rgb:dem,0;rep,1;ind,0}, minimum size=1.5pt] (143) at (-3.56, 171.73) {};
\node[node,color={rgb:dem,0;rep,1;ind,0}, minimum size=1.5pt] (144) at (-2.34, 99.336) {};
\node[node,color={rgb:dem,0;rep,1;ind,0}, minimum size=1.5pt] (145) at (-3.56, 96.456) {};
\node[node,color={rgb:dem,1;rep,0;ind,0}, minimum size=1.5pt] (146) at (-4.68, 64.0) {};
\node[node,color={rgb:dem,1;rep,0;ind,0}, minimum size=1.5pt] (147) at (-3.56, 60.656) {};
\node[node,color={rgb:dem,1;rep,0;ind,0}, minimum size=1.5pt] (148) at (-4.28, 66.316) {};
\node[node,color={rgb:dem,1;rep,0;ind,0}, minimum size=1.5pt] (149) at (-2.34, 47.996) {};
\node[node,color={rgb:dem,1;rep,0;ind,0}, minimum size=1.5pt] (150) at (-2.34, 86.206) {};
\node[node,color={rgb:dem,0;rep,1;ind,0}, minimum size=1.5pt] (151) at (-3.56, 168.73) {};
\node[node,color={rgb:dem,0;rep,1;ind,0}, minimum size=1.5pt] (152) at (-3.56, 162.34) {};
\node[node,color={rgb:dem,1;rep,0;ind,0}, minimum size=1.5pt] (153) at (-2.34, 63.466) {};
\node[node,color={rgb:dem,0;rep,2;ind,0}, minimum size=2.121pt] (154) at (-1.62, 154.1) {};
\node[node,color={rgb:dem,1;rep,0;ind,0}, minimum size=1.5pt] (155) at (-1.62, 144.47) {};
\node[node,color={rgb:dem,0;rep,1;ind,0}, minimum size=1.5pt] (156) at (-2.34, 145.6) {};
\node[node,color={rgb:dem,0;rep,1;ind,0}, minimum size=1.5pt] (157) at (-3.56, 145.03) {};
\node[node,color={rgb:dem,1;rep,0;ind,0}, minimum size=1.5pt] (158) at (-2.34, 39.326) {};
\node[node,color={rgb:dem,1;rep,0;ind,0}, minimum size=1.5pt] (159) at (-3.56, 100.89) {};
\node[node,color={rgb:dem,1;rep,0;ind,0}, minimum size=1.5pt] (160) at (-2.34, 106.87) {};
\node[node,color={rgb:dem,1;rep,0;ind,0}, minimum size=1.5pt] (161) at (-2.34, 91.486) {};
\node[node,color={rgb:dem,1;rep,0;ind,0}, minimum size=1.5pt] (162) at (-3.56, 6.906) {};
\node[node,color={rgb:dem,1;rep,0;ind,0}, minimum size=1.5pt] (163) at (-3.56, 94.206) {};
\node[node,color={rgb:dem,1;rep,0;ind,0}, minimum size=1.5pt] (164) at (-2.34, 82.506) {};
\node[node,color={rgb:dem,1;rep,1;ind,0}, minimum size=2.121pt] (165) at (-2.34, 88.456) {};
\node[node,color={rgb:dem,1;rep,0;ind,0}, minimum size=1.5pt] (166) at (-1.62, 90.136) {};
\node[node,color={rgb:dem,0;rep,1;ind,0}, minimum size=1.5pt] (167) at (-2.34, 58.936) {};
\node[node,color={rgb:dem,0;rep,1;ind,0}, minimum size=1.5pt] (168) at (-2.34, 121.16) {};
\node[node,color={rgb:dem,1;rep,0;ind,0}, minimum size=1.5pt] (169) at (-2.34, 74.376) {};
\node[node,color={rgb:dem,1;rep,0;ind,0}, minimum size=1.5pt] (170) at (-3.56, 80.266) {};
\node[node,color={rgb:dem,1;rep,0;ind,0}, minimum size=1.5pt] (171) at (-3.56, 65.196) {};
\node[node,color={rgb:dem,1;rep,0;ind,0}, minimum size=1.5pt] (172) at (-3.56, 34.866) {};
\node[node,color={rgb:dem,1;rep,0;ind,0}, minimum size=1.5pt] (173) at (-2.34, 51.386) {};
\node[node,color={rgb:dem,1;rep,0;ind,0}, minimum size=1.5pt] (174) at (-4.68, 66.0) {};
\node[node,color={rgb:dem,1;rep,0;ind,0}, minimum size=1.5pt] (175) at (-4.68, 68.0) {};
\node[node,color={rgb:dem,1;rep,0;ind,0}, minimum size=1.5pt] (176) at (-3.56, 25.056) {};
\node[node,color={rgb:dem,1;rep,0;ind,0}, minimum size=1.5pt] (177) at (-2.34, 90.346) {};
\node[node,color={rgb:dem,0;rep,3;ind,0}, minimum size=2.598pt] (178) at (-3.56, 153.15) {};
\node[node,color={rgb:dem,1;rep,0;ind,0}, minimum size=1.5pt] (179) at (-3.56, 64.446) {};
\node[node,color={rgb:dem,1;rep,0;ind,0}, minimum size=1.5pt] (180) at (-2.34, 69.116) {};
\node[node,color={rgb:dem,0;rep,1;ind,0}, minimum size=1.5pt] (181) at (-2.34, 154.29) {};
\node[node,color={rgb:dem,1;rep,0;ind,0}, minimum size=1.5pt] (182) at (-1.62, 56.936) {};
\node[node,color={rgb:dem,0;rep,1;ind,0}, minimum size=1.5pt] (183) at (-2.34, 156.93) {};
\node[node,color={rgb:dem,0;rep,1;ind,0}, minimum size=1.5pt] (184) at (-3.56, 165.37) {};
\node[node,color={rgb:dem,0;rep,1;ind,0}, minimum size=1.5pt] (185) at (-1.62, 157.5) {};
\node[node,color={rgb:dem,1;rep,0;ind,0}, minimum size=1.5pt] (186) at (-3.56, 88.926) {};
\node[node,color={rgb:dem,1;rep,0;ind,0}, minimum size=1.5pt] (187) at (-3.56, 27.326) {};
\node[node,color={rgb:dem,1;rep,0;ind,0}, minimum size=1.5pt] (188) at (-3.56, 43.466) {};
\node[node,color={rgb:dem,0;rep,1;ind,0}, minimum size=1.5pt] (189) at (-2.34, 80.916) {};
\node[node,color={rgb:dem,1;rep,0;ind,0}, minimum size=1.5pt] (190) at (-1.26, 38.096) {};
\node[node,color={rgb:dem,0;rep,1;ind,0}, minimum size=1.5pt] (191) at (-2.34, 125.3) {};
\node[node,color={rgb:dem,0;rep,1;ind,0}, minimum size=1.5pt] (192) at (-3.56, 143.33) {};
\node[node,color={rgb:dem,0;rep,1;ind,0}, minimum size=1.5pt] (193) at (-3.56, 157.88) {};
\node[node,color={rgb:dem,0;rep,1;ind,0}, minimum size=1.5pt] (194) at (-3.56, 77.626) {};
\node[node,color={rgb:dem,0;rep,1;ind,0}, minimum size=1.5pt] (195) at (-2.34, 113.65) {};
\node[node,color={rgb:dem,1;rep,0;ind,0}, minimum size=1.5pt] (196) at (-4.68, 70.0) {};
\node[node,color={rgb:dem,1;rep,0;ind,0}, minimum size=1.5pt] (197) at (-3.56, 66.696) {};
\node[node,color={rgb:dem,0;rep,1;ind,0}, minimum size=1.5pt] (198) at (-2.34, 149.35) {};
\node[node,color={rgb:dem,1;rep,0;ind,0}, minimum size=1.5pt] (199) at (-2.34, 36.306) {};
\node[node,color={rgb:dem,0;rep,1;ind,0}, minimum size=1.5pt] (200) at (-2.34, 165.24) {};
\node[node,color={rgb:dem,0;rep,1;ind,0}, minimum size=1.5pt] (201) at (-2.34, 70.616) {};
\node[node,color={rgb:dem,0;rep,2;ind,0}, minimum size=2.121pt] (202) at (-2.34, 178.08) {};
\node[node,color={rgb:dem,0;rep,1;ind,0}, minimum size=1.5pt] (203) at (-3.56, 20.936) {};
\node[node,color={rgb:dem,1;rep,0;ind,0}, minimum size=1.5pt] (204) at (-3.56, 174.76) {};
\node[node,color={rgb:dem,1;rep,0;ind,0}, minimum size=1.5pt] (205) at (-2.34, 105.73) {};
\node[node,color={rgb:dem,0;rep,1;ind,0}, minimum size=1.5pt] (206) at (-2.34, 135.43) {};
\node[node,color={rgb:dem,1;rep,0;ind,0}, minimum size=1.5pt] (207) at (-2.34, 44.616) {};
\node[node,color={rgb:dem,0;rep,1;ind,0}, minimum size=1.5pt] (208) at (-4.68, 134.0) {};
\node[node,color={rgb:dem,0;rep,1;ind,0}, minimum size=1.5pt] (209) at (-2.34, 85.456) {};
\node[node,color={rgb:dem,0;rep,1;ind,0}, minimum size=1.5pt] (210) at (-3.56, 164.62) {};
\node[node,color={rgb:dem,0;rep,1;ind,0}, minimum size=1.5pt] (211) at (-4.28, 173.44) {};
\node[node,color={rgb:dem,1;rep,0;ind,0}, minimum size=1.5pt] (212) at (-3.56, 176.26) {};
\node[node,color={rgb:dem,1;rep,0;ind,0}, minimum size=1.5pt] (213) at (-2.34, 28.386) {};
\node[node,color={rgb:dem,0;rep,1;ind,0}, minimum size=1.5pt] (214) at (-2.34, 118.18) {};
\node[node,color={rgb:dem,0;rep,1;ind,0}, minimum size=1.5pt] (215) at (-2.34, 92.626) {};
\node[node,color={rgb:dem,1;rep,0;ind,0}, minimum size=1.5pt] (216) at (-4.68, 72.0) {};
\node[node,color={rgb:dem,0;rep,1;ind,0}, minimum size=1.5pt] (217) at (-3.56, 170.98) {};
\node[node,color={rgb:dem,0;rep,1;ind,0}, minimum size=1.5pt] (218) at (-4.68, 136.0) {};
\node[node,color={rgb:dem,0;rep,1;ind,0}, minimum size=1.5pt] (219) at (-3.56, 177.76) {};
\node[node,color={rgb:dem,1;rep,0;ind,0}, minimum size=1.5pt] (220) at (-4.68, 74.0) {};
\node[node,color={rgb:dem,0;rep,1;ind,0}, minimum size=1.5pt] (221) at (-3.56, 103.89) {};
\node[node,color={rgb:dem,0;rep,1;ind,0}, minimum size=1.5pt] (222) at (-3.56, 113.58) {};
\node[node,color={rgb:dem,1;rep,0;ind,0}, minimum size=1.5pt] (223) at (-3.56, 88.176) {};
\node[node,color={rgb:dem,0;rep,1;ind,0}, minimum size=1.5pt] (224) at (-1.62, 118.81) {};
\node[node,color={rgb:dem,0;rep,1;ind,0}, minimum size=1.5pt] (225) at (-2.34, 103.11) {};
\node[node,color={rgb:dem,1;rep,0;ind,0}, minimum size=1.5pt] (226) at (-2.34, 49.496) {};
\node[node,color={rgb:dem,0;rep,1;ind,0}, minimum size=1.5pt] (227) at (-3.56, 152.01) {};
\node[node,color={rgb:dem,0;rep,1;ind,0}, minimum size=1.5pt] (228) at (-3.56, 110.26) {};
\node[node,color={rgb:dem,1;rep,0;ind,0}, minimum size=1.5pt] (229) at (-4.68, 76.0) {};
\node[node,color={rgb:dem,0;rep,1;ind,0}, minimum size=1.5pt] (230) at (-3.56, 155.42) {};
\node[node,color={rgb:dem,0;rep,1;ind,0}, minimum size=1.5pt] (231) at (-2.34, 168.65) {};
\node[node,color={rgb:dem,1;rep,0;ind,0}, minimum size=1.5pt] (232) at (-3.56, 2.176) {};
\node[node,color={rgb:dem,0;rep,1;ind,0}, minimum size=1.5pt] (233) at (-2.34, 173.57) {};
\node[node,color={rgb:dem,0;rep,1;ind,0}, minimum size=1.5pt] (234) at (-2.34, 144.08) {};
\node[node,color={rgb:dem,1;rep,0;ind,0}, minimum size=1.5pt] (235) at (-2.34, 6.156) {};
\node[node,color={rgb:dem,0;rep,1;ind,0}, minimum size=1.5pt] (236) at (-2.34, 166.76) {};
\node[node,color={rgb:dem,0;rep,1;ind,0}, minimum size=1.5pt] (237) at (-3.56, 108.76) {};
\node[node,color={rgb:dem,1;rep,0;ind,0}, minimum size=1.5pt] (238) at (-3.56, 44.216) {};
\node[node,color={rgb:dem,1;rep,0;ind,0}, minimum size=1.5pt] (239) at (-3.56, 100.14) {};
\node[node,color={rgb:dem,1;rep,0;ind,0}, minimum size=1.5pt] (240) at (-3.56, 56.516) {};
\node[node,color={rgb:dem,1;rep,0;ind,0}, minimum size=1.5pt] (241) at (-2.34, 40.466) {};
\node[node,color={rgb:dem,0;rep,1;ind,0}, minimum size=1.5pt] (242) at (-4.28, 147.51) {};
\node[node,color={rgb:dem,1;rep,0;ind,0}, minimum size=1.5pt] (243) at (-2.34, 64.946) {};
\node[node,color={rgb:dem,0;rep,1;ind,0}, minimum size=1.5pt] (244) at (-2.34, 89.206) {};
\node[node,color={rgb:dem,0;rep,1;ind,0}, minimum size=1.5pt] (245) at (-3.56, 120.41) {};
\node[node,color={rgb:dem,1;rep,0;ind,0}, minimum size=1.5pt] (246) at (-4.68, 78.0) {};
\node[node,color={rgb:dem,0;rep,1;ind,0}, minimum size=1.5pt] (247) at (-3.56, 105.39) {};
\node[node,color={rgb:dem,0;rep,1;ind,0}, minimum size=1.5pt] (248) at (-3.56, 150.51) {};
\node[node,color={rgb:dem,1;rep,0;ind,0}, minimum size=1.5pt] (249) at (-3.56, 175.51) {};
\node[node,color={rgb:dem,1;rep,0;ind,0}, minimum size=1.5pt] (250) at (-1.62, 34.336) {};
\node[node,color={rgb:dem,1;rep,0;ind,0}, minimum size=1.5pt] (251) at (-3.56, 33.726) {};
\node[node,color={rgb:dem,0;rep,1;ind,0}, minimum size=1.5pt] (252) at (-2.34, 111.38) {};
\node[node,color={rgb:dem,0;rep,1;ind,0}, minimum size=1.5pt] (253) at (-3.56, 107.62) {};
\node[node,color={rgb:dem,1;rep,0;ind,0}, minimum size=1.5pt] (254) at (-4.28, 74.146) {};
\node[node,color={rgb:dem,1;rep,0;ind,0}, minimum size=1.5pt] (255) at (-2.34, 73.236) {};
\node[node,color={rgb:dem,0;rep,1;ind,0}, minimum size=1.5pt] (256) at (-1.62, 160.04) {};
\node[node,color={rgb:dem,1;rep,0;ind,0}, minimum size=1.5pt] (257) at (-4.28, 58.406) {};
\node[node,color={rgb:dem,0;rep,1;ind,0}, minimum size=1.5pt] (258) at (-2.34, 122.66) {};
\node[node,color={rgb:dem,0;rep,1;ind,0}, minimum size=1.5pt] (259) at (-3.56, 81.786) {};
\node[node,color={rgb:dem,0;rep,1;ind,0}, minimum size=1.5pt] (260) at (-2.34, 148.6) {};
\node[node,color={rgb:dem,0;rep,1;ind,0}, minimum size=1.5pt] (261) at (-4.28, 109.22) {};
\node[node,color={rgb:dem,1;rep,0;ind,0}, minimum size=1.5pt] (262) at (-4.28, 76.866) {};
\node[node,color={rgb:dem,0;rep,1;ind,0}, minimum size=1.5pt] (263) at (-2.34, 142.58) {};
\node[node,color={rgb:dem,0;rep,2;ind,0}, minimum size=2.121pt] (264) at (-2.34, 121.91) {};
\node[node,color={rgb:dem,1;rep,0;ind,0}, minimum size=1.5pt] (265) at (-3.56, 44.966) {};
\node[node,color={rgb:dem,1;rep,0;ind,0}, minimum size=1.5pt] (266) at (-2.34, 26.526) {};
\node[node,color={rgb:dem,1;rep,0;ind,0}, minimum size=1.5pt] (267) at (-4.68, 80.0) {};
\node[node,color={rgb:dem,1;rep,0;ind,0}, minimum size=1.5pt] (268) at (-2.34, 110.63) {};
\node[node,color={rgb:dem,1;rep,0;ind,0}, minimum size=1.5pt] (269) at (-3.56, 29.226) {};
\node[node,color={rgb:dem,0;rep,1;ind,0}, minimum size=1.5pt] (270) at (-2.34, 108.01) {};
\node[node,color={rgb:dem,1;rep,0;ind,0}, minimum size=1.5pt] (271) at (-3.56, 174.01) {};
\node[node,color={rgb:dem,1;rep,0;ind,0}, minimum size=1.5pt] (272) at (-2.34, 33.686) {};
\node[node,color={rgb:dem,1;rep,0;ind,0}, minimum size=1.5pt] (273) at (-3.56, 79.126) {};
\node[node,color={rgb:dem,2;rep,0;ind,0}, minimum size=2.121pt] (274) at (-4.28, 1) {};
\node[node,color={rgb:dem,0;rep,1;ind,0}, minimum size=1.5pt] (275) at (-3.56, 154.66) {};
\node[node,color={rgb:dem,0;rep,1;ind,0}, minimum size=1.5pt] (276) at (-4.28, 134.86) {};
\node[node,color={rgb:dem,0;rep,1;ind,0}, minimum size=1.5pt] (277) at (-2.34, 139.18) {};
\node[node,color={rgb:dem,1;rep,0;ind,0}, minimum size=1.5pt] (278) at (-2.34, 54.386) {};
\node[node,color={rgb:dem,1;rep,0;ind,0}, minimum size=1.5pt] (279) at (-1.62, 56.186) {};
\node[node,color={rgb:dem,1;rep,0;ind,0}, minimum size=1.5pt] (280) at (-2.34, 35.166) {};
\node[node,color={rgb:dem,0;rep,1;ind,0}, minimum size=1.5pt] (281) at (-2.34, 123.8) {};
\node[node,color={rgb:dem,1;rep,0;ind,0}, minimum size=1.5pt] (282) at (-2.34, 37.806) {};
\node[node,color={rgb:dem,1;rep,0;ind,0}, minimum size=1.5pt] (283) at (-3.56, 41.186) {};
\node[node,color={rgb:dem,1;rep,0;ind,0}, minimum size=1.5pt] (284) at (-3.56, 32.976) {};
\node[node,color={rgb:dem,1;rep,0;ind,0}, minimum size=1.5pt] (285) at (-2.34, 97.086) {};
\node[node,color={rgb:dem,1;rep,0;ind,0}, minimum size=1.5pt] (286) at (-3.56, 21.686) {};
\node[node,color={rgb:dem,0;rep,1;ind,0}, minimum size=1.5pt] (287) at (-1.62, 137.6) {};
\node[node,color={rgb:dem,0;rep,1;ind,0}, minimum size=1.5pt] (288) at (-3.56, 130.39) {};
\node[node,color={rgb:dem,1;rep,0;ind,0}, minimum size=1.5pt] (289) at (-2.34, 67.226) {};
\node[node,color={rgb:dem,1;rep,0;ind,0}, minimum size=1.5pt] (290) at (-2.34, 77.826) {};
\node[node,color={rgb:dem,1;rep,0;ind,0}, minimum size=1.5pt] (291) at (-2.34, 59.686) {};
\node[node,color={rgb:dem,0;rep,1;ind,0}, minimum size=1.5pt] (292) at (-2.34, 120.41) {};
\node[node,color={rgb:dem,0;rep,1;ind,0}, minimum size=1.5pt] (293) at (-3.56, 121.87) {};
\node[node,color={rgb:dem,1;rep,0;ind,0}, minimum size=1.5pt] (294) at (-4.68, 82.0) {};
\node[node,color={rgb:dem,0;rep,1;ind,0}, minimum size=1.5pt] (295) at (-3.56, 141.83) {};
\node[node,color={rgb:dem,0;rep,1;ind,0}, minimum size=1.5pt] (296) at (-3.56, 145.78) {};
\node[node,color={rgb:dem,2;rep,0;ind,0}, minimum size=2.121pt] (297) at (-2.34, 60.436) {};
\node[node,color={rgb:dem,0;rep,1;ind,0}, minimum size=1.5pt] (298) at (-2.34, 98.586) {};
\node[node,color={rgb:dem,0;rep,1;ind,0}, minimum size=1.5pt] (299) at (-3.56, 122.66) {};
\node[node,color={rgb:dem,1;rep,0;ind,0}, minimum size=1.5pt] (300) at (-2.34, 48.746) {};
\node[node,color={rgb:dem,1;rep,0;ind,0}, minimum size=1.5pt] (301) at (-3.56, 95.706) {};
\node[node,color={rgb:dem,1;rep,0;ind,0}, minimum size=1.5pt] (302) at (-2.34, 66.086) {};
\node[node,color={rgb:dem,0;rep,1;ind,0}, minimum size=1.5pt] (303) at (-1.62, 170.83) {};
\node[node,color={rgb:dem,0;rep,1;ind,0}, minimum size=1.5pt] (304) at (-3.56, 159.36) {};
\node[node,color={rgb:dem,1;rep,0;ind,0}, minimum size=1.5pt] (305) at (-3.56, 25.806) {};
\node[node,color={rgb:dem,1;rep,0;ind,0}, minimum size=1.5pt] (306) at (-3.56, 70.836) {};
\node[node,color={rgb:dem,0;rep,1;ind,0}, minimum size=1.5pt] (307) at (-1.62, 130.76) {};
\node[node,color={rgb:dem,0;rep,1;ind,0}, minimum size=1.5pt] (308) at (-4.28, 92.136) {};
\node[node,color={rgb:dem,0;rep,1;ind,0}, minimum size=1.5pt] (309) at (-2.34, 160.68) {};
\node[node,color={rgb:dem,1;rep,0;ind,0}, minimum size=1.5pt] (310) at (-2.34, 93.376) {};
\node[node,color={rgb:dem,1;rep,0;ind,0}, minimum size=1.5pt] (311) at (-3.56, 59.156) {};
\node[node,color={rgb:dem,1;rep,0;ind,0}, minimum size=1.5pt] (312) at (-2.34, 32.936) {};
\node[node,color={rgb:dem,0;rep,1;ind,0}, minimum size=1.5pt] (313) at (-4.28, 123.56) {};
\node[node,color={rgb:dem,0;rep,1;ind,0}, minimum size=1.5pt] (314) at (-3.56, 149.76) {};
\node[node,color={rgb:dem,1;rep,0;ind,0}, minimum size=1.5pt] (315) at (-3.56, 90.816) {};
\node[node,color={rgb:dem,0;rep,1;ind,0}, minimum size=1.5pt] (316) at (-3.56, 83.266) {};
\node[node,color={rgb:dem,1;rep,0;ind,0}, minimum size=1.5pt] (317) at (-3.56, 72.726) {};
\node[node,color={rgb:dem,0;rep,1;ind,0}, minimum size=1.5pt] (318) at (-2.34, 124.55) {};
\node[node,color={rgb:dem,1;rep,0;ind,0}, minimum size=1.5pt] (319) at (-3.56, 1.036) {};
\node[node,color={rgb:dem,0;rep,1;ind,0}, minimum size=1.5pt] (320) at (-2.34, 109.15) {};
\node[node,color={rgb:dem,0;rep,1;ind,0}, minimum size=1.5pt] (321) at (-2.34, 112.9) {};
\node[node,color={rgb:dem,0;rep,1;ind,0}, minimum size=1.5pt] (322) at (-3.56, 166.87) {};
\node[node,color={rgb:dem,0;rep,1;ind,0}, minimum size=1.5pt] (323) at (-2.34, 167.51) {};
\node[node,color={rgb:dem,0;rep,1;ind,0}, minimum size=1.5pt] (324) at (-2.34, 172.07) {};
\node[node,color={rgb:dem,0;rep,1;ind,0}, minimum size=1.5pt] (325) at (-1.62, 70.616) {};
\node[node,color={rgb:dem,1;rep,0;ind,0}, minimum size=1.5pt] (326) at (-3.56, 53.156) {};
\node[node,color={rgb:dem,1;rep,0;ind,0}, minimum size=1.5pt] (327) at (-1.62, 4.696) {};
\node[node,color={rgb:dem,0;rep,1;ind,0}, minimum size=1.5pt] (328) at (-2.34, 127.94) {};
\node[node,color={rgb:dem,0;rep,1;ind,0}, minimum size=1.5pt] (329) at (-2.34, 128.69) {};
\node[node,color={rgb:dem,1;rep,0;ind,0}, minimum size=1.5pt] (330) at (-3.56, 86.286) {};
\node[node,color={rgb:dem,1;rep,0;ind,0}, minimum size=1.5pt] (331) at (-3.56, 71.586) {};
\node[node,color={rgb:dem,1;rep,0;ind,0}, minimum size=1.5pt] (332) at (-3.56, 17.636) {};
\node[node,color={rgb:dem,1;rep,0;ind,0}, minimum size=1.5pt] (333) at (-3.56, 32.226) {};
\node[node,color={rgb:dem,1;rep,0;ind,0}, minimum size=1.5pt] (334) at (-2.34, 81.726) {};
\node[node,color={rgb:dem,0;rep,1;ind,0}, minimum size=1.5pt] (335) at (-2.34, 126.05) {};
\node[node,color={rgb:dem,0;rep,1;ind,0}, minimum size=1.5pt] (336) at (-3.56, 111.74) {};
\node[node,color={rgb:dem,1;rep,0;ind,0}, minimum size=1.5pt] (337) at (-3.56, 52.406) {};
\node[node,color={rgb:dem,1;rep,0;ind,0}, minimum size=1.5pt] (338) at (-3.56, 23.916) {};
\node[node,color={rgb:dem,0;rep,1;ind,0}, minimum size=1.5pt] (339) at (-3.56, 170.23) {};
\node[node,color={rgb:dem,1;rep,0;ind,0}, minimum size=1.5pt] (340) at (-2.34, 52.886) {};
\node[node,color={rgb:dem,1;rep,0;ind,0}, minimum size=1.5pt] (341) at (-4.68, 84.0) {};
\node[node,color={rgb:dem,0;rep,1;ind,0}, minimum size=1.5pt] (342) at (-2.34, 117.04) {};
\node[node,color={rgb:dem,1;rep,0;ind,0}, minimum size=1.5pt] (343) at (-3.56, 42.326) {};
\node[node,color={rgb:dem,0;rep,1;ind,0}, minimum size=1.5pt] (344) at (-3.56, 163.48) {};
\node[node,color={rgb:dem,0;rep,1;ind,0}, minimum size=1.5pt] (345) at (-3.56, 166.12) {};
\node[node,color={rgb:dem,1;rep,0;ind,0}, minimum size=1.5pt] (346) at (-4.68, 86.0) {};
\node[node,color={rgb:dem,0;rep,1;ind,0}, minimum size=1.5pt] (347) at (-2.34, 136.57) {};
\node[node,color={rgb:dem,1;rep,0;ind,0}, minimum size=1.5pt] (348) at (-2.34, 141.83) {};
\node[node,color={rgb:dem,1;rep,0;ind,0}, minimum size=1.5pt] (349) at (-2.34, 10.926) {};
\node[node,color={rgb:dem,1;rep,0;ind,0}, minimum size=1.5pt] (350) at (-4.28, 16.176) {};
\node[node,color={rgb:dem,1;rep,0;ind,0}, minimum size=1.5pt] (351) at (-3.56, 55.016) {};
\node[node,color={rgb:dem,1;rep,0;ind,0}, minimum size=1.5pt] (352) at (-3.56, 87.426) {};
\node[node,color={rgb:dem,1;rep,0;ind,0}, minimum size=1.5pt] (353) at (-4.28, 78.756) {};
\node[node,color={rgb:dem,0;rep,1;ind,0}, minimum size=1.5pt] (354) at (-2.34, 116.29) {};
\node[node,color={rgb:dem,1;rep,0;ind,0}, minimum size=1.5pt] (355) at (-2.34, 57.416) {};
\node[node,color={rgb:dem,1;rep,0;ind,0}, minimum size=1.5pt] (356) at (-3.56, 37.816) {};
\node[node,color={rgb:dem,1;rep,0;ind,0}, minimum size=1.5pt] (357) at (-2.34, 31.416) {};
\node[node,color={rgb:dem,1;rep,0;ind,0}, minimum size=1.5pt] (358) at (-4.28, 25.056) {};
\node[node,color={rgb:dem,1;rep,0;ind,0}, minimum size=1.5pt] (359) at (-3.56, 76.486) {};
\node[node,color={rgb:dem,1;rep,0;ind,0}, minimum size=1.5pt] (360) at (-3.56, 57.656) {};
\node[node,color={rgb:dem,1;rep,0;ind,0}, minimum size=1.5pt] (361) at (-3.56, 62.546) {};
\node[node,color={rgb:dem,1;rep,0;ind,0}, minimum size=1.5pt] (362) at (-2.34, 87.706) {};
\node[node,color={rgb:dem,1;rep,0;ind,0}, minimum size=1.5pt] (363) at (-3.56, 22.436) {};
\node[node,color={rgb:dem,0;rep,1;ind,0}, minimum size=1.5pt] (364) at (-1.62, 140.7) {};
\node[node,color={rgb:dem,1;rep,0;ind,0}, minimum size=1.5pt] (365) at (-2.34, 79.066) {};
\node[node,color={rgb:dem,0;rep,1;ind,0}, minimum size=1.5pt] (366) at (-4.68, 138.0) {};
\node[node,color={rgb:dem,0;rep,1;ind,0}, minimum size=1.5pt] (367) at (-3.56, 117.14) {};
\node[node,color={rgb:dem,0;rep,1;ind,0}, minimum size=1.5pt] (368) at (-2.34, 102.36) {};
\node[node,color={rgb:dem,0;rep,1;ind,0}, minimum size=1.5pt] (369) at (-2.34, 161.82) {};
\node[node,color={rgb:dem,0;rep,1;ind,0}, minimum size=1.5pt] (370) at (-2.34, 176.19) {};
\node[node,color={rgb:dem,1;rep,0;ind,0}, minimum size=1.5pt] (371) at (-2.34, 69.866) {};
\node[node,color={rgb:dem,0;rep,1;ind,0}, minimum size=1.5pt] (372) at (-3.56, 101.64) {};
\node[node,color={rgb:dem,0;rep,1;ind,0}, minimum size=1.5pt] (373) at (-0.9, 165.42) {};
\node[node,color={rgb:dem,1;rep,0;ind,0}, minimum size=1.5pt] (374) at (-4.68, 88.0) {};
\node[node,color={rgb:dem,1;rep,0;ind,0}, minimum size=1.5pt] (375) at (-2.34, 53.636) {};
\node[node,color={rgb:dem,2;rep,0;ind,0}, minimum size=2.121pt] (376) at (-3.56, 0.286) {};
\node[node,color={rgb:dem,1;rep,0;ind,0}, minimum size=1.5pt] (377) at (-4.28, 45.986) {};
\node[node,color={rgb:dem,1;rep,0;ind,0}, minimum size=1.5pt] (378) at (-2.34, 43.116) {};
\node[node,color={rgb:dem,1;rep,0;ind,0}, minimum size=1.5pt] (379) at (-2.34, 30.666) {};
\node[node,color={rgb:dem,1;rep,0;ind,0}, minimum size=1.5pt] (380) at (-2.34, 17.936) {};
\node[node,color={rgb:dem,0;rep,1;ind,0}, minimum size=1.5pt] (381) at (-3.56, 125.3) {};
\node[node,color={rgb:dem,1;rep,0;ind,0}, minimum size=1.5pt] (382) at (-1.62, 37.806) {};
\node[node,color={rgb:dem,1;rep,0;ind,0}, minimum size=1.5pt] (383) at (-2.34, 37.056) {};
\node[node,color={rgb:dem,1;rep,0;ind,0}, minimum size=1.5pt] (384) at (-3.56, 74.246) {};
\node[node,color={rgb:dem,1;rep,0;ind,0}, minimum size=1.5pt] (385) at (-4.68, 90.0) {};
\node[node,color={rgb:dem,1;rep,0;ind,0}, minimum size=1.5pt] (386) at (-4.28, 86.176) {};
\node[node,color={rgb:dem,1;rep,0;ind,0}, minimum size=1.5pt] (387) at (-2.34, 55.526) {};
\node[node,color={rgb:dem,0;rep,1;ind,0}, minimum size=1.5pt] (388) at (-2.34, 137.32) {};
\node[node,color={rgb:dem,0;rep,1;ind,0}, minimum size=1.5pt] (389) at (-3.56, 153.9) {};
\node[node,color={rgb:dem,0;rep,1;ind,0}, minimum size=1.5pt] (390) at (-2.34, 151.63) {};
\node[node,color={rgb:dem,1;rep,0;ind,0}, minimum size=1.5pt] (391) at (-4.68, 92.0) {};
\node[node,color={rgb:dem,1;rep,0;ind,0}, minimum size=1.5pt] (392) at (-2.34, 75.856) {};
\node[node,color={rgb:dem,1;rep,0;ind,0}, minimum size=1.5pt] (393) at (-4.68, 94.0) {};
\node[node,color={rgb:dem,0;rep,1;ind,0}, minimum size=1.5pt] (394) at (-2.34, 1.036) {};
\node[node,color={rgb:dem,1;rep,0;ind,0}, minimum size=1.5pt] (395) at (-2.34, 114.4) {};
\node[node,color={rgb:dem,1;rep,0;ind,0}, minimum size=1.5pt] (396) at (-3.56, 109.51) {};
\node[node,color={rgb:dem,1;rep,0;ind,0}, minimum size=1.5pt] (397) at (-1.62, 53.126) {};
\node[node,color={rgb:dem,0;rep,1;ind,0}, minimum size=1.5pt] (398) at (-3.56, 172.87) {};
\node[node,color={rgb:dem,0;rep,1;ind,0}, minimum size=1.5pt] (399) at (-3.56, 141.08) {};
\node[node,color={rgb:dem,0;rep,1;ind,0}, minimum size=1.5pt] (400) at (-4.68, 140.0) {};
\node[node,color={rgb:dem,1;rep,0;ind,0}, minimum size=1.5pt] (401) at (-3.56, 94.956) {};
\node[node,color={rgb:dem,1;rep,0;ind,0}, minimum size=1.5pt] (402) at (-2.34, 176.94) {};
\node[node,color={rgb:dem,0;rep,1;ind,0}, minimum size=1.5pt] (403) at (-3.56, 135.43) {};
\node[node,color={rgb:dem,1;rep,0;ind,0}, minimum size=1.5pt] (404) at (-3.56, 39.296) {};
\node[node,color={rgb:dem,1;rep,0;ind,0}, minimum size=1.5pt] (405) at (-2.34, 24.636) {};
\node[node,color={rgb:dem,1;rep,0;ind,0}, minimum size=1.5pt] (406) at (-4.68, 96.0) {};
\node[node,color={rgb:dem,0;rep,1;ind,0}, minimum size=1.5pt] (407) at (-2.34, 158.07) {};
\node[node,color={rgb:dem,0;rep,1;ind,0}, minimum size=1.5pt] (408) at (-2.34, 134.29) {};
\node[node,color={rgb:dem,0;rep,1;ind,0}, minimum size=1.5pt] (409) at (-2.34, 129.44) {};
\node[node,color={rgb:dem,0;rep,1;ind,0}, minimum size=1.5pt] (410) at (-2.34, 119.66) {};
\node[node,color={rgb:dem,0;rep,1;ind,0}, minimum size=1.5pt] (411) at (-3.56, 160.86) {};
\node[node,color={rgb:dem,1;rep,0;ind,0}, minimum size=1.5pt] (412) at (-2.34, 71.756) {};
\node[node,color={rgb:dem,0;rep,1;ind,0}, minimum size=1.5pt] (413) at (-2.34, 132.05) {};
\node[node,color={rgb:dem,1;rep,0;ind,0}, minimum size=1.5pt] (414) at (-2.34, 52.136) {};
\node[node,color={rgb:dem,1;rep,0;ind,0}, minimum size=1.5pt] (415) at (-4.68, 98.0) {};
\node[node,color={rgb:dem,1;rep,0;ind,0}, minimum size=1.5pt] (416) at (-2.34, 94.126) {};
\node[node,color={rgb:dem,1;rep,0;ind,0}, minimum size=1.5pt] (417) at (-3.56, 55.766) {};
\node[node,color={rgb:dem,0;rep,1;ind,0}, minimum size=1.5pt] (418) at (-4.68, 142.0) {};
\node[node,color={rgb:dem,1;rep,0;ind,0}, minimum size=1.5pt] (419) at (-3.56, 46.446) {};
\node[node,color={rgb:dem,1;rep,0;ind,0}, minimum size=1.5pt] (420) at (-1.62, 105.73) {};
\node[node,color={rgb:dem,0;rep,1;ind,0}, minimum size=1.5pt] (421) at (-3.56, 169.48) {};
\node[node,color={rgb:dem,0;rep,1;ind,0}, minimum size=1.5pt] (422) at (-3.56, 157.13) {};
\node[node,color={rgb:dem,1;rep,0;ind,0}, minimum size=1.5pt] (423) at (-2.34, 97.836) {};
\node[node,color={rgb:dem,1;rep,0;ind,0}, minimum size=1.5pt] (424) at (-2.34, 62.326) {};
\node[node,color={rgb:dem,1;rep,0;ind,0}, minimum size=1.5pt] (425) at (-3.56, 78.376) {};
\node[node,color={rgb:dem,1;rep,0;ind,0}, minimum size=1.5pt] (426) at (-2.34, 18.686) {};
\node[node,color={rgb:dem,1;rep,0;ind,0}, minimum size=1.5pt] (427) at (-3.56, 68.946) {};
\node[node,color={rgb:dem,1;rep,0;ind,0}, minimum size=1.5pt] (428) at (-3.56, 65.946) {};
\node[node,color={rgb:dem,1;rep,0;ind,0}, minimum size=1.5pt] (429) at (-3.56, 59.906) {};
\node[node,color={rgb:dem,0;rep,1;ind,0}, minimum size=1.5pt] (430) at (-3.56, 114.86) {};
\node[node,color={rgb:dem,1;rep,0;ind,0}, minimum size=1.5pt] (431) at (-4.28, 27.326) {};
\node[node,color={rgb:dem,0;rep,1;ind,0}, minimum size=1.5pt] (432) at (-3.56, 116.39) {};
\node[node,color={rgb:dem,0;rep,1;ind,0}, minimum size=1.5pt] (433) at (-1.62, 102.83) {};
\node[node,color={rgb:dem,1;rep,0;ind,0}, minimum size=1.5pt] (434) at (-3.56, 69.696) {};
\node[node,color={rgb:dem,1;rep,0;ind,0}, minimum size=1.5pt] (435) at (-3.56, 93.456) {};
\node[node,color={rgb:dem,1;rep,0;ind,0}, minimum size=1.5pt] (436) at (-3.56, 49.406) {};
\node[node,color={rgb:dem,0;rep,1;ind,0}, minimum size=1.5pt] (437) at (-3.56, 91.566) {};
\node[node,color={rgb:dem,0;rep,1;ind,0}, minimum size=1.5pt] (438) at (-3.56, 118.92) {};
\node[node,color={rgb:dem,0;rep,1;ind,0}, minimum size=1.5pt] (439) at (-3.56, 151.26) {};
\node[node,color={rgb:dem,0;rep,1;ind,0}, minimum size=1.5pt] (440) at (-2.34, 22.026) {};
\node[node,color={rgb:dem,1;rep,0;ind,0}, minimum size=1.5pt] (441) at (-4.68, 100.0) {};
\node[node,color={rgb:dem,1;rep,0;ind,0}, minimum size=1.5pt] (442) at (-2.34, 43.866) {};
\node[node,color={rgb:dem,0;rep,1;ind,0}, minimum size=1.5pt] (443) at (-3.56, 178.51) {};
\node[node,color={rgb:dem,1;rep,0;ind,0}, minimum size=1.5pt] (444) at (-3.56, 67.446) {};
\node[node,color={rgb:dem,0;rep,1;ind,0}, minimum size=1.5pt] (445) at (-4.28, 117.14) {};
\node[node,color={rgb:dem,1;rep,0;ind,0}, minimum size=1.5pt] (446) at (-2.34, 41.216) {};
\node[node,color={rgb:dem,1;rep,0;ind,0}, minimum size=1.5pt] (447) at (-3.56, 84.766) {};
\node[node,color={rgb:dem,1;rep,0;ind,0}, minimum size=1.5pt] (448) at (-3.56, 68.196) {};
\node[node,color={rgb:dem,0;rep,1;ind,0}, minimum size=1.5pt] (449) at (-2.34, 130.19) {};
\node[node,color={rgb:dem,1;rep,0;ind,0}, minimum size=1.5pt] (450) at (-2.34, 56.276) {};
\node[node,color={rgb:dem,0;rep,1;ind,0}, minimum size=1.5pt] (451) at (-3.56, 90.066) {};
\node[node,color={rgb:dem,0;rep,1;ind,0}, minimum size=1.5pt] (452) at (-3.56, 92.706) {};
\node[node,color={rgb:dem,0;rep,1;ind,0}, minimum size=1.5pt] (453) at (-2.34, 156.18) {};
\node[node,color={rgb:dem,1;rep,0;ind,0}, minimum size=1.5pt] (454) at (-3.56, 35.616) {};
\node[node,color={rgb:dem,0;rep,1;ind,0}, minimum size=1.5pt] (455) at (-4.68, 144.0) {};
\begin{pgfonlayer}{bg}
\draw[arc] (1) -- (20);
\draw[arc] (2) -- (164);
\draw[arc] (2) -- (334);
\draw[arc] (4) -- (3);
\draw[arc] (4) -- (28);
\draw[arc] (4) -- (55);
\draw[arc] (5) -- (72);
\draw[arc] (5) -- (43);
\draw[arc] (5) -- (292);
\draw[arc] (6) -- (51);
\draw[arc] (6) -- (55);
\draw[arc] (9) -- (378);
\draw[arc] (9) -- (120);
\draw[arc] (9) -- (58);
\draw[arc] (9) -- (61);
\draw[arc] (10) -- (397);
\draw[arc] (10) -- (173);
\draw[arc] (10) -- (54);
\draw[arc] (10) -- (414);
\draw[arc] (14) -- (382);
\draw[arc] (17) -- (285);
\draw[arc] (18) -- (30);
\draw[arc] (18) -- (7);
\draw[arc] (22) -- (16);
\draw[arc] (23) -- (383);
\draw[arc] (25) -- (58);
\draw[arc] (25) -- (131);
\draw[arc] (26) -- (96);
\draw[arc] (26) -- (74);
\draw[arc] (27) -- (129);
\draw[arc] (27) -- (66);
\draw[arc] (27) -- (235);
\draw[arc] (27) -- (48);
\draw[arc] (27) -- (24);
\draw[arc] (27) -- (28);
\draw[arc] (27) -- (349);
\draw[arc] (29) -- (67);
\draw[arc] (29) -- (74);
\draw[arc] (29) -- (43);
\draw[arc] (29) -- (16);
\draw[arc] (29) -- (24);
\draw[arc] (30) -- (16);
\draw[arc] (31) -- (274);
\draw[arc] (32) -- (110);
\draw[arc] (33) -- (405);
\draw[arc] (34) -- (56);
\draw[arc] (34) -- (289);
\draw[arc] (34) -- (317);
\draw[arc] (35) -- (91);
\draw[arc] (35) -- (67);
\draw[arc] (35) -- (19);
\draw[arc] (35) -- (74);
\draw[arc] (36) -- (83);
\draw[arc] (37) -- (16);
\draw[arc] (37) -- (83);
\draw[arc] (37) -- (368);
\draw[arc] (38) -- (67);
\draw[arc] (39) -- (105);
\draw[arc] (39) -- (270);
\draw[arc] (39) -- (79);
\draw[arc] (39) -- (16);
\draw[arc] (39) -- (50);
\draw[arc] (39) -- (51);
\draw[arc] (41) -- (256);
\draw[arc] (41) -- (67);
\draw[arc] (42) -- (334);
\draw[arc] (42) -- (22);
\draw[arc] (42) -- (30);
\draw[arc] (44) -- (8);
\draw[arc] (46) -- (234);
\draw[arc] (46) -- (44);
\draw[arc] (46) -- (48);
\draw[arc] (46) -- (21);
\draw[arc] (46) -- (154);
\draw[arc] (46) -- (413);
\draw[arc] (47) -- (33);
\draw[arc] (47) -- (51);
\draw[arc] (47) -- (286);
\draw[arc] (48) -- (67);
\draw[arc] (49) -- (8);
\draw[arc] (49) -- (16);
\draw[arc] (49) -- (54);
\draw[arc] (52) -- (112);
\draw[arc] (53) -- (74);
\draw[arc] (53) -- (166);
\draw[arc] (59) -- (110);
\draw[arc] (59) -- (303);
\draw[arc] (60) -- (72);
\draw[arc] (60) -- (289);
\draw[arc] (60) -- (250);
\draw[arc] (60) -- (67);
\draw[arc] (62) -- (82);
\draw[arc] (62) -- (11);
\draw[arc] (62) -- (416);
\draw[arc] (63) -- (16);
\draw[arc] (64) -- (67);
\draw[arc] (65) -- (128);
\draw[arc] (65) -- (45);
\draw[arc] (65) -- (81);
\draw[arc] (65) -- (50);
\draw[arc] (65) -- (58);
\draw[arc] (65) -- (383);
\draw[arc] (66) -- (67);
\draw[arc] (66) -- (327);
\draw[arc] (68) -- (244);
\draw[arc] (68) -- (15);
\draw[arc] (69) -- (16);
\draw[arc] (69) -- (55);
\draw[arc] (71) -- (24);
\draw[arc] (71) -- (43);
\draw[arc] (71) -- (140);
\draw[arc] (71) -- (141);
\draw[arc] (73) -- (54);
\draw[arc] (75) -- (320);
\draw[arc] (75) -- (1);
\draw[arc] (75) -- (7);
\draw[arc] (75) -- (43);
\draw[arc] (75) -- (15);
\draw[arc] (76) -- (48);
\draw[arc] (76) -- (20);
\draw[arc] (78) -- (82);
\draw[arc] (78) -- (119);
\draw[arc] (78) -- (376);
\draw[arc] (78) -- (319);
\draw[arc] (79) -- (297);
\draw[arc] (80) -- (56);
\draw[arc] (80) -- (51);
\draw[arc] (81) -- (360);
\draw[arc] (81) -- (82);
\draw[arc] (81) -- (250);
\draw[arc] (82) -- (266);
\draw[arc] (84) -- (52);
\draw[arc] (87) -- (307);
\draw[arc] (87) -- (155);
\draw[arc] (88) -- (58);
\draw[arc] (89) -- (136);
\draw[arc] (89) -- (408);
\draw[arc] (89) -- (110);
\draw[arc] (92) -- (256);
\draw[arc] (92) -- (321);
\draw[arc] (92) -- (117);
\draw[arc] (92) -- (127);
\draw[arc] (94) -- (320);
\draw[arc] (95) -- (256);
\draw[arc] (95) -- (99);
\draw[arc] (95) -- (107);
\draw[arc] (95) -- (307);
\draw[arc] (97) -- (185);
\draw[arc] (97) -- (307);
\draw[arc] (98) -- (307);
\draw[arc] (100) -- (244);
\draw[arc] (102) -- (144);
\draw[arc] (102) -- (298);
\draw[arc] (103) -- (105);
\draw[arc] (103) -- (138);
\draw[arc] (103) -- (86);
\draw[arc] (103) -- (120);
\draw[arc] (103) -- (190);
\draw[arc] (106) -- (320);
\draw[arc] (106) -- (409);
\draw[arc] (106) -- (307);
\draw[arc] (107) -- (373);
\draw[arc] (108) -- (130);
\draw[arc] (109) -- (136);
\draw[arc] (109) -- (310);
\draw[arc] (111) -- (256);
\draw[arc] (111) -- (303);
\draw[arc] (111) -- (154);
\draw[arc] (113) -- (168);
\draw[arc] (113) -- (433);
\draw[arc] (113) -- (373);
\draw[arc] (113) -- (215);
\draw[arc] (115) -- (140);
\draw[arc] (115) -- (207);
\draw[arc] (116) -- (190);
\draw[arc] (117) -- (433);
\draw[arc] (118) -- (202);
\draw[arc] (118) -- (373);
\draw[arc] (118) -- (183);
\draw[arc] (120) -- (119);
\draw[arc] (122) -- (256);
\draw[arc] (122) -- (156);
\draw[arc] (123) -- (373);
\draw[arc] (123) -- (303);
\draw[arc] (124) -- (8);
\draw[arc] (124) -- (320);
\draw[arc] (124) -- (85);
\draw[arc] (125) -- (256);
\draw[arc] (125) -- (99);
\draw[arc] (125) -- (453);
\draw[arc] (125) -- (107);
\draw[arc] (125) -- (110);
\draw[arc] (125) -- (309);
\draw[arc] (125) -- (185);
\draw[arc] (126) -- (224);
\draw[arc] (126) -- (256);
\draw[arc] (128) -- (289);
\draw[arc] (128) -- (164);
\draw[arc] (128) -- (360);
\draw[arc] (128) -- (424);
\draw[arc] (128) -- (442);
\draw[arc] (128) -- (115);
\draw[arc] (128) -- (250);
\draw[arc] (134) -- (307);
\draw[arc] (135) -- (424);
\draw[arc] (135) -- (105);
\draw[arc] (135) -- (289);
\draw[arc] (137) -- (127);
\draw[arc] (143) -- (107);
\draw[arc] (143) -- (309);
\draw[arc] (145) -- (416);
\draw[arc] (145) -- (93);
\draw[arc] (147) -- (291);
\draw[arc] (147) -- (424);
\draw[arc] (147) -- (105);
\draw[arc] (147) -- (140);
\draw[arc] (147) -- (142);
\draw[arc] (147) -- (112);
\draw[arc] (147) -- (114);
\draw[arc] (147) -- (149);
\draw[arc] (147) -- (383);
\draw[arc] (148) -- (197);
\draw[arc] (148) -- (326);
\draw[arc] (148) -- (360);
\draw[arc] (148) -- (169);
\draw[arc] (148) -- (428);
\draw[arc] (148) -- (52);
\draw[arc] (151) -- (200);
\draw[arc] (151) -- (256);
\draw[arc] (152) -- (200);
\draw[arc] (152) -- (234);
\draw[arc] (152) -- (236);
\draw[arc] (152) -- (111);
\draw[arc] (152) -- (369);
\draw[arc] (152) -- (181);
\draw[arc] (152) -- (309);
\draw[arc] (154) -- (107);
\draw[arc] (156) -- (307);
\draw[arc] (157) -- (224);
\draw[arc] (157) -- (258);
\draw[arc] (157) -- (198);
\draw[arc] (157) -- (263);
\draw[arc] (157) -- (390);
\draw[arc] (157) -- (234);
\draw[arc] (157) -- (236);
\draw[arc] (157) -- (298);
\draw[arc] (157) -- (303);
\draw[arc] (157) -- (181);
\draw[arc] (157) -- (342);
\draw[arc] (157) -- (154);
\draw[arc] (158) -- (105);
\draw[arc] (159) -- (423);
\draw[arc] (162) -- (129);
\draw[arc] (163) -- (215);
\draw[arc] (165) -- (155);
\draw[arc] (167) -- (303);
\draw[arc] (169) -- (382);
\draw[arc] (170) -- (112);
\draw[arc] (171) -- (289);
\draw[arc] (171) -- (164);
\draw[arc] (171) -- (140);
\draw[arc] (171) -- (383);
\draw[arc] (172) -- (199);
\draw[arc] (172) -- (272);
\draw[arc] (172) -- (241);
\draw[arc] (172) -- (280);
\draw[arc] (172) -- (383);
\draw[arc] (173) -- (279);
\draw[arc] (178) -- (130);
\draw[arc] (178) -- (67);
\draw[arc] (178) -- (200);
\draw[arc] (178) -- (307);
\draw[arc] (178) -- (181);
\draw[arc] (179) -- (424);
\draw[arc] (179) -- (412);
\draw[arc] (182) -- (190);
\draw[arc] (184) -- (200);
\draw[arc] (184) -- (130);
\draw[arc] (186) -- (209);
\draw[arc] (186) -- (142);
\draw[arc] (186) -- (207);
\draw[arc] (187) -- (266);
\draw[arc] (188) -- (207);
\draw[arc] (189) -- (307);
\draw[arc] (189) -- (279);
\draw[arc] (191) -- (307);
\draw[arc] (192) -- (136);
\draw[arc] (192) -- (390);
\draw[arc] (192) -- (127);
\draw[arc] (193) -- (407);
\draw[arc] (193) -- (181);
\draw[arc] (193) -- (231);
\draw[arc] (194) -- (289);
\draw[arc] (194) -- (362);
\draw[arc] (195) -- (224);
\draw[arc] (197) -- (140);
\draw[arc] (201) -- (307);
\draw[arc] (201) -- (20);
\draw[arc] (201) -- (325);
\draw[arc] (202) -- (303);
\draw[arc] (203) -- (440);
\draw[arc] (203) -- (266);
\draw[arc] (204) -- (96);
\draw[arc] (205) -- (420);
\draw[arc] (206) -- (307);
\draw[arc] (210) -- (200);
\draw[arc] (210) -- (107);
\draw[arc] (210) -- (181);
\draw[arc] (211) -- (256);
\draw[arc] (211) -- (231);
\draw[arc] (211) -- (143);
\draw[arc] (211) -- (370);
\draw[arc] (211) -- (118);
\draw[arc] (211) -- (443);
\draw[arc] (211) -- (123);
\draw[arc] (212) -- (96);
\draw[arc] (212) -- (402);
\draw[arc] (212) -- (164);
\draw[arc] (213) -- (250);
\draw[arc] (214) -- (307);
\draw[arc] (217) -- (99);
\draw[arc] (217) -- (236);
\draw[arc] (217) -- (111);
\draw[arc] (217) -- (370);
\draw[arc] (217) -- (123);
\draw[arc] (219) -- (202);
\draw[arc] (219) -- (107);
\draw[arc] (221) -- (433);
\draw[arc] (221) -- (127);
\draw[arc] (222) -- (134);
\draw[arc] (222) -- (234);
\draw[arc] (222) -- (138);
\draw[arc] (222) -- (433);
\draw[arc] (222) -- (342);
\draw[arc] (222) -- (91);
\draw[arc] (223) -- (209);
\draw[arc] (225) -- (307);
\draw[arc] (225) -- (166);
\draw[arc] (226) -- (250);
\draw[arc] (227) -- (200);
\draw[arc] (227) -- (234);
\draw[arc] (227) -- (390);
\draw[arc] (228) -- (320);
\draw[arc] (228) -- (307);
\draw[arc] (228) -- (270);
\draw[arc] (230) -- (136);
\draw[arc] (230) -- (369);
\draw[arc] (230) -- (181);
\draw[arc] (230) -- (309);
\draw[arc] (230) -- (373);
\draw[arc] (230) -- (126);
\draw[arc] (232) -- (55);
\draw[arc] (236) -- (256);
\draw[arc] (236) -- (373);
\draw[arc] (237) -- (244);
\draw[arc] (237) -- (86);
\draw[arc] (238) -- (207);
\draw[arc] (239) -- (285);
\draw[arc] (239) -- (423);
\draw[arc] (240) -- (291);
\draw[arc] (242) -- (200);
\draw[arc] (242) -- (329);
\draw[arc] (242) -- (409);
\draw[arc] (242) -- (122);
\draw[arc] (242) -- (155);
\draw[arc] (243) -- (397);
\draw[arc] (245) -- (292);
\draw[arc] (247) -- (354);
\draw[arc] (247) -- (373);
\draw[arc] (247) -- (127);
\draw[arc] (248) -- (96);
\draw[arc] (248) -- (409);
\draw[arc] (248) -- (206);
\draw[arc] (248) -- (407);
\draw[arc] (249) -- (96);
\draw[arc] (251) -- (312);
\draw[arc] (251) -- (164);
\draw[arc] (251) -- (357);
\draw[arc] (251) -- (298);
\draw[arc] (251) -- (207);
\draw[arc] (251) -- (241);
\draw[arc] (251) -- (149);
\draw[arc] (251) -- (405);
\draw[arc] (251) -- (119);
\draw[arc] (251) -- (88);
\draw[arc] (251) -- (379);
\draw[arc] (251) -- (412);
\draw[arc] (251) -- (190);
\draw[arc] (253) -- (160);
\draw[arc] (253) -- (234);
\draw[arc] (253) -- (205);
\draw[arc] (253) -- (433);
\draw[arc] (253) -- (307);
\draw[arc] (254) -- (197);
\draw[arc] (254) -- (105);
\draw[arc] (254) -- (112);
\draw[arc] (254) -- (412);
\draw[arc] (257) -- (289);
\draw[arc] (257) -- (360);
\draw[arc] (257) -- (337);
\draw[arc] (257) -- (442);
\draw[arc] (257) -- (255);
\draw[arc] (259) -- (320);
\draw[arc] (259) -- (378);
\draw[arc] (260) -- (256);
\draw[arc] (260) -- (307);
\draw[arc] (261) -- (160);
\draw[arc] (261) -- (354);
\draw[arc] (261) -- (228);
\draw[arc] (261) -- (330);
\draw[arc] (261) -- (365);
\draw[arc] (261) -- (206);
\draw[arc] (261) -- (430);
\draw[arc] (261) -- (150);
\draw[arc] (262) -- (194);
\draw[arc] (262) -- (140);
\draw[arc] (263) -- (256);
\draw[arc] (263) -- (307);
\draw[arc] (265) -- (442);
\draw[arc] (265) -- (138);
\draw[arc] (265) -- (140);
\draw[arc] (265) -- (383);
\draw[arc] (269) -- (442);
\draw[arc] (271) -- (96);
\draw[arc] (275) -- (307);
\draw[arc] (275) -- (99);
\draw[arc] (276) -- (398);
\draw[arc] (276) -- (307);
\draw[arc] (276) -- (168);
\draw[arc] (277) -- (130);
\draw[arc] (277) -- (420);
\draw[arc] (277) -- (364);
\draw[arc] (277) -- (303);
\draw[arc] (277) -- (287);
\draw[arc] (283) -- (141);
\draw[arc] (284) -- (289);
\draw[arc] (284) -- (266);
\draw[arc] (284) -- (140);
\draw[arc] (284) -- (442);
\draw[arc] (284) -- (250);
\draw[arc] (284) -- (383);
\draw[arc] (286) -- (405);
\draw[arc] (288) -- (409);
\draw[arc] (288) -- (307);
\draw[arc] (293) -- (258);
\draw[arc] (293) -- (168);
\draw[arc] (293) -- (264);
\draw[arc] (293) -- (410);
\draw[arc] (295) -- (348);
\draw[arc] (295) -- (133);
\draw[arc] (296) -- (185);
\draw[arc] (296) -- (449);
\draw[arc] (296) -- (181);
\draw[arc] (299) -- (258);
\draw[arc] (301) -- (416);
\draw[arc] (302) -- (86);
\draw[arc] (304) -- (110);
\draw[arc] (305) -- (105);
\draw[arc] (305) -- (132);
\draw[arc] (306) -- (289);
\draw[arc] (306) -- (131);
\draw[arc] (306) -- (420);
\draw[arc] (306) -- (169);
\draw[arc] (306) -- (268);
\draw[arc] (306) -- (334);
\draw[arc] (306) -- (112);
\draw[arc] (306) -- (241);
\draw[arc] (306) -- (243);
\draw[arc] (306) -- (310);
\draw[arc] (306) -- (153);
\draw[arc] (306) -- (255);
\draw[arc] (308) -- (224);
\draw[arc] (308) -- (430);
\draw[arc] (308) -- (16);
\draw[arc] (311) -- (226);
\draw[arc] (311) -- (297);
\draw[arc] (311) -- (140);
\draw[arc] (311) -- (86);
\draw[arc] (313) -- (296);
\draw[arc] (313) -- (329);
\draw[arc] (313) -- (237);
\draw[arc] (313) -- (410);
\draw[arc] (314) -- (96);
\draw[arc] (314) -- (256);
\draw[arc] (314) -- (453);
\draw[arc] (314) -- (136);
\draw[arc] (314) -- (11);
\draw[arc] (314) -- (364);
\draw[arc] (314) -- (433);
\draw[arc] (314) -- (309);
\draw[arc] (314) -- (373);
\draw[arc] (314) -- (183);
\draw[arc] (314) -- (127);
\draw[arc] (315) -- (177);
\draw[arc] (315) -- (138);
\draw[arc] (315) -- (142);
\draw[arc] (315) -- (423);
\draw[arc] (316) -- (164);
\draw[arc] (318) -- (256);
\draw[arc] (318) -- (307);
\draw[arc] (318) -- (166);
\draw[arc] (319) -- (12);
\draw[arc] (322) -- (231);
\draw[arc] (322) -- (168);
\draw[arc] (322) -- (107);
\draw[arc] (322) -- (236);
\draw[arc] (322) -- (369);
\draw[arc] (322) -- (309);
\draw[arc] (322) -- (183);
\draw[arc] (322) -- (123);
\draw[arc] (323) -- (256);
\draw[arc] (323) -- (303);
\draw[arc] (324) -- (373);
\draw[arc] (324) -- (303);
\draw[arc] (326) -- (442);
\draw[arc] (328) -- (256);
\draw[arc] (331) -- (258);
\draw[arc] (331) -- (300);
\draw[arc] (331) -- (255);
\draw[arc] (332) -- (380);
\draw[arc] (332) -- (426);
\draw[arc] (332) -- (131);
\draw[arc] (332) -- (132);
\draw[arc] (333) -- (312);
\draw[arc] (333) -- (382);
\draw[arc] (333) -- (105);
\draw[arc] (333) -- (241);
\draw[arc] (333) -- (280);
\draw[arc] (335) -- (307);
\draw[arc] (336) -- (320);
\draw[arc] (336) -- (252);
\draw[arc] (338) -- (442);
\draw[arc] (339) -- (390);
\draw[arc] (339) -- (200);
\draw[arc] (339) -- (233);
\draw[arc] (339) -- (107);
\draw[arc] (339) -- (110);
\draw[arc] (339) -- (303);
\draw[arc] (340) -- (105);
\draw[arc] (342) -- (307);
\draw[arc] (343) -- (424);
\draw[arc] (343) -- (442);
\draw[arc] (343) -- (383);
\draw[arc] (343) -- (207);
\draw[arc] (344) -- (256);
\draw[arc] (344) -- (200);
\draw[arc] (344) -- (155);
\draw[arc] (344) -- (307);
\draw[arc] (344) -- (181);
\draw[arc] (344) -- (123);
\draw[arc] (345) -- (200);
\draw[arc] (345) -- (256);
\draw[arc] (345) -- (83);
\draw[arc] (345) -- (231);
\draw[arc] (347) -- (307);
\draw[arc] (350) -- (51);
\draw[arc] (351) -- (224);
\draw[arc] (351) -- (450);
\draw[arc] (351) -- (99);
\draw[arc] (351) -- (291);
\draw[arc] (351) -- (167);
\draw[arc] (351) -- (199);
\draw[arc] (351) -- (266);
\draw[arc] (351) -- (300);
\draw[arc] (351) -- (241);
\draw[arc] (351) -- (180);
\draw[arc] (351) -- (340);
\draw[arc] (351) -- (86);
\draw[arc] (351) -- (312);
\draw[arc] (351) -- (90);
\draw[arc] (351) -- (414);
\draw[arc] (351) -- (383);
\draw[arc] (352) -- (334);
\draw[arc] (353) -- (290);
\draw[arc] (353) -- (425);
\draw[arc] (353) -- (273);
\draw[arc] (353) -- (434);
\draw[arc] (354) -- (307);
\draw[arc] (355) -- (279);
\draw[arc] (356) -- (140);
\draw[arc] (356) -- (349);
\draw[arc] (358) -- (226);
\draw[arc] (358) -- (207);
\draw[arc] (358) -- (176);
\draw[arc] (358) -- (338);
\draw[arc] (358) -- (383);
\draw[arc] (359) -- (226);
\draw[arc] (359) -- (290);
\draw[arc] (359) -- (164);
\draw[arc] (359) -- (112);
\draw[arc] (359) -- (412);
\draw[arc] (359) -- (383);
\draw[arc] (360) -- (207);
\draw[arc] (361) -- (302);
\draw[arc] (361) -- (371);
\draw[arc] (361) -- (166);
\draw[arc] (363) -- (312);
\draw[arc] (363) -- (132);
\draw[arc] (367) -- (320);
\draw[arc] (367) -- (98);
\draw[arc] (369) -- (256);
\draw[arc] (370) -- (373);
\draw[arc] (370) -- (303);
\draw[arc] (371) -- (86);
\draw[arc] (372) -- (164);
\draw[arc] (372) -- (329);
\draw[arc] (372) -- (298);
\draw[arc] (372) -- (369);
\draw[arc] (372) -- (371);
\draw[arc] (376) -- (12);
\draw[arc] (377) -- (129);
\draw[arc] (377) -- (265);
\draw[arc] (377) -- (280);
\draw[arc] (377) -- (153);
\draw[arc] (377) -- (283);
\draw[arc] (377) -- (412);
\draw[arc] (377) -- (434);
\draw[arc] (377) -- (326);
\draw[arc] (377) -- (207);
\draw[arc] (377) -- (337);
\draw[arc] (377) -- (101);
\draw[arc] (377) -- (103);
\draw[arc] (377) -- (241);
\draw[arc] (377) -- (255);
\draw[arc] (381) -- (258);
\draw[arc] (381) -- (133);
\draw[arc] (381) -- (263);
\draw[arc] (381) -- (264);
\draw[arc] (381) -- (281);
\draw[arc] (381) -- (409);
\draw[arc] (381) -- (413);
\draw[arc] (381) -- (287);
\draw[arc] (381) -- (292);
\draw[arc] (381) -- (168);
\draw[arc] (381) -- (183);
\draw[arc] (381) -- (191);
\draw[arc] (381) -- (320);
\draw[arc] (381) -- (321);
\draw[arc] (381) -- (449);
\draw[arc] (381) -- (328);
\draw[arc] (381) -- (329);
\draw[arc] (381) -- (335);
\draw[arc] (381) -- (214);
\draw[arc] (381) -- (224);
\draw[arc] (381) -- (98);
\draw[arc] (381) -- (99);
\draw[arc] (381) -- (354);
\draw[arc] (381) -- (108);
\draw[arc] (381) -- (368);
\draw[arc] (382) -- (190);
\draw[arc] (384) -- (416);
\draw[arc] (384) -- (101);
\draw[arc] (384) -- (392);
\draw[arc] (384) -- (169);
\draw[arc] (384) -- (140);
\draw[arc] (384) -- (142);
\draw[arc] (384) -- (110);
\draw[arc] (384) -- (180);
\draw[arc] (384) -- (312);
\draw[arc] (386) -- (112);
\draw[arc] (386) -- (330);
\draw[arc] (387) -- (325);
\draw[arc] (387) -- (141);
\draw[arc] (387) -- (397);
\draw[arc] (387) -- (279);
\draw[arc] (387) -- (250);
\draw[arc] (388) -- (307);
\draw[arc] (388) -- (287);
\draw[arc] (389) -- (307);
\draw[arc] (389) -- (183);
\draw[arc] (394) -- (20);
\draw[arc] (395) -- (224);
\draw[arc] (396) -- (270);
\draw[arc] (398) -- (96);
\draw[arc] (399) -- (136);
\draw[arc] (399) -- (292);
\draw[arc] (399) -- (133);
\draw[arc] (401) -- (215);
\draw[arc] (403) -- (224);
\draw[arc] (403) -- (256);
\draw[arc] (403) -- (258);
\draw[arc] (403) -- (320);
\draw[arc] (403) -- (133);
\draw[arc] (403) -- (303);
\draw[arc] (403) -- (433);
\draw[arc] (403) -- (181);
\draw[arc] (403) -- (309);
\draw[arc] (403) -- (409);
\draw[arc] (403) -- (28);
\draw[arc] (403) -- (287);
\draw[arc] (404) -- (199);
\draw[arc] (404) -- (282);
\draw[arc] (404) -- (300);
\draw[arc] (404) -- (397);
\draw[arc] (404) -- (209);
\draw[arc] (404) -- (250);
\draw[arc] (404) -- (383);
\draw[arc] (407) -- (185);
\draw[arc] (408) -- (307);
\draw[arc] (410) -- (307);
\draw[arc] (411) -- (200);
\draw[arc] (411) -- (99);
\draw[arc] (411) -- (309);
\draw[arc] (413) -- (307);
\draw[arc] (417) -- (289);
\draw[arc] (417) -- (291);
\draw[arc] (417) -- (132);
\draw[arc] (417) -- (297);
\draw[arc] (417) -- (140);
\draw[arc] (417) -- (207);
\draw[arc] (417) -- (241);
\draw[arc] (417) -- (243);
\draw[arc] (417) -- (405);
\draw[arc] (417) -- (278);
\draw[arc] (417) -- (215);
\draw[arc] (417) -- (280);
\draw[arc] (417) -- (279);
\draw[arc] (417) -- (285);
\draw[arc] (417) -- (375);
\draw[arc] (419) -- (101);
\draw[arc] (419) -- (365);
\draw[arc] (419) -- (241);
\draw[arc] (419) -- (149);
\draw[arc] (419) -- (442);
\draw[arc] (419) -- (190);
\draw[arc] (421) -- (130);
\draw[arc] (421) -- (99);
\draw[arc] (421) -- (258);
\draw[arc] (421) -- (324);
\draw[arc] (421) -- (233);
\draw[arc] (421) -- (234);
\draw[arc] (421) -- (111);
\draw[arc] (421) -- (370);
\draw[arc] (421) -- (407);
\draw[arc] (421) -- (123);
\draw[arc] (422) -- (200);
\draw[arc] (422) -- (268);
\draw[arc] (422) -- (309);
\draw[arc] (422) -- (183);
\draw[arc] (422) -- (185);
\draw[arc] (425) -- (140);
\draw[arc] (427) -- (140);
\draw[arc] (427) -- (141);
\draw[arc] (427) -- (149);
\draw[arc] (427) -- (150);
\draw[arc] (427) -- (412);
\draw[arc] (428) -- (140);
\draw[arc] (429) -- (105);
\draw[arc] (429) -- (289);
\draw[arc] (430) -- (368);
\draw[arc] (430) -- (409);
\draw[arc] (430) -- (258);
\draw[arc] (430) -- (307);
\draw[arc] (431) -- (197);
\draw[arc] (431) -- (405);
\draw[arc] (431) -- (86);
\draw[arc] (431) -- (190);
\draw[arc] (432) -- (160);
\draw[arc] (432) -- (270);
\draw[arc] (432) -- (342);
\draw[arc] (432) -- (214);
\draw[arc] (432) -- (408);
\draw[arc] (435) -- (433);
\draw[arc] (435) -- (90);
\draw[arc] (435) -- (127);
\draw[arc] (436) -- (226);
\draw[arc] (436) -- (300);
\draw[arc] (436) -- (149);
\draw[arc] (437) -- (160);
\draw[arc] (437) -- (244);
\draw[arc] (437) -- (142);
\draw[arc] (438) -- (98);
\draw[arc] (439) -- (8);
\draw[arc] (439) -- (99);
\draw[arc] (444) -- (433);
\draw[arc] (444) -- (140);
\draw[arc] (444) -- (45);
\draw[arc] (445) -- (325);
\draw[arc] (445) -- (234);
\draw[arc] (445) -- (367);
\draw[arc] (446) -- (105);
\draw[arc] (447) -- (190);
\draw[arc] (447) -- (215);
\draw[arc] (448) -- (289);
\draw[arc] (448) -- (140);
\draw[arc] (448) -- (371);
\draw[arc] (448) -- (412);
\draw[arc] (449) -- (433);
\draw[arc] (451) -- (130);
\draw[arc] (451) -- (165);
\draw[arc] (451) -- (362);
\draw[arc] (451) -- (12);
\draw[arc] (451) -- (177);
\draw[arc] (451) -- (83);
\draw[arc] (451) -- (28);
\draw[arc] (452) -- (161);
\draw[arc] (452) -- (83);
\draw[arc] (452) -- (297);
\draw[arc] (454) -- (383);
\end{pgfonlayer}

\node[anchor=west] at (307) {\tiny \textit{a} \label{node:McCathy}};
\node[anchor=west] at (13) {\tiny \textit{b}};
\node[anchor=north] at (65) {\tiny \textit{c}};
\end{tikzpicture}
    \caption{Preorder of Twitter accounts of members of the 117th US Congress.
    Each node corresponds to a cluster of accounts and the size of the node is proportional to the size of the cluster (most clusters contain just a single account.). 
    The color of the node corresponds to the party membership (Democrats: blue, Republicans: red, Other: gray), where clusters with mixed membership have a mixed color.
    The depicted edges are those of the transitive reduction of the partial order on the clusterings.
    The layout is computed using the dot algorithm implemented in Graphviz \citep{ellson2004graphviz}.}
    \label{fig:congress-preorder}
\end{figure}

