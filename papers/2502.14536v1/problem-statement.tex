\section{Preordering problem}
\label{sec:problem-statement}

\begin{figure}
    \centering
    \begin{tikzpicture}
    
    \node[vertex] (0) at (1, 2.4) {\scriptsize 0};
    \node[vertex] (1) at (-0.2, 1.5) {\scriptsize 1};
    \node[vertex] (2) at (2.2, 1.5) {\scriptsize 2};
    \node[vertex] (3) at (1, 0.8) {\scriptsize 3};
    \node[vertex] (4) at (1, 0) {\scriptsize 4};

    \def\sep{0.1pt}

    \draw[-latex] (0) edge[bend right=5]  node[auto, inner sep=\sep] {\scriptsize 3}(1);
    \draw[-latex] (0) edge[bend left=15]  node[auto, inner sep=\sep] {\scriptsize 3}(3);
    \draw[-latex] (1) edge[bend left=25]  node[auto, inner sep=\sep] {\scriptsize 4}(0);
    \draw[-latex] (1) edge[bend left=20]  node[auto, inner sep=3*\sep] {\scriptsize -1}(2);
    \draw[-latex] (1) edge[bend left=5]  node[auto, inner sep=\sep] {\scriptsize -1}(3);
    \draw[-latex] (1) edge[bend right=15]  node[auto, inner sep=\sep] {\scriptsize 1}(4);
    \draw[-latex] (2) edge[bend right=20]  node[auto, swap, inner sep=\sep] {\scriptsize -1}(0);
    \draw[-latex] (2) edge[bend left=10]  node[auto, inner sep=\sep] {\scriptsize 2}(3);
    \draw[-latex] (2) edge[bend left=30]  node[auto, inner sep=\sep] {\scriptsize 1}(4);
    \draw[-latex] (3) edge[bend left=15]  node[auto, inner sep=\sep] {\scriptsize -1}(0);
    \draw[-latex] (3) edge[bend left=15]  node[auto, inner sep=\sep] {\scriptsize 1}(4);
    \draw[-latex] (4) edge[bend right=35]  node[auto, inner sep=\sep] {\scriptsize -1}(0);
    \draw[-latex] (4) edge[bend left=35]  node[auto, inner sep=\sep] {\scriptsize -1}(1);
    \draw[-latex] (4) edge[bend right=10]  node[auto, inner sep=\sep] {\scriptsize 2}(2);
    \draw[-latex] (4) edge[bend left=20]  node[auto, inner sep=\sep] {\scriptsize -1}(3);

\end{tikzpicture}%
    \hspace{5ex}
    \begin{tikzpicture}
    
    \node[vertex] (0) at (1, 2.4) {\scriptsize 0};
    \node[vertex] (1) at (-0.2, 1.5) {\scriptsize 1};
    \node[vertex] (2) at (2.2, 1.5) {\scriptsize 2};
    \node[vertex] (3) at (1, 0.8) {\scriptsize 3};
    \node[vertex] (4) at (1, 0) {\scriptsize 4};

    \draw[-latex] (0) edge[bend left=20] (1);
    \draw[-latex] (0) edge (3);
    \draw[-latex] (0) edge[bend left=20] (4);
    \draw[-latex] (1) edge[bend left=20] (0);
    \draw[-latex] (1) edge (3);
    \draw[-latex] (1) edge (4);
    \draw[-latex] (2) edge (3);
    \draw[-latex] (2) edge[bend left=15] (4);
    \draw[-latex] (3) edge (4);
\end{tikzpicture}%
    \hspace{5ex}
    \begin{tikzpicture}
    
    \node[cluster, minimum size=0.8cm] (C01) at (0.4, 2) {};
    \node[cluster, minimum size=0.4cm] (C2) at (1.8, 2) {};
    \node[cluster, minimum size=0.4cm] (C3) at (1, 1) {};
    \node[cluster, minimum size=0.4cm] (C4) at (1, 0) {};
    \node[vertex, fill=white, xshift=-0.2cm] (0) at (C01) {\scriptsize 0};
    \node[vertex, fill=white, xshift=0.2cm] (1) at (C01) {\scriptsize 1};
    \node[vertex, fill=white] (2) at (C2) {\scriptsize 2};
    \node[vertex, fill=white] (3) at (C3) {\scriptsize 3};
    \node[vertex, fill=white] (4) at (C4) {\scriptsize 4};

    \draw[cluster_arc] (C01) edge (C3);
    \draw[cluster_arc] (C2) edge (C3);
    \draw[cluster_arc] (C3) edge (C4);

\end{tikzpicture}
    \caption{Depicted on the left is an instance of the preordering problem with $\lvert V \rvert = 5$. 
    The value $c_{ij}$ is written next to the arc from $i$ to $j$, and arcs that are not depicted have a value of $0$.
    Depicted in the middle is an optimal solution to the preordering problem.
    The value of this solution is $\maxpo(c) = 14$.
    The upper bound is $B(c) = 17$ and thus, the transitivity index is $T(c) = \frac{14}{17}$.
    Depicted on the right is the same preorder as a partial order on clusters (equivalence classes).
    Here and in all subsequent figures, partial orders are depicted by their unique transitive reduction.
    E.g., the arc $(2, 4)$ is not shown because it is implied by the arcs $(2, 3)$ and $(3, 4)$.}
    \label{fig:example}
\end{figure}

Throughout this article we consider some finite, non-empty set $V$, e.g., the accounts of a social network.
For simplicity and conciseness, we let $V_n$ denote the set of all $n$-tuples of pairwise distinct elements of $V$.
We refer to the elements of $V$ and $V_2$ as \emph{nodes} and \emph{arcs}, respectively.
We cast the preordering problem (\Cref{definition:preordering-problem}) with respect to $V$ and values $c\colon V_{2}\to \R$ as an ILP:
%
\begin{align}
	\max \quad & \sum_{ij \in V_2} c_{ij} x_{ij} \label{eq:objective} \\
	\textrm{subject~to} \quad & x_{ij} \in \{0,1\} && \forall\; ij \in V_2 \label{eq:binary} \\
	& x_{ij} + x_{jk} - x_{ik} \leq 1 && \forall\; ijk \in V_3  \label{eq:triangle}
\end{align}
%
An example is depicted in \Cref{fig:example}.
Let $\maxpo(c)$ denote the objective value of an optimal solution.
Note that the map from $x$ to $\mathord{\lesssim_{x}} \coloneqq \{ ij \in V_{2} \mid x_{ij} = 1 \} \cup \{ ii \in V^{2} \mid i \in V \}$ is a bijection from the feasible solutions of the ILP \eqref{eq:objective}--\eqref{eq:triangle} to the set of preorders on $V$.
Indeed, the objective value of a feasible solution $x$ to the ILP \eqref{eq:objective}--\eqref{eq:triangle}  is equal to the value of $\lesssim_{x}$ according to \Cref{definition:preordering-problem}.
See \Cref{sec:discussion-transitivity} for further details.

As shown by \citet{wakabayashi1998complexity}, the preordering problem is \textsc{NP}-hard even for values in $\order(\lvert V \rvert^6)$.
We strengthen this result by \Cref{thm:np-hard} that we prove in \Cref{sec:proof-np-hard}.
\begin{theorem}\label{thm:np-hard}
    The preordering problem is \textsc{NP}-hard even for values $c\colon V_2 \to \{-1, 0, 1\}$.
\end{theorem}

The sum of all positive values clearly is an upper bound to the preordering problem. 
Let
\begin{align}\label{eq:transitivity-measure}
    T(c) = \frac{\maxpo(c)}{B(c)} 
    \quad \text{with} \quad
    B(c) = \sum_{ij \in V_2\colon c_{ij} > 0} \hspace{-3ex} c_{ij}
\end{align}
denote the proportion of the optimal value to the upper bound.
We call it the \emph{transitivity index}.
It quantifies how transitive the values $c$ are.
If there exists a transitive relation that contains all pairs with positive value and no pair with negative value, then $\maxpo(c) = B(c)$, hence $T(c) = 1$.
If no such relation exists, then $T(c) < 1$.
As the empty relation is transitive, $\maxpo(c) \geq 0$, and thus, $T(c) \in [0, 1]$ for all $c\colon V_2 \to \R$.
