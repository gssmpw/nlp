\documentclass[pmlr]{jmlr}% new name PMLR (Proceedings of Machine Learning Research)

 % The following packages will be automatically loaded:
 % amsmath, amssymb, natbib, graphicx, url, algorithm2e

 %\usepackage{rotating}% for sideways figures and tables
\usepackage{longtable}% for long tables

 % The booktabs package is used by this sample document
 % (it provides \toprule, \midrule and \bottomrule).
 % Remove the next line if you don't require it.
\usepackage{booktabs}
 % The siunitx package is used by this sample document
 % to align numbers in a column by their decimal point.
 % Remove the next line if you don't require it.
% \usepackage[load-configurations=version-1]{siunitx} % newer version
 %\usepackage{siunitx}



%%%%%%%%%%%%%%%%%%%%%%%%%%%%%%%%%%%%%%%%%%%%%%%%%%%%%%%%%%%%%%%%%%%%%%%%%%%%%%%%
%% Custom packages and commands
%%%%%%%%%%%%%%%%%%%%%%%%%%%%%%%%%%%%%%%%%%%%%%%%%%%%%%%%%%%%%%%%%%%%%%%%%%%%%%%%

\usepackage{placeins} % for Appendix images
\usepackage{xcolor}
\newcommand{\todo}[1]{\textcolor{red}{[*** #1 ***]}}
\newcommand{\tom}[1]{\textcolor{blue}{[***Tom: #1 ***]}}
\newcommand{\robin}[1]{\textcolor{teal}{[***Robin: #1 ***]}}

%%%%%%%%%%%%%%%%%%%%%%%%%%%%%%%%%%%%%%%%%%%%%%%%%%%%%%%%%%%%%%%%%%%%%%%%%%%%%%%%




 % The following command is just for this sample document:
\newcommand{\cs}[1]{\texttt{\char`\\#1}}

 % Define an unnumbered theorem just for this sample document:
\theorembodyfont{\upshape}
\theoremheaderfont{\scshape}
\theorempostheader{:}
\theoremsep{\newline}
\newtheorem*{note}{Note}

 % change the arguments, as appropriate, in the following:
 % TODO: change this
\jmlrvolume{273}
\jmlryear{2025}
\jmlrworkshop{iRAISE Workshop at AAAI 2025}

\title[Spotting Early Issues in Computer Science Proposals]{AI Mentors for Student Projects:\\ Spotting Early Issues in Computer Science Proposals}

 % Use \Name{Author Name} to specify the name.

 % Spaces are used to separate forenames from the surname so that
 % the surnames can be picked up for the page header and copyright footer.
 
 % If the surname contains spaces, enclose the surname
 % in braces, e.g. \Name{John {Smith Jones}} similarly
 % if the name has a "von" part, e.g \Name{Jane {de Winter}}.
 % If the first letter in the forenames is a diacritic
 % enclose the diacritic in braces, e.g. \Name{{\'E}louise Smith}

 % *** Make sure there's no spurious space before \nametag ***

 % Two authors with the same address
  % \author{\Name{Anonymous Author(s)}} %\Email{abc@sample.com}\and
   %\Name{Author Name2} \Email{xyz@sample.com}\\
   %\addr Address}

 % Three or more authors with the same address:
 \author{\Name{Gati Aher} \Email{gaher@cs.cmu.edu}\\
  \Name{Robin Schmucker} \Email{rschmuck@cs.cmu.edu}\\
  \Name{Tom Mitchell} \Email{tom.mitchell@cs.cmu.edu}\\
  \Name{Zachary C. Lipton} \Email{zlipton@cs.cmu.edu}\\
 %  \Name{Author Name5} \Email{an5@sample.com}\\
 %  \Name{Author Name6} \Email{an6@sample.com}\\
 %  \Name{Author Name7} \Email{an7@sample.com}\\
 %  \Name{Author Name8} \Email{an8@sample.com}\\
 %  \Name{Author Name9} \Email{an9@sample.com}\\
 %  \Name{Author Name10} \Email{an10@sample.com}\\
 %  \Name{Author Name11} \Email{an11@sample.com}\\
 %  \Name{Author Name12} \Email{an12@sample.com}\\
 %  \Name{Author Name13} \Email{an13@sample.com}\\
 %  \Name{Author Name14} \Email{an14@sample.com}\\
  \addr Carnegie Mellon University, Pittsburgh, PA 15213, USA}


 % Authors with different addresses:
 % \author{\Name{Author Name1} \Email{abc@sample.com}\\
 % \addr Address 1
 % \AND
 % \Name{Author Name2} \Email{xyz@sample.com}\\
 % \addr Address 2
 %}
 % \editor{Editor's name}
 % \editors{List of editors' names}


%%%%%%%%%%%%%%%%%%%%%%%%%%%%%%%%%%%%%%%%%%%%%%%%%%%%%%%%%%%%%%%%%%%%%%%%%%%%%%%%
\begin{document}

\maketitle

\begin{abstract}
% -- (START) NEW
% sentence 1. (INTRODUCTION PARAGRAPH)
When executed well, project-based learning (PBL) engages students' intrinsic motivation, encourages students to learn far beyond a course's limited curriculum, and prepares students to think critically and maturely about the skills and tools at their disposal.
% sentence 2. (RELATED WORK PARAGRAPH)
However, educators experience mixed results when using PBL in their classrooms: some students thrive with minimal guidance and others flounder. Early evaluation of project proposals could help educators determine which students need more support, yet evaluating project proposals and student aptitude is time-consuming and difficult to scale.
% sentence 3. (SYSTEM DESIGN)
In this work, we design, implement, and conduct an initial user study ($n = 36$) for a software system that collects project proposals and aptitude information to support educators in determining whether a student is ready to engage with PBL.
% sentence 4. (SYSTEM EVALUTATION)
We find that (1) users perceived the system as helpful for writing project proposals and identifying tools and technologies to learn more about, (2) educator ratings indicate that users with less technical experience in the project topic tend to write lower-quality project proposals, and (3) GPT-4o's ratings show agreement with educator ratings.
% sentence 5. (DISCUSSION)
While the prospect of using LLMs to rate the quality of students' project proposals is promising, its long-term effectiveness strongly hinges on future efforts at characterizing indicators that reliably predict students' success and motivation to learn.
% --- (END) NEW

% --- (START) OLD
% sentence 1. (INTRODUCTION PARAGRAPH)
% Project-based learning (PBL) motivates students by allowing them to choose problems and design solutions that pique their interest. 

% sentence 2. (RELATED WORK PARAGRAPH)
% However, without guidance, students may struggle with vague, overly ambitious, or out-of-scope project ideas. 

% sentence 3. (SYSTEM DESIGN)
% To investigate the feasibility of an automated feedback feature to support students in the early stages of designing a project, we developed a scaffolded computer science project proposal activity with self-evaluation steps and ran a user study with 36 self-identified students on the Prolific crowdworker platform. 

% sentence 4. (SYSTEM EVALUTATION)
% We found that two teaching assistants and GPT-4o all labeled approximately 50\% of skills students wanted to develop as either unrelated to computer science or overly vague ($>80\%, \kappa>0.6$). Importantly, evaluations by teaching assistants and GPT-4o indicated that students with less computer science experience struggled more: they wrote down fewer specific computer science skills, struggled to link the skills they wanted to learn to computer science career paths, and produced lower-quality project proposals. 

% sentence 5. (DISCUSSION)
% These findings highlight the challenges faced by novice students and motivate the potential of generative AI as a tool to detect early issues in student projects within PBL contexts.
% ---- (END) OLD ---

% --- (START) ZACK's STRUCTURE ---
% sentence 1. (INTRODUCTION PARAGRAPH) PBL is compelling because fires up people's intrinsic motivation, learn far beyond material covered in class, nurtures autodidactic tendencies

% sentence 2. (RELATED WORK PARAGRAPH) BUT PBL is challenging to scale because some students thrive with minimal educator guidance and other flounder. Early evaluation of project proposals can help educators determine which students need more support before stuents run into major issues, yet making this judgement call in may be challenging and time-consuming because it involves reviewing freeform text on arbitary topics, classifying student aptitude based on incomplete knowledge, and depends on the instructor's ability to provide helpful and relevant feedback without overshadowing student motivation for a specific project topic.

% sentence 3. (SYSTEM DESIGN) IN THIS WORK, we design, build, and conduct user study for using a software system to collect project propsoals and auxiliary information that educators can use to evaulate a student's preparedness to engage with project-based learning.

% sentence 4. (SYSTEM EVALUTATION) We find that (1) users are excited to use our system to write project proposals and identify tools and technologies to learn more about, (2) users with less experience with the project topic write lower quality project propsoals according to educator ratings, and (3) LLMs' ratings of quality show agreement with educator ratings.

% sentence 5. (DISCUSSION) Our work suggests that LLMs have potential to detect issues in students' project proposals. However more work needs to be done to characterize which indicators reliably predict students' success and motivation to learn.
% --- (END) ZACK's STRUCTURE ---

\end{abstract}
\begin{keywords}
project-based learning, generative AI in education, learning analytics
\end{keywords}

\section{Introduction}
\label{sec:introduction}
The business processes of organizations are experiencing ever-increasing complexity due to the large amount of data, high number of users, and high-tech devices involved \cite{martin2021pmopportunitieschallenges, beerepoot2023biggestbpmproblems}. This complexity may cause business processes to deviate from normal control flow due to unforeseen and disruptive anomalies \cite{adams2023proceddsriftdetection}. These control-flow anomalies manifest as unknown, skipped, and wrongly-ordered activities in the traces of event logs monitored from the execution of business processes \cite{ko2023adsystematicreview}. For the sake of clarity, let us consider an illustrative example of such anomalies. Figure \ref{FP_ANOMALIES} shows a so-called event log footprint, which captures the control flow relations of four activities of a hypothetical event log. In particular, this footprint captures the control-flow relations between activities \texttt{a}, \texttt{b}, \texttt{c} and \texttt{d}. These are the causal ($\rightarrow$) relation, concurrent ($\parallel$) relation, and other ($\#$) relations such as exclusivity or non-local dependency \cite{aalst2022pmhandbook}. In addition, on the right are six traces, of which five exhibit skipped, wrongly-ordered and unknown control-flow anomalies. For example, $\langle$\texttt{a b d}$\rangle$ has a skipped activity, which is \texttt{c}. Because of this skipped activity, the control-flow relation \texttt{b}$\,\#\,$\texttt{d} is violated, since \texttt{d} directly follows \texttt{b} in the anomalous trace.
\begin{figure}[!t]
\centering
\includegraphics[width=0.9\columnwidth]{images/FP_ANOMALIES.png}
\caption{An example event log footprint with six traces, of which five exhibit control-flow anomalies.}
\label{FP_ANOMALIES}
\end{figure}

\subsection{Control-flow anomaly detection}
Control-flow anomaly detection techniques aim to characterize the normal control flow from event logs and verify whether these deviations occur in new event logs \cite{ko2023adsystematicreview}. To develop control-flow anomaly detection techniques, \revision{process mining} has seen widespread adoption owing to process discovery and \revision{conformance checking}. On the one hand, process discovery is a set of algorithms that encode control-flow relations as a set of model elements and constraints according to a given modeling formalism \cite{aalst2022pmhandbook}; hereafter, we refer to the Petri net, a widespread modeling formalism. On the other hand, \revision{conformance checking} is an explainable set of algorithms that allows linking any deviations with the reference Petri net and providing the fitness measure, namely a measure of how much the Petri net fits the new event log \cite{aalst2022pmhandbook}. Many control-flow anomaly detection techniques based on \revision{conformance checking} (hereafter, \revision{conformance checking}-based techniques) use the fitness measure to determine whether an event log is anomalous \cite{bezerra2009pmad, bezerra2013adlogspais, myers2018icsadpm, pecchia2020applicationfailuresanalysispm}. 

The scientific literature also includes many \revision{conformance checking}-independent techniques for control-flow anomaly detection that combine specific types of trace encodings with machine/deep learning \cite{ko2023adsystematicreview, tavares2023pmtraceencoding}. Whereas these techniques are very effective, their explainability is challenging due to both the type of trace encoding employed and the machine/deep learning model used \cite{rawal2022trustworthyaiadvances,li2023explainablead}. Hence, in the following, we focus on the shortcomings of \revision{conformance checking}-based techniques to investigate whether it is possible to support the development of competitive control-flow anomaly detection techniques while maintaining the explainable nature of \revision{conformance checking}.
\begin{figure}[!t]
\centering
\includegraphics[width=\columnwidth]{images/HIGH_LEVEL_VIEW.png}
\caption{A high-level view of the proposed framework for combining \revision{process mining}-based feature extraction with dimensionality reduction for control-flow anomaly detection.}
\label{HIGH_LEVEL_VIEW}
\end{figure}

\subsection{Shortcomings of \revision{conformance checking}-based techniques}
Unfortunately, the detection effectiveness of \revision{conformance checking}-based techniques is affected by noisy data and low-quality Petri nets, which may be due to human errors in the modeling process or representational bias of process discovery algorithms \cite{bezerra2013adlogspais, pecchia2020applicationfailuresanalysispm, aalst2016pm}. Specifically, on the one hand, noisy data may introduce infrequent and deceptive control-flow relations that may result in inconsistent fitness measures, whereas, on the other hand, checking event logs against a low-quality Petri net could lead to an unreliable distribution of fitness measures. Nonetheless, such Petri nets can still be used as references to obtain insightful information for \revision{process mining}-based feature extraction, supporting the development of competitive and explainable \revision{conformance checking}-based techniques for control-flow anomaly detection despite the problems above. For example, a few works outline that token-based \revision{conformance checking} can be used for \revision{process mining}-based feature extraction to build tabular data and develop effective \revision{conformance checking}-based techniques for control-flow anomaly detection \cite{singh2022lapmsh, debenedictis2023dtadiiot}. However, to the best of our knowledge, the scientific literature lacks a structured proposal for \revision{process mining}-based feature extraction using the state-of-the-art \revision{conformance checking} variant, namely alignment-based \revision{conformance checking}.

\subsection{Contributions}
We propose a novel \revision{process mining}-based feature extraction approach with alignment-based \revision{conformance checking}. This variant aligns the deviating control flow with a reference Petri net; the resulting alignment can be inspected to extract additional statistics such as the number of times a given activity caused mismatches \cite{aalst2022pmhandbook}. We integrate this approach into a flexible and explainable framework for developing techniques for control-flow anomaly detection. The framework combines \revision{process mining}-based feature extraction and dimensionality reduction to handle high-dimensional feature sets, achieve detection effectiveness, and support explainability. Notably, in addition to our proposed \revision{process mining}-based feature extraction approach, the framework allows employing other approaches, enabling a fair comparison of multiple \revision{conformance checking}-based and \revision{conformance checking}-independent techniques for control-flow anomaly detection. Figure \ref{HIGH_LEVEL_VIEW} shows a high-level view of the framework. Business processes are monitored, and event logs obtained from the database of information systems. Subsequently, \revision{process mining}-based feature extraction is applied to these event logs and tabular data input to dimensionality reduction to identify control-flow anomalies. We apply several \revision{conformance checking}-based and \revision{conformance checking}-independent framework techniques to publicly available datasets, simulated data of a case study from railways, and real-world data of a case study from healthcare. We show that the framework techniques implementing our approach outperform the baseline \revision{conformance checking}-based techniques while maintaining the explainable nature of \revision{conformance checking}.

In summary, the contributions of this paper are as follows.
\begin{itemize}
    \item{
        A novel \revision{process mining}-based feature extraction approach to support the development of competitive and explainable \revision{conformance checking}-based techniques for control-flow anomaly detection.
    }
    \item{
        A flexible and explainable framework for developing techniques for control-flow anomaly detection using \revision{process mining}-based feature extraction and dimensionality reduction.
    }
    \item{
        Application to synthetic and real-world datasets of several \revision{conformance checking}-based and \revision{conformance checking}-independent framework techniques, evaluating their detection effectiveness and explainability.
    }
\end{itemize}

The rest of the paper is organized as follows.
\begin{itemize}
    \item Section \ref{sec:related_work} reviews the existing techniques for control-flow anomaly detection, categorizing them into \revision{conformance checking}-based and \revision{conformance checking}-independent techniques.
    \item Section \ref{sec:abccfe} provides the preliminaries of \revision{process mining} to establish the notation used throughout the paper, and delves into the details of the proposed \revision{process mining}-based feature extraction approach with alignment-based \revision{conformance checking}.
    \item Section \ref{sec:framework} describes the framework for developing \revision{conformance checking}-based and \revision{conformance checking}-independent techniques for control-flow anomaly detection that combine \revision{process mining}-based feature extraction and dimensionality reduction.
    \item Section \ref{sec:evaluation} presents the experiments conducted with multiple framework and baseline techniques using data from publicly available datasets and case studies.
    \item Section \ref{sec:conclusions} draws the conclusions and presents future work.
\end{itemize}
\section{RELATED WORK}
\label{sec:relatedwork}
In this section, we describe the previous works related to our proposal, which are divided into two parts. In Section~\ref{sec:relatedwork_exoplanet}, we present a review of approaches based on machine learning techniques for the detection of planetary transit signals. Section~\ref{sec:relatedwork_attention} provides an account of the approaches based on attention mechanisms applied in Astronomy.\par

\subsection{Exoplanet detection}
\label{sec:relatedwork_exoplanet}
Machine learning methods have achieved great performance for the automatic selection of exoplanet transit signals. One of the earliest applications of machine learning is a model named Autovetter \citep{MCcauliff}, which is a random forest (RF) model based on characteristics derived from Kepler pipeline statistics to classify exoplanet and false positive signals. Then, other studies emerged that also used supervised learning. \cite{mislis2016sidra} also used a RF, but unlike the work by \citet{MCcauliff}, they used simulated light curves and a box least square \citep[BLS;][]{kovacs2002box}-based periodogram to search for transiting exoplanets. \citet{thompson2015machine} proposed a k-nearest neighbors model for Kepler data to determine if a given signal has similarity to known transits. Unsupervised learning techniques were also applied, such as self-organizing maps (SOM), proposed \citet{armstrong2016transit}; which implements an architecture to segment similar light curves. In the same way, \citet{armstrong2018automatic} developed a combination of supervised and unsupervised learning, including RF and SOM models. In general, these approaches require a previous phase of feature engineering for each light curve. \par

%DL is a modern data-driven technology that automatically extracts characteristics, and that has been successful in classification problems from a variety of application domains. The architecture relies on several layers of NNs of simple interconnected units and uses layers to build increasingly complex and useful features by means of linear and non-linear transformation. This family of models is capable of generating increasingly high-level representations \citep{lecun2015deep}.

The application of DL for exoplanetary signal detection has evolved rapidly in recent years and has become very popular in planetary science.  \citet{pearson2018} and \citet{zucker2018shallow} developed CNN-based algorithms that learn from synthetic data to search for exoplanets. Perhaps one of the most successful applications of the DL models in transit detection was that of \citet{Shallue_2018}; who, in collaboration with Google, proposed a CNN named AstroNet that recognizes exoplanet signals in real data from Kepler. AstroNet uses the training set of labelled TCEs from the Autovetter planet candidate catalog of Q1–Q17 data release 24 (DR24) of the Kepler mission \citep{catanzarite2015autovetter}. AstroNet analyses the data in two views: a ``global view'', and ``local view'' \citep{Shallue_2018}. \par


% The global view shows the characteristics of the light curve over an orbital period, and a local view shows the moment at occurring the transit in detail

%different = space-based

Based on AstroNet, researchers have modified the original AstroNet model to rank candidates from different surveys, specifically for Kepler and TESS missions. \citet{ansdell2018scientific} developed a CNN trained on Kepler data, and included for the first time the information on the centroids, showing that the model improves performance considerably. Then, \citet{osborn2020rapid} and \citet{yu2019identifying} also included the centroids information, but in addition, \citet{osborn2020rapid} included information of the stellar and transit parameters. Finally, \citet{rao2021nigraha} proposed a pipeline that includes a new ``half-phase'' view of the transit signal. This half-phase view represents a transit view with a different time and phase. The purpose of this view is to recover any possible secondary eclipse (the object hiding behind the disk of the primary star).


%last pipeline applies a procedure after the prediction of the model to obtain new candidates, this process is carried out through a series of steps that include the evaluation with Discovery and Validation of Exoplanets (DAVE) \citet{kostov2019discovery} that was adapted for the TESS telescope.\par
%



\subsection{Attention mechanisms in astronomy}
\label{sec:relatedwork_attention}
Despite the remarkable success of attention mechanisms in sequential data, few papers have exploited their advantages in astronomy. In particular, there are no models based on attention mechanisms for detecting planets. Below we present a summary of the main applications of this modeling approach to astronomy, based on two points of view; performance and interpretability of the model.\par
%Attention mechanisms have not yet been explored in all sub-areas of astronomy. However, recent works show a successful application of the mechanism.
%performance

The application of attention mechanisms has shown improvements in the performance of some regression and classification tasks compared to previous approaches. One of the first implementations of the attention mechanism was to find gravitational lenses proposed by \citet{thuruthipilly2021finding}. They designed 21 self-attention-based encoder models, where each model was trained separately with 18,000 simulated images, demonstrating that the model based on the Transformer has a better performance and uses fewer trainable parameters compared to CNN. A novel application was proposed by \citet{lin2021galaxy} for the morphological classification of galaxies, who used an architecture derived from the Transformer, named Vision Transformer (VIT) \citep{dosovitskiy2020image}. \citet{lin2021galaxy} demonstrated competitive results compared to CNNs. Another application with successful results was proposed by \citet{zerveas2021transformer}; which first proposed a transformer-based framework for learning unsupervised representations of multivariate time series. Their methodology takes advantage of unlabeled data to train an encoder and extract dense vector representations of time series. Subsequently, they evaluate the model for regression and classification tasks, demonstrating better performance than other state-of-the-art supervised methods, even with data sets with limited samples.

%interpretation
Regarding the interpretability of the model, a recent contribution that analyses the attention maps was presented by \citet{bowles20212}, which explored the use of group-equivariant self-attention for radio astronomy classification. Compared to other approaches, this model analysed the attention maps of the predictions and showed that the mechanism extracts the brightest spots and jets of the radio source more clearly. This indicates that attention maps for prediction interpretation could help experts see patterns that the human eye often misses. \par

In the field of variable stars, \citet{allam2021paying} employed the mechanism for classifying multivariate time series in variable stars. And additionally, \citet{allam2021paying} showed that the activation weights are accommodated according to the variation in brightness of the star, achieving a more interpretable model. And finally, related to the TESS telescope, \citet{morvan2022don} proposed a model that removes the noise from the light curves through the distribution of attention weights. \citet{morvan2022don} showed that the use of the attention mechanism is excellent for removing noise and outliers in time series datasets compared with other approaches. In addition, the use of attention maps allowed them to show the representations learned from the model. \par

Recent attention mechanism approaches in astronomy demonstrate comparable results with earlier approaches, such as CNNs. At the same time, they offer interpretability of their results, which allows a post-prediction analysis. \par


\section{\projecttitle Library}
\label{sec:abstraction}
\label{sec:recipe-implementation}
%\dimitra{system level design details here and the low-level API!}




\subsection{\projecttitle{} Architecture Overview} \label{subsec:overview}
\myparagraph{Distributed systems architecture} 
A distributed data store system uses a tiered architecture with a distributed data store layer for routing requests, a replication layer (provided by \projecttitle{}) for consistent data replication, and a data layer (Key-Value stores) for storage.  \projecttitle{} provides a trusted computing base using trusted execution environments (TEEs) to protect consensus fault tolerance protocols, including secure initialization of replicas and a direct I/O layer for efficient, secure communication.


Figure~\ref{fig:overview} shows the overview of a distributed data store system that builds on top of the \projecttitle{} system.  Distributed data stores implement a tiered architecture consisting of a \emph{distributed data store layer}, \emph{replication layer}, and  \emph{data layer}. In our case, the replication and data layers are provided by \projecttitle{}. The distributed data store layer maintains a routing table that matches the keyspace with the owners' nodes. This layer is responsible for forwarding client requests to the appropriate coordinator nodes (e.g., leader of the replication protocol) for execution. The \projecttitle{} replication layer is responsible for consistently replicating the data by executing the implemented protocol. After the protocol execution, \projecttitle{} nodes store the data in their Key-Value stores (KVs), the data layer, and they reply to the client~\cite{redis, rocksdb, leveldb, memcached2004}.

\myparagraph{\projecttitle{} architecture} \projecttitle{} design is based on a distributed setting of TEEs that implement a (distributed) trusted computing base (TCB) and shield the execution of unmodified CFT protocols against Byzantine failures. \projecttitle{}'s TCB contains the CFT protocol's code along with some metadata specific to the protocol. 

The code and TEEs of all replicas are attested before instantiating the protocol to ensure that the TEE hardware and the residing code are genuine. All authenticated replicas receive secrets (e.g., signing or encryption keys) and configuration data securely at initialization. 

Further, \projecttitle{} builds a \emph{direct I/O layer} comprised of a networking library for low-latency communication between nodes ($\S$~\ref{subsec:networkin}). The library bypasses the kernel stack for performance and shields the communication to guarantee non-equivocation and transferable authentication against Byzantine actors in the network. \projecttitle{} guarantees both properties by layering the non-equivocation and authentication layers on top of the direct I/O layer. In addition,  to strengthen \projecttitle{}'s security properties and eliminate syscalls, we map the network library software stack to the TEE's address space.

Lastly, \projecttitle{} builds the  \emph{data layer} on top of local KV store instances. Our design of the KV store increases the trust in individual nodes, allowing for local reads ($\S$~\ref{subsec:KV}). Our KV store achieves two goals: first, we guarantee trust to individual replicas to serve reads locally, and second, we limit the TCB size, optimizing the enclave memory usage. As shown in Figure~\ref{fig:overview}, \projecttitle{} keeps bulk data (values) in the host memory and stores only minimal data (keys + metadata) in the TEE area. The metadata, e.g., hash of the value, timestamps, etc., are kept along with keys in the TEE for integrity verification. 



Our work shows how to leverage modern hardware to build efficient, robust, and easily adaptable distributed protocols by meeting the aforementioned transformation requirements.
%(see Q1---Q3 below). Motivated by the recently launched cloud-hosted blockchain systems, we also argue that confidential BFT protocols are required to satisfy modern applications' needs for confidentiality (see Q4 below).
To achieve our goal, we need to address the following technical questions discussed in~$\S$~\ref{sec:motivation}.
Next, we present the implementation details or our work focusing on four core components of \projecttitle{} ($\S$~\ref{sec:recipe_impl_apis}). Table~\ref{tab:api} summarizes the \projecttitle{}'s API for each component. 

\begin{figure*}[t]
    \begin{center}
        \includegraphics[width=1\textwidth]{figs/recipe_full_system.pdf}
        %\vspace{-2pt}
        \caption{\projecttitle{}'s system architecture.}
       % \vspace{-1pt}
        \label{fig:overview}
    \end{center}
\end{figure*}


\section{Bellman Error Centering}

Centering operator $\mathcal{C}$ for a variable $x(s)$ is defined as follows:
\begin{equation}
\mathcal{C}x(s)\dot{=} x(s)-\mathbb{E}[x(s)]=x(s)-\sum_s{d_{s}x(s)},
\end{equation} 
where $d_s$ is the probability of $s$.
In vector form,
\begin{equation}
\begin{split}
\mathcal{C}\bm{x} &= \bm{x}-\mathbb{E}[x]\bm{1}\\
&=\bm{x}-\bm{x}^{\top}\bm{d}\bm{1},
\end{split}
\end{equation} 
where $\bm{1}$ is an all-ones vector.
For any vector $\bm{x}$ and $\bm{y}$ with a same distribution $\bm{d}$,
we have
\begin{equation}
\begin{split}
\mathcal{C}(\bm{x}+\bm{y})&=(\bm{x}+\bm{y})-(\bm{x}+\bm{y})^{\top}\bm{d}\bm{1}\\
&=\bm{x}-\bm{x}^{\top}\bm{d}\bm{1}+\bm{y}-\bm{y}^{\top}\bm{d}\bm{1}\\
&=\mathcal{C}\bm{x}+\mathcal{C}\bm{y}.
\end{split}
\end{equation}
\subsection{Revisit Reward Centering}


The update (\ref{src3}) is an unbiased estimate of the average reward
with  appropriate learning rate $\beta_t$ conditions.
\begin{equation}
\bar{r}_{t}\approx \lim_{n\rightarrow\infty}\frac{1}{n}\sum_{t=1}^n\mathbb{E}_{\pi}[r_t].
\end{equation}
That is 
\begin{equation}
r_t-\bar{r}_{t}\approx r_t-\lim_{n\rightarrow\infty}\frac{1}{n}\sum_{t=1}^n\mathbb{E}_{\pi}[r_t]= \mathcal{C}r_t.
\end{equation}
Then, the simple reward centering can be rewrited as:
\begin{equation}
V_{t+1}(s_t)=V_{t}(s_t)+\alpha_t [\mathcal{C}r_{t+1}+\gamma V_{t}(s_{t+1})-V_t(s_t)].
\end{equation}
Therefore, the simple reward centering is, in a strict sense, reward centering.

By definition of $\bar{\delta}_t=\delta_t-\bar{r}_{t}$,
let rewrite the update rule of the value-based reward centering as follows:
\begin{equation}
V_{t+1}(s_t)=V_{t}(s_t)+\alpha_t \rho_t (\delta_t-\bar{r}_{t}),
\end{equation}
where $\bar{r}_{t}$ is updated as:
\begin{equation}
\bar{r}_{t+1}=\bar{r}_{t}+\beta_t \rho_t(\delta_t-\bar{r}_{t}).
\label{vrc3}
\end{equation}
The update (\ref{vrc3}) is an unbiased estimate of the TD error
with  appropriate learning rate $\beta_t$ conditions.
\begin{equation}
\bar{r}_{t}\approx \mathbb{E}_{\pi}[\delta_t].
\end{equation}
That is 
\begin{equation}
\delta_t-\bar{r}_{t}\approx \mathcal{C}\delta_t.
\end{equation}
Then, the value-based reward centering can be rewrited as:
\begin{equation}
V_{t+1}(s_t)=V_{t}(s_t)+\alpha_t \rho_t \mathcal{C}\delta_t.
\label{tdcentering}
\end{equation}
Therefore, the value-based reward centering is no more,
 in a strict sense, reward centering.
It is, in a strict sense, \textbf{Bellman error centering}.

It is worth noting that this understanding is crucial, 
as designing new algorithms requires leveraging this concept.


\subsection{On the Fixpoint Solution}

The update rule (\ref{tdcentering}) is a stochastic approximation
of the following update:
\begin{equation}
\begin{split}
V_{t+1}&=V_{t}+\alpha_t [\bm{\mathcal{T}}^{\pi}\bm{V}-\bm{V}-\mathbb{E}[\delta]\bm{1}]\\
&=V_{t}+\alpha_t [\bm{\mathcal{T}}^{\pi}\bm{V}-\bm{V}-(\bm{\mathcal{T}}^{\pi}\bm{V}-\bm{V})^{\top}\bm{d}_{\pi}\bm{1}]\\
&=V_{t}+\alpha_t [\mathcal{C}(\bm{\mathcal{T}}^{\pi}\bm{V}-\bm{V})].
\end{split}
\label{tdcenteringVector}
\end{equation}
If update rule (\ref{tdcenteringVector}) converges, it is expected that
$\mathcal{C}(\mathcal{T}^{\pi}V-V)=\bm{0}$.
That is 
\begin{equation}
    \begin{split}
    \mathcal{C}\bm{V} &= \mathcal{C}\bm{\mathcal{T}}^{\pi}\bm{V} \\
    &= \mathcal{C}(\bm{R}^{\pi} + \gamma \mathbb{P}^{\pi} \bm{V}) \\
    &= \mathcal{C}\bm{R}^{\pi} + \gamma \mathcal{C}\mathbb{P}^{\pi} \bm{V} \\
    &= \mathcal{C}\bm{R}^{\pi} + \gamma (\mathbb{P}^{\pi} \bm{V} - (\mathbb{P}^{\pi} \bm{V})^{\top} \bm{d_{\pi}} \bm{1}) \\
    &= \mathcal{C}\bm{R}^{\pi} + \gamma (\mathbb{P}^{\pi} \bm{V} - \bm{V}^{\top} (\mathbb{P}^{\pi})^{\top} \bm{d_{\pi}} \bm{1}) \\  % 修正双重上标
    &= \mathcal{C}\bm{R}^{\pi} + \gamma (\mathbb{P}^{\pi} \bm{V} - \bm{V}^{\top} \bm{d_{\pi}} \bm{1}) \\
    &= \mathcal{C}\bm{R}^{\pi} + \gamma (\mathbb{P}^{\pi} \bm{V} - \bm{V}^{\top} \bm{d_{\pi}} \mathbb{P}^{\pi} \bm{1}) \\
    &= \mathcal{C}\bm{R}^{\pi} + \gamma (\mathbb{P}^{\pi} \bm{V} - \mathbb{P}^{\pi} \bm{V}^{\top} \bm{d_{\pi}} \bm{1}) \\
    &= \mathcal{C}\bm{R}^{\pi} + \gamma \mathbb{P}^{\pi} (\bm{V} - \bm{V}^{\top} \bm{d_{\pi}} \bm{1}) \\
    &= \mathcal{C}\bm{R}^{\pi} + \gamma \mathbb{P}^{\pi} \mathcal{C}\bm{V} \\
    &\dot{=} \bm{\mathcal{T}}_c^{\pi} \mathcal{C}\bm{V},
    \end{split}
    \label{centeredfixpoint}
    \end{equation}
where we defined $\bm{\mathcal{T}}_c^{\pi}$ as a centered Bellman operator.
We call equation (\ref{centeredfixpoint}) as centered Bellman equation.
And it is \textbf{centered fixpoint}.

For linear value function approximation, let define
\begin{equation}
\mathcal{C}\bm{V}_{\bm{\theta}}=\bm{\Pi}\bm{\mathcal{T}}_c^{\pi}\mathcal{C}\bm{V}_{\bm{\theta}}.
\label{centeredTDfixpoint}
\end{equation}
We call equation (\ref{centeredTDfixpoint}) as \textbf{centered TD fixpoint}.

\subsection{On-policy and Off-policy Centered TD Algorithms
with Linear Value Function Approximation}
Given the above centered TD fixpoint,
 mean squared centered Bellman error (MSCBE), is proposed as follows:
\begin{align*}
    \label{argminMSBEC}
 &\arg \min_{{\bm{\theta}}}\text{MSCBE}({\bm{\theta}}) \\
 &= \arg \min_{{\bm{\theta}}} \|\bm{\mathcal{T}}_c^{\pi}\mathcal{C}\bm{V}_{\bm{{\bm{\theta}}}}-\mathcal{C}\bm{V}_{\bm{{\bm{\theta}}}}\|_{\bm{D}}^2\notag\\
 &=\arg \min_{{\bm{\theta}}} \|\bm{\mathcal{T}}^{\pi}\bm{V}_{\bm{{\bm{\theta}}}} - \bm{V}_{\bm{{\bm{\theta}}}}-(\bm{\mathcal{T}}^{\pi}\bm{V}_{\bm{{\bm{\theta}}}} - \bm{V}_{\bm{{\bm{\theta}}}})^{\top}\bm{d}\bm{1}\|_{\bm{D}}^2\notag\\
 &=\arg \min_{{\bm{\theta}},\omega} \| \bm{\mathcal{T}}^{\pi}\bm{V}_{\bm{{\bm{\theta}}}} - \bm{V}_{\bm{{\bm{\theta}}}}-\omega\bm{1} \|_{\bm{D}}^2\notag,
\end{align*}
where $\omega$ is is used to estimate the expected value of the Bellman error.
% where $\omega$ is used to estimate $\mathbb{E}[\delta]$, $\omega \doteq \mathbb{E}[\mathbb{E}[\delta_t|S_t]]=\mathbb{E}[\delta]$ and $\delta_t$ is the TD error as follows:
% \begin{equation}
% \delta_t = r_{t+1}+\gamma
% {\bm{\theta}}_t^{\top}\bm{{\bm{\phi}}}_{t+1}-{\bm{\theta}}_t^{\top}\bm{{\bm{\phi}}}_t.
% \label{delta}
% \end{equation}
% $\mathbb{E}[\delta_t|S_t]$ is the Bellman error, and $\mathbb{E}[\mathbb{E}[\delta_t|S_t]]$ represents the expected value of the Bellman error.
% If $X$ is a random variable and $\mathbb{E}[X]$ is its expected value, then $X-\mathbb{E}[X]$ represents the centered form of $X$. 
% Therefore, we refer to $\mathbb{E}[\delta_t|S_t]-\mathbb{E}[\mathbb{E}[\delta_t|S_t]]$ as Bellman error centering and 
% $\mathbb{E}[(\mathbb{E}[\delta_t|S_t]-\mathbb{E}[\mathbb{E}[\delta_t|S_t]])^2]$ represents the the mean squared centered Bellman error, namely MSCBE.
% The meaning of (\ref{argminMSBEC}) is to minimize the mean squared centered Bellman error.
%The derivation of CTD is as follows.

First, the parameter  $\omega$ is derived directly based on
stochastic gradient descent:
\begin{equation}
\omega_{t+1}= \omega_{t}+\beta_t(\delta_t-\omega_t).
\label{omega}
\end{equation}

Then, based on stochastic semi-gradient descent, the update of 
the parameter ${\bm{\theta}}$ is as follows:
\begin{equation}
{\bm{\theta}}_{t+1}=
{\bm{\theta}}_{t}+\alpha_t(\delta_t-\omega_t)\bm{{\bm{\phi}}}_t.
\label{theta}
\end{equation}

We call (\ref{omega}) and (\ref{theta}) the on-policy centered
TD (CTD) algorithm. The convergence analysis with be given in
the following section.

In off-policy learning, we can simply multiply by the importance sampling
 $\rho$.
\begin{equation}
    \omega_{t+1}=\omega_{t}+\beta_t\rho_t(\delta_t-\omega_t),
    \label{omegawithrho}
\end{equation}
\begin{equation}
    {\bm{\theta}}_{t+1}=
    {\bm{\theta}}_{t}+\alpha_t\rho_t(\delta_t-\omega_t)\bm{{\bm{\phi}}}_t.
    \label{thetawithrho}
\end{equation}

We call (\ref{omegawithrho}) and (\ref{thetawithrho}) the off-policy centered
TD (CTD) algorithm.

% By substituting $\delta_t$ into Equations (\ref{omegawithrho}) and (\ref{thetawithrho}), 
% we can see that Equations (\ref{thetawithrho}) and (\ref{omegawithrho}) are formally identical 
% to the linear expressions of Equations (\ref{rewardcentering1}) and (\ref{rewardcentering2}), respectively. However, the meanings 
% of the corresponding parameters are entirely different.
% ${\bm{\theta}}_t$ is for approximating the discounted value function.
% $\bar{r_t}$ is an estimate of the average reward, while $\omega_t$ 
% is an estimate of the expected value of the Bellman error.
% $\bar{\delta_t}$ is the TD error for value-based reward centering, 
% whereas $\delta_t$ is the traditional TD error.

% This study posits that the CTD is equivalent to value-based reward 
% centering. However, CTD can be unified under a single framework 
% through an objective function, MSCBE, which also lays the 
% foundation for proving the algorithm's convergence. 
% Section 4 demonstrates that the CTD algorithm guarantees 
% convergence in the on-policy setting.

\subsection{Off-policy Centered TDC Algorithm with Linear Value Function Approximation}
The convergence of the  off-policy centered TD algorithm
may not be guaranteed.

To deal with this problem, we propose another new objective function, 
called mean squared projected centered Bellman error (MSPCBE), 
and derive Centered TDC algorithm (CTDC).

% We first establish some relationships between
%  the vector-matrix quantities and the relevant statistical expectation terms:
% \begin{align*}
%     &\mathbb{E}[(\delta({\bm{\theta}})-\mathbb{E}[\delta({\bm{\theta}})]){\bm{\phi}}] \\
%     &= \sum_s \mu(s) {\bm{\phi}}(s) \big( R(s) + \gamma \sum_{s'} P_{ss'} V_{\bm{\theta}}(s') - V_{\bm{\theta}}(s)  \\
%     &\quad \quad-\sum_s \mu(s)(R(s) + \gamma \sum_{s'} P_{ss'} V_{\bm{\theta}}(s') - V_{\bm{\theta}}(s))\big)\\
%     &= \bm{\Phi}^\top \mathbf{D} (\bm{TV}_{\bm{{\bm{\theta}}}} - \bm{V}_{\bm{{\bm{\theta}}}}-\omega\bm{1}),
% \end{align*}
% where $\omega$ is the expected value of the Bellman error and $\bm{1}$ is all-ones vector.

The specific expression of the objective function 
MSPCBE is as follows:
\begin{align}
    \label{MSPBECwithomega}
    &\arg \min_{{\bm{\theta}}}\text{MSPCBE}({\bm{\theta}})\notag\\ 
    % &= \arg \min_{{\bm{\theta}}}\big(\mathbb{E}[(\delta({\bm{\theta}}) - \mathbb{E}[\delta({\bm{\theta}})]) \bm{{\bm{\phi}}}]^\top \notag\\
    % &\quad \quad \quad\mathbb{E}[\bm{{\bm{\phi}}} \bm{{\bm{\phi}}}^\top]^{-1} \mathbb{E}[(\delta({\bm{\theta}}) - \mathbb{E}[\delta({\bm{\theta}})]) \bm{{\bm{\phi}}}]\big) \notag\\
    % &=\arg \min_{{\bm{\theta}},\omega}\mathbb{E}[(\delta({\bm{\theta}})-\omega) \bm{\bm{{\bm{\phi}}}}]^{\top} \mathbb{E}[\bm{\bm{{\bm{\phi}}}} \bm{\bm{{\bm{\phi}}}}^{\top}]^{-1}\mathbb{E}[(\delta({\bm{\theta}}) -\omega)\bm{\bm{{\bm{\phi}}}}]\\
    % &= \big(\bm{\Phi}^\top \mathbf{D} (\bm{TV}_{\bm{{\bm{\theta}}}} - \bm{V}_{\bm{{\bm{\theta}}}}-\omega\bm{1})\big)^\top (\bm{\Phi}^\top \mathbf{D} \bm{\Phi})^{-1} \notag\\
    % & \quad \quad \quad \bm{\Phi}^\top \mathbf{D} (\bm{TV}_{\bm{{\bm{\theta}}}} - \bm{V}_{\bm{{\bm{\theta}}}}-\omega\bm{1}) \notag\\
    % &= (\bm{TV}_{\bm{{\bm{\theta}}}} - \bm{V}_{\bm{{\bm{\theta}}}}-\omega\bm{1})^\top \mathbf{D} \bm{\Phi} (\bm{\Phi}^\top \mathbf{D} \bm{\Phi})^{-1} \notag\\
    % &\quad \quad \quad \bm{\Phi}^\top \mathbf{D} (\bm{TV}_{\bm{{\bm{\theta}}}} - \bm{V}_{\bm{{\bm{\theta}}}}-\omega\bm{1})\notag\\
    % &= (\bm{TV}_{\bm{{\bm{\theta}}}} - \bm{V}_{\bm{{\bm{\theta}}}}-\omega\bm{1})^\top {\bm{\Pi}}^\top \mathbf{D} {\bm{\Pi}} (\bm{TV}_{\bm{{\bm{\theta}}}} - \bm{V}_{\bm{{\bm{\theta}}}}-\omega\bm{1}) \notag\\
    &= \arg \min_{{\bm{\theta}}} \|\bm{\Pi}\bm{\mathcal{T}}_c^{\pi}\mathcal{C}\bm{V}_{\bm{{\bm{\theta}}}}-\mathcal{C}\bm{V}_{\bm{{\bm{\theta}}}}\|_{\bm{D}}^2\notag\\
    &= \arg \min_{{\bm{\theta}}} \|\bm{\Pi}(\bm{\mathcal{T}}_c^{\pi}\mathcal{C}\bm{V}_{\bm{{\bm{\theta}}}}-\mathcal{C}\bm{V}_{\bm{{\bm{\theta}}}})\|_{\bm{D}}^2\notag\\
    &= \arg \min_{{\bm{\theta}},\omega}\| {\bm{\Pi}} (\bm{\mathcal{T}}^{\pi}\bm{V}_{\bm{{\bm{\theta}}}} - \bm{V}_{\bm{{\bm{\theta}}}}-\omega\bm{1}) \|_{\bm{D}}^2\notag.
\end{align}
In the process of computing the gradient of the MSPCBE with respect to ${\bm{\theta}}$, 
$\omega$ is treated as a constant.
So, the derivation process of CTDC is the same 
as for the TDC algorithm \cite{sutton2009fast}, the only difference is that the original $\delta$ is replaced by $\delta-\omega$.
Therefore, the updated formulas of the centered TDC  algorithm are as follows:
\begin{equation}
 \bm{{\bm{\theta}}}_{k+1}=\bm{{\bm{\theta}}}_{k}+\alpha_{k}[(\delta_{k}- \omega_k) \bm{\bm{{\bm{\phi}}}}_k\\
 - \gamma\bm{\bm{{\bm{\phi}}}}_{k+1}(\bm{\bm{{\bm{\phi}}}}^{\top}_k \bm{u}_{k})],
\label{thetavmtdc}
\end{equation}
\begin{equation}
 \bm{u}_{k+1}= \bm{u}_{k}+\zeta_{k}[\delta_{k}-\omega_k - \bm{\bm{{\bm{\phi}}}}^{\top}_k \bm{u}_{k}]\bm{\bm{{\bm{\phi}}}}_k,
\label{uvmtdc}
\end{equation}
and
\begin{equation}
 \omega_{k+1}= \omega_{k}+\beta_k (\delta_k- \omega_k).
 \label{omegavmtdc}
\end{equation}
This algorithm is derived to work 
with a given set of sub-samples—in the form of 
triples $(S_k, R_k, S'_k)$ that match transitions 
from both the behavior and target policies. 

% \subsection{Variance Minimization ETD Learning: VMETD}
% Based on the off-policy TD algorithm, a scalar, $F$,  
% is introduced to obtain the ETD algorithm, 
% which ensures convergence under off-policy 
% conditions. This paper further introduces a scalar, 
% $\omega$, based on the ETD algorithm to obtain VMETD.
% VMETD by the following update:
% \begin{equation}
% \label{fvmetd}
%  F_t \leftarrow \gamma \rho_{t-1}F_{t-1}+1,
% \end{equation}
% \begin{equation}
%  \label{thetavmetd}
%  {{\bm{\theta}}}_{t+1}\leftarrow {{\bm{\theta}}}_t+\alpha_t (F_t \rho_t\delta_t - \omega_{t}){\bm{{\bm{\phi}}}}_t,
% \end{equation}
% \begin{equation}
%  \label{omegavmetd}
%  \omega_{t+1} \leftarrow \omega_t+\beta_t(F_t  \rho_t \delta_t - \omega_t),
% \end{equation}
% where $\rho_t =\frac{\pi(A_t | S_t)}{\mu(A_t | S_t)}$ and $\omega$ is used to estimate $\mathbb{E}[F \rho\delta]$, i.e., $\omega \doteq \mathbb{E}[F \rho\delta]$.

% (\ref{thetavmetd}) can be rewritten as
% \begin{equation*}
%  \begin{array}{ccl}
%  {{\bm{\theta}}}_{t+1}&\leftarrow& {{\bm{\theta}}}_t+\alpha_t (F_t \rho_t\delta_t - \omega_t){\bm{{\bm{\phi}}}}_t -\alpha_t \omega_{t+1}{\bm{{\bm{\phi}}}}_t\\
%   &=&{{\bm{\theta}}}_{t}+\alpha_t(F_t\rho_t\delta_t-\mathbb{E}_{\mu}[F_t\rho_t\delta_t|{{\bm{\theta}}}_t]){\bm{{\bm{\phi}}}}_t\\
%  &=&{{\bm{\theta}}}_t+\alpha_t F_t \rho_t (r_{t+1}+\gamma {{\bm{\theta}}}_t^{\top}{\bm{{\bm{\phi}}}}_{t+1}-{{\bm{\theta}}}_t^{\top}{\bm{{\bm{\phi}}}}_t){\bm{{\bm{\phi}}}}_t\\
%  & & \hspace{2em} -\alpha_t \mathbb{E}_{\mu}[F_t \rho_t \delta_t]{\bm{{\bm{\phi}}}}_t\\
%  &=& {{\bm{\theta}}}_t+\alpha_t \{\underbrace{(F_t\rho_tr_{t+1}-\mathbb{E}_{\mu}[F_t\rho_t r_{t+1}]){\bm{{\bm{\phi}}}}_t}_{{b}_{\text{VMETD},t}}\\
%  &&\hspace{-7em}- \underbrace{(F_t\rho_t{\bm{{\bm{\phi}}}}_t({\bm{{\bm{\phi}}}}_t-\gamma{\bm{{\bm{\phi}}}}_{t+1})^{\top}-{\bm{{\bm{\phi}}}}_t\mathbb{E}_{\mu}[F_t\rho_t ({\bm{{\bm{\phi}}}}_t-\gamma{\bm{{\bm{\phi}}}}_{t+1})]^{\top})}_{\textbf{A}_{\text{VMETD},t}}{{\bm{\theta}}}_t\}.
%  \end{array}
% \end{equation*}
% Therefore, 
% \begin{equation*}
%  \begin{array}{ccl}
%   &&\textbf{A}_{\text{VMETD}}\\
%   &=&\lim_{t \rightarrow \infty} \mathbb{E}[\textbf{A}_{\text{VMETD},t}]\\
%   &=& \lim_{t \rightarrow \infty} \mathbb{E}_{\mu}[F_t \rho_t {\bm{{\bm{\phi}}}}_t ({\bm{{\bm{\phi}}}}_t - \gamma {\bm{{\bm{\phi}}}}_{t+1})^{\top}]\\  
%   &&\hspace{1em}- \lim_{t\rightarrow \infty} \mathbb{E}_{\mu}[  {\bm{{\bm{\phi}}}}_t]\mathbb{E}_{\mu}[F_t \rho_t ({\bm{{\bm{\phi}}}}_t - \gamma {\bm{{\bm{\phi}}}}_{t+1})]^{\top}\\
%   &=& \lim_{t \rightarrow \infty} \mathbb{E}_{\mu}[{\bm{{\bm{\phi}}}}_tF_t \rho_t ({\bm{{\bm{\phi}}}}_t - \gamma {\bm{{\bm{\phi}}}}_{t+1})^{\top}]\\   
%   &&\hspace{1em}-\lim_{t \rightarrow \infty} \mathbb{E}_{\mu}[ {\bm{{\bm{\phi}}}}_t]\lim_{t \rightarrow \infty}\mathbb{E}_{\mu}[F_t \rho_t ({\bm{{\bm{\phi}}}}_t - \gamma {\bm{{\bm{\phi}}}}_{t+1})]^{\top}\\
%   && \hspace{-2em}=\sum_{s} d_{\mu}(s)\lim_{t \rightarrow \infty}\mathbb{E}_{\mu}[F_t|S_t = s]\mathbb{E}_{\mu}[\rho_t\bm{{\bm{\phi}}}_t(\bm{{\bm{\phi}}}_t - \gamma \bm{{\bm{\phi}}}_{t+1})^{\top}|S_t= s]\\   
%   &&\hspace{1em}-\sum_{s} d_{\mu}(s)\bm{{\bm{\phi}}}(s)\sum_{s} d_{\mu}(s)\lim_{t \rightarrow \infty}\mathbb{E}_{\mu}[F_t|S_t = s]\\
%   &&\hspace{7em}\mathbb{E}_{\mu}[\rho_t(\bm{{\bm{\phi}}}_t - \gamma \bm{{\bm{\phi}}}_{t+1})^{\top}|S_t = s]\\
%   &=& \sum_{s} f(s)\mathbb{E}_{\pi}[\bm{{\bm{\phi}}}_t(\bm{{\bm{\phi}}}_t- \gamma \bm{{\bm{\phi}}}_{t+1})^{\top}|S_t = s]\\   
%   &&\hspace{1em}-\sum_{s} d_{\mu}(s)\bm{{\bm{\phi}}}(s)\sum_{s} f(s)\mathbb{E}_{\pi}[(\bm{{\bm{\phi}}}_t- \gamma \bm{{\bm{\phi}}}_{t+1})^{\top}|S_t = s]\\
%   &=&\sum_{s} f(s) \bm{\bm{{\bm{\phi}}}}(s)(\bm{\bm{{\bm{\phi}}}}(s) - \gamma \sum_{s'}[\textbf{P}_{\pi}]_{ss'}\bm{\bm{{\bm{\phi}}}}(s'))^{\top}  \\
%   &&-\sum_{s} d_{\mu}(s) {\bm{{\bm{\phi}}}}(s) * \sum_{s} f(s)({\bm{{\bm{\phi}}}}(s) - \gamma \sum_{s'}[\textbf{P}_{\pi}]_{ss'}{\bm{{\bm{\phi}}}}(s'))^{\top}\\
%   &=&{\bm{\bm{\Phi}}}^{\top} \textbf{F} (\textbf{I} - \gamma \textbf{P}_{\pi}) \bm{\bm{\Phi}} - {\bm{\bm{\Phi}}}^{\top} {d}_{\mu} {f}^{\top} (\textbf{I} - \gamma \textbf{P}_{\pi}) \bm{\bm{\Phi}}  \\
%   &=&{\bm{\bm{\Phi}}}^{\top} (\textbf{F} - {d}_{\mu} {f}^{\top}) (\textbf{I} - \gamma \textbf{P}_{\pi}){\bm{\bm{\Phi}}} \\
%   &=&{\bm{\bm{\Phi}}}^{\top} (\textbf{F} (\textbf{I} - \gamma \textbf{P}_{\pi})-{d}_{\mu} {f}^{\top} (\textbf{I} - \gamma \textbf{P}_{\pi})){\bm{\bm{\Phi}}} \\
%   &=&{\bm{\bm{\Phi}}}^{\top} (\textbf{F} (\textbf{I} - \gamma \textbf{P}_{\pi})-{d}_{\mu} {d}_{\mu}^{\top} ){\bm{\bm{\Phi}}},
%  \end{array}
% \end{equation*}
% \begin{equation*}
%  \begin{array}{ccl}
%   &&{b}_{\text{VMETD}}\\
%   &=&\lim_{t \rightarrow \infty} \mathbb{E}[{b}_{\text{VMETD},t}]\\
%   &=& \lim_{t \rightarrow \infty} \mathbb{E}_{\mu}[F_t\rho_tR_{t+1}{\bm{{\bm{\phi}}}}_t]\\
%   &&\hspace{2em} - \lim_{t\rightarrow \infty} \mathbb{E}_{\mu}[{\bm{{\bm{\phi}}}}_t]\mathbb{E}_{\mu}[F_t\rho_kR_{k+1}]\\  
%   &=& \lim_{t \rightarrow \infty} \mathbb{E}_{\mu}[{\bm{{\bm{\phi}}}}_tF_t\rho_tr_{t+1}]\\
%   &&\hspace{2em} - \lim_{t\rightarrow \infty} \mathbb{E}_{\mu}[  {\bm{{\bm{\phi}}}}_t]\mathbb{E}_{\mu}[{\bm{{\bm{\phi}}}}_t]\mathbb{E}_{\mu}[F_t\rho_tr_{t+1}]\\ 
%   &=& \lim_{t \rightarrow \infty} \mathbb{E}_{\mu}[{\bm{{\bm{\phi}}}}_tF_t\rho_tr_{t+1}]\\
%   &&\hspace{2em} - \lim_{t \rightarrow \infty} \mathbb{E}_{\mu}[ {\bm{{\bm{\phi}}}}_t]\lim_{t \rightarrow \infty}\mathbb{E}_{\mu}[F_t\rho_tr_{t+1}]\\  
%   &=&\sum_{s} f(s) {\bm{{\bm{\phi}}}}(s)r_{\pi} - \sum_{s} d_{\mu}(s) {\bm{{\bm{\phi}}}}(s) * \sum_{s} f(s)r_{\pi}  \\
%   &=&\bm{\bm{\bm{\Phi}}}^{\top}(\textbf{F}-{d}_{\mu} {f}^{\top}){r}_{\pi}.
%  \end{array}
% \end{equation*}



\if 0
This section describes four core components of \projecttitle{}. Table~\ref{tab:api} summarizes the \projecttitle{}'s API for each component.%  (a) networking library, (b) KV store, (c) secure runtime, and (d) attestation and configuration management.


%(\projecttitle{}-lib): \emph{(i)} the shielded networking library which leverages direct I/O while also preventing Byzantine behaviors in the untrusted network infrastructure, \emph{(ii)} the KV store which guarantees trust to local reads and, \emph{(iii)} the attestation and secrets distribution service which ensures that only trusted nodes know the configuration, keys, etc.


%\pramod{fix missing citations and a lot of typos and grammar errors.}


\subsection{\projecttitle Implementation and APIs}
\label{sec:recipe_impl_apis}
\myparagraph{\projecttitle{} networking}
\label{subsec:networkin}
%\myparagraph{Networking overview}
\projecttitle{} adopts the Remote Procedure Call (RPC) paradigm~\cite{286500} over a generic network library with various transportation layers (Infiniband, RoCE, and DPDK), which is also favorable in the context of TEEs where traditional kernel-based networking is impractical~\cite{kuvaiskii2017sgxbounds}. %Below, we explain how the networking layer is initialized in \projecttitle{}, the requests workflow and the core implementation details.



\begin{table}[t]
%\small
\fontsize{7}{9}\selectfont 

\begin{center}
\begin{tabular}{ |c|c| }
 \hline
 \bf{Attestation API} &  \\ \hline
 \multirow{1}{*}{\texttt{attest(measurement)}} & Attests the node based on  a measurement.  \\  \hline \hline
 \bf{Initialization API} &  \\ \hline
 \texttt{create\_rpc(app\_ctx)} & Initializes an RPCobj. \\
  \texttt{init\_store()} & Initializes the KV store. \\
  \texttt{reg\_hdlr(\&func)} & Registers request handlers. \\ \hline \hline
 \bf{Network API} &  \\ \hline
 \texttt{send(\&msg\_buf)} & Prepares a req for transmission. \\
 %\hline
% \texttt{multicast(\&msg\_buf, nodes)} & Prepares a request for multicast. \\
 \multirow{1}{*}{\texttt{respond(\&msg\_buf)}} & Prepares a resp for transmission. \\
 \texttt{poll()} & Polls for incoming messages. \\\hline \hline
% \texttt{aggregates\_multicast()} &  \\ 
 \bf{Security API} &  \\ \hline
 \texttt{verify\_msg(\&msg\_buf)} & Verifies the authenticity/integrity and cnt of a msg. \\
 \texttt{shield\_msg(\&msg\_buf)} & Generates a shielded msg. \\ \hline \hline
% \texttt{aggregates\_multicast()} &  \\ 
 \bf{KV Store API} &  \\ \hline
 \texttt{write(key, value)} & Writes a KV to the store. \\
 \hline
 \multirow{2}{*}{\texttt{get(key, \&v$_{TEE}$)}} & Reads the value into \texttt{v$_{TEE}$} \\ & and verifies integrity. \\ \hline %\hline
 \if 0
 \bf{Trusted Leases API} &  \\ \hline
 \texttt{init\_lease(node\_id, thread\_id)} & Requests a lease from the grander.\\ \hline
 \texttt{renew\_lease(\&lease)} & Updates a lease.\\ \hline
 \texttt{grand\_or\_update\_lease(node\_id, thread\_id)} & Grands a lease.\\ \hline
 \texttt{exec\_with\_lease(\&lease, \&func, \&args\_list)} & Executes the func within the lease ownership.\\ [1ex] \hline
 \fi
\end{tabular}
\end{center}
%\vspace{-10pt}
\caption{\projecttitle{} library APIs.} \label{tab:api}
\vspace{-6pt}
\end{table}





\if 0

Developer effort – initialization. The developer must spec-
ify the number and the nature of the logical message flows
they require. In RDMA parlance each flow corresponds to
one queue pair (QP), i.e., a send and a receive queue. For
instance, consider Hermes where a write requires two broad-
cast rounds: invalidations (invs) and validations (vals). Each
worker in each node sets up three QPs: 1) to send and re-
ceive invs, 2) to send and receive acks (for the invs) and 3) to
send and receive vals. Splitting the communication in mes-
sage flows is the responsibility of the developer. To create
the QP for each message flow, the developer simply calls a
Odyssey function, passing details about the nature of the QP.

\fi





% \myparagraph{Initialization}
% Prior to application's execution, developers need to initialize the networking layer by specifying the number of concurrent available connections, the types of the available requests and by registering the appropriate (custom) request handlers. In \projecttitle{} terms, a communication endpoint corresponds to a per-thread RPC object (RPCobj) with private send/receive queues. All RPCobjs are registered to the same physical port (configurable). Initially, \projecttitle{} creates a handle to the NIC which is passed to all RPCobjs. Developers need to define the types of the RPC requests, each of which might be served by a different request handler. Request handlers are functions written by developers that are registered with the handle prior to the creation of the communication endpoints. Lastly, before executing the application's code, the connections between RPCobjs need to be correctly established.

\if 0
\projecttitle{} offers a \texttt{create\_rpc()} function that creates Remote Procedure Call (RPC) objects (rpc) bound to the NIC. Specifically this function takes the application context as an argument, i.e., node's NIC specification and port, remote IP and port, creates a communication endpoint and continuously tries to establish connection with the remote side. The function returns after the connection establishment. An rpc offers bidirectional communication between the two sides. Additionally, we need to register the request handler functions to the rpcs, i.e., pass a pointer function a the construction of the endpoint which states what will happen when a request of a specific type is received. The developer might to overwrite/implement the \texttt{init\_store()} function which will keep an application's state and metadata in the trusted enclave. By default \projecttitle{} comes with a thread-safe and lock-free hybrid skiplist based on~\cite{avocado, folly}. While implementing our use cases in $\$$~\ref{sec:eval}, we used two  \projecttitle{} skiplists for metadata and data accordingly.  %Lastly, we need to register the request handler functions to the \texttt{rpc}s, i.e., pass a pointer function a the construction of the endpoint which states what will happen when a request of a specific type is received.
\fi

\if 0
Developer effort – send and receive. For each QP, Odys-
sey maintains a send-FIFO and a receive-FIFO. Sending re-
quires that the developer first inserts messages in the send-
FIFO via an Odyssey insert function; later they can call a send
function to trigger the sending of all inserted messages. To re-
ceive messages, the developer need only call an Odyssey func-
tion that polls the receive-FIFO. Notably, the developer can
specify and register handlers to be called when calling any
one of the Odyssey functions. Therefore, the Odyssey polling
function will deliver the incoming messages, if any, to the
developer-specified handler.
\fi





%\myparagraph{send/receive operations}
We offer asynchronous network operations following the RPC paradigm. For each RPCobj, \projecttitle{} keeps a transmission (TX) and reception (RX) queue, organized as ring buffers. Developers enqueue requests and responses to requests via \projecttitle{}'s specific functions which place the message in the RPCobj's TX queue. Later, they can call a polling function that flushes the messages in the TX and drains the RX queues of an RPCobj. The function will trigger the sending of all queued messages and process all received requests and responses. Reception of a request triggers the execution of the request handler for that specific type. Reception of a response to a request triggers a cleanup function that releases all resources allocated for the request, e.g., message buffers and rate limiters (for congestion). %The cleanup functions can be overwritten by the developers for extra functionalities.

\if 0
\projecttitle{} offers high performance RPCs by extending eRPC~\cite{erpc} and DPDK~\cite{dpdk} in the context of TEEs. eRPC is .. Specifically, we place the message buffers outside the trusted enclave to both overcome the limited enclave memory and enable DMA operations~\footnote{DMA mappings are prohibited in the trusted area of a TEE as this violates their security properties~\cite{intel-sgx}}. We design a \texttt{send()} operation is used to submit a message for transmission. The message buffer is allocated by our library in \texttt{Hugepage} memory area and is later copied to the transmission queue (TX). Further, we provide a multicast() operation which creates identical copies of a message for all the recipient group.    Upon a reception of a request, the program control passes to the registered request handler where the function \texttt{respond()} can submit a response or \texttt{ACK} to that request. Lastly, the function \texttt{poll()} needs to be called regularly to fetch and process and send the incoming responses or requests and send the queued responses and requests respectively. 
\fi




%\myparagraph{Non-equivocation and authentication layers} %\manos{changed the paragraph titles to match the title and the first sentence below}
\label{non-equivocation-design}
%extends the properties of conventional CFT protocols to tolerate Byzantine settings by 
\projecttitle{}'s networking library embodies a non-equivocation and an authentication layer through two TEE-assisted primitives, the shield\_request() and verify\_request().%, shown in Algorithm~\ref{algo:primitives}.% (the \texttt{Attest()} primitive is used for initialization and is discussed in~$\S$~\ref{subsec:attestation}). We now explain the mechanisms, correctness arguments are presented in $\S$~\ref{sec:theory}.



\noindent{\underline{Non-equivocation layer}}: \projecttitle{} prevents replay attacks in the network with sequence numbers for the exchanged messages. Each replica maintains local sequence tuples of the form (view, cq, cnt$_{cq}$) where view is the current view number, cq is the communication endpoint(s) between two nodes, and cnt$_{cq}$ is the current trusted counter value in that view for the latest request sent over the cq. The sender assigns to messages a unique tuple of the form (view, cq, cnt$_{cq}$) and then increments cnt$_{cq}$ to guarantee monotonicity. % and rollback/forking attacks resilience.  %the request with the correct tuple and increments the $seq$.
Replicas execute the implemented CFT protocol for verified valid requests. Replicas can verify the freshness of a message by examining its cnt$_{cq}$ (verify\_request() primitive). The primitive verifies that the message's id (as part of the metadata) is consistent with the receiver's local counter rcnt$_{cq}$ (rcnt$_{cq}$ is the last seen valid message counter for received messages in cq). \projecttitle{}'s replicas are willing to accept ``future'' valid messages as these might come out of order, i.e., messages whose cnt$_{cq}$ is $>$ (rcnt$_{cq}$+1). These messages are processed and committed according to the CFT protocol. %~\footnote{The attestation that takes place at the TEE setup ($\S$~\ref{subsec:attestation}) ensures that only trusted nodes are capable of generating valid messages.}

\noindent{\underline{Authentication layer}}: For the authentication, we use cryptographic primitives (e.g., MAC and encryption functions when \projecttitle{} aims for confidentiality) to verify the integrity and the authenticity of the messages. Each message $m$ sent from a node $n_i$ to a node $n_j$ over a communication channel cq is accompanied by metadata (e.g., cnt$_{cq}$, view, sender and receiver nodes id) and the calculated message authentication code (MAC) $h_{cq}_{\sigma}_q$. The MAC is calculated over the payload and the metadata, then follows the message $m$. The sender node calls into the shield\_request(req, cq) and generates such a trusted message for the request req. %The trusted message is of the form [(req, (ReplicaView, cq, cnt\_{cq})), ${h_{cq}_{\sigma_{cq}})}$$>$.% containing the encrypted metadata and hash of the $req$ and the $req$ payload. %This function will marshal the current value of its trusted monotonic counter, the current view number, and the (cryptographic) hash of the message into one string. 

%\myparagraph{Non-equivocation} \projecttitle{} limits the equivocation of Byzantine (malicious) faults in the networking infrastructure using TEEs. Specifically, we guarantee non-equivocation via trusted counter assignment and verification. Each replica maintains a local sequence tuple $(v, cq, seq)$ where $v$ is the current view number, $cq$ is the communication (pair) channel between two nodes and $seq$ is the
%current trusted counter value in that view for the latest committed request sent in that communication channel. Each request is assigned a unique tuple $(v, cq, seq)$ which is maintained by the TEE of each replica to guarantee monotonic increments and rollback/forking attacks resilience. The coordinator node of a request assigns the request with the correct tuple and increments the $seq$.  Once a replica receives a request, they only accept it after its verification. The accepted requests pass through the underlying CFT protocol. Replicas verify the received requests using the \texttt{VerifyCounter}$(<req, (v, cq, seq)>, {h_{cq}})$ function. Specifically, the replica verifies the freshness of the message/request by examining its counter id. The message passes through the non-equivocation layer to verify that the counter associated with the received request (as part of the message metadata) is consistent with its local counter. Replicas in \projecttitle{} are willing to accept ``future'' valid messages as these might come out of order, i.e., messages whose seq number is $> (seq'_{cq}+1)$ ($seq'_{cq}$ is the last seen request number from that communication channel). Such messages are valid so \projecttitle{} accepts them. However, they are processed and committed when the underlying CFT protocol allows that.

%\dimitra{fix this}
%\myparagraph{Integrity verification} In \projecttitle{} we leverage basic primitives in modern cryptography such as hash functions to check and verify the integrity of the data the might reside in the untrusted areas including, the host memory and the network infrastructure. Each message $m$ sent from $n_i$ to $n_j$ over a communication channel $cq$ is accompanied by its calculated hash $h_{cq}$ that allows the recipient $n_j$ to verify that the message payload is genuine. A node that drives the client's request (coordinator) before sending the request to replicas need to call \texttt{ShieldRequest}$(req, cq)$ to generate an integrity-protected message for that request. This function will marshal the current value of its trusted monotonic counter, the current view number, and the (cryptographic) hash of the message into one string. The output is a bytestream of the form $<req, (\texttt{(ReplicaView}, cq, cnt_{cq})>, {h_{cq})}$ containing the metadata, the request payload and the computed hash of metadata and payload.

\if 0
\myparagraph{API} We offer a create\_rpc() function that creates a bound-to-the-NIC RPCobj. The function takes as an argument the application context, i.e., NIC specification and port, remote IP and port, creates a communication endpoint and establishes connection with the remote side. The function returns after the connection establishment. RPCobjs offer bidirectional communication between the two sides. Prior to the creation of RPCobj, developers need to specify and register the request types and handlers using the reg\_hdlr() which takes as an argument a reference to the preferred handler function. %The developer might to overwrite/implement the \texttt{init\_store()} function which will keep an application's state and metadata in the trusted enclave. By default \projecttitle{} comes with a thread-safe and lock-free hybrid skiplist based on~\cite{avocado, folly}. While implementing our use cases in $\$$~\ref{sec:eval}, we used two  \projecttitle{} skiplists for metadata and data accordingly.  %Lastly, we need to register the request handler functions to the \texttt{rpc}s, i.e., pass a pointer function a the construction of the endpoint which states what will happen when a request of a specific type is received.

For exchanging network messages, we designed a send() function which takes as arguments the session (connection) identifier, the message buffer to be sent, the request type and the cleanup function. This function submits a message for transmission. Upon a reception of a request, the program control passes to the registered request handler where the function respond() can submit a response or ACK to that request. Lastly, the function poll() needs to be called regularly to fetch or transmit the network messages in the TX and RX queues.




\fi 
\begin{comment}
~\footnote{DMA mappings are prohibited in the trusted area of a TEE as this violates TEE's security properties~\cite{intel-sgx, avocado, treaty}}
\end{comment}

\if 0
\myparagraph{Implementation details}
We designed \projecttitle{}'s high performance RPCs by extending eRPC~\cite{erpc} in the context of TEEs. We place the message buffers outside the enclave to overcome the limited enclave memory and enable DMA operations~\cite{intel-sgx, avocado, treaty}. The message buffers are allocated in Hugepage area and are later copied or mapped to the TX/RX queues. The networking buffers residing outside the TEE follow the trusted message format we discussed in \ref{subsec:overview}. As such, while outside the trusted area, their integrity (or confidentiality) can be verified upon reception.

We also adopted a rate limiter which can be configured to limit read and/or write requests. The use of a rate limiter was found to be useful in protocols which vastly saturated the available host memory (we found out that the R-ABD protocol was quickly exhausting all available memory in the system while the tail-node in the R-CR protocol also ran out-of-memory for read-heavy workloads). Lastly, we implement on top of \projecttitle{}'s network library a batching technique which queues the message buffers and merges them into a bigger buffer before transmission. The batching factor is configurable and has been proven extremely efficient for small messages (e.g., \SI{256}{\byte}).

\fi 


\myparagraph{Secure runtime} We build our codebase in C++ using \scone{} to access the TEE hardware. \scone{} exposes a modified libc library and combines user-level threading and asynchronous syscalls~\cite{flexsc} to reduce the cost of syscall execution. While we limit the number of syscalls, leveraging  \scone{}'s exit-less approach allows us to optimize the initialization phase that vastly allocates host memory for the network stack and the KV store. To enable NIC's DMA operations and memory mappings to the hugepages (for message buffers and TX/RX queues) ($\S$~\ref{subsec:networkin}), we overwrite the \texttt{mmap()} syscall of \scone{} to bypass its shield layer and allow the allocation of (untrusted) host memory. 

%For the cryptographic primitives, we build on OpenSSL~\cite{openssl}. Lastly, we build on a lease mechanism~\cite{t-lease} in \scone{} for auxiliary operations, e.g., failures detection and leader's election.

\if 0
\begin{algorithm}
\SetAlgoLined
%\fontsize
\small
%\fontsize{9}{10}\selectfont 

%\texttt{$seq'_{cq}$ the last committed sequence for that cq} \\

$\triangleright$ cnt$_{cq}$: the latest sent message id from  cq\\$\triangleright$ rcnt$_{cq}$: the last committed message id from cq


%\vspace{0.1cm}

\textbf{function} shield\_request(req, cq) \{ \\
\Indp
cnt$_{cq}$ $\leftarrow$ cnt$_{cq}$+1; t$\leftarrow$ (view, cq, cnt$_{cq}$);\\
$[$$h_{\sigma_{cq}}$, (req,t)$]$  $\leftarrow$ singed\_hash(req, t);\\
\textbf{return} $[$$h_{\sigma_{cq}}$, (req,t)$]$;\\
\Indm
\} \\



%\vspace{0.1cm}
\textbf{function} verify\_request($h_{\sigma_{cq}}$, req, (view, cq, cnt$_{cq}$)) \{ \\
\Indp
    \textbf{if} verify\_signature($h_{\sigma_{cq}}$, req, (view, cq, cnt$_{cq}$)) == True \textbf{then}\\
    \Indp
        \textbf{if} view == current\_view \textbf{then}\\
        \Indp
            \textbf{if} cnt$_{cq}$ <= rcnt$_{cq}$ \textbf{then}\\
            \Indp
                \textbf{return} [False, req, (view, cq, cnt$_{cq}$)]; \\
            \Indm
            \textbf{if} cnt$_{cq}$ == rcnt$_{cq}$+1) \textbf{then} rcnt$_{cq}$ $\leftarrow$ rcnt$_{cq}$+1;
            buffer\_locally(req, (view, cq, cnt$_{cq}$));\\
                \textbf{return} [True, req, (view, cq, cnt$_{cq}$)]; \\
            
        \Indm
    \Indm
    \textbf{return} [False, req, (view, cq, cnt$_{cq}$)]; \\

\Indm
\} \\
%\vspace{0.1cm}
\vspace{-1pt}
\caption{\projecttitle{}'s authentication primitives.}
\vspace{-3pt}
\label{algo:primitives}
\end{algorithm}

\fi


\myparagraph{\projecttitle{} key-value store}
\label{subsec:KV}
\projecttitle{} provides a lock-free, high-performant KV store based on a skip-list. We partition the keys from the values' space by placing the keys along with metadata (and a pointer to the value in host memory) inside the TEE's memory area, the {\em enclave}, and storing the values in the host memory. %Our partitioned KVs reduce the number of calculations for integrity checks, compared to prior work~\cite{shieldstore}, which implements (per-bucket) Merkle trees and re-calculates the root on each update. Importantly, separating the (keys + metadata) and the values between the enclave and untrusted unlimited memory decreases the Enclave Page Cache (EPC) pressure~\cite{speicher-fast}. %Our lock-free data structure supports concurrent operations and it is well-suited
%for increased parallelism.
\projecttitle{}'s KV store design resolves Byzantine errors since the metadata (and the code that accesses them) reside in the enclave. That said, \projecttitle{} allows for local reads as nodes can verify the integrity of the stored values.

\if 0
\myparagraph{Implementation and API} The developer might want to overwrite/implement the init\_store() function which will keep an application's state and metadata in the trusted enclave. \projecttitle{} implements its hybrid skiplist based on folly library~\cite{folly}. The write() function updates the KV while the get() function copies the value of the given key in the protected area. The function also verifies the value's integrity. We implement an allocator for host memory that is given as an initialization parameter to the KV store.

\fi 
%\myparagraph{Data properties}
 %Our partitioned scheme {\em seamlessly} strengthens the system's security properties further and can offer confidentiality by encrypting the values outside the TEE. \projecttitle{}-transformed protocols that further offer confidentiality outperform the BFT systems ($\S$~\ref{sec:eval}).


%that offers parallel write and read operations, however the developer might wish to overwrite those functions. Both operations calculate the hash of the given data which is placed in the enclave memory. The hash is used to verify the integrity of the data stored in the host memory (if any).




\myparagraph{Attestation and secrets distribution}~\label{subsec:attestation}
Remote attestation is the building block to verify the authenticity of a TEE, i.e., the code and the TEE state are the expected~\cite{Parno2010}. As such, \projecttitle{} provides attest(), generate\_quote() and remote\_attestation() primitives  that allow replicas to prove their trustworthiness to other replicas or clients. The attestation takes place before the control passes to the protocol's code. Only successfully attested nodes get access to secrets (e.g., signing or encryption keys, etc.) and configurations. 

%Essentially, \projecttitle{} needs to \emph{(1)} offer low-latency attestation of the joiner nodes (for fast recovery) and \emph{(2)} securely distribute the secrets and configuration data. \projecttitle{}'s attestation shields against Sybil attacks~\cite{sybilAttack}.

\if 0
\begin{algorithm}
\SetAlgoLined
%\fontsize
\small
%\fontsize{9}{10}\selectfont 

\textbf{function} remote\_attestation() \{ \\
 \Indp
 nonce $\leftarrow$ generate\_nonce();\\
 \textbf{send}(nonce, k$_{pub}$); \textbf{DHKE}(); quote$_{\sigma_{k_{pub}}}$ $\leftarrow$ \textbf{recv}();\\
 \textbf{if} verify\_signature(quote$_{\sigma_{k_{pub}}}$) == True \textbf{then}\\
    \Indp
        $\mu_{TEE}$ $\leftarrow$ decrypt(quote$_{\sigma_{k_{pub}}}$, k$_{priv}$);\\
        \textbf{if} (verify\_quote$(\mu_{TEE})$ == True) send\_secrets();\\
    \Indm
 \Indm
\} \\


%\vspace{0.1cm}
\textbf{function} attest() \{ \\
\Indp
    $\mu$ $\leftarrow$ gen\_enclave\_report(); \textbf{return} $\mu$;\\ 
\Indm
\} \\

%\vspace{0.1cm}
\textbf{function} generate\_quote($\mu$, k$_{pub}$) \{ \\
\Indp
    key$_{hw}$ $\leftarrow$ EGETKEY();\\
    quote $\leftarrow$ sign($\mu$, key$_{hw}$); 
    quote$_{\sigma_{k_{pub}}}$ $\leftarrow$ sign(quote, k$_{pub}$);\\
    \textbf{return } quote$_{\sigma_{k_{pub}}}$;\\
\Indm
\} \\
\caption{\projecttitle{}'s attestation primitive.}
\label{algo:attestation}
\vspace{-3pt}
\end{algorithm}
\fi 

%\vspace{-8pt}



%\myparagraph{Attestation design} 

%The application and the challenger first establish communication and, then, the application asks the challenger to provision secrets. 
The attestation process is initialized by the \emph{challenger}, a remote process that can verify the authenticity of a specific TEE. The challenger executes the remote\_attestation() function to send an attestation request to the application---usually in the form a nonce (a random number). The challenger and the application, then, pass through a Diffie-Hellman key exchange process~\cite{10.1145/359460.359473}. The application generates an ephemeral public key which is used by the challenger later to provision any secrets.

%When the TEE receives the nonce, it calls the attest() and generates a \emph{measurement} ($\mu$) of its state and loaded code. Following this, the TEE calls into the generate\_quote$(\mu, k_{pub})$ to sign $\mu$ (quote) with the $key_{hw}$ which is fetched from the TEE's h/w. The TEE signs and encrypts the quote quote$_{\sigma_{k_{pub}}}$ over the challenger's public key $k_{pub}$ which is, then, sent back to the challenger. Upon successful verifications of the quote$_{\sigma_{k_{pub}}}$, the challenger shares secrets and configurations.

%To offer low-latency attestations within the same datacenter that \projecttitle{} runs, we build a Configuration and Attestation service (CAS). The Protocol Designer (PD) deploys the CAS inside a TEE and attests it through the hardware vendor's attestation service---e.g., Intel Attestation Service (IAS~\cite{ias}). Once the CAS is attested, it is trusted and the PB can upload secrets and configurations. 

%The challenger asserts upon a failed verification and denies to share any secret or configuration data. Otherwise, it distributes all necessary shared secrets.

\if 0 
\myparagraph{Implementation details} 
%\projecttitle{} builds on top of a Configuration and Attestation Service (CAS) that is shipped with \scone{} and ultimately relies on hardware-based TEEs for secure secrets and configuration management. 
A trusted entity, i.e., the developer, must deploy the  Configuration and Attestation Service  CAS inside a TEE in a node that can also be part of the membership. Afterward, they need to attest the CAS through the TEE's attestation service---in our case, Intel Attestation Service (IAS~\cite{ias}). Once the CAS is attested, it can replace the TEE's attestation service. The trusted entity needs to spawn further Local Attestation Services (LASes) that perform local (intra-platform) attestation of processes and offer low-latency attestation of the new nodes. Before the CFT protocol commences, the CAS must also attest all LASes. 

When a \projecttitle{} process is loaded in the TEE (before the control passes to the application code), the LAS instructs a local attestation. The attestee generates a quote of the enclave (measurement). The LAS forwards the enclave's measurement to the CAS, which replies with a success or failure message indicating the authenticity of the process. After a successful attestation, the CAS stores the node's IP and provides the trusted process with secrets and configurations.

\myparagraph{API}
\projecttitle{} provides an attestation API to developers. Particularly, we provide the attest function that takes as arguments the IP of a trusted third-party service, CAS's IP for \projecttitle{}, and a generated enclave measurement of the code. Then, this service verifies that both the enclave signer and measurement are in the expected state and replies accordingly.
\fi 
%\dimitra{fix this}
%\myparagraph{Attestation and secrets management} Attestation is the process of demonstrating that the correct software is securely running within a TEE-enclave on an enabled platform. Secrets (e.g., certificates, encryption keys, etc.) and configurations (membership IPs and information) should only be provided to a replica after its successful attestation. Once an enclave is initialized and before the control is relinquished to the application's code, the attestation process is launched to verify the integrity and authenticity of the included code and data. Essentially, \projecttitle{} needs to (1.) offer low-latency attestation of the joiner nodes and (2.) securely provide the trusted enclave applications with secrets and configuration data (usually over the network).

%\projecttitle{} leverages TEEs to implement \texttt{RemoteAttestation}$()$, \texttt{Attest}$()$ and \texttt{GenerateQuote}$()$ primitives that allow a third party to attest an enclave. Next we describe the workflow of the attestation and the properties of those primitives.

%The attestation process is initialized by the challenger (or verifier)---a remote process (typically provided by the hardware vendor) that can prove the authenticity of a specific enclave. The application and the challenger first establish communication and, then, the application asks the challenger to provision secrets. The challenger executes the \texttt{RemoteAttestation}$()$ function where it first replies with an attestation request to the application---usually in the form a nonce (a random number
%generated for this one occasion). The challenger and the application also pass through a Diffie-Hellman key exchange (DHKE) process. The application generates an ephemeral public key which is used by the challenger later for provisioning secrets to the enclave.

%After receiving the nonce, the enclave calls into the \texttt{Attest}$()$ and generates a \emph{measurement} or \emph{report} that includes information about the enclave. The report needs to be sent to the challenger for verification. The enclave calls into \texttt{GenerateQuote$(\mu, k_{public})$} which sings the measurement $\mu$ over $key_{hw}$, where $key_{hw}$ is fetched from the TEE's hardware. Additionally, the function encrypts the quote over challenger's public key $k_{public}$. The encrypted quote can be sent and verified by the challenger with the corresponding verification key.

%The challenger asserts upon a failed verification and denies to share any secret or configuration data. Otherwise, it distributes all necessary shared secrets.

\if 0

\subsection{\projecttitle{} API}

\if 0
\subsection{\projecttitle{} client API}
\projecttitle{} exposes a simple \texttt{PUT}/\texttt{GET} API to clients. Both functions take as arguments the coordinator node's id, the view and the leader identifier (if any) that are known to the client. The API assigns a unique monotonically increased id to every request executed by the client. That is to help CFT distinguish already executed requests.
\fi

%\subsection{\projecttitle{} system-level API}
Table~\ref{tab:api} presents the core library that \projecttitle{} exposes to the developers. We implement our \projecttitle{} on top of \scone~\cite{arnautov2016scone} and \textsc{Palaemon}~\cite{palaemon} that use Intel SGX~\cite{intel-sgx} as the TEE and we extend eRPC~\cite{erpc} on top of DPDK~\cite{dpdk} for fast networking. Next we discuss the use-cases and the implementation details for each core function of \projecttitle{}.

\myparagraph{Attestation API} 

\myparagraph{Initialization API} Developers need to initialize the protocol by creating the communication endpoints between replicas. \projecttitle{} offers a \texttt{create\_rpc()} function that creates Remote Procedure Call (RPC) objects (rpc) bound to the NIC. Specifically this function takes the application context as an argument, i.e., node's NIC specification and port, remote IP and port, creates a communication endpoint and continuously tries to establish connection with the remote side. The function returns after the connection establishment. An rpc offers bidirectional communication between the two sides. Additionally, we need to register the request handler functions to the rpcs, i.e., pass a pointer function a the construction of the endpoint which states what will happen when a request of a specific type is received. The developer might to overwrite/implement the \texttt{init\_store()} function which will keep an application's state and metadata in the trusted enclave. By default \projecttitle{} comes with a thread-safe and lock-free hybrid skiplist based on~\cite{avocado, folly}. While implementing our use cases in $\$$~\ref{sec:eval}, we used two  \projecttitle{} skiplists for metadata and data accordingly.  %Lastly, we need to register the request handler functions to the \texttt{rpc}s, i.e., pass a pointer function a the construction of the endpoint which states what will happen when a request of a specific type is received.

\myparagraph{Network API} \projecttitle{} offers high performance RPCs by extending eRPC~\cite{erpc} in the context of TEEs. Specifically, we place the message buffers outside the trusted enclave to both overcome the limited enclave memory and enable DMA operations~\footnote{DMA mappings are prohibited in the trusted area of a TEE as this violates their security properties~\cite{intel-sgx}}. We design a \texttt{send()} operation is used to submit a message for transmission. The message buffer is allocated by our library in \texttt{Hugepage} memory area and is later copied to the transmission queue (TX). Further, we provide a \texttt{multicast()} operation which creates identical copies of a message for all the recipient group.    Upon a reception of a request, the program control passes to the registered request handler where the function \texttt{respond()} can submit a response or \texttt{ACK} to that request. Lastly, the function \texttt{poll()} needs to be called regularly to fetch and process and send the incoming responses or requests and send the queued responses and requests respectively. 

\myparagraph{KV Store API} 

\dimitra{
\myparagraph{Trusted Leases API} \projecttitle{} exposes an API for leases to guarantee linearizable local reads when the CFT protocol allows that. A thread on a node initializes a lease (\texttt{init\_lease()}) and afterwards can exec a function within the lease's ownership (\texttt{exec\_with\_lease()}). In case the lease has expired, the function is not executed. The protocol updates the expiration date of a current lease with the \texttt{renew\_lease()}. The lease granter node/service updates its lease table with the \texttt{grand\_or\_update\_lease()} function.
}
\fi


\fi



%\section{\projecttitle Library}
%\label{sec:abstraction}
%\label{sec:recipe-implementation}
%\dimitra{system level design details here and the low-level API!}


%This section describes four core components of \projecttitle{}. .%  (a) networking library, (b) KV store, (c) secure runtime, and (d) attestation and configuration management.


%(\projecttitle{}-lib): \emph{(i)} the shielded networking library which leverages direct I/O while also preventing Byzantine behaviors in the untrusted network infrastructure, \emph{(ii)} the KV store which guarantees trust to local reads and, \emph{(iii)} the attestation and secrets distribution service which ensures that only trusted nodes know the configuration, keys, etc.


%\pramod{fix missing citations and a lot of typos and grammar errors.}

\subsection{\projecttitle Implementation and APIs}
\label{sec:recipe_impl_apis}
\label{subsec:networkin}
\label{subsec:KV}
\label{subsec:attestation}
\label{non-equivocation-design}

Table~\ref{tab:api} summarizes the \projecttitle{}'s API for each system component.

\myparagraph{\projecttitle{} networking} \projecttitle{} adopts the Remote Procedure Call (RPC) paradigm~\cite{286500} over a generic network library with various transportation layers (Infiniband, RoCE, and DPDK), which is also favorable in the context of TEEs where traditional kernel-based networking is impractical~\cite{kuvaiskii2017sgxbounds}. %Below, we explain how the networking layer is initialized in \projecttitle{}, the requests workflow and the core implementation details.



\begin{table}[t]
\small
%\fontsize{7}{10}\selectfont 

\begin{center}
\begin{tabular}{ |c|c| }
 \hline
 \bf{Attestation API} &  \\ \hline
 \multirow{1}{*}{\texttt{attest(measurement)}} & Attests the node based on  a measurement.  \\  \hline \hline
 \bf{Initialization API} &  \\ \hline
 \texttt{create\_rpc(app\_ctx)} & Initializes an RPCobj. \\
  \texttt{init\_store()} & Initializes the KV store. \\
  \texttt{reg\_hdlr(\&func)} & Registers request handlers. \\ \hline \hline
 \bf{Network API} &  \\ \hline
 \texttt{send(\&msg\_buf)} & Prepares a req for transmission. \\
 %\hline
% \texttt{multicast(\&msg\_buf, nodes)} & Prepares a request for multicast. \\
 \multirow{1}{*}{\texttt{respond(\&msg\_buf)}} & Prepares a resp for transmission. \\
 \texttt{poll()} & Polls for incoming messages. \\\hline \hline
% \texttt{aggregates\_multicast()} &  \\ 
 \bf{Security API} &  \\ \hline
 \texttt{verify\_msg(\&msg\_buf)} & Verifies the authenticity/integrity and cnt of a msg. \\
 \texttt{shield\_msg(\&msg\_buf)} & Generates a shielded msg. \\ \hline \hline
% \texttt{aggregates\_multicast()} &  \\ 
 \bf{KV Store API} &  \\ \hline
 \texttt{write(key, value)} & Writes a KV to the store. \\
 \hline
 \multirow{2}{*}{\texttt{get(key, \&v$_{TEE}$)}} & Reads the value into \texttt{v$_{TEE}$} \\ & and verifies integrity. \\ \hline %\hline
 \if 0
 \bf{Trusted Leases API} &  \\ \hline
 \texttt{init\_lease(node\_id, thread\_id)} & Requests a lease from the grander.\\ \hline
 \texttt{renew\_lease(\&lease)} & Updates a lease.\\ \hline
 \texttt{grand\_or\_update\_lease(node\_id, thread\_id)} & Grands a lease.\\ \hline
 \texttt{exec\_with\_lease(\&lease, \&func, \&args\_list)} & Executes the func within the lease ownership.\\ [1ex] \hline
 \fi
\end{tabular}
\end{center}
%\vspace{-10pt}
\caption{\projecttitle{} library APIs.} \label{tab:api}
\vspace{-6pt}
\end{table}





\if 0

Developer effort – initialization. The developer must spec-
ify the number and the nature of the logical message flows
they require. In RDMA parlance each flow corresponds to
one queue pair (QP), i.e., a send and a receive queue. For
instance, consider Hermes where a write requires two broad-
cast rounds: invalidations (invs) and validations (vals). Each
worker in each node sets up three QPs: 1) to send and re-
ceive invs, 2) to send and receive acks (for the invs) and 3) to
send and receive vals. Splitting the communication in mes-
sage flows is the responsibility of the developer. To create
the QP for each message flow, the developer simply calls a
Odyssey function, passing details about the nature of the QP.

\fi





\noindent\underline{Initialization.} Prior to the application's execution, developers need to initialize the networking layer by specifying the number of concurrent available connections, the types of the available requests, and by registering the appropriate (custom) request handlers. In \projecttitle{} terms, a communication endpoint corresponds to a per-thread RPC object (RPCobj) with private send/receive queues. All RPCobjs are registered to the same physical port (configurable). Initially, \projecttitle{} creates a handle to the NIC, which is passed to all RPCobjs. Developers need to define the types of RPC requests, each of which might be served by a different request handler. Request handlers are functions written by developers that are registered with the handle prior to the creation of the communication endpoints. Lastly, before executing the application's code, the connections between RPCobjs need to be correctly established.

\if 0
\projecttitle{} offers a \texttt{create\_rpc()} function that creates Remote Procedure Call (RPC) objects (rpc) bound to the NIC. Specifically, this function takes the application context as an argument, i.e., node's NIC specification and port, remote IP and port, create a communication endpoint and continuously tries to establish a connection with the remote side. The function returns after the connection establishment. An RPC offers bidirectional communication between the two sides. Additionally, we need to register the request handler functions to the rpcs, i.e., pass a pointer function to the construction of the endpoint, which states what will happen when a request of a specific type is received. The developer might overwrite/implement the \texttt{init\_store()} function, which will keep an application's state and metadata in the trusted enclave. By default \projecttitle{} comes with a thread-safe and lock-free hybrid skiplist based on~\cite{avocado, folly}. While implementing our use cases in $\$$~\ref{sec:eval}, we used two  \projecttitle{} skiplists for metadata and data accordingly.  %Lastly, we need to register the request handler functions to the \texttt{rpc}s, i.e., pass a pointer function a the construction of the endpoint which states what will happen when a request of a specific type is received.
\fi

\if 0
Developer effort – send and receive. For each QP, Odys-
sey maintains a send-FIFO and a receive-FIFO. Sending re-
quires that the developer first inserts messages in the send-
FIFO via an Odyssey insert function; later they can call a send
function to trigger the sending of all inserted messages. To re-
ceive messages, the developer need only call an Odyssey func-
tion that polls the receive-FIFO. Notably, the developer can
specify and register handlers to be called when calling any
one of the Odyssey functions. Therefore, the Odyssey polling
function will deliver the incoming messages, if any, to the
developer-specified handler.
\fi





\noindent\underline{send/receive operations.} We offer asynchronous network operations following the RPC paradigm. For each RPCobj, \projecttitle{} keeps a transmission (TX) and reception (RX) queue, organized as ring buffers. Developers enqueue requests and responses to requests via \projecttitle{}'s specific functions, which place the message in the RPCobj's TX queue. Later, they can call a polling function that flushes the messages in the TX and drains the RX queues of an RPCobj. The function will trigger the sending of all queued messages and process all received requests and responses. Reception of a request triggers the execution of the request handler for that specific type. Reception of a response to a request triggers a cleanup function that releases all resources allocated for the request, e.g., message buffers and rate limiters (for congestion). %The cleanup functions can be overwritten by the developers for extra functionalities.

\if 0
\projecttitle{} offers high performance RPCs by extending eRPC~\cite{erpc} and DPDK~\cite{dpdk} in the context of TEEs. eRPC is .. Specifically, we place the message buffers outside the trusted enclave to both overcome the limited enclave memory and enable DMA operations~\footnote{DMA mappings are prohibited in the trusted area of a TEE as this violates their security properties~\cite{intel-sgx}}. We design a \texttt{send()} operation is used to submit a message for transmission. The message buffer is allocated by our library in \texttt{Hugepage} memory area and is later copied to the transmission queue (TX). Further, we provide a multicast() operation which creates identical copies of a message for all the recipient group.    Upon a reception of a request, the program control passes to the registered request handler where the function \texttt{respond()} can submit a response or \texttt{ACK} to that request. Lastly, the function \texttt{poll()} needs to be called regularly to fetch and process and send the incoming responses or requests and send the queued responses and requests respectively. 
\fi



 %The trusted message is of the form [(req, (ReplicaView, cq, cnt\_{cq})), ${h_{cq}_{\sigma_{cq}})}$$>$.% containing the encrypted metadata and hash of the $req$ and the $req$ payload. %This function will marshal the current value of its trusted monotonic counter, the current view number, and the (cryptographic) hash of the message into one string. 

%\myparagraph{Non-equivocation} \projecttitle{} limits the equivocation of Byzantine (malicious) faults in the networking infrastructure using TEEs. Specifically, we guarantee non-equivocation via trusted counter assignment and verification. Each replica maintains a local sequence tuple $(v, cq, seq)$ where $v$ is the current view number, $cq$ is the communication (pair) channel between two nodes and $seq$ is the
%current trusted counter value in that view for the latest committed request sent in that communication channel. Each request is assigned a unique tuple $(v, cq, seq)$ which is maintained by the TEE of each replica to guarantee monotonic increments and rollback/forking attacks resilience. The coordinator node of a request assigns the request with the correct tuple and increments the $seq$.  Once a replica receives a request, they only accept it after its verification. The accepted requests pass through the underlying CFT protocol. Replicas verify the received requests using the \texttt{VerifyCounter}$(<req, (v, cq, seq)>, {h_{cq}})$ function. Specifically, the replica verifies the freshness of the message/request by examining its counter id. The message passes through the non-equivocation layer to verify that the counter associated with the received request (as part of the message metadata) is consistent with its local counter. Replicas in \projecttitle{} are willing to accept ``future'' valid messages as these might come out of order, i.e., messages whose seq number is $> (seq'_{cq}+1)$ ($seq'_{cq}$ is the last seen request number from that communication channel). Such messages are valid so \projecttitle{} accepts them. However, they are processed and committed when the underlying CFT protocol allows that.

%\dimitra{fix this}
%\myparagraph{Integrity verification} In \projecttitle{} we leverage basic primitives in modern cryptography such as hash functions to check and verify the integrity of the data the might reside in the untrusted areas including, the host memory and the network infrastructure. Each message $m$ sent from $n_i$ to $n_j$ over a communication channel $cq$ is accompanied by its calculated hash $h_{cq}$ that allows the recipient $n_j$ to verify that the message payload is genuine. A node that drives the client's request (coordinator) before sending the request to replicas need to call \texttt{ShieldRequest}$(req, cq)$ to generate an integrity-protected message for that request. This function will marshal the current value of its trusted monotonic counter, the current view number, and the (cryptographic) hash of the message into one string. The output is a bytestream of the form $<req, (\texttt{(ReplicaView}, cq, cnt_{cq})>, {h_{cq})}$ containing the metadata, the request payload and the computed hash of metadata and payload.

\noindent\underline{API.} We offer a create\_rpc() function that creates a bound-to-the-NIC RPCobj. The function takes the application context, i.e., NIC specification and port, remote IP and port, as an argument, creates a communication endpoint, and establishes a connection with the remote side. The function returns after the connection establishment. RPCobjs offer bidirectional communication between the two sides. Prior to the creation of RPCobj, developers need to specify and register the request types and handlers using the reg\_hdlr() which takes as an argument a reference to the preferred handler function. %The developer might to overwrite/implement the \texttt{init\_store()} function which will keep an application's state and metadata in the trusted enclave. By default \projecttitle{} comes with a thread-safe and lock-free hybrid skiplist based on~\cite{avocado, folly}. While implementing our use cases in $\$$~\ref{sec:eval}, we used two  \projecttitle{} skiplists for metadata and data accordingly.  %Lastly, we need to register the request handler functions to the \texttt{rpc}s, i.e., pass a pointer function a the construction of the endpoint which states what will happen when a request of a specific type is received.

For exchanging network messages, we designed a send() function that takes the session (connection) identifier, the message buffer to be sent, the request type, and the cleanup function as arguments. This function submits a message for transmission. Upon reception of a request, the program control passes to the registered request handler, where the function respond() can submit a response or ACK to that request. Lastly, the function poll() needs to be called regularly to fetch or transmit the network messages in the TX and RX queues.





\begin{comment}
~\footnote{DMA mappings are prohibited in the trusted area of a TEE as this violates TEE's security properties~\cite{intel-sgx, avocado, treaty}}
\end{comment}

\if 0
\myparagraph{Implementation details}
We designed \projecttitle{}'s high-performance RPCs by extending eRPC~\cite{erpc} in the context of TEEs. We place the message buffers outside the enclave to overcome the limited enclave memory and enable DMA operations~\cite{intel-sgx, avocado, treaty}. The message buffers are allocated in Hugepage area and are later copied or mapped to the TX/RX queues. The networking buffers residing outside the TEE follow the trusted message format we discussed in \ref{subsec:overview}. As such, while outside the trusted area, their integrity (or confidentiality) can be verified upon reception.

We also adopted a rate limiter which can be configured to limit read and/or write requests. The use of a rate limiter was found to be useful in protocols which vastly saturated the available host memory (we found out that the R-ABD protocol was quickly exhausting all available memory in the system while the tail-node in the R-CR protocol also ran out-of-memory for read-heavy workloads). Lastly, we implement on top of \projecttitle{}'s network library a batching technique which queues the message buffers and merges them into a bigger buffer before transmission. The batching factor is configurable and has been proven extremely efficient for small messages (e.g., \SI{256}{\byte}).

\fi 


\myparagraph{Secure runtime} We build our codebase in C++ using \scone{} to access the TEE hardware. \scone{} exposes a modified libc library and combines user-level threading and asynchronous syscalls~\cite{flexsc} to reduce the cost of syscall execution. While we limit the number of syscalls, leveraging  \scone{}'s exit-less approach allows us to optimize the initialization phase that vastly allocates host memory for the network stack and the KV store. To enable NIC's DMA operations and memory mappings to the hugepages (for message buffers and TX/RX queues) ($\S$~\ref{subsec:networkin}), we overwrite the \texttt{mmap()} syscall of \scone{} to bypass its shield layer and allow the allocation of (untrusted) host memory. 

For the cryptographic primitives, we build on OpenSSL~\cite{openssl}. Lastly, we build on a lease mechanism~\cite{t-lease} in \scone{} for auxiliary operations, e.g., failures detection and leader's election.




\myparagraph{\projecttitle{} key-value store} \projecttitle{} provides a lock-free, high-performant KV store based on a skip-list. We partition the keys from the values' space by placing the keys along with metadata (and a pointer to the value in host memory) inside the TEE's memory area, the {\em enclave}, and storing the values in the host memory. Our partitioned KVs reduces the number of calculations for integrity checks, compared to prior work~\cite{shieldstore} which implements (per-bucket) merkle trees and re-calculates the root on each update. Importantly, separating the (keys + metadata) and the values between the enclave and untrusted unlimited memory decreases the Enclave Page Cache (EPC) pressure~\cite{speicher-fast}. %Our lock-free data structure supports concurrent operations and it is well-suited
%for increased parallelism.

The developer might want to overwrite/implement the init\_store() function, which will keep an application's state and metadata in the trusted enclave. \projecttitle{} implements its hybrid skiplist based on folly library~\cite{folly}. The write() function updates the KV, while the get() function copies the value of the given key in the protected area. The function also verifies the value's integrity. We implement an allocator for host memory that is given as an initialization parameter to the KV store.

\if 0
\myparagraph{Implementation and API} 

\fi 
%\myparagraph{Data properties}
\projecttitle{}'s KV store design resolves Byzantine errors since the metadata (and the code that accesses them) reside in the enclave. That said, \projecttitle{} allows for local reads as nodes can verify the integrity of the stored values. Our partitioned scheme {\em seamlessly} strengthens the system's security properties further and can offer confidentiality by encrypting the values outside the TEE. \projecttitle{}-transformed protocols that further offer confidentiality outperform the BFT systems ($
\S$~\ref{sec:eval}).


%that offers parallel write and read operations, however the developer might wish to overwrite those functions. Both operations calculate the hash of the given data which is placed in the enclave memory. The hash is used to verify the integrity of the data stored in the host memory (if any).





%\vspace{-8pt}
%The application and the challenger first establish communication, and then the application asks the challenger to provide secrets.


\myparagraph{Attestation process} The attestation process is initialized by the \emph{challenger}, a remote process that can verify the authenticity of a specific TEE. The challenger executes the remote\_attestation() function to send an attestation request to the application---usually in the form of a nonce (a random number). The challenger and the application then pass through a Diffie-Hellman key exchange process~\cite{10.1145/359460.359473}. The application generates an ephemeral public key which is used by the challenger later to provision any secrets.

When the TEE receives the nonce, it calls the attest() and generates a \emph{measurement} ($\mu$) of its state and loaded code. Following this, the TEE calls into the generate\_quote$(\mu, k_{pub})$ to sign $\mu$ (quote) with the $key_{hw}$ which is fetched from the TEE's h/w. The TEE signs and encrypts the quote quote$_{\sigma_{k_{pub}}}$ over the challenger's public key $k_{pub}$, which is then sent back to the challenger. Upon successful verifications of the quote$_{\sigma_{k_{pub}}}$, the challenger shares secrets and configurations.

To offer low-latency attestations within the same datacenter that \projecttitle{} runs, we build a Configuration and Attestation service (CAS). The Protocol Designer (PD) deploys the CAS inside a TEE and attests it through the hardware vendor's attestation service---e.g., Intel Attestation Service (IAS~\cite{ias}). Once the CAS is attested, it is trusted, and the PB can upload secrets and configurations. 

The challenger asserts upon a failed verification and denies sharing any secret or configuration data. Otherwise, it distributes all necessary shared secrets.

\if 0 
\myparagraph{Implementation details} 
%\projecttitle{} builds on top of a Configuration and Attestation Service (CAS) that is shipped with \scone{} and ultimately relies on hardware-based TEEs for secure secrets and configuration management. 
A trusted entity, i.e., the developer, must deploy the  Configuration and Attestation Service  CAS inside a TEE in a node that can also be part of the membership. Afterward, they need to attest the CAS through the TEE's attestation service---in our case, Intel Attestation Service (IAS~\cite{ias}). Once the CAS is attested, it can replace the TEE's attestation service. The trusted entity needs to spawn further Local Attestation Services (LASes) that perform local (intra-platform) attestation of processes and offer low-latency attestation of the new nodes. Before the CFT protocol commences, the CAS must also attest all LASes. 

When a \projecttitle{} process is loaded in the TEE (before the control passes to the application code), the LAS instructs a local attestation. The attestee generates a quote of the enclave (measurement). The LAS forwards the enclave's measurement to the CAS, which replies with a success or failure message indicating the authenticity of the process. After a successful attestation, the CAS stores the node's IP and provides the trusted process with secrets and configurations.

\myparagraph{API}
\projecttitle{} provides an attestation API to developers. Particularly, we provide the attest function that takes as arguments the IP of a trusted third-party service, CAS's IP for \projecttitle{}, and a generated enclave measurement of the code. Then, this service verifies that both the enclave signer and measurement are in the expected state and replies accordingly.
\fi 
%\dimitra{fix this}
%\myparagraph{Attestation and secrets management} Attestation is the process of demonstrating that the correct software is securely running within a TEE-enclave on an enabled platform. Secrets (e.g., certificates, encryption keys, etc.) and configurations (membership IPs and information) should only be provided to a replica after its successful attestation. Once an enclave is initialized and before the control is relinquished to the application's code, the attestation process is launched to verify the integrity and authenticity of the included code and data. Essentially, \projecttitle{} needs to (1.) offer low-latency attestation of the joiner nodes and (2.) securely provide the trusted enclave applications with secrets and configuration data (usually over the network).

%\projecttitle{} leverages TEEs to implement \texttt{RemoteAttestation}$()$, \texttt{Attest}$()$ and \texttt{GenerateQuote}$()$ primitives that allow a third party to attest an enclave. Next we describe the workflow of the attestation and the properties of those primitives.

%The attestation process is initialized by the challenger (or verifier)---a remote process (typically provided by the hardware vendor) that can prove the authenticity of a specific enclave. The application and the challenger first establish communication and, then, the application asks the challenger to provision secrets. The challenger executes the \texttt{RemoteAttestation}$()$ function where it first replies with an attestation request to the application---usually in the form a nonce (a random number
%generated for this one occasion). The challenger and the application also pass through a Diffie-Hellman key exchange (DHKE) process. The application generates an ephemeral public key which is used by the challenger later for provisioning secrets to the enclave.

%After receiving the nonce, the enclave calls into the \texttt{Attest}$()$ and generates a \emph{measurement} or \emph{report} that includes information about the enclave. The report needs to be sent to the challenger for verification. The enclave calls into \texttt{GenerateQuote$(\mu, k_{public})$} which sings the measurement $\mu$ over $key_{hw}$, where $key_{hw}$ is fetched from the TEE's hardware. Additionally, the function encrypts the quote over challenger's public key $k_{public}$. The encrypted quote can be sent and verified by the challenger with the corresponding verification key.

%The challenger asserts upon a failed verification and denies to share any secret or configuration data. Otherwise, it distributes all necessary shared secrets.

\if 0

\subsection{\projecttitle{} API}

\if 0
\subsection{\projecttitle{} client API}
\projecttitle{} exposes a simple \texttt{PUT}/\texttt{GET} API to clients. Both functions take as arguments the coordinator node's id, the view and the leader identifier (if any) that are known to the client. The API assigns a unique monotonically increased id to every request executed by the client. That is to help CFT distinguish already executed requests.
\fi

%\subsection{\projecttitle{} system-level API}
Table~\ref{tab:api} presents the core library that \projecttitle{} exposes to the developers. We implement our \projecttitle{} on top of \scone~\cite{arnautov2016scone} and \textsc{Palaemon}~\cite{palaemon} that use Intel SGX~\cite{intel-sgx} as the TEE and we extend eRPC~\cite{erpc} on top of DPDK~\cite{dpdk} for fast networking. Next we discuss the use-cases and the implementation details for each core function of \projecttitle{}.

\myparagraph{Attestation API} 

\myparagraph{Initialization API} Developers need to initialize the protocol by creating the communication endpoints between replicas. \projecttitle{} offers a \texttt{create\_rpc()} function that creates Remote Procedure Call (RPC) objects (rpc) bound to the NIC. Specifically this function takes the application context as an argument, i.e., node's NIC specification and port, remote IP and port, creates a communication endpoint and continuously tries to establish connection with the remote side. The function returns after the connection establishment. An rpc offers bidirectional communication between the two sides. Additionally, we need to register the request handler functions to the rpcs, i.e., pass a pointer function a the construction of the endpoint which states what will happen when a request of a specific type is received. The developer might to overwrite/implement the \texttt{init\_store()} function which will keep an application's state and metadata in the trusted enclave. By default \projecttitle{} comes with a thread-safe and lock-free hybrid skiplist based on~\cite{avocado, folly}. While implementing our use cases in $\$$~\ref{sec:eval}, we used two  \projecttitle{} skiplists for metadata and data accordingly.  %Lastly, we need to register the request handler functions to the \texttt{rpc}s, i.e., pass a pointer function a the construction of the endpoint which states what will happen when a request of a specific type is received.

\myparagraph{Network API} \projecttitle{} offers high performance RPCs by extending eRPC~\cite{erpc} in the context of TEEs. Specifically, we place the message buffers outside the trusted enclave to both overcome the limited enclave memory and enable DMA operations~\footnote{DMA mappings are prohibited in the trusted area of a TEE as this violates their security properties~\cite{intel-sgx}}. We design a \texttt{send()} operation is used to submit a message for transmission. The message buffer is allocated by our library in \texttt{Hugepage} memory area and is later copied to the transmission queue (TX). Further, we provide a \texttt{multicast()} operation which creates identical copies of a message for all the recipient group.    Upon a reception of a request, the program control passes to the registered request handler where the function \texttt{respond()} can submit a response or \texttt{ACK} to that request. Lastly, the function \texttt{poll()} needs to be called regularly to fetch and process and send the incoming responses or requests and send the queued responses and requests respectively. 

\myparagraph{KV Store API} 

\dimitra{
\myparagraph{Trusted Leases API} \projecttitle{} exposes an API for leases to guarantee linearizable local reads when the CFT protocol allows that. A thread on a node initializes a lease (\texttt{init\_lease()}) and afterwards can exec a function within the lease's ownership (\texttt{exec\_with\_lease()}). In case the lease has expired, the function is not executed. The protocol updates the expiration date of a current lease with the \texttt{renew\_lease()}. The lease granter node/service updates its lease table with the \texttt{grand\_or\_update\_lease()} function.
}
\fi

\section{System Evaluation}
\label{sec:evaluation}

Each project proposal was independently evaluated by two human domain experts with experience as head teaching assistant (TA) for undergraduate computer science classes and GPT-4o \citep{hurst2024gpt}. All raters used the same 23-item quality checklist with four subtasks: (1) 10-question quality checklist students had used to self-assess their own work, (2) 3 questions judging the quality of 3 skill descriptions written by students, (3) 9 questions judging the appropriateness of skill-career pairings ($3 \times 3$), and (4) an overall quality judgment question (``I would recommend a student include this project on their resume.”). We did not perform any prompt-engineering on GPT-4o beyond evaluating subtasks and each of the student's three written skill descriptions in a separate prompt to avoid holistic evaluations. We provide the evaluation rubric in Appendix \ref{app:expert_evaluation_rubric} and GPT-4o prompts in Appendix \ref{app:llm-prompts}.

This evaluation protocol captures two main signals of user readiness for PBL. First, by comparing users' self-assessments to expert human evaluations on the same 10-point \textbf{quality checklist} adapted from \cite{WPIRubric} and \cite{lawlor2012smart}, we could determine whether users can accurately report on the quality of their own work. 

Second, by asking human expert raters to classify the \textbf{quality of skill descriptions} (``Good" vs. ``Irrelevant", ``Vague", or ``Not Core Computer Science Skill") of three skills users want to develop by working on their project and the \textbf{appropriateness of how users matched those skills to predefined mentor profiles} (such as ``Data Scientist" or ``Software Developer"), career-specific tasks, and trending technologies sourced from the O*Net Online Database \citep{Onet2024:Online}, we could capture signal on whether users' are capable of identifying perspectives and types of industry knowledge relevant to their project ideas. Users were explicitly instructed to leverage internet resources for assistance if needed, so this metric was intended to measure general reasoning and searching ability rather than ability to recall specific definitions. However, misinterpreting instructions, inability to parse technical language used by O*Net Online Database \citep{Onet2024:Online}, and shoehorning skills to match one of a limited number of options could weaken the strength of this measurement. We sanity check the quality of our grading criteria by comparing scores from users with higher levels of computer science experience against users with less experience. We expect that students with computer science experience should be able to write higher quality project proposals.

Additionally, we asked the human experts raters and GPT-4o to answer a question meant to capture the overall quality of the project proposal (``I would recommend a student include this project on their resume. Yes/No").

We checked the motivational benefits of our project proposal writing activity with a post-activity experience survey.

% --- (START) OLD

% To evaluate the feasibility of using GPT-4o to detect potential issues in students’ project proposals, we compared its ratings against ratings provided by two human domain experts experienced in teaching computer sciences classes using four assessments: (1) a \textbf{skill classification task} to detect whether the skills written by students were good or bad (irrelevant, not computer science, overly vague), (2) a \textbf{skill-paring classification task} to assess the appropriateness of pairing each skill to the selected career, selected tasks, and selected hot technologies (3) \textbf{10-question quality checklist}, and (4) A \textbf{binary choice question to simulate mentorship} (``I would recommend a student include this project on their resume.”)

% We did not perform any prompt-engineering on GPT-4o beyond evaluating each of the student's three written skills in a separate prompt to avoid holistic evaluations. Prompts are given in the appendix.

\textbf{Experimental Procedure.} We recruited 40 participants online via Prolific. Our recruitment call was shown to crowd workers (i) 18 years or older; (ii) located in the USA; (iii) fluent in English; (iv) possess at least a high-school (HS) degree or equivalent; (v) using a desktop device (no mobile or tablets); and (vi) answered yes to ``Are you a student?" on Prolific (see recruitment call in Appendix \ref{app:prolific_recruitment}). 2 participants were filtered out for not completing the activity, and 2 were filtered for failing attention checks. Of the 36 remaining participants, 17 were 18-25 years old, 14 were 26-35 years old, and 5 were over 46 years old. 25 were male, 11 were female. 11 participants reported Computer Science as their field of work/study, and 19 claimed to have computer science experience, i.e, have built a website or interactive application before using ``Programming Language (e.g., Python, JavaScript)", ``Backend Technologies (e.g., Node.js, Django, Flask)", ``Cloud Platforms (e.g., Firebase, AWS)", and/or ``Testing and Deployment (e.g., unit testing libraries, Docker)". Interestingly, some of these users claimed experience in categories like Backend Technologies or Cloud Platforms while simultaneously reporting no experience with Programming Languages. This discrepancy suggests a potential need for more fine-grained and specific questions to accurately capture their technical experience. 

All participants encountered the same set of activities and questions in the same order, and could not go back to previous stages. Participants were told the activity would take 30-45 minutes, but we did not control for time in any phase. The median time to complete the activity was 40 minutes. The recruitment and study process was approved by CMU's Institutional Review Board (research study 2024\_00000405).

\subsection{Results}
As shown in Table \ref{tab:agreement}, human expert raters and GPT-4o showed promising agreement on the relative quality of proposals. As shown in Table~\ref{tab:positive_grades}, human expert raters and GPT-4o tended to give lower quality scores to users with less computer science experience.


\begin{table}[t]
\centering
\caption{Rater Agreement. We report agreement percentage and Cohen’s $\kappa$ value between the two human expert raters (TA1 and TA2) and GPT-4o with respect to the 4 grading subtasks. As an overall index of agreement, we compute kappa for all rater pairs then report the arithmetic mean of these estimates.
}
\begin{tabular}{lcccc}
\hline 
& TA1 / TA2 & TA1 / GPT-4o & TA2 / GPT-4o  & Avg. $\kappa$ \\
\hline
 Skill Quality Classification & $86.1\%$, $\kappa = 0.72$ & $84.3\%$, $\kappa = 0.68$ & $81.5\%$, $\kappa = 0.63$ & $0.68$ \\
 Skill Pairing Classification & $68.6\%$, $\kappa = 0.26$ & $74.2\%$, $\kappa = 0.46$ & $67.2\%$, $\kappa = 0.20$  & $0.29$ \\
 Quality Checklist& $87.2\%$, $\kappa = 0.49$ & $71.4\%$, $\kappa = 0.28$ & $74.7\%$, $\kappa = 0.28$ & $0.38$ \\
 Recommend for Resume & $80.6\%$, $\kappa = 0.50$ & $75.0\%$, $\kappa = 0.43$ & $77.8\%$, $\kappa = 0.45$ & $0.46$ \\
\hline
\end{tabular}
\label{tab:agreement}
\end{table}

% grade table figure

\begin{table}[t]
    \centering
    \caption{Mean Positive Grades for Experienced vs. Novice Students. Each cell represents the mean proportion of student responses graded positively per subtask and rater, grouped by whether students reported having previous computer science experience. Subtasks averaged over 3, 9, 10, and 1 rater items respectively}
    \begin{tabular}{rrcccc}
    \hline
     Task & Experience & TA1 & TA2 & GPT-4o & Self-Rating \\
    \hline
     Skill Classification & Novice & 35.3\% & 39.2\% & 31.4\% & - \\
     & Experienced & 70.2\% & 68.4\% & 68.4\% & - \\
     \hline
     Skill Pairing Classification & Novice & 60.8\% & 86.9\% & 57.5\% & - \\
     & Experienced & 53.7\% & 92.4\% & 64.9\% & -  \\
     \hline
     Quality Checklist & Novice & 83.5\% & 82.4\% & 58.2\% & 85.9\% \\
     & Experienced & 84.7\% & 90.0\% & 71.1\% & 95.8\% \\
     \hline
     Recommend for Resume & Novice & 58.8\% & 70.6\% & 58.8\% & - \\
     & Experienced & 78.9\% & 84.2\% & 73.7\% & - \\
    \hline
    \end{tabular}
    \label{tab:positive_grades}
\end{table}


\textbf{Skill Classification.} Human raters and GPT-4o classifed ~50\% of skills as irrelevant, vague, or not relevant to computer science ($>80\%$, $\kappa>0.6$). Consistent across all three graders, students with less computer science knowledge had an average of 1 out of 3 skills accepted, while students with more computer science knowledge had an average of 2 of 3 skills accepted.


\begin{figure}[t]
\floatconts
  {fig:mean-score-spearman-corr}
  {\caption{Spearman correlation of students' total score on the 10-point quality checklist. Although GPT-4o's scores for the quality checklist task are lower than both teaching assistants and student self-evaluations, GPT-4o's scores preserve the rank order of teaching assistants' scores better than students' self-evaluation scores do.}}
  {\includegraphics[width=0.5\textwidth]{figures/spearman_quality_checklist}}
\end{figure}


\textbf{Appropriateness of Skill Pairing Classification.} The low quality of skills written down by the students might have made the subsequent activity to pair skills with careers, tasks, and technologies more challenging to grade. Human rater agreement was minimal ($68\%, \kappa=0.25$). GPT-4o showed weak agreement with TA 1 ($74\%, \kappa=0.46$) and minimal agreement with TA 2 ($67\%, \kappa=0.20$).

\textbf{Quality Checklist.}  GPT-4o’s scores tended to be lower than either human rater's scores, leading to lower agreement between GPT-4o and humans ($71-75\%, \kappa=0.28$). However, as seen in Figure \ref{fig:mean-score-spearman-corr}, GPT-4o roughly maintained the rank of student project proposal quality relative to each other, shown by GPT-4o’s Spearman correlations with TA 1’s (Spearman=0.70) and TA 2’s (Spearman=0.53). In contrast, students’ self-evaluations had a much weaker correlation with either human expert's ratings (Spearman=0.16, Spearman=0.38).

\textbf{Experience Survey.} As shown in Figure \ref{fig:experience}, the large majority of users enjoyed the process of writing project proposals, with $88.8\%$ of users wanting to use our system in the future to choose skills and technologies to learn more about, and $91.6\%$ of users wanting to use our system in the future to design project ideas that motivate them to learn more.

\begin{figure}[t]
\floatconts
  {fig:experience}
  {\caption{\textbf{Experience Survey, Response Counts.} The majority of users reported high level of excitement, motivation, and wanting to use the activity in the future to choose skills and technologies to learn more about.}}
  {\includegraphics[width=0.9\textwidth]{figures/experience_survey}}
\end{figure}

\section{Discussion of Assumptions}\label{sec:discussion}
In this paper, we have made several assumptions for the sake of clarity and simplicity. In this section, we discuss the rationale behind these assumptions, the extent to which these assumptions hold in practice, and the consequences for our protocol when these assumptions hold.

\subsection{Assumptions on the Demand}

There are two simplifying assumptions we make about the demand. First, we assume the demand at any time is relatively small compared to the channel capacities. Second, we take the demand to be constant over time. We elaborate upon both these points below.

\paragraph{Small demands} The assumption that demands are small relative to channel capacities is made precise in \eqref{eq:large_capacity_assumption}. This assumption simplifies two major aspects of our protocol. First, it largely removes congestion from consideration. In \eqref{eq:primal_problem}, there is no constraint ensuring that total flow in both directions stays below capacity--this is always met. Consequently, there is no Lagrange multiplier for congestion and no congestion pricing; only imbalance penalties apply. In contrast, protocols in \cite{sivaraman2020high, varma2021throughput, wang2024fence} include congestion fees due to explicit congestion constraints. Second, the bound \eqref{eq:large_capacity_assumption} ensures that as long as channels remain balanced, the network can always meet demand, no matter how the demand is routed. Since channels can rebalance when necessary, they never drop transactions. This allows prices and flows to adjust as per the equations in \eqref{eq:algorithm}, which makes it easier to prove the protocol's convergence guarantees. This also preserves the key property that a channel's price remains proportional to net money flow through it.

In practice, payment channel networks are used most often for micro-payments, for which on-chain transactions are prohibitively expensive; large transactions typically take place directly on the blockchain. For example, according to \cite{river2023lightning}, the average channel capacity is roughly $0.1$ BTC ($5,000$ BTC distributed over $50,000$ channels), while the average transaction amount is less than $0.0004$ BTC ($44.7k$ satoshis). Thus, the small demand assumption is not too unrealistic. Additionally, the occasional large transaction can be treated as a sequence of smaller transactions by breaking it into packets and executing each packet serially (as done by \cite{sivaraman2020high}).
Lastly, a good path discovery process that favors large capacity channels over small capacity ones can help ensure that the bound in \eqref{eq:large_capacity_assumption} holds.

\paragraph{Constant demands} 
In this work, we assume that any transacting pair of nodes have a steady transaction demand between them (see Section \ref{sec:transaction_requests}). Making this assumption is necessary to obtain the kind of guarantees that we have presented in this paper. Unless the demand is steady, it is unreasonable to expect that the flows converge to a steady value. Weaker assumptions on the demand lead to weaker guarantees. For example, with the more general setting of stochastic, but i.i.d. demand between any two nodes, \cite{varma2021throughput} shows that the channel queue lengths are bounded in expectation. If the demand can be arbitrary, then it is very hard to get any meaningful performance guarantees; \cite{wang2024fence} shows that even for a single bidirectional channel, the competitive ratio is infinite. Indeed, because a PCN is a decentralized system and decisions must be made based on local information alone, it is difficult for the network to find the optimal detailed balance flow at every time step with a time-varying demand.  With a steady demand, the network can discover the optimal flows in a reasonably short time, as our work shows.

We view the constant demand assumption as an approximation for a more general demand process that could be piece-wise constant, stochastic, or both (see simulations in Figure \ref{fig:five_nodes_variable_demand}).
We believe it should be possible to merge ideas from our work and \cite{varma2021throughput} to provide guarantees in a setting with random demands with arbitrary means. We leave this for future work. In addition, our work suggests that a reasonable method of handling stochastic demands is to queue the transaction requests \textit{at the source node} itself. This queuing action should be viewed in conjunction with flow-control. Indeed, a temporarily high unidirectional demand would raise prices for the sender, incentivizing the sender to stop sending the transactions. If the sender queues the transactions, they can send them later when prices drop. This form of queuing does not require any overhaul of the basic PCN infrastructure and is therefore simpler to implement than per-channel queues as suggested by \cite{sivaraman2020high} and \cite{varma2021throughput}.

\subsection{The Incentive of Channels}
The actions of the channels as prescribed by the DEBT control protocol can be summarized as follows. Channels adjust their prices in proportion to the net flow through them. They rebalance themselves whenever necessary and execute any transaction request that has been made of them. We discuss both these aspects below.

\paragraph{On Prices}
In this work, the exclusive role of channel prices is to ensure that the flows through each channel remains balanced. In practice, it would be important to include other components in a channel's price/fee as well: a congestion price  and an incentive price. The congestion price, as suggested by \cite{varma2021throughput}, would depend on the total flow of transactions through the channel, and would incentivize nodes to balance the load over different paths. The incentive price, which is commonly used in practice \cite{river2023lightning}, is necessary to provide channels with an incentive to serve as an intermediary for different channels. In practice, we expect both these components to be smaller than the imbalance price. Consequently, we expect the behavior of our protocol to be similar to our theoretical results even with these additional prices.

A key aspect of our protocol is that channel fees are allowed to be negative. Although the original Lightning network whitepaper \cite{poon2016bitcoin} suggests that negative channel prices may be a good solution to promote rebalancing, the idea of negative prices in not very popular in the literature. To our knowledge, the only prior work with this feature is \cite{varma2021throughput}. Indeed, in papers such as \cite{van2021merchant} and \cite{wang2024fence}, the price function is explicitly modified such that the channel price is never negative. The results of our paper show the benefits of negative prices. For one, in steady state, equal flows in both directions ensure that a channel doesn't loose any money (the other price components mentioned above ensure that the channel will only gain money). More importantly, negative prices are important to ensure that the protocol selectively stifles acyclic flows while allowing circulations to flow. Indeed, in the example of Section \ref{sec:flow_control_example}, the flows between nodes $A$ and $C$ are left on only because the large positive price over one channel is canceled by the corresponding negative price over the other channel, leading to a net zero price.

Lastly, observe that in the DEBT control protocol, the price charged by a channel does not depend on its capacity. This is a natural consequence of the price being the Lagrange multiplier for the net-zero flow constraint, which also does not depend on the channel capacity. In contrast, in many other works, the imbalance price is normalized by the channel capacity \cite{ren2018optimal, lin2020funds, wang2024fence}; this is shown to work well in practice. The rationale for such a price structure is explained well in \cite{wang2024fence}, where this fee is derived with the aim of always maintaining some balance (liquidity) at each end of every channel. This is a reasonable aim if a channel is to never rebalance itself; the experiments of the aforementioned papers are conducted in such a regime. In this work, however, we allow the channels to rebalance themselves a few times in order to settle on a detailed balance flow. This is because our focus is on the long-term steady state performance of the protocol. This difference in perspective also shows up in how the price depends on the channel imbalance. \cite{lin2020funds} and \cite{wang2024fence} advocate for strictly convex prices whereas this work and \cite{varma2021throughput} propose linear prices.

\paragraph{On Rebalancing} 
Recall that the DEBT control protocol ensures that the flows in the network converge to a detailed balance flow, which can be sustained perpetually without any rebalancing. However, during the transient phase (before convergence), channels may have to perform on-chain rebalancing a few times. Since rebalancing is an expensive operation, it is worthwhile discussing methods by which channels can reduce the extent of rebalancing. One option for the channels to reduce the extent of rebalancing is to increase their capacity; however, this comes at the cost of locking in more capital. Each channel can decide for itself the optimum amount of capital to lock in. Another option, which we discuss in Section \ref{sec:five_node}, is for channels to increase the rate $\gamma$ at which they adjust prices. 

Ultimately, whether or not it is beneficial for a channel to rebalance depends on the time-horizon under consideration. Our protocol is based on the assumption that the demand remains steady for a long period of time. If this is indeed the case, it would be worthwhile for a channel to rebalance itself as it can make up this cost through the incentive fees gained from the flow of transactions through it in steady state. If a channel chooses not to rebalance itself, however, there is a risk of being trapped in a deadlock, which is suboptimal for not only the nodes but also the channel.

\section{Conclusion}
This work presents DEBT control: a protocol for payment channel networks that uses source routing and flow control based on channel prices. The protocol is derived by posing a network utility maximization problem and analyzing its dual minimization. It is shown that under steady demands, the protocol guides the network to an optimal, sustainable point. Simulations show its robustness to demand variations. The work demonstrates that simple protocols with strong theoretical guarantees are possible for PCNs and we hope it inspires further theoretical research in this direction.
%\section{Conclusion}
In this work, we propose a simple yet effective approach, called SMILE, for graph few-shot learning with fewer tasks. Specifically, we introduce a novel dual-level mixup strategy, including within-task and across-task mixup, for enriching the diversity of nodes within each task and the diversity of tasks. Also, we incorporate the degree-based prior information to learn expressive node embeddings. Theoretically, we prove that SMILE effectively enhances the model's generalization performance. Empirically, we conduct extensive experiments on multiple benchmarks and the results suggest that SMILE significantly outperforms other baselines, including both in-domain and cross-domain few-shot settings.

% Add those for the camera ready
%\acks{Acknowledgements go here.}

\bibliography{bibliography}

\subsection{Lloyd-Max Algorithm}
\label{subsec:Lloyd-Max}
For a given quantization bitwidth $B$ and an operand $\bm{X}$, the Lloyd-Max algorithm finds $2^B$ quantization levels $\{\hat{x}_i\}_{i=1}^{2^B}$ such that quantizing $\bm{X}$ by rounding each scalar in $\bm{X}$ to the nearest quantization level minimizes the quantization MSE. 

The algorithm starts with an initial guess of quantization levels and then iteratively computes quantization thresholds $\{\tau_i\}_{i=1}^{2^B-1}$ and updates quantization levels $\{\hat{x}_i\}_{i=1}^{2^B}$. Specifically, at iteration $n$, thresholds are set to the midpoints of the previous iteration's levels:
\begin{align*}
    \tau_i^{(n)}=\frac{\hat{x}_i^{(n-1)}+\hat{x}_{i+1}^{(n-1)}}2 \text{ for } i=1\ldots 2^B-1
\end{align*}
Subsequently, the quantization levels are re-computed as conditional means of the data regions defined by the new thresholds:
\begin{align*}
    \hat{x}_i^{(n)}=\mathbb{E}\left[ \bm{X} \big| \bm{X}\in [\tau_{i-1}^{(n)},\tau_i^{(n)}] \right] \text{ for } i=1\ldots 2^B
\end{align*}
where to satisfy boundary conditions we have $\tau_0=-\infty$ and $\tau_{2^B}=\infty$. The algorithm iterates the above steps until convergence.

Figure \ref{fig:lm_quant} compares the quantization levels of a $7$-bit floating point (E3M3) quantizer (left) to a $7$-bit Lloyd-Max quantizer (right) when quantizing a layer of weights from the GPT3-126M model at a per-tensor granularity. As shown, the Lloyd-Max quantizer achieves substantially lower quantization MSE. Further, Table \ref{tab:FP7_vs_LM7} shows the superior perplexity achieved by Lloyd-Max quantizers for bitwidths of $7$, $6$ and $5$. The difference between the quantizers is clear at 5 bits, where per-tensor FP quantization incurs a drastic and unacceptable increase in perplexity, while Lloyd-Max quantization incurs a much smaller increase. Nevertheless, we note that even the optimal Lloyd-Max quantizer incurs a notable ($\sim 1.5$) increase in perplexity due to the coarse granularity of quantization. 

\begin{figure}[h]
  \centering
  \includegraphics[width=0.7\linewidth]{sections/figures/LM7_FP7.pdf}
  \caption{\small Quantization levels and the corresponding quantization MSE of Floating Point (left) vs Lloyd-Max (right) Quantizers for a layer of weights in the GPT3-126M model.}
  \label{fig:lm_quant}
\end{figure}

\begin{table}[h]\scriptsize
\begin{center}
\caption{\label{tab:FP7_vs_LM7} \small Comparing perplexity (lower is better) achieved by floating point quantizers and Lloyd-Max quantizers on a GPT3-126M model for the Wikitext-103 dataset.}
\begin{tabular}{c|cc|c}
\hline
 \multirow{2}{*}{\textbf{Bitwidth}} & \multicolumn{2}{|c|}{\textbf{Floating-Point Quantizer}} & \textbf{Lloyd-Max Quantizer} \\
 & Best Format & Wikitext-103 Perplexity & Wikitext-103 Perplexity \\
\hline
7 & E3M3 & 18.32 & 18.27 \\
6 & E3M2 & 19.07 & 18.51 \\
5 & E4M0 & 43.89 & 19.71 \\
\hline
\end{tabular}
\end{center}
\end{table}

\subsection{Proof of Local Optimality of LO-BCQ}
\label{subsec:lobcq_opt_proof}
For a given block $\bm{b}_j$, the quantization MSE during LO-BCQ can be empirically evaluated as $\frac{1}{L_b}\lVert \bm{b}_j- \bm{\hat{b}}_j\rVert^2_2$ where $\bm{\hat{b}}_j$ is computed from equation (\ref{eq:clustered_quantization_definition}) as $C_{f(\bm{b}_j)}(\bm{b}_j)$. Further, for a given block cluster $\mathcal{B}_i$, we compute the quantization MSE as $\frac{1}{|\mathcal{B}_{i}|}\sum_{\bm{b} \in \mathcal{B}_{i}} \frac{1}{L_b}\lVert \bm{b}- C_i^{(n)}(\bm{b})\rVert^2_2$. Therefore, at the end of iteration $n$, we evaluate the overall quantization MSE $J^{(n)}$ for a given operand $\bm{X}$ composed of $N_c$ block clusters as:
\begin{align*}
    \label{eq:mse_iter_n}
    J^{(n)} = \frac{1}{N_c} \sum_{i=1}^{N_c} \frac{1}{|\mathcal{B}_{i}^{(n)}|}\sum_{\bm{v} \in \mathcal{B}_{i}^{(n)}} \frac{1}{L_b}\lVert \bm{b}- B_i^{(n)}(\bm{b})\rVert^2_2
\end{align*}

At the end of iteration $n$, the codebooks are updated from $\mathcal{C}^{(n-1)}$ to $\mathcal{C}^{(n)}$. However, the mapping of a given vector $\bm{b}_j$ to quantizers $\mathcal{C}^{(n)}$ remains as  $f^{(n)}(\bm{b}_j)$. At the next iteration, during the vector clustering step, $f^{(n+1)}(\bm{b}_j)$ finds new mapping of $\bm{b}_j$ to updated codebooks $\mathcal{C}^{(n)}$ such that the quantization MSE over the candidate codebooks is minimized. Therefore, we obtain the following result for $\bm{b}_j$:
\begin{align*}
\frac{1}{L_b}\lVert \bm{b}_j - C_{f^{(n+1)}(\bm{b}_j)}^{(n)}(\bm{b}_j)\rVert^2_2 \le \frac{1}{L_b}\lVert \bm{b}_j - C_{f^{(n)}(\bm{b}_j)}^{(n)}(\bm{b}_j)\rVert^2_2
\end{align*}

That is, quantizing $\bm{b}_j$ at the end of the block clustering step of iteration $n+1$ results in lower quantization MSE compared to quantizing at the end of iteration $n$. Since this is true for all $\bm{b} \in \bm{X}$, we assert the following:
\begin{equation}
\begin{split}
\label{eq:mse_ineq_1}
    \tilde{J}^{(n+1)} &= \frac{1}{N_c} \sum_{i=1}^{N_c} \frac{1}{|\mathcal{B}_{i}^{(n+1)}|}\sum_{\bm{b} \in \mathcal{B}_{i}^{(n+1)}} \frac{1}{L_b}\lVert \bm{b} - C_i^{(n)}(b)\rVert^2_2 \le J^{(n)}
\end{split}
\end{equation}
where $\tilde{J}^{(n+1)}$ is the the quantization MSE after the vector clustering step at iteration $n+1$.

Next, during the codebook update step (\ref{eq:quantizers_update}) at iteration $n+1$, the per-cluster codebooks $\mathcal{C}^{(n)}$ are updated to $\mathcal{C}^{(n+1)}$ by invoking the Lloyd-Max algorithm \citep{Lloyd}. We know that for any given value distribution, the Lloyd-Max algorithm minimizes the quantization MSE. Therefore, for a given vector cluster $\mathcal{B}_i$ we obtain the following result:

\begin{equation}
    \frac{1}{|\mathcal{B}_{i}^{(n+1)}|}\sum_{\bm{b} \in \mathcal{B}_{i}^{(n+1)}} \frac{1}{L_b}\lVert \bm{b}- C_i^{(n+1)}(\bm{b})\rVert^2_2 \le \frac{1}{|\mathcal{B}_{i}^{(n+1)}|}\sum_{\bm{b} \in \mathcal{B}_{i}^{(n+1)}} \frac{1}{L_b}\lVert \bm{b}- C_i^{(n)}(\bm{b})\rVert^2_2
\end{equation}

The above equation states that quantizing the given block cluster $\mathcal{B}_i$ after updating the associated codebook from $C_i^{(n)}$ to $C_i^{(n+1)}$ results in lower quantization MSE. Since this is true for all the block clusters, we derive the following result: 
\begin{equation}
\begin{split}
\label{eq:mse_ineq_2}
     J^{(n+1)} &= \frac{1}{N_c} \sum_{i=1}^{N_c} \frac{1}{|\mathcal{B}_{i}^{(n+1)}|}\sum_{\bm{b} \in \mathcal{B}_{i}^{(n+1)}} \frac{1}{L_b}\lVert \bm{b}- C_i^{(n+1)}(\bm{b})\rVert^2_2  \le \tilde{J}^{(n+1)}   
\end{split}
\end{equation}

Following (\ref{eq:mse_ineq_1}) and (\ref{eq:mse_ineq_2}), we find that the quantization MSE is non-increasing for each iteration, that is, $J^{(1)} \ge J^{(2)} \ge J^{(3)} \ge \ldots \ge J^{(M)}$ where $M$ is the maximum number of iterations. 
%Therefore, we can say that if the algorithm converges, then it must be that it has converged to a local minimum. 
\hfill $\blacksquare$


\begin{figure}
    \begin{center}
    \includegraphics[width=0.5\textwidth]{sections//figures/mse_vs_iter.pdf}
    \end{center}
    \caption{\small NMSE vs iterations during LO-BCQ compared to other block quantization proposals}
    \label{fig:nmse_vs_iter}
\end{figure}

Figure \ref{fig:nmse_vs_iter} shows the empirical convergence of LO-BCQ across several block lengths and number of codebooks. Also, the MSE achieved by LO-BCQ is compared to baselines such as MXFP and VSQ. As shown, LO-BCQ converges to a lower MSE than the baselines. Further, we achieve better convergence for larger number of codebooks ($N_c$) and for a smaller block length ($L_b$), both of which increase the bitwidth of BCQ (see Eq \ref{eq:bitwidth_bcq}).


\subsection{Additional Accuracy Results}
%Table \ref{tab:lobcq_config} lists the various LOBCQ configurations and their corresponding bitwidths.
\begin{table}
\setlength{\tabcolsep}{4.75pt}
\begin{center}
\caption{\label{tab:lobcq_config} Various LO-BCQ configurations and their bitwidths.}
\begin{tabular}{|c||c|c|c|c||c|c||c|} 
\hline
 & \multicolumn{4}{|c||}{$L_b=8$} & \multicolumn{2}{|c||}{$L_b=4$} & $L_b=2$ \\
 \hline
 \backslashbox{$L_A$\kern-1em}{\kern-1em$N_c$} & 2 & 4 & 8 & 16 & 2 & 4 & 2 \\
 \hline
 64 & 4.25 & 4.375 & 4.5 & 4.625 & 4.375 & 4.625 & 4.625\\
 \hline
 32 & 4.375 & 4.5 & 4.625& 4.75 & 4.5 & 4.75 & 4.75 \\
 \hline
 16 & 4.625 & 4.75& 4.875 & 5 & 4.75 & 5 & 5 \\
 \hline
\end{tabular}
\end{center}
\end{table}

%\subsection{Perplexity achieved by various LO-BCQ configurations on Wikitext-103 dataset}

\begin{table} \centering
\begin{tabular}{|c||c|c|c|c||c|c||c|} 
\hline
 $L_b \rightarrow$& \multicolumn{4}{c||}{8} & \multicolumn{2}{c||}{4} & 2\\
 \hline
 \backslashbox{$L_A$\kern-1em}{\kern-1em$N_c$} & 2 & 4 & 8 & 16 & 2 & 4 & 2  \\
 %$N_c \rightarrow$ & 2 & 4 & 8 & 16 & 2 & 4 & 2 \\
 \hline
 \hline
 \multicolumn{8}{c}{GPT3-1.3B (FP32 PPL = 9.98)} \\ 
 \hline
 \hline
 64 & 10.40 & 10.23 & 10.17 & 10.15 &  10.28 & 10.18 & 10.19 \\
 \hline
 32 & 10.25 & 10.20 & 10.15 & 10.12 &  10.23 & 10.17 & 10.17 \\
 \hline
 16 & 10.22 & 10.16 & 10.10 & 10.09 &  10.21 & 10.14 & 10.16 \\
 \hline
  \hline
 \multicolumn{8}{c}{GPT3-8B (FP32 PPL = 7.38)} \\ 
 \hline
 \hline
 64 & 7.61 & 7.52 & 7.48 &  7.47 &  7.55 &  7.49 & 7.50 \\
 \hline
 32 & 7.52 & 7.50 & 7.46 &  7.45 &  7.52 &  7.48 & 7.48  \\
 \hline
 16 & 7.51 & 7.48 & 7.44 &  7.44 &  7.51 &  7.49 & 7.47  \\
 \hline
\end{tabular}
\caption{\label{tab:ppl_gpt3_abalation} Wikitext-103 perplexity across GPT3-1.3B and 8B models.}
\end{table}

\begin{table} \centering
\begin{tabular}{|c||c|c|c|c||} 
\hline
 $L_b \rightarrow$& \multicolumn{4}{c||}{8}\\
 \hline
 \backslashbox{$L_A$\kern-1em}{\kern-1em$N_c$} & 2 & 4 & 8 & 16 \\
 %$N_c \rightarrow$ & 2 & 4 & 8 & 16 & 2 & 4 & 2 \\
 \hline
 \hline
 \multicolumn{5}{|c|}{Llama2-7B (FP32 PPL = 5.06)} \\ 
 \hline
 \hline
 64 & 5.31 & 5.26 & 5.19 & 5.18  \\
 \hline
 32 & 5.23 & 5.25 & 5.18 & 5.15  \\
 \hline
 16 & 5.23 & 5.19 & 5.16 & 5.14  \\
 \hline
 \multicolumn{5}{|c|}{Nemotron4-15B (FP32 PPL = 5.87)} \\ 
 \hline
 \hline
 64  & 6.3 & 6.20 & 6.13 & 6.08  \\
 \hline
 32  & 6.24 & 6.12 & 6.07 & 6.03  \\
 \hline
 16  & 6.12 & 6.14 & 6.04 & 6.02  \\
 \hline
 \multicolumn{5}{|c|}{Nemotron4-340B (FP32 PPL = 3.48)} \\ 
 \hline
 \hline
 64 & 3.67 & 3.62 & 3.60 & 3.59 \\
 \hline
 32 & 3.63 & 3.61 & 3.59 & 3.56 \\
 \hline
 16 & 3.61 & 3.58 & 3.57 & 3.55 \\
 \hline
\end{tabular}
\caption{\label{tab:ppl_llama7B_nemo15B} Wikitext-103 perplexity compared to FP32 baseline in Llama2-7B and Nemotron4-15B, 340B models}
\end{table}

%\subsection{Perplexity achieved by various LO-BCQ configurations on MMLU dataset}


\begin{table} \centering
\begin{tabular}{|c||c|c|c|c||c|c|c|c|} 
\hline
 $L_b \rightarrow$& \multicolumn{4}{c||}{8} & \multicolumn{4}{c||}{8}\\
 \hline
 \backslashbox{$L_A$\kern-1em}{\kern-1em$N_c$} & 2 & 4 & 8 & 16 & 2 & 4 & 8 & 16  \\
 %$N_c \rightarrow$ & 2 & 4 & 8 & 16 & 2 & 4 & 2 \\
 \hline
 \hline
 \multicolumn{5}{|c|}{Llama2-7B (FP32 Accuracy = 45.8\%)} & \multicolumn{4}{|c|}{Llama2-70B (FP32 Accuracy = 69.12\%)} \\ 
 \hline
 \hline
 64 & 43.9 & 43.4 & 43.9 & 44.9 & 68.07 & 68.27 & 68.17 & 68.75 \\
 \hline
 32 & 44.5 & 43.8 & 44.9 & 44.5 & 68.37 & 68.51 & 68.35 & 68.27  \\
 \hline
 16 & 43.9 & 42.7 & 44.9 & 45 & 68.12 & 68.77 & 68.31 & 68.59  \\
 \hline
 \hline
 \multicolumn{5}{|c|}{GPT3-22B (FP32 Accuracy = 38.75\%)} & \multicolumn{4}{|c|}{Nemotron4-15B (FP32 Accuracy = 64.3\%)} \\ 
 \hline
 \hline
 64 & 36.71 & 38.85 & 38.13 & 38.92 & 63.17 & 62.36 & 63.72 & 64.09 \\
 \hline
 32 & 37.95 & 38.69 & 39.45 & 38.34 & 64.05 & 62.30 & 63.8 & 64.33  \\
 \hline
 16 & 38.88 & 38.80 & 38.31 & 38.92 & 63.22 & 63.51 & 63.93 & 64.43  \\
 \hline
\end{tabular}
\caption{\label{tab:mmlu_abalation} Accuracy on MMLU dataset across GPT3-22B, Llama2-7B, 70B and Nemotron4-15B models.}
\end{table}


%\subsection{Perplexity achieved by various LO-BCQ configurations on LM evaluation harness}

\begin{table} \centering
\begin{tabular}{|c||c|c|c|c||c|c|c|c|} 
\hline
 $L_b \rightarrow$& \multicolumn{4}{c||}{8} & \multicolumn{4}{c||}{8}\\
 \hline
 \backslashbox{$L_A$\kern-1em}{\kern-1em$N_c$} & 2 & 4 & 8 & 16 & 2 & 4 & 8 & 16  \\
 %$N_c \rightarrow$ & 2 & 4 & 8 & 16 & 2 & 4 & 2 \\
 \hline
 \hline
 \multicolumn{5}{|c|}{Race (FP32 Accuracy = 37.51\%)} & \multicolumn{4}{|c|}{Boolq (FP32 Accuracy = 64.62\%)} \\ 
 \hline
 \hline
 64 & 36.94 & 37.13 & 36.27 & 37.13 & 63.73 & 62.26 & 63.49 & 63.36 \\
 \hline
 32 & 37.03 & 36.36 & 36.08 & 37.03 & 62.54 & 63.51 & 63.49 & 63.55  \\
 \hline
 16 & 37.03 & 37.03 & 36.46 & 37.03 & 61.1 & 63.79 & 63.58 & 63.33  \\
 \hline
 \hline
 \multicolumn{5}{|c|}{Winogrande (FP32 Accuracy = 58.01\%)} & \multicolumn{4}{|c|}{Piqa (FP32 Accuracy = 74.21\%)} \\ 
 \hline
 \hline
 64 & 58.17 & 57.22 & 57.85 & 58.33 & 73.01 & 73.07 & 73.07 & 72.80 \\
 \hline
 32 & 59.12 & 58.09 & 57.85 & 58.41 & 73.01 & 73.94 & 72.74 & 73.18  \\
 \hline
 16 & 57.93 & 58.88 & 57.93 & 58.56 & 73.94 & 72.80 & 73.01 & 73.94  \\
 \hline
\end{tabular}
\caption{\label{tab:mmlu_abalation} Accuracy on LM evaluation harness tasks on GPT3-1.3B model.}
\end{table}

\begin{table} \centering
\begin{tabular}{|c||c|c|c|c||c|c|c|c|} 
\hline
 $L_b \rightarrow$& \multicolumn{4}{c||}{8} & \multicolumn{4}{c||}{8}\\
 \hline
 \backslashbox{$L_A$\kern-1em}{\kern-1em$N_c$} & 2 & 4 & 8 & 16 & 2 & 4 & 8 & 16  \\
 %$N_c \rightarrow$ & 2 & 4 & 8 & 16 & 2 & 4 & 2 \\
 \hline
 \hline
 \multicolumn{5}{|c|}{Race (FP32 Accuracy = 41.34\%)} & \multicolumn{4}{|c|}{Boolq (FP32 Accuracy = 68.32\%)} \\ 
 \hline
 \hline
 64 & 40.48 & 40.10 & 39.43 & 39.90 & 69.20 & 68.41 & 69.45 & 68.56 \\
 \hline
 32 & 39.52 & 39.52 & 40.77 & 39.62 & 68.32 & 67.43 & 68.17 & 69.30  \\
 \hline
 16 & 39.81 & 39.71 & 39.90 & 40.38 & 68.10 & 66.33 & 69.51 & 69.42  \\
 \hline
 \hline
 \multicolumn{5}{|c|}{Winogrande (FP32 Accuracy = 67.88\%)} & \multicolumn{4}{|c|}{Piqa (FP32 Accuracy = 78.78\%)} \\ 
 \hline
 \hline
 64 & 66.85 & 66.61 & 67.72 & 67.88 & 77.31 & 77.42 & 77.75 & 77.64 \\
 \hline
 32 & 67.25 & 67.72 & 67.72 & 67.00 & 77.31 & 77.04 & 77.80 & 77.37  \\
 \hline
 16 & 68.11 & 68.90 & 67.88 & 67.48 & 77.37 & 78.13 & 78.13 & 77.69  \\
 \hline
\end{tabular}
\caption{\label{tab:mmlu_abalation} Accuracy on LM evaluation harness tasks on GPT3-8B model.}
\end{table}

\begin{table} \centering
\begin{tabular}{|c||c|c|c|c||c|c|c|c|} 
\hline
 $L_b \rightarrow$& \multicolumn{4}{c||}{8} & \multicolumn{4}{c||}{8}\\
 \hline
 \backslashbox{$L_A$\kern-1em}{\kern-1em$N_c$} & 2 & 4 & 8 & 16 & 2 & 4 & 8 & 16  \\
 %$N_c \rightarrow$ & 2 & 4 & 8 & 16 & 2 & 4 & 2 \\
 \hline
 \hline
 \multicolumn{5}{|c|}{Race (FP32 Accuracy = 40.67\%)} & \multicolumn{4}{|c|}{Boolq (FP32 Accuracy = 76.54\%)} \\ 
 \hline
 \hline
 64 & 40.48 & 40.10 & 39.43 & 39.90 & 75.41 & 75.11 & 77.09 & 75.66 \\
 \hline
 32 & 39.52 & 39.52 & 40.77 & 39.62 & 76.02 & 76.02 & 75.96 & 75.35  \\
 \hline
 16 & 39.81 & 39.71 & 39.90 & 40.38 & 75.05 & 73.82 & 75.72 & 76.09  \\
 \hline
 \hline
 \multicolumn{5}{|c|}{Winogrande (FP32 Accuracy = 70.64\%)} & \multicolumn{4}{|c|}{Piqa (FP32 Accuracy = 79.16\%)} \\ 
 \hline
 \hline
 64 & 69.14 & 70.17 & 70.17 & 70.56 & 78.24 & 79.00 & 78.62 & 78.73 \\
 \hline
 32 & 70.96 & 69.69 & 71.27 & 69.30 & 78.56 & 79.49 & 79.16 & 78.89  \\
 \hline
 16 & 71.03 & 69.53 & 69.69 & 70.40 & 78.13 & 79.16 & 79.00 & 79.00  \\
 \hline
\end{tabular}
\caption{\label{tab:mmlu_abalation} Accuracy on LM evaluation harness tasks on GPT3-22B model.}
\end{table}

\begin{table} \centering
\begin{tabular}{|c||c|c|c|c||c|c|c|c|} 
\hline
 $L_b \rightarrow$& \multicolumn{4}{c||}{8} & \multicolumn{4}{c||}{8}\\
 \hline
 \backslashbox{$L_A$\kern-1em}{\kern-1em$N_c$} & 2 & 4 & 8 & 16 & 2 & 4 & 8 & 16  \\
 %$N_c \rightarrow$ & 2 & 4 & 8 & 16 & 2 & 4 & 2 \\
 \hline
 \hline
 \multicolumn{5}{|c|}{Race (FP32 Accuracy = 44.4\%)} & \multicolumn{4}{|c|}{Boolq (FP32 Accuracy = 79.29\%)} \\ 
 \hline
 \hline
 64 & 42.49 & 42.51 & 42.58 & 43.45 & 77.58 & 77.37 & 77.43 & 78.1 \\
 \hline
 32 & 43.35 & 42.49 & 43.64 & 43.73 & 77.86 & 75.32 & 77.28 & 77.86  \\
 \hline
 16 & 44.21 & 44.21 & 43.64 & 42.97 & 78.65 & 77 & 76.94 & 77.98  \\
 \hline
 \hline
 \multicolumn{5}{|c|}{Winogrande (FP32 Accuracy = 69.38\%)} & \multicolumn{4}{|c|}{Piqa (FP32 Accuracy = 78.07\%)} \\ 
 \hline
 \hline
 64 & 68.9 & 68.43 & 69.77 & 68.19 & 77.09 & 76.82 & 77.09 & 77.86 \\
 \hline
 32 & 69.38 & 68.51 & 68.82 & 68.90 & 78.07 & 76.71 & 78.07 & 77.86  \\
 \hline
 16 & 69.53 & 67.09 & 69.38 & 68.90 & 77.37 & 77.8 & 77.91 & 77.69  \\
 \hline
\end{tabular}
\caption{\label{tab:mmlu_abalation} Accuracy on LM evaluation harness tasks on Llama2-7B model.}
\end{table}

\begin{table} \centering
\begin{tabular}{|c||c|c|c|c||c|c|c|c|} 
\hline
 $L_b \rightarrow$& \multicolumn{4}{c||}{8} & \multicolumn{4}{c||}{8}\\
 \hline
 \backslashbox{$L_A$\kern-1em}{\kern-1em$N_c$} & 2 & 4 & 8 & 16 & 2 & 4 & 8 & 16  \\
 %$N_c \rightarrow$ & 2 & 4 & 8 & 16 & 2 & 4 & 2 \\
 \hline
 \hline
 \multicolumn{5}{|c|}{Race (FP32 Accuracy = 48.8\%)} & \multicolumn{4}{|c|}{Boolq (FP32 Accuracy = 85.23\%)} \\ 
 \hline
 \hline
 64 & 49.00 & 49.00 & 49.28 & 48.71 & 82.82 & 84.28 & 84.03 & 84.25 \\
 \hline
 32 & 49.57 & 48.52 & 48.33 & 49.28 & 83.85 & 84.46 & 84.31 & 84.93  \\
 \hline
 16 & 49.85 & 49.09 & 49.28 & 48.99 & 85.11 & 84.46 & 84.61 & 83.94  \\
 \hline
 \hline
 \multicolumn{5}{|c|}{Winogrande (FP32 Accuracy = 79.95\%)} & \multicolumn{4}{|c|}{Piqa (FP32 Accuracy = 81.56\%)} \\ 
 \hline
 \hline
 64 & 78.77 & 78.45 & 78.37 & 79.16 & 81.45 & 80.69 & 81.45 & 81.5 \\
 \hline
 32 & 78.45 & 79.01 & 78.69 & 80.66 & 81.56 & 80.58 & 81.18 & 81.34  \\
 \hline
 16 & 79.95 & 79.56 & 79.79 & 79.72 & 81.28 & 81.66 & 81.28 & 80.96  \\
 \hline
\end{tabular}
\caption{\label{tab:mmlu_abalation} Accuracy on LM evaluation harness tasks on Llama2-70B model.}
\end{table}

%\section{MSE Studies}
%\textcolor{red}{TODO}


\subsection{Number Formats and Quantization Method}
\label{subsec:numFormats_quantMethod}
\subsubsection{Integer Format}
An $n$-bit signed integer (INT) is typically represented with a 2s-complement format \citep{yao2022zeroquant,xiao2023smoothquant,dai2021vsq}, where the most significant bit denotes the sign.

\subsubsection{Floating Point Format}
An $n$-bit signed floating point (FP) number $x$ comprises of a 1-bit sign ($x_{\mathrm{sign}}$), $B_m$-bit mantissa ($x_{\mathrm{mant}}$) and $B_e$-bit exponent ($x_{\mathrm{exp}}$) such that $B_m+B_e=n-1$. The associated constant exponent bias ($E_{\mathrm{bias}}$) is computed as $(2^{{B_e}-1}-1)$. We denote this format as $E_{B_e}M_{B_m}$.  

\subsubsection{Quantization Scheme}
\label{subsec:quant_method}
A quantization scheme dictates how a given unquantized tensor is converted to its quantized representation. We consider FP formats for the purpose of illustration. Given an unquantized tensor $\bm{X}$ and an FP format $E_{B_e}M_{B_m}$, we first, we compute the quantization scale factor $s_X$ that maps the maximum absolute value of $\bm{X}$ to the maximum quantization level of the $E_{B_e}M_{B_m}$ format as follows:
\begin{align}
\label{eq:sf}
    s_X = \frac{\mathrm{max}(|\bm{X}|)}{\mathrm{max}(E_{B_e}M_{B_m})}
\end{align}
In the above equation, $|\cdot|$ denotes the absolute value function.

Next, we scale $\bm{X}$ by $s_X$ and quantize it to $\hat{\bm{X}}$ by rounding it to the nearest quantization level of $E_{B_e}M_{B_m}$ as:

\begin{align}
\label{eq:tensor_quant}
    \hat{\bm{X}} = \text{round-to-nearest}\left(\frac{\bm{X}}{s_X}, E_{B_e}M_{B_m}\right)
\end{align}

We perform dynamic max-scaled quantization \citep{wu2020integer}, where the scale factor $s$ for activations is dynamically computed during runtime.

\subsection{Vector Scaled Quantization}
\begin{wrapfigure}{r}{0.35\linewidth}
  \centering
  \includegraphics[width=\linewidth]{sections/figures/vsquant.jpg}
  \caption{\small Vectorwise decomposition for per-vector scaled quantization (VSQ \citep{dai2021vsq}).}
  \label{fig:vsquant}
\end{wrapfigure}
During VSQ \citep{dai2021vsq}, the operand tensors are decomposed into 1D vectors in a hardware friendly manner as shown in Figure \ref{fig:vsquant}. Since the decomposed tensors are used as operands in matrix multiplications during inference, it is beneficial to perform this decomposition along the reduction dimension of the multiplication. The vectorwise quantization is performed similar to tensorwise quantization described in Equations \ref{eq:sf} and \ref{eq:tensor_quant}, where a scale factor $s_v$ is required for each vector $\bm{v}$ that maps the maximum absolute value of that vector to the maximum quantization level. While smaller vector lengths can lead to larger accuracy gains, the associated memory and computational overheads due to the per-vector scale factors increases. To alleviate these overheads, VSQ \citep{dai2021vsq} proposed a second level quantization of the per-vector scale factors to unsigned integers, while MX \citep{rouhani2023shared} quantizes them to integer powers of 2 (denoted as $2^{INT}$).

\subsubsection{MX Format}
The MX format proposed in \citep{rouhani2023microscaling} introduces the concept of sub-block shifting. For every two scalar elements of $b$-bits each, there is a shared exponent bit. The value of this exponent bit is determined through an empirical analysis that targets minimizing quantization MSE. We note that the FP format $E_{1}M_{b}$ is strictly better than MX from an accuracy perspective since it allocates a dedicated exponent bit to each scalar as opposed to sharing it across two scalars. Therefore, we conservatively bound the accuracy of a $b+2$-bit signed MX format with that of a $E_{1}M_{b}$ format in our comparisons. For instance, we use E1M2 format as a proxy for MX4.

\begin{figure}
    \centering
    \includegraphics[width=1\linewidth]{sections//figures/BlockFormats.pdf}
    \caption{\small Comparing LO-BCQ to MX format.}
    \label{fig:block_formats}
\end{figure}

Figure \ref{fig:block_formats} compares our $4$-bit LO-BCQ block format to MX \citep{rouhani2023microscaling}. As shown, both LO-BCQ and MX decompose a given operand tensor into block arrays and each block array into blocks. Similar to MX, we find that per-block quantization ($L_b < L_A$) leads to better accuracy due to increased flexibility. While MX achieves this through per-block $1$-bit micro-scales, we associate a dedicated codebook to each block through a per-block codebook selector. Further, MX quantizes the per-block array scale-factor to E8M0 format without per-tensor scaling. In contrast during LO-BCQ, we find that per-tensor scaling combined with quantization of per-block array scale-factor to E4M3 format results in superior inference accuracy across models. 


\end{document}
