\appendix

\section{Illustrations of User Interface}
\label{app:user_interface}

Here, we provide additional details regarding the system workflow and user interface. The system guides users through a series of steps prompting the user to (1) describe a problem they want to work on (Figure~\ref{fig:s1} and~\ref{fig:s2}), (2) propose a solution (Figure~\ref{fig:s3}), (3) recall and analyze design inspirations (Figure~\ref{fig:s2}), (4) predict the effects of their design (Figure~\ref{fig:s4}), (5) plan to evaluate and iterate on their project, (6) describe skills they want to develop, and (7) connect those skills to computer science careers, technical tasks, and popular industry technologies (Figure~\ref{fig:s5},~\ref{fig:s6},~\ref{fig:s7},~\ref{fig:s8} and ~\ref{fig:s9}).


\begin{figure}[t]
\floatconts
  {fig:s1}
  {\caption{Introduction of project-based learning (PBL) task and possible projects.}}
  {\includegraphics[width=0.85\textwidth]{figures/assignment_s1.png}}
\end{figure}


\begin{figure}[t]
\floatconts
  {fig:s2}
  {\caption{In the first step, the system prompts the student to specify a project and describe background and objectives.}}
  {\includegraphics[width=0.82\textwidth]{figures/s2.png}}
\end{figure}


\begin{figure}[t]
\floatconts
  {fig:s3}
  {\caption{In the second step, the system prompts the student to find inspiration and to connect the problem to existing examples.}}
  {\includegraphics[height=0.82\textheight]{figures/s3.png}}
\end{figure}


\begin{figure}[t]
\floatconts
  {fig:s4}
  {\caption{In the third step, the system prompts the student to specify their design in greater detail.}}
  {\includegraphics[width=0.72\textwidth]{figures/s4.png}}
\end{figure}


\begin{figure}[t]
\floatconts
  {fig:s5}
  {\caption{The system prompts the student to reflect on their proposal in terms of background and problem.}}
  {\includegraphics[width=0.9\textwidth]{figures/s5.png}}
\end{figure}


\begin{figure}[t]
\floatconts
  {fig:s6}
  {\caption{The system prompts the student to reflect on their proposal in terms of project objectives. After answering Q1 (choosing topic), the second question in the checklist is populated with the chosen topic's name.}}
  {\includegraphics[width=0.9\textwidth]{figures/s6.png}}
\end{figure}



\begin{figure}[t]
\floatconts
  {fig:s7}
  {\caption{The system prompts the student to reflect on their proposal in terms of related work and references.}}
  {\includegraphics[width=0.9\textwidth]{figures/s7.png}}
\end{figure}



\begin{figure}[t]
\floatconts
  {fig:s8}
  {\caption{The system prompts the student to reflect on their proposal in terms of design hypothesis.}}
  {\includegraphics[width=0.9\textwidth]{figures/s8.png}}
\end{figure}


\begin{figure}[t]
\floatconts
  {fig:s9}
  {\caption{The system prompts the student to reflect on their proposal in terms of evaluation plan.}}
  {\includegraphics[width=0.9\textwidth]{figures/s9.png}}
\end{figure}


\begin{figure}[t]
\floatconts
  {fig:s10}
  {\caption{The system prompts the student to describe three skills they want to develop while working on their project.}}
  {\includegraphics[width=0.9\textwidth]{figures/s10.png}}
\end{figure}



\begin{figure}[t]
\floatconts
  {fig:s11}
  {\caption{The system prompts the student to connect the first skill to a mentor whose perspective would be valuable.}}
  {\includegraphics[width=0.9\textwidth]{figures/s11.png}}
\end{figure}



\begin{figure}[t]
\floatconts
  {fig:s12}
  {\caption{The system prompts the student to connect the first skill to one of the top 5 important tasks for their selected mentor's career (tasks sourced from O*Net Online).}}
  {\includegraphics[width=0.9\textwidth]{figures/s12.png}}
\end{figure}



\begin{figure}[t]
\floatconts
  {fig:s13}
  {\caption{The system prompts the student to connect the first skill to tools and technologies associated with their selected mentor's career (technologies sourced from O*Net Online).}}
  {\includegraphics[width=0.9\textwidth]{figures/s13.png}}
\end{figure}

\FloatBarrier

\section{Human Expert Evaluation Rubric}\label{app:expert_evaluation_rubric}

Here, we provide a crowd worker's project proposal and the 23-item quality checklist that human experts and GPT-4o used to grade the project proposals. The rubric had four subtasks: 
\begin{itemize}
    \item Using the same quality checklist students used to self-assess their own work (10 items).
    \item Judging the appropriateness of skill-career pairings (9 items; 3 items per skill).
    \item An overall quality judgment question (“I would recommend a student include this project on their resume”) (1 item).
    \item Judging the quality of skill descriptions written by students (3 items; 1 item per skill). This set of questions was asked in a separate form developed to capture factors that could explain low rater agreement on the appropriateness of skill-career pairings.
\end{itemize}

\begin{figure}[t]
\floatconts
  {fig:e4}
  {\caption{Evaluation task description, shown to the human experts.}}
  {\includegraphics[width=0.85\textwidth]{figures/e4.png}}
\end{figure}

\begin{figure}[t]
\floatconts
  {fig:e5}
  {\caption{Instructions for evaluating project proposals, shown to the human experts.}}
  {\includegraphics[height=0.85\textheight]{figures/e5.png}}
\end{figure}

\begin{figure}[t]
\floatconts
  {fig:e6}
  {\caption{Items for evaluating a student's project proposal in terms of background and problem.}}
  {\includegraphics[width=0.85\textwidth]{figures/e6.png}}
\end{figure}

\begin{figure}[t]
\floatconts
  {fig:e7}
  {\caption{Items for evaluating a student's project proposal in terms of objectives.}}
  {\includegraphics[width=0.85\textwidth]{figures/e7.png}}
\end{figure}

\begin{figure}[t]
\floatconts
  {fig:e8}
  {\caption{Items for evaluating a student's project proposal in terms of related work.}}
  {\includegraphics[width=0.85\textwidth]{figures/e8.png}}
\end{figure}

\begin{figure}[t]
\floatconts
  {fig:e9}
  {\caption{Items for evaluating a student's project proposal in terms of design hypothesis.}}
  {\includegraphics[width=0.85\textwidth]{figures/e9.png}}
\end{figure}

\begin{figure}[t]
\floatconts
  {fig:e10}
  {\caption{Items for evaluating a student's project proposal in terms of evaluation plan.}}
  {\includegraphics[width=0.85\textwidth]{figures/e10.png}}
\end{figure}

\begin{figure}[t]
\floatconts
  {fig:e11}
  {\caption{Items for evaluating whether student has matched their skill \#1 to appropriate technologies, computer science career, and career task.}}
  {\includegraphics[width=0.85\textwidth]{figures/e11.png}}
\end{figure}

\begin{figure}[t]
\floatconts
  {fig:e12}
  {\caption{Items for evaluating whether student has matched their skill \#2 to appropriate technologies, computer science career, and career task.}}
  {\includegraphics[width=0.85\textwidth]{figures/e12.png}}
\end{figure}

\begin{figure}[t]
\floatconts
  {fig:e13}
  {\caption{Items for evaluating whether student has matched their skill \#3 to appropriate technologies, computer science career, and career task.}}
  {\includegraphics[width=0.85\textwidth]{figures/e13.png}}
\end{figure}

\begin{figure}[t]
\floatconts
  {fig:e14}
  {\caption{Item for evaluating overall quality and open-ended comments on the project proposal.}}
  {\includegraphics[width=0.85\textwidth]{figures/e14.png}}
\end{figure}

\begin{figure}[t]
\floatconts
  {fig:e1}
  {\caption{Instructions for classifying the skills written by students.}}
  {\includegraphics[width=0.85\textwidth]{figures/e1.png}}
\end{figure}

\begin{figure}[t]
\floatconts
  {fig:e2}
  {\caption{Project proposal and skill \#1 classification question.}}
  {\includegraphics[width=0.85\textwidth]{figures/e2.png}}
\end{figure}

\begin{figure}[t]
\floatconts
  {fig:e3}
  {\caption{The skill \#1 and skill \#2 classification questions.}}
  {\includegraphics[width=0.85\textwidth]{figures/e3.png}}
\end{figure}

\FloatBarrier

\section{LLM-as-a-Judge Prompt GPT-4o}\label{app:llm-prompts}

To produce ratings from GPT-4o, we call OpenAI API with the following prompt templates, populated with each project proposal's text. The rubric items are the same as the items used in the human expert grader's rubric given in Appendix \ref{app:expert_evaluation_rubric}.

\begin{table}[h!]
\centering
 \begin{tabular}{||p{0.85\textwidth}||} 
 \hline
 System Prompt - Quality Checklist \& Resume Project \\ [0.5ex] 
 \hline\hline
 You are a teaching assistant. Your job is to check project proposals for quality and higher order thinking skills in Bloom's taxonomy (evaluation, synthesis). For each item in the checklist (Q1-Q11), respond with Yes or No. \\ [1ex] 
 \hline
 \end{tabular}
\end{table}

\begin{table}
\centering
 \begin{tabular}{||p{0.95\textwidth}||} 
 \hline
 User Prompt - Quality Checklist \& Resume Project \\ [0.5ex] 
 \hline\hline
Project Title: \`{}\`{}\`{}project\_title\`{}\`{}\`{} \\
Background/Problem. \\
Briefly describe the focus of your project and motivate its importance: \`{}\`{}\`{}q2\_problem\`{}\`{}\`{} \\
* Q1. This proposal describes a specific focus and motivation (beyond describing video game design, interactive web design, writing useful programs and scripts, or practicing cloud computing in general) \\
* Q2. This proposal describes a good use of computer science skills \\ \\

Objectives: \\
Topic: \`{}\`{}\`{}q1\_topic\`{}\`{}\`{} \\
Describe the software you want to build and how it will work: \`{}\`{}\`{}q3\_objective\`{}\`{}\`{} \\
* Q3. This proposal describes specific, tangible features that will be built in the project \\
* Q4. Working on the project is relevant to learning about q1\_topic \\ \\

Related Work. \\
Inspirations: \`{}\`{}\`{}q4\_inspirations\`{}\`{}\`{} \\
Which existing features are easiest to copy: \`{}\`{}\`{}q5\_analysis\_copy\`{}\`{}\`{} \\
Which existing features would you learn the most from building: \`{}\`{}\`{}q6\_analysis\_learn\`{}\`{}\`{} \\
* Q5. The proposal analyzes similar products, papers, or applications \\ \\

Testable Design Hypothesis. \\
Name an interesting difference between your project idea and existing solutions. What effect could this difference have: \`{}\`{}\`{}q7\_your\_design\`{}\`{}\`{} \\
* Q6. The design hypothesis describes a specific feature that will be built in the project \\ 
* Q7. The predicted effects of the design hypothesis can be tested quickly \\ \\

Evaluation Plan. \\
Select the types of feedback you plan on collecting to evaluate your project (and motivate yourself to build better iterations of your project): \`{}\`{}\`{}q8\_feedback\`{}\`{}\`{} \\
Briefly describe how you will use each one to evaluate and improve small, manageable parts of your project: \`{}\`{}\`{}q9\_evaluation\_plan\`{}\`{}\`{} \\
* Q8. The project has an objective measure of success or learning  \\
* Q9. The design hypothesis has an objective measure of success or learning \\
* Q10. The evaluation plan can be carried out within a 4 week sprint \\ \\

Project Plan. \\
Steps: \`{}\`{}\`{}steps\`{}\`{}\`{} \\
* Q11. I would recommend a student include this project on their resume. \\ \\

Desired format: \\
* QX. \{Yes $\mid$ No\} \\ [1ex] 
 \hline
 \end{tabular}
\end{table}

\begin{table}[h!]
\centering
 \begin{tabular}{||p{0.85\textwidth}||}  
 \hline
 System Prompt - Skill Pairing Classification \\ [0.5ex] 
 \hline\hline
 You are a teaching assistant. Your job is to check whether the skill students want to learn is appropriate for the listed career and career-specific task. The skill students want to learn is in double quotes. The mentor and the task are listed on the following line. For each item in the checklist (Q1-Q3), respond with Yes or No. \\ [1ex] 
 \hline
 \end{tabular}
\end{table}

\begin{table}[h!]
\centering
 \begin{tabular}{||p{0.85\textwidth}||}
 \hline
 User Prompt - Skill Pairing Classification \\ [0.5ex] 
 \hline\hline
Background/Problem: \`{}\`{}\`{}q2\_problem\`{}\`{}\`{} \\

Objectives: \`{}\`{}\`{}q3\_objective\`{}\`{}\`{} \\

Testable Design Hypothesis: \`{}\`{}\`{}q7\_your\_design\`{}\`{}\`{} \\

Skill: \`{}\`{}\`{}skill\`{}\`{}\`{} \\
* Q1. The technologies chosen for learning the skill are a good fit. \\
* Q2. The selected career is relevant to the skill. \\
* Q3. The selected career-specific task is relevant for learning the skill. \\

Desired format:
* QX. \{Yes $\mid$ No\} \\ [1ex] 
 \hline
 \end{tabular}
\end{table}

\begin{table}[h!]
\centering
 \begin{tabular}{||p{0.85\textwidth}||}
 \hline
 System Prompt - Skill Classification \\ [0.5ex] 
 \hline\hline
 You are a teaching assistant. Rate each skill as either: \\
* Irrelevant - not obvious how this skill is relevant to the project description \\
* Auxiliary - the skill not a computer science technical career skill -- for example ``Art design for video games", ``researching on health and well-being", or non-technical project management skills like ``communication" or ``make documents" \\
* Vague - an overly broad computer science skill; giving concrete advice for improving this skill over the course of a project is difficult -- for example ``coding" or ``debugging" \\
* Good - practicing this skill will improve STEM or Computer Science knowledge -- for example ``front end development" \\
Respond with Irrelevant, Auxiliary, Vague, or Good. \\ [1ex] 
 \hline
 \end{tabular}
\end{table}

\begin{table}[h!]
\centering
 \begin{tabular}{||p{0.85\textwidth}||} 
 \hline
 User Prompt - Skill Classification \\ [0.5ex] 
 \hline\hline
 Background/Problem: \`{}\`{}\`{}q2\_problem\`{}\`{}\`{} \\

Objectives: \`{}\`{}\`{}q3\_objective\`{}\`{}\`{} \\

Testable Design Hypothesis: \`{}\`{}\`{}q7\_your\_design\`{}\`{}\`{} \\

Skill: \`{}\`{}\`{}skill\`{}\`{}\`{} \\
Desired format: \\
\{Irrelevant $\mid$ Auxiliary $\mid$ Vague $\mid$ Good\} \\ [1ex] 
 \hline
 \end{tabular}
\end{table}

\FloatBarrier

\section{Recruitment Call Posted on Prolific}
\label{app:prolific_recruitment}

\begin{figure}[t]
\floatconts
  {fig:prolific}
  {\caption{Recruitment call displayed on Prolific crowd worker platform.}}
  {\includegraphics[width=0.85\textwidth]{figures/Prolific_Recruitment_Call.png}}
\end{figure}