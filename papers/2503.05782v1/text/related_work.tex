\section{Related Work}
\label{sec:related_work}

% sentence 2. (RELATED WORK PARAGRAPH) BUT PBL is challenging to scale because some students thrive with minimal educator guidance and other flounder. Early evaluation of project proposals can help educators determine which students need more support before stuents run into major issues, yet making this judgement call in may be challenging and time-consuming because it involves reviewing freeform text on arbitary topics, classifying student aptitude based on incomplete knowledge, and depends on the instructor's ability to provide helpful and relevant feedback without overshadowing student motivation for a specific project topic.

% sentence 2. However, PBL has shown mixed effectiveness in the classroom, with learning and motivation depending on students' prior ability, educators' ability to support specific projects and topics \citep{Barnes2008:Increasing}, and whether the project idea originated from students or teachers (CITE: 500 projects). Evaluating project proposals may help educators identify which students require additional guidance or alert educators to major issues that could consume limited motivation, time, and resources. Yet evaluating project proposals is challenging and time-consuming, requiring educators to review freeform text on arbitrary topics and infer student aptitude from incomplete or vague details.

\cite{Barnes2008:Increasing} found that student engagement in an Algebra 1 class improved when activities incorporated opportunities for choice, goal-setting, and one-on-one interactions with teachers. Similarly, \cite{pucher2011project} reported increased motivation among students when employing PBL in computer science classes. However, both studies highlighted challenges tied to students’ limited meta-cognitive skills and domain knowledge. For instance, many students struggled to maintain their goal portfolios \citep{Barnes2008:Increasing}, and student-defined projects received significantly lower grades from faculty \citep{pucher2011project}. \cite{kirschner2006unguided} critique PBL's effectiveness in supporting student learning, arguing that under minimal guidance, inexperienced students often struggle to develop effective strategies for independently searching for and applying information. Recent advances in LLMs enabled new types of learning technologies in various educational domains~\citep{Kasneci2023:Chatgpt}. LLMs' ability to provide feedback to student submissions~\citep{Botelho2023:Leveraging} and to support student self-reflection~\citep{Yazici2024:Gelex} inspires the present development of a system to support effective execution of PBL.  



% --- (START) OLD

%\paragraph{The Need To Scaffold Discovery-Based Learning.} \cite{kirschner2006unguided} discuss reasons for why minimal guidance in discovery-based learning sets students up for failure. Students who are not familiar with the domain they are trying to explore will struggle to search for information that could be useful for solving their problem. This holds true for the PBL proposal-writing process as well---the process needs to guide students through the strategies they need to succeed.

%\textbf{Higher Levels of Blooms Taxonomy in Computer Science Education.} Numerous studies have used Bloom’s Taxonomy \citep{bloom1956taxonomy} as the standard for judging whether test items require lower order or higher order thinking \citep{thompson2008mathematics}. Bloom's Taxonomy has been used to specify learning outcomes in computer science prior to assessment \citep{starr2008bloom}. \cite{scott2003bloom} argues that assessments in computer science should cover each level of Bloom’s Taxonomy to ensure that students have the opportunity to demonstrate their achieved level of mastery. We explicitly use Bloom's Taxonomy to guide the structure of our example-based design activity.%

%\paragraph{Novices Struggle to Use Interfaces Designed for Experts.} Traditional information-rich interaction modes like textbooks and search engines are not beginner friendly because they require the student to have a clear idea of a concept to search successfully and efficiently. \cite{neuman1991organizing, neuman1993designing} found evidence that the linguistic and conceptual gaps between students’ abstractions and those of systems designed for adults often stymied successful searching. We believe this observation holds for students trying to navigate the O*Net Online Database of careers, tasks, and hot technologies. More work has to be done to bridge the gap between novices’ understanding and the topics they want to explore.

% \paragraph{STEM Mentorship and AI.} \cite{nora2007mentoring} conceptualized four domains of mentorship: psychological and emotional support, academic subject knowledge support, goal setting and career paths, and serving as a role model. The capabilities of generative AI may allow it to perform the first two functions in ways that surpass human capabilities. Scaffolding critical evaluation exercises is necessary when giving students access to an ``expert'' mentor---students may accept an expert’s advice without carefully considering the merits of their arguments \citep{hovland1951influence}. Prior works have explored the ethical questions of privacy in AI-supported mentoring \citep{kobis2021ethical}, scaling mentoring with AI \citep{klamma2020scaling}, and using Google Gemini as an AI mentor to navigate STEM careers \citep{chang2024navigating}.

% --- (END) OLD