\section{System Design}
\label{sec:design}

% --- (START) NEW

% sentence 3. (SYSTEM DESIGN) IN THIS WORK, we design, build, and conduct user study for using a software system to collect project propsoals and auxiliary information that educators can use to evaulate a student's preparedness to engage with project-based learning.

Our software system is implemented as an React.js and firebase-based web application that collects project proposals and aptitude information that may help educators determine whether a student is ready to engage with PBL. The system logs the time and type of user actions, such as keystroke patterns, answers to multiple-choice selections, and when users navigate away from and back to the web page. Our system first asks users questions about their aptitude for problem-solving, experience with computer science technologies, and prior experience with PBL. Then the system asks users to write down their idea for a high school-level computer science project on developing an interactive web application or video game. Our project proposal form, adapted from \cite{CSPathway, Byrdseed}, asks users to (1) describe a problem they want to work on, (2) describe the software they want to build, (3) recall and analyze design inspirations, (4) predict the effects of their design, (5) make a plan to evaluate and iterate on their project, (6) self-evaluate using a 10-point quality checklist, (7) describe skills they want to develop, and (8) connect those skills to computer science careers, technical tasks, and popular industry technologies. Due to space limitations, we included UI images in Appendix~\ref{app:user_interface}.

% --- (START) OLD

% CS Pathways program offers high school seniors an opportunity to work on a computer science project of their choosing under the guidance of volunteer mentors who are computer science professionals from local companies. The teachers leading the initiative told us about two main challenges they face in trying to scale their program: (1) students not providing enough detail in their initial project proposals, which makes matching students to relevant mentors more difficult, and (2) gaps in the teachers' technical expertise, which hindered their ability to offer relevant guidance, especially in classes where industry mentors are not readily available or for students with diverse career goals. To help students submit more detailed proposals and better link their project goals to career-relevant skills, we augmented the CS Pathways project proposal form by adding three components: (1) an example-based design activity that scaffolds higher order thinking, (2) a 10-question quality checklist, and (3) an activity that asked students to connect the skills they wanted to learn to careers, tasks, and technologies from the O*Net Online Database.

% \textbf{Example Based Design Activity.} A small pilot study on Prolific revealed confusion and low-effort responses on three questions in the original CS Pathway proppsal form --- ``Methodology”, and ``Impact”, and ``Resources/Tutorials”. Since ``Impact” and ``Methodology” questions correspond to the higher level ``synthesis” thinking skills according to Bloom’s Taxonomy \citep{bloom1956taxonomy}, success requires support from lower levels of Bloom’s Taxonomy, namely recall, understanding, analysis and evaluation.
% We scaffold the thinking expected in the ``Impact” section (``If you are successful what difference will it make? What customers would be interested in this work?”) with an example-based design activity that leads students to recall, analyze, and evaluate examples from their prior knowledge in order to predict the impact of a tangible change of their choosing. This approach prepares students to use iterative testing and remain motivated to make improvements.

% \textbf{10-question Quality Checklist.} We designed this survey as a quick self-evaluation tool, enabling comparison between students self-ratings, teaching assistants, and GPT-4o. The rubric is adapted from Worcester Polytechnic Institute's Project Proposal Rubric for a mobile development class \citep{WPIRubric}. Our survey uses binary items based on the S.M.A.R.T. goals framework (Specific, Measurable, Achievable, Relevant, Time-Bound), which have been shown to improve student performance on projects \citep{lawlor2012smart}, alongside students' responses to nine project proposal questions in editable text boxes. To move on to the ``Career-Relevant Skills" activity, students only needed to check off one item.

% \textbf{Connect Skills to O*Net Online Database.} As part of the CS Pathways proposal, students identified three skills they hoped to develop by working on their computer science project. To better understand how students perceived the skills they wanted to learn, we asked them to match each skill to one of nine computer science careers, one of the five key tasks of that career, and any relevant hot technologies associated with the career. Careers, tasks, and technologies were sourced from the O*Net Online Database. By comparing the ratings from TAs and GPT-4o on how well students completed this task, we were able to estimate how difficult this task was for students, particularly novices, and assess whether GPT-4o could be used to discriminate between appropriate or inappropriate skill pairings.