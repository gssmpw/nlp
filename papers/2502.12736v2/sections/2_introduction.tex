%%%%%%%%%%
\section[Introduction]{Introduction}
%%%%%%%%%%
\label{sec: intro}

Integrated sensing and communications~(ISAC) techniques are widely recognized as a crucial enabler for 6G communications~\cite{Dong23TWC_Sensing} 
as it helps establish cognitive and perceptive wireless networks in a cost-efficient manner without requiring extensive additional infrastructure~\cite{Liu22JSAC_ISAC}.
Specifically, by utilizing the readily available channel state information (CSI), ISAC enables edge devices (EDs), such as cellular base stations and Wi-Fi access points, to perceive channel variations caused by human motions and environment dynamics.
By handling the channel variations with powerful signal processing techniques empowered by edge intelligence~(EI), i.e., the artificial intelligence on the edge, the EDs can develop various sensing functionalities.
These sensing functionalities, in which human activity recognition~(HAR) is a prominent example~\cite{Li23COMMAG_Integrated}, can retrospectively benefit communications through improved context and mobility awareness~\cite{Strinati24EuCNC_Distributed,Saleem21TWC_Mobility}.
With advances in EI and the increasingly large bandwidth and antenna arrays used by wireless transceivers, the sensing capability of EDs in wireless ISAC networks is expected to grow rapidly, paving the way for more intelligent and adaptive wireless networks.


Nevertheless, the EI tends to be highly \emph{domain-dependent}.
\Copy{E-1-1}{\frev{According to~\cite{Akrout23CST_Domain}, a domain in this context can be defined as the statistical relationship between CSI variations and user activities.
In practice, the domains should be designated as the variable that changes during the EI's deployment and whose change has a significant influence on this CSI-activity relationship.}}
Existing studies pursue the cross-domain sensing capability of EI through extracting domain-invariant features from CSI, either by using the Doppler frequency shifts~(DFS)~\cite{Zhang21TPAMI_Widar3,Niu22TMC_Understand} or adversarial learning~\cite{Jiang18MobiCom_Sulu, Wang23TMC_AirFi, Li21IMWUT_CrossGR}.
However, domain-invariant features discard domain-relevant information in CSI~\cite{Bui21NIPS_Exploiting} and thus suffer from deficient accuracy.
Moreover, in common situations where users are in close proximity to their devices, their individual characteristics, such as body contours and subtle movements, have a significant impact on CSI~\cite{Hu2023Mobicom_Muse}, which can hardly be removed without substantially degrading the sensing potential of CSI.
Due to the unpredictable nature of environment dynamics and user characteristics in practice, substantial discrepancies exist between the domains in offline pre-training and those encountered in online deployments, making it challenging for EI to obtain robust and high-accuracy cross-domain sensing capabilities.

Essentially, to guarantee practical cross-domain sensing capability, the EI needs to actually learn for the domains encountered during online deployment~\cite{Gulrajani21ICLR_In}.
This requires the collection of sufficient training datasets and timely update of the EI to extend its sensing capability to new domains. 
However, as new domains continually emerge, the CSI datasets expand over time; storing these datasets in an ED, which generally has rather limited memory resources, becomes increasingly prohibitive~\cite{Mao17CST_Survey,Haibeh22Access_Survey}, especially when CSI data is collected over large bandwidths of multiple bands for high sensing precision~\cite{Li24MobiSys_UWB}.
Although it is possible for the ED to upload the datasets and its EI to a cloud server and conduct training remotely to alleviate its local burden~\cite{Lin19PIEEE_Computation}, this approach will inevitably incur substantial communication overheads to the backbone network and lead to high latency~\cite{Barbera13INFOCOM_Offload}.


\begin{figure}[t] 
    \centering
    \includegraphics[width=0.9\linewidth]{./figures/system_model/sys_model_1010_clip.pdf}
        \caption{Working principle of \name: The CSI dataset of current domain and the knowledge learned from previous domains are jointly used for the cross-domain continual learning of EI.}
        \label{fig_sys_mod}
    \vspace{-.5em}
    \end{figure}

Fortunately, as different domains emerge at different times, the datasets for each domain, referred to as \emph{domain datasets}, are gathered sequentially. 
This allows an ED to train its EI on each domain dataset one by one.
For example, consider a scenario where a new domain dataset is collected periodically to accommodate newly emerged users within the period.
At the end of each period, the EI can be trained solely on this new domain dataset, which is then discarded to free up memory resources on the ED, preparing for subsequent domain datasets.
Nevertheless, such sequential training suffers from the \emph{catastrophic forgetting} problem~\cite{Goodfellow14ICLR_Empirical,Kemker18AAAI_Measuring}, i.e., the EI tends to rapidly forget its knowledge of previous domains when trained on the current domain dataset.
This problem severely hinders the EI from gaining the cross-domain sensing capability and, consequently, limits its potential for enabling ubiquitous sensing applications in resource-constrained EDs.


In this paper, we propose \emph{\name}, a resource-efficient framework that enables the EI to continually learn-then-discard domain datasets without catastrophic forgetting, achieving cross-domain continual learning for discriminative sensing tasks in wireless ISAC networks.
The working principle of \name is illustrated in Fig.~\ref{fig_sys_mod}.
Treating EI as a parameterized neural model, we formulate its continual learning over a sequence of domain datasets as a sequence of optimizations of neural parameters.
\name jointly considers the loss on the current domain dataset and a regularization term employed to preserve the knowledge that the EI has learned from previous domain datasets.
After each training, the knowledge is updated, and the used domain dataset is discarded to free up memory occupation, preparing the system to learn subsequent domains.
In particular, in stark contrast with existing studies~\cite{Hu23ICC_Digital, Ji22MobiSys_SiFall} that consider continual learning of EI in ISAC but focus only on the adaptation to the newest domain, \name aims to achieve high sensing accuracy across \emph{all} domains.
As far as the authors know, there are no existing proposals for this important problem in wireless ISAC networks.


Developing \name involves addressing three major challenges.
\emph{Firstly}, unlike video or radar images, the low spatial resolution of EDs and the multipath effects of wireless transmissions make the relationship between CSI and user activities highly intricate, which is further complicated by the nonequispaced sampling time, requiring specialized neural model design.
\emph{Secondly}, both the knowledge and its corresponding regularization term are essential for cross-domain continual learning, yet they are challenging to derive. 
In addition, they must be memory and computational efficient to be applicable given the limited resources of EDs.
\emph{Thirdly}, the training on subsequent domain datasets inevitably leads to deviation of neural parameters from the optimal point for the current domain, degrading its sensing accuracy.
However, such deviations are highly unpredictable, making it difficult to mitigate the accuracy degradation.


To handle the above challenges efficiently, we first design a transformer-based discriminative neural model capable of handling sequences of noisy and nonequispaced CSI samples, which has a compact architecture and relies on no complicated time-frequency transformations.
Then, we propose a regularization term for knowledge retention based on distilled core-sets of few exemplar CSI data, which are significantly smaller than the domain dataset, and improve their efficacy by a novel hybrid clustering-herding exemplar selection method.
Finally, we prevent the neural model from rapid performance loss by endowing it with enhanced robustness towards worst-case parameter deviations.
Our main contributions are:
\begin{itemize}[leftmargin=*]
    \item We propose \name, the first cross-domain continual learning framework for resource-limited EDs in wireless ISAC networks, enabling the EI to learn across a sequence of domain datasets without catastrophic forgetting.
    \item We design an efficient algorithm to handle nonequispaced noisy CSI samples, preserve previous knowledge by replaying distilled core-sets of exemplars, and enhance robustness against unpredictable parameter deviations.
    \item We perform practical experimental evaluations using a real-world wireless ISAC system for HAR, confirming that \name significantly improves the cross-domain capability of EI, achieving 89\% of the cross-domain training accuracy of cumulative training while consuming only 3\% of its memory, mitigating catastrophic forgetting by 79\%.
    \end{itemize}

    The rest of this paper is organized as follows: In Section~\ref{sec_sysmod}, we establish the system model, including the models for CSI and EI in wireless ISAC networks.
    We formulate the \name framework for the cross-domain continual learning of EI in Section~\ref{sec_prob}, and design the algorithm to solve it in Section~\ref{sec_algorithm}.
    Experimental setups and results are presented in Section~\ref{sec_eva}.
    Finally, a conclusion is drawn in Section~\ref{sec_conclu}.
    
 

\endinput