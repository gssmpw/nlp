\section{Conclusion}
%
% Why the problem we solve is important:
The generation of high-quality conformational ensembles of proteins is a key task in many protein-related fields.
%
% Our key contribution:
Inspired by generative models for protein design, we propose BBFlow, a method for sampling high-quality ensembles with state-of-the-art efficiency while, at the same time, also avoiding problems with \textit{de novo} proteins and over-stabilization observed in current state-of-the-art models.
%
% How we achieve this:
We achieve this by introducing a conditional prior distribution and encoding the protein's equilibrium backbone structure as condition in a flow-matching model that is based on backbone geometry, decoupling the structure prediction task from conformational sampling.
%
% Additional sentence on a nice thing of our approach that is not 100% necessary to state:
This eliminates the need for evolutionary information and enables to train the model from scratch, without requiring pre-trained weights from a folding model that is trained on a much larger dataset.
%
% Impact of the proposed methods/outlook:
We expect that both of these ideas -- using a conditional prior in flow-matching and replacing evolutionary information by structure-conditioning -- can also be applied to other problems in generative modeling and structural biology.
%

%
% Practical impact:
AlphaFlow \cite{jing2024alphafoldmeetsflowmatching} is widely used by practitioners for sampling protein conformational ensembles without MD -- it goes well beyond a proof-of-concept.
We also see BBFlow as highly relevant in practice, given its significantly increased efficiency at accuracy comparable with AlphaFlow, allowing accurate conformational ensemble generation on a much larger scale.
% Example application/outlook:
In particular, BBFlow can be applied in pipelines for \textit{de novo} protein design, where it could enable the screening of generated structures for desired dynamics -- a property that is challenging to incorporate into designed proteins so far.
