\newcommand{\error}[3]{{#1}^{\, #2}_{\, #3}} % for formatting reported values

\section{Experiments}
\label{sec:experiments}

\paragraph{Training}
In order to directly compare the proposed model to the current state-of-the-art conformational ensemble generator for proteins, AlphaFlow \cite{jing2024alphafoldmeetsflowmatching}, we train BBFlow on the ATLAS dataset \cite{vander2024atlas} with the same split into training, validation and test proteins.
The ATLAS dataset consists of three trajectories of 100\,ns long all-atom Molecular Dynamics (MD) simulations for 1390 structurally diverse proteins, of which \citet{jing2024alphafoldmeetsflowmatching} select 1265 for training, 39 for validation and 82 for testing.
We train the model, and variants where we leave out key features for an ablation study, for 3 days on two NVIDIA A100-40GB GPUs from scratch, i.e. without initial weights from a pre-trained folding model\footnote{The source code, model weights and the \textit{de novo} MD dataset will be published along with the final version of this paper.}.
For all experiments, we use the same hyperparameters as in FrameFlow \cite{yim2023frameflow} and GAFL \cite{wagner2024generating}, except for the number of timesteps, which we set to 20.
Also the respective feature dimensions are increased by 128 for embedding the amino acid identity as node feature and by 22 or 25, respectively, for embedding the equilibrium structure encoding with or without direction as edge feature.

\paragraph{Baselines}
We compare BBFlow with models from \citep{jing2024alphafoldmeetsflowmatching} that were fine-tuned on the training set of BBFlow, but rely on pre-trained weights from the folding models AlphaFold 2 \cite{jumper2021highly} and ESMFold \cite{lin2023evolutionary} that were trained on much larger datasets.
Next to the original AlphaFlow-MD model (referred to in this work as \textbf{AlphaFlow}), we also evaluate AlphaFlow-MD with templates (\textbf{AlphaFlow-T}), which receives the equilibrium structure as input, encoded as template in AlphaFold.
\citet{jing2024alphafoldmeetsflowmatching} also introduce a model that relies not directly on the expensive MSA but on the protein language model ESM (\textbf{ESMFlow-T}).
Additionally, we compare BBFlow with models based on AlphaFlow-MD with templates that are optimized for efficiency by distillation (\textbf{AlphaFlow-T$_\text{dist}$}), decreasing the timesteps required from 10 to 1, and by reducing the number of layers (\textbf{AlphaFlow-T$_\text{12L,dist}$}).
We evaluate all models above using the conformations deposited in the AlphaFlow GitHub repository\footnote{https://github.com/bjing2016/alphaflow}.
We also include \textbf{ConfDiff} \cite{wang2024proteinconfdiff} in our comparison -- a diffusion model with pre-trained sequence representation from AlphaFold 2 that is trained on the large PDB dataset and fine-tuned on ATLAS (more details in Appendix~\ref{sec:confdiff}).

\begin{table*}[h!]
    \centering
    \caption{Performance of BBFlow and baselines (Sec.~\ref{sec:experiments}) on the ATLAS test set. For each protein, we evaluate the metrics described in Sec.~\ref{sec:metrics} and report the median of all proteins. We also report RMSF medians over all residues and pairwise RMSD medians over all proteins, respectively, and indicate the MD reference value in parentheses. Inference time is reported per generated conformation of the length 302 protein 7c45A. All metrics except for RMSF $r$ and time are reported in \AA. Errors are estimated as described in Sec.~\ref{sec:metrics} and are shown in parentheses if they are above the displayed precision. Best values are \textbf{bold}, second best are \underline{underlined}.}
    \begin{tabular}{lccccccc}
    \toprule
     & \multicolumn{3}{c}{RMSF} & \multicolumn{2}{c}{Pairwise RMSD} & &\\
    \cmidrule(lr){2-4} \cmidrule(lr){5-6}
     & $r$ ($\uparrow$) & MAE ($\downarrow$) & Median (1.48) & MAE ($\downarrow$) & Median (2.90) & PCA $\mathcal{W}_2$ ($\downarrow$) & Time [s] \\
    \midrule
    AlphaFlow & 0.86 & 0.59 (0.01) & \underline{1.51} & 1.34 (0.01) & \textbf{2.89} & 1.47 (0.03) & 32.0 \\
    ConfDiff & 0.88 & 0.62 (0.01) & 2.00 & 1.45 (0.01) & 3.43 & 1.41 (0.03) & 20.2 \\
    AlphaFlow-T & \textbf{0.92} & \textbf{0.41} (0.01) & 1.17 & \underline{0.91} (0.01) & 2.18 & \textbf{1.28} (0.03) & 32.6 \\
    ESMFlow-T & \textbf{0.92} & 0.52 (0.01) & 0.94 & 1.22 (0.01) & 2.00 & 1.48 (0.03) & 11.2 \\
    AlphaFlow-T$_\text{dist}$ & \textbf{0.92} & 0.68 (0.01) & 0.90 & 1.41 (0.01) & 1.73 & 1.44  (0.03) & 3.3 \\
    AlphaFlow-T$_\text{12L,dist}$ & 0.90 & 0.85 (0.01) & 0.68 & 1.80 (0.01) & 1.40 & 1.60  (0.03) & \underline{1.2} \\
    \midrule
    BBFlow (Ours) & 0.89 & \underline{0.47} (0.01) & \textbf{1.48} & \textbf{0.78} (0.01) & \underline{2.76} & \underline{1.34}  (0.03) & \textbf{0.9} \\
    \bottomrule
    \end{tabular}
    % }
    \label{tab:benchmark}
\end{table*}

\subsection{Metrics}
\label{sec:metrics}
We evaluate the performance of the compared models by reporting key metrics introduced by \citet{jing2024alphafoldmeetsflowmatching}, which quantify how well statistical properties of the generated ensembles -- listed below -- agree with those obtained by MD.
In all experiments, we generate 250 conformations per protein, as in AlphaFlow, and bootstrap the set of MD conformations 100 times in order to estimate the error caused by sampling finitely many states.
All metrics are calculated using the C$\alpha$ atoms of the protein structures.

\paragraph{RMSF}
The Root Mean Square Fluctuation (RMSF) of C$\alpha$ atoms measures the magnitude of positional deviations of individual residues across the set of conformations.
For a given residue, these fluctuations are calculated in a reference frame that is defined by aligning the entire protein to the equilibrium structure and thus implicitly depend on the positions of all other residues.
Consequently, RMSF can be interpreted as measure for flexibility, but also encodes global collective behaviour.
As in AlphaFlow, we calculate the Pearson correlation between RMSF profiles (for an example see Figure~\ref{fig:rmsf_denovo}) obtained from MD and generated ensembles in order to quantify how well the shapes of the profiles match. We also include the Mean Absolute Error (MAE) of per-residue RMSF to measure how well RMSF amplitudes are reproduced, and compare the median over all residues with the ground truth in order to quantify systematic over- or understabilization.

\paragraph{Pairwise RMSD}
For each protein, we calculate the average C$\alpha$ RMSD between any two conformations $x$ as
\begin{equation}
\text{pwRMSD} \equiv \frac{1}{N^2}\sum_{i,j=1}^{N_\text{confs}} \text{RMSD}\left(x_i,x_j\right).
\end{equation}
This quantifies the magnitude of conformational changes without requiring a specified reference state.
We report the MAE of pairwise RMSD across all proteins and the median pairwise RMSD, the latter of which also quantifies systematic over- or understabilization.


\paragraph{PCA}
A metric that explicitly accounts for conformational changes, and quantifies how well the respective conformations are captured, relies on the Principal Component Analysis of the C$\alpha$ positions across the sampled conformations.
We project the generated states on the first two principal components obtained from MD, thus receiving a two-dimensional PCA-projection of each conformation (for examples see Figure~\ref{fig:app_pca_components_atlas}).
We report the Wasserstein-2-distance between the distributions of PCA-projections.


\paragraph{Inference time}
Inference time is, even if orders of magnitude smaller than MD, a critical factor for applications of conformational ensemble generators such as large-scale annotation of datasets or screening of proteins for a target motion.
As in \cite{huguet2024foldflow2}, we evaluate the inference time per generated conformation of the 302-residue protein 7c45A using an NVIDIA A100-40GB GPU.



\subsection{ATLAS benchmark}
\label{sec:benchmark}
We report the performance of BBFlow and the baselines evaluated on the ATLAS test set from AlphaFlow \cite{jing2024alphafoldmeetsflowmatching} in Table~\ref{tab:benchmark}.
We find that BBFlow generates high-quality conformational ensembles faster than all baselines.
For proteins of length 300, it is around 40 times faster than AlphaFlow with templates (AlphaFlow-T), at comparable accuracy.
While AlphaFlow-T is slightly more accurate in terms of RMSF and principal components, BBFlow outperforms it in capturing flexibility quantified by pairwise RMSD and median RMSF.
BBFlow outperforms AlphaFlow, ESMFlow-T, the two distilled models and ConfDiff in almost all metrics while, at the same time, generating the ensembles faster.
Indicated by small median RMSF and pairwise RMSD, AlphaFlow-T and AlphaFlow-T$_\text{12L,dist}$ systematically over-stabilize the proteins.
Also for other metrics from \citet{jing2024alphafoldmeetsflowmatching}, BBFlow is competitive (Table~\ref{tab:app_full_table}).
Additionally, we investigate the performance of BBFlow, AlphaFlow-T and AlphaFlow-T$_\text{12L,dist}$ for different protein lengths (Figure~\ref{fig:benchmark_boxplots}, Figure~\ref{fig:app_full_benchmark_boxplots}) and find that the trends described above hold true for all lengths considered, while the over-stabilization of AlphaFlow-T is particularly prominent for large proteins.


\begin{table*}[h!]
    \centering
    \caption{Performance of BBFlow and baselines for \textit{de novo} proteins. As in Table~\ref{tab:benchmark}, we evaluate the metrics described in Sec.~\ref{sec:metrics} and report the median of all proteins. For median RMSF and pairwise RMSD, we report the reference value in parentheses. All units except for RMSF $r$ are reported in \AA. Errors are shown in parentheses if they are above precision. Best values are \textbf{bold}, second best are \underline{underlined}.}
    \vspace{0.1cm}
    \begin{tabular}{lcccccc}
    \toprule
     & \multicolumn{3}{c}{RMSF} & \multicolumn{2}{c}{Pairwise RMSD} &\\
    \cmidrule(lr){2-4} \cmidrule(lr){5-6}
     & $r$ ($\uparrow$) & MAE ($\downarrow$) & Median (0.91) & MAE ($\downarrow$) & Median (1.59) & PCA $\mathcal{W}_2$ ($\downarrow$)\\
    \midrule
    AlphaFlow & 0.48 & 4.76 (0.01) & 7.09 & 7.40 & 8.08 & 1.63 (0.03) \\
    ConfDiff & 0.62 & 3.82 (0.01) & 6.35 & 7.26 & 7.27 & 1.71 (0.02) \\
    AlphaFlow-T & \textbf{0.89} & \underline{0.25} & \underline{0.74} & \underline{0.38} & \underline{1.25} & \underline{0.66} (0.01) \\
    ESMFlow-T & \textbf{0.89} & 0.28 & 0.68 & 0.43 & 1.20 & \textbf{0.63} (0.01)\\
    AlphaFlow-T$_\text{dist}$ & 0.88 & 0.46 & 0.51 & 0.77 & 0.87 & 0.75  (0.01) \\
    AlphaFlow-T$_\text{12L,dist}$ & 0.87 & 0.58 & 0.41 & 0.97 & 0.68 & 0.75  (0.01) \\
    \midrule
    BBFlow (Ours) & 0.85 & \textbf{0.23} & \textbf{0.86} & \textbf{0.29} & \textbf{1.51} & 0.68 (0.01) \\
    \bottomrule
    \end{tabular}
    % }
    \label{tab:de_novo}
\end{table*}

\begin{figure*}[h]
\centering
\vspace{0.2cm}
\includegraphics[width=2.0\columnwidth]{figures/experiments/combined_box_time.png}
\vspace{-0.2cm}
% \vspace{-1cm}
\caption{Performance of BBFlow, AlphaFlow-T and AlphaFlow-T$_\text{12L,dist}$ on the ATLAS test set for different protein lengths. We divide the protein lengths in three bins and calculate RMSF MAE, the absolute error of pairwise RMSD and PCA $\mathcal{W}_2$ of each protein (lower is better) with length in the respective bin. The boxes depict the 0.25 and 0.75 quantile, minimum, maximum and median of all test proteins.
% The other metrics from Table~\ref{tab:benchmark} can be found in Figure~\ref{fig:app_full_benchmark_boxplots}.
We also show inference time per generated conformation as function of protein length, including the distilled model AlphaFlow-T$_\text{dist}$.
}
\label{fig:benchmark_boxplots}
\end{figure*}

\subsection{De novo proteins}
\label{sec:de_novo}
We hypothesize that BBFlow's greatly reduced inference time for generating high-quality conformational ensembles makes the method interesting for application in protein design pipelines.
% , in which large amounts of generated \textit{de novo} proteins are screened for desired properties.
Efficient generation of conformational ensembles would allow to screen for dynamical properties.
However, since \textit{de novo} proteins often have scarce evolutionary information available, the applicability of models that rely on said evolutionary information is unclear.

In order to evaluate conformational ensembles of \textit{de novo} proteins, we generate a small dataset of 50 proteins sampled with the established models RFdiffusion \cite{watson2023novo} and FrameFlow \cite{yim2024improved}, respectively, and perform three 100\,ns long MD simulations for each, similar to ATLAS (Appendix~\ref{sec:app_de_novo_proteins}).

In Table~\ref{tab:de_novo}, we report the performance of the models considered in Section~\ref{sec:benchmark} for \textit{de novo} proteins.
We find that AlphaFlow, which heavily relies on evolutionary information, experiences a strong decline of performance compared to naturally occurring proteins (Table~\ref{tab:benchmark}).
The relative differences between BBFlow and the other baselines are comparable to the performance on natural proteins, with slight improvements for BBFlow in terms of RMSF MAE and median pairwise RMSD.
We show performance for different protein lengths in Figure~\ref{fig:app_denovo_benchmark}.
Figure~\ref{fig:rmsf_denovo} displays structures and predicted RMSF profiles of four \textit{de novo} proteins.

\begin{table*}[h!]
    \centering
    \caption{Ablation study for key components of BBFlow. The metrics are defined as in Table~\ref{tab:benchmark} and reported in \AA. Errors are calculated as described in Section~\ref{sec:experiments} and displayed in parentheses or left out if they are below the displayed precision.}
    \vspace{0.1cm}
    \begin{tabular}{lccccccccc}
    \toprule
    \multirow{2}{*}{Name} & Cond. & Distance & Direction & Amino & & RMSF & RMSF & Pw-RMSD & PCA \\
    & prior & encoding & encoding & acid enc. & & Med. (1.48) & MAE ($\downarrow$) & MAE ($\downarrow$) & $\mathcal{W}_2$ ($\downarrow$) \\
    \midrule
    BBFlow & \checkmark & \checkmark &            & \checkmark & & 1.48 & 0.47 (0.01) & 0.78 (0.01) & 1.34 (0.03) \\
    \midrule
    a &            & \checkmark &            & \checkmark & & 1.55 & 0.52 (0.01) & 1.15 (0.01) & 1.47 (0.04) \\
    b &            & \checkmark & \checkmark & \checkmark & & 1.30 & 0.48 (0.01) & 0.90 (0.01) & 1.45 (0.04) \\
    c & \checkmark & \checkmark & \checkmark & \checkmark & & 1.25 & 0.42 (0.01) & 0.82 (0.01) & 1.32 (0.03) \\
    d & \checkmark & \checkmark &            &            & & 1.34 & 0.53 (0.01) & 0.92 (0.01) & 1.47 (0.04) \\
    e & \checkmark &            &            & \checkmark & & 8.26 & 5.88 (0.01) & 7.08 (0.01) & 1.32 (0.03) \\
    \bottomrule
    \end{tabular}
    % }
    \label{tab:ablation}
\end{table*}


\begin{figure}[h]
\centering
% \vspace{0.2cm}
\includegraphics[width=1.0\columnwidth]{figures/experiments/de_novo_rmsf_profiles.png}
\vspace{-0.6cm}
\caption{RMSF profiles of \textit{de novo} proteins. We show structures and RMSF profiles predicted by BBFlow and MD of four selected proteins from the dataset of \textit{de novo} proteins along with Pearson correlation $r$ and MAE as reported in Table~\ref{tab:de_novo}.
In the sequential coloring of the structures, blue corresponds to the N-terminus.
}
\label{fig:rmsf_denovo}
\vspace{-0.1cm}
\end{figure}


\subsection{Ablation}
For quantifying the contributions of key components proposed or discussed in this work, we perform an ablation study on the ATLAS dataset and report the results in Table~\ref{tab:ablation}.
We find that using the novel conditional prior instead of the standard unconditional prior benefits all metrics (Table~\ref{tab:ablation} a,b).
The proposed direction embedding (c) decreases the mean absolute error of RMSF but leads to over-stabilization and impedes the performance on other metrics -- it is not used in the BBFlow model referred to in the rest of this work.
Additionally, we train a model that is entirely backbone structure-based, without any sequence information (d), and find that it is on par with non-template AlphaFlow.

We also demonstrate the need for the proposed distance encoding of the equilibrium structure if no evolutionary information is used, by training a model without passing the equilibrium structure encoding but only one-hot encoded amino acid identities as input (e).
While the model drastically overestimates flexibility, we find that, surprisingly, it performs well in PCA Wasserstein-2 distance, which could be explained by our observation that local environments like membership in $\alpha$-helices is accurate but the global fold is not predicted correctly.




\subsection{Discussion}
The results show that BBFlow achieves a favorable trade-off between speed and quality of the generated ensembles.
At comparable accuracy, it is more than an order of magnitude faster than the current state-of-the-art model AlphaFlow-T and also faster than the distilled model AlphaFlow-T$_\text{12L,dist}$.
Crucially, BBFlow does not suffer from the over-stabilization observed in AlphaFlow-T and AlphaFlow-T$_\text{12L,dist}$, which seems to be caused by using templates with AlphaFlow models, impeding the exploration of conformational space.

While using no templates in AlphaFlow is required to avoid over-stabilization, AlphaFlow without templates fails for \textit{de novo} proteins.
As a consequence, BBFlow is the only model considered that accurately captures overall flexibility for \textit{de novo} proteins and improves in relative performance compared to AlphaFlow-T.
We attribute these observations to BBFlow being based entirely on backbone geometry instead of on evolutionary information, which can be scarce for non-natural proteins.

In our ablation study, we find that the proposed conditional prior and distance encoding are crucial for the performance of BBFlow.
We also show that the approach of conditioning on the equilibrium backbone structure can not only be used to eliminate the need of evolutionary information, but even performs well without any sequence information as input.