\onecolumn
\section{Appendix}

\setcounter{figure}{0}
\renewcommand\thefigure{\thesection.\arabic{figure}}
\setcounter{table}{0}
\renewcommand{\thetable}{\thesection.\arabic{table}}
\renewcommand*{\theHtable}{\thetable}
\renewcommand*{\theHfigure}{\thefigure}




\subsection{De novo proteins dataset}
\label{sec:app_de_novo_proteins}

\paragraph{Protein generation}
As described in Section \ref{sec:de_novo}, we assess the performance of BBFlow on a set of \textit{de novo} proteins. We sample 20 protein backbones with FrameFlow \cite{yim2024improved} and RFdiffusion \cite{watson2023novo} for each length $L \in [60, 65, \dots, 512]$. For each individual generated backbone, we carry out a self-consistency evaluation pipeline as previously proposed \cite{yim2023framediff, lin2023genie} by designing 8 sequence with ProteinMPNN \cite{dauparas2022protmpnn} and refolding candidate sequences with ESMfold \cite{lin2023evolutionary}. We then compute the length distribution of the ATLAS dataset and select 50 refolded backbones that have a self-consistency RMSD (scRMSD) of $\le$ 2.0 \AA \, to the originally generated backbone that follow a size distribution similar to the ATLAS dataset \cite{vander2024atlas} each for FrameFlow and RFdiffusion. 


\paragraph{MD setup}
MD simulations are performed using GROMACS v2023~\cite{gromacs}, utilizing the CHARMM27 all-atom force field. Proteins are embedded in a periodic dodecahedron box, ensuring a minimum separation of 1 nm from the box boundaries. The simulation system is hydrated using the TIP3P water model~\cite{tip3p_1983}, and the ionic strength is adjusted to a NaCl concentration of 150 mM. An initial energy minimization is carried out for 5000 steps.
The system undergoes NVT equilibration for 1 ns with a timestep of 2 fs, employing the leap-frog integrator. Temperature control is achieved at 300K using the Berendsen thermostat. This is followed by NPT equilibration for 1 ns, where the pressure is maintained at 1 bar using the Parrinello-Rahman barostat.
The production run of the simulation extends over three 100 ns replicas. Throughout the simulations, covalent bonds involving hydrogen are constrained using the LINCS algorithm~\cite{hess_p-lincs_2008}. Long-range electrostatic interactions are treated using the Particle-Mesh Ewald (PME) method.

\subsection{ConfDiff inference setup}
\label{sec:confdiff}
We evaluate ConfDiff using the ConfDiff-OF-r3-MD model, which is fine-tuned on the ATLAS dataset, available on GitHub\footnote{\url{https://github.com/bytedance/ConfDiff}, commit 9cfae1c14121e423d8d455d03506c7e8ee580e48}.
We use the default hyperparameters for generating conformations.

\begin{figure*}[h]
\centering
\includegraphics[width=0.8\textwidth]{figures/appendix/combined_metrics_vs_length_bin.png}
\caption{Additional metrics for the performance of BBFlow, AlphaFlow-T and AlphaFlow-T$_\text{12L,dist}$ on the ATLAS test set for different protein lengths. We divide the protein lengths in three bins and calculate per-residue RMSF, RMSF MAE, RMSF correlation $r$, per-protein RMSD, the absolute error of pairwise RMSD and PCA $\mathcal{W}_2$ of each protein with length in the respective bin. The boxes depict minimum, maximum, median, and the 0.25 and 0.75 quantile.
}
\label{fig:app_full_benchmark_boxplots}
\end{figure*}

\begin{figure*}[h]
\centering
\includegraphics[width=0.8\textwidth]{figures/appendix/combined_metrics_vs_length_bin_denovo.png}
\caption{Performance of BBFlow, AlphaFlow-T and AlphaFlow-T$_\text{12L,dist}$ on the \textit{de novo} protein dataset for different protein lengths. We divide the protein lengths in three bins and calculate per-residue RMSF, RMSF MAE, RMSF correlation $r$, per-protein RMSD, the absolute error of pairwise RMSD and PCA $\mathcal{W}_2$ of each protein with length in the respective bin. The boxes depict minimum, maximum, median, and the 0.25 and 0.75 quantile.
}
\label{fig:app_denovo_benchmark}
\end{figure*}

\subsection{Further performance on the ATLAS test set}
\label{sec:app_ATLAS_performance}
We show an extension of Figure~\ref{fig:benchmark_boxplots} to all metrics from Table~\ref{tab:benchmark} in Figure~\ref{fig:app_full_benchmark_boxplots}.
In Table~\ref{tab:app_full_table}, we report the performance of BBFlow and baselines on all metrics used in \cite{jing2024alphafoldmeetsflowmatching}.


\begin{table*}[h]
    \centering
    \begin{tabular}{l|ccccc|c}
    \toprule
     & AlphaFlow & AlphaFlow-T & ESMFlow-T & AlphaFlow-T$_\text{12L,dist}$ & ConfDiff & BBFlow \\
    \midrule
    Pairwise RMSD (=2.9) & 2.89 & 2.18 & 2.00 & 1.40 & 3.43 & 2.76 \\
    Pairwise RMSD $r$ & 0.48 & 0.94 & 0.85 & 0.76 & 0.62 & 0.83 \\
    C$_\alpha$ RMSF (=1.48) & 1.51 & 1.17 & 0.94 & 0.68 & 2.00 & 1.48 \\
    Global RMSF $r$ & 0.58 & 0.91 & 0.84 & 0.74 & 0.70 & 0.83 \\
    Per target RMSF $r$ & 0.86 & 0.92 & 0.92 & 0.90 & 0.88 & 0.89 \\
    \midrule
    Root mean $\mathcal{W}_2$-distance & 2.37 & 1.72 & 1.91 & 2.13 & 2.56 & 2.07 \\
    -- Translation contrib. & 2.16 & 1.47 & 1.52 & 1.73 & 2.02 & 1.82 \\
    -- Variance contrib. & 1.18 & 0.82 & 0.92 & 1.20 & 1.22 & 0.94 \\
    MD PCA $\mathcal{W}_2$-distance & 1.47 & 1.28 & 1.48 & 1.60 & 1.41 & 1.34 \\
    Joint PCA $\mathcal{W}_2$-distance & 2.25 & 1.58 & 1.75 & 1.93 & 2.19 & 1.90 \\
    \midrule
    \% PC-sim $> 0.5$ & 43.70 & 44.72 & 47.95 & 39.09 & 37.72 & 44.82 \\
    Weak contacts $J$ & 0.62 & 0.62 & 0.59 & 0.56 & 0.63 & 0.54 \\
    Transient contacts $J$ & 0.41 & 0.47 & 0.47 & 0.24 & 0.39 & 0.35 \\
    \bottomrule
    \end{tabular}
    \caption{Evaluation on the ATLAS dataset with the metrics from \cite{jing2024alphafoldmeetsflowmatching}. RMSF and RMWD are calculated from C$\alpha$ atoms.}
    \label{tab:app_full_table}
\end{table*}

\subsection{PCA plots}
For all proteins in the ATLAS test set, we visualize the PCA components of MD and conformations generated with BBFlow and AlphaFlow-T in Figure~\ref{fig:app_pca_components_atlas}.

\begin{figure*}
\centering
\includegraphics[width=0.8\textwidth]{figures/appendix/pca_components_atlas.png}
\caption{PCA components for the ATLAS test set. We show the first (horizontal axis) and second (vertical axis) principal components of MD, and the projection of conformations generated with BBFlow and AlphaFlow-T on these principal components.
}
\label{fig:app_pca_components_atlas}
\end{figure*}



\subsection{Timestep Analysis}

We also perform a timesteps analysis for BBFlow and report the results in~\ref{tab:app_timestep_analysis}.

\begin{table*}[h!]
    \centering
    \begin{tabular}{lccccccc}
    \toprule
     & \multicolumn{3}{c}{RMSF} & \multicolumn{2}{c}{Pairwise RMSD} & &\\
    \cmidrule(lr){2-4} \cmidrule(lr){5-6}
    Timesteps & $r$ ($\uparrow$) & MAE ($\downarrow$) & Median (1.48) & MAE  ($\downarrow$) & Median (2.90) & PCA $\mathcal{W}_2$  ($\downarrow$) & Time [s] ($\downarrow$) \\
    \midrule
    5 & 0.74 & 3.33 (0.01) & 5.12 & 4.17 (0.01) & 7.82 & 1.58 (0.03) & 0.3 \\
    10 & 0.89 & 0.48 (0.01) & 1.42 & 0.85 (0.01) & 2.62 & 1.40 (0.03) & 0.5 \\
    20 & 0.89 & 0.47 & 1.48 & 0.78 (0.01) & 2.76 & 1.34 (0.03) & 0.9 \\
    50 & 0.89 & 0.48 & 1.53 & 0.74 (0.01) & 2.92 & 1.34 (0.03) & 2.1 \\
    \bottomrule
    \end{tabular}
    \caption{Performance of BBFlow on the ATLAS test set for different number of timesteps. Metrics, errors and units as in Table~\ref{tab:benchmark}.}
    \label{tab:app_timestep_analysis}
\end{table*}


