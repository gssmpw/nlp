\subsection{Reproducibility of \eval Backbone}
To assess whether \eval produces consistent results across different backbone LLMs, we analyze two open-source models (DeepSeek-R1-8B and Llama-3.1-8B) against the GPT-4o-mini, which is the original backbone of \eval. Specifically, we measure how closely the evaluation scores from each open-source model align with those from GPT-4o-mini by calculating pairwise correlation.
From the results in Table~\ref{tab:apx_evalbaseline}, we can observe that even when replacing the backbone with open-source models, the high correlation persists, demonstrating that our evaluation method is both reproducible and robust.

\begin{table}[htb]
\centering
\begin{small}
\renewcommand{\arraystretch}{1.0}

\begin{tabularx}{\linewidth}{%
l
>{\centering\arraybackslash}X
>{\centering\arraybackslash}X}  

\toprule
\textbf{Backbone} & \textbf{Faithfulness}& \textbf{Completeness} \\ \midrule

DeepSeek R1-8B     & 74.12 & 77.85 \\
Llama-3.1-8B       & 80.78 & 72.82 \\

\bottomrule
\end{tabularx}
\end{small}
\caption{Pearson correlation scores varying \eval backbone compared to GPT-4o mini.}
\label{tab:apx_evalbaseline}
\end{table}

\subsection{Effect of Parameter Scaling on \bench Performance}
From the results in Table~\ref{tab:generator}, we observe that proprietary models with larger model sizes generally show higher performance compared to open-source models that have relatively fewer parameters. 
To further investigate this, we conduct additional experiments by differ the parameter sizes of open-source LLM to understand the effect of model size in the table reasoning performance.
Specifically, we evaluate DeepSeek-R1, which is most powerful open-source LLM among our baselines ranging from 8B to 70B parameters.
From the results in Table~\ref{tab:apx_deepseek}, we confirm a general trend where increasing model size correlates with improved performance, with a particularly notable improvement in the faithfulness score.
This discrepancy suggests that the model’s capacity to accurately interpret and reason about table structures (which is a prerequisite for faithfulness) is more directly enhanced by parameter scaling, compared to the broader task coverage implied by completeness.
\definecolor{tab_gray}{HTML}{7d7d7d}
\newcommand{\downlogo}[2][{-1em}]{%
  \raisebox{#1}{\includegraphics[width=0.8em,keepaspectratio]{#2}}%
}

\begin{table}[htb]
\centering
\begin{small}
\renewcommand{\arraystretch}{1.0}

\begin{tabularx}{\linewidth} & \textbf{Score} & \textbf{$\Delta$\%} \\
\midrule

Distill-Llama-8B  & 35.55 & \textcolor{tab_gray}{-} & 60.96 & \textcolor{tab_gray}{-} \\ \midrule
Distill-Qwen-14B  & 37.26 & \textcolor{tab_gray}{+ 4.81} & 61.07 & \textcolor{tab_gray}{+ 0.18} \\
Distill-Qwen-32B  & \textbf{39.02} & \textcolor{tab_gray}{+ 9.76} & 60.62 & \textcolor{tab_gray}{\textminus\;0.56} \\
Distill-Llama-70B & 38.11 & \textcolor{tab_gray}{+ 7.20} & \textbf{62.13} & \textcolor{tab_gray}{+ 1.92} \\

\bottomrule
\end{tabularx}
\end{small}
\caption{\eval results varying parameter size of DeepSeek-R1. $\Delta$\% denotes the relative improvement in performance compared to the Distill-Llama-8B.}
\label{tab:apx_deepseek}
\end{table}