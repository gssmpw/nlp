\begin{table*}[htbp]
\centering
\begin{small}
\begin{tabularx}{\textwidth}{%
>{\arraybackslash}p{0.15\textwidth}
>{\arraybackslash}X
}
\toprule
\multicolumn{2}{l}{\textbf{\eval Completeness : GPT-4o (score:} 68.97\textbf{) v.s. Qwen2-7B (score:} 9.84\textbf{)}} \\ \midrule
\textbf{Question}
& Summarize the performance of all Belarusian clubs in the European Cups across the 2006, 2007, and 2008 seasons, focusing on the rounds they reached and the teams they faced in each leg of the competition. \\ \midrule
\textbf{Insight}
& \textcolor{cyan}{Across the 2006, 2007, and 2008 seasons, Belarusian clubs had varied performances in European Cups}. \textcolor{cyan}{In 2006, MTZ-RIPO Minsk participated} in the UEFA Intertoto Cup, losing to Shakhter Karagandy in the first round and subsequently to Moscow in the second round. \textcolor{cyan}{Dinamo Minsk and BATE Borisov also competed} in the UEFA Cup, with Dinamo Minsk eliminated by Artmedia Bratislava and BATE Borisov losing to Rubin Kazan. \textcolor{cyan}{Shakhtyor Soligorsk faced} defeat in the UEFA Champions League against Široki Brijeg. \textcolor{cyan}{In 2007, Shakhtyor Soligorsk advanced} past Ararat Yerevan in the Intertoto Cup but was eliminated by Chernomorets Odessa in the second round. \textcolor{cyan}{Dinamo Brest and Dinamo Minsk had mixed results} in the UEFA Cup, with Dinamo Brest advancing past Liepājas Metalurgs but Dinamo Minsk failing against Odense. \textcolor{cyan}{BATE Borisov faced} Villarreal in the UEFA Cup and was eliminated, while in the Champions League, they had a narrow defeat against APOEL Nicosia but managed to progress past Hafnarfjarðar before losing to Steaua Bucureşti. \textcolor{cyan}{In 2008, Shakhtyor Soligorsk} was eliminated in the Intertoto Cup by Cracovia, while Gomel and MTZ-RIPO Minsk both lost in the UEFA Cup qualifying rounds. \textcolor{cyan}{BATE Borisov had a notable run} in the Champions League, defeating Valur and reaching the group stage, where they faced tough competition from Real Madrid, Juventus, and Zenit St. Petersburg, ultimately finishing with mixed results. \textcolor{cyan}{Overall, Belarusian clubs struggled to advance past the early rounds in most competitions during these seasons, with BATE Borisov achieving the most significant success by reaching the group stage of the Champions League in 2008.} \\ \midrule
\textbf{GPT-4o}
& \textcolor{cyan}{Across the 2006, 2007, and 2008 seasons, Belarusian clubs had mixed performances in European Cups}, with BATE Borisov emerging as the most successful team. \textcolor{cyan}{In 2006, BATE} reached the second qualifying round of the UEFA Cup, defeating Nistru Otaci (3-0 aggregate) but losing to Rubin Kazan (5-0 aggregate). \textcolor{cyan}{Shakhtyor Soligorsk} exited in the first qualifying round of the UEFA Champions League against Široki Brijeg (0-2 aggregate), while \textcolor{cyan}{MTZ-RIPO Minsk advanced} past Shakhtyor Karagandy in the Intertoto Cup first round (6-4 aggregate) but fell to Moscow in the second round (3-0 aggregate). \textcolor{cyan}{In 2007, BATE reached} the third qualifying round of the UEFA Champions League, defeating APOEL Nicosia (3-2 aggregate) and Hafnarfjarðar (4-2 aggregate) before losing to Steaua Bucureşti (4-2 aggregate). In the UEFA Cup, \textcolor{cyan}{Dinamo Minsk} defeated Skonto Riga (3-1 aggregate) in the first qualifying round but lost to Odense (5-1 aggregate) in the second, while Dinamo Brest was eliminated by Liepājas Metalurgs (3-2 aggregate) in the first qualifying round. \textcolor{cyan}{Shakhtyor Soligorsk advanced} to the second round of the Intertoto Cup, beating Ararat Yerevan (4-3 aggregate) but losing to Chernomorets Odessa (6-2 aggregate). \textcolor{cyan}{In 2008, BATE had a historic run} in the UEFA Champions League, progressing from the first qualifying round by defeating Valur (3-0 aggregate), then overcoming Anderlecht (4-3 aggregate) and Levski Sofia (2-1 aggregate) to reach the group stage, where they faced Real Madrid, Juventus, and Zenit St. Petersburg, earning two draws but failing to advance. In the UEFA Cup, \textcolor{cyan}{MTZ-RIPO Minsk} lost to Žilina (3-2 aggregate) in the first qualifying round, while Gomel was eliminated by Legia Warszawa (4-1 aggregate). \textcolor{cyan}{Shakhtyor Soligorsk participated} in the Intertoto Cup, defeating Cracovia (5-1 aggregate) in the first round but losing to Sturm Graz (2-0 aggregate) in the second. \textcolor{cyan}{Overall, BATE's group stage appearance in 2008 marked a significant achievement for Belarusian football, while other clubs struggled to progress beyond the early rounds.} \\ \midrule
\textbf{Qwen2-7B}
& \textcolor{cyan}{In the European Cups across the 2006, 2007, and 2008 seasons, Belarusian clubs displayed a commendable performance, advancing to various stages of the tournaments}. \textcolor{cyan}{In 2006, BATE Borisov} reached the UEFA Champions League group stage, defeating Anderlecht and Juventus, while Shakhtyor Soligorsk was eliminated in the UEFA Intertoto Cup's second round. \textcolor{cyan}{In 2007, BATE Borisov made it} to the UEFA Champions League group stage, overcoming APOEL Nicosia, Hafnarfjarðar, and Steaua Bucureşti, and Shakhtyor Soligorsk advanced to the UEFA Intertoto Cup's second round. \textcolor{cyan}{In 2008, BATE Borisov reached} the UEFA Champions League group stage, facing Real Madrid, Juventus, and Zenit St. Petersburg, and Shakhtyor Soligorsk was eliminated in the UEFA Intertoto Cup's first round. \textcolor{cyan}{These results highlight the competitive spirit and skill of Belarusian clubs in European football}. \\ \midrule
\textbf{Comparative}
& \textcolor{cyan}{GPT-4o provides a thorough breakdown of how Belarusian clubs performed in European competitions} \\
\textbf{analysis}
& \textcolor{cyan}{during the 2006, 2007, and 2008 seasons}, detailing the rounds they reached, the teams they faced, and the exact match results. In contrast, \textcolor{purple}{Qwen2-7B gives a much more general summary, primarily focusing on BATE Borisov while omitting other clubs like Shakhtyor Soligorsk and Dinamo Minsk, leaving out key details} such as specific opponents, progression through different stages, and match outcomes. GPT-4o delivers a far richer response by fully covering each season’s results and matchups, making it a much stronger answer. On the other hand, Qwen2-7B’s response lacks depth due to its narrow focus on BATE and contains inaccuracies (e.g., claiming BATE defeated Juventus in 2006), resulting in a lower score. \\
\bottomrule
\end{tabularx}
\end{small}
\caption{Comparison of \eval Completeness results between GPT-4o and Qwen2-7B.}
\label{case:comparison_ii}
\end{table*}