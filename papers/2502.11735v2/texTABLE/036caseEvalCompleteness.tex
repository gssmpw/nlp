\begin{table*}[htbp]
\centering
\begin{small}
\begin{tabularx}{\textwidth}{%
>{\arraybackslash}p{0.15\textwidth}
>{\arraybackslash}X
}
\toprule
\multicolumn{2}{l}{\textbf{\eval Completeness score :} 69.57} \\ \midrule
\multicolumn{2}{l}{\textbf{Step : Question-aware insight decomposition}} \\ \midrule
\textbf{Question}
& Summarize the attendance figures for the 1968 Buffalo Bills season and the 1969 Buffalo Bills season, focusing on the games against the Houston Oilers and Oakland Raiders. What might these figures indicate about fan engagement and support for the team during these years? \\ \midrule
\textbf{Ground-truth}
& The attendance figures for the 1968 Buffalo Bills season against the Houston Oilers were 34,339 on \\
\textbf{insight}
& October 27 and 34,110 on December 7, while the games against the Oakland Raiders drew 43,056 on September 15 and 39,883 on November 28, indicating a moderate level of fan engagement despite the team's struggles that season. In contrast, the 1969 season saw increased attendance figures against the Houston Oilers, with 40,146 on September 21 and 46,485 on October 5, and a significant turnout of 54,418 for the game against the Oakland Raiders on October 19. These rising attendance numbers from 1968 to 1969 suggest a growing support and enthusiasm for the team, reflecting an improvement in fan engagement as the Bills likely began to show better performance or potential, which could have contributed to increased interest and attendance at their games. \\ \midrule
\textbf{Decomposed}
& 1. Attendance figures for 1968 Buffalo Bills season. \\
\textbf{ground-truth}
& 2. Attendance figures for 1969 Buffalo Bills season. \\
\textbf{topic set}
& 3. Attendance figures against Houston Oilers in 1968. \\
& 4. Attendance figures against Oakland Raiders in 1968. \\
& 5. Attendance figures against Houston Oilers in 1969. \\
& 6. Attendance figures against Oakland Raiders in 1969. \\
& 7. Moderate level of fan engagement in 1968. \\
& 8. Increased attendance figures in 1969. \\
& 9. Growing support for the team. \\
& 10. Improvement in fan engagement. \\
& 11. Team performance impact on attendance. \\
& 12. Enthusiasm for the team. \\ \midrule
\textbf{Predicted}
& In the 1968 Buffalo Bills season, the home game against the Houston Oilers on October 27 drew an \\
\textbf{insight}
& attendance of 34,339, while the away game against the Oakland Raiders on November 28 had 39,883 attendees. In the 1969 season, the home game against the Houston Oilers on September 21 saw a slightly higher attendance of 40,146, and the away game against the Oakland Raiders on October 19 attracted 54,418 fans. These figures suggest that despite the Bills' poor performance in both seasons, with records of 1-12-1 in 1968 and 4-10 in 1969, there was a modest increase in attendance for these matchups, particularly in 1969. The higher attendance for the away games, especially against the Raiders, may reflect stronger fan engagement in cities with more competitive teams, while the relatively lower home attendance could indicate waning local support due to the Bills' struggles on the field. This trend highlights how team performance and the competitiveness of opponents can influence fan turnout and engagement. \\ \midrule
\textbf{Decomposed}
& 1. Attendance figures for 1968 Buffalo Bills season. \\
\textbf{predicted}
& 2. Attendance figures for 1969 Buffalo Bills season. \\
\textbf{topic set}
& 3. Attendance at home game against Houston Oilers. \\
& 4. Attendance at away game against Oakland Raiders. \\
& 5. Comparison of attendance figures between seasons. \\
& 6. Modest increase in attendance for matchups. \\
& 7. Higher attendance for away games. \\
& 8. Fan engagement in cities with competitive teams. \\
& 9. Lower home attendance for Buffalo Bills. \\
& 10. Impact of team performance on local support. \\
& 11. Influence of opponent competitiveness on fan turnout. \\ \midrule
\multicolumn{2}{l}{\textbf{Step : Topic matching}} \\ \midrule
\textbf{Ground-truth}
& \textit{As referenced above} \\
\textbf{topic set} & \\ \midrule
\textbf{Predicted}
& \textit{As referenced above} \\
\textbf{topic set} & \\ \midrule
\textbf{Matched}
& [1, 2, 3, 4, 5, 6, 8, 11] \\
\textbf{ground-truth} & \\
\textbf{topic subset} & \\ \midrule
\textbf{Matched}
& [1, 2, 3, 4, 5, 6, 10, 11] \\
\textbf{predicted} & \\
\textbf{topic subset} & \\
\bottomrule
\end{tabularx}
\end{small}
\caption{\eval Completeness evaluation example for each step based on GPT-4o.}
\label{case:eval_comp}
\end{table*}