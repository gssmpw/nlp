\begin{table*}[htbp]
\centering
\begin{small}
\begin{tabularx}{\textwidth}{%
>{\arraybackslash}p{0.15\textwidth}
>{\arraybackslash}X
}
\toprule
\multicolumn{2}{l}{\textbf{\eval Faithfulness : Claude 3.5 Sonnet (score:} 60.53\textbf{) v.s. Llama 3.1-8B (score:} 24.32\textbf{)}} \\ \midrule
\textbf{Tables}
& \textcolor{cyan}{\textbf{Table 1}} \\
& [TITLE] 1970 John Player League - Batting averages - Batting averages \\
& [HEADER] Player | County | Matches | Innings | Runs | Average | Highest\_Score | \_100s | \_50s \\
& [ROW 1] Barry Richards Category:Articles with hCards | Hampshire | 12 | 11 | 592 | 65.77 | 155* | 3 | 0 \\
& [ROW 2] Clive Lloyd Category:Articles with hCards | Lancashire | 14 | 14 | 521 | 57.88 | 134* | 1 | 2 \\
& [ROW 3] Harry Pilling Category:Articles with hCards | Lancashire | 16 | 15 | 625 | 52.08 | 85 | 0 | 5 \\
& ... \\
& \textcolor{purple}{\textbf{Table 2}} \\
& [TITLE] Somerset County Cricket Club in 2010 - Friends Provident t20 - Batting averages \\
& [HEADER] Player | Matches | Innings | Runs | Average | Strike\_rate | Highest\_Score | \_50s \\
& [ROW 1] James Hildreth Category:Articles with hCards | 19 | 19 | 459 | 32.78 | 110.60 | 77* | 2 \\
& [ROW 2] Kieron Pollard Category:Articles with hCards | 17 | 16 | 354 | 32.18 | 175.24 | 89* | 2 \\
& [ROW 3] Marcus Trescothick Category:Articles with hCards | 19 | 19 | 572 | 31.77 | 157.14 | 83 | 6 \\
& ... \\
& \textcolor{purple}{\textbf{Table 10}} \\
& [TITLE] Somerset County Cricket Club in 2009 - Match logs and statistics - Bowling averages \\
& [HEADER] Player | Matches | Overs | Wickets | Average | Economy | BBI | \_4wi \\
& [ROW 1] Ben Phillips Category:Articles with hCards | 8 | 47.0 | 14 | 14.00 | 4.17 | 3/24 | 0 \\
& [ROW 2] Charl Willoughby Category:Articles with hCards | 6 | 42.4 | 11 | 19.18 | 4.94 | 3/36 | 0 \\
& [ROW 3] Zander de Bruyn Category:Articles with hCards | 8 | 32.4 | 6 | 28.33 | 5.20 | 4/20 | 1 \\
& ... \\ \midrule
\textbf{Question}
& What can be inferred about the batting styles and strategies of players in the 1970 John Player League versus those in the Somerset County Cricket Club in 1891 based on their average scores and highest scores? \\ \midrule
\textbf{Claude 3.5 Sonnet}
& \textcolor{cyan}{The data provided does not include information about the Somerset County Cricket Club in 1891, so a direct comparison between that year and the 1970 John Player League is not possible}. However, we can draw insights from the 1970 John Player League data. \textcolor{cyan}{In the 1970 season, batsmen like Barry Richards and Clive Lloyd demonstrated exceptional batting prowess, with Richards averaging an impressive 65.77 runs and Lloyd 57.88 runs}. These high averages, coupled with their ability to score centuries (Richards had 3 and Lloyd had 1), suggest that the top batsmen of that era were capable of both consistency and explosive performances. \textcolor{cyan}{The presence of multiple players with averages above 40 and the ability to score centuries or half-centuries} indicates that the 1970 John Player League featured aggressive batting strategies, with players aiming for high individual scores while maintaining solid averages throughout the season. This style of play likely contributed to entertaining matches and high team totals, reflecting the evolving nature of limited-overs cricket in that period. \\ \midrule
\textbf{Llama 3.1-8B}
& The batting styles and strategies of players in the 1970 John Player League and \textcolor{purple}{the Somerset County Cricket Club in 2010 are distinct from those in the 1969 Player's County League and the Somerset County Cricket Club in 2009, as evidenced by their average scores and highest scores}. In the 1970 John Player League, players like Barry Richards and Clive Lloyd achieved high average scores and highest scores, indicating a more aggressive batting approach. In contrast, \textcolor{purple}{players in the Somerset County Cricket Club in 2010, such as James Hildreth and Marcus Trescothick, had lower average scores and highest scores, suggesting a more defensive approach}. This difference in batting style is also reflected in the \textcolor{purple}{1969 Player's County League, where players like Keith Boyce and Peter Sainsbury had high average scores and highest scores}, indicating an aggressive approach. However, \textcolor{purple}{the Somerset County Cricket Club in 2009, with players like Zander de Bruyn and Craig Kieswetter, had lower average scores and highest scores}, suggesting a more defensive approach. This contrast highlights the evolution of batting styles and strategies over time, with teams and players adapting to different conditions and opponents. \textcolor{purple}{The Somerset County Cricket Club in 2010, with its more defensive \textit{...Repeat similar contents...}} \\ \midrule
\textbf{Comparative}
& \textcolor{cyan}{Claude 3.5 Sonnet’s response clearly establishes the premise that “a conclusion cannot be drawn due} \\
\textbf{analysis}
& \textcolor{cyan}{to the lack of data from 1891”} and maintains consistency by focusing solely on the statistics within the 1970 John Player League. In contrast, \textcolor{purple}{Llama 3.1-8B’s response expands the scope excessively by referencing multiple seasons and tournaments (1969, 2009, 2010, etc.), including information that is not directly relevant to the original question}, which compares 1891 to 1970. As a result, while neither response reaches a perfect conclusion (due to retrieved tables), Claude 3.5 Sonnet’s answer aligns more closely with the intent of the question by acknowledging the limitations of the available data and maintaining a coherent interpretation. \\
\bottomrule
\end{tabularx}
\end{small}
\caption{Comparison of \eval Faithfulness results between Claude 3.5 Sonnet and Llama 3.1-8B.}
\label{case:comparison_i}
\end{table*}