\subsection{Single-Agent Long-Term Fairness.}
To overcome the limitations of static fairness formulations, several approaches have re-framed fairness as a dynamic systems problem. Effort-based fairness analyzes the differing efforts required by groups to achieve outcomes~\cite{heidari2019long, guldogan2022equal}, while causal models use structural causal models and interventions to introduce fairness~\cite{hu2020fair, hu2022achieving}. Another approach incorporates fairness within dynamic systems through reinforcement learning (RL), with early work using multi-armed bandits~\cite{joseph2016fairness} and recent efforts employing Markov Decision Processes (MDPs). Puranik et al.~\cite{puranik2022dynamic} introduce the Fair-Greedy policy in an admissions case study, balancing applicants’ scores with group proportions. Yin et al.~\cite{yin2024long} frame the long-term fairness RL problem to maximize profits while minimizing unfairness, measured by regret and distortion. To address temporal bias, Xu et al.~\cite{xuadapting} propose a fairness measure based on the ratio-after-aggregation and modify the proximal policy optimization algorithm (PPO) to satisfy this constraint. Though these works reduce temporal disparities, they do not analyze their source. Deng et al.~\cite{deng2024hides} use causal analysis to trace sources of inequality over time. While these works extend static fairness to long-term outcomes, Hu et al.~\cite{hu2023striking} argue that long-term fairness should focus on the convergence of input feature distributions, proposing a PPO variant with pre-processing and regularization to balance short- and long-term fairness.

\subsection{Multi-Agent Long-Term Fairness.} 
In systems with multiple decision-making entities, modeling fairness explicitly across agents becomes crucial for understanding their interventions and their effects on system dynamics. Several studies have explored fairness in multi-agent contexts. Jiang and Lu~\cite{jiang2019learning} introduce the Fair-Efficient Network, a hierarchical RL model where homogeneous agents aim to balance fairness and efficiency. Zheng et al.~\cite{zheng2022ai} use two-level deep RL to design agents that reduce income inequality via taxation and redistribution, with equity measured by the Gini Index. Reuel and Ma~\cite{reuel2024fairness} provide a survey on fairness in RL, covering both single- and multi-agent systems. They highlight key gaps, such as fairness in RL from human feedback, and emphasize the challenges of ensuring fairness in dynamic real-world environments, which underscores the need for realistic simulation environments.

\subsection{Long-Term Fairness Environments.}
A major challenge in long-term fairness research is designing appropriate environments for measuring, simulating, and assessing fairness algorithms. Among the growing body of research on long-term fairness, some works have introduced environments that consider the complexities of real-world decision-making. For example, D’Amour et al.~\cite{d2020fairness} introduce lending and attention environments, while Atwood et al.~\cite{atwood2019fair} focus on infectious disease environments. However, these environments are single-agent based. Real-world systems, by contrast, often consist of multiple interacting entities that influence outcomes. By not explicitly modeling these entities as agents, such environments limit the ability to flexibly analyze the various forms of intervention and the effects that these different entities may have on the system’s underlying dynamics.

Although there are existing multi-agent fair environments, such as those developed by Jiang and Lu~\cite{jiang2019learning}, their approach is limited to focusing on fairness among homogeneous agents. By modeling fairness at the agent level rather than for a broader population, their environments lack the necessary structure to analyze group fairness. Additionally, their environments are simpler compared to real-world social systems, where stakeholders in fields like healthcare and finance have diverse decision-making processes. In contrast, our proposed framework supports heterogeneous agents with varied decision-making strategies and emphasizes group fairness, ensuring equitable outcomes across demographic groups and enabling a more comprehensive analysis of societal impacts. Furthermore, while Zheng et al.’s~\cite{zheng2022ai} environment offers a detailed model, its context is restricted to economic outcomes. Our framework spans multiple domains---including loan allocation, healthcare, and education---each requiring tailored approaches and supporting multiple fairness measures across diverse contexts.