\label{sec::general_MAFE}

While each of our MAFEs has unique elements, they also share several common structural characteristics derived from their Fair Dec-POMPs. In this section, we outline the key similarities in their designs.

\subsection{Observations}

At a given time step, $t$, Agent $n$ receives an observation $o_{n,t} \subseteq \mathcal{O}_n$. We design the observation space for every agent in each of our environments to take the following form, $\mathcal{O}_n = \{o | o \in \Pi_{m=0}^M \mathbb{R}^{m\times k_n}\}$. Here, $M$ represents the global population size in a given MAFE and $k_n$ denotes the dimensionality of the feature vector associated with each individual containing the features that Agent $n$ can use when deciding on an action.

Moreover, while there may be overlap in the features provided to different agents, this is not guaranteed. As a result, the size of the feature vector $k_n$ varies across agents. For instance, an employer agent may have access to an individual’s undergraduate GPA when determining salary offers, but this feature would not be available to a university admissions agent, since high school students do not have an undergraduate GPA.

\subsection{Actions}

Agent actions take the general form $\mathcal{A}_n = \{a | a \in \Pi_{m=0}^M \mathbb{R}^{m}\}$. There are two particular categories of actions that serve as special instances of this structure: (1) \textbf{individual-level actions} and (2) \textbf{group-level} actions. 

For Agent $n$ with observation matrix, $o_{n,t}$, of size $m_{n,t}\times k_n$, an individual-level action takes the form $a_{n,t}\in\mathbb{R}^{m_{n,t}}$. In this case, Agent $n$ produces an action vector, where the $i^{th}$ element corresponds to a decision for the $i^{th}$ individual, whose feature vector is represented by the $i^{th}$ row of $o_{n,t}$. For instance, in the Healthcare MAFE, the Hospital agent could generate an action vector in which each element represents the priority rank assigned to an individual, determining their position in the queue for receiving an available hospital bed.

In contrast, a group-level action affects a subset of individuals in the entire population (subset of the rows of the  observation matrix). The structure of a group-level action is $a_{n,t}\in\mathbb{R}^{f_n}$, where $f_n$ represents the number of decisions Agent $n$ must make, which affect all $m_{n,t}$ individuals. For example, in the Loan MAFE, the Debt Management Agent could output a single percentage value that determines the debt adjustment percentage applied to every customer's payment at that time step. In this case the group is the entire customer repayment population.


\subsection{Agents}

A MAFE is defined as a fair Dec-POMDP, where the decentralization reflects the interaction of $N$ agents with the environment through their respective input actions and output observations, rewards, and fairness components. Specifically, $N$ agents correspond to $N$ distinct input actions provided to the environment and $N$ corresponding output observations, reward component vectors, and fairness component vectors generated by the environment. This decentralization does not necessarily mean that $N$ separate models must be used to generate the actions for each agent, though.

For instance, the $N$ observations, $\{o_{n,t}\}$, could be aggregated into a single global observation, processed by a single AI model, which outputs a unified action vector. This vector can then be split into $N$ individual,  actions, $\{a_{n,t}\}$—one for each agent—before being input back into the environment. Alternatively, in a fully decentralized setup, $N$ separate models can process the individual observations independently to generate $N$ actions. A hybrid setup might involve partial aggregation of observations, with subsets of agents sharing models. Thus, while the environment enforces decentralization in terms of interactions with agents, the AI model architecture (centralized, decentralized, or hybrid) remains a design choice and is independent of the underlying MAFE formulation.

However, we require $a_{n,t}$ to be permutation-equivariant with respect to the rows of $o_{n,t}$. For global-level actions, permutation-equivariance ensures that the arbitrary ordering of the rows in an observation does not affect the global decision applied to all individuals influenced by the action. For individual-level actions, permutation-equivariance guarantees that the $i^{th}$ element of the action vector corresponds to the decision for the $i^{th}$ individual in the agent’s observation matrix, rather than being associated with any other individual.

\subsection{Sensitive Attribute}

The sensitive attribute refers to the feature for which bias mitigation is necessary, as measured using the binary or $D$-ary metrics defined in Equations~\ref{eq::binary_measure} and~\ref{eq::std_measure} in Section~\ref{sec::fair_struct}. In the Loan and Education MAFEs, the sensitive attribute is a binary feature indicating whether an individual belongs to an advantaged or disadvantaged group. In the Loan MAFE, this could represent attributes such as sex or race, both of which are protected characteristics under U.S. anti-discrimination laws in financial institutions~\cite{FDIC}. Similarly, in the Education MAFE, the sensitive attribute reflects whether an individual belongs to an underrepresented minority group at the university level.

In contrast, the Healthcare MAFE underscores that much of the disparity in health outcomes across demographic groups is driven by geographic location. For example, families of color—particularly Black families—are more likely to live in areas with limited access to healthcare facilities~\cite{hhs2024blackhistory}. In this context, geographic location serves as the sensitive attribute, with four distinct regions, each associated with specific health outcome disparities.

\vspace{-2mm}

\subsection{Reward and Fairness Component Functions}
\label{sec::component_func_remark}
In the MAFE framework, the use of component functions for reward and fairness allows for greater flexibility in how these metrics are calculated. Specifically, this design choice enables the calculation of aggregation-based fairness and reward metrics as opposed to step-wise metrics that are computed at each individual time step.

The primary advantage of using component functions rather than directly outputting rewards or fairness values at each time step is that it allows the construction of rate-based terms that aggregate the rewards and fairness violations over time. Directly computing values at each time step would constrain the system to use step-wise measures of fairness (e.g., fairness ratios calculated at each step), which can be sensitive to outliers and fluctuations in the data, as pointed out by Xu et al.~\cite{xuadapting}. Instead, our approach supports the calculation of aggregation-based metrics, which aggregate over time, offering a more holistic view of fairness across the entire decision-making process.

For example, using step-wise fairness metrics might yield values like:

\resizebox{.49 \textwidth}{!}{
$
    \sum_t^{T}\frac{\text{\#insured}_t}{\text{\#population}_t} \ \ \ \  \text{ and } \ \ \ \ 
    \sum_t^{T}\bigg|\frac{\text{\#insured}^A_t}{\text{\#population}^A_t}-\frac{\text{\#insured}^B_t}{\text{\#population}^B_t}\bigg|.
$
}

While this approach is valid, it only captures fairness at each time step and can be influenced by short-term fluctuations. On the other hand, aggregation-based fairness metrics enable the calculation of measures like:

\resizebox{.49 \textwidth}{!}{
$
    \frac{\sum_t^{T}\text{\#insured}_t}{\sum_t^{T}\text{\#population}_t} \ \ \ \  \text{ and } \ \ \ \
    \bigg|\frac{\sum_t^{T}\text{\#insured}^A_t}{\sum_t^{T}\text{\#population}^A_t}-\frac{\sum_t^{T}\text{\#insured}^B_t}{\sum_t^{T}\text{\#population}^B_t} \bigg|.
$
}

These metrics aggregate relevant quantities across all time steps before computing the fairness ratios, leading to more stable, long-term views of fairness that are less sensitive to the variance at each individual time step.

This flexibility in defining fairness and reward measures provides greater versatility in capturing long-term patterns and overall fairness in decision-making processes, making the MAFE framework adaptable to different applications.

\subsection{Transition Function}

The transition function defines system dynamics, updating the state from time $t$ to $t+1$ based on agent actions. This updated state forms the basis for future observations. While each MAFE’s transition function is unique, they all capture complex interactions between agents and individuals, reflecting real-world processes such as loan repayment cycles, health resource allocation, and educational progression. 

These state transitions continue until a MAFE episode is terminated. This occurs when one of the following conditions is met:
\begin{enumerate}
    \item \underline{Financial Failure:} Entities like an insurance company, employer, or university may go bankrupt after incuring losses that lead to net negative profits or prevent them from paying employees.
    \item \underline{Terminal Time Step:} The episode ends at a user-specified terminal time step.
\end{enumerate}

\begin{figure*}[ht!]
    \centering
    \includegraphics[width=\linewidth]{figures/Loan_Diagram_v2.png}
    \caption{Loan MAFE Diagram} 
    \label{fig:Loan_Diagram}
    \vspace{-0mm}
\end{figure*}

\begin{table*}[t]
\caption{Loan Processing MAFE Features}
\label{tab::Loan_indicators}
\centering
\begin{tabular}{ p{0.19\linewidth} || p{0.15\linewidth} || p{0.20\linewidth} || p{0.35\linewidth}}
 \hline
 Variable & Origin & How it is updated & Description\\
 \hline
  RACE & Lending Club & None& Main racial background\\  
 \hline
 INTRATE & Lending Club & None& Loan Interest Rate\\ 
 \hline
 BALANCE & Lending Club & Environment Dynamics& Loan Balance\\ 
 \hline
 ANNUALINC & Lending Club & Environment Dynamics & Annual income\\  
 \hline
 DTI & Lending Club & Environment Dynamics & Debt-to-income ratio\\
 \hline
 FICO\_RANGE\_LOW & Lending Club & Environment Dynamics & Lower boundary of individual's FICO score range\\ 
 \hline
 FICO\_RANGE\_HIGH & Lending Club & Environment Dynamics & Upper boundary of individual's FICO score range\\ 
 \hline
  TIMETOMATURITY & Environment& Environment Dynamics & Remaining time until loan maturity\\ 
 \hline
 WARNING & Environment & Environment Dynamics & Flag that loan in danger of default\\ 
 \hline
 TOTREQUEST & Environment & Environment Dynamics & Total amount requested by bank on current loan\\ 
\hline
 TOTRECEIVE & Environment & Environment Dynamics & Total amount received by bank on current loan\\
 \hline
 QUALSCORE & Environment & Environment Dynamics  & Qualification score\\
 \hline
 TOTBANKPROF & Environment & Environment Dynamics & Bank's accumulated profits\\
  \hline
 CURRINSTALL & Environment & Debt Agent ($\boldsymbol{\pi}_3$) & Amount of current installment\\ 
 \hline
\end{tabular}
\end{table*}
