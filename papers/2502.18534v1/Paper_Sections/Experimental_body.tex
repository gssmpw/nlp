\label{Sec::Experiments}

\begin{figure*}[ht!]
  \begin{subfigure}{0.28\textwidth}
    \includegraphics[width=\linewidth]{figures/Qual_distributions.png}
    \caption{Loan Score Distributions} \label{fig:qual_dist}
  \end{subfigure}%
  \hspace{0.05\textwidth}   
  \begin{subfigure}{0.28\textwidth}
    \includegraphics[width=\linewidth]{figures/Health_distributions.png}
    \caption{Healthcare Score Distributions} \label{fig:health_dist}
  \end{subfigure}%
  \hspace{0.05\textwidth}
  \begin{subfigure}{0.28\textwidth}
    \includegraphics[width=\linewidth]{figures/gpa_distributions.png}
    \caption{Education Score Distributions} \label{fig:gpa_dists}
  \end{subfigure}%
\caption{Distribution plots that illustrate disparities in (a) the qualification score distributions of customers in the Loan Environment, (b) health risk score distributions among geographic sub-populations the Healthcare Environment, and (c) GPA score distributions of students in the Education Environment at the beginning of an episode.} \label{fig:Distributions}
\end{figure*}

\begin{figure*}[ht!]  
    \captionsetup[subfigure]{justification=centering, font=footnotesize}
  \centering
  \begin{subfigure}{0.25\textwidth}
    \includegraphics[width=\linewidth]{figures/debt_relief_hardcode.png}
    \caption{Debt Management \\ (Entire Population)} \label{fig:debt_hardcode}
  \end{subfigure}%
  \begin{subfigure}{0.25\textwidth}
    \includegraphics[width=\linewidth]{figures/Beds_hardcode.png}
    \caption{Hospital Bed Availability \\ (Entire Population)} \label{fig:bed_hardcode}
  \end{subfigure}%
  \begin{subfigure}{0.25\textwidth}
    \includegraphics[width=\linewidth]{figures/Insurance_hardcode.png}
    \caption{Insurance Availability\\ (Entire Population)} \label{fig:Insurance_hardcode}
  \end{subfigure}%
  \begin{subfigure}{0.25\textwidth}
    \includegraphics[width=\linewidth]{figures/Public_invest_hardcode.png}
    \caption{Public Health Investment\\ (Entire Population)} \label{fig:public_hardcode}
  \end{subfigure}%

    \vspace{-0mm}
  \begin{subfigure}{0.25\textwidth}
    \includegraphics[width=\linewidth]{figures/Tertiary_hardcode.png}
    \caption{Tertiary Investment\\ (Entire Population)} \label{fig:tert_hardcode}
  \end{subfigure}%
  \hspace*{\fill}   
  \begin{subfigure}{0.25\textwidth}
    \includegraphics[width=\linewidth]{figures/Scholarship_hardcode.png}
    \caption{Scholarships\\ (Entire Population)} \label{fig:scholar_hardcode}
  \end{subfigure}%
  \begin{subfigure}{0.25\textwidth}
    \includegraphics[width=\linewidth]{figures/Mentorship_hardcode.png}
    \caption{Mentorship Programs \\ (Disadvantaged Population)} \label{fig:ment_hardcode}
    \vspace{0.0cm}
  \end{subfigure}%
  \begin{subfigure}{0.25\textwidth}
    \includegraphics[width=\linewidth]{figures/Emp_diversity_hardcode.png}
    \caption{Employer Div. Incent.\\ (Disadvantaged Population)} \label{fig:div_hardcode}
    \vspace{0cm}
  \end{subfigure}%
\caption{Plots illustrating the impact of various interventions in each environment, isolating their effects while holding other factors constant. (a) In the Loan MAFE, the effect of 20\% debt relief on qualification scores for the full population. (b)-(d) In the Healthcare MAFE, the effects of providing hospital beds, universal health insurance, and unlimited public health investment on mortality rates. (e)-(g) In the Education MAFE, the effects of unlimited tertiary investment, full scholarships, and mentorship on graduation rates for the full population (e) and (f) and the disadvantaged population (g). (h) In the Education MAFE, the effect of unlimited diversity incentives for the Employer Agent on the average utility of workers from disadvantaged groups.} 
\label{fig:hardcode_edu}
\vspace{-3mm}
\end{figure*}

In Sections~\ref{exp::hardcode}, \ref{exp::learnability}, and \ref{exp::action_analysis}, we demonstrate how MAFE actions address system disparities, verify agent learnability, and analyze agent action strategies. We introduce the \textit{Fair Multi-Agent Cross Entropy Method (F-MACEM)} for optimizing these tasks, with details in Appendix~\ref{sus_fair} and additional results in Appendix~\ref{Sec::Experiments_ext}.


\begin{figure*}[ht!] 
   \centering
  \begin{subfigure}{0.28\textwidth}
    \includegraphics[width=\linewidth]{figures/Direct_Rewards_Loan.png}
    \caption{Loan: Direct Rewards} \label{fig:Direct_Rewards_Loan}
  \end{subfigure}%
  \hspace{0.05\textwidth}
  \begin{subfigure}{0.28\textwidth}
    \includegraphics[width=\linewidth]{figures/Equity_Rewards_Loan.png}
    \caption{Loan: Fair Rewards} \label{fig:Equity_Rewards_Loan}
  \end{subfigure}%
  \hspace{0.05\textwidth}
  \begin{subfigure}{0.28\textwidth}
    \includegraphics[width=\linewidth]{figures/Rate_Rewards_Loan.png}
    \caption{Loan: Rate Rewards} \label{fig:Rate_Rewards_Loan}
  \end{subfigure}%

  \begin{subfigure}{0.28\textwidth}
    \includegraphics[width=\linewidth]{figures/Direct_Rewards_Healthcare.png}
    \caption{Healthcare: Direct Rewards} \label{fig:Direct_Rewards_Healthcare}
  \end{subfigure}%
  \hspace{0.05\textwidth}
  \begin{subfigure}{0.28\textwidth}
    \includegraphics[width=\linewidth]{figures/Equity_Rewards_Healthcare.png}
    \caption{Healthcare: Fair Rewards} \label{fig:Equity_Rewards_Healthcare}
  \end{subfigure}%
  \hspace{0.05\textwidth}
  \begin{subfigure}{0.28\textwidth}
    \includegraphics[width=\linewidth]{figures/Rate_Rewards_Healthcare.png}
    \caption{Healthcare: Rate Rewards} \label{fig:Rate_Rewards_Healthcare}
  \end{subfigure}%

  \begin{subfigure}{0.28\textwidth}
    \includegraphics[width=\linewidth]{figures/Direct_Rewards_Education.png}
    \caption{Education: Direct Rewards} \label{fig:Direct_Rewards_Education}
  \end{subfigure}%
  \hspace{0.05\textwidth}
  \begin{subfigure}{0.28\textwidth}
    \includegraphics[width=\linewidth]{figures/Equity_Rewards_Education.png}
    \caption{Education: Fair Rewards} \label{fig:Equity_Rewards_Education}
  \end{subfigure}%
  \hspace{0.05\textwidth}
  \begin{subfigure}{0.28\textwidth}
    \includegraphics[width=\linewidth]{figures/Rate_Rewards_Education.png}
    \caption{Education: Rate Rewards} \label{fig:Rate_Rewards_Education}
  \end{subfigure}%
\caption{Learning curves showing realized rewards obtained during training for models with different combinations of reward terms explicitly included in the F-MACEM’s objective function: ``Direct"; ``Direct + Fair"; or ``Direct+Fair+Rate" in the objective.} 
\label{fig:learnability}
\vspace{-3mm}
\end{figure*}
\vspace{-2mm}
\subsection{Validating Interventions for Correcting Disparities}
\label{exp::hardcode}

This section shows that actions shaped in our MAFEs can effectively mitigate disparities. Each MAFE is designed in a way that can incorporate structural biases, which may lead to disparate outcomes across demographic groups. The core attributes influencing outcomes vary by environment: qualification scores in the Loan MAFE, health risk scores in the Healthcare MAFE, and baseline GPA in the Education MAFE. These attributes reflect inherent biases across sensitive groups, calculated by regressing over dataset features used to construct each MAFE's feature vectors. To enhance these biases for the purpose of supporting fairness research, we have resampled the original feature distributions, exacerbating disparities. Figure~\ref{fig:Distributions} illustrates these biased distributions at the start of each MAFE episode.

To assess whether agent actions can correct disparities, we conducted fixed intervention experiments, summarized in Figure~\ref{fig:hardcode_edu}. Using a fixed random seed, we compare the impact of specific interventions on environmental indicators with and without the intervention, repeating the process across five seeds. In the Loan MAFE, we examined debt management's effect on qualification scores. In the Healthcare MAFE, we evaluated incidence and mortality rates under varying conditions like hospital bed availability, insurance coverage, and public health investments. In the Education MAFE, we analyzed the impact of investments, scholarships, mentorship programs, and diversity incentives on graduation rates and employer utility.

The results shown in Figure~\ref{fig:hardcode_edu} illustrate significant improvements when interventions are applied (dashed red lines) compared to baseline scenarios (solid black lines). In each plot, there is significant bias in the red dash line when compared with the black solid lines. The direction of the arrow (upward or downward) above each plot signifies improvement in the indicator of interest, indicating the positive impacts that each intervention has on improving outcomes for members of the population.  Thus, applying these interventions strategically for sub-population groups should allow agents to effectively mitigate disparities among different sensitive attribute groups.

\vspace{-2mm}
\subsection{Compound Effects of Reward Terms}
\label{exp::learnability}

In this section, we explore the cumulative impact of incorporating different terms into the F-MACEM's objective function for each MAFE, specifically examining how various combinations of terms influence the observed outcomes for each individual term. We categorize these terms into three distinct groups, as outlined in Sections~\ref{sec::reward_struct} and ~\ref{sec::fair_struct}: direct rewards, fairness penalties, and rate-based rewards. To analyze their effects, we train the F-MACEM using three configurations of the objective function: (1) including only direct rewards, (2) including both direct rewards and fairness penalties, and (3) including direct rewards, fairness penalties, and rate-based rewards. For consistency, all elements in each objective function are uniformly weighted.

\begin{figure*}[ht!]    
  \hspace{0.6cm}
  \begin{subfigure}{0.28\textwidth}
    \includegraphics[width=\linewidth]{figures/Education_employer_allocation.png}
    \caption{Employer Agent} \label{fig:Education_employer_allocation_body}
    \vspace{0.34cm}
  \end{subfigure}%
  \hspace{0.25cm}
  \begin{subfigure}{0.28\textwidth}
    \includegraphics[width=\linewidth]{figures/Education_planner_level1_allocation.png}
    \caption{Central Planner Agent \\ (General Intervention)} \label{fig:Education_planner_level1_allocation_body}
  \end{subfigure}%
  \hspace{0.25cm}
  \begin{subfigure}{0.28\textwidth}
    \includegraphics[width=\linewidth]{figures/Education_univ_budget_allocation_overall.png}
    \caption{University Budget Allocation Agent} \label{fig:Education_univ_budget_allocation_body}
    \vspace{-0.0cm}
  \end{subfigure}%

\caption{Average actions taken by agents over training epochs in Education MAFE.} \label{fig:education_action_summary}
\vspace{-3mm}
\end{figure*}

The results of this analysis are presented in Figure~\ref{fig:learnability}. Each row corresponds to a different environment, while each column tracks the evolution of a specific reward category throughout training. Within each plot, the plotted curves differentiate the explicit reward terms included in the objective function. As expected, the red line---representing the objective function that explicitly incorporates all reward categories---shows steady improvement across all reward types during training. In contrast, configurations excluding certain terms often exhibit less consistent and volatile performance. For example, in the Education environment, the rate-based reward curve for the F-MACEM, trained solely with direct rewards, declines from its initial value during training and only approximately returns to its starting point by the final epoch on average. Similarly, in the Loan environment, excluding rate-based rewards causes the corresponding reward curve to plateau at a significantly lower value than observed in the fully-inclusive configuration. These patterns underscore the utility of integrating diverse reward terms to balance learning objectives effectively within each MAFE.

This analysis also highlights environment-specific characteristics. Notably, the Healthcare MAFE shows smaller performance differences between training configurations compared to the Loan and Education MAFEs. While this might seem counterintuitive, it reflects the MAFE's design: individuals transition between healthy, sick, and deceased states, with insurance profit as the primary reward. Insurers benefit most when the population maintains a high insured rate and remains healthy, minimizing claims. As a result, agents learn to balance interventions that optimize profitability and health outcomes. This alignment between agent objectives and system well-being offers a key insight: even when explicit stakeholder priorities diverge, overlapping indirect objectives can foster cooperative strategies that outperform narrow, self-serving approaches.

\vspace{-2mm}
\subsection{Policy Action Analysis}
\label{exp::action_analysis}

In this section, we analyze the actions that the F-MACEM learns to produce over the training process when direct rewards, rate-based rewards, and fairness penalties receive uniform weighting in the objective function for the Education MAFE. We visualize how the Central Planner Agent distributes funds across different interventions, how the Employer Agent sets salaries, and how the university distributes the resources it receives for different interventions that improve student academic success.

Figures~\ref{fig:Education_employer_allocation_body}-\ref{fig:Education_univ_budget_allocation_body} illustrate these agent actions. The Central Planner Agent primarily invests in tertiary resources and employer diversity incentives (Figure~\ref{fig:Education_planner_level1_allocation_body}), suggesting that tuition revenue adequately covers university operations. The University Budget Allocation Agent shifts its strategy the training process (Figure~\ref{fig:Education_univ_budget_allocation_body}). Initially, it allocates a significant portion of its budget to faculty salaries to ensure stability, but since faculty salaries are fixed, the agent refines its strategy by directing more resources to student-specific interventions, like scholarships and mentorship programs for underrepresented groups. This change helps reduce disparities in GPAs between majority and underrepresented students, improving overall educational and career outcomes.

Notably, Figure~\ref{fig:Education_employer_allocation_body} shows a significant trend reversal in the Employer Agent’s salary-setting behavior midway through the training process. Initially, the Employer Agent decreases average salaries; however, this trend inverts as training progresses, leading to a steady increase in salaries. This shift results from multiple factors. First, the Central Planner Agent’s investment in diversity incentives directly boosts the salaries of underrepresented minority groups. Second, as the Central Planner Agent and University Budget Allocation Agent optimize their investments in tertiary resources and university student aid, overall student performance improves. These enhancements in educational outcomes translate to better career success, indirectly driving higher salaries.

Coordination among agents in each MAFE can create a positive feedback loop for improving system rewards, enabled by the flexible intervention structure our MAFEs offer. This structure facilitates the development of coordinated strategies that are more realistic and useful than the simplified abstractions used in single-action environments.

