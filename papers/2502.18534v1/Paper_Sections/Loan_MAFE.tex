\label{sec::loan_MAFE}

In this section we provide a detailed explanation of how we design the Loan MAFE introduced in Section~\ref{env} of the main paper.

\textbf{Overview:} 
A diagram illustrating the design of our Loan MAFE is provided in Figure~\ref{fig:Loan_Diagram}. This environment simulates the loan processing pipeline of a financial institution. The agents in this system represent three main branches of the bank. The first is the Admissions Agent ($\boldsymbol{\pi}_{1}$), responsible for determining who will be approved for loans. The second is the Disbursement Agent ($\boldsymbol{\pi}_{2}$), which handles the timing of loan disbursements. The third is the Debt Management Agent ($\boldsymbol{\pi}_{3}$), which oversees loan repayment and manages defaults.

At each time step, a sample of individuals from the applicant population applies for loans. These applicants are either approved or rejected by the Admissions Agent. Rejected applicants are re-entered into the population and may be considered for loans in subsequent time steps. Approved applicants move into the disbursement phase of the loan processing pipeline.

In the disbursement phase, individuals must wait for their loan funds to be disbursed by the institution. The disbursement process is constrained by human resources, meaning only a fixed number of loans can be processed per time step, which may introduce delays. The Disbursement Agent controls who receives their funds first by sorting the queue of individuals waiting for their loans at every time step.

Once an applicant receives a loan, they begin making regular payments in each subsequent time step. If the borrower consistently makes on-time payments until the loan's maturity, the loan is fully paid off. Conversely, if the borrower fails to make timely payments, they will default on the loan. In this phase, the Debt Management Agent has the ability to adjust repayment requests to alleviate financial strain on an individual and help them avoid default.

An individual’s features are updated when their loan is terminated, but the nature of the update differs depending on how the loan is terminated: the individual’s features improve in the case of successful repayment and deteriorate in the case of default. The individual is then reinserted into the applicant pool to be resampled for future loan applications.

We now elaborate on each entity in the environment by explaining the operations that take place during a given time step, $t$.

\textbf{Population:} At the beginning of the loan simulation, a global population is initialized consisting of $N$ individuals. Each individual has an associated feature vector, $\mathbf{v}=[\mathbf{v}_{c}^{T} \ \ \mathbf{v}_{v}^{T}]^{T} \in \mathbb{R}^k$, which contains both financial and demographic attributes used by the agents to make decisions. The vector $\mathbf{v}_{c}$ represents constant features that remain unchanged throughout the simulation, while $\mathbf{v}_{v}$ contains variable features that are influenced by the dynamics of the MAFE system.

To ensure that the data used in the simulation is realistic, we leverage real-world data from LendingClub, a financial services company that connects borrowers with investors for peer-to-peer lending~\cite{lendingclub_kaggle}. Our population is constructed using loans from this dataset, with initial balances ranging from \$1,000 to \$40,000. Approximately half of the features in the feature vector are directly derived from the loan data, as outlined in Table~\ref{tab::Loan_indicators}. These feature vectors are then augmented with additional information relevant to the dynamics of the environment, such as QUALSCORE, which indicates an individual's qualification score and serves as a proxy for the likelihood of loan repayment. 

The global population is divided into distinct subpopulations based on the phase of the loan processing system each individual inhabits. These include the \textbf{application population}, which consists of individuals not yet in the loan processing system but who wish to apply for loans; the \textbf{waiting population}, which includes individuals who have been approved for loans and are awaiting disbursement of funds; and the \textbf{repayment population}, which contains individuals who have received their loan funds and are currently repaying them.

The features associated with individuals in each of these categories provide the observations for the various agents involved in the MAFE system, including the Admissions, Disbursement, and Debt Management Agents, at each time step. These features, particularly those in $\mathbf{v}_{v}$, are influenced by the actions taken by different agents within the system. For example, the bank may adjust an individual's installment plan as they continue to repay their loan. This not only updates the current loan balance (CURRINSTALL), but can also improve or deteriorate financial indicators like DTI and FICO scores over time, depending on the individual’s payment behavior. These evolving features provide context to enable the agents to adjust their strategies to, for example, modify installment amounts to help prevent default or encouraging timely repayments.

In the remainder of this section, we use subscript notation to refer to the value of a particular variable for an arbitrary individual or group at time $t$. For instance, $BALANCE_t$ refers to the balance of an individual's loan at time $t$, while $BALANCE_{g,t}$ refers to the loan balance for an individual belonging to sensitive group $g$ at time $t$. Similarly, other features in the individual’s vector, such as CURRINSTALL, DTI, or FICO scores, will be indexed by subscripts to refer to specific individuals or groups at different points in time.

Further details on how each agent affects these features are provided in the following discussion.

\textbf{Admissions Agent ($\boldsymbol{\pi}_1$):} At time step $t$, the Admissions Agent samples a group of $N_{1,t}$ applicants to form the application population for this time step and is tasked with deciding which of these applicants should be approved or rejected for a loan. Let $\mathbf{V}\in\mathcal{V}$ represent the matrix whose rows represent the feature vectors associated with these $N_{1,t}$ individuals. A scoring function $\mathbf{s}:\mathbb{R}^k\rightarrow [0,1]$ produces a score which represents how qualified an individual is for repaying the loan that they have requested. The Admissions Agent, $\boldsymbol{\pi}_1: \mathcal{V} \rightarrow [0,1]^{g}$, is tasked with setting $g$ thresholds used to determine which individuals are admitted or rejected from the system. Two configurations of the agent's action space are considered: $g=1$ ($g=2$) indicates that the agent outputs a global (group-specific) threshold for approving individuals for loans at time step $t$. Admitted individuals are removed from the application population and enter the next phase of the loan system where they wait for their funds to be disbursed starting in time step $t+1$. Rejected individuals are returned to the population and wait for another opportunity to be sampled and considered for a loan.

\textbf{Disbursement Agent ($\boldsymbol{\pi}_2$):} Once a person has been approved for a loan, he/she is removed from the application population pool and enters the funds disbursement stage of the pipeline. At time step $t$, $N_{2,t}$ individuals comprise the waiting population and wait in a queue for their funds to be disbursed. There is a fixed cap on the number of individuals who may have their funds disbursed at any given time step, which is used to mimic the real-world human resource constraints of a bank. Let $\mathbf{D}\in\mathcal{D}$ represent the matrix whose rows represent the feature vectors associated with these $N_{2,t}$ individuals. The Disbursement Agent, $\boldsymbol{\pi}_2:\mathcal{D}\rightarrow[0,1]^{N_{2,t}}$, reorders the queue at every time step by producing a score in the range $[0,1]$ for every customer waiting for their funds to be disbursed. At each time step, the queue is re-sorted in descending order of the scores produced by this agent. Individuals at the top of the queue are then provided with funds until the disbursement cap is hit.

\textbf{Debt Management Agent ($\boldsymbol{\pi}_3$):} Once individuals receive their funds, they enter the loan repayment phase of the pipeline. At time step $t$, $N_{3,t}$ individuals in the repayment population make payments on their loans. Let $\mathbf{B} \in \mathcal{B}$ represent the matrix whose rows are the feature vectors associated with these $N_{3,t}$ individuals. Each individual is required to make payments according to a fixed payment schedule until their loan reaches maturity or they default. To support customers and reduce the likelihood of default, the Debt Management Agent, $\boldsymbol{\pi}_3:\mathcal{B} \rightarrow [0,1]^{g}$, can adjust repayment terms to alleviate financial strain. Two configurations of the agent's action space are considered: $g=1$ ($g=2$) indicates that the agent outputs a global (group-specific) adjustment percentage for the installments of all individuals repaying their loans at time step $t$. Once an individual's loan is terminated, they reenter the application population pool, from which the bank samples individuals for future loans.

\begin{table}[t]
\caption{Loan MAFE Component Indicators}
\label{tab::Loan_indicators}
\centering
\resizebox{0.95\columnwidth}{!}{%
\begin{tabular}{ p{0.15\linewidth} || p{0.75\linewidth}}
 \hline
 Indicator& Description\\
 \hline
 $P_t$ & Bank profits at time step $t$ \\ 
 \hline
 $N_{L,t}^{g}$ & Number of people who applied for loans from Group $g$ at time step $t$ \\ 
 \hline
 $N_{A,t}^{g}$ & Number of people approved for loans from Group $g$ at time step $t$ \\ 
 \hline
 $N_{D,t}^{g}$ & Number of people from Group $g$ that had their fund disbursed at time step $t$ \\ 
 \hline
 $N_{T,t}^{g}$ & Sum of the number of time steps waited to receive loan funds for everyone from Group $g$ that received their funds at time step $t$.\\ 
 \hline
 $N_{R,t}^{g}$ & Number of terminated loans by members of Group $g$ at time step $t$.\\  
  \hline
 $N_{F,t}^{g}$ & Number of defaulted loans by members of Group $g$ at time step $t$.\\  
 \hline
\end{tabular}
}
\end{table}

\textbf{Reward and Disparity Component Indicators:} At the end of time step $t$, the environment returns a collection of reward and disparity component indicators used for reward and fairness violation measurement. A summary of these indicators is provided in Table~\ref{tab::Loan_indicators}. Each agent in this environment represents a functioning part of one institution, namely, a bank which has one primary objective---maximizing profits ($P_t$). Thus, the
total amount of money made by the bank at time step $t$ represents the primary reward returned by the environment. Two other rewards can constructed from this list of indicators to guide learning models to avoid poor local minima; namely overall admissions rates ($\frac{\sum_t \sum_g N_{A,t}^{g}}{\sum_t \sum_g N_{L,t}^{g}}$) and (negative) overall default rates ($-\frac{\sum_t \sum_gN_{F,t}^{g}}{\sum_t \sum_g N_{R,t}^{g}}$). 

The remaining environmental indicators provided by the system are used to measure fairness violations by tracking disparities among different rates provided for each demographic group at time step $t$. In particular, this information can be used to analyze three fairness disparities within the system among the two sensitive groups; namely, we can analyze disparities in: admissions rates ($\frac{\sum_tN_{A,t}^{g}}{\sum_t N_{L,t}^{g}}$), funds disbursement wait times ($\frac{\sum_tN_{T,t}^{g}}{\sum_t N_{D,t}^{g}}$), and default rates ($\frac{\sum_tN_{F,t}^{g}}{\sum_t N_{R,t}^{g}}$). Hence the indicators provided by the environment at each time step are used to measure three rewards and three fairness disparities.

\textbf{Mathematical Modeling:} A variety of environmental dynamics must be accounted for explicitly to ensure that the different underlying processes within the loan system function properly. These include modeling things such as a customer's financial rating or qualification to repay a loan, which is used by the Admissions Agent to set a threshold to determine who is and is not approved for a loan; loan payment schedule, which determines the amount a customer's loan installment at a given time step; and propensity to make a payment, which ultimately will determine whether or not he/she defaults. These design choices are outlined as follows.

\underline{Customer Qualification Scores:} 

A logistic regression is trained to take a customer's feature vector, $\mathbf{v}$, and produce a score in the range, $[0,1]$. This model uses only a features from the Lending Club dataset, excluding any features from Table~\ref{tab::Loan_indicators} augmented from environmental dynamics.

\underline{Payment Schedule:} 

Each loan is characterized by its duration (in time steps, representing its maturity), denoted as $\text{TIMETOMATURITY}_t$; interest rate, $\text{INTRATE}$; and initial balance, $\text{BALANCE}_{t_0}$. For simplicity, we respectively use $m$, $r$, and $B$ to refer to these variables in the ensuing discussion. At each time step, the customer is requested to make a payment, $Y_t$. In response, the customer will make a payment, $X_t$, where $0 \leq X_t \leq Y_t$. A payment below $Y_t$ indicates that the customer is falling behind on their loan obligations. The loan balance at each time step is updated using the following recursive formula:
\begin{equation}
    B_t = (1+r)B_{t-1}-X_t
\end{equation}
The bank’s goal is for the loan to be fully repaid by its maturity date, $m$. Assuming a fixed-rate payment schedule, at time step $t$, the payment request, $Y_t$, is set so that, if the customer were to pay the full amount of $Y_t$ at each time step until maturity, the loan balance would reach zero by time step $m$. To calculate this payment, we expand $B_m$ in terms of $B_t$ as follows:
\begin{align}
    B_{m} &= (1+r)B_{m-1}-Y_t \notag \\
        &= (1+r)^{m-t}B_{t}-\sum_{k=0}^{m-t-1}Y_t(1+r)^k \notag\\
        &= (1+r)^{m-t}B_{t}-Y_t\frac{(1+r)^{m-t}-1}{r} \notag\\
\end{align}
Setting this equation equal to zero and solving for $Y_t$ yields the required payment amount, which depends on the loan’s current balance, the interest rate, and the time remaining until maturity:
\begin{align}
    Y_{t}= \frac{r}{1-(1+r)^{t-m}}B_t \notag
\end{align}
This payment ensures that, if paid in full at each time step, the loan balance will be entirely paid off by the maturity date, $m$.

\begin{figure*}[ht!]
    \centering
    \includegraphics[width=0.9\linewidth]{figures/Healthcare_Diagram_v2.png}
    \caption{Healthcare MAFE Diagram} 
    \label{fig:Healthcare_Diagram}
    \vspace{-0mm}
\end{figure*}

\underline{Customer Payment:} 

The following equation is used to calculate the payment received by the bank on the installment requested at time step $t$:
\begin{equation}
    X_t = clip(p_t+N_t,0,1)\cdot Y_t,
\end{equation}
where $p_t$ is a propensity score that represents the percentage of $Y_t$ that a customer is willing to pay and $N_t\sim \mathcal{N}(\mu,\sigma^2)$ is Gaussian noise used to make the propensity score stochastic. The propensity scores for a customer are produced by a linear regression model trained to take the subset of a customer's feature vector, $\mathbf{v}$, containing features from the Lending Club dataset as input and output a percentage in the range $[0,1]$. The labels for training this model are constructed by dividing the number of months it took for an individual's loan to terminate by the term of the loan for each individual in the training dataset. If the individual did not default, this label value is $1$ (meaning they are completely likely to repay their loan). Moreover, the propensity scores of customers that default much earlier are lower than those of the customers that took a longer time to default.

\underline{Customer Default:} 

Default occurs if the applicant falls behind by more than $10\%$ on all payments that the bank has requested from them for at least two consecutive time steps.

\underline{Bank Lending \& Profits:} 

To finance the loans provided to its customers, we assume that the bank ``borrows" money. That is, the bank pools deposits on which it, too, pays interest. Its profits are thus made by paying a lower interest rate than the rate it charges its customers. Thus the profits at a given time step are calculated as the difference between the sum of the payments received on the outstanding loans of its customers and the amount it is required to pay to its depositors.

\textbf{Loan Termination Feature Update Rule:}
In reality, termination of a loan impacts an individual's financial well-being. For example, defaulting on a loan may reduce a person's FICO score, but the reverse may happen should a person repay his/her loan. Thus, each time a loan is terminated in this MAFE, we adjust a subset of features in $\mathbf{v}_{v}$ to reflect such a change, with the cause of termination (repayment versus default) determining whether the features will deteriorate or improve. In particular, we apply the following linear feature update rule to adjust these values:
\begin{equation}
    \mathbf{v}_v = 
    \begin{cases} 
        \mathbf{v}_v + \mathbf{c} \ \  \text{, if} \ \  \text{Customer Repays Loan} \\
        \mathbf{v}_v - \mathbf{c}  \ \ \text{, if} \ \  \text{Customer Defaults on Loan}
    \end{cases}
\end{equation}
for some constant vector $c$.

\textbf{Episode Termination:} An episode in the Loan MAFE may terminate for two reasons: (1) The maximal number of time steps set by a user has been reached and (2) the bank goes bankrupt. Bankruptcy occurs if at any point during the simulation, the total amount of money lost by the bank is greater than the total amount of money it has received.
