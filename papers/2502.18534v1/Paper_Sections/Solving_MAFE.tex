\label{Fair_Sec}

In this section, we focus on a cooperative multi-agent setting as a use case for analyzing the MAFEs introduced in Section~\ref{env}, where agents work towards shared objectives and fairness concerns arise from disparities in outcomes over time. This design choice serves to demonstrate the MAFE framework’s application in a concrete scenario, while the framework itself remains adaptable to other scenarios.

\subsection{Formulating the Multi-agent Decision Problem}
In this section, we formalize the concept of success in our cooperative decision-making scenario. Let $o_{n,t}$ and $a_{n,t}$ represent the observation received and the action taken by the $n^{th}$ agent at time $t$ and $\mathbf{o}_{t}$ and $\mathbf{a}_{t}$ represent the collections of all observations seen and actions produced by every agent at time $t$. Colon notation over these temporal indices denotes a time interval. Let $R_n^{(k)}(\mathbf{o}_{1:\infty},\mathbf{a}_{1:\infty})=R^{(k)}_n(c_n^{(R)}(\mathbf{o}_{1:\infty},\mathbf{a}_{1:\infty}))$ represent the \textit{total reward} for the $k^{th}$ of $K$ rewards for Agent $n$, and similarly, let $F_n^{(m)}(\mathbf{o}_{1:\infty},\mathbf{a}_{1:\infty})=F_n^{(m)}(c_n^{(F)}(\mathbf{o}_{1:\infty},\mathbf{a}_{1:\infty}))$ represent the \textit{total violation} of the $m^{th}$ of $M$ fairness measures for Agent $n$. For brevity, we refer to these values as $R_n^{(k)}$ and $F_n^{(m)}$, respectively. We describe the functional forms that we select for $R^{(k)}_n$ and $F_n^{(m)}$ in use case in Sections~\ref{sec::reward_struct} and ~\ref{sec::fair_struct}. Finally, let $\theta_n$ represent the parameters of the model used to produce the action taken by Agent $n$. Then, a maximization problem for Agent $n$ may be given by Equation~\ref{problem1}:
\begin{align}
    \label{problem1}
     \max_{\theta_n} \ \ & \sum_{k=1}^{K}
    \alpha_{n,k}\mathbb{E}_{\theta_n}[R_n^{(k)}]\notag\\
     &\text{s.t. } \ \ \ \ \mathbb{E}_{\theta_n}[F_n^{(m)}] \leq \epsilon^{(m)}, \ \ 1\leq m \leq M,
\end{align}
where $\alpha_{n,k}$ is a user-defined weight for the $k^{th}$ reward in Agent $n$'s objective function.

Using regularization, this problem can be rewritten as:
\begin{align}
    \label{problem1_reg}
    \max_{\theta_n} \ \ \sum_{k=1}^{K}
    \alpha_{n,k}\mathbb{E}_{\theta_n}[R_n^{(k)}]+ \sum_{m=1}^{M}\beta_{m,n} \mathbb{E}_{\theta_n}[F_n^{(m)}],
\end{align}
where $\beta_{m,k}$ is a user-defined weight for the $m^{th}$ fairness penalty in Agent $n$'s objective function.

The cooperative setting is a special case of this problem in which, for all $n$ agents, $\alpha_{n,k}=\alpha_k$, $\beta_{m,n}=\beta_{m}$, $c^{(R)}=c^{(R)}_n$ and $c^{(F)}=c^{(F)}_n$. In this scenario, the objective function becomes identical for all agents. Thus, we can rewrite Problem~\ref{problem1_reg} in the following form:
\begin{align}
    \label{problem1_reg_coop}
    \max_{\theta_n} \ \ \ \ \sum_{k=1}^{K}
    \alpha_{k}\mathbb{E}_{\theta_n}[R^{(k)}] + \sum_{m=1}^{M}\beta_{m} \mathbb{E}_{\theta_n}[F^{(m)}].
\end{align}
Thus, success over an episode can be measured directly computing the following value once an episode is terminated:
\begin{equation}
    \label{eq::episode_success}
    \sum_{k=1}^{K}
    \alpha_{k}R^{(k)} + \sum_{m=1}^{M}\beta_{m} F^{(m)}.
\end{equation}

\subsection{Reward Structure Customization}
\label{sec::reward_struct}

We design two types of rewards for agents: \textbf{direct} rewards and \textbf{rate-based} rewards. Direct rewards are explicit values, such as profits, that an agent aims to optimize. Rate-based rewards are expressed as ratios, such as the proportion of insured individuals to the total population, representing relative measures that agents aim to optimize. With this, we now provide the form of the reward summation in Problem~\ref{problem1_reg_coop}.

Let $K=j+l$, and define the reward components $[r_{1,t},...,r_{j+2l,t}]=c^{(R)}(\mathbf{o}_{1:\infty},\mathbf{a}_{1:\infty})$, where $r_{1,t},...,r_{j,t}$ are the direct rewards, $r_{j+1,t},...,r_{j+l,t}$ are numerators for rate-based rewards, and $r_{j+l+1,t},...,r_{j+2l,t}$ are denominators for the rate-based rewards at time $t$. Then, the final structure of the rewards summation in Equation~\ref{problem1_reg_coop} can be rewritten as the sum of its direct and rate-based constituents:
\begin{equation}
    \sum_{i=1}^{j}\alpha_i\left[\sum_{t=0}^{\infty} r_{i,t}\right] + \sum_{i=j+1}^{j+l}\alpha_i \left[\frac{\sum_{t=0}^{\infty}r_{i,t}}{\sum_{t=0}^{\infty}r_{i+l,t}}\right].
\end{equation}

\subsection{Fairness Measure Structure Customization}
\label{sec::fair_struct}

Given that the most common disparities in algorithmic fairness are rate-based, such as differences in insured rates across geographic regions in healthcare, we now describe how $F^{(m)}$ in Problem~\ref{problem1_reg_coop} is structured to measure these disparities when the number of groups is two or more.

\textbf{Two-group case.} In the two-group case, the disparity between two groups is measured using the directly interpretable absolute difference in rates. Define the fairness components $[f_{1,t},...,f_{4M,t}]=c^{(F)}(\mathbf{o}_{1:\infty},\mathbf{a}_{1:\infty})$, where $f_{4m-3,t},...,f_{4m,t}$ represent the numerator and denominator for the rates of Groups 1 and 2 for the $m^{th}$ fairness measure. Then, the fairness violation is given by:
\begin{equation}
    \label{eq::binary_measure}
    F^{(m)} = -\bigg|\frac{\sum_{t=0}^{\infty}f_{4m-3,t}}{\sum_{t=0}^{\infty}f_{4m-2,t}}-\frac{\sum_{t=0}^{\infty}f_{4m-1,t}}{\sum_{t=0}^{\infty}f_{4m,t}} \bigg|
\end{equation}

\textbf{$D$-group case.} When the number of groups, $D$, exceeds two, an absolute difference is inadequate for capturing disparities, as it fails to reflect the distribution of rates across multiple groups. To address this, we use standard deviation to quantify fairness disparities in the $D$-group case. Its simplicity provides an interpretable measure of how evenly rates are distributed among groups, making it particularly suitable for assessing fairness in multi-group settings. We define this measure as follows. Let the fairness components, $[f_{1,t},...,f_{2DM,t}]=c^{(F)}(\mathbf{o}_{1:\infty},\mathbf{a}_{1:\infty})$, where $f_{2D(m-1)+1,t}, ..., f_{2Dm,t}$, provide the numerator and denominator of each of $D$ groups for which we use for measuring the $m^{th}$ rate. Let $Y_d^{(m)}=\frac{\sum_{t=0}^{\infty}f_{2D(m-1)+d,t}}{\sum_{t=0}^{\infty}f_{2D(m-1)+d+1,t}}$ and $\mu^{(m)}=\frac{1}{D}\sum_{d=1}^DY_d^{(m)}$. Then, the fairness measure is given by:
\begin{equation}
    \label{eq::std_measure}
    F^{(m)} = -\sqrt{ \frac{\sum_{d=1}^D \left(Y_d^{(m)}-\mu^{(m)}\right)^2 }{D}}
\end{equation}
As the value of $F^{(m)}$ approaches its upper limit of 0, the disparity in rates across different demographic groups diminishes, improving the parity among them.