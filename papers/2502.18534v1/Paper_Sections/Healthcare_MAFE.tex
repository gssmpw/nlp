\label{sec::healthcare_mafe}

In this section we provide a detailed explanation of how we design the Healthcare MAFE introduced in Section~\ref{env} of the main paper.

\textbf{Overview:} A diagram illustrating the design of our Healthcare MAFE is shown in Figure~\ref{fig:Healthcare_Diagram}. This environment models the interactions among three core agents: an insurance company, a hospital, and a central planner. These agents collectively impact the health and insurance coverage of the population.

At each time step, the Insurance Agent offers a premium to each individual, who decides whether to accept the plan based on its cost. The premium affects the likelihood of obtaining insurance, which influences the individual’s access to routine medical care. Thus, uninsured individuals face greater health risks due to limited access to early disease detection and regular treatment.

Individuals are categorized into three health states: \textbf{healthy}, \textbf{ill}, and \textbf{deceased}. Healthy individuals may become ill, and sick individuals may either recover or pass away. Upon diagnosis, a sick individual joins a hospital queue, where they await treatment. The allocation of hospital beds depends on the hospital’s capacity, with individuals prioritized for treatment according to the queue-ordering scores produced by the Hospital Agent. The likelihood of recovery is higher for individuals who are treated early, which is more likely if they are insured.

The Central Planner Agent allocates a healthcare budget at each time step, distributing funds across hospital infrastructure, public health initiatives, and insurance subsidies. The planner may also save funds for future investments in the healthcare system.

When moralities occur, deceased individuals are reintroduced into the population to simulate real-world population replenishment. However, in contrast with the Loan MAFE, where all agents act at every time step, in this system, the Hospital Agent acts at every time step, while the Insurance and Central Planner Agents take actions every $k$ time steps. This reflects real-world scenarios where premiums and budgets are set periodically, while healthcare needs can arise at any time. 

Ultimately, the collective decisions made by these agents affect mortality rates within the system. In the following sections, we provide a detailed description of the roles and operations of each agent within the environment at a given time step $t$.

\begin{table*}[ht!]
\caption{Healthcare MAFE Features}
\label{tab::Healthcare_indicators}
\centering
\begin{tabular}{ p{0.15\linewidth} || p{0.15\linewidth} || p{0.20\linewidth} || p{0.35\linewidth}}
 \hline
 Variable & Origin & How it is updated & Description\\
 \hline
 YEAR & IPUMS MEPS& None& Survey Year\\ 
 \hline
 AGE & IPUMS MEPS& None& Age\\ 
 \hline
 SEX & IPUMS MEPS& None& Sex\\ 
 \hline
 REGION & IPUMS MEPS& None& Census region as of 12/31 of the survey year\\ 
 \hline
 FAMSIZE & IPUMS MEPS& None& Number of persons in family\\ 
 \hline
 RACE & IPUMS MEPS& None& Main racial background\\  
  \hline
 USBORN & IPUMS MEPS& None& Born in United States\\  
 \hline
 EDUC & IPUMS MEPS& None& Educational Attainment\\
 \hline
 HICOV & IPUMS MEPS& Insurance Agent ($\boldsymbol{\pi}_1$) & Has health insurance\\ 
 \hline
 CHOLHIGHEV & IPUMS MEPS& None& Ever told had high cholesterol\\ 
 \hline
 SMOKENOW & IPUMS MEPS& None& Smoke cigarettes now\\ 
\hline
 INCTOT & IPUMS MEPS& Central Planner Agent ($\boldsymbol{\pi}_3$) & Total personal income\\
 \hline
 FTOTVAL & IPUMS MEPS& Central Planner Agent ($\boldsymbol{\pi}_3$) & Total family income\\
 \hline
 POVLEV & IPUMS MEPS& Central Planner Agent ($\boldsymbol{\pi}_3$) & Family income as a percentage of the poverty line\\
\hline
 AEFFORT & IPUMS MEPS& Central Planner Agent ($\boldsymbol{\pi}_3$) & Felt everything an effort, past 30 days\\
 \hline
 ANERVOUS & IPUMS MEPS& Central Planner Agent ($\boldsymbol{\pi}_3$) & How often felt nervous, past 30 days\\
 \hline
 ARESTLESS & IPUMS MEPS& Central Planner Agent ($\boldsymbol{\pi}_3$) & How often felt restless, past 30 days\\
 \hline
 AHOPELESS & IPUMS MEPS& Central Planner ($\boldsymbol{\pi}_3$) & How often felt hopeless, past 30 days\\
 \hline
 ASAD & IPUMS MEPS& Central Planner ($\boldsymbol{\pi}_3$) & How often felt sad, past 30 days\\
 \hline
 AWORTHLESS & IPUMS MEPS& Central Planner Agent ($\boldsymbol{\pi}_3$) & How often felt worthless, past 30 days\\
 \hline
  HEALTH & IPUMS MEPS& Environment Dynamics & Health status\\ 
 \hline
 NEEDBED & Environment & Environment Dynamics & Waiting for hospital bed\\
 \hline
 INHOSP & Environment & Hospital Agent ($\boldsymbol{\pi}_2$)& Person is in the hospital\\
 \hline
 ILLNESS & Environment & Environment Dynamics & How long person has been ill\\
 \hline
 DECEASED & Environment & Environment Dynamics & Person is deceased\\
 \hline
 NGEOBED & Environment & Environment Dynamics & Number of beds in each region\\
 \hline
 HIPCOST & Environment & Environment Dynamics & Health insurance premium\\
 \hline
 HIPFULLCOST & Environment & Environment Dynamics & Amount paid to health insurance by all members in same region \\
 \hline
 HOSPCOST & Environment & Environment Dynamics & Cost of hospital stay\\
 \hline
 WAITBED & Environment & Environment Dynamics& Waiting for a bed\\
 \hline
 ILLTIME & Environment & Environment Dynamics & How long sick with current illness\\
 \hline
 PLANBUDGET & Environment & Environment Dynamics & Central Planner current budget\\
 \hline
\end{tabular}
\end{table*}

\textbf{Population:} 
At the beginning of the healthcare simulation, a global population is initialized which consists of $N$ healthy individuals, each of whom has an associated global feature vector $\mathbf{v}=[\mathbf{v}_{c}^{T} \ \  \mathbf{v}_{v}^{T}]^{T}\in\mathbb{R}^k$ which contain \textbf{all} demographic information and indicators correlated with a person's health which the agents use to make their decisions. $\mathbf{v}_{c}$ represents the subset of constant features in $\mathbf{v}$ which remain constant throughout the entire simulation, while $\mathbf{v}_{v}$ represents a person's variable features which are updated based on the actions made by the different agents. 


To ensure that data we use contain realistic features, we use realworld census data curated from the Integrated Public Use Microdata Series (IPUMS) Medical Expenditure Panel Survey (MEPS) available under IPUMS Health Surveys~\cite{blewett2024ipums}. Our population is constructed from survey responses from 2014 to 2016. These responses are converted to feature vectors using the variables listed in Table~\ref{tab::Healthcare_indicators}. All responses that contain missing values for any survey questions associated with these variables are filtered from the population. Each of these feature vectors is then augmented to include information associated with the dynamics of the environment, such as INSURED, which specifies whether or not a person has insurance at a particular time step.

The variables in $\mathbf{v}_{v}$ may be influenced by the actions taken by different agents. For example, public health subsidies funded by the Central Planner Agent can improve general health variables, while insurance subsidies can increase the likelihood of an individual having health coverage. These evolving features provide the necessary observations for the agents to adjust their strategies at each time step.

In the remainder of this section, we use subscript notation to refer to the value of a particular variable for an arbitrary individual or group at time $t$. For instance, $\text{INCTOT}_t$ refers to the total income of an individual at time $t$, while $\text{INCTOT}_{g,t}$ refers to the total income of an individual in sensitive group $g$ at time $t$. Similarly, other variables such as insurance status (INSURED), health indicators, and demographic factors will be indexed with subscripts to track changes over time for specific individuals or groups.

Further details on how each agent influences these features are provided in the following discussion.

\textbf{Insurance Agent ($\boldsymbol{\pi}_1$):} Every $k$ time steps the Insurance Agent must decide to offer an insurance package containing of a set premium to all individuals in the global population. Let $\mathbf{V}\in\mathcal{V}$ represent the matrix whose rows represent the feature vectors associated with these $N$ individuals. The Insurance Agent, $\boldsymbol{\pi}_1: \mathcal{V} \rightarrow [0,1]^{N}$, is responsible for determining the premium offered to each individual in the system by producing a value in the range $[0,1]$. This value is then scaled to establish a recurring premium over the next $k$ time steps, with the scaling factor ensuring that the premium falls within the allowable range, from 0 to the maximum permissible amount. Each customer then decides whether or not he/she will accept this premium for the duration of the ensuing cycle or not. We elaborate on how we model customer decisions in the mathematical modeling discussion we provide later in this section.

\textbf{Hospital Agent ($\boldsymbol{\pi}_2$):} Once a person becomes sick, they are reclassified from the healthy population to become part of the sick population. At time step $t$, $N_{2,t}$ individuals are waiting for a hospital bed. Let $\mathbf{D}\in\mathcal{D}$ represent the matrix whose rows represent the feature vectors associated with these $N_{2,t}$ individuals. The Hospital Agent, $\boldsymbol{\pi}_2:\mathcal{D}\rightarrow[0,1]^{N_{2,t}}$, 
produces a score for each one of these individuals in the range $[0,1]$ which are used to reorder the global hospital queue (in descending order). The queue for each local hospital is then determined by segmenting the sorted scores of the individuals in the global hospital queue that belong to a particular geographic regions. Individuals with scores at the top of the queue are then provided with beds based on their local hospital's availability.

\begin{figure}[t!]
    \centering
    \includegraphics[width=0.99\linewidth]{figures/health_omn_tree_v2.png}
    \caption{Action structure of Central Planner.} 
    \label{fig::health_omn_tree}
    \vspace{-0mm}
\end{figure}

\textbf{Central Planner Agent ($\boldsymbol{\pi}_3$):} The Central Planner Agent makes decisions that improve outcomes for the different entities within the system by allocating its budget to three types of subsidies---insurance subsidies for customers, public health subsidies, and hospital infrastructure subsidies. To make informed decisions, it receives the feature information of the global population. Namely, let $\mathbf{D}\in\mathcal{D}$ represent the matrix whose rows represent the feature vectors associated with all $N_{3,t}$ individuals in the global population at time $t$ and assume that there are $N_g$ geographic regions in the environment. Then, the Central Planner Agent, $\boldsymbol{\pi}_3:\mathcal{D}\rightarrow[0,1]^{3N_g+3}$, produces actions that can be represented by a tree structure, as illustrated in Figure~\ref{fig::health_omn_tree}. Given the Central Planner Agent's budget at time $t$, the first four elements of its action vector correspond with the middle level of nodes in this tree and represent the percentage of budget allocated to each of the three categories of subsidies and rollover funds for the next time step. The remaining $3N_g$ values represent the leaves of this tree and determine the percentage of each subsidy allocated to each of the $N_g$ geographic regions. Letting $a_{3,t}$ represent the action taken by the Central Planner Agent, $\boldsymbol{\pi}_3$, at time $t$, we have that $\sum_{i=0}^{3}a_{3,t}(i)$, $\sum_{i=4}^{N_g+3}a_{3,t}(i)$, $\sum_{i=N_g+4}^{2N_g+3}a_{3,t}(i)$, and $\sum_{i=2N_g+4}^{3N_g+3}a_{3,t}(i)$ should all equal 1. Thus, the product of actions taken along a path from the root of the tree to an arbitrary leaf provides the percentage of the agent's budget allocated to a particular subsidy in a given geographic region or rollover investment.

\begin{table}[t]
\caption{Healthcare MAFE Component Indicators}
\label{tab::health_indicators}
\centering
\resizebox{0.95\columnwidth}{!}{%
\begin{tabular}{ p{0.15\linewidth} || p{0.75\linewidth}}
 \hline
 Indicator& Description\\
 \hline
 $P_t$ & Insurance profits at time step $t$ \\ 
 \hline
 $N_{g,t}^{G}$ & Total number of people in Region $g$ at time step $t$ \\ 
  \hline
 $N_{g,t}^{I}$ & Number of people insured in Region $g$ at time step $t$ \\ 
 \hline
 $N_{g,t}^{H}$ & Number of healthy people in Region $g$ at the start of time step $t$ \\
 \hline
 $N_{g,t}^{S}$ & Number of people who become sick in Region $g$ at time step $t$ \\
 \hline
 $N_{g,t}^{T}$ & Number of people whose illnesses terminated in Region $g$ at time step $t$ \\
 \hline
 $N_{g,t}^{M}$ & Number of moralities in Region $g$ at time step $t$ \\
 \hline
\end{tabular}
}
\end{table}

\textbf{Indicators for Measuring Rewards and Fairness:} 
At the end of time step $t$, the environment returns a collection of indicators used to measure rewards and fairness violations within the system. A summary of these indicators is provided in Table~\ref{tab::health_indicators}. These indicators can be used to construct the following set of rewards that motivate these agents in the real world: insurance profits ($P_{t}$), insured rates ($\frac{\sum_t \sum_g N_{g,t}^{I}}{\sum_t \sum_g N_{g,t}^{G}}$), (negative) incidence rates ($-\frac{\sum_t \sum_g N_{g,t}^{S}}{\sum_t \sum_g N_{g,t}^{H}}$), and (negative) mortality rates ($-\frac{\sum_t \sum_g N_{g,t}^{M}}{\sum_t \sum_g N_{g,t}^{T}}$). 

The remaining environmental indicators provided by the system are used to measure fairness by tracking disparities in different rates over different geographic regions in the environment over time. In particular, this information can be used to analyze three fairness disparities within the system among $N_g$ geographic regions; namely, we can analyze disparities in insured rates ($\frac{\sum_t N_{g,t}^{I}}{\sum_t N_{g,t}^{G}}$), incidence rates ($\frac{\sum_tN_{g,t}^{S}}{\sum_t N_{g,t}^{H}}$), and mortality rates ($\frac{\sum_tN_{g,t}^{M}}{\sum_t N_{g,t}^{T}}$) across geographic regions using the standard deviation measure from Equation~\ref{eq::std_measure}. Hence, the indicators provided by the environment at each time step are used to measure four rewards and three fairness disparities.

\textbf{Mathematical Modeling:} 

\underline{Health Risk Scores:} 

A linear regression is trained to take a customer's feature vector at time $t$, $\mathbf{v}_t$, and produce a health risk score, $\text{HEALTH}_t$, in the range $[1,5]$ using the IPUMs health dataset. A higher value of $\text{HEALTH}_t$ indicates that a participant has worse health and is thus at increased risk of illness at time $t$. To ensure that the outputs of the linear regression are bounded within this range, the final health score is given after applying the clip operation to the original health score outputs, e.g. $clip(\text{HEALTH}_t,1,5)$. 


\begin{figure}[t!]
    \centering
    \includegraphics[width=0.9\linewidth]{figures/state_transition.png}
    \caption{Health state transition.} 
    \label{fig:state_transition}
    \vspace{-0mm}
\end{figure}

\underline{Health Transition Likelihoods:} 

An individual in this MAFE may transition across three health states in this simulation---namely, they may be healthy, ill, or deceased, as illustrated by the graph shown in Figure~\ref{fig:state_transition}. At the beginning of the simulation, every individual resides in the healthy state. As an episode progresses, each person may transition between states according to the state transition probabilities. As depicted in Figure~\ref{fig:state_transition}, let $P^{Sick}_t, P^{Death}_t, \text{ and } P^{Cured}_t$ represent the conditional probabilities that individuals who are healthy become ill, individuals who are ill to pass away, and individuals who are ill become healthy at time $t$. 
    
These transition probabilities are directly and indirectly influenced by the actions taken by the agents within the system. We model the likelihood of an individual who is not sick becomes sick as being positively correlated with a person having poor health (e.g. positively correlated with the value of $\text{HEALTH}_t$) and negatively correlated with having health insurance (e.g. negatively correlated with the binary value of $\text{HICOV}_t$, with a value of 1 indicating that a person has health insurance), given by the following equation:
\begin{equation}
    P^{Sick}_t = A(1-\text{HICOV}_t) + \frac{B}{5}\text{HEALTH}_t.
\end{equation}
To ensure that $P^{Sick}_t$ is a probability, $A \text{ and } B$ must be chosen to ensure that $A+\frac{B}{5}\in [0,1]$ (where the factor of 5 is included since $\text{HEALTH}_t\in[1,5]$).

We model the probability that a sick person passes away, $P^{Death}_t$, as the product of two probabilities: the probability that their illness terminates, $P^{Terminate}$, and the probability that the termination is due to mortality (rather than recovery), $P^{Mortality}$. That is,
\begin{equation}
    P^{Death}_t = P^{Terminate}_t P^{Mortality}_t.
\end{equation}

Similarly, the probability that a person that is sick is cured is given by
\begin{equation}
    P^{Cured}_t = P^{Terminate}_t(1-P^{Mortality}_t).
\end{equation}

Both $P^{Terminate}_t$ and $P^{Mortality}_t$ are modeled using an exponential family of functions of the form:
\begin{equation}
    C + D^{E\cdot\text{ILLTIME}_t+F \cdot\text{WAITBED}_t + G \cdot\text{HEALTH}_t +H},
\end{equation}
where $\text{ILLTIME}_t$ represents the number of consecutive time steps that a person with an illness has had it as of time step $t$, $\text{WAITBED}_t$ represents the amount of time that a person who is ill had to wait before receiving a hospital bed as of time step $t$, and $\text{HEALTH}_t$ specifies a person's general health quality as of time step $t$.
    
We now provide the intuition we consider for making our parameter selections, though we note that this is only one way of modeling these probabilities. These parameter choices, and the functional forms, themselves, can be adapted by users of our MAFEs as they see fit. 

We select $\text{ILLTIME}_t$ to be negatively correlated with $P^{Terminate}_t$ and positively correlated with $P^{Mortality}_t$ as an illness may be more likely to be resolved the longer one has it, but a longer illness could indicated it is more serve and may increase the likelihood that someone dies from it. One the other hand, and increase value of $\text{HEALTH}_t$ means someone has poorer overall health. Since it may take someone with poorer health more time to fend off an illness, putting them at increased risk of mortality, $\text{HEALTH}_t$ we specify its coefficient parameter to make it positively correlated with $P^{Terminate}_t$ and $P^{Mortality}_t$. Similarly, the longer it takes someone to receive a hospital bed, the longer and illness may fester since he/she may be unable to receive the appropriate care needed to cure it. As a result, we ensure that $\text{WAITBED}_t$ is positively correlated with $P^{Terminate}_t$ and $P^{Mortality}_t$.
  
\underline{Cost of Hospital Infrastructure:}

Hospital infrastructure refers to the physical facilities needed to increase the number of available beds in a hospital. Building new infrastructure involves two main costs: a base cost, which is incurred for any construction plan, and a proportional cost, which depends on the number of new beds being built. The total cost of building new infrastructure is modeled as a linear function, where the base cost is added to the cost that increases with the number of new beds. This creates a trade-off for the Central Planner Agent, which must decide when to invest in infrastructure. Investing in small projects repeatedly can become expensive due to the base cost, while waiting to fund a larger project may lead to insufficient hospital resources and more deaths.

\underline{Time to Build Hospital Infrastructure:}

The time required to build new hospital infrastructure is modeled similarly to the cost of infrastructure, with a different interpretation of the variables. The time required for construction depends on the size of the project. There is a base amount of time required for planning and setting up the project, and additional time required is linearly proportional to the number of new beds added by the project.

\underline{Individual's Likelihood of Accepting Insurance:}

An individual's willingness to pay for insurance depends on a number of factors whether or not his/her insurance premiums is reasonably priced (which is relatively determined by a person's financial well-being, e.g. their net worth), their age, and their health, the size of their family, and so on. To strike a balance between complexity and fidelity, we model this as a function of the following factors: net family income ($\text{FTOTVAL}_t$), household size ($\text{FAMSIZE}_t$), and the monthly premium ($\text{HIPCOST}_t$) a customer would be required to pay should he/she accept health insurance. This is done by sampling a Bernoulli distribution, ${\rm Bernoulli}(P^{Insured}_t)$, where $P^{Insured}_t$ is given by:
\begin{equation}
     P^{Insured}_t = 1-e^{\frac{\text{FTOTVAL}_t}{\text{HIPCOST}_t(\text{FAMSIZE}_t)}}.
\end{equation}

\underline{Distributing Insurance Subsidies:}

The final premium for health insurance that a customer is offered is determined by subtracting the amount subsidized by the Central Planner Agent from the initial price set by the Insurance Agent. However, rather than making case-by-case decisions on subsidy allocation, the Central Planner Agent designates a fixed budget for subsidizing insurance within each geographic region, as described in the description of the Central Planner Agent. A rule is then applied to distribute these funds proportionally to all individuals within each region. Specifically, subsidies are inversely weighted by each individual's per capita household income. Let $\text{FTOTVAL}_{g,t}(i)$ represent the per capita income of the 
$i^{th}$  individual among $N_g$ members living in Region $g$ at time $t$. The fraction of the total subsidy allocated to this individual is calculated as:
\begin{equation}
    w_i = \frac{\frac{1}{\text{FTOTVAL}_{g,t}(i)}}{\sum_{n=1}^{N_g} \frac{1}{\text{FTOTVAL}_{g,t}(n)}}.
\end{equation}

\underline{Effect of Public Health Investment:}

In each time step, a subset of the updateable features in $\mathbf{v}_v$ associated with each individual in Region $g$ will improve with probability $P_{g,t}^{improve}$, remain unchanged with constant probability $U$, or deteriorate with probability $1-P_{g,t}^{improve}-U$. We treat $U$ as a user specified constant. The value of $P_{g,t}^{improve}$ is affected by the amount of the Central Planner Agent's budget that is used on public health expenditures in Region $g$ at time step $t$. In particular, we model $P_{g,t}^{improve}$ as a function of the amount of the planners budget invested in the region in which this individual is located at time $t$. For constant hyperparameters $Q,R,V, \text{ and } W$, this is given by the following equation:
\begin{equation}
    \label{eq:improve_function}
    P_{g,t}^{improve}(x) = Q + R\sigma(V\cdot x + W)
\end{equation}
where $\sigma$ represents a sigmoid function. We assume this equation is tuned so that $P_{g,t}^{improve}$ is non-negative and 
\begin{equation}
    \sup_{x} P_{g,t}^{improve}(x) + U= 1.
\end{equation}
To determine if an individuals features improve, deteriorate, or remain unchanged we sample a uniform distribution over the range $[0,1]$ and update the features appropriately based on the segment in which the output value lands---$[0,P_{g,t}^{improve}]$, $(P_{g,t}^{improve},P_{g,t}^{improve}+U]$, or $(P_{g,t}^{improve}+U,1]$.


\textbf{Episode Termination:} An episode may terminate for three reasons. First, if the agents produce actions that lead them to successfully reach the user specified terminal time step, the episode terminates. Conversely, the environment may also terminate early if any entity in the institution fails. Particularly, if the Insurance Agent ever has net negative profits. at is, if the income it receives from premium payments is outweighed by the cost of paying for customer's hospital stays over the entirety of an episode. The episode also fails if the entire living population in the simulation is depleted, we consider the episode a failure.