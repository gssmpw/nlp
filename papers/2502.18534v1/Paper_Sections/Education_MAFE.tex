\label{sec::education_MAFE}

\begin{figure*}
    \centering
    \includegraphics[width=0.99\linewidth]{figures/Education_Diagram_v2.png}
    \caption{Education MAFE Diagram} 
    \label{fig:Education_Diagram}
    \vspace{-0mm}
\end{figure*}

\textbf{Overview:} A diagram outlining the design of our Education MAFE is provided in Figure~\ref{fig:Education_Diagram}. This environment is designed to simulate the school-to-employment pipeline by modeling three key entities involved in this process: a university system, the employers of each individual, and an central planner (which functions as a central planner or government-like entity). The Central Planner Agent ($\boldsymbol{\pi}_{4}$) and Employer Agent ($\boldsymbol{\pi}_{3}$) are both modeled using a single agent. However, two separate agents are used to model distinct processes within the university: the Admissions Agent ($\boldsymbol{\pi}_{1}$), which determines which applicants are admitted or rejected, and the University Budget Allocation Agent ($\boldsymbol{\pi}_{2}$), which decides how to allocate the university’s budget across various expenses. The decisions made by these agents collectively shape the students' future success.

At each time step, the population is categorized into three groups: the \textbf{tertiary population} (individuals not actively involved in the simulation), the \textbf{higher education population} (degree-seeking students within the university system), and the \textbf{working population} (individuals employed in the workforce). The tertiary population consists of individuals who are not currently involved in the higher education pipeline. At each time step, a subset of these individuals is sampled from the tertiary population, with each passing through the education system for a fixed number of time steps, representing their journey from enrollment to career termination, before being returned to the tertiary population for future resampling.

When individuals sampled from the tertiary population apply to college, the University Admissions Agent decides who will be accepted into the higher education system to pursue one or more degrees. Those who are rejected immediately enter the workforce. At any given time step, an individual within the university system may choose to exit and join the workforce, with the length of time they have spent in the university system determining the highest degree they have earned. The longer they stay in the university, the higher the degree attained.

The number of individuals the university can accept and support successfully depends on the University Budget Allocation Agent, which determines how the university allocates the funds it has accrued at each time step. These funds are distributed across various resources that the university believes will lead to the best student outcomes, as measured by the rewards provided by the system.

The Central Planner Agent also operates with a budget at each time step, which it allocates across various expenditures that influence individuals' educational and career success. These expenditures include tertiary investments (which improve the quality of education children receive in their formative years), university budget investments (which serve as a secondary source of funding, aside from tuition), and diversity incentives (which may be provided to the employer agent to encourage salary equity in the workforce).

Once an individual enters the workforce, they remain there until the number of time steps they have spent in the simulation reaches the limit, $N$. During this time, the Employer Agent sets the salary for each worker, which directly affects their productivity. Upon reaching the terminal time step, the individual is removed from the environment, their features are updated, and they are returned to the tertiary population, where they may be resampled for a future pass through the system. This process continues until the episode is terminated.

Ultimately, the collective decisions made by these agents determine individuals' academic and career success within the system. In the following sections, we provide a detailed description of the roles and operations of each agent at a given time step, $t$.

\textbf{Population:} 
At the beginning of the education simulation, a global population is initialized which consists of $N$ individuals, each of whom has an associated global feature vectors, $\mathbf{v}=[\mathbf{v}_{c}^{T} \ \  \mathbf{v}_{v}^{T}]^{T}\in\mathbb{R}^k$ which contain \textbf{all} demographic information and indicators correlated with a person's experience and academic merits which the agents use to make their decisions. $\mathbf{v}_{c}$ represents the subset of constant features in $\mathbf{v}$ which remain constant throughout the entire simulation, while $\mathbf{v}_{v}$ represents a person's variable features which are updated based on the actions made by the different agents. 


To ensure that data we use contain realistic features, we use real-world census data curated from the Integrated Public Use Microdata Series (IPUMS) Higher Ed (EDUC) Surveys~\cite{ipums_higher_ed_2016}. Our population is constructed from survey responses from 2014 to 2016. These responses are converted to feature vectors using the variables listed in Table~\ref{tab::Education_indicators}. All responses that contain missing values for any survey questions associated with these variables are filtered from the population. Each of these feature vectors is then augmented to include information associated with the dynamics of the environment, such as TIMEINUNIV, which specifies the amount of time  an individual has spent in the university through the current time step.

The variables in $\mathbf{v}_{v}$ may be influenced by the actions taken by different agents. For example, if the university detects structural performance disparities among different demographic groups, it could allocate more of its budget to providing mentorship programs to the disadvantaged group, thereby increasing their likelihood of obtaining higher GPAs and affecting the CURRENTGPA feature. Alternatively, the Central Planner could allocate funds for employer incentives to mitigate salary-based disparities among members of different demographic groups, thus affecting the SALARY feature.

In the remainder of this section, we use subscript notation to refer to the value of a particular variable for an arbitrary individual or group at time $t$. For example, $\text{GPA}_t$ refers to an individual's cumulative GPA at time $t$, while $\text{GPA}_{g,t}$ refers to the GPA of an individual with sensitive attribute $g$ at time $t$. This subscript notation allows us to track how variables, such as GPA and time in university, evolve over time for specific individuals or groups, including those based on demographic characteristics.

Further details on how each agent influences these features are provided in the following discussion.

\begin{table*}[ht!]
\caption{Education MAFE Features}
\label{tab::Education_indicators}
\centering
\begin{tabular}{ p{0.20\linewidth} || p{0.15\linewidth} || p{0.20\linewidth} || p{0.33\linewidth}}
 \hline
 Variable & Origin & How it is Updated & Description\\
 \hline
 SEX & IPUMS EDUC& None& Sex\\ 
 \hline
 MINRTY & IPUMS EDUC& None& Minority indicator\\ 
 \hline
 RACE & IPUMS EDUC& None& Main racial background\\  
 \hline
  NBAMEMG & IPUMS EDUC& None& Field of major first degree\\
 \hline
 NDGMEMG & IPUMS EDUC& None& Field of major highest degree\\
 \hline
 REGION & IPUMS EDUC& None & Region of the country lived in\\
 \hline
 NOCPRMG & IPUMS EDUC& None & Job code for principal job (major group)\\
 \hline
 SALARY & IPUMS EDUC& Employer ($\boldsymbol{\pi}_3$)& Salary (annualized) \\ 
 \hline
 HRSWK & IPUMS EDUC& Central Planner ($\boldsymbol{\pi}_4$) & Principal job hours worked\\
\hline
  EMSEC & IPUMS EDUC& Central Planner ($\boldsymbol{\pi}_4$) & Employer sector\\
 \hline
 EMSIZE & IPUMS EDUC& Central Planner ($\boldsymbol{\pi}_4$) & Size of employer\\
 \hline
 UGLOAN & IPUMS EDUC& Central Planner ($\boldsymbol{\pi}_4$) & Total amount taken out for undergraduate loans\\
 \hline
 GRLOAN & IPUMS EDUC& Central Planner ($\boldsymbol{\pi}_4$) & Total amount taken out for graduate loans\\
 \hline
  DGRDG & IPUMS EDUC& Environment Dynamics & Type of highest certificate or degree\\ 
 \hline
  GPA & IPUMS EDUC& Environment Dynamics, Central Planner ($\boldsymbol{\pi}_4$)& Cumulative College GPA \\ 
\hline
 INENV & Environment & Environment Dynamics & Indicator specifying if person was sampled to become part of the environment\\
 \hline
 INWORKF & Environment & Environment Dynamics & Indicator specifying if person in environment is in workforce\\
 \hline
 INUNIV & Environment & Environment Dynamics & Indicator specifying if person in environment is in university\\
 \hline
 INMINTYPGRM & Environment & Environment Dynamics & Indicator specifying if person in university if in minority mentorship program\\
 \hline
 CURRENTGPA & Environment & Environment Dynamics & GPA of student in university at current time step\\
 \hline
 PLANBUDGET & Environment & Environment Dynamics & Central planner current budget\\
 \hline
 UNIVBUDGET & Environment & Environment Dynamics & University's current budget\\
 \hline
 ANNUALTUIT & Environment & Environment Dynamics & Student's annual tuition (scholarship adjusted)\\
 \hline
 N\_UNIV\_UNITS & Environment & Environment Dynamics & Number of university infrastructure units\\
 \hline
 N\_FACULTY & Environment & Environment Dynamics & Number of university faculty\\
 \hline
 N\_STUDENTS\_CURR & Environment & Environment Dynamics & number of students in university\\
 \hline
 TIMEINUNIV & Environment & Environment Dynamics & Time student has spent in university (nonzero if INENV=1 and INUNIV=1)\\
 \hline
 TIMEINWORKF & Environment & Environment Dynamics & Number of time steps person has been in university (nonzero if INENV=1 and INWORF=1)\\
 \hline
 TIMEINENV & Environment & Environment Dynamics & Number of time steps person has been in environment (nonzero if INENV=1)\\
 \hline
 DIVINVEST & Environment & Environment Dynamics & Amount of money Central Planner allocates to employer diversity incentives\\
 \hline
 AGE & Environment & Environment Dynamics & Age of person in environment \\
 \hline
 AVE\_SALARY & Environment & Environment Dynamics & Average salary of person over entirety of work career\\
 \hline
\end{tabular}
\end{table*}


\textbf{University Admissions Agent ($\boldsymbol{\pi}_1$):}
Different from the standard ML setup in which an admissions agent is represented by a classifier who accepts any students whose scores fall above a given (typically 0.5) threshold, we take a resource constrained approach to modeling admissions. In particular, we assume that for the university to provide quality instruction to students, there is a cap on the size of the student-instructor ratio. Thus, there is a limit to the number of students that may be admitted to the university at time $t$ which depends on the number of students already in the university and the number of instructors employed by the university at time $t$. At the same time, it is essential for the university to raise money to pay for expenses such as teacher salaries and infrastructure. Thus, the university should always admit as many students as it can without violating the student-instructor ratio cap so as to ensure that no available classroom seats are left empty. With this in mind, our admission agent operates as follows.

At time step $t$, a collection of $N_{1,t}$ individuals are sampled from the tertiary population to apply for college. Let $\mathbf{D}\in\mathcal{D}$ represent the matrix whose rows represent the feature vectors associated with these $N_{1,t}$ individuals. The admissions agent, $\boldsymbol{\pi}_1:\mathcal{D}\rightarrow[0,1]^{N_{1,t}}$, 
produces a score for each of these individuals in the range $[0,1]$, which is used to rank students in terms of who the university most desires to admit. Students are then admitted in order of their rank until all available slots at the university have been filled. Those who are rejected immediately enter the workforce.

\textbf{University Budget Allocation Agent ($\boldsymbol{\pi}_2$):}
The University Budget Allocation Agent makes decisions that affect the proper functioning of the university, which have consequences for student success. In particular, given a budget, this agent allocates these funds to four primary expenses---university infrastructure, staff salaries, scholarships, and minority mentorship programs which have the potential to improve the performance of underrepresented groups in higher education. To make informed decisions, it receives the feature information for the higher education population. Namely, let $\mathbf{D}\in\mathcal{D}$ represent the matrix whose rows represent the feature vectors associated with all $N_{2,t}$ students currently in the university system at time $t$. Then, the University Budget Allocation Agent, $\boldsymbol{\pi}_2:\mathcal{D}\rightarrow[0,1]^{5}$, produces four actions that represent the percentages of its budget that are allocated to each of the four expenses it is allowed to pay, plus an amount that it is allowed to roll over for future budgeting, such as for investing in larger infrastructure projects than it currently can afford. Thus, letting $a_{2,t}$ represents the actions taken by the University Budget Allocation Agent, $\boldsymbol{\pi}_2$, at time $t$, we have that this action must be constrained such that  $\sum_{i=1}^{5}a_{2,t}(i)$ equals 1.

\textbf{Employer Agent ($\boldsymbol{\pi}_3$):} 
At time step $t$, the workforce population consists of $N_{3,t}$ people, for each of whom the Employer Agent provides a salary. Let $\mathbf{V}\in\mathcal{V}$ represent the matrix whose rows represent the feature vectors associated with these $N_{3,t}$ individuals. The Employer Agent, $\boldsymbol{\pi}_3: \mathcal{V} \rightarrow [0,1]^{N_{3,t}}$, is responsible for determining the salary for each individual in the workforce by producing a value in the range $[0,1]$. This value is then scaled to establish an annual salary for the ensuring time step, with the scaling factor ensuring that the salary falls within the allowable range, from 0 to the maximum permissible amount. Here, the employer agent is not meant to be interpreted as a single employer. Rather, it can be thought of as a tool that decides the salary of a particular person for the job at which they work, whatever that job may be. The goal of this agent is to set this salary so that the utility the employer received from each worker is maximized. We elaborate on how we quantify utility in our ensuing discussion.

\begin{figure}[t!]
    \centering
    \includegraphics[width=0.99\linewidth]{figures/education_omn_tree_v2.png}
    \caption{Action structure of Education Central Planner.} 
    \label{fig::education_omn_tree_v2}
    \vspace{-0mm}
\end{figure}

\textbf{Central Planner Agent ($\boldsymbol{\pi}_4$):}
The Central Planner Agent makes decisions that improve outcomes for the different entities within the system by allocating its budget for three types of investments---investments in tertiary education resources, university funding, and diversity incentives for employers. To make informed decisions, it receives the feature information of the global population. Namely, let $\mathbf{D}\in\mathcal{D}$ represent the matrix whose rows represent the feature vectors associated with all $N$ individuals in the global population at time $t$ and assume that there are $N_g$ geographic regions in which these students may have received their tertiary education in the environment. Then, the Central Planner Agent, $\boldsymbol{\pi}_4:\mathcal{D}\rightarrow[0,1]^{N_g+3}$ produces actions that can be represented by a tree structure, as illustrated in Figure~\ref{fig::education_omn_tree_v2}. Given its budget at time $t$, $B_t$, the first three elements of its action vector correspond with the middle level of nodes in this tree and represent the percentage of $B_t$ allocated to each of the three investment categories. Note, no rollover action is provided to this agent since there are no incentives for it to budget for future investment. The remaining $N_g$ values represent the leaves under the tertiary investment node in Figure~\ref{fig::education_omn_tree_v2} and determine the percentage of tertiary investment allocated to each of the $N_g$ geographic regions. Letting $a_{4,t}$ represent the action taken by the Central Planner Agent at time $t$, we have that $\sum_{i=1}^{3}a_{4,t}(i)$ and  $\sum_{i=4}^{N_g+3}a_{4,t}(i)$ should all equal 1.

\begin{table}[t!]
\caption{Education MAFE Component Indicators}
\label{tab::education_indicators}
\centering
\resizebox{0.95\columnwidth}{!}{%
\begin{tabular}{ p{0.15\linewidth} || p{0.75\linewidth}}
 \hline
 Indicator& Description\\
 \hline
 $P_t$ & Employer Profits at time step $t$ \\ 
 \hline
 $A_{U,t}^{g}$ & Number of people that applied to university from Group $g$ at time step $t$\\ 
  \hline
 $E_{g,t}^{U}$ & Number of students that entered university from Group $g$ at time step $t$\\ 
  \hline
 $C_{g,t}^{U}$ & Initial number of students in undergraduate class currently graduating from Group $g$ at time step $t$\\ 
  \hline
 $G_{g,t}^{U}$ & Number of students that graduated from undergraduate program from Group $g$ at time step $t$ \\ 
 \hline
  $C_{g,t}^{M}$ & Initial number of students in undergraduate class currently graduating from Group $g$ at time step $t$\\ 
  \hline
 $G_{g,t}^{M}$ & Number of students that graduated from master's program from Group $g$ at time step $t$ \\ 
 \hline
  $C_{g,t}^{D}$ & Initial number of students in undergraduate class currently graduating from Group $g$ at time step $t$\\ 
  \hline
 $G_{g,t}^{D}$ & Number of students that graduated from doctoral program from Group $g$ at time step $t$ \\ 
 \hline
 $W_{g, t}$ & Number of people in the workforce from Group $g$ at time step $t$ \\ 
 \hline
 $S_{g,t}$ & Sum of all salaries of people in workforce from Group $g$ at time step $t$ \\ 
 \hline
\end{tabular}
}
\end{table}

\textbf{Indicators for Measuring Rewards and Fairness:}

At the end of time step $t$, the environment returns a collection of indicators used to measure rewards and fairness violations within the system. A summary of these indicators is provided in Table~\ref{tab::education_indicators}. These indicators can be used to construct the following set of rewards that motivate these agents in the real world: employer profits ($P_t$), admissions rates ($\frac{\sum_t \sum_g  E_{g,t}^{U}}{\sum_t  \sum_g  A_{U,t}^{g}}$), and graduation rates for undergraduate, Master's, or doctoral degrees ($\frac{\sum_t \sum_g G_{g,t}^{U}}{\sum_t \sum_g C_{g,t}^{U}}$, $\frac{\sum_t \sum_gG_{g,t}^{M}}{\sum_t \sum_gC_{g,t}^{M}}$, and $\frac{\sum_t \sum_gG_{g,t}^{D}}{\sum_t \sum_gC_{g,t}^{D}}$), and average salaries ($\frac{\sum_t \sum_gS_{g,t}}{\sum_t\sum_g W_{g, t}}$). 

The remaining environmental indicators provided by the system are used to measure fairness by tracking disparities among different rates provided for each demographic group at time step $t$. In particular, this information can be used to analyze five fairness disparities within the system among the two sensitive groups; namely, we can analyze disparities in: admissions rates ($\frac{\sum_t E_{g,t}^{U}}{\sum_t A_{U,t}^{g}}$); graduations rates for undergraduate, Master's and doctoral programs ($\frac{\sum_t G_{g,t}^{U}}{\sum_t C_{g,t}^{U}}$, $\frac{\sum_t G_{g,t}^{M}}{\sum_t C_{g,t}^{M}}$, and $\frac{\sum_t G_{g,t}^{D}}{\sum_t C_{g,t}^{D}}$); and salaries ($\frac{\sum_t S_{g,t}}{\sum_t W_{g, t}}$). Hence, the indicators provided by the environment at each time step are used to measure five rewards and five fairness disparities.

\textbf{Mathematical Modeling:} 
    
\underline{Student GPA Dynamics:}

We model a student's cumulative at time step $t$, $\text{GPA}_t$, as a random process given by the following recursion:
\begin{equation}
    \label{eq:GPA}
    \text{GPA}_t = \frac{(t-1)\text{GPA}_{t-1} + \widehat{\text{GPA}}_{t}}{t},
\end{equation}
where $\widehat{GPA}_{t}$ represents a student's semester GPA at time step $t$. We model $\widehat{\text{GPA}}_{t}$ as being a noisy estimate of the student's previous semester GPA, $\widehat{\text{GPA}}_{t-1}$, assuming that the GPA that the student most recently received is most indicative of the trajectory of their performance in classes. That is,
\begin{equation}
    \widehat{\text{GPA}}_{t} = \widehat{\text{GPA}}_{t-1}+ \epsilon,
\end{equation}
where $\epsilon\sim {\rm Uniform}[-\Delta,\Delta]$ for some constant $\Delta$.

The final critical ingredient required for completing the modeling of a student's GPA is to determine how to set $\widehat{\text{GPA}}_{0}$, the initial condition for Equation~\ref{eq:GPA}. For this task, we model $\widehat{\text{GPA}}_{0}$ as a noisy function of the subset of an individual's feature vector, $\mathbf{u}\subset \mathbf{v}$, containing features from the IPUMS EDUC dataset given by: 
\begin{align}
    \label{eq:init_GPA}
    \widehat{\text{GPA}}_{0} = f(\mathbf{u}) &+ \gamma_0 + \gamma_1\cdot (1-\text{ANNUALTUIT}) \notag \\
    &+ \gamma_2\cdot \text{INMINTYPGRM} 
\end{align}
We obtain $f$ through training a regressor using the samples available in the IPUMS EDUC dataset where all IPUMS EDUC features from Table~\ref{tab::Education_indicators} are treated as the independent variables and $\text{GPA}$ is treated as the dependent variable. We particularly use ridge regression for this task. $\gamma_1$ and $\gamma_2$ are user-specified weights that introduce the effect that student supports provided by the University Budget Allocation Agent have on improving student progress through the university. For these terms, we assume that ANNUALTUIT is normalized to be a percentage (between 0 and 1) and INMINTYPGRM is a binary valued variable.$\gamma_0\sim {\rm Uniform}[-\delta+C,\delta+C]$ is used to introduce stochasticity in baseline GPAs and is represented by uniform random noise over a window of length $2\delta$. $C$ centers this window and is adjusted based on the academic supports provided to as student. If an individual receives a significant scholarship or is provided an academic mentor, then $C>0$. Otherwise, $C=0$.  Taken collectively, $f$ represents measures an individual's baseline academic merits, while $V$ represents intervention adjusted uncertainty in an individual's performance.

\underline{Likelihood of Leaving College:}

When deciding whether remaining enrolled in school is beneficial, a student must way a variety of factors, his/her performance thus far, the tradeoff in time that could be spent elsewhere, and the price paid for tuition. Thus, we obtain the likelihood that an individual leaves college at time step $t$ through sampling Bernoulli distribution, ${\rm Bernoulli}(P^{Leave}_t)$, where $P^{Leave}_t$ is given by:
\begin{align}
    P^{Leave}_t = \sigma(&\alpha_0+\alpha_1\text{GPA}_t + \alpha_2\text{ANNUALTUIT}_t \notag\\
    &+ \alpha_3\text{TIMEINUNIV}_t + \alpha_4\text{TIMEINUNIV}_t^2).
\end{align}
$\text{GPA}_t$ and $\text{ANNUALTUIT}_t$ are modeled as linear functions with negative and positive effects, respectively, on a student's likelihood of leaving college. Therefore, we assume $\alpha_1 < 0$ and $\alpha_2 > 0$.

We represent the effect of enrollment duration on the likelihood of departure using an inverted quadratic function, reflecting the intuition that students are less likely to leave immediately after enrolling. Consequently, $\alpha_3 < 0$ and $\alpha_4 > 0$.

The rationale is as follows: During the initial period after enrollment, students may be more inclined to leave if their academic performance is poor or their expectations are unmet. However, as time progresses, the likelihood of departure decreases. This is because students invest increasing resources into their degree and draw closer to completion, making withdrawal less advantageous.

Finally, note that tuition is influenced by the amount of scholarship funding provided by the university.

\underline{Student-Teacher-Infrastructure Ratio:}

As previously discussed, we assume that the university’s ability to provide quality instruction to students is limited by the number of students it can enroll at any given time. This enrollment cap is dependent on the size of the faculty. However, the number of faculty members that can be supported on campus is in turn limited by the availability of infrastructure, such as classrooms, offices, and laboratories, which are necessary for both faculty research and instruction. Therefore, the number of faculty members and the available student seats on campus are both determined by the amount of infrastructure the university has.

Specifically,  the number of faculty members supported by the university at time $t$ is linearly proportional to the amount of infrastructure available. Similarly, the student enrollment capacity at any time is also linearly proportional to the infrastructure available. To align with common intuition, we set the proportionality constants governing faculty size and student enrollment to values significantly greater than one. This reflects the fact that multiple faculty members can occupy a single building, and many students are taught by a single faculty member. The ratio between the student enrollment capacity and the number of faculty indicates the student-to-faculty ratio, with larger ratios corresponding to larger class sizes.

\underline{Cost of Building University Infrastructure:}

By university infrastructure, we refer to all construction (including classrooms, laboratories, offices, etc.) that must take place to increase the student and faculty population capacities on a university campus. We use the same equations used to the model cost of building new hospital infrastructure here for building new university infrastructure, though the interpretation is changed. That is, building new infrastructure involves two main costs: a base cost, which is incurred for any construction plan, and a proportional cost, which depends on the number of new university infrastructure units built. The total cost of building new infrastructure is modeled as a linear function, where the base cost is added to the cost that increases with the number of new beds. This creates a trade-off for the university budget allocator planner, who must decide when to invest in infrastructure. Investing in small projects repeatedly can become expensive due to the base cost, while waiting to fund a larger project may limit the number of students the university can admit.

Notably, counter to the hospital MAFE, in the university MAFE, we also assume that building new university infrastructure comes with an additional recurring cost which represents then additional salaries for faculty and staff that are supported by the addition of this new infrastructure. 


\underline{Time to Build University Infrastructure:}

We model the time to build university infrastructure identically to cost of hospital infrastructure, but with a different interpretation. Specifically, the time required for construction depends on the size of the project. There is a base amount of time required for planning and setting up the project, and additional time required is linearly proportional to the number of new beds added by the project.

\underline{An Individual's Utility to An Employer:}

An employee's value to an employer may depend on a variety of factors that comprise his/her merits, including his/her years of experience, level of degree attainment, cumulative GPA, the salary he/she receives, and whether or not his/her hiring affects an employer's diversity incentives. Moreover, these factor may interact, making modeling the effect that they have on the profits made by an employer non-linear and thus more complicated. With this in mind, we model the profits an employee brings to an employer at time step $t$ using an inverted quadratic function of a person's salary, $SALARY_t$:
\begin{align}
    \label{eq::employer_profit}
    U(\text{SALARY}_t) =&\alpha_0 + \alpha_1 (\text{SALARY}_t+\text{DIVINVEST}_t) \notag\\
    &- \alpha_2 \text{SALARY}_t^2,
\end{align}
where $\alpha_0$ and $\alpha_1>0$ are user-defined parameters and $\alpha_2$ is a function of a person's cumulative college, $GPA$; the level of a persons highest degree attained, $DEGREE$; and the number of years of experience a person has working, $EXPERIENCE_t$. That is,  $\alpha_2$ takes the following form with user defined parameter's $\beta_0,...,\beta_3$:
\begin{align}
    \label{eq::employer_profit_coeff}
    \alpha_2= & \beta_0 + \beta_1 \text{GPA}_t + \beta_2 \text{TIMEINUNIV}_t \notag\\
    &+ \beta_3 (\text{EXPERIENCE}_t-\text{EXPERIENCE}_t^2)
\end{align}
To ensure that Equation~\ref{eq::employer_profit} takes an inverted quadratic form, The parametrization of Equation~\ref{eq::employer_profit_coeff} must be selected so that $\alpha_2>0$.

The intuition behind the design of Equation~\ref{eq::employer_profit} is as follows. An increase in employee income leads to a marginal improvement in productivity, which directly benefits employer profits. This positive relationship is captured by the linear term in Equation~\ref{eq::employer_profit}. On the other hand, paying an employee a higher salary also directly reduces the employer’s profits, which is modeled by the negative quadratic term in the same equation. The balance between these two effects depends on the interactions between employee salary and other factors captured by $\alpha_2$. The coefficients $\beta_0, \dots, \beta_3$ can be adjusted to reflect the relative influence of these factors on employer profits. We set these values based on the intuition that higher education and better educational performance justify higher wages for employees, as they are likely to increase productivity. The quadratic term for experience captures the dual effects of greater experience: while more experience may enhance job performance, it could also lead to less flexibility in work habits and reduced exposure to the latest industry developments, as newer educational techniques and trends are typically acquired earlier in a career.

\underline{Effect of Tertiary Investment:}

We use the same modeling as was performed to model the effect of public investment in Section~\ref{sec::healthcare_mafe} to model the effect of tertiary investment for the Education MAFE, just with different application interpretation. Namely, in each time step, a subset of the updateable features in $\mathbf{v}_v$ associated with each individual in Region $g$ will improve with probability $P_{g,t}^{improve}$, remain unchanged with constant probability $U$, or deteriorate with probability $1-P_{g,t}^{improve}-U$. We treat $U$ as a user specified constant. The value of $P_{g,t}^{improve}$ is affected by the amount of the Central Planner's budget that is used on tertiary investment in in Region $g$ at time step $t$. In particular, we model $P_{g,t}^{improve}$ as a function of the amount of the planners budget invested in the region in which this individual is located at time $t$. For constant hyperparameters $Q,R,V, \text{ and } W$, this is given by the following equation:
\begin{equation}
    \label{eq:improve_function}
    P_{g,t}^{improve}(x) = Q + R\sigma(V\cdot x + W)
\end{equation}
where $\sigma$ represents a sigmoid function. We assume this equation is tuned so that $P_{g,t}^{improve}$ is non-negative and 
\begin{equation}
    \sup_{x} P_{g,t}^{improve}(x) + U= 1.
\end{equation}
To determine if an individuals features improve, deteriorate, or remain unchanged we sample a uniform distribution over the range $[0,1]$ and update the features appropriately based on the segment in which the output value lands---$[0,P_{g,t}^{improve}]$, $(P_{g,t}^{improve},P_{g,t}^{improve}+U]$, or $(P_{g,t}^{improve}+U,1]$.


\textbf{Episode Termination:} An episode may terminate for three reasons. First, if the agents produce actions that lead them to successfully reach the user specified terminal time step, the episode terminates. Conversely, the environment may also terminate early if any entity in the institution fails. Particularly, if the university is ever unable to support the salaries of its staff and faculty due to improper allocation of its budget or a lack of enough money in the budget. An episode may also fail if net profits accumulated by the employer agent are ever negative.