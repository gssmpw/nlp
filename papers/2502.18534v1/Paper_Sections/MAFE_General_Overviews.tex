\label{env}

In this section, we introduce three distinct MAFEs, which simulate the dynamics of a loan processing pipeline, healthcare system, and higher education system. These MAFEs are implemented within a flexible framework for multi-agent environments, similar to popular MARL ecosystems like Gym, Gymnasium, and PettingZoo. Each MAFE follows the standard pattern of producing \texttt{observations}, \texttt{rewards}, and \texttt{dones}, and allows agents to interact through the \texttt{env.step(action)} method, which updates the environment based on the current action. In our implementation, \texttt{rewards} comprises both the reward and fairness component vectors. This design enables easy integration with existing MARL libraries, allowing researchers to extend the environments by defining new agents, reward functions, and fairness constraints. We summarize the three MAFEs below, and a detailed description of them is provided in Appendices~\ref{sec::loan_MAFE}-\ref{sec::education_MAFE}.

\paragraph{Loan MAFE:}
This MAFE simulates a financial institution's loan pipeline with three agents: (1) an \textit{Admissions Agent} that approves or rejects loan applications; (2) a \textit{Funds Disbursement Agent} that controls the timing of loan disbursements; and (3) a \textit{Debt Management Agent} that manages the percentage of debt adjustments applied to individuals' payments. Throughout an episode, individuals apply for loans, some of whom are approved while others remain in the applicant pool. Approved borrowers await fund disbursement before beginning to make regular payments. A borrower’s financial profile is positively updated after the full loan is repaid, while defaults have negative impacts. Borrowers rejoin the applicant pool after repayment or default, reflecting real-world financial cycles.
\vspace{-3mm}
\paragraph{Healthcare MAFE} 
This MAFE simulates a healthcare system involving three agents that represent an insurance company, a hospital, and a central planner. The \textit{Insurance Agent} determines insurance price offerings to individuals, influencing healthcare access. The \textit{Hospital Agent} provides treatment based on capacity constraints, allocating hospital beds to sick individuals in the population at each time step. The \textit{Central Planner Agent} allocates budgets periodically for hospital infrastructure, public health initiatives, and insurance subsidies. Individuals choose whether to purchase insurance, affecting their healthcare access and health outcomes. Patients in sick states seek treatment, where hospital capacity and treatment timing impact recovery.
\vspace{-3mm}
\paragraph{Education MAFE} 
This MAFE tracks population transitions across three stages: tertiary population, university students, and workforce members, reflecting progressions from higher education to employment. It includes four agents: a \textit{University Admissions Agent} that selects applicants for enrollment; a \textit{University Budget Allocation Agent} that distributes university funds for different resources, affecting student success and resource quality; an \textit{Employer Agent} that sets the salaries for the workforce; and a \textit{Central Planner Agent} that allocates resources for tertiary education, university funding, and workforce diversity incentives. Individuals transition from the tertiary population to the university and then to the workforce or directly to the workforce if rejected from the university. Students may leave the university at any time step, with the duration of their education determining the highest degree and qualifications they attain. Employers set salaries based on worker qualifications, linking educational outcomes to career trajectories.
