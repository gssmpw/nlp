\label{sec:more-related-works}
\textbf{Agent-based Social Simulations.} 
Agent-based models (ABMs) have been employed to study various societal phenomena, such as the spread of misinformation in social networks, the propagation of epidemics, resource management, and economic systems \cite{perez2009agent,asgharpour2010impact,giabbanelli2021application,benthall2021boundary,gausen2022using}. ABMs offer a bottom-up approach to understanding sociological phenomena, where the interactions between individual agents can lead to emergent behaviors~\cite{elsenbroich2023agent}.
Traditionally, such modeling has been conducted using surveys, network analysis, data mining, and game theory \cite{bonabeau2002agent}. Recently, MARL has emerged as a powerful tool for analyzing complex group dynamics \cite{busoniu2008comprehensive}. However, the majority of existing MARL environments focus on specialized applications, such as games or autonomous navigation \cite{terry2021pettingzoo,li2022metadrive} with limited relevance to fairness-oriented research.  In contrast, our work analyzes fairness---an essential metric for assessing social and institutional interactions---in an MARL context.