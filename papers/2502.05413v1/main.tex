%%%%%%%%%%%%%%%%%%%%%%%%%%%%%%%%%%%%%%%%%%%%%%%%%%%%%%%%%%%%%%%%%%%%%%%%%%%%%%%%
% Template for USENIX papers.
%
% History:
%
% - TEMPLATE for Usenix papers, specifically to meet requirements of
%   USENIX '05. originally a template for producing IEEE-format
%   articles using LaTeX. written by Matthew Ward, CS Department,
%   Worcester Polytechnic Institute. adapted by David Beazley for his
%   excellent SWIG paper in Proceedings, Tcl 96. turned into a
%   smartass generic template by De Clarke, with thanks to both the
%   above pioneers. Use at your own risk. Complaints to /dev/null.
%   Make it two column with no page numbering, default is 10 point.
%
% - Munged by Fred Douglis <douglis@research.att.com> 10/97 to
%   separate the .sty file from the LaTeX source template, so that
%   people can more easily include the .sty file into an existing
%   document. Also changed to more closely follow the style guidelines
%   as represented by the Word sample file.
%
% - Note that since 2010, USENIX does not require endnotes. If you
%   want foot of page notes, don't include the endnotes package in the
%   usepackage command, below.
% - This version uses the latex2e styles, not the very ancient 2.09
%   stuff.
%
% - Updated July 2018: Text block size changed from 6.5" to 7"
%
% - Updated Dec 2018 for ATC'19:
%
%   * Revised text to pass HotCRP's auto-formatting check, with
%     hotcrp.settings.submission_form.body_font_size=10pt, and
%     hotcrp.settings.submission_form.line_height=12pt
%
%   * Switched from \endnote-s to \footnote-s to match Usenix's policy.
%
%   * \section* => \begin{abstract} ... \end{abstract}
%
%   * Make template self-contained in terms of bibtex entires, to allow
%     this file to be compiled. (And changing refs style to 'plain'.)
%
%   * Make template self-contained in terms of figures, to
%     allow this file to be compiled. 
%
%   * Added packages for hyperref, embedding fonts, and improving
%     appearance.
%   
%   * Removed outdated text.
%
%%%%%%%%%%%%%%%%%%%%%%%%%%%%%%%%%%%%%%%%%%%%%%%%%%%%%%%%%%%%%%%%%%%%%%%%%%%%%%%%

\documentclass[letterpaper,twocolumn,10pt]{article}
\usepackage{usenix}


\section{A Benchmark for Grounded Persuasion}

\paragraph{Motivations and Challenges}
The first task of our study is to build a consistent and effective evaluation benchmark. However, this effort faces several key challenges. A fundamental challenge is the inherent subjectivity of persuasion, which depends heavily on human feedback. Unlike many other LLM capabilities, such as reasoning and planning, 
which have objective criteria for evaluation, persuasiveness lacks standardized metrics. The persuasiveness of a message is determined by its recipient and can vary significantly with individual preferences and context.  

Another challenge arises from the multifaceted nature of persuasion --- a process broadly studied in fields such as psychology, economics, and communication.  Each field offers distinct models of influence. 
Thus, effective persuasion techniques can differ significantly across domains. In the realm of LLMs, most existing research has focused on political or opinion-based contexts, where persuasion often takes on an adversarial nature. Such contexts complicate evaluation due to the strong influence of subjective beliefs and cognitive biases. For instance, studies like those conducted by \citet{hackenburg2024evaluating} and \citet{matz2024potential} reached differing conclusions regarding the effectiveness of LLMs in personalized persuasion, despite using similar experimental setups. 
Moreover, \citet{durmus2024measuring} noted that opinion-based persuasion is susceptible to the anchoring effect, where participants' initial beliefs strongly influence them, making opinion shifts difficult to measure accurately.
Additionally, persuasion in these settings often resorts to deception, as fact-checking is hard even for experienced humans. \citet{durmus2024measuring} found that prompting models to fabricate information was the most effective strategy under current evaluation. 

These challenges underscore the critical need to go beyond existing studies and develop new benchmarks that evaluate persuasion in more controlled, fact-based contexts.
Thus, we develop an evaluation framework customized for the task of grounded persuasion. 

\textbf{Real Estate Marketing as Testbed}\quad A crucial aspect of our design is to identify a domain that aligns well with grounded persuasion. We select real estate marketing as the primary testbed for several reasons. First, the real estate sector is characterized by high-stakes economic decisions, where potential buyers tend to hold more fact-based, rational beliefs compared to political or emotionally charged domains. In this environment, persuasive language must not only resonate with potential buyers but also remain truthful and contextually relevant. This makes it an ideal setting for testing the principles of grounded persuasion. Second, real estate marketing involves complex decision-making processes, where strong persuasive capabilities can significantly influence outcomes. Experienced realtors earn commissions reflecting the economic value of their persuasion skills. In addition, the potential of generative AI in this domain has been highlighted by a notable anecdote~\citep{reddit_2023} claiming an LLM successfully facilitated a home sale without an agent.
Third, the availability of extensive datasets in the real estate sector enables us to extract valuable domain-specific knowledge for training LLMs and conducting thorough empirical evaluations. 
Leveraging these resources, our benchmark provides a robust and consistent means to assess LLMs' grounded persuasion abilities, offering a scalable solution for evaluating persuasion in high-stakes scenarios.

\textbf{A Realistic Evaluation Interface}\quad
We design the framework to ensure a realistic evaluation interface based on two key criteria. First, we aim to create an immersive experience to gather authentic human feedback on marketing content persuasiveness.  
Second, we need to naturally elicit human preferences to properly test dynamic personalized content generation. As such, we collect real data from more than 50k real estate listings on the market and design a web interface that mimics online platforms, allowing the model to observe buyers' general profiles and behaviors (e.g., recently browsed or liked listing). See Appendix~\ref{app: interface} and \ref{app: dataset} for a full description of the web interface and dataset.

\textbf{Measuring the Fact-based Persuasiveness}\quad
To evaluate model performance in 
grounded persuasion,
we are particularly interested in how buyer behaviors are influenced by the generated content and whether it is factually accurate. Hence, we set an interface to present to the potential buyers each time a single listing along with two descriptions generated by two distinct models, then ask them to choose which description makes them more interested.
We use the Elo score~\citep{elo1967proposed} to measure the relative persuasiveness of text generated by different models.
Additionally, we assess whether the generated content is factually accurate. We defer the detailed experiment setup to \cref{sec: evaluation}.



\newcommand{\sysname}{{XPUT\textsc{imer}}}
\newcommand{\groupname}{{Ant Group}}
% \newcommand{\sysname}{{XPUT\textsc{imer}}}



% \newcommand{\sysname}{{R\textsc{adar}}}
%\newcommand{\cell}{{\emph{Cell}}}
%\newcommand{\peaktpt}{{1.49}}
%\newcommand{\queuet}{{71.0\%}}
%\newcommand{\jct}{{48.9\%}}

% to be able to draw some self-contained figs
% \usepackage{tikz}
% \usepackage{amsmath}

% inlined bib file
% \usepackage{filecontents}

%-------------------------------------------------------------------------------
% \begin{filecontents}{\jobname.bib}
% %-------------------------------------------------------------------------------
% @Book{arpachiDusseau18:osbook,
%   author =       {Arpaci-Dusseau, Remzi H. and Arpaci-Dusseau Andrea C.},
%   title =        {Operating Systems: Three Easy Pieces},
%   publisher =    {Arpaci-Dusseau Books, LLC},
%   year =         2015,
%   edition =      {1.00},
%   note =         {\url{http://pages.cs.wisc.edu/~remzi/OSTEP/}}
% }
% @InProceedings{waldspurger02,
%   author =       {Waldspurger, Carl A.},
%   title =        {Memory resource management in {VMware ESX} server},
%   booktitle =    {USENIX Symposium on Operating System Design and
%                   Implementation (OSDI)},
%   year =         2002,
%   pages =        {181--194},
%   note =         {\url{https://www.usenix.org/legacy/event/osdi02/tech/waldspurger/waldspurger.pdf}}}
% \end{filecontents}

%-------------------------------------------------------------------------------
\begin{document}
%-------------------------------------------------------------------------------

%don't want date printed
\date{}

% make title bold and 14 pt font (Latex default is non-bold, 16 pt)
\title{\Large \bf \sysname{}: Anomaly Diagnostics for Divergent LLM Training in \\ GPU Clusters of Thousand-Plus Scale}

%for single author (just remove % characters)
\author{
{\rm Weihao Cui$^{1}$\thanks{Weihao Cui and Ji Zhang contributed equally to this work.}, Ji Zhang$^{2}$\footnotemark[1], Han Zhao$^{1}$, Chao Liu$^{2}$, Wenhao Zhang$^{1}$, Jian Sha$^{2}$\thanks{Jian Sha and Quan Chen are the corresponding authors}}\\ 
{\rm Quan Chen$^{1}$\footnotemark[2], Bingsheng He$^{3}$, Minyi Guo$^{1}$}\\
$^{1}$Shanghai Jiao Tong University,$^{2}$Ant Group,$^{3}$National University of Singapore
} % end author

\maketitle

%-------------------------------------------------------------------------------
\begin{abstract}
%-------------------------------------------------------------------------------
The rapid proliferation of large language models has driven the need for efficient GPU training clusters. However, ensuring high-performance training in these clusters is challenging due to the complexity of software-hardware interactions and the frequent occurrence of training anomalies. Since existing diagnostic tools are narrowly tailored to specific issues, there are gaps in their ability to address anomalies spanning the entire training stack. In response, we introduce \sysname{}, a real-time diagnostic framework designed for distributed LLM training at scale. \sysname{} first integrates a lightweight tracing daemon to monitor key code segments with minimal overhead. Additionally, it features a diagnostic engine that employs novel intra-kernel tracing and holistic aggregated metrics to efficiently identify and resolve anomalies. Deployment of \sysname{} across 6,000 GPUs over eight months demonstrated significant improvements across the training stack, validating its effectiveness in real-world scenarios.
\end{abstract}

\documentclass[../main.tex]{subfiles}
\graphicspath{{../images/}}
\makeatletter
\def\input@path{{../images/}}
\makeatother
\begin{document}
\section{Introduction}
\begin{figure}
\centering
\begin{tikzpicture}
\node[inner sep=0pt] (ws) at (0, 0) {
\includegraphics[height=.4\textwidth, trim={10cm 0 10cm 0},clip]{world_space.png}};
\node[inner sep=0pt] (cs) at (6,0) {\includegraphics[height=.4\textwidth, trim={10cm 1cm 10cm 4cm},clip]{conf_space.png}};
\end{tikzpicture}
\vspace{-5pt}
\label{fig:pbrm_intro}
\caption{\textbf{Left}: Shows world space obstacles as grey spheres. Robots start and goal configuration is colored red and green, respectively. Configurations along the computed path are colored transparent blue. \textbf{Right:} Mapped world space scenario to configuration space. Obstacle region is the grey mesh. Red spheres are collision-free regions computed by the neural SCDF. The optimized shortest path in the convex corridor is the blue curve.}
\vspace{-25pt}
\end{figure}
Motion planning is the problem of finding a collision-free trajectory that connects a given start and goal configuration. The planning takes place in the configuration space of the robot. For single body robots, like mobile robots or drones, the configuration space and the world space are usually the same. This simplifies the planning, since explicit obstacle representations are available which enables geometrical tools like separating hyperplanes, smallest distance to obstacles etc., to be used when designing motion planning algorithms. For multi-body robots like manipulators, the situation is completely different. The world space obstacles are usually mapped to non-convex regions, and to make the problem even harder, the mapping is usually not known. Forming explicit representations of the obstacle region in the configuration space is usually too expensive or intractable. Despite all of this, sampling based planners are used with great success, which mainly is due to their use of implicit representations of the obstacle region. The basic idea is to construct a graph in the configuration space that covers and connects the collision-free region. From this graph, a path can be extracted that connects a given start and goal configuration. The approach is computationally expensive, since the graph is constructed with the smallest geometrical building block available, points, which represents a collision-check. Furthermore, the extracted paths from the graph are non-smooth and jagged due to the stochastic nature of the approach. This adds an additional post-processing step to the process, where the paths are shortcutted and smoothened, before the path can be used for tracking. Clearly a lot of time is invested to form this graph and produce smooth paths. Thus, if the obstacles start to move, then all of this work is done in no use, since all points that make up this graph need to be re-verified, which is simply too time consuming to be done in real time.
\\\\
In this work, we want to address the existing drawbacks of the sampling based planners. Our main contribution is an improved motion planner where each vertex in the graph covers a collision-free region in the form of a sphere instead of a point and where the edges are formed with neighboring intersecting spheres. This representation has the advantage of instead of returning piecewise linear paths, returning a sequence of overlapping spheres, i.e. a convex corridor, that connects a given start and goal configuration, illustrated in Figure \ref{fig:pbrm_intro}. This convex corridor allows us to use convex optimization to produce smooth trajectories, instead of computationally expensive post-processing methods. The representation further allows us to estimate the coverage of the collision-free space, which gives us awareness and feedback in the offline roadmap construction phase. Finally, our representation is simple to adapt to moving obstacles, simply requery for the new radii and recheck for intersections. 
\\\\
The spherical collision-free regions are formed using a signed distance function (SDF), which is a function that returns the smallest distance from an arbitrary point to the boundary of an obstacle. As the name implies, the distance is signed, thus if the point is inside the obstacle it is negative otherwise positive. If the distance is positive, a sphere with radius equal to the distance is guaranteed to cover a collision-free region. Using an SDF in motion planning is not new, but what is novel about our approach is that we express the distance in the configuration space instead of the world space and by doing so allows us to form these convex collision-free regions. We refer to the resulting SDF as a signed configuration distance function (SCDF). Computing an SCDF analytically is non-trivial, our approach is therefore to parameterize the SCDF with a deep neural network and learn the mapping by supervised learning. Our resulting neural SCDF can compute distances for different parameter values of obstacle shapes and we also show how multiple distances can be combined, thus making our approach flexible.
\section{Related work}
Motion planning algorithms can roughly be divided into three families, grid-based, sampling based and optimization based methods. Grid-based methods (GBM) discretize the planning space from which a graph is then compiled. A standard search method is A$^\star$ \citep{a_star}, which is classified as an \textit{informed} search method, since it employs a heuristic function to speed up the search. A$^\star$ guarantees to return an optimal path at the level of discretization used. GBMs usually discretize the planning space by a regular lattice and this limits the GBMs to problems with low dimensionality due to the curse of dimensionality. Thus, GBMs are usually limited to single-body robots where the degrees of freedom (DOF) are low. To overcome the inherent scaling problem with the GBMs, stochastic methods are usually used for multi-body robots. These methods are termed as sampling-based methods (SBM) and core members within this family are the rapidly-exploring random trees (RRT) \citep{rrt} and the probabilistic roadmap (PRM) \citep{prm}. RRT grows a tree from the start configuration and explores the collision-free region in a rapid way until it is able to connect to the goal region. RRT is usually improved by bi-directional planning \citep{rrt_connect}, i.e. an additional tree is grown from the goal configuration and the trees are tested for connection after any tree has been expanded. RRT is a single-query method, thus it searches for a path from scratch each time it is queried. Contrary to this, PRM is a multi-query method, which solves for multiple queries without starting from scratch. PRM does this by creating a roadmap (graph) that covers the collision-free space as an offline step. The graph is then used to solve for multiple queries. PRMs are used in cases where the environment does not change since the extra offline step is too computationally costly and needs to be re-done if the environment is changed. In our work, we address this inherent issue by using a different roadmap representation. Our vertices in the graph cover a collision-free region in the form of spheres and we form the edges by checking for intersecting spheres. If something in the environment changes, we recompute the spheres radii and recheck the intersections, without relying on collision detection. We use a trained neural network to compute the sphere radius, therefore querying for the radius can be done fast, hence our representation enables the PRM for dynamic environments.
\\\\
In the recent decades, optimization based methods (OBM) \citep{chomp, schulman, itomp, stomp} have been introduced as an alternative to SBM for multi-body robots. Like the SBM, the OBMs scale well to higher dimensional problems and produce smoother motion. It is common to use a SDF in the optimization since it is a smooth function, thus enabling gradient-based methods. However, the standard way of expressing the SDF is in world space. The distance therefore needs to be mapped to the configuration space by the forward kinematics. This mapping makes the optimization problem a non-linear program (NLP), which is computationally expensive to solve. Recently, a different approach has been proposed. In \cite{mp_gcs} motion planning is formulated as a convex optimization problem by using the graph of convex sets framework \citep{gcs}. The underlying idea is to decompose the collision-free space into intersecting convex sets from which a convex optimization problem is formulated. In cases where an explicit representation of the obstacles in the configuration space exists, like for single-body robots, creating collision-free convex regions can be done fast \citep{iris}. For multi-body robots, this is non-trivial. Existing work does this successfully \citep{iris_nlp, iris_c} by an optimization based approach, but the methods are still too time consuming to be used in the presence of moving obstacles. Our approach is instead to use deep learning to learn an SDF expressed in the configuration space. With this, we can query for shortest distances to the collision boundary, which allows us to expand spherical regions which are collision-free. Our approach is fast and therefore enables our suggested roadmap planner to be used in dynamic environments.
\\\\
Recent research has focused on learning collision detection \citep{fk_kernel_distance, diffco, graphdistnet} by predicting the signed distance between the robot links and the surrounding obstacles in the world space. The learned SDF is used in trajectory optimization but since the distance is expressed in the world space, the problem becomes an NLP and therefore takes a long time to solve. We take a novel approach and suggest to instead express the signed distance in the configuration space. This allows us to improve the PRM at the same time as it enables convex optimization for trajectory optimization, which runs faster and is more reliable than NLP solvers. In \cite{cspf} a learned signed distance function in the configuration space is proposed similar to our approach. However, their approach is restricted to point cloud representations, while we propose to represent the obstacles as parameterized geometric shapes, e.g. spheres. Furthermore, we also show how to use our learned SCDF to improve an existing roadmap planner.
\section{Problem formulation}
A robot is located in the world space, $\W \subset \R^3 $. The unique location of the robot is given by its configuration $\q \in \C$, where $\C$ is the configuration space. The set of points covered by the robots bodies at a certain configuration is expressed as $\B(\q) \subset \W$. The robot is surrounded by $\NrObst$ obstacles $\O = \bigcup_{i=1}^{\NrObst} \O_i$, where  $\O_i \subset \W$. The representation of the obstacle in the configuration space is the set $\C\O_i = \{\q \in \C \: |\: \B(\q) \cap \O_i \neq \emptyset \}$. The obstacle space is formed as $\Co = \bigcup_{i=1}^{\NrObst} \C \O_i$. The complement is referred to as the free space, $\Cf = \C \setminus \Co$. The path planning problem is a tuple, ($\Cf$, $\qStart$, $\qGoal$), where we want to connect a query pair, consisting of a start, $\qStart$, and goal configuration, $\qGoal$, with a geometric path, $\q(s): [0, 1] \mapsto \Cf$, such that $\q(0)=\qStart$ and $\q(1)=\qGoal$, or report correctly when such a path does not exist.
\end{document}

\section{Basic Background: Supervised Learning and the PAC Model}
\label{sec:background}

At this point almost everyone has heard of machine learning (ML). Anyone likely to stumble upon this article will have also heard of its most influential special case, supervised learning, and those theoretically inclined will also be familiar with the PAC model. Nonetheless, I will set the stage by  recapping the basics.

\subsection{Basics of Supervised Learning}%Let's set the stage in any case

\emph{Supervised Learning} is the task of ``coming up'' with a function $f: \X \to \Y$ to ``explain'' or ``fit'' a sequence of input/output examples   $(x_1,y_1), \ldots, (x_n,y_n)$, with $x_i \in \X$ and $y_i \in \Y$.  Here $\X$ is a \emph{data domain} consisting of \emph{datapoints} $x \in \X$, $\Y$ is a \emph{label set} consisting of \emph{labels} $y \in \Y$, and the sequence $(x_1,y_1),\ldots,(x_n,y_n)$ is the \emph{training data} consisting of \emph{labeled examples (a.k.a. samples)}~$(x_i,y_i)$.  I~will refer to the chosen function $f$ as a \emph{predictor}, and to $n$ as the \emph{sample size}. A \emph{learning algorithm} takes as input training data, and outputs (some representation of) a predictor $f \in \Y^\X$.\footnote{Note that this describes the usual \emph{batch}, a.k.a.~\emph{offline}, setting of supervised learning. I do not discuss other paradigms such as online or active learning in this article.} 



Success in supervised learning is defined as \emph{generalization} to  future examples: For a typical \emph{test example}  $(x_{\tst},y_{\tst})$, the predicted label $y'_{\tst}=f(x_{\tst})$ should ``equal'' $y_{\tst}$, perhaps approximately. We usually assume the test example is drawn from the same  ``source'' as the training data  --- commonly, i.i.d.~from the same distribution. The quality of the prediction is quantified by $\ell(y'_{\tst},y_{\tst})$, where $\ell:~\Y~\times~\Y \to \RR_{\geq 0}$ is a \emph{loss function} chosen as part of the problem definition. Common loss functions include the 0-1 loss $\ell_{0-1}(y',y) = [y' \neq y]$ for \emph{classification} problems,\footnote{The notation $[P]$ denotes $1$ when predicate $P$ is true, and denotes $0$ when $P$ is false.} as well as the absolute loss $|y'-y|$ or squared loss $(y'-y)^2$ for \emph{regression problems} featuring $\Y  \sse \RR$.

Nontrivial generalization properties are typically only possible if one assumes something about the data.\footnote{The need for such an assumption is formalized by the  \emph{no free lunch theorems} of supervised learning \cite{wolpert_connection_1992,wolpert_lack_1996,schaffer_conservation_1994}.} The Bayesian approach to  machine learning, common in many applications, assumes some parametric form for the distribution generating the data, and postulates a prior on the parameters. This is not the approach I will take in this article. Instead, I will focus on the frequentist --- and some would say ``worst-case'' or ``adversarial'' ---  approach that is common in the computational learning theory community, embodied by the PAC model. Here we assume that the (training and test) data can be explained, perhaps approximately, by a function in some ``simple enough to learn'' class of functions $\H \sse \Y^\X$, often called the \emph{hypotheses}. Equivalently, we  seek a predictor which explains the unseen data roughly  as well as the best hypothesis $h^* \in \H$, whether or not we assume that $h^*$ itself provides a perfect explanation.



 \paragraph{Common Algorithmic Templates.} Perhaps the best known general-purpose supervised learning algorithm is \emph{empirical risk minimization (ERM)}, which chooses as its predictor a hypothesis $f \in \H$ minimizing $\frac{1}{n} \sum_{i=1}^n \ell(f(x_i),y_i)$ --- a quantity called the \emph{training error}, \emph{empirical error}, or \emph{empirical risk} of $f$. %\footnote{When multiple hypotheses minimize the empirical risk, we assume ERM breaks ties arbitrarily.}
A common template for generalizing ERM involves adding a \emph{regularization term} $\psi(f)$ to the  objective function, typically chosen to measure some notion of ``hypothesis complexity.'' An algorithm instantiating this template is known as a \emph{structural risk minimizer (SRM)}, and chooses as its predictor the hypothesis $f \in \H$ minimizing the \emph{structural risk} $\frac{1}{n} \sum_{i=1}^n \ell(f(x_i),y_i) + \psi(f)$. Other well-known algorithms, such as gradient descent and its variations,  can frequently be interpreted as approximate implementations of ERM or SRM.


\paragraph{Proper vs Improper Learning.} A learning algorithm is said to be \emph{proper} if its predictor $f$ is always chosen from the hypothesis class, i.e., $f \in \H$, otherwise it is said to be \emph{improper}. ERM  is an example of a proper learning algorithm, as are SRM algorithms of the form described above.  In the \emph{proper regime} of learning, algorithms are required to be proper. This article will be concerned with the more flexible \emph{improper regime} (a.k.a \emph{representation-independent learning}), where no such constraint is placed on the learner. In other words, all we care about is predictive power at test time, rather than any insights derived from the functional form or representation of the predictor~itself.


\subsection{The PAC Model}
A standard mathematical setup for evaluation of supervised learning algorithms, at least in the theoretical computer science community, is Valiant's \emph{Probably Approximately Correct (PAC) model} of learning (see e.g.~\cite{kearns_introduction_1994,mohri_foundations_2018}). Here, we assume there is an unknown distribution $\D$ on $\X \times \Y$ from which training and test data are  drawn.  Specifically, the labeled datapoints of the training set  $(x_1,y_1), \ldots, (x_n,y_n)$, as well as the test data  $(x_\tst,y_\tst)$, are i.i.d.~from $\D$. Often it is assumed that $\D$ lies in some class of distributions of interest. The \emph{true expected loss}, or simply \emph{loss}, of a predictor $f: \X \to \Y$ is the expected loss it incurs on draws from $\D$, written $L_\D(f) = \Ex_{(x,y) \sim \D} \ell(f(x),y)$.


There are two main ``settings'' in PAC learning. The  \emph{realizable setting} only requires that the data be perfectly explained by some hypothesis in $\H$. More generally, the \emph{agnostic setting} makes no assumption relating the data to the hypotheses, but shifts the goalposts as necessary to allow nontrivial guarantees: the expected loss at test time is evaluated only ``relative'' to that of the best hypothesis $h^* \in \H$. There are other settings which make more nuanced assumptions, such as $\D$ being of a particular parametric form or its support living in some (unknown) lower-dimensional space, etc. I will mostly discuss the realizable and agnostic settings in this article, those being the simplest and most studied from a theoretical perspective. %TODO:We will briefly discuss other settings in Section ??

The PAC model demands high probability guarantees of learners, in the worst case over distributions of interest. Consider first the realizable setting, where $\D$ is such that $\min_{h \in \H} L_{\D}(h) = 0$. A PAC learner has \emph{error} $\epsilon=\epsilon(n)$ and \emph{confidence} $\delta=\delta(n)$ if, when training data consists of $n$ i.i.d~samples from a realizable distribution $\D$, it produces a predictor $f$  satisfying $L_\D(f) \leq \epsilon$ with probability at least $1-\delta$. In the agnostic setting, where $\D$ can be arbitrary, we require $L_\D(f) - \min_{h \in \H} L_\D(h) \leq \epsilon$ with probability $1-\delta$.

In both the realizable and agnostic settings, we look for PAC learners with small $\epsilon$ and $\delta$ as a function of the sample size $n$. An equivalent perspective looks at the sample complexity $m(\epsilon,\delta)$, which is the minimum sample size which guarantees error  at most $\epsilon$ with probability at least $1-\delta$. We say a problem is \emph{PAC learnable} if its PAC sample complexity is finite whenever $\epsilon,\delta > 0$.

For most PAC learning problems, learnability and sample complexity are characterized in terms of a  ``dimension'' of the hypothesis class. Most prominently this is the \emph{VC dimension} for binary classification, the \emph{fat shattering dimension} for agnostic regression, and the \emph{DS dimension} for multiclass classification (see \cite{anthony_neural_1999,daniely_optimal_2014,brukhim_characterization_2022}). Treatment of these is beyond the scope of this article. The unfamiliar reader need not worry, however,  as dimensions will feature only tangentially in our~discussion.




%\paragraph{Learning settings: Realizable, Agnostic, etc.} In learning theory, evaluating a supervised learning algorithm requires specifying a data model and an objective. We will leave the details of the data model flexible for now, to allow for both the PAC model and the adversarial transductive model. Nonetheless we will describe two variations, which we call ``settings'', which cut across different models. The  \emph{realizable setting}  requires only that the data be perfectly explained by some hypothesis $h \in \H$ --- i.e., there exists a hypothesis which is guaranteed to suffer a loss of $0$ on training and test data. The performance of the learning algorithm is its expected loss at test time for some ``worst case'' realizable instance. More generally, the \emph{agnostic setting} makes no assumption relating the data to the hypotheses, but shifts the goalposts as necessary to allow nontrivial guarantees: the expected loss at test time is evaluated only ``relative'' to that of the best hypothesis $h^* \in \H$, again for some ``worst case'' instance. There are other settings which make more nuanced assumptions about the data, such as it is drawn from a distribution of a particular parametric form, or that it lives in some (unknown) lower-dimensional space, etc. We will mostly discuss the realizable and agnostic settings, those being the simplest and most studied from a theoretical perspective.




%%% Local Variables:
%%% mode: latex
%%% TeX-master: "learning_matching"
%%% End:

\begin{figure*}[t]
\begin{center}
\includegraphics[width=.85\linewidth]{fig_overview_v3.pdf}
\end{center}
\caption{
FastAtlas Overview: In each frame, we compute charts spanning fully or partially visible triangles (a), determine texture space bounding boxes for the visible portions of the view-space projections of each chart, and tightly pack these boxes into atlases (b, here $2K \times 2K$). We simultaneously bijectively parameterize and shade the charts into their atlas boxes, obtaining high quality texture space shading (c), and use this shading to render the shaded frames (d).}
\label{fig:overview}
\label{fig:alg_overview}
\end{figure*}

\section{Overview}
\label{sec:overview}
Our work has two core contributions: a real-time, GPU-based algorithm for tight packing of general parameterized charts into compact atlases; and a real-time TSS method that
utilizes this packing.  

\paragraph*{FastAtlas Packing.}
FastAtlas runs entirely on the GPU as a series of compute shaders. It takes the bounding boxes of parameterized charts as input, and packs them into an atlas (Fig~\ref{fig:overview}b, Sec.~\ref{sec:pack}). As such, the only input it requires are the dimensions of the bounding boxes.
Its outputs are deterministic; identical input charts are packed into identical atlases. This is critical for TSS and similar applications, as it ensures that consecutive frames taken from the same camera view have the same shading. Even minute shading differences across such frames can cause sampling jitter, leading to undesirable flicker \cite{baker2012rock}. 
While prior methods such as \cite{mueller2018shading,hladky2019tessellated,hladky2021snakebinning,Neff2022MSA} cap the dimensions of the charts that can be packed as-is for a given atlas size, and scale down all charts that exceed these dimensions, we scale all charts by the same factor, and do so only when strictly necessary to achieve packing success (Figs~\ref{fig:atlas},~\ref{fig:sas_issues}). 

\paragraph*{TSS using FastAtlas.}
Our end-to-end TSS atlas generation method combines the packing method above with a novel approach for computing seamless per-frame charts. 
We define our charts as the connected components of the visible surfaces in each frame (Fig.~\ref{fig:overview}a), and efficiently compute them using a parallel union-find algorithm (Sec.~\ref{sec:visible}). Since the boundaries of these charts coincide with the contours of the rendered surface, they are {\em invisible} to the viewer. This approach 
eliminates the artifacts caused by shading discontinuities along visible seams (Fig.~\ref{fig:seams}). 

\begin{parWithWrapFigure}
\begin{wrapfigure}{l}{.27\columnwidth}%
\includegraphics[width=\linewidth]{fig_inset_view_plane.pdf}%
\end{wrapfigure}
We bijectively parametrize the {\em visible portions} of our charts by projecting them to view space (inset). This maps a constant number of texels to each pixel in the final rendered output, evenly distributing residual undersampling error across all image pixels. While conceptually straightforward, efficiently parameterizing charts containing partially visible triangles using viewspace projection is non-trivial, as the visible portions may no longer be triangular (e.g. green triangle in the inset); applying naive projection to triangles with vertices behind the camera may produce ill-posed results. Clipping triangles before projection is both computationally expensive and significantly complicates downstream operations. We avoid explicit clipping by observing that all that is required for atlas packing is the dimensions of, potentially conservative, bounding boxes of these projected visible portions. We compute such bounding boxes without explicit chart clipping by adapting a conservative screen coverage estimator \shortcite{Blinn:CalculatingScreenCoverage} (Sec.~\ref{sec:box}). We then pack the computed boxes using FastAtlas. 
\end{parWithWrapFigure}

Finally, we shade the visible portion of each chart into its corresponding atlas bounding box (Fig~\ref{fig:overview}c). 
The resulting texture is then used during rasterization as a standard texture map (Fig. ~\ref{fig:overview}d). 
Our framework is compatible with all existing approaches for texture space shading, including forward shading, raytraced illumination, or deferred shading in texture space \cite{baker:2016}. In the examples shown, we use the standard forward shading based rendering pipeline included in the G3D Innovation Engine \cite{G3D17}, a commercial grade renderer.

\section{Lightweight Selective Tracing}
To collect sufficient real-time data for anomaly diagnosis in the LLM training cluster, \sysname{}’s tracing daemon offers backbone-agnostic and lightweight full-spectrum tracing. Its design focuses on two key aspects: determining what information to collect and establishing how to collect it efficiently.

% To gather sufficient real-time data for anomaly detection within LLM training clusters, \sysname{}’s tracing daemon provides backbone-agnostic and lightweight full-spectrum tracing.
% Its design focuses on two aspects: identifying what information to collect and determining how to collect it efficiently.

\subsection{Key Segment Instrumentation}
% Collecting all infor

Since profiling APIs like CUPTI\cite{cupti2024} can operate in a background thread, the runtime data collection overhead primarily stems from high memory usage rather than interference with computing resources. For instance, profiling a Llama-70B model trained on 512 H800 GPUs using PyTorch’s built-in profiler produces a log file of $5.5\text{GB}$ (in JSON format, compressed to $451\text{MB}$) for each training step. This substantial memory overhead renders such arbitrary profiling methods impractical for continuously collecting real-time data to support anomaly diagnostics.

% Collecting comprehensive real-time information, including data from all launched GPU kernels, introduces significant overhead. Profiling APIs, such as CUPTI~\cite{cupti2024}, can operate in a background thread, meaning the primary overhead arises not from the slowdown of the running training but from the high memory usage incurred during real-time data collection.
% For instance, profiling a Llama-70B model trained on 512 A100-80GB GPUs using PyTorch’s built-in profiler generates a log file of $5.5\text{GB}$ (in JSON format, compressed to $451\text{MB}$) for each training step, while the profiling process only causes noticeable slowdowns when the profiler dumps the log file after the final profiling step.
% This substantial memory overhead renders such arbitrary profiling methods impractical for continuously collecting real-time data to support anomaly detection.

% \weihao{impact on cluster}
\begin{figure}
    \centering
    \includegraphics[width=.9\linewidth]{figure/2-instrument.pdf}
    \caption{Instrumented key code segments in \sysname{}.}
    \label{fig:key-segment}
    \vspace{-4mm}
\end{figure}

Therefore, \sysname{} selectively instruments code segments of key APIs and kernels to collect real-time information. This design is based on an insight into LLM training on large-scale GPUs: LLM training is predominantly dominated by a limited set of deep learning operators. These operators mainly include matrix multiplication and cross-GPU communication operators. \autoref{fig:key-segment} presents the specific code segments instrumented by \sysname{} for efficient anomaly diagnostics.

% Therefore, \sysname{} turns to selectively instrument code segment of key APIs and kernels for collecting real-time information.
% This design is grounded in the insight regarding the training of LLMs on large-scale GPUs:
% LLM training is predominantly dominated by a limited set of deep learning operators, with matrix multiplication and cross-GPU communication being particularly critical.
% Exhaust profiling, such as tracing all function calls across Python and C++ runtime, is unnecessary and would introduce significant overhead without corresponding benefits.
% \autoref{fig:key-segment} highlights the specific code segments instrumented by \sysname{} for efficient anomaly detection.

As shown in the figure, the instrumented code segments can be broadly categorized into two groups. The first category involves intercepting key API calls, including those related to Python’s garbage collection (GC), PyTorch’s dataloader, and GPU synchronization. These APIs are carefully selected based on empirical insights into performance issues and optimization opportunities. These insights are detailed in \S\ref{sec:slowdown-anomalies}, with corresponding cases discussed in \S\ref{sec:case_study}.

The second category focuses on intercepting critical GPU computation and communication kernels executed at the C++ runtime level. These kernels, primarily provided by optimized libraries\cite{cublas2024,daoFlashAttentionFast,daoFlashAttention2Faster,nccl2024}, account for the majority of the workload during large-scale training. Additionally, there are customized kernels developed by the infrastructure team.

Notably, the above design enables \sysname{} to support backbone-agnostic and extensible tracing capabilities. Extending tracing capabilities for Python-related APIs is straightforward, requiring only the configuration of the specific environment variables in the training scripts, as shown below.
\begin{minted}[
    frame=none,
    obeytabs=true,
    framesep=0mm,
    baselinestretch=0.8,
    fontsize=\footnotesize,
    xleftmargin=1.6em,
    breaklines,
    escapeinside=||,
]{shell}
export TRACED_PYTHON_API="torch.cuda@synchronize"
\end{minted}
Meanwhile, intercepting C++ kernels necessitates explicit registration through a C++ interface. This requirement is feasible, as the infrastructure team takes charge of the development of both these customized operators and \sysname{}, ensuring seamless integration and functionality.


% The code segment instrumentation also enables \sysname{} to support backbone-agnostic and extensible tracing capabilities.
% The instrumented code segments can be broadly categorized into two groups.
% The first category involves intercepting key API calls, including those related to Python’s garbage collection (GC), PyTorch’s dataloader, and GPU synchronization.
% These APIs are carefully chosen based on empirical insights into performance issues and optimization opportunities.
% For example, GC operations and explicit GPU synchronization may introduce stalls of kernel issue during the training, as elaborated in \S\ref{sec:case:stall-free}.
% Extending the tracing capabilities for Python-related APIs is straightforward and only requires setting specific environment variables in the training scripts.

% In the aforementioned example, tracing GPU synchronization in PyTorch is enabled through this approach.
% The second category focuses on intercepting critical GPU computation and communication kernels executed at the C++ runtime level. These kernels, mainly provided by optimized libraries~\cite{cublas2024,daoFlashAttentionFast,daoFlashAttention2Faster,nccl2024}, constitute the majority of the workload during large-scale training.
% For customized kernels, interception requires explicit registration through a C++ interface.
% This requirement is feasible, as the infrastructure team takes charge of the development of both these customized operators and \sysname{}, ensuring seamless integration and functionality.

% Modeling teams develop new modeling algorithms for improving the LLM performance, which are pure python APIs.
% These APIs are often the attributions that incur the slowdown of training jobs compared to the original version of LLM without the new modeling algorithms.


\subsection{Timing in the Background}

With intercepted Python APIs and GPU kernels, FLARE measures their elapsed latencies, as shown in \autoref{fig:timing}. Specifically, a dedicated tracing thread runs in the background to efficiently manage timing data. It employs different timing mechanisms for Python APIs and GPU kernels.

For synchronous Python API calls, FLARE directly records their start and end timestamps and forwards them to the timing manager. For GPU kernels, which execute asynchronously, FLARE injects CUDA events\cite{cudaevents2024} after an interception to record execution status. These events are enqueued for further processing. The timing manager queries the status of the queued events in the background, avoiding any disruption to the training thread. Additionally, during GPU kernel interception, FLARE extracts input specifications, such as memory layout, to support subsequent anomaly diagnostics in \S\ref{sec:diagnose_obvious}.

\begin{figure}
    \centering
    \includegraphics[width=0.8\linewidth]{figure/3-timing.pdf}
    \caption{Intercepting and timing the training in the background.}
    \label{fig:timing}
    \vspace{-4mm}
\end{figure}

% With intercepted Python APIs and GPU kernels, \sysname{} measures their elapsed latencies, as depicted in \autoref{fig:timing}. A dedicated tracing thread operates in the background of the main training thread to manage timing data efficiently.
% The timing mechanisms for Python APIs and GPU kernels are implemented differently. For synchronous Python API calls, \sysname{} directly records their start and end timestamps and forwards them to the timing manager.
% For GPU kernels, which execute asynchronously, \sysname{} injects CUDA events~\cite{cudaevents2024} after interception to record execution status.
% These events are enqueued for further processing.
% The timing manager queries the status of the queued events in the background, avoiding any disruption to the training thread.
% To reduce overhead, the events are reused.
% Additionally, during GPU kernel interception, \sysname{} extracts input specifications, such as memory layout, from kernel arguments to support subsequent anomaly detection in \S\ref{sec:slowdown-anomalies}.

As training progresses, the timing manager proactively transmits all real-time data to \sysname{}’s diagnosis engine.
By employing key segment instrumentation and running timing tasks in the background, \sysname{} minimizes both computing resource and memory overhead, ensuring efficient data collection for real-time anomaly diagnostics.
A detailed evaluation of \sysname{}’s real-time overhead is provided in \S\ref{sec:eval:overhead}.

\section{Anomaly Detection and Diagnosis}
Using the real-time data collected by the tracing daemon, \sysname{}’s diagnostic engine identifies and analyzes anomalies encountered during distributed LLM training.
In this section, we present \sysname{}’s holistic diagnostic workflow for addressing two common anomaly symptoms: errors and slowdowns.

\subsection{Fast Runtime Error Diagnosis}

As shown in the left of \autoref{tb:anomaly-analysis}, errors encountered at the beginning of a training job are typically caused by bugs in the training scripts, which can often be addressed easily by the algorithm teams and infrastructure team.
However, diagnosing errors that occur during the training progress is more challenging and critical.
Such errors often stem from issues like operating system crashes, GPU failures, or network disruptions, which can generally be resolved by isolating the problematic machines and restarting the training job.

A typical symptom associated with these errors is the hanging of the training job.
Training LLMs across numerous GPUs in a distributed manner inherently relies on the coordination of training processes.
When the aforementioned errors occur, they rarely affect all training processes simultaneously.
% , leading to a ``hang'' state where training progress stalls.
In this context, \sysname{} focuses on rapidly diagnosing hang errors by identifying faulty machines. Then, \sysname{} routes this information to the operations team, enabling the training job to restart with healthy machines.

Specifically, \sysname{}’s diagnostic engine first detects hang errors by examining the status of tracing daemons.
The tracing daemon operates in the background of the training thread and continuously queries events recorded during job execution.
If it fails to confirm the completion of an event within a predefined timeout interval, it proactively reports a potential hang error to the diagnostic engine.
Similarly, if a tracing daemon does not transmit any real-time data within the specified timeout interval, the diagnostic engine also interprets this as an indication of a hang error.

After hang errors are reported, they are classified as either communication or non-communication errors. \sysname{} diagnoses these errors in two steps: first, a coarse-grained diagnosis through call stack analysis; and second, a fine-grained diagnosis using intra-kernel tracing

\paragraph{Diagnosis using call stack analysis.} 
This diagnosis is used to identify problematic machines encountering non-communication errors. 
% This diagnosis using call stack analysis is coarse-grained and can only identify problematic machines encountering non-communication errors.
\autoref{fig:hang-non-comm} illustrates an example of hang-error diagnosis via call stack analysis.
% involving $N$ training processes.
As shown in the left of \autoref{fig:hang-non-comm}, when the training process of rank-$0$ crashes or is suspended due to a non-communication error, it halts at a call stack corresponding to a non-communication function.
In contrast, the training processes of other ranks continue executing correctly and eventually stop at a call stack associated with a communication-related function that depends on coordination with rank-$0$.
In this scenario, the machine associated with rank-$0$ is identified as the source of the error.
It should be noted that, although these non-communication errors may cause direct crashes, the call stack analysis could also locate the faulty machine. 

% It should be noted that, at times, low-level issues directly cause crashes, which can also be identified through stack analysis.

However, communication hang errors cannot be identified through call stack analysis.
As shown in the right of \autoref{fig:hang-non-comm}, the training processes of all ranks terminate at the same call stack corresponding to a communication function, such as allreduce or allgather.
In this scenario, there are no distinct differences between ranks based on call stack analysis.

\begin{figure}
    \centering
    \includegraphics[width=.9\linewidth]{figure/4-hang-stack.pdf}
    \caption{Diagnosing hang errors via call stack analysis.}
    \label{fig:hang-non-comm}
    \vspace{-4mm}
\end{figure}

We further investigate the symptoms of communication hang errors and obtain two observations. Firstly, some communication hang errors generate error logs. For instance, if the link between RDMA NICs breaks, an error code of \texttt{12} is produced. Secondly, more hang errors result in an endless loop within the launched communication kernels, ultimately leading to job termination after a predefined timeout. To identify the unhealthy machine responsible for such errors, a straightforward solution is to perform a binary search by executing communication tests across all involved GPUs. This approach has a complexity of $O(\log N)$ and requires hours to pinpoint the faulty machine among thousands of GPUs~\cite{wuFALCONPinpointing}.

\paragraph{Diagnosis using intra-kernel tracing.}
Faced with this problem, \sysname{} introduces a minute-level diagnostic approach using intra-kernel tracing. This intra-kernel tracing leverages CUDA-GDB, the debugging tool for CUDA programming.

Specifically, \sysname{}'s diagnostic engine instructs the tracing daemon to attach the halted training processes with CUDA-GDB before terminating them.
Once attached, the tracing daemon executes a script capable of automatically extracting detailed communication statuses to identify unhealthy machines.
\autoref{fig:hang-comm} depicts an example of diagnosing communication hang errors in a hanging ring-allreduce kernel.

In the ring-allreduce kernel, each thread block is responsible for transmitting data between linked adjacent ranks within the kernel’s constructed ring.
The data are split into chunks and thread blocks of adjacent ranks work together to transmit the chunks step by step.
Thus, \sysname{} could retrieve the register values corresponding to the loop steps used for data transmission between linked ranks. 
Theoretically, the connection with the minimum step reveals the related GPUs experiencing errors.
This intra-kernel tracing process is performed in parallel across all involved GPUs.
As a result, its complexity is $O(1)$, enabling completion within a few minutes.

\sysname{} then routes the diagnostic information for detected errors to the operations team, assisting with tasks such as isolating faulty machines and restarting the training job.

\subsection{Aggregation for Slowdown Diagnosis}


\label{sec:slowdown-anomalies}
As demonstrated in \S\ref{sec:anomalies}, slowdowns can be attributed to changes across the entire training stack.
Meantime, slowdowns caused by software changes introduced by the algorithm and infrastructure teams are often subtle and challenging to detect.
Identifying these changes typically requires comparisons across historical training jobs and prior training steps.
In contrast, hardware changes, such as GPU underclocking or network jitter, are more apparent and can be detected solely through comparisons across training steps.

To holistically identify these anomalies, \sysname{} aggregates real-time data collected from the tracing daemon into five primary metrics, shown in \autoref{fig:aggregated-metric}.
\textit{
These metrics are based on the consensus that a “healthy” training pipeline should exhibit a timeline saturated with GPU kernels dedicated to either computation or communication.}
Computation kernels should achieve high FLOPS, while communication kernels are expected to utilize high bandwidth.
Any deviations from these characteristics point to idle GPU resources, signaling potential slowdowns in training jobs.
Of the five metrics, three are commonly used in existing works~\cite{jiangMegaScaleScaling,wuFALCONPinpointing}, while the other two are newly introduced by \sysname{}.
% to assess training efficiency.

\begin{figure}
    \centering
    \includegraphics[width=.9\linewidth]{figure/5-hang-intra-kernel.pdf}
    \caption{An example of diagnosing communication hang errors in a ring-allreduce kernel using intra-kernel tracing.}
    \label{fig:hang-comm}
    \vspace{-4mm}
\end{figure}
\begin{figure*}
    \centering
    \includegraphics[width=.85\linewidth]{figure/6-aggregated-metric.pdf}
    \vspace{-2mm}
    \caption{A timeline of a distributed training job annotated with aggregated metrics used for diagnosing slowdowns in \sysname{}.}
    \label{fig:aggregated-metric}
    \vspace{-4mm}
\end{figure*}
\subsubsection{Diagnosing Obvious Slowdown}
\label{sec:diagnose_obvious}
\paragraph{\protect\circlenumber{1} Training throughput for detecting slowdown.}
Training throughput is the most straightforward metric for detecting slowdowns.
\sysname{} measures training throughput by timing the rate at which input data is consumed by the training pipeline.
This is achieved by instrumenting the dataloader API of Pytorch.
As a macro performance metric, training throughput directly reflects slowdowns in training efficiency through comparison to historical training jobs and between training steps of the same job.
However, \sysname{} cannot diagnose the specific factors contributing to the slowdown.
To address this, \sysname{} relies on the following four micro metrics to further investigate the underlying causes.

\paragraph{\protect\circlenumber{2} FLOPS for slow critical kernel.}
\sysname{} monitors the FLOPS of instrumented critical computation kernels, leveraging timing data and input layout.
By comparing the FLOPS of identical kernels across different ranks, \sysname{} diagnoses GPUs that exhibit poor computational performance, often caused by issues like GPU underclocking.
Machines affected by GPU underclocking are then routed to the operations team for isolation.
Additionally, by analyzing FLOPS, \sysname{} identifies un-optimized kernels in training jobs, particularly those with large input sizes but low FLOPS.
These anomalies are detected without interrupting training jobs and are subsequently routed to the infrastructure team for further investigation.
Notably, when analyzing FLOPS data, \sysname{} accounts for the impact of communication kernels that overlap with computation kernels.%\weihao{real data}
This ensures that computation kernels with falsely low FLOPS are not mistakenly flagged. 

\paragraph{\protect\circlenumber{3} Bandwidth for slow connection.}
\sysname{} monitors the bandwidth of communication kernels. A communication operator requires launching the communication kernels on all ranks. Since variations in kernel-issue timestamps exist across different ranks, \sysname{} calculates the communication bandwidth by utilizing the start and end timestamps of the final communication kernels issued across all participating ranks.
The captured communication bandwidth is compared with offline profiled data.
If low-bandwidth communication is detected, \sysname{} conducts a communication test using binary search to pinpoint machines experiencing issues such as network congestion.
These slowdowns are then identified and routed to the operations team for resolution.

% \begin{figure}
%     \centering
%     \includegraphics[width=0.5\linewidth]{figure/7-issue-distribution.pdf}
%     \caption{Typical issue distribution patterns in communication kernels for both healthy and unhealthy LLM training jobs.}
%     \label{fig:issue-distribution}
% \end{figure}


\subsubsection{Diagnosing Obscured Slowdown}
While the above three metrics ensure that both critical computation and communication GPU kernels operate at high performance, they do not cover the less critical operations, such as various CPU operations and element-wise activation GPU kernels. Meantime, \sysname{}’s selective key segment instrumentation also omits the monitoring of these operations. 
% Metric-\circlenumber{1},\circlenumber{2} and \circlenumber{3} primarily ensure that both critical computation and communication GPU kernels operate at high performance.
% Furthermore, \sysname{}’s selective key segment instrumentation omits the monitoring of less critical operations, such as various CPU operations and element-wise activation GPU kernels.
% Their impact on the training efficiency is obscured, but this does not mean that \sysname{} disregards their potential contributions to slowdowns.

To diagnose their potential contributions to slowdowns, we further classify these not-instrumented operations into three categories: intra-step CPU operations, inter-step CPU operations, and minority GPU kernels.
Intra-step CPU operations and inter-step CPU operations differ due to their occurrences within the timeline of training steps.
Minority GPU kernels refer to those GPU kernels that often occupy little GPU computation resources. 
Specifically, two metrics are introduced for the diagnostics: issue latency distribution for intra-step CPU operations and void percentage for inter-step CPU operations and minority GPU kernels.

\begin{figure*}
    \centering
    \includegraphics[width=.95\linewidth]{figure/9-latency-overhead.pdf}
    \vspace{-2mm}
    \caption{Runtime overhead in terms of latency with various models, backbones, and number of GPUs.}
    \label{fig:eval:latency-overhead}
    \vspace{-4mm}
\end{figure*}

\paragraph{\protect\circlenumber{4} Issue latency distribution for kernel-issue stall.}\label{sec:diagnose:issue}
In a well-optimized parallel backbone, only the necessary intra-step CPU operations for launching GPU kernels or coordinating the training processes are expected. However, algorithm teams may inadvertently introduce unnecessary GPU synchronizations when modifying the LLM model. Meantime, certain function calls, such as GC~\cite{jiangMegaScaleScaling,shoeybiMegatronLMTraining}, may be implicitly triggered by the Python runtime. These intra-step CPU operations can occur repeatedly during the model’s forward pass, bringing considerable overhead. In such cases, these operations cause an anomaly known as a kernel-issue stall, leading to GPU idle time within the training step.

\protect\circlenumber{4}--1 in \autoref{fig:aggregated-metric} shows the example of Python runtime GC.
In the figure, the Python runtime GC stalls the CPU thread and causes the lagging of GPU kernels on rank-1.
Although the communication kernel on rank-0 is issued without stalling, it simply waits for the one on rank-1, ultimately causing the overall training speed to decline.
\protect\circlenumber{4}--2 in \autoref{fig:aggregated-metric} shows an example of unnecessary GPU synchronization introduced by the developers from the algorithm teams.
As all ranks wait for the completion of communication kernels, the kernel issue of follow-up kernels is stalled and not overlapped with GPU computation.
When such unnecessary synchronization occurs repeatedly across the model’s forward pass, it ultimately results in a slowdown of the training speed.

Originally, detecting these anomalies of kernel-issue stall requires investigating the aggregated timeline with much human effort.
Faced with this issue, \sysname{} proposes a new metric, named issue latency distribution, for diagnosing this issue without human intervention.
Kernel-issue latency is defined as the time elapsed between the kernel’s issue timestamp and the start timestamp of its execution on the GPU.
% \autoref{fig:issue-distribution} illustrates the pattern differences in kernel issue latency distribution between healthy and unhealthy LLM training jobs. 
% In the figure, two cases of kernel issue stall are: one caused by unmanaged Python runtime GC, and the other by unnecessary GPU synchronization introduced by algorithm teams.
Based on our observation of anomalies of kernel-issue stall, the kernel-issue latencies of unhealthy training jobs should be much shorter than those of a healthy training job.

By monitoring runtime issue latency distribution, and comparing it with the historical data, \sysname{} could identify the anomaly of kernel-issue stall. Then, \sysname{} routes them to algorithm and infrastructure teams for resolution, as they are commonly software issues.
\S\ref{sec:eval:issue} and \S\ref{sec:case:stall-free} demonstrate the effectiveness of issue latency distribution in diagnosing kernel-issue stalls, even when the slowdown is minimal.



\paragraph{\protect\circlenumber{5} Void percentage for other un-covered operations.}
While the tracing daemon only instruments the critical operators, inter-step CPU operations and minority GPU kernels both manifest as empty time slots in the visualized timeline, as shown in \autoref{fig:aggregated-metric}.
Consequently, \sysname{} introduces a metric, termed the void percentage, to identify slowdowns caused by these factors.

% Analyzing the GPU timeline of a visualized training job reveals that inter-step CPU operations and minority GPU kernels both manifest as empty GPU slots, as shown in \autoref{fig:aggregated-metric}. Consequently, \sysname{} introduces a metric, termed the void percentage, to identify slowdowns caused by these factors.

As for inter-step CPU operations, as depicted by \circlenumber{5}–2 in \autoref{fig:aggregated-metric}, \sysname{} measures the latency between the last kernel preceding the dataloader and the first kernel following the same dataloader. \sysname{} then computes the void percentage for inter-step CPU operations using the following equation:
\begin{equation}
V_{inter} = T_{inter}\ /\ T_{step} 
\end{equation}
where $T_{inter}$ represents the latency associated with inter-step CPU operations, and $T_{step}$ denotes the total latency of the training step.

As for minority GPU kernels, as shown by \circlenumber{5}–1 in \autoref{fig:aggregated-metric}, \sysname{} first automatically detects empty slots where GPU kernels are launched but remain un-executed. These empty slots signify that the GPUs are occupied by kernels outside the scope of \sysname{}’s tracing mechanism. \sysname{} subsequently accumulates these slots for each training step and computes the void percentage using the following equation:
\begin{equation}
V_{minority} = T_{minority}\ /\ (T_{step} - T_{inter})
\end{equation}
where $T_{minority}$ is the latency of all minority GPU kernels.

When the void percentages ($V_{inter}$ and $V_{minority}$) surpass the predefined thresholds for a specific parallel backbone, \sysname{} annotates the training job with potential slowdowns attributed to inter-step CPU operations or minority GPU kernels. Then, \sysname{} notifies the algorithm and infrastructure team for further investigation.

% \subsection{Diagnostic Summary}
% \weihao{we may give a brief summary here.}
\section{Implementation Environment}
\label{sec:implementation_environment}

Here we introduce the detailed implementation details and environment for reproducibility purpose. For our model, we choose hyperparameters based on the performance on validation set (Document classification task in the main paper explains how we split validation set). The results in the main paper are obtain by 5 independent runs. The standard deviations reported in the main paper are 1-sigma error bars and are obtained by calling its corresponding function in Excel library. All the experiments were done on Linux server with an NVIDIA A40 GPU with 46,068 MiB. Its operating system is CentOS Linux 7 (Core). We implemented our proposed model GTFormer using Python 3.10 as programming language and PyTorch 2.0.0 as deep learning library. Other frameworks include NumPy 1.23.1, sklearn 0.23.2, and scipy 1.5.2. We emphasize that the main focus of our model is effectiveness, instead of running efficiency. But for completeness, we still make a short comment on execution time. Our model is efficient, on the largest dataset Web, the training takes less than 40 hours to converge. We will release code and datasets upon publication.
\section{Evaluation}
We provide three sets of insights into this section, organised as \textit{findings (F*)}. We quantitatively study the effect of the adversarial and counterfactual perturbations on the performance of informal reasoners and autoformalisation methods. Then, we dive deeper into method variants. Finally, 
we analyse the nature of formalisation errors made by the models.

\subsection{Robustness Analysis}
\paragraph{\textbf{\emph{F1: Noise perturbations have a stronger effect on formalisation methods than informal \ac{LLM} reasoners.}}}
Table~\ref{tab:distraction_k4_formalisation} shows that, on average, the accuracy of both direct and \ac{CoT} informal reasoning remains between $73\%$ and $74\%$ in the face of added noise. While the autoformalisation method performs similarly to informal reasoners on the original dataset, its performance decreases between $4\%$ and $11\%$. The accuracy drops especially with logical (L) and tautological (T) distractions, whose logical language formats trick the \ac{LLM} into formalizing the noisy clauses. On the other hand, the linguistically complex and more natural sentences of encyclopedic distractions show a minor effect, suggesting that \acp{LLM} successfully avoids formalizing the more complicated sentences.

\paragraph{\textbf{\emph{F2: All \ac{LLM}-based reasoning methods suffer a drop for counterfactual perturbations.}}} % influence .}}}
Table~\ref{tab:distraction_k4_formalisation} shows that counterfactual statements cause a significant decrease in performance for both the informal reasoners and autoformalisation methods of between $12\%$ and $13\%$ on average. 
Moreover, this observation also holds for all tested models, i.e., none are robust towards counterfactual perturbations across every evaluated dimension. Even the strongest model, GPT 4o-mini, yields a performance of 63-68\%, which is relatively close to the random performance of 50\%. The high impact of counterfactual statements (the single ``not'' inserted) could be due to the inability of \acp{LLM} to overwrite prior knowledge with explicitly stated information or memorization of the answers. We study the error sources further in §\ref{subsec:errors}.  

\noindent \paragraph{\textbf{\emph{F3: Introducing multiple noise sentences has an effect only for logical distractions.}}}
We show the impact of introducing between one and four sentences for the two top-performing autoformalisation models in Figure~\ref{fig:length_distraction}. The figure shows similar trends with and without counterfactual perturbations.
As additional logical distractions are introduced, the model performance consistently decreases. Tautological (T) distractions lead to a decline in accuracy with a single disruptive sentence, yet adding more noise does not worsen the outcome. 
The tautological corpus introduces truth constants for all sentences as a persistent unseen logical construct. Given that this leads only to a decrease for a single occurrence, we can assume that a model can consistently handle the same unseen logical construct. In contrast, the logical corpus increases the chance of adding text, requiring new, previously unseen reasoning constructs for each added sentence. The impact of encyclopedic noise remains negligible, generalising F1 to $k$ sentences. Similarly, counterfactual perturbations remain much more effective for all settings, generalising F2.

\begin{table}[!t]
\small
\setlength{\modelspacing}{2pt}
\setlength{\tabcolsep}{1.7pt} % Default value: 6pt
\setlength{\belowrulesep}{4pt}
\begin{threeparttable}
    \centering
    \begin{tabular}{cc l r rrr @{\quad} rrrr}
\toprule
\multirow{2}{*}{} & \multirow{2}{*}{} & Reasoning & \multirow{2}{*}{O} & \multicolumn{3}{c}{Distraction} & \multicolumn{4}{c}{Counterfactual} \\
 & & Format & & E& L & T & $\text{O}_C$ & $\text{E}_C$& $\text{L}_C$ & $\text{T}_C$\\
\midrule
\multirow{6}{*}{\rotatebox{90}{Gemma-2}} & \multirow{3}{*}{\rotatebox{90}{9b}}
   & Informal (direct) & \textbf{0.78} & \textbf{0.80} & \textbf{0.79} & \textbf{0.77} & 0.58 & 0.52 & 0.50 & 0.59 \\
 & & Informal (CoT) & 0.72 & 0.78 & 0.73 & 0.76 & 0.61 & \textbf{0.57} & \textbf{0.60} & \textbf{0.66} \\
 & & Formal (FOL) & 0.62 & 0.58 & 0.52 & 0.53 & \textbf{0.63} & 0.52 & 0.46 & 0.46 \\[\modelspacing]
\cmidrule{2-11}
 & \multirow{3}{*}{\rotatebox{90}{27b}} 
   & Informal (direct) & 0.71 & 0.69 & \textbf{0.66} & \textbf{0.68} & 0.59 & 0.51 & 0.54 & 0.59 \\
 & & Informal (CoT) & 0.66 & 0.65 & 0.64 & 0.63 & 0.62 & 0.58 & \textbf{0.62} & \textbf{0.64} \\
 & & Formal (FOL) & \textbf{0.74} & \textbf{0.74} & 0.61 & 0.61 & \underline{\textbf{0.72}} & \underline{\textbf{0.67}} & 0.58 & 0.51 \\[\modelspacing]
\midrule
\multirow{6}{*}{\rotatebox{90}{Mistral}} & \multirow{3}{*}{\rotatebox{90}{7B}} 
   & Informal (direct) & 0.77 & \textbf{0.77} & 0.75 & \textbf{0.79} & \textbf{0.63} & \textbf{0.54} & \textbf{0.54} & \textbf{0.66} \\
 & & Informal (CoT) & \textbf{0.79} & 0.75 & \textbf{0.77} & 0.78 & 0.55 & 0.52 & \textbf{0.54} & 0.58 \\
 & & Formal (FOL) & 0.62 & 0.58 & 0.54 & 0.57 & 0.50 & \textbf{0.54} & 0.51 & 0.52 \\[\modelspacing]
\cmidrule{2-11}
 & \multirow{3}{*}{\rotatebox{90}{Small}} 
   & Informal (direct) & \textbf{0.77} & \textbf{0.76} & \textbf{0.76} & \textbf{0.75} & 0.61 & 0.51 & 0.56 & 0.59 \\
 & & Informal (CoT) & 0.72 & 0.72 & 0.72 & 0.71 & \textbf{0.62} & \textbf{0.59} & \textbf{0.62} & \textbf{0.68} \\
 & & Formal (FOL) & 0.68 & 0.59 & 0.53 & 0.64 & 0.54 & 0.55 & 0.49 & 0.51 \\[\modelspacing]
\midrule
\multirow{6}{*}{\rotatebox{90}{Llama-3.1}} & \multirow{3}{*}{\rotatebox{90}{8B}} 
   & Informal (direct) & 0.63 & 0.61 & 0.64 & 0.66 & 0.61 & \textbf{0.62} & 0.59 & 0.61 \\
 & & Informal (CoT) & 0.73 & \textbf{0.73} & \textbf{0.71} & \textbf{0.72} & \textbf{0.62} & 0.59 & \textbf{0.61} & \textbf{0.65} \\
 & & Formal (FOL) & \textbf{0.77} & 0.71 & 0.63 & 0.52 & 0.60 & 0.58 & 0.55 & 0.52 \\[\modelspacing]
\cmidrule{2-11}
 & \multirow{3}{*}{\rotatebox{90}{70B}} 
   & Informal (direct) & 0.77 & 0.74 & 0.74 & 0.73 & 0.62 & 0.53 & 0.56 & 0.64 \\
 & & Informal (CoT) & \textbf{0.78} & \textbf{0.75} & \textbf{0.76} & \textbf{0.76} & 0.64 & 0.61 & \textbf{0.66} & \underline{\textbf{0.73}} \\
 & & Formal (FOL) & 0.74 & 0.73 & 0.71 & 0.71 & \textbf{0.66} & \textbf{0.62} & 0.59 & 0.57 \\[\modelspacing]
 \midrule
\multirow{3}{*}{\rotatebox{90}{GPT}} & \multirow{3}{*}{\rotatebox{90}{4o-mini}} 
   & Informal (direct) & 0.78 & 0.77 & 0.79 & 0.79 & 0.64 & 0.61 & 0.61 & 0.63 \\
 & & Informal (CoT) & 0.80 & 0.80 & \underline{\textbf{0.81}} & \underline{\textbf{0.82}} & \textbf{0.68} & \textbf{0.63} & \underline{\textbf{0.68}} & \textbf{0.64} \\
 & & Formal (FOL) & \underline{\textbf{0.84}} & \underline{\textbf{0.82}} & 0.73 & 0.79 & 0.63 & 0.62 & 0.57 & 0.54 \\[\modelspacing]
 \midrule
\multicolumn{2}{c}{\multirow{3}{*}{\textbf{Avg}}} 
 & Informal (direct) & 0.74 & 0.73 & 0.73 & 0.73 & 0.61 & 0.55 & 0.56 & 0.62 \\
 & & Informal (CoT) & 0.74 & 0.74 & 0.73 & 0.74 & 0.62 & 0.58 & 0.62 & 0.65 \\
  & & Formal (FOL) & 0.72 & 0.68 &	0.61 & 0.62 & 0.61 & 0.59 & 0.54 & 0.52 \\
\bottomrule
\end{tabular}
\caption{Accuracies of informal and autoformalisation-based deductive reasoners. The best overall model per dataset is underlined; the best model version is marked in bold.}
\label{tab:distraction_k4_formalisation}
\end{threeparttable}
\end{table} 

\begin{figure}[!t]
    \centering
    \scriptsize
    \begin{tikzpicture}
        \begin{axis}[name=gpt,
            title={GPT-4o-mini},
            width=0.6\linewidth,
            height=0.6\linewidth,
            xlabel={\# Noise sentences},
            ylabel={Accuracy},
            xmin=-0.1, xmax=4.1,
            ymin=0.5, ymax=0.9,
            xtick={1,2,4},
            ytick={0.55, 0.6, 0.65, 0.75, 0.8, 0.85},
            title style={yshift=-0.6em},
            legend style={at={(1,-0.15)},
	           anchor=north,legend columns=-1},
            x label style={at={(axis description cs:1,-0.05)},anchor=north},
            y label style={at={(axis description cs:-0.15,0.5)},anchor=south},
            ymajorgrids=true,
            grid style=dashed,
        ]
            \addplot[color=blue, mark=square,]
                coordinates {
                (0,0.848076939582825)(1,0.823076903820038)(2,0.826923072338104)(4,0.821153819561005)
                };
            \addplot[color=red, mark=triangle,]
                coordinates {
                (0,0.848076939582825)(1,0.817307710647583)(2,0.801923096179962)(4,0.759615361690521)
                };
            \addplot[color=green, mark=diamond,] 
                coordinates {
                (0,0.848076939582825)(1,0.767307698726654)(2,0.769230782985687)(4,0.803846180438995)
                };
            \addplot[color=blue, mark=square*] 
                coordinates {
                (0,0.627777755260468)(1,0.622222244739533)(2,0.600000023841858)(4,0.633333325386047)
                };
            \addplot[color=red, mark=triangle*,] 
                coordinates {
                (0,0.627777755260468)(1,0.611111104488373)(2,0.611111104488373)(4,0.594444453716278)
                };
            \addplot[color=green, mark=diamond*,] 
                coordinates {
                (0,0.627777755260468)(1,0.572222232818604)(2,0.538888871669769)(4,0.555555582046509)
                };
                \legend{E,L,T,$\text{E}_C$, $\text{L}_C$ , $\text{T}_C$}
        \end{axis}

        \begin{axis}[name=llama, at={($(gpt.east)+(0.1cm,0)$)},anchor=west,
            title={Llama 3.1 70b},
            width=0.6\linewidth,
            height=0.6\linewidth,
            xmin=-0.1,, xmax=4.1,
            ymin=0.5, ymax=0.9,
            xtick={1,2,4},
            ytick={0.55, 0.6, 0.65, 0.75, 0.8, 0.85},
            title style={yshift=-0.6em},
            yticklabel=\empty,
            ymajorgrids=true,
            grid style=dashed,
        ]
            \addplot[color=blue, mark=square,]
                coordinates {
                (0,0.838461518287659)(1,0.817307710647583)(2,0.805769205093384)(4,0.817307710647583)
                };
            \addplot[color=red, mark=triangle,]
                coordinates {
                (0,0.838461518287659)(1,0.819230794906616)(2,0.803846180438995)(4,0.771153867244721)
                };
            \addplot[color=green, mark=diamond,]
                coordinates {
                (0,0.838461518287659)(1,0.803846180438995)(2,0.807692289352417)(4,0.805769205093384)
                };
            \addplot[color=blue, mark=square*]
                coordinates {
                (0,0.627777755260468)(1,0.622222244739533)(2,0.577777802944183)(4,0.594444453716278)
                };
            \addplot[color=red, mark=triangle*,]
                coordinates {
                (0,0.627777755260468)(1,0.583333313465118)(2,0.561111092567444)(4,0.577777802944183)
                };
            \addplot[color=green, mark=diamond*,]
                coordinates {
                (0,0.627777755260468)(1,0.627777755260468)(2,0.566666662693024)(4,0.577777802944183)
                };
        \end{axis}
    \end{tikzpicture}
    \caption{Influence of the number of noisy sentences for FOL.}
    \label{fig:length_distraction}
\end{figure}



\subsection{Impact of Method Design}
\paragraph{\textbf{\emph{F4: \ac{CoT} prompting is most impactful when both noise and counterfactual perturbations are applied.}}}
The accuracies for the individual \acp{LLM} in Table~\ref{tab:distraction_k4_formalisation} show that the impact of \ac{CoT} is negligible for noise-only datasets (first four columns). Meanwhile, the benefit from \ac{CoT} is most pronounced in the datasets that combine noise and counterfactual perturbations.
The better-performing informal prompting strategy for a model remains stable for all types of distractions. Still, the decline in performance due to counterfactuals leads to a less consistent preference for a specific prompting style.

\paragraph{\textbf{\emph{F5: The best-performing grammar differs per model and is unstable across data versions.}}}

The evaluation of different logical forms for formal \ac{LLM}-based reasoning in Table~\ref{tab:distraction_k4_logical_form} shows the preference of some models for specific syntactic formats.
Llama 3.1 70B has a considerable improvement of $12\%$ with TPTP syntax on the original set, while Llama 3.1 8B benefits from the R-FOL syntax. However, all grammars show a declining accuracy trend and increased syntax errors for noise perturbations, where the best grammar loses its advantage over the rest. 
When comparing the grammars on the counterfactual partitions, we observe that TPTP is consistently more robust than the standard first-order logic grammar. Here, GPT 4o-mini shows a reduction from $O$ to $O_C$ of $20\%$ for FOL and only $12\%$ for the TPTP grammar. Since this does not correlate with fewer syntax errors, the formalisation in TPTP prevents semantical errors for counterfactual premises. 
A positive reading of these results, especially the minor differences between FOL and R-FOL, is that autoformalisation \acp{LLM} can adapt to the grammar syntax prescribed in the prompt without further loss in performance.

\begin{table}[!t]
\small
\setlength{\modelspacing}{2pt}
\setlength{\tabcolsep}{1.7pt} % Default value: 6pt
\setlength{\belowrulesep}{4pt}
\begin{threeparttable}
    \centering
    \begin{tabular}{cc l r rrr @{\quad} rrrr}
\toprule
\multirow{2}{*}{} & \multirow{2}{*}{} & Grammar & \multirow{2}{*}{O} & \multicolumn{3}{c}{Distraction} & \multicolumn{4}{c}{Counterfactual} \\
 & & Syntax & & E& L & T & $\text{O}_C$ & $\text{E}_C$& $\text{L}_C$ & $\text{T}_C$\\
\midrule
\multirow{6}{*}{\rotatebox{90}{Llama-3.1}} & \multirow{3}{*}{\rotatebox{90}{8B}} 
   & FOL & 0.77 & \textbf{0.71} & 0.61 & \textbf{0.53} & 0.58 & \textbf{0.55} & 0.52 & \textbf{0.56} \\
 & & R-FOL & \textbf{0.78} & 0.69 & \textbf{0.62} & \textbf{0.53} & 0.58 & \textbf{0.55} & \textbf{0.54} & 0.52 \\
 & & TPTP & 0.73 & 0.67 & 0.55 & 0.51 & \textbf{0.68} & 0.54 & 0.46 & 0.51 \\[\modelspacing]
\cmidrule{2-11}
 & \multirow{3}{*}{\rotatebox{90}{70B}} 
   & FOL & 0.76 & 0.73 & 0.71 & \textbf{0.72} & 0.67 & 0.57 & 0.63 & 0.56 \\
 & & R-FOL & 0.76 & 0.73 & 0.67 & 0.71 & 0.64 & 0.57 & 0.53 & 0.64 \\
 & & TPTP & \underline{\textbf{0.88}} & \underline{\textbf{0.84}} & \underline{\textbf{0.81}} & \textbf{0.72} & \underline{\textbf{0.81}} & \underline{\textbf{0.68}} & \underline{\textbf{0.67}} & \underline{\textbf{0.68}} \\[\modelspacing]
\midrule
\multirow{3}{*}{\rotatebox{90}{GPT}} & \multirow{3}{*}{\rotatebox{90}{4o-mini}} 
   & FOL & \textbf{0.84} & \textbf{0.82} & \textbf{0.72} & \underline{\textbf{0.78}} & 0.64 & \textbf{0.63} & \textbf{0.61} & 0.51 \\
 & & R-FOL & \textbf{0.84} & 0.77 & 0.70 & \underline{\textbf{0.78}} & \textbf{0.72} & 0.56 & 0.54 & \textbf{0.63} \\
 & & TPTP & 0.83 & \textbf{0.82} & 0.71 & 0.71 & 0.69 & \textbf{0.63} & 0.57 & 0.57 \\
\bottomrule
\end{tabular}
\caption{Accuracies of different formalisation grammars for autoformalisation.}
\label{tab:distraction_k4_logical_form}
\end{threeparttable}
\end{table} 

\paragraph{\textbf{\emph{F6: Feedback does not help \acp{LLM} self-correct to mitigate robustness issues.}}}
\autoref{tab:distraction_k4_feedback} shows the results with different error recovery mechanisms. The results indicate that no feedback strategy emerges as a winner in the different datasets. 
All feedback variants reduce syntax errors for noise perturbations, but given the lack of a consistent increase in accuracy, the corrected formalisations are most likely to contain semantic errors still. 
The type of feedback message only has a minor influence on correcting syntax errors, whereas Llama 3.1 70b and GPT 4o-mini correct slightly more syntax errors with specific error messages. This finding aligns with \cite{huang2023large}, who also found that \acp{LLM} cannot consistently self-correct their reasoning after receiving relevant feedback.

\begin{table}[!ht]
\small
\setlength{\modelspacing}{2pt}
\setlength{\tabcolsep}{1.7pt} % Default value: 6pt
\setlength{\belowrulesep}{4pt}
\begin{threeparttable}
    \centering
    \begin{tabular}{cc l r rrr @{\quad} rrrr}
\toprule
\multirow{2}{*}{} & \multirow{2}{*}{} & \multirow{2}{*}{Feedback} & \multirow{2}{*}{O} & \multicolumn{3}{c}{Distraction} & \multicolumn{4}{c}{Counterfactual} \\
 & & & & E& L & T & $\text{O}_C$ & $\text{E}_C$& $\text{L}_C$ & $\text{T}_C$\\
\midrule
\multirow{8}{*}{\rotatebox{90}{Llama-3.1}} & \multirow{4}{*}{\rotatebox{90}{8B}} 
   & No recovery & 0.77 & \textbf{0.72} & 0.62 & 0.53 & 0.59 & 0.58 & 0.56 & \textbf{0.56} \\
 & & Error type & \textbf{0.79} & 0.71 & 0.63 & \textbf{0.56} & \textbf{0.66} & 0.54 & 0.52 & 0.51 \\
 & & Error message & 0.78 & 0.71 & \textbf{0.67} & 0.55 & 0.59 & 0.53 & \underline{\textbf{0.64}} & 0.49 \\
 & & Warning & 0.74 & 0.66 & 0.58 & 0.55 & 0.55 & \textbf{0.60} & 0.49 & 0.49 \\[\modelspacing]
\cmidrule{2-11}
 & \multirow{4}{*}{\rotatebox{90}{70B}} 
   & No recovery & \textbf{0.77} & \textbf{0.72} & \textbf{0.73} & 0.71 & \textbf{0.64} & 0.59 & \textbf{0.61} & 0.56 \\
 & & Error type & 0.72 & 0.70 & 0.72 & \textbf{0.73} & 0.62 & 0.56 & 0.60 & 0.58 \\
 & & Error message & 0.71 & 0.70 & \textbf{0.73} & 0.71 & \textbf{0.64} & 0.59 & 0.54 & \underline{\textbf{0.64}} \\
 & & Warning & 0.69 & \textbf{0.72} & 0.72 & 0.72 & 0.62 & \underline{\textbf{0.65}} & \textbf{0.61} & 0.63 \\[\modelspacing]
\midrule
\multirow{4}{*}{\rotatebox{90}{GPT}} & \multirow{4}{*}{\rotatebox{90}{4o-mini}} 
   & No recovery & \underline{\textbf{0.84}} & \underline{\textbf{0.82}} & 0.73 & 0.79 & 0.64 & \textbf{0.62} & 0.56 & \textbf{0.56} \\
 & & Error type & 0.83 & 0.79 & 0.74 & 0.76 & 0.67 & 0.57 & 0.56 & \textbf{0.56} \\
 & & Error message & \underline{\textbf{0.84}} & 0.78 & \underline{\textbf{0.77}} & \underline{\textbf{0.80}} & 0.62 & 0.59 & 0.56 & \textbf{0.56} \\
 & & Warning & \underline{\textbf{0.84}} & 0.75 & 0.73 & 0.76 & \underline{\textbf{0.70}} & 0.61 & \textbf{0.61} & 0.55 \\
 \bottomrule
\end{tabular}
\caption{Accuracies of error recovery strategies.}
\label{tab:distraction_k4_feedback}
\end{threeparttable}
\end{table} 

\subsection{Error Analysis}
\label{subsec:errors}
\paragraph{\textbf{\emph{F7: Autoformalisation increases syntax errors for noise perturbations.}}}
The low performance for noise perturbations correlates with more syntax errors for all models and distraction categories (cf. execution rates in Table~\ref{tab:appendix_k4_formalisation_exec}). The three worst-performing models (both Mistral models, Gemma-2 9b) generate, at best, for $37\%$  and, at worst, for only $4\%$ of the samples, a valid logical form.
Gemma-2 9b and Llama3.1 8b produce more syntax errors than the larger counterparts, suggesting that larger models are more robust towards noise perturbations. 
The accuracy of syntactically valid samples is higher than the informal reasoning methods for most distractions (Table~\ref{tab:appendix_k4_formalisation_vacc}), motivating informal reasoning as a backup strategy for formal reasoning. The error message feedback reveals two common syntax errors: 1) errors by models with an initial low execution rate exhibit issues with the template structure, including using incorrect keywords or adding conversational phrases;
2) perturbation-related errors, the most common of which is using undefined truth constants as part of tautological distractions. 

\paragraph{\textbf{\emph{F8: Autoformalisation increases semantic errors for counterfactuals.}}}
Unlike the introduced noise, counterfactual perturbations do not lead to more syntax errors. The execution rate in Table~\ref{tab:appendix_k4_formalisation_exec} is stable or improves for counterfactuals. However, we see a drop in accuracy for the counterfactual column $\text{O}_C$ in Table~\ref{tab:distraction_k4_formalisation} and can conclude that the number of logical forms with semantic errors has to increase. This suggests that the introduced negation is not correctly formalised. Looking at the warnings generated by the feedback mechanism, for GPT 4o-mini, $161$ warning messages are generated on the unperturbed data. $54$ of these were fixed with a single iteration. Not considering predicates and individuals as part of the context is the most frequent warning across all models. 
\section{Related Work}
% \subsection{Vision Language Model}
% 시각장애인에서 상황을 설명할 DB가 없으니 만들었다. 그리고 이를 VLM에 튜닝했다.
\subsection{Technical approaches for assisting the visually-impaired}


\subsection{Datasets for visual instruction tuning}

\section*{Conclusion}
This paper aims to enhance our understanding of the computational complexity of computing various Shapley value variants. We found that for various ML models --- including decision trees, regression tree ensembles, weighted automata, and linear regression --- both local and global interventional and baseline SHAP can be computed in polynomial time under HMM modeled distributions. This extends popular algorithms, such as TreeSHAP, beyond their empirical distributional scope. We also establish strict complexity gaps between the various SHAP variants (baseline, interventional, and conditional) and prove the intractability of computing SHAP for tree ensembles and neural networks in simplified scenarios. Overall, we present SHAP as a versatile framework whose complexity depends on four key factors: \begin{inparaenum}[(i)] \item model type, \item SHAP variant, \item distribution modeling approach, \item and local vs. global explanations\end{inparaenum}. We believe this perspective provides deeper insight into the computational complexity of SHAP, paving the way for future work.




%We believe that our framework provides a more intricate understanding of SHAP computation complexity across different models, distributions, and variants, paving the way for further research.

Our work opens promising directions for future research. First, expanding our computational analysis to other SHAP-related metrics, such as asymmetric SHAP~\citep{frye20} and SAGE~\citep{covert2020understanding}, would be valuable. Additionally, we aim to explore more expressive distribution classes and relaxed assumptions beyond those in Section \ref{sec:tractable} while maintaining tractable SHAP computation. Finally, when exact computation is intractable (Section \ref{sec:intractable}), investigating the approximability of SHAP metrics through approximation and parameterized complexity theory~\citep{downey2012parameterized} is an important direction.

%Our work opens several promising avenues for future research on the computational properties of explainable AI methods, with a particular focus on SHAP. First, it would be interesting to broaden the computational analysis conducted in this work to include other popular SHAP-related metrics in the literature, such as asymmetric SHAP \cite{frye20} and SAGE \cite{covert2020understanding}. Also, in the future, we aim to explore more expressive distribution classes and relaxed distributional assumptions—extending beyond those examined in Section \ref{sec:tractable} —that still yield tractable SHAP computation. Finally, when exact computation proves intractable (Section \ref{sec:intractable}), it is worthwhile to theoretically investigate the question of the approximability of computing the SHAP metrics across various configurations, through the lens of approximation and parametrized complexity theory \cite{arora2009computational}.

%This paper aims to deepen our understanding of the computational complexity involved in obtaining different Shapley value variants. We found that for a variety of ML models, including decision trees, tree ensembles for regression, weighted automata, and linear regression models — computing both local and global interventional and baseline SHAP can be done in polynomial time when distributions are modeled by HMMs. This extends the distributional scope of popular algorithms like TreeSHAP, which is limited to empirical distributions. Additionally, we demonstrate a strict complexity gap between SHAP variants, showing that interventional and baseline SHAP can be strictly easier to compute than conditional SHAP. Despite these positive results, we uncovered intractability for various SHAP variants in neural networks and tree ensembles. Finally, we provided generalized complexity relations across SHAP variants. We believe that our framework offers a deeper understanding of the complexity involved in computing SHAP across various variants, models, distributions, as well as in both local and global computations, laying the groundwork for future research.


\begin{comment}
\section*{Acknowledgments}
%-------------------------------------------------------------------------------

The USENIX latex style is old and very tired, which is why
there's no \textbackslash{}acks command for you to use when
acknowledging. Sorry.

%-------------------------------------------------------------------------------
\section*{Availability}
%-------------------------------------------------------------------------------

USENIX program committees give extra points to submissions that are
backed by artifacts that are publicly available. If you made your code
or data available, it's worth mentioning this fact in a dedicated
section.
    
\end{comment}

%-------------------------------------------------------------------------------
\bibliographystyle{unsrt}
\bibliography{references/others,references/papers}

%%%%%%%%%%%%%%%%%%%%%%%%%%%%%%%%%%%%%%%%%%%%%%%%%%%%%%%%%%%%%%%%%%%%%%%%%%%%%%%%
\end{document}
%%%%%%%%%%%%%%%%%%%%%%%%%%%%%%%%%%%%%%%%%%%%%%%%%%%%%%%%%%%%%%%%%%%%%%%%%%%%%%%%

%%  LocalWords:  endnotes includegraphics fread ptr nobj noindent
%%  LocalWords:  pdflatex acks
