\section{Related work}
\paragraph{Steering vectors}
\citet{li2024inference} train linear probes on model activations to obtain steering vectors that induce truthfulness when added to certain attention heads.
\citet{arditi2024refusal} find steering vectors, across many models, that modulate refusal of harmful requests. \citet{panickssery2023steering} introduce \enquote{contrastive activation addition} (CAA), a method for obtaining steering vectors from a contrastive dataset (which we use as a skyline), and use it to find steering vectors for a variety of behaviors, including sycophancy. \citet{zou2023representation} obtain steering vectors using various probing methods and use it to control behaviors including honesty, fairness, and knowledge editing. \citet{liu2023context} obtain steering vectors from in-context learning activations and use them to modulate toxicity. 

\paragraph{Optimization methods} \citet{subramani2022extractinglatentsteeringvectors} optimize steering vectors using a process that we call \textit{promotion steering} in order to maximize the probability that an LLM generates a given sequence starting with the beginning of sentence token. However, unlike us, they do not use these vectors to induce general changes of behavior on a variety of inputs. Instead, to find behavior modification vectors, they first generate steering vectors for sequences in a contrastive dataset, and then take the difference of the means of the two classes of steering vectors. \citet{hernandez2023inspecting} optimize affine transformations on a dataset of inputs to perform knowledge editing; we, in contrast, focus on only optimizing steering vectors on a single input. \citet{mack2024melbo} introduce an \textit{unsupervised} optimization-based method for finding steering vectors that induce behavioral changes on a single prompt. In particular, they use this method to find anti-refusal steering vectors, anticipating our findings in \S\ref{sec:refusal}. However, because their method is unsupervised, it does not directly address our setting, in which we seek to target a specific behavior; for example, in order to find these anti-refusal vectors, the authors had to manually test 32 vectors.

\paragraph{Concurrent work on low-shot steering}
While we were writing our manuscript, \citet{turner2025bidpo} released a report detailing their attempts to optimize steering vectors in Gemini models that induce truthfulness. This uses a method called BiPO introduced by \citet{cao2024personalizedsteeringlargelanguage}, who use it to train steering vectors on large contrastive datasets (over 300 examples). In their investigations, \citet{turner2025bidpo} looked at the efficacy of BiPO in low-shot settings, including the single training example regime (as we focus on). However, they find that for the more powerful Gemini 1.5v2 model, optimizing steering vectors no longer beats baselines such as multi-shot prompting. While these results might initially seem to suggest that steering vectors have limited utility compared to prompting, \textit{we believe otherwise}, particularly because our results demonstrate that steering optimization allows model behavior to be controlled \textit{even in settings where it is unclear how to write prompts that elicit the desired behavior}.
(In \S\ref{sec:poser}, we optimize steering vectors to mediate harmful behavior in an alignment-faking model, using only training examples on which the model does not display the harmful behavior. 
And in \S\ref{sec:refusal}, we are able to optimize steering vectors that mediate the model's refusal of harmful requests, even though finding \enquote{jailbreaking} prompts that do the same is highly non-trivial.)
We think that this is particularly important in safety-relevant scenarios: prompting alone might not be enough to prevent a misaligned model from behaving harmfully, but directly intervening on model activations with steering vectors can more effectively do so.

\paragraph{Steering vector evaluation methods} \citet{tan2024analyzing} evaluate the performance of steering vectors on a variety of multiple-choice datasets and finds that many steering vectors fail to generalize. \citet{pres2024reliableevaluationbehaviorsteering} introduce a method for evaluating steering efficacy based on the probabilities assigned to steered vs. unsteered completions on a contrastive dataset, allowing for evaluations of more open-ended generations. While not directly related to steering vectors, \citet{burden2024conversational} quantify the \enquote{conversational complexity} of jailbreak attempts by looking at the probabilities assigned to them, anticipating our evaluation methods in \S\ref{sec:steeringEval}.