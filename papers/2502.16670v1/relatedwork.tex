\section{Related Work}
\label{sec: related work}

The paper's reference domain is empirical software engineering focusing on
software security. The domain is large, but with some suitable framings, it is
possible to narrow it down. The first obvious framing is toward empirical
research. The second equally obvious framing is toward the PHP programming
language. With these two framings, a few examples can be given about implicitly
related work. For instance, technical debt of PHP packages for web applications
has been investigated empirically~\cite{Amanatidis17}. Regarding web
applications and websites in general, another example would be a large-scale
empirical study investigating the adoption of PHP interpreter releases among
website deployments~\cite{Ruohonen17APSEC}. A third framing can be done toward
software security. The previous two examples are in the scope also with this
further software security \text{framing---technical} debt is related to software
quality, which is a superset for software security, while the adoption (or a
lack thereof) of new PHP interpreter releases contains security risks in case
many (or some important) websites do not update their interpreters for a reason
or another. The reason for the risks is that also the PHP interpreter has seen
many vulnerabilities over the years.

A lot of empirical research has also been done to examine the security of
software written in PHP more generally. For instance, the security of plugins
for popular PHP-based web frameworks has been investigated~\cite{Niemietz21,
  Ruohonen19EASE}. Regarding software vulnerabilities and their underlying
weaknesses, and despite the availability of defensive solutions~\cite{Dahse15},
cross-site scripting and input validation more generally have been the most
typical weaknesses in PHP software~\cite{Ruohonen19EASE, Santos17}. Though, it
can be noted that this observation is hardly unique to PHP. In other words, the
same weaknesses typically lead the scoreboards also with other popular
interpreted programming languages used in the web
domain~\cite{Ruohonen18IWESEP}. Having said that, it should be emphasized that
empirical software security research on PHP applications is not limited only on
vulnerabilities and weaknesses. A good example would be a forensics
investigation of PHP applications in containerized cloud environments based on
empirical log analysis~\cite{Hyder24}. Furthermore, a third framing can be done
toward replication research.

Despite a long debate on a real, perceived, or alleged need to align with
empirical sciences in cyber security research~\cite{Herley17}, the empirical
foundations are arguably still rather limited. Like in empirical software
engineering~\cite{Shepperd13}, which also partially frames the current paper,
systematic literature reviews have improved the systematization of knowledge but
actual cumulation of empirical evidence, whether done through meta-analyses or
by other means, has been limited. Many reasons for this limitation could be
pointed out and further speculated. Among other things: like in many fields,
there has been a lack of incentives in cyber security research to share
datasets~\cite{Zheng18}. Although there are replications also in cyber
security~\cite{Ruohonen15COSE}, the data sharing limitation has supposedly
contributed to their volume. To this end, the paper contributes to the efforts
to strengthen the empirical knowledge base in cyber security.

A fourth and final clarifying framing can be done toward research on software
ecosystems and particularly the security or insecurity of software distributed
via them. The empirical security-oriented ecosystem research is again vast. A
recurring theme is that software packages distributed in software ecosystems via
language-specific package managers are generally of poor equality, often
containing security issues of various kinds, including concrete
vulnerabilities~\cite{Ruohonen21PST}. In addition, the ecosystems exhibit a risk
of malware being uploaded, as is often done together with so-called
typo-squatting \cite{Ruohonen18IWESEP, Vaidya19}. The underlying security risks
are intensified by the heavy use of software dependencies in these
ecosystems~\cite{Zerouali22}. Given this background, it is understandable that
there are also various ongoing funded projects to improve the ecosystems and the
supply chain security of open source software in
general~\cite{Ruohonen24JSS}. The software ecosystem research and dependencies
are also important to underline because the concept of (software) popularity is
closely related to them~\cite{Kula18, Qiu18}. The framing toward software
ecosystems is also relevant as it is also directly related to the dataset
examined. On that note, the materials and methods should also be elaborated.