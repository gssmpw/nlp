%%%%%%%% ICML 2025 EXAMPLE LATEX SUBMISSION FILE %%%%%%%%%%%%%%%%%

\documentclass{article}

% Recommended, but optional, packages for figures and better typesetting:
\usepackage{microtype}
\usepackage{graphicx}
\usepackage{subfigure}
\usepackage{booktabs} % for professional tables

% hyperref makes hyperlinks in the resulting PDF.
% If your build breaks (sometimes temporarily if a hyperlink spans a page)
% please comment out the following usepackage line and replace
% \usepackage{icml2025} with \usepackage[nohyperref]{icml2025} above.
\usepackage{hyperref}


% Attempt to make hyperref and algorithmic work together better:
\newcommand{\theHalgorithm}{\arabic{algorithm}}

% Use the following line for the initial blind version submitted for review:
% \usepackage{icml2025}

% If accepted, instead use the following line for the camera-ready submission:
\usepackage[accepted]{icml2025}

% For theorems and such
\usepackage{amsmath}
\usepackage{amssymb}
\usepackage{mathtools}
\usepackage{amsthm}
\usepackage{multirow}
\usepackage{pgfplots}
\usepackage{subcaption} %subcaption a b c
\usepackage{pifont}
\usepackage{float}
\usepackage{marvosym}

\newcommand{\Checkmark}{\ding{51}}       % 勾选符号
\newcommand{\XSolidBrush}{\ding{55}}    % 叉号符号
\newcommand{\yx}[1]{{\color{blue} #1}}

% if you use cleveref..
\usepackage[capitalize,noabbrev]{cleveref}

%%%%%%%%%%%%%%%%%%%%%%%%%%%%%%%%
% THEOREMS
%%%%%%%%%%%%%%%%%%%%%%%%%%%%%%%%
\theoremstyle{plain}
\newtheorem{theorem}{Theorem}[section]
\newtheorem{proposition}[theorem]{Proposition}
\newtheorem{lemma}[theorem]{Lemma}
\newtheorem{corollary}[theorem]{Corollary}
\theoremstyle{definition}
\newtheorem{definition}[theorem]{Definition}
\newtheorem{assumption}[theorem]{Assumption}
\theoremstyle{remark}
\newtheorem{remark}[theorem]{Remark}

% Todonotes is useful during development; simply uncomment the next line
%    and comment out the line below the next line to turn off comments
%\usepackage[disable,textsize=tiny]{todonotes}
\usepackage[textsize=tiny]{todonotes}


% The \icmltitle you define below is probably too long as a header.
% Therefore, a short form for the running title is supplied here:
\icmltitlerunning{IPSeg: Image Posterior Mitigates Semantic Drift in Class-Incremental Segmentation}

\begin{document}

\twocolumn[{
\icmltitle{IPSeg: Image Posterior Mitigates Semantic Drift in \\Class-Incremental Segmentation}

% It is OKAY to include author information, even for blind
% submissions: the style file will automatically remove it for you
% unless you've provided the [accepted] option to the icml2025
% package.

% List of affiliations: The first argument should be a (short)
% identifier you will use later to specify author affiliations
% Academic affiliations should list Department, University, City, Region, Country
% Industry affiliations should list Company, City, Region, Country

% You can specify symbols, otherwise they are numbered in order.
% Ideally, you should not use this facility. Affiliations will be numbered
% in order of appearance and this is the preferred way.
\icmlsetsymbol{equal}{*}
\begin{center}
\textbf{
Xiao Yu$^{1,2,\ast}$ \quad
Yan Fang$^{1,2,\ast}$ \quad
Yao Zhao$^{1,2}$ \quad
Yunchao Wei$^{1,2,}$\textsuperscript{\Letter} \quad
}\\
{
$^1$ Institute of Information Science, Beijing Jiaotong University \quad
$^2$ Visual Intelligence + X International Joint Laboratory \\
} $^\ast$equal contributors \quad \textsuperscript{\Letter}corresponding author \\
 {\tt\small wychao1987@gmail.com}
 % \vspace{-1cm}
\end{center}

\iffalse
\begin{icmlauthorlist}
\icmlauthor{Xiao Yu}{equal,b,x}
\icmlauthor{Yan Fang}{equal,b,x}
\icmlauthor{YunChao Wei}{b,x}
\icmlauthor{Yao Zhao}{b,x}
\end{icmlauthorlist}

\icmlaffiliation{b}{Institute of Information Science, Beijing Jiaotong University}
\icmlaffiliation{x}{Visual Intelligence + X International Joint Laboratory}

\icmlcorrespondingauthor{Yunchao Wei}{wychao1987@gmail.com}

% You may provide any keywords that you
% find helpful for describing your paper; these are used to populate
% the "keywords" metadata in the PDF but will not be shown in the document
\icmlkeywords{Machine Learning, ICML, }
\fi
\vskip 0.3in
}]

\definecolor{mygreen}{RGB}{93,173,85}
\definecolor{myred}{RGB}{192, 0, 0}
\makeatother
\newcommand{\pub}[1]{\color{gray}{\tiny{[{#1}]}}}
\newcommand{\reshll}[2]{
{#1} \fontsize{7.5pt}{1em}\selectfont\color{mygreen}{$\!\uparrow\!$ {#2}}
}
\newcommand{\reshl}[2]{
\textbf{#1} \fontsize{7.5pt}{1em}\selectfont\color{mygreen}{$\!\uparrow\!$ \textbf{#2}}
}

% this must go after the closing bracket ] following \twocolumn[ ...

% This command actually creates the footnote in the first column
% listing the affiliations and the copyright notice.
% The command takes one argument, which is text to display at the start of the footnote.
% The \icmlEqualContribution command is standard text for equal contribution.
% Remove it (just {}) if you do not need this facility.

%\printAffiliationsAndNotice{}  % leave blank if no need to mention equal contribution

% \printAffiliationsAndNotice{\icmlEqualContribution} % otherwise use the standard text.

\begin{abstract}
% 1. background
    Class incremental learning aims to enable models to learn from sequential, non-stationary data streams across different tasks without catastrophic forgetting. 
    In class incremental semantic segmentation (CISS), the semantic content of image pixels evolves over incremental phases, known as \textbf{ semantic drift}. 
    In this work, we identify two critical challenges in CISS that contribute to semantic drift and degrade performance. First, we highlight the issue of separate optimization, where different parts of the model are optimized in distinct incremental stages, leading to misaligned probability scales. Second, we identify noisy semantics arising from inappropriate pseudo-labeling, which results in sub-optimal results.
    To address these challenges, we propose a novel and effective approach, Image Posterior and Semantics Decoupling for Segmentation (IPSeg). IPSeg introduces two key mechanisms: (1) leveraging image posterior probabilities to align optimization across stages and mitigate the effects of separate optimization, and (2) employing semantics decoupling to handle noisy semantics and tailor learning strategies for different semantics. Extensive experiments on the Pascal VOC 2012 and ADE20K datasets demonstrate that IPSeg achieves superior performance compared to state-of-the-art methods, particularly in challenging long-term incremental scenarios. Our code is now available at \href{https://github.com/YanFangCS/IPSeg}{https://github.com/YanFangCS/IPSeg}.
\end{abstract}

\section{Introduction}
\label{sec:introduction}
The business processes of organizations are experiencing ever-increasing complexity due to the large amount of data, high number of users, and high-tech devices involved \cite{martin2021pmopportunitieschallenges, beerepoot2023biggestbpmproblems}. This complexity may cause business processes to deviate from normal control flow due to unforeseen and disruptive anomalies \cite{adams2023proceddsriftdetection}. These control-flow anomalies manifest as unknown, skipped, and wrongly-ordered activities in the traces of event logs monitored from the execution of business processes \cite{ko2023adsystematicreview}. For the sake of clarity, let us consider an illustrative example of such anomalies. Figure \ref{FP_ANOMALIES} shows a so-called event log footprint, which captures the control flow relations of four activities of a hypothetical event log. In particular, this footprint captures the control-flow relations between activities \texttt{a}, \texttt{b}, \texttt{c} and \texttt{d}. These are the causal ($\rightarrow$) relation, concurrent ($\parallel$) relation, and other ($\#$) relations such as exclusivity or non-local dependency \cite{aalst2022pmhandbook}. In addition, on the right are six traces, of which five exhibit skipped, wrongly-ordered and unknown control-flow anomalies. For example, $\langle$\texttt{a b d}$\rangle$ has a skipped activity, which is \texttt{c}. Because of this skipped activity, the control-flow relation \texttt{b}$\,\#\,$\texttt{d} is violated, since \texttt{d} directly follows \texttt{b} in the anomalous trace.
\begin{figure}[!t]
\centering
\includegraphics[width=0.9\columnwidth]{images/FP_ANOMALIES.png}
\caption{An example event log footprint with six traces, of which five exhibit control-flow anomalies.}
\label{FP_ANOMALIES}
\end{figure}

\subsection{Control-flow anomaly detection}
Control-flow anomaly detection techniques aim to characterize the normal control flow from event logs and verify whether these deviations occur in new event logs \cite{ko2023adsystematicreview}. To develop control-flow anomaly detection techniques, \revision{process mining} has seen widespread adoption owing to process discovery and \revision{conformance checking}. On the one hand, process discovery is a set of algorithms that encode control-flow relations as a set of model elements and constraints according to a given modeling formalism \cite{aalst2022pmhandbook}; hereafter, we refer to the Petri net, a widespread modeling formalism. On the other hand, \revision{conformance checking} is an explainable set of algorithms that allows linking any deviations with the reference Petri net and providing the fitness measure, namely a measure of how much the Petri net fits the new event log \cite{aalst2022pmhandbook}. Many control-flow anomaly detection techniques based on \revision{conformance checking} (hereafter, \revision{conformance checking}-based techniques) use the fitness measure to determine whether an event log is anomalous \cite{bezerra2009pmad, bezerra2013adlogspais, myers2018icsadpm, pecchia2020applicationfailuresanalysispm}. 

The scientific literature also includes many \revision{conformance checking}-independent techniques for control-flow anomaly detection that combine specific types of trace encodings with machine/deep learning \cite{ko2023adsystematicreview, tavares2023pmtraceencoding}. Whereas these techniques are very effective, their explainability is challenging due to both the type of trace encoding employed and the machine/deep learning model used \cite{rawal2022trustworthyaiadvances,li2023explainablead}. Hence, in the following, we focus on the shortcomings of \revision{conformance checking}-based techniques to investigate whether it is possible to support the development of competitive control-flow anomaly detection techniques while maintaining the explainable nature of \revision{conformance checking}.
\begin{figure}[!t]
\centering
\includegraphics[width=\columnwidth]{images/HIGH_LEVEL_VIEW.png}
\caption{A high-level view of the proposed framework for combining \revision{process mining}-based feature extraction with dimensionality reduction for control-flow anomaly detection.}
\label{HIGH_LEVEL_VIEW}
\end{figure}

\subsection{Shortcomings of \revision{conformance checking}-based techniques}
Unfortunately, the detection effectiveness of \revision{conformance checking}-based techniques is affected by noisy data and low-quality Petri nets, which may be due to human errors in the modeling process or representational bias of process discovery algorithms \cite{bezerra2013adlogspais, pecchia2020applicationfailuresanalysispm, aalst2016pm}. Specifically, on the one hand, noisy data may introduce infrequent and deceptive control-flow relations that may result in inconsistent fitness measures, whereas, on the other hand, checking event logs against a low-quality Petri net could lead to an unreliable distribution of fitness measures. Nonetheless, such Petri nets can still be used as references to obtain insightful information for \revision{process mining}-based feature extraction, supporting the development of competitive and explainable \revision{conformance checking}-based techniques for control-flow anomaly detection despite the problems above. For example, a few works outline that token-based \revision{conformance checking} can be used for \revision{process mining}-based feature extraction to build tabular data and develop effective \revision{conformance checking}-based techniques for control-flow anomaly detection \cite{singh2022lapmsh, debenedictis2023dtadiiot}. However, to the best of our knowledge, the scientific literature lacks a structured proposal for \revision{process mining}-based feature extraction using the state-of-the-art \revision{conformance checking} variant, namely alignment-based \revision{conformance checking}.

\subsection{Contributions}
We propose a novel \revision{process mining}-based feature extraction approach with alignment-based \revision{conformance checking}. This variant aligns the deviating control flow with a reference Petri net; the resulting alignment can be inspected to extract additional statistics such as the number of times a given activity caused mismatches \cite{aalst2022pmhandbook}. We integrate this approach into a flexible and explainable framework for developing techniques for control-flow anomaly detection. The framework combines \revision{process mining}-based feature extraction and dimensionality reduction to handle high-dimensional feature sets, achieve detection effectiveness, and support explainability. Notably, in addition to our proposed \revision{process mining}-based feature extraction approach, the framework allows employing other approaches, enabling a fair comparison of multiple \revision{conformance checking}-based and \revision{conformance checking}-independent techniques for control-flow anomaly detection. Figure \ref{HIGH_LEVEL_VIEW} shows a high-level view of the framework. Business processes are monitored, and event logs obtained from the database of information systems. Subsequently, \revision{process mining}-based feature extraction is applied to these event logs and tabular data input to dimensionality reduction to identify control-flow anomalies. We apply several \revision{conformance checking}-based and \revision{conformance checking}-independent framework techniques to publicly available datasets, simulated data of a case study from railways, and real-world data of a case study from healthcare. We show that the framework techniques implementing our approach outperform the baseline \revision{conformance checking}-based techniques while maintaining the explainable nature of \revision{conformance checking}.

In summary, the contributions of this paper are as follows.
\begin{itemize}
    \item{
        A novel \revision{process mining}-based feature extraction approach to support the development of competitive and explainable \revision{conformance checking}-based techniques for control-flow anomaly detection.
    }
    \item{
        A flexible and explainable framework for developing techniques for control-flow anomaly detection using \revision{process mining}-based feature extraction and dimensionality reduction.
    }
    \item{
        Application to synthetic and real-world datasets of several \revision{conformance checking}-based and \revision{conformance checking}-independent framework techniques, evaluating their detection effectiveness and explainability.
    }
\end{itemize}

The rest of the paper is organized as follows.
\begin{itemize}
    \item Section \ref{sec:related_work} reviews the existing techniques for control-flow anomaly detection, categorizing them into \revision{conformance checking}-based and \revision{conformance checking}-independent techniques.
    \item Section \ref{sec:abccfe} provides the preliminaries of \revision{process mining} to establish the notation used throughout the paper, and delves into the details of the proposed \revision{process mining}-based feature extraction approach with alignment-based \revision{conformance checking}.
    \item Section \ref{sec:framework} describes the framework for developing \revision{conformance checking}-based and \revision{conformance checking}-independent techniques for control-flow anomaly detection that combine \revision{process mining}-based feature extraction and dimensionality reduction.
    \item Section \ref{sec:evaluation} presents the experiments conducted with multiple framework and baseline techniques using data from publicly available datasets and case studies.
    \item Section \ref{sec:conclusions} draws the conclusions and presents future work.
\end{itemize}

\section{RELATED WORK}
\label{sec:relatedwork}
In this section, we describe the previous works related to our proposal, which are divided into two parts. In Section~\ref{sec:relatedwork_exoplanet}, we present a review of approaches based on machine learning techniques for the detection of planetary transit signals. Section~\ref{sec:relatedwork_attention} provides an account of the approaches based on attention mechanisms applied in Astronomy.\par

\subsection{Exoplanet detection}
\label{sec:relatedwork_exoplanet}
Machine learning methods have achieved great performance for the automatic selection of exoplanet transit signals. One of the earliest applications of machine learning is a model named Autovetter \citep{MCcauliff}, which is a random forest (RF) model based on characteristics derived from Kepler pipeline statistics to classify exoplanet and false positive signals. Then, other studies emerged that also used supervised learning. \cite{mislis2016sidra} also used a RF, but unlike the work by \citet{MCcauliff}, they used simulated light curves and a box least square \citep[BLS;][]{kovacs2002box}-based periodogram to search for transiting exoplanets. \citet{thompson2015machine} proposed a k-nearest neighbors model for Kepler data to determine if a given signal has similarity to known transits. Unsupervised learning techniques were also applied, such as self-organizing maps (SOM), proposed \citet{armstrong2016transit}; which implements an architecture to segment similar light curves. In the same way, \citet{armstrong2018automatic} developed a combination of supervised and unsupervised learning, including RF and SOM models. In general, these approaches require a previous phase of feature engineering for each light curve. \par

%DL is a modern data-driven technology that automatically extracts characteristics, and that has been successful in classification problems from a variety of application domains. The architecture relies on several layers of NNs of simple interconnected units and uses layers to build increasingly complex and useful features by means of linear and non-linear transformation. This family of models is capable of generating increasingly high-level representations \citep{lecun2015deep}.

The application of DL for exoplanetary signal detection has evolved rapidly in recent years and has become very popular in planetary science.  \citet{pearson2018} and \citet{zucker2018shallow} developed CNN-based algorithms that learn from synthetic data to search for exoplanets. Perhaps one of the most successful applications of the DL models in transit detection was that of \citet{Shallue_2018}; who, in collaboration with Google, proposed a CNN named AstroNet that recognizes exoplanet signals in real data from Kepler. AstroNet uses the training set of labelled TCEs from the Autovetter planet candidate catalog of Q1–Q17 data release 24 (DR24) of the Kepler mission \citep{catanzarite2015autovetter}. AstroNet analyses the data in two views: a ``global view'', and ``local view'' \citep{Shallue_2018}. \par


% The global view shows the characteristics of the light curve over an orbital period, and a local view shows the moment at occurring the transit in detail

%different = space-based

Based on AstroNet, researchers have modified the original AstroNet model to rank candidates from different surveys, specifically for Kepler and TESS missions. \citet{ansdell2018scientific} developed a CNN trained on Kepler data, and included for the first time the information on the centroids, showing that the model improves performance considerably. Then, \citet{osborn2020rapid} and \citet{yu2019identifying} also included the centroids information, but in addition, \citet{osborn2020rapid} included information of the stellar and transit parameters. Finally, \citet{rao2021nigraha} proposed a pipeline that includes a new ``half-phase'' view of the transit signal. This half-phase view represents a transit view with a different time and phase. The purpose of this view is to recover any possible secondary eclipse (the object hiding behind the disk of the primary star).


%last pipeline applies a procedure after the prediction of the model to obtain new candidates, this process is carried out through a series of steps that include the evaluation with Discovery and Validation of Exoplanets (DAVE) \citet{kostov2019discovery} that was adapted for the TESS telescope.\par
%



\subsection{Attention mechanisms in astronomy}
\label{sec:relatedwork_attention}
Despite the remarkable success of attention mechanisms in sequential data, few papers have exploited their advantages in astronomy. In particular, there are no models based on attention mechanisms for detecting planets. Below we present a summary of the main applications of this modeling approach to astronomy, based on two points of view; performance and interpretability of the model.\par
%Attention mechanisms have not yet been explored in all sub-areas of astronomy. However, recent works show a successful application of the mechanism.
%performance

The application of attention mechanisms has shown improvements in the performance of some regression and classification tasks compared to previous approaches. One of the first implementations of the attention mechanism was to find gravitational lenses proposed by \citet{thuruthipilly2021finding}. They designed 21 self-attention-based encoder models, where each model was trained separately with 18,000 simulated images, demonstrating that the model based on the Transformer has a better performance and uses fewer trainable parameters compared to CNN. A novel application was proposed by \citet{lin2021galaxy} for the morphological classification of galaxies, who used an architecture derived from the Transformer, named Vision Transformer (VIT) \citep{dosovitskiy2020image}. \citet{lin2021galaxy} demonstrated competitive results compared to CNNs. Another application with successful results was proposed by \citet{zerveas2021transformer}; which first proposed a transformer-based framework for learning unsupervised representations of multivariate time series. Their methodology takes advantage of unlabeled data to train an encoder and extract dense vector representations of time series. Subsequently, they evaluate the model for regression and classification tasks, demonstrating better performance than other state-of-the-art supervised methods, even with data sets with limited samples.

%interpretation
Regarding the interpretability of the model, a recent contribution that analyses the attention maps was presented by \citet{bowles20212}, which explored the use of group-equivariant self-attention for radio astronomy classification. Compared to other approaches, this model analysed the attention maps of the predictions and showed that the mechanism extracts the brightest spots and jets of the radio source more clearly. This indicates that attention maps for prediction interpretation could help experts see patterns that the human eye often misses. \par

In the field of variable stars, \citet{allam2021paying} employed the mechanism for classifying multivariate time series in variable stars. And additionally, \citet{allam2021paying} showed that the activation weights are accommodated according to the variation in brightness of the star, achieving a more interpretable model. And finally, related to the TESS telescope, \citet{morvan2022don} proposed a model that removes the noise from the light curves through the distribution of attention weights. \citet{morvan2022don} showed that the use of the attention mechanism is excellent for removing noise and outliers in time series datasets compared with other approaches. In addition, the use of attention maps allowed them to show the representations learned from the model. \par

Recent attention mechanism approaches in astronomy demonstrate comparable results with earlier approaches, such as CNNs. At the same time, they offer interpretability of their results, which allows a post-prediction analysis. \par



\section{Method}
\label{sec:Method}

% 现在的逻辑太差了,太散了,读下来割裂感很强。
% 1. notations and problem definition -- 给出问题定义,引出semantic overlap (考虑是否应当从method section中拆分)
% 2.method overview -- 承上起下,先直接给出我们对1中问题的解决方法,然后简单描述方法pipeline,统领后续两段的具体方法表述 (通过紧凑的语言给出的我们的solution)
% 3.Method Part 1 -- 
% 4.Method Part 2 -- 

% Methods 5.21
% 1.Preliminary 
% 2.Methods 
% 

% 首先是问题的形式化定义
% In this section, we first present the necessary notation and question definition and our analysis of the problem \textit{semantic drift} in ~\cref{sec3-1:preliminary}. Then, we introduce our proposed IPSeg with details in ~\cref{sec3-3} and ~\cref{sec3-4}. 
In this section, we begin by presenting the necessary notation and definition of the problem, followed by our analysis of \textit{semantic drift} in Section~\ref{sec3-1:preliminary}. Next, we introduce our proposed method, IPSeg, with detailed designs including image posterior and semantics decoupling in Section~\ref{sec3-3} and Section~\ref{sec3-4}.


% In this section, we first present an intuitive analysis of class incremental learning and give the definition of the semantic overlap phenomenon. Then, we present our solution in detail and claim its effectiveness in this challenge.

\subsection{Preliminary}
\label{sec3-1:preliminary}

% 1. 给出统一的符号/概念定义 
% 2. 给出具体的问题描述 
% 重点 -- 为什么现有范式有问题?为什么Image Posterior对这个问题有效?

\paragraph{Notation and problem formulation} Following previous works~\citep{SSUL_cha2021ssul,microseg_zhang2022mining,coinseg_zhang2023coinseg}, 
in CISS, a model needs to learn the target classes $\mathcal{C}_{1:T}$ from a series of incremental tasks as $t=1,2,3,...,T$. For task $t$, the model learns from a unique training dataset $\mathcal{D}_t$ which consists of training data and ground truth pairs $\mathcal{D}_t = \{(x_{i}^t, y_{i}^t)\}_{i=1}^{\left|\mathcal{D}_t\right|}$. Here $i$ denotes the sample index, $t$ for the task index, and $\left|\mathcal{D}_t\right|$ for the training dataset scale. 
$x_{i,j}^t$ and $y_{i, j}^t$ denote the $j$-th pixels and the annotation in the image $x_i^t$.
In each incremental phase $t$, the model can only access the class set $\mathcal{C}_t \cup c_b$ where $\mathcal{C}_t$ denotes the class set of current task $t$ and $c_b$ for background class. 
% After completing the $t$-th $(t>1)$ tasks, the model $f_t$ is expected to be able to predict pixels with ever-seen label set, $\mathcal{C}_{1:t} = c_b \cup \mathcal{C}_{1:t-1} \cup \mathcal{C}_{t}$.
% Specifically, we denote $x_{i}^t$ as an image belonging to task $t$ while $x$ represents an image without any task assignments. 

To prevent catastrophic forgetting, architecture-based methods allocate and optimize distinct sets of parameters for each class, instead of directly updating the whole model $f_t$. Typically, $f_t$ is composed of a frozen backbone $h_\theta$ and a series of learnable task heads $\phi_{1:t}$, with one task head corresponding to a specific task. 
% The model only needs to optimize the parameters of the newly added task head $\phi_t$ in task \(t\). 
In task $t$, only the new task head $\phi_t$ is set to be optimized.
In inference, the prediction for the $j$-th pixel in image $x_i$ can be obtained by:
\begin{equation}
    \vspace{-10pt}
    \hat{y}_{i,j} = f_{t}(x_{i,j}) = \mathop{\arg\max}\limits_{c\in\mathcal{C}_{1:T}} \phi_{1:T}^c(h_\theta(x_{i,j})).
    % \vspace{-10pt}
\end{equation}

Where \(\phi_{1:T}^c(\cdot)\) denotes the $C$-dimension outputs. 
% In addition to the notations mentioned above, 
Additionally, we introduce the image-level labels $\mathcal{Y}_i$ of the image $x_i$, a memory buffer $\mathcal{M}$, and an extra image classification head $\psi$ in our implementation. A comprehensive list and explanation of symbols can be found in the appendix.
% image-level labels, 
% image classification heads \psi


% \textbf{Semantic Drift} 
% is a problem that refers to the gradual change of the semantic content of the background class as the learning foreground classes change, according to the definition in previous works~\citep{MiB_cermelli2020modeling, PLOP_douillard2021plop}. So it is also treated as the same concept as \textit{background drift}.
% Previous studies ~\citep{SSUL_cha2021ssul, coinseg_zhang2023coinseg} have attempted to mitigate \textit{semantic drift} by decoupling the background class \( c_b \) into several subclasses. However, it works with the limited effect but introduces more tricky noisy dummy labels $c_u$. It still leaves the problem of noisy knowledge coupling. Moreover, the inconsistent output scale caused by unaligned training and testing targets exacerbates \textit{semantic drift}. During training, the task heads \( \cup_{c \in \mathcal{C}_t}{\phi_t^c} \) can only access supervision from the current classes \( c_b \cup \mathcal{C}_t \), whereas the testing target requires the integrated incremental model \( \cup_{c \in \mathcal{C}_{1:T}}{\phi_t^c} \) to output predictions for the full class domain \( \mathcal{C}_{1:T} \). In incremental learning scenarios, it is impractical to jointly optimize models of different tasks \( \phi_{t_1} \) and \( \phi_{t_2} \) (\forall \( t_1 \neq t_2 \)) concurrently under unified supervision without compromising memory stability. Therefore, ensuring that  models from different phases \( \phi_{t_1} \) and \( \phi_{t_2} \) (\forall \( t_1 \neq t_2 \)) produce consistent output scales is challenging.

% Upon rethinking the problem of \textit{semantic drift}, we attribute it to two aspects \textbf{separate optimization} and \textbf{noisy concept}. \textbf{Separate optimization} refers to the absence of unified supervision to update all task heads and results in inconsistent output scales across different task heads. \textbf{Noisy concept} refers to the noisy and complex semantic contents of classes \( c_b \) and \( c_u \), which is also related to the lack of effective modeling to fulfill their potential.

% \iffalse

% Separate optimization refers to the absence of unified supervision for updating all task heads, leading to inconsistent output scales across different tasks. And noisy concepts refer to the noisy and complex semantic contents within classes \( c_b \). We propose to address these two problems in IPSeg.

% Different from commonly seen jointly training all classes, target classes are separately optimized and learned in different phases in class incremental learning. Without unified learning, the incremental learning models always 

% Moreover, the inconsistency in output scales across different task heads further exacerbates \textit{semantic drift}. During training, the task head \( \phi_t \) receives supervision exclusively from the current classes and is frozen to solve catastrophic forgetting as the incremental process continues. Consequently, \( \phi_t \) will classify unknown objectives as familiar classes simply because they are similar.
% 这里的表意也有问题,需要重写



% , making the model learning and optimization hard.
% Despite these efforts, \textit{semantic drift} remains unresolved as the semantic content of \( c_b \) is overly complex and noisy. Extracting partial concepts from the background region and designing only a single architecture for learning are far from sufficient. 
% lacking the corresponding learning and optimization, making the models failed to effectively learn from chaos and inaccurate labels.
% 这里也应该更具体的阐述这一方式的不足,比如同时学习模糊的类别导致的优化困难or...


\iffalse
Previous works~\citep{ewc_kirkpatrick2017overcoming} attempt to mitigate \textit{semantic drift} by decoupling the background class \( c_b \) into several subclasses. 
Specifically, the background class \( c_b \) is splited into \( c'_b \) and \( c_u \), representing the ``pure'' background and unknown classes respectively.
% Specifically, the background class \( c_b \) is splited into \( c'_b \), representing the ``pure'' background, and \( c_u \), representing unknown classes that complement to the current visible classes \( c_b' \cup \mathcal{C}_t \). 
The most advanced methods~\citep{microseg_zhang2022mining,SSUL_cha2021ssul} further subdivide the unknown classes \( c_u \) into past classes \( \mathcal{C}_{1:t-1} \) and dummy unknown classes \( c'_u \) using pseudo-label techniques.

Despite these efforts, \textit{semantic drift} remains unresolved as the semantic content of \( c_b \) is overly complex and noisy. Extracting partial concepts from the background region and designing only a single architecture for learning are far from sufficient. 
\fi

% Additionally, previous solutions~\citep{PLOP_douillard2021plop} that assign pseudo-label to unknown classes \( c_u \) show limited effectiveness in enhancing learning plasticity and memory stability. Other approaches based on unknown classes \( c_u \), such as further decoupling and weight transfer~\citep{SSUL_cha2021ssul}, also have limited improvement in preventing \textit{semantic drift}.


% Moreover, the inconsistent output scale among different task heads exacerbates \textit{semantic drift}. During training, the task head \( \phi_t \) only receives supervision from the current classes, whereas the testing target requires the incremental model \( f_t=\{h_\theta, \phi_{1:t}\} \) to output predictions for the full class domain \( \mathcal{C}_{1:T} \). In incremental learning scenarios, it is impractical to jointly optimize task heads \( \phi_{t_1} \) and \( \phi_{t_2} \) $(\forall t_1 \neq t_2 )$ for different tasks under unified supervision. Consequently, producing consistent output scales for different models \( f_{t_1} \) and \( f_{t_2} \) proves to be highly challenging.


% \fi
% A significant reason for \textit{semantic drift} is the absence of unified supervision to update all task heads, resulting in inconsistent output scales across different task heads. Additionally, the noisy and complex semantic contents of \( c_b \) and \( c_u \) lack effective modeling to fulfill their roles.

% It is common practice to train a task head \( \phi_t^{c_u} \) separately with weight transfer to improve the training stability of CISS, but this does not effectively prevent \textit{semantic drift}.

% Previous works~\citep{SSUL_cha2021ssul,coinseg_zhang2023coinseg} try to alleviate \textit{semantic drift} by decoupling the single background class $c_b$ into several classes. In detail, the background class $c_b$ is decoupled into multiple classes $c'_b$ and $c_u$ where $c'_b$ stands for the ``pure'' background and $c_u$ for unknown classes complementary to the currently visible classes $c_b \cup \mathcal{C}_t$. Meanwhile, the current best method further divides unknown classes $c_u$ into the past classes $\mathcal{C}_{1:t-1}$ and dummy unknown classes $c'_u$. 

% Though many attempts at this orientation, we find \textit{semantic drift} have not been solved well as shown in ~\cref{fig:vis_intro}. Based on our empirical conclusion, there remain several challenges. The most notable one is that the semantic content of $c_u$ is too heavy and noisy to learn. In other words, \textit{semantic drift} is transferred from $c_b$ to $c_u$. 
% Besides this, previous solutions assign pixels with unknown classes $c_u$, but with limited effect on learning plasticity and memory stability. It is typical to separately train a task head $\phi_t^{c_u}$ in each with weight transfer to improve the training stability of CISS but can not help prevent \textit{semantic drift}. 

% Besides, inconsistent output scale caused by unaligned training and testing targets also deteriorates the \textit{semantic drift}. The training target constrains the current task heads $\cup_{c\in\mathcal{C}_t}{\phi_t^c}$ output correct predictions on the current classes $c_b \cup \mathcal{C}_t$. While the testing target constrains the whole model $\cup_{c\in\mathcal{C}_{1:T}}{\phi_t^c}$ output correct predictions on the full class domain $\mathcal{C}_{1:T}$. In incremental learning scenarios, it is not feasible to jointly optimize different $\phi_{t_1}$ and $\phi_{t_2}, t_1 \neq t_2$ at the same current by the unified supervision without changing their memory stability. Thus, it is hard to ensure that different models $\phi_{t_1}$ and $\phi_{t_2}, t_1 \neq t_2$ have consistent output scale.

% Rethinking the problem of \textit{semantic drift}, we summarize it as a form of \textit{catastrophic forgetting}. An important reason for \textit{semantic drift} is the lack of unified supervision to update all task heads which leads to inconsistent output scales of different task heads. On the other hand, the noisy and heavy semantic contents of $c_b$ and $c_u$ lack effective modeling to play their roles.

% \footnote{The side effect of it is also discussed in our Appendix.}


% To further mitigate semantic drift, the training supervision setting in CISS needs to be re-considered. In the incremental task $t$, the current training supervision classes are $\mathcal{C}_{t} \cup c_b$, where $c_b$ only denotes the pure background class. Previous works~\citep{} decouples the single background class $c_b$ into multiple classes $c_b, c_u$ where $c_u$ denotes the unseen classes compared to the $\mathcal{C}_{t} \cup c_b$. And ~\citep{} further adds the previous classes $\mathcal{C}_{1:t-1}$ into the current training supervision by pseudo labels from $\cup_{c\in\mathcal{C}_{1:t-1}}{\psi_c}$. Thus, the heavy semantic class $c_b$ is decomposed into multiple semantic expressions, the classes from the past and the classes not ever seen. In this pixel label assignment mechanism, the training supervision of task $t$ is $\{c_b \cup \mathcal{C}_{1:t}\cup \mathcal{C}_{1:t}\}$. 

\paragraph{Semantic Drift}  
Previous work~\citep{ewc_kirkpatrick2017overcoming} mainly attributes the \textit{semantic drift} to \textit{noisy semantics} within the background class $c_b$. They attempt to mitigate this challenge by decoupling the class \( c_b \) into subclasses $c'_b$ and $c_u$, where $c'_b$ denotes the pure background and $c_u$ denotes the unknown class. The most advanced methods~\citep{microseg_zhang2022mining,SSUL_cha2021ssul} further decouple the unknown classes \( c_u \) into past seen classes \( \mathcal{C}_{1:t-1} \) and dummy unknown class \( c'_u \) using pseudo labeling. 
However, \textit{semantic drift} remains unresolved as the decoupled classes are still evolving across incremental phases while the coupled training strategy is not able to cope with noisy pseudo labels.
% the decoupled classes are still changing across incremental phases and models are always hard to learn these chaotic classes.

Additionally, another essential challenge, \textit{separate optimization} inherent within incremental learning also contributes to \textit{semantic drift} but attracts little attention. Recent work~\citep{eclipse_kim2024eclipse} finds a similar phenomenon that freezing parameters from the old stage can preserve the model's prior knowledge but introduces error propagation and confusion between similar classes. 
In architecture-based methods, the task head \( \phi_t \) is exclusively trained by supervision from the current classes and will be frozen to resist catastrophic forgetting in the following incremental phases. In the following task $t_1, t_1 > t$, \( \phi_t \) may predict high scores on objects from other appearance-similar classes, without any penalty and optimization. 
In this incremental learning manner, task heads trained in different stages always have misaligned probability scales, and generate error predictions, especially on similar classes.
% Meanwhile, the new task head $\phi_{t_1}$ just predicts moderate scores which might be slightly lower than error predictions from \( \phi_t \).
% In this way, it is common that earlier incremental task heads may have larger output scales than the later heads, especially in similar classes.
This \textit{separate optimization} manner ultimately causes the incremental models to misclassify some categories and makes \textit{semantic drift} more difficult to thoroughly address. In the appendix, some cases can be found to help understand this challenge.
% And newer task heads have to predict higher scores. The output scales are inconsistent across different task heads and predictions are always wrong in similar classes, making \textit{semantic drift} harder to tackle.

% Upon rethinking the two factors of \textit{semantic drift} problem, we propose our method, IPSeg.

\begin{figure*}[t]
    \centering
    \includegraphics[width=0.9\textwidth]{figs/iclr_rebuttal_pdf_yuxiao/pipeline.pdf}
    \caption{Overall architecture of our proposed IPSeg, mainly composed of image posterior and permanent-temporary semantics decoupling two parts. In the latter part, $\phi_p$ denotes the permanent learning branch and $\phi_1, \phi_2, ..., \phi_t$ for temporary ones. The black solid lines are used to indicate the data flow in training and the green ones are for inference.}
    \label{fig:overview}
\end{figure*}


\subsection{Overview}
\label{sec3-2:overview}

As illustrated in Figure~\ref{fig:overview}, we propose \textbf{I}mage \textbf{P}osterior and Semantics Decoupling for Class-incremental Semantic \textbf{Seg}mentation (IPSeg) to mitigate \textit{semantic drift} through two main strategies: image posterior guidance and permanent-temporary semantics decoupling. In Section~\ref{sec3-3}, we describe how the IPSeg model uses image posterior guidance to mitigate \textit{separate optimization}. 
To address \textit{noisy semantics}, IPSeg employs branches with different learning cycles to decouple the learning of noisy semantics. Detailed explanations of this approach are provided in Section~\ref{sec3-4}.
% 表达啰嗦不干练

% Based on our analysis in ~\cref{sec3-1:preliminary}, we propose our method, \textbf{I}mage \textbf{P}osterior for Class-incremental Semantic \textbf{Seg}mentation (IPSeg) to solve \textit{semantic drift} from two aspects, image posterior guidance and permanent and temporary knowledge decoupling. 
% In ~\cref{sec3-3}, IPSeg models image posterior to compensate for the insufficient \textbf{separate training} by the design of building an extra class incremental classifier. To mitigate the side effects of heavy and noisy concepts $c_b$ and $c_u$,  IPSeg adopts two branches to decouple permanent and temporary knowledge and mitigate the noisy effect with details in ~\cref{sec3-4}.
% IPSeg solves pixel semantic drift from two unique designs, image posterior and unknown object mining. The image posterior method directly adjusts pixel-level class-wise predictions from different heads.
% The unknown object mining method is designed to dig knowledge from the current unknown class objects by concept decoupling. We will introduce them in the following subsections in detail.
% \fy{Or we write this overview by overall introducing our method.}



\subsection{Image Posterior Guidance}
\label{sec3-3}

% As previously discussed, the most straightforward solution to solve separate optimization is to rescale the inconsistent outputs using global statistical values. 
% 这里的表达不太好,前文没有做铺垫
As previously discussed, the \textit{separate optimization} leads to misaligned probability scales across different incremental task heads and error predictions. 
We propose leveraging the image-level posterior as the global guidance to correct the probability distributions of different task heads. The rationale for using the image posterior probabilities is based on the following fact:

\textbf{Fact}: \textit{For any image, if its image-level class domain is \(\mathcal{C}_I\) and its pixel-level class domain is \(\mathcal{C}_P\), the class domains \(\mathcal{C}_I\) and \(\mathcal{C}_P\) are the same, i.e., \(\mathcal{C}_I = \mathcal{C}_P\).}

Inspired by this fact, we propose to use an extra image posterior branch $\psi$ to predict image classification labels and train it in an incremental learning manner. As illustrated in Figure~\ref{fig:overview}, $\psi$ is composed of Pooling, Fully connected (FC) layers, and Multi-Layer Perceptrons (with one MLP per step) with the input dimension of 4096 and the output dimension of \(\left|\mathcal{C}_{1:T}\right|\), where the FC layers serve as shared intermediate feature processors, and the MLPs serve as incremental classification heads for incremental classes. 
% During inference, this branch predicts the image-level posterior probabilities on the target class set \(\mathcal{C}_{1:T}\). 

% This image posterior prediction branch is implemented using simple MLPs with the input dimension of 4096 and the output dimension of \(\left|\mathcal{C}_{1:t}\right|\).


% Data sample \( x_i^m \) from \(\mathcal{M}\) is associated with its corresponding image-level ground truth \(\mathcal{Y}_i^m\), where the classes in \(\mathcal{Y}_i^m\) depend on specific tasks. For example, \(\mathcal{Y}_i^m \subset \mathcal{C}_{t_1}\) if \((x_i^m, y_i^m) \in \mathcal{D}_{t_1}\). Similarly, data sample \( x_i^t \) from \(\mathcal{D}_t\) has image-level labels \(\mathcal{Y}_i^t\) from the current task class domain \(\mathcal{C}_{t}\).
% During incremental training for task \( t \),

In task \( t \) (\( t > 1 \)), the model can only access data \( x^{m}_i \) from the memory buffer \(\mathcal{M}\) and \( x^{t}_i \) from the current training dataset \(\mathcal{D}_t\). Previous works~\cite{SSUL_cha2021ssul, coinseg_zhang2023coinseg} put \( x^{m}_i \) into the training phase to revisit and reinforce prior knowledge of segmentation by simply rehearsal. IPSeg further takes advantage of the rich class distribution knowledge in \( x^{m}_i \) to train and enhance the image posterior branch.
% However, we observe that \( x^{m}_i \) contains rich category information, which can be leveraged not only to guide pixel-wise predictions but also to effectively train image-level tasks.
% 这个符号可能会引起歧义
% 这个表达也很差
% are associated with their corresponding image-level ground truth \(\tilde{\mathcal{Y}}^{t}_i\) and

In IPSeg, the mixed data samples \( x^{m,t}_i \) from  \(\mathcal{M}\) and \(\mathcal{D}_t\) are processed by the network backbone \( h_{\theta} \) into the image feature \( h_{\theta}(x^{m,t}_i) \), and further processed by image posterior branch \(\psi\) into the image classification prediction \(\hat{\mathcal{Y}}_i^{m,t}\). The objective function for training $\psi$ is: 
\vspace{-5pt}
\begin{equation}
    % \vspace{-10pt}
    \begin{aligned}
        \mathcal{L}_{\text{\tiny IP}} &= \mathcal{L}_{\text{\tiny BCE}}(\hat{\mathcal{Y}}^{m,t}_i, \tilde{\mathcal{Y}}^{m,t}_i) = \mathcal{L}_{\text{\tiny BCE}}(\psi(h_\theta(x^{m,t}_i)), \tilde{\mathcal{Y}}^{m,t}_i), \\
        \tilde{\mathcal{Y}}^{m,t}_i &= \mathcal{Y}^{m,t}_i \cup \tilde{\mathcal{Y}}_{\phi_{1:t-1}(h_\theta(x^{m,t}_i))}.
    \end{aligned}
    % \vspace{-10pt}
\end{equation}
Where image classification label \(\tilde{\mathcal{Y}}^{m,t}_i\) consists of two parts, the ground truth label \(\mathcal{Y}^{m,t}_i\) of the data \( x^{m,t}_i \) and pseudo label \(\tilde{\mathcal{Y}}_{\phi_{1:t-1}(h_\theta(x^{m,t}_i))}\) on past seen classes $\mathcal{C}_{1:t-1}$. 
Instead of relying solely on the label \(\mathcal{Y}^{m,t}_i\), we use the image-level pseudo labels from previous task heads prediction to enhance the model's discriminative ability on prior classes.

% 后面这一项太复杂了,增加了阅读难度
% The entire \(\tilde{\mathcal{Y}}^{t}_i\) is processed into pseudo-label to train image posterior branch. 
% Instead of relying solely on the label \(\mathcal{Y}^{t}_i\), we use the pseudo-label \(\tilde{\mathcal{Y}}^{t}_i\) to provide more comprehensive and diverse supervision. 
% ..., we also use the image-level pseudo labels from the previous task heads prediction, to provide informative signals of previous incremental tasks.

During inference, the image posterior branch predicts posterior probabilities on all classes \(\mathcal{C}_{1:T}\). For a testing image \( x_i \), the final pixel-wise scores are computed by element-wise multiplication between the image posterior probabilities from \(\psi\) and the pixel-wise probabilities from \(\phi_{0:T}\): 
% \begin{equation}
%     p_{i} = 
%     \underbrace{\texttt{Concat}(~~\alpha_{\text{\tiny BC}},~\sigma(~\psi(h_\theta(x_i))~)~~) }_{\text{Image Posterior Probability}}
%     ~~\cdot  ~~
%     \sigma (~\underbrace{\phi_{\yx{0}:T}(h_{\theta}(x_i))}_{\text{Pixel-wise Probability}}~).
%     \label{equ_3}
% \end{equation}
\begin{equation}
    p_{i} = 
    \underbrace{\texttt{Concat}(\alpha_{\text{\tiny BC}},\sigma(~\psi(h_\theta(x_i)))) }_{\text{Image Posterior Probability}}
    \cdot  
    \sigma (\underbrace{\phi_{0:T}(h_{\theta}(x_i))}_{\text{Pixel-wise Probability}}).
    \label{equ_3}
\end{equation}

Where \(\sigma(\cdot)\) denotes the Sigmoid function.
The hyperparameter \(\alpha_{\text{\tiny BC}}\) is used to compensate for the lack of background posterior probability, with the default value \(\alpha_{\text{\tiny BC}}=0.9\). 
The result \( p_i \) is the rectified pixel-wise prediction with a shape of \([C, HW]\), and \( p^c_{i,j} \) is prediction of the \( j \)-th pixel on class $c$. The prediction of the $j$-th pixel can be written as:
\begin{equation}
    \hat{y}_{i,j} = \mathop{\arg\max}\limits_{c\in\mathcal{C}_{1:t}} p_{i,j}^c.
    \label{equ_4}
\end{equation}

% The prediction \(\hat{y}_{i,j}\) can be written as:
% \begin{equation}
%     \hat{y}_{i,j} = \mathop{\arg\max}\limits_{c\in\mathcal{C}_{1:t}} p_{i,j}^c.
% \end{equation}
\subsection{Permanent-Temporary Semantics Decoupling}
\label{sec3-4}

To further address \textit{semantic drift} caused by the coupled learning of complex and noisy pseudo labels \( c_b \) and \( c_u \) along with incomplete yet accurate label \(\mathcal{C}_t\), we propose a decoupling strategy that segregates the learning process for different semantics. Here is our empirical observation:

\textbf{Observation}: \textit{Given an image in incremental task t, the semantic contents of it can be divided into four parts: past classes \(\mathcal{C}_{1:t-1}\), target classes \(\mathcal{C}_{t}\), unknown foreground \(c'_u\) and pure background \(c'_b\).}
% ..., its content can be divided into four semantics: ...

Based on this observation, we first introduce dummy label \(c_f=\mathcal{C}_{1:t-1} \cup c'_u\) to represent the foreground regions that encompass both past seen classes and unknown classes, which are not the primary targets in the current task. Subsequently, we decouple the regions of a training image into two sets: \( \mathcal{C}_t  \cup  c_f\) and \( c'_b \cup c'_u \). The former set  \( \mathcal{C}_t  \cup  c_f\) are current target classes and other foreground objects, which are temporary concepts belonging to specific incremental steps, and change drastically as the incremental steps progress. In contrast, \( c'_b \cup c'_u \) are pseudo labels representing pure background and unknown objects, which are permanent concepts, exist across the whole incremental steps and maintain stable (\( c'_b \) remains fixed, \( c'_u \) shrinks but does not disappear). 
% For instance, ``cat'' and ``horse'' are target classes in the current task but change into the region of past seen foreground classes in subsequent tasks, while the concepts of background and unknown objects remain consistent. For instance, given all target classes \{\(c'_b, \mathcal{C}_1, \mathcal{C}_2, \mathcal{C}_3, ..., \mathcal{C}_T\)\}, 

The learning of these two sets is also decoupled. The current task head \(\phi_t\) serves as the temporary branch to learn the semantics \( \mathcal{C}_t \) $\cup$ \( c_f \) existing in the current incremental phase. Besides, we introduce a permanent branch \(\phi_p\) to learn the permanent dummy semantics \( c'_b \) and \( c'_u \). \(\phi_p\) has the same network architecture as \(\phi_{t}\). They are composed of three 3x3 convolution layers and several upsampling layers. It's worth noting that $\phi_p$ and $\phi_t$  have different learning cycles as illustrated in Figure~\ref{fig:overview}. The permanent branch $\phi_p$ is trained and optimized across all incremental phases to distinguish unknown objects and the background. While temporary branch $\phi_t$ ($t=1,2,...,T$) is temporarily trained in the corresponding task phase $t$ to recognize target classes $\mathcal{C}_t$.
Following our decoupling strategy, we can reassign the labels of image \( x_i \) as:
% .... 
% Compared to the popular taxonomy in previous works, which is based on foreground classes, we group all classes into two groups: the static group and the dynamic group. In different incremental phases, the fine-grained class concepts of known foreground change—for example, the target class set may be ``cat, horse'' in the current task and changes into ``cow, sofa'' in the next. However, the concepts of background and unknown foreground remain constant.
% % However, the distinction between background and foreground remains constant. 
% % This static group includes the permanent concepts across different incremental phases that is the background and foreground.

% Based on this taxonomy, we decouple the regions of a given image into two groups: \( c'_b \cup c_f \) and \( \mathcal{C}_t  \cup  c'_{f}\). Here, \( c'_b \) and \( c_f \) are dummy labels representing ``pure'' background and foreground objects, respectively. While \(\mathcal{C}_t \) and \(c'_{f}\) are target foreground classes and other foreground objects in current task $t$. The group \( c'_b \cup c_f \) stands for permanent concepts across all incremental phases, while the group \( \mathcal{C}_t  \cup  c'_{f}\) includes detailed but temporary concepts that change with different phases.

% \begin{equation}
% \scriptsize
% \begin{aligned}
% \tilde{y}^p_{i} = \begin{cases}
% c_i, & \text{if} ~y^t_{i}\in\mathcal{C}_t \vee ((y^t_{i}=c_b)\wedge(f_{t-1}(x_i)\in\mathcal{C}_{1:t-1})) \\
% % c_i, & \text{if} ~(y^t_{i}=c_b)\wedge(f_{t-1}(x_i)\in\mathcal{C}_{1:t-1}) \\
% c'_u, & \text{if} ~(y^t_{i}=c_b)\wedge (f_{t-1}(x_i)\notin\mathcal{C}_{1:t-1})\wedge(S(x_i)=1) \\
% c^{'}_b, & \text{else,}
% \end{cases},
% \qquad
% \tilde{y}^t_{i} = \begin{cases}
% y^t_{i}, & \text{if} ~y^t_{i}\in\mathcal{C}_t \\
% c_f, & \text{if} ~(y^t_{i}=c_b)\wedge(S(x_i)=1) \\
% c^{'}_b, & \text{else,}
% \end{cases}
% \end{aligned}
% \normalsize
% \end{equation}
% \begin{equation}
% % \scriptsize
% \tilde{y}^p_{i} = \begin{cases}
% c_i, & \text{if} ~y^t_{i}\in\mathcal{C}_t \vee ((y^t_{i}=c_b)\wedge(f_{t-1}(x_i)\in\mathcal{C}_{1:t-1})) \\
% % c_i, & \text{if} ~(y^t_{i}=c_b)\wedge(f_{t-1}(x_i)\in\mathcal{C}_{1:t-1}) \\
% c'_u, & \text{if} ~(y^t_{i}=c_b)\wedge (f_{t-1}(x_i)\notin\mathcal{C}_{1:t-1})\wedge(S(x_i)=1) \\
% c^{'}_b, & \text{else,}
% \end{cases},
% \end{equation}

% \begin{equation}
% \tilde{y}^t_{i} = \begin{cases}
% y^t_{i}, & \text{if} ~y^t_{i}\in\mathcal{C}_t \\
% c_f, & \text{if} ~(y^t_{i}=c_b)\wedge(S(x_i)=1) \\
% c^{'}_b, & \text{else,}
% \end{cases}
% % \normalsize
% \end{equation}
\vspace{-10pt}
\begin{equation}
    \scriptsize
    \begin{aligned}
        \tilde{y}^p_{i} &= \begin{cases}
            c_i, & \text{if} ~y^t_{i}\in\mathcal{C}_t \vee \left( (y^t_{i}=c_b) \wedge \left( f_{t-1}(x_i)\in\mathcal{C}_{1:t-1} \right) \right) \\
            c'_u, & \text{if} ~\left( y^t_{i}=c_b \right) \wedge \left( f_{t-1}(x_i) \notin \mathcal{C}_{1:t-1} \right) \wedge \left( S(x_i)=1 \right) \\
            c'_b, & \text{else,}
        \end{cases}, \\
        \tilde{y}^t_{i} &= \begin{cases}
            y^t_{i}, & \text{if} ~y^t_{i}\in\mathcal{C}_t \\
            c_f, & \text{if} ~\left( y^t_{i}=c_b \right) \wedge \left( S(x_i)=1 \right) \\
            c'_b, & \text{else,}
        \end{cases}.
    \end{aligned}
    \normalsize
    \label{equ_label}
\end{equation}
\vspace{-10pt}

% \begin{equation}
% \scriptsize
% \begin{aligned}
% \tilde{y}^p_{i} = \begin{cases}
% c_{i}, & \text{if}  ~(y^t_{i}=c_b)\wedge(f_{t-1}(x_i)\in\mathcal{C}_{1:t-1} ~or~ ~y^t_{i}\in\mathcal{C}_t\\
% c'_u, & \text{if} ~(y^t_{i}=c_b)\wedge (f_{t-1}(x_i)\notin\mathcal{C}_{1:t-1})\wedge(S(x_i)=1) \\
% c^{'}_b, & \text{else,}
% \end{cases},
% \qquad
% \tilde{y}^t_{i} = \begin{cases}
% y^t_{i}, & \text{if} ~y^t_{i}\in\mathcal{C}_t \\
% c_f, & \text{if} ~(y^t_{i}=c_b)\wedge(S(x_i)=1) \\
% c^{'}_b, & \text{else,}
% \end{cases}
% \end{aligned}
% \normalsize
% \end{equation}
Where \(f_{t-1}(\cdot)\) is the model of task $t-1$ and \( S(\cdot) \) is the salient object detector as used in SSUL~\citep{SSUL_cha2021ssul}. \(\tilde{y}_{i}^{p}\) is the label used to train \(\phi_p\), and \(\tilde{y}_{i}^{t}\) is the label used to train \(\phi_t\) for the current task \( t \). \(c_i\) is the ignored region not included in the loss calculation. The visualization of semantics decoupling is provided in the appendix. 


The objective functions for these two branches is defined as:
\begin{equation}
    \begin{aligned}
    \mathcal{L}_{p} = \mathcal{L}_{\text{\tiny BCE}}(~\phi_p(h_\theta(x_i^t)), \tilde{y}_i^p ~), \\
    \mathcal{L}_{\text{\tiny current}} = \mathcal{L}_{\text{\tiny BCE}}(~\phi_t(h_\theta(x_i^t)), \tilde{y}_i^t~).
    \end{aligned}
\end{equation}

% It's worth noting that the learning lifecycle of permanent branch \(\phi_p\) spans all incremental tasks from $1$ to $T$, and \(\phi_p\) generates the logit of \( c'_b \) and \( c'_u \). While the temporary branch \(\phi_t\) updates merely during task $t$, which produces the logit of \( c'_b \), \( \mathcal{C}_t \) and \(c_f\), helping the model distinguish target classes from other foreground. 

% To better learn these two groups of concepts, we propose to train them with separate branches. Specifically, we introduce a permanent branch \(\phi_p\) to learn the permanent dummy classes \( c'_b \) and \( c'_u \). \(\phi_p\) has a simple network structure the same as the other parallel task heads. It consists of three 3x3 convolution layers and several upsampling layers. Furthermore, the existing incremental heads \(\phi_t\) are served as the temporary branch designed to learn the concepts \( \mathcal{C}_t \) and \( c_f \) existing in the current incremental phase. It's worth noting that the learning lifecycle of permanent branch \(\phi_p\) spans all incremental tasks from $1$ to $T$, and \(\phi_p\) generates the logit of \( c'_b \) and \( c'_u \). While the temporary branch \(\phi_t\) updates merely during task $t$, which produces the logit of \( c'_b \), \( \mathcal{C}_t \) and \(c_f\), helping the model distinguish target classes from other foreground. 
% 最后关于permanent branch 和 temporary branch 的叙述不好。


% Where \(\{\tilde{y}_i^p\}_{\mathcal{C} \in \{ c'_u, c'_b \}}\) indicates the annotations of \(c'_u\) and \(c'_b\) within \(\tilde{y}_i^p\). Since \(y^t_i\) and \(f_{t-1}(x_i)\) are merely designed to obtain the regions corresponding to \(c'_u\) and \(c'_b\), they do not participate in the loss calculation. 
\vspace{-10pt}
Finally, the total optimization objective function is:
\begin{equation}
    \mathcal{L}_{total}=\mathcal{L}_{\text{\tiny IP}}+\lambda_{1}\mathcal{L}_{\text{\tiny current}}+\lambda_{2}\mathcal{L}_{p},
\end{equation}
where \(\lambda_1\) and \(\lambda_2\) are trade-off hyperparameters to balance different training objective functions.

During inference, as illustrated by the green lines in Figure 2, the permanent branch $\phi_p$ predicts on the background $c_b'$ and unknown objects $c_u'$, with only $c_b'$ used for inference. Meanwhile, the temporary branch $\phi_t$ ($t=1,2,...,T$) predicts for the target classes $\mathcal{C}_t$, the foreground region \( c_f \) and the background $c_b'$, where $\mathcal{C}_t$ and \( c_f \) are used for inference. The pixel-level prediction $\phi_{0:T}(h_\theta(x_i))$ is formulated as:
\begin{equation}
\phi_{0:T}(h_{\theta}(x_i)) = \texttt{Concat}( ~\phi_{p}(h_\theta(x_i))~,~\phi_{1:T}(h_\theta(x_i))).
\end{equation} 
Where $\phi_{p}(h_\theta(x_i))$ and $\phi_{1:T}(h_\theta(x_i))$ represent background prediction from permanent branch and the aggregated foreground predictions from all temporary branches. The pixel-level prediction is then producted by image posterior probability to form the final prediction maps as Eq~\ref{equ_3} and Eq~\ref{equ_4}.


Furthermore, to mitigate the issue of inaccurate predictions on other foreground classes \(c_f\) within each task head $\phi_t$ during inference, we introduce a Noise Filtering trick, filtering out prediction errors associated with \(c_f\). The prediction for the \( j \)-th pixel \(\hat{y}_{i,j}\) is processed as:
\begin{equation}
\hat{y}_{i,j}=\begin{cases}\alpha_{\text{\tiny NF}} \cdot \hat{y}_{i,j}&\text{if} ~max(~p^f_{i,j},~p^c_{i,j}~)=p^f_{i,j}\\\hat{y}_{i,j}&\text{if}~max(~p^f_{i,j},~p^c_{i,j}~)=p^c_{i,j}\end{cases}
\end{equation}
Where \(\alpha_{\text{\tiny NF}}\) is noise filtering term with the default value \(\alpha_{\text{\tiny NF}}=0.4\). And \(p^f_{i,j}\) and \(p^c_{i,j}\) are the \( j \)-th pixel logit outputs on the foreground \(c_f\) and target class $\mathcal{C}_t$ respectively.
% As we analyze the class incremental learning challenge in ~\cref{sec3-1:preliminary}, the image-level posterior probability $P(x \in \mathcal{X}_c \mid \mathcal{D}, \theta)$ plays a strong role to discriminate the pixel-level class prediction of class $c$, $P(x_i \in \mathcal{X}_c \mid \mathcal{D}, \theta)$ for any pixel $x_i$ in this image. This can also be explained by the following physical fact.

% As we claim in above, we faced the challenge of inconsistent output scales due to separate optimization, the best way to solve this problem is to re-scale them with global constraint factors.
\iffalse
As we claim above, we face the challenge of inconsistent output scales due to individual optimizations. The best and most straightforward solution to this problem is to rescale them with global statistical values. In this way, we propose to use the image-level posterior as this global guidance. The reason for choosing the image posterior is based on the following fact.

\textbf{Fact 1}: \textit{For any image, if its image-level class domain is $\mathcal{C}_I$ and its pixel-level class domain is $\mathcal{C}_P$, then the class domains $\mathcal{C}_I$ and $\mathcal{C}_P$ is the same, that is, $\mathcal{C}_I = \mathcal{C}_P$.}

% This fact tells us that we can constrain the model only to predict the score of classes in $\mathcal{C}_x$ through post-filtering. However, in practical CISS scenarios, image-level labels can not be directly obtained off-the-shell. 

Inspired by the above fact, we propose to build an extra prediction branch for predicting image classification labels and train it in an incremental learning manner. In testing, it predicts image-level posterior from the whole class set $\mathcal{C}_{1:T}$. In Detail, This image posterior prediction branch is implemented by a simple MLP with input dim of $4096$ and output dim of $\left|\mathcal{C}_{1:T}\right|$. 

In task $t, t > 1$, the model can get access to data from both the memory buffer $\mathcal{M}$ and training data $\mathcal{D}_t$ from the current task $t$. 
Data of the memory buffer $x_j^m$ is with the corresponding ground truth $\mathcal{Y}_j^{m}$. The class domain of $\mathcal{Y}_j^{m}$ depends on the specific task it belongs to, like $\mathcal{Y}_j^{m} \subset \mathcal{C}_{t_1}$, if $(x_j^m, y_j^m) \in \mathcal{D}_{t_1}$. 
While the image $x_i^t$ of the current task is with image-level labels $\mathcal{Y}_i^t$ from the current task class domain $\mathcal{C}_{t}$. 
During the incremental training task $t$, a data sample $x_k$ from these two sources is processed by network backbone $h_{\theta}$ into image feature $h_{\theta}(x_k)$. Then feature $h_{\theta}(x_k)$ is processed by \texttt{Adaptive Average Pooling} and \texttt{MLP} to get the final prediction $\hat{\mathcal{Y}}_k^t$, which is supervised by
% Feature $\psi(x_k)$ is processed by \texttt{Adaptive Average Pooling} and \texttt{Flatten} into a 4096-dim tensor and then mapped into a $\left|\mathcal{C}_{1:t}\right|$-dim prediction $\hat{y}_k$ by image posterior branch. Finally, the training objective function of the image posterior is 
\begin{equation}
    \mathcal{L}_{\text{\tiny IP}} = \mathcal{L}_{\text{\tiny BCE}}(\hat{\mathcal{Y}}_k, \tilde{\mathcal{Y}}_k) = \mathcal{L}_{\text{\tiny BCE}}(\psi(h_\theta(x_k)), \tilde{\mathcal{Y}}_k), ~~\tilde{\mathcal{Y}}_k = (\cup_{c\in\mathcal{C}_{1:t-1}}\phi^c(h_\theta(x_k))))\cup \mathcal{Y}_k
\end{equation}
% where $\hat{y}_k$ is the prediction of the image posterior branch on $\left|\mathcal{C}_{1:t}\right|$ classes and $y_k$ is the corresponding image-level ground-truth in one-hot form, originated from the pixel-wise annotation of task $t$.

Instead of directly using partial label $\mathcal{Y}_k$, pseudo label $\tilde{\mathcal{Y}}_k$ is preferred for rich knowledge behind it. 

In the testing, the image posterior branch directly predicts posterior probabilities on all classes $\mathcal{C}_{1:T}$. Given a test image $x_i$, its final prediction scores can be obtained by element-wise multiplying between image posterior probabilities from $\psi$ and pixel-wise predictions from $\cup_{c\in\mathcal{C}_{1:T}}\phi^c$, 
\begin{equation}
    p_{i} = \texttt{Concat}(\alpha_{\text{\tiny BR}} \cdot \phi^{c_b}(h_\theta(x_{i})), \sigma(\psi(h_\theta(x_i))) \cdot \sigma(\texttt{Concat}(\cup_{c\in\mathcal{C}_{1:T}}\phi^{c}(h_\theta(x_{i}))))).
\end{equation}
$\alpha_{\text{\tiny BR}}$ is a hyper-parameter used to compensate for the background class due to its absence in posterior probabilities. The result $p_i$ is a tensor of the shape $[C, HW]$, where $p_{i,j}$ denotes the $C$-dim prediction results on the $j$-th pixel. And $p_{i,j}^c$ is the score of the $c$-th class. The prediction of the $j$-th pixel is
\begin{equation}
    \hat{y}_{i,j} = \mathop{\arg\max}\limits_{c\in\mathcal{C}_{1:t}} p_{i,j}^c.
\end{equation}
\fi
% In the practical implementation, we adopt an improved implementation by utilizing already existing class-wise knowledge from the trained $\psi_{c_i}, c_i \in \mathcal{C}_{1:t-1}$. Instead of directly using image level classes from $\mathcal{C}_{t_1}$ for a memory sample of task $t-1$, IPSeg extends its class scope to  $\mathcal{C}_{1:t}$ by merging prediction from all segmentation heads.
% \fy{using formulation to better express.}

\iffalse
To further mitigate \textit{semantic drift} due to the mixture of noisy dummy labels $c_b, c_u$ and correct but partial labels $\mathcal{C}_t$, we propose to further decouple their learning and optimization from the semantic concepts. 

Compared to this popular taxonomy in previous works which is based on foreground classes, we group all classes into two groups: the static group and the dynamic group. In different incremental phases, the fine-grained class concepts of foreground objects change from phase to phase, for example, the foreground class set is ``cat, horse'' in the current task, and it could rapidly change into ``cow, sofa'' in the next. However, compared to these fine-grained class concepts, the meaning of background and foreground never changes. This is the static concepts group across different incremental phases. 

Based on this taxonomy, we decouple all classes into two groups: $c'_b \cup c_f$ and $c_b \cup \mathcal{C}_t$. $c'_b$ and $c_f$ are dummy labels represent for ``pure'' background and foreground objects respectively. The group $c'_b \cup c_f$ stands for the permanent concepts across all incremental phases. The group $c_b \cup \mathcal{C}_t$ are detailed but temporary concepts, changing with different phases. 

To better learn the two different groups of concepts, we also propose to learn them with different branches. Specifically, we propose to build a permanent learning head $\phi_p$ to learn the permanent classes dummy classes $c_b$ and $c_u$. In addition, the existing incremental heads $\cup_{c\in\mathcal{C}_t}\{\phi_t^c\}$ are designed to learn from the temporary concepts $c_b \cup \mathcal{C}_t$.

Based on our decoupling rules, we also re-assign the labels of image $x_i$ as 
\begin{equation}
    \left\{
    \begin{array}{l}
    \tilde{y}_{i}^{p} = y_i^t \vee (\neg y_i^t \wedge S(x_i))     &  \\
    \tilde{y}_{i}^{t} = y_i^t \vee (\neg y_i^t \wedge \phi^{t-1}(x_i)) \vee (\neg y_i^t \wedge \neg \phi^{t-1}(x_i) \wedge S(x_i))     & 
    \end{array}
    \right.
\end{equation}
where $S(\cdot)$ is the salient object detector as the same as in~\citep{}. $\tilde{y}_{i}^{p}$ is the labels assigned to the learning of permanent branch $\phi_p$. And $\tilde{y}_{i}^{t}$ is the labels assigned to the incremental learning heads $\cup_{c\in\mathcal{C}_t}\{\phi_t^c\}$ of current task $t$. Note that the proposed label re-assignment mechanism only works for this decoupling.

Based on the above pixel-level label re-assignment, the optimization for these two branches are 
\begin{equation}
    \mathcal{L}_{\text{\tiny current}} = \mathcal{L}_{\text{\tiny BCE}}(\texttt{Concat}(\cup_{c \in \mathcal{C}_{inner}} \phi_c^t(x_i^t)), \tilde{y}_i^t).
\end{equation}
\begin{equation}
    \mathcal{L}_{p} = \mathcal{L}_{\text{\tiny BCE}}(\phi_p(x_i^t), \check{y}_i^p).
\end{equation}

Finally, the total optimization objective function of IPSeg is 
\begin{equation}
    \mathcal{L}_{total}=\mathcal{L}_{\text{\tiny IP}}+\lambda_{1}\mathcal{L}_{\text{\tiny current}}+\lambda_{2}\mathcal{L}_{p},
\end{equation}
where $\lambda_1$ and $\lambda_2$ are super-parameters for trade-off between different supervision.
\fi


% $\cup_{c\in\mathcal{C}_{1:t}}\phi_t^c$. 

% Different from previous works training mixed and noisy classes $\{c_b \cup c_u \cup \mathcal{C}_t\}$ all together, IPSeg decouples these classes into two clusters, noisy and changing classes $c'_b \cup c_u$ and relatively clean classes $c_b \cup \mathcal{C}_t$.

% The reason for this taxonomy is that the concepts of background and foreground are relatively cleaner than fine-grained class concepts, e.g. cow, horse, bus, and et.al.

% IPSeg considers maintaining a long-term task branch to learn the unchanging class concepts of background and foreground which is not changed across tasks and task class scopes. $\phi_{o}$.


% in addition to adding new task heads to gradually learn different inter-task classes. 
% In this way, the learning labels for the inter-task branch are $\tilde{y}_{i}^{t} = y_i^t \vee (\neg y_i^t \wedge S(x_i))$ and labels for the outer-task branch $\psi_u$ are $\check{y}_{i}^t = y_i^t \vee (\neg y_i^t \wedge \psi^{t-1}(x_i)) \vee (\neg y_i^t \wedge \neg \psi^{t-1}(x_i) \wedge S(x_i))$.

% via an individual branch instead of mixed training together with foreground classes.

% the training supervision setting in CISS needs to be re-considered. In the incremental task $t$, the current training supervision classes are $\mathcal{C}_{t} \cup c_b$, where $c_b$ only denotes the pure background class. Previous works~\citep{} decouples the single background class $c_b$ into multiple classes $c_b, c_u$ where $c_u$ denotes the unseen classes compared to the $\mathcal{C}_{t} \cup c_b$. And ~\citep{} further adds the previous classes $\mathcal{C}_{1:t-1}$ into the current training supervision by pseudo labels from $\cup_{c\in\mathcal{C}_{1:t-1}}{\psi_c}$. Thus, the heavy semantic class $c_b$ is decomposed into multiple semantic expressions, the classes from the past and the classes not ever seen. In this pixel label assignment mechanism, the training supervision of task $t$ is $\{c_b \cup \mathcal{C}_{1:t}\cup \mathcal{C}_{1:t}\}$. 

\iffalse
However, the task heads $\psi^t = \{\cup_{c\in\mathcal{C}_t}\psi_c^t, \psi_{c_b}^t, \psi_{c_u}^t\}$ are not well organized to align with these classes. The classes $c_u$ and $\mathcal{C}_{1:t-1}$ are mixed and learned by a single task head $\psi_{c_u}$, leading to heavy semantic expression. On the other hand, noisy labels always lead to unstable optimization, and make the inconsistent output problems more severe due to coupled classes candidates in $c_u$.

Based on this discovery, we propose a decoupled class assignment mechanism that decouples the noisy and heavy label assignment from both global (inter-task) and current (inner-task) perspectives. 

In contrast to assigning a single image with messy class labels and proposing together, IPSeg proposes to separate them into the current object classes $\mathcal{C}_t \cup c_b$ and classes not belonging to current classes $\mathcal{C}_{1:t} \cup c_u$. For brevity, we name them inner-task classes $\mathcal{C}_{inner}$ and outer-task classes $\mathcal{C}_{outer}$. 

\fy{There is a problem that task prediction heads might be ambiguous referring to previous works or IPSeg.}

The learning of these two class scopes is also decoupled. Different from training a set of prediction head $\{\cup_{c\in\mathcal{C}_t}\psi_c^t, \psi_{c_b}^t, \psi_{c_u}^t\}$ using all noisy labels in a phase $k$, IPSeg considers maintain a long-term task branch to learn the changing outer-task classes across tasks in addition to adding new task heads to gradually learn different inter-task classes. In this way, the learning labels for the inter-task branch are $\tilde{y}_{i}^{t} = y_i^t \vee (\neg y_i^t \wedge S(x_i))$ and labels for the outer-task branch $\psi_u$ are $\check{y}_{i}^t = y_i^t \vee (\neg y_i^t \wedge \psi^{t-1}(x_i)) \vee (\neg y_i^t \wedge \neg \psi^{t-1}(x_i) \wedge S(x_i))$. \fy{The reasoning for this design is that.}

Based on the above pixel-level class label assignment, the optimization for these two branches are 
\begin{equation}
    \mathcal{L}_{\text{\tiny inner}} = \mathcal{L}_{\text{\tiny BCE}}(\texttt{Concat}(\cup_{c \in \mathcal{C}_{inner}} \psi_c^t(x_i^t)), \tilde{y}_i^t).
\end{equation}
\begin{equation}
    \mathcal{L}_{\text{\tiny outer}} = \mathcal{L}_{\text{\tiny BCE}}(\psi_u(x_i^t), \check{y}_i^t).
\end{equation}

Finally, the total optimization objective function is 
\begin{equation}
    L_{total}=L_{IP}+\lambda_{1}\mathcal{L}_{\text{\tiny inner}}+\lambda_{2}\mathcal{L}_{\text{\tiny outer}},
\end{equation}
where $\lambda_1$ and $\lambda_2$ are super-parameters for trade-off between different supervision.
\fi

\subsection{Improving Memory Buffer}

% Besides the distinctive design outlined above, the memory buffer \(\mathcal{M}\) plays a crucial role in supporting implementation. Following prior works, the memory buffer \(\mathcal{M}\) utilizes minimal capacity to store past samples and classes while adhering to privacy policies. We enhance the efficiency of the memory buffer with a class-balanced sampling strategy and storage cost reduction.

% 这部分并不重要,不需要单独拆两段论述了,可以整合一下
% Besides the distinctive design outlined above, 
The memory buffer \(\mathcal{M}\) plays a crucial role in our implementation and we implement the memory buffer based on unbiased learning and storage efficiency. 
% \paragraph{Class rebalance design}
IPSeg employs a class-balanced sampling strategy, ensuring the image posterior branch can adequately access samples from all classes. Specifically, given the memory size \(\left|\mathcal{M}\right|\) and the number of already seen classes \(\left|\mathcal{C}_{1:t}\right|\), the sampling strategy ensures there are at least \(\left|\mathcal{M}\right|//\left|\mathcal{C}_{1:t}\right|\) samples for each class.
% To ensure that the image posterior branch can adequately access samples from all classes and is trained with the least bias towards any specific class, IPSeg employs a class-balanced sampling strategy. Specifically, for a given memory size \(\left|\mathcal{M}\right|\) and the number of already seen classes \(\left|\mathcal{C}_{1:t}\right|\), we ensure that samples from each class appear at least \(\left|\mathcal{M}\right|//\left|\mathcal{C}_{1:t}\right|\) times.
IPSeg also optimizes the storage cost of $\mathcal{M}$ by only storing image-level labels and object salient masks for samples. Image-level labels are required for the image posterior branch for unbiased classification. While the salient masks split images into background and foreground objects, labeled with 0 and 1 respectively. This simplification mechanism requires less storage cost compared to previous methods that store the whole pixel-wise annotations on all classes. More details can be found in the appendix.
% \paragraph{Storage cost reduction} In IPSeg, only image-level labels and object salient masks of samples from \(\mathcal{M}\) are needed. Image-level labels are required for the image posterior (IP) branch to obtain IP guidance. While the salient masks simply divide the image into background and foreground objects, labeled with 0 and 1 respectively. This simplification mechanism requires less storage cost compared to previous methods that store the whole pixel-wise annotations of all classes, allowing IPSeg to store the same number of samples more efficiently.


\iffalse
Besides the above distinctive design, the memory buffer $\mathcal{M}$ is also important to support our method implementation. Following previous works, the memory buffer $\mathcal{M}$ utilizes minimal capacity to access past samples and classes without violating privacy policies. We improve the implementation of the memory buffer with a class-balanced sampling strategy.

\paragraph{Class rebalance design for memory buffer} To ensure that the image posterior branch can adequately access samples from all classes and 
is trained without bias to a specific class, IPSeg selects representative samples from both old and new classes and employs a class-balanced strategy for sample selection. Specifically, for a given memory size $\left|\mathcal{M}\right|$ and the number of already seen classes $\left|\mathcal{C}_{1:t}\right|$, we strictly ensure that each class samples appear at least $\left|M\right|/\left|\mathcal{C}_{1:t}\right|$ times. 
Further, the number of foreground object classes and pixels are counted as candidate selection criteria.

\paragraph{Simplified annotation for memory} In IPSeg, only image-level labels and object salient masks are needed. Image-level labels are needed for the image posterior branch. The salient mask simply divides the image into the background and foreground objects, labeled with $0$ and $1$ respectively. This simplification method needs just $1/\lceil\log_2C\rceil$ of storage cost to store the same number of samples as previous works do. 
\fi

\iffalse
\subsection{Image Posterior Guidance}
\label{sec3-3}

\textbf{Fact 1} \textit{For one image, if its image-level class scope is $\mathcal{C}_I$ and its pixels class scope is $\mathcal{C}_P$, class scope $\mathcal{C}_I$ and $\mathcal{C}_P$ are the same, that is $\mathcal{C}_I = \mathcal{C}_P$.}

This fact tells us that we can constrain the model only to predict the score of classes in $\mathcal{C}_x$ through post-filtering. However, in practical scenarios, image-level labels can not be directly obtained off-the-shell. 

To well utilize the above cues from the image level, we propose to build an image posterior prediction branch which trained in an incremental manner. In Detail, This image posterior prediction branch is implemented by a simple MLP with input dim of $4096$ and output dim of $\left|C\right|$. 

\fy{Image level gt/prediction is not differentiated from pixel-level...}
In task $t, t > 0$, the model can get access to two kinds of data, data in memory pool $x^M$ and data of the current task $x^t$. Data of the memory pool is with the corresponding class scope $\mathcal{C}_{1:t-1}$, while data of the current task is with the current task class scope $\mathcal{C}_{t}$. It ensures the image posterior branch can only get limited access to the previous class scope. In image posterior construction, a data sample $x_k$ from these two sources is processed by network backbone $h_{\phi}$ into image feature $h_{\phi}(x_k)$. Feature $h_{\phi}(x_k)$ is processed by \texttt{Adaptive Average Pooling} and \texttt{Flatten} into a 4096-dim tensor and then mapped into a $\left|\mathcal{C}_{1:t}\right|$-dim prediction $\hat{y}_k$ by image posterior branch. Finally, the training objective function of the image posterior is 
\begin{equation}
    \mathcal{L}_{\text{\tiny IP}} = \mathcal{L}_{\text{\tiny BCE}}(\hat{y}_k, y_k) 
\end{equation}
where $\hat{y}_k$ is the prediction of the image posterior branch on $\left|\mathcal{C}_{1:t}\right|$ classes and $y_k$ is the corresponding image-level ground-truth in one-hot form, originated from the pixel-wise annotation of task $t$.

In the inference, the trained image posterior branch directly predicts posterior probabilities on all classes. Given an image sample $x$ to be tested, the final prediction score on its pixel $x_{j}$ of a certain class $c$ can be obtained by element-wise multiplying between image posterior probability and pixel-wise prediction, 
\begin{equation}
    p_{j}^{c} = \sigma(\hat{y}^c) \times \sigma(\psi_{c}(h_\phi(x_{i,j}))).
\end{equation}
The final prediction on this pixel is,
\begin{equation}
    \hat{y}_{j} = \mathop{\arg\max}\limits_{c\in\mathcal{C}_{1:t}} p_{j}^c
\end{equation}

In the practical implementation, we adopt an improved implementation by utilizing already existing class-wise knowledge from the trained $\psi_{c_i}, c_i \in \mathcal{C}_{1:t-1}$. Instead of directly using image level classes from $\mathcal{C}_{t_1}$ for a memory sample of task $t-1$, IPSeg extends its class scope to  $\mathcal{C}_{1:t}$ by merging prediction from all segmentation heads.
\fy{using formulation to better express.}

\textbf{Class rebalance design for memory pool.} To ensure that the image posterior branch can adequately receive samples from all classes and maintain class scaling capability, for a given memory constraint, IPSeg selects representative samples from both old and new classes and employs a class-balanced strategy for collection to support effective training of image posterior branch. Specifically, for a given memory size $\left|M\right|$ and the number of already seen classes $\left|\mathcal{C}_{1:t}\right|$, we strictly ensure that each class's data appears at least $\left|M\right|/\left|\mathcal{C}_{1:t}\right|$ times. 
Further, for the memory pool candidates selection, the number of foreground pixels and classes are tallied as selection criteria. 
\fi

\iffalse
\fy{ During the incremental process, each phase only accesses annotated data for specific classes, leading to a complete loss of information for unknown foregrounds and foregrounds that should belong to old classes. Our method integrates saliency information to enhance the ability of each newly added classification head to recognize background and non-current classes. We add and maintain a continuously trained \(head[0]\), which is specifically responsible for identifying background and unknown classes. Pseudo-labeling is utilized to help classification heads maintain responsiveness to knowledge of old classes. Each classification head can not only recognize the classes of the current phase but also effectively distinguish between the background and non-current phase classes, thereby addressing the issue of complete loss of information for non-current classes during the incremental process, as discussed in section~\cref{sec3-1:preliminary}.

\paragraph{Step-wise knowledge mining for non-current}
We added phase-specific background and non-current recognition capabilities to the classification heads at different phases of the main network branch. By using saliency information to highlight potential foregrounds and combining class labels specific to this phase, we integrate the recognition of backgrounds and non-current classes into the training. This ensures that each classification head "sees" non-phase classes during training, achieving feature scale consistency within the heads at different phases of the model. During the training phase, auxiliary information such as a saliency mask and pseudo-labels are added to the ground truth, resulting in two new ground truths: ground truth for global and ground truth for current, corresponding to \(global\_gt\) and \(curr\_gt\) in Figure~\cref{fig:overview}, respectively. This process can be described as follows:
\[gt_{current}=gt_t\cup(\sim gt_t\cap Sal(x_m))\]
\[gt_{global}=gt_{t}\cup(\sim gt_{t}\cap\Theta_{t-1}(x_{m}))\cup(\sim gt_{t}\cap\sim\Theta_{t-1}(x_{m})\cap Sal(x_{m}))\]

Where, \(gt\) denotes the true class labels for this phase, \(Sal\) represents the saliency detection, and \( \Theta_{t-1}(x_{m})\) denotes the output from the model of the previous phase.  \( \Theta_{t-1}(x_{m})\)' is obtained according to the following rules:

\[\theta_{t-1}(x_m)_{(i,j)}=\begin{cases}\theta_{t-1}(x_m)_{(i,j)}~~~where~\theta_{t-1}(x_m)_{(i,j)}>\tau\\0~~~~~~~~~~~~~~~~~~~~~~~others\end{cases}\]

Additionally, to ensure better recognition of both current and other classes, this method fully mines foreground knowledge hidden at different depths within the model. It incorporates a feature integration mechanism that operates layer by layer to enhance its generalization capability and accuracy:

\begin{equation}
    {fm}_{{new}}={M}^{'}({G}_{1}(\Psi_{1}),G_{2}(\Psi_{2}),G_{3}(\Psi_{3}), ASPP(\Psi_{4}))
\end{equation}

% \[{fm}_{{new}}={M}^{'}({G}_{1}(\Psi_{1}),G_{2}(\Psi_{2}),G_{3}(\Psi_{3}), ASPP(\Psi_{4}))\]

Where $Psi_{i}$ represents the semantic information extracted from different layers of the backbone. $ASPP$ is the original method used in DeepLab V3. $G_{i}$ stands for the information mining function, and $M^{'}$ is the feature integration function, which aligns the processed semantic information from different layers into a new feature map.

\paragraph{Dual-loss for global and current}
After obtaining the two types of ground truth, we designed a composite loss strategy that combines global output and current output. In this approach, \(global\_gt\) focuses on constraining \(head[0]\) to recognize unknown classes, while \(curr\_gt\) focuses on constraining the new phase classification head to distinguish between its current and non-current items. This design makes the model more efficient when dealing with images with complex backgrounds and multiple classes, enhancing the new classification head's ability to recognize non-phase classes while also improving the identification of backgrounds and unknown classes. It provides a structured solution to balance the recognition of various classes. This can be described as follows:
\[output_{global}=H_0^c(fm_{new})\oplus Concat_{i=1}^tS(H_i^c(fm_{new}))\]
\[output_{current}=S^{-1}(H_t^c(fm_{new}))\oplus Concat_{i=1}^tS(H_i^c(fm_{new}))\]
\[L_{global}=BCE(output_{global},gt\_global)\]
\[L_{current}=BCE( output_{current} , gt\_curr)\]

Where, \(S\) denotes the removal of results from background and non-current in the classifier, \(S^{-1}\) denotes the removal of results from various classes, and \(\oplus\) represents the concatenation operation. Finally, the design of the loss during the training phase can be described as follows:
\[L_{total}=L_{scale}+\lambda_{1}L_{global}+\lambda_{2}L_{current}\]
Where \(\lambda_{i}\) is a hyper-parameter controlling the loss.}
\fi

\iffalse
\paragraph{Intra-head soft mask for output}
During the inference phase, a \(softmax\) operation is used within each classifier, in conjunction with a soft mask to process the semantic segmentation results, as illustrated in the bottom of Figure~\cref{fig:overview}. This approach further leverages non-current information within the classifier, and the soft mask enables decision-making without compromising the integrity of the original semantic outputs, effectively preventing classification errors. This can be described as follows:
\[M_t^c=Softmax(h_t^c)\]
\[h_{t_{(i,j)}}^{c}=\begin{cases}\lambda_{mask}\cdot class~t~~~~where~ M_{t_{(i,j)}}^{c}=non\_current\\class~t~~~~~~~~~~~~~~~~~where~M_{t_{(i,j)}}^{c}=class~t\\bg~~~~~~~~~~~~~~~~~~~~~~~~~where~ M_{t_{(i,j)}}^{c}=bg\end{cases}\]

Where,  \(h_t^c\) represents the output results from the classification head, \(M_t^c\) denotes the integrated information after applying a \(softmax\) operation to  \(h_t^c\), and \(\lambda_{mask}\) represents the soft mask coefficient.
\fi


\section{Experiments}
\label{sec:experiments}
The experiments are designed to address two key research questions.
First, \textbf{RQ1} evaluates whether the average $L_2$-norm of the counterfactual perturbation vectors ($\overline{||\perturb||}$) decreases as the model overfits the data, thereby providing further empirical validation for our hypothesis.
Second, \textbf{RQ2} evaluates the ability of the proposed counterfactual regularized loss, as defined in (\ref{eq:regularized_loss2}), to mitigate overfitting when compared to existing regularization techniques.

% The experiments are designed to address three key research questions. First, \textbf{RQ1} investigates whether the mean perturbation vector norm decreases as the model overfits the data, aiming to further validate our intuition. Second, \textbf{RQ2} explores whether the mean perturbation vector norm can be effectively leveraged as a regularization term during training, offering insights into its potential role in mitigating overfitting. Finally, \textbf{RQ3} examines whether our counterfactual regularizer enables the model to achieve superior performance compared to existing regularization methods, thus highlighting its practical advantage.

\subsection{Experimental Setup}
\textbf{\textit{Datasets, Models, and Tasks.}}
The experiments are conducted on three datasets: \textit{Water Potability}~\cite{kadiwal2020waterpotability}, \textit{Phomene}~\cite{phomene}, and \textit{CIFAR-10}~\cite{krizhevsky2009learning}. For \textit{Water Potability} and \textit{Phomene}, we randomly select $80\%$ of the samples for the training set, and the remaining $20\%$ for the test set, \textit{CIFAR-10} comes already split. Furthermore, we consider the following models: Logistic Regression, Multi-Layer Perceptron (MLP) with 100 and 30 neurons on each hidden layer, and PreactResNet-18~\cite{he2016cvecvv} as a Convolutional Neural Network (CNN) architecture.
We focus on binary classification tasks and leave the extension to multiclass scenarios for future work. However, for datasets that are inherently multiclass, we transform the problem into a binary classification task by selecting two classes, aligning with our assumption.

\smallskip
\noindent\textbf{\textit{Evaluation Measures.}} To characterize the degree of overfitting, we use the test loss, as it serves as a reliable indicator of the model's generalization capability to unseen data. Additionally, we evaluate the predictive performance of each model using the test accuracy.

\smallskip
\noindent\textbf{\textit{Baselines.}} We compare CF-Reg with the following regularization techniques: L1 (``Lasso''), L2 (``Ridge''), and Dropout.

\smallskip
\noindent\textbf{\textit{Configurations.}}
For each model, we adopt specific configurations as follows.
\begin{itemize}
\item \textit{Logistic Regression:} To induce overfitting in the model, we artificially increase the dimensionality of the data beyond the number of training samples by applying a polynomial feature expansion. This approach ensures that the model has enough capacity to overfit the training data, allowing us to analyze the impact of our counterfactual regularizer. The degree of the polynomial is chosen as the smallest degree that makes the number of features greater than the number of data.
\item \textit{Neural Networks (MLP and CNN):} To take advantage of the closed-form solution for computing the optimal perturbation vector as defined in (\ref{eq:opt-delta}), we use a local linear approximation of the neural network models. Hence, given an instance $\inst_i$, we consider the (optimal) counterfactual not with respect to $\model$ but with respect to:
\begin{equation}
\label{eq:taylor}
    \model^{lin}(\inst) = \model(\inst_i) + \nabla_{\inst}\model(\inst_i)(\inst - \inst_i),
\end{equation}
where $\model^{lin}$ represents the first-order Taylor approximation of $\model$ at $\inst_i$.
Note that this step is unnecessary for Logistic Regression, as it is inherently a linear model.
\end{itemize}

\smallskip
\noindent \textbf{\textit{Implementation Details.}} We run all experiments on a machine equipped with an AMD Ryzen 9 7900 12-Core Processor and an NVIDIA GeForce RTX 4090 GPU. Our implementation is based on the PyTorch Lightning framework. We use stochastic gradient descent as the optimizer with a learning rate of $\eta = 0.001$ and no weight decay. We use a batch size of $128$. The training and test steps are conducted for $6000$ epochs on the \textit{Water Potability} and \textit{Phoneme} datasets, while for the \textit{CIFAR-10} dataset, they are performed for $200$ epochs.
Finally, the contribution $w_i^{\varepsilon}$ of each training point $\inst_i$ is uniformly set as $w_i^{\varepsilon} = 1~\forall i\in \{1,\ldots,m\}$.

The source code implementation for our experiments is available at the following GitHub repository: \url{https://anonymous.4open.science/r/COCE-80B4/README.md} 

\subsection{RQ1: Counterfactual Perturbation vs. Overfitting}
To address \textbf{RQ1}, we analyze the relationship between the test loss and the average $L_2$-norm of the counterfactual perturbation vectors ($\overline{||\perturb||}$) over training epochs.

In particular, Figure~\ref{fig:delta_loss_epochs} depicts the evolution of $\overline{||\perturb||}$ alongside the test loss for an MLP trained \textit{without} regularization on the \textit{Water Potability} dataset. 
\begin{figure}[ht]
    \centering
    \includegraphics[width=0.85\linewidth]{img/delta_loss_epochs.png}
    \caption{The average counterfactual perturbation vector $\overline{||\perturb||}$ (left $y$-axis) and the cross-entropy test loss (right $y$-axis) over training epochs ($x$-axis) for an MLP trained on the \textit{Water Potability} dataset \textit{without} regularization.}
    \label{fig:delta_loss_epochs}
\end{figure}

The plot shows a clear trend as the model starts to overfit the data (evidenced by an increase in test loss). 
Notably, $\overline{||\perturb||}$ begins to decrease, which aligns with the hypothesis that the average distance to the optimal counterfactual example gets smaller as the model's decision boundary becomes increasingly adherent to the training data.

It is worth noting that this trend is heavily influenced by the choice of the counterfactual generator model. In particular, the relationship between $\overline{||\perturb||}$ and the degree of overfitting may become even more pronounced when leveraging more accurate counterfactual generators. However, these models often come at the cost of higher computational complexity, and their exploration is left to future work.

Nonetheless, we expect that $\overline{||\perturb||}$ will eventually stabilize at a plateau, as the average $L_2$-norm of the optimal counterfactual perturbations cannot vanish to zero.

% Additionally, the choice of employing the score-based counterfactual explanation framework to generate counterfactuals was driven to promote computational efficiency.

% Future enhancements to the framework may involve adopting models capable of generating more precise counterfactuals. While such approaches may yield to performance improvements, they are likely to come at the cost of increased computational complexity.


\subsection{RQ2: Counterfactual Regularization Performance}
To answer \textbf{RQ2}, we evaluate the effectiveness of the proposed counterfactual regularization (CF-Reg) by comparing its performance against existing baselines: unregularized training loss (No-Reg), L1 regularization (L1-Reg), L2 regularization (L2-Reg), and Dropout.
Specifically, for each model and dataset combination, Table~\ref{tab:regularization_comparison} presents the mean value and standard deviation of test accuracy achieved by each method across 5 random initialization. 

The table illustrates that our regularization technique consistently delivers better results than existing methods across all evaluated scenarios, except for one case -- i.e., Logistic Regression on the \textit{Phomene} dataset. 
However, this setting exhibits an unusual pattern, as the highest model accuracy is achieved without any regularization. Even in this case, CF-Reg still surpasses other regularization baselines.

From the results above, we derive the following key insights. First, CF-Reg proves to be effective across various model types, ranging from simple linear models (Logistic Regression) to deep architectures like MLPs and CNNs, and across diverse datasets, including both tabular and image data. 
Second, CF-Reg's strong performance on the \textit{Water} dataset with Logistic Regression suggests that its benefits may be more pronounced when applied to simpler models. However, the unexpected outcome on the \textit{Phoneme} dataset calls for further investigation into this phenomenon.


\begin{table*}[h!]
    \centering
    \caption{Mean value and standard deviation of test accuracy across 5 random initializations for different model, dataset, and regularization method. The best results are highlighted in \textbf{bold}.}
    \label{tab:regularization_comparison}
    \begin{tabular}{|c|c|c|c|c|c|c|}
        \hline
        \textbf{Model} & \textbf{Dataset} & \textbf{No-Reg} & \textbf{L1-Reg} & \textbf{L2-Reg} & \textbf{Dropout} & \textbf{CF-Reg (ours)} \\ \hline
        Logistic Regression   & \textit{Water}   & $0.6595 \pm 0.0038$   & $0.6729 \pm 0.0056$   & $0.6756 \pm 0.0046$  & N/A    & $\mathbf{0.6918 \pm 0.0036}$                     \\ \hline
        MLP   & \textit{Water}   & $0.6756 \pm 0.0042$   & $0.6790 \pm 0.0058$   & $0.6790 \pm 0.0023$  & $0.6750 \pm 0.0036$    & $\mathbf{0.6802 \pm 0.0046}$                    \\ \hline
%        MLP   & \textit{Adult}   & $0.8404 \pm 0.0010$   & $\mathbf{0.8495 \pm 0.0007}$   & $0.8489 \pm 0.0014$  & $\mathbf{0.8495 \pm 0.0016}$     & $0.8449 \pm 0.0019$                    \\ \hline
        Logistic Regression   & \textit{Phomene}   & $\mathbf{0.8148 \pm 0.0020}$   & $0.8041 \pm 0.0028$   & $0.7835 \pm 0.0176$  & N/A    & $0.8098 \pm 0.0055$                     \\ \hline
        MLP   & \textit{Phomene}   & $0.8677 \pm 0.0033$   & $0.8374 \pm 0.0080$   & $0.8673 \pm 0.0045$  & $0.8672 \pm 0.0042$     & $\mathbf{0.8718 \pm 0.0040}$                    \\ \hline
        CNN   & \textit{CIFAR-10} & $0.6670 \pm 0.0233$   & $0.6229 \pm 0.0850$   & $0.7348 \pm 0.0365$   & N/A    & $\mathbf{0.7427 \pm 0.0571}$                     \\ \hline
    \end{tabular}
\end{table*}

\begin{table*}[htb!]
    \centering
    \caption{Hyperparameter configurations utilized for the generation of Table \ref{tab:regularization_comparison}. For our regularization the hyperparameters are reported as $\mathbf{\alpha/\beta}$.}
    \label{tab:performance_parameters}
    \begin{tabular}{|c|c|c|c|c|c|c|}
        \hline
        \textbf{Model} & \textbf{Dataset} & \textbf{No-Reg} & \textbf{L1-Reg} & \textbf{L2-Reg} & \textbf{Dropout} & \textbf{CF-Reg (ours)} \\ \hline
        Logistic Regression   & \textit{Water}   & N/A   & $0.0093$   & $0.6927$  & N/A    & $0.3791/1.0355$                     \\ \hline
        MLP   & \textit{Water}   & N/A   & $0.0007$   & $0.0022$  & $0.0002$    & $0.2567/1.9775$                    \\ \hline
        Logistic Regression   &
        \textit{Phomene}   & N/A   & $0.0097$   & $0.7979$  & N/A    & $0.0571/1.8516$                     \\ \hline
        MLP   & \textit{Phomene}   & N/A   & $0.0007$   & $4.24\cdot10^{-5}$  & $0.0015$    & $0.0516/2.2700$                    \\ \hline
       % MLP   & \textit{Adult}   & N/A   & $0.0018$   & $0.0018$  & $0.0601$     & $0.0764/2.2068$                    \\ \hline
        CNN   & \textit{CIFAR-10} & N/A   & $0.0050$   & $0.0864$ & N/A    & $0.3018/
        2.1502$                     \\ \hline
    \end{tabular}
\end{table*}

\begin{table*}[htb!]
    \centering
    \caption{Mean value and standard deviation of training time across 5 different runs. The reported time (in seconds) corresponds to the generation of each entry in Table \ref{tab:regularization_comparison}. Times are }
    \label{tab:times}
    \begin{tabular}{|c|c|c|c|c|c|c|}
        \hline
        \textbf{Model} & \textbf{Dataset} & \textbf{No-Reg} & \textbf{L1-Reg} & \textbf{L2-Reg} & \textbf{Dropout} & \textbf{CF-Reg (ours)} \\ \hline
        Logistic Regression   & \textit{Water}   & $222.98 \pm 1.07$   & $239.94 \pm 2.59$   & $241.60 \pm 1.88$  & N/A    & $251.50 \pm 1.93$                     \\ \hline
        MLP   & \textit{Water}   & $225.71 \pm 3.85$   & $250.13 \pm 4.44$   & $255.78 \pm 2.38$  & $237.83 \pm 3.45$    & $266.48 \pm 3.46$                    \\ \hline
        Logistic Regression   & \textit{Phomene}   & $266.39 \pm 0.82$ & $367.52 \pm 6.85$   & $361.69 \pm 4.04$  & N/A   & $310.48 \pm 0.76$                    \\ \hline
        MLP   &
        \textit{Phomene} & $335.62 \pm 1.77$   & $390.86 \pm 2.11$   & $393.96 \pm 1.95$ & $363.51 \pm 5.07$    & $403.14 \pm 1.92$                     \\ \hline
       % MLP   & \textit{Adult}   & N/A   & $0.0018$   & $0.0018$  & $0.0601$     & $0.0764/2.2068$                    \\ \hline
        CNN   & \textit{CIFAR-10} & $370.09 \pm 0.18$   & $395.71 \pm 0.55$   & $401.38 \pm 0.16$ & N/A    & $1287.8 \pm 0.26$                     \\ \hline
    \end{tabular}
\end{table*}

\subsection{Feasibility of our Method}
A crucial requirement for any regularization technique is that it should impose minimal impact on the overall training process.
In this respect, CF-Reg introduces an overhead that depends on the time required to find the optimal counterfactual example for each training instance. 
As such, the more sophisticated the counterfactual generator model probed during training the higher would be the time required. However, a more advanced counterfactual generator might provide a more effective regularization. We discuss this trade-off in more details in Section~\ref{sec:discussion}.

Table~\ref{tab:times} presents the average training time ($\pm$ standard deviation) for each model and dataset combination listed in Table~\ref{tab:regularization_comparison}.
We can observe that the higher accuracy achieved by CF-Reg using the score-based counterfactual generator comes with only minimal overhead. However, when applied to deep neural networks with many hidden layers, such as \textit{PreactResNet-18}, the forward derivative computation required for the linearization of the network introduces a more noticeable computational cost, explaining the longer training times in the table.

\subsection{Hyperparameter Sensitivity Analysis}
The proposed counterfactual regularization technique relies on two key hyperparameters: $\alpha$ and $\beta$. The former is intrinsic to the loss formulation defined in (\ref{eq:cf-train}), while the latter is closely tied to the choice of the score-based counterfactual explanation method used.

Figure~\ref{fig:test_alpha_beta} illustrates how the test accuracy of an MLP trained on the \textit{Water Potability} dataset changes for different combinations of $\alpha$ and $\beta$.

\begin{figure}[ht]
    \centering
    \includegraphics[width=0.85\linewidth]{img/test_acc_alpha_beta.png}
    \caption{The test accuracy of an MLP trained on the \textit{Water Potability} dataset, evaluated while varying the weight of our counterfactual regularizer ($\alpha$) for different values of $\beta$.}
    \label{fig:test_alpha_beta}
\end{figure}

We observe that, for a fixed $\beta$, increasing the weight of our counterfactual regularizer ($\alpha$) can slightly improve test accuracy until a sudden drop is noticed for $\alpha > 0.1$.
This behavior was expected, as the impact of our penalty, like any regularization term, can be disruptive if not properly controlled.

Moreover, this finding further demonstrates that our regularization method, CF-Reg, is inherently data-driven. Therefore, it requires specific fine-tuning based on the combination of the model and dataset at hand.

\vspace{-0.2cm}
\section{Impact: Why Free Scientific Knowledge?}
\vspace{-0.1cm}

Historically, making knowledge widely available has driven transformative progress. Gutenberg’s printing press broke medieval monopolies on information, increasing literacy and contributing to the Renaissance and Scientific Revolution. In today's world, open source projects such as GNU/Linux and Wikipedia show that freely accessible and modifiable knowledge fosters innovation while ensuring creators are credited through copyleft licenses. These examples highlight a key idea: \textit{access to essential knowledge supports overall advancement.} 

This aligns with the arguments made by Prabhakaran et al. \cite{humanrightsbasedapproachresponsible}, who specifically highlight the \textbf{ human right to participate in scientific advancement} as enshrined in the Universal Declaration of Human Rights. They emphasize that this right underscores the importance of \textit{ equal access to the benefits of scientific progress for all}, a principle directly supported by our proposal for Knowledge Units. The UN Special Rapporteur on Cultural Rights further reinforces this, advocating for the expansion of copyright exceptions to broaden access to scientific knowledge as a crucial component of the right to science and culture \cite{scienceright}. 

However, current intellectual property regimes often create ``patently unfair" barriers to this knowledge, preventing innovation and access, especially in areas critical to human rights, as Hale compellingly argues \cite{patentlyunfair}. Finding a solution requires carefully balancing the imperative of open access with the legitimate rights of authors. As Austin and Ginsburg remind us, authors' rights are also human rights, necessitating robust protection \cite{authorhumanrights}. Shareable knowledge entities like Knowledge Units offer a potential mechanism to achieve this delicate balance in the scientific domain, enabling wider dissemination of research findings while respecting authors' fundamental rights.

\vspace{-0.2cm}
\subsection{Impact Across Sectors}

\textbf{Researchers:} Collaboration across different fields becomes easier when knowledge is shared openly. For instance, combining machine learning with biology or applying quantum principles to cryptography can lead to important breakthroughs. Removing copyright restrictions allows researchers to freely use data and methods, speeding up discoveries while respecting original contributions.

\textbf{Practitioners:} Professionals, especially in healthcare, benefit from immediate access to the latest research. Quick access to newer insights on the effectiveness of drugs, and alternative treatments speeds up adoption and awareness, potentially saving lives. Additionally, open knowledge helps developing countries gain access to health innovations.

\textbf{Education:} Education becomes more accessible when teachers use the latest research to create up-to-date curricula without prohibitive costs. Students can access high-quality research materials and use LM assistance to better understand complex topics, enhancing their learning experience and making high-quality education more accessible.

\textbf{Public Trust:} When information is transparent and accessible, the public can better understand and trust decision-making processes. Open access to government policies and industry practices allows people to review and verify information, helping to reduce misinformation. This transparency encourages critical thinking and builds trust in scientific and governmental institutions.

Overall, making scientific knowledge accessible supports global fairness. By viewing knowledge as a common resource rather than a product to be sold, we can speed up innovation, encourage critical thinking, and empower communities to address important challenges.

\vspace{-0.2cm}
\section{Open Problems}
\vspace{-0.1cm}

Moving forward, we identify key research directions to further exploit the potential of converting original texts into shareable knowledge entities such as demonstrated by the conversion into Knowledge Units in this work:


\textbf{1. Enhancing Factual Accuracy and Reliability:}  Refining KUs through cross-referencing with source texts and incorporating community-driven correction mechanisms, similar to Wikipedia, can minimize hallucinations and ensure the long-term accuracy of knowledge-based datasets at scale.

\textbf{2. Developing Applications for Education and Research:}  Using KU-based conversion for datasets to be employed in practical tools, such as search interfaces and learning platforms, can ensure rapid dissemination of any new knowledge into shareable downstream resources, significantly improving the accessibility, spread, and impact of KUs.

\textbf{3. Establishing Standards for Knowledge Interoperability and Reuse:}  Future research should focus on defining standardized formats for entities like KU and knowledge graph layouts \citep{lenat1990cyc}. These standards are essential to unlock seamless interoperability, facilitate reuse across diverse platforms, and foster a vibrant ecosystem of open scientific knowledge. 

\textbf{4. Interconnecting Shareable Knowledge for Scientific Workflow Assistance and Automation:} There might be further potential in constructing a semantic web that interconnects publicly shared knowledge, together with mechanisms that continually update and validate all shareable knowledge units. This can be starting point for a platform that uses all collected knowledge to assist scientific workflows, for instance by feeding such a semantic web into recently developed reasoning models equipped with retrieval augmented generation. Such assistance could assemble knowledge across multiple scientific papers, guiding scientists more efficiently through vast research landscapes. Given further progress in model capabilities, validation, self-repair and evolving new knowledge from already existing vast collection in the semantic web can lead to automation of scientific discovery, assuming that knowledge data in the semantic web can be freely shared.

We open-source our code and encourage collaboration to improve extraction pipelines, enhance Knowledge Unit capabilities, and expand coverage to additional fields.

\vspace{-0.2cm}
\section{Conclusion}
\vspace{-0.1cm}

In this paper, we highlight the potential of systematically separating factual scientific knowledge from protected artistic or stylistic expression. By representing scientific insights as structured facts and relationships, prototypes like Knowledge Units (KUs) offer a pathway to broaden access to scientific knowledge without infringing copyright, aligning with legal principles like German \S 24(1) UrhG and U.S. fair use standards. Extensive testing across a range of domains and models shows evidence that Knowledge Units (KUs) can feasibly retain core information. These findings offer a promising way forward for openly disseminating scientific information while respecting copyright constraints.

\section*{Author Contributions}

Christoph conceived the project and led organization. Christoph and Gollam led all the experiments. Nick and Huu led the legal aspects. Tawsif led the data collection. Ameya and Andreas led the manuscript writing. Ludwig, Sören, Robert, Jenia and Matthias provided feedback. advice and scientific supervision throughout the project. 

\section*{Acknowledgements}

The authors would like to thank (in alphabetical order): Sebastian Dziadzio, Kristof Meding, Tea Mustać, Shantanu Prabhat for insightful feedback and suggestions. Special thanks to Andrej Radonjic for help in scaling up data collection. GR and SA acknowledge financial support by the German Research Foundation (DFG) for the NFDI4DataScience Initiative (project number 460234259). AP and MB acknowledge financial support by the Federal Ministry of Education and Research (BMBF), FKZ: 011524085B and Open Philanthropy Foundation funded by the Good Ventures Foundation. AH acknowledges financial support by the Federal Ministry of Education and Research (BMBF), FKZ: 01IS24079A and the Carl Zeiss Foundation through the project "Certification and Foundations of Safe ML Systems" as well as the support from the International Max Planck Research School for Intelligent Systems (IMPRS-IS). JJ acknowledges funding by the Federal Ministry of Education and Research of Germany (BMBF) under grant no. 01IS22094B (WestAI - AI Service Center West), under grant no. 01IS24085C (OPENHAFM) and under the grant DE002571 (MINERVA), as well as co-funding by EU from EuroHPC Joint Undertaking programm under grant no. 101182737 (MINERVA) and from Digital Europe Programme under grant no. 101195233 (openEuroLLM) 




% In the unusual situation where you want a paper to appear in the
% references without citing it in the main text, use \nocite
\nocite{langley00}
% \newpage
\bibliography{example_paper}
\bibliographystyle{icml2025}


%%%%%%%%%%%%%%%%%%%%%%%%%%%%%%%%%%%%%%%%%%%%%%%%%%%%%%%%%%%%%%%%%%%%%%%%%%%%%%%
%%%%%%%%%%%%%%%%%%%%%%%%%%%%%%%%%%%%%%%%%%%%%%%%%%%%%%%%%%%%%%%%%%%%%%%%%%%%%%%
% APPENDIX
%%%%%%%%%%%%%%%%%%%%%%%%%%%%%%%%%%%%%%%%%%%%%%%%%%%%%%%%%%%%%%%%%%%%%%%%%%%%%%%
%%%%%%%%%%%%%%%%%%%%%%%%%%%%%%%%%%%%%%%%%%%%%%%%%%%%%%%%%%%%%%%%%%%%%%%%%%%%%%%
\newpage
\appendix
\onecolumn
\subsection{Lloyd-Max Algorithm}
\label{subsec:Lloyd-Max}
For a given quantization bitwidth $B$ and an operand $\bm{X}$, the Lloyd-Max algorithm finds $2^B$ quantization levels $\{\hat{x}_i\}_{i=1}^{2^B}$ such that quantizing $\bm{X}$ by rounding each scalar in $\bm{X}$ to the nearest quantization level minimizes the quantization MSE. 

The algorithm starts with an initial guess of quantization levels and then iteratively computes quantization thresholds $\{\tau_i\}_{i=1}^{2^B-1}$ and updates quantization levels $\{\hat{x}_i\}_{i=1}^{2^B}$. Specifically, at iteration $n$, thresholds are set to the midpoints of the previous iteration's levels:
\begin{align*}
    \tau_i^{(n)}=\frac{\hat{x}_i^{(n-1)}+\hat{x}_{i+1}^{(n-1)}}2 \text{ for } i=1\ldots 2^B-1
\end{align*}
Subsequently, the quantization levels are re-computed as conditional means of the data regions defined by the new thresholds:
\begin{align*}
    \hat{x}_i^{(n)}=\mathbb{E}\left[ \bm{X} \big| \bm{X}\in [\tau_{i-1}^{(n)},\tau_i^{(n)}] \right] \text{ for } i=1\ldots 2^B
\end{align*}
where to satisfy boundary conditions we have $\tau_0=-\infty$ and $\tau_{2^B}=\infty$. The algorithm iterates the above steps until convergence.

Figure \ref{fig:lm_quant} compares the quantization levels of a $7$-bit floating point (E3M3) quantizer (left) to a $7$-bit Lloyd-Max quantizer (right) when quantizing a layer of weights from the GPT3-126M model at a per-tensor granularity. As shown, the Lloyd-Max quantizer achieves substantially lower quantization MSE. Further, Table \ref{tab:FP7_vs_LM7} shows the superior perplexity achieved by Lloyd-Max quantizers for bitwidths of $7$, $6$ and $5$. The difference between the quantizers is clear at 5 bits, where per-tensor FP quantization incurs a drastic and unacceptable increase in perplexity, while Lloyd-Max quantization incurs a much smaller increase. Nevertheless, we note that even the optimal Lloyd-Max quantizer incurs a notable ($\sim 1.5$) increase in perplexity due to the coarse granularity of quantization. 

\begin{figure}[h]
  \centering
  \includegraphics[width=0.7\linewidth]{sections/figures/LM7_FP7.pdf}
  \caption{\small Quantization levels and the corresponding quantization MSE of Floating Point (left) vs Lloyd-Max (right) Quantizers for a layer of weights in the GPT3-126M model.}
  \label{fig:lm_quant}
\end{figure}

\begin{table}[h]\scriptsize
\begin{center}
\caption{\label{tab:FP7_vs_LM7} \small Comparing perplexity (lower is better) achieved by floating point quantizers and Lloyd-Max quantizers on a GPT3-126M model for the Wikitext-103 dataset.}
\begin{tabular}{c|cc|c}
\hline
 \multirow{2}{*}{\textbf{Bitwidth}} & \multicolumn{2}{|c|}{\textbf{Floating-Point Quantizer}} & \textbf{Lloyd-Max Quantizer} \\
 & Best Format & Wikitext-103 Perplexity & Wikitext-103 Perplexity \\
\hline
7 & E3M3 & 18.32 & 18.27 \\
6 & E3M2 & 19.07 & 18.51 \\
5 & E4M0 & 43.89 & 19.71 \\
\hline
\end{tabular}
\end{center}
\end{table}

\subsection{Proof of Local Optimality of LO-BCQ}
\label{subsec:lobcq_opt_proof}
For a given block $\bm{b}_j$, the quantization MSE during LO-BCQ can be empirically evaluated as $\frac{1}{L_b}\lVert \bm{b}_j- \bm{\hat{b}}_j\rVert^2_2$ where $\bm{\hat{b}}_j$ is computed from equation (\ref{eq:clustered_quantization_definition}) as $C_{f(\bm{b}_j)}(\bm{b}_j)$. Further, for a given block cluster $\mathcal{B}_i$, we compute the quantization MSE as $\frac{1}{|\mathcal{B}_{i}|}\sum_{\bm{b} \in \mathcal{B}_{i}} \frac{1}{L_b}\lVert \bm{b}- C_i^{(n)}(\bm{b})\rVert^2_2$. Therefore, at the end of iteration $n$, we evaluate the overall quantization MSE $J^{(n)}$ for a given operand $\bm{X}$ composed of $N_c$ block clusters as:
\begin{align*}
    \label{eq:mse_iter_n}
    J^{(n)} = \frac{1}{N_c} \sum_{i=1}^{N_c} \frac{1}{|\mathcal{B}_{i}^{(n)}|}\sum_{\bm{v} \in \mathcal{B}_{i}^{(n)}} \frac{1}{L_b}\lVert \bm{b}- B_i^{(n)}(\bm{b})\rVert^2_2
\end{align*}

At the end of iteration $n$, the codebooks are updated from $\mathcal{C}^{(n-1)}$ to $\mathcal{C}^{(n)}$. However, the mapping of a given vector $\bm{b}_j$ to quantizers $\mathcal{C}^{(n)}$ remains as  $f^{(n)}(\bm{b}_j)$. At the next iteration, during the vector clustering step, $f^{(n+1)}(\bm{b}_j)$ finds new mapping of $\bm{b}_j$ to updated codebooks $\mathcal{C}^{(n)}$ such that the quantization MSE over the candidate codebooks is minimized. Therefore, we obtain the following result for $\bm{b}_j$:
\begin{align*}
\frac{1}{L_b}\lVert \bm{b}_j - C_{f^{(n+1)}(\bm{b}_j)}^{(n)}(\bm{b}_j)\rVert^2_2 \le \frac{1}{L_b}\lVert \bm{b}_j - C_{f^{(n)}(\bm{b}_j)}^{(n)}(\bm{b}_j)\rVert^2_2
\end{align*}

That is, quantizing $\bm{b}_j$ at the end of the block clustering step of iteration $n+1$ results in lower quantization MSE compared to quantizing at the end of iteration $n$. Since this is true for all $\bm{b} \in \bm{X}$, we assert the following:
\begin{equation}
\begin{split}
\label{eq:mse_ineq_1}
    \tilde{J}^{(n+1)} &= \frac{1}{N_c} \sum_{i=1}^{N_c} \frac{1}{|\mathcal{B}_{i}^{(n+1)}|}\sum_{\bm{b} \in \mathcal{B}_{i}^{(n+1)}} \frac{1}{L_b}\lVert \bm{b} - C_i^{(n)}(b)\rVert^2_2 \le J^{(n)}
\end{split}
\end{equation}
where $\tilde{J}^{(n+1)}$ is the the quantization MSE after the vector clustering step at iteration $n+1$.

Next, during the codebook update step (\ref{eq:quantizers_update}) at iteration $n+1$, the per-cluster codebooks $\mathcal{C}^{(n)}$ are updated to $\mathcal{C}^{(n+1)}$ by invoking the Lloyd-Max algorithm \citep{Lloyd}. We know that for any given value distribution, the Lloyd-Max algorithm minimizes the quantization MSE. Therefore, for a given vector cluster $\mathcal{B}_i$ we obtain the following result:

\begin{equation}
    \frac{1}{|\mathcal{B}_{i}^{(n+1)}|}\sum_{\bm{b} \in \mathcal{B}_{i}^{(n+1)}} \frac{1}{L_b}\lVert \bm{b}- C_i^{(n+1)}(\bm{b})\rVert^2_2 \le \frac{1}{|\mathcal{B}_{i}^{(n+1)}|}\sum_{\bm{b} \in \mathcal{B}_{i}^{(n+1)}} \frac{1}{L_b}\lVert \bm{b}- C_i^{(n)}(\bm{b})\rVert^2_2
\end{equation}

The above equation states that quantizing the given block cluster $\mathcal{B}_i$ after updating the associated codebook from $C_i^{(n)}$ to $C_i^{(n+1)}$ results in lower quantization MSE. Since this is true for all the block clusters, we derive the following result: 
\begin{equation}
\begin{split}
\label{eq:mse_ineq_2}
     J^{(n+1)} &= \frac{1}{N_c} \sum_{i=1}^{N_c} \frac{1}{|\mathcal{B}_{i}^{(n+1)}|}\sum_{\bm{b} \in \mathcal{B}_{i}^{(n+1)}} \frac{1}{L_b}\lVert \bm{b}- C_i^{(n+1)}(\bm{b})\rVert^2_2  \le \tilde{J}^{(n+1)}   
\end{split}
\end{equation}

Following (\ref{eq:mse_ineq_1}) and (\ref{eq:mse_ineq_2}), we find that the quantization MSE is non-increasing for each iteration, that is, $J^{(1)} \ge J^{(2)} \ge J^{(3)} \ge \ldots \ge J^{(M)}$ where $M$ is the maximum number of iterations. 
%Therefore, we can say that if the algorithm converges, then it must be that it has converged to a local minimum. 
\hfill $\blacksquare$


\begin{figure}
    \begin{center}
    \includegraphics[width=0.5\textwidth]{sections//figures/mse_vs_iter.pdf}
    \end{center}
    \caption{\small NMSE vs iterations during LO-BCQ compared to other block quantization proposals}
    \label{fig:nmse_vs_iter}
\end{figure}

Figure \ref{fig:nmse_vs_iter} shows the empirical convergence of LO-BCQ across several block lengths and number of codebooks. Also, the MSE achieved by LO-BCQ is compared to baselines such as MXFP and VSQ. As shown, LO-BCQ converges to a lower MSE than the baselines. Further, we achieve better convergence for larger number of codebooks ($N_c$) and for a smaller block length ($L_b$), both of which increase the bitwidth of BCQ (see Eq \ref{eq:bitwidth_bcq}).


\subsection{Additional Accuracy Results}
%Table \ref{tab:lobcq_config} lists the various LOBCQ configurations and their corresponding bitwidths.
\begin{table}
\setlength{\tabcolsep}{4.75pt}
\begin{center}
\caption{\label{tab:lobcq_config} Various LO-BCQ configurations and their bitwidths.}
\begin{tabular}{|c||c|c|c|c||c|c||c|} 
\hline
 & \multicolumn{4}{|c||}{$L_b=8$} & \multicolumn{2}{|c||}{$L_b=4$} & $L_b=2$ \\
 \hline
 \backslashbox{$L_A$\kern-1em}{\kern-1em$N_c$} & 2 & 4 & 8 & 16 & 2 & 4 & 2 \\
 \hline
 64 & 4.25 & 4.375 & 4.5 & 4.625 & 4.375 & 4.625 & 4.625\\
 \hline
 32 & 4.375 & 4.5 & 4.625& 4.75 & 4.5 & 4.75 & 4.75 \\
 \hline
 16 & 4.625 & 4.75& 4.875 & 5 & 4.75 & 5 & 5 \\
 \hline
\end{tabular}
\end{center}
\end{table}

%\subsection{Perplexity achieved by various LO-BCQ configurations on Wikitext-103 dataset}

\begin{table} \centering
\begin{tabular}{|c||c|c|c|c||c|c||c|} 
\hline
 $L_b \rightarrow$& \multicolumn{4}{c||}{8} & \multicolumn{2}{c||}{4} & 2\\
 \hline
 \backslashbox{$L_A$\kern-1em}{\kern-1em$N_c$} & 2 & 4 & 8 & 16 & 2 & 4 & 2  \\
 %$N_c \rightarrow$ & 2 & 4 & 8 & 16 & 2 & 4 & 2 \\
 \hline
 \hline
 \multicolumn{8}{c}{GPT3-1.3B (FP32 PPL = 9.98)} \\ 
 \hline
 \hline
 64 & 10.40 & 10.23 & 10.17 & 10.15 &  10.28 & 10.18 & 10.19 \\
 \hline
 32 & 10.25 & 10.20 & 10.15 & 10.12 &  10.23 & 10.17 & 10.17 \\
 \hline
 16 & 10.22 & 10.16 & 10.10 & 10.09 &  10.21 & 10.14 & 10.16 \\
 \hline
  \hline
 \multicolumn{8}{c}{GPT3-8B (FP32 PPL = 7.38)} \\ 
 \hline
 \hline
 64 & 7.61 & 7.52 & 7.48 &  7.47 &  7.55 &  7.49 & 7.50 \\
 \hline
 32 & 7.52 & 7.50 & 7.46 &  7.45 &  7.52 &  7.48 & 7.48  \\
 \hline
 16 & 7.51 & 7.48 & 7.44 &  7.44 &  7.51 &  7.49 & 7.47  \\
 \hline
\end{tabular}
\caption{\label{tab:ppl_gpt3_abalation} Wikitext-103 perplexity across GPT3-1.3B and 8B models.}
\end{table}

\begin{table} \centering
\begin{tabular}{|c||c|c|c|c||} 
\hline
 $L_b \rightarrow$& \multicolumn{4}{c||}{8}\\
 \hline
 \backslashbox{$L_A$\kern-1em}{\kern-1em$N_c$} & 2 & 4 & 8 & 16 \\
 %$N_c \rightarrow$ & 2 & 4 & 8 & 16 & 2 & 4 & 2 \\
 \hline
 \hline
 \multicolumn{5}{|c|}{Llama2-7B (FP32 PPL = 5.06)} \\ 
 \hline
 \hline
 64 & 5.31 & 5.26 & 5.19 & 5.18  \\
 \hline
 32 & 5.23 & 5.25 & 5.18 & 5.15  \\
 \hline
 16 & 5.23 & 5.19 & 5.16 & 5.14  \\
 \hline
 \multicolumn{5}{|c|}{Nemotron4-15B (FP32 PPL = 5.87)} \\ 
 \hline
 \hline
 64  & 6.3 & 6.20 & 6.13 & 6.08  \\
 \hline
 32  & 6.24 & 6.12 & 6.07 & 6.03  \\
 \hline
 16  & 6.12 & 6.14 & 6.04 & 6.02  \\
 \hline
 \multicolumn{5}{|c|}{Nemotron4-340B (FP32 PPL = 3.48)} \\ 
 \hline
 \hline
 64 & 3.67 & 3.62 & 3.60 & 3.59 \\
 \hline
 32 & 3.63 & 3.61 & 3.59 & 3.56 \\
 \hline
 16 & 3.61 & 3.58 & 3.57 & 3.55 \\
 \hline
\end{tabular}
\caption{\label{tab:ppl_llama7B_nemo15B} Wikitext-103 perplexity compared to FP32 baseline in Llama2-7B and Nemotron4-15B, 340B models}
\end{table}

%\subsection{Perplexity achieved by various LO-BCQ configurations on MMLU dataset}


\begin{table} \centering
\begin{tabular}{|c||c|c|c|c||c|c|c|c|} 
\hline
 $L_b \rightarrow$& \multicolumn{4}{c||}{8} & \multicolumn{4}{c||}{8}\\
 \hline
 \backslashbox{$L_A$\kern-1em}{\kern-1em$N_c$} & 2 & 4 & 8 & 16 & 2 & 4 & 8 & 16  \\
 %$N_c \rightarrow$ & 2 & 4 & 8 & 16 & 2 & 4 & 2 \\
 \hline
 \hline
 \multicolumn{5}{|c|}{Llama2-7B (FP32 Accuracy = 45.8\%)} & \multicolumn{4}{|c|}{Llama2-70B (FP32 Accuracy = 69.12\%)} \\ 
 \hline
 \hline
 64 & 43.9 & 43.4 & 43.9 & 44.9 & 68.07 & 68.27 & 68.17 & 68.75 \\
 \hline
 32 & 44.5 & 43.8 & 44.9 & 44.5 & 68.37 & 68.51 & 68.35 & 68.27  \\
 \hline
 16 & 43.9 & 42.7 & 44.9 & 45 & 68.12 & 68.77 & 68.31 & 68.59  \\
 \hline
 \hline
 \multicolumn{5}{|c|}{GPT3-22B (FP32 Accuracy = 38.75\%)} & \multicolumn{4}{|c|}{Nemotron4-15B (FP32 Accuracy = 64.3\%)} \\ 
 \hline
 \hline
 64 & 36.71 & 38.85 & 38.13 & 38.92 & 63.17 & 62.36 & 63.72 & 64.09 \\
 \hline
 32 & 37.95 & 38.69 & 39.45 & 38.34 & 64.05 & 62.30 & 63.8 & 64.33  \\
 \hline
 16 & 38.88 & 38.80 & 38.31 & 38.92 & 63.22 & 63.51 & 63.93 & 64.43  \\
 \hline
\end{tabular}
\caption{\label{tab:mmlu_abalation} Accuracy on MMLU dataset across GPT3-22B, Llama2-7B, 70B and Nemotron4-15B models.}
\end{table}


%\subsection{Perplexity achieved by various LO-BCQ configurations on LM evaluation harness}

\begin{table} \centering
\begin{tabular}{|c||c|c|c|c||c|c|c|c|} 
\hline
 $L_b \rightarrow$& \multicolumn{4}{c||}{8} & \multicolumn{4}{c||}{8}\\
 \hline
 \backslashbox{$L_A$\kern-1em}{\kern-1em$N_c$} & 2 & 4 & 8 & 16 & 2 & 4 & 8 & 16  \\
 %$N_c \rightarrow$ & 2 & 4 & 8 & 16 & 2 & 4 & 2 \\
 \hline
 \hline
 \multicolumn{5}{|c|}{Race (FP32 Accuracy = 37.51\%)} & \multicolumn{4}{|c|}{Boolq (FP32 Accuracy = 64.62\%)} \\ 
 \hline
 \hline
 64 & 36.94 & 37.13 & 36.27 & 37.13 & 63.73 & 62.26 & 63.49 & 63.36 \\
 \hline
 32 & 37.03 & 36.36 & 36.08 & 37.03 & 62.54 & 63.51 & 63.49 & 63.55  \\
 \hline
 16 & 37.03 & 37.03 & 36.46 & 37.03 & 61.1 & 63.79 & 63.58 & 63.33  \\
 \hline
 \hline
 \multicolumn{5}{|c|}{Winogrande (FP32 Accuracy = 58.01\%)} & \multicolumn{4}{|c|}{Piqa (FP32 Accuracy = 74.21\%)} \\ 
 \hline
 \hline
 64 & 58.17 & 57.22 & 57.85 & 58.33 & 73.01 & 73.07 & 73.07 & 72.80 \\
 \hline
 32 & 59.12 & 58.09 & 57.85 & 58.41 & 73.01 & 73.94 & 72.74 & 73.18  \\
 \hline
 16 & 57.93 & 58.88 & 57.93 & 58.56 & 73.94 & 72.80 & 73.01 & 73.94  \\
 \hline
\end{tabular}
\caption{\label{tab:mmlu_abalation} Accuracy on LM evaluation harness tasks on GPT3-1.3B model.}
\end{table}

\begin{table} \centering
\begin{tabular}{|c||c|c|c|c||c|c|c|c|} 
\hline
 $L_b \rightarrow$& \multicolumn{4}{c||}{8} & \multicolumn{4}{c||}{8}\\
 \hline
 \backslashbox{$L_A$\kern-1em}{\kern-1em$N_c$} & 2 & 4 & 8 & 16 & 2 & 4 & 8 & 16  \\
 %$N_c \rightarrow$ & 2 & 4 & 8 & 16 & 2 & 4 & 2 \\
 \hline
 \hline
 \multicolumn{5}{|c|}{Race (FP32 Accuracy = 41.34\%)} & \multicolumn{4}{|c|}{Boolq (FP32 Accuracy = 68.32\%)} \\ 
 \hline
 \hline
 64 & 40.48 & 40.10 & 39.43 & 39.90 & 69.20 & 68.41 & 69.45 & 68.56 \\
 \hline
 32 & 39.52 & 39.52 & 40.77 & 39.62 & 68.32 & 67.43 & 68.17 & 69.30  \\
 \hline
 16 & 39.81 & 39.71 & 39.90 & 40.38 & 68.10 & 66.33 & 69.51 & 69.42  \\
 \hline
 \hline
 \multicolumn{5}{|c|}{Winogrande (FP32 Accuracy = 67.88\%)} & \multicolumn{4}{|c|}{Piqa (FP32 Accuracy = 78.78\%)} \\ 
 \hline
 \hline
 64 & 66.85 & 66.61 & 67.72 & 67.88 & 77.31 & 77.42 & 77.75 & 77.64 \\
 \hline
 32 & 67.25 & 67.72 & 67.72 & 67.00 & 77.31 & 77.04 & 77.80 & 77.37  \\
 \hline
 16 & 68.11 & 68.90 & 67.88 & 67.48 & 77.37 & 78.13 & 78.13 & 77.69  \\
 \hline
\end{tabular}
\caption{\label{tab:mmlu_abalation} Accuracy on LM evaluation harness tasks on GPT3-8B model.}
\end{table}

\begin{table} \centering
\begin{tabular}{|c||c|c|c|c||c|c|c|c|} 
\hline
 $L_b \rightarrow$& \multicolumn{4}{c||}{8} & \multicolumn{4}{c||}{8}\\
 \hline
 \backslashbox{$L_A$\kern-1em}{\kern-1em$N_c$} & 2 & 4 & 8 & 16 & 2 & 4 & 8 & 16  \\
 %$N_c \rightarrow$ & 2 & 4 & 8 & 16 & 2 & 4 & 2 \\
 \hline
 \hline
 \multicolumn{5}{|c|}{Race (FP32 Accuracy = 40.67\%)} & \multicolumn{4}{|c|}{Boolq (FP32 Accuracy = 76.54\%)} \\ 
 \hline
 \hline
 64 & 40.48 & 40.10 & 39.43 & 39.90 & 75.41 & 75.11 & 77.09 & 75.66 \\
 \hline
 32 & 39.52 & 39.52 & 40.77 & 39.62 & 76.02 & 76.02 & 75.96 & 75.35  \\
 \hline
 16 & 39.81 & 39.71 & 39.90 & 40.38 & 75.05 & 73.82 & 75.72 & 76.09  \\
 \hline
 \hline
 \multicolumn{5}{|c|}{Winogrande (FP32 Accuracy = 70.64\%)} & \multicolumn{4}{|c|}{Piqa (FP32 Accuracy = 79.16\%)} \\ 
 \hline
 \hline
 64 & 69.14 & 70.17 & 70.17 & 70.56 & 78.24 & 79.00 & 78.62 & 78.73 \\
 \hline
 32 & 70.96 & 69.69 & 71.27 & 69.30 & 78.56 & 79.49 & 79.16 & 78.89  \\
 \hline
 16 & 71.03 & 69.53 & 69.69 & 70.40 & 78.13 & 79.16 & 79.00 & 79.00  \\
 \hline
\end{tabular}
\caption{\label{tab:mmlu_abalation} Accuracy on LM evaluation harness tasks on GPT3-22B model.}
\end{table}

\begin{table} \centering
\begin{tabular}{|c||c|c|c|c||c|c|c|c|} 
\hline
 $L_b \rightarrow$& \multicolumn{4}{c||}{8} & \multicolumn{4}{c||}{8}\\
 \hline
 \backslashbox{$L_A$\kern-1em}{\kern-1em$N_c$} & 2 & 4 & 8 & 16 & 2 & 4 & 8 & 16  \\
 %$N_c \rightarrow$ & 2 & 4 & 8 & 16 & 2 & 4 & 2 \\
 \hline
 \hline
 \multicolumn{5}{|c|}{Race (FP32 Accuracy = 44.4\%)} & \multicolumn{4}{|c|}{Boolq (FP32 Accuracy = 79.29\%)} \\ 
 \hline
 \hline
 64 & 42.49 & 42.51 & 42.58 & 43.45 & 77.58 & 77.37 & 77.43 & 78.1 \\
 \hline
 32 & 43.35 & 42.49 & 43.64 & 43.73 & 77.86 & 75.32 & 77.28 & 77.86  \\
 \hline
 16 & 44.21 & 44.21 & 43.64 & 42.97 & 78.65 & 77 & 76.94 & 77.98  \\
 \hline
 \hline
 \multicolumn{5}{|c|}{Winogrande (FP32 Accuracy = 69.38\%)} & \multicolumn{4}{|c|}{Piqa (FP32 Accuracy = 78.07\%)} \\ 
 \hline
 \hline
 64 & 68.9 & 68.43 & 69.77 & 68.19 & 77.09 & 76.82 & 77.09 & 77.86 \\
 \hline
 32 & 69.38 & 68.51 & 68.82 & 68.90 & 78.07 & 76.71 & 78.07 & 77.86  \\
 \hline
 16 & 69.53 & 67.09 & 69.38 & 68.90 & 77.37 & 77.8 & 77.91 & 77.69  \\
 \hline
\end{tabular}
\caption{\label{tab:mmlu_abalation} Accuracy on LM evaluation harness tasks on Llama2-7B model.}
\end{table}

\begin{table} \centering
\begin{tabular}{|c||c|c|c|c||c|c|c|c|} 
\hline
 $L_b \rightarrow$& \multicolumn{4}{c||}{8} & \multicolumn{4}{c||}{8}\\
 \hline
 \backslashbox{$L_A$\kern-1em}{\kern-1em$N_c$} & 2 & 4 & 8 & 16 & 2 & 4 & 8 & 16  \\
 %$N_c \rightarrow$ & 2 & 4 & 8 & 16 & 2 & 4 & 2 \\
 \hline
 \hline
 \multicolumn{5}{|c|}{Race (FP32 Accuracy = 48.8\%)} & \multicolumn{4}{|c|}{Boolq (FP32 Accuracy = 85.23\%)} \\ 
 \hline
 \hline
 64 & 49.00 & 49.00 & 49.28 & 48.71 & 82.82 & 84.28 & 84.03 & 84.25 \\
 \hline
 32 & 49.57 & 48.52 & 48.33 & 49.28 & 83.85 & 84.46 & 84.31 & 84.93  \\
 \hline
 16 & 49.85 & 49.09 & 49.28 & 48.99 & 85.11 & 84.46 & 84.61 & 83.94  \\
 \hline
 \hline
 \multicolumn{5}{|c|}{Winogrande (FP32 Accuracy = 79.95\%)} & \multicolumn{4}{|c|}{Piqa (FP32 Accuracy = 81.56\%)} \\ 
 \hline
 \hline
 64 & 78.77 & 78.45 & 78.37 & 79.16 & 81.45 & 80.69 & 81.45 & 81.5 \\
 \hline
 32 & 78.45 & 79.01 & 78.69 & 80.66 & 81.56 & 80.58 & 81.18 & 81.34  \\
 \hline
 16 & 79.95 & 79.56 & 79.79 & 79.72 & 81.28 & 81.66 & 81.28 & 80.96  \\
 \hline
\end{tabular}
\caption{\label{tab:mmlu_abalation} Accuracy on LM evaluation harness tasks on Llama2-70B model.}
\end{table}

%\section{MSE Studies}
%\textcolor{red}{TODO}


\subsection{Number Formats and Quantization Method}
\label{subsec:numFormats_quantMethod}
\subsubsection{Integer Format}
An $n$-bit signed integer (INT) is typically represented with a 2s-complement format \citep{yao2022zeroquant,xiao2023smoothquant,dai2021vsq}, where the most significant bit denotes the sign.

\subsubsection{Floating Point Format}
An $n$-bit signed floating point (FP) number $x$ comprises of a 1-bit sign ($x_{\mathrm{sign}}$), $B_m$-bit mantissa ($x_{\mathrm{mant}}$) and $B_e$-bit exponent ($x_{\mathrm{exp}}$) such that $B_m+B_e=n-1$. The associated constant exponent bias ($E_{\mathrm{bias}}$) is computed as $(2^{{B_e}-1}-1)$. We denote this format as $E_{B_e}M_{B_m}$.  

\subsubsection{Quantization Scheme}
\label{subsec:quant_method}
A quantization scheme dictates how a given unquantized tensor is converted to its quantized representation. We consider FP formats for the purpose of illustration. Given an unquantized tensor $\bm{X}$ and an FP format $E_{B_e}M_{B_m}$, we first, we compute the quantization scale factor $s_X$ that maps the maximum absolute value of $\bm{X}$ to the maximum quantization level of the $E_{B_e}M_{B_m}$ format as follows:
\begin{align}
\label{eq:sf}
    s_X = \frac{\mathrm{max}(|\bm{X}|)}{\mathrm{max}(E_{B_e}M_{B_m})}
\end{align}
In the above equation, $|\cdot|$ denotes the absolute value function.

Next, we scale $\bm{X}$ by $s_X$ and quantize it to $\hat{\bm{X}}$ by rounding it to the nearest quantization level of $E_{B_e}M_{B_m}$ as:

\begin{align}
\label{eq:tensor_quant}
    \hat{\bm{X}} = \text{round-to-nearest}\left(\frac{\bm{X}}{s_X}, E_{B_e}M_{B_m}\right)
\end{align}

We perform dynamic max-scaled quantization \citep{wu2020integer}, where the scale factor $s$ for activations is dynamically computed during runtime.

\subsection{Vector Scaled Quantization}
\begin{wrapfigure}{r}{0.35\linewidth}
  \centering
  \includegraphics[width=\linewidth]{sections/figures/vsquant.jpg}
  \caption{\small Vectorwise decomposition for per-vector scaled quantization (VSQ \citep{dai2021vsq}).}
  \label{fig:vsquant}
\end{wrapfigure}
During VSQ \citep{dai2021vsq}, the operand tensors are decomposed into 1D vectors in a hardware friendly manner as shown in Figure \ref{fig:vsquant}. Since the decomposed tensors are used as operands in matrix multiplications during inference, it is beneficial to perform this decomposition along the reduction dimension of the multiplication. The vectorwise quantization is performed similar to tensorwise quantization described in Equations \ref{eq:sf} and \ref{eq:tensor_quant}, where a scale factor $s_v$ is required for each vector $\bm{v}$ that maps the maximum absolute value of that vector to the maximum quantization level. While smaller vector lengths can lead to larger accuracy gains, the associated memory and computational overheads due to the per-vector scale factors increases. To alleviate these overheads, VSQ \citep{dai2021vsq} proposed a second level quantization of the per-vector scale factors to unsigned integers, while MX \citep{rouhani2023shared} quantizes them to integer powers of 2 (denoted as $2^{INT}$).

\subsubsection{MX Format}
The MX format proposed in \citep{rouhani2023microscaling} introduces the concept of sub-block shifting. For every two scalar elements of $b$-bits each, there is a shared exponent bit. The value of this exponent bit is determined through an empirical analysis that targets minimizing quantization MSE. We note that the FP format $E_{1}M_{b}$ is strictly better than MX from an accuracy perspective since it allocates a dedicated exponent bit to each scalar as opposed to sharing it across two scalars. Therefore, we conservatively bound the accuracy of a $b+2$-bit signed MX format with that of a $E_{1}M_{b}$ format in our comparisons. For instance, we use E1M2 format as a proxy for MX4.

\begin{figure}
    \centering
    \includegraphics[width=1\linewidth]{sections//figures/BlockFormats.pdf}
    \caption{\small Comparing LO-BCQ to MX format.}
    \label{fig:block_formats}
\end{figure}

Figure \ref{fig:block_formats} compares our $4$-bit LO-BCQ block format to MX \citep{rouhani2023microscaling}. As shown, both LO-BCQ and MX decompose a given operand tensor into block arrays and each block array into blocks. Similar to MX, we find that per-block quantization ($L_b < L_A$) leads to better accuracy due to increased flexibility. While MX achieves this through per-block $1$-bit micro-scales, we associate a dedicated codebook to each block through a per-block codebook selector. Further, MX quantizes the per-block array scale-factor to E8M0 format without per-tensor scaling. In contrast during LO-BCQ, we find that per-tensor scaling combined with quantization of per-block array scale-factor to E4M3 format results in superior inference accuracy across models. 


%%%%%%%%%%%%%%%%%%%%%%%%%%%%%%%%%%%%%%%%%%%%%%%%%%%%%%%%%%%%%%%%%%%%%%%%%%%%%%%
%%%%%%%%%%%%%%%%%%%%%%%%%%%%%%%%%%%%%%%%%%%%%%%%%%%%%%%%%%%%%%%%%%%%%%%%%%%%%%%


\end{document}


% This document was modified from the file originally made available by
% Pat Langley and Andrea Danyluk for ICML-2K. This version was created
% by Iain Murray in 2018, and modified by Alexandre Bouchard in
% 2019 and 2021 and by Csaba Szepesvari, Gang Niu and Sivan Sabato in 2022.
% Modified again in 2023 and 2024 by Sivan Sabato and Jonathan Scarlett.
% Previous contributors include Dan Roy, Lise Getoor and Tobias
% Scheffer, which was slightly modified from the 2010 version by
% Thorsten Joachims & Johannes Fuernkranz, slightly modified from the
% 2009 version by Kiri Wagstaff and Sam Roweis's 2008 version, which is
% slightly modified from Prasad Tadepalli's 2007 version which is a
% lightly changed version of the previous year's version by Andrew
% Moore, which was in turn edited from those of Kristian Kersting and
% Codrina Lauth. Alex Smola contributed to the algorithmic style files.
