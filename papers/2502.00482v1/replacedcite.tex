\section{Related Work}
Previous research explored various aspects of microservice granularity and its impact on system performance.
____ investigated the impact of microservice granularity on performance and resource utilization in cloud applications.
Their experiment showed that separating heavily loaded components into distinct microservices improved response times and reduced CPU usage under high load.

Similarly, ____ investigated the relationship between microservice granularity and performance.
They compared microservices deployed in a single container versus those partitioned across separate containers.
Their results suggest that services co-located in the same container can benefit from finer granularity, while distributed ones should be designed with coarse granularity for resilience.

Using Service Weaver, which allows developing applications as modular monoliths while deploying them as distributed systems of varying granularity, ____ evaluated different decompositions of an open-source system.
They concluded that granularity levels beneficial for maintainability had a negative impact on performance.

____ proposed \enquote{microservice ambients}, an architecture meta-modeling approach to address optimal microservice granularity.
Their work extends aspect-oriented ambient modeling with a granularity adaptation aspect.
They demonstrated how the approach supports architecture evolution and runtime analysis.
In a follow-up paper____, the authors also extended their approach into a runtime adaptation framework to dynamically evaluate different granularity decisions.
While they consider the generic \textit{architectural value} of the different granularity levels, energy consumption is not studied.

Beyond granularity considerations, some researchers have specifically addressed energy efficiency in microservices deployments.
____ introduced Elergy, a proactive elasticity model for microservices applications that aims to reduce energy consumption while maintaining performance.
Using ARIMA time series forecasting to predict CPU usage for microservice scaling, their model achieved energy savings of 1.93\% to 27.92\% compared to non-elastic scenarios.

____ empirically evaluated the energy and performance overhead of monitoring tools on Docker-based microservices.
Their experiments with the Train Ticket benchmark system revealed strong correlations between CPU usage, CPU load, execution time, and energy consumption.
The study provides valuable insights on the trade-offs between monitoring capabilities and system efficiency in microservices.

____ conducted a comprehensive energy consumption analysis of Docker-based Linux distributions for IoT devices.
Using a microservice-based benchmarking architecture with hardware power sensors, they compared the energy efficiency of different operating systems for containerized workloads.
Their results revealed variations in energy consumption patterns, with some systems like RancherOS showing lower power draw but longer task completion times. 

Recent research has also explored hardware approaches to energy optimization.
____ introduced SIMR (Single Instruction Multiple Request), a novel approach for processing microservices that combines GPU-like SIMT execution with out-of-order CPU capabilities.
By executing similar microservice requests, their Request Processing Unit (RPU) achieves 5.7x better efficiency than traditional CPUs.
Their evaluation can significantly reduce both frontend and memory-system energy consumption.

While these studies provide valuable insights into various related aspects, the relationship between granularity and energy consumption remains understudied.
Our research aims to bridge this gap by investigating how different levels of granularity impact both energy consumption and performance.