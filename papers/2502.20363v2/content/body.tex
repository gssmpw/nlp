\paragraph{\textbf{\textit{Introduction.}---}}The study of nuclear properties and the interactions that govern them is critical to understanding how complex many-body systems such as nuclei, atoms, and molecules emerge from the fundamental forces and particles of nature. At the most fundamental level, the nuclear force is described by Quantum Chromodynamics (QCD), the theory of strong interactions. Due to the non-perturbative complexity of QCD in the low-energy regime, effective interactions derived from Chiral Effective Field Theory ($\chi$EFT)~\cite{Epelbaum:2009,Machleidt:2011} are commonly employed to describe the interactions between neutrons and protons inside the atomic nucleus.
$\chi$EFT provides a low-energy expansion of QCD in terms of nucleon-nucleon and three-nucleon interactions, with undetermined Low-Energy Constants (LECs) adjusted to reproduce few-nucleon observables. However, nuclear properties can be highly sensitive to the value of these LECs~\cite{Hergert:2014, Hebeler:2015, 
Lapoux:2016, Garcia_Ruiz_2016, Hagen:2016, Simonis:2017, Duguet:2017, Lu:2019, Liu:2019, Gysbers:2019, Soma:2020PRC, Belley:2022, Ekström:2023, Arthuis:2024, Belley2024prl}. Therefore, it is crucial to understand the impact of the LECs on nuclear observables and how their uncertainties propagate through \emph{ab initio} many-body calculations.

\emph{Ab initio} methods~\cite{BARRETT:2013,hagen2014coupled, Stroberg:2017, Yao:2020, Soma:2020, Hergert:2020, Tichai:2020, Tichai:2023} rely on systematically improvable approximations to solve the many-body Schrödinger equation, starting from $\chi$EFT interactions.  
Despite major advances in recent years, \emph{ab initio} nuclear calculations of medium and heavy nuclei remain prohibitively expensive due to the exponential scaling of many-body wave functions. While approximate polynomially-scaling methods (e.g.~\cite{Stroberg:2017,hagen2014coupled}) have been developed, this complexity still imposes stringent resource demands and methodological challenges~\cite{Morris:2018,Vernon2022,Karthein2024}. 
Statistical procedures, such as global sensitivity studies, are necessary to evaluate nuclear observables for samples in the range of millions, even for the simplest realistic nuclear interactions ~\cite{Ekstrom:2019}.
As a result, surveying the entire nuclear landscape with realistic uncertainty estimation and establishing a connection to the LECs remains an outstanding challenge.

Nuclear emulators have emerged as powerful tools to address these problems, offering efficient approximations of many-body calculations at substantially reduced computational cost and allowing for uncertainty quantification~\cite{Boehnlein:2022}. Techniques such as multi-output multi-fidelity Deep Gaussian Processes~\cite{Belley2024prl, Belley_thesis}, eigenvector continuation~\cite{Frame:2018,Ekstrom:2019, Drischler2023, Duguet:2024, Jiang:2024} and many-more (e.g.~\cite{Yoshida2023, Giuliani2024, Lay2024}) have been proposed as surrogate models that approximate the outputs of nuclear calculations with significantly reduced computational costs. However, these methods are so far limited to individual isotopes, restricting their ability to capture correlations and trends across isotopic chains and more globally. Moreover, they still demand costly \emph{ab initio} calculation train them.

To overcome the challenges mentioned above, we introduce BAyesian Neural Network for Atomic Nuclei Emulation (BANNANE), a hierarchical Bayesian neural network (BNN) framework that integrates multi-fidelity datasets with categorical and positional embeddings to emulate nuclear properties globally. This approach introduces a flexible architecture tailored for the prediction of different nuclear properties with vastly reduced computational time.
We demonstrate the overarching capabilities of BANNANE by calculating the nuclear binding energies and charge radii of the oxygen isotopic chain, achieving accurate predictions of these observables. Notably, BANNANE enables a global analysis of the sensitivity of these observables to the LECs used in the interactions. This allows us to directly study how macroscopic nuclear properties are impacted by specific details of the nuclear force, and how these sensitivities vary along isotopic chains, which, to our knowledge, is not possible with current nuclear emulators.

%Furthermore, we showcase its strong performance in both interpolation and extrapolation, significantly enhancing the predictive power of nuclear emulators.

\begin{figure*}[t]
    \centering
    \includegraphics[width=0.9\linewidth]{figures/bannane_diagram_deltas_shared_emax.pdf}
    \caption{\textbf{BANNANE architecture overview.} The input LECs, along with categorical embeddings for proton number (\(Z\)), neutron number (\(N\)), and fidelity level (\(e_{\text{max}}\)), are processed through a Bayesian shared-latent layer and a multi-head attention mechanism. Fidelity-specific prediction heads refine the base prediction at higher \(e_{\text{max}}\) for each fidelity ranging from 2, outputting mean (\(\mu_N^f\)) and uncertainty (\(\sigma_N^f\)) estimates for nuclear observables.}
    \label{fig:diagram_bannane}
\end{figure*}

\paragraph{\textbf{\textit{Methods.}---}} We construct a multi-fidelity Bayesian emulator using a \emph{hierarchical} architecture tailored for global nuclear modeling. Our pipeline is divided into three main stages: (i) \emph{data loading and preprocessing}, (ii) \emph{hierarchical BNN construction}, and (iii) \emph{variational training and inference}.

%%%%%%%%Data generation and preprocessing:
\textit{Data generation and preprocessing:} LEC samples utilized in this work are taken from Ref.~\cite{Jiang:2024}, where history matching was used to constrain the LECs to physically plausible ones. More specifically, these samples are for a formulation of $\chi$EFT where $\Delta$-isobars are considered explicitly, up to next-to-next-to leading order (N2LO) in the chiral expansion. In particular, 17 LECs are required to parametrize this theory. We construct data sets for training, validation, and testing randomly from the 8188 samples given in Ref.~\cite{Jiang:2024}.
We employ the Valence-Space formulation of the In-Medium Similarity Renormalization Group (VS-IMSRG)~\cite{Stroberg:2017, Stroberg:2019} to solve the many-body problem and obtain the ground-state energies and charge radii of the different isotopes. In practice, the decoupling of the valence space is done using the \texttt{imsrg++} code~\cite{Stroberg_IMSRG_2018}, and the valence space is then diagonalized using the \texttt{KSHELL} shell-model code~\cite{Shimizu2019}. In particular, we use calculations with different model space sizes, given by including all harmonic oscillator states with $e = 2n+l \leq e_{\rm max}$ where $n$ and $l$ are the principal quantum number and angular momentum of the wave functions respectively. We note that increasing $e_{\rm max}$ results in a more accurate calculation, at the cost of a rapid rise in computational power required to solve the problem. We therefore consider the values of the observables at $e_{\rm max}\in \{4,6,8,10\}$ to constitute different fidelities in the context of the emulator. 

%%%%%%%Hierarchical Bayesian neural network
\textit{Hierarchical BNN:} 
Figure~\ref{fig:diagram_bannane} presents a diagrammatic representation of 
the multi-fidelity architecture used in BANNANE. Our core model uses distinct embeddings for the proton number $Z$, and neutron number $N$, via a learnable positional encoding, and $e_{\text{max}}$ of each sample. These embeddings are concatenated with the LECs, then passed through a \emph{shared} multi-head self-attention layer that captures correlations across nuclei and is queried for each fidelity. 
To accommodate multiple fidelities, we combine a \emph{base} predictor (representing the lowest-$e_{\text{max}}$ dataset) with additive \emph{delta} blocks for each higher-$e_{\text{max}}$ level. Concretely, the network first produces a common latent representation, which is mapped to predictions via (i) a base model for the lowest $e_{\text{max}}$ and (ii) incrementally learned offsets for any higher-$e_{\text{max}}$ data present. This ensures consistent cross-$e_{\text{max}}$ learning, allowing the emulator to leverage coarse information from lower-$e_{\text{max}}$ calculations and refine it where higher-$e_{\text{max}}$ data exist. 


%%%%%%%%%Variational training and inference
\textit{Variational training and inference:} We implement the hierarchical BNN with Pyro’s stochastic variational inference (SVI)~\cite{bingham2019pyro}, employing a diagonal Gaussian variational posterior. All weights, biases, and $e_{\text{max}}$-specific parameters (including output variance terms) are learned by minimizing evidence lower bound (ELBO)~\cite{hoffman2016elbo}. After each SVI iteration, the validation loss is monitored for early stopping. During inference, our posterior predictive distribution is sampled for each $(Z, N)$ and $e_{\text{max}}$. This yields both mean predictions and standard deviations, thus quantifying uncertainties across isotopic chains. By design, BANNANE can \emph{extrapolate} to new or sparse regions of the nuclear chart simply by evaluating the learned embeddings of $(Z, N)$. 
Further implementation details and additional ablation studies are provided in the supplemental material~\cite{SupplementalMaterial}.

\paragraph{\textbf{\textit{Results.}---}}In this section, we demonstrate BANNANE’s performance under a range of conditions:

\textit{Performance on the oxygen isotopic chain:} We assess BANNANE on the oxygen isotopic chain ($^{12}$O -- $^{24}$O) including up to $e_{\text{max}}=10$ for all the isotopes. Figure~\ref{fig:full_oxygen_performance} displays the predicted ground-state energies at \( e_{\text{max}}=10\) for all oxygen isotopes, compared to the reference IMSRG calculations. 

\begin{figure}[h!]
    \centering
    \includegraphics[width=1
    \linewidth]{figures/predictions_vs_true_energy_radii.pdf}
    \caption{\label{fig:full_oxygen_performance}
    \textbf{Full-chain training on the oxygen isotopes.}
    BANNANE predictions versus IMSRG reference results for test samples at the highest fidelity \(e_{\text{max}}=10\), for \(E_{B}\) and $R_{ch}$. Error bars indicate \(1\sigma\) uncertainty from BANNANE's posterior, and the average Root Mean Squared Error (RMSE) is displayed for each one.
    }
\end{figure}

We observe that BANNANE correctly reproduces both total binding energies $E_B$ and charge radii $R_{ch}$ across the chain with 0.8 MeV and 0.01 fm Root Mean Squared Error (RMSE), respectively. This represents a significant improvement compared to the Eigenvector Continuation-based subspace-projected coupled-cluster~\cite{Ekstrom:2019} which obtains RMSEs of 3 MeV and 0.02 fm for the ground state energies and charge radii respectively in $^{16}$O using 128 training points. While BANNANE uses more training points in total for this isotope (715  in total), only 51 samples are of the highest fidelity, making BANNANE much more computationally efficient to train, while still achieving an RMSE of 0.337 MeV and 0.02 fm, respectively, for the same isotope.
Further study of the residuals and benchmark is provided in the supplemental material~\cite{SupplementalMaterial}.

Moreover, we note that the discontinuity caused by the shell closure at $N=8$ is well captured by BANNANE.  To further study this phenomenon, we performed a dimensional reduction to perform an analysis of the latent space of the model after the attention mechanism using a projection t-distributed Stochastic Neighbor Embedding (t-SNE) dimensionality reduction~\cite{van2008visualizing} for visualization. As illustrated in Fig.~\ref{fig:tsne_attention_output_by_N}, there is an evident clustering of the isotopes in the $sd$ shell, after the shell closure at $N=8$, whereas the isotopes in the $p$ shell appear to be mapped to distinct clusters, hinting of the model's ability to capture an underlying nuclear structure.

\begin{figure}[!h]
    \centering
    \includegraphics[width=1\linewidth]{figures/tsne_attention_output_by_N.pdf}
    \caption{\textbf{Learnt embedding space.} Projection of the attention map to 2 dimensions using t-distributed Stochastic Neighbor Embedding (t-SNE) dimensionality reduction applied to the latent space for test LEC samples. 
    The line and colored background show the decision boundary of a simple linear classifier to the reduced space to illustrate the separation between shells.
    \label{fig:tsne_attention_output_by_N}}
    
\end{figure}



%%%%%%%Zero-Shot isotope Interpolation
\textit{Zero-shot extrapolation\label{sec:zero_shot}}: A major advantage of BANNANE is its ability to make reliable predictions for isotopes beyond those included in the training. BANNANE’s architecture enables true zero-shot generalization, meaning it can infer nuclear properties of isotopes it has never seen before, solely from learned trends across isotopic chains.
 To illustrate this, we simultaneously withheld all the data corresponding to $^{15}$O samples and evaluated the model's regression. 

Figure~\ref{fig:zero_shot_interpolation} (top) shows the relative errors (as a percentage) of $E_B$ once all the samples for this isotope are left out of the training set. Interestingly, the inductive biases on the hierarchical architecture are such that the model can capture both convergence and nuclear structure even for unseen isotopes, as shown by the small residuals at each fidelity. Besides the noticeable shift in the regressed values, it is still surprising that no samples for this isotope were provided. We note that the performance is weaker for the charge radii, especially near the shell closure as discussed along with further analysis for extrapolation over the complete chain in the supplemental material~\cite{SupplementalMaterial}.


This fit can be improved significantly when including data only from the lower fidelity, namely from $e{\rm max} = 4$, as illustrated in Figure~\ref{fig:zero_shot_interpolation} (bottom). We find that including the lower fidelity greatly reduces the spread of the residuals, reduces the systematic bias, and makes the distribution closer to a Gaussian, making for much better-behaved residuals on all fronts. 


\begin{figure}[!h]
    \centering
    \includegraphics[width=1
    \linewidth]{figures/relative_error_eb_N_7.pdf} \caption{\textbf{Extrapolation to $^{15}$O.} (a) Residual distribution of the binding energy $E_B$ (\%) for BANNANE predictions compared to IMSRG reference values at $e_{\text{max}} = 6, 8, 10$, when no training samples for $^{15}$O were included. (b) Residual distribution for $E_B$ predictions when only low-fidelity data ($e_{\text{max}} = 4$) was used in training. The inclusion of low-fidelity data significantly reduces systematic bias and improves the overall accuracy of the extrapolated predictions.
\label{fig:zero_shot_interpolation}}
\end{figure}

This extrapolation feature shows great advantages for future applications to heavy systems where high-precision calculations of many LEC samples are infeasible for an entire isotopic chain. In practice, one could choose to compute $e_{\text{max}}=10$ results only for a small \emph{strategic} set of isotopes (e.g., a few in mid-shell regions where higher-order correlations are most relevant and around shell-closure where discontinuity are more likely to happen) and low-fidelity data for the rest of the isotopes. 

Crucially, the Bayesian nature of BANNANE captures the remaining uncertainties due to limited high-fidelity coverage. 
These uncertainty estimates can inform experimental proposals or guide more selective future high-fidelity calculations, focusing effort on regions where the emulator’s confidence is lowest. This phenomenon can be further studied by analyzing how the model's emulation residual converges as more high-fidelity training data is included, for which results are presented in the Result section of the supplemental material~\cite{SupplementalMaterial}.


\begin{figure}[h]
    \centering
    \includegraphics[width=\linewidth]{figures/energy_predictions.pdf}
    \caption{Emulator-driven convergence of binding energies $E_B$ for oxygen isotopes as a function of $e_\text{max}$. Results are compared to fully converged VS-IMSRG results~\cite{Stroberg:2021} at $e_\text{max} =14$ using the EM(1.8/2.0) nuclear interaction in purple squares. Solid lines represent experimental results from \cite{wang2021ame}. \label{fig:energy_predictions}}
\end{figure}
\textit{Physical Convergence:}
BANNANE not only interpolates and extrapolates nuclear observables but also emulates the physical convergence of many-body methods. Figure~\ref{fig:energy_predictions} shows the emulator's predictions for \(E_B\) across the oxygen chain as a function of \(e_{\text{max}}\), compared to experimental values after doing a weighted resampling of 8188 LEC samples~\cite{Jiang:2024} as done in Ref.~\cite{Belley2024prl}. The emulator accurately reproduces convergence trends, including nontrivial shell effects, with deviations well within its uncertainty estimates, and in good agreement with previous results~\cite{Lapoux_2016}. Notably, these uncertainties stem solely from the emulator and the LECs, not from many-body truncation effects. This highlights BANNANE’s ability to capture systematic trends while reducing computational costs, which in practice could take several years on HPC clusters to just a few seconds for the resampling. 

\begin{figure*}[t!]
    \centering
    \includegraphics[width=1\linewidth]{figures/sobol_sensitivity.pdf}
    \caption{\textbf{Variance-based global sensitivity analysis for the oxygen isotopes.} 
Shown are the main-effect indices $S_i$ (colored bars) and total-effect indices $S_T$ (black outline) for each LEC, indicating their direct and combined contributions to the variance of the binding energy $E_B$ (top) and charge radius $R_{ch}$ (bottom). Notable differences between $S_i$ and $S_T$ point to strong interactions among particular LECs.
}
    \label{fig:sobols}
\end{figure*}

\textit{Unified Sensitivity Estimation:}
Beyond providing fast and accurate emulation, BANNANE enables global sensitivity analysis (GSA) that would be computationally prohibitive using brute-force \emph{ab initio} methods. With this method, we can probe how much the variance of each LEC (and their covariance) impacts the variance of the observables.
In Fig.~\ref{fig:sobols}, we apply a variance-based Sobol approach~\cite{sobol2001global} to assess how each LEC affects $E_B$ and $R_{ch}$ across oxygen isotopes from $^{12}$O to $^{24}$O. 
We plot the main-effect index $S_i$ (filled bars), capturing the direct contribution of each LEC to the variance of an observable, and the total-effect index $S_T$ (white outlines), accounting for higher-order interactions (see supplemental material~\cite{SupplementalMaterial} for more details.).

A few features stand out: For binding energies (top panel), certain two-nucleon (2N) couplings (e.g., $C_{1S0}$ and $c_2$) and three-nucleon (3N) couplings (e.g., $C_D$, $C_E$) dominate the variance while showing small progressive changes across the chain. This is consistent with findings of previous methods~\cite{Ekstrom:2019, Ekström:2023,Belley2024prl} in multiple nuclei. It indicates that the dependency of the ground state energy to the LEC exhibits small variations in different nuclear systems. In contrast, the hierarchy of LEC sensitivities differs considerably for the charge radii (bottom panel): large changes can be observed at the shell-closure, with for example $C_{3P2}$ being important and $C_{E1}$ being negligible in the sd-shell but the opposite holding true in the p-shell. Moreover, the sensitivities for $R_{ch}$ are inherently more nonlinear and show the emerging importance of cross-couplings rather than individual influences. This highlights that the charge radius can be highly sensitive to certain LECs, making it strongly complementary to the experimental constraints derived from binding energies. In particular, experimental data on neutron-deficient oxygen isotopes, which have not yet been measured, will be crucial for guiding these developments and benchmarking our predictions.

\paragraph{\textbf{\textit{Conclusions.}---}}

We have introduced a hierarchical multi-fidelity Bayesian Neural Network for Atomic Nuclei Emulation (\textbf{BANNANE}), which synthesizes low- and high-fidelity datasets across different nuclei. By leveraging learnable embeddings for nucleon numbers and an additive fidelity-specific architecture, BANNANE not only interpolates faithfully among known nuclei, achieving better accuracy than current emulators but also extrapolates with remarkable accuracy to hitherto uncalculated (or unobserved) nuclei. Due to this, BANNANE offers a computationally feasible way to calculate nuclear properties with realistic uncertainties and with an accuracy comparable to those obtained by considerably costly \emph{ab initio} methods. These developments have enabled us to investigate the overall dependence of nuclear observables on the LECs that govern inter-nucleon interactions. We found that the nuclear binding energy and charge radii of oxygen isotopes exhibit distinct sensitivities to these LECs, with particular interest in data from neutron-deficient oxygen isotopes that have yet to be measured. We hope our findings will motivate charge radius measurements of these isotopes, which are now feasible via laser spectroscopy experiments \cite{Yan23} at the new Facility for Rare Isotope Beams (FRIB) \cite{FRIB25}. More broadly, we anticipate that BANNANE will serve as a powerful tool to guide future experiments and establish direct connections between measurements of binding energy, charge radii, and specific components of the nuclear force.

Beyond its computational advantages, BANNANE provides a powerful framework to guide future experimental and theoretical efforts. It enables targeted high-precision calculations and experimental searches where they matter most, optimizing resource allocation for both theory and experiments. Its versatile architecture can be used with any many-body method and can include as many observables as desired. While we have only varied the neutron number in this study, BANNANE architecture also allows the emulation of different proton numbers, which we will test in future work.

\paragraph{\textbf{\textit{Data Availability.}---}}
The data for reproducing the results in this work can be found online~\cite{munozariasjm_paper_o_bannane} along with scripts used in the data analysis. Additionally, the source code for training the model is available upon request.

\begin{acknowledgments}

We thank A. Ekström, C. Forssén, G. Hagen, and W. G. Jiang for providing the interaction samples used in this work, and J. Holt for insightful discussions. The IMSRG code used in this work makes use of the Armadillo \texttt{C++} library \cite{Sanderson2016, Sanderson2018}.  Computational resources were provided by subMIT at MIT Physics. This work was supported by the Office of Nuclear Physics, U.S. Department of Energy, under grants DESC0021176 and DE-SC0021179. We acknowledge the support of the Natural Sciences and Engineering Research Council of Canada (NSERC) [PDF-587464-2024].

\end{acknowledgments}