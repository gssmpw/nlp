\section{Related Work}
\paragraph{Modeling Infectious Disease Spread.} 
The classical SIR model, introduced in~\cite{kermack1927contribution}, is a widely used dynamical system for modeling disease evolution, assuming deterministic and homogeneous dynamics across the entire population. To account for uncertainty in disease spread, various stochastic extensions of the SIR model have been developed~\citep{bartlett1956deterministic,bailey1975mathematical,beretta1998stability,andersson2012stochastic,britton2010stochastic,kiss2017mathematics,cai2015stochastic,cai2017stochastic,karako2020analysis,martcheva2015introduction,allen2017primer,britton2019stochastic,zhou2021ergodic,laaribi2023generalized,greenwood2009stochastic}. \cite{Aurell_Carmona_Dayanıklı_Lauriere_2022b} explore a graphon game of epidemic control, with one special case involving multi-population modeling that considers different age and city groups. Their model examines the effects of different policies when individuals choose their own socialization level (control) in the game setup. However, their approach is based on an individualized version of \textit{multi-group SIR} whereas our framework focuses on the group level. In \cite{acemoglu2021optimal}, a non-homogeneous (i.e., multi-group) version of the deterministic SIR model is considered, incorporating different age groups---young, middle-aged, and old---and deriving age-specific policies to minimize overall costs. In this paper, we extend their framework to the stochastic counterpart. In addition, while the motivation behind their multi-group approach is to derive group-specific policies, our method also considers fairness, which, to the best of our knowledge, has not been explored in existing work.
 
\paragraph{Fairness in AI for Social Decision Making.} With the advancement of AI, automated decision-making and policymaking have become increasingly prevalent in critical social domains such as college admissions, loan approvals, and criminal justice. However, these AI-driven systems have raised many concerns, as they may not be entirely objective and can even exacerbate existing human biases~\citep{dastin2022amazon,angwin2022machine,samorani2022overbooked}. 
Many studies have explored fairness considerations in AI-driven decision-making across various applications. For example, \cite{azizi2018designing} study the problem of housing allocations for homeless youth and propose fair, efficient, and interpretable policies. In the context of micro-lending, \cite{liu2019personalized}~develop a fairness-aware re-ranking algorithm that balances recommendation accuracy with borrower-side fairness while also accounting for lenders’ preferences for diversity. Similarly, \cite{berk2021fairness}~provide an integrated examination of fairness and accuracy trade-offs in criminal justice risk assessments, demonstrating the inherent challenges in satisfying multiple fairness criteria simultaneously. Furthermore, \cite{kallus2022assessing}~assess disparities in lending and healthcare applications when the protected class membership is not observed in the data. They provide exact characterizations of the tightest possible set of all true disparities that are consistent with the available data. Beyond these studies, many other works examine the broader challenges and trade-offs involved in designing fair AI systems across various domains~\citep{mouzannar2019fair,corbett2023measure, kleinberg2017inherent,dai2025balancing,aghaei2019learning,raghavan2020mitigating,chen2023algorithmic,jia2024learning,wang2024wasserstein,liu2018delayed, baker2022algorithmic,taskesen2020distributionally,bertsimas2013fairness,nguyen2021scarce,corbett2017algorithmic,rahmattalabi2022learning,mashiat2022trade,rahmattalabi2021fair,freeman2020best,athanassoglou2011house}. 

\paragraph{Path Integral Control.}
Path integral control has emerged as a promising solution scheme for solving a certain class of nonlinear stochastic optimal control problems~\citep{kappen2005path}. It has recently been applied in many reinforcement learning domains, including autonomous driving~\citep{mohamed2022autonomous,williams2016aggressive,gandhi2021robust,ha2019topology}, robotics~\citep{theodorou2010generalized,chebotar2017path,williams2017information,yin2023risk,park2024distributionally,patil2022chance}, visual serving techniques~\citep{mohamed2021sampling,costanzo2023modeling,mohamed2021mppi} and finance~\citep{ingber2000high,decamps2006path,perkowski2016pathwise}. The key idea behind path integral control is to convert the value function into an expectation over uncontrolled trajectory costs. Therefore, it does not involve any optimization processes that can be intractable for solving nonlinear stochastic control problems. Instead, the method generates independent trajectories through Monte Carlo sampling and computes their associated expected costs. Furthermore, since the trajectories are independent, various parallelization techniques can be applied to significantly speed up the computation process~\citep{williams2017model}.

\paragraph{Notations.} 
Bold lower-case letter $\bm{x}\in\RR^n$ and upper-case letter $\boldsymbol{X}\in\RR^{n\times m}$ represent an $n$-dimensional vector and an $n\times m$ matrix, respectively.
$\bm{X}\in\mathbb{S}^n_{+}$ denotes an $n\times n$ positive semidefinite matrix. We define diag$(\bm{x})$ as a diagonal matrix with the vector $\bm{x}$ on its main diagonal. Similarly, diag$(\bm{X}_1,\ldots,\bm{X}_J)$ denotes a block diagonal matrix of matrices $\bm{X}_1,\ldots,\bm{X}_J$. For any $K \in \mathbb N$, we define $[K]$ as the index set $\{1,\dots,K\}$.