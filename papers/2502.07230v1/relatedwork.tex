\section{Literature Review and Research Gap}
Previous studies have explored the parameter estimation of gas pipeline systems, which is critical for their efficient and safe operation. 
These methods can be broadly classified into several categories: optimization-based methods, Kalman filtering techniques, neural network models, and other miscellaneous approaches.

Optimization-based approaches aim to identify parameters by minimizing an objective function that quantifies error. 
For instance, a reduced-order model was developed in \cite{sundarDynamicStateParameter2019} and \cite{sundarStateParameterEstimation2019}, where nonlinear programming and a least squares objective function were employed to estimate both the dynamic states and parameters of gas pipeline networks. 
Similarly, maximum likelihood and least squares methods were utilized to identify parameters in \cite{chenIdentificationCharacteristicParameters2022}.
A Bayesian approach to parameter estimation, specifically targeting the friction coefficient in gas networks, was proposed in \cite{hajianBayesianApproachParameter2019}. 
Additionally, \cite{sukharevIdentificationModelFlow2021} employed a lumped-parameter model under both stationary and non-stationary gas flows to estimate pipeline parameters. 
The work in \cite{cuiDatadrivenComputationNatural2020} presented an optimization problem to determine the steady-state of gas networks, where a neural network was used to initialize the optimization process and expedite its solution.

Neural networks have also been explored extensively for parameter estimation, capitalizing on their ability to handle dynamic processes in gas systems. 
A gray-box neural network model was presented in \cite{cenGrayboxNeuralNetworkbased2013} to identify faults within gas systems. 
Long Short-Term Memory (LSTM) models, known for their proficiency in sequence-based data, were applied to predict pipeline shutdown pressures in \cite{zhengDeeppipeTheoryguidedLSTM2021}. 
In addition, a shortcut Elman network was proposed in \cite{zhouDynamicSimulationNatural2022} for dynamic simulations of gas networks.

Kalman filtering is another effective method, particularly for dealing with noisy measurements in parameter estimation. 
For example, \cite{chenRobustDynamicState2022, chenDynamicStateEstimation2022} utilized the Kalman filter to estimate the dynamic states of natural gas pipelines. 
Furthermore, a robust Kalman filter-based approach, designed to mitigate the impact of poor-quality data, was introduced in \cite{chenRobustKalmanFilterBased2021}. 
The work in \cite{suGraphFrequencyDomainKalman2024} proposed a graph-frequency domain Kalman filter that reduces the influence of outliers in the data.

Several additional techniques have been proposed for the identification of gas network models and state estimation.
A data-driven approach, developed in \cite{huangDataDrivenStateEstimation2023}, involves solving a weighted low-rank approximation problem to estimate the gas state in the presence of measurement noise. 
Another method, based on Fourier transformation, was presented in \cite{yinEnergyCircuitTheory2020}. This approach transforms the partial differential equations governing gas dynamics from the time domain to the frequency domain, yielding algebraic equations that can be used for state estimation.

In summary, much of the existing research focuses on simulating the dynamic behavior of gas networks, often incorporating physical information to enhance model performance \cite{suHybridPhysicalData2021, yinHighaccuracyOnlineTransient2023}. 
However, less attention has been directed towards parameter identification in gas networks—a process that is essentially the inverse of simulation. 
Furthermore, current data-driven solutions for gas pipeline parameter estimation tend to focus on steady-state conditions, with limited attention paid to dynamic state transitions, which are either oversimplified or ignored. 
This shortcoming restricts their applicability to real-world gas system operations. 
For example, the widely used quasi-dynamic models, which rely on terminal node states to represent the entire pipeline, can introduce significant inaccuracies in optimal dispatch problems \cite{yangEffectNaturalGas2018}. 
Additionally, traditional neural network models face challenges in safety-critical applications due to their poor interpretability. 
Thus, developing interpretable models capable of accurately capturing the physical dynamics of gas pipeline networks is critical for improving operational efficiency and safety.