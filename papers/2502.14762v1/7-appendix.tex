\clearpage
\appendix
\section{Appendix}

\subsection{Compared Methods and TOSCA}
\label{Compared Methods and TOSCA}
Here, we provide an overview of the methods evaluated in the main paper. To ensure a fair and consistent basis for comparison, all methods utilize the same PTM. This standardization allows us to isolate the contributions of each method’s unique approach and compare their performance more accurately. Additionally, we present the pseudocode for TOSCA, providing a clear and detailed description of its working algorithm. This helps to better understand how TOSCA operates, offering insights into its efficiency and functionality within the context of class-incremental learning.

\textbf{Joint}: This method adheres to the traditional supervised batch learning paradigm, where all classes are presented simultaneously and trained over multiple epochs. It serves as the upper bound for class-incremental learning methods, as it does not experience forgetting.
    
\textbf{Finetune}: This method updates all parameters of the pretrained model when continually trained on new tasks. While it can achieve strong performance, it is susceptible to catastrophic forgetting, where previous knowledge is lost when learning new tasks.
        
\textbf{SimpleCIL} \cite{simplecil_aper}: SimpleCIL uses the PTM in its original form, combined with a prototypical classifier. It constructs a prototype for each class and utilizes a cosine classifier for classification, aiming for efficient task learning without additional adaptations.
    
\textbf{L2P} \cite{l2p}: L2P integrates visual prompt tuning into class-incremental learning with a pre-trained Vision Transformer (ViT). The method places the prompt only in the initial embedding layer, ensuring that the prompt adjusts the features at the early stage of the model while maintaining the frozen structure of the rest of the pretrained model.
    
\textbf{DualPrompt} \cite{dualprompt}: DualPrompt builds on L2P by introducing two types of prompts: general prompts (G-Prompt) and expert prompts (E-Prompt). The G-Prompts are applied to the earlier transformer blocks, allowing for broad task-specific adaptation. E-Prompts, on the other hand, are used in the latter blocks of the transformer, providing more specialized tuning for later stages of task processing. This separation allows for more efficient adaptation.
    
\textbf{CODA-Prompt} \cite{codaprompt}: It addresses the challenges of selecting instance-specific prompts by introducing prompt reweighting. It enhances the selection process through an attention mechanism that dynamically weights prompts, improving task-specific performance.

\textbf{APER} \cite{simplecil_aper}: This approach builds on SimpleCIL by introducing an adapter to each transformer layer, but only for the initial task. This adapter helps the pre-trained model to extract task-specific features during the first incremental phase, ensuring better adaptation to the new task while minimizing forgetting in subsequent tasks.
    
\textbf{EASE} \cite{ease}: This method adds adapters to each transformer layer for every task. This approach leads to good performance by concatenating the feature representations of multiple task-specific backbones, but it comes with an increase in model complexity due to the addition of task-specific adapters at every stage.
    
\textbf{MOS} \cite{mos}: It also trains adapters for each transformer layer for every task. However, MOS introduces the concept of adapter merging and replay generation for classifier correction. These processes increases computational complexity, particularly during training, as the model must handle the merging of multiple task-specific adapters with an increasing number of parameters.


\begin{algorithm}[]
\caption{\textbf{TOSCA for PTM-based CIL}}
\label{alg:ease}
\begin{algorithmic}[1]
\REQUIRE Incremental datasets: $\{\mathcal{D}^{1}, \mathcal{D}^{2}, \ldots, \mathcal{D}^{B}\}$, Pre-trained embedding: $\phi(\mathbf{x})$
\FOR{$b = 1, 2, \ldots, B$}
    \STATE Get the incremental training set $\mathcal{D}^{b}$
    \STATE Initialize a new LuCA module $L_{b}$ on top of last \texttt{[CLS]} token
    \STATE Optimize the parameters of the module $L_{b}$ via \cref{eq:training-loss}
    \STATE Test the model with all classes seen so far via \cref{eq:entropy-fusion}
\ENDFOR
\end{algorithmic}
\end{algorithm}

\subsection{Additional Results}
\label{Additional Results}
In this section, we provide a detailed table for pre-trained \textbf{ViT-B/16-IN1K} and figures that illustrate each incremental step with  \textbf{ViT-B/16-IN21K}, showcasing the performance of our proposed method. As demonstrated, our approach consistently achieves superior results in terms of the last incremental accuracy across all datasets. Importantly, this performance improvement is accompanied by a minimal overhead cost during both the training and inference phases, emphasizing the efficiency of our method, and making it highly suitable for real-world applications. Overall, these results underscore the effectiveness of our method in achieving high accuracy while maintaining computational efficiency, positioning it as a competitive solution for class-incremental learning scenarios.

\renewcommand\thetable{A}
\begin{table}[h]
\captionsetup{font=small}
\caption{Average and last accuracy [\%] performance on six datasets with \textbf{ViT-B/16-IN1K} as the backbone. ‘IN-R/A’ stands for ‘ImageNet-R/A,’ and ‘OmniBench’ stands for ‘OmniBenchmark.’ We report all compared methods with their source code and show the best performance in bold.}
\label{tab:in1k}
\fontsize{15}{25}\selectfont
%\setlength{\arrayrulewidth}{0.01pt}
\resizebox{\textwidth}{!}{%
\begin{tabular}{lcccccccccccc}
\hline
\multirow{2}{*}{Method} &
  \multicolumn{2}{c}{CIFAR B0 Inc5} &
  \multicolumn{2}{c}{CUB B0 Inc10} &
  \multicolumn{2}{c}{IN-R B0 Inc20} &
  \multicolumn{2}{c}{IN-A B0 Inc20} &
  \multicolumn{2}{c}{OmniBench B0 Inc30} &
  \multicolumn{2}{c}{VTAB B0 Inc10} \\
 &
  \multicolumn{1}{c}{$\bar{\mathcal{A}}$} &
  \multicolumn{1}{c}{$\mathcal{A}_B$} &
  \multicolumn{1}{c}{$\bar{\mathcal{A}}$} &
  \multicolumn{1}{c}{$\mathcal{A}_B$} &
  \multicolumn{1}{c}{$\bar{\mathcal{A}}$} &
  \multicolumn{1}{c}{$\mathcal{A}_B$} &
  \multicolumn{1}{c}{$\bar{\mathcal{A}}$} &
  \multicolumn{1}{c}{$\mathcal{A}_B$} &
  \multicolumn{1}{c}{$\bar{\mathcal{A}}$} &
  \multicolumn{1}{c}{$\mathcal{A}_B$} &
  \multicolumn{1}{c}{$\bar{\mathcal{A}}$} &
  \multicolumn{1}{c}{$\mathcal{A}_B$} \\ \hline
Joint &
  $-$ &
  95.88 \footnotesize{± 1.0} &
  $-$ &
  90.19 \footnotesize{± 1.4} &
  $-$ &
  83.87 \footnotesize{± 1.4} &
  $-$ &
  74.05 \footnotesize{± 1.9} &
  $-$ &
  83.08 \footnotesize{± 1.1} &
  $-$ &
  93.24 \footnotesize{± 1.8} \\ \hline
Finetune &
  44.4 \footnotesize{± 8.4} &
  39.7 \footnotesize{± 6.1} &
  57.27 \footnotesize{± 2.9} &
  34.76 \footnotesize{± 1.1} &
  66.96 \footnotesize{± 3.2} &
  53.64 \footnotesize{± 1.5} &
  28.64 \footnotesize{± 4.5} &
  14.26 \footnotesize{± 1.8} &
  63.35 \footnotesize{± 2.1} &
  45.70 \footnotesize{± 1.0} &
  67.84 \footnotesize{± 4.9} &
  51.12 \footnotesize{± 5.6} \\
SimpleCIL &
  82.21 \footnotesize{± 0.7} &
  76.24 \footnotesize{± 0.1} &
  90.42 \footnotesize{± 1.4} &
  85.16 \footnotesize{± 0.1} &
  66.89 \footnotesize{± 0.5} &
  61.27 \footnotesize{± 0.1} &
  58.70 \footnotesize{± 1.1} &
  49.44 \footnotesize{± 0.1} &
  78.67 \footnotesize{± 1.4} &
  72.20 \footnotesize{± 0.1} &
  90.50 \footnotesize{± 1.2} &
  83.61 \footnotesize{± 0.1} \\
L2P &
  83.37 \footnotesize{± 1.7} &
  78.64 \footnotesize{± 1.6} &
  70.64 \footnotesize{± 1.7} &
  58.70 \footnotesize{± 1.1} &
  77.22 \footnotesize{± 0.5} &
  72.35 \footnotesize{± 0.3} &
  52.32 \footnotesize{± 2.2} &
  44.30 \footnotesize{± 0.8} &
  72.76 \footnotesize{± 1.8} &
  63.10 \footnotesize{± 0.6} &
  81.25 \footnotesize{± 3.0} &
  66.71 \footnotesize{± 1.7} \\
DualPrompt &
  82.41 \footnotesize{± 1.7} &
  76.39 \footnotesize{± 0.6} &
  75.78 \footnotesize{± 2.2} &
  63.47 \footnotesize{± 1.5} &
  74.37 \footnotesize{± 0.5} &
  69.58 \footnotesize{± 2.0} &
  56.42 \footnotesize{± 1.1} &
  46.99 \footnotesize{± 0.3} &
  73.21 \footnotesize{± 1.8} &
  63.63 \footnotesize{± 0.8} &
  82.84 \footnotesize{± 4.7} &
  70.39 \footnotesize{± 5.5} \\
CODA-Prompt &
  86.67 \footnotesize{± 0.5} &
  80.68 \footnotesize{± 1.1} &
  70.75 \footnotesize{± 1.1} &
  61.61 \footnotesize{± 1.1} &
  78.37 \footnotesize{± 0.5} &
  73.07 \footnotesize{± 0.5} &
  63.61 \footnotesize{± 0.9} &
  52.32 \footnotesize{± 0.4} &
  72.22 \footnotesize{± 0.3} &
  68.26 \footnotesize{± 0.6} &
  84.88 \footnotesize{± 1.1} &
  82.94 \footnotesize{± 1.6} \\
APER-Adapter &
  88.46 \footnotesize{± 0.8} &
  83.16 \footnotesize{± 0.4} &
  87.64 \footnotesize{± 1.2} &
  80.63 \footnotesize{± 0.1} &
  78.25 \footnotesize{± 0.5} &
  72.07 \footnotesize{± 0.8} &
  66.86 \footnotesize{± 1.3} &
  58.83 \footnotesize{± 0.2} &
  77.66 \footnotesize{± 1.0} &
  70.72 \footnotesize{± 0.4} &
  89.59 \footnotesize{± 1.2} &
  82.60 \footnotesize{± 0.1} \\
EASE &
  89.94 \footnotesize{± 1.0} &
  84.39 \footnotesize{± 0.6} &
  87.93 \footnotesize{± 1.2} &
  81.00 \footnotesize{± 0.3} &
  82.96 \footnotesize{± 0.3}&
  77.45 \footnotesize{± 0.1}&
  70.49 \footnotesize{± 1.6}&
  62.36 \footnotesize{± 0.5} &
  78.40 \footnotesize{± 0.8} &
  71.60 \footnotesize{± 1.0} &
  90.71 \footnotesize{± 1.6} &
  83.39 \footnotesize{± 0.7} \\
MOS &
  92.71 \footnotesize{± 1.1} &
  88.82 \footnotesize{± 0.7} &
  \textbf{92.24} \footnotesize{± 0.9} &
  88.02 \footnotesize{± 0.2} &
  83.53 \footnotesize{± 0.7} &
  78.94 \footnotesize{± 0.3} &
  69.14 \footnotesize{± 1.1} &
  61.24 \footnotesize{± 1.8} &
  \textbf{85.33} \footnotesize{± 1.1} &
  78.28 \footnotesize{± 0.5} &
  91.81 \footnotesize{± 0.5} &
  91.77 \footnotesize{± 0.2} \\ \hline
\rowcolor{pink!20}
TOSCA (ours) &
  \textbf{96.03} \footnotesize{± 0.9} &
  \textbf{95.37} \footnotesize{± 0.7} &
  91.55 \footnotesize{± 1.8} &
  \textbf{89.05} \footnotesize{± 1.9} &
  \textbf{83.57} \footnotesize{± 0.6} &
  \textbf{82.25} \footnotesize{± 0.6} &
  \textbf{74.48} \footnotesize{± 2.1} &
  \textbf{72.30} \footnotesize{± 1.8} &
  82.48 \footnotesize{± 1.8} &
  \textbf{78.65} \footnotesize{± 1.2} &
  \textbf{94.33} \footnotesize{± 2.0} &
  \textbf{91.80} \footnotesize{± 1.9} \\ \hline
\end{tabular}%
}
\end{table}

\renewcommand\thefigure{A}
\begin{figure}[h]
\centering
\begin{tikzpicture}
    \begin{groupplot}[
        group style={
            group size=3 by 2,
            horizontal sep=1cm,
            vertical sep=1.75cm,
        },
        width=0.36\textwidth,
        height=0.28\textwidth,
        xlabel={Number of Classes},
        ylabel={Accuracy [\%]},
        xmin=0, xmax=100,        % Adjust based on your data range
        ymin=0, ymax=100,       % Adjust based on your data range
        grid=major,
        legend columns=4,
        legend style={font=\small,
        draw=none,
        align=left,
        text width=2.5cm,
        at={(0,-2.2)}, 
        anchor=north west},
        legend image post style={scale=0.8}, 
        cycle list={% Define styles for 9 baselines + our method
            {blue, mark size= 0.5pt},
            {green!60!black, mark size= 0.5pt},
            {red, mark size= 0.5pt},
            {cyan, mark size= 0.5pt},
            {violet, mark size= 0.5pt},
            {brown, mark size= 0.5pt},
            {orange, mark size= 0.5pt},
            {black, line width=1.2pt, mark size= 0.6pt}
        },
        every axis plot/.append style={semithick}
    ]
    
    % First row
    \nextgroupplot[title={CIFAR B0 Inc5}, xmin=5, xmax=100, ymin=75, xlabel={}]
        \addplot+[] coordinates {(5, 90.4) (10, 89.7) (15, 91.13) (20, 90.25) (25, 90.52) (30, 89.97) (35, 89.06) (40, 88.65) (45, 88.31) (50, 86.98) (55, 86.33) (60, 85.85) (65, 85.37) (70, 84.29) (75, 83.43) (80, 82.75) (85, 82.01) (90, 81.94) (95, 81.35) (100, 81.28)};
        
        \addplot+[] coordinates {(5, 96.6) (10, 94.5) (15, 90.8) (20, 88.75) (25, 88.76) (30, 85.6) (35, 81.86) (40, 82.65) (45, 81.24) (50, 81.18) (55, 80.89) (60, 80.65) (65, 81.8) (70, 80.11) (75, 79.33) (80, 80.03) (85, 79.81) (90, 78.67) (95, 78.6) (100, 78.94)};
        
        \addplot+[] coordinates {(5, 98.0) (10, 95.5) (15, 92.67) (20, 89.45) (25, 87.96) (30, 84.63) (35, 83.89) (40, 83.62) (45, 82.38) (50, 80.72) (55, 81.87) (60, 81.82) (65, 82.18) (70, 81.84) (75, 81.55) (80, 81.35) (85, 81.35) (90, 80.56) (95, 79.36) (100, 79.79)};
       
        \addplot+[] coordinates {(5, 99.2) (10, 98.2) (15, 92.93) (20, 92.85) (25, 92.36) (30, 89.63) (35, 89.74) (40, 88.4) (45, 87.67) (50, 86.5) (55, 86.0) (60, 84.6) (65, 83.52) (70, 83.64) (75, 82.43) (80, 82.94) (85, 82.81) (90, 83.37) (95, 82.79) (100, 81.48)};
        
        \addplot+[] coordinates {(5, 99.0) (10, 97.3) (15, 96.33) (20, 94.65) (25, 94.4) (30, 93.67) (35, 92.23) (40, 91.95) (45, 91.18) (50, 90.18) (55, 88.96) (60, 88.82) (65, 88.42) (70, 88.46) (75, 87.52) (80, 86.24) (85, 85.92) (90, 85.51) (95, 85.53) (100, 84.97)};
        
        \addplot+[] coordinates {(5, 99.4) (10, 98.1) (15, 97.4) (20, 96.5) (25, 96.36) (30, 94.97) (35, 93.4) (40, 93.08) (45, 92.24) (50, 90.98) (55, 89.95) (60, 89.82) (65, 89.22) (70, 89.23) (75, 88.32) (80, 86.69) (85, 86.26) (90, 86.02) (95, 86.14) (100, 85.66)};
        
        \addplot+[] coordinates {(5, 99.2) (10, 98.3) (15, 97.6) (20, 96.8) (25, 95.76) (30, 94.87) (35, 94.63) (40, 94.25) (45, 94.24) (50, 94.36) (55, 93.71) (60, 93.32) (65, 93.14) (70, 92.19) (75, 91.72) (80, 91.74) (85, 91.07) (90, 90.59) (95, 90.13) (100, 90.24)};
        
        \addplot+[] coordinates {(5, 99.2) (10, 96.1) (15, 96.6) (20, 96.45) (25, 96.96) (30, 97.1) (35, 96.57) (40, 96.65) (45, 96.53) (50, 96.54) (55, 96.4) (60, 96.15) (65, 96.26) (70, 96.33) (75, 96.23) (80, 96.01) (85, 96.02) (90, 95.96) (95, 95.94) (100, 95.8)};

        \node at (axis cs:90, 95.7) [anchor=south] {\scriptsize{\textbf{5.55}$\uparrow$}};
        
        \legend{SimpleCIL, L2P, DualPrompt, CODAPrompt, APER, EASE, MOS, TOSCA}
    
    \nextgroupplot[title={CUB B0 Inc10}, xmin=10, xmax=200, ymin=55, xlabel={}, ylabel={}]
        \addplot+[] coordinates {(10, 96.26) (20, 93.43) (30, 93.33) (40, 91.32) (50, 91.61) (60, 90.21) (70, 90.97) (80, 91.06) (90, 90.36) (100, 90.03) (110, 88.05) (120, 87.51) (130, 87.71) (140, 87.82) (150, 87.31) (160, 87.1) (170, 86.91) (180, 86.72) (190, 86.35) (200, 86.73)};
        
        \addplot+[] coordinates {(10, 100.0) (20, 95.2) (30, 85.47) (40, 76.04) (50, 72.6) (60, 74.31) (70, 73.29) (80, 71.35) (90, 71.54) (100, 72.26) (110, 70.38) (120, 69.84) (130, 68.24) (140, 69.1) (150, 68.54) (160, 67.66) (170, 64.35) (180, 63.86) (190, 63.02) (200, 60.94)};
        
        \addplot+[] coordinates {(10, 100.0) (20, 95.57) (30, 97.12) (40, 95.34) (50, 91.26) (60, 89.51) (70, 83.94) (80, 83.15) (90, 81.87) (100, 79.03) (110, 78.94) (120, 78.04) (130, 76.71) (140, 77.08) (150, 76.14) (160, 75.55) (170, 74.49) (180, 73.54) (190, 72.31) (200, 70.7)};
                
        \addplot+[] coordinates {(10, 97.3) (20, 90.48) (30, 84.64) (40, 85.71) (50, 80.24) (60, 78.12) (70, 74.76) (80, 77.26) (90, 78.36) (100, 79.25) (110, 78.19) (120, 77.46) (130, 76.88) (140, 75.9) (150, 72.96) (160, 70.68) (170, 70.13) (180, 68.53) (190, 67.74) (200, 67.98)};
        
        \addplot+[] coordinates {(10, 99.19) (20, 97.93) (30, 97.23) (40, 96.73) (50, 94.83) (60, 93.79) (70, 91.68) (80, 91.89) (90, 91.15) (100, 90.73) (110, 89.72) (120, 90.06) (130, 89.31) (140, 88.31) (150, 87.84) (160, 87.21) (170, 87.19) (180, 87.61) (190, 86.56) (200, 86.73)};
        
        \addplot+[] coordinates {(10, 98.39) (20, 97.51) (30, 96.95) (40, 96.94) (50, 94.83) (60, 93.93) (70, 91.19) (80, 91.14) (90, 90.47) (100, 89.77) (110, 89.64) (120, 89.99) (130, 89.04) (140, 88.0) (150, 87.38) (160, 87.16) (170, 87.09) (180, 87.37) (190, 86.2) (200, 86.47)};
        
        \addplot+[] coordinates {(10, 98.31) (20, 97.98) (30, 97.97) (40, 97.36) (50, 97.15) (60, 95.54) (70, 93.7) (80, 94.02) (90, 93.54) (100, 92.68) (110, 92.71) (120, 92.37) (130, 91.92) (140, 91.41) (150, 91.23) (160, 91.22) (170, 90.86) (180, 90.93) (190, 90.31) (200, 90.03)  
};

        \addplot+[] coordinates {(10, 100.0) (20, 95.57) (30, 95.21) (40, 96.5) (50, 97.45) (60, 96.55) (70, 96.74) (80, 95.96) (90, 95.03) (100, 95.15) (110, 95.27) (120, 95.43) (130, 95.54) (140, 94.13) (150, 93.75) (160, 93.23) (170, 93.27) (180, 92.88) (190, 92.83) (200, 92.79)};

        \node at (axis cs:179, 92) [anchor=south] {\scriptsize{\textbf{2.76}$\uparrow$}};
    
    \nextgroupplot[title={ImageNet-R B0 Inc20}, xmin=20, xmax=200, ymin=55, ytick={55, 75, 95}, ymax=95, xlabel={}, ylabel={}]
        \addplot+[] coordinates {(20, 78.28) (40, 70.85) (60, 64.1) (80, 60.27) (100, 58.61) (120, 58.03) (140, 57.0) (160, 56.79) (180, 56.82) (200, 55.55)};
        
        \addplot+[] coordinates {(20, 85.25) (40, 80.5) (60, 78.47) (80, 76.82) (100, 75.92) (120, 74.72) (140, 72.69) (160, 73.06) (180, 71.51) (200, 70.85)};
       
        \addplot+[] coordinates {(20, 80.45) (40, 77.8) (60, 74.89) (80, 71.37) (100, 70.77) (120, 69.11) (140, 68.01) (160, 67.04) (180, 66.9) (200, 66.0)};
            
        \addplot+[] coordinates {(20, 84.91) (40, 81.74) (60, 79.27) (80, 77.54) (100, 75.58) (120, 74.7) (140, 72.77) (160, 71.37) (180, 72.25) (200, 71.43)};
        
        \addplot+[] coordinates {(20, 91.79) (40, 85.13) (60, 79.67) (80, 76.66) (100, 74.44) (120, 72.39) (140, 71.31) (160, 69.75) (180, 67.74) (200, 66.9)};
        
        \addplot+[] coordinates {(20, 93.4) (40, 89.23) (60, 85.96) (80, 83.02) (100, 81.3) (120, 79.32) (140, 78.44) (160, 77.38) (180, 76.11) (200, 75.73)};
        
        \addplot+[] coordinates {(20, 90.42) (40, 88.19) (60, 86.04) (80, 83.63) (100, 81.57) (120, 80.49) (140, 80.09) (160, 79.85) (180, 78.42) (200, 77.7)};
        
        \addplot+[] coordinates {(20, 88.79) (40, 86.94) (60, 83.06) (80, 82.87) (100, 82.56) (120, 82.57) (140, 81.8) (160, 82.49) (180, 81.46) (200, 79.9)};

        \node at (axis cs:183, 82) [anchor=south] {\scriptsize{\textbf{1.77}$\uparrow$}};
    
    % Second row
    \nextgroupplot[title={ImageNet-A B0 Inc20}, xmin=20, xmax=200, ymin=40, ymax=85, ytick={45, 65, 85}]
        \addplot+[] coordinates {(20, 77.06) (40, 66.99) (60, 62.3) (80, 57.85) (100, 56.5) (120, 56.9) (140, 55.13) (160, 53.03) (180, 51.56) (200, 48.78)};
        
        \addplot+[] coordinates {(20, 69.57) (40, 58.63) (60, 55.31) (80, 48.32) (100, 48.41) (120, 45.56) (140, 43.5) (160, 42.48) (180, 41.43) (200, 42.66)};
        
        \addplot+[] coordinates {(20, 72.67) (40, 62.95) (60, 60.25) (80, 52.43) (100, 51.01) (120, 49.01) (140, 45.96) (160, 43.45) (180, 41.74) (200, 40.36)};
         
        \addplot+[] coordinates {(20, 79.48) (40, 70.59) (60, 63.0) (80, 60.57) (100, 58.38) (120, 55.66) (140, 53.06) (160, 51.7) (180, 48.27) (200, 47.0)};
        
        \addplot+[] coordinates {(20, 74.53) (40, 69.78) (60, 64.69) (80, 59.7) (100, 58.65) (120, 54.69) (140, 53.53) (160, 50.4) (180, 48.9) (200, 49.24)};
        
        \addplot+[] coordinates {(20, 81.71) (40, 76.39) (60, 70.59) (80, 67.48) (100, 64.57) (120, 63.08) (140, 59.5) (160, 57.45) (180, 55.41) (200, 54.71)};
        
        \addplot+[] coordinates {(20, 76.43) (40, 71.8) (60, 70.29) (80, 67.1) (100, 65.03) (120, 61.97) (140, 59.37) (160, 57.11) (180, 55.02) (200, 54.18)};
        
        \addplot+[] coordinates {(20, 78.29) (40, 77.95) (60, 76.5) (80, 72.77) (100, 71.23) (120, 71.18) (140, 71.31) (160, 70.17) (180, 67.97) (200, 67.52)};
        
        \node at (axis cs:182, 68) [anchor=south] {\scriptsize{\textbf{11.8}$\uparrow$}};
    
    \nextgroupplot[title={Omnibenchmark B0 Inc30}, xmin=30, xmax=300, ymin=60, ymax=95, ylabel={}, ytick={65, 80, 95}]
        \addplot+[] coordinates {(30, 88.11) (60, 81.57) (90, 77.82) (120, 76.27) (150, 76.14) (180, 75.53) (210, 76.16) (240, 75.7) (270, 74.62) (300, 73.13)};
        \addplot+[] coordinates {(30, 93.31) (60, 83.88) (90, 80.95) (120, 77.69) (150, 75.24) (180, 71.99) (210, 71.29) (240, 67.91) (270, 65.65) (300, 64.61)};
        \addplot+[] coordinates {(30, 90.67) (60, 83.32) (90, 80.53) (120, 78.0) (150, 75.45) (180, 72.49) (210, 70.42) (240, 67.69) (270, 66.75) (300, 66.52)};
        \addplot+[] coordinates {(30, 90.8) (60, 86.05) (90, 66.04) (120, 66.32) (150, 69.85) (180, 72.59) (210, 73.03) (240, 71.42) (270, 69.95) (300, 69.67)};
        \addplot+[] coordinates {(30, 89.61) (60, 83.17) (90, 79.55) (120, 77.06) (150, 76.84) (180, 76.37) (210, 76.9) (240, 76.45) (270, 75.51) (300, 74.29)};
        \addplot+[] coordinates {(30, 94.0) (60, 87.99) (90, 85.98) (120, 84.22) (150, 81.48) (180, 78.45) (210, 77.96) (240, 76.24) (270, 75.83) (300, 74.2)};
        \addplot+[] coordinates {(30, 96.66) (60, 91.82) (90, 86.63) (120, 84.49) (150, 83.08) (180, 82.25) (210, 81.43) (240, 81.52) (270, 80.66) (300, 79.92)};
        \addplot+[] coordinates {(30, 93.81) (60, 90.48) (90, 83.9) (120, 84.75) (150, 83.39) (180, 84.03) (210, 84.14) (240, 84.75) (270, 81.93) (300, 82.32)};

        \node at (axis cs:275, 82) [anchor=south] {\scriptsize{\textbf{2.40}$\uparrow$}};

    \nextgroupplot[title={VTAB B0 Inc10}, xmin=10, xmax=50, ymin=60, ylabel={}, xtick={20, 30, 40, 50}, ytick={65, 80, 100}]
        \addplot+[] coordinates {(10, 93.49) (20, 94.99) (30, 92.19) (40, 88.47) (50, 84.43)};
        \addplot+[] coordinates {(10, 95.78) (20, 79.56) (30, 72.42) (40, 70.48) (50, 65.54)};
        \addplot+[] coordinates {(10, 98.07) (20, 80.6) (30, 84.49) (40, 79.01) (50, 76.53)};
        \addplot+[] coordinates {(10, 92.3) (20, 85.0) (30, 87.09) (40, 85.87) (50, 86.77)};
        \addplot+[] coordinates {(10, 93.45) (20, 94.96) (30, 92.19) (40, 88.31) (50, 84.06)};
        \addplot+[] coordinates {(10, 99.29) (20, 96.63) (30, 93.55) (40, 87.44) (50, 82.03)};
        \addplot+[] coordinates {(10, 94.7) (20, 91.71) (30, 92.49) (40, 91.89) (50, 92.59)};
        \addplot+[] coordinates {(10, 99.03) (20, 97.75) (30, 96.54) (40, 96.75) (50, 94.07)};
        
        \node at (axis cs:46.5, 93.5) [anchor=south] {\scriptsize{\textbf{1.48}$\uparrow$}};
        
    \end{groupplot}
\end{tikzpicture}
\captionsetup{font=small}
\caption{Performance curve of different methods on different benchmarks. All methods are initialized with \textbf{ViT-B/16-IN21K}. We annotate the relative improvement of TOSCA above the runner-up method with numerical numbers at the last incremental stage.}
\label{fig:cil_results}
\end{figure}

