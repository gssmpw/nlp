\section{Related Work}
%Our work lies at the intersection of several active research areas within the interpretability of deep learning models. 
In this section we highlight the key connections and distinctions between our approach and prior work, emphasizing how we build upon existing methods while forging a novel path towards understanding the internal mechanisms of transformers. We 
%organize the discussion into 
focus on
three main themes: 1) the geometric perspective of neural network representations, particularly the view of features as directions; 2) the circuit-centric framework for mechanistic interpretability; and 3) the application of computational mechanics and belief state geometry to neural network analysis. 

%\paragraph{Features as directions in activation space}
\textbf{Features as directions in activation space.}
---
% Modern interpretability research often views neural network representations through the lens of linear geometry.
% Rather than focusing on individual neurons, this framework analyzes how activation patterns align with specific directions in the network's activation space~\cite{park2024linearrepresentationhypothesisgeometry}.
% These directions are thought to encode fundamental features or concepts learned by the network.
% This perspective is particularly useful in light of superposition~\cite{elhage2022superposition}, where networks represent more features than available neurons by encoding them as 
% non-orthogonal vectors.
%
% Conceptualizing features as linear directions in activation space has been instrumental \cite{cunningham2023sae, bricken2023monosemanticity, templeton2024scaling}, in understanding \textit{what} information is represented in transformers, and the geometric relationships between these feature directions often capture meaningful semantic relationships, suggesting that the network's internal representation space is highly structured~\cite{Engels24_Not}. Our work complements this line of research by providing a mechanistic explanation for the emergence of specific non-orthogonal geometric structures within transformer intermediate representations. We connect these structures to a theory of constrained belief updating, providing the theoretical ``why'' to complement the ``what'' of feature representations.
%
Modern interpretability research views neural network representations through the lens of linear geometry, analyzing how activation patterns align with specific directions that encode fundamental features~\cite{park2024linearrepresentationhypothesisgeometry}. This perspective is particularly useful given superposition~\cite{elhage2022superposition}, where networks encode more features than available neurons using non-orthogonal vectors. Conceptualizing features as linear directions has been instrumental~\cite{cunningham2023sae, bricken2023monosemanticity, templeton2024scaling} in understanding what information transformers represent, with geometric relationships between features revealing structured internal representations~\cite{Engels24_Not}. Our work complements this line of research by providing a mechanistic explanation for these non-orthogonal geometric structures, providing the theoretical ``why'' to complement the ``what'' of feature representations.

%\paragraph{From features to circuits}
\textbf{From features to circuits.}
---
While feature directions reveal what information is encoded, understanding how networks process this information benefits from identifying computational circuits—subnetworks that implement specific algorithmic operations.
These circuits typically combine simpler features into more complex ones as information flows through the network.
Notable examples include circuits that detect syntax in language models \cite{elhage2021mathematical}, implement indirect object identification \cite{wang2022interpretability}, or perform basic arithmetic \cite{nanda2023mechanistic}.
However, identifying these circuits remains largely a manual process, starting from observed behaviors and working backwards to discover relevant components
(although active research is developing automated approaches; see \citet{conmy2023automatedcircuitdiscoverymechanistic, marks2024sparsefeaturecircuitsdiscovering}).

Our work contributes to this area by demonstrating that a principled, top-down theoretical framework, based on constrained belief updating, can guide the search for circuits and provide a deeper understanding of their function within the larger network. We show how specific circuits in the attention mechanism directly implement the computations predicted by our theory.

% \paragraph{A mathematical framework for transformer interpretability}
%\paragraph{Belief state geometry and computational mechanics}
\textbf{Belief state geometry and computational mechanics.}
---
Our work draws inspiration from computational mechanics, a framework for studying the physics of information processing in dynamical systems~\cite{shalizi2001computational, Crutchfield12_Between, Riec18_SSAC1}. When applied to sequential data, computational mechanics, in accordance with the POMDP framework~\cite{Kaelbling98_Planning}, 
% \cite{DeepMind_paper}
shows that optimal prediction requires maintaining beliefs about the underlying latent states of the data-generating process~\cite{Upper97_Theory}. 
These belief states can be visualized as points on a probability simplex, evolving according to Bayesian updating rules, and forming characteristic geometric patterns~\cite{Crutchfield94_Calculi, Marzen_2017}.
Recent work shows that transformer networks naturally discover and encode these belief state geometries in their activation patterns \cite{shai2024transformersrepresentbeliefstate}.
This connection offers a principled way to analyze network representations: rather than reverse-engineering observed behaviors, we can study how architectural constraints shape the network's implementation of theoretically optimal prediction strategies.

This is the approach taken 
here. %in this paper. 
We move beyond prior work by proposing and validating a theory of constrained belief updating, demonstrating how specific architectural elements, like the attention mechanism, modify the idealized belief state dynamics. This perspective shifts the focus from reverse-engineering learned features to understanding why particular geometric patterns emerge during training as a consequence of the interplay between optimal prediction and architectural constraints. Our work provides a concrete example of how this theoretical framework can be applied to understand the internal mechanisms of transformers.