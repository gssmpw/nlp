We provide an analysis of the performance of heterogeneous learning agents in asset markets with stochastic payoffs. Our agents aim to maximize the expected growth rate of their wealth, but have different theories on how to learn to do this best. Our main focus is on comparing Bayesian learners and no-regret learners in market dynamics. Bayesian learners with a prior over a finite set of models that assign positive prior probability to the correct model have posterior predicted probabilities that converge exponentially fast to the correct model. Consequently, such Bayesians survive even in the presence of agents who invest according to a correct model of the stochastic process. Bayesian learners with a continuum prior converge to the correct model at a rate of $O((\log T) / T)$. 

Online learning theory provides no-regret algorithms for maximizing the log of wealth in this asset market setting, achieving a worst-case regret bound of $O(\log T)$ without assuming that there is a steady underlying stochastic process, but comparing to the best fixed investment rule. This regret, as we observe, is of the same order of magnitude as that of a Bayesian learner with a continuum prior. However, we show that even such low regret may not be sufficient for survival in asset markets: an agent can have regret as low as $O(\log T)$, converging to the correct model at a rate of $O((\log T) / T)$, but still vanish in market dynamics when competing against an agent who always invests according to the correct model, or even against a perfect Bayesian with a finite prior. On the other hand, we show that Bayesian learning is fragile, while no-regret learning requires less knowledge of the environment and is therefore more robust. Specifically, any no-regret learner will drive out of the market an imperfect Bayesian whose finite prior or update rule has even small errors. In our analysis, we formally establish the relationship between the notions of survival, vanishing, and market domination studied in economics and the framework of regret minimization, thus bridging these theories. More broadly, our work contributes to the study of dynamics with heterogeneous types of learning agents and their impact on markets. Our results highlight the importance of exploring this area further.\\ 
