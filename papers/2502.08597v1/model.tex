We use the market structure as in \cite{blumeeasley1992} in our analysis of wealth and price dynamics. 
At each time in discrete steps indexed by $t$, a state is realized. There is a finite number of states $s\in \{1, \dots, S\}$ distributed i.i.d. with probability $q=(q_1, \dots, q_S)$ where $q_s \geq \Delta > 0$ for all $s$. 
At each time, there is a complete set of assets. Assets are also indexed by $s$ with one unit of asset $s$ purchased at time $t$ paying one unit of wealth at time $t+1$ if state $s$ occurs at time $t+1$ and $0$ otherwise. These assets, known as Arrow securities, are more general than they may seem; any asset structure that permits arbitrary reallocation of wealth across states can be written as a simple transformation of Arrow securities. This asset market model, originally introduced in \cite{Arrow}, is the standard general equilibrium model of an economy with a complete set of assets, see \cite{MWG} Chapter 19. Asset prices at time $t$ are represented by $p_t=(p_{1t},\dots, p_{St})$. 

There are $N$ agents indexed by $n=1, \dots, N$. The wealth of agent $n$ at time $t$ is $w^n_t$. Agents begin with initial wealths $w^n_0>0$ for agent $n$. These initial wealths are normalized to sum to one.\footnote{This normalization is done only for convenience. 
We analyze relative wealths. The actual size of the economy (the aggregate wealth) does not play a role in relative wealth dynamics and may follow an arbitrary stochastic process, as long as it is bounded away from zero and bounded from above.}
Thus, if state $s$ occurs at time $t+1$, agent $n$ will have time $t+1$ wealth, $w^n_{t+1} = \frac{\alpha^n_{st}w^n_t}{p_{st}}$. 

In a market equilibrium, the price of asset $s$ must be such that the aggregate demand for asset $s$ equals the aggregate supply of one. So the price is given by 
\begin{equation}
p_{st} = \sum_n \alpha^n_{st}w^n_t.
\end{equation}
Note that asset prices sum to one as the aggregate wealth is one.

Using this structure, we can describe the evolution of wealths and asset prices as functions of the agents' investments. The step-by-step stochastic process of agent $n$'s wealth is

\begin{equation}\label{eq:wealth-evolution}
w^n_{t+1} = \prod_{s=1}^S  \bigg(\frac{\alpha^n_{st}}{p_{st}}\bigg)^{\mathbf{1}_{\{s_t = s\}}} w^n_t,
\end{equation}
where $\mathbf{1}_{\{s_t = s\}}$ is the indicator function taking on the value $1$ if the state realized at time $t$ is state $s$, i.e., $s_t=s$, and equals zero otherwise. Equivalently, the wealth at time $t > 0$ is 
\begin{equation}
w^n_{t}   = \prod_{\tau=1}^{t} \prod_{s=1}^S  \bigg(\frac{\alpha^n_{s\tau}}{p_{s\tau}}\bigg)^{\mathbf{1}_{\{s_\tau = s\}}} w^n_0.
\end{equation}

Directly analyzing individual wealth processes is potentially complex as they depend on prices which are endogenous. Instead, we analyze relative wealths. Let $r^{nm}_t$ be the ratio of the wealth of agent $n$ to the wealth of agent $m$ at time $t$. Prices cancel, so wealth ratios follow
\begin{align}
	r^{nm}_{t+1} &=  \prod_{s=1}^S  \bigg(\frac{\alpha^n_{st}}{\alpha^m_{st}}\bigg)^{\mathbf{1}_{\{s_t = s\}}} r^{nm}_t,\\
	r^{nm}_{T}   &= \prod_{t=1}^{T} \prod_{s=1}^S \bigg (\frac{\alpha^n_{st}}{\alpha^m_{st}}\bigg)^{\mathbf{1}_{\{s_t = s\}}} r^{nm}_0.
\end{align}

Taking the logarithm of both sides yields
\begin{equation}\label{eq:logshares}
	\log(r^{nm}_{T}) = \sum_{t=1}^{T}\sum_{s=1} ^S \mathbf{1}_{\{s_t = s\}} \log (\frac{\alpha^n_{st}}{\alpha^m_{st}}) + \log(r^{nm}_{0}). 
\end{equation}

To provide some insight into how wealth ratios evolve, it is useful to first consider a special case in which investment sequences are constant over time: $\alpha^n_t=\alpha^n$ for all $t$. With constant investments, dividing both sides of equation (\ref{eq:logshares}) by $T$, taking the limit as $T\to \infty$, omitting the constant term $\log(r_0^{nm})$ that vanishes in the limit, and applying the Law of Large Numbers yields\footnote{Note that if $\alpha^n_{st}$ were a randomized strategy with expectation $\alpha^n$ we could not simply take the expectation, but by Jensen's inequality, we would get a wealth ratio bounded by the entropy difference $\lim_{T\to \infty} T^{-1} \log(r^{nm}_{T}) \leq I_q(\alpha^m) - I_q(\alpha^n)$.}

\begin{equation}\label{eq:wealthshares}
  \lim_{T\to \infty} T^{-1} \log(r^{nm}_{T}) = \sum_{s=1}^S q_s \log(\frac{\alpha^n_s}{\alpha^m_s}). 
\end{equation}

To interpret equation (\ref{eq:wealthshares}) it is useful to define the relative entropy (a.k.a. KL-divergence) of investment rule $\alpha$ relative to the probability distribution $q$ as

\begin{equation}
    I_q(\alpha) = \sum_{s=1}^S q_s \log(\frac{q_s}{\alpha_s}).
\end{equation}

Then, rewriting equation (\ref{eq:wealthshares}) using relative entropies we have 

\begin{align}
\label{eq:fixed invext entropy}
    \lim_{T\to \infty} T^{-1} \log(r^{nm}_{T}) = I_q(\alpha^m) - I_q(\alpha^n). 
\end{align}

Thus, in the constant investment rules environment, the fate of agent $n$ depends on $I_q(\alpha^n)$ and $I_q(\alpha^m)$ for all other traders $m$. For example, if $n$ has the unique lowest relative entropy then $n$'s wealth converges to $1$ and all other agents' wealth converge to $0$. Entropy is minimized at $\alpha=q$, so if a single agent invests according to $q$, that agent's wealth share converges to $1$.

In the next two sections, we consider Bayesian learners and regret minimizers. Their investment rules are not constant, so we need to generalize this analysis to allow for non-constant rules. 

Before considering specific learning strategies, it is useful to note that due to the multiplicative nature of wealth evolution (see Equation (\ref{eq:wealth-evolution})), any learning strategy that assigns zero weight to any state risks losing all wealth, with no possibility of recovery; formally:
\begin{definition}\label{def:survive-and-vanish}
    An agent $n$ {\em survives} in the market if there exists a constant market share $c$ such that 
    $$
    \lim_{T \rightarrow \infty} \Pr\left[
    \E[w^n_T] > c\right] = 1.
    $$
    We say that agent $n$ {\em vanishes} if $n$'s expected wealth converges to zero in the limit. That is, 
    $$
    \lim_{T \rightarrow \infty} 
    \E[w^n_T]  = 0 \text{ with probability } 1.
    $$
\end{definition}

\begin{observation}
    Let $\{\alpha^n_t\}_{t=0}^\infty$ be a sequence of investment profiles for agent $n$. If there exists a atate $s \in S$ for which $\alpha^n_{st} = 0$ infinitely often, then agent $n$ vanishes from the market.
\end{observation}

This straightforward observation immediately rules out certain learning approaches, such as those that treat asset selection as pure actions. It demonstrates that any reasonable learning strategy must be restricted to investment profiles with full support on the assets. Specifically, all learning strategies we consider will assign at least a small, constant probability $\minProb$ to each state.
