
To the best of our knowledge, this work is the first to analyze the dynamics of no-regret learners against agents using other theories of learning, to study the survival of no-regret learners in investment scenarios (in contrast to only their regret level), and to establish a connection between regret minimization and Bayesian learning in investment settings.

We show that, on the one hand, a Bayesian learner dominates the market against any player whose regret increases with time (including logarithmic regret). Such a player may have vanishing regret, but will also have a vanishing share of the market value. On the other hand, we show that Bayesian learning is fragile. If the Bayesian learner is not perfect but, instead, has some level of noise in its prior or update rule, it may initially gain a large share of the market, but in the long run, a no-regret learner will dominate, driving the Bayesian out of the market.

The paper is structured as follows. In Section \ref{sec:model} we present the market model and provide useful preliminary analysis, discussing relative wealths and the role of relative entropy in characterizing survival for stationary investment rules.  
Section \ref{sec:compare} analyzes the relation between regret and wealth shares in the market, showing how regret can be used to compare different learners.  
In Section \ref{sec:Bayesians}, we restate in the language of our market setting several known analyses of Bayesian learning that will be important for our comparison of Bayesian learners, no-regret learners, and imperfect Bayesians.  
In Section \ref{sec:no-regret}, we analyze the regret of a player who knows the correct stochastic model of market states and show that this analysis leads to a characterization of survival conditions in competition with a perfect Bayesian.  
Section \ref{sec:imperfect-Bayesians} studies imperfect Bayesians, showing that Bayesian learners with inaccurate priors or updates vanish when competing against no-regret learners that converge to the correct model. 
Finally, in Section \ref{sec:simulations}, we present and discuss simulation results for several scenarios of competition between no-regret learners, perfect Bayesians, and imperfect ones, to complement our analysis and provide further intuition.  

We conclude the introduction with an a informal summary of our main theorems and analyses:

\empar{Regret and Survival}  
Theorem \ref{thm:survival-by-finite-regret-gap} characterizes the regret condition for survival in a competitive asset market: an agent survives if and only if its regret remains bounded by a constant relative to every competitor at all times.  
Remarkably, if two players both have similar regret asymptotics with respect to the best strategy in hindsight (e.g., both have $O(\log T)$ regret) but with different coefficients, the agent with the larger constant almost surely vanishes from the market.  

\empar{Regret of the Optimal Investment Strategy}  
Regret with respect to the best strategy in hindsight is a strong benchmark; as we discuss in Section \ref{sec:compare}, due to the stochasticity in market dynamics, even the best implementable strategy (i.e., one that does not depend on future outcomes) has positive regret. In Theorem \ref{thm:regret-of-q}, in Section \ref{sec:no-regret}, we calculate this expected regret level, showing that it is constant and depends only on the number of assets in the market. As a corollary of this result and Theorem \ref{thm:survival-by-finite-regret-gap}, any agent who survives in competition with an agent who invests knowing the true distribution of states must have regret bounded by a constant at all times. 

\empar{Survival against Perfect Bayesians}  
Theorem \ref{thm:regret-of-a-Bayesian} shows that perfect Bayesian learners with a finite support have constant expected regret.  
Consequently, a learning agent survives in competition with perfect Bayesians if and only if it also has constant regret.  
The proof uses Proposition \ref{thm:Bayesians-suvive-against-q} about Bayesian learners together with the results described above.  

\empar{Bayesians with Inaccurate Priors}  
Theorem \ref{thm:wrong-Bayesians-vanish} shows that a Bayesian agent with the correct state distribution not within its support suffers linear regret, and, as a result, vanishes from the market in competition with any no-regret learner.  
This theorem is followed by an analysis of the expected survival times of Bayesians with inaccurate priors, as a function of the size of the errors in the models considered in their priors and the regret rate of the competitor.  

\empar{Bayesians with Inaccurate Updates}  
In Section \ref{sec:noisy-Bayesian}, we analyze a scenario where a Bayesian learner performs ``trembling hand'' updates, making small zero-mean errors in the weight given to the current prior and the new data at each step.  
Theorem \ref{thm:wrong-update-Bayesians-vanish} demonstrates that even such tiny errors break Bayesian learning, causing the learner to fail to converge.  
As a result, any no-regret (or other) learner that converges to the correct model at any rate outperforms such an inaccurate Bayesian, driving them out of the market.  
\vspace{5pt}
