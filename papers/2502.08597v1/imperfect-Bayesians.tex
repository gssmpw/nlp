In the preceding sections, we saw that a perfect Bayesian is optimal in the sense that it survives almost surely in the market and drives out any player with regret increasing over time. In this section, we explore a scenario where a Bayesian learner suffers small errors, either in its set of prior distributions or in its update rule. We find that, while perfect Bayesians are optimal, Bayesian learning is also very fragile; for example, it is key that the correct probabilities is one of the model they  consider. Specifically, we show that an imperfect Bayesian would eventually vanish from the market, either against the state distribution $q$ or in competition with any player with sub-linear regret. This holds even if the Bayesian's errors are very small and have zero mean.
 
\subsection{Bayesian Learning with Inaccurate Priors}\label{sec:noisy-priors-Bayesian}

As mentioned in Section \ref{sec:Bayesians}, a Bayesian learner with $q$ not within the support of its prior converges to using the best strategy within its prior (i.e., the one closest to $q$ in relative entropy), denoted here by $q'$. 
This convergence property leads to the following result. 
\begin{theorem}\label{thm:wrong-Bayesians-vanish}
    A Bayesian agent with the state distribution $q$ not in its support incurs regret 
    linear in $T$, and vanishes from the market in competition against any no-regret learner.   
\end{theorem}
Note that this contrasts with the case of Theorem \ref{thm:regret-of-a-Bayesian}, where a perfect Bayesian dominates the market in competition with any learners with regret increasing in time.
\begin{proof}
Let $n$ be the index of a Bayesian agent with $q$ not within its support, and let $q' \neq q$ be the strategy within the support that is closest to $q$ in relative entropy. By Proposition \ref{thm:bayesian-convergence}, the strategy used by the Bayesian agent, $\alpha^n_t$, converges almost surely to $q'$. Hence, intuitively, for sufficiently large $t$, the strategies remain bounded away from $q$, leading to linear regret from that point onward. In more detail, define $I_0 = \frac{1}{2} I_q(q') > 0$. In every sample path $i$ (i.e., an infinite sequence, indexed by $i$, of state realizations and the ensuing sequence of $\alpha^n_t$), except on a set of measure zero, there exists a time $T_i$ such that for all $t \geq T_i$, we have $I_q(\alpha^n_t) > I_0$. The regret accumulated up to time $T_i$ is constant (possibly positive or negative). The expected regret difference relative to strategy $q$ at $t > T_i$ is $\E[U^t(q) - U^t(\alpha^n_t)] \geq I_0$, and so the expected regret relative to $q$ until time $T$ is at least $I_0 \cdot (T - T_i) + const$. By Theorem \ref{thm:regret-of-q}, the strategy $q$ incurs only constant regret, so the regret of the Bayesian agent converging to $q'$ grows linearly in $T$. Now, consider a no-regret learner. Such a learner has sublinear worst-case regret and thus sublinear expected regret. By Theorem \ref{thm:survival-by-finite-regret-gap}, the imperfect Bayesian agent who has linear regret vanishes.    
\end{proof}
\noindent
{\em Remark:}
The possible strategies that the Bayesian can play span the entire convex hull of the prior. 
Bayesians with a finite prior can (and do) play convex combinations of the prior during the dynamic, but still, they always converge to the vertices---even when a more profitable strategy lies in the interior. 
In this sense, this convergence property of Bayesians is both a strength (if they have an accurate prior) and a weakness (if they do not). 
\vspace{5pt}

\bfpar{Typical Survival Time}
A question that arises now is what happens when errors are small.  
In the following analysis, we consider ``$\epsilon$-inaccurate Bayesians'' who have in their prior a close approximation of $q$, but still with some small error. Specifically, the total variation distance between the best strategy in the prior and the state distribution $q$ satisfies $TV(q', q) = \epsilon$ for some $\epsilon > 0$. 

On the one hand, we know that such an agent asymptotically fails to survive against a regret-minimizing agent for any error $\epsilon$. On the other hand, when $\epsilon$ is small, the inaccurate Bayesian will initially converge very quickly to a distribution close to the correct state distribution, and may capture a significant share of the market value during the early stages. 

Our goal now is to analyze the survival time during which the inaccurate Bayesian still maintains a significant market share, and to understand how this survival time is related to the level of error $\epsilon$. This, of course, also depends on the best competitor and how fast they converge.

To estimate this relation and provide an upper bound on the survival time of an inaccurate Bayesian learner, we consider a player using $q'$ with $TV(q', q) = \epsilon$ (having already converged to this distribution), competing against a second player playing a dynamic strategy with expected regret level increasing as $f(t)$ that is sub-linear with $t$.

First, we would like to express the first player's error in terms of entropy (i.e., KL divergence with $q$). We can bound the entropy using Pinsker's inequality \cite{pinsker1964information} for a lower bound and its inverse for an upper bound, were the latter holds for finite distributions with full support, as in our case. We have (where recall that $\Delta = \min_s q_s$)
\[
2\epsilon^2 \leq 
% D_{KL}(q || q') 
I_q(q')
\leq \frac{2}{\Delta} \epsilon^2.
\]
Using Lemma \ref{thm:expected-regret-lemma}, this can be translated into the regret of the inaccurate agent playing strategy $q'$ where $R$ is the regret of an investor using the true state distribution $q$ (see Theorem \ref{thm:regret-of-q}): 
\[
% D_{KL}(q \| q') 
I_q(q')
= \frac{1}{T} \big(\mathbb{E}[R^{T}(q'_{1:T})] - R \big).
\]

To maximize the survival time, we pick the distribution $q'$ to be the one that has the smallest regret, which is when $I_q(q')$ is the smallest possible: $I_q(q') = 2\epsilon^2$. So we have\footnote{If the inaccurate agent uses the worst $q'$ with error $\epsilon$, the regret calcualtion is similar, but with $\epsilon$ rescaled by a factor of $1/\sqrt{\Delta}$, using the inverse of Pinsker's inequality: 
$2 \epsilon^2 \tau/\Delta  + R$. The smallest-regret  $q'$ is used for the bound.} 
$\mathbb{E}[R^{T}(q'_{1:T})] = 2\epsilon^2 T + R$. 
% 
By Equation (\ref{eq:logshare-as-regret}), the regret difference between two players is equal to the log of their wealth ratios  plus a constant. By comparing regret levels between the agents, we obtain a bound on the typical survival time $\tau$ up to which the inaccurate player holds, in expectation, more than half the market value; beyond this point, their expected share is less and continues decaying to zero:\footnote{An alternative question one may want to ask is, ``How accurate should my prior be if I want to hold a share of the market for $T$ steps against the competitor?'' For this, one can invert the equation to get $\epsilon = \sqrt{(f(T) - R)/T}$.}
\[
f(\tau) = 2\epsilon^2 \tau + R.
\]

\vspace{5pt}
\noindent
{\em Remark:} Note that $f(t)$ is the competitor's actual regret. Regret bounds for known learning algorithms (e.g., bandit algorithms such as UCB \cite{slivkins2019introduction} or gradient-descent and second-order methods like those described in  \cite{hazan2016introduction}) are typically worst-case bounds. Estimating the expected regret (or the most likely one) in game dynamics in general, or specifically in our investment scenario, is an interesting open problem.
\vspace{5pt}

For the special cases that the competitor has constant regret (e.g., an accurate Bayesian), or the competitor has $\log T$ or $\sqrt{T}$ regret we get the following results. 
\begin{observation}\label{obs:survival-time-gainst-constant-regret}
    The expected survival time of a Bayesian learner with $\epsilon$-inaccurate prior against any player with constant regret (e.g., a perfect Bayesian) is $O\big(\frac{1}{\epsilon^2}\big)$. 
\end{observation}
For the the competitor with logarithmic regret, we get an equation of the form $\tau = c_1 \cdot \exp(c_2 \epsilon^2\tau)$. For small errors $\epsilon$, by expanding the exponent, we have the following bound.
\begin{observation}\label{obs:survival-time-gainst-log-regret}
    The expected survival time of a Bayesian learner with $\epsilon$-inaccurate prior against any player with logarithmic regret (e.g., a no-regret convex optimizer) is $O\big(\frac{1}{\epsilon^3}\big)$.   
\end{observation}
Finally, in competition against a learner with regret $f(T) = \sqrt{T}$ the survival time is longer.
\begin{observation}\label{obs:survival-time-gainst-sqrt-regret}
    The expected survival time of a Bayesian learner with $\epsilon$-inaccurate prior against any player with $\sqrt{T}$ regret is $O\big(\frac{1}{\epsilon^4}\big)$. 
    % For $c$ is $\tau \leq \frac{c-R^q}{2 \epsilon^2}$.  
\end{observation}

The interpretation of the above analysis depends, of course, on the time scale we are interested in. When $T$ is large, which is our main focus (e.g., when trade occurs at high frequency or if investments are long-term), only the long-term survival matters. In this case, a Bayesian with an inaccurate prior will eventually vanish in competition with any learner who converges to the truth (such as a regret minimizer). However, when considering transient states as well, it is possible that a Bayesian  with a prior that is inaccurate but relatively close to the true distribution could hold a significant market share for some time before eventually vanishing. See Section \ref{sec:simulations} for an example of such a scenario.


\subsection{Bayesian Learning with Noisy Updates}\label{sec:noisy-Bayesian}
Next, we consider a different type of imperfect Bayesian learner that does have $q$ in its prior, but in every step performs slightly inaccurate ``trembling hand'' updates. Also here, we demonstrate that the optimality of Bayesian learning is very fragile, even when the correct distribution lies in the support. To model this, we define a noisy Bayesian learner as one who at each step, either slightly over-weights the current observation or slightly over-weights its current prior, such that, in expectation, both the data and the prior receive the correct weights in every step (i.e., the errors in weight have zero mean).

For concreteness, consider the following scenario of a learner attempting to learn a distribution of states. Suppose there are two states, $s_t = 0$ with a fixed probability $q \in (0,1)$, and $s_t = 1$ otherwise. The learner considers two models: $\theta_a = q$, which is the correct model, and $\theta_b \neq q$, with $\theta_b < 1$. The log-likelihood is given by
\begin{align}
    L(s_t) = 
    (1 - s_t) \log\Big(\frac{\theta_a}{\theta_b}\Big) + 
    s_t \log\Big(\frac{1 - \theta_a}{1 - \theta_b}\Big).
\end{align}

Now suppose that in every step $t$ the Bayesian learner performs $\lmb$-noisy updates where it over- or under-weights the data compared to the prior with a small excess weight $\lmbt$, where $\lmbt$ has zero mean. Specifically, for a parameter $\lmb > 0$, $\lmbt= \lmb$ with probability $1/2$ and $\eta_t = -\lmb$ otherwise. 
We find that even a tiny error with zero mean can have a significant impact, essentially breaking the learning process.

\begin{theorem}\label{thm:wrong-update-Bayesians-vanish}
    For any $\lmb > 0$, the Bayesian learner with $\lmb$-noisy updates does not converge. 
\end{theorem}

\begin{proof}
Let $\epsilon > 0$. 
The log-likelihood ratio under the noisy update rule takes the following form:
\begin{align}\label{eq:inacurate-update-log-ratio}
    \log\Big(
    \frac{P_t(\theta_a)}{P_t(\theta_b)}
    \Big) 
    &=  
    (1 + \lmbt) L(s_t) + 
    (1 - \lmbt) \log \Big(\frac{P_{t-1}(\theta_a)}{P_{t-1}(\theta_b)}\Big) \nonumber \\
    &= 
    (1 + \lmbt)\sum_{\tau=0}^{t-1} L(s_{t - \tau}) \prod_{k=0}^\tau (1 - \eta_k)  
     + 
    \log\Big(
    \frac{P_0(\theta_a)}{P_0(\theta_b)}
    \Big)
    \prod_{\tau=0}^t
    (1 - \eta_{\tau}),
\end{align}
where the empty product equals one, and we define $\eta_0 = 0$.
The products can be simplified:
\begin{align*}
    \prod_{\tau=0}^t
    (1 - \eta_\tau)  = 
    (1 - \lmb)^{n_+(t)}
    (1 + \lmb)^{n_-(t)},
\end{align*}
where $n_+(t)$ is a binomial random variable counting the number of times $\eta_{\tau \leq t} = \lmb$, and $n_-(t) = t - n_+(t)$. 
The last term can be written as  
\begin{align*}
    &
    \log\Big(
    \frac{P_0(\theta_a)}{P_0(\theta_b)}
    \Big)
    \cdot 
    (1 - \lmb)^{n_+(t)}
    (1 + \lmb)^{n_-(t)} 
    = 
    \log\Big(
    \frac{P_0(\theta_a)}{P_0(\theta_b)}
    \Big)
    \cdot 
    \Big[
    \Big(
    \frac{1 - \lmb}
    {1 + \lmb}\Big)^{n_+(t)}
    \Big] \cdot
    (1 + \lmb)^t.
\end{align*}

Intuitively, since $\E[n_+] = T/2$, this should be close to $(1 - \lmb^2)^{\nicefrac{t}{2}}$ with high probability as $t \rightarrow \infty$, which converges to zero for $\eta < 1$. Formally, we have the following claim.
    \begin{claim}
        Let $\epsilon > 0$.
        $
        \lim_{t \rightarrow \infty} \Pr \Big[ \left(\frac{1 - \lmb}{1 + \lmb}\right)^{n_+(t)} \cdot (1 + \lmb)^t > \epsilon \Big] = 0.
        $
    \end{claim}
    
    \begin{proof}
        Let $\delta > 0$. We need to show that there exists $T$ such that for all $t > T$, the event
        \[
        \Big(\frac{1 - \lmb}{1 + \lmb}\Big)^{n_+(t)} \cdot (1 + \lmb)^t > \epsilon
        \]
        has probability less than $\delta$. 
        % We will call this the high-probability event. 
        Dividing by $(1 + \lmb)^t$ and taking the logarithm, we get 
        \[
        n_+(t) \cdot \big(\log(1 - \lmb) - \log(1 + \lmb)\big) > \log(\epsilon) - t \log(1 + \lmb).
        \]
        Rearranging, we have
        \[
        n_+(t) < \frac{t \log(1 + \lmb) - \log(\epsilon)}{\log(1 + \lmb) - \log(1 - \lmb)}.
        \]
        Denote $c_{\epsilon} = \frac{\log(\epsilon)}{\log(1 + \lmb) - \log(1 - \lmb)}$,
        so
        \[
        n_+(t) < \frac{\log(1 + \lmb)}{\log(1 + \lmb) - \log(1 - \lmb)} \cdot t - c_{\epsilon}.
        \]
        Next, we observe that the coefficient of $t$ is strictly less than $1/2$. To see this, define
        \[
        c = \frac{1}{2} - \frac{\log(1 + \lmb)}{\log(1 + \lmb) - \log(1 - \lmb)} = -\frac{\log(1 + \lmb) + \log(1 - \lmb)}{2 \big(\log(1 + \lmb) - \log(1 - \lmb)\big)} = \frac{-\log(1 - \lmb^2)}{\text{Positive Number}} > 0.
        \]
        Thus, we obtain that the event we want to bound is
        $
        n_+(t) < \Big(\frac{1}{2} - c\Big)t - c_{\epsilon}.
        $
        Using Hoeffding's inequality, using that $\E[n_+] = \nicefrac{t}{2}$, we bound the probability of this event:
        \[
        \Pr\Big[\big|n_+(t) - \nicefrac{t}{2}\big| > ct + c_{\epsilon}\Big] < 2e^{-\frac{2(ct + c_{\epsilon})^2}{4t \lmb^2}} < 2e^{-\frac{1}{2}ct}.
        \]
        Setting $T = \lceil \frac{2}{c} \ln\left(\frac{2}{\delta}\right)
        \rceil$, we conclude that for all $t > T$, the probability of the event
        \[
        (1 + \lmb)^t \cdot \Big(\frac{1 - \lmb}{1 + \lmb}\Big)^{n_+(t)} > \epsilon
        \]
        is less than $\delta$, as required. Hence, the product approaches zero for large $t$:
        $
        \lim_{t \rightarrow \infty} \prod_{\tau=1}^t (1 - \eta_\tau) = 0
        $ almost surely.
    \end{proof}
    \noindent
    {\em Remark:} It is worth noting here that the almost-sure behavior of the product is very different from its expectation, which equals one: 
    $$\sum_{k=0}^T \binom{T}{k} p^k(1 - \lmb)^k \cdot (1 - p)^{T - k} (1 + \lmb)^{T - k} = \big(p (1 - \lmb) + (1 - p) (1 + \lmb)\big)^T = 1 \ \text{ (for $p = 1/2$)}.$$

    The above claim implies that the second term in Equation (\ref{eq:inacurate-update-log-ratio}) converges to zero almost surely. In other words, the learner forgets the prior, much like a perfect Bayesian learner (albeit at a somewhat faster rate in this case). For the first term of Equation (\ref{eq:inacurate-update-log-ratio})
    $$(1 + \lmbt)\sum_{\tau=0}^{t-1} L(s_{t - \tau}) \prod_{k=0}^\tau (1 - \eta_k),$$
    a similar argument applies: 
    with high probability the product for large values of $\tau$ contributes negligibly to the sum, but there remains a random contribution from the smaller $\tau$ terms.  
    To see why there is no convergence, observe that the coefficient $(1 + \lmbt)$ oscillates randomly around $1$. More formally, let 
    $Y_\tau = \prod_{k=0}^\tau (1 - \eta_k)$, noting that $Y_0 = 1$, and let $Z_t  = \sum_{\tau=0}^{t-1}L(s_{t-\tau})Y_\tau$. Expanding the sum, we have: $Z_t = L(s_t)Y_0 + \dots + L(s_1)Y_{t-1}$ and for the next time step $Z_{t+1} = L(s_t)Y_0 + \dots + L(s_1)Y_{t}$. Since $Y_t = (1-\lmbt)Y_{t-1}$, this simplifies to $Z_{t+1} = (1 - \lmbt)Z_t + L(s_{t+1})$. 
    Clearly, this sequence does not converge, as both $L(s_{t+1})$ and $\lmbt$ take independent random values at each step.
\end{proof}
