%% 
%% Copyright 2007-2020 Elsevier Ltd
%% 
%% This file is part of the 'Elsarticle Bundle'.
%% ---------------------------------------------
%% 
%% It may be distributed under the conditions of the LaTeX Project Public
%% License, either version 1.2 of this license or (at your option) any
%% later version.  The latest version of this license is in
%%    http://www.latex-project.org/lppl.txt
%% and version 1.2 or later is part of all distributions of LaTeX
%% version 1999/12/01 or later.
%% 
%% The list of all files belonging to the 'Elsarticle Bundle' is
%% given in the file `manifest.txt'.
%% 

%% Template article for Elsevier's document class `elsarticle'
%% with numbered style bibliographic references
%% SP 2008/03/01
%%
%% 
%%
%% $Id: elsarticle-template-num.tex 190 2020-11-23 11:12:32Z rishi $
%%
%%
\documentclass[preprint,11pt]{elsarticle}

%% Use the option review to obtain double line spacing
%% \documentclass[authoryear,preprint,review,12pt]{elsarticle}

%% Use the options 1p,twocolumn; 3p; 3p,twocolumn; 5p; or 5p,twocolumn
%% for a journal layout:
%\documentclass[final,1p,times]{elsarticle}
%% \documentclass[final,1p,times,twocolumn]{elsarticle}
%%\documentclass[final,3p,times]{elsarticle}
%% \documentclass[final,3p,times,twocolumn]{elsarticle}
%%\documentclass[final,5p,times]{elsarticle}
%% \documentclass[final,5p,times,twocolumn]{elsarticle}

%% For including figures, graphicx.sty has been loaded in
%% elsarticle.cls. If you prefer to use the old commands
%% please give \usepackage{epsfig}

%% The amssymb package provides various useful mathematical symbols
\usepackage{amssymb}
\usepackage{soul}
\usepackage{xcolor}
\usepackage{booktabs}
\usepackage{multirow}
\usepackage{xcolor}
%\usepackage[demo]{graphicx} % for \includegraphics
\usepackage{lipsum} %for filler text
\usepackage{afterpage}
\usepackage{float} % To control table placement
\usepackage{amsmath}
\usepackage{caption}

%% The amsthm package provides extended theorem environments
%% \usepackage{amsthm}

%% The lineno packages adds line numbers. Start line numbering with
%% \begin{linenumbers}, end it with \end{linenumbers}. Or switch it on
%% for the whole article with \linenumbers.
%% \usepackage{lineno}

%\journal{Transportation Research Part F: Traffic Psychology and Behaviour}

\begin{document}

\begin{frontmatter}

%% Title, authors and addresses

%% use the tnoteref command within \title for footnotes;
%% use the tnotetext command for theassociated footnote;
%% use the fnref command within \author or \address for footnotes;
%% use the fntext command for theassociated footnote;
%% use the corref command within \author for corresponding author footnotes;
%% use the cortext command for theassociated footnote;
%% use the ead command for the email address,
%% and the form \ead[url] for the home page:
%% \title{Title\tnoteref{label1}}
%% \tnotetext[label1]{}
%% \author{Name\corref{cor1}\fnref{label2}}
%% \ead{email address}
%% \ead[url]{home page}
%% \fntext[label2]{}
%% \cortext[cor1]{}
%% \affiliation{organization={},
%%             addressline={},
%%             city={},
%%             postcode={},
%%             state={},
%%             country={}}
%% \fntext[label3]{}

\title{Adoption of AI-Assisted E-Scooters: The Role of Perceived Trust, Safety, and Demographic Drivers}

%Trust and Perceived Safety of Road Users Towards Regular and AI-assisted E-scooters}

%% use optional labels to link authors explicitly to addresses:
%% \author[label1,label2]{}
%% \affiliation[label1]{organization={},
%%             addressline={},
%%             city={},
%%             postcode={},
%%             state={},
%%             country={}}
%%
%% \affiliation[label2]{organization={},
%%             addressline={},
%%             city={},
%%             postcode={},
%%             state={},
%%             country={}}

\author[inst1]{Amit Kumar}

\affiliation[inst1]{organization={William States Lee College of Engineering, University of North Carolina at Charlotte},%Department and Organization
            addressline={9201 University City Blvd}, 
            city={Charlotte},
            postcode={28223}, 
            state={North Carolina},
            country={USA}}

\author[inst2]{Arman Hosseini}
\author[inst1]{Arghavan Azarbayjani}
\author[inst3]{Arsalan Heydarian}
\author[inst1]{Omidreza Shoghli \corref{cor1}}

\cortext[cor1]{Corresponding author. Email: \texttt{oshoghli@charlotte.edu}}

\affiliation[inst2]{organization={Systems and Information Engineering, University of Virginia},%Department and Organization
            addressline={Olsson Hall}, 
            city={Charlottesville},
            postcode={22903}, 
            state={Virginia},
            country={USA}}

\affiliation[inst3]{organization={Civil and Environmental Engineering, University of Virginia},%Department and Organization
            addressline={Olsson Hall}, 
            city={Charlottesville},
            postcode={22903}, 
            state={Virginia},
            country={USA}}
\begin{abstract}
E-scooters have become a more dominant mode of transport in recent years. However, the rise in their usage has been accompanied by an increase in injuries, affecting the trust and perceived safety of both users and non-users. Artificial intelligence (AI), as a cutting-edge and widely applied technology, has demonstrated potential to enhance transportation safety, particularly in driver assistance systems. The integration of AI into e-scooters presents a promising approach to addressing these safety concerns. This study aims to explore the factors influencing individuals’ willingness to use AI-assisted e-scooters. Data were collected using a structured questionnaire, capturing responses from 405 participants. The questionnaire gathered information on demographic characteristics, micromobility usage frequency, road users' perception of safety around e-scooters, perceptions of safety in AI-enabled technology, trust in AI-enabled e-scooters, and involvement in e-scooter crash incidents. To examine the impact of demographic factors on participants' preferences between AI-assisted and regular e-scooters, decision tree analysis is employed, indicating that ethnicity, income, and age significantly influence preferences. 
%The results from the direct effects hypothesis testing model revealed a small but significant influence of perceived safety of road users around e-scooters on perception of safety in AI enabled technology. Most notably, the cause and effect model uncovered strong positive relationships between the perception of safety in AI enabled technology and both trust and willingness to use AI-enabled e-scooter. Additionally, trust in AI-enabled e-scooters strongly predicted willingness to use AI-enabled e-scooter. 
To analyze the impact of other factors on the willingness to use AI-enabled e-scooters, a full-scale Structural Equation Model (SEM) is applied, revealing that the perception of safety in AI enabled technology and the level of trust in AI-enabled e-scooters are the strongest predictors. 
\end{abstract}


%%Graphical abstract
%\begin{graphicalabstract}
%\includegraphics{grabs}
%\end{graphicalabstract}

%%Research highlights
%\begin{highlights}
%\item Ethnicity, income, and age significantly influence AI-assisted e-scooter adoption
%\item Perceived safety of road users around e-scooters impacts trust in AI-enabled e-scooters
%\item Perceived safety in AI-enabled technology influences trust in AI-enabled e-scooters
%\item Perceived safety in AI-enabled technology impacts adoption of AI-enabled e-scooters
%\item Perceived trust in AI-enabled e-scooters impacts willingness to use them
%\end{highlights}

\begin{keyword}
%% keywords here, in the form: keyword \sep keyword
E-scooter, AI\sep Perceived Safety\sep Perceived Trust\sep Adoption, Structural Equation Model\sep Decision Tree Analysis
%AI-assisted e-scooters\sep Willingness to Use \sep Structural Equation Model (SEM) \sep Decision Tree Analysis \sep Trust and Perceived Safety
%% PACS codes here, in the form: \PACS code \sep code
%\PACS 0000 \sep 1111
%% MSC codes here, in the form: \MSC code \sep code
%% or \MSC[2008] code \sep code (2000 is the default)
%\MSC 0000 \sep 1111
\end{keyword}

\end{frontmatter}

%% \linenumbers

%% main text
\section{Introduction}
Electric scooters (e-scooters) have emerged as a transformative mode of mobility experiencing a remarkable rise globally, becoming a prominent feature in urban transportation networks \cite{shaheen2020sharing}. 
Their adoption has been particularly evident in US, where micromobility trips increased from 321,000 in 2010 to 157 million in 2023, with shared e-scooters playing a key role in this growth \cite{nacto2024shared}.
E-scooters offer a convenient and sustainable alternative to traditional urban transit, providing essential first- and last-mile connectivity \cite{burt2023scooter}.
Their ease of use has made them a viable option for short-distance travel, often surpassing the utility of shared bicycle services \cite{hardt2019usage}.

However, several challenges hinder their integration into urban transportation systems, creating difficulties for municipal and regulatory agencies in addressing their impacts on urban spaces and road safety \cite{comi2022innovative}. 
%Safety is one of the most significant issues that has emerged with the increase in the use of e-scooters. 
One of the most pressing issues associated with the increasing use of e-scooters is safety. Infrastructure inadequacies exacerbate these challenges, as many cities lack dedicated lanes for e-scooters, forcing riders to share roads with vehicles and sidewalks with pedestrians \cite{chen2024impact}. The predominance of car-centric urban design often creates hazardous situations, with vehicles posing substantial risks to e-scooter users. A primary concern among riders is the threat of being struck by a moving vehicle or colliding with one \cite{sievert2023survey}.
%Infrastructure inadequacies further compound these challenges. Many cities lack designated lanes for e-scooters, forcing them to share roads and sidewalks with cars and pedestrians respectively\cite{chen2024impact}. The car-centric urban design often leads to precarious situations, with vehicles posing significant threats to e-scooter users. A substantial concern of e-scooter riders is being hit by a moving vehicle or colliding with one \cite{sievert2023survey}.
Additionally, the absence of well-defined laws for e-scooter usage creates ambiguity, with regulations on parking zones, speed limits, and permitted riding areas varying across locations \cite{arun2021systematic}.
%A growing trend of crashes involving e-scooters has been observed, with studies revealing that 
Riders are particularly vulnerable to traumatic injuries due to the lack of protective gear and the ergonomic design of e-scooters \cite{trivedi2019craniofacial}. 
The spontaneous nature of e-scooter usage discourages helmet use, as riders typically do not carry helmets for impromptu trips \cite{serra2021head}. 

The safety challenges of e-scooters are not limited to the e-scooter riders. Riding e-scooters on sidewalks has led to discomfort and crashes between riders and other vulnerable road users, particularly pedestrians. Accidents involving collisions with vehicles or infrastructure have raised liability issues and highlighted the inexperience of some riders \cite{kazemzadeh2023electric}. 
%In the case of off-road facilities, 
The speed differential between pedestrians and e-scooter users can result in heightened collision risks, as approximately 40\% of e-scooter trips occur within four feet of a pedestrian \cite{haworth2021comparing,feng2022estimating}. 

Although these challenges persist, significant advancements in AI and autonomous vehicle (AV) technologies provide a promising framework for addressing such issues. In particular, 
%the rise of Advanced Driver Assistance Systems (ADAS) showcases the potential of 
integrating cutting-edge sensors and AI-assisted driving technologies has shown potential to enhance safety. Advanced driver assistance systems, operating at various levels of autonomy, can identify and mitigate potential hazards, alert drivers to dangers, and assist in accident prevention \cite{greenblatt2015automated}. E-scooters can benefit from these technologies by incorporating AI-enabled systems such as real-time object detection \cite{chen2024performance}, collision avoidance \cite{li2023modeling}, and adaptive speed controls. 
%The safety concerns surrounding e-scooters can be significantly mitigated by leveraging these technologies to ensure safer interactions with pedestrians, cyclists, and other road users while fostering broader acceptance and integration of e-scooters into urban environments. 

Despite the numerous studies on e-scooter adoption \citep{mckenzie2020urban, rejali2021assessing}, research on trust and perceived safety has primarily centered on regular e-scooters. This has left the impact of AI integration into e-scooters on road users' perceptions under-explored \citep{samadzad2023factors, ari2024investigating}. There is limited understanding of the adoption of AI-assisted e-scooters and the factors that influence the willingness to use them.
This paper addresses these gaps by answering the following research questions: (1) What is the impact of sociodemographic characteristics on adoption of AI-enabled e-scooters? (2) How do individual perceptions, behaviors, and trust dynamics interact and shape the decision to adopt AI-enabled e-scooters?
 
 %how factors such as micromobility usage frequency, safety perceptions around e-scooters, perceptions of AI technology, trust in AI-enabled e-scooters, and willingness to use AI-enabled e-scooters interact with one another; and (3) which factors are the most influential in shaping people's willingness to use AI-enabled e-scooters. 

%NOTE: This paragprah should frame the gaps in the body of knowlede and be closely conencted to the research obejctvies and results of this stuyd. This includes:
%- impact of race, age, and other demographic factors on willingenss to use of both regulare and ai enabled e-scooters
% Trust and percieved safey of road users around reguarlar and ai-enabled e-scooters
% other gaps that this apper is addressing


%use and the interrelationships among various saftey considerations and perceptions. When the literature is examined, it has been observed that factors affecting the use of electric scooters are primarily associated with perceptions of safety \cite{hardt2019usage}. 
%Recent studies have revealed the behavioral purpose of use, and stated the most important factors affecting users' decisions were perceived usefulness and trust \cite{samadzad2023factors,ari2024investigating}. Building upon this foundation, this study aims to investigate the complex interplay between micromobility use patterns, safety perceptions, and trust in AI technology, particularly as e-scooters evolve to incorporate AI-enabled features. Understanding these relationships is crucial for the future development and implementation of AI-enhanced micromobility systems, as it will help bridge the gap between current usage patterns and future technological adoption.


%The remainder of this paper is organized as follows. Section \ref{sec:literature} reviews the relevant literature and provides necessary background information. Section \ref{sec:distribution} describes the survey data and characteristics of the participants. Section \ref{sec: Method} introduces the research methodology and presents the results of the analysis. Section \ref{sec:discussion} discusses the implications and findings of the study. Finally, conclusions and directions for future research are provided in Sections \ref{sec:conclusion} and \ref{sec:limitation}, respectively.

\section{Literature Review} \label{sec:literature}


\subsection{Perceptions of Safety in E-Scooter Use: Rider and Non-Rider Perspectives}
Safety is a critical concern regarding e-scooter use, and many studies focused on injury and crash data to address these concerns. However, another equally important aspect is the perception of safety, including how both users and non-users view e-scooters in terms of safety. This includes factors such as riding behavior, interactions with other road users (such as pedestrians, cyclists, and drivers), and the general feeling of safety while riding.

From the riders' perspective, one of the most significant safety concerns is road infrastructure. Many riders report a low sense of safety when navigating major streets, largely due to the risk of collisions with vehicles. A particularly prominent concern is the possibility of encountering potholes, which can destabilize the rider and lead to crashes \cite{sievert2023survey}. Both riders and non-riders perceive riding e-scooters in vehicle lanes as unsafe \cite{pourfalatoun2023shared}, reflecting a shared apprehension about the risks posed by mixed-traffic environments.
%Additionally, severe vibration events, often caused by uneven road surfaces or poor infrastructure, can result in a loss of control and accidents \cite{sievert2023survey}.
The study by Tian et al. \cite{tian2022characteristics} revealed that, although sidewalks accounted for the majority of e-scooter-related injuries (44\%, primarily minor injuries), they are still perceived by riders as the safest type of road infrastructure. This perception likely stems from the reduced interaction with vehicles, which riders associate with a lower risk of severe collisions, despite the higher incidence of minor injuries on sidewalks.

Several studies highlighted the safety concerns that e-scooters pose from the perspective of other road users. Všucha et al. \cite{vsucha2023scooter} conducted surveys across five countries, finding that non-riders generally perceived e-scooter use as dangerous, with women and older individuals more likely to consider them unsafe.
Derrick et al. \cite{derrick2020perceptions} surveyed participants in Singapore to assess pedestrian safety concerns and support for an e-scooter ban. The findings showed that 64\% of respondents supported a ban on e-scooters, citing their danger to pedestrians. Key factors influencing this support included age, negative personal experiences, and social norms (e.g., family or peer opinions), and speeding was identified as the most common safety concern.
James et al. \cite{james2023pedestrians} reported that pedestrians and drivers felt less safe walking or driving around dockless e-scooters compared to bicycles. In another study, most pedestrians and cyclists perceived abandoned e-scooters on sidewalks as hazards to other road users \cite{burt2023scooter}.
%This discomfort was likely due to the novelty of e-scooters, which reduces familiarity and confidence in navigating shared spaces with them. Increased exposure, as seen with bicycles, might mitigate these concerns over time.Pedestrians and cyclists viewd abandoned e-scooters obstructing sidewalks. Burt et al. \cite{burt2023scooter} found that 78\% of participants considered abandoned e-scooters a hazard to other road users. 
%Additionally, tandem riding—carrying a passenger—was identified as a factor contributing to both injuries and heightened perceived risk.

%Several studies have explored the perception of risky behaviors associated with e-scooter riders. Gioldasis et al. \cite{gioldasis2021risk} investigated three key risky behaviors: riding under the influence of alcohol, drug use, and smartphone use while riding. The survey revealed that young and male riders are more likely to engage in these behaviors, with longer trips correlating with increased risk-taking.
%Similarly, Burt et al. \cite{burt2023scooter} highlighted risky behaviors such as riding on pavements, alcohol consumption, and minimal helmet use. An observational study in Berlin, Germany\cite{siebert2021safety}, documented a high prevalence of rule violations among shared e-scooter users. Over 25\% of riders were observed misusing infrastructure, while 10\% rode against traffic flow, 5\% engaged in dual occupancy, and none used helmets.
%Regarding helmet usage, Pourfalatoun et al. \cite{pourfalatoun2023shared} found differing perceptions between users and non-users. Non-users placed greater emphasis on the importance of helmet use, while users expressed more moderate views about its necessity. These findings underscore the need for targeted interventions to address risky behaviors and improve safety awareness among e-scooter riders.

Although previous studies examined safety perceptions of regular e-scooters, the perception of safety in AI-integrated e-scooters remains unexplored. Further studies are required to assess how safety perceptions of AI enabled technology and road users' views on e-scooters influence overall safety perception and trust in AI-enabled e-scooter. 

%Safety is one of the most important issues that has emerged with the rise in the use of E-scooters. The use of scooters on sidewalks is arguably the most contentious issue and source of frustration for local officials. A separate category of problems arises with the introduction of a new mode of transportation, including compatibility issues with existing modes and inadequate riding/driving experience for users.
%Accidents involving pedestrians and cyclists have caused injuries and increased responsibility concerns in the communities. It is possible to attribute part of the misuse of the dockless vehicles to users' inexperience with them and the city's regulations governing their use. The users must ultimately educate themselves about the many city regulations governing the use of dockless automobiles. The manufacturers of E-scooters have also made a number of attempts to inform the public about neighborhood laws and the risks associated with riding on sidewalks \Cite: Kazemzadeh et al. (2023).
%Another inherent challenge associated with the use of E-Scooter emerges from the absence of adequate infrastructure in many communities to support alternative transportation methods. Typically designed around accommodating cars, these transportation networks face potential issues when E-Scooter vehicles begin utilizing the same street space. The prevalent car-centric layout in numerous cities might lead to unsafe or precarious situations. In fact, cities might witness an increased risk where cars pose a threat to E-Scooter vehicles on the streets, additional to the already existing bicycles and scooters present on the sidewalks. This unfortunate reality is highlighted often in the form of collisions with normal cars and E-Scooter automobiles. Drivers, unaccustomed to sharing the road with diverse vehicles, often encounter challenges when small, vulnerable scooters and bicycles operate within the same zones as cars, resulting in accidents and, tragically, fatalities. This issue is further complicated by the absence of an efficient accident tracking system implemented by scooter companies \cite Folco et. al 2023
%The obstacles encountered have motivated numerous cities to invest in specific infrastructure capable of supporting alternative transportation methods. Several cities have initiated the establishment of bicycle lanes, either by re purposing areas that were once designated for curbside parking or by introducing physical barriers between bicycle lanes and lanes for vehicles. The configuration of a bike lane requires a thorough consideration of existing traffic levels and behaviors, adequate safety buffers to protect bicyclists and other small mobility users from parked and moving vehicles, and enforcement to prohibit motorized vehicle encroachment and double-parking. Bike Lanes may be distinguished using color, lane markings, signage, and intersection treatments \Cite: Hossein et al. (2023).

%Such policies and initiatives contribute to fostering a stronger E-scooter riding culture by simplifying, securing, and enhancing the efficiency of E-scooters and the use of alternative modes. With an increasing number of residents opting for alternative modes of transport, motorists will gradually grow accustomed to sharing road space. This adjustment holds the potential for a cumulative positive impact on safety and the environment.
%Additionally, another significant safety concern confronting providers and cities involves the usage of helmets. Many injuries associated with scooters directly result from riders not wearing helmets. The lack of legal obligation to wear a helmet and other personal protective equipment’s along with the absence of an adequate and feasible concept of protective equipment for sharing services are the main barriers to helmet use among riders. However, the nature of shared systems offers pedestrians the liberty to utilize a bike or scooter at their convenience, granting considerable freedom but also potentially exposing riders to unanticipated risks. Typically, a traditional cyclist, using their personal bike, is more inclined to possess their own helmet compared to users of dockless devices. These users engage with the vehicles spontaneously and might not prefer carrying bulky helmets without certainty regarding their use of a scooter \Cite: Serra et al. (2021). 
%While scooter providers strongly encourage riders to wear helmets while using their services, they do not push for cities to enforce helmet mandates. Addressing this challenge proves notably complex because these providers lack a system for enforcement and benefit from maintaining riders' flexible engagement with their vehicles. This situation underscores the challenging position that scooters pose for local law enforcement and traffic officials. Enforcing helmet usage demands time, and instances of non-compliance are widespread. Numerous cities continue to struggle with identifying appropriate regulatory measures to enhance resident safety in this area. 
\subsection{Perception of Safety and Trust in AI-assisted Vehicles}
The level of trust in AI-assisted vehicles has been a significant topic of interest since the advent of autonomous vehicles (AVs). Public concerns about AVs are often amplified by reports of accidents involving these vehicles, which can further hinder their acceptance. Previous experiences with AVs play a crucial role in shaping public attitudes toward this technology. Individuals with prior exposure to AVs tended to exhibit more positive perceptions and a greater willingness to adopt them compared to those without such experiences \cite{othman2021public}. Previous studies examined the impact of sociodemographics on trust in AI-assisted vehicles. One study revealed that individuals aged 36 to 65 expressed greater apprehension and resistance toward driving autonomous vehicles compared to both younger individuals aged 18 to 35 and older individuals aged 65 and above \cite{thomas2020perception}. In other study, older people generally held more pessimistic views about AVs \cite{othman2021public}. Gender differences also emerged, with males expressing more positive attitudes toward AVs than females.
In terms of education, individuals with a university degree (Bachelor's, Master's, or PhD) exhibited lower levels of concern regarding accident liability and system failures in autonomous vehicles compared to those without a degree \cite{thomas2020perception}. This finding aligns with another study, which found that people with higher education levels are more likely to have a positive view of AVs than those with lower levels of education \cite{othman2021public}. Interestingly, individuals in countries with lower GDP levels tended to have more positive attitudes toward AVs compared to those in medium- or high-GDP countries\cite{othman2021public}.

%A review study by Othman et al. \cite{othman2021public} highlights several intriguing insights into public attitudes toward autonomous vehicles (AVs). Interestingly, individuals in countries with lower GDP levels tended to have more positive attitudes toward AVs compared to those in medium- or high-GDP countries. Moreover, the survey results indicated that older people generally held more pessimistic views about AVs.
%Willingness to pay for new technology is another critical factor influencing its adoption. However, surveys reveal that only a small percentage of individuals are willing to pay a premium for AVs. Gender differences also emerge, with males expressing more positive attitudes toward AVs than females. Similarly, individuals with higher levels of education are more likely to view AVs favorably compared to those with lower educational backgrounds \cite{othman2021public}.

% Hilgarter et al. \cite{hilgarter2020public} compared safety perceptions of autonomous vehicles (AVs) in rural and urban environments, revealing that AVs receive a more favorable reception in rural areas than in urban settings. In rural regions, AVs are seen as having the potential to shift individuals from private car use to public transportation. Notably, survey participants generally view AVs as an alternative to, rather than a complete replacement for, existing transportation methods.

Pyrialakou et al. \cite{pyrialakou2020perceptions} examined safety perceptions among vulnerable road users interacting with AVs. Their findings highlighted gender differences, with females feeling less safe around AVs than males. Among various activities, cycling near AVs was perceived as the least safe, followed by walking and then driving. The study also found that feeling safe near AVs was positively associated with reduced concerns about threats such as hacking or terrorism. Additionally, direct experience with AVs significantly improved pedestrian safety perceptions.

Previous studies primarily examined the trust and safety perceptions in autonomous vehicles, with limited attention to other AI-assisted modes of transport. To the best of our knowledge, no study has specifically investigated the perception of trust and safety in AI-assisted e-scooters, either from the perspective of riders or other road users. This study explores the level of trust in AI-assisted e-scooters and its influence on the willingness to adopt this emerging system.


%Hilgarter et al. explored public acceptance of autonomous vehicles (AVs) and perceived passenger safety after riding an autonomous shuttle in Carinthia, Austria, through interviews with 19 participants. The findings show that AVs are more positively received in rural areas, where they are seen as a way to shift from private cars to public transportation, and older adults perceive more benefits from AVs than younger people, particularly in sustaining mobility. While AVs are viewed as an alternative to existing transport, respondents emphasized that societal challenges like job loss, privacy concerns, and the high cost of AVs must be addressed. Safety perceptions were largely positive, with 84\% of participants feeling safe, influenced by the slow speed of the shuttle (10 km/h) and the experience of riding an AV, which improved their confidence in the technology. However, some concerns about hacking and the need for better public awareness and training were noted \cite{hilgarter2020public}.Othman et al. conducted a literature review to examine survey papers on public perceptions of autonomous vehicles (AVs), focusing on how safety, ethics, liability, and regulations impact AV acceptance. The review finds that public concern about AVs rises with reports of accidents, and those with prior experience with AVs are more accepting. Interestingly, while older adults are considered early adopters due to increased accessibility, surveys reveal they are the most pessimistic toward AVs. Willingness to pay for AVs remains low, with men and individuals with higher education levels being more positive. Concerns about cybersecurity are widespread, and people in lower-GDP countries tend to be more optimistic about AVs than those in wealthier nations \cite{othman2021public}.Pyrialkou et al. conducted a survey study in Phoenix, Arizona, exploring the safety perceptions of road users interacting with autonomous vehicles (AVs), including drivers, cyclists, and pedestrians. The findings reveal that feeling safe near AVs is positively linked to a lack of concern about threats like hacking or terrorism, while direct experience with AVs improves pedestrian safety perceptions. However, distrust in strangers and a focus on travel time negatively impact these perceptions. Cycling near AVs is seen as the least safe activity, followed by walking and then driving. Females are less likely than males to feel safe around AVs, and overall exposure and experience with AVs can have both positive and negative effects on safety perceptions  \cite{pyrialakou2020perceptions}.The study by Thomas et al. explored public perceptions of the benefits and challenges of autonomous vehicles (AVs), focusing on sociodemographic factors and acceptance at both individual and societal levels through an online survey. Respondents identified key concerns, including AV malfunctions, crash risks, purchase price, liability for incidents, interactions with non-AV vehicles, unexpected situations, hacking, and overall safety. While AVs were generally perceived as low-risk to drive, those with university degrees were less concerned about system failures and liability issues. In contrast, older respondents (36–65 years) expressed more concern and were less willing to drive AVs compared to both younger (18–35 years) and older (65+) age groups \cite{thomas2020perception}.

\subsection{Adoption of E-scooter}
Previous studies explored factors influencing the adoption of electric scooters. 
Sanders et al. \cite{sanders2020scoot} investigated the willingness to use e-scooters in Arizona, U.S., revealing significant gender-based differences in barriers to usage, especially regarding safety concerns. Moreover, African American and non-white Hispanic respondents were more likely than non-Hispanic white respondents to express an intention to try e-scooters. Nikiforiadis et al. \cite{nikiforiadis2021analysis} examined shared e-scooter usage in Thessaloniki, Greece, and found that females were less inclined to use e-scooters compared to males.
Teixeira et al. \cite{teixeira2023barriers} conducted a survey across five European capital cities to explore the barriers preventing non-users from adopting e-scooters. The results implied that these obstacles are largely external and infrastructural, including the convenience of alternative transport modes, safety concerns about riding in traffic and inadequate road conditions. Similarly, She et al.\cite{she2017barriers} identified barriers to the widespread adoption of electric scooters and found that younger individuals showed more positive attitudes toward using them compared to older generations.

Structural Equation Modeling (SEM), often based on the extended Technology Acceptance Model (TAM), is a widely used approach for analyzing factors influencing e-scooter adoption\cite{ari2024investigating,samadzad2023factors,javadinasr2022eliciting}. Ari et al. \cite{ari2024investigating} found that social influences and enjoyment were the most significant predictors of perceived ease of use and perceived usefulness. Samadzad et al. \cite{samadzad2023factors} highlighted that perceived usefulness, trust, and subjective norms are key factors shaping the adoption and willingness to use shared e-scooters. Javadinasr et al. \cite{javadinasr2022eliciting} focused on factors driving the continuous use of e-scooters in Chicago, showing that perceived usefulness is the most influential factor, followed by perceived reliability.

Despite significant research on the adoption of regular e-scooters, no studies, to the best of our knowledge, have specifically examined the willingness to use AI-assisted e-scooters. Key questions remain unresolved, such as identifying the profiles of individuals who prefer AI-assisted e-scooters over regular ones and uncovering the factors that influence the willingness to adopt this new system.
%The examination of previously published literature highlights how various factors such as perceived usefulness, perceived ease of use, environmental concerns, lack of appropriate infrastructure. performance, social impact, and safety problems for electric scooter user affect people's attitudes towards using electric scooters (Eccarius and Lu, 2020 \cite{eccarius2020adoption}; Sovacool et al 2019 \cite{sovacool2019pleasure}; Sun et al, \cite{Sunetal2011} Erkan and Ari \cite{ari2024investigating}).


%\subsection{Use of E-scooter}Previous studies explored the factors shaping e-scooter usage, focusing on sociodemographic characteristics, general attitudes, and usage patterns.
%While studies from different regions reveal some consistent trends, they also highlight unique findings specific to particular geographic contexts.Christoforou et al. conducted a survey study in Paris, France, revealing that most e-scooter users are male, aged between 18 and 29, possess a high level of education, and are less likely to own their own e-scoooter \cite{christoforou2021using}.
% Another research by Almanaa et al. \cite{almannaa2021perception} in Saudi Arabia indicated that males are more likely to use e-scooters than females, with the majority of potential users falling within the 18 to 45 age range.Similarly, Nikiforiadis et al. \cite{nikiforiadis2021analysis} examined shared e-scooter usage in Thessaloniki, Greece, and found that females were less inclined to use e-scooters compared to males. Furthermore, their study revealed that individuals residing in downtown areas are more regular e-scooter users compared to those living farther from the city center.


%Similarly, a study conducted in Colorado, U.S., highlighted the profile of e-scooter users as predominantly younger, employed individuals residing in urban areas with longer commutes \cite{pourfalatoun2023shared}. Compared to non-users, e-scooter users generally held more positive perceptions regarding the safety, sanitary conditions, comfort while riding intoxicated, and overall usability of shared e-scooters. 

%Research on the perceived usefulness of e-scooters has uncovered several key insights. A survey conducted in the United Kingdom found that e-scooters are widely viewed as an efficient, sustainable, and practical solution for first- and last-mile transportation needs \cite{burt2023scooter}.Despite the extensive body of research on the use of regular e-scooters, to the best of our knowledge, no studies have specifically examined the willingness to use AI-assisted e-scooters. Critical questions remain unanswered, including identifying the profiles of individuals who favor AI-assisted e-scooters over regular ones, exploring general perceptions surrounding the use of AI-assisted systems, and uncovering the factors that drive willingness to adopt these cutting-edge technologies.


\section{Model Structure and Hypothesis}
To address the research questions, we developed two models: (1) a Decision Tree Classifier to find patterns in sociodemographic features that shape participants' preference between regular vs AI-assisted e-scooters. (2) Structural Equation Modeling (SEM) to detect factors significantly impacting the willingness to use AI-assisted e-scooters. 
 
The SEM framework was designed to analyze the causal relationships between latent variables, enabling the assessment of both direct and indirect effects within a theoretical model. By incorporating structural paths, we aimed to evaluate the underlying mechanisms shaping participants’ attitudes and behaviors toward AI-assisted e-scooters. Specifically, the model considers five key constructs: (1) Frequency of regular micromobility use, (2) Perceived safety of road users around e-scooters, (3) Perception of safety in AI-technology, (4) Perception of trust in AI-enabled e-scooter, (5) Willingness to use AI-enabled e-scooter.
%Structural paths were developed to examine cause-and-effect relationships between latent variables, allowing to test theoretical models by specifying and evaluating direct and indirect relationships within a network of variables. 
%SEM is widely used in social sciences, business, and education for its ability to model measurement errors and assess causal pathways simultaneously \cite{kline2023principles}. 
%We focused on the following factors: 
%\begin{enumerate}
%    \item Frequency of regular micromobility use
%    \item Perceived safety of road users around e-scooters
%    \item Perception of safety in AI-technology
%    \item Perception of trust in AI-enabled e-scooter
%    \item Willingness to use AI-enabled e-scooter
%\end{enumerate}
Figure \ref{SEM modelZ} illustrates the proposed constructs of our model along with their hypothesized relationships. The rationale for selecting these factors, as well as the detailed explanation of the proposed hypotheses, are provided in the following subsections.
%Figure \ref{SEM modelZ} presents the proposed factors (constructs) of our model along with their hypothesized relationships. The logic behind choosing these factors and proposed hypotheses are illustrated in details in following subsection.

\begin{figure}[!ht]
    \centering
    \includegraphics[width=0.8\linewidth]{The_Proposed_Model.png}
    \caption{The proposed model to assess willingness to use AI enabled driving assistance technology}
    \label{SEM modelZ}
\end{figure}
    
\subsection{Frequency of Regular Micromobility Use}
The frequency of micromobility use varies significantly across different urban environments and demographic groups. In dense city centers, many people use bicycles, e-scooters, and e-bikes multiple times per week or even daily for commuting and short trips, particularly in areas with well-developed cycling infrastructure and favorable weather conditions \citep{blazanin2022scooter,bachand2012better, xu2019analysis}.
At the same time, the rapid integration of artificial intelligence (AI) technologies into micromobility solutions, including AI-assisted e-scooters, holds the potential to significantly influence urban mobility landscapes in the near future \cite{alp2025micromobility}. AI-powered features such as real-time obstacle detection, adaptive speed control, and collision detection aim to enhance both the safety and trust of vulnerable road users (VRUs). However, perceptions of these advancements which are key in technology acceptance are underexplored.

Users with frequent engagement in regular e-scooters and shared micromobility may develop a more nuanced understanding of road risks, operational dynamics, and vehicle behavior, potentially influencing their perception of AI-enhanced safety features. Similarly, regular micromobility users may develop adaptive strategies for interacting with e-scooters, fostering greater acceptance. However, non-users or infrequent users may be more likely to associate e-scooters with concerns such as pedestrian safety conflicts, regulatory ambiguity, and infrastructure inadequacy. This suggests that prior micromobility engagement may influence attitudes toward AI-enabled e-scooters \cite{jevinger2024artificial}. 

Trust in AI-powered mobility solutions is a critical determinant of adoption intention \cite{kuberkar2020factors}. Frequent users may trust AI for better navigation and safety, while less experienced users may doubt its ability to handle urban complexities and non-motorized interactions.
Based on this information, we propose the following hypothesis:
%Research indicates that e-bike users tend to ride more frequently and for longer distances than traditional cyclists, with many users reporting three to five trips per week. E-scooter usage shows more variability, with higher frequency during warmer months and weekends \cite{kimpton2022weather}, and particular popularity among younger users for last-mile connectivity and leisure trips. Factors such as weather, seasonality, and local infrastructure quality significantly influence usage patterns, with cities like Amsterdam showing consistently high year-round micromobility use due to their robust cycling infrastructure and supportive urban planning \cite{podesta2022decision}.

\begin{itemize}
    \item H1: Frequency of regular micromobility use positively influences road users' perception of safety in AI enabled technology.
    \item H2: Frequency of regular micromobility use positively influences perceived safety of road users around e-scooters.
    \item H3: Frequency of regular micromobility use positively influences perception of trust in AI-enabled e-scooter.
\end{itemize}

\subsection{Perceived safety of road users around e-scooters}
%It is common for a general commuter to come across an e-scooter in the urban environment \cite{james2019pedestrians}. In 2022, Useche et al. \cite{useche2022unsafety} investigated road interactions with bicycle and e-scooter riders, and their surrounding perceptions of riding safety. 
Public attitudes toward regular e-scooters may shape how users evaluate AI-driven E-Scooters. 
Perceived safety of road users around e-scooters aims to evaluate how different road users perceive their safety when e-scooters are in their proximity. As a result, we need to assess the safety perception of other road users, incluing: pedestrians, cyclists, e-bike riders, vehicle drivers, and e-scooter riders, in respone to presence of regular e-scooters.
%Participants are asked specific questions, such as, "When driving a car, how safe do you feel when e-scooters are around you?" Findings from the study conducted by \cite{al2024assessing} shows that e-scooter riders are typically aware of physical dangers to their safety from other road users and made a case for suitable infrastructure to separate them from motor vehicles and pedestrians. 
%, the more autonomous and anthropomorphic a system is, the higher its safety and trust levels \cite{watanabe2021cheaper}. 
%The questions assessing the impact of these five type of road users  explore safety perceptions of various road users, providing an understanding of how e-scooters influence feelings of security, comfort, and willingness to adopt among different road users. 
Positive experiences with traditional e-scooters could lead to greater openness toward AI enhancements, while negative perceptions may result in hesitation or resistance to adopting AI-powered alternatives, and lack of trust in AI-enabled technologies.
Based on this construct model, we propose the following hypothesis: 
\begin{itemize}
    \item H4: Perceived safety of road users around e-scooters positively influences perception of safety in AI-enabled technology.
    \item H5: Perceived safety of road users around e-scooters positively influences perception of trust in AI-enabled e-scooter.
\end{itemize}

\subsection{Perception of Safety in AI-enabled technologies}
Perception of Safety in AI-enabled technologies assesses individuals' confidence in the ability of AI systems in vehicles to operate securely and minimize risks in real-world scenarios. It explores users' beliefs about the safety measures, reliability, and decision-making accuracy of AI-powered solutions across different applications, particularly in mobility and transportation. Previously, Jevinger et al.\cite{jevinger2024artificial} used machine learning technique to present a case for AI improving public transport safety while Azarbayjani \cite{azarbayjani2024trust} investigated the trust and perceived safety of vulnerable adult road users towards regular and AI-enabled E-scooters. A strong perception of safety is essential for building user trust, as it alleviates concerns about potential errors, malfunctions, or unforeseen hazards. This construct is crucial for evaluating public acceptance and safety in AI enabled technologies, particularly because concerns about system failures, unpredictability, or decision- making in AI algorithms may erode trust, leading to reluctance in adoption.
Based on this construct model, we propose the following hypothesis: 
\begin{itemize}
    \item H6: Perception of safety in AI-enabled technologies positively influences perception of trust in AI-enabled e-scooter.
    \item H7: Perception of safety in AI-enabled technologies positively influences willingness to use AI-enabled e-scooter.
\end{itemize}

\subsection{Perception of Trust in AI-Enabled E-Scooter}
Trust is the most relevant psychological state which is derived from the willingness to accept vulnerability and has been identified as a key factor in the adoption and use of technology \citep{rousseau1998not,hohenberger2017not}. 
The perception of trust in AI-enabled e-scooters construct focuses on evaluating users' perception of the ability of the AI technology integrated into e-scooters to handle unpredictable situations, functionality, and maneuvering. It examines how well users trust the AI to enhance safety, provide accurate navigation, and ensure smooth operation in various environments. Xu et al. \cite{xu2019analysis} identified that trust has a strong and positive correlation with both perceived usefulness and ease of use. Trust is crucial for fostering widespread adoption of AI-enabled e-scooters, as it may influence users' willingness to use AI-enabled e-scooters. 
Based on this construct model, we propose the following hypothesis: 
\begin{itemize}
    \item H8: Perception of trust in AI-enabled e-scooter positively influences willingness to use AI-enabled e-scooters.
\end{itemize}

%\subsection{Willingness to use AI enabled e-scooter} The construct Willingness to Use AI-enabled e-scooter evaluates an individual's openness and intention to adopt AI-driven systems designed to assist in driving. It reflects users' perceptions of the benefits, ease of use, and trustworthiness of AI-enabled driving assisted technologies, as well as their readiness to rely on AI for improved driving efficiency, safety, and convenience. This construct is influenced by factors such as familiarity with AI, confidence in its reliability, and perceived improvement in driving performance and road safety. Understanding this willingness is critical for predicting adoption rates and guiding the development of user-centric AI-enabled driving solutions.

   
%To examine factors that influence the adoption of regular and AI-enabled electric scooters, this study develops a research model that incorporates key determinants such as perceived usefulness, perceived ease of use, environmental concerns, infrastructure availability, performance, social influence, and safety considerations. These factors are selected based on their relevance to user attitudes and behavioral intentions toward electric scooter adoption.
%Additionally, this research identifies a gap in existing studies regarding the role of AI-enabled driving assistance technology in influencing adoption decisions. While various frameworks, including the Theory of Planned Behavior (TPB) and the Unified Theory of Acceptance and Use of Technology (UTAUT), have been utilized to understand user intentions, the integration of AI-driven support systems remains largely unexplored. This study aims to address this gap by extending existing models to include AI-related variables and assessing their impact on user perceptions and willingness to adopt electric scooters.
%By employing Structural Equation Modeling (SEM) for data analysis, this study systematically evaluates the relationships between these variables. The model structure is designed to capture the direct and indirect effects of identified factors on adoption intentions, providing an  understanding of how technological and contextual elements interact in shaping user acceptance of electric scooters.
%In this study, building on the gaps of the previous studies, the effects of perceived safety of road users around e-scooters and experiences on the willingness to use AI enabled e-scooter in the context of e-scooter were investigated. Additionally, the impact of crash experience on the willingness to use AI-enabled e-scooter was also analyzed. Figure \ref{SEM modelZ} presents the proposed constructs of our model along with their hypothesized relationships. Below, we provide detailed explanations of the background and rationale for these hypotheses. This adapted version includes elements such as use frequency, perception, safety, trust, and willingness to use. The rationale for selecting each factor is detailed in the subsequent subsections. The model consists of five components as depicted in Table \ref{tab:reliability-validity}, with its core and primary foundation being the AI technology, enhanced to the willingness to use AI-enabled e-scooter in the context of E-Scooters. Additionally, three key components are integrated: Perceived safety of road users around e-scooters, Perception of Trust in AI-enabled E-Scooter, Perception of Safety in AI enabled Technology, and Use Frequency of regular Micromobility vehicles. Alongside these, previous crash experience is included as an important factor in the full model, collectively serving as effective measures of willingness to adopt AI-enabled Driving Assistance Technology.



\section{Method} \label{sec: Method}
The research methodology of this study involved designing and distributing an online survey to collect data across the United States. The collected data were analyzed using decision tree classification to explore the impact of sociodemographic factors on preferences for AI-assisted e-scooters, and structural equation modeling (SEM) to examine complex relationships between perceived safety, trust, and willingness to use AI-enabled e-scooters. In the following sections, we explain the detailed methodology used for survey design, distribution, participant selection, data analysis, and model development.

\subsection{Survey Design}\label{sec:design}
To collect data for the model outlined in the previous section and test the hypotheses, we developed a survey questionnaire aligned with the model's structure and hypotheses. The full list of questions used for hypothesis testing can be found in Table \ref{tab:reliability-validity}. The survey is organized into four sections as follows.
%The survey questionnaire was designed based on the model structure and hypothesis in four main sections:
\begin{enumerate}
\item Perceived safety of road users around e-scooters:
The survey began by exploring participants' interactions with e-scooters, tailored to their role as road users—pedestrians, car drivers, bike riders, e-bike riders, or e-scooter riders. Respondents were asked about their perceived safety of e-scooters, while those with prior e-scooter experience provided more detailed information, including their frequency of use and any previous accident experiences. These questions are listed from question 1 to question 8 under the Survey Questions in Table \ref{tab:reliability-validity}.

\item Perceptions of safety in AI-enabled technologies:
To assess participants' perceptions of safety in AI-enabled technology, the survey included questions examining their overall safety perceptions, comparative views on autonomous versus human-driven vehicles, and their evaluation of AI-assisted e-scooter features. Responses were collected using both open-ended and Likert scale formats to capture nuanced perspectives on safety in AI-enabled technology. The relevant questions in this section are listed from question 9 to 11 in Table \ref{tab:reliability-validity}.
%The second section focused on participants’ perceptions of safety regarding AI enabled technology and their level of trust in AI-assisted technologies.

\item Perceived trust in AI-enabled e-scooters:
In this section, participants, regardless of their prior experience with e-scooters, were presented with a conceptual scenario involving e-scooters equipped with AI systems. They were asked to share their perceptions of trust and their preferences for how such AI systems could assist them with notifications. The questions were designed to assess participant's confidence in the system's ability to enhance ride safety, handle unexpected situations, and operate reliably without malfunctioning. The appropriate questions for this section are listed from question 12 to 14 in Table \ref{tab:reliability-validity}.

\item Demographic information:
The final section included demographic questions to collect information on participants' personal characteristics, such as age, gender, race, ethnicity, education level, income, and geographic location. These details were essential for understanding the composition of the respondent pool and categorizing populations within the study. By analyzing demographic factors, the study aimed to develop a comprehensive analysis of trends and patterns across different population segments.
%The final section gathered demographic details about the participants to enable a comprehensive analysis of trends and patterns across different population segments.
\end{enumerate}

\subsection{Survey Distribution} \label{sec:distribution}
The online survey was administered using Qualtrics, a cloud-based platform for survey design and distribution. 
%By leveraging Qualtrics' intuitive interface, questions and answer options were carefully designed to ensure clarity and ease of use.
Participants accessed the survey using an active link generated by the platform, which was compatible with a wide range of devices, ensuring accessibility and convenience. The survey was distributed through multiple channels, including email, pamphlets in transit systems including buses and light rails,  social media platforms, public space gatherings including weekend farmer's markets, public parks, and Thanksgiving parade, university classes at the University of North Carolina at Charlotte and the University of Virginia, and campus communications. Each distribution was accompanied by a brief description of the study’s purpose, assurance of confidentiality, and clear instructions for participants. To encourage participation, respondents were offered the option to receive a \$3 online gift card incentive upon completion of the survey. Ethical approval for the study was granted by the Institutional Review Boards of the University of North Carolina Charlotte (IRB-24-0118) and the University of Virginia (IRB-6120).
Data collection spanned a 12-month period, from November 2023 to November 2024. The survey, with an estimaed completion time of 15 minutes, targeted a diverse population of road users across the United States, including both e-scooter riders and non-riders.
 
\subsection{Participants}
Participation was limited to individuals aged 18 or older who resided in the United States. A total of 405 valid responses were collected, comprising 188 e-scooter users (46\%) and 218 non-users (54\%). 
%Among e-scooter users, the majority, 137 participants (31\%), reported using e-scooters rarely. Smaller groups indicated varying usage patterns: 20 participants (5\%) used e-scooters several times a month, 13 participants (3\%) used them several times a week, and 18 participants (4\%) reported daily or near-daily use.
The majority of respondents (56\%) were young adults aged between 18 and 25, which aligns with the target population of e-scooter users.
%, as most participants were undergraduate and graduate students from the University of Virginia and UNC Charlotte. 
A detailed distribution of age ranges is provided in Table \ref{tab:demographics}.

Regarding gender identity, 233 (58\%) participants identified as male, 162(40\%) as female, 2 as non-binary, and 8 (2\%) preferred not to disclose their gender. The racial distribution of the sample was as follows: (42\%) identified as White, (24\%) as Asian, (11\%) as Hispanic or Latino, (9\%) as African American,(9\%) as Middle Eastern, (2\%) as others, and (2\%) of participants preferred not to disclose their ethnic background. A summary of the respondents' demographics is presented in Table \ref{tab:demographics}.


\begin{table}[H]
\centering
\scriptsize % Further reduces the font size
\setlength{\tabcolsep}{3pt} % Reduces column spacing
\renewcommand{\arraystretch}{1} % Keeps row spacing tight
\caption{Demographic Distribution of Survey Respondents}
\label{tab:demographics}
\resizebox{0.7\textwidth}{!}{ % Resize to 70% of text width
\begin{tabular}{llrr}
\toprule
Variable & Category & Count & Percentage (\%) \\
\midrule
Gender Identity & Female & 162 & 40\% \\
 & Male & 233 & 58\% \\
 & Non-binary & 2 & 1\% \\
 & Prefer not to disclose & 8 & 1\% \\
\midrule
Age & 18--21 & 113 & 28\% \\
 & 22--25 & 112 & 28\% \\
 & 26--30 & 51 & 13\% \\
 & 31--35 & 39 & 10\% \\
 & 36--40 & 31 & 8\% \\
 & 41--50 & 35 & 9\% \\
 & 51 or older & 24 & 3\% \\
\midrule
Racial Background & White/Caucasian & 170 & 42\% \\
 & Asian & 98 & 24\% \\
 & Hispanic/Latino & 44 & 11\% \\
 & Middle Eastern & 35 & 9\% \\
 & African American & 35 & 9\% \\
 & Native American & 2 & 1\% \\
 & Other & 10 & 2\% \\
 & Prefer not to disclose & 11 & 2\% \\
\midrule
Education Level & High school graduate & 99 & 25\% \\
 & Associate degree & 49 & 12\% \\
 & Bachelor's degree & 115 & 28\% \\
 & Master's degree & 94 & 23\% \\
 & Advanced degree & 47 & 12\% \\
\midrule
Income & Less than \$20,000 & 67 & 17\% \\
 & \$20,000 - \$50,000 & 87 & 21\% \\
 & \$50,000 - \$100,000 & 76 & 19\% \\
 & \$100,000 - \$150,000 & 68 & 16\% \\
 & \$150,000 - \$200,000 & 32 & 8\% \\
 & More than \$200,000 & 30 & 7\% \\
 & Prefer not to disclose & 45 & 11\% \\ % Add missing \\
\bottomrule
\end{tabular}
}
\end{table}

Regarding the highest level of education, 28\% of participants held a bachelor's degree, while 25\% reported having a high school diploma or equivalent. Additionally, 23\% had obtained a master's degree, 12\% held an associate degree, 11\% possessed a doctorate, and 1\% reported holding a professional healthcare degree. 

Concerning household income, 17\% of participants reported annual incomes below USD 20,000, 21\% fell within the USD 20,000– 50,000 range, and 19\% reported incomes between USD 50,000 –100,000. Additionally, 16\% were in the USD 100,000–150,000 bracket, 8\% in the USD 150,000 –200,000 range and 7\% indicated annual incomes exceeding USD 200,000. The remaining 11\% preferred not to disclose their income. Further details on education and income are provided in Table 1. 

\subsection{Decision Tree Classification Approach for Evaluating the Impact of Sociodemographic Features}
Decision trees are among the most widely used classification methods due to their simplicity, interpretability, and effectiveness in handling large datasets. 
%They uncover key characteristics and identify useful patterns through an intuitive 'if-else' structure and graphical representation, making results easy to understand and interpret.
As a nonparametric and nonlinear approach, decision trees do not rely on assumptions about data distribution or linearity, making them highly adaptable to various data types while maintaining satisfactory accuracy \cite{priyam2013comparative, de2013decision}.
This section explains the process of designing the decision tree classifier to analyze the influence of sociodemographic characteristics on their preferences for choosing between regular and AI-assisted e-scooters both of which were priced identically. 
Participants were presented with three options to express their choice: (a) I will definitely choose the AI-assisted e-scooter, (b) I am uncertain about the decision, and (c) I would prefer the regular e-scooter. The responses were analyzed in relation to various demographic factors, including age, gender, race, education level, and income, to identify any potential correlations. A classification tree analysis drawing from the methodology developed by Nikiforiadis et al.  \cite{nikiforiadis2021analysis} was conducted utilizing Python and scikit-learn library to examine these relationships. The distribution of choices among participants revealed that 205 observations were recorded for AI-assisted e-scooters, while 104 participants indicated they were unsure, and 93 observations were noted for regular e-scooters.

%For data preprocessing, fortunately, there were no missing values in the demographic data. 
%The distribution of choices was as follows:
%\begin{itemize}
%    \item AI-assisted e-scooter: 205 observations
%    \item Not sure: 104 observations
%    \item Regular e-scooter: 93 observations
%\end{itemize}

The class imbalance posed a potential bias for the decision tree model, as it could lead to a dominant class skewing the results. While methods such as undersampling and oversampling could address this issue, they come with their own drawbacks, such as the loss of valuable data or the creation of synthetic data. To mitigate this, the "Not sure" and "Regular e-scooter" categories were combined into a single class labeled "Not AI-assisted." This class represents participants who either chose the regular e-scooter or were uncertain about their decision, effectively consolidating non-AI-assisted choices into one category. 

The data was split into a training set (80\%, 321 data points) and a test set (20\%, 81 data points). To fine-tune the hyperparameters, a grid search cross-validation \cite{bramer2007avoiding} was performed, testing nine different combinations of parameters, including maximum depth, minimum samples split, and minimum samples leaf. Two criteria, Gini and entropy, were applied to evaluate the splits, with accuracy used as the scoring metric. The best-performing hyperparameters, yielding an accuracy of approximately 65.3\%, were selected for training the decision tree model. The result of decision tree analysis is provided in section \ref{sec: Result}.
 
\subsection{Structural Equation Modeling (SEM)}
SEM is a comprehensive statistical approach that combines factor analysis and multiple regression to analyze complex relationships among observed and latent variables \cite{ullman2012structural}.
The technique involves building a hypothesized model, estimating parameters, and evaluating model fit using indices such as the Root Mean Square of Approximation, Comparative Fit Index, and Tucker Lewis Index. A well-specified SEM provides valuable insights into the structural relationships and underlying mechanisms within data \cite{kline2023principles}. 
The SEM based approach used in the research aims to fit a theoretical model to observed data, making it particularly suitable for confirmatory research. SEM assumes that constructs are common factors and estimates the model accordingly, using a statistical model to analyze correlations between dependent and independent variables, as well as the hidden structures \cite{SmartPLS4}. The method employs sophisticated algorithms, such as maximum likelihood estimation, to simultaneously calculate model coefficients using all available information from the observed variables. This comprehensive approach allows researchers to test hypotheses and assess the consistency of their theoretical models with empirical data, providing valuable insights into causal relationships and latent constructs.
In this study, the full SEM was developed using IBM SPSS AMOS, it considered the simultaneous assessment of the impact of all the constructs, C1: 'Frequency of regular micromobility use', C2: 'Perceived safety of road users around e-scooters', C3: 'Perception of safety in AI-enabled technologies.', and C4: 'Perception of trust in AI enabled e-scooter', on C5: 'Willingness to Use AI enabled e-scooter.'

\subsubsection{Data Screening and Preparation}
%There may be Common Method Bias (CMB) in the data due to the occurrence of artificial correlations between variables and the erroneous reflection of these results in the parameter estimates. Controllable and uncontrolled circumstances may be the cause of this. CMB and inconsistency may also result from the respondents, the data collection tool's statement sequence and writing style, the setting in which the survey was administered, etc. In our survey, to avoid this issue, the statements were presented in a way that participants could easily understand the questions. Secondly, the questions in the data collecting instrument that belonged to the same factor were mixed together so that they were not consecutive in order to avoid false and strong correlations between variables. Moreover, in the preliminary data preparation phase, we also utilized the Variance Inflation Factor, Mahalaboris distance, and Cook's distance method to detect strong correlations, identify inconsistent responses, and eliminate outliers.
%December 19 Content (Amit) DM 3.0 
Common Method Bias (CMB) may arise due to artificial correlations between variables, potentially distorting parameter estimates. This bias may stem from controllable and uncontrollable factors, including respondent behavior, questionnaire design, and survey administration conditions. To mitigate CMB in this study, survey questions were structured for clarity and comprehension. Additionally, items measuring the same construct were mixed together so that they were not consecutive in order to avoid false and strong correlations between variables.

%Data screening in Structural Equation Modeling (SEM) refers to the process of assessing the quality and suitability of the data before conducting the SEM analysis. It involves various procedures aimed at identifying and addressing issues such as missing data, outliers, non-normality, and other data-related problems. The purpose of data screening is to ensure the validity and reliability of the data used in the SEM analysis. When we are accessing the quality of the data for indicators of the same construct, we check the reliability and validity of the construct. The measures of Reliability  and Convergent Validity are Cronbach Alpha (CA), Composite Reliability (CR), and Average Variance Extracted (AVE). Alpha values of 0.6-0.7 show an acceptable level of reliability \cite{eisinga2013reliability}, AVE values of 0.5 or more is desired while CR  values of 0.6 or more is acceptable \citep{fornell1981evaluating, hair2019use, diba2023autonomous}.

Furthermore, the data screening in Structural Equation Modeling (SEM) involved identifying and addressing missing data issues using mean data imputation, outlier detection, and non-normality to ensure the accuracy and reliability of results. Assessing data quality for indicators within the same construct requires evaluating both, reliability, and validity. Key measures of reliability and convergent validity include Cronbach’s Alpha (CA), Composite Reliability (CR), and Average Variance Extracted (AVE). A Cronbach’s Alpha between 0.6 and 0.7 is considered acceptable \cite{eisinga2013reliability}, while an AVE of 0.5 or higher and CR of at least 0.6 indicate satisfactory construct validity \citep{fornell1981evaluating, hair2019use, diba2023autonomous}.The CA and CR was found to be more than 0.6 for all the latent variables with the exception of C1 for which the values were 0.45 and 0.46 respectively.

%To assess the quality of the data for the indicators of different construct, we check the divergent validity of the construct. The measures of Divergent Validity are Heterotrait Monitrait Ratio, and Variance Inflation Factor (VIF). The quality of the data row-wise was assessed with the help of SPSS and Microsoft Excel. We calculated this with the help of standard deviation and average. If a respondant gave the same score (like 5) to all the answers, It's called the unengaged respondent. With respect to Likert scales, there are times, especially with lengthy self-reported questionnaires, when respondents provide "linear" answers, i.e., thy give, constantly, the same rating to all questions regardless if the questions are reversed or not. The respondents having a mean value less than 0.5 or more than 4.5 for the 5-point Likert Scale were deleted. Additionally, respondents having a standard deviation less than 0.25, were removed. 
%means there is no variation, respondent has ticked all 1's or 5's. So we needed to delete rows where SD was less than 0.25
To ensure the quality of data across different constructs, discriminant validity was assessed using the Heterotrait-Monotrait (HTMT) ratio and the Variance Inflation Factor (VIF). Additionally, row-wise data quality was examined using SPSS and Microsoft Excel, focusing on response patterns to identify unengaged respondents. Participants who provided identical ratings (e.g., consistently selecting "5" across all items) were flagged as unengaged. In self-reported Likert scale surveys, particularly lengthy ones, respondents may exhibit response bias by selecting the same rating for all questions, regardless of item phrasing. To mitigate this issue, responses with a mean value below 0.5 or above 4.5 on the 5-point Likert scale were excluded. Furthermore, cases where the standard deviation was less than 0.25—indicating a lack of variability in responses—were removed to enhance data reliability. The correlation matrix was used to assess multicollinearity and identify variables for exclusion or combination. A correlation coefficient above 0.7 typically indicates a high degree of multicollinearity, requiring further evaluation \cite{hair2013multivariate}.

Multivariate normality assesses the distribution of multiple variables at the same time. Multivariate outliers are data points that significantly deviate from the collective pattern, potentially distorting parameter estimates and introducing bias in predictive models. Unlike univariate outliers, which affect individual variables, multivariate outliers exhibit unusual combinations and relationships among multiple variables simultaneously. To detect such outliers, Mahalanobis distance and Cook’s distance were employed. The maximum Mahalanobis distance for the composite scores of C1, C2, C3, C4, and C5 was 20.557, exceeding the chi-square threshold of 9.488 at four degrees of freedom (95\% CI), indicating the presence of multivariate outliers, these datapoints were subsequently removed. Cook’s distance values were all below 1, so no observations were excluded based on this criterion. Additionally, Variance Inflation Factor (VIF) statistics for all constructs remained below 3, and collinearity tolerance values for all constructs were less than 1, confirming the absence of multicollinearity.

%Total Response: 472 Remaining based on Eligibility Criteria and Consent: 406
%Missing Data were tackled with the mean/median data imputation. It is a technique used in data analysis and machine learning to fill in missing values in a dataset. 
%Missing data can occur due to various reasons such as entry errors, equipment malfunction, or participant non-response in surveys.
%Data screening included examining the presence of multicollinearity, which refers to high correlations among predictor variables. Multicollinearity can affect the stability and reliability of parameter estimates in SEM. 
%Correlation matrices technique was used to assess the degree of multicollinearity and determine if any variables needed to be excluded or combined. The threshold value to avoid multicollinearity in correlation matrix can vary depending upon the specific context and analysis being performed. A commonly used guideline is to consider correlation coefficient greater or less than 0.7 indicates high degree of multicollenarity. \cite{hair2013multivariate}.
%Multivariate normality is the testing normality of more than one variable at the same time. Multivariate outliers are observations in a dataset that deviate significantly from the overall pattern or distribution of the variables considered collectively. Unlike uni-variate outliers, which involve extreme values in individual variables, multivariate outliers exhibit unusual combinations or relationships among multiple variables simultaneously. These outliers can distort parameter estimates, affect the accuracy of predictive models, and lead to bias conclusions or unreliable inferences. There are two distances to detect multivariate outliers, Mahalanboris distance, and Cook's distance. Maximum Mahalaboris distance for the composite score of C1, C2, C3, C4, and C5, came up to be 20.557 The value of chi-square at four degrees of freedom (number of predictor variables) came up to be 9.488 at 95\% CI. Since Mahalanobis maximum value is more than the Chi-square value at 5\% level of significance, multivariate normality does not exist due to the presence of outliers. The major outliers were selected and deleted. Cook's distance was also tested, no observation was removed on Cook's distance basis because all values were less than 1. Additionally, VIF statistics for all the constructs C1, C2, C3, C4, and C5 were less than 3, and Collinearity tolerance was less than 1 which means multicollinearity did not exist.
%Moreover, to assess the internal consistancy of the items, we tested Cronbach alpha for all the questions, which came up to be 0.823 which is considered good \cite{samadzad2023factors}


% Figure \ref{fig:DecisionTree} represents the final result of the classification tree. Each node in the decision tree contains several key elements: the "samples" value represents the number of data points reaching that node, the "value" shows the distribution of classes at the node (AI-assisted vs. Not AI-assisted), and the "class" indicates the predicted class for that node. 
% \begin{figure}[!ht]
%     \centering
%     %\rotatebox{90}
%     {\includegraphics[width=1\textwidth]
%     {FinalTree.png}}
%     %{\includegraphics[width=1\textwidth]
%     \caption{Classification tree for demographic analysis of AI-assisted vs regular e-scooter choice}
%     \label{fig:DecisionTree}
% \end{figure}

% The decision tree classification results reveal that ethnicity is the most influential feature in participants' preferences between AI-assisted and non-AI-assisted e-scooters. At the root node, participants of Middle Eastern ethnicity demonstrated a strong preference for AI-assisted e-scooters. However, within this group, individuals aged 31–35 and 18–21 showed a greater inclination toward non-AI-assisted options.

% Following Middle Eastern participants, a significant proportion of Asian participants also leaned toward AI-assisted e-scooters, especially those holding advanced degrees such as doctorate or professional healthcare qualifications. Interestingly, within the Asian group, participants with higher income ranges (\$150k–\$200k) or even mid-range incomes (\$50k–\$100k) exhibited a tendency to prefer Not-AI-assisted options.

% For the White/Caucasian group, no strong overall preference was observed. However, education level emerged as an important factor, with individuals holding advanced degrees predominantly selecting AI-assisted e-scooters. Additionally, among White/Caucasian participants, those aged 36–40 were more likely to choose AI-assisted options.

% Among other ethnic groups, including African American, Hispanic/Latino, and others/undisclosed, a larger proportion of participants generally preferred non-AI-assisted e-scooters. Within these groups, age emerged as a significant factor influencing their choices.
% Younger participants aged 18–21 with lower incomes (less than \$50k) tended to favor non-AI-assisted e-scooters. However, within the same age range, those with incomes between \$50k - \$100k were more inclined to select AI-assisted e-scooters. Additionally, participants aged 41–50 predominantly opted for AI-assisted e-scooters, whereas those in other age ranges primarily preferred non-AI-assisted options.



%\subsection{Confirmatory Factor Analysis}
%Confirmatory Factor Analysis (CFA) is a statistical technique used to test whether a set of observed variables aligns with a predefined set of latent constructs or factors based on theoretical expectations. It is often employed in social sciences, psychology, and education research to validate the structure of measurement models. Unlike exploratory factor analysis, which identifies potential underlying factor structures without prior assumptions, CFA requires the researcher to specify the expected relationships between observed variables and their respective latent factors beforehand. The model’s fit is then evaluated using indices such as the Chi-square test, Root Mean Square Error of Approximation (RMSEA), Comparative Fit Index (CFI), and others. A well-fitting CFA model provides evidence that the data supports the hypothesized factor structure, ensuring the validity and reliability of the constructs that are being measured.

%\subsection{Structural Equation Modeling}
 %The following figure \ref{SEM model developmentl} depicts the major steps involved in designing a SEM model adapted from Cheah et al., \cite{cheah2023multigroup}.

%\begin{figure} [!ht]
  %  \centering
  %  \includegraphics[width=0.8\linewidth]{SEM_Flowchart.png}
 %   \caption{The SEM model development flowchart}
 %   \label{SEM model developmentl}
%\end{figure}

%distrbution, etc
%step by step process and statisctal methods used from t-tes/ Chi-square to EFA and CFA to SEM

%\subsection{Hypothesis Development}
For the purpose of hypothesis testing, we developed eight hypotheses with five latent variables, consisting of eighteen observed variables on a five-point Likert scale. The model construct is depicted in Figure \ref{SEM modelZ}. One can wonder why the measuring model did not use every statement from the questionnaire included in the appendix. It is well known that in hypothesis testing, several statements in the questionnaire are not included in the analysis due to their failure to meet the required criteria (discriminant validity, CR and AVE conditions, factor loading of less than 0.60, or problematic internal linkages between items). This scenario may arise more frequently, particularly in online surveys. A small number of the study's questions were eliminated from analysis due to responses that did not fit the Likert scale. A few more also failed to meet the measurement model's validity and reliability requirements. 
%Researchers have proposed various models to evaluate the factors influencing the adoption of electric scooters. Sovacool et al.\cite{sovacool2019pleasure} examined the intentions to use electric vehicles in China, highlighting that people’s willingness to adopt these vehicles is influenced by their perceived usefulness (PU) and overall performance. Similarly, She et al.\cite{she2017barriers} identified barriers to the widespread adoption of electric scooters and found that younger individuals showed more positive attitudes toward using them compared to older generations.

%Eccarius and Lu (2020) \cite{eccarius2020adoption} studied students’ acceptance of electric scooters using the Theory of Planned Behavior (TPB), incorporating additional factors such as environmental values, perceived harmony, and awareness information. Their findings revealed that all of these variables significantly impacted acceptance. Rejali et al. \cite{rejali2021assessing} explored the factors that influence the use of electric scooters in Iran through an online survey of 1,078 participants, employing Structural Equation Modeling (SEM) to analyze the data. Finally, Oztas Karl et al. \cite{karli2022investigating} investigated the determinants of the intention to use electric scooters by developing a research model based on the Unified theory of Acceptance and Use of Technology. They extended the model by including variables such as price sensitivity and environmental awareness, which were found to influence user intentions. 

%The examination of previously published literature highlights how various factors such as perceived usefulness, perceived ease of use, environmental concerns, lack of appropriate infrastructure. performance, social impact, and safety problems for electric scooter user affect people's attitudes towards using electric scooters (Eccarius and Lu, 2020 \cite{eccarius2020adoption}; Sovacool et al 2019 \cite{sovacool2019pleasure}; Sun et al, \cite{Sunetal2011} Erkan and Ari \cite{ari2024investigating}). A review of the literature allows us to identify that AI-enabled driving assistance technology acceptance factors that have been left mostly unexplored.


\section{Data Analysis and Results} \label{sec: Result}
\subsection{Impact of Sociodemographic Features on Willingness to Use AI-enabled E-scooter}
%Decision Tree Analysis: AI-Assisted vs. Regular E-Scooter Chooser Profiles}
Figure \ref{fig:DecisionTree} represents the final result of the classification tree. Each node in the decision tree contains several key elements: the "samples" value represents the number of data points reaching that node, the "value" shows the distribution of classes at the node (AI-assisted vs. Not AI-assisted), and the "class" indicates the predicted class for that node. 

\begin{figure}[!ht]
    \centering
    %\rotatebox{90}
    {\includegraphics[width=1\textwidth]
    {FinalTree.png}}
    %{\includegraphics[width=1\textwidth]
    \caption{Classification tree for demographic analysis of AI-assisted vs regular e-scooter choice}
    \label{fig:DecisionTree}
\end{figure}

The decision tree classification results reveal that ethnicity is the most influential feature in participants' preferences between AI-assisted and regular e-scooters. At the root node, participants of Middle Eastern ethnicity demonstrated a strong preference for AI-assisted e-scooters. However, within this group, individuals aged 31–35 and 18–21 showed a greater inclination toward regular options.

Following Middle Eastern participants, a significant proportion of Asian participants also leaned toward AI-assisted e-scooters, especially those holding advanced degrees such as doctorates or professional healthcare qualifications. Interestingly, within the Asian group, participants with higher income ranges (\$150k–\$200k) or even mid-range incomes (\$50k–\$100k) exhibited a tendency to prefer regular options.

For the White/Caucasian group, no strong overall preference was observed. However, education level emerged as an important factor, with individuals holding advanced degrees predominantly selecting AI-assisted e-scooters. Additionally, among White/Caucasian participants, those aged 36–40 were more likely to choose AI-assisted options.

Among other ethnic groups, including African American, Hispanic/ Latino, and others/ undisclosed, a larger proportion of participants generally preferred regular e-scooters. Within these groups, age emerged as a significant factor influencing their choices.
Younger participants aged 18–21 with lower incomes (less than \$50k) tended to favor non-AI-assisted e-scooters. However, within the same age range, those with incomes between \$50k - \$100k were more inclined to select AI-assisted e-scooters. Additionally, participants aged 41–50 predominantly opted for AI-assisted e-scooters, whereas those in other age ranges primarily preferred non-AI-assisted options.
In the following, we outline the conducted SEM analysis to validate the process and ensure that the constructs and their indicators meet the necessary reliability and validity criteria before hypothesis testing.

\subsection{Validation of SEM Measurement Model}
%SEM is a widely used statistical method for estimating and testing complex relationships between variables in various scientific fields, including social sciences, behavioral sciences, and economics \cite{leguina2015primer}. 
The analysis in Table \ref{tab:reliability-validity} indicates that the construct reliability (CR) and Cronbach's alpha (CA) are above 0.50 for each factor with and exception of C1  with CA=0.45 and CR=0.46 which is acceptable in this case, given the context. In their paper, An applied Orientation, Malhotra et al. \cite{malhotra2010applied} stated that average variance extracted (AVE) is often too strict, and reliability can be established through CR alone. The factor loading for each unobserved variable is 0.3 or above which is within the acceptable limit \citep{hair2019use, field2024discovering}.  Also, the AVE for C1 and C2 are 0.23 and 0.33 while for C3, C4 and C5 are 0.32, 0.35, and 0.60 respectively. The CR was found to be more than 0.6 for all the latent variables with the exception of C1 for which CR was calculated as 0.46.

\begin{table}[!ht]
\centering
\caption{Evaluation of Reliability and Validity of Latent Constructs}
\label{tab:reliability-validity}
\renewcommand{\arraystretch}{1.3} % Improves row spacing for readability
\setlength{\tabcolsep}{4pt} % Reduces column spacing to fit within margins
\resizebox{0.98\textwidth}{!}{ 
\begin{tabular}{p{4.5cm} c p{7.5cm} c c c c c}
\toprule
\textbf{Constructs} & \textbf{Code} & \textbf{Survey Questions} & \textbf{Mean} & \textbf{Factor Loading} & \textbf{CA} & \textbf{CR} & \textbf{AVE} \\ 
\midrule
\multirow{3}{=}{\raggedright Frequency of Regular Micromobility Use} 
& A & \raggedright Q1: How frequently do you use any of the following vehicles? - E-scooter & 1.64 & 0.47 & 0.45 & 0.46 & 0.23 \\  
& B & \raggedright Q2: How frequently do you use any of the following vehicles? - Bicycle & 1.92 & 0.35 &  &  &  \\  
& C & \raggedright Q3: How frequently do you use any of the following vehicles? - E-bike & 1.41 & 0.58 &  &  &  \\  
\midrule
\multirow{5}{=}{\raggedright Perceived Safety of Road Users Around E-Scooters} 
& F & \raggedright Q4: As a pedestrian, how safe do you feel when e-scooters are around you? & 3.38 & 0.53 & 0.72 & 0.71 & 0.33 \\  
& G & \raggedright Q5: When riding a bicycle, how safe do you feel when e-scooters are around you? & 3.66 & 0.65 &  &  &  \\  
& H & \raggedright Q6: When riding an e-bike, how safe do you feel when e-scooters are around you? & 3.91 & 0.61 &  &  &  \\  
& I & \raggedright Q7: When driving a car, how safe do you feel when e-scooters are around you? & 3.12 & 0.45 &  &  &  \\  
& J & \raggedright Q8: When riding an e-scooter, how safe do you feel when e-scooters are around you? & 3.79 & 0.61 &  &  &  \\  
\midrule
\multirow{3}{=}{\raggedright Perception of Safety in AI-Enabled Technology} 
& S & \raggedright Q9: How do you feel about the safety of these driving assistance technologies? & 3.47 & 0.45 & 0.6 & 0.61 & 0.32 \\  
& T & \raggedright Q10: Do you think autonomous vehicles (self-driving cars) offer a higher or safer level of safety compared to vehicles driven manually by humans? & 3.02 & 0.42 &  &  &  \\  
& W & \raggedright Q11: Compared to regular e-scooters, the AI-assisted features will reduce the likelihood of accidents. & 3.48 & 0.77 &  &  &  \\  
\midrule
\multirow{3}{=}{\raggedright Perception of Trust in AI-Enabled E-Scooters} 
& U & \raggedright Q12: I trust this system for a safer ride. & 2.97 & 0.78 & 0.78 & 0.78 & 0.55 \\  
& X & \raggedright Q13: I trust the AI's ability to handle unexpected situations while I'm on the e-scooter. & 3.11 & 0.78 &  &  &  \\  
& Y & \raggedright Q14: I trust that the AI won't malfunction and compromise my safety while I'm using the e-scooter. & 3.00 & 0.64 &  &  &  \\  
\midrule
\multirow{2}{=}{\raggedright Willingness to Use AI-Enabled E-Scooters} 
& V & \raggedright Q15: Compared to regular e-scooters, I feel more confident using the AI-assisted e-scooter in various traffic conditions because of its AI capabilities. & 3.41 & 0.84 & 0.7 & 0.74 & 0.60 \\  
& Z & \raggedright Q16: Given a choice between two e-scooters priced the same, which one would you buy/select from the deck? & 3.6 & 0.69 &  &  &  \\  
\bottomrule
\end{tabular}}
\end{table}


Table \ref{tab:constructs} represents the correlation matrix and discriminant validity for five constructs labeled as C1, C2, C3, C4, and C5. Each diagonal element (in bold) represents the square root of the average variance extracted (AVE) for the respective construct. The off-diagonal values represent the inter-construct correlations \cite{henseler2015new}. C1 and C2 demonstrate strong discriminant validity and appear to be well differentiated, while C3, C4, and C5 exhibit inter-construct correlations due to their conceptual overlap, as they represent closely related dimensions but on investigating the underlying conceptual definitions of these constructs, we can see that they are genuinely distinct categories and are conceptually valid. 

The correlation analysis reveals a strong positive relationship between trust and safety (0.738) and an even stronger correlation between trust and use (0.952), indicating that higher trust levels are closely linked to both perceived safety and willingness to use the system. Similarly, safety and use (1.037) show an unexpectedly high correlation, suggesting potential multicollinearity or measurement issues. Perspective and safety (0.143) exhibit a weak positive correlation, implying a minimal connection between user perspective and safety perception. Conversely, frequency of use and perceived safety of road users (-0.134) show a slight negative correlation, indicating that increased frequency of micromobility use may not necessarily enhance user perspectives on safety around e-scooters.
%To check for multicollinearity, Variance Inflation Factor (VIF) was examined and found to be less than 3, indicating no significant multicollinearity concerns, and 
HTMT ratio analysis was also conducted to confirm discriminant validity using the HTMT $<$ 0.85 rule, as depicted in following section. 


\begin{table}[!ht]
    \centering
    \footnotesize % Further reduces the font size
    \setlength{\tabcolsep}{3pt} % Reduces column spacing
    \renewcommand{\arraystretch}{1.2} % Keeps row spacing tight
    \caption{Correlation matrix and discriminant validity}
    \label{tab:constructs}
    \begin{tabular}{clccccc}
        \toprule
        \textbf{} & \textbf{Construct} & \textbf{C1} & \textbf{C2} & \textbf{C3} & \textbf{C4} & \textbf{C5} \\
        \midrule
        \textbf{1} & Frequency of regular micromobility use & \textbf{0.475} \\
        \textbf{2} & Perceived Safety of Road Users around E-Scooters & -0.134 & \textbf{0.571} \\
        \textbf{3} & Perception of Safety in AI-enabled technology & -0.024 & 0.143 & \textbf{0.566} \\
        \textbf{4} & Perception of Trust in AI-enabled E-Scooter & -0.031 & 0.176 & 0.982 & \textbf{0.738} \\
        \textbf{5} & Willingness to Use AI-enabled e-scooter & -0.089 & 0.063 & 0.967 & 0.952 & \textbf{0.768} \\
        \bottomrule
    \end{tabular}
    \captionsetup{justification=centering, font=footnotesize} 
    \caption*{\textit{Note: Bold numbers on the diagonal are square roots of AVE's.}}
\end{table}


The Hetrotrait-Monotrait Ratio (HTMT) matrix has been shown in Table \ref{tab:HMT}. It demonstrates the discriminant validity of the constructs (C1, C2, C3, C4, and C5). Based on the thresholds of 0.850 (strict) and 0.900 (liberal) \cite{fornell1981evaluating}, the majority of the construct pairs exhibit satisfactory discriminant validity. Specifically, the HTMT values between C1, C2, and other constructs are well below both thresholds, indicating strong evidence of discriminant validity. However, C3 and C4 (HTMT = 0.71) and C4 and C5 (HTMT = 0.707) are close to the liberal threshold but remain within acceptable limits. This suggests moderate overlap but still supports discriminant validity under both strict and liberal criteria. Overall, the matrix highlights that the constructs are distinct, with no violations of the thresholds, affirming the robustness of the measurement model.


\begin{table}[!ht]
    \centering
    \footnotesize
    \setlength{\tabcolsep}{3pt} % Reduces column spacing
    \renewcommand{\arraystretch}{1.15} % Keeps row spacing tight
    \caption{Heterotrait-Monotrait Ratio (HTMT)}
    \label{tab:HMT}
    \begin{tabular}{clccccc}
        \toprule
        \textbf{} & \textbf{Construct} & \textbf{C1} & \textbf{C2} & \textbf{C3} & \textbf{C4} & \textbf{C5} \\
        \midrule
        \textbf{1} & C1 & \textbf{-} \\
        \textbf{2} & C2 & 0.058 & \textbf{-} \\
        \textbf{3} & C3 & 0.001 & 0.120 & \textbf{-} \\
        \textbf{4} & C4 & 0.039 & 0.130 & 0.711 & \textbf{-} \\
        \textbf{5} & C5 & 0.081 & 0.003 & 0.655 & 0.707 & \textbf{-} \\
        \bottomrule
    \end{tabular}
    \captionsetup{justification=centering, font=footnotesize} 
    \caption*{\textit{Note:} Thresholds are 0.850 for strict and 0.900 for liberal discriminant validity.}
\end{table}

\begin{figure}[!ht]
    \centering
    \includegraphics[width=0.77\linewidth]{The_Proposed_Model2.png}
    \caption{The proposed model to assess willingness to use AI enabled e-scooter}
    \label{SEM model2}
\end{figure}

\begin{table}[!ht]
\centering
\caption{Hypothesis Testing Results}
\label{tab:hypothesis-results}
\renewcommand{\arraystretch}{1.5} % Improves row spacing for readability
\setlength{\tabcolsep}{5pt} % Adjusts column spacing to prevent overflow
\footnotesize 
\resizebox{0.99\textwidth}{!}{ 
\begin{tabular}{p{7.5cm} c c c c}
\toprule
\textbf{Hypothesis} & \textbf{Label} & \textbf{Path Coefficient} ($\boldsymbol{\beta}$) & \textbf{p-value} & \textbf{Conclusion} \\ 
\midrule
\raggedright H1: Frequency of regular micromobility use influences road users' perception of safety in AI-enabled technology. 
& H1 & 0.024 & 0.798 & Not Supported \\ 

\raggedright H2: Frequency of regular micromobility use influences perceived safety of road users around e-scooters. 
& H2 & -0.155 & 0.096 & Not Supported \\ 

\raggedright H3: Frequency of regular micromobility use influences perception of trust in AI-enabled e-scooters. 
& H3 & -0.051 & 0.519 & Not Supported \\ 

\raggedright H4: Perceived safety of road users around e-scooters influences perception of safety in AI-enabled technology. 
& H4 & 0.218 & 0.009 & \textbf{Supported} \\ 

\raggedright H5: Perceived safety of road users around e-scooters influences perception of trust in AI-enabled e-scooters. 
& H5 & 0.145 & 0.033 & \textbf{Supported} \\ 

\raggedright H6: Perception of safety in AI-enabled technology influences perception of trust in AI-enabled e-scooters. 
& H6 & 1.075 & 0.001 & \textbf{Supported} \\ 

\raggedright H7: Perception of safety in AI-enabled technology influences willingness to use AI-enabled e-scooters. 
& H7 & 1.040 & 0.001 & \textbf{Supported} \\ 

\raggedright H8: Perception of trust in AI-enabled e-scooters influences willingness to use AI-enabled e-scooters. 
& H8 & 0.935 & 0.001 & \textbf{Supported} \\ 
\bottomrule
\end{tabular}}
\end{table}



\subsection{Hypothesis Testing Results from Structural Equation Modeling}
This section presents the results of the Structural Equation Modeling (SEM) analysis, focusing on the direct effects of the hypothesized relationships. The standardized regression weights (SRW) and p-values indicate the strength and statistical significance of each relationship. The results are shown in Table \ref{tab:hypothesis-results} and interpreted as follows:
%The result from Structural Equation Modeling (SEM) using the direct effects method can be interpreted as follows:

The first hypothesis (H1) examined the relationship between frequency of regular micromobility use (C1) and perception of safety in AI-enabled technology (C3). The standardized estimate of 0.024 suggests a weak positive effect; however, the high p-value of 0.798 indicates that this relationship is not statistically significant. Consequently, there is no strong evidence to suggest that frequency of micromobility use significantly influences perceptions of AI safety. Similarly, H2 tested the impact of C1 on perceived safety of road users around e-scooters (C2). The results indicate a weak negative relationship (SRW = -0.134, p = 0.096), which does not reach statistical significance at the 95\% confidence level. Although this near-threshold p-value suggests a potential trend, the effect remains small and may require a larger sample size for conclusive results.

For H3, the relationship between C1 and perception of trust in AI-enabled e-scooters (C4) was analyzed. The unstandardized coefficient of -0.063 and a standardized regression weight of -0.051 indicate an extremely weak negative relationship, further supported by a non-significant p-value of 0.519. This suggests that frequency of micromobility use does not have a meaningful impact on trust in AI-enabled e-scooters. In contrast, H4 revealed a moderate positive relationship between perceived safety of road users (C2) and perception of safety in AI-enabled technology (C3). The standardized regression weight of 0.218 with a p-value of 0.009 confirms a statistically significant effect, suggesting that individuals who perceive greater safety around e-scooters are also more likely to perceive AI-assisted technologies as safe.

H5 examined the influence of C2 on C4, indicating a weak but statistically significant relationship (SRW = 0.145, p = 0.033). This result suggests that perceived safety of road users has a small but noticeable positive influence on trust in AI-enabled e-scooters. Although the effect size is not substantial, the statistical significance confirms that perceptions of general road safety contribute to AI trust formation.

A particularly strong relationship was observed for H6, which tested the effect of perception of safety in AI-enabled technology (C3) on perception of trust in AI-enabled e-scooters (C4). The standardized regression weight of 1.07 suggests a substantial positive effect, and the relationship is highly statistically significant \(p < 0.001\). These findings indicate that perceptions of AI safety strongly enhance user trust in AI-enabled e-scooters, making C3 a critical predictor of C4.

The results for H7 further reinforce the importance of AI safety perceptions in user adoption. A standardized regression weight of 1.04 \(p < 0.001\) indicates a strong and significant positive relationship between C3 and willingness to use AI-enabled e-scooters (C5). This suggests that as confidence in AI-assisted micromobility safety increases, users are substantially more willing to adopt such technology. Similarly, H8 confirmed a strong positive effect between C4 and C5, with a standardized regression weight of 0.935 \(p < 0.001\). This result suggests that higher trust in AI-enabled e-scooters significantly increases willingness to use them.

Overall, the hypothesis testing results reveal that frequency of micromobility use (C1) does not significantly impact safety perceptions or trust in AI-assisted e-scooters, as evidenced by the non-significant results in H1–H3. However, perceived safety of road users (C2) positively influences both AI safety perceptions (C3) and trust in AI-enabled e-scooters (C4), as seen in H4 and H5. The most influential factors driving willingness to use AI-enabled e-scooters (C5) are perceptions of AI safety (C3) and trust in AI-enabled e-scooters (C4), which demonstrated strong and highly significant relationships in H6–H8. These findings suggest that enhancing user trust and AI safety perceptions may be more effective in increasing AI micromobility adoption than simply relying on prior micromobility usage experience. Future research should explore potential moderating factors that may refine these relationships further.

%For H1: The standardized estimate of 0.024 represents the moderate strength of the relationship between C1 (independent variable) and C3 (dependent variable). A positive value suggests that as C1 increases, C3 is expected to increase, though the magnitude of the effect is small. The p-value of 0.798 is much greater than 0.05, indicating that the relationship between C1 and C3 is not statistically significant. This means that we fail to reject the null hypothesis, suggesting there is no strong evidence of a meaningful direct effect of C1 on C3.

%For H2: The standardized Regression Weight (SRW) indicates a weak negative relationship between C1 (independent variable) and C2 (dependent variable). A negative value suggests that as C1 increases, C2 tends to decrease, but the effect size is small. The p-value at 0.096 is slightly above 0.05, indicating that the relationship does not reach statistical significance at the 95 percent confidence level.The relationship between C1 and C2 is weak and not statistically significant suggesting that C1 has a small negative influence on C2, but the effect is not strong enough to be considered statistically reliable. However, the near-threshold p-value indicates a potential trend that may become significant with a larger sample size.

%For H3: The unstandardized coefficient of -0.063 indicates the raw effect of C1 on C4. A one-unit increase in C1 leads to a 0.063-unit decrease in C4, which is a very small effect. The standardized regression weight of -0.051 indicates an extremely weak negative relationship between C1 (independent variable) and C4 (dependent variable). A negative value suggests that as C1 increases, C4 slightly decreases, but the effect is negligible. A p-value greater than 0.05 indicates that the relationship is not statistically significant. There is no strong evidence to suggest that C1 significantly influences C4. Frequency of regular micromobility use does not significantly influence perception of trust in AI-enabled E-Scooter (SRW = -0.051, p = 0.519).

%For H4: The standardized regression weigh of 0.218 indicates a moderate positive relationship between C2 (independent variable) and C3 (dependent variable).A p-value less than 0.05 confirms that the relationship between C2 and C3 is statistically significant. Perceived safety of road users around e-scooters positively influences perception of safety in AI enabled technology (SRW = 0.218, p = 0.009).As C2 increases, C3 also increases, with a moderate effect size.

%For H5:The standardized regression weight of 0.145 means that for every 1 standard deviation increase in C2, C4 is expected to increase by 0.145 standard deviations.This suggests that C2 has a weak but still noticeable influence on C4.The p-value of 0.033 indicates that the relationship between C2 and C4 is statistically significant at the 0.05 level. This means that there is enough evidence to reject the null hypothesis (which suggests no relationship between C2 and C4) and confirm that the relationship is real and not due to chance, though it's less robust than those with lower p-values. Perceived safety of road users around e-scooters positively influences perception of trust in AI-enabled E-Scooter (SRW = 0.145, p = 0.033).


%For H6: The standardized regression weight value of 1.07 indicates a strong positive relationship between C3 and C4. This means that for every 1 standard deviation increase in C3, C4 is expected to increase by 1.07 standard deviations. This suggests a substantial effect of C3 on C4. The relationship between C3 and C4 is highly statistically significant, with a p-value less than 0.001. This means that there is very strong evidence to reject the null hypothesis (which would suggest no relationship between C3 and C4) and confirm that the relationship observed in our model is real and not due to random chance.

%For H7: The standardized regression weight of 1.04 means that for every 1 standard deviation increase in C3, C5 is expected to increase by 1.04 standard deviations.This indicates a strong positive relationship between C3 and C5, suggesting that C3 has a notable influence on C5.Given that the weight is above 1, it suggests that C3 has a substantial impact on C5 relative to other predictors in our model. The p-value \(p < 0.001\) indicates that the relationship between C3 and C5 is highly statistically significant.C3 to C5 is highly statistically significant, with a strong positive effect as indicated by the standardized regression weight of 1.04. This suggests that changes in C3 have a significant and substantial impact on C5, and the result is highly reliable.

%For H8: The standardized regression weight of 0.935 suggests a strong positive relationship between C3 and C5. The relationship remains substantial, with C3 having a significant impact on C5.A value just under 1 indicates a robust effect, though slightly smaller compared to a path with a value greater than 1. The p-value of \(p < 0.001\) shows that the relationship is highly statistically significant. The path from C3 to C5 is statistically significant, with a strong positive effect indicated by the standardized regression weight of 0.935. This shows that C3 has a substantial influence on C5.

\subsection{Full Scale Structure Equation Modeling}
The five constructs and their item descriptions are illustrated in Table \ref{tab:regression_results}. We used Cronbach's Alpha (CA) scale to check the reliability of the full scale model, the summary statistics are - CA =0.759, mean = 3.088, with a minimum value of 1.418 and maximum value of 3.919, range = 2.501 with a variance of 0.573. The CR value for each construct C1, C2, C3, C4, and C5 in the model were calculated to be 0.48, 0.72, 0.73, 0.81, and 0.53 respectively. We also included an independent construct in the full model, C6: "Crash Experience" which consisted of two items -
\begin{itemize}
    \item Have you ever been involved in an accident while riding an e-scooter?
    \item How would you categorize the severity of your injury from the e-scooter incident?
\end{itemize}
%Please see that the figure can be changed to display the vealues more prominantly if needed. 
The summary statistics for this construct consisted of Cronbach's alpha = 0.676, cumulative mean of 4.81, SD = 0.65, and variance =0.13. 

The full-scale model consists of five exogenous variables C1, C2, C3, C4, and C6 which have an effect on one endogenous variable i.e. variable C5. Since, Willingness to use AI enabled e-scooter. (C5) is a dependent variable, it needs to have an error term called disturbance added to it, i.e. e21 depicted in the full-scale figure. 

\begin{table}[h!]
\centering
\caption{Regression Results}
\resizebox{0.9\textwidth}{!}{
\begin{tabular}{lccccc}
\toprule
\textbf{Predictor} & \textbf{Standardized Estimate} & \textbf{S.E.} & \textbf{C.R.} & \textbf{p-value} & \textbf{Interpretation} \\ \midrule
C5 $\leftarrow$ C2 & -0.091 & 0.07 & -1.674 & 0.094 & Negative, marginally significant \\
C5 $\leftarrow$ C4 & 0.597 & 0.059 & 9.46 & 0.001 & Strong positive, highly significant \\
C5 $\leftarrow$ C3 & 0.775 & 0.147 & 6.176 & 0.001 & Strong positive, highly significant \\
C5 $\leftarrow$ C1 & -0.089 & 0.093 & -1.296 & 0.195 & Negative, non-significant \\
C5 $\leftarrow$ C6 & -0.621 & 12.039 & -0.336 & 0.737 & Negative, non-significant \\ \bottomrule
\end{tabular}
}

\label{tab:regression_results}
\end{table}

The summary of the key findings from the SEM results regarding the direct effects of the latent variables (C1, C2, C3, C4, and C6) on the dependent variable C5 are depicted below in figure \ref{SEM Direct Effects Full ModelZ}.

\begin{figure}[H]
    \centering
    \includegraphics[width=0.65\textwidth]{SEM_Full_Sclae_Direct_Effects.png} % Adjust width as needed
    \caption{Structure Equation Modeling Full Scale Model}
    \label{SEM Direct Effects Full ModelZ}
\end{figure}




Effect of C1 (frequency of regular micromobility use) on C5 (Willingness to use AI-enabled e-scooter): The standardized regression weight (–0.089) indicates a weak negative relationship between frequency of regular micromobility use and willingness to use AI-enabled e-scooter. The critical ratio (C.R. = -1.296) and p-value (0.195) indicate that this effect is not statistically significant. C1 does not have a meaningful impact on C5 in this model, and the relationship may not be reliable. It could be because people who have been using other modes of micromobility like e-scooters, bicycles, and e-bikes, may not be willing to use AI-enabled technology because they are already comfortable with their current mode of locomotion. 
Effect of C2 (perceived safety of road users around e-scooters) and C5 (willingness to use AI-enabled e-scooter): The standardized regression weight (–0.091) indicates a slight negative relationship between perceived safety of road users around e-scooters and willingness to use AI-enabled e-scooter. The critical ratio (C.R. = -1.674) and p-value (0.094) indicate that this effect is not statistically significant. C2 does not appear to have a meaningful impact on C5 in this model.
Effect of C3 (perception of safety in AI-enabled technology) on C5 (willingness to use AI-enabled e-scooter): The standardized regression weight (0.775) indicates a strong positive relationship between perception of safety in AI-enabled technology and willingness to use AI-enabled e-scooter. For every 1-unit increase in C3, C5 is expected to increase by 0.775 standard deviations in the standardized scale. The effect is highly significant (C.R. = 6.176, $p < 0.001$). This demonstrates that C3 is a major determinant of C5 with a substantial impact. 
Effect of C4 (perception of trust in AI-enabled E-Scooter) on C5 (willingness to use AI-enabled e-scooter): The standardized regression weight (0.597) indicates a positive relationship between perception of trust in AI-enabled E-Scooter and willingness to use AI-enabled e-scooter. For every 1-unit increase in C4, C5 is expected to increase by 0.597 standard deviations in the standardized scale. The effect is highly significant (C.R. = 9.46, $p < 0.001$). This suggests that C4 is an important and statistically significant predictor of C5.
Effect of C6 (Crash Experience) on C5 (willingness to use AI-enabled e-scooter): The standardized regression weight (–0.621) indicates a slight negative relationship between previous Crash experience and Willingness to Use AI-enabled e-scooter. This seems logical because people who have been previously involved in an accident using an E-scooter, may not be interested in using it again even with the AI-enabled features. The critical ratio (C.R. = -0.336) and p-value (0.621) suggest that this effect is not statistically significant. Crash experience does not significantly influence Willingness to use AI enabled e-scooter.

By observing the findings from the Full SEM model, we can see that perception of safety in AI-enabled technologies has the strongest positive effect on Willingness to use AI-enabled e-scooter, followed by perception of trust in AI-enabled E-Scooters, both of which are statistically significant. Both perception of safety and trust are significant predictors while frequency of regular micromobility use, perceived safety of road users around e-scooters, and Crash experience do not significantly affect willingness to use AI-enabled e-scooters. Their p-values are greater than 0.05, and their critical ratios are not substantial enough to indicate statistical significance. For the purpose of practical implications, efforts to influence willingness to use AI-enabled e-scooters should focus on factors that enhance perception of safety and perception of trust, as these are the most impactful predictors. Variables like frequeny of micromobility use, perceived safety of road users around e-scooters, and previous crash experience are not the driving factors in the adoption and willingness to use AI-enabled e-scooters. 


\section{Discussion} \label{sec:discussion}
\subsection{Impact of Sociodemographic Factors on Adoption of AI-Assisted E-Scooters}
Sociodemographic analysis revealed that gender was not an influential factor in participants' inclination towards AI-assisted e-scooters. This finding contrasts with previous studies indicating that males are more likely to use e-scooters, a trend linked to the gender mobility gap and males' greater risk-taking behavior \cite{laa2020survey,nikiforiadis2021analysis}. However, AI-assisted e-scooters could potentially mitigate risks, serving as a tool to narrow the gender disparity in e-scooter adoption.
%Based on prior studies, males are more likely than females to use e-scooters \cite{laa2020survey,nikiforiadis2021analysis}, which has been linked to the gender mobility gap and males’ greater risk-taking behavior. Given that AI-assisted e-scooters can potentially mitigate risks, they could serve as a bridge to narrow this gender disparity in e-scooter adoption.
Younger adults, especially those aged 18-21, showed a stronger preference for regular e-scooters. 
This aligns with research suggesting that older adults perceive more benefits from AI-assisted vehicles than younger individuals \cite{hilgarter2020public}.
%This aligns with findings from Hilgartner et al \cite{hilgarter2020public}, which suggested that older adults perceive more benefits from AI-assisted vehicles than younger individuals. 
The tendency of younger riders to avoid AI-assisted e-scooters could be attributed to their greater propensity for risky behaviors while riding, as highlighted by \cite{gioldasis2021risk}. Additionally, e-scooter trauma patients tend to be younger \cite{burt2023scooter}, reinforcing the notion that younger riders may exhibit more risk-taking behavior, which AI-assisted technology could help mitigate.

Education also played a significant role in individuals' preferences, particularly among those with advanced degrees, including doctorate degrees (e.g., PhD, EdD) or professional healthcare degrees (e.g., MD, DDS, DVM). This preference could be attributed to their greater exposure to AI knowledge and its applications. Individuals with such educational backgrounds are often more engaged in research and are regularly exposed to the benefits and advancements of AI across various scientific fields, making them more likely to appreciate the potential advantages of AI-assisted e-scooters. 

\subsection{Influence of Micromobility Use and Crash Experience on AI-Assisted E-Scooter Preferences}
Findings from the hypothesis testing revealed that H1 which proposed that frequent use of micormobility vehicles positively influences perceptions of safety in AI-enabled technologies, was not supported.
A Previous study by Pourfalatoun et al. \cite{pourfalatoun2023shared} has shown that e-scooter users were more likely to be early adopters of new technology. While the analysis yielded a non-significant relationship ($\beta$ = 0.024, p = 0.798), the positive coefficient suggests a potential association, the lack of statistical significance implies that familiarity with micromobility might not inherently translate to a more favorable attitude towards AI safety technology. This could be due to variations in user experiences with micromobility and AI technologies or a general lack of public awareness connecting the two domains. Similarly, H2 examined whether frequent micromobility use affects perceived safety of road users around e-scooters. The negative coefficient ($\beta$ = -0.155, p = 0.096) aligns with the previous study \cite{nikiforiadis2021analysis} and our hypothesis that it does have an effect which is negative in nature, but the lack of statistical significance again points to a non-conclusive relationship. This weak relationship could stem from diverse factors such as variations in e-scooter regulations, cultural norms, or differences in micromobility vehicle design that dilute a clear pattern of perception. This implies that frequent users' experiences with micromobiliy do not strongly shape their broader opinion about E-scooter use. 

In the full model, it seems logical that crash experience had a negative effect on willingness to use AI-enabled e-scooter. This finding is in line with previous studies \cite{he2017impact} which found prior accidents had a significant impact on the preference of travel mode where the riders were reported to have greater fears for personal safety, worries about driving, and negative physical and physiological symptoms \cite{marasini2022psychological}. Individuals who have been involved in accidents while riding conventional e-scooters may develop a reluctance to continue using such vehicles, even if AI-assisted features are introduced. This finding underscores the potential psychological barriers to AI adoption among riders with prior negative experiences. 

Similarly, perceived safety of road users around e-scooters negatively influenced willingness to use AI-enabled e-scooters. This suggests that general attitudes toward regular e-scooters, whether positive or negative, do not necessarily extend to AI-assisted versions. Users may perceive AI-enabled e-scooters as fundamentally different from traditional ones, making their pre-existing perceptions of conventional e-scooters less relevant to their willingness to adopt AI-based alternatives.

Finally, the frequency of other types of micromobility use (e-scooters, bicycles, and e-bikes) was negatively associated with willingness to adopt AI-assisted e-scooters. This is in line with similar findings by Sharpe \cite{sharpe2013aesthetics} who showed that people may continue using familiar transport modes due to habitual reinforcement. This indicates that users who are already accustomed to their current micromobility vehicles may be less inclined to transition to AI-assisted models. Resistance to change \cite{lattarulo2019resistance}, comfort with familiar transportation modes, and skepticism about AI enhancements could contribute to this reluctance, even when AI features offer potential safety improvements.

\subsection{Implications for Urban Mobility and AI Adoption}

The findings of this study highlight several important considerations for urban mobility stakeholders, policymakers, and AI developers. Firstly, AI-assisted e-scooters may serve as a tool to reduce existing demographic disparities in micromobility adoption, particularly regarding gender and safety concerns. By enhancing safety features, AI-assisted e-scooters may appeal more to females and other underrepresented groups, contributing to a more equitable urban mobility landscape.
Secondly, educational initiatives aimed at improving AI literacy could enhance public trust and acceptance of AI-driven transportation solutions. Increasing awareness about the benefits and capabilities of AI technology can help mitigate skepticism and foster a more receptive environment for AI-assisted e-scooters.
Thirdly, strategies to overcome psychological barriers among users with prior crash experiences should be explored, as safety-related trauma may hinder the adoption of AI-assisted e-scooters, underscoring the need for targeted interventions to address these concerns. By addressing these challenges, stakeholders can develop more effective strategies to promote the adoption of AI-assisted e-scooters and enhance urban mobility.

%as people who have been involved in an accident, riding regular e-scooters might not be willing to use them again even if the E-scooter comes loaded with the AI-assisted features. It would be counter intuitive if crash had positive effect on willing to use AI enabled driving assisted technology in this scenario.
%Similarly, perceived safety of road users around e-scooters was found to have a negative effect on the willingness to use AI enabled e-scooters. The reason could be what road users perceive of E-scooters might not affect their willingness to use AI-enabled e-scooters. Finally, Frequency of use of regular micro-mobility vehicles like normal e-scooters, bicycles, and e-bikes were found to have a negative effect on the users' willingness to use AI-enabled e-scooters. This can be due to the fact that the users who are already acclimated to using their current mode of micromobility might not be excited to use a different vehicle even if comes with added AI-assisted features.

\section{Conclusion} \label{sec:conclusion}
This study examined the factors influencing the adoption of AI-assisted e-scooters, focusing on sociodemographics, safety perceptions, trust, and prior micromobility experiences. The result of decision tree analysis on demographics indicates that ethnicity, income, and age, significantly shape adoption preferences, with Middle Eastern and Asian participants showing greater willingness to use AI-enabled e-scooters, while younger, lower-income individuals from other ethnic groups tended to prefer regular e-scooters.
Results from the hypothesis testing revealed that road users’ perceptions of e-scooters demonstrated meaningful associations with safety and trust. Perceived safety of road users around e-scooters was linked to higher perceived safety in AI-enabled technologies and greater trust in AI-enabled e-scooters. These findings highlight the role of positive road users' interaction with regular e-scooters in shaping AI adoption. Conversely, the frequency of regular micromobility use did not significantly impact perceptions of safety in AI-enabled technologies, e-scooters, or trust in AI-enabled e-scooters.
%A positive attitude toward e-scooters was linked to higher perceived safety in AI-enabled technologies ($\beta$ = 0.218, p = 0.009) and greater trust in AI-enabled e-scooters ($\beta$ = 0.145, p = 0.033). These findings highlight the role of positive user experiences with conventional e-scooters in shaping AI adoption.

It should be noted that perception of safety in AI-enabled technologies strongly influences both trust in AI-enabled e-scooters and willingness to use them. Moreover, trust in AI-enabled e-scooters significantly affects willingness to use them. These results highlight the foundational role of safety and trust in AI-assisted e-scooter adoption.

%Particularly noteworthy are the strong relationships observed between safety perceptions, trust, and willingness to adopt AI technology. The perception of safety in AI-enabled technologies strongly influences both trust in AI-enabled e-scooters ($\beta$ = 1.075, p = 0.001) and willingness to use them ($\beta$ = 1.04, p = 0.001). Similarly, trust in AI-enabled e-scooters significantly affects willingness to use them ($\beta$ = 0.935, p = 0.001). These results highlight the foundational role of safety and trust in driving AI-assisted mobility adoption.
%These findings have important implications for practitioners and policymakers in the urban mobility sector. First, they suggest that building trust in AI-enabled mobility solutions may be more effectively achieved through positive experiences with both traditional and advanced mobility options rather than frequency of use alone. Second, the strong influence of safety perceptions on both trust and willingness to adopt AI technology emphasizes the need for stakeholders to prioritize and communicate safety features effectively. Finally, the results indicate that trust-building initiatives should focus on demonstrating the reliability and safety of AI enabled technology, as these factors appear to be crucial determinants of user acceptance.

Perceived safety in AI-enabled technology emerged as the strongest predictor of willingness to use AI-assisted e-scooters closely followed by perception of trust in AI-enabled e-scooter which showed a highly significant and strong positive relationship with willingness to use AI-enabled e-scooter. In contrast, perceived safety of road users around e-scooters, frequency of regular micromobility use, and previous crash experience showed weaker negative associations that were statistically insignificant. This reinforces the dominant role of safety and trust perceptions over other factors in influencing AI-assisted e-scooter adoption.

From a policy and industry perspective, fostering trust in AI-assisted micromobility requires more than just increased exposure. Rather than assuming that frequent micromobility users will naturally transition to AI-assisted models, stakeholders should focus on enhancing safety perceptions and transparency in AI functionality. Given the strong influence of perceived safety on both trust and adoption, manufacturers and urban mobility planners must prioritize clear communication of AI safety features, real-world performance data, and user education programs.


%From a policy and industry perspective, these findings offer several critical insights. First, fostering trust in AI-enabled mobility solutions may be more effective through positive user experiences rather than frequent exposure alone. Second, the strong influence of safety perceptions on trust and adoption underscores the need for developers and policymakers to prioritize and communicate safety features transparently. Finally, trust-building initiatives should focus on demonstrating AI reliability and safety, as these are key determinants of user acceptance.

%In the full-scale SEM model, the relationships between willingness to use AI-enabled e-scooters and predictors like - perception of trust in AI-enabled e-scooter and perception of safety in AI-enabled technology were found to be the most notable latent variables. Specifically, the standardized regression weight for perception of trust in AI-enabled e-scooter (0.597) indicates a moderate positive influence, while the weight for perception of safety in AI-enabled technology (0.775) suggests a strong positive impact on willingness to use AI-enabled e-scooters. %These results highlight the significant role of perception of safety in AI enabled technology and perception of trust in AI enabled E-Scooter in explaining or predicting changes in Willingness to Use AI enabled e-scooter. On the other hand, the standardized weights for perceived safety of road users around e-scooters (-0.091), frequency of regular micromobility use (-0.089), and Crash experience (-0.621) suggest relatively weak negative associations with Willingness to Use AI enabled e-scooter. These predictors appear to have minimal influence, with their effects being substantially smaller in magnitude compared to perception of trust in AI enabled E-Scooter and perception of safety in AI enabled technology.
%The full-scale SEM model further validates these relationships. Perceived trust in AI-enabled e-scooters ($\beta$ = 0.597) and perceived safety in AI enabled technology ($\beta$ = 0.775) emerged as the strongest predictors of willingness to use AI-assisted e-scooters. In contrast, perceived safety of road users around e-scooters (-0.091), other micromobility use frequency (-0.089), and previous crash experience (-0.621) showed weaker negative associations. This reinforces the dominant role of safety and trust perceptions over other factors in influencing AI-assisted e-scooter adoption.
%In summary, the analysis underscores the dominant contributions of perception of safety in AI-enabled technology and perception of trust in AI-enabled E-Scooter to willingness to use AI-enabled e-scooters, while the roles of perceived safety of road users around e-scooters, frequency of regular micromobility use, and crash experience are comparatively negligible. This finding reinforces the importance of prioritizing perception of safety in AI-enabled technologies and perception of trust in AI-enabled E-Scooter when examining the factors influencing willingness to use AI-enabled e-scooters.
%SEM direct effects models typically assume unidirectional relationships and are not designed to accommodate feedback loops, limiting their ability to capture the reciprocal causality often present in real-world systems. Additionally, these models require large sample sizes to ensure reliable estimates; insufficient samples can weaken the model's power and increase the risk of estimation errors. 

\section{Limitation and Future Work} \label{sec:limitation}

%Like all the studies, this study has certain limitations, 
The SEM direct effects models typically assume unidirectional relationships and are not designed to accommodate feedback loops, limiting their ability to capture the reciprocal causality often present in real-world systems. 
Contextual or situational factors, such as time or environment, that might moderate relationships between variables were not accommodated in this study. Furthermore, SEM models are highly context-dependent, meaning that the findings from a specific model may not generalize to other populations or settings without further validation. %Additionally, many new and complex models of SEM like (TAM, MIMIC Model) have been used in research with varying degree of success. Important variables addressed by these models might have been overlooked, therefore, researchers can include other variables not covered here in the research model and compare their results. 
%Additionally, an extension of this study could be converting this into a Technology Acceptance Model (TAM) with respect to AI enabled E-Scooters.
%In this study, most respondents were college or graduate students aged 18–25. 
Furthermore, future studies should include a more diverse group of respondents in terms of age and ethnicity to develop a deeper understanding of public perception regarding these technologies. There is a significant gap in the literature concerning adolescents' perception of safety and trust in using e-scooters, particularly as this mode of transport gains popularity among this age group, coinciding with the rise in e-scooter-related injuries among adolescents \cite{douglas2024high}.
%Moreover, the integration of AI in e-scooters in routine locomotion requires further investigation. A real-world implementation of AI-assisted e-scooters could improve people's understanding of what it feels like to use such a system, aligning with findings in the literature suggests that prior experience with autonomous vehicles positively influences trust in AI-assisted transportation \cite{othman2021public}. 
%An effective AI-assisted e-scooter system must be able to assess the rider's surroundings in real time, including road conditions and nearby objects. It must be able to detect road obstacles such as potholes and identify other road users, including vehicles and pedestrians. Additionally, a real-time feedback system could provide riders with situational awareness and hazard notifications while minimizing distractions.
Future research could explore the development of a high-fidelity virtual reality (VR) simulator of an AI-enabled e-scooter, allowing participants to virtually experience how the system functions \cite{guo2023psycho, MAD-IVE}. The 3D simulation in VR can enable users to experience an almost real-life scenario in a safe, environment, allowing the researchers to access the impact of AI-enabled e-scooters on users' perceptions of safety, trust, and usability. Furtehrmore, recent research has demonstrated the potential of multimodal augmented reality safety warnings in improving situational awareness and reducing reaction times in workers \cite{SABETI2024104867, Sabeti02012024} AR’s capability in enhancing real-time safety interventions. Similarly, EEG analysis was used to measure cognitive responses to AR safety warnings \cite{BANANIARDECANI2025106802}, revealing improved attention and situational awareness in high-risk environments. Future research on AI-assisted e-scooters could leverage these findings by integrating multi modal safety alerts into e-scooter systems, and study the human factors of warnings in the context of micromobiliy. Such an approach could be used to evaluate warnings impact on users' trust and perception of safety in AI-assisted micromobility systems.

%The decision tree analysis on demographics revealed ethnicity, age, and education as the most influential factors in preferences for AI-assisted e-scooters. Middle Eastern participants showed the strongest preference for AI-assisted options, though younger subgroups (18–21 and 31–35) leaned toward non-AI-assisted e-scooters. Asian participants, particularly those with advanced degrees, favored AI-assisted options, while higher income levels shifted preferences toward non-AI-assisted ones. Among White/Caucasian participants, advanced education and ages 36–40 were linked to AI-assisted preferences, while African American, Hispanic/Latino, and other groups generally preferred non-AI-assisted e-scooters, with younger, lower-income participants favoring non-AI-assisted options and older participants (41–50) choosing AI-assisted ones.

%% If you have bibdatabase file and want bibtex to generate the
%% bibitems, please use
%%
 \bibliographystyle{elsarticle-num} 
 \bibliography{cas-refs}

%% else use the following coding to input the bibitems directly in the
%% TeX file.

% \begin{thebibliography}{00}

% %% \bibitem{label}
% %% Text of bibliographic item

% \bibitem{}

% \end{thebibliography}
\end{document}
\endinput
%%
%% End of file `elsarticle-template-num.tex'.
