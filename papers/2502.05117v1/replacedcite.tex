\section{Literature Review}
\label{sec:literature}


\subsection{Perceptions of Safety in E-Scooter Use: Rider and Non-Rider Perspectives}
Safety is a critical concern regarding e-scooter use, and many studies focused on injury and crash data to address these concerns. However, another equally important aspect is the perception of safety, including how both users and non-users view e-scooters in terms of safety. This includes factors such as riding behavior, interactions with other road users (such as pedestrians, cyclists, and drivers), and the general feeling of safety while riding.

From the riders' perspective, one of the most significant safety concerns is road infrastructure. Many riders report a low sense of safety when navigating major streets, largely due to the risk of collisions with vehicles. A particularly prominent concern is the possibility of encountering potholes, which can destabilize the rider and lead to crashes ____. Both riders and non-riders perceive riding e-scooters in vehicle lanes as unsafe ____, reflecting a shared apprehension about the risks posed by mixed-traffic environments.
%Additionally, severe vibration events, often caused by uneven road surfaces or poor infrastructure, can result in a loss of control and accidents ____.
The study by Tian et al. ____ revealed that, although sidewalks accounted for the majority of e-scooter-related injuries (44\%, primarily minor injuries), they are still perceived by riders as the safest type of road infrastructure. This perception likely stems from the reduced interaction with vehicles, which riders associate with a lower risk of severe collisions, despite the higher incidence of minor injuries on sidewalks.

Several studies highlighted the safety concerns that e-scooters pose from the perspective of other road users. Všucha et al. ____ conducted surveys across five countries, finding that non-riders generally perceived e-scooter use as dangerous, with women and older individuals more likely to consider them unsafe.
Derrick et al. ____ surveyed participants in Singapore to assess pedestrian safety concerns and support for an e-scooter ban. The findings showed that 64\% of respondents supported a ban on e-scooters, citing their danger to pedestrians. Key factors influencing this support included age, negative personal experiences, and social norms (e.g., family or peer opinions), and speeding was identified as the most common safety concern.
James et al. ____ reported that pedestrians and drivers felt less safe walking or driving around dockless e-scooters compared to bicycles. In another study, most pedestrians and cyclists perceived abandoned e-scooters on sidewalks as hazards to other road users ____.
%This discomfort was likely due to the novelty of e-scooters, which reduces familiarity and confidence in navigating shared spaces with them. Increased exposure, as seen with bicycles, might mitigate these concerns over time.Pedestrians and cyclists viewd abandoned e-scooters obstructing sidewalks. Burt et al. ____ found that 78\% of participants considered abandoned e-scooters a hazard to other road users. 
%Additionally, tandem riding—carrying a passenger—was identified as a factor contributing to both injuries and heightened perceived risk.

%Several studies have explored the perception of risky behaviors associated with e-scooter riders. Gioldasis et al. ____ investigated three key risky behaviors: riding under the influence of alcohol, drug use, and smartphone use while riding. The survey revealed that young and male riders are more likely to engage in these behaviors, with longer trips correlating with increased risk-taking.
%Similarly, Burt et al. ____ highlighted risky behaviors such as riding on pavements, alcohol consumption, and minimal helmet use. An observational study in Berlin, Germany____, documented a high prevalence of rule violations among shared e-scooter users. Over 25\% of riders were observed misusing infrastructure, while 10\% rode against traffic flow, 5\% engaged in dual occupancy, and none used helmets.
%Regarding helmet usage, Pourfalatoun et al. ____ found differing perceptions between users and non-users. Non-users placed greater emphasis on the importance of helmet use, while users expressed more moderate views about its necessity. These findings underscore the need for targeted interventions to address risky behaviors and improve safety awareness among e-scooter riders.

Although previous studies examined safety perceptions of regular e-scooters, the perception of safety in AI-integrated e-scooters remains unexplored. Further studies are required to assess how safety perceptions of AI enabled technology and road users' views on e-scooters influence overall safety perception and trust in AI-enabled e-scooter. 

%Safety is one of the most important issues that has emerged with the rise in the use of E-scooters. The use of scooters on sidewalks is arguably the most contentious issue and source of frustration for local officials. A separate category of problems arises with the introduction of a new mode of transportation, including compatibility issues with existing modes and inadequate riding/driving experience for users.
%Accidents involving pedestrians and cyclists have caused injuries and increased responsibility concerns in the communities. It is possible to attribute part of the misuse of the dockless vehicles to users' inexperience with them and the city's regulations governing their use. The users must ultimately educate themselves about the many city regulations governing the use of dockless automobiles. The manufacturers of E-scooters have also made a number of attempts to inform the public about neighborhood laws and the risks associated with riding on sidewalks \Cite: Kazemzadeh et al. (2023).
%Another inherent challenge associated with the use of E-Scooter emerges from the absence of adequate infrastructure in many communities to support alternative transportation methods. Typically designed around accommodating cars, these transportation networks face potential issues when E-Scooter vehicles begin utilizing the same street space. The prevalent car-centric layout in numerous cities might lead to unsafe or precarious situations. In fact, cities might witness an increased risk where cars pose a threat to E-Scooter vehicles on the streets, additional to the already existing bicycles and scooters present on the sidewalks. This unfortunate reality is highlighted often in the form of collisions with normal cars and E-Scooter automobiles. Drivers, unaccustomed to sharing the road with diverse vehicles, often encounter challenges when small, vulnerable scooters and bicycles operate within the same zones as cars, resulting in accidents and, tragically, fatalities. This issue is further complicated by the absence of an efficient accident tracking system implemented by scooter companies \cite Folco et. al 2023
%The obstacles encountered have motivated numerous cities to invest in specific infrastructure capable of supporting alternative transportation methods. Several cities have initiated the establishment of bicycle lanes, either by re purposing areas that were once designated for curbside parking or by introducing physical barriers between bicycle lanes and lanes for vehicles. The configuration of a bike lane requires a thorough consideration of existing traffic levels and behaviors, adequate safety buffers to protect bicyclists and other small mobility users from parked and moving vehicles, and enforcement to prohibit motorized vehicle encroachment and double-parking. Bike Lanes may be distinguished using color, lane markings, signage, and intersection treatments \Cite: Hossein et al. (2023).

%Such policies and initiatives contribute to fostering a stronger E-scooter riding culture by simplifying, securing, and enhancing the efficiency of E-scooters and the use of alternative modes. With an increasing number of residents opting for alternative modes of transport, motorists will gradually grow accustomed to sharing road space. This adjustment holds the potential for a cumulative positive impact on safety and the environment.
%Additionally, another significant safety concern confronting providers and cities involves the usage of helmets. Many injuries associated with scooters directly result from riders not wearing helmets. The lack of legal obligation to wear a helmet and other personal protective equipment’s along with the absence of an adequate and feasible concept of protective equipment for sharing services are the main barriers to helmet use among riders. However, the nature of shared systems offers pedestrians the liberty to utilize a bike or scooter at their convenience, granting considerable freedom but also potentially exposing riders to unanticipated risks. Typically, a traditional cyclist, using their personal bike, is more inclined to possess their own helmet compared to users of dockless devices. These users engage with the vehicles spontaneously and might not prefer carrying bulky helmets without certainty regarding their use of a scooter \Cite: Serra et al. (2021). 
%While scooter providers strongly encourage riders to wear helmets while using their services, they do not push for cities to enforce helmet mandates. Addressing this challenge proves notably complex because these providers lack a system for enforcement and benefit from maintaining riders' flexible engagement with their vehicles. This situation underscores the challenging position that scooters pose for local law enforcement and traffic officials. Enforcing helmet usage demands time, and instances of non-compliance are widespread. Numerous cities continue to struggle with identifying appropriate regulatory measures to enhance resident safety in this area. 
\subsection{Perception of Safety and Trust in AI-assisted Vehicles}
The level of trust in AI-assisted vehicles has been a significant topic of interest since the advent of autonomous vehicles (AVs). Public concerns about AVs are often amplified by reports of accidents involving these vehicles, which can further hinder their acceptance. Previous experiences with AVs play a crucial role in shaping public attitudes toward this technology. Individuals with prior exposure to AVs tended to exhibit more positive perceptions and a greater willingness to adopt them compared to those without such experiences ____. Previous studies examined the impact of sociodemographics on trust in AI-assisted vehicles. One study revealed that individuals aged 36 to 65 expressed greater apprehension and resistance toward driving autonomous vehicles compared to both younger individuals aged 18 to 35 and older individuals aged 65 and above ____. In other study, older people generally held more pessimistic views about AVs ____. Gender differences also emerged, with males expressing more positive attitudes toward AVs than females.
In terms of education, individuals with a university degree (Bachelor's, Master's, or PhD) exhibited lower levels of concern regarding accident liability and system failures in autonomous vehicles compared to those without a degree ____. This finding aligns with another study, which found that people with higher education levels are more likely to have a positive view of AVs than those with lower levels of education ____. Interestingly, individuals in countries with lower GDP levels tended to have more positive attitudes toward AVs compared to those in medium- or high-GDP countries____.

%A review study by Othman et al. ____ highlights several intriguing insights into public attitudes toward autonomous vehicles (AVs). Interestingly, individuals in countries with lower GDP levels tended to have more positive attitudes toward AVs compared to those in medium- or high-GDP countries. Moreover, the survey results indicated that older people generally held more pessimistic views about AVs.
%Willingness to pay for new technology is another critical factor influencing its adoption. However, surveys reveal that only a small percentage of individuals are willing to pay a premium for AVs. Gender differences also emerge, with males expressing more positive attitudes toward AVs than females. Similarly, individuals with higher levels of education are more likely to view AVs favorably compared to those with lower educational backgrounds ____.

% Hilgarter et al. ____ compared safety perceptions of autonomous vehicles (AVs) in rural and urban environments, revealing that AVs receive a more favorable reception in rural areas than in urban settings. In rural regions, AVs are seen as having the potential to shift individuals from private car use to public transportation. Notably, survey participants generally view AVs as an alternative to, rather than a complete replacement for, existing transportation methods.

Pyrialakou et al. ____ examined safety perceptions among vulnerable road users interacting with AVs. Their findings highlighted gender differences, with females feeling less safe around AVs than males. Among various activities, cycling near AVs was perceived as the least safe, followed by walking and then driving. The study also found that feeling safe near AVs was positively associated with reduced concerns about threats such as hacking or terrorism. Additionally, direct experience with AVs significantly improved pedestrian safety perceptions.

Previous studies primarily examined the trust and safety perceptions in autonomous vehicles, with limited attention to other AI-assisted modes of transport. To the best of our knowledge, no study has specifically investigated the perception of trust and safety in AI-assisted e-scooters, either from the perspective of riders or other road users. This study explores the level of trust in AI-assisted e-scooters and its influence on the willingness to adopt this emerging system.


%Hilgarter et al. explored public acceptance of autonomous vehicles (AVs) and perceived passenger safety after riding an autonomous shuttle in Carinthia, Austria, through interviews with 19 participants. The findings show that AVs are more positively received in rural areas, where they are seen as a way to shift from private cars to public transportation, and older adults perceive more benefits from AVs than younger people, particularly in sustaining mobility. While AVs are viewed as an alternative to existing transport, respondents emphasized that societal challenges like job loss, privacy concerns, and the high cost of AVs must be addressed. Safety perceptions were largely positive, with 84\% of participants feeling safe, influenced by the slow speed of the shuttle (10 km/h) and the experience of riding an AV, which improved their confidence in the technology. However, some concerns about hacking and the need for better public awareness and training were noted ____.Othman et al. conducted a literature review to examine survey papers on public perceptions of autonomous vehicles (AVs), focusing on how safety, ethics, liability, and regulations impact AV acceptance. The review finds that public concern about AVs rises with reports of accidents, and those with prior experience with AVs are more accepting. Interestingly, while older adults are considered early adopters due to increased accessibility, surveys reveal they are the most pessimistic toward AVs. Willingness to pay for AVs remains low, with men and individuals with higher education levels being more positive. Concerns about cybersecurity are widespread, and people in lower-GDP countries tend to be more optimistic about AVs than those in wealthier nations ____.Pyrialkou et al. conducted a survey study in Phoenix, Arizona, exploring the safety perceptions of road users interacting with autonomous vehicles (AVs), including drivers, cyclists, and pedestrians. The findings reveal that feeling safe near AVs is positively linked to a lack of concern about threats like hacking or terrorism, while direct experience with AVs improves pedestrian safety perceptions. However, distrust in strangers and a focus on travel time negatively impact these perceptions. Cycling near AVs is seen as the least safe activity, followed by walking and then driving. Females are less likely than males to feel safe around AVs, and overall exposure and experience with AVs can have both positive and negative effects on safety perceptions  ____.The study by Thomas et al. explored public perceptions of the benefits and challenges of autonomous vehicles (AVs), focusing on sociodemographic factors and acceptance at both individual and societal levels through an online survey. Respondents identified key concerns, including AV malfunctions, crash risks, purchase price, liability for incidents, interactions with non-AV vehicles, unexpected situations, hacking, and overall safety. While AVs were generally perceived as low-risk to drive, those with university degrees were less concerned about system failures and liability issues. In contrast, older respondents (36–65 years) expressed more concern and were less willing to drive AVs compared to both younger (18–35 years) and older (65+) age groups ____.

\subsection{Adoption of E-scooter}
Previous studies explored factors influencing the adoption of electric scooters. 
Sanders et al. ____ investigated the willingness to use e-scooters in Arizona, U.S., revealing significant gender-based differences in barriers to usage, especially regarding safety concerns. Moreover, African American and non-white Hispanic respondents were more likely than non-Hispanic white respondents to express an intention to try e-scooters. Nikiforiadis et al. ____ examined shared e-scooter usage in Thessaloniki, Greece, and found that females were less inclined to use e-scooters compared to males.
Teixeira et al. ____ conducted a survey across five European capital cities to explore the barriers preventing non-users from adopting e-scooters. The results implied that these obstacles are largely external and infrastructural, including the convenience of alternative transport modes, safety concerns about riding in traffic and inadequate road conditions. Similarly, She et al.____ identified barriers to the widespread adoption of electric scooters and found that younger individuals showed more positive attitudes toward using them compared to older generations.

Structural Equation Modeling (SEM), often based on the extended Technology Acceptance Model (TAM), is a widely used approach for analyzing factors influencing e-scooter adoption____. Ari et al. ____ found that social influences and enjoyment were the most significant predictors of perceived ease of use and perceived usefulness. Samadzad et al. ____ highlighted that perceived usefulness, trust, and subjective norms are key factors shaping the adoption and willingness to use shared e-scooters. Javadinasr et al. ____ focused on factors driving the continuous use of e-scooters in Chicago, showing that perceived usefulness is the most influential factor, followed by perceived reliability.

Despite significant research on the adoption of regular e-scooters, no studies, to the best of our knowledge, have specifically examined the willingness to use AI-assisted e-scooters. Key questions remain unresolved, such as identifying the profiles of individuals who prefer AI-assisted e-scooters over regular ones and uncovering the factors that influence the willingness to adopt this new system.
%The examination of previously published literature highlights how various factors such as perceived usefulness, perceived ease of use, environmental concerns, lack of appropriate infrastructure. performance, social impact, and safety problems for electric scooter user affect people's attitudes towards using electric scooters (Eccarius and Lu, 2020 ____; Sovacool et al 2019 ____; Sun et al, ____ Erkan and Ari ____).


%\subsection{Use of E-scooter}Previous studies explored the factors shaping e-scooter usage, focusing on sociodemographic characteristics, general attitudes, and usage patterns.
%While studies from different regions reveal some consistent trends, they also highlight unique findings specific to particular geographic contexts.Christoforou et al. conducted a survey study in Paris, France, revealing that most e-scooter users are male, aged between 18 and 29, possess a high level of education, and are less likely to own their own e-scoooter ____.
% Another research by Almanaa et al. ____ in Saudi Arabia indicated that males are more likely to use e-scooters than females, with the majority of potential users falling within the 18 to 45 age range.Similarly, Nikiforiadis et al. ____ examined shared e-scooter usage in Thessaloniki, Greece, and found that females were less inclined to use e-scooters compared to males. Furthermore, their study revealed that individuals residing in downtown areas are more regular e-scooter users compared to those living farther from the city center.


%Similarly, a study conducted in Colorado, U.S., highlighted the profile of e-scooter users as predominantly younger, employed individuals residing in urban areas with longer commutes ____. Compared to non-users, e-scooter users generally held more positive perceptions regarding the safety, sanitary conditions, comfort while riding intoxicated, and overall usability of shared e-scooters. 

%Research on the perceived usefulness of e-scooters has uncovered several key insights. A survey conducted in the United Kingdom found that e-scooters are widely viewed as an efficient, sustainable, and practical solution for first- and last-mile transportation needs ____.Despite the extensive body of research on the use of regular e-scooters, to the best of our knowledge, no studies have specifically examined the willingness to use AI-assisted e-scooters. Critical questions remain unanswered, including identifying the profiles of individuals who favor AI-assisted e-scooters over regular ones, exploring general perceptions surrounding the use of AI-assisted systems, and uncovering the factors that drive willingness to adopt these cutting-edge technologies.