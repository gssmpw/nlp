\documentclass{article}

\usepackage{microtype}
% \usepackage{subfigure}
\usepackage{booktabs} 
\usepackage{hyperref}
\usepackage{tabularx}
\newcommand{\theHalgorithm}{\arabic{algorithm}}

\usepackage[accepted]{icml2025}
% \usepackage{icml2025}
\usepackage{amsmath}
\usepackage{amssymb}
\usepackage{mathtools}
\usepackage{amsthm}
\usepackage{graphicx}
\usepackage{caption}
\usepackage{subcaption}
\usepackage{wrapfig}
\usepackage{bm}
\usepackage{bbm}
\usepackage{url}
\usepackage{multirow}
\usepackage{enumitem}
\usepackage{diagbox}
\usepackage{color,colortbl}
\usepackage{xcolor}
\usepackage{makecell}
\usepackage{tcolorbox}
\definecolor{COLOR_MEAN}{HTML}{f0f0f0}

% if you use cleveref..
\usepackage[capitalize,noabbrev]{cleveref}

%%%%%%%%%%%%%%%%%%%%%%%%%%%%%%%%
% THEOREMS
%%%%%%%%%%%%%%%%%%%%%%%%%%%%%%%%
\theoremstyle{plain}
\newtheorem{theorem}{Theorem}[section]
\newtheorem{proposition}[theorem]{Proposition}
\newtheorem{lemma}[theorem]{Lemma}
\newtheorem{corollary}[theorem]{Corollary}
\theoremstyle{definition}
\newtheorem{definition}[theorem]{Definition}
\newtheorem{assumption}[theorem]{Assumption}
\theoremstyle{remark}
\newtheorem{remark}[theorem]{Remark}

% Todonotes is useful during development; simply uncomment the next line
%    and comment out the line below the next line to turn off comments
%\usepackage[disable,textsize=tiny]{todonotes}
\usepackage[textsize=tiny]{todonotes}


% The \icmltitle you define below is probably too long as a header.
% Therefore, a short form for the running title is supplied here:
\icmltitlerunning{A Comprehensive Analysis on LLM-based Node Classification Algorithms}

\begin{document}

\twocolumn[
\icmltitle{A Comprehensive Analysis on LLM-based Node Classification Algorithms}

% It is OKAY to include author information, even for blind
% submissions: the style file will automatically remove it for you
% unless you've provided the [accepted] option to the icml2025
% package.

% List of affiliations: The first argument should be a (short)
% identifier you will use later to specify author affiliations
% Academic affiliations should list Department, University, City, Region, Country
% Industry affiliations should list Company, City, Region, Country

% You can specify symbols, otherwise they are numbered in order.
% Ideally, you should not use this facility. Affiliations will be numbered
% in order of appearance and this is the preferred way.
\icmlsetsymbol{equal}{*}

\begin{icmlauthorlist}
\icmlauthor{Xixi Wu}{cuhk}
\icmlauthor{Yifei Shen}{msra}
\icmlauthor{Fangzhou Ge}{cuhk}
\icmlauthor{Caihua Shan}{msra}
\icmlauthor{Yizhu Jiao}{uiuc}
\icmlauthor{Xiangguo Sun}{cuhk}
\icmlauthor{Hong Cheng}{cuhk}
%\icmlauthor{}{sch}
%\icmlauthor{}{sch}
\end{icmlauthorlist}

\icmlaffiliation{cuhk}{The Chinese University of Hong Kong}
\icmlaffiliation{msra}{Microsoft Research Asia}
\icmlaffiliation{uiuc}{University of Illinois Urbana-Champaign}

\icmlcorrespondingauthor{Yifei Shen}{\texttt{yifeishen@microsoft.com}}
\icmlcorrespondingauthor{Hong Cheng}{\texttt{hcheng@se.cuhk.edu.hk}}


% You may provide any keywords that you
% find helpful for describing your paper; these are used to populate
% the "keywords" metadata in the PDF but will not be shown in the document
\icmlkeywords{Large Language Models, Graph Neural Networks, Node Classification}

\vskip 0.3in
]

% this must go after the closing bracket ] following \twocolumn[ ...

% This command actually creates the footnote in the first column
% listing the affiliations and the copyright notice.
% The command takes one argument, which is text to display at the start of the footnote.
% The \icmlEqualContribution command is standard text for equal contribution.
% Remove it (just {}) if you do not need this facility.

% \printAffiliationsAndNotice{}  % leave blank if no need to mention equal contribution
\printAffiliationsAndNotice{} % otherwise use the standard text.

\begin{abstract}
Node classification is a fundamental task in graph analysis, with broad applications across various fields. Recent breakthroughs in Large Language Models (LLMs) have enabled LLM-based approaches for this task. Although many studies demonstrate the impressive performance of LLM-based methods, the lack of clear design guidelines may hinder their practical application.  In this work, we aim to establish such guidelines through a fair and systematic comparison of these algorithms. As a first step, we developed LLMNodeBed, a comprehensive codebase and testbed for node classification using LLMs. It includes ten datasets, eight LLM-based algorithms, and three learning paradigms, and is designed for easy extension with new methods and datasets. Subsequently, we conducted extensive experiments, training and evaluating over 2,200 models, to determine the key settings (e.g., learning paradigms and homophily) and components (e.g., model size) that affect performance. Our findings uncover eight insights, e.g., (1) LLM-based methods can significantly outperform traditional methods in a semi-supervised setting, while the advantage is marginal in a supervised setting; (2) Graph Foundation Models can beat open-source LLMs but still fall short of strong LLMs like GPT-4o in a zero-shot setting. We hope that the release of LLMNodeBed, along with our insights, will facilitate reproducible research and inspire future studies in this field. Codes and datasets are released at \href{https://llmnodebed.github.io/}{\texttt{https://llmnodebed.github.io/}}.


\end{abstract}

\section{Introduction}

Node classification is a fundamental task in graph analysis, with a wide range of applications such as item tagging \cite{Mao2020ItemTF}, user profiling \cite{Yan2021RelationawareHG}, and financial fraud detection \cite{Zhang2022eFraudComAE}. Developing effective algorithms for node classification is crucial, as they can significantly impact commercial success. For instance, US banks lost 6 billion USD to fraudsters in 2016. Therefore, even a marginal improvement in fraud detection accuracy could result in substantial financial savings.

Given its practical importance, node classification has been a long-standing research focus in both academia and industry. The earliest attempts to address this task adopted techniques such as Laplacian regularization \cite{belkin2006manifold}, graph embeddings \cite{yang2016revisiting}, and label propagation \cite{zhu2003semi}. Over the past decade, GNN-based methods have been developed and have quickly become prominent due to their superior performance, as demonstrated by works such as \citet{kipf2017GCN}, \citet{velickovic2018GAT}, and \citet{hamilton2017SAGE}. Additionally, the incorporation of encoded textual information has been shown to further complement GNNs' node features, enhancing their effectiveness \cite{jin2023patton, zhao2022GLEM}.

Inspired by the recent success of LLMs, there has been a surge of interest in leveraging LLMs for node classification \cite{li2023survey}. LLMs, pre-trained on extensive text corpora, possess context-aware knowledge and superior semantic comprehension, overcoming the limitations of the non-contextualized shallow embeddings used by traditional GNNs. Typically, supervised methods fall into three categories: Encoder, Reasoner, and Predictor. In the Encoder paradigm, LLMs employ their vast parameters to encode nodes' textual information, producing more expressive features that surpass shallow embeddings \cite{Zhu2024ENGINE}. The Reasoner approach utilizes LLMs' reasoning capabilities to enhance node attributes and the task descriptions with a more detailed text \cite{chen2024exploring, he2023TAPE}. This generated text augments the nodes' original information, thereby enriching their attributes. Lastly, the Predictor role involves LLMs integrating graph context through graph encoders, enabling direct text-based predictions  \cite{chen23llaga,tang2023graphgpt,chai2023graphllm,Huang2024GraphAdapter}. For zero-shot learning with LLMs, methods can be categorized into two types: Direct Inference and Graph Foundation Models (GFMs). Direct Inference involves guiding LLMs to directly perform classification tasks via crafted prompts \cite{Huang2023CanLE}. In contrast, GFMs entail pre-training on extensive graph corpora before applying the model to target graphs, thereby equipping the model with specialized graph intelligence \cite{li2024zerog}. An illustration of these methods is shown in Figure \ref{fig:llm_role}. 

Despite tremendous efforts and promising results, the design principles for LLM-based node classification algorithms remain elusive. Given the significant training and inference costs associated with LLMs, practitioners may opt to deploy these algorithms only when they provide substantial performance enhancements compared to costs. This study, therefore, seeks to identify \textbf{(1) the most suitable settings for each algorithm category, and (2) the scenarios where LLMs surpass traditional LMs such as BERT}. While recent work like GLBench \cite{Li2024GLBench} has evaluated various methods using consistent data splits in semi-supervised and zero-shot settings, differences in backbone architectures and implementation codebases still hinder fair comparisons and rigorous conclusions. To address these limitations, we introduce a new benchmark that further standardizes backbones and codebases. Additionally, we extend GLBench by incorporating three new E-Commerce datasets relevant to practical applications and expanding the evaluation settings. Specifically, we assess the impact of supervision signals (e.g., supervised, semi-supervised), different language model backbones (e.g., RoBERTa, Mistral, LLaMA, GPT-4o), and various prompt types (e.g., CoT, ToT, ReAct). These enhancements enable a more detailed and reliable analysis of LLM-based node classification methods. In summary, our contributions to the field of LLMs for graph analysis are as follows:


% A fair comparison necessitates a benchmark that evaluates all methods using consistent data splitting ratios, learning paradigms, backbone architectures, and implementation codebases. A very recent work, GLBench~\cite{Li2024GLBench}, tested various methods on several datasets in a semi-supervised/zero-shot setting, maintaining the same data splits. However, differences in the underlying backbones and implementation codebases still pose challenges for a fair comparison and drawing rigorous conclusions of the above questions. This paper introduces a benchmark that further standardizes the backbones and implementation codebases. Moreover, we expand upon GLBench by providing additional datasets and evaluation settings. Specifically, we include three new datasets from the E-Commerce sector, which are more relevant for practical commercial applications. We also assess the influence of supervision signals (e.g., supervised or semi-supervised), various language model backbones (e.g., RoBERTa, Mistral, GPT-4o), and prompts (e.g., CoT, ToT, and ReAct). These datasets and settings enable a detailed analysis of the aforementioned questions. 



% However, existing works lack the necessary standardization for such comparisons. An algorithm that performs exceptionally well in its original paper might underperform when used as a baseline in subsequent studies. This discrepancy often arises from variations in data splitting, learning paradigms, backbone architectures, and implementation codebases.  The backbone architecture and implementations are adopted from the original papers, which 

% To address this issue, this paper introduces a testbed for LLM-based node classification algorithms and conducts extensive experiments to derive insights and guidelines. 

\begin{itemize}
    \item \textbf{A Testbed:} We release LLMNodeBed, a PyG-based testbed designed to facilitate reproducible and rigorous research in LLM-based node classification algorithms. The initial release includes ten datasets, eight LLM-based algorithms, and three learning configurations. LLMNodeBed allows for easy addition of new algorithms or datasets, and a single command to run all experiments, and to automatically generate all tables included in this work.
    
    \item \textbf{Comprehensive Experiments:} By training and evaluating over 2,200 models, we analyzed how the learning paradigm, homophily, language model type and size, and prompt design impact the performance of each algorithm category.
    
    \item \textbf{Insights and Tips:} Detailed experiments were conducted to analyze each influencing factor. We identified the settings where each algorithm category performs best and the key components for achieving this performance. Our work provides intuitive explanations, practical tips, and insights about the strengths and limitations of each algorithm category.
\end{itemize}




%It has been a research focus in both academia and industry due to its wide range of applications, including item tagging \cite{Mao2020ItemTF}, user profiling \cite{Yan2021RelationawareHG}, and financial fraud detection \cite{Zhang2022eFraudComAE}. 


%Building effective algorithms for node classification is a long-standing topic as it has a direct impact on commercial success \cite{Lo2022InspectionLSG}.

%Before the popularity of LLMs, node classification is typically tackled by graph neural networks (GNNs) or language models (LMs) such as BERT \cite{Devlin2019BERTPO}. GNNs \cite{kipf2017GCN,velickovic2018GAT,hamilton2017SAGE} enhance node representations by aggregating information from neighboring nodes, thereby capturing the structural context essential for accurate classification. In contrast, LMs \cite{Wang2022e5-large, Liu2019roberta} focus on semantic representations by encoding the textual information associated with each node, transforming the node classification into a text classification task. The encoded textual information can further complement GNNs' node features \cite{jin2023patton, zhao2022GLEM}. Yifei: I think the current intro is too long, to move it to related works

%Over the past decade, we have witnessed great progress in node classification algorithms. The classical ones include Graph Neural Networks (GNNs) \cite{kipf2017GCN,velickovic2018GAT,hamilton2017SAGE} and additional language modeling to enhance the node features \cite{jin2023patton, zhao2022GLEM}. Recently, there has been a surge of interest in applying LLMs for node classification \cite{li2023survey}. In these studies, the roles performed by LLMs can be primarily 


% Despite the importance of this area, the literature of LLM-based node classification is scattered: the algorithms are evaluated under different datasets, learning paradigms, baselines, and implementation codebases. The purpose of this work is to perform rigorous comparisons among algorithms, as well as to open-source our software for anyone to replicate and extend our analysis. This manuscript investigates the question: \emph{How useful are LLMs for node classification under a fair setting?}

% To answer this question, we implement and tune eight LLM-based node classification algorithms, to compare them across ten datasets and three learning paradigms.  There are four major takeaways from our investigations: (1) \textbf{LLM-as-Encoder is effective for low-homophily graphs:} These methods outperform classic LM counterparts on low-homophily graphs, with the advantages being more obvious under limited supervision.
% (2) \textbf{LLM-as-Reasoner is the most effective when LLMs have prior knowledge of the target graph:} These methods achieve superior performance on datasets where the LLMs possess prior knowledge like academic and web link datasets, and benefit from more powerful models like GPT-4o. 
% (3) \textbf{LLM-as-Predictor methods is highly effective when labeled data is abundant}: Predictor methods require extensive supervision for model training, with their performance improving as larger LLMs adhering to scaling laws \cite{Kaplan2020ScalingLF} are utilized. Among different LLMs, Mistral-7B \cite{Jiang2023Mistral7B} consistently serves as a robust backbone. (4) \textbf{Zero-shot methods are most effective when neighbor information is injected:} Although Graph Foundation Models (GFMs) \cite{liu2023one, li2024zerog, Zhu2024GraphCLIPET} outperform open-source LLMs in zero-shot settings, they still lag behind advanced models like GPT-4o. The most effective zero-shot approaches involve injecting neighbor information to guide LLMs for direct inference.

% As a result of this paper, we release LLMNodeBed, a PyTorch-based testbed designed to facilitate reproducible and rigorous research in node classification algorithms. The initial release includes ten datasets, eight algorithms, three learning configurations, and the infrastructure to run all experiments. Our experimental framework can be easily extended to include new methods and datasets. We are committed to updating this repository with new algorithms and datasets and welcome pull requests from fellow researchers to ensure its ongoing development.


%While a myriad of algorithms exists, diverse datasets, architectures, learning configurations, and implementation codebases, rendering fair and realistic comparisons difficult and conclusions inconsistent. Inspired by standardized benchmarks in computer vision like ImageNet, this paper conducts a rigorous comparison of various LLM-based node classification methods to assess the true efficacy of LLMs. This investigation addresses the following research question:

%\textit{Under What Circumstances do LLMs Help Node Classification Task?}

%At a first step, we implement LLMNodeBed, a codebase and testbed for node classification with LLMs. It includes ten multi-domain graph datasets with varying scales and levels of homophily, supports eight representative algorithms that represent diverse LLM roles, and offers three learning configurations: semi-supervised, fully-supervised, and zero-shot. Through extensive experiments, we provide empirical insights into when LLMs contribute to node classification performance: 



% In summary, we make the following contributions: 

% \begin{enumerate}
%     \item \textbf{LLMNodeBed:} We introduce LLMNodeBed, a comprehensive and extensible testbed for evaluating LLM-based node classification algorithms. It comprises ten datasets, eight representative algorithms, and three learning scenarios, and can easily accommodate new datasets, methods, and backbones.
%     \item \textbf{Comprehensive Evaluation:} We conduct extensive empirical analysis across different datasets, algorithms, and learning settings to elucidate the efficacy of different LLM roles in node classification performance. 
%     \item \textbf{Practical Guidelines:} Based on our findings, we provide actionable guidelines for effectively applying LLMs to diverse real-world node classification tasks, enhancing their performance and applicability in various scenarios.
% \end{enumerate}

\begin{figure}[!t]
    \centering
    \includegraphics[width=\linewidth]{figs/LibFramework.pdf}
    \vspace*{-10pt}
    \caption{\textbf{Overview of LLMNodeBed.}}
    \vspace*{-10pt}
    \label{fig:system_implementation}
\end{figure}



\begin{figure*}[!t]
    \centering
    \includegraphics[width=0.86\linewidth]{figs/LLMRole.pdf}
    \vspace*{-0.2cm}
    \caption{\textbf{Illustrations of LLM-based node classification algorithms under supervised and zero-shot settings.}}
    \vspace*{-0.2cm}
    \label{fig:llm_role}
\end{figure*}


\section{Preliminaries on Node Classification}
To leverage the language abilities of LLMs, we study the node classification task within the context of text-attributed graphs (TAGs) \cite{ma2021deep}. TAGs are represented as $\mathcal{G} = (\mathcal{V}, \mathcal{E}, \mathcal{S})$, where $\mathcal{V}$ represents the set of nodes, $\mathcal{E}$ the set of edges, and $\mathcal{S}$ the collection of textual descriptions associated with each node $v \in \mathcal{V}$. Some of the nodes are associated with labels, represented as $\mathcal{V}_l \subset \mathcal{V}$. The remaining nodes do not have labels, and are denoted as $\mathcal{V}_u$. The goal of node classification is to train a neural network based on the graph $\mathcal{G}$ and the labels of $\mathcal{V}_l$, which can predict the labels of the unlabeled nodes in $\mathcal{V}_u$.

Traditionally, the textual attributes of nodes can be encoded into shallow embeddings as $\bm{X} = [\bm{x}_1, \ldots, \bm{x}_{|\mathcal{V}|}] \in \mathbb{R}^{|\mathcal{V}| \times d}$ using naive methods like bag-of-words or TF-IDF \cite{Salton1988TermWeightingAI}, where $d$ represents the dimensionality of the embeddings. Such transformation is adopted in most GNN papers. Instead, the input for LLM-based approaches is the raw text and one may expect that the pre-trained knowledge in LLMs can improve performance.







\section{LLMNodeBed: A Testbed for LLM-based Node Classification}
In this section, we present the datasets, baselines, and learning paradigms within LLMNodeBed (Figure \ref{fig:system_implementation}).

\subsection{Datasets}

To provide guidelines for applying algorithms across diverse real-world applications, the selection of datasets in LLMNodeBed considers several key factors: (1) \textbf{Multi-domain Diversity} to reflect different contexts, (2) \textbf{Varying Scales} to examine algorithm scalability and the associated costs of leveraging LLMs, and (3) \textbf{Diverse Levels of Homophily} to understand its impact on performance. Therefore, LLMNodeBed comprises ten datasets spanning the academic, web link, social, and E-Commerce domains. These datasets vary significantly in scale, ranging from thousands of nodes to millions of edges, and exhibit differing levels of homophily. Such diversity in domain, scale, and homophily enables the assessment of algorithms across a wide range of contexts. 

For datasets where raw text has been preprocessed into vector embeddings using bag-of-words or TF-IDF techniques, we utilize collected versions including Cora and Pubmed \cite{he2023TAPE}, Citeseer \cite{chen2024exploring}, and WikiCS \cite{liu2023one}. The remaining datasets already include text attributes in their official releases, including arXiv \cite{hu2020open}, Instagram and Reddit \cite{Huang2024GraphAdapter}, and Books, Computer, and Photo \cite{yan2023comprehensive}. Detailed statistics and information for these datasets are provided in Table \ref{tab:dataset} and Table \ref{tab:dataset_detail} in the Appendix.




\subsection{Baselines} 
The initial release of LLMNodeBed includes eight LLM-based baseline algorithms alongside classic methods. We selected these LLM-based algorithms based on three key criteria: (1) \textbf{Diverse Roles of LLMs} to thoroughly evaluate their effectiveness, (2) \textbf{Straightforward Design} to facilitate clear comparisons by avoiding complex and intertwined architectures, and (3) \textbf{Representativeness} to ensure benchmark relevance by including widely recognized methods. Therefore, the LLM-based baselines include:

\textbf{LLM-as-Encoder: }We include \textbf{ENGINE} \cite{Zhu2024ENGINE} and introduce \textbf{GNN\textsubscript{LLMEmb}}. ENGINE aggregates hidden embeddings from each LLM layer to create comprehensive node representations. In contrast, GNN\textsubscript{LLMEmb} initializes node embeddings using the LLM's last hidden layer before feeding them into GNNs for classification.

\textbf{LLM-as-Reasoner: }We select \textbf{TAPE} \cite{he2023TAPE}, a representative LLM-as-Reasoner method. TAPE prompts the LLM to reason over nodes by generating predictions along with explanations, thereby enriching the node's text attributes and enhancing classification performance.

Both Encoder and Reasoner methods require processing the entire dataset, either by encoding each node's text or by performing reasoning over nodes, which introduces additional processing time before the actual model training begins.

\textbf{LLM-as-Predictor: }We select \textbf{GraphGPT} \cite{tang2023graphgpt}, \textbf{LLaGA} \cite{chen23llaga}, and implement \textbf{LLM Instruction Tuning} (\textbf{LLM\textsubscript{IT}}). GraphGPT employs a multi-stage pre-training and instruction tuning process to classify nodes based on both text and graph context. \textbf{LLaGA} integrates tokenized task instructions and graph context into LLMs to generate predictions.  We implement LLM\textsubscript{IT} to evaluate whether LLMs alone can function as effective predictors. This involves fine-tuning the LLM using task prompts and ground-truth labels formatted as $\langle \texttt{Question}, \texttt{Answer} \rangle$ pairs. 

\textbf{LLM Direct Inference: } This category refers to LLMs generating prediction labels directly from a node's text without additional training or labels. We employ two types of prompt templates: (1) \textbf{Advanced Prompts} that improve LLMs' reasoning abilities such as Chain-of-Thought (CoT) \cite{Wei2022ChainOT} and Tree-of-Thought (ToT) \cite{yao2023tree}, and (2) \textbf{Enriched Prompts} that incorporate neighboring node information to provide structural context.


\textbf{GFMs: } GFMs are foundation models trained on large-scale source graph datasets to acquire general classification knowledge, which can then be seamlessly applied to target graphs. We include \textbf{ZeroG} \cite{li2024zerog} as a representative GFM due to its superior performance in zero-shot settings. Additionally, LLM-as-Predictor methods trained with extensive graph corpora are also considered within this category for zero-shot applications.

Besides LLM-based methods, LLMNodeBed also integrates classic algorithms, including \textbf{MLPs}, \textbf{GNNs}, and \textbf{LMs}. MLPs generate predicted label matrices from node embeddings, while GNNs combine shallow embeddings with graph structures for label prediction. LMs process a node's text through hidden layers and use a classification head to produce label distributions. Further discussions of existing node classification algorithms are provided in Appendix \ref{sec:related_works}.

Prompt templates for LLM-as-Predictor and Direct Inference are listed in Appendix \ref{sec:predictor_prompt} and \ref{sec:zeroshot_prompt}, respectively. Appendix \ref{sec:hyperparam} describes the implementation details, backbone selections, and hyperparameter search spaces for all algorithms. Additionally, we highlight the distinctions of LLMNodeBed in Appendix \ref{sec:distinct_llmnodebed}.


\begin{table*}[!t]
    \centering
    \caption{\textbf{Performance comparison under semi-supervised and supervised settings with Accuracy ($\%$) reported.} \\\small{The \colorbox{orange!25}{\textbf{best}} and \colorbox{orange!10}{second-best} results are highlighted. Results of Macro-F1 are shown in Table \ref{tab:mainexp_f1} in the Appendix. LLM\textsubscript{IT} on the arXiv dataset requires 30+ hours per run, preventing multiple executions.}} 
    \vspace*{-8pt}
    \resizebox{\linewidth}{!}{
    \begin{tabular}{cc|cccccccccc}
      \toprule
     \rowcolor{COLOR_MEAN} \multicolumn{2}{c|}{\textbf{Semi-supervised}}  & \textbf{Cora} & \textbf{Citeseer} & \textbf{Pubmed} & \textbf{WikiCS} & \textbf{Instagram} & \textbf{Reddit} & \textbf{Books} & \textbf{Photo} & \textbf{Computer} & \textbf{Avg.} \\ \midrule
       \multirow{5}{*}{\textbf{Classic}} & {GCN\tiny{ShallowEmb}} & 82.30$_{\pm \text{0.19}}$ & 70.55$_{\pm \text{0.32}}$ & \cellcolor{orange!10} 78.94$_{\pm \text{0.27}}$ & 79.86$_{\pm \text{0.19}}$ & 63.50$_{\pm \text{0.11}}$ & 61.44$_{\pm \text{0.38}}$ & 68.79$_{\pm \text{0.46}}$ & 69.25$_{\pm \text{0.81}}$ & 71.44$_{\pm \text{1.19}}$ & 71.79 \\ 
        & SAGE\tiny{ShallowEmb} & 82.27$_{\pm \text{0.37}}$ & 69.56$_{\pm \text{0.43}}$ & 77.88$_{\pm \text{0.44}}$ & 79.67$_{\pm \text{0.25}}$ & 63.57$_{\pm \text{0.10}}$ & 56.65$_{\pm \text{0.33}}$ & 72.01$_{\pm \text{0.33}}$& 78.50$_{\pm \text{0.15}}$ & 81.43$_{\pm \text{0.27}}$ & 73.50 \\ 
         & {GAT\tiny{ShallowEmb}} & 81.30$_{\pm \text{0.67}}$ & 69.94$_{\pm \text{0.74}}$ & 78.49$_{\pm \text{0.70}}$ & 79.99$_{\pm \text{0.65}}$ & 63.56$_{\pm \text{0.04}}$ & 60.60$_{\pm \text{1.17}}$ & 74.35$_{\pm \text{0.35}}$ & 80.40$_{\pm \text{0.45}}$ & 83.39$_{\pm \text{0.22}}$ & 74.67 \\ 
         & SenBERT-66M & 66.66$_{\pm \text{1.42}}$ & 60.52$_{\pm \text{1.62}}$ & 36.04$_{\pm \text{2.92}}$ & 77.77$_{\pm \text{0.75}}$ & 59.00$_{\pm \text{1.17}}$ & 56.05$_{\pm \text{0.41}}$ & 83.68$_{\pm \text{0.19}}$ & 73.89$_{\pm \text{0.31}}$ & 70.76$_{\pm \text{0.15}}$ & 64.93 \\
         & {RoBERTa-355M} & 72.24$_{\pm \text{1.14}}$ & 66.68$_{\pm \text{2.03}}$ & 42.32$_{\pm \text{1.56}}$ & 76.81$_{\pm \text{1.04}}$ & 63.52$_{\pm \text{0.44}}$ & 59.27$_{\pm \text{0.34}}$ & \cellcolor{orange!10} 84.62$_{\pm \text{0.16}}$ & 74.79$_{\pm \text{1.13}}$ & 72.31$_{\pm \text{0.37}}$ & 68.06 \\ \midrule
         
        \multirow{2}{*}{\textbf{Encoder}} 
       & $\text{GCN}_{\text{LLMEmb}}$ & 83.33$_{\pm \text{0.75}}$ & 71.39$_{\pm \text{0.90}}$ & 78.71$_{\pm \text{0.45}}$ & \cellcolor{orange!10} 80.94$_{\pm \text{0.16}}$ & \cellcolor{orange!25} \textbf{67.49$_{\pm \text{0.43}}$} & 68.65$_{\pm \text{0.75}}$ &  83.03$_{\pm \text{0.34}}$ & \cellcolor{orange!10} 84.84$_{\pm \text{0.47}}$ & \cellcolor{orange!10} 88.22$_{\pm \text{0.16}}$ & \cellcolor{orange!25} \textbf{78.51} \\ 
       & ENGINE & \cellcolor{orange!25} \textbf{84.22$_{\pm \text{0.46}}$} & \cellcolor{orange!25} \textbf{72.14$_{\pm \text{0.74}}$}
        & 77.84$_{\pm \text{0.27}}$ & 80.94$_{\pm \text{0.19}}$ & \cellcolor{orange!10} 67.14$_{\pm \text{0.46}}$ & \cellcolor{orange!10} 69.67$_{\pm \text{0.16}}$ & 82.89$_{\pm \text{0.14}}$ & 84.33$_{\pm \text{0.57}}$ & 86.42$_{\pm \text{0.23}}$ & 78.40  \\ \midrule
        
       \textbf{Reasoner} & TAPE &  \cellcolor{orange!10} 84.04$_{\pm \text{0.24}}$ & \cellcolor{orange!10} 71.87$_{\pm \text{0.35}}$ & 78.61$_{\pm \text{1.23}}$ & \cellcolor{orange!25} \textbf{81.94$_{\pm \text{0.16}}$} & 66.07$_{\pm \text{0.10}}$ & 62.43$_{\pm \text{0.47}}$ & \cellcolor{orange!25} \textbf{84.92$_{\pm \text{0.26}}$} & \cellcolor{orange!25} \textbf{86.46$_{\pm \text{0.12}}$} & \cellcolor{orange!25} \textbf{89.52$_{\pm \text{0.04}}$} & \cellcolor{orange!10} 78.43  \\  \midrule
       
      \multirow{3}{*}{\textbf{Predictor}} & $\text{LLM}_{\text{IT}}$  &  67.00$_{\pm \text{0.16}}$ & 54.26$_{\pm \text{0.22}}$ & \cellcolor{orange!25} \textbf{80.99$_{\pm \text{0.43}}$} & 75.02$_{\pm \text{0.16}}$ & 41.83$_{\pm \text{0.47}}$ & 54.09$_{\pm \text{1.02}}$ & 80.92$_{\pm \text{1.38}}$ & 71.28$_{\pm \text{1.81}}$ & 66.99$_{\pm \text{2.02}}$ & 65.76 \\ 
       & GraphGPT & 64.72$_{\pm \text{1.50}}$ & 64.58$_{\pm \text{1.55}}$ & 70.34$_{\pm \text{2.27}}$ & 70.71$_{\pm \text{0.37}}$ & 62.88$_{\pm \text{2.14}}$ & 58.25$_{\pm \text{0.37}}$ & 81.13$_{\pm \text{1.52}}$ & 77.48$_{\pm \text{0.78}}$ & 80.10$_{\pm \text{0.76}}$ & 70.02 \\ 
       & LLaGA & 78.94$_{\pm \text{1.14}}$ & 62.61$_{\pm \text{3.63}}$ & 65.91$_{\pm \text{2.09}}$ & 76.47$_{\pm \text{2.20}}$ & 65.84$_{\pm \text{0.72}}$ &  \cellcolor{orange!25} \textbf{70.10$_{\pm \text{0.38}}$} & 83.47$_{\pm \text{0.45}}$ & 84.44$_{\pm \text{0.90}}$ & 87.82$_{\pm \text{0.53}}$  & 75.07 \\
       \bottomrule
    \end{tabular}
    }

   \vspace*{5pt}
    \resizebox{\linewidth}{!}{
    \begin{tabular}{cc|ccccccccccc}
      \toprule
      \rowcolor{COLOR_MEAN} \multicolumn{2}{c|}{\textbf{Supervised}} & \textbf{Cora} & \textbf{Citeseer} & \textbf{Pubmed} & \textbf{arXiv} & \textbf{WikiCS} & \textbf{Instagram} & \textbf{Reddit} & \textbf{Books} & \textbf{Photo} & \textbf{Computer} & \textbf{Avg.} \\ \midrule

     \multirow{5}{*}{\textbf{Classic}} & {GCN\tiny{ShallowEmb}} & 87.41$_{\pm \text{2.08}}$ & 75.74$_{\pm \text{1.20}}$ & 89.01$_{\pm \text{0.59}}$ & 71.39$_{\pm \text{0.28}}$ & 83.67$_{\pm \text{0.63}}$ & 63.94$_{\pm \text{0.61}}$ & 65.07$_{\pm \text{0.38}}$ & 76.94$_{\pm \text{0.26}}$ & 73.34$_{\pm \text{1.34}}$ & 77.16$_{\pm \text{3.80}}$ & 76.37 \\ 
     & {SAGE\tiny{ShallowEmb}} & 87.44$_{\pm \text{1.74}}$ & 74.96$_{\pm \text{1.20}}$ & 90.47$_{\pm \text{0.25}}$ & 71.21$_{\pm \text{0.18}}$ & 84.86$_{\pm \text{0.91}}$ & 64.14$_{\pm \text{0.47}}$ & 61.52$_{\pm \text{0.60}}$ & 79.40$_{\pm \text{0.45}}$ & 84.59$_{\pm \text{0.32}}$ & 87.77$_{\pm \text{0.34}}$ & 78.64 \\

     & {GAT\tiny{ShallowEmb}} & 86.68$_{\pm \text{1.12}}$ & 73.73$_{\pm \text{0.94}}$ & 88.25$_{\pm \text{0.47}}$ & 71.57$_{\pm \text{0.25}}$ & 83.94$_{\pm \text{0.61}}$ & 64.93$_{\pm \text{0.75}}$ & 64.16$_{\pm \text{1.05}}$ & 80.61$_{\pm \text{0.49}}$ & 84.84$_{\pm \text{0.69}}$ & 88.32$_{\pm \text{0.24}}$ & 78.70 \\ 
     & SenBERT-66M & 79.61$_{\pm \text{1.40}}$ & 74.06$_{\pm \text{1.26}}$ & \cellcolor{orange!10} 94.47$_{\pm \text{0.33}}$ & 72.66$_{\pm \text{0.24}}$ & 86.51$_{\pm \text{0.86}}$ & 60.11$_{\pm \text{0.93}}$ & 58.70$_{\pm \text{0.54}}$ & \cellcolor{orange!10} 85.99$_{\pm \text{0.58}}$ & 77.72$_{\pm \text{0.35}}$ & 74.22$_{\pm \text{0.21}}$ & 76.40 \\ 
     & {RoBERTa-355M} & 83.17$_{\pm \text{0.84}}$ & 75.90$_{\pm \text{1.69}}$ & \cellcolor{orange!25} \textbf{94.84$_{\pm \text{0.06}}$} & 74.12$_{\pm \text{0.12}}$ & \cellcolor{orange!25}\textbf{87.47$_{\pm \text{0.83}}$} & 63.75$_{\pm \text{1.13}}$ & 60.61$_{\pm \text{0.24}}$ & 
      \cellcolor{orange!25} \textbf{86.65$_{\pm \text{0.38}}$} & 79.45$_{\pm \text{0.37}}$ & 75.76$_{\pm \text{0.30}}$ & 78.17 \\ \midrule
      
      \multirow{2}{*}{\textbf{Encoder}} & $\text{GCN}_{\text{LLMEmb}}$ & 
      \cellcolor{orange!25} \textbf{88.15$_{\pm \text{1.79}}$} & \cellcolor{orange!10} 76.45$_{\pm \text{1.19}}$ & 88.38$_{\pm \text{0.68}}$ & 74.39$_{\pm \text{0.31}}$ & 84.78$_{\pm \text{0.86}}$ & 68.27$_{\pm \text{0.45}}$ & 70.65$_{\pm \text{0.75}}$ & 84.23$_{\pm \text{0.20}}$ & 86.07$_{\pm \text{0.20}}$ & 89.52$_{\pm \text{0.31}}$ & 81.09 \\ 
      & ENGINE & 87.00$_{\pm \text{1.60}}$ & 75.82$_{\pm \text{1.52}}$ & 90.08$_{\pm \text{0.16}}$ & 74.69$_{\pm \text{0.36}}$ & 85.44$_{\pm \text{0.53}}$ & \cellcolor{orange!10} 68.87$_{\pm \text{0.25}}$ & \cellcolor{orange!25} \textbf{71.21$_{\pm \text{0.77}}$} & 84.09$_{\pm \text{0.09}}$ & 86.98$_{\pm \text{0.06}}$ & 89.05$_{\pm \text{0.13}}$ & 81.32 \\  \midrule
      \textbf{Reasoner} & TAPE &  \cellcolor{orange!10} 88.05$_{\pm \text{1.76}}$ & 76.45$_{\pm \text{1.60}}$ & 93.00$_{\pm \text{0.13}}$ & 74.96$_{\pm \text{0.14}}$ & \cellcolor{orange!10} 87.11$_{\pm \text{0.66}}$ & 68.11$_{\pm \text{0.54}}$ & 66.22$_{\pm \text{0.83}}$ &  85.95$_{\pm \text{0.59}}$ & \cellcolor{orange!25} \textbf{87.72$_{\pm \text{0.28}}$} & \cellcolor{orange!25} \textbf{90.46$_{\pm \text{0.18}}$} & \cellcolor{orange!25}\textbf{81.80} \\  \midrule
      \multirow{3}{*}{\textbf{Predictor}} & $\text{LLM}_{\text{IT}}$ & 71.93$_{\pm \text{1.47}}$ & 60.97$_{\pm \text{3.97}}$ &  94.16$_{\pm \text{0.19}}$ & \cellcolor{orange!25} \textbf{76.08} & 80.61$_{\pm \text{0.47}}$ & 44.20$_{\pm \text{3.06}}$ & 58.30$_{\pm \text{0.48}}$ & 84.80$_{\pm \text{0.13}}$ & 78.27$_{\pm \text{0.54}}$ & 74.51$_{\pm \text{0.53}}$ & 72.38 \\ 
      & GraphGPT & 82.29$_{\pm \text{0.26}}$ & 74.67$_{\pm \text{1.15}}$ & 93.54$_{\pm \text{0.22}}$ & \cellcolor{orange!10} 75.15$_{\pm \text{0.14}}$ & 82.54$_{\pm \text{0.23}}$ & 67.00$_{\pm \text{1.22}}$  & 60.72$_{\pm \text{1.47}}$ & 85.38$_{\pm \text{0.72}}$ & 84.46$_{\pm \text{0.36}}$ & 86.78$_{\pm \text{1.14}}$ & 79.25  \\ 
      & LLaGA & 87.55$_{\pm \text{1.15}}$ & \cellcolor{orange!25} \textbf{76.73$_{\pm \text{1.70}}$} & 90.28$_{\pm \text{0.91}}$ & 74.49$_{\pm \text{0.23}}$ & 84.03$_{\pm \text{1.10}}$ & \cellcolor{orange!25} \textbf{69.16$_{\pm \text{0.72}}$} & \cellcolor{orange!10} 71.06$_{\pm \text{0.38}}$ & 85.56$_{\pm \text{0.30}}$ & \cellcolor{orange!10} 87.62$_{\pm \text{0.30}}$ & \cellcolor{orange!10} 90.41$_{\pm \text{0.12}}$ & \cellcolor{orange!10}81.69 \\ \bottomrule
    \end{tabular}
    }

    \label{tab:mainexp}
\end{table*}


\subsection{Learning Paradigms}

We evaluate the baselines under three learning configurations: Semi-supervised, Supervised, and Zero-shot. These configurations are defined as follows:

% [leftmargin=*, topsep=2pt]
\begin{itemize}
    \item \textbf{Semi-supervised Learning:} A small subset of nodes $\mathcal{V}_l \subseteq \mathcal{V}$ with known labels $\mathcal{Y}_l$ is provided. This setting assesses the model's ability to effectively utilize limited labeled data, reflecting real-world scenarios where labeling is scarce. For experimental datasets, we adopt the official splits designed for semi-supervised settings to ensure standardized evaluation.
   
    \item \textbf{Supervised Learning:} A larger subset of nodes $\mathcal{V}_l$ with known labels is provided, assessing the model's performance with abundant supervision. Specifically, we use a 60\% training, 20\% validation, and 20\% testing split for most datasets. This consistent split facilitates fair comparisons across baselines. Detailed data splits are provided in Table \ref{tab:dataset_detail} in the Appendix.
    
    \item \textbf{Zero-shot Learning:} No labeled data is provided for training. The model predicts labels solely based on node textual descriptions and the graph structure, assessing its ability to generalize to new, unseen data. For test samples, we follow existing literature \cite{Zhu2024GraphCLIPET} by selecting one smaller dataset from each domain and using 20\% of its nodes as test samples.
\end{itemize}

\section{Comparisons among Algorithm Categories}
In this section, we present empirical results among algorithm categories, along with key insights derived from them.

\subsection{Semi-supervised and Supervised Performance}
\textbf{Settings: }To ensure a fair comparison of baseline algorithms, all methods are implemented with consistent components: GCN \cite{kipf2017GCN} for GNNs, RoBERTa-355M \cite{Liu2019roberta} for LMs, and Mistral-7B \cite{Jiang2023Mistral7B} for LLM components where applicable. This uniformity guarantees that performance differences are attributable to model designs rather than underlying architectures. Each experiment was conducted over \textbf{4 runs}. Based on the results of Accuracy (Table \ref{tab:mainexp}) and Macro-F1 (Table \ref{tab:mainexp_f1} in Appendix), we summarize the following takeaways:

\textbf{Takeaway 1: Appropriately incorporating LLMs consistently improves the performance.} According to the table, the best performance is often achieved by LLM-based methods compared to classic methods. It suggests that using LLM to exploit the textual information is useful. 

\textbf{Takeaway 2: LLM-based methods provide greater improvements in semi-supervised settings than in supervised settings.} By comparing the tables, we observe that performance gains are more significant in semi-supervised scenarios. From an information-theoretic perspective, the node classification task with cross-entropy loss aims to maximize the mutual information between the graph and the provided labels, denoted as $I(\mathcal{G}; \mathcal{Y}_{l})$. If we consider graph $\mathcal{G}$ as a joint distribution of node attributes $\bm{X}$ and structure $\mathcal{E}$, we have: 
\begin{equation}
\label{eq:mutual_info}
    \begin{aligned}
    I(\mathcal{G}; \mathcal{Y}_{l}) = I(\bm{X}, \mathcal{E}; \mathcal{Y}_{l}) = I(\mathcal{E}; \mathcal{Y}_{l}) + I(\bm{X}; \mathcal{Y}_{l} | \mathcal{E}).
\end{aligned}
\end{equation}

The first term represents the information encoded in the graph structure, utilized by classic GNNs, while the second term represents information from node features, leveraged by LLMs. In semi-supervised settings, the mutual information between structure and labels is relatively low, allowing LLMs to contribute more significantly to performance. 

%In the semi-supervised setting, the training of GNNs is easier to overfit due to the lack of label information. The abundant pretrained knowledge can help under such circumstance. 

\textbf{Takeaway 3: LLM-as-Reasoner methods are highly effective when labels heavily depend on text.} TAPE achieves top or runner-up performance on academic and web link datasets like Cora and WikiCS, where structural information is less relevant to labels \cite{zhang2021graphless}. However, TAPE struggles with social networks that require deeper structural understanding, such as predicting popular users (high-degree nodes) on Reddit. 


\textbf{Takeaway 4: LLM-as-Encoder methods balance computational cost and accuracy effectively.} LLM-as-Encoder methods perform satisfactorily across all datasets. Further experiments in Section \ref{exp:encoder_comp} reveal that \textbf{LLM-as-Encoder methods are more effective than their LM counterparts when graphs are heterophilic.} Regarding cost-effectiveness, LLM-as-Reasoner should generate long reasoning text, which is far more time-consuming than encoding texts in LLM-as-Encoder (see Appendix \ref{sec:detail_cost}). Therefore, LLM-as-Encoder methods strike a balance between computational efficiency and accuracy.


\textbf{Takeaway 5: LLM-as-Predictor methods are more effective when labeled data is abundant.} In supervised scenarios, LLM-as-Predictor methods enhance performance across most datasets. Especially, the LLaGA method achieves superior results among 5 of 10 datasets. Conversely, in semi-supervised settings, LLM-as-Predictor methods exhibit unstable performance, evidenced by low Macro-F1 scores, and imbalanced output distributions (detailed discussion in Appendix \ref{sec:llm_bias_pred}). These findings indicate that LLM-as-Predictor methods are most effective when ample supervision is available, with LLaGA being an especially strong choice. Furthermore, within the predictor methods, LLM Instruction Tuning typically falls behind the other two methods and incurs substantial time costs (Table \ref{tab:timecost} and Table \ref{tab:timecost_supervised} in the Appendix). This shows that standalone LLMs are weak predictors and incorporating graph context is essential for achieving satisfactory performance.



\subsection{Zero-shot Performance}
\textbf{Settings:} For LLM Direct Inference, we utilize both closed-source and open-source LLMs, including GPT-4o \cite{Achiam2023GPT4TR}, DeepSeek-Chat \cite{Shao2024DeepSeekV2AS}, LLaMA3.1-8B \cite{llama3modelcard}, and Mistral-7B. The prompt templates include \textbf{Direct}, \textbf{CoT}, \textbf{ToT}, and \textbf{ReAct} \cite{Yao2022ReActSR}. Additionally, we incorporate a node's neighboring information into extended prompts, referred to as ``\textbf{w. Neighbor}'', and have the LLMs first reason over neighbors to generate a summary that facilitates the subsequent classification task, referred to as ``\textbf{w. Summary}''. Prompt templates are listed in Appendix \ref{sec:zeroshot_prompt}. Besides, we assess the transferability of GFMs by evaluating \textbf{ZeroG} \cite{li2024zerog}, \textbf{LLM Instruction Tuning}, and \textbf{LLaGA}. GFMs are applied following the intra-domain manner: each model is pre-trained on a larger dataset within the same domain (e.g., arXiv from the academic domain) before being evaluated on the target dataset. Results for Accuracy and Macro-F1 are shown in Table \ref{tab:zeroshot} and Table \ref{tab:zeroshot_supple} in the Appendix, where we have the following takeaway: 

\begin{table*}[!t]
    \centering
    \caption{\textbf{Performance comparison under zero-shot setting with Accuracy (\%) and Macro-F1 (\%) reported.} \\ \small{Numbers in brackets represent the dataset's homophily ratio (\%). Results of other LLMs are shown in Table \ref{tab:zeroshot_supple} in Appendix.}}
    \vspace*{-8pt}
    \resizebox{0.92\linewidth}{!}{
    \begin{tabular}{cc|cc|cc|cc|cc|cc}
       \toprule
       \rowcolor{COLOR_MEAN} & &  \multicolumn{2}{c|}{\textbf{Cora} (82.52)} & \multicolumn{2}{c|}{\textbf{WikiCS} (68.67)} & \multicolumn{2}{c|}{\textbf{Instagram} (63.35)}  & \multicolumn{2}{c|}{\textbf{Photo} (78.50)} & \multicolumn{2}{c}{\textbf{Avg.}}  \\ 
       \rowcolor{COLOR_MEAN} \multirow{-2}{*}{\textbf{Type \& LLM}} &  \multirow{-2}{*}{\textbf{Method}} & Acc & Macro-F1 & Acc & Macro-F1 & Acc & Macro-F1 &  Acc & Macro-F1  & Acc & Macro-F1 \\  \midrule
       \multirow{6}{*}{\begin{tabular}{c}
            \textbf{LLM} \\ GPT-4o
       \end{tabular}} & Direct  & 68.08  & 69.25  & 68.59  & 63.21  & 44.53  &  42.77 & 63.99  & 61.09 & 61.30 & 59.08 \\
       & CoT  & 68.89 &	69.86 &	70.75 &	\textbf{66.23} &	\textbf{47.87} &	\textbf{47.57} &	61.61 &	60.62 & 62.28 & 61.07 \\ 
       & ToT  & 68.29 &	69.13 &	70.78 &	65.69 &	44.16 &	42.68 &	60.84 &	59.16 & 61.02 & 59.16  \\
       & ReAct & 68.21 & 69.28  & 69.45 &	66.03 & 44.49 &	43.16 &	63.63 &	60.82 & 61.44 & 59.82 \\ 
       & w. Neighbor & 70.30 & 71.44 & 69.69 & 64.51 & 42.42 & 39.79 & 69.93 & 68.55 & 63.09 & 61.07 \\ 
       & w. Summary & \textbf{71.40} &	\textbf{72.13}  &	\textbf{70.90} &	65.42 &	45.02 &	44.62 &	\textbf{72.63} & \textbf{70.84} & \textbf{64.99} & \textbf{63.25} \\ \midrule

       % \multirow{6}{*}{\begin{tabular}{c}
       %      \textbf{LLM} \\ Mistral-7B
       % \end{tabular}} & Direct & 59.65 & 58.34  &  70.13 &	67.80 &	44.29 &	42.16 &	\textbf{57.54} &	\textbf{55.50} \\
       % & CoT  & 58.02 & 57.13 & 69.00 &	66.17 &	45.48 &	44.56 &	49.56 &	51.42 \\ 
       % & ToT  & 58.78 &	57.20 & 	67.56 &	64.52 &	45.39 &	44.73 &	44.25 &	46.87 \\
       % & ReAct & 60.32 & 60.89 &71.02 &	67.31 	& 	\textbf{46.26} &	\textbf{46.09} &	52.47 	& 50.92 \\ 
       % & w. Neighbor \\ 
       % & w. Summary & \textbf{68.12} & \textbf{67.45} & 70.52 &	67.87 & 	41.94 &	38.93 &	56.01 &	56.22 \\
       % \midrule

       \multirow{6}{*}{\begin{tabular}{c}
            \textbf{LLM} \\ LLaMA-8B 
       \end{tabular}} & Direct & 62.64 & 63.02 & 56.77 & 53.04 & 37.58 & 29.70 & 41.23 & 44.26 & 49.56 & 47.50 \\ 
       & CoT & 62.04 & 62.61 & 58.88 & 56.00 & 42.00 & 39.06 & 44.22 & 47.13 & 51.78 & 51.20 \\ 
       & ToT & 34.06 & 33.30 & 40.35 & 41.15 & \textbf{45.33} & \textbf{45.27} & 31.31 & 34.00 & 37.76 & 38.43 \\ 
       & ReAct & 36.55 & 38.04 & 22.40 & 25.76 & 44.67 & 44.42 & 27.03 & 28.96  & 32.66 & 34.30 \\ 
       & w. Neighbor & 64.55 & 64.41 & 59.43 & 54.16 & 36.98 & 28.32 & 45.49 & 50.44 & 51.61 & 49.33 \\ 
       & w. Summary & \textbf{64.69} & \textbf{64.62} & \textbf{62.69} & \textbf{56.40}  & 37.59 & 30.91 & \textbf{48.11} & \textbf{52.20} & \textbf{53.27} & \textbf{51.03} \\ \midrule

       \multirow{3}{*}{\begin{tabular}{c}
            \textbf{GFM} 
       \end{tabular}} & ZeroG & \textbf{62.55} & \textbf{57.56}  & \textbf{62.71} & \textbf{57.87} & \textbf{50.71} & \textbf{50.43} & 46.27 & \textbf{51.52} & \textbf{55.56} & \textbf{54.35} \\ 
       & LLM\textsubscript{IT} & 52.58 & 51.89 & 60.83 & 53.59 & 41.58 & 26.26 & \textbf{49.23} & 44.85  & 51.06 & 44.15 \\ 
       & LLaGA & 18.82 & 8.49 & 8.20 & 8.29 & 47.93 & 47.70 & 39.18 & 4.71 & 28.53 & 17.30 \\ 
       \bottomrule
    \end{tabular}
    }
    \label{tab:zeroshot}
\end{table*}




\textbf{Takeaway 6: GFMs can outperform open-source LLMs but still fall short of strong LLMs like GPT-4o.} ZeroG outperforms LLaMA-8B in most cases, achieving up to a 6\% average improvement in accuracy. However, it still falls short of GPT-4o and DeepSeek-Chat. Among GFMs, LLaGA performs poorly because it uses a projector to align the source graph's tokens with LLM input tokens. This projector may be dataset-specific, leading to reduced performance on different datasets, as also observed in \citet{Zhu2024GraphCLIPET}. These findings highlight the need for further research to improve the generalization of GFMs to match the performance of more powerful LLMs.


\textbf{Takeaway 7: LLM direct inference can be improved by appropriately incorporating structural information.} Our results reveal that advanced prompt templates such as CoT, ToT, and ReAct, offer only minor performance improvements. Specifically, models like LLaMA exhibit limited instruction-following abilities, often producing unexpected and over-length outputs when processing complex prompts such as ReAct \cite{wu2025webwalker}. This makes parsing classification results challenging and leads to suboptimal performance. The advanced prompts are generally designed for broad reasoning tasks and lack graph- or classification-specific knowledge, thereby limiting their benefits for the node classification task. In contrast, enriched prompts that incorporate structural information, i.e., ``w. Neighbor'' and ``w. Summary'', demonstrate performance enhancements across LLMs. The performance gains are particularly evident on homophilic datasets such as Cora and Photo (3\%-10\%), where neighboring nodes are likely to share the same labels as the central node. High homophily means that information from neighboring nodes provides crucial clues about a central node's label, thereby improving classification performance. Among these enriched prompts, ``\textbf{w. Summary}'' is especially effective as it not only provides structural context but also leverages the self-reflection abilities of LLMs to further utilize structural information. 

%This finding aligns with previous studies \cite{Huang2023CanLE, Hu2023BeyondTA}.

\begin{table*}[!t]
    \centering
    \caption{\textbf{Comparison of LLM- and LM-as-Encoder with Accuracy (\%) reported under semi-supervised setting. LLM-as-Encoder outperforms LMs in heterophilic graphs.}\\ \small{The \colorbox{blue!10}{\textbf{best encoder}} within each method on a dataset is highlighted. Results in supervised settings are shown in Table \ref{tab:encoder_comp_fullysupervised} in Appendix.}}
    \vspace*{-6pt}
    \resizebox{\linewidth}{!}{
    \begin{tabular}{cc|ccccccccc}
      \toprule
     \rowcolor{COLOR_MEAN} \textbf{Method}  & \textbf{Encoder}  & \textbf{Computer} & \textbf{Cora} & \textbf{Pubmed} & \textbf{Photo} & \textbf{Books} & \textbf{Citeseer} & \textbf{WikiCS} & \textbf{Instagram} & \textbf{Reddit} \\ \midrule
     \multicolumn{2}{c}{Homophily Ratio (\%)} & 85.28 & 82.52 & 79.24 & 78.50 & 78.05 & 72.93 & \textbf{\textcolor{blue!30}{68.67}} & \textbf{\textcolor{blue!30}{63.35}} & \textbf{\textcolor{blue!30}{55.52}} \\ \midrule

      \multirow{4}{*}{MLP} & SenBERT & \cellcolor{blue!10}\textbf{69.57$_{\pm \text{0.18}}$} & 64.61$_{\pm \text{1.34}}$ & 74.67$_{\pm \text{0.63}}$ & 72.28$_{\pm \text{0.36}}$ & \cellcolor{blue!10}\textbf{81.93$_{\pm \text{0.08}}$} & 66.83$_{\pm \text{0.58}}$ & 71.48$_{\pm \text{0.33}}$ & 64.98$_{\pm \text{0.38}}$ & 57.23$_{\pm \text{0.51}}$ \\
    &  RoBERTa & 69.42$_{\pm \text{0.10}}$ & 73.84$_{\pm \text{0.55}}$ & 73.21$_{\pm \text{0.78}}$ & 72.95$_{\pm \text{0.34}}$ & 81.71$_{\pm \text{0.14}}$ & \cellcolor{blue!10}\textbf{70.59$_{\pm \text{0.31}}$} & 75.82$_{\pm \text{0.13}}$ & 66.39$_{\pm \text{0.24}}$ & 59.66$_{\pm \text{0.53}}$  \\
    &  Qwen-3B & 67.54$_{\pm \text{0.29}}$ & \cellcolor{blue!10}\textbf{74.03$_{\pm \text{0.57}}$} & 75.30$_{\pm \text{0.72}}$ & 72.72$_{\pm \text{0.23}}$ & 81.60$_{\pm \text{0.53}}$ & 68.26$_{\pm \text{0.79}}$ & 78.64$_{\pm \text{0.37}}$ & 66.53$_{\pm \text{0.37}}$ & 60.49$_{\pm \text{0.17}}$  \\ 
    &  Mistral-7B & 69.37$_{\pm \text{0.28}}$ & 73.90$_{\pm \text{0.59}}$ & \cellcolor{blue!10}\textbf{75.70$_{\pm \text{1.00}}$} & \cellcolor{blue!10} \textbf{74.16$_{\pm \text{0.24}}$} & 81.91$_{\pm \text{0.25}}$ & 69.66$_{\pm \text{0.38}}$ & \cellcolor{blue!10}\textbf{79.56$_{\pm \text{0.41}}$} & \cellcolor{blue!10}\textbf{66.68$_{\pm \text{0.24}}$} & \cellcolor{blue!10}\textbf{61.91$_{\pm \text{0.21}}$}  \\ \midrule
    
     \multirow{4}{*}{GCN} & SenBERT & \cellcolor{blue!10}\textbf{88.92$_{\pm \text{0.19}}$} & 81.76$_{\pm \text{0.75}}$ & 78.24$_{\pm \text{0.66}}$ & 85.18$_{\pm \text{0.16}}$ & \cellcolor{blue!10} \textbf{83.47$_{\pm \text{0.20}}$} & 70.97$_{\pm \text{0.81}}$ & 80.41$_{\pm \text{0.18}}$ & 65.78$_{\pm \text{0.14}}$ & 64.97$_{\pm \text{0.82}}$ \\ 
    &  RoBERTa & 88.90$_{\pm \text{0.14}}$ & \cellcolor{blue!10}\textbf{84.56$_{\pm \text{0.41}}$} & 78.08$_{\pm \text{0.52}}$ & \cellcolor{blue!10} \textbf{85.19$_{\pm \text{0.17}}$} & 83.22$_{\pm \text{0.29}}$ & \cellcolor{blue!10}\textbf{73.52$_{\pm \text{0.58}}$} & 80.97$_{\pm \text{0.22}}$ & 66.64$_{\pm \text{0.21}}$ & 65.69$_{\pm \text{1.01}}$ \\ 
    &  Qwen-3B & 87.55$_{\pm \text{0.14}}$ & 83.62$_{\pm \text{0.41}}$ & 78.50$_{\pm \text{0.80}}$ & 84.26$_{\pm \text{0.33}}$ & 82.83$_{\pm \text{0.24}}$ & 71.50$_{\pm \text{0.92}}$ & \cellcolor{blue!10} \textbf{81.02$_{\pm \text{0.33}}$} & 66.69$_{\pm \text{0.59}}$ & \cellcolor{blue!10} \textbf{69.40$_{\pm \text{0.56}}$} \\ 
    &  Mistral-7B & 88.22$_{\pm \text{0.16}}$ & 83.33$_{\pm \text{0.75}}$ & \cellcolor{blue!10} \textbf{78.71$_{\pm \text{0.45}}$} & 84.84$_{\pm \text{0.47}}$ & 83.03$_{\pm \text{0.34}}$ & 71.39$_{\pm \text{0.90}}$ & 80.94$_{\pm \text{0.16}}$ & \cellcolor{blue!10}\textbf{67.49$_{\pm \text{0.43}}$} & 68.65$_{\pm \text{0.75}}$ \\ \midrule

     \multirow{4}{*}{SAGE} & SenBERT & \cellcolor{blue!10}\textbf{89.08$_{\pm \text{0.06}}$} & 80.45$_{\pm \text{0.79}}$ & 77.29$_{\pm \text{0.45}}$ & 85.54$_{\pm \text{0.16}}$ & \cellcolor{blue!10} \textbf{83.93$_{\pm \text{0.17}}$} & 69.42$_{\pm \text{1.42}}$ & 80.02$_{\pm \text{0.24}}$ & 65.34$_{\pm \text{0.44}}$ & 61.65$_{\pm \text{0.17}}$ \\ 
    & RoBERTa & 88.97$_{\pm \text{0.09}}$ & \cellcolor{blue!10} \textbf{84.06$_{\pm \text{0.52}}$} & 75.82$_{\pm \text{0.59}}$ & \cellcolor{blue!10} \textbf{85.57$_{\pm \text{0.17}}$} & 83.74$_{\pm \text{0.22}}$ & \cellcolor{blue!10}\textbf{72.58$_{\pm \text{0.45}}$} & 80.77$_{\pm \text{0.29}}$ & 66.53$_{\pm \text{0.50}}$ & 63.65$_{\pm \text{0.32}}$ \\ 
    & Qwen-3B & 86.24$_{\pm \text{0.32}}$ & 83.31$_{\pm \text{0.63}}$ & 76.76$_{\pm \text{0.35}}$ & 84.28$_{\pm \text{0.42}}$ & 82.84$_{\pm \text{0.31}}$ & 71.11$_{\pm \text{0.98}}$ & 80.85$_{\pm \text{0.22}}$ & 66.73$_{\pm \text{0.37}}$ & 63.82$_{\pm \text{0.38}}$ \\ 
    & Mistral-7B & 88.48$_{\pm \text{0.20}}$ & 82.73$_{\pm \text{0.99}}$ & \cellcolor{blue!10} \textbf{77.64$_{\pm \text{1.73}}$} & 85.50$_{\pm \text{0.20}}$ & 83.32$_{\pm \text{0.16}}$ & 71.42$_{\pm \text{0.47}}$ & \cellcolor{blue!10} \textbf{81.47$_{\pm \text{0.32}}$} & \cellcolor{blue!10}\textbf{67.44$_{\pm \text{0.06}}$} &  \cellcolor{blue!10} \textbf{65.02$_{\pm \text{0.13}}$} \\ 
    \bottomrule
    \end{tabular}
    }
    \label{tab:encoder_comp}
    % \vspace*{-5pt}
\end{table*}


% \begin{table*}[!t]
%     \centering
%     \caption{\textbf{Comparison of LLM- and LM-as-Encoder with Accuracy reported under Semi-supervised Setting.} The \colorbox{brown!10}{\textbf{best encoder}} within each method on a dataset is highlighted. Results under fully-supervised settings are shown in Table \ref{tab:encoder_comp_fullysupervised} in Appendix.}
%     \resizebox{\linewidth}{!}{

%     \begin{tabular}{cc|ccccccccc}
%       \toprule
%      \rowcolor{COLOR_MEAN} & & \multicolumn{3}{c}{\textbf{Academic}} & \textbf{Web Link} & \multicolumn{2}{c}{\textbf{Social}} & \multicolumn{3}{c}{\textbf{E-Commerce}} \\
%      \rowcolor{COLOR_MEAN} \multirow{-2}{*}{\textbf{Method}}  & \multirow{-2}{*}{\textbf{Encoder}}  & Cora & Citeseer & Pubmed  & WikiCS & Instagram & Reddit & Books-History & Ele-Photo & Ele-Computers \\ \midrule
%      \multicolumn{2}{c}{Homophily Ratio (\%)} & 82.52 & 72.93 & 79.24 & \textbf{\textcolor{brown}{68.67}} & \textbf{\textcolor{brown}{63.53}} & \textbf{\textcolor{brown}{55.22}} & 78.05 & 78.50  & 85.28 \\ \midrule

%       \multirow{4}{*}{MLP} & SenBERT & 64.61$_{\pm \text{1.34}}$ & 66.83$_{\pm \text{0.58}}$ & 74.67$_{\pm \text{0.63}}$ & 71.48$_{\pm \text{0.33}}$ & 64.98$_{\pm \text{0.38}}$ & 57.23$_{\pm \text{0.51}}$ & \cellcolor{blue!10}\textbf{81.93$_{\pm \text{0.08}}$} & 72.28$_{\pm \text{0.36}}$ & \cellcolor{blue!10}\textbf{69.57$_{\pm \text{0.18}}$} \\
%     &  RoBERTa & 73.84$_{\pm \text{0.55}}$ & \cellcolor{blue!10}\textbf{70.59$_{\pm \text{0.31}}$} & 73.21$_{\pm \text{0.78}}$ & 75.82$_{\pm \text{0.13}}$ & 66.39$_{\pm \text{0.24}}$ & 59.66$_{\pm \text{0.53}}$ & 81.71$_{\pm \text{0.14}}$ & 72.95$_{\pm \text{0.34}}$ & 69.42$_{\pm \text{0.10}}$  \\
%     &  Qwen-3B & \cellcolor{blue!10}\textbf{74.03$_{\pm \text{0.57}}$} & 68.26$_{\pm \text{0.79}}$ & 75.30$_{\pm \text{0.72}}$ & 78.64$_{\pm \text{0.37}}$ & 66.53$_{\pm \text{0.37}}$ & 60.49$_{\pm \text{0.17}}$ & 81.60$_{\pm \text{0.53}}$ & 72.72$_{\pm \text{0.23}}$ & 67.54$_{\pm \text{0.29}}$  \\ 
%     &  Mistral-7B & 73.90$_{\pm \text{0.59}}$ & 69.66$_{\pm \text{0.38}}$ & \cellcolor{blue!10}\textbf{75.70$_{\pm \text{1.00}}$} & \cellcolor{blue!10}\textbf{79.56$_{\pm \text{0.41}}$} & \cellcolor{blue!10}\textbf{66.68$_{\pm \text{0.24}}$} & \cellcolor{blue!10}\textbf{61.91$_{\pm \text{0.21}}$} & 81.91$_{\pm \text{0.25}}$ & \cellcolor{blue!10}\textbf{74.16$_{\pm \text{0.24}}$} & 69.37$_{\pm \text{0.28}}$ \\ \midrule
    
%      \multirow{4}{*}{GCN} & SenBERT & 81.76$_{\pm \text{0.75}}$ & 70.97$_{\pm \text{0.81}}$ & 78.24$_{\pm \text{0.66}}$ & 80.41$_{\pm \text{0.18}}$ & 65.78$_{\pm \text{0.14}}$ & 64.97$_{\pm \text{0.82}}$ & \cellcolor{blue!10}\textbf{83.47$_{\pm \text{0.20}}$} & 85.18$_{\pm \text{0.16}}$ & \cellcolor{blue!10}\textbf{88.92$_{\pm \text{0.19}}$}  \\ 
%     &  RoBERTa & \cellcolor{blue!10}\textbf{84.56$_{\pm \text{0.41}}$} & \cellcolor{blue!10}\textbf{73.52$_{\pm \text{0.58}}$} & 78.08$_{\pm \text{0.52}}$ & 80.97$_{\pm \text{0.22}}$ & 66.64$_{\pm \text{0.21}}$ & 65.69$_{\pm \text{1.01}}$ & 83.22$_{\pm \text{0.29}}$ & \cellcolor{blue!10}\textbf{85.19$_{\pm \text{0.17}}$} & 88.90$_{\pm \text{0.14}}$\\ 
%     &  Qwen-3B & 83.62$_{\pm \text{0.41}}$ & 71.50$_{\pm \text{0.92}}$ & 78.50$_{\pm \text{0.80}}$ & \cellcolor{blue!10}\textbf{81.02$_{\pm \text{0.33}}$} & 66.69$_{\pm \text{0.59}}$ & \cellcolor{blue!10}\textbf{69.40$_{\pm \text{0.56}}$} & 82.83$_{\pm \text{0.24}}$ & 84.26$_{\pm \text{0.33}}$ & 87.55$_{\pm \text{0.14}}$ \\ 
%     &  Mistral-7B & 83.33$_{\pm \text{0.75}}$ & 71.39$_{\pm \text{0.90}}$ & \cellcolor{blue!10}\textbf{78.71$_{\pm \text{0.45}}$} & 80.94$_{\pm \text{0.16}}$ & \cellcolor{blue!10}\textbf{67.49$_{\pm \text{0.43}}$} & 68.65$_{\pm \text{0.75}}$ & 83.03$_{\pm \text{0.34}}$ & 84.84$_{\pm \text{0.47}}$ & 88.22$_{\pm \text{0.16}}$ \\ \midrule

%      \multirow{4}{*}{SAGE} & SenBERT & 80.45$_{\pm \text{0.79}}$ & 69.42$_{\pm \text{1.42}}$ & 77.29$_{\pm \text{0.45}}$ & 80.02$_{\pm \text{0.24}}$ & 65.34$_{\pm \text{0.44}}$ & 61.65$_{\pm \text{0.17}}$ & \cellcolor{blue!10}\textbf{83.93$_{\pm \text{0.17}}$} & 85.54$_{\pm \text{0.16}}$ & \cellcolor{blue!10}\textbf{89.08$_{\pm \text{0.06}}$}\\ 
%     & RoBERTa & \cellcolor{blue!10}\textbf{84.06$_{\pm \text{0.52}}$} & \cellcolor{blue!10}\textbf{72.58$_{\pm \text{0.45}}$} & 75.82$_{\pm \text{0.59}}$ & 80.77$_{\pm \text{0.29}}$ & 66.53$_{\pm \text{0.50}}$ & 63.65$_{\pm \text{0.32}}$ & 83.74$_{\pm \text{0.22}}$ & \cellcolor{blue!10}\textbf{85.57$_{\pm \text{0.17}}$} & 88.97$_{\pm \text{0.09}}$ \\ 
%     & Qwen-3B & 83.31$_{\pm \text{0.63}}$ & 71.11$_{\pm \text{0.98}}$ & 76.76$_{\pm \text{0.35}}$ & 80.85$_{\pm \text{0.22}}$ & 66.73$_{\pm \text{0.37}}$ & 63.82$_{\pm \text{0.38}}$ & 82.84$_{\pm \text{0.31}}$ & 84.28$_{\pm \text{0.42}}$ & 86.24$_{\pm \text{0.32}}$ \\ 
%     & Mistral-7B & 82.73$_{\pm \text{0.99}}$ & 71.42$_{\pm \text{0.47}}$ & \cellcolor{blue!10}\textbf{77.64$_{\pm \text{1.73}}$} & \cellcolor{blue!10}\textbf{81.47$_{\pm \text{0.32}}$} & \cellcolor{blue!10}\textbf{67.44$_{\pm \text{0.06}}$} & \cellcolor{blue!10}\textbf{65.02$_{\pm \text{0.13}}$} & 83.32$_{\pm \text{0.16}}$ & 85.50$_{\pm \text{0.20}}$ & 88.48$_{\pm \text{0.20}}$ \\ \midrule

%     \multirow{4}{*}{ENGINE} & SenBERT & 82.17$_{\pm \text{0.66}}$ & 71.23$_{\pm \text{0.84}}$ & 76.24$_{\pm \text{1.17}}$ & 78.76$_{\pm \text{0.95}}$ & 65.45$_{\pm \text{0.66}}$ & 66.97$_{\pm \text{0.71}}$ & 82.81$_{\pm \text{0.22}}$ & 84.08$_{\pm \text{0.24}}$ & \cellcolor{blue!10}\textbf{86.45$_{\pm \text{0.52}}$}  \\ 
%     & RoBERTa & 84.04$_{\pm \text{0.44}}$ & \cellcolor{blue!10}\textbf{73.14$_{\pm \text{0.81}}$} & 74.04$_{\pm \text{0.54}}$ & 79.42$_{\pm \text{0.34}}$ & 66.42$_{\pm \text{0.62}}$ & 68.27$_{\pm \text{0.47}}$ & 82.78$_{\pm \text{0.25}}$ & 84.25$_{\pm \text{0.29}}$ & 86.16$_{\pm \text{0.19}}$   \\ 
%     & Qwen-3B & 82.35$_{\pm \text{0.95}}$ & 70.58$_{\pm \text{0.30}}$ & 76.27$_{\pm \text{0.88}}$ & 79.81$_{\pm \text{0.31}}$ & 66.95$_{\pm \text{0.38}}$ & 69.62$_{\pm \text{0.21}}$ & \cellcolor{blue!10}\textbf{83.00$_{\pm \text{0.26}}$} & 82.54$_{\pm \text{0.87}}$ & 84.00$_{\pm \text{0.75}}$ \\ 
%     & Mistral-7B & \cellcolor{blue!10}\textbf{84.22$_{\pm \text{0.46}}$} & 72.14$_{\pm \text{0.74}}$ & \cellcolor{blue!10}\textbf{77.84$_{\pm \text{0.27}}$} & \cellcolor{blue!10}\textbf{80.94$_{\pm \text{0.19}}$} & \cellcolor{blue!10}\textbf{67.14$_{\pm \text{0.46}}$} & \cellcolor{blue!10}\textbf{69.67$_{\pm \text{0.16}}$} & 82.89$_{\pm \text{0.14}}$ & \cellcolor{blue!10}\textbf{84.33$_{\pm \text{0.57}}$} & 86.42$_{\pm \text{0.23}}$ \\
%     \bottomrule
%     \end{tabular}
%     }
%     \label{tab:encoder_comp}
%     % \vspace*{-5pt}
% \end{table*}


\subsection{Computational Cost Analysis}
 
We evaluate the training and inference times of various methods in both semi-supervised and supervised settings. Detailed training times are provided in Tables \ref{tab:timecost} and \ref{tab:timecost_supervised} in the Appendix, while inference times are presented in Table \ref{tab:inference_cost}. All measurements were conducted on a single NVIDIA H100-80G GPU to ensure consistency.

Based on the results, we can conclude that classic methods are highly efficient, with GNNs typically converging within seconds (e.g., 5.2 seconds for GCN\textsubscript{ShallowEmb} on Pubmed) and LMs fine-tuning completed within minutes. In contrast, LLM-as-Reasoner approaches are the most time-consuming (e.g., 5.9 hours for TAPE on Pubmed) because they require generating reasoning text for each node and subsequently processing this augmented text through both an LM and a GNN. This three-stage computational process significantly extends the overall computation time. LLM-as-Encoder methods are the most efficient among LLM-based approaches (e.g., 13.4 minutes for GCN$_{\text{LLMEmb}}$ on Pubmed), utilizing LLMs solely for feature encoding, which allows GNN training to remain efficient and complete within minutes. Although LLM-as-Predictor methods are more efficient than Reasoner approaches, they still require hours for effective model training. Among predictor methods, LLaGA is the most efficient (e.g., 25.6 minutes on Pubmed) as it encodes both the node's textual and structural information into embeddings instead of processing raw text.

During inference, a significant efficiency gap remains between LLM-based and classic methods. Classic methods can complete the entire inference process for thousands of cases within milliseconds, making them suitable for industrial deployments that demand real-time responses. In contrast, LLM-based methods are limited to processing one case within the same timeframe, highlighting the urgent need to improve their efficiency.
\section{Fine-grained Analysis Within Each Category}
In this section, we present empirical results within each category. For LLM-as-Encoder, we explore the conditions under which LLMs outperform traditional LMs. Additionally, we examine how key components (e.g., model type and size) influence the effectiveness of LLM-as-Reasoner and LLM-as-Predictor. 

\subsection{LLM-as-Encoder: Compared with LMs}\label{exp:encoder_comp}
\begin{table*}[!t]
    \centering
     \caption{\textbf{Performance ($\%$) of TAPE with different LLM backbones under semi-supervised setting}.}
     \vspace*{-8pt}
    \resizebox{\linewidth}{!}{
    \begin{tabular}{cc|cccccccccc}
       \toprule
      \rowcolor{COLOR_MEAN} \textbf{Metrics} & \textbf{LLM}  &  \textbf{Cora} & \textbf{Citeseer} & \textbf{Pubmed} & \textbf{WikiCS} & \textbf{Instagram} & \textbf{Reddit} & \textbf{Books} & \textbf{Photo} & \textbf{Computer} & \textbf{Avg.} \\ \midrule
      \multirow{2}{*}{\textbf{Acc}} & Mistral  &  84.04$_{\pm \text{0.24}}$ & 71.87$_{\pm \text{0.35}}$ & 78.61 $_{\pm \text{1.23}}$ & \textbf{81.94$_{\pm \text{0.16}}$} & 66.07$_{\pm \text{0.10}}$ & \textbf{62.43$_{\pm \text{0.47}}$} & 84.92$_{\pm \text{0.26}}$ & 86.46$_{\pm \text{0.12}}$ & 89.52$_{\pm \text{0.04}}$ & 78.43 \\  
      & GPT-4o & \textbf{84.30$_{\pm \text{0.36}}$} & \textbf{73.75$_{\pm \text{0.67}}$} & \textbf{82.70$_{\pm \text{1.78}}$} & 81.93$_{\pm \text{0.33}}$ & \textbf{66.25$_{\pm \text{0.38}}$} & 62.22$_{\pm \text{1.24}}$ & \textbf{85.08$_{\pm \text{0.17}}$} & \textbf{86.65$_{\pm \text{0.17}}$} & \textbf{89.62$_{\pm \text{0.13}}$} & \textbf{79.17} \\  \midrule

     \multirow{2}{*}{\textbf{F1}} & Mistral & 81.89$_{\pm \text{0.31}}$ & 66.80$_{\pm \text{0.33}}$ & 78.46$_{\pm \text{1.13}}$ & 80.03$_{\pm \text{0.23}}$ & 50.01$_{\pm \text{1.60}}$ & \textbf{61.23$_{\pm \text{0.69}}$} & 47.12$_{\pm \text{3.26}}$ & 82.31$_{\pm \text{0.19}}$ & \textbf{84.90$_{\pm \text{1.14}}$} & 70.31 \\ 
      & GPT-4o & \textbf{82.62$_{\pm \text{0.60}}$} & \textbf{67.41$_{\pm \text{0.82}}$} & \textbf{82.45$_{\pm \text{1.65}}$} & \textbf{80.27$_{\pm \text{0.34}}$} & \textbf{51.16$_{\pm \text{3.54}}$} & 61.11$_{\pm \text{1.52}}$ & \textbf{47.51$_{\pm \text{2.92}}$} & \textbf{82.54$_{\pm \text{0.18}}$} & 84.28$_{\pm \text{2.98}}$ & \textbf{71.04} \\ 
      
       \bottomrule
    \end{tabular}
    }
    \label{tab:tape_llm_semi}
\end{table*}

\begin{table*}[!t]
    \centering
     \caption{\textbf{Performance ($\%$) of TAPE with different LLM backbones under supervised setting}.}
    \vspace*{-8pt}
    \resizebox{\linewidth}{!}{
    \begin{tabular}{cc|ccccccccccc}
       \toprule
      \rowcolor{COLOR_MEAN} \textbf{Metrics} & \textbf{LLM}  &  \textbf{Cora} & \textbf{Citeseer} & \textbf{Pubmed} & \textbf{arXiv} & \textbf{WikiCS} & \textbf{Instagram} & \textbf{Reddit} & \textbf{Books} & \textbf{Photo} & \textbf{Computer} & \textbf{Avg.} \\ \midrule

       \multirow{2}{*}{\textbf{Acc}} & Mistral  & 88.05$_{\pm \text{1.76}}$ & \textbf{76.45$_{\pm \text{1.60}}$} & 93.00$_{\pm \text{0.13}}$ & 74.96$_{\pm \text{0.14}}$ &
        \textbf{87.11$_{\pm \text{0.66}}$} & \textbf{68.11$_{\pm \text{0.54}}$} & 66.22$_{\pm \text{0.83}}$ & 85.95$_{\pm \text{0.59}}$ & \textbf{87.72$_{\pm \text{0.28}}$} & 90.46$_{\pm \text{0.18}}$ & 81.80 \\  

      & GPT-4o & \textbf{88.24$_{\pm \text{1.23}}$} & 76.41$_{\pm \text{1.38}}$ & \textbf{94.12$_{\pm \text{0.03}}$} & \textbf{75.08$_{\pm \text{0.08}}$} & 87.10$_{\pm \text{0.62}}$ & 67.99$_{\pm \text{0.51}}$ & \textbf{66.33$_{\pm \text{0.89}}$} & \textbf{86.19$_{\pm \text{0.60}}$} & 87.65$_{\pm \text{0.47}}$ & \textbf{90.56$_{\pm \text{0.21}}$} & \textbf{81.97} \\   \midrule
    
      \multirow{2}{*}{\textbf{F1}} & Mistral & 87.21$_{\pm \text{1.60}}$ & \textbf{73.33$_{\pm \text{1.57}}$} & 92.39$_{\pm \text{0.02}}$ & \textbf{57.79$_{\pm \text{0.45}}$} & \textbf{86.03$_{\pm \text{1.14}}$} & \textbf{58.31$_{\pm \text{1.15}}$} & 65.91$_{\pm \text{0.71}}$ & 54.07$_{\pm \text{2.01}}$ & \textbf{83.41$_{\pm \text{0.42}}$} & 86.78$_{\pm \text{0.53}}$ & 74.52 \\ 
      & GPT-4o & \textbf{87.34$_{\pm \text{1.06}}$} & 73.17$_{\pm \text{2.00}}$ & \textbf{93.58$_{\pm \text{0.09}}$} & 57.69$_{\pm \text{0.23}}$ & 85.93$_{\pm \text{1.05}}$ & 57.49$_{\pm \text{1.93}}$ & \textbf{66.09$_{\pm \text{0.80}}$} & \textbf{54.32$_{\pm \text{3.30}}$} & 83.40$_{\pm \text{0.41}}$ & \textbf{86.91$_{\pm \text{0.55}}$} & \textbf{74.59} \\ 
      
       \bottomrule
    \end{tabular}
    }
    \label{tab:tape_llm_fully}
\end{table*}




\textbf{Motivation and Settings: }Both LLMs and small-scale LMs can encode nodes' associated texts. This raises the question: When do LLMs surpass LMs as encoders? To address this, we evaluate various methods using node features derived from LLMs and LMs, observing the resulting performance differences. For LMs, we select SenBERT-66M \cite{reimers-2019-sentence-bert} and RoBERTa-355M \cite{Liu2019roberta}. For LLMs, we choose Qwen2.5-3B \cite{Yang2024Qwen2TR} and Mistral-7B \cite{Jiang2023Mistral7B}. The considered methods include: (1) \textbf{MLP}, which solely utilizes node features as input to predict labels without incorporating any graph information, (2) \textbf{GCN}, and (3) \textbf{GraphSAGE}. For each method, we input node features initialized from various LM or LLM backbones while keeping all other components consistent.  From the results shown in Table \ref{tab:encoder_comp} and Table \ref{tab:encoder_comp_fullysupervised} in the Appendix, we can observe that: 


\textbf{Takeaway 8: LLM-as-Encoder significantly outperforms LMs in heterophilic graphs.} In heterophilic datasets such as Reddit, LLM-based encoders achieve 2\% - 8\% higher accuracy than their LM counterparts. This performance gap is most evident with the MLP method (4\%–8\%). The performance gain is less obvious in homophilic settings. We can also leverage the mutual information \eqref{eq:mutual_info} to give theoretical insights. In homophilic graphs, edges often connect nodes with the same labels, whereas this does not hold in heterophilic graphs. Therefore, in homophilic graphs, the first term in \eqref{eq:mutual_info} dominates, and the room for a better encoder, e.g., LLM, to improve is limited. 


\begin{table*}[!t]
    \centering
    \caption{\textbf{Accuracy ($\%$) of LLaGA to different LLM backbones under supervised settings}.\\ \small{The best LLM backbone within \colorbox{red!10}{\textbf{each series}} and \colorbox{yellow!20}{\textbf{at similar scales}} is highlighted. Semi-supervised performance is shown in Table \ref{tab:llaga_llm}.}}
    \vspace*{-8pt}
    \resizebox{\linewidth}{!}{
    \begin{tabular}{cc|ccccccccccc}
      \toprule
     \rowcolor{COLOR_MEAN}  & \textbf{LLM} & \textbf{Cora} & \textbf{Citeseer} & \textbf{Pubmed} & \textbf{arXiv} & \textbf{WikiCS} & \textbf{Instagram} & \textbf{Reddit} & \textbf{Books} & \textbf{Photo} & \textbf{Computer} & \textbf{Avg.} \\ \midrule
      \multirow{4}{*}{\rotatebox[origin=c]{90}{\small \begin{tabular}{c}
           \textbf{Same} \\ \textbf{series}
      \end{tabular}}}  & Qwen-3B & 84.91$_{\pm \text{2.19}}$ & 74.83$_{\pm \text{2.46}}$ & 88.61$_{\pm \text{1.24}}$ & 71.82$_{\pm \text{1.37}}$ & 82.23$_{\pm \text{3.14}}$ & 62.49$_{\pm \text{0.98}}$ & 67.96$_{\pm \text{0.90}}$ & 83.56$_{\pm \text{1.86}}$ & \cellcolor{red!10} \textbf{85.20$_{\pm \text{1.63}}$} & \cellcolor{red!10} \textbf{89.37$_{\pm \text{0.29}}$} & 79.10 \\ 
     & Qwen-7B & 85.33$_{\pm \text{1.50}}$ & 70.75$_{\pm \text{5.18}}$ & \cellcolor{red!10} \textbf{90.53$_{\pm \text{0.49}}$} & 71.60$_{\pm \text{1.59}}$ & 82.57$_{\pm \text{1.67}}$ & 63.86$_{\pm \text{2.76}}$ & \cellcolor{red!10} \textbf{68.62$_{\pm \text{0.53}}$} & \cellcolor{red!10} \textbf{84.23$_{\pm \text{0.51}}$} & 83.55$_{\pm \text{1.35}}$ & 87.21$_{\pm \text{1.88}}$  & 78.82 \\ 
    & Qwen-14B & \cellcolor{red!10} \textbf{87.25$_{\pm \text{1.63}}$} & \cellcolor{red!10} \textbf{75.49$_{\pm \text{2.03}}$} & 89.93$_{\pm \text{0.27}}$ & \cellcolor{red!10} \textbf{73.15$_{\pm \text{0.74}}$} & 82.26$_{\pm \text{1.51}}$ & 63.88$_{\pm \text{2.49}}$ & 67.60$_{\pm \text{1.77}}$ & 83.94$_{\pm \text{0.41}}$ & 84.83$_{\pm \text{0.77}}$ & 87.06$_{\pm \text{0.80}}$ & 79.54 \\  
    & Qwen-32B & 85.93$_{\pm \text{0.99}}$ & 75.39$_{\pm \text{1.90}}$ & 89.97$_{\pm \text{0.26}}$ & 72.84$_{\pm \text{0.67}}$ & \cellcolor{red!10} \textbf{83.49$_{\pm \text{0.91}}$} & \cellcolor{red!10} \textbf{64.33$_{\pm \text{1.69}}$} & 68.47$_{\pm \text{0.09}}$ & 84.18$_{\pm \text{0.29}}$ & 84.77$_{\pm \text{0.23}}$ & 88.49$_{\pm \text{0.49}}$ & \cellcolor{red!10}\textbf{79.79} \\ 
    \midrule
    
    \multirow{3}{*}{\rotatebox[origin=c]{90}{\small \begin{tabular}{c}
           \textbf{Similar} \\ \textbf{scales}
      \end{tabular}}}  & Mistral-7B & \cellcolor{yellow!10}\textbf{87.55$_{\pm \text{1.15}}$} &\cellcolor{yellow!10} \textbf{76.73$_{\pm \text{1.70}}$} & 90.28$_{\pm \text{0.91}}$ & \cellcolor{yellow!10}\textbf{74.49$_{\pm \text{0.23}}$} & \cellcolor{yellow!10}\textbf{84.03$_{\pm \text{1.10}}$} & \cellcolor{yellow!10}\textbf{69.16$_{\pm \text{0.72}}$} & \cellcolor{yellow!10}\textbf{71.06$_{\pm \text{0.38}}$} & \cellcolor{yellow!10}\textbf{85.56$_{\pm \text{0.30}}$} &  \cellcolor{yellow!10}\textbf{87.62$_{\pm \text{0.30}}$} &\cellcolor{yellow!10} \textbf{90.41$_{\pm \text{0.12}}$} & \cellcolor{yellow!10} \textbf{81.69} \\ 
     & Qwen-7B & 85.33$_{\pm \text{1.50}}$ & 70.75$_{\pm \text{5.18}}$ & \cellcolor{yellow!10}\textbf{90.53$_{\pm \text{0.49}}$} & 70.47$_{\pm \text{1.12}}$ & 82.57$_{\pm \text{1.67}}$ & 63.86$_{\pm \text{2.76}}$ & 68.62$_{\pm \text{0.53}}$ & 84.23$_{\pm \text{0.51}}$ & 83.55$_{\pm \text{1.35}}$ & 87.21$_{\pm \text{1.88}}$  & 78.71\\ 
    & LLaMA-8B & 85.77$_{\pm \text{1.34}}$ & 74.84$_{\pm \text{1.09}}$ & 89.57$_{\pm \text{0.24}}$ & 72.72$_{\pm \text{0.26}}$  & 82.25$_{\pm \text{1.65}}$ & 61.12$_{\pm \text{0.45}}$ & 67.70$_{\pm \text{0.44}}$ & 84.05$_{\pm \text{0.26}}$ & 85.57$_{\pm \text{0.41}}$ & 89.42$_{\pm \text{0.12}}$ & 79.30 \\ 
      \bottomrule
    \end{tabular}
    }
    \label{tab:llaga_llm_s}
   % \vspace*{-13pt}
\end{table*}

\begin{table*}[!t]
    \centering
    \caption{\textbf{Macro-F1($\%$) of LLaGA to different LLM backbones under supervised settings}.}
    \vspace*{-8pt}
    \resizebox{\linewidth}{!}{
    \begin{tabular}{cc|ccccccccccc}
      \toprule
     \rowcolor{COLOR_MEAN}  & \textbf{LLM} & \textbf{Cora} & \textbf{Citeseer} & \textbf{Pubmed} & \textbf{arXiv} & \textbf{WikiCS} & \textbf{Instagram} & \textbf{Reddit} & \textbf{Books} & \textbf{Photo} & \textbf{Computer} & \textbf{Avg.} \\ \midrule
     %  & \# Train Samples & 1,624 & 1,911 & 11,830 & 90,941 & 7,020 & 6,803 & 20,060 & 24,930 & 29,017 & 52,337 \\ \midrule
      \multirow{4}{*}{\rotatebox[origin=c]{90}{\begin{tabular}{c}
           \textbf{Same-} \\ \textbf{series}
      \end{tabular}}}  & Qwen-3B & 77.92$_{\pm \text{6.14}}$ & 66.52$_{\pm \text{5.69}}$ & 78.88$_{\pm \text{10.43}}$ & 51.30$_{\pm \text{0.83}}$ & 78.81$_{\pm \text{7.68}}$ & 50.93$_{\pm \text{7.72}}$ & 65.77$_{\pm \text{1.38}}$ & \cellcolor{red!10} \textbf{49.87$_{\pm \text{1.52}}$} & 77.51$_{\pm \text{3.24}}$ & \cellcolor{red!10} \textbf{80.77$_{\pm \text{3.27}}$} & 67.83 \\
    & Qwen-7B & 82.50$_{\pm \text{4.12}}$ & 64.03$_{\pm \text{4.86}}$ & \cellcolor{red!10}\textbf{90.29$_{\pm \text{0.52}}$} & 51.97$_{\pm \text{0.83}}$ & 77.35$_{\pm \text{4.26}}$ & 56.50$_{\pm \text{1.15}}$ & \cellcolor{red!10}\textbf{68.55$_{\pm \text{0.60}}$} & 46.21$_{\pm \text{1.78}}$ & 75.76$_{\pm \text{5.34}}$ & 78.86$_{\pm \text{5.90}}$ & 69.20 \\  
   & Qwen-14B &\cellcolor{red!10} \textbf{85.64$_{\pm \text{1.89}}$} & \cellcolor{red!10}\textbf{69.92$_{\pm \text{3.95}}$} & 89.69$_{\pm \text{0.39}}$ & \cellcolor{red!10}\textbf{53.32$_{\pm \text{0.38}}$} & \cellcolor{red!10}\textbf{79.13$_{\pm \text{1.78}}$} & \cellcolor{red!10}\textbf{57.58$_{\pm \text{0.83}}$} & 67.10$_{\pm \text{2.18}}$ & 44.26$_{\pm \text{2.27}}$ & 76.09$_{\pm \text{2.71}}$ & 80.17$_{\pm \text{4.74}}$ & \cellcolor{red!10} \textbf{70.29} \\   
    & Qwen-32B & 82.85$_{\pm \text{4.10}}$ & 68.11$_{\pm \text{3.69}}$ & 89.57$_{\pm \text{0.37}}$ & 52.52$_{\pm \text{0.65}}$ & 77.31$_{\pm \text{3.89}}$ & 57.28$_{\pm \text{2.21}}$ & 68.22$_{\pm \text{0.02}}$ & 48.25$_{\pm \text{1.29}}$ & \cellcolor{red!10}\textbf{79.51$_{\pm \text{0.87}}$} & 77.15$_{\pm \text{7.53}}$ & 70.08 \\ 
    \midrule
    
    \multirow{3}{*}{\rotatebox[origin=c]{90}{\small \begin{tabular}{c}
           \textbf{Similar} \\ \textbf{scales}
      \end{tabular}}}  & Mistral-7B & \cellcolor{yellow!10}\textbf{84.97$_{\pm \text{3.97}}$} & \cellcolor{yellow!10}\textbf{72.59$_{\pm \text{1.70}}$} & 90.00$_{\pm \text{0.80}}$ & \cellcolor{yellow!10}\textbf{58.08$_{\pm \text{0.29}}$} & \cellcolor{yellow!10}\textbf{82.37$_{\pm \text{1.73}}$} & \cellcolor{yellow!10} \textbf{57.96$_{\pm \text{2.40}}$} & 62.14$_{\pm \text{15.59}}$ & \cellcolor{yellow!10} \textbf{54.89$_{\pm \text{2.29}}$} & \cellcolor{yellow!10} \textbf{83.56$_{\pm \text{0.40}}$} & \cellcolor{yellow!10} \textbf{86.97$_{\pm \text{0.34}}$} &  \cellcolor{yellow!10}\textbf{73.35} \\ 
     & Qwen-7B & 82.50$_{\pm \text{4.12}}$ & 64.03$_{\pm \text{4.86}}$ & \cellcolor{yellow!10} \textbf{90.29$_{\pm \text{0.52}}$} & 45.74$_{\pm \text{9.78}}$ & 77.35$_{\pm \text{4.26}}$ & 56.50$_{\pm \text{1.15}}$ & \cellcolor{yellow!10} \textbf{68.55$_{\pm \text{0.60}}$} & 46.21$_{\pm \text{1.78}}$ & 75.76$_{\pm \text{5.34}}$ & 78.86$_{\pm \text{5.90}}$ & 69.09 \\  
    & LLaMA-8B & 81.40$_{\pm \text{5.46}}$ & 69.87$_{\pm \text{3.68}}$ & 89.30$_{\pm \text{0.23}}$ & 55.23$_{\pm \text{0.59}}$ & 80.14$_{\pm \text{2.09}}$ & 54.58$_{\pm \text{1.24}}$ & 67.40$_{\pm \text{0.61}}$ & 51.65$_{\pm \text{0.17}}$ & 78.87$_{\pm \text{2.38}}$ & 85.54$_{\pm \text{0.59}}$ & 71.40 \\
      \bottomrule
    \end{tabular}
    }
     \vspace*{-8pt}
    \label{tab:llaga_llm_s_f1}
\end{table*}





\subsection{LLM-as-Reasoner: Impact of LLM Reasoning Capabilities}

\textbf{Motivation and Settings: }In the LLM-as-Reasoner paradigm, as the language model should generate reasoning texts, the adopted language models should be auto-regressive and the model size should be large. To investigate how the advanced reasoning capabilities of LLMs influence overall performance, we replace the default Mistral-7B model in the TAPE method with the more powerful GPT-4o model, keeping all other components unchanged. 

\textbf{Results and Analysis: } As presented in Table \ref{tab:tape_llm_semi} (semi-supervised settings) and Table \ref{tab:tape_llm_fully} (supervised settings), the effectiveness of LLM-as-Reasoner methods positively correlates with the strength of the underlying LLMs. In semi-supervised settings, TAPE utilizing GPT-4o consistently outperforms its Mistral-7B counterpart, achieving performance gains of up to 4\% on the Pubmed dataset. However, in supervised scenarios, the performance gap between GPT-4o and Mistral-7B narrows. This reduction is attributed to the abundance of labeled data, which increases the mutual information $I(\mathcal{E}; \mathcal{Y}_l)$ in \eqref{eq:mutual_info}. Consequently, the dependency on node attributes decreases, thereby diminishing the relative advantages of more powerful LLMs. 

Based on these findings, we recommend that when abundant supervision is available, practitioners may opt for open-source LLMs instead of more powerful and costlier models for the LLM-as-Reasoner method. This practice can achieve comparable performance without incurring additional costs.

\subsection{LLM-as-Predictor: Sensitivity to LLM Backbones}

\textbf{Motivation and Settings: } For most LLM-as-Predictor methods, only open-source LLMs are compatible. Given the diverse choices and varying scales of these models, we aim to investigate the sensitivity of performance to different LLM backbones. This examination seeks to identify potential scaling laws and determine which LLMs excel at the node classification task. Therefore, we choose the best predictor method LLaGA as the baseline, include models of different sizes within the same series, i.e., Qwen2.5-series \cite{Yang2024Qwen2TR}. Additionally, we consider similar-scaled models to identify the most suitable for this task, including Qwen2.5-7B, Mistral-7B, and LLaMA3.1-8B. All experiments maintain consistency by only varying the backbone LLMs while keeping other components, training configurations, and hyperparameters unchanged.

\textbf{Results and Analysis: } \textbf{(1) Scaling within the same series: }Comparing Qwen-3B to Qwen-32B (performance shown in Tables \ref{tab:llaga_llm_s} and \ref{tab:llaga_llm_s_f1}, efficiency in Table \ref{tab:qwen_cost}, and performance trends in Figures \ref{fig:llaga_scaling} and \ref{fig:llaga_scaling_s} in the Appendix), we observe that performance generally improves with larger model sizes. However, beyond Qwen-7B and Qwen-14B, the performance gains become marginal while training and inference times increase significantly. For instance, Qwen-32B takes over 200 milliseconds per sample for inference, which is five times longer than Qwen-7B. Therefore, Qwen-7B or Qwen-14B are recommended as practical choices balancing performance and efficiency. \textbf{(2) Model selection at similar scales: }When comparing models of similar sizes (Tables \ref{tab:llaga_llm_s} and \ref{tab:llaga_llm_s_f1}, and Table \ref{tab:llaga_llm} in the Appendix), Mistral-7B outperforms other LLMs of comparable scale. Its superior performance makes Mistral-7B the recommended backbone LLM for node classification tasks.




\section{Conclusion}

This paper provides guidelines for leveraging LLMs to enhance node classification tasks across diverse real-world applications. We introduce LLMNodeBed, a codebase and testbed for systematic comparisons, featuring ten datasets, eight LLM-based algorithms, and three learning paradigms. Through extensive experiments involving 2,200 models, we uncover key insights: In supervised settings, each category offers unique advantages, but LLM-based approaches deliver marginal improvements over classic methods when ample supervision is available. In zero-shot scenarios, directing powerful LLMs to perform inference with integrated structural context yields the best performance.

Our findings offer practical guidance for practitioners applying LLMs to node classification tasks and highlight research gaps, e.g., the limited exploration of LLMs on heterophilic graphs and the scarcity of such text-rich datasets. We intend to address these gaps in future work. We hope that LLMNodeBed will inspire and serve as a valuable toolkit for further research.

\clearpage
\newpage

\section*{Impact Statement}
This paper presents work whose goal is to advance the field of Machine Learning. There are many potential societal consequences of our work, none of which we feel must be specifically highlighted here.



\bibliography{reference}
\bibliographystyle{icml2025}



%%%%%%%%%%%%%%%%%%%%%%%%%%%%%%%%%%%%%%%%%%%%%%%%%%%%%%%%%%%%%%%%%%%%%%%%%%%%%%%
%%%%%%%%%%%%%%%%%%%%%%%%%%%%%%%%%%%%%%%%%%%%%%%%%%%%%%%%%%%%%%%%%%%%%%%%%%%%%%%
% APPENDIX
%%%%%%%%%%%%%%%%%%%%%%%%%%%%%%%%%%%%%%%%%%%%%%%%%%%%%%%%%%%%%%%%%%%%%%%%%%%%%%%
%%%%%%%%%%%%%%%%%%%%%%%%%%%%%%%%%%%%%%%%%%%%%%%%%%%%%%%%%%%%%%%%%%%%%%%%%%%%%%%
\newpage
\appendix
\onecolumn

\section{RELATED WORK}
\label{sec:relatedwork}
In this section, we describe the previous works related to our proposal, which are divided into two parts. In Section~\ref{sec:relatedwork_exoplanet}, we present a review of approaches based on machine learning techniques for the detection of planetary transit signals. Section~\ref{sec:relatedwork_attention} provides an account of the approaches based on attention mechanisms applied in Astronomy.\par

\subsection{Exoplanet detection}
\label{sec:relatedwork_exoplanet}
Machine learning methods have achieved great performance for the automatic selection of exoplanet transit signals. One of the earliest applications of machine learning is a model named Autovetter \citep{MCcauliff}, which is a random forest (RF) model based on characteristics derived from Kepler pipeline statistics to classify exoplanet and false positive signals. Then, other studies emerged that also used supervised learning. \cite{mislis2016sidra} also used a RF, but unlike the work by \citet{MCcauliff}, they used simulated light curves and a box least square \citep[BLS;][]{kovacs2002box}-based periodogram to search for transiting exoplanets. \citet{thompson2015machine} proposed a k-nearest neighbors model for Kepler data to determine if a given signal has similarity to known transits. Unsupervised learning techniques were also applied, such as self-organizing maps (SOM), proposed \citet{armstrong2016transit}; which implements an architecture to segment similar light curves. In the same way, \citet{armstrong2018automatic} developed a combination of supervised and unsupervised learning, including RF and SOM models. In general, these approaches require a previous phase of feature engineering for each light curve. \par

%DL is a modern data-driven technology that automatically extracts characteristics, and that has been successful in classification problems from a variety of application domains. The architecture relies on several layers of NNs of simple interconnected units and uses layers to build increasingly complex and useful features by means of linear and non-linear transformation. This family of models is capable of generating increasingly high-level representations \citep{lecun2015deep}.

The application of DL for exoplanetary signal detection has evolved rapidly in recent years and has become very popular in planetary science.  \citet{pearson2018} and \citet{zucker2018shallow} developed CNN-based algorithms that learn from synthetic data to search for exoplanets. Perhaps one of the most successful applications of the DL models in transit detection was that of \citet{Shallue_2018}; who, in collaboration with Google, proposed a CNN named AstroNet that recognizes exoplanet signals in real data from Kepler. AstroNet uses the training set of labelled TCEs from the Autovetter planet candidate catalog of Q1–Q17 data release 24 (DR24) of the Kepler mission \citep{catanzarite2015autovetter}. AstroNet analyses the data in two views: a ``global view'', and ``local view'' \citep{Shallue_2018}. \par


% The global view shows the characteristics of the light curve over an orbital period, and a local view shows the moment at occurring the transit in detail

%different = space-based

Based on AstroNet, researchers have modified the original AstroNet model to rank candidates from different surveys, specifically for Kepler and TESS missions. \citet{ansdell2018scientific} developed a CNN trained on Kepler data, and included for the first time the information on the centroids, showing that the model improves performance considerably. Then, \citet{osborn2020rapid} and \citet{yu2019identifying} also included the centroids information, but in addition, \citet{osborn2020rapid} included information of the stellar and transit parameters. Finally, \citet{rao2021nigraha} proposed a pipeline that includes a new ``half-phase'' view of the transit signal. This half-phase view represents a transit view with a different time and phase. The purpose of this view is to recover any possible secondary eclipse (the object hiding behind the disk of the primary star).


%last pipeline applies a procedure after the prediction of the model to obtain new candidates, this process is carried out through a series of steps that include the evaluation with Discovery and Validation of Exoplanets (DAVE) \citet{kostov2019discovery} that was adapted for the TESS telescope.\par
%



\subsection{Attention mechanisms in astronomy}
\label{sec:relatedwork_attention}
Despite the remarkable success of attention mechanisms in sequential data, few papers have exploited their advantages in astronomy. In particular, there are no models based on attention mechanisms for detecting planets. Below we present a summary of the main applications of this modeling approach to astronomy, based on two points of view; performance and interpretability of the model.\par
%Attention mechanisms have not yet been explored in all sub-areas of astronomy. However, recent works show a successful application of the mechanism.
%performance

The application of attention mechanisms has shown improvements in the performance of some regression and classification tasks compared to previous approaches. One of the first implementations of the attention mechanism was to find gravitational lenses proposed by \citet{thuruthipilly2021finding}. They designed 21 self-attention-based encoder models, where each model was trained separately with 18,000 simulated images, demonstrating that the model based on the Transformer has a better performance and uses fewer trainable parameters compared to CNN. A novel application was proposed by \citet{lin2021galaxy} for the morphological classification of galaxies, who used an architecture derived from the Transformer, named Vision Transformer (VIT) \citep{dosovitskiy2020image}. \citet{lin2021galaxy} demonstrated competitive results compared to CNNs. Another application with successful results was proposed by \citet{zerveas2021transformer}; which first proposed a transformer-based framework for learning unsupervised representations of multivariate time series. Their methodology takes advantage of unlabeled data to train an encoder and extract dense vector representations of time series. Subsequently, they evaluate the model for regression and classification tasks, demonstrating better performance than other state-of-the-art supervised methods, even with data sets with limited samples.

%interpretation
Regarding the interpretability of the model, a recent contribution that analyses the attention maps was presented by \citet{bowles20212}, which explored the use of group-equivariant self-attention for radio astronomy classification. Compared to other approaches, this model analysed the attention maps of the predictions and showed that the mechanism extracts the brightest spots and jets of the radio source more clearly. This indicates that attention maps for prediction interpretation could help experts see patterns that the human eye often misses. \par

In the field of variable stars, \citet{allam2021paying} employed the mechanism for classifying multivariate time series in variable stars. And additionally, \citet{allam2021paying} showed that the activation weights are accommodated according to the variation in brightness of the star, achieving a more interpretable model. And finally, related to the TESS telescope, \citet{morvan2022don} proposed a model that removes the noise from the light curves through the distribution of attention weights. \citet{morvan2022don} showed that the use of the attention mechanism is excellent for removing noise and outliers in time series datasets compared with other approaches. In addition, the use of attention maps allowed them to show the representations learned from the model. \par

Recent attention mechanism approaches in astronomy demonstrate comparable results with earlier approaches, such as CNNs. At the same time, they offer interpretability of their results, which allows a post-prediction analysis. \par



\definecolor{titlecolor}{rgb}{0.9, 0.5, 0.1}
\definecolor{anscolor}{rgb}{0.2, 0.5, 0.8}
\definecolor{labelcolor}{HTML}{48a07e}
\begin{table*}[h]
	\centering
	
 % \vspace{-0.2cm}
	
	\begin{center}
		\begin{tikzpicture}[
				chatbox_inner/.style={rectangle, rounded corners, opacity=0, text opacity=1, font=\sffamily\scriptsize, text width=5in, text height=9pt, inner xsep=6pt, inner ysep=6pt},
				chatbox_prompt_inner/.style={chatbox_inner, align=flush left, xshift=0pt, text height=11pt},
				chatbox_user_inner/.style={chatbox_inner, align=flush left, xshift=0pt},
				chatbox_gpt_inner/.style={chatbox_inner, align=flush left, xshift=0pt},
				chatbox/.style={chatbox_inner, draw=black!25, fill=gray!7, opacity=1, text opacity=0},
				chatbox_prompt/.style={chatbox, align=flush left, fill=gray!1.5, draw=black!30, text height=10pt},
				chatbox_user/.style={chatbox, align=flush left},
				chatbox_gpt/.style={chatbox, align=flush left},
				chatbox2/.style={chatbox_gpt, fill=green!25},
				chatbox3/.style={chatbox_gpt, fill=red!20, draw=black!20},
				chatbox4/.style={chatbox_gpt, fill=yellow!30},
				labelbox/.style={rectangle, rounded corners, draw=black!50, font=\sffamily\scriptsize\bfseries, fill=gray!5, inner sep=3pt},
			]
											
			\node[chatbox_user] (q1) {
				\textbf{System prompt}
				\newline
				\newline
				You are a helpful and precise assistant for segmenting and labeling sentences. We would like to request your help on curating a dataset for entity-level hallucination detection.
				\newline \newline
                We will give you a machine generated biography and a list of checked facts about the biography. Each fact consists of a sentence and a label (True/False). Please do the following process. First, breaking down the biography into words. Second, by referring to the provided list of facts, merging some broken down words in the previous step to form meaningful entities. For example, ``strategic thinking'' should be one entity instead of two. Third, according to the labels in the list of facts, labeling each entity as True or False. Specifically, for facts that share a similar sentence structure (\eg, \textit{``He was born on Mach 9, 1941.''} (\texttt{True}) and \textit{``He was born in Ramos Mejia.''} (\texttt{False})), please first assign labels to entities that differ across atomic facts. For example, first labeling ``Mach 9, 1941'' (\texttt{True}) and ``Ramos Mejia'' (\texttt{False}) in the above case. For those entities that are the same across atomic facts (\eg, ``was born'') or are neutral (\eg, ``he,'' ``in,'' and ``on''), please label them as \texttt{True}. For the cases that there is no atomic fact that shares the same sentence structure, please identify the most informative entities in the sentence and label them with the same label as the atomic fact while treating the rest of the entities as \texttt{True}. In the end, output the entities and labels in the following format:
                \begin{itemize}[nosep]
                    \item Entity 1 (Label 1)
                    \item Entity 2 (Label 2)
                    \item ...
                    \item Entity N (Label N)
                \end{itemize}
                % \newline \newline
                Here are two examples:
                \newline\newline
                \textbf{[Example 1]}
                \newline
                [The start of the biography]
                \newline
                \textcolor{titlecolor}{Marianne McAndrew is an American actress and singer, born on November 21, 1942, in Cleveland, Ohio. She began her acting career in the late 1960s, appearing in various television shows and films.}
                \newline
                [The end of the biography]
                \newline \newline
                [The start of the list of checked facts]
                \newline
                \textcolor{anscolor}{[Marianne McAndrew is an American. (False); Marianne McAndrew is an actress. (True); Marianne McAndrew is a singer. (False); Marianne McAndrew was born on November 21, 1942. (False); Marianne McAndrew was born in Cleveland, Ohio. (False); She began her acting career in the late 1960s. (True); She has appeared in various television shows. (True); She has appeared in various films. (True)]}
                \newline
                [The end of the list of checked facts]
                \newline \newline
                [The start of the ideal output]
                \newline
                \textcolor{labelcolor}{[Marianne McAndrew (True); is (True); an (True); American (False); actress (True); and (True); singer (False); , (True); born (True); on (True); November 21, 1942 (False); , (True); in (True); Cleveland, Ohio (False); . (True); She (True); began (True); her (True); acting career (True); in (True); the late 1960s (True); , (True); appearing (True); in (True); various (True); television shows (True); and (True); films (True); . (True)]}
                \newline
                [The end of the ideal output]
				\newline \newline
                \textbf{[Example 2]}
                \newline
                [The start of the biography]
                \newline
                \textcolor{titlecolor}{Doug Sheehan is an American actor who was born on April 27, 1949, in Santa Monica, California. He is best known for his roles in soap operas, including his portrayal of Joe Kelly on ``General Hospital'' and Ben Gibson on ``Knots Landing.''}
                \newline
                [The end of the biography]
                \newline \newline
                [The start of the list of checked facts]
                \newline
                \textcolor{anscolor}{[Doug Sheehan is an American. (True); Doug Sheehan is an actor. (True); Doug Sheehan was born on April 27, 1949. (True); Doug Sheehan was born in Santa Monica, California. (False); He is best known for his roles in soap operas. (True); He portrayed Joe Kelly. (True); Joe Kelly was in General Hospital. (True); General Hospital is a soap opera. (True); He portrayed Ben Gibson. (True); Ben Gibson was in Knots Landing. (True); Knots Landing is a soap opera. (True)]}
                \newline
                [The end of the list of checked facts]
                \newline \newline
                [The start of the ideal output]
                \newline
                \textcolor{labelcolor}{[Doug Sheehan (True); is (True); an (True); American (True); actor (True); who (True); was born (True); on (True); April 27, 1949 (True); in (True); Santa Monica, California (False); . (True); He (True); is (True); best known (True); for (True); his roles in soap operas (True); , (True); including (True); in (True); his portrayal (True); of (True); Joe Kelly (True); on (True); ``General Hospital'' (True); and (True); Ben Gibson (True); on (True); ``Knots Landing.'' (True)]}
                \newline
                [The end of the ideal output]
				\newline \newline
				\textbf{User prompt}
				\newline
				\newline
				[The start of the biography]
				\newline
				\textcolor{magenta}{\texttt{\{BIOGRAPHY\}}}
				\newline
				[The ebd of the biography]
				\newline \newline
				[The start of the list of checked facts]
				\newline
				\textcolor{magenta}{\texttt{\{LIST OF CHECKED FACTS\}}}
				\newline
				[The end of the list of checked facts]
			};
			\node[chatbox_user_inner] (q1_text) at (q1) {
				\textbf{System prompt}
				\newline
				\newline
				You are a helpful and precise assistant for segmenting and labeling sentences. We would like to request your help on curating a dataset for entity-level hallucination detection.
				\newline \newline
                We will give you a machine generated biography and a list of checked facts about the biography. Each fact consists of a sentence and a label (True/False). Please do the following process. First, breaking down the biography into words. Second, by referring to the provided list of facts, merging some broken down words in the previous step to form meaningful entities. For example, ``strategic thinking'' should be one entity instead of two. Third, according to the labels in the list of facts, labeling each entity as True or False. Specifically, for facts that share a similar sentence structure (\eg, \textit{``He was born on Mach 9, 1941.''} (\texttt{True}) and \textit{``He was born in Ramos Mejia.''} (\texttt{False})), please first assign labels to entities that differ across atomic facts. For example, first labeling ``Mach 9, 1941'' (\texttt{True}) and ``Ramos Mejia'' (\texttt{False}) in the above case. For those entities that are the same across atomic facts (\eg, ``was born'') or are neutral (\eg, ``he,'' ``in,'' and ``on''), please label them as \texttt{True}. For the cases that there is no atomic fact that shares the same sentence structure, please identify the most informative entities in the sentence and label them with the same label as the atomic fact while treating the rest of the entities as \texttt{True}. In the end, output the entities and labels in the following format:
                \begin{itemize}[nosep]
                    \item Entity 1 (Label 1)
                    \item Entity 2 (Label 2)
                    \item ...
                    \item Entity N (Label N)
                \end{itemize}
                % \newline \newline
                Here are two examples:
                \newline\newline
                \textbf{[Example 1]}
                \newline
                [The start of the biography]
                \newline
                \textcolor{titlecolor}{Marianne McAndrew is an American actress and singer, born on November 21, 1942, in Cleveland, Ohio. She began her acting career in the late 1960s, appearing in various television shows and films.}
                \newline
                [The end of the biography]
                \newline \newline
                [The start of the list of checked facts]
                \newline
                \textcolor{anscolor}{[Marianne McAndrew is an American. (False); Marianne McAndrew is an actress. (True); Marianne McAndrew is a singer. (False); Marianne McAndrew was born on November 21, 1942. (False); Marianne McAndrew was born in Cleveland, Ohio. (False); She began her acting career in the late 1960s. (True); She has appeared in various television shows. (True); She has appeared in various films. (True)]}
                \newline
                [The end of the list of checked facts]
                \newline \newline
                [The start of the ideal output]
                \newline
                \textcolor{labelcolor}{[Marianne McAndrew (True); is (True); an (True); American (False); actress (True); and (True); singer (False); , (True); born (True); on (True); November 21, 1942 (False); , (True); in (True); Cleveland, Ohio (False); . (True); She (True); began (True); her (True); acting career (True); in (True); the late 1960s (True); , (True); appearing (True); in (True); various (True); television shows (True); and (True); films (True); . (True)]}
                \newline
                [The end of the ideal output]
				\newline \newline
                \textbf{[Example 2]}
                \newline
                [The start of the biography]
                \newline
                \textcolor{titlecolor}{Doug Sheehan is an American actor who was born on April 27, 1949, in Santa Monica, California. He is best known for his roles in soap operas, including his portrayal of Joe Kelly on ``General Hospital'' and Ben Gibson on ``Knots Landing.''}
                \newline
                [The end of the biography]
                \newline \newline
                [The start of the list of checked facts]
                \newline
                \textcolor{anscolor}{[Doug Sheehan is an American. (True); Doug Sheehan is an actor. (True); Doug Sheehan was born on April 27, 1949. (True); Doug Sheehan was born in Santa Monica, California. (False); He is best known for his roles in soap operas. (True); He portrayed Joe Kelly. (True); Joe Kelly was in General Hospital. (True); General Hospital is a soap opera. (True); He portrayed Ben Gibson. (True); Ben Gibson was in Knots Landing. (True); Knots Landing is a soap opera. (True)]}
                \newline
                [The end of the list of checked facts]
                \newline \newline
                [The start of the ideal output]
                \newline
                \textcolor{labelcolor}{[Doug Sheehan (True); is (True); an (True); American (True); actor (True); who (True); was born (True); on (True); April 27, 1949 (True); in (True); Santa Monica, California (False); . (True); He (True); is (True); best known (True); for (True); his roles in soap operas (True); , (True); including (True); in (True); his portrayal (True); of (True); Joe Kelly (True); on (True); ``General Hospital'' (True); and (True); Ben Gibson (True); on (True); ``Knots Landing.'' (True)]}
                \newline
                [The end of the ideal output]
				\newline \newline
				\textbf{User prompt}
				\newline
				\newline
				[The start of the biography]
				\newline
				\textcolor{magenta}{\texttt{\{BIOGRAPHY\}}}
				\newline
				[The ebd of the biography]
				\newline \newline
				[The start of the list of checked facts]
				\newline
				\textcolor{magenta}{\texttt{\{LIST OF CHECKED FACTS\}}}
				\newline
				[The end of the list of checked facts]
			};
		\end{tikzpicture}
        \caption{GPT-4o prompt for labeling hallucinated entities.}\label{tb:gpt-4-prompt}
	\end{center}
\vspace{-0cm}
\end{table*}
\section{Supplementary Materials for LLMNodeBed}

\subsection{Datasets}\label{sec:dataset}

\begin{table}[!h]
    \centering
    \caption{\textbf{Statistics of supported datasets in LLMNodeBed.}}
    \resizebox{0.95\linewidth}{!}{
    \begin{tabular}{c|cccc|c|cc|ccc}
       \toprule
       \rowcolor{COLOR_MEAN}  &  \multicolumn{4}{c|}{\textbf{Academic}} & \textbf{Web Link} & \multicolumn{2}{c|}{\textbf{Social}} & \multicolumn{3}{c}{\textbf{E-Commerce}} \\ 
       \rowcolor{COLOR_MEAN} \multirow{-2}{*}{\textbf{Statistics}} & Cora & Citeseer & Pubmed & arXiv & WikiCS & Instagram & Reddit & Books & Photo & Computer \\ \midrule
        \# Classes & 7 & 6 & 3 & 40 & 10 & 2 & 2 & 12 & 12 & 10 \\
       \# Nodes & 2,708 & 3,186 & 19,717 & 169,343 & 11,701 & 11,339 & 33,434 & 41,551 & 48,362 & 87,229 \\ 
       \# Edges & 5,429 & 4,277 & 44,338 & 1,166,243 & 215,863 & 144,010 & 198,448 & 358,574 & 500,928 & 721,081 \\ 
       Avg. \# Token & 183.4 & 210.0  & 446.5 & 239.8 & 629.9 & 56.2 & 197.3 & 337.0 & 201.5 & 123.1  \\
       Homophily (\%) & 82.52 & 72.93 & 79.24 & 63.53 & 68.67 & 63.35 & 55.52 & 78.05 & 78.50 & 85.28 \\
       \bottomrule
    \end{tabular}
    }
    \label{tab:dataset}
\end{table}

\begin{table}[!t]
    \centering
    \caption{\textbf{Details of datasets: label space and training data percentages with supervision}.}
    \resizebox{0.95\linewidth}{!}{
    \begin{tabular}{c|cc} 
      \toprule
      \rowcolor{COLOR_MEAN} \textbf{Domain} &  \textbf{Dataset}  & \textbf{Label Space} \\ \midrule
       \multirow{9}{*}{\textbf{Academic}} & Cora  & \begin{tabular}{c}
          Rule\_Learning, Neural\_Networks, Case\_Based, Genetic\_Algorithms,\\ Theory, Reinforcement\_Learning, Probabilistic\_Methods
       \end{tabular} \\
       & Citeseer & \begin{tabular}{c}
            Agents, ML (Machine Learning), IR (Information Retrieval), DB (Databases), \\ HCI (Human-Computer Interaction), AI (Artificial Intelligence)
       \end{tabular} \\ 
       & Pubmed & Experimentally induced diabetes, Type 1 diabetes, Type 2 diabetes \\ 
      &  arXiv & \begin{tabular}{c}
           cs.NA, cs.MM, cs.LO, cs.CY, cs.CR, cs.DC, cs.HC, cs.CE, cs.NI, cs.CC,\\ cs.AI, cs.MA, cs.GL, cs.NE, cs.SC, cs.AR, cs.CV, cs.GR, cs.ET, cs.SY,\\ cs.CG, cs.OH, cs.PL, cs.SE, cs.LG, cs.SD, cs.SI, cs.RO, cs.IT, cs.PF,\\ cs.CL, cs.IR, cs.MS, cs.FL, cs.DS, cs.OS, cs.GT, cs.DB, cs.DL, cs.DM
      \end{tabular}\\ \midrule 

      \textbf{Web Link} & WikiCS & \begin{tabular}{c} 
           Computational Linguistics, Databases, Operating Systems, Computer Architecture,\\ Computer Security, Internet Protocols, Computer File Systems,\\ Distributed Computing Architecture, Web Technology, Programming Language Topics
      \end{tabular} \\ \midrule

      \multirow{2}{*}{\textbf{Social}} & Instagram & Normal User, Commercial User \\ 
      & Reddit & Normal User, Popular User \\ \midrule

      \multirow{6}{*}{\textbf{E-Commerce}} & Books & \begin{tabular}{c} 
          World, Americas, Asia, Military, Europe, Russia, Africa, \\ Ancient Civilizations,  Middle East, Historical Study \& Educational Resources,\\ Australia \& Oceania, Arctic \& Antarctica
      \end{tabular} \\ 
      & Photo & \begin{tabular}{c} \small
           Video Surveillance, Accessories, Binoculars \& Scopes, Video,\\ Lighting \& Studio, Bags \& Cases,  Tripods \& Monopods, Flashes, \\ Digital Cameras, Film Photography, Lenses, Underwater Photography
      \end{tabular} \\ 
      & Computer & \begin{tabular}{c} 
          Computer Accessories \& Peripherals, Tablet Accessories, Laptop Accessories, \\ Computers \& Tablets,  Computer Components, Data Storage, Networking Products,\\ Monitors, Servers, Tablet Replacement Parts
      \end{tabular}
      
      \\ \bottomrule 
        
    \end{tabular}
    }
    \vspace*{10pt}
    \resizebox{0.95\linewidth}{!}{
     \begin{tabular}{c|cccccccccc} 
         \toprule
        \rowcolor{COLOR_MEAN} \textbf{Setting} &  \textbf{Cora} & \textbf{Citeseer} & \textbf{Pubmed} & \textbf{arXiv} & \textbf{WikiCS} & \textbf{Instagram} & \textbf{Reddit} & \textbf{Books} & \textbf{Photo} & \textbf{Computer} \\ \midrule
         Semi-supervised & 5.17\% & 3.77\% & 0.30\% & - & 4.96\% & 10.00\% & 10.00\% & 10.00\% & 10.00\% & 10.00\%  \\
         Supervised & 60.0\% & 60.0\% & 60.0\% & 53.7\% & 60.0\% & 60.0\% & 60.0\% & 60.0\% & 60.0\% & 60.0\% \\
        \bottomrule
     \end{tabular}
    }
    \label{tab:dataset_detail}
\end{table}

We selected 10 datasets from academic, web link, social, and e-commerce domains to create a diverse graph database. Within LLMNodeBed, each dataset is stored in \texttt{.pt} format using PyTorch, which includes shallow embeddings, raw text of nodes, edge indices, labels, and data splits for convenient loading. The datasets are described below, while their statistics and additional details are provided in Table \ref{tab:dataset} and Table \ref{tab:dataset_detail}, respectively. 

\begin{itemize} % [leftmargin=*, topsep=2pt]
    \item \textbf{Academic Networks: }The \textbf{Cora} \cite{Sen2008CollectiveCora}, \textbf{Citeseer} \cite{Giles1998CiteSeerAA}, \textbf{Pubmed} \cite{Yang2016RevisitingSL}, and \textbf{ogbn-arXiv} (abbreviated as ``arXiv'') \cite{hu2020open} datasets consist of nodes representing papers, with edges indicating citation relationships. The associated text attributes include each paper's title and abstract, which we use the collected version as follows: Cora and Pubmed from \citet{he2023TAPE}, Citeseer from \citet{chen2024exploring}. Within the dataset, each node is labeled according to its category. For example, the arXiv dataset includes 40 CS sub-categories such as cs.AI (Artificial Intelligence) and cs.DB (Databases).

    \item  \textbf{Web Link Network: }In the \textbf{WikiCS} dataset \cite{Mernyei2020WikiCSAW}, each node represents a Wikipedia page, and edges indicate reference links between pages. The raw text for each node includes the page name and content, which was collected by \citet{liu2023one}. The classification goal is to categorize each entity into different Wikipedia categories. 
    
    \item \textbf{Social Networks: }The \textbf{Reddit} and \textbf{Instagram} datasets, originally released in \citet{Huang2024GraphAdapter}, feature nodes representing users, with edges denoting social connections like following relationships. For Reddit, each user's associated text consists of their historically published sub-reddits, while for Instagram, it includes the user’s profile page introduction. In Reddit, nodes are labeled to indicate whether the user is popular or normal, while in Instagram, labels specify whether a user is commercial or normal.  

    \item \textbf{E-Commerce Networks: }The \textbf{Ele-Photo} (abbreviated as ``Photo'') and \textbf{Ele-Computer} (abbreviated as ``Computer'') datasets are derived from the Amazon Electronics dataset \cite{Ni2019Amazon}, where each node represents an item in the Photo or Computer category. The \textbf{Books-History} (abbreviated as ``Books'') dataset comes from the Amazon Books dataset, where each node corresponds to a book in the history category. We utilize the processed datasets released in \citet{yan2023comprehensive}. In these e-commerce networks, edges indicate co-purchase or co-view relationships. The associated text for each item includes descriptions, e.g., book titles and summaries, or user reviews. The classification task involves categorizing these products into fine-grained sub-categories.
\end{itemize}



\subsection{Implementation Details and Hyperparameters Setting}\label{sec:hyperparam}

\begin{itemize}
    \item For \textbf{GNNs} with arbitrary input embeddings, either from shallow embeddings or those generated by LMs or LLMs, we perform a grid-search on the hyperparameters as follows:
    
    \texttt{num\_layers} in $[2, 3, 4]$, \texttt{hidden\_dimension} in $[32, 64, 128, 256]$, and \texttt{dropout} in $[0.3, 0.5, 0.7]$.

    Additionally, we explore the design space by considering the inclusion or exclusion of \texttt{batch\_normalization} and \texttt{residual\_connection}. 
    
    For \textbf{shallow embeddings}, the Cora, Citeseer, Pubmed, WikiCS, and arXiv datasets provide initialized embeddings in their released versions \cite{hu2020open, Sen2008CollectiveCora}. For remaining datasets lacking shallow embeddings, we construct these embeddings using \textbf{Node2Vec} \cite{Grover2016node2vecSF} techniques, generating a fixed $300$-dimensional embedding for each node based on a walk length of $30$ and a total of $10$ walks. 
    
    \item For \textbf{MLPs} with arbitrary input embeddings, we perform a grid-search on the hyperparameters as follows: 

    \texttt{num\_layers} in $[2, 3, 4]$, \texttt{hidden\_dimension} in $[128, 256, 512]$, and \texttt{dropout} in $[0.5, 0.6, 0.7]$.

    For both GNNs and MLPs across experimental datasets, the \texttt{learning\_rate} is consistently set to $1e-2$, following previous studies \cite{he2023TAPE, Li2024GLBench}. The total number of epochs is set to $500$ with a patience of $100$. 


    \item For \textbf{SenBERT-66M} and \textbf{RoBERTa-355M}, we set the training epochs to $10$ for semi-supervised settings and $4$ for supervised settings. The \texttt{batch\_size} is set to $32$, and the \texttt{learning\_rate} is set to $2e-5$.  


    \item For \textbf{ENGINE} \cite{Zhu2024ENGINE}, we refer to the hyperparameter settings outlined in the original paper to determine the hyperparameter search space as follows:
    
    \texttt{num\_layers} in $[1, 2, 3]$, $\texttt{hidden\_dimension}$ in $[64, 128]$, and \texttt{learning\_rate} in $[5e-4, 1e-3]$. 
    
    The neighborhood sampler is set to ``Random Walk'' for Cora while ``$k$-Hop'' with $k=2$ for the remaining datasets. 


    \item For \textbf{TAPE} \cite{he2023TAPE}, we utilize the provided prompt templates to guide Mistral-7B and GPT-4o in conducting reasoning. The LM, RoBERTa-355M, is fine-tuned based on its default parameter settings, while the GNN hyperparameters are explored with \texttt{num\_layers} in $[2,3,4]$, \texttt{hidden\_dimension} in $[128,256]$, and with or without \texttt{batch\_normalization}.

    \item For \textbf{LLM Instruction Tuning}, we use the LoRA \cite{Hu2021LoRALA} techniques to fine-tune LLMs. The \texttt{lora\_r} parameter (dimension for LoRA update matrices) is set to $8$ and the \texttt{lora\_alpha} (scaling factor) to $16$. The \texttt{dropout} ratio is set to $0.1$, the \texttt{batch\_size} to $16$, and the \texttt{learning\_rate} to $1e-5$. For each dataset, the input consists of the node's original text along with a carefully crafted task prompt designed to guide the LLMs in performing the classification task. The expected output is the corresponding label. For small-scale datasets such as Cora, Citeseer, and Instagram, the number of training epochs is $10$ in semi-supervised settings and $4$ in supervised settings. For the remaining datasets, the training epochs are $2$ and $1$ for semi-supervised and supervised settings, respectively. The maximum input and output lengths are determined based on the average token lengths of each dataset.

    \item For \textbf{LLaGA} \cite{chen23llaga}, we empirically find that the HO templates consistently outperform the ND templates. Therefore, we set the HO templates as the default configuration, with \texttt{num\_hop} set to $4$. We use the text encoder as RoBERTa-355M. The linear projection layer $\phi_{\theta}(\cdot)$ consists of a $2-$layer MLP with a \texttt{hidden\_dimension} of $2048$. The \texttt{batch\_size} is set to $64$ and \texttt{learning\_rate} to $1e-4$. The number of training epochs is set to $10$ for semi-supervised settings and $4$ for supervised settings. For Qwen2.5-series, we encounter over-fitting issues in the Photo, Computer, and Books datasets,  leading us to adjust the learning rate to $5e-5$ and reduce the number of epochs to $2$ under supervised settings.

    \item For \textbf{GraphGPT} \cite{tang2023graphgpt}, it includes three distinct stages: (1) text-graph grounding, (2) self-supervised instruction tuning, and (3) task-specific instruction tuning. Our empirical findings indicate that the inclusion of stage (1) does not consistently lead to performance improvements, thereby rendering this stage optional. For stage (2), we construct self-supervised training data for each dataset to perform dataset-specific graph matching tasks, adhering to the provided data format\footnote{\href{https://huggingface.co/datasets/Jiabin99/graph-matching}{https://huggingface.co/datasets/Jiabin99/graph-matching}}. In stage (3), we utilize the training data to create $\langle$instruction, ground-truth label$\rangle$ pairs following the original prompt design. The training parameters for stage (2) include $2$ epochs with a \texttt{learning\_rate} of $1e-4$ and a \texttt{batch\_size} of $16$. For stage (3), we train for $10$ epochs in semi-supervised settings and $6$ epochs in supervised settings, with a \texttt{batch\_size} of $32$. Additionally, we adjust the maximum input and output lengths for each stage based on the dataset's text statistics.

    \item For \textbf{LLM Direct Inference}, we adopt two distinct categories of prompt templates: (1) advanced prompts that enhance the reasoning capabilities of LLMs, and (2) prompts enriched with structural information. These templates are illustrated in Appendix \ref{sec:zeroshot_prompt} and strictly adhere to the zero-shot setting.

    \item For \textbf{ZeroG}, we adhere to its original parameter configurations by setting $k=2$, the number of SGC iterations to $10$, and the \texttt{learning\_rate} to $1e-4$. In experiments involving \textbf{GFMs}, the intra-domain training mode utilizes the following source-target pairs: arXiv $\rightarrow$ Cora, arXiv $\rightarrow$ WikiCS, Reddit $\rightarrow$ Instagram, and Computer $\rightarrow$ Photo. 
\end{itemize}


\subsection{Distinct Features}\label{sec:distinct_llmnodebed}
A fair comparison necessitates a benchmark that evaluates all methods using consistent dataloaders, learning paradigms, backbone architectures, and implementation codebases. Our LLMNodeBed carefully follows these guidelines to support systematic and comprehensive evaluation of LLM-based node classification algorithms. Unlike existing benchmarks \cite{Li2024GLBench}, which primarily rely on each algorithm's official implementation, LLMNodeBed distinguishes itself in the following ways:

\begin{itemize}
    \item \textbf{Systematical Implementation: }We consolidate common components (e.g., DataLoader, Evaluation, Backbones) across algorithms to avoid code redundancy and enable fair comparisons and streamlined deployment. For example, several official implementations involve extensive code snippets, and we have produced cleaner, more streamlined versions that enhance both readability and usability. This systematic approach makes LLMNodeBed easily extendable to new datasets or algorithms.

    \item  \textbf{Flexible Selection of Backbones: } LLMNodeBed incorporates a diverse selection of GNNs, LMs, and LLMs, which can be seamlessly integrated as components in baseline methods. 
    \begin{itemize}
        \item \textbf{GNNs: }Our framework supports a wide range of variants, including GCN \cite{kipf2017GCN}, GraphSAGE \cite{hamilton2017SAGE}, GAT \cite{velickovic2018GAT}, GIN \cite{xu2018GIN}, and Graph Transformers \cite{Shi2020GraphTransformer}. These GNNs can be customized with various layers and embedding dimensions.
        \item \textbf{LMs and LLMs: }Open-source models can be easily loaded via the Transformers library\footnote{\href{https://huggingface.co/docs/transformers}{https://huggingface.co/docs/transformers}}. In our experiments, we primarily utilize SenBERT-66M \cite{reimers-2019-sentence-bert}, RoBERTa-355M \cite{Liu2019roberta}, Qwen2.5-Series \cite{Yang2024Qwen2TR}, Mistral-7B \cite{Jiang2023Mistral7B}, and LLaMA3.1-8B \cite{llama3modelcard}. For close-source LLMs, we have formatted the invocation functions of DeepSeek-Chat \cite{Shao2024DeepSeekV2AS} and GPT-4o \cite{Achiam2023GPT4TR}. Additionally, LLMNodeBed allows users to specify and invoke any LM or LLM of their choice, providing flexibility for diverse research needs.
    \end{itemize}

    \item \textbf{Robust Evaluation Protocols: }LLMNodeBed incorporates comprehensive hyperparameter tuning and design space exploration to fully leverage the potential of the algorithms. For instance, recent research \cite{luo2024classic} highlights that classic GNNs remain strong baselines for node classification tasks, especially when the design space is expanded through techniques like residual connections, jumping knowledge, and selectable batch normalization. LLMNodeBed supports these enhancements, enabling the full utilization of GNNs. Furthermore, we conduct multiple experimental runs to enhance reliability and account for variability, which was often overlooked in previous studies.

\end{itemize}


\clearpage
\newpage

\section{Supplementary Materials for Comparisons among Algorithm Categories} 
\begin{table*}[!h]
    \centering
    \caption{\textbf{Performance comparison under semi-supervised and supervised settings with Macro-F1 ($\%$) reported.} \\\small{The \colorbox{orange!25}{\textbf{best}} and \colorbox{orange!10}{second-best} results are highlighted. LLM\textsubscript{IT} on the arXiv dataset requires extensive training time, preventing repeated experiments.}}
    \vspace*{-8pt}
   \resizebox{\linewidth}{!}{
    \begin{tabular}{cc|cccccccccc}
      \toprule
     \rowcolor{COLOR_MEAN} \multicolumn{2}{c}{\textbf{Semi-supervised}}   & \textbf{Cora} & \textbf{Citeseer} & \textbf{Pubmed} & \textbf{WikiCS} & \textbf{Instagram} & \textbf{Reddit} & \textbf{Books} & \textbf{Photo} & \textbf{Computer} & \textbf{Avg.} \\ \midrule
         \multirow{5}{*}{\textbf{Classic}} & {GCN\tiny{ShallowEmb}} & 80.76$_{\pm \text{0.30}}$ & 66.00$_{\pm \text{0.17}}$ & \cellcolor{orange!25} \textbf{79.00$_{\pm \text{0.30}}$} & 77.87$_{\pm \text{0.18}}$ & 52.44$_{\pm \text{1.02}}$ & 61.15$_{\pm \text{0.56}}$ & 22.18$_{\pm \text{1.12}}$ & 62.65$_{\pm \text{1.46}}$ & 61.33$_{\pm \text{2.55}}$ & 62.60 \\ 
          & {SAGE\tiny{ShallowEmb}} & 80.88$_{\pm \text{0.47}}$ & 65.06$_{\pm \text{0.32}}$ & 77.88$_{\pm \text{0.41}}$ & 77.02$_{\pm \text{0.59}}$ & 50.74$_{\pm \text{0.26}}$ & 56.39$_{\pm \text{0.44}}$ & 24.37$_{\pm \text{0.72}}$ & 74.17$_{\pm \text{0.32}}$ & 71.67$_{\pm \text{0.61}}$ & 64.24 \\ 
        & {GAT\tiny{ShallowEmb}} & 79.65$_{\pm \text{1.03}}$ & 64.82$_{\pm \text{1.20}}$ & 78.43$_{\pm \text{0.73}}$ & 77.57$_{\pm \text{1.01}}$ & 40.19$_{\pm \text{1.56}}$ & 60.37$_{\pm \text{1.22}}$ & 28.93$_{\pm \text{3.65}}$ & 75.89$_{\pm \text{0.60}}$ & 77.06$_{\pm \text{2.98}}$ & 64.77 \\ 
        & SenBERT-66M & 64.54$_{\pm \text{1.18}}$ & 56.65$_{\pm \text{1.40}}$ & 32.08$_{\pm \text{3.35}}$ & 74.97$_{\pm \text{0.96}}$ & 55.47$_{\pm \text{0.76}}$ & 55.75$_{\pm \text{0.58}}$ & 43.15$_{\pm \text{1.53}}$ & 66.08$_{\pm \text{0.76}}$ & 59.79$_{\pm \text{1.30}}$  & 56.50 \\
         & {RoBERTa-355M} & 70.41$_{\pm \text{0.83}}$ & 63.36$_{\pm \text{1.75}}$ & 40.82$_{\pm \text{2.05}}$ & 73.98$_{\pm \text{1.72}}$ & \cellcolor{orange!10} 57.43$_{\pm \text{0.42}}$ & 59.23$_{\pm \text{0.36}}$ & \cellcolor{orange!25} \textbf{51.34$_{\pm \text{0.95}}$} & 67.92$_{\pm \text{0.49}}$ & 63.38$_{\pm \text{2.17}}$ & 60.87 \\ \midrule
          
        \multirow{2}{*}{\textbf{Encoder}}  
       & $\text{GCN}_{\text{LLMEmb}}$ & 81.19$_{\pm \text{0.59}}$ & \cellcolor{orange!25} \cellcolor{orange!25} \textbf{67.17$_{\pm \text{0.73}}$} & 78.39$_{\pm \text{0.36}}$ & 78.58$_{\pm \text{0.51}}$ & \cellcolor{orange!25} \textbf{58.97$_{\pm \text{0.85}}$} & \cellcolor{orange!10} 68.46$_{\pm \text{0.91}}$ & 39.64$_{\pm \text{0.85}}$ & \cellcolor{orange!10} 79.87$_{\pm \text{0.57}}$ & 77.36$_{\pm \text{0.70}}$ &  \cellcolor{orange!10} 
 69.95 \\ 
       & ENGINE & \cellcolor{orange!25} \textbf{82.52$_{\pm \text{0.45}}$} & \cellcolor{orange!10} \textbf{67.15$_{\pm \text{0.15}}$} & 77.53$_{\pm \text{0.34}}$ & \cellcolor{orange!10} 78.89$_{\pm \text{0.38}}$ & 57.25$_{\pm \text{2.50}}$ & \cellcolor{orange!25} \textbf{69.56$_{\pm \text{0.21}}$} & 34.04$_{\pm \text{1.10}}$ & 78.55$_{\pm \text{1.12}}$ & 75.86$_{\pm \text{0.60}}$ & 69.04 \\  \midrule
       
       \textbf{Reasoner} & TAPE & \cellcolor{orange!10} 81.89$_{\pm \text{0.31}}$ & 66.80$_{\pm \text{0.33}}$ & \cellcolor{orange!10} 78.46$_{\pm \text{1.13}}$ & \cellcolor{orange!25} \textbf{80.03$_{\pm \text{0.23}}$} & 50.01$_{\pm \text{1.60}}$ & 61.23$_{\pm \text{0.69}}$ & \cellcolor{orange!10} 47.12$_{\pm \text{3.26}}$ & \cellcolor{orange!25} \textbf{82.31$_{\pm \text{0.19}}$} & \cellcolor{orange!25} \textbf{84.90$_{\pm \text{1.14}}$} & \cellcolor{orange!25} \textbf{70.31} \\  \midrule
       
      \multirow{3}{*}{\textbf{Predictor}} & $\text{LLM}_{\text{IT}}$ & 56.35$_{\pm \text{0.22}}$ & 47.34$_{\pm \text{0.68}}$ & 62.81$_{\pm \text{0.21}}$ & 65.75$_{\pm \text{0.17}}$ & 38.30$_{\pm \text{0.94}}$ & 44.41$_{\pm \text{8.86}}$ & 39.44$_{\pm \text{0.44}}$ & 60.71$_{\pm \text{0.09}}$ & 57.38$_{\pm \text{0.65}}$ & 52.50 \\
       & GraphGPT & 58.33$_{\pm \text{0.81}}$ & 54.21$_{\pm \text{1.11}}$ & 56.09$_{\pm \text{0.88}}$ & 62.04$_{\pm \text{0.62}}$ & 38.78$_{\pm \text{0.60}}$ & 38.88$_{\pm \text{0.28}}$ & 42.85$_{\pm \text{0.94}}$ & 65.77$_{\pm \text{1.34}}$ & 66.69$_{\pm \text{1.49}}$ & 53.74  \\ 
       & LLaGA  & 71.14$_{\pm \text{4.47}}$ & 52.53$_{\pm \text{3.59}}$ & 45.12$_{\pm \text{7.63}}$ & 70.48$_{\pm \text{2.94}}$ & 50.12$_{\pm \text{10.45}}$ & 54.67$_{\pm \text{11.24}}$ & 39.70$_{\pm \text{2.44}}$ & 79.32$_{\pm \text{2.42}}$ & \cellcolor{orange!10} 78.01$_{\pm \text{1.36}}$ & 60.12 \\ \bottomrule
    \end{tabular}
    }

    \vspace*{5pt}
    \resizebox{\linewidth}{!}{
    \begin{tabular}{cc|ccccccccccc}
      \toprule
      \rowcolor{COLOR_MEAN} \multicolumn{2}{c}{\textbf{Supervised}}   & \textbf{Cora} & \textbf{Citeseer} & \textbf{Pubmed} & \textbf{arXiv} & \textbf{WikiCS} & \textbf{Instagram} & \textbf{Reddit} & \textbf{Books} & \textbf{Photo} & \textbf{Computer} & \textbf{Avg.} \\ \midrule
      \multirow{5}{*}{\textbf{Classic}} &{GCN\tiny{ShallowEmb}} & 86.54$_{\pm \text{1.44}}$ & 71.52$_{\pm \text{1.71}}$ & 88.54$_{\pm \text{0.60}}$ & 50.28$_{\pm \text{0.84}}$ & 82.11$_{\pm \text{0.61}}$ & 54.91$_{\pm \text{0.48}}$ & 65.00$_{\pm \text{0.42}}$ & 34.39$_{\pm \text{1.26}}$ & 66.04$_{\pm \text{2.85}}$ & 64.60$_{\pm \text{4.99}}$ & 66.39 \\ 

      & {SAGE\tiny{ShallowEmb}} & 86.37$_{\pm \text{1.26}}$ & 71.87$_{\pm \text{1.34}}$ & 90.16$_{\pm \text{0.27}}$ & 49.73$_{\pm \text{0.49}}$ & 82.78$_{\pm \text{1.53}}$ & 51.37$_{\pm \text{1.67}}$ & 61.39$_{\pm \text{0.54}}$ & 38.29$_{\pm \text{2.54}}$ & 80.37$_{\pm \text{0.34}}$ & 82.93$_{\pm \text{0.49}}$ & 69.53 \\ 
      & {GAT\tiny{ShallowEmb}} & 85.64$_{\pm \text{0.87}}$ & 69.27$_{\pm \text{2.15}}$ & 87.70$_{\pm \text{0.48}}$ & 49.71$_{\pm \text{0.23}}$ & 82.14$_{\pm \text{1.04}}$ & 50.26$_{\pm \text{3.16}}$ & 64.11$_{\pm \text{1.06}}$ & 42.85$_{\pm \text{1.62}}$ & 80.82$_{\pm \text{0.89}}$ & 84.74$_{\pm \text{0.79}}$ & 69.72 \\ 
      & SenBERT-66M & 77.13$_{\pm \text{2.19}}$ & 71.25$_{\pm \text{1.03}}$ & \cellcolor{orange!10} 93.95$_{\pm \text{0.39}}$ & 52.48$_{\pm \text{0.12}}$ & 84.43$_{\pm \text{1.58}}$ & 56.12$_{\pm \text{0.66}}$ & 58.31$_{\pm \text{0.76}}$ & 52.96$_{\pm \text{1.78}}$ & 70.39$_{\pm \text{0.54}}$ & 65.08$_{\pm \text{0.37}}$ & 68.21 \\
      & {RoBERTa-355M} & 81.38$_{\pm \text{1.17}}$ & 72.31$_{\pm \text{1.45}}$ & \cellcolor{orange!25} \textbf{94.33$_{\pm \text{0.14}}$} & 57.25$_{\pm \text{0.53}}$ & \cellcolor{orange!25} \textbf{86.10$_{\pm \text{1.11}}$} & 59.10$_{\pm \text{1.38}}$ & 60.16$_{\pm \text{0.94}}$ & \cellcolor{orange!25} \textbf{57.24$_{\pm \text{1.27}}$} & 72.89$_{\pm \text{0.50}}$ & 70.64$_{\pm \text{0.58}}$  & 71.14 \\ \midrule
      \multirow{2}{*}{\textbf{Encoder}} & $\text{GCN}_{\text{LLMEmb}}$ & \cellcolor{orange!25} \textbf{87.23$_{\pm \text{1.34}}$} & 
      \cellcolor{orange!10} 72.71$_{\pm \text{0.86}}$ & 87.76$_{\pm \text{0.76}}$ & 55.22$_{\pm \text{0.46}}$ & 82.78$_{\pm \text{1.68}}$ & \cellcolor{orange!25} \textbf{60.38$_{\pm \text{0.53}}$} & \cellcolor{orange!10} 70.64$_{\pm \text{0.75}}$ & 48.18$_{\pm \text{2.29}}$ & 80.51$_{\pm \text{0.65}}$ & 85.10$_{\pm \text{1.00}}$ & 73.05 \\ 
      & ENGINE & 85.72$_{\pm \text{1.58}}$ & 71.22$_{\pm \text{2.17}}$ & 89.63$_{\pm \text{0.14}}$ & 56.32$_{\pm \text{0.60}}$ & 83.80$_{\pm \text{1.06}}$ & \cellcolor{orange!10} 60.02$_{\pm \text{1.26}}$ & \cellcolor{orange!25} \textbf{71.17$_{\pm \text{0.75}}$} & 48.24$_{\pm \text{2.64}}$ & 82.77$_{\pm \text{0.28}}$ & 84.15$_{\pm \text{0.84}}$ & 73.30 \\  \midrule
      \textbf{Reasoner} & TAPE & \cellcolor{orange!10} 87.21$_{\pm \text{1.60}}$ & \cellcolor{orange!25} \textbf{73.33$_{\pm \text{1.57}}$} & 92.39$_{\pm \text{0.02}}$ & \cellcolor{orange!10} 57.79$_{\pm \text{0.49}}$ & \cellcolor{orange!10} 86.03$_{\pm \text{1.14}}$ & 58.31$_{\pm \text{1.15}}$ & 65.91$_{\pm \text{0.71}}$ & 54.07$_{\pm \text{2.01}}$ & \cellcolor{orange!10} 83.41$_{\pm \text{0.42}}$ & \cellcolor{orange!10} 86.78$_{\pm \text{0.53}}$  & \cellcolor{orange!25} \textbf{74.52} \\  \midrule
      \multirow{3}{*}{\textbf{Predictor}} & $\text{LLM}_{\text{IT}}$ & 66.93$_{\pm \text{4.54}}$ & 52.22$_{\pm \text{2.71}}$ & 93.45$_{\pm \text{0.25}}$ & 57.48 & 78.39$_{\pm \text{1.20}}$ & 42.15$_{\pm \text{4.54}}$ & 56.65$_{\pm \text{0.85}}$ & 49.86$_{\pm \text{0.71}}$ & 68.74$_{\pm \text{2.54}}$ & 62.78$_{\pm \text{2.83}}$ & 62.86 \\ 
      & GraphGPT & 74.08$_{\pm \text{4.36}}$ & 61.04$_{\pm \text{1.24}}$ & 80.98$_{\pm \text{11.22}}$ & 56.80$_{\pm \text{0.08}}$ & 73.92$_{\pm \text{0.61}}$ & 40.07$_{\pm \text{2.10}}$ & 39.97$_{\pm \text{1.77}}$ & 47.97$_{\pm \text{1.94}}$ & 74.22$_{\pm \text{0.43}}$ & 74.19$_{\pm \text{1.75}}$ & 62.32 \\ 
      & LLaGA & 84.97$_{\pm \text{3.97}}$ & 72.59$_{\pm \text{1.70}}$ & 90.00$_{\pm \text{0.80}}$ & \cellcolor{orange!25} \textbf{58.08$_{\pm \text{0.29}}$} & 82.37$_{\pm \text{1.73}}$ & 57.96$_{\pm \text{2.40}}$ & 62.14$_{\pm \text{15.59}}$ & \cellcolor{orange!10} 54.89$_{\pm \text{2.29}}$ & \cellcolor{orange!25} \textbf{83.56$_{\pm \text{0.40}}$} & \cellcolor{orange!25} \textbf{86.97$_{\pm \text{0.34}}$} & \cellcolor{orange!10} 73.35 \\ \bottomrule
    \end{tabular}
    }
    \label{tab:mainexp_f1}
\end{table*}



\begin{table*}[!h]
    \centering
    \caption{\textbf{Performance comparison under zero-shot setting with both Accuracy (\%) and Macro-F1 (\%) reported.} }
    \vspace*{-8pt}
    \resizebox{0.9\linewidth}{!}{
    \begin{tabular}{cc|cc|cc|cc|cc|cc}
       \toprule
       \rowcolor{COLOR_MEAN} & &  \multicolumn{2}{c}{\textbf{Cora} (82.52)} & \multicolumn{2}{c}{\textbf{WikiCS} (68.67)} & \multicolumn{2}{c}{\textbf{Instagram} (63.35)}  & \multicolumn{2}{c}{\textbf{Photo} (78.50)} & \multicolumn{2}{c}{\textbf{Avg.}}  \\ 
       \rowcolor{COLOR_MEAN} \multirow{-2}{*}{\textbf{Type \& LLM}} &  \multirow{-2}{*}{\textbf{Method}} & Acc & Macro-F1 & Acc & Macro-F1 & Acc & Macro-F1 &  Acc & Macro-F1  & Acc & Macro-F1 \\  \midrule
       \multirow{6}{*}{\begin{tabular}{c}
            \textbf{LLM} \\ DeepSeek-Chat
       \end{tabular}} & Direct  & 68.06 & 60.07 & 71.41 & 65.21 & 42.42 & 39.42 & 65.25 & 56.92 & 61.78 & 55.40 \\
       & CoT  &  68.08 & 60.32 & 71.83 & 60.73 & \textbf{43.47} & 28.22 & 63.97 & 57.61 & 61.84 & 51.72 \\ 
       & ToT & 66.61 & 59.10 & 57.35 & 54.68 & 36.95 & 23.94 & 59.25 & 54.39 & 55.04 & 48.03 \\ 
       & ReAct & 65.68 & 58.68 & 71.24 & 60.97 & 43.30 & 28.42 & 63.66 & 56.23 & 60.97 & 51.08 \\ 
       & w. Neighbor & 68.63 & \textbf{69.21} & 70.26 & 64.26 & 43.34 & \textbf{41.57} & 61.57 & 54.84 & 60.95 & 57.47 \\ 
       & w. Summary & \textbf{73.62} & 64.80 & \textbf{72.53} & \textbf{67.28} & 41.18 & 37.56 & \textbf{72.73} & \textbf{70.58} & \textbf{65.02} & \textbf{60.05} \\ \midrule

       \multirow{6}{*}{\begin{tabular}{c}
            \textbf{LLM} \\ Mistral-7B
       \end{tabular}} & Direct & 59.65 & 58.34  &  70.13 &	67.80 &	44.29 &	42.16 &	\textbf{57.54} & 55.50 & 57.90 & 55.95 \\
       & CoT  & 58.02 & 57.13 & 69.00 &	66.17 &	45.48 & 44.56 & 49.56 &	51.42 & 55.52 & 54.82 \\ 
       & ToT  & 58.78 &	57.20 & 	67.56 &	64.52 &	45.39 &	44.73 &	44.25 &	46.87 & 54.00 & 53.33 \\
       & ReAct & 60.32 & 60.89 & \textbf{71.02} &	67.31 	& 	\textbf{46.26} &	\textbf{46.09} &	52.47 	& 50.92 & 57.52 & 56.30\\ 
       & w. Neighbor & 67.69 & 66.62 & 68.32 & 65.58 & 37.05 & 28.23 &  53.39 & 56.06 & 56.61 & 54.12  \\ 
       & w. Summary & \textbf{68.12} & \textbf{67.45} & 70.52 & \textbf{67.87} & 	41.94 &	38.93 &	56.01 &	\textbf{56.22} & \textbf{59.15} & \textbf{57.62} \\
       \bottomrule
    \end{tabular}
    }
    \label{tab:zeroshot_supple}
\end{table*}



\clearpage
\newpage

\section{Supplementary Materials for Fine-grained Analysis}

\subsection{LLM-as-Encoder: Compared with LMs}
\begin{table*}[!h]
    \centering
    \caption{\textbf{Comparison of LLM- and LM-as-Encoder with Accuracy (\%) reported under supervised setting. LLM-as-Encoder outperforms LMs in heterophilic graphs.} The \colorbox{blue!10}{\textbf{best encoder}} within each method on a dataset is highlighted.}
    \resizebox{\linewidth}{!}{

    \begin{tabular}{cc|cccccccccc}
      \toprule
     \rowcolor{COLOR_MEAN} {\textbf{Method}}  & {\textbf{Encoder}}  & \textbf{Computer} &  \textbf{Cora} & \textbf{Pubmed} & \textbf{Photo} & \textbf{Books} & \textbf{Citeseer} & \textbf{WikiCS} & \textbf{arXiv} & \textbf{Instagram} & \textbf{Reddit} \\ \midrule
      \multicolumn{2}{c}{Homophily Ratio (\%)} & 85.28 & 82.52 & 79.24 &  78.50 & 78.05 & 72.93 & \textcolor{blue!30}{\textbf{68.67}} & \textcolor{blue!30}{\textbf{63.53}} & \textcolor{blue!30}{\textbf{63.35}} & \textcolor{blue!30}{\textbf{55.22}} \\ \midrule

    \multirow{4}{*}{MLP} & SenBERT & 71.75\textsubscript{±0.20} & 75.72\textsubscript{±2.19} & 90.26\textsubscript{±0.37} & 74.59\textsubscript{±0.19} & 83.61\textsubscript{±0.43} & 69.84\textsubscript{±2.21} & 81.13\textsubscript{±1.01} & 68.60\textsubscript{±0.16} & 67.12\textsubscript{±1.01} & 58.40\textsubscript{±0.56} \\
    & RoBERTa & \cellcolor{blue!10} \textbf{72.36\textsubscript{±0.09}} & 80.92\textsubscript{±2.65} & 90.54\textsubscript{±0.37} & 75.50\textsubscript{±0.25} & \cellcolor{blue!10} \textbf{84.06\textsubscript{±0.37}} & \cellcolor{blue!10} \textbf{74.11\textsubscript{±1.33}} & 83.18\textsubscript{±0.96} & 73.65\textsubscript{±0.11} & 68.57\textsubscript{±0.84} & 61.06\textsubscript{±0.31} \\
    & Qwen-3B & 69.25\textsubscript{±0.34} & 81.29\textsubscript{±1.05} & 92.02\textsubscript{±0.38} & 74.35\textsubscript{±0.47} & 83.43\textsubscript{±0.56} & 72.88\textsubscript{±1.66} & 85.46\textsubscript{±0.84} & 74.62\textsubscript{±0.19} & 68.62\textsubscript{±0.54} & 61.39\textsubscript{±0.36}\\ 
    & Mistral-7B & 71.38\textsubscript{±0.17} & \cellcolor{blue!10} \textbf{81.62\textsubscript{±0.63}} & \cellcolor{blue!10} \textbf{92.73\textsubscript{±0.24}} & \cellcolor{blue!10} \textbf{75.83\textsubscript{±0.25}} & 83.96\textsubscript{±0.46} & 73.85\textsubscript{±1.75}  & \cellcolor{blue!10} \textbf{86.92\textsubscript{±0.90}} & \cellcolor{blue!10} \textbf{75.29\textsubscript{±0.16}} & \cellcolor{blue!10} \textbf{69.03\textsubscript{±0.35}}  &  \cellcolor{blue!10}\textbf{62.49\textsubscript{±0.20}} \\ \midrule
    
    \multirow{4}{*}{GCN} & SenBERT & 
    \cellcolor{blue!10}
    \textbf{90.53\textsubscript{±0.18}} & \cellcolor{blue!10} \textbf{88.33\textsubscript{±0.90}} & \cellcolor{blue!10} \textbf{89.26\textsubscript{±0.24}} & \cellcolor{blue!10} \textbf{86.79\textsubscript{±0.18}} & \cellcolor{blue!10} \textbf{84.60\textsubscript{±0.38}} & 75.31\textsubscript{±0.55} & 84.28\textsubscript{±0.26} & 73.29\textsubscript{±0.22} & 68.13\textsubscript{±0.18} & 69.04\textsubscript{±0.46}\\ 
     & RoBERTa & 90.16\textsubscript{±0.13} & 87.96\textsubscript{±1.98} & 89.00\textsubscript{±0.21} & 86.79\textsubscript{±0.48} & 84.42\textsubscript{±0.38} & \cellcolor{blue!10} \textbf{76.57\textsubscript{±0.99}} & 84.64\textsubscript{±0.27} & 74.13\textsubscript{±0.19} & \cellcolor{blue!10} \textbf{68.43\textsubscript{±0.51}} & 69.28\textsubscript{±0.50} \\ 
     & Qwen-3B & 88.06\textsubscript{±0.35} & 88.24\textsubscript{±1.79} & 88.42\textsubscript{±0.51} & 85.20\textsubscript{±0.38} & 84.34\textsubscript{±0.61} & 76.37\textsubscript{±0.97} & 84.05\textsubscript{±0.55} & 73.62\textsubscript{±0.33} & 68.32\textsubscript{±0.73} & \cellcolor{blue!10} \textbf{71.04\textsubscript{±0.42}} \\ 
    & Mistral-7B & 89.52\textsubscript{±0.31} & 88.15\textsubscript{±1.79} & 88.38\textsubscript{±0.68} & 86.07\textsubscript{±0.63} & 84.23\textsubscript{±0.20} & 76.45\textsubscript{±1.19} & \cellcolor{blue!10} \textbf{84.78\textsubscript{±0.86}} & \cellcolor{blue!10} \textbf{74.39\textsubscript{±0.31}} & 68.27\textsubscript{±0.45} & 70.65\textsubscript{±0.75} \\ \midrule
    
    \multirow{4}{*}{SAGE} & SenBERT & \cellcolor{blue!10} \textbf{90.86\textsubscript{±0.18}} & 87.36\textsubscript{±1.79} & \cellcolor{blue!10} \textbf{90.93\textsubscript{±0.13}} & 87.41\textsubscript{±0.33} & \cellcolor{blue!10} \textbf{85.13\textsubscript{±0.27}} & 74.73\textsubscript{±0.67} & 85.94\textsubscript{±0.52} & 73.43\textsubscript{±0.23} & 67.72\textsubscript{±0.43} & 64.13\textsubscript{±0.41} \\ 
     & RoBERTa & 90.70\textsubscript{±0.25} & 87.36\textsubscript{±1.69} & 90.38\textsubscript{±0.09} & \cellcolor{blue!10} \textbf{87.42\textsubscript{±0.51}} & 85.13\textsubscript{±0.41} & \cellcolor{blue!10} \textbf{75.90\textsubscript{±0.41}} & 86.31\textsubscript{±0.68} & 75.28\textsubscript{±0.31} & 68.84\textsubscript{±0.54} & \cellcolor{blue!10}  \textbf{64.85\textsubscript{±0.31}} \\
    & Qwen-3B & 87.44\textsubscript{±0.66} & \cellcolor{blue!10} \textbf{87.36\textsubscript{±1.10}} & 89.98\textsubscript{±0.38} & 85.17\textsubscript{±0.44}  & 84.69\textsubscript{±0.31} & 75.63\textsubscript{±0.94} & 85.58\textsubscript{±0.58} & 75.20\textsubscript{±0.49} & 68.43\textsubscript{±0.57} & 63.98\textsubscript{±0.69} \\ 
    & Mistral-7B & 90.16\textsubscript{±0.26} & 87.22\textsubscript{±1.24} & 90.54\textsubscript{±0.50} & 87.34\textsubscript{±0.43} & 85.01\textsubscript{±0.49} & 75.20\textsubscript{±1.34} & \cellcolor{blue!10} \textbf{87.87\textsubscript{±0.35}} & \cellcolor{blue!10} \textbf{76.18\textsubscript{±0.34}} & \cellcolor{blue!10} \textbf{69.39\textsubscript{±0.52}} & 64.34\textsubscript{±0.23} \\ 

    \bottomrule
    \end{tabular}
    }
    \label{tab:encoder_comp_fullysupervised}
\end{table*}




% \begin{table*}[!h]
%     \centering
%     \caption{\textbf{Comparison of LLM- and LM-as-Encoder with Accuracy (\%) reported under supervised setting. LLM-as-Encoder outperforms LMs in heterophilic graphs.} The \colorbox{blue!10}{\textbf{best encoder}} within each method on a dataset is highlighted.}
%     \resizebox{\linewidth}{!}{

%     \begin{tabular}{cc|cccccccccc}
%       \toprule
%      \rowcolor{COLOR_MEAN} & & \multicolumn{4}{c}{\textbf{Academic}} & \textbf{Web Link} & \multicolumn{2}{c}{\textbf{Social}} & \multicolumn{3}{c}{\textbf{E-Commerce}} \\
%      \rowcolor{COLOR_MEAN} \multirow{-2}{*}{\textbf{Method}}  & \multirow{-2}{*}{\textbf{Encoder}}  & Cora & Citeseer & Pubmed & arXiv & WikiCS & Instagram & Reddit & Books & Photo & Computer \\ \midrule
%       \multicolumn{2}{c}{Homophily Ratio (\%)} & 82.52 & 72.93 & 79.24 & \textbf{\textcolor{blue!30}{63.53}} & \textbf{\textcolor{blue!30}{68.67}} & \textbf{\textcolor{blue!30}{63.35}} & \textbf{\textcolor{blue!30}{55.22}} & 78.05 & 78.50  & 85.28 \\ \midrule

%     \multirow{4}{*}{MLP} & SenBERT & 75.72$_{\pm \text{2.19}}$ & 69.84$_{\pm \text{2.21}}$ & 90.26$_{\pm \text{0.37}}$ & 68.60$_{\pm \text{0.16}}$ & 81.13$_{\pm \text{0.74}}$ & 67.12$_{\pm \text{1.01}}$ & 58.40$_{\pm \text{0.56}}$ & 83.61$_{\pm \text{0.43}}$ & 74.59$_{\pm \text{0.19}}$ & 71.75$_{\pm \text{0.20}}$ \\
%     & RoBERTa & 80.92$_{\pm \text{2.65}}$ & \cellcolor{blue!10}\textbf{74.11$_{\pm \text{1.33}}$} & 90.54$_{\pm \text{0.37}}$ & 73.65$_{\pm \text{0.11}}$ & 83.18$_{\pm \text{0.96}}$ & 68.57$_{\pm \text{0.84}}$ & 61.06$_{\pm \text{0.31}}$ & \cellcolor{blue!10}\textbf{84.06$_{\pm \text{0.37}}$} & 75.50$_{\pm \text{0.25}}$ & \cellcolor{blue!10}\textbf{72.36$_{\pm \text{0.09}}$} \\
%     & Qwen-3B & 81.29$_{\pm \text{1.05}}$ & 72.88$_{\pm \text{1.66}}$ & 92.02$_{\pm \text{0.38}}$ & 74.62$_{\pm \text{0.19}}$ & 85.46$_{\pm \text{0.84}}$ & 68.62$_{\pm \text{0.54}}$ & 61.39$_{\pm \text{0.36}}$ & 83.43$_{\pm \text{0.56}}$ & 74.35$_{\pm \text{0.47}}$ & 69.25$_{\pm \text{0.34}}$ \\ 
%     & Mistral-7B & \cellcolor{blue!10}\textbf{81.62$_{\pm \text{0.63}}$} & 73.85$_{\pm \text{1.75}}$ & \cellcolor{blue!10}\textbf{92.73$_{\pm \text{0.24}}$} &  \cellcolor{blue!10} \textbf{75.29$_{\pm \text{0.16}}$} & \cellcolor{blue!10}\textbf{86.92$_{\pm \text{0.90}}$} & \cellcolor{blue!10}\textbf{69.03$_{\pm \text{0.35}}$} & \cellcolor{blue!10}\textbf{62.49$_{\pm \text{0.20}}$} & 83.96$_{\pm \text{0.46}}$ & \cellcolor{blue!10}\textbf{75.83$_{\pm \text{0.25}}$} & 71.38$_{\pm \text{0.17}}$ \\ \midrule
%     \multirow{4}{*}{GCN} & SenBERT & \cellcolor{blue!10} \textbf{88.33$_{\pm \text{0.90}}$} & 75.31$_{\pm \text{0.55}}$ & \cellcolor{blue!10} \textbf{89.26$_{\pm \text{0.24}}$} & 73.29$_{\pm \text{0.22}}$ & 84.28$_{\pm \text{0.26}}$ & 68.13$_{\pm \text{0.18}}$ & 69.04$_{\pm \text{0.46}}$ & \cellcolor{blue!10} \textbf{84.60$_{\pm \text{0.38}}$} & \cellcolor{blue!10} \textbf{86.79$_{\pm \text{0.18}}$} & \cellcolor{blue!10} \textbf{90.53$_{\pm \text{0.18}}$} \\ 
%      & RoBERTa & 87.96$_{\pm \text{1.98}}$ & \cellcolor{blue!10} \textbf{76.57$_{\pm \text{0.99}}$} & 89.00$_{\pm \text{0.21}}$ & 74.13$_{\pm \text{0.19}}$ & 84.64$_{\pm \text{0.27}}$ & \cellcolor{blue!10} \textbf{68.43$_{\pm \text{0.51}}$} & 69.28$_{\pm \text{0.50}}$ & 84.42$_{\pm \text{0.38}}$ & 86.79$_{\pm \text{0.48}}$ & 90.16$_{\pm \text{0.13}}$ \\ 
%      & Qwen-3B & 88.24$_{\pm \text{1.79}}$ & 76.37$_{\pm \text{0.97}}$ & 88.42$_{\pm \text{0.51}}$ & 73.62$_{\pm \text{0.33}}$ & 84.05$_{\pm \text{0.55}}$ & 68.32$_{\pm \text{0.73}}$ & \cellcolor{blue!10} \textbf{71.04$_{\pm \text{0.42}}$} & 84.34$_{\pm \text{0.61}}$ & 85.20$_{\pm \text{0.38}}$ & 88.06$_{\pm \text{0.35}}$ \\ 
%     & Mistral-7B & 88.15$_{\pm \text{1.79}}$ & 76.45$_{\pm \text{1.19}}$ & 88.38$_{\pm \text{0.68}}$ & \cellcolor{blue!10} \textbf{74.39$_{\pm \text{0.31}}$} & \cellcolor{blue!10} \textbf{84.78$_{\pm \text{0.86}}$} & 68.27$_{\pm \text{0.45}}$ & 70.65$_{\pm \text{0.75}}$ & 84.23$_{\pm \text{0.20}}$ & 86.07$_{\pm \text{0.63}}$ & 89.52$_{\pm \text{0.31}}$ \\ \midrule
%     \multirow{4}{*}{SAGE} & SenBERT & 87.36$_{\pm \text{1.79}}$ & 74.73$_{\pm \text{0.67}}$ & \cellcolor{blue!10} \textbf{90.93$_{\pm \text{0.13}}$} & 73.43$_{\pm \text{0.23}}$ & 85.94$_{\pm \text{0.52}}$ & 67.72$_{\pm \text{0.43}}$ & 64.13$_{\pm \text{0.41}}$ & \cellcolor{blue!10} \textbf{85.13$_{\pm \text{0.27}}$} & 87.41$_{\pm \text{0.33}}$ & \cellcolor{blue!10} \textbf{90.86$_{\pm \text{0.18}}$} \\ 
%      & RoBERTa & 87.36$_{\pm \text{1.69}}$ & \cellcolor{blue!10} \textbf{75.90$_{\pm \text{0.41}}$} & 90.38$_{\pm \text{0.09}}$ & 
%  75.28$_{\pm \text{0.31}}$ & 86.31$_{\pm \text{0.68}}$ & 68.84$_{\pm \text{0.54}}$ & \cellcolor{blue!10} \textbf{64.85$_{\pm \text{0.31}}$} & 85.13$_{\pm \text{0.41}}$ & \cellcolor{blue!10} \textbf{87.42$_{\pm \text{0.51}}$} & 90.70$_{\pm \text{0.25}}$ \\
%     & Qwen-3B & \cellcolor{blue!10} \textbf{87.36$_{\pm \text{1.10}}$} & 75.63$_{\pm \text{0.94}}$ & 89.98$_{\pm \text{0.38}}$ & 75.20$_{\pm \text{0.49}}$ & 85.58$_{\pm \text{0.58}}$ & 68.43$_{\pm \text{0.57}}$ & 63.98$_{\pm \text{0.69}}$ & 84.69$_{\pm \text{0.31}}$ & 85.17$_{\pm \text{0.44}}$ & 87.44$_{\pm \text{0.66}}$ \\ 
%     & Mistral-7B & 87.22$_{\pm \text{1.24}}$ & 75.20$_{\pm \text{1.34}}$ & 90.54$_{\pm \text{0.50}}$ & \cellcolor{blue!10} \textbf{76.18$_{\pm \text{0.34}}$} & \cellcolor{blue!10} \textbf{87.87$_{\pm \text{0.35}}$} & \cellcolor{blue!10} \textbf{69.39$_{\pm \text{0.52}}$} & 64.34$_{\pm \text{0.23}}$ & 85.01$_{\pm \text{0.49}}$ & 87.34$_{\pm \text{0.43}}$ & 90.16$_{\pm \text{0.26}}$ \\ 
%    %  \midrule
%    %  \multirow{4}{*}{ENGINE} & SenBERT & \cellcolor{blue!10} \textbf{87.50$_{\pm \text{1.56}}$} & 74.73$_{\pm \text{1.50}}$ & 88.72$_{\pm \text{0.27}}$ & 71.70$_{\pm \text{0.35}}$ & 83.86$_{\pm \text{0.67}}$ & 67.41$_{\pm \text{0.41}}$ & 70.95$_{\pm \text{0.26}}$ & 84.00$_{\pm \text{0.32}}$ & 86.44$_{\pm \text{0.12}}$ & \cellcolor{blue!10} \textbf{89.17$_{\pm \text{0.21}}$} \\ 
%    %  & RoBERTa & 86.77$_{\pm \text{1.60}}$ & 75.59$_{\pm \text{0.95}}$ & 88.70$_{\pm \text{0.47}}$ & 73.10$_{\pm \text{0.45}}$ & 83.96$_{\pm \text{0.37}}$ & 68.64$_{\pm \text{0.65}}$ & 71.14$_{\pm \text{0.23}}$ & 83.89$_{\pm \text{0.19}}$ & 86.57$_{\pm \text{0.20}}$ & 88.95$_{\pm \text{0.13}}$ \\ 
%    %  & Qwen-3B & 86.34$_{\pm \text{1.22}}$ & 74.69$_{\pm \text{1.74}}$ & 89.89$_{\pm \text{0.40}}$ & 74.14$_{\pm \text{0.25}}$ & 85.21$_{\pm \text{0.56}}$ & 68.86$_{\pm \text{0.41}}$ & 71.17$_{\pm \text{0.56}}$ & 83.81$_{\pm \text{0.15}}$ & 85.06$_{\pm \text{0.27}}$ & 88.38$_{\pm \text{0.24}}$ \\ 
%    %  & Mistral-7B & 87.00$_{\pm \text{1.60}}$ & \cellcolor{blue!10} \textbf{75.82$_{\pm \text{1.52}}$} & \cellcolor{blue!10} \textbf{90.08$_{\pm \text{0.16}}$} & \cellcolor{blue!10} \textbf{74.69$_{\pm \text{0.36}}$}
%    % & \cellcolor{blue!10} \textbf{85.44$_{\pm \text{0.53}}$} & \cellcolor{blue!10} \textbf{68.87$_{\pm \text{0.25}}$} & \cellcolor{blue!10} \textbf{71.21$_{\pm \text{0.77}}$} & \cellcolor{blue!10} \textbf{84.09$_{\pm \text{0.09}}$} & \cellcolor{blue!10} \textbf{86.98$_{\pm \text{0.06}}$} & 89.05$_{\pm \text{0.13}}$ \\
%     \bottomrule
%     \end{tabular}
%     }
%     \label{tab:encoder_comp_fullysupervised}
% \end{table*}


We supplement the comparison between LLM-as-Encoder and LM-as-Encoder under supervised settings in Table \ref{tab:encoder_comp_fullysupervised}. The key takeaway that \textbf{LLMs outperform LMs as Encoders in heterophilic graphs} remains valid. This conclusion is particularly evident on the arXiv dataset, where the performance gap between LM- and LLM-generated embeddings reaches up to $7\%$ on MLP and $3\%$ on GraphSAGE. Additionally, we observe that in supervised settings, the performance gap between LM- and LLM-as-Encoders becomes less pronounced compared to semi-supervised settings. We still consider the theoretical insights in Equation \eqref{eq:mutual_info} for explanation: Increased supervision enhances the mutual information between labels and graph structure, i.e., $I(\mathcal{E}, \mathcal{Y}_{l})$, thereby rendering the second term less significant and diminishing the advantages provided by more powerful encoders like LLMs.



\subsection{LLM-as-Predictor: Sensitivity to LLM Backbones}

\begin{table*}[!h]
    \centering
    \caption{\textbf{Sensitivity of LLaGA to different LLM backbones under semi-supervised Settings}.\\ \small{The best LLM backbone within \colorbox{red!10}{\textbf{each series}} and \colorbox{yellow!20}{\textbf{at similar scales}} is highlighted.}}
    \resizebox{\linewidth}{!}{
    \begin{tabular}{ccc|ccccccccc}
      \toprule
     \rowcolor{COLOR_MEAN} & & \textbf{LLM} & \textbf{Cora} & \textbf{Citeseer} & \textbf{Pubmed} & \textbf{WikiCS} & \textbf{Instagram} & \textbf{Reddit} & \textbf{Books} & \textbf{Photo} & \textbf{Computer} \\ \midrule
       % & & \# Train Samples & 140 & 120 & 60 & 580 & 1,160 & 3,344 & 4,155 & 4,836 & 8,722 \\ \midrule
      \multirow{8}{*}{\rotatebox[origin=c]{90}{\textbf{Accuracy (\%)}}} & \multirow{4}{*}{ \rotatebox[origin=c]{90}{\small \begin{tabular}{c}
         \textbf{Same}\\ \textbf{series}
      \end{tabular}}} & Qwen-3B  & 74.15$_{\pm \text{2.41}}$ & 62.74$_{\pm \text{10.42}}$ & 54.97$_{\pm \text{10.71}}$ & 71.64$_{\pm \text{1.34}}$ & \cellcolor{red!10}\textbf{61.35$_{\pm \text{2.35}}$} & 65.11$_{\pm \text{1.59}}$ & 82.26$_{\pm \text{0.43}}$ & \cellcolor{red!10} \textbf{83.85$_{\pm \text{0.77}}$} & \cellcolor{red!10} \textbf{85.84$_{\pm \text{1.12}}$} \\ 
     & & Qwen-7B  & 74.23$_{\pm \text{1.58}}$ & 64.79$_{\pm \text{1.77}}$ & \cellcolor{red!10} \textbf{62.58$_{\pm \text{1.36}}$} & 71.40$_{\pm \text{2.28}}$ & 59.09$_{\pm \text{3.32}}$ & 66.07$_{\pm \text{0.32}}$ & 81.38$_{\pm \text{1.58}}$ & 80.75$_{\pm \text{1.60}}$ & 84.47$_{\pm \text{1.73}}$ \\ 
     & & Qwen-14B & 76.63$_{\pm \text{1.79}}$ & \cellcolor{red!10} \textbf{66.06$_{\pm \text{1.86}}$} & 62.17$_{\pm \text{6.86}}$ & \cellcolor{red!10} \textbf{73.76$_{\pm \text{0.42}}$} & 61.14$_{\pm \text{3.84}}$ & \cellcolor{red!10} \textbf{66.59$_{\pm \text{1.23}}$} & 81.40$_{\pm \text{0.20}}$ & 82.47$_{\pm \text{0.80}}$ & 85.08$_{\pm \text{0.36}}$  \\ 
     & & Qwen-32B & \cellcolor{red!10} \textbf{77.01$_{\pm \text{3.62}}$} & 64.01$_{\pm \text{2.59}}$ & 58.60$_{\pm \text{7.93}}$ & 71.31$_{\pm \text{3.05}}$ & 60.24$_{\pm \text{4.03}}$ & 66.19$_{\pm \text{2.11}}$ & \cellcolor{red!10} \textbf{82.34$_{\pm \text{0.44}}$} & 82.85$_{\pm \text{1.52}}$ & 85.74$_{\pm \text{1.65}}$ \\  \cmidrule(r){2-12}
     &  \multirow{3}{*}{\rotatebox[origin=c]{90}{\small \begin{tabular}{c}
          \textbf{Similar}\\ \textbf{scales}
     \end{tabular}}} & Mistral-7B & \cellcolor{yellow!20}\textbf{78.94$_{\pm \text{1.14}}$} & 62.61$_{\pm \text{3.63}}$ & \cellcolor{yellow!20}\textbf{65.91$_{\pm \text{2.09}}$} & \cellcolor{yellow!20}\textbf{76.47$_{\pm \text{2.20}}$} & \cellcolor{yellow!20}\textbf{65.84$_{\pm \text{0.72}}$} & \cellcolor{yellow!20}\textbf{70.10$_{\pm \text{0.38}}$} & \cellcolor{yellow!20}\textbf{83.47$_{\pm \text{0.45}}$} & \cellcolor{yellow!20}\textbf{84.44$_{\pm \text{0.90}}$} & \cellcolor{yellow!20}\textbf{87.82$_{\pm \text{0.53}}$} \\ 
    & & Qwen-7B  & 74.23$_{\pm \text{1.58}}$ & \cellcolor{yellow!20}\textbf{64.79$_{\pm \text{1.77}}$} & 62.58$_{\pm \text{1.36}}$ & 71.40$_{\pm \text{2.28}}$ & 59.09$_{\pm \text{3.32}}$ & 66.07$_{\pm \text{0.32}}$ & 81.38$_{\pm \text{1.58}}$ & 80.75$_{\pm \text{1.60}}$ & 84.47$_{\pm \text{1.73}}$ \\ 
   & & LLaMA-8B  & 75.34$_{\pm \text{1.09}}$ & 61.33$_{\pm \text{2.11}}$ & 61.84$_{\pm \text{3.62}}$ & 72.15$_{\pm \text{3.32}}$ & 55.77$_{\pm \text{3.07}}$ & 65.09$_{\pm \text{1.39}}$ & 81.30$_{\pm \text{0.07}}$ & 82.26$_{\pm \text{1.67}}$ & 86.43$_{\pm \text{0.25}}$ \\ \midrule

    
    \multirow{8}{*}{\rotatebox[origin=c]{90}{\textbf{Macro-F1 (\%)}}} & \multirow{4}{*}{ \rotatebox[origin=c]{90}{\small \begin{tabular}{c}
         \textbf{Same}\\ \textbf{series}
      \end{tabular}}} & Qwen-3B & 64.48$_{\pm \text{4.32}}$ & 53.25$_{\pm \text{2.21}}$ & 44.35$_{\pm \text{3.64}}$ & 59.09$_{\pm \text{3.00}}$ & 51.92$_{\pm \text{8.02}}$ & 42.42$_{\pm \text{0.50}}$ & \cellcolor{red!10}\textbf{37.56$_{\pm \text{2.56}}$} & \cellcolor{red!10} \textbf{70.94$_{\pm \text{0.97}}$} & \cellcolor{red!10} \textbf{70.65$_{\pm \text{3.31}}$} \\ 
     & & Qwen-7B  & 67.30$_{\pm \text{5.70}}$ & 53.43$_{\pm \text{1.63}}$ & \cellcolor{red!10} \textbf{47.55$_{\pm \text{1.16}}$} & 58.97$_{\pm \text{3.67}}$ & 49.78$_{\pm \text{8.13}}$ & \cellcolor{red!10} \textbf{51.32$_{\pm \text{10.10}}$} & 34.33$_{\pm \text{4.37}}$ & 66.94$_{\pm \text{4.64}}$ & 69.23$_{\pm \text{3.26}}$ \\ 
     & & Qwen-14B  & 67.85$_{\pm \text{4.53}}$ & \cellcolor{red!10} \textbf{55.39$_{\pm \text{3.99}}$} & 45.90$_{\pm \text{7.69}}$ & \cellcolor{red!10} \textbf{64.10$_{\pm \text{0.24}}$} & \cellcolor{red!10} \textbf{55.40$_{\pm \text{1.26}}$} & 44.35$_{\pm \text{0.86}}$ &37.50$_{\pm \text{0.35}}$ & 70.33$_{\pm \text{1.12}}$ & 69.19$_{\pm \text{5.16}}$ \\ 
     & & Qwen-32B & \cellcolor{red!10} \textbf{68.27$_{\pm \text{6.24}}$} & 53.11$_{\pm \text{2.59}}$ & 46.52$_{\pm \text{7.54}}$ & 60.50$_{\pm \text{3.40}}$ & 39.50$_{\pm \text{8.07}}$ & 43.96$_{\pm \text{1.45}}$ & 35.30$_{\pm \text{1.80}}$ & 70.08$_{\pm \text{2.39}}$ & 67.26$_{\pm \text{3.89}}$ \\  \cmidrule(r){2-12}
     &  \multirow{3}{*}{\rotatebox[origin=c]{90}{\small \begin{tabular}{c}
          \textbf{Similar}\\ \textbf{scales}
     \end{tabular}}} & Mistral-7B & \cellcolor{yellow!20}\textbf{71.14$_{\pm \text{4.47}}$} & 52.53$_{\pm \text{3.59}}$ & 45.12$_{\pm \text{7.63}}$ & \cellcolor{yellow!20}\textbf{70.48$_{\pm \text{2.94}}$} & \cellcolor{yellow!20}\textbf{50.12$_{\pm \text{10.45}}$} & \cellcolor{yellow!20}\textbf{54.67$_{\pm \text{11.24}}$} & \cellcolor{yellow!20}\textbf{39.70$_{\pm \text{2.44}}$} & \cellcolor{yellow!20}\textbf{79.32$_{\pm \text{2.42}}$} & \cellcolor{yellow!20}\textbf{78.01$_{\pm \text{1.36}}$}  \\ 
    & & Qwen-7B  & 67.30$_{\pm \text{5.70}}$ & \cellcolor{yellow!20}\textbf{53.43$_{\pm \text{1.63}}$} & 47.55$_{\pm \text{1.16}}$ & 58.97$_{\pm \text{3.67}}$ & 49.78$_{\pm \text{8.13}}$ & 51.32$_{\pm \text{10.10}}$ & 34.33$_{\pm \text{4.37}}$ & 66.94$_{\pm \text{4.64}}$ & 69.23$_{\pm \text{3.26}}$  \\ 
   & & LLaMA-8B & 67.50$_{\pm \text{4.73}}$ & 51.22$_{\pm \text{1.27}}$ & \cellcolor{yellow!20}\textbf{47.80$_{\pm \text{2.78}}$} & 64.17$_{\pm \text{6.00}}$ & 48.56$_{\pm \text{6.92}}$ & 43.31$_{\pm \text{0.94}}$ & 34.49$_{\pm \text{1.48}}$ & 72.45$_{\pm \text{0.35}}$ & 71.43$_{\pm \text{4.43}}$ \\ 

      \bottomrule
    \end{tabular}
    }
    \label{tab:llaga_llm}
\end{table*}

\begin{figure}[!h]
    \centering
    \begin{subfigure}[b]{0.25\textwidth}
        \centering
        \includegraphics[width=\textwidth]{figs/exp_new/qwen_cora.pdf}        
        \caption{Cora}
        \label{fig:qwen_cora}
    \end{subfigure}
    \begin{subfigure}[b]{0.25\textwidth}
        \centering
        \includegraphics[width=\textwidth]{figs/exp_new/qwen_wikics.pdf}
        \caption{WikiCS}
        \label{fig:qwen_citeseer}
    \end{subfigure}
    \begin{subfigure}[b]{0.25\textwidth}
        \centering
        \includegraphics[width=\textwidth]{figs/exp_new/qwen_instagram.pdf}
        \caption{Instagram}
        \label{fig:qwen_instagram}
    \end{subfigure}
    \caption{\textbf{Performance trends within Qwen-series in different scales using LLaGA framework in semi-supervised settings.}}
    \label{fig:llaga_scaling}
\end{figure}


\begin{figure}[!h]
    \centering
    \begin{subfigure}[b]{0.25\textwidth}
        \centering
        \includegraphics[width=\textwidth]{figs/exp_new/qwen_cora_s.pdf}        
        \caption{Cora}
        \label{fig:qwen_cora_s}
    \end{subfigure}
    \begin{subfigure}[b]{0.25\textwidth}
        \centering
        \includegraphics[width=\textwidth]{figs/exp_new/qwen_wikics_s.pdf}
        \caption{WikiCS}
        \label{fig:qwen_citeseer_s}
    \end{subfigure}
    \begin{subfigure}[b]{0.25\textwidth}
        \centering
        \includegraphics[width=\textwidth]{figs/exp_new/qwen_instagram_s.pdf}
        \caption{Instagram}
        \label{fig:qwen_instagram_s}
    \end{subfigure}
    \caption{\textbf{Performance trends within Qwen-series in different scales using LLaGA framework in supervised settings.}}
    \label{fig:llaga_scaling_s}
\end{figure}

\begin{table}[!h]
    \centering
    \caption{\textbf{Training and inference times of Qwen-series models ranging from 3B to 32B parameters.}}
    \resizebox{\textwidth}{!}{
    \begin{tabular}{cc|ccc|ccc|ccc}
       \toprule
        \rowcolor{COLOR_MEAN} &  & \multicolumn{3}{c|}{\textbf{Semi-supervised Training Times}} & \multicolumn{3}{c}{\textbf{Supervised Training Times}} & \multicolumn{3}{|c}{\textbf{Avg. Inference Times Per Case}} \\
       \rowcolor{COLOR_MEAN} \multirow{-2}{*}{\textbf{GPU Device}}  & \multirow{-2}{*}{\textbf{LLM}} & Cora & WikiCS & Instagram  &  Cora & WikiCS & Instagram &  Cora & WikiCS & Instagram \\  \midrule 
        &  Qwen-3B & 2.2min & 7.1min & 5.8min & 8.6min & 33.5min & 20.2min & 32.3ms & 37.3ms & 26.9ms \\ 
        \multirow{-2}{*}{1 NVIDIA A6000-48G} & Qwen-7B & 4.7min & 15.3min & 8.2min & 13.3min & 59.4min & 43.2min & 50.9ms & 55.9ms & 43.4ms \\ 
       & Qwen-14B & 8.9min & 25.5min & 15.7min & 30.8min & 2.3h & 1.5h & 97.6ms & 103.0ms & 83.2ms \\ 
        \multirow{-2}{*}{2 NVIDIA A6000-48G} & Qwen-32B & 18.9min & 43.6min & 30.7min & 52.2min & 3.3h & 2.0h & 254.7ms & 262.6ms & 232.4ms \\ 
       \bottomrule
    \end{tabular}
    }
    \label{tab:qwen_cost}
\end{table}


We evaluate LLaGA with different LLM backbones in semi-supervised settings, as detailed in Table \ref{tab:llaga_llm}. We examine two primary trends: (1) \textbf{Scaling within the same series}: Assessing whether scaling laws apply to node classification tasks by using LLMs from the same series, and (2) \textbf{Model selection at similar scales}: Identifying the most suitable LLM for node classification tasks by comparing models of similar scales.

\textbf{Scaling within the same series}: We plot performance trends across several datasets under both semi-supervised and supervised settings to clearly illustrate these dynamics. From Figure \ref{fig:llaga_scaling} and Figure \ref{fig:llaga_scaling_s}, we conclude that scaling laws generally hold: as the Qwen model size increases from 3B to 32B parameters, performance improves, indicating that larger model sizes enhance task performance. However, the 7B and 14B models are sufficiently large, typically representing the point beyond which further increases in model size yield only marginal improvements but introducing huge computational costs (Table \ref{tab:qwen_cost}). Unexpectedly, in the Instagram dataset under semi-supervised settings, the Qwen-32B model experiences a performance drop. This may be because 32B models require extensive data to train effectively, making them less robust and stable compared to smaller models. Based on these findings, we recommend the 7B or 14B models as they offer an optimal balance between performance and computational costs.

\textbf{Model selection at similar scales:} By comparing the performance of Mistral-7B, Qwen-7B, and LLaMA-8B in Table \ref{tab:llaga_llm}, we conclude that Mistral-7B outperforms the other two similarly scaled LLMs in most cases. This makes Mistral-7B the optimal choice as a backbone LLM for node classification tasks.



\clearpage
\newpage
\subsection{LLM-as-Predictor: Biased and Hallucinated Predictions}\label{sec:llm_bias_pred}

During our experiments, we found that LLM-as-Predictor methods are vulnerable to limited supervision. In addition to standard metrics such as Accuracy (Table \ref{tab:mainexp}) and Macro-F1 scores (Table \ref{tab:mainexp_f1}), their predictions also exhibit significant  \textbf{biases} and \textbf{hallucinations}.

\textbf{Biased Predictions:} For datasets with fewer labels, LLM-as-Predictor methods tend to disproportionately predict certain labels while under-predicting others. To illustrate this phenomenon, we compare the ground-truth label distributions with the predicted label distributions. Specifically, we present different LLM-as-Predictor methods, LLM\textsubscript{IT}, GraphGPT, and LLaGA, in Figure \ref{fig:instagram_predictor}, and the LLaGA method with various LLM backbones in Figure \ref{fig:instagram_llm}, using the Instagram dataset in semi-supervised settings, which has two labels.

From Figure \ref{fig:instagram_predictor}, we can directly observe that LLaGA and GraphGPT predominantly bias towards the first class, while LLM\textsubscript{IT} tends to predict the second class more frequently. The predicted label distributions of LLMs are more \textbf{polarized} compared to the ground-truth distributions, where the two labels are roughly in a $6:4$ proportion. In contrast, LLMs tend to predict in ratios such as $8:2$ or $1:9$. This observation also holds across different LLMs, as shown in Figure \ref{fig:instagram_llm}, where both Qwen-7B and LLaMA-8B tend to bias towards the first label. A similar example on the Pubmed dataset, which contains three classes in a semi-supervised setting, is shown in Figure \ref{fig:distribution_pubmed}. Here, the predictor methods tend to bias towards the third class, while LLaGA with Qwen-7B tends to predict the second class. Additionally, in the semi-supervised setting for Pubmed, the training data consists of only $60$ samples, which is insufficient to train a robust predictor model, leading to high levels of hallucinations across all methods.

\begin{figure}[!t]
    \centering
    \begin{subfigure}[b]{0.4\textwidth}
        \centering
        \includegraphics[width=\textwidth]{figs/exp_new/instagram_predictor.pdf}
        \caption{Different LLM-as-Predictor Methods}
        \label{fig:instagram_predictor}
    \end{subfigure}
    \begin{subfigure}[b]{0.4\textwidth}
        \centering
        \includegraphics[width=\textwidth]{figs/exp_new/instagram_llm.pdf}
        \caption{Different LLM Backbones within LLaGA}
        \label{fig:instagram_llm}
    \end{subfigure}
    \caption{\textbf{Biased predictions by LLM-as-Predictor methods on the Instagram dataset:} Comparison of ground-truth label distributions with predictor-generated label distributions.}
    \label{fig:distribution_instagram}
\end{figure}

\begin{figure}[!t]
    \centering
    \begin{subfigure}[b]{0.38\textwidth}
        \centering
        \includegraphics[width=\textwidth]{figs/exp_new/pubmed_predictor.pdf}
        \caption{Different LLM-as-Predictor Methods}
        \label{fig:pubmed_predictor}
    \end{subfigure}
    \begin{subfigure}[b]{0.38\textwidth}
        \centering
        \includegraphics[width=\textwidth]{figs/exp_new/pubmed_llm.pdf}
        \caption{Different LLM Backbones within LLaGA}
        \label{fig:pubmed_llm}
    \end{subfigure}
    \caption{\textbf{Biased predictions by LLM-as-Predictor methods on the Pubmed dataset.}}
    \label{fig:distribution_pubmed}
\end{figure}


\textbf{Hallucinations: }In addition to biased predictions, we observed that a certain portion of the LLMs' outputs \textbf{fall outside the valid label space} or \textbf{contain unexpected content that cannot be parsed}. In semi-supervised settings, the limited training data restricts these predictor methods from developing effective models, resulting in failures to follow instructions and understand dataset-specific classification requirements. To illustrate that, we provide both quantitative and qualitative analyses as follows:

\begin{itemize}
    \item \textbf{Quantitative Analysis: }Table \ref{tab:predictor_hall} presents the hallucination rates of each LLM-as-Predictor method across various experimental datasets in both semi-supervised and supervised settings. The hallucination rate is calculated as the proportion of outputs containing invalid labels or unexpected content among all test cases, where higher values indicate poorer classification performance. Hallucinations are most severe on the Pubmed and Citeseer datasets within semi-supervised settings, where the number of training samples does not exceed hundreds, making effective model training challenging. \textbf{This demonstrates that the number of training samples significantly impacts the mitigation of hallucinations}: even in semi-supervised settings, larger datasets like Books and Photo provide thousands of training samples, resulting in hallucination ratios consistently below $1\%$. Therefore, this further verifies that LLM-as-Predictor methods require extensive labeled data for effective model training.
    
    \item \textbf{Qualitative Analysis: }We provide several examples to facilitate the comprehension of hallucinated predictions, which we categorize into three types: (1) misspellings of existing labels, (2) generation of non-existent types, and (3) unexpected content that cannot be parsed. Illustrative examples of these types from GraphGPT's outputs on the Citeseer and Pubmed datasets are presented in Table \ref{tab:predictor_hall_example}.
    
\end{itemize}

\begin{table}[!t]
    \centering
    \caption{\textbf{Average hallucination ratios ($\%$) of LLM-as-Predictor methods.} The hallucination rate is calculated as the proportion of outputs containing invalid labels or unexpected content across all test cases, where higher values indicate poorer classification ability. Hallucinations $>1\%$ are \textcolor{brown}{\textbf{highlighted}}.}
    \vspace*{-8pt}
    \resizebox{0.9\linewidth}{!}{
      \begin{tabular}{cc|ccccccccc}
        \toprule
        \rowcolor{COLOR_MEAN} \textbf{Setting} &  \textbf{Method} & \textbf{Cora} & \textbf{Citeseer} & \textbf{Pubmed} & \textbf{WikiCS} & \textbf{Instagram} & \textbf{Reddit} & \textbf{Books} & \textbf{Photo} & \textbf{Computer}  \\ \midrule
       \multirow{4}{*}{\textbf{Semi-supervised}} &  \# Train Samples & 140 & 120 & 60 & 580 & 1,160 & 3,344 & 4,155 & 4,836 & 8,722 \\ 
       & {LLM\textsubscript{IT}} & 0.43 & \textcolor{brown}{\textbf{13.08}} & \textcolor{brown}{\textbf{9.24}} & 0.06 & 0.00 & 0.00 & 0.01 & 0.02 & 0.02  \\ 
       & {GraphGPT} & \textcolor{brown}{\textbf{7.56}} & \textcolor{brown}{\textbf{2.51}} & \textcolor{brown}{\textbf{15.97}} & \textcolor{brown}{\textbf{7.76}} & 0.28 & \textcolor{brown}{\textbf{1.78}} & 0.51 & 0.72 & 0.27 \\
       &  LLaGA & 0.35 & 0.20 & 0.29 & 0.00 & 0.01 & 0.02 & 0.00 & 0.00 & 0.00 \\  \midrule

       \multirow{4}{*}{\textbf{Supervised}} & \# Train Samples & 1,624 & 1,911 & 11,830 &  7,020 & 6,803 & 20,060 & 24,930 & 29,017 & 52,337 \\ % \cmidrule(r){2-11} 
       & LLM\textsubscript{IT} & 0.06 & \textcolor{brown}{\textbf{13.61}} & 0.00 & 0.00 & 0.00 & 0.00 & 0.00 & 0.01 & 0.02 \\ 
       & GraphGPT & \textcolor{brown}{\textbf{1.29}} & 0.63 & 0.08 & \textcolor{brown}{\textbf{1.64}} & 0.13 & 0.50 & 0.11 & 0.11 & 0.10 \\ 
       & LLaGA & 0.03 & 0.00 & 0.00 & 0.00 & 0.00 & 0.01 & 0.00 & 0.00 & 0.00 \\
       
         \bottomrule
    \end{tabular}
    }
    \label{tab:predictor_hall}
\end{table}

\begin{table}[!t]
    \centering
    \caption{\textbf{Examples of hallucinations in GraphGPT's outputs on the Citeseer and Pubmed datasets.}}
     \vspace*{-8pt}
    \resizebox{\linewidth}{!}{
    \begin{tabular}{c|cc|cc}
      \toprule
      \rowcolor{COLOR_MEAN}  &  \multicolumn{2}{c|}{\textbf{Citeseer}} & \multicolumn{2}{c}{\textbf{Pubmed}} \\ 
        \rowcolor{COLOR_MEAN} \multirow{-2}{*}{\textbf{Error Type}} & \textbf{Prediction} & \textbf{Ground-truth} &  \textbf{Prediction} & \textbf{Ground-truth} \\ \midrule
        \textbf{Misspelling} & AGents & Agents & Type II diabetes & Type 2 diabetes \\ \midrule

        \multirow{4}{*}{\begin{tabular}{c}
             \textbf{Non-existent} \\ \textbf{Types}
        \end{tabular}} & Logic and Mathematics & ML (Machine Learning) & Type 3 diabetes of the young (MODY) & Type 2 diabetes \\
        & Information Extraction &	IR (Information Retrieval) & Genetic Studies of Wolfram Syndrome & Type 2 diabetes\\ 
        & Pattern Recognition & ML (Machine Learning) & Experimentally induced insulin resistance & Type 2 diabetes \\ 
        & Multiagent Systems & Agents & Experimentally induced oxidative stress & Experimentally induced diabetes \\ \midrule

        \multirow{2}{*}{\begin{tabular}{c}
             \textbf{Unexpected} \\ \textbf{Contents}
        \end{tabular}} & \multicolumn{2}{c|}{H.4.1 Office Automation: Workflow Management} & \multicolumn{2}{c}{membrane is not altered by diabetes.} \\ 

        & \multicolumn{2}{c|}{The citation graph is given by the following: ...} & \multicolumn{2}{c}{What is the sensitivity and specificity of the IgA-EMA test ...} \\

        \bottomrule
    \end{tabular}
    }
    \label{tab:predictor_hall_example}
\end{table}


\clearpage 
\newpage

\section{Supplementary Materials for Computational Cost Analysis}\label{sec:detail_cost}

\begin{table}[!h]
    \centering
    \caption{\textbf{Total training times of different methods in semi-supervised settings.} All recorded experiment times are based on a single NVIDIA H100-80G GPU.}
    \resizebox{\linewidth}{!}{

      \begin{tabular}{cc|ccccccccc}
       \toprule
       \rowcolor{COLOR_MEAN}  \textbf{Type} & \textbf{Method} & \textbf{Cora} & \textbf{Citeseer} & \textbf{Pubmed} & \textbf{WikiCS} & \textbf{Instagram} & \textbf{Reddit} & \textbf{Books} & \textbf{Photo} & \textbf{Computer} \\ \midrule
       \multicolumn{2}{c|}{\# Training Samples} & 140 & 120 & 60 & 580 & 1,160 & 3,344 & 4,155 & 4,836 & 8,722 \\ \midrule
       \multirow{5}{*}{\textbf{Classic}} & GCN$_{\text{ShallowEmb}}$ & 2.8s & 2.7s & 2.7s & 3.2s & 1.8s & 7.7s & 12.7s & 13.8s & 33.5s \\ 
       & {GAT$_{\text{ShallowEmb}}$} & 1.9s & 2.4s & 3.8s & 3.3s & 2.0s & 6.0s & 10.5s & 12.3s & 38.0s \\ 
       & {SAGE$_{\text{ShallowEmb}}$} & 1.9s & 3.9s & 5.0s & 2.9s & 1.8s & 6.4s & 16.9s & 21.1s & 33.3s \\ 
       & {SenBERT-66M} & 8.5s & 7.9s & 5.9s & 27.9s & 14.7s & 1.2m & 1.5m & 1.8m & 3.3m \\ 
       & {RoBERTa-355M} & 21.2s & 18.9s & 12.7s & 1.2m & 2.3m & 6.5m & 8.1m & 3.8m & 6.9m \\ \midrule 

      \multirow{2}{*}{\textbf{Encoder}} & GCN$_{\text{LLMEmb}}$ & 1.2m & 1.4m & 13.4m & 7.4m & 4.5m & 16.0m & 23.5m & 26.8m & 44.7m \\ 
      & ENGINE  & 2.2m & 2.4m & 16.1m & 15.2m & 9.3m & 22.9m & 31.1m & 38.8m & 1.1h \\ \midrule

      \textbf{Reasoner} & TAPE & 25.5m & 27.8m & 5.6h & 2.7h & 2.0h & 8.0h & 9.9h & 11.7h & 14.5h \\ \midrule

      \multirow{3}{*}{\textbf{Predictor}} & {LLM$_{\text{IT}}$} & 25.6m & 22.0m & 3.9m & 1.1h & 49.1m & 1.1m & 2.0h & 2.4h & 2.7h \\ 
      & GraphGPT & 16.4m & 15.5m & 1.2h & 48.5m & 30.1m & 1.8h & 2.4h & 2.2h & 5.8h \\ 
       & LLaGA & 1.7m & 2.2m & 5.2m & 5.8m & 3.0m & 20.9m & 19.5m & 23.5m & 43.1m  \\    
        \bottomrule
      \end{tabular}
    }
    \label{tab:timecost}
\end{table}


\begin{table}[!h]
    \centering
    \caption{\textbf{Total training times of different methods in supervised settings.} All recorded experiment times are based on a single NVIDIA H100-80G GPU.}
    \resizebox{\linewidth}{!}{

      \begin{tabular}{cc|cccccccccc}
       \toprule
       \rowcolor{COLOR_MEAN}  \textbf{Type} & \textbf{Method} & \textbf{Cora} & \textbf{Citeseer} & \textbf{Pubmed} & \textbf{WikiCS} & \textbf{arXiv} & \textbf{Instagram} & \textbf{Reddit} & \textbf{Books} & \textbf{Photo} & \textbf{Computer} \\ \midrule
       \multicolumn{2}{c|}{\# Training Samples} & 1,624 & 1,911 & 11,830 & 7,020 &  90,941  & 6,803 & 20,060 & 24,930 & 29,017 & 52,337  \\ \midrule
       \multirow{5}{*}{\textbf{Classic}} & {GCN$_{\text{ShallowEmb}}$} & 1.8s & 1.7s & 5.2s & 5.1s & 51.2s & 19.5s & 8.5s & 14.9s & 19.7s & 25.8s \\ 
       & {GAT$_{\text{ShallowEmb}}$} & 2.1s & 1.9s & 7.9s & 5.7s & 1.5m & 2.7s & 6.9s & 16.6s & 28.0s & 44.6s \\ 
       & {SAGE$_{\text{ShallowEmb}}$} & 1.7s & 3.0s & 7.6s & 4.0s & 1.3m & 2.0s & 7.2s & 19.6s & 20.1s & 43.2s \\ 
       & {SenBERT-66M} &  35s & 41s & 2.6m & 2.5m & 7.4m & 1.2m & 4.4m & 1.8m & 2.2m & 4.1m \\ 
       & {RoBERTa-355M} & 1.3m & 1.6m & 9.2m & 5.5m & 40.8m & 5.3m & 15.9m & 9.7m & 11.9m & 22.4m \\ \midrule 

      \multirow{2}{*}{\textbf{Encoder}} & GCN$_{\text{LLMEmb}}$ & 1.2m & 1.4m & 13.4m & 7.5m & 1.4h & 4.5m & 16.1m & 23.6m & 26.8m & 44.8m  \\ 
      & ENGINE & 2.2m & 2.4m & 16.1m & 19.4m & 2.6h & 8.9m & 24.2m & 35.2m & 44.2m & 1.2h \\ \midrule

      \textbf{Reasoner} & TAPE & 27.4m & 30.3m & 5.9h & 2.8h & 37.4h &  2.1h & 8.3h & 10.0h & 12.0h & 15.0h \\ \midrule

      \multirow{3}{*}{\textbf{Predictor}} & {LLM$_{\text{IT}}$} & 1.0h & 1.3h & 9.9h & 4.2h & 36.3h & 2.7h & 3.4h & 5.7h & 7.4h & 12.4h \\ 
      & GraphGPT & 26.4m & 29.5m & 2.7h & 1.7h & 7.8h & 49.1m & 3.4h & 3.8h & 3.6h & 7.8h \\ 
       & LLaGA &  5.6m & 7.7m & 25.6m & 18.8m & 7.7h & 10.6m & 32.2m & 1.0h & 1.4h & 2.5h \\    
        \bottomrule
      \end{tabular}
    }
    \label{tab:timecost_supervised}
\end{table}


\begin{table}[!h]
    \centering
    \caption{\textbf{Inference times of different methods.} Values in brackets denote the average inference time per case in milliseconds (ms). All recorded experiment times are based on a single NVIDIA H100-80G GPU.}
    \resizebox{0.8\linewidth}{!}{
     \begin{tabular}{cc|ccccc}
     \toprule
        \rowcolor{COLOR_MEAN} \multicolumn{2}{c|}{\textbf{Method}} & \textbf{Cora} & \textbf{arXiv}  & \textbf{Instagram} & \textbf{Photo} & \textbf{WikiCS} \\ \midrule
        \multicolumn{2}{c|}{\# Test Samples} &  542 & 48,603 & 5,847 & 2,268 & 9,673 \\ \midrule
        \textbf{Classic} & GCN & 0.9ms & 21.8ms & 2.0ms & 7.5ms & 4.4ms \\ \midrule 
        \textbf{Encoder} & GCN$_{\text{LLMEmb}}$ & 14.0s (26ms) & 23.8m (29ms) & 53.6s (24ms) & 5.3m (33ms) & 3.7m (38ms) \\  \midrule
        \textbf{Reasoner} & TAPE & 5.0m (551ms) & 10.4h (767ms) & 23.7m (627ms) & 2.3h (863ms) & 1.3h (813ms) \\ \midrule
        \multirow{3}{*}{\textbf{Predictor}} & LLM$_{\text{IT}}$ & 1.2m (129ms) & 3.3h (243ms) & 2.7m (71ms) & 24.1m (149ms) & 5.8m (60ms) \\ 
        & GraphGPT & 1m (104ms) & 1.2h (87ms) & 2.0m (52ms) & 10.4m (64ms) & 11.0m (112ms) \\ 
        & LLaGA & 11.2s (21ms) & 57.1m (70ms) & 1.3m (35ms) & 4.4m (27ms) & 2.4m (25ms) \\
        \bottomrule
    \end{tabular}
    }
    \label{tab:inference_cost}
\end{table}




\end{document}

