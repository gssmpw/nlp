Generative AI has further embroiled society in a “data episteme'' \cite{Koopman2019-ef}, where we know ourselves through the acquisition of data and our knowledge can only be validated by gathering more and more data.  Where we began with indexing ourselves, our economies, and our institutions, we are now in a moment where culture is ``content''; moments of ingenuity, mundanity, and deviance abstracted to records of human activity suitable for extraction.  Statistical and algorithmic interpretations of the world---models and training data corpora---are aggressively promoted as ``reality'', absent acknowledgement of the libidinal tensions suffusing the economic and political ends for which these representations are being deployed.

A concern with the moral, expressive, and contemplative aspects of living \cite{mesthene_1969} are central to humanistic inquiry.  But AI recenters inquiry as the interpretation and synthesis of data, rather than a contemplation of the world. This recentering depends on the data episteme’s enactment of the premises of European modernity: the abstraction of resources---including being---in order to yield quantifiable measures of progress, efficiency at someone else’s expense.  For example, Amazon’s  practice of algorithmically monitoring workers fosters dreadful work conditions in pursuit of ``frictionless'', ``timely'' deliveries.

Charles Mills (2021), writing on antiblackness, offers a startling assessment of what it means to be modern. For Mills, antiblackness is foundational to the systematized racial subordination of chattel slavery that is essential to European modernity’s function and structure.  He writes, ``race becomes the signifier of full or diminished humanity, a signifier that is enforced by material practices in a modern racialized world'' (2018, 28).  The Enlightenment project of equality, liberalism, capital, and democracy was structured by antiblackness and colonialism, shaping the modern world by demarcating who is human through information, commerce, and citizenship.  Ramon Amaro (Amaro )(cite) expands antiblackness to computer vision, writing ``By regressing complex environmental data into a generalized pattern, the lived and multivalent specificities of black lives are represented ``as if'' the visual matrix maintains no connections to historical category, stereotype, or a moral imaginary'' (2019, 4). As a result, ``research in computation is an adaptation of the fictive and compulsive ordering of human attributes into a single coherent image of species'' (Amaro 2019, 5).  

The internet’s early days featured similar promises of liberal ideals: freedom of expression, freedom of identity, freedom of association.  As Silicon Valley sought venture capital, however, their promises of free speech and expression were by the dictates of Wall Streets, but the last decade has seen those aesthetic, inventive, and quotidian digital expressions be reframed as ``content''.  Where AI vision reduces the visual complexity of humanity, LLMs similarly distort the inventive and cultural capacities of culture to ``content''.  These abstractions afford AI the coercive claim that they are a ‘universal knowledge’ machine, rather than a statistical enactment of the ideology that ``one can improve --- control --- a deviant subpopulation by enumeration and classification'' (Hacking 1990/2013, (Hacking 2013, page 3) 
 
While AI proponents declaim that ``creativity'' and ``freedom of speech'' are essential to artificial general intelligence, their theft and expropriation of copyrighted works, public and private discourse, and user-generated multimedia as ``data'' should not be solely understood as neoliberal capitalism (although that certainly applies).  Instead, humanistic inquiry offers us the alternative of seeing AI’s libidinal economic enterprise; where in addition to inheriting the antiblack capacities of Western arts, letters, and public discourse, the AI being proffered to us is deeply anti-human.

Humanities inquiry illuminates AI’s reordering of society as a network of relations between abstraction, acquisition, informational and material violence.  Although the genie is out of the bottle, humanities scholars work to redirect AI to become a convivial technology, one in service to society and culture. 
