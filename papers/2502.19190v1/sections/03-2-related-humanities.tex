The launch of ChatGPT, in November 2022, brought humanities researchers to the forefront of national conversations about generative AI and its impact on research, writing and creativity (e.g. \cite{schmidt_representation_2023,tilton_relating_2023,jones_ai_2023,crawford_archeologies_2023,broussard2023,underwood_mapping_2022}. Those in the fields of digital humanities, digital pedagogy, and the history of technology, in particular---fields that require both technical and humanistic expertise---became some of the earliest expositors of its limitations and its potential e.g. (\cite{Kirschenbaum_2023, Rogers_2023, Klein_2022}). While these fields continue to lead within the humanities---for example, the task force convened jointly by the MLA and CCCC in Spring 2023 that has since published a series of working papers on AI and writing (\cite{Byrd_Flores, Generative, Building}---the national conversation about AI has been usurped by corporate spokespeople parroting AI hype, whose claims in turn convince academic administrators to adopt costly and potentially harmful AI systems \cite{seybold_ed_2023}. In fact, the increasingly politicized relationship between for-profit corporations, public resources, public institutions (like universities and research centers) and political power should make humanists, computer scientists, and researchers broadly alarmed.  

At the same time, humanities researchers have continued to push forward research about and/or involving AI in their own fields. This includes new journals (e.g. Critical AI, edited by two literary scholars \cite{Goodlad_2023}), and themed sections and special issues of flagship humanities journals (e.g. American Literature’s special issue on ``Critical AI: A Field in Formation'' \cite{Raley_Rhee_2023}) and PMLA’s themed section on ``AI and the University as Service'' \cite{Kirschenbaum_Raley_2024}. It includes monographs (e.g. \citep{Pasquinelli, Gunkel_2012, Katz_2020, Tenen_2024}) and long-form essays that theorize the language \cite{Slater_2024} and images \cite{Wasielewski_2024} generated by LLMs, historicize and critique ML approaches such as sentiment analysis \cite{Chun_Hong_Nakamura_2024}, topic modeling \cite{Binder_2016}, and image generation \cite{Offert_Phan_2024}, analyze the output of generative AI according to literary critical \cite{Walsh_Preus_Gronski_2024} and art historical expertise \cite{Malevé_Sluis_2023}, and more. However, this valuable work is rarely read outside of humanities disciplines. The goal of this paper is to argue that this important work should have a direct impact on the development of this technology. 