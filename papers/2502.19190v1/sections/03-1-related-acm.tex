Historically, humanities research has not featured prominently within the FAccT conference. Most mentions of the humanities that appear in FAccT publications appear in the umbrella term of ``social science and humanities research,'' or because of the ACM category of ``arts and humanities'' in which papers on ML, AI, and artistic practice appear. There has been a consistent acknowledgment of the unique perspectives and areas of expertise brought by humanities researchers, however, e.g. Ganesh et al.’s paper documenting a workshop in which researchers from different disciplines were brought together to discuss issues of AI fairness \cite{Ganesh_Dechesne_Waseem_2020}, Lünich and Keller’s paper documenting a similar workshop on the topic of explainable AI \cite{Lünich_Keller_2024}, Bates et al.’s auto-ethnographic account of incorporating critical data studies perspectives into the computer science classroom \cite{Bates_2020}, and Dotan et al.’s survey of approaches to incorporating generative AI responsibly in the classroom \cite{Dotan_Parker_Radzilowicz_2024}. Most pointedly, and ironically, Raji et al.’s quantitative analysis of AI ethics syllabi across the academy, ```You Can’t Sit With Us’: Exclusionary Pedagogy in AI Ethics Education,'' makes explicit mention of multiple humanities fields in the service of an argument about how, in refusing to look outside the discipline of computer science for their assigned readings, AI ethics courses are reproducing existing academic hierarchies \cite{Raji_Scheuerman_Amironesei_2021}. 

An exception is philosophy, which is one of the core disciplines of the FAccT community. A recent retrospective of the FAccT conference categorized 11\% of papers presented at FAccT as belonging to philosophy, and philosophy was the only humanities discipline prominent enough to be categorized individually \citet{Laufer_Jain_Cooper_Kleinberg_Heidari_2022}. But why is philosophy privileged in this way, not only in publications at FAccT but also in public discourse about AI? Unlike other humanities disciplines, which, as discussed above, prioritize the human and the specific, philosophy can allow researchers to abstract away from the messiness of real humans toward scenarios that isolate variables of interest and are more easily interpretable; a tempting proposition for scientists who value more tractable theories of knowing, and a useful method that can more easily mesh with mathematical and technical solutions. We can see this pattern reflected in works published at FAccT which have brought philosophical approaches to algorithmic decision making \citep{Cooper_Moss_Laufer_Nissenbaum_2022, Susser_2022, Jain_Suriyakumar_Creel_Wilson_2024}. The contributions of philosophy to FAccT and to the field of AI have been significant, but we clarify that the humanities has much more to offer than philosophy alone.

With that said, concepts and ideas from the humanities have consistently made their way into papers published at FAccT, often in substantive and impactful ways. A 2020 CRAFT workshop centered on speculative, queer, and feminist engagements with archives \cite{Pritchard_Snodgrass_Morrison_Britton_Moll_2020} while that same year saw Jo and Gebru’s ``Lessons from Archives: Strategies for Collecting Sociocultural Data in Machine Learning,'' a paper that has only grown in significance as the issue of LLM training data has entered broad consciousness \cite{Jo_Gebru_2020}. Additional papers have focused on specific theoretical constructs from the humanities, using them to question core concepts in ML/AI research (e.g. Corbett and Denton on transparency \cite{Corbett_Denton_2023} and Stark on animation \cite{Stark_2024}), and to imagine alternatives to existing approaches (e.g. Klumbyté et al., who explore the feminist and Black feminist concepts of situatedness, figuration, diffraction, and critical fabulation to think beyond existing modeling approaches \cite{Klumbyte_Draude_Taylor_2022}. Indeed, feminist and gender theories have perhaps received the most substantial attention at FAccT, e.g. \cite{Devinney_Björklund_Björklund_2022} and \cite{Klein_D’Ignazio_2024}. But engagement with other humanities fields and practices---most notably, art history and art practice, can be found as well, e.g. \cite{Huang_Liem_2022}, \cite{Divakaran_Sridhar_Srinivasan_2023}, and \cite{Srinivasan_2024}. We contribute to this work by documenting the core methods and contributions of humanities research and synthesizing the major concepts that carry across its fields. 