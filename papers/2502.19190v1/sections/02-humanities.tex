The humanities have an extensive institutional and disciplinary history \cite{bod_new_2016,noauthor_philology_nodate,reitter_permanent_2021}, but here we focus on humanities research as it is practiced in the present: to study the humanities is to investigate the human: to investigate people and groups, and the cultural objects they create \cite{small_introduction_2013,haufe_humanities_2024}. We do so in order to understand how these cultural objects create or reflect new forms of knowledge. In order to do so, we are trained in the history of knowledge production within our individual and overlapping fields of expertise. We are also trained in multiple languages and are trained to translate cultural and historical meaning across objects, audiences, and in an array of forms. 

Humanities research requires both minute specificity and sweeping breadth. To engage with Claude McKay's ``Constab Ballads'' (1912), a researcher must first read both normative English and Jamaican dialect filled with elisions and diacritical marks. They must then be able to analyze poetry in the context of Caribbean history, American literature, and Black studies. They must know both the material culture that informs the work's physical form as well as text encoding and web development. The humanities encompass the range of cultural objects that humanities researchers analyze, including language, history, philosophy, theory, aesthetics, and phenomena. Humanities researchers are trained to understand these concepts by examining and interpreting specific examples, and by asking what ways of knowing and thinking (individually and collectively) might have gone into their creation \cite{noauthor_claude_nodate}.

What methods do we employ to conduct this work? Methods shared across humanities fields include theorization, interpretation, contextualization---and, increasingly but not universally, collaboration. Some concrete examples might include: archival research, textual explication, disciplinary techniques such as close reading (literary studies), historical synthesis (historical fields), and media-specific analysis (film and media studies), among others. These methods are what generate the bulk of the evidence that appears in humanities scholarship. However, what more commonly travels to technical fields are the theories that this evidence points towards. These are humanistic theories: written articulations of complex ideas that, until that point, we did not yet fully understand (or we thought we did, but the evidence shows that there is still more to learn). In this paper, we summarize and synthesize many of these theories. We encourage readers who want to learn more about the methods and mechanisms of humanities research to consult the original books and papers in which they appear.    

As should be clear, we reject formulations of humanistic thinking that have been proffered by scholars in computer science as, for example, ``basically a style or mode'' \cite{Bardzell_Bardzell_2015}. As outlined above, our methods are instilled through rigorous training, refined through ongoing, reflection-driven interpretation, and grounded in the history of how that object or culture or concept has been interpreted to that point. Like the sciences, the humanities are also iterative; humanities disciplines build on past knowledge and they grow and they change. And the impact of this research is substantial. Humanities research teaches us what specific cultural objects might have represented to the people who created them or used them, and about the cultures that gave rise to them. These objects and cultures, in turn, point to how humans have existed in cultural and social contexts, both past and present, and how these contexts are part of what defines us at any given moment. Also key to these contexts is an understanding of how they consist of that which has not yet been discovered, digitized, transcribed, or annotated, as well as the of unrecoverable voices, the silenced narratives, and the stolen artifacts that have become symbols of empire. Humanities researchers track these complex meanings across time and place, and we share a commitment to protecting the stories of the few over the many. As part of this work we study technologies to understand how they affect societies and cultures \cite{Hicks_2020, Morgan_2021, Rosenthal_2018, Benjamin_2024} and we use technologies in order to tell new stories and make new discoveries about the past \cite{gallon_chapter_2016}. We have learned how interpretations of the past become the subject of future interpretations, which is why we are so invested in an understanding of generative AI, its conceptual underpinnings, its inputs, and its outputs, that is informed by humanistic expertise. 