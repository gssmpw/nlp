There is no question that the capabilities of AI systems have grown tremendously, especially over the past 3-5 years. As noted by Whittaker \cite{Whittaker_2021}, however, much of this epochal growth in capability has been fueled not by new insights, but by the concentration of resources. While the capabilities of these new models are qualitatively different, the change that enabled it is almost purely quantitative: massive datasets, massive computation clusters, and massive neural network models that are well-adapted to take advantage of current hardware. Universities and even governments have difficulty participating in this resource-intensive computing ecosystem; individual scholars have no chance. 

Ironically, current AI models’ dependency on massive scale presents a nightmare for corporate interests---one that in turn further motivates the consolidation of resources in corporate hands. For most of the history of computing, there has been a one-way march towards greater capability for lower cost and higher efficiency. The success of AI has inverted this curve. The cost to achieve minimal gains in performance is now growing exponentially. The environmental impact in terms of carbon emissions and water consumption is horrifying to those who must endure its direct consequences, as it is to all those seeking to work towards climate justice. But it is also worth observing that these costs are ruinously expensive for companies as well, since they must find a way to pay for these fixed costs in a market that has no fundamental barriers to entry (beyond vast sums of money). Because of the clear environmental harms, this less visible reality is often unremarked upon. But it is important to acknowledge because it illuminates yet another dimension of the corporate investment in AI. It is not just that companies may attempt to control AI; their very solvency depends on it. 

What these corporate forces all point towards is the phase of late capitalism that characterizes the world in which we all live. While the term ``late capitalism'' has become a convenient catchphrase to describe the range of injustices we face today---from minor indignities of data-based ad tracking, to the incontrovertible harms of collusion between governments and corporations---we believe that here again, humanities research can serve as a guide. This work allows us to understand the economic mechanisms of late capitalism with more precision, so as to push back against them and to envision alternatives. For example, we might look to Ulises Mejias and Nick Couldry’s theory of data colonialism \cite{Couldry_Mejias_2019}, which explains how corporations have actively worked to separate the data generated by web browsing and social media sites from the people who produce it, enabling it to become a form of ``surplus value'' that can be monetized by corporations without compensation for the people who produce it. Coupled with observations by scholars such as Paola Ricaurte, Mary Gray, and Siddarth Suri, among others, about how the patterns of exploitative digital labor (data cleaning and labeling, content moderation, and the like) parallel historical patterns of colonial labor exploitation, we can more fully understand how we have arrived at a time and place in which decisions of global significance are being made by five US-based tech CEOs, and how their power has been enabled by and continues to depend upon the further extraction and exploitation of the resources and labor of everyone else.   

While these forces may feel difficult to confront, placing them within a longer history of known forces and mechanisms–as histories and theories of labor, colonialism, racial capitalism, and more, enable us to do–also enrich and strengthen our strategies of resistance. We might look backwards to early twentieth century examples of community organizing in the wake of reconstruction, or further back to nineteenth-century strategies of mutual aid, as with the person-to-person and community-based networks that emerged among Black Americans, enslaved and free, as they supported those seeking freedom and economic justice \cite{Colored}. We might also look to ourselves to imagine futures outside of the constraints of capitalism, or at least less determined by the logics of capital. Examples of Indigenous data futures \cite{harjo_spiral_2019,brown_2023} and viral justice \cite{benjamin_viral_2022} provide glimpses of what is possible when we use our imagination to ``push us beyond the constraints of what we think, and are told, is politically possible'' \citep{benjamin_imagination_2024}.