Against the dispiriting backdrop of late capitalism and its fueling of the corporate capture of technical research, it has been heartening to see how the substance if not the development of this research has brought attention to the role and importance of humanistic thinking in ways that we, the authors, have not experienced in our near two-decade long involvement at the forefront of computational humanities research. This has opened up new possibilities for conversation and collaboration between computing, engineering, and the humanities. Humanities scholars are needed to augment a wide range of computational research conversations, policy debates, and public-facing social and cultural AI products. This has been demonstrated  by a trend within many subfields of computer science to turn to histories and theories developed within the humanities for additional context, fresh inspiration, and new ideas. Recognizing these contributions, some information schools have begun to hire researchers with humanities PhDs into faculty positions.  

These are all welcome developments. However, placed against the backdrop of the continued defunding and devaluation of humanities disciplines that has taken place since the Obama administration, combined with the assault on the humanities adding vitriol to culture wars we face the ever-increasing risk of another form of extraction: the stripping of humanities expertise from the disciplines that produce it, as well as---more distressingly---from the disciplinary structures that can ensure the continued development of this expertise. This phenomenon is akin to the phenomenon of ``elite capture'' as theorized by 
Ol\'uf\d{\'e}mi T\'a\'iw\`o \cite{taiwo_elite_2022}: the simultaneous valorization of certain groups and their contributions (here ``humanities research``) while taking control of the resources required to sustain them–and therefore taking credit for these contributions while depriving the original group of the ability to chart any future course. 

This leads us to two final sets of observations: first, a set of practical suggestions for pushing back against these forms of capture and control; and second, a set of general conclusions about the value and significance of humanities research for the present moment in AI. First, some practical suggestions. For one, humanities scholars must be brought to the table as equal partners with technical researchers. This means bringing humanities scholars into research projects in the initial phases of research collaboration, crediting them as coauthors, and not simply asking for feedback after the fact. For this to be possible, technical researchers must recognize the institutional asymmetries that exist between their own disciplines and humanities fields. Put plainly: humanities researchers teach more and are compensated less. They have less administrative support and fewer funding mechanisms available to them. For humanities scholars and technical researchers to meet as equal partners when conducting AI research, their time, participation, students, and staff must be funded like their technical counterparts. This funding must come from sources that fund technical work, since funding for humanities research are capped at several orders of magnitude less than what is available to technical researchers \cite{newfield_humanities_2025}. Furthermore, when new institutional initiatives are announced that make claims to engage with history and culture, they must include humanities researchers in their leadership structures. If humanities scholars are not included, it becomes incumbent upon the technical researchers already involved to call this out. Otherwise, in spite of any amount of increased attention to or value of the work of the humanities, it will simply not be able to continue into the future at the level required to sustain the production of new ideas and new researchers who can carry the work forward.   

Second, a more general observation. Much of the advance in AI technology has come from stochastic optimization, data management, and GPU coding. It is reasonable to expect that experts in those fields should have a place of influence. But we argue that both the capabilities and the challenges of contemporary AI look increasingly familiar to the humanities: collating archives, developing theories, identifying nuance, and generating new arguments, all through the lens of power, production, interpretation, and preservation. It is equally reasonable to expect that experts in these fields should have similar input. By acknowledging the expertise of humanities scholars and by taking active steps to ensure the continuity of humanities research, we see the best hope of employing AI technologies to improve our understanding of the human condition and its range of cultures, and enlisting them in support of a future–or futures–in which all of us can thrive.

With this work, we seek to spark conversations across technical and humanities fields about the uses and limits of AI. We write in the spirit of true interdisciplinary collaboration and exchange, and we hope this work serves as an invitation to scholars across varied fields to join together on the basis of mutual respect, working towards a goal of building AI systems that enhance our collective knowledge of human experiences and cultures when appropriate, while resisting the extractive and outright harmful AI systems that have been provided to us.  
