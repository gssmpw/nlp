%%
%% This is file `sample-acmsmall.tex',
%% generated with the docstrip utility.
%%
%% The original source files were:
%%
%% samples.dtx  (with options: `all,journal,bibtex,acmsmall')
%% 
%% IMPORTANT NOTICE:
%% 
%% For the copyright see the source file.
%% 
%% Any modified versions of this file must be renamed
%% with new filenames distinct from sample-acmsmall.tex.
%% 
%% For distribution of the original source see the terms
%% for copying and modification in the file samples.dtx.
%% 
%% This generated file may be distributed as long as the
%% original source files, as listed above, are part of the
%% same distribution. (The sources need not necessarily be
%% in the same archive or directory.)
%%
%%
%% Commands for TeXCount
%TC:macro \cite [option:text,text]
%TC:macro \citep [option:text,text]
%TC:macro \citet [option:text,text]
%TC:envir table 0 1
%TC:envir table* 0 1
%TC:envir tabular [ignore] word
%TC:envir displaymath 0 word
%TC:envir math 0 word
%TC:envir comment 0 0
%%
%% The first command in your LaTeX source must be the \documentclass
%% command.
%%
%% For submission and review of your manuscript please change the
%% command to \documentclass[manuscript, screen, review]{acmart}.
%%
%% When submitting camera ready or to TAPS, please change the command
%% to \documentclass[sigconf]{acmart} or whichever template is required
%% for your publication.
%%
%%
% \documentclass[acmsmall]{acmart}
\documentclass[acmsmall,screen,nonacm,review=false,timestamp=false]{acmart}
\usepackage[utf8]{inputenc}
%%
%% \BibTeX command to typeset BibTeX logo in the docs
\AtBeginDocument{%
  \providecommand\BibTeX{{%
    Bib\TeX}}}

\usepackage{fancyhdr}
\AtBeginDocument{%
    \addtolength{\footskip}{1.0\baselineskip}%
    \fancyfoot[L]{\textit{\textbf{Preprint --- please check authors' websites for updated versions.}}}%
}

%%%%%%%%%%%---SETME-----%%%%%%%%%%%%%
%replace @@ with the submission number submission site.
\newcommand{\thiswork}{INF$^2$\xspace}
%%%%%%%%%%%%%%%%%%%%%%%%%%%%%%%%%%%%


%\newcommand{\rev}[1]{{\color{olivegreen}#1}}
\newcommand{\rev}[1]{{#1}}


\newcommand{\JL}[1]{{\color{cyan}[\textbf{\sc JLee}: \textit{#1}]}}
\newcommand{\JW}[1]{{\color{orange}[\textbf{\sc JJung}: \textit{#1}]}}
\newcommand{\JY}[1]{{\color{blue(ncs)}[\textbf{\sc JSong}: \textit{#1}]}}
\newcommand{\HS}[1]{{\color{magenta}[\textbf{\sc HJang}: \textit{#1}]}}
\newcommand{\CS}[1]{{\color{navy}[\textbf{\sc CShin}: \textit{#1}]}}
\newcommand{\SN}[1]{{\color{olive}[\textbf{\sc SNoh}: \textit{#1}]}}

%\def\final{}   % uncomment this for the submission version
\ifdefined\final
\renewcommand{\JL}[1]{}
\renewcommand{\JW}[1]{}
\renewcommand{\JY}[1]{}
\renewcommand{\HS}[1]{}
\renewcommand{\CS}[1]{}
\renewcommand{\SN}[1]{}
\fi

%%% Notion for baseline approaches %%% 
\newcommand{\baseline}{offloading-based batched inference\xspace}
\newcommand{\Baseline}{Offloading-based batched inference\xspace}


\newcommand{\ans}{attention-near storage\xspace}
\newcommand{\Ans}{Attention-near storage\xspace}
\newcommand{\ANS}{Attention-Near Storage\xspace}

\newcommand{\wb}{delayed KV cache writeback\xspace}
\newcommand{\Wb}{Delayed KV cache writeback\xspace}
\newcommand{\WB}{Delayed KV Cache Writeback\xspace}

\newcommand{\xcache}{X-cache\xspace}
\newcommand{\XCACHE}{X-Cache\xspace}


%%% Notions for our methods %%%
\newcommand{\schemea}{\textbf{Expanding supported maximum sequence length with optimized performance}\xspace}
\newcommand{\Schemea}{\textbf{Expanding supported maximum sequence length with optimized performance}\xspace}

\newcommand{\schemeb}{\textbf{Optimizing the storage device performance}\xspace}
\newcommand{\Schemeb}{\textbf{Optimizing the storage device performance}\xspace}

\newcommand{\schemec}{\textbf{Orthogonally supporting Compression Techniques}\xspace}
\newcommand{\Schemec}{\textbf{Orthogonally supporting Compression Techniques}\xspace}



% Circular numbers
\usepackage{tikz}
\newcommand*\circled[1]{\tikz[baseline=(char.base)]{
            \node[shape=circle,draw,inner sep=0.4pt] (char) {#1};}}

\newcommand*\bcircled[1]{\tikz[baseline=(char.base)]{
            \node[shape=circle,draw,inner sep=0.4pt, fill=black, text=white] (char) {#1};}}


%% Rights management information.  This information is sent to you
%% when you complete the rights form.  These commands have SAMPLE
%% values in them; it is your responsibility as an author to replace
%% the commands and values with those provided to you when you
%% complete the rights form.
\setcopyright{acmlicensed}
\copyrightyear{2018}
\acmYear{2018}
\acmDOI{XXXXXXX.XXXXXXX}

%%
%% These commands are for a JOURNAL article.
\acmJournal{JACM}
\acmVolume{37}
\acmNumber{4}
\acmArticle{111}
\acmMonth{8}

%%
%% Submission ID.
%% Use this when submitting an article to a sponsored event. You'll
%% receive a unique submission ID from the organizers
%% of the event, and this ID should be used as the parameter to this command.
%%\acmSubmissionID{123-A56-BU3}

%%
%% For managing citations, it is recommended to use bibliography
%% files in BibTeX format.
%%
%% You can then either use BibTeX with the ACM-Reference-Format style,
%% or BibLaTeX with the acmnumeric or acmauthoryear sytles, that include
%% support for advanced citation of software artefact from the
%% biblatex-software package, also separately available on CTAN.
%%
%% Look at the sample-*-biblatex.tex files for templates showcasing
%% the biblatex styles.
%%

%%
%% The majority of ACM publications use numbered citations and
%% references.  The command \citestyle{authoryear} switches to the
%% "author year" style.
%%
%% If you are preparing content for an event
%% sponsored by ACM SIGGRAPH, you must use the "author year" style of
%% citations and references.
%% Uncommenting
%% the next command will enable that style.
%%\citestyle{acmauthoryear}


%%
%% end of the preamble, start of the body of the document source.
\begin{document}

%%
%% The "title" command has an optional parameter,
%% allowing the author to define a "short title" to be used in page headers.
\title{Provocations from the Humanities for Generative AI Research}

%%
%% The "author" command and its associated commands are used to define
%% the authors and their affiliations.
%% Of note is the shared affiliation of the first two authors, and the
%% "authornote" and "authornotemark" commands
%% used to denote shared contribution to the research.

\author{Lauren Klein}
%%\authornote{Both authors contributed equally to this research.}
\affiliation{%
  \institution{Emory University}
  \city{Atlanta}
  \state{GA}
  \country{USA}
}
\email{lauren.klein@emory.edu}

\author{Meredith Martin}
%%\authornote{Both authors contributed equally to this research.}
\affiliation{%
  \institution{Princeton University}
  \city{Princeton}
  \state{NJ}
  \country{USA}
}
\email{mm4@princeton.edu}

\author{Andre Brock}
%%\authornote{Both authors contributed equally to this research.}
\affiliation{%
  \institution{Georgia Institute of Technology}
  \city{Atlanta}
  \state{GA}
  \country{USA}
}
\email{andre.brock@lmc.gatech.edu}

\author{Maria Antoniak}
%%\authornote{Both authors contributed equally to this research.}
\affiliation{%
  \institution{University of Copenhagen}
  \city{Copenhagen}
  \state{Capital}
  \country{Denmark}
}
\email{maria.antoniak@colorado.edu}

\author{Melanie Walsh}
%%\authornote{Both authors contributed equally to this research.}
\affiliation{%
  \institution{University of Washington}
  \city{Seattle}
  \state{WA}
  \country{USA}
}
\email{melwalsh@uw.edu}

\author{Jessica Marie Johnson}
%%\authornote{Both authors contributed equally to this research.}
\affiliation{%
  \institution{Johns Hopkins University}
  \city{Baltimore}
  \state{MD}
  \country{USA}
}
\email{jmj@jhu.edu}

\author{Lauren Tilton}
%%\authornote{Both authors contributed equally to this research.}
\affiliation{%
  \institution{University of Richmond}
  \city{Richmond}
  \state{VA}
  \country{USA}
}
\email{ltilton@richmond.edu}

\author{David Mimno}
%%\authornote{Both authors contributed equally to this research.}
\affiliation{%
  \institution{Cornell University}
  \city{Ithaca}
  \state{NY}
  \country{USA}
}
\email{mimno@cornell.edu}

%%
%% By default, the full list of authors will be used in the page
%% headers. Often, this list is too long, and will overlap
%% other information printed in the page headers. This command allows
%% the author to define a more concise list
%% of authors' names for this purpose.
\renewcommand{\shortauthors}{Klein et al.}

%%
%% The abstract is a short summary of the work to be presented in the
%% article.
\begin{abstract}
This paper presents a set of provocations for considering the uses, impact, and harms of generative AI from the perspective of humanities researchers. We provide a working definition of humanities research, summarize some of its most salient theories and methods, and apply these theories and methods to the current landscape of AI. Drawing from foundational work in critical data studies, along with relevant humanities scholarship, we elaborate eight claims with broad applicability to current conversations about generative AI: 1) Models make words, but people make meaning; 2) Generative AI requires an expanded definition of culture; 3) Generative AI can never be representative; 4) Bigger models are not always better models; 5) Not all training data is equivalent; 6) Openness is not an easy fix; 7) Limited access to compute enables corporate capture; and 8) AI universalism creates narrow human subjects. We conclude with a discussion of the importance of resisting the extraction of humanities research by computer science and related fields.  	
\end{abstract}

%%
%% The code below is generated by the tool at http://dl.acm.org/ccs.cfm.
%% Please copy and paste the code instead of the example below.
%%

\begin{CCSXML}
<ccs2012>
   <concept>
       <concept_id>10010147.10010178</concept_id>
       <concept_desc>Computing methodologies~Artificial intelligence</concept_desc>
       <concept_significance>500</concept_significance>
       </concept>
   <concept>
       <concept_id>10010405.10010469</concept_id>
       <concept_desc>Applied computing~Arts and humanities</concept_desc>
       <concept_significance>500</concept_significance>
       </concept>
   <concept>
       <concept_id>10003120.10003121.10003126</concept_id>
       <concept_desc>Human-centered computing~HCI theory, concepts and models</concept_desc>
       <concept_significance>300</concept_significance>
       </concept>
 </ccs2012>
\end{CCSXML}

\ccsdesc[500]{Computing methodologies~Artificial intelligence}
\ccsdesc[500]{Applied computing~Arts and humanities}
\ccsdesc[300]{Human-centered computing~HCI theory, concepts and models}

%%
%% Keywords. The author(s) should pick words that accurately describe
%% the work being presented. Separate the keywords with commas.
\keywords{humanities research, humanities theory, humanities methods, humanistic approaches, digital humanities, digital pedagogy, media studies, critical data studies}

\received{22 January 2025}
%%\received[revised]{12 March 2009}
%%\received[accepted]{5 June 2009}

%%
%% This command processes the author and affiliation and title
%% information and builds the first part of the formatted document.
\maketitle

\section{Introduction}
\section{Introduction}

Tutoring has long been recognized as one of the most effective methods for enhancing human learning outcomes and addressing educational disparities~\citep{hill2005effects}. 
By providing personalized guidance to students, intelligent tutoring systems (ITS) have proven to be nearly as effective as human tutors in fostering deep understanding and skill acquisition, with research showing comparable learning gains~\citep{vanlehn2011relative,rus2013recent}.
More recently, the advancement of large language models (LLMs) has offered unprecedented opportunities to replicate these benefits in tutoring agents~\citep{dan2023educhat,jin2024teach,chen2024empowering}, unlocking the enormous potential to solve knowledge-intensive tasks such as answering complex questions or clarifying concepts.


\begin{figure}[t!]
\centering
\includegraphics[width=1.0\linewidth]{Figs/Fig.intro.pdf}
\caption{An illustration of coding tutoring, where a tutor aims to proactively guide students toward completing a target coding task while adapting to students' varying levels of background knowledge. \vspace{-5pt}}
\label{fig:example}
\end{figure}

\begin{figure}[t!]
\centering
\includegraphics[width=1.0\linewidth]{Figs/Fig.scaling.pdf}
\caption{\textsc{Traver} with the trained verifier shows inference-time scaling for coding tutoring (detailed in \S\ref{sec:scaling_analysis}). \textbf{Left}: Performance vs. sampled candidate utterances per turn. \textbf{Right}: Performance vs. total tokens consumed per tutoring session. \vspace{-15pt}}
\label{fig:scale}
\end{figure}


Previous research has extensively explored tutoring in educational fields, including language learning~\cite{swartz2012intelligent,stasaski-etal-2020-cima}, math reasoning~\cite{demszky-hill-2023-ncte,macina-etal-2023-mathdial}, and scientific concept education~\cite{yuan-etal-2024-boosting,yang2024leveraging}. 
Most aim to enhance students' understanding of target knowledge by employing pedagogical strategies such as recommending exercises~\cite{deng2023towards} or selecting teaching examples~\cite{ross-andreas-2024-toward}. 
However, these approaches fall short in broader situations requiring both understanding and practical application of specific pieces of knowledge to solve real-world, goal-driven problems. 
Such scenarios demand tutors to proactively guide people toward completing targeted tasks (e.g., coding).
Furthermore, the tutoring outcomes are challenging to assess since targeted tasks can often be completed by open-ended solutions.



To bridge this gap, we introduce \textbf{coding tutoring}, a promising yet underexplored task for LLM agents.
As illustrated in Figure~\ref{fig:example}, the tutor is provided with a target coding task and task-specific knowledge (e.g., cross-file dependencies and reference solutions), while the student is given only the coding task. The tutor does not know the student's prior knowledge about the task.
Coding tutoring requires the tutor to proactively guide the student toward completing the target task through dialogue.
This is inherently a goal-oriented process where tutors guide students using task-specific knowledge to achieve predefined objectives. 
Effective tutoring requires personalization, as tutors must adapt their guidance and communication style to students with varying levels of prior knowledge. 


Developing effective tutoring agents is challenging because off-the-shelf LLMs lack grounding to task-specific knowledge and interaction context.
Specifically, tutoring requires \textit{epistemic grounding}~\citep{tsai2016concept}, where domain expertise and assessment can vary significantly, and \textit{communicative grounding}~\citep{chai2018language}, necessary for proactively adapting communications to students' current knowledge.
To address these challenges, we propose the \textbf{Tra}ce-and-\textbf{Ver}ify (\textbf{\model}) agent workflow for building effective LLM-powered coding tutors. 
Leveraging knowledge tracing (KT)~\citep{corbett1994knowledge,scarlatos2024exploring}, \model explicitly estimates a student's knowledge state at each turn, which drives the tutor agents to adapt their language to fill the gaps in task-specific knowledge during utterance generation. 
Drawing inspiration from value-guided search mechanisms~\citep{lightman2023let,wang2024math,zhang2024rest}, \model incorporates a turn-by-turn reward model as a verifier to rank candidate utterances. 
By sampling more candidate tutor utterances during inference (see Figure~\ref{fig:scale}), \model ensures the selection of optimal utterances that prioritize goal-driven guidance and advance the tutoring progression effectively. 
Furthermore, we present \textbf{Di}alogue for \textbf{C}oding \textbf{T}utoring (\textbf{\eval}), an automatic protocol designed to assess the performance of tutoring agents. 
\eval employs code generation tests and simulated students with varying levels of programming expertise for evaluation. While human evaluation remains the gold standard for assessing tutoring agents, its reliance on time-intensive and costly processes often hinders rapid iteration during development. 
By leveraging simulated students, \eval serves as an efficient and scalable proxy, enabling reproducible assessments and accelerated agent improvement prior to final human validation. 



Through extensive experiments, we show that agents developed by \model consistently demonstrate higher success rates in guiding students to complete target coding tasks compared to baseline methods. We present detailed ablation studies, human evaluations, and an inference time scaling analysis, highlighting the transferability and scalability of our tutoring agent workflow.


\section{Background: What are the humanities anyway?}
The humanities have an extensive institutional and disciplinary history \cite{bod_new_2016,noauthor_philology_nodate,reitter_permanent_2021}, but here we focus on humanities research as it is practiced in the present: to study the humanities is to investigate the human: to investigate people and groups, and the cultural objects they create \cite{small_introduction_2013,haufe_humanities_2024}. We do so in order to understand how these cultural objects create or reflect new forms of knowledge. In order to do so, we are trained in the history of knowledge production within our individual and overlapping fields of expertise. We are also trained in multiple languages and are trained to translate cultural and historical meaning across objects, audiences, and in an array of forms. 

Humanities research requires both minute specificity and sweeping breadth. To engage with Claude McKay's ``Constab Ballads'' (1912), a researcher must first read both normative English and Jamaican dialect filled with elisions and diacritical marks. They must then be able to analyze poetry in the context of Caribbean history, American literature, and Black studies. They must know both the material culture that informs the work's physical form as well as text encoding and web development. The humanities encompass the range of cultural objects that humanities researchers analyze, including language, history, philosophy, theory, aesthetics, and phenomena. Humanities researchers are trained to understand these concepts by examining and interpreting specific examples, and by asking what ways of knowing and thinking (individually and collectively) might have gone into their creation \cite{noauthor_claude_nodate}.

What methods do we employ to conduct this work? Methods shared across humanities fields include theorization, interpretation, contextualization---and, increasingly but not universally, collaboration. Some concrete examples might include: archival research, textual explication, disciplinary techniques such as close reading (literary studies), historical synthesis (historical fields), and media-specific analysis (film and media studies), among others. These methods are what generate the bulk of the evidence that appears in humanities scholarship. However, what more commonly travels to technical fields are the theories that this evidence points towards. These are humanistic theories: written articulations of complex ideas that, until that point, we did not yet fully understand (or we thought we did, but the evidence shows that there is still more to learn). In this paper, we summarize and synthesize many of these theories. We encourage readers who want to learn more about the methods and mechanisms of humanities research to consult the original books and papers in which they appear.    

As should be clear, we reject formulations of humanistic thinking that have been proffered by scholars in computer science as, for example, ``basically a style or mode'' \cite{Bardzell_Bardzell_2015}. As outlined above, our methods are instilled through rigorous training, refined through ongoing, reflection-driven interpretation, and grounded in the history of how that object or culture or concept has been interpreted to that point. Like the sciences, the humanities are also iterative; humanities disciplines build on past knowledge and they grow and they change. And the impact of this research is substantial. Humanities research teaches us what specific cultural objects might have represented to the people who created them or used them, and about the cultures that gave rise to them. These objects and cultures, in turn, point to how humans have existed in cultural and social contexts, both past and present, and how these contexts are part of what defines us at any given moment. Also key to these contexts is an understanding of how they consist of that which has not yet been discovered, digitized, transcribed, or annotated, as well as the of unrecoverable voices, the silenced narratives, and the stolen artifacts that have become symbols of empire. Humanities researchers track these complex meanings across time and place, and we share a commitment to protecting the stories of the few over the many. As part of this work we study technologies to understand how they affect societies and cultures \cite{Hicks_2020, Morgan_2021, Rosenthal_2018, Benjamin_2024} and we use technologies in order to tell new stories and make new discoveries about the past \cite{gallon_chapter_2016}. We have learned how interpretations of the past become the subject of future interpretations, which is why we are so invested in an understanding of generative AI, its conceptual underpinnings, its inputs, and its outputs, that is informed by humanistic expertise. 

\section{Related Work}

\subsection{Humanities Research at FAccT}
Historically, humanities research has not featured prominently within the FAccT conference. Most mentions of the humanities that appear in FAccT publications appear in the umbrella term of ``social science and humanities research,'' or because of the ACM category of ``arts and humanities'' in which papers on ML, AI, and artistic practice appear. There has been a consistent acknowledgment of the unique perspectives and areas of expertise brought by humanities researchers, however, e.g. Ganesh et al.’s paper documenting a workshop in which researchers from different disciplines were brought together to discuss issues of AI fairness \cite{Ganesh_Dechesne_Waseem_2020}, Lünich and Keller’s paper documenting a similar workshop on the topic of explainable AI \cite{Lünich_Keller_2024}, Bates et al.’s auto-ethnographic account of incorporating critical data studies perspectives into the computer science classroom \cite{Bates_2020}, and Dotan et al.’s survey of approaches to incorporating generative AI responsibly in the classroom \cite{Dotan_Parker_Radzilowicz_2024}. Most pointedly, and ironically, Raji et al.’s quantitative analysis of AI ethics syllabi across the academy, ```You Can’t Sit With Us’: Exclusionary Pedagogy in AI Ethics Education,'' makes explicit mention of multiple humanities fields in the service of an argument about how, in refusing to look outside the discipline of computer science for their assigned readings, AI ethics courses are reproducing existing academic hierarchies \cite{Raji_Scheuerman_Amironesei_2021}. 

An exception is philosophy, which is one of the core disciplines of the FAccT community. A recent retrospective of the FAccT conference categorized 11\% of papers presented at FAccT as belonging to philosophy, and philosophy was the only humanities discipline prominent enough to be categorized individually \citet{Laufer_Jain_Cooper_Kleinberg_Heidari_2022}. But why is philosophy privileged in this way, not only in publications at FAccT but also in public discourse about AI? Unlike other humanities disciplines, which, as discussed above, prioritize the human and the specific, philosophy can allow researchers to abstract away from the messiness of real humans toward scenarios that isolate variables of interest and are more easily interpretable; a tempting proposition for scientists who value more tractable theories of knowing, and a useful method that can more easily mesh with mathematical and technical solutions. We can see this pattern reflected in works published at FAccT which have brought philosophical approaches to algorithmic decision making \citep{Cooper_Moss_Laufer_Nissenbaum_2022, Susser_2022, Jain_Suriyakumar_Creel_Wilson_2024}. The contributions of philosophy to FAccT and to the field of AI have been significant, but we clarify that the humanities has much more to offer than philosophy alone.

With that said, concepts and ideas from the humanities have consistently made their way into papers published at FAccT, often in substantive and impactful ways. A 2020 CRAFT workshop centered on speculative, queer, and feminist engagements with archives \cite{Pritchard_Snodgrass_Morrison_Britton_Moll_2020} while that same year saw Jo and Gebru’s ``Lessons from Archives: Strategies for Collecting Sociocultural Data in Machine Learning,'' a paper that has only grown in significance as the issue of LLM training data has entered broad consciousness \cite{Jo_Gebru_2020}. Additional papers have focused on specific theoretical constructs from the humanities, using them to question core concepts in ML/AI research (e.g. Corbett and Denton on transparency \cite{Corbett_Denton_2023} and Stark on animation \cite{Stark_2024}), and to imagine alternatives to existing approaches (e.g. Klumbyté et al., who explore the feminist and Black feminist concepts of situatedness, figuration, diffraction, and critical fabulation to think beyond existing modeling approaches \cite{Klumbyte_Draude_Taylor_2022}. Indeed, feminist and gender theories have perhaps received the most substantial attention at FAccT, e.g. \cite{Devinney_Björklund_Björklund_2022} and \cite{Klein_D’Ignazio_2024}. But engagement with other humanities fields and practices---most notably, art history and art practice, can be found as well, e.g. \cite{Huang_Liem_2022}, \cite{Divakaran_Sridhar_Srinivasan_2023}, and \cite{Srinivasan_2024}. We contribute to this work by documenting the core methods and contributions of humanities research and synthesizing the major concepts that carry across its fields. 

\subsection{Humanities Research on AI}
The launch of ChatGPT, in November 2022, brought humanities researchers to the forefront of national conversations about generative AI and its impact on research, writing and creativity (e.g. \cite{schmidt_representation_2023,tilton_relating_2023,jones_ai_2023,crawford_archeologies_2023,broussard2023,underwood_mapping_2022}. Those in the fields of digital humanities, digital pedagogy, and the history of technology, in particular---fields that require both technical and humanistic expertise---became some of the earliest expositors of its limitations and its potential e.g. (\cite{Kirschenbaum_2023, Rogers_2023, Klein_2022}). While these fields continue to lead within the humanities---for example, the task force convened jointly by the MLA and CCCC in Spring 2023 that has since published a series of working papers on AI and writing (\cite{Byrd_Flores, Generative, Building}---the national conversation about AI has been usurped by corporate spokespeople parroting AI hype, whose claims in turn convince academic administrators to adopt costly and potentially harmful AI systems \cite{seybold_ed_2023}. In fact, the increasingly politicized relationship between for-profit corporations, public resources, public institutions (like universities and research centers) and political power should make humanists, computer scientists, and researchers broadly alarmed.  

At the same time, humanities researchers have continued to push forward research about and/or involving AI in their own fields. This includes new journals (e.g. Critical AI, edited by two literary scholars \cite{Goodlad_2023}), and themed sections and special issues of flagship humanities journals (e.g. American Literature’s special issue on ``Critical AI: A Field in Formation'' \cite{Raley_Rhee_2023}) and PMLA’s themed section on ``AI and the University as Service'' \cite{Kirschenbaum_Raley_2024}. It includes monographs (e.g. \citep{Pasquinelli, Gunkel_2012, Katz_2020, Tenen_2024}) and long-form essays that theorize the language \cite{Slater_2024} and images \cite{Wasielewski_2024} generated by LLMs, historicize and critique ML approaches such as sentiment analysis \cite{Chun_Hong_Nakamura_2024}, topic modeling \cite{Binder_2016}, and image generation \cite{Offert_Phan_2024}, analyze the output of generative AI according to literary critical \cite{Walsh_Preus_Gronski_2024} and art historical expertise \cite{Malevé_Sluis_2023}, and more. However, this valuable work is rarely read outside of humanities disciplines. The goal of this paper is to argue that this important work should have a direct impact on the development of this technology. 

\section{Provocations from the Humanities for AI Research}

In the sections that follow, we employ boyd and Crawford’s framework of ``provocations'' for big data \citeyear{boyd_Crawford_2012} to develop a new set of provocations that encompass the expanded research landscape of generative AI. Ours are intended to establish a stronger foundation for future AI research and cross-disciplinary collaboration, to allow for the development of more informed research questions, and to prevent non-experts from making thin claims about the full breadth of human culture, past and present. We list the full set of provocations in Table \ref{table:provocations-list}.

% \begin{table}[h]
%     \centering
%     \footnotesize
%     \begin{tabular}{l}
%         \toprule
%         \textbf{Models make words, but people make meaning}  \\
%         \midrule
%         \textit{Question 1} \\
%         \textit{Question 1} \\
%         \textit{Question 1} \\
%         \midrule
%         \textbf{AI requires an understanding of culture} \\
%         \midrule
%         \textit{Question 1} \\
%         \textit{Question 2} \\
%         \textit{Question 3} \\
%         \midrule
%         \textbf{AI can never be ``representative''} \\
%         \midrule
%         \textit{Question 1} \\
%         \textit{Question 2} \\
%         \textit{Question 3} \\
%         \midrule
%         \textbf{Bigger models are not always better models} \\
%         \midrule
%         \textit{Question 1} \\
%         \textit{Question 2} \\
%         \textit{Question 3} \\
%         \midrule
%         \textbf{Not all training data is equivalent} \\
%         \midrule
%         \textit{Question 1} \\
%         \textit{Question 2} \\
%         \textit{Question 3} \\
%         \midrule
%         \textbf{Openness is not an easy fix} \\
%         \midrule
%         \textit{Question 1} \\
%         \textit{Question 2} \\
%         \textit{Question 3} \\
%         \midrule
%         \textbf{Limited access to compute enables corporate capture} \\
%         \midrule
%         \textit{Question 1} \\
%         \textit{Question 2} \\
%         \textit{Question 3} \\
%         \midrule
%         \textbf{AI universalism creates narrow human subjects} \\
%         \midrule
%         \textit{Question 1} \\
%         \textit{Question 2} \\
%         \textit{Question 3} \\
%         \bottomrule
%     \end{tabular}
%     \vspace{0.2cm}
%     \caption{The full list of provocations from the humanities for generative AI research.}
%     \label{table:provocations-list}
% \end{table}

\begin{table}[h]
    \centering
    \footnotesize
    \begin{tabular}{l}
        1. Models make words, but people make meaning  \\
        2. AI requires an understanding of culture \\
        3. AI can never be ``representative'' \\
        4. Bigger models are not always better models \\
        5. Not all training data is equivalent \\
        6. Openness is not an easy fix \\
        7. Limited access to compute enables corporate capture \\
        8. AI universalism creates narrow human subjects \\
    \end{tabular}
    \vspace{0.2cm}
    \caption{The full list of provocations from the humanities for generative AI research.}
    \label{table:provocations-list}
\end{table}


\subsection{Models make words, but people make meaning}
Humanists have wrestled for decades with the interplay between language and intention.
What is the implication of language that is produced in a way that is---by construction---devoid of human intention?
The field of natural language processing has been shaped by decades of research in both machine learning and linguistics. This has led to explanations (and rejections) of the language generated by LLMs and related AI systems that rely on linguistic theories of ``communicative intent'' \cite{Bender_Gebru_McMillan-Major_Shmitchell_2021}. This framing remains essential for understanding the output of these models in conversational contexts, but there are additional theories from the humanities that can be used to understand the broader significance (or lack thereof) of the language these models produce. 

More concretely, the theories of meaning-making first developed by literary scholars in the 1960s and 1970s offer ways of making sense of AI-generated language that are not dependent upon being able to identify any particular speaker or source. As early as 1967, philosopher Roland Barthes proposed that the meaning of any particular text is determined not by the author’s intention, but by the reader’s interpretation \cite{barthes1977image}. Literary theorist Jacques Derrida \citeyear{Derrida_1998} articulated a similar sentiment in his critique of logocentrism, or the idea that words are bound to a static meaning in the world. Scholars in the fields of Black studies, ethnic studies, gender and sexuality studies, and more, see the literal grammar of the English language, which has long functioned to deny people their humanity, as something that must be challenged wholesale. Far from enabling a descent into relativism, theories such as these become the basis for an expansion of the sources of meaning, as well as the techniques we can employ (or ourselves develop) for making sense of the range of writing we encounter in the world. 

Consider how LLMs generate text by predicting sequences of words. These predictions are based on both observed patterns and on human preferences and feedback. The result is often output that is factually wrong yet linguistically fluent and seemingly coherent---these are the ``hallucinations'' that have become a topic of research interest \cite{koenecke_careless_2024} and (justifiable) public concern. The theories cited above allow us to understand how the human tendency to create meaning through the interpretive expectations of intention and care can confer the illusion that the model ``knows'' something or someone. This is compounded by our current climate of ``AI hype'' and the nonconsensual infusion of AI into our lives, which have together naturalized artificial ``knowing'' over the complicated, messy, and polyrhythmic meaning-making conducted by humans when using language (oral, written, and otherwise) in social contexts. Humanities scholars have already been at the forefront of analyzing the output of genAI models, both text and image, not for what they ``know'' but for what meaning their output elicits about the cultures from which they emerged. 

Meaning is always people-fueled, socially-driven, irreverent, and impossible to map consistently. Especially in the context of the Global Minority, making meaning from language also requires an awareness of the history of evading and erasing dominant meanings, and extracting additional meaning from guarded words. Take for example the work of the Translation Lab, which set out to translate a famed interview with Ousmane Sembène, the Senegalese filmmaker, using SmartCat, a professional AI translation platform. The team quickly found that there were few Twi and Wolof terms on suggestion, and the terms that were suggested varied widely from the meaning expressed in the original French. As humanists, we are not surprised. AI ``knowing'' is an easier solution to a complicated reality of meaning forged between language, place, time, and social relations. 

\subsection{AI requires an understanding of culture}

Several years after the release of generative AI models to the public, we have ample evidence of how the text and images that they produce do not perform well in the context of non-dominant cultures, and at times actively harm them (e.g. \cite{Bender_Gebru_McMillan-Major_Shmitchell_2021,Prabhakaran_Qadri_Hutchinson_2022,cetinic_myth_2022,Cao_Zhou_Lee_Cabello_Chen_Hershcovich_2023}). Admirably, the technical community has recognized the problem and has proposed solutions such as improved training data \cite{Pawar_Park_Jin_Arora_Myung_Yadav_Haznitrama_Song_Oh_Augenstein_2024}, additional fine-tuning processes \cite{Masoud_Liu_Ferianc_Treleaven_Rodrigues_2024}, enhanced prompting strategies \cite{AlKhamissi_ElNokrashy_Alkhamissi_Diab_2024}, and new benchmarks \cite{Santurkar_Durmus_Ladhak_Lee_Liang_Hashimoto_2023}. These interventions are framed under the broader category of systems that improve users' acceptance of AI output: ``cultural alignment'' \cite{AlKhamissi_ElNokrashy_Alkhamissi_Diab_2024} or ``cultural inclusion'' \cite{Karamolegkou_Rust_Cui_Cao_Søgaard_Hershcovich_2024, Wadern}.

And yet much of the work on these topics does not ask what "culture" is, or seek involvement from the many humanities fields that are defined by that question. (There is increasingly involvement from the social sciences, especially anthropology \cite{Pawar_Park_Jin_Arora_Myung_Yadav_Haznitrama_Song_Oh_Augenstein_2024,hershcovich_challenges_2022,Adilazuarda_Mukherjee_Lavania_Singh_Aji_O’Neill_Modi_Choudhury_2024}. The result is a narrow definition of culture that rests on the terms of European modernity–e.g. a geographic region or a unifying nationality, language, or racial/ethnic/religious identity. This definition is far more rigid than how humanities scholars understand the term. In the humanities, ``culture'' may be used to refer to ``the way of life of a people, group, or humanity in general,'' but it can also be employed to describe ``the works and practices of intellectual and artistic activity'' that emerge from a particular community or group \citep{Williams_1976, Culture_2014}. These complementary yet distinct definitions are important to keep in mind, since it is not only that, for example, a model’s training data might be produced by people from different cultures, in the first sense of the word, but are that the training data is itself an expressions of culture, as in the second sense of the word, as is the model itself. 

The distinction between people as cultures and objects or expressions of culture, and our awareness of how both definitions are engaged by genAI models, is crucial for our understanding of their development and their output. Understanding how training data both reflects cultures, and consists of expressions of those cultures, can lead to more intentional data curation practices. In the commercial arena, we have seen how EleutherAI has developed the Pile with heightened attention to scientific cultures, as evidenced by their inclusion of data from PubMed and arXiv, among others \cite{gao_pile_2020}, and how Pleias recently focused its Common Corpus on cultural heritage (e.g., newspapers, monographs) and under-resourced languages \cite{noauthor_releasing_nodate}. This expanded definition of culture can also lead to new ideas for model development, as in recent works that treat pretraining datasets as curations worthy of examination and have studied the spread of books, poetry, and other creative content in these datasets \cite{Chang_Cramer_Soni_Bamman_2023,D’Souza_Mimno,walsh_sonnet_2024}, in work that considers the creative outputs of large models \cite{Lucy_Bamman_2021}, or in work that trains models with capabilities tuned for historical languages \cite{Yamshchikov_Tikhonov_Pantis_Schubert_Jost_2022}.

This expanded definition of culture also opens up the possibility for understanding cultural objects as expressions of larger structures of power, or as active challenges to those structures. In fact, when humanities scholars study culture, we analyze each of these dimensions of culture and more. Carried over to the context of AI, we can not only better understand for example how and why certain perspectives end up captured in training data and others do not, but how and why the models themselves must be understood as cultural objects and analyzed accordingly, meaningful both for the text and other media that they produce, and as expressions of contemporary tech culture in and of themselves.  

\subsection{AI can never be ``representative''}

A recognition of the cultural complexity of historical sources, including training data, also offers a fundamentally different way to approach the issue of bias. As is widely recognized, gaps in training data, coupled with the range of biases embedded in existing training datasets, have contributed to a range of harms perpetuated by AI systems \citep{Buolamwini_Gebru_2018, Noble_2018, Eubanks_2018}. In response, researchers and government agencies alike often call for strategies of bias mitigation, ranging from attempts to ``de-bias'' datasets and AI systems \citep{Bolukbasi_Chang_Zou_Saligrama_Kalai_2016}, to improved documentation of both datasets \cite{ Gebru_Morgenstern_Vecchione_Vaughan_Wallach_III_Crawford_2021} and models \cite{Mitchell_Wu_Zaldivar_Barnes_Vasserman_Hutchinson_Spitzer_Raji_Gebru_2019}, to research into model interpretability \cite{Bhatt_Ravikumar_Moura_2019} and explainability \cite{Danilevsky_Qian_Aharonov_Katsis_Kawas_Sen_2020}. Each of these interventions are necessary, but the issue of bias reflects a deeper structural problem, one that will never be fixed unless the power differentials that cause structural inequalities are challenged at their source. 

Here is where additional concepts and strategies from the humanities enter in. Among them, we might draw from theories and practices that emerge from feminist, Black feminist, and archival theory, which offer examples of how to engage with historical sources that contain biases–what humanities researchers would call silences \cite{trouillot_silencing_2015} or absences \cite{Hartman_2008}---that can never be ``de-biased'' or modeled away \cite{bode_why_2020}. Reframing bias not as "bad data" but as an effect of unequal structural power redirects our attention to the social, historical, and political conditions that gave rise to the biased data, as well as to the spaces of indeterminacy, and to the irrecoverably missing parts, of any dataset past or present \cite{Sherman_Morrison_Klein_Rosner_2024}. In an algorithmic context, this raises new research questions: what new methods might be required in order to mark missing data, or intentionally amplify the significance of sparse data, rather than merely pass over these silences or gaps? This view also raises questions of personal responsibility: are we informed enough about, or connected enough to the people or cultures that the data represents in order to make these decisions? Have we read until we understand? \cite{Griffin_2021}. 

Humanities researchers devote significant amounts of time and energy to supplementing our own knowledge, expanding our shared archives, and contributing to a collective understanding, but we do not deceive ourselves into thinking that there is ever an end-point to this process. We know we will never have all of the ``data,'' so to speak, so as to fully understand the past. How do we grapple with this lack of complete representativeness? What are some lessons we might carry over to our work with AI? We might seek to learn from scholars of the archive of slavery who have spent decades developing strategies for making meaning from damaged archival records. We might look to the technique of critical fabulation as conceived by Saidiya Hartman, for example, which seeks to amplify the significance of sparse and at times violent archival records with narrative detail \cite{Hartman_2020}; we might model our work after the mode of historically-informed speculation as illustrated by Marisa Fuentes, who posits questions about what might have happened on the basis of historical fact \cite{fuentes_2016}; or we might look to Jessica Marie Johnson’s theorization of the ``null value,''---the same concept as in the relational database---which she employs to hold space for the people whose stories that cannot be recovered at all \cite{Johnson_2020}. We can see some of these approaches already being applied to the output of biased models in the present, as in Curry Hackett’s use of text-to-image models to generate images of city streetscapes inspired by examples of twentieth-century African American art \cite{noauthor_architect_2023}. But we are eager to see them applied to the construction and framing of models themselves. 

There is an additional, broader set of lessons to be learned about the impossibility of complete knowledge and the need for technical researchers to come to terms with that fact: any act of dataset creation is a political act. This is true because our datasets represent languages, peoples, and cultures, which are in themselves embedded in larger structures of power. Here we can look to the work of statistician Taylor Arnold and visual culture scholar Lauren Tilton and their approach of distant viewing \cite{Arnold_2023}. Using AI to analyze photographs from the US National Archives, they show how the very process of assigning an annotation to an image entails a set of political as well as social and cultural decisions. Put another way, there is always a perspective that is shaped by training data, annotations, parameters, and prompts. Rather than chasing a ``representative'' dataset---which, of course, can never be achieved---we would be better served by acknowledging that there are always perspectives encoded in our models, and by taking the time to document and name them. 

\subsection{Bigger models are not always better models}

As boyd and Crawford assert in their original provocations, ``bigger data is not always better data'' \cite{boyd_Crawford_2012}. This applies to models as well. For the first several years of LLM development, the growth in numbers of parameters was rivaled only by the growth in the number of charts documenting the rising number of those parameters. But the axiom that bigger is better has now met its match. Increasingly, studies have found that more data, more parameters, and more compute no longer lead to more accurate output (e.g. (\cite{Zhou_Schellaert_Martínez-Plumed_Moros-Daval_Ferri_Hernández-Orallo_2024,mckenzie_inverse_2024}. These findings have raised interesting technical questions, as well anxieties that we are "running out of data" \cite{noauthor_openai_nodate}. But to humanities researchers, these findings serve to prove a more philosophical point: that the goal of a single universal source of ``intelligence'' is a fraught endeavor. More than that, it will never lead to complete knowledge, as feminist philosophers as early as the seventeenth century have shown \citep{Newcastle_2019, Emecheta_1983, Haraway_1988}. 

Arguably, the humanities’ most universal commitment is to the diversity and complexity of the human experience. More than that, centuries of humanities scholarship has confirmed the asymptotic relationship between increased understanding and complete knowledge. The idea that we could ever build a model of artificial ``general'' intelligence is not only a fool’s errand; it is uninformed by how intelligence actually works. Somewhat paradoxically, this same body of work---that is to say, humanities scholarship---offers the clearest path forward. Smaller models, with intentionally-curated datasets, and trained or fine-tuned for precise tasks, promise to bring us closer to a goal we all share: of more informed, more accurate, and more precise knowledge. In the field of computational humanities, this work has already begun. For example, Giselle Gonzalez Garcia and Christian Weilbach brought their respective backgrounds in the fields of history and computer science to customize an LLM for research assistant in the field of Irish migration studies \cite{Garcia_Weilbach}. Ted Underwood and collaborators at UIUC are currently training an LLM from scratch on only nineteenth-century English-language writing, so as to be able to run counterfactual experiments on the past.

The promise of smaller, more bespoke models also points to the need for a shift in thinking. Rather than being driven by the ``bigger is better'' agenda that characterizes much of the current corporate research landscape, we might reestablish a research agenda born of hundreds of years of humanities scholarship, and informed by the evidence of thousands of years of actual human experience. This will lead us to value the expertise of domain experts---not only as prompt engineers, as has been the case thus far, but as collaborators in model development. And it will lead to a goal we all share: of more informed, more accurate, and more precise knowledge.


\subsection{Not all training data is equivalent}

A related point to the previous provocation has to do with the nature of training data. Up until very recently, advances in generative AI have relied upon on vast, heterogeneous pretraining datasets. These draw from a wide range of sources: internet content, copyrighted books, scientific articles, and now even synthetic data from other AI models, not to mention a variety (but to be clear, not the entirety) of languages, time periods, communities, and genres \cite{Anil_Dai_Firat_Johnson_Lepikhin_Passos_Shakeri_Taropa_Bailey_Chen_etal._2023,brown_language_2020,nostalgebraist_2022}. Despite this range, pretraining data is often treated as a homogenous, undifferentiated resource—often likened to ``oil''—where individual pieces of data are assumed to be interchangeable and valued primarily for their downstream utility. When individual points within these massive datasets are considered, they are typically evaluated through metrics like ``toxicity'' or ``quality'' \citep{Lucy_Gururangan_Soldaini_Strubell_Bamman_Klein_Dodge_2024}, framings that prioritize how the data will impact a model’s performance. This shallow approach to pretraining data is further compounded by  ``documentation debt'' \cite{Bandy_Vincent_2021}; the actual contents of pretraining data and why specific sources are chosen are rarely disclosed. 

This reduction and obfuscation of training data is now widely recognized as an ethical problem as well as a technical one. Inspired by practices in the electronics industry, the influential ``Datasheets for Datasets'' \cite{Gebru_Morgenstern_Vecchione_Vaughan_Wallach_III_Crawford_2021} paradigm introduced a structured approach to documenting machine learning datasets, which has been implemented for high-profile data repositories such as HuggingFace. We have also increasingly seen empirical tests that assess how different factors such as dataset age, degree of toxicity, and quality level, impact model performance \cite{longpre-etal-2024-pretrainers}. " While these are urgent and necessary interventions, approaching pretraining data from a humanistic perspective can offer additional affordances. For example, the development of rich metadata can help to ensure that datasets more accurately represent the contents they seek to include. For a dataset of books, this might include both computationally and manually derived metadata like the book’s title, publication date, genre, author, and perhaps even demographic information about the author like race, gender, and geography, as in a recent project aimed at improving the analysis of the more than 17 million volumes of digitized text in the HathiTrust Digital Library \cite{underwood_page-level_2014,Underwood_Kimutis_Witte_2020}. 

Humanities researchers are also keenly aware that both metadata and data curation choices radically impact any downstream task, and have developed several detailed strategies for documenting these choices and their potential impact. For example, digital archeologies of datasets place specific data sources in the broader context of the conditions of their creation, revealing biases and limitations \cite{lee_compounded_2021,fyfe_archaeology_2016}. Data narratives seek to situate datasets in human contexts and social histories beyond their immediate technical use \cite{pouchard_data_2014}. Another recent approach used by initiatives such as Post45 Data Collective, the Nineteenth Century Data Collective, and Responsible Datasets in Context Project, is the data essay. These longform works describe the data’s historical context, collection and curation, limitations, and ethical considerations, build on important prior work like ``datasheets''  but expand to include more abstract considerations of how any dataset is shaped by, and must be considered in light of, history, society, power, and specific cultural phenomena.

What these examples all point towards is how a consideration of a dataset’s conceptual characteristics, in addition to its technical characteristic and downstream utility, can improve the transparency and accountability of AI models and have other benefits.  Knowing more about training data can help model developers avoid the perpetual cycle of after-the-fact fixes, such as guardrails, RLHF, post-hoc content moderation, and other patchwork solutions. Knowing more about where data actually came from and who it belongs to can help developers address ongoing intellectual property issues. And knowing more about training data can even lead to innovation. Because most developers have not cared to understand their training data in great depth, how models are influenced by finer features of training data remains underexplored.


\subsection{Openness is not an easy fix}

As an increasing number of open and open-source models have been released for public use, many researchers in both technical and humanistic fields have moved on from the issues surrounding black-boxed, proprietary, and pay-for-access models. But the question of openness---both what it means and what it implies---remains unresolved. In short: there are no easy fixes when working with objects of culture (which include both training data and models, as has been explained). In this section we introduce some key considerations from the humanities aimed at illustrating how questions of openness, ownership, and access, rarely have yes-or-no answers and must be consistently reevaluated as contexts and conditions change.   

It may be obvious at this point to assert that Meta’s ``open'' models, while available for use, remain closed in terms of information about their underlying data and training processes. They also still require significant computational resources, which becomes economically restrictive as discussed more in the next section. This limitation also holds for open-source models such as AI2’s OLMo models, and while we welcome additional entries in this area, as well as the proliferation of smaller desktop models, the questions of which and whose data and models should be open, for which purposes, and for whose goals, remains unresolved. Consider current debates over the inclusion of copyrighted content in training data. On the surface, this would seem to have an easy answer: individuals should control access to their data, including their creative content, and if they do not want it included in training data, it should not be. But consider the downstream effects on scholarship aimed at understanding those people and their cultures. If their data is not included in the model, future researchers cannot employ it to learn about the past. Set against the backdrop of corporate extraction, this use case may not be the most important one to consider. (And to be clear, preserving the ability of writers and artists to create their art and be fairly compensated for it should remain paramount). But there are examples from computational research in the humanities that point to how this work can be done on a case-by-case basis. For example, \cite{Bamman_Samberg_So_Zhou_2024} use fair use exceptions and an exemption to the Digital Millennium Copyright Act (DMCA) to create one of the largest digitized collections of copyrighted films. The HathiTrust Digital Library, which consists of over 18 million digitized books, offers derivative data and virtual environments that enable researchers to access and analyze copyright-restricted materials for educational purposes \cite{hathitrust}. However, the time, labor, and cost involved in maintaining this environment is substantial, and it is unclear if it will be able to continue into the future. 

In addition, there remain questions of access to data from communities that have not or cannot provide consent. These questions apply to community data that might even technically be ``open'' and scrapeable from the web. For example, fanfiction writers and readers often operate with an expectation of privacy, participating in intimate, close-knit communities, despite their work being openly published online. But because of the ease of collecting data from the web, users’ stories and interactions can be gathered and shared even without their knowledge or consent, violating their expectations and community norms. According to \citet{Dym_Fiesler_2020}, members of fanfiction communities have expressed concerns about the potential negative repercussions of their data being shared with broader audiences, such as being outed or facing professional consequences. Similarly, the Documenting the Now project has highlighted the potential harms that can arise when social media data, especially data related to protests like the Black Lives Matter movement, is archived or shared. These concerns must also be placed in the larger context of the history of data, and of data extraction, both of which can be traced to specific and violent pasts. We discuss this further in the section below.

There are also deeper questions that emerge from the use of synthetic data, as well as from the ability of models to generate ``new'' content that would seem to circumvent these real-world harms. Beyond banning specific keywords, we must consider the impact of our ability to prompt models to generate what Saidiya Hartman would describe as ``scenes of subjection''---text and images that, in restaging scenes of historical violence, even through computational means, introduces questions of our own complicity---even with respect to cultures or phenomena that we might simply seek to learn more about. We must also contend with the underlying motivation---and very often, the final use case---of much of the research on minoritized groups. As we survey the devastation that has been brought upon Ukraine and Gaza, in part due to AI technologies \cite{noauthor_tech_nodate,tharoor_analysis_2024}, we must remind ourselves that the end goal of much of AI research is the surveillance of people and communities---and at times, their outright destruction---even if these are not the goals that compel us forward in our research.  

\subsection{Limited access to compute enables corporate capture}

There is no question that the capabilities of AI systems have grown tremendously, especially over the past 3-5 years. As noted by Whittaker \cite{Whittaker_2021}, however, much of this epochal growth in capability has been fueled not by new insights, but by the concentration of resources. While the capabilities of these new models are qualitatively different, the change that enabled it is almost purely quantitative: massive datasets, massive computation clusters, and massive neural network models that are well-adapted to take advantage of current hardware. Universities and even governments have difficulty participating in this resource-intensive computing ecosystem; individual scholars have no chance. 

Ironically, current AI models’ dependency on massive scale presents a nightmare for corporate interests---one that in turn further motivates the consolidation of resources in corporate hands. For most of the history of computing, there has been a one-way march towards greater capability for lower cost and higher efficiency. The success of AI has inverted this curve. The cost to achieve minimal gains in performance is now growing exponentially. The environmental impact in terms of carbon emissions and water consumption is horrifying to those who must endure its direct consequences, as it is to all those seeking to work towards climate justice. But it is also worth observing that these costs are ruinously expensive for companies as well, since they must find a way to pay for these fixed costs in a market that has no fundamental barriers to entry (beyond vast sums of money). Because of the clear environmental harms, this less visible reality is often unremarked upon. But it is important to acknowledge because it illuminates yet another dimension of the corporate investment in AI. It is not just that companies may attempt to control AI; their very solvency depends on it. 

What these corporate forces all point towards is the phase of late capitalism that characterizes the world in which we all live. While the term ``late capitalism'' has become a convenient catchphrase to describe the range of injustices we face today---from minor indignities of data-based ad tracking, to the incontrovertible harms of collusion between governments and corporations---we believe that here again, humanities research can serve as a guide. This work allows us to understand the economic mechanisms of late capitalism with more precision, so as to push back against them and to envision alternatives. For example, we might look to Ulises Mejias and Nick Couldry’s theory of data colonialism \cite{Couldry_Mejias_2019}, which explains how corporations have actively worked to separate the data generated by web browsing and social media sites from the people who produce it, enabling it to become a form of ``surplus value'' that can be monetized by corporations without compensation for the people who produce it. Coupled with observations by scholars such as Paola Ricaurte, Mary Gray, and Siddarth Suri, among others, about how the patterns of exploitative digital labor (data cleaning and labeling, content moderation, and the like) parallel historical patterns of colonial labor exploitation, we can more fully understand how we have arrived at a time and place in which decisions of global significance are being made by five US-based tech CEOs, and how their power has been enabled by and continues to depend upon the further extraction and exploitation of the resources and labor of everyone else.   

While these forces may feel difficult to confront, placing them within a longer history of known forces and mechanisms–as histories and theories of labor, colonialism, racial capitalism, and more, enable us to do–also enrich and strengthen our strategies of resistance. We might look backwards to early twentieth century examples of community organizing in the wake of reconstruction, or further back to nineteenth-century strategies of mutual aid, as with the person-to-person and community-based networks that emerged among Black Americans, enslaved and free, as they supported those seeking freedom and economic justice \cite{Colored}. We might also look to ourselves to imagine futures outside of the constraints of capitalism, or at least less determined by the logics of capital. Examples of Indigenous data futures \cite{harjo_spiral_2019,brown_2023} and viral justice \cite{benjamin_viral_2022} provide glimpses of what is possible when we use our imagination to ``push us beyond the constraints of what we think, and are told, is politically possible'' \citep{benjamin_imagination_2024}.

\subsection{AI universalism creates narrow human subjects}

Generative AI has further embroiled society in a “data episteme'' \cite{Koopman2019-ef}, where we know ourselves through the acquisition of data and our knowledge can only be validated by gathering more and more data.  Where we began with indexing ourselves, our economies, and our institutions, we are now in a moment where culture is ``content''; moments of ingenuity, mundanity, and deviance abstracted to records of human activity suitable for extraction.  Statistical and algorithmic interpretations of the world---models and training data corpora---are aggressively promoted as ``reality'', absent acknowledgement of the libidinal tensions suffusing the economic and political ends for which these representations are being deployed.

A concern with the moral, expressive, and contemplative aspects of living \cite{mesthene_1969} are central to humanistic inquiry.  But AI recenters inquiry as the interpretation and synthesis of data, rather than a contemplation of the world. This recentering depends on the data episteme’s enactment of the premises of European modernity: the abstraction of resources---including being---in order to yield quantifiable measures of progress, efficiency at someone else’s expense.  For example, Amazon’s  practice of algorithmically monitoring workers fosters dreadful work conditions in pursuit of ``frictionless'', ``timely'' deliveries.

Charles Mills (2021), writing on antiblackness, offers a startling assessment of what it means to be modern. For Mills, antiblackness is foundational to the systematized racial subordination of chattel slavery that is essential to European modernity’s function and structure.  He writes, ``race becomes the signifier of full or diminished humanity, a signifier that is enforced by material practices in a modern racialized world'' (2018, 28).  The Enlightenment project of equality, liberalism, capital, and democracy was structured by antiblackness and colonialism, shaping the modern world by demarcating who is human through information, commerce, and citizenship.  Ramon Amaro (Amaro )(cite) expands antiblackness to computer vision, writing ``By regressing complex environmental data into a generalized pattern, the lived and multivalent specificities of black lives are represented ``as if'' the visual matrix maintains no connections to historical category, stereotype, or a moral imaginary'' (2019, 4). As a result, ``research in computation is an adaptation of the fictive and compulsive ordering of human attributes into a single coherent image of species'' (Amaro 2019, 5).  

The internet’s early days featured similar promises of liberal ideals: freedom of expression, freedom of identity, freedom of association.  As Silicon Valley sought venture capital, however, their promises of free speech and expression were by the dictates of Wall Streets, but the last decade has seen those aesthetic, inventive, and quotidian digital expressions be reframed as ``content''.  Where AI vision reduces the visual complexity of humanity, LLMs similarly distort the inventive and cultural capacities of culture to ``content''.  These abstractions afford AI the coercive claim that they are a ‘universal knowledge’ machine, rather than a statistical enactment of the ideology that ``one can improve --- control --- a deviant subpopulation by enumeration and classification'' (Hacking 1990/2013, (Hacking 2013, page 3) 
 
While AI proponents declaim that ``creativity'' and ``freedom of speech'' are essential to artificial general intelligence, their theft and expropriation of copyrighted works, public and private discourse, and user-generated multimedia as ``data'' should not be solely understood as neoliberal capitalism (although that certainly applies).  Instead, humanistic inquiry offers us the alternative of seeing AI’s libidinal economic enterprise; where in addition to inheriting the antiblack capacities of Western arts, letters, and public discourse, the AI being proffered to us is deeply anti-human.

Humanities inquiry illuminates AI’s reordering of society as a network of relations between abstraction, acquisition, informational and material violence.  Although the genie is out of the bottle, humanities scholars work to redirect AI to become a convivial technology, one in service to society and culture. 


\section{Conclusion: Against Humanities Extraction}

\section{Conclusion}
\label{sec:conclusion}
We demonstrate the \chat{} system to interactively build AI pipelines using \sys{} and \archytas{}.
Although our demo did not extensively cover physical optimization aspects, more details can be found in~\cite{palimpzestCIDR}.
The \chat{} interface offers a convenient tool for data practitioners to build complex data processing pipelines with little effort and a soft learning curve.
Our vision is that on the one hand, the future of data engineering will include more and more sophisticated frameworks to build complex applications that mix LLMs and traditional data processing.
On the other hand, tools like \chat{} will assist developers and make it easier to adopt new technologies and programming paradigms.

%%
%% The acknowledgments section is defined using the "acks" environment
%% (and NOT an unnumbered section). This ensures the proper
%% identification of the section in the article metadata, and the
%% consistent spelling of the heading.
\begin{acks}
The authors wish to thank the Data + Feminism lab and the Liberatory AI Ecosystems working group for feedback on this paper. This project has also benefited from conversations with members of the Atlanta Interdisciplinary AI Network and LifexCode: Digital Humanities Against Enclosure. 

Work on this paper has been supported by the following grants: Mellon Foundation G-2211-14240.
\end{acks}

%%
%% The next two lines define the bibliography style to be used, and
%% the bibliography file.
\bibliographystyle{ACM-Reference-Format}
\bibliography{provocations}

% \appendix

%\section{Appendix: Research Ethics and Social Impact}
%\subsection{Ethical Considerations Statement}

%This paper is primarily a literature review and theory contribution, thus we did not engage in any human subjects research, systems development, or deployment. Our work has been guided by considerations for citational justice, and we have specifically sought to cite the work of scholars from marginalized backgrounds, especially BIWOC and queer people of color.

%\subsection{Researcher Positionality Statement}

%We are a three-person writing team that brings domain expertise in the humanities (including historical, literary, and media studies scholarship), digital humanities (including NLP/ML/AI), communication and rhetoric, software development, data science, and data visualization. As a multi-gender and multiracial project team, we share an interest in and commitment to gender and racial justice. This commitment draws us to the politically-engaged humanities scholarship that we seek to engage and amplify in this paper. WIth that said, there are many experiences of intersectional oppression that we do not have – we are all settler scholars, cisgender, (mostly) heterosexual, and (mostly) non-disabled. For these reasons, we have tried to include and learn from scholars who write from these intersections. 

%\subsection{Adverse Impact Statement}

%With this work, we seek to spark conversations across technical and humanities fields about the uses and limits of AI. We write in the spirit of true interdisciplinary collaboration and exchange, and we hope this work serves as an invitation to scholars across varied fields to join together on the basis of mutual respect, working towards a goal of building AI systems that enhance our collective knowledge of human experiences and cultures when appropriate, while resisting the extractive and outright harmful AI systems that have been provided to us.  






\end{document}
\endinput
%%
%% End of file `sample-acmsmall.tex'.
