\begin{abstract}
Bimodal data, such as image-text pairs, has become increasingly prevalent in the digital era. The Hybrid Vector Query (\hvq) is an effective approach for querying such data and has recently garnered considerable attention from researchers. It calculates similarity scores for objects represented by two vectors using a weighted sum of each individual vector's similarity, with a query-specific parameter $\alpha$ to determine the weight. Existing methods for \hvq typically construct Approximate Nearest Neighbors Search (ANNS) indexes with a fixed $\alpha$ value. This leads to significant performance degradation when the query's $\alpha$ dynamically changes based on the different scenarios and needs.

In this study, we introduce the Dynamic Edge Navigation Graph (\method), a graph-based ANNS index that maintains efficiency and accuracy with changing 
{\cheng $\alpha$ values}. It includes three novel components: (1) a greedy Pareto frontier search algorithm to compute a candidate %edge
neighbor set for each node, which comprises the node’s approximate nearest neighbors 
% at any $\alpha$ value; 
{\cheng for all possible $\alpha$ values;}
(2) a dynamic edge pruning strategy to determine the final edges from the candidate set and assign each edge an active range. This active range enables the dynamic use of the Relative Neighborhood Graph's pruning strategy based on the query's $\alpha$ values, skipping redundant edges at query time and achieving a better accuracy-efficiency trade-off; {\cheng and} (3) an edge seed method that accelerates the querying process. Extensive experiments on real-world datasets show that \method demonstrates superior performance compared to existing methods under varying 
% query 
{\cheng $\alpha$ values}.

\end{abstract}

