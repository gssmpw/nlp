\section{Convergence of Optimal Policies}
\label{sec:convergence-of-the-optimal-policy}
In Section~\ref{sec:approximating-the-optimal-policy}, we showed that finite horizon analysis suffices to approximate optimal policies. Here, we establish an even more fundamental property for a broad family of instances: After a finite number of rounds, the optimal policy becomes fixed, transitioning from exploration to pure exploitation.

We focus our attention on instances where the preference matrix $\bm{P}$ has \emph{distinct preferences} (DP), meaning all entries of $\bm{P}$ are unique. This ensures that every observation provides unambiguous information about user types, enabling informative observations. A discussion of how this theorem generalizes to other instances is provided in \apxref{sec:convergence-of-the-optimal-policy-proofs}. Our main result is as follows:

\begin{theorem}\label{thm:convergence}
   Fix any DP-$\prob$ instance. There exists $T < \infty$ such that for any $t \geq T$, it holds that $\pi^\star_{t+1}=\pi^\star_{t}$.
\end{theorem}

Theorem~\ref{thm:convergence} introduces a natural complexity measure through the time until convergence. Through the proof's construction, we can identify both when convergence occurs and determine the optimal policy from that point onward. Hence, instances with faster convergence require less computational effort as fewer possibilities need to be explored before identifying the final repeating recommendation.

While Theorem~\ref{thm:convergence} only establishes the convergence, our empirical analysis in Section~\ref{sec:experiments} reveals that this transition typically occurs relatively fast in practice.

\begin{proof}[\normalfont\bfseries Proof Sketch of Theorem~\ref{thm:convergence}]
We begin by introducing an instance-dependent parameter $c$ that captures crucial structural 
relationships within the user preference matrix. While the exact definition of $c$ is deferred 
to \apxref{sec:convergence-of-the-optimal-policy-proofs}, intuitively, $c$ encodes values like the degree of heterogeneity among user types---quantified by the minimal separation between entries in each 
row of $\bm{P}$---and the minimum probability in the user-type distribution $\bm{q}$. Crucially, 
$c > 0$ holds for all UP-$\prob$ instances.

%Notably, $c > 0$ for all UP-$\prob$ instances (see \apxref{sec:convergence-of-the-optimal-policy-proofs} for the detailed construction of $c$ and auxiliary proofs). 
%This parameter encapsulates several key characteristics, like the heterogeneity among user types (measured by the minimal difference between entries in rows of $\bm{P}$) and the minimum probability in $\bm{q}$ (corresponding to the rarest user types).

Given a user type $m \in M$ and $\delta > 0$, we say a belief $\bm{b}$ is \emph{$(\delta,m)$-concentrated} if $\bm{b}(m) \geq 1-\delta$. Typically, the prior $\bm{q}$ is \emph{$c$-unconcentrated}, meaning $\bm{q}(m) < 1-c$ for all $m \in M$, with probability mass distributed across multiple types.

Our first result demonstrates that for unconcentrated beliefs, the value function strictly increases by a significant gap during consecutive steps of the optimal policy.

\begin{theorem}\label{thm:gap-between-value-function}
For any $\delta$-unconcentrated belief $\bm{b} \in \Delta(M)$ it holds that $V^{\star}(\tau(\bm{b}, \pi^{\star}_1(\bm{b}))) - V^{\star}(\bm{b}) \geq \frac{\delta \cdot (1 - \delta) \cdot c^2}{1 - c}$.
\end{theorem}
Lemma~\ref{lemma:closed-form-representations-of-the-value-function} implies that the value function is bounded above by $\frac{1-c}{c}$, since
\begin{equation}\label{eq:bounded-value}
V^\pi(\bm{b}) = \sum_{t=1}^{\infty} \prod_{j=1}^{t} p_{\pi_j}(\bm{b}^{\pi, \bm{b}}_j)    \leq \frac{p_{\max}}{1-p_{\max}} \leq \frac{1-c}{c}.
\end{equation}
Using potential arguments, we derive the following.
\begin{corollary}\label{corr:limited-unconcetrated}
For a fixed $\delta > 0$, the optimal belief walk initiating from prior $\bm{q}$ can contain at most $H = \left\lceil\frac{(1 - c)^2}{\delta \cdot (1 - \delta) \cdot c^3}\right\rceil$ $\delta$-unconcentrated beliefs.
\end{corollary}
Corollary~\ref{corr:limited-unconcetrated} ensures that the belief walk ultimately enters a sub-space of concentrated beliefs. The next theorem suggests that the optimal recommendation in concentrated beliefs is myopic; therefore, as long as the belief walk remains concentrated, the optimal policy is fixed.
\begin{theorem}\label{thm:myopic-near-boundary}
For every user type $m \in M$ and $(\frac{c^2}{4}, m)$-concentrated belief $\bm{b}$, $\pi^{\star}_1(\bm{b}) = \argmax_{k \in K} \bm{P}(k, m)$.
\end{theorem}
However, we still have one edge case to cover: The belief walk could potentially transition from a $(\frac{c^2}{4},m)$-concentrated belief for some arbitrary $m \in M$ to another $(\frac{c^2}{4},m')$-concentrated belief for $m' \neq m$. If such a case occurs, the optimal policy will diverge. To that end, we provide the following lemma.
\begin{lemma}\label{lemma:concentrated-transition}
For any two distinct user types $m, m' \in M$ and $(\frac{c^2}{4},m)$-concentrated belief $\bm{b}$, $\tau(\bm b, \pi^\star_1(\bm b))$ is not $(\frac{c^2}{4},m')$-concentrated.
\end{lemma}
Lemma~\ref{lemma:concentrated-transition} guarantees that transitions from concentrated beliefs of one type to another must include rounds with unconcentrated beliefs; however, Corollary~\ref{corr:limited-unconcetrated} limits the number of such transitions. Together, Corollary~\ref{corr:limited-unconcetrated} and Lemma~\ref{lemma:concentrated-transition} indicate that after some finite time $T$ the belief remains $(\frac{c^2}{4},m')$-concentrated for one single type $m$ indefinitely, and Theorem~\ref{thm:myopic-near-boundary} suggests that the optimal policy converges to the myopic, fixed policy from $T$ onward.
\end{proof}