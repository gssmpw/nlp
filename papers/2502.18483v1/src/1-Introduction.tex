\section{Introduction}

Recommender systems (RSs) have become essential in digital media and e-commerce. They collect massive amounts of data and apply sophisticated techniques to improve user engagement and satisfaction. RSs leverage past user interaction, demographics, and possibly additional information to provide personalized recommendations. For instance, Netflix and Amazon utilize these techniques to offer tailored movie recommendations and product suggestions, respectively. While RSs typically collect and record user data, some have limited access to individual user information or none at all. For instance, regulations such as the General Data Protection Regulation (GDPR) and the California Consumer Privacy Act (CCPA) impose stringent rules on data usage and user consent. Prioritization of privacy is also a trend within commercial companies, e.g., Apple's iOS 14 opt-in device tracking modification that has affected targeted advertising~\cite{kollnig2022goodbye}. In some cases, the RS can record \emph{user sessions} of varying length, but cannot identify users. While these regulatory shifts aim to protect privacy, they pose substantial challenges to RSs in maintaining the same level of service. Due to these privacy-preserving limitations, RSs may at times be limited to utilizing \emph{aggregated user information}, such as clusters of users or personas.

Relying solely on aggregated information significantly affects the recommendation process. When interacting in a user session without accurate individual data, the RS must explore more intensively to understand the current user's preferences. This initial exploration phase is critical for gathering enough data to make accurate recommendations later in the session. However, aggressive exploration bears the risk of \emph{user churn}, where users may leave the system due to receiving unsatisfactory recommendations. For example, when a song-recommendation RS encounters a user with unknown preferences, it could suggest a song from an unconventional genre, which some users enjoy while most users dislike. Although user feedback on that unconventional genre can provide valuable insights into their preferences, it can cause users to leave the RS if the recommendation is off-target; thus, the RS should address the possibility of churn as part of its design.

In this paper, we propose a model to study the intertwined challenges of aggregated user information and churn risk. We call our model $\prob$, standing for \textbf{Rec}ommendation with \textbf{A}ggregated \textbf{P}references under \textbf{C}hurn. Our model assumes that the RS has a probabilistic prior over user types (clusters, personas, etc.) and aggregated satisfaction levels for each content type (genre, etc.) with respect to each user type. Each user session involves an unidentified user whose type is sampled from this prior distribution and is unknown to the RS. The RS recommends content sequentially, receiving binary feedback (like or dislike) from the user. This feedback allows the RS to infer the user type in a Bayesian sense to improve future recommendations. However, careless present recommendations can lead to user churn. The objective of the RS is to maximize user utility, defined as the number of contents the user likes.

\subsection{Our contribution}
Our contribution is three-fold. First, we are among the first to address aggregated user preferences and the risk of churn simultaneously. We propose a model where the general population preferences are known and the type (cluster, persona, etc.) of users are hidden but can be deduced through interaction. This interaction drives user engagement, which we model as the RS's reward. However, due to uncertainty about user type, the RS could generate unsatisfactory recommendations that can cause user churn. Our model poses a novel exploration-exploitation trade-off with the risk of churn, an aspect under-explored in the current literature.

Our second contribution is technical. Analyzing the model shows that optimal policies converge after a finite number of recommendations, symbolizing a transition to pure exploitation. Informally,
\begin{theorem}[Informal statement of Theorem~\ref{thm:convergence}]
For a broad range of instances, the (infinite-length) optimal policy converges.
\end{theorem}

Our third contribution is algorithmic. We leverage our theoretical results to develop a straightforward yet state-of-the-art branch-and-bound algorithm designed explicitly for our setting. As the problem we address can be described as a partially observable Markov decision process (POMDP), we compare our algorithm with the state-of-the-art POMDP benchmark~\cite{kurniawati2009sarsop}. In practice, our algorithm performs better when there is a large variety of user types but is less effective when the number of contents is significantly larger than the number of user types.

%While significant research has been conducted in the areas of privacy-preserving techniques, addressing the cold start problem, optimizing sequential recommendations, and user retention, there is limited work that effectively integrates all four aspects. The intersection of these challenges is particularly complex, as they introduce competing requirements for the RS. For instance, while some research addresses the cold-start problem alongside sequential recommendations, it may not fully consider the complexities introduced by the user-churn concerns, which makes initial exploration more challenging. Conversely, approaches that deal with the cold start problem in a privacy-sensitive manner might not account for the long-term dynamics of user satisfaction in sequential settings. Our work aims to bridge this gap by proposing a unified model that addresses the intertwined issues of privacy preservation, user churn, and sequential recommendations in RS.

%In our model we leverage aggregated user data to plan recommendations for new users, ensuring privacy and anonymity while maximizing user engagement. We introduce a probabilistic prior over user types to guide the recommendation process, minimizing the need for specific user data and preserving privacy. We thoroughly examine the complexity of this issue, suggest approximation techniques for identifying optimal recommendation policies, assess the attributes of the optimal recommendation strategy, and introduce a branch-and-bound algorithm to solve the recommendation challenge efficiently. Our empirical validation shows enhanced efficiency compared to standard methods, underscoring the practical relevance of our approach in real-world applications. This work pushes the field of RS forward by providing a solution that strikes a balance between privacy issues and the necessity for effective user recommendations.

\subsection{Related work}
Our work captures several phenomena: First, we assume that the RS has aggregated user information, but no access to individual user information. Second, we have sequential interaction, where we can balance exploration-exploitation trade-offs. And third, our model includes the risk of user churn. Below, we review the relevant literature strands.

\paragraph{Aggregated user information}
Our model follows the trend of using clustered data in the recommendation process~\cite{recommender-systems-for-large-scale-e-commerce-scalable-neighborhood-formation-using-clustering}. In addition to improved efficiency, the use of clustering can increase the diversity and reliability of recommendations \cite{a-clustering-approach-for-personalizing-diversity-in-collaborative-recommender-systems, robust-collaborative-filtering-based-on-multiple-clustering} and handle the sparsity of user preference matrices \cite{recommender-systems-clustering-using-bayesian-non-negative-matrix-factorization}.

Aggregated information also relates to privacy, a topic that has gained much attention recently, following the seminal work of \citet{differential-privacy} on differential privacy. Several works propose RSs that satisfy differential privacy \cite{differentially-private-recommender-systems-building-privacy-into-the-netflix-prize-contenders, differentially-private-collaborative-coupling-learning-for-recommender-systems, differential-privacy-for-collaborative-filtering-recommender-algorithm}. In a broader context, \citet{the-effect-of-online-privacy-information-on-purchasing-behavior-an-experimental-study} have empirically shown that users value their privacy and are willing to pay for it. Several other definitions of privacy were suggested in the literature \cite{an-agent-based-approach-for-privacy-preserving-recommender-systems,
enhancing-privacy-and-preserving-accuracy-of-a-distributed-collaborative-filtering,svd-based-collaborative-filtering-with-privacy}. In our work, we assume that the RS has access to aggregated data, akin to other recent works addressing lookalike clustering~\cite{anonymous-learning-via-look-alike-clustering-a-precise-analysis-of-model-generalization, interactive-and-explainable-point-of-interest-recommendation-using-look-alike-groups}.

The cases where the RS has little information about users or items are called \emph{cold-start} problems. This issue relates to our work because, while we assume access to aggregated information, every user interaction starts from a tabula rasa. Broadly, solutions are divided into data-driven approaches~\cite{a-heterogeneous-information-network-based-cross-domain-insurance-recommendation-system-for-cold-start-users, transfer-meta-framework-for-cross-domain-recommendation-to-cold-start-users, alleviating-data-sparsity-and-cold-start-in-recommender-systems-using-social-behaviour} and method-driven approaches~\cite{personalized-adaptive-meta-learning-for-cold-start-user-preference-prediction, task-adaptive-neural-process-for-user-cold-start-recommendation, meta-matrix-factorization-for-federated-rating-predictions} (see \citet{user-cold-start-problem-in-recommendation-systems-a-systematic-review} for a recent survey). 

This paper is \emph{inspired} by clustering, privacy, and cold-start problems. However, our model only assumes access to aggregated information and abstracts the reasons why individual information is unavailable. Notably, we do not propose techniques to cluster users, address privacy concerns, or provide new approaches for the cold-start problem.

\paragraph{Sequential recommendation with churn}
Our model falls under Markov Decision Process (MDP) modeling for RSs~\cite{an-mdp-based-recommender-system}, where the belief over user types represents the state. Alternatively, we can formalize it as a Partially Observable MDP (POMDP), where the state corresponds to the user type that remains constant but is initially unknown. Both MDP and POMDP modeling are well-studied in the literature of RSs~\cite{optimal-recommendation-to-users-that-react-online-learning-for-a-class-of-pomdps, interactive-recommendation-with-user-specific-deep-reinforcement-learning, recommendation-as-a-stochastic-sequential-decision-problem}, and typically the main task is to learn the underlying model~\cite{empirical-evaluation-of-gated-recurrent-neural-networks-on-sequence-modeling, usage-based-web-recommendations-a-reinforcement-learning-approach}.

We model user churn as part of the sequential recommendation process, thereby generating an exploration-exploitation trade-off. User churn is an integral part of many systems, and most of the literature addresses user churn prediction~\cite{user-retention-a-causal-approach-with-triple-task-modeling,quantifying-and-leveraging-user-fatigue-for-interventions-in-recommender-systems} and techniques to retain users~\cite{e-government-deep-recommendation-system-based-on-user-churn,surrogate-for-long-term-user-experience-in-recommender-systems}. Several recent works~\cite {maximizing-cumulative-user-engagement-in-sequential-recommendation-an-online-optimization-perspective, returning-is-believing-optimizing-long-term-user-engagement-in-recommender-systems, partially-observable-markov-decision-process-for-recommender-systems, cao2020fatigue} adopt a similar approach to ours, directly modeling user churn due to irrelevant recommendations. 

% \omer{GUR TO PUT THE REFERENCES IN PLACE}. \omer{GUR - please take a look at the last paragraph of \url{https://arxiv.org/pdf/2203.13423}}. 

The paper most relevant to ours is the one by \citet{modeling-attrition-in-recommender-systems-with-departing-bandits}. They focus on the problem of online learning with the risk of user churn under user uncertainty. While they study the stochastic variant, where the user-type preferences are initially unknown, their analysis is restricted to one user and multiple categories, or two users and two categories. In contrast, we assume complete information, but we study general matrices of any size.

% \omer{BELOW - OLD}

% tackled maximizing cumulative user engagement by treating recommendation as an MDP, optimizing immediate and long-term browsing behavior with dynamic programming.

% Another layer of difficulty to sequential recommendation is the challenge of user retention and churn. User retention is a critical metric for the success of RS, reflecting the system's ability to keep users engaged and returning to the platform over time. High retention rates indicate that users are satisfied with the recommendations, leading to continued interaction with the system. On the other hand, user churn is the opposite phenomenon, where users stop engaging with the RS, either by reducing their interactions or abandoning the platform altogether. \gur{concise this paragraph}
% As retaining existing customers is considered easier and less expensive than acquiring new ones (\citet{acquisition-versus-retention-competitive-customer-relationship-management}), improving user retention is a key focus for many RS. Some methods focus on predicting churn to identify at-risk users and intervene before they leave the platform (\cite{user-modeling-for-churn-prediction-in-e-commerce,user-retention-a-causal-approach-with-triple-task-modeling,interpretable-user-retention-modeling-in-recommendation,quantifying-and-leveraging-user-fatigue-for-interventions-in-recommender-systems}). 
% Others aim to optimize long-term user engagement directly by providing personalized recommendations (\cite{maximizing-cumulative-user-engagement-in-sequential-recommendation-an-online-optimization-perspective,e-government-deep-recommendation-system-based-on-user-churn,surrogate-for-long-term-user-experience-in-recommender-systems,an-integrated-framework-to-recommend-personalized-retention-actions-to-control-b2c-e-commerce-customer-churn,returning-is-believing-optimizing-long-term-user-engagement-in-recommender-systems}).











