\section{Model Extensions and POMDP Framework}
\label{sec:model-extensions}

Our model provides a foundational framework for understanding recommendation systems under aggregated user information and churn risk. However, its binary interaction model -- where users either like and stay or dislike and leave -- represents a simplification of real-world user behavior. In this section, we discuss how our model can be extended within the broader POMDP framework to capture more nuanced user interactions.

\subsection{Richer User Interactions}

The current model can be extended to accommodate more complex user behaviors and reward structures. For instance, users might dislike a recommendation but remain in the system, potentially with a modified engagement level. Similarly, users might express varying degrees of satisfaction, from ``superlike'' to mild approval, each contributing differently to the system's reward. These extensions can be naturally formulated within the POMDP framework while maintaining our state space (user types) and action space (recommendation categories). The key enrichment comes through more complex transition, observation, and reward models. The transition function would capture various user responses, including remaining in the system after negative feedback and possibly changing preferences over time; the observation model would accommodate different levels of feedback (e.g., ratings on a scale), and the reward function would assign different values to these varied interaction types. This richer probabilistic structure better reflects real-world user behavior while preserving the fundamental representation of states as user types.

\subsection{Learning from Historical Data}

While our current model assumes known parameters ($\mathbf{P}$ and $\mathbf{q}$), a more practical approach would involve estimating these parameters along with the transition, observation, and reward models from historical data before deployment. The POMDP framework readily accommodates such model estimation - historical interaction data can approximate how users transition between states, how different types of feedback relate to underlying user types, and what rewards are obtained from various interactions. The quality of these empirical estimates directly impacts the effectiveness of the resulting recommendation policy obtained through POMDP solvers. While this introduces additional complexity in model specification, it allows the system to better reflect real-world user behavior patterns observed in the data.

\subsection{Computational Considerations}

The extension to more complex POMDPs introduces significant computational challenges that fundamentally alter our solution approach. While our current model admits specialized solutions leveraging the structure of belief walks, this structure breaks down in the general POMDP setting. 

First, in our simplified model, we could effectively represent policies as sequences of actions by implicitly conditioning on the user remaining in the system. This representation was computationally advantageous as it reduced the policy search space. However, with multiple possible feedback types, we need to find a proper policy -- a mapping from beliefs to actions -- that specifies optimal recommendations for every possible belief state. This more complex policy structure is inherently harder to optimize than our current sequence-based approach.

This fundamental shift in policy representation directly impacts our branch-and-bound algorithm. In our current model, the algorithm exploits the fact that only positive feedback (``like'') is informative for future recommendations, as negative feedback ends the episode. However, when policies must account for multiple feedback types, every observation could potentially influence future recommendations. As a result, evaluating each policy prefix requires considering multiple possible feedback branches, significantly expanding the solution space that must be explored.

These challenges suggest two potential paths forward. One approach would be to analyze the specific dynamics induced by the extended model and develop specialized POMDP solvers that exploit any structural properties we can identify. The feasibility of this approach depends heavily on the particular model extensions chosen and the resulting state-action-observation structure. Alternatively, we could leverage existing general-purpose POMDP solvers, which are already quite efficient but might be further optimized for our specific recommendation domain by incorporating domain-specific heuristics or approximations.

\subsection{Theoretical Implications}

The extension to general POMDPs significantly impacts the theoretical analysis presented in this paper. The clean theoretical properties that we established, particularly in Section~\ref{sec:convergence-of-the-optimal-policy}, rely fundamentally on the deterministic nature of belief walks in our current model. When this determinism breaks down, several key theoretical challenges emerge:

\begin{itemize}
    \item Multiple possible future beliefs from each action prevent the definition of a unique belief walk
    \item Convergence results based on monotonic progression through the belief simplex no longer apply
\end{itemize}

New theoretical frameworks would be needed to analyze the behavior of optimal policies in these more general settings, likely focusing on probabilistic convergence properties rather than deterministic ones.

\subsection{Future Research Directions}

The extension to general POMDPs opens several promising research directions. First, developing specialized solvers that exploit the particular structure of recommendation systems could improve computational efficiency. Second, theoretical analysis of approximate solution methods could provide performance guarantees for practical implementations.

These extensions represent a natural evolution of our model toward more practical applications, though at the cost of some theoretical elegance. The challenge for future research lies in finding the right balance between model complexity and analytical tractability while maintaining computational feasibility.