\section{MovieLens Experiments}\label{sec:movielens}

This section demonstrates an end-to-end procedure for applying our Branch and Bound algorithm to real-world recommendation data. Specifically, we show how to extract model parameters from the MovieLens 1M dataset, a widely used benchmark for collaborative filtering algorithms, and evaluate the performance of our approach against the baseline. This demonstrates that our method not only works theoretically but can be effectively applied to actual recommendation system data, providing insights into its practical efficiency and scalability.

\subsection{Dataset Description}
We evaluate our Branch and Bound approach using the MovieLens 1M dataset, containing $1$ a million ratings from $6,040$ users on $3,706$ movies. Each rating is an integer value between $1$ and $5$ stars, representing a realistic recommendation system scenario with typical characteristics such as sparsity and varying user activity levels.

\subsection{Clustering Methodology}

As part of our end-to-end pipeline for extracting model parameters, we first need to identify user types and item categories from raw rating data. This step is crucial for obtaining the inputs required by our Branch and Bound algorithm: the probabilities-to-like matrix and the prior distribution over user types.

We employ Spectral Co-clustering \citep{coclustering}, a well-established method for simultaneously clustering users and items in recommendation matrices \citep{george2005scalable}. This approach captures the dual nature of user-item interactions, identifying subgroups with distinct rating patterns that can serve as our user types and item categories.

For larger systems that might experience an imbalance between number of user types and item categories, we suggest applying additional clustering methods (e.g., k-means or hierarchical clustering) to decompose larger clusters. Since spectral co-clustering returns the same number of clusters on both domains, this post-processing step helps maintain balanced representation while preserving the computational advantages of our approach.

\subsection{Score Transformation}

The second step in our parameter extraction pipeline involves converting the clustered rating matrix into the probability matrix required by our model. For each user-type and item-category pair (identified through clustering), we calculate the mean rating and normalize it to the $[0,1]$ interval by dividing by the maximum rating ($5.0$). This transforms the raw ratings into probabilities that represent the likelihood of a user type "liking" a particular content category.

While this is a straightforward approach, alternative transformations can be employed, such as using percentile rankings or more sophisticated normalization techniques. Our choice prioritizes simplicity and interpretability while maintaining the relative preferences between clusters. The resulting probability matrix, combined with the prior distribution derived from cluster sizes, provides a complete instance of our model derived entirely from real-world data.

\subsection{Experimental Setup and Results}

Using the parameters extracted from the MovieLens dataset through our pipeline, we compare the Branch and Bound algorithm with SARSOP across different matrix sizes. This evaluation uses real-world derived parameters rather than synthetic ones, providing insight into the algorithms' performance in practical scenarios. The initial state distribution (prior) is derived from the proportional size of user clusters, reflecting the actual distribution of user types in the system.

To assess robustness to uncertainty in preference estimation, we add Gaussian noise ($\sigma = 0.01$) to the extracted probability matrices. We test both algorithms on $10$ random noise samples for each matrix size, and run each input $1000$ times to account for variations in runtime due to internal computational randomness. The resultant 95\% bootstrap confidence intervals are plotted as shaded regions around the mean curves, though they appear notably narrow due to the high consistency of runtime performance across experimental runs.

This evaluation with real-world derived parameters demonstrates that both algorithms achieve optimal solutions with markedly distinct computational characteristics. As illustrated in Figure~\ref{fig:movielens}, the Branch and Bound algorithm consistently outperforms SARSOP across all matrix dimensions, with the performance differential widening substantially as the problem size increases. This systematic superiority in computational efficiency, coupled with maintained solution quality, strongly indicates the algorithm's particular suitability for large-scale real-world recommendation systems. 

\begin{figure}
    \centering
    \includegraphics[width=0.5\linewidth]{simulations/Algorithm_Runtime_Comparison_on_MovieLens_Data}
    \caption{Comparative runtime analysis of Branch and Bound versus SARSOP algorithms on MovieLens dataset. Runtime (in seconds) is plotted against matrix dimension, with shaded regions representing 95\% bootstrap confidence intervals (indistinguishable due to high consistency across runs).}
    \label{fig:movielens}
\end{figure}