\section{Related Works}
\subsection{Social Media for Networking} The web is full of networking solutions to facilitate connections for academia. Several existing platforms and research studies have explored similar objectives, aiming to address the challenges associated with traditional networking. From approaching peers in person to connecting on digital platforms, there have been significant advancements in networking methodologies. Udenze and Silas \cite{ref1} in their paper study academic awareness and the usage of LinkedIn for academic networking using a non-probability-based sampling method and conclude that scholars have not fully embraced the power of LinkedIn to build academic collaboration due to the low level of awareness. Similarly, Ritesh Chugh et al \cite{ref11} conducted a scoping survey review focusing on the use of social media by academics highlighting the need for higher education institutions to provide awareness to its stakeholders of the applicability of social media for academia. A study on the benefits and problems of using general social networking sites in the scholastic domain has concluded to be a lot more negative than positive as observed by Katy Jordan and Martin Weller \cite{ref2} in their research. Generic social networking platforms prioritize commercial usage and marketing rather than upholding academic interests. On a more fundamental side, Heffernan \cite{ref12} studies consequences of academic networking and surveys the effect on overall career opportunities and faculty relationships. His findings suggest that merit-based achievements can be overshadowed by network opportunities and good connections impact individual's aspirations proving to be a key motivation to our work.

\subsection{Academic Specific Networking and Recommendation Systems} Conole et al \cite{ref13} present 'Cloudworks' a web application developed to share learning and teaching ideas across users. The work also explores various theoretical frameworks towards connecting people and the downside of maintaining a self sustaining user base for such platforms. 
Adeniyi et al \cite{ref3} in their paper, investigate the significance of academic-specific social networking tools in contrast to the generic platforms discussed previously. The study highlights the poor usability of such niche-specific platforms and proves to be clumsy for a new user, spending a significant amount of time getting acquainted with the interface. Further, academic sites lack a personalized touch of giving you relevant collaboration information due to the sheer volume of data as concluded by Kong et al \cite{ref4} in their publication. Ko et al \cite{ref16} review recommendation models and techniques giving comprehensive overview of existing systems in application service fields further enhancing the basic understanding of such systems for this study . Zhang et al \cite{ref6} investigate various scholarly recommendation systems including literature and author collaborator systems emphasising the importance for better techniques than content based and collaborative filtering. Nikhat Akhtar and Devendra Agarwal \cite{ref7} explore a research paper recommendation system employing several machine learning paradigms . Similarly, a program and course recommendation algorithm was discussed by Mohammed Ezz and Ayman Elshenawy \cite{ref8} in their research . StudieMe is a college suggestion system published by Vidish Sharma et al \cite{ref9} that uses text-based Cosine Similarity to match relevant data. Presenting a collaborative filtering recommendation system, Jianjun Ni et al \cite{ref15} use a two step approach applying TF-IDF and fuzzy logic to recommend data. 

\subsection{Deep Learning Based Recommendation Systems} The previous references primarily focused on conventional methods for building recommendation systems. Shifting our literature survery towards deep learning architectures, Rodríguez-Hernández et al \cite{ref17} conducted an experimental study where they compare traditional recommendation algorithms with BERT based ones and conclude that while BERT techniques have been used well in Natural Language Processing tasks the exploitation of BERT in recommendation systems is a sparsely unexplored topic. Juarto and Girsang \cite{ref19} in their work present a neural system integrating Sentence BERT with a collaborative news recommender. This integration results in a significant increase in hit ratio indicating the positive impact using deep architectures. A text recommendation system fused with Convoluted neural networks (CNN) and BERT semantic information was offered by Xingyun Xie et al \cite{ref20} demonstrating an improved feature extraction mechanism when performing small scale Top-N recommendations.