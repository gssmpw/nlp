\section{Introduction}

Hyperspectral image (HSI) classification is crucial in remote sensing applications, such as environmental monitoring, agriculture, and mineral exploration~\cite{zhang2016deep, paoletti2019deep}. However, high dimensionality and spectral redundancy in HSI data—often termed the “curse of dimensionality”—pose unique challenges, complicating effective classification~\cite{he2017recent, li2013spectral}. This redundancy hampers performance, particularly in cases where different objects share similar spectra or the same object exhibits spectral variability under different conditions~\cite{theiler2019spectral, yao2024spectralmamba}. Figure~\ref{fig:challenges} illustrates these phenomena, emphasizing the need for methods that capture essential spatial-spectral features while reducing redundant information~\cite{li2024latent,deng2023psrt}.

\begin{figure}[H]
    \centering
    \begin{subfigure}[b]{1\linewidth}
        \includegraphics[width=\linewidth]{figures/sameobj.png}
        \caption{Same object, different spectra: Spectral variability in corn types (\textit{Corn-notill} and \textit{Corn-mintill}). Highlighted curves represent specific types, while gray curves show other corn samples, reflecting intra-class variability influenced by spectral redundancy and high correlation.}
    \end{subfigure}
    \vfill
    \begin{subfigure}[b]{1\linewidth}
        \includegraphics[width=\linewidth]{figures/difobj.png}
        \caption{Different objects, same spectra: Spectral similarity between \textit{Buildings-Grass-Trees-Drives} and \textit{Stone-Steel-Towers}. Highlighted curves represent these classes, with gray curves showing other land cover types, illustrating inter-class similarity caused by spectral overlap and strong correlation.}
    \end{subfigure}
    \caption{Spectral response functions illustrating HSI classification challenges due to spectral redundancy and high correlation.}
    \label{fig:challenges}
\end{figure}
\begin{figure*}[t]
    \centering
    \includegraphics[width=1\linewidth]{figures/flowchart.png}
    \caption{DCT-Mamba3D framework.}
    \label{fig:dct-mamba3d-flowchart}
\end{figure*}

\textbf{Challenges of Spectral Variability and Redundancy}: Spectral variability (caused by changing illumination, atmospheric conditions, or intrinsic material differences) and similarity between materials exacerbate classification challenges~\cite{theiler2019spectral, yao2024spectralmamba, hong2024spectralgpt}. High inter-band correlation leads to redundant information, complicating differentiation, particularly in mixed pixels where each pixel may represent multiple materials~\cite{be2014tgrs}.

\textbf{Frequency-Domain Transformations for Enhanced Feature Extraction}: Frequency-domain transformations can improve spectral separation and feature extraction in HSI classification~\cite{yan2024exploiting, qiao2023dual}. Discrete cosine transform (DCT) specifically enables decorrelation by transforming data into the frequency domain, facilitating refined feature extraction~\cite{shen2021dct, ulicny2022harmonic, xu2020learning, ulicny2019harmonic}.

Recent approaches in HSI classification explore CNN-based, Transformer-based, and Mamba-based architectures. CNN-based methods, such as 2D-CNN~\cite{yang2018hyperspectral}, 3D-CNN~\cite{yang2020synergistic}, and HybridSN~\cite{roy2019hybridsn}, primarily focus on spatial features but often overlook complex spectral correlations~\cite{chen2016deep, Jia2022tgrs, ulicny2022harmonic, Xu2023grsl}. Transformer-based models, including ViT~\cite{dosovitskiy2020image}, HiT~\cite{yang2022hyperspectral}, CAT~\cite{feng2024cat}, and MorphF~\cite{roy2023spectral}, utilize self-attention mechanisms to capture spectral dependencies but are computationally intensive and often require large datasets~\cite{scheibenreif2023masked, feng2024cat}. Mamba-based models, such as MiM~\cite{zhou2024mamba}, SpectralMamba~\cite{yao2024spectralmamba}, WaveMamba~\cite{ahmad2024wavemamba}, and Vision Mamba~\cite{zhu2024vision}, employ state-space representations to model spatial-spectral relationships without convolutional structures but face limitations in addressing spectral redundancy and inter-band correlation.

In this paper, we propose \textit{DCT-Mamba3D}, an HSI classification model that integrates a 3D Spatial-Spectral Decorrelation Module (3D-SSDM), a 3D-Mamba module, and a Global Residual Enhancement (GRE) module to reduce spectral redundancy and enhance feature extraction. 3D-SSDM uses 3D DCT basis functions to transform data into the frequency domain, enabling both spectral and spatial decorrelation and improving feature clarity for subsequent extraction layers. The 3D-Mamba module leverages a 3D state-space model to capture intricate spatial-spectral dependencies. Finally, the GRE module stabilizes feature representation, enhancing robustness and convergence.

Our contributions are as follows:
\begin{itemize}
    \item \textbf{Spectral-Spatial Decorrelation with 3D-SSDM}: The 3D Spatial-Spectral Decorrelation Module (3D-SSDM) reduces spectral and spatial redundancy using 3D DCT basis functions, enabling comprehensive feature separability in complex HSI scenarios.
    \item \textbf{Efficient Spatial-Spectral Dependency Modeling with 3D-Mamba}: The 3D-Mamba module captures both local and global spatial-spectral dependencies, enhancing efficiency and feature interaction.
    \item \textbf{Robust Feature Stability with GRE}: The Global Residual Enhancement (GRE) module stabilizes feature representation, improving robustness and convergence.
\end{itemize}

The rest of this paper is organized as follows: Section~\ref{sec:related_work} reviews related work, Section~\ref{sec:methodology} details our approach, Section~\ref{sec:experiments} presents results, Section~\ref{sec:discussion} discusses model advantages, and Section~\ref{sec:conclusion} concludes the paper.





