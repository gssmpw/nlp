\section{Methodology} \label{sec:methodology}

Our proposed \textbf{DCT-Mamba3D} comprises three main components, as shown in Fig.~\ref{fig:dct-mamba3d-flowchart}. First,  \textbf{3D Spatial-Spectral Decorrelation Module (3D-SSDM)} applies 3D DCT basis functions to convert spatial pixels into decorrelated frequency components, reducing redundancy and isolating essential features. Second, the \textbf{3D-Mamba module} uses state-space modeling and selective scanning to capture complex spatial-spectral dependencies within the decorrelated data. Finally, the \textbf{GRE module} stabilizes feature representation across layers by integrating global context, enhancing robustness and classification accuracy.


\subsection{3D-SSDM Module}

\begin{figure}[t]
    \centering
    \includegraphics[width=1\linewidth]{figures/3dssdm2.png}
    \caption{3D Spatial-Spectral Decorrelation Module (3D-SSDM), applying 3D DCT basis functions to decompose the image into independent frequency components, aiding decorrelation and feature extraction.}
    \label{fig:3dssdm}
\end{figure}

The \textbf{3D-SSDM Module} begins with a \textbf{Stem} stage for shallow feature extraction and normalization. HSIs contain hundreds of contiguous spectral bands with high inter-band correlation. To address this, 3D-SSDM applies \textbf{3D DCT basis functions} to convert spatial pixels into decorrelated frequency components across both spatial and spectral dimensions. It enhances feature clarity by consolidating most of the energy into distinct frequency components, as shown in Fig.~\ref{fig:3dssdm}.

The 3D DCT generates a set of spatial-spectral frequency components, capturing varying HSI characteristics. In a \(3 \times 3 \times 3\) setup, the 3D DCT yields 27 basis functions, with low frequencies capturing smooth variations and high frequencies capturing fine-grained details.

Representing the HSI as \( X \in \mathbb{R}^{H \times W \times C} \) (where \( H \), \( W \), and \( C \) are spatial and spectral dimensions), the 3D DCT is applied as:
\begin{equation}
X_{\text{freq}}(i, j, k) = \sum_{x=0}^{H-1} \sum_{y=0}^{W-1} \sum_{z=0}^{C-1} X(x, y, z) \cdot \psi_{i,j,k}(x, y, z),
\end{equation}
where \( \psi_{i,j,k}(x, y, z) \) represents the 3D DCT basis functions, decorrelating both spatial and spectral dimensions and extracting spatial-spectral frequency features.

The basis functions \( \psi_{i,j,k}(x, y, z) \) are defined as:
\begin{equation}
\begin{aligned}
&\psi_{i,j,k}(x, y, z) = \alpha_i \alpha_j \alpha_k \cos\left(\frac{\pi (2x + 1) i}{2H}\right) \\
& \times \cos\left(\frac{\pi (2y + 1) j}{2W}\right) \cos\left(\frac{\pi (2z + 1) k}{2C}\right),
\end{aligned}
\end{equation}
where normalization factors \( \alpha_n \) are:
\[
\alpha_n =
\begin{cases}
\sqrt{\frac{1}{N}} & \text{if } n = 0, \\
\sqrt{\frac{2}{N}} & \text{if } n > 0,
\end{cases}
\]
with \( N \) as \( H \), \( W \), or \( C \).

\subsection{3D-Mamba Module}

The \textbf{3D-Mamba module} employs a bidirectional state-space model (SSM) to capture spatial-spectral dependencies within the frequency-domain data \( X_{\text{freq}} \) from the 3D-SSDM. It operates directly on decorrelated data, distinguishing similar features and refining spatial-spectral information across all dimensions.

After decorrelation, the input \( X_{\text{freq}} \) undergoes the following stages:

\begin{itemize}
    \item \textbf{Patch Embeddings:} The 3D-Mamba module decomposes \( X_{\text{freq}} \) into spatial, spectral, and residual components via specialized embedding layers in the frequency domain. This setup separates essential features and prepares data for selective scanning:
    \begin{equation}
    \begin{aligned}
    x_{\text{spatial}} &= F_{\text{PE}_1}(X_{\text{freq}}), \quad x_{\text{spectral}} = F_{\text{PE}_2}(X_{\text{freq}}), \\
    x_{\text{residual}} &= F_{\text{PE}_3}(X_{\text{freq}}).
    \end{aligned}
    \end{equation}

    \item \textbf{Frequency Spatial and Spectral bidirectional SSM:} Using SiLU activation for non-linearity, \( x_{\text{spatial}} \) and \( x_{\text{spectral}} \) undergo selective scanning within the SSM framework, capturing spatial-spectral dependencies in independent frequency components:
        \begin{equation}
        \begin{aligned}
            h_{\text{spatial}}(t) &= A_s \cdot s(t) + B_s \cdot \text{SiLU}(x_{\text{spatial}}(t)), \\
            h_{\text{spectral}}(t) &= A_v \cdot v(t) + B_v \cdot \text{SiLU}(x_{\text{spectral}}(t)),
        \end{aligned}
        \end{equation}
        where \( s(t) \) and \( v(t) \) denote latent spatial and spectral states, and \( h_{\text{spatial}}(t) \), \( h_{\text{spectral}}(t) \) are the outputs, enhancing independence across frequency domains.

    \item \textbf{Feature Aggregation and Normalization:} The outputs \( h_{\text{spatial}}(t) \) and \( h_{\text{spectral}}(t) \) are combined with residuals \( x_{\text{residual}} \), and normalized for stability:
        \begin{equation}
        y_{\text{mamba}}(t) = \gamma_0 \cdot x_{\text{residual}} + \gamma_1 \cdot h_{\text{spatial}}(t) + \gamma_2 \cdot h_{\text{spectral}}(t),
        \end{equation}
        where \( y_{\text{mamba}}(t) \) is the final spatial-spectral feature map, combining initial decorrelated features with refined updates.
\end{itemize}

\subsection{GRE Module}

The \textbf{GRE module} enhances feature robustness by integrating global context with spatial-spectral features extracted by the 3D-Mamba module. By introducing a residual connection, the GRE module stabilizes training and preserves key information across layers.

The GRE module receives \( y_{\text{mamba}} \) from the 3D-Mamba module and combines it with \( X_{\text{freq}} \) from the 3D-SSDM to form the final output \( F_{\text{out}} \):
\begin{equation}
F_{\text{out}} = y_{\text{mamba}} + \alpha X_{\text{freq}},
\end{equation}
where \( F_{\text{out}} \) is the final feature map, and \( \alpha \) is a learnable parameter balancing contributions from \( y_{\text{mamba}} \) and \( X_{\text{freq}} \).

The output \( F_{\text{out}} \) feeds into the classification layer, completing the feature extraction pipeline for HSI classification.

For optimization, a composite loss function is used, combining cross-entropy loss and an optional regularization term for spectral decorrelation:
\begin{equation}
\mathcal{L} = \mathcal{L}_{\text{CE}} + \lambda \mathcal{L}_{\text{reg}},
\end{equation}
where \( \mathcal{L}_{\text{CE}} \) is the cross-entropy loss, \( \mathcal{L}_{\text{reg}} \) penalizes spectral redundancy, and \( \lambda \) is a regularization weight.



