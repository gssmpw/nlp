\section{Discussion}
\label{sec:discussion}

Our \textbf{DCT-Mamba3D} model addresses HSI classification challenges by leveraging spectral-spatial decorrelation and effective feature extraction through its integrated modules: \textbf{3D-SSDM}, \textbf{3D-Mamba}, and \textbf{GRE}.

\textbf{3D-SSDM for Spectral Decorrelation and Feature Extraction:}  
Using 3D DCT basis functions, 3D-SSDM reduces spectral and spatial redundancy while extracting critical features across dimensions. Its enhanced decorrelation improves the model’s ability to manage cases of high spectral similarity, such as in 'different objects, same spectra' scenarios. Ablation studies show that the 3D-SSDM alone achieves competitive accuracy, underscoring its effectiveness in isolating relevant features.

\textbf{3D-Mamba for Enhanced Spatial-Spectral Dependencies:}  
3D-Mamba effectively captures spatial-spectral dependencies through bidirectional state-space layers, particularly aiding in differentiating subtle spectral variations, as in "same object, different spectra" scenarios. This structure balances complexity while maintaining efficient feature interaction across spatial and spectral dimensions.

\textbf{GRE for Stability and Robust Training:}  
GRE integrates residual spatial-spectral features, stabilizing feature representation across layers. This stability accelerates convergence, contributing to DCT-Mamba3D’s efficient training times and robustness during learning, despite its complex architecture.

\textbf{Performance with Limited Data and Complexity Efficiency:}  
DCT-Mamba3D demonstrates strong feature discrimination even with limited training data, achieving high accuracy and Kappa scores at smaller sample sizes. Its balanced computational complexity, as shown in Table~\ref{complexity}, allows for superior accuracy compared to other models, with manageable FLOPS and parameter counts, making it suitable for resource-constrained applications.


