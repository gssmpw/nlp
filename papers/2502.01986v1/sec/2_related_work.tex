\section{Related Work}
\label{sec:related_work}

This section reviews key approaches in hyperspectral image (HSI) classification, focusing on frequency-domain analysis, Transformer-based models, and Mamba-based models for capturing spatial-spectral dependencies.

\subsection{Frequency-Domain Techniques for Image Classification}

Frequency-domain techniques, including the discrete cosine transform (DCT)~\cite{shen2021dct, ulicny2022harmonic, xu2020learning, ulicny2019harmonic}, discrete Fourier transform (DFT)~\cite{zhang2024three, wang2019frequency}, and wavelet transformations~\cite{yao2022wave}, have proven effective in enhancing feature extraction for image classification. Techniques such as Harmonic Neural Networks (HNN)~\cite{ulicny2022harmonic, ulicny2019harmonic} apply 2D DCT to capture subtle variations in the frequency domain, exploiting DCT’s ability to decorrelate and concentrate energy, thereby reducing redundancy and improving classification performance in natural image classification~\cite{ahmed1974discrete}. Similarly, 2D FFT-based methods~\cite{qiao2023dual, zhang2024three, wang2019frequency} leverage frequency information to extract discriminative features for HSI classification. WaveViT~\cite{yao2022wave} utilizes discrete wavelet transformations for multi-scale feature extraction.

\subsection{Mamba-Based HSI Classification Approaches}

Mamba-based models present a highly promising approach for hyperspectral image (HSI) classification, focusing on capturing spatial-spectral dependencies through state-space representations. These models are particularly well-suited to the high-dimensional nature of HSI data, enabling effective feature extraction without relying on traditional convolutional or attention mechanisms. Notable Mamba-based methods, such as MiM~\cite{zhou2024mamba}, SpectralMamba~\cite{yao2024spectralmamba}, and WaveMamba~\cite{ahmad2024wavemamba}, leverage state-space models to capture spatial-spectral relationships, demonstrating the potential of Mamba for HSI classification. Recent advancements, including Li et al.~\cite{li2024mambahsi}, emphasize the importance of integrating spatial and spectral features. Despite their success in modeling long-range dependencies, these approaches still face significant challenges related to the redundancy between spectral bands, which remains a critical gap in current Mamba-based methods.
