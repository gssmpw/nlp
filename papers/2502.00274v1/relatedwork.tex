\section{Related work}
The first queueing-theoretic study on AoI was presented in \cite{6195689}, where the average AoI for M/M/1, D/M/1, and M/D/1 first-come first-served (FCFS) queueing models was derived. 
As demonstrated in \cite{6310931,7415972}, implementing appropriate packet management policies in status update systems—either in the waiting queue, the server, or both—can significantly enhance information freshness.
The performance of various packet management policies in queueing systems with exponentially distributed service times and Poisson arrivals has been extensively analyzed in the literature \cite{8469047,8437591,8406966,8437907,9013935,9048914,9252168,9162681,Moltafet2020mgf,9611498,9705518}.


Besides exponentially distributed service time and Poisson arrivals, AoI has been studied under various arrival processes and service time distributions. 
 The distribution of the AoI and PAoI for the single-source PH/PH/1/1 and M/PH/1/2 queueing models were derived in \cite{9119460}.
The average AoI of a single-source D/G/1 FCFS  queueing model was derived in \cite{8406909}. A closed-form expression of the average AoI for a single-source  M/G/1/1 preemptive queueing model with hybrid automatic repeat request was derived in \cite{8006504}. The distributions of the AoI and PAoI of single-source  M/G/1/1 FCFS and G/M/1/1 FCFS queueing models were derived in \cite{8006592}.
A general formula for the distribution of the AoI in single-source single-server queueing systems was derived in \cite{8820073}. 
The average AoI and PAoI of a single-source status update system with Poisson arrivals and a service time with gamma distribution under the last-come first-served (LCFS) policy were analyzed in \cite{7541764}. 
The average AoI of a single-source G/G/1/1 queueing model was studied in \cite{9048933}. 
The distribution of the AoI in a generate-at-will single-source dual-server system was derived in \cite{Arxakar2024age}. The authors considered that the servers serve packets according to exponentially distributed service times where the sampling and transmission process is frozen for a period of time with  Erlang distribution upon each transmission.

The average AoI and PAoI in a multi-source M/G/1 FCFS queueing model were studied in    
\cite{9099557,inoue2024exact}.
The average AoI of a queueing system with two classes of Poisson arrivals with different priorities under a general service time distribution was studied in \cite{8886357}. 
The average AoI and PAoI of a multi-source M/G/1/1 queueing model under the globally preemptive packet management policy were derived in \cite{8406928}.  According to the globally preemptive packet management policy, a new arriving packet preempts the possible under-service packet independent of the source index of the packets.
The average AoI and PAoI of a multi-source M/G/1/1 queueing model under the non-preemptive policy were derived in \cite{9500775}. According to the non-preemptive policy, when the server is busy, any arriving packet, independent of their source index, is blocked and cleared.  
 The authors of \cite{9519697} considered a multi-source system with Poisson arrivals where the server serves packets according to a phase-type distribution. 
 %Using the theory of Markov fluid queues, they proposed a method to 
 They numerically obtain the distributions of the AoI and PAoI under a probabilistically preemptive policy. 
 %According to the policy, a new packet arriving from source $c$ can preempt a packet from source $c'$ in service with a probability depending on $c$ and $c'$.
The MGFs of the AoI and PAoI of a multi-source M/G/1/1 queueing model under the self-preemptive policy were derived in \cite{9869867}. According to the self-preemptive policy, a new arriving packet preempts a possible under-service packet only if they have the same source index. In addition, the authors of \cite{9869867} derived the MGFs of the AoI and PAoI of the models studied in \cite{9500775} and \cite{9869867}.
The distributions of the AoI and PAoI for a generate-at-will multi-source system with a phase-type service time distribution were derived in \cite{10139823}.
The Laplace-Stieltjes transform of the AoI for a two-source system with Poisson arrivals and a generally distributed service time was derived in \cite{10038591}. The authors assumed that there is a buffer for each source and studied three versions of the self-preemptive policy. 





%