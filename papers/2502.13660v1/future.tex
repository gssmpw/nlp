In this section, we discuss several future research directions. An intriguing open question is whether two layers of a GNN with unique identifiers (IDs) suffice to achieve additional expressiveness beyond the 1-WL test.\newline In Section~\ref{sec:theory} we demonstrated that three layers already provide greater expressiveness than 1-WL. The question of whether this can be achieved with only two layers would be interesting to study.

We showed that using ID-invariance on all layers does not enhance expressiveness, whereas enforcing invariance solely in the final layer does. Fully understanding the conditions on layers regarding ID-invariance and identifying which functions can be computed under various settings presents an interesting direction for future research.


Our experiments show that ID utilization can be improved with {\ourmethod}, by \textit{learning} to produce an invariant network. Another interesting open question is how to design GNNs that benefit from the added theoretical expressiveness of IDs \textit{and} are invariant to IDs \textit{by design}. We conjecture that this can be achieved by combining a matching oracle with a GNN architecture. \newline
A matching oracle is a function that takes two nodes from the graph as input and returns 1 if they are the same node and 0 otherwise. 
The following result (see appendix for proof) shows that any ID-invariant function can be computed with such an oracle. 
\begin{theorem}\label{thm:matching_oracle}
      Any ID-invariant function can be computed with a matching oracle.
\end{theorem}


It would be very interesting to combine such an oracle with Message-Passing GNNs in an effective way, that also results in an improved empirical performance.