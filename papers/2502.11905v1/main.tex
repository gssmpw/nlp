\documentclass[10pt]{wlscirep} %\documentclass[fleqn,10pt]{wlscirep}
\usepackage[utf8]{inputenc}
\usepackage[T1]{fontenc}
% ========================================================================
\usepackage{braket}
\usepackage{subcaption}
\usepackage{float}
\usepackage{dblfloatfix}
\usepackage{placeins}
\usepackage{adjustbox}
\usepackage{tabularx}
\usepackage{seqsplit}
\usepackage{multirow}
\usepackage{array}
\usepackage{soul}
\renewcommand{\arraystretch}{1.75}
\usepackage{colortbl}
\usepackage{pgfplots}
\usepackage{pgfplotstable}
\usepackage{pgf} % for calculating the values for gradient
% \usepackage[table]{xcolor}
\usepackage{etoolbox}
\definecolor{high}{HTML}{08ef15}  % the color for the highest number in your data set
\definecolor{low}{HTML}{2f5652}  % the color for the lowest number in your data set
\newcommand*{\opacity}{50}% here you can change the opacity of the background color!
%======================================
% Data set related!
\newcommand*{\minval}{0.0}% define the minimum value on your data set
\newcommand*{\maxval}{2.5}% define the maximum value in your data set!
%======================================
% gradient function!
\newcommand{\gradient}[1]{
    \small
    % The values are calculated linearly between \minval and \maxval
    \ifdimcomp{#1pt}{>}{\maxval pt}{#1}{
        \ifdimcomp{#1pt}{<}{\minval pt}{#1}{
            \pgfmathparse{int(round(100*(#1/(\maxval-\minval))-(\minval*(100/(\maxval-\minval)))))}
            \xdef\tempa{\pgfmathresult}
            \cellcolor{high!\tempa!low!\opacity} #1
    }}
}

%======================================


\usepackage{bm}
\makeatletter
\AtBeginDocument{\DeclareMathVersion{bold}
\SetSymbolFont{operators}{bold}{T1}{times}{b}{n}
% \SetSymbolFont{NewLetters}{bold}{T1}{times}{b}{it}
\SetMathAlphabet{\mathrm}{bold}{T1}{times}{b}{n}
\SetMathAlphabet{\mathit}{bold}{T1}{times}{b}{it}
\SetMathAlphabet{\mathbf}{bold}{T1}{times}{b}{n}
\SetMathAlphabet{\mathtt}{bold}{OT1}{pcr}{b}{n}
\SetSymbolFont{symbols}{bold}{OMS}{cmsy}{b}{n}
\renewcommand\boldmath{\@nomath\boldmath\mathversion{bold}}}
\makeatother

\def\BibTeX{{\rm B\kern-.05em{\sc i\kern-.025em b}\kern-.08em
    T\kern-.1667em\lower.7ex\hbox{E}\kern-.125emX}}
\pgfplotsset{compat=1.18}
% ========================================================================


\title{Exploring Quantum Control Landscape and Solution Space Complexity through Optimization Algorithms \& Dimensionality Reduction}

\author[1,3,*]{Haftu W. Fentaw}
\author[2,3]{Steve Campbell}
\author[1,3]{Simon Caton}
\affil[1]{School of Computer Science, University College Dublin, Dublin, Ireland}
\affil[2]{School of Physics, University College Dublin, Dublin, Ireland}
\affil[3]{Centre for Quantum Engineering, Science, and Technology, University College Dublin, Dublin, Ireland}
\affil[*]{haftu.fentaw@ucdconnect.ie}


\keywords{
Genetic Algorithms (GA), Principal Component Analysis (PCA), Quantum Control Landscape (QL), Reinforcement Learning (RL), Stochastic Gradient Descent (SGD)}
%\keywords{Keyword1, Keyword2, Keyword3}

\begin{abstract}
Understanding the quantum control landscape (QCL) is important for designing effective quantum control strategies. In this study, we analyze the QCL for a single two-level quantum system (qubit) using various control strategies. We employ Principal Component Analysis (PCA), to visualize and analyze the QCL for higher dimensional control parameters. Our results indicate that dimensionality reduction techniques such as PCA, can play an important role in understanding the complex nature of quantum control in higher dimensions. Evaluations of traditional control techniques and machine learning algorithms reveal that Genetic Algorithms (GA) outperform Stochastic Gradient Descent (SGD), while Q-learning (QL) shows great promise compared to Deep Q-Networks (DQN) and Proximal Policy Optimization (PPO). Additionally, our experiments highlight the importance of reward function design in DQN and PPO demonstrating that using immediate reward results in improved performance rather than delayed rewards for systems with short time steps. A study of solution space complexity was conducted by using Cluster Density Index (CDI) as a key metric for analyzing the density of optimal solutions in the landscape. The CDI reflects cluster quality and helps determine whether a given algorithm generates regions of high fidelity or not. Our results provide insights into effective quantum control strategies, emphasizing the significance of parameter selection and algorithm optimization.
\end{abstract}
\begin{document}

\flushbottom
\maketitle

\thispagestyle{empty}

\section{Introduction}

\label{sec:introduction}
As the demand for advanced computational capabilities continues to rise, there is a growing need for alternatives to solve complex problems that are either beyond the reach of classical computers or could use a boost with improved computational capability. Quantum computing has emerged as a viable alternative with the potential to tackle certain classes of problems currently not amenable with today's classical algorithms, such as factoring large numbers (Shor's algorithm) \cite{Shor_1997}, or speeding up certain algorithms as in quantum machine learning \cite{Biamonte_2017}. However, the inherent fragility of quantum states means that to harness this potential, effective quantum control strategies are essential for manipulating quantum states. In order to design an effective control strategy it is therefore important to understand the quantum control landscape (QCL). 

The QCL refers to the multidimensional space of control variables and represents the relationship between the control parameters of a quantum system and the associated performance measures, such as the distribution of candidate optimal solutions in any local neighborhood or the trajectories through control space to the optimal solution~\cite{C4CP03853C, 9502048}. The study of QCLs is critical for understanding the controllability, and corresponding optimality, that is achievable by manipulating the quantum systems using, e.g., external control fields. Exploration of this landscape helps in identifying the optimal path a quantum system takes during a transition to a target state \cite{Chakrabarti_2007}. By varying the properties of the control pulses (i.e. applying variable external fields), the state of the system can be driven to a desired state from its initial configuration~\cite{PhysRevA-84-012109}. 

For a closed system, the process during which this change in the state occurs is governed by the time-dependent Schr\"odinger equation
\begin{equation}
    \label{eq:SE}
    i\hbar\frac{\partial}{\partial t}\ket{\psi(t)} = H(t)\ket{\psi(t)}
\end{equation}
where $H(t)$ is the Hamiltonian of the system which describes the total energy and $\ket{\psi(t)}$ is the time evolved state of the system. In general, solving~\eqref{eq:SE} is a difficult problem for an arbitrary time-dependence. One approach is to determine $\ket{\psi(t)}$ using a discrete time analysis by assuming the overall evolution time, $T$, is divided into $N$ equal partitions of size $\Delta t=T/N$ such that the state at time $t+\Delta t$ is given by
\begin{equation}
\ket{\psi(t+\Delta t)} = e^{-\frac{i}{\hbar}H(t)\Delta t} \ket{\psi(t)}
\label{timeEvolution}
\end{equation}

In quantum control problems it is common to express the time-dependence in the Hamiltonian, $H(t)$ as
\begin{equation}
    H(t) = H_d + u(t)H_c  
    \label{Hamiltonian}
\end{equation}
where $H_d$ is the drift Hamiltonian and $u(t) H_c$ is the control Hamiltonian. The objective of quantum control is, therefore, to find a $u(t)$ that will maximize the probability of achieving some target objective, e.g. evolving to a given final state~\cite{Larocca_2018, Koch2022-bo}. The cost function for the control is then given by the quantum state fidelity
\begin{equation}
    F = |\braket{\psi(T)|\Psi}|^2
    \label{fidelity}
\end{equation}
where $\ket{\psi(T)}$ is the state at the end of the evolution and $\ket{\Psi}$ is the desired target state.

The QCL allows us to analyze the complexity of the solution space by plotting the combination of discrete field values that will result in high target state fidelity values. While for simple piece-wise constant pulses consisting of only two segments, i.e. $N=2$, %in equation \eqref{timeEvolution},
 the full QCL can be plotted, in order to visually inspect the complexity of the solution space for higher dimensions, i.e. $N>2$, we need to resort to the use of \textit{dimensionality reduction} techniques.

Finding the right combination of control parameters which will result in achieving a desired target is a complex task that requires the use of different optimization algorithms. There is a rich body of work exploring different approaches, including notable examples such as Gradient Ascent Pulse Engineering (GRAPE)~\cite{KHANEJA2005296}, stochastic gradient descent (SGD)-based works such as~\cite{Ferrie190404}, and reinforcement learning (RL)-based approaches as demonstrated in \cite{Zhang_2019}.  

In this work, the performance and suitability of traditional optimization algorithms, such as SGD, genetic algorithms (GA), and RL algorithms, for exploring the QCL are presented. The high level description of the state transfer problem and steps followed in our work is presented in Fig.~\ref{fig:highleveldiagram}. We focus on the control of a single two-level quantum system (qubit) initially in state $\ket{0}$ and aim to perform a bit flip operation to state $\ket{1}$ as seen in Fig.~\ref{fig:highleveldiagram}(a) using a control pulse. The qubit's evolution is governed by~\eqref{timeEvolution} and \eqref{Hamiltonian}, leading to a piece-wise constant control signal, as illustrated in Fig.~\ref{fig:highleveldiagram}(b). The QCL for different input combinations is generated using a brute-force approach (for later comparison), the results for each parameter are passed through a 2 component PCA to obtain a 2D landscape and the PCA loadings (which will be used in later stages of the analysis). Details of this are presented in section~\ref{sec:dimensionalityReduction}. Following this, Fig.~\ref{fig:highleveldiagram}(c), shows the experiments we perform using different optimization algorithms to generate the control pulses (we refer to section~\ref{sec:experimentalsetup}). After the control pulses are generated,  they are passed through PCA (using PCA loadings from previous step) to obtain the 2D landscape. Once the 2D landscapes are plotted, we perform a number of analyses to better understand the complexity of the landscape, and find the algorithms which can explore the landscape well. Details of this can be found in sections~\ref{sec:results} and~\ref{sec:solutionspacecomplexity}.

\begin{figure}[htbp]
    \centering
        \includegraphics[width=0.9\textwidth]{high_level_diagram/High_level_diagram.png} 
    \caption{High level diagram: The basic steps involved in this work: (a) the state transfer problem, (b) the brute-force solution (by converting the continuous control signal into discrete values ), (c) optimization algorithms based solution}
    \label{fig:highleveldiagram}
\end{figure}


The main contributions of this work are:
\begin{itemize}
    \item Investigation of how well different algorithms explore the solution space / landscape with the view to characterize the potential suitability of different techniques. 
    \item Introduced Dimensionality Reduction (DR) techniques to visualize and study higher dimensional quantum control landscapes
    \item Designed proper RL networks and reward signals specifically for short time step (small steps per episode) quantum control problems
    \item Analysis of the solution space complexity - to better understand how easy or difficult it is to find high fidelity solutions in certain regions of the landscape and which algorithms are best suited for this. 
\end{itemize}

\section{Prior Related work}
\label{sec:relatedwork}
The study of QCLs is well established at this point and one of the key insights we can extract from QCL analysis is whether the landscape is trap-free or it contains local minima that can hinder optimization~\cite{PhysRevA-84-012109}. It therefore follows that understanding the structure of the landscape can help avoid these traps and improve the efficiency of control strategies~\cite{PhysRevA-84-012109}. When no constraints are imposed on the available controls for the system, the QCL is devoid of local minima and is trap-free~\cite{Russell_2017}. However, in many real-world scenarios, some constraints are unavoidable. As shown, for example, in~\cite{Larocca_2020_PhysRevA.101.023410}, the QCL is not trap-free in most practical situations where constraints are imposed on the system, making it necessary to use more sophisticated optimization techniques to escape traps. 

In Ref.~\cite{Larocca_2018}, the authors conducted a detailed analysis of the QCL for a single qubit bit-flip using a two-parameter control system for various evolution times, both below and above the so-called quantum speed limit~\cite{Deffner2017}. The quantum speed limit, $T_{min}$ sets a fundamental lower bound on the time necessary to connect two quantum states and for a single qubit its value can be analytically determined as in~\cite{Hegerfeldt_2013, Poggi2013}. Their findings indicate that for $T/T_{min}<1$, the landscape is relatively simple, featuring only a single global maximum but importantly with a fidelity, $F<1$. As the total evolution time is increased well beyond the quantum speed limit, the landscape becomes more complex and there are a number of local maxima which are able to achieve perfect state transfer. We extend their work to explore details of the landscape in higher dimensions, and how the possible ridges and valleys in higher dimensions can influence the solution space.

It is worth highlighting that the QCL can also reveal important physical insights into the underlying quantum dynamics. For example, the presence of broad plateaus or ridges in the landscape may suggest that a wide range of control parameters can achieve near-optimal performance, highlighting the robustness of certain quantum systems to noise and perturbations\cite{PhysRevA.86.052117, PhysRevA-84-012109}. On the contrary, if a quantum system has steep landscapes, this suggests that the system is highly sensitive and it requires more precise control strategies to find the optimal control pulses. 

In order to study the details of the QCL at higher dimensions, a technique to reduce the number of dimensions or features of a dataset while retaining as much information as possible is necessary. Dimensionality reduction algorithms such as Principal Component Analysis (PCA)~\cite{MACKIEWICZ1993303}, t-distributed Stochastic Neighbor Embedding (t-SNE)~\cite{JMLR:v9:vandermaaten08a} and Uniform Manifold Approximation and Projection for Dimension Reduction~(UMAP)~\cite{mcinnes2020umapuniformmanifoldapproximation} are among the most commonly used techniques. The decision of choosing which algorithm to use depends on several factors including how sensitive the algorithm is to small changes in its parameters and how well the algorithm preserves the structure of the data after applying it. There are studies into these details (e.g. \cite{Huang2022_comprehensive_DR}) that can assist in this decision process.  For example, authors in Ref.~\cite{berger2024dimensionalityreductionclosedloopquantum} applied dimensionality reduction in quantum gate calibration, PCA is used in Ref.~\cite{li2018visualizinglosslandscapeneural} to visualize the loss landscape in neural networks, and in Ref.~\cite{fan2024manifoldconnectednessquantumcontrol} PCA is used to project quantum control trajectories into three dimensions. 
%In our study, we experimented with t-SNE, UMAP and PCA. 
Both t-SNE and UMAP require tuning parameters that significantly influence the results, often leading to inconsistent outcomes that can be difficult to interpret~\cite{10.5555/3546258.3546459}. For instance, t-SNE relies on parameters such as perplexity, learning rate, and the number of iterations, while UMAP depends on the number of neighbors and minimum distance between points. In contrast, PCA requires only the number of principal components to be specified, making it simpler to use and resulting in more stable and interpretable outcomes. For these reasons (and also based on initial experimentation with each of these methods), we will focus on PCA to visualise the QCL. 
%Nevertheless, as with any dimensionality reduction technique, PCA entails some degree of information loss, meaning the lower dimensional landscapes  may not capture all fine details. The QCLs in this work are based on the PCA-transformed topology.
%In this work, we chose PCA as our dimensionality reduction technique since PCA is less sensitive to parameter choices unlike  tSNE or UMAP~\cite{10.5555/3546258.3546459} as PCA does not require multiple input parameters, making the results easier to interpret and less sensitive to input variations. 

The QCL is typically studied using various optimization techniques to find the best control parameters that guide the system. Traditional control algorithms including Gradient-based methods, Genetic Algorithms (GA) and Bang-Bang Control are widely used for navigating the QCL and optimizing control parameters. While the authors in Ref.~\cite{Ferrie190404} introduced a central difference based gradient approach (self guided quantum tomography) and argued this iterative approach is more efficient than other methods, those in Ref.~\cite{Brown_2023} utilized genetic algorithms for optimal quantum state control and claimed they observed fast preparation times, and resilience to noise when using genetic algorithms. The drawbacks of gradient-based algorithms is that they can get trapped in local minima, especially in systems with complex landscapes, and their sensitivity to initial conditions~\cite{Netrapalli2019-gu}. While GAs can avoid local minima more effectively than gradient-based methods, they may still suffer from inefficiencies in high-dimensional search spaces~\cite{10.1371/journal.pone.0303088}. Authors in Ref.~\cite{Barnes2015-zj} introduced an analytical robust quantum control approach that not only yields explicit constraints on the control field but also ensure that the leading-order noise-induced errors in a qubit’s evolution cancel exactly.


In recent years, Reinforcement Learning (RL) based approaches have emerged as alternatives for quantum optimal control. The major advantage of RL is its ability to explore vast control landscapes autonomously, finding creative control solutions that may be difficult for traditional methods to uncover. For instance, authors in Ref.~\cite{s41534-019-0141-3} used trusted-region-policy-optimization (TRPO) RL algorithms to train agents for the control of two-qubit unitary gates by adding control noise into training environments and they found the agent demonstrates a two-order-of-magnitude reduction in average-gate-error compared to gradient based methods. This shows that RL algorithms are robust to noise and model uncertainties because they can adapt to noisy environments by learning policies that are more resilient to fluctuations in control parameters, making them practical for real-world quantum systems where noise and environmental factors affect control fidelity. RL algorithms can also handle high-dimensional control spaces and complex quantum dynamics by learning from interactions with the environment, thus reducing the need for precise system modeling. In Ref.~\cite{PhysRevX.12.011059}, the authors trained a model-free RL agent which learns through trial-and-error interaction with the quantum system and hence there is no need for the precise modeling of the physical system. They concluded that this is of immediate relevance allowing the adaptation of quantum control policies to the specific system in which they are deployed and complete elimination of model bias. 

Different RL based methods were compared with traditional optimization techniques in Ref.~\cite{Zhang_2019}, where the authors demonstrated that RL algorithms (TQL~\cite{sutton1999reinforcement}, DQN~\cite{mnih2015human} and PG~\cite{NIPS1999_464d828b}) can outperform traditional methods (SGD~\cite{Ferrie190404} and Krotov~\cite{krotov1996global}) when the problem size is scaled up. The authors showed that for a relatively small number of iterations (up to 500) SGD would fail to reach optimal solutions compared with ML techniques, however it is worth noting that allowing SGD more iterations can result in improved performance as detailed in section \ref{sec:experimentalsetup}.

Despite RL-based algorithms showing better performance, there are drawbacks to using RL in quantum control. RL methods often require a large number of training episodes, particularly for complex quantum systems. Additionally, RL systems are sensitive to hyper-parameters, such as the learning rate and exploration-exploitation trade-off, which must be carefully tuned for optimal performance~\cite{sutton1999reinforcement}. In this work, an effort is made to answer the question of when one should consider using RL techniques, which RL methods are beneficial, and which are excessive by visualizing the landscape and studying the solution space complexity.


Indeed, the fact that there are different options for quantum control notwithstanding, there has been limited research into visualizing and understanding the QCL in higher-dimensional parameter spaces. Therefore, this work contributes by providing: (a) a framework for the exploration and visualization of QCL in such complex spaces and (b) algorithms that we can utilize to get optimal control pulses, offering new insights and tools for optimizing quantum control.

\section{Visualizing Higher Dimensional Landscapes Using Principal Component Analysis (PCA)}
\label{sec:dimensionalityReduction}
PCA is a statistical technique for dimensionality reduction that transforms a higher dimensional data set into a lower dimensional representation. It seeks to maximise the amount of information retained while reducing the dimensionality of the data. By projecting data onto a new set of orthogonal axes, called principal components, PCA identifies the directions of maximum variance in the data. The first principal component accounts for the largest possible variance, with each subsequent component capturing the next highest variance under the constraint of being orthogonal to the preceding components. This process helps in extracting key features and making it easier to visualize and analyze high-dimensional datasets.

% Results of using t-SNE and UMAP is presented in the Appendix section. 

By applying PCA on 3-parameter or 4-parameter quantum landscapes, we can create 2-dimensional visualisations of the landscapes which can allow us to gain insight into the complexities and characteristics of the QCL for the Hamiltonian. To make things concrete, in this work we focus on the QCL generated for $N$-parameter pulses, i.e. the continuous control field $u(t)$ over total evolution time $T$ is divided into $N$ equal partitions (time steps), each partition with its corresponding amplitude, resulting in a piece-wise constant function for $u(t)$. 
\begin{figure*}[htbp]
    \centering
    \begin{subfigure}[b]{0.295\textwidth}
        \centering
        \includegraphics[width=\textwidth]{landscape_images/2_param_brute_force_raw_data_20241003_2.png} 
        \caption{}
        \label{fig:2paramraw}
    \end{subfigure}
    \begin{subfigure}[b]{0.295\textwidth}
        \centering
        \includegraphics[width=\textwidth]{landscape_images/3_param_brute_force_raw_data_20241003_2.png} 
        \caption{}
        \label{fig:3paramraw}
    \end{subfigure}
    \begin{subfigure}[b]{0.295\textwidth}
        \centering
        \includegraphics[width=\textwidth]{landscape_images/3_param_brute_force_PCA_20241003.png} 
        \caption{}
        \label{fig:3paramPCAraw}
    \end{subfigure}
    \caption{Quantum landscape using brute-force data: (a) 2 parameter~\cite{Larocca_2018},  (b) 3 parameter, and (c) 3 parameter after applying PCA. Data points were created by generating every possible combination of values in the range [-1, 1] for each axis, with each axis divided into 100 intervals. Axis labels a1, a2 and a3 represent parameters 1, 2 and 3 in each axis, pc1 and pc2 represent the first two principal components.}
    \label{fig:bruteforce}
\end{figure*}
We consider the evolution of a single qubit with $T/T_{min}=2$, initial state of $\ket{0}$ and target state of $\ket{1}$ where $\ket{0}$ and $\ket{1}$ are the eigenstates of $\sigma_z$ and assume the system is governed by the Hamiltonian      
\begin{equation}
\label{eq:LZ}
    H(t) = \frac{\sigma_{x}}{2} + 2u(t)\sigma_{z}
\end{equation} 
The Hamiltonian in Eq.~\eqref{eq:LZ} is closely related to the celebrated Landau-Zener model which captures a remarkably diverse range of physical settings, as discussed, for example in Ref.~\cite{IVAKHNENKO20231} and we refer to Refs.~\cite{Larocca_2018, Hegerfeldt_2013} where similar control techniques as will be considered below were employed.

The two parameter and three parameter quantum landscapes resulting from using brute-force combination of 100 values for each parameter in the range [-1, 1] are shown in Fig.~\ref{fig:2paramraw} and Fig.~\ref{fig:3paramraw}. As can be seen from Fig.~\ref{fig:3paramraw}, details of the landscape for the 3 parameter case are hard to see. (Note: throughout this paper, the axis labels a1, a2, and a3 represent parameter 1, parameter 2, and parameter 3, respectively, while pc1 and pc2 denote principal component 1 and principal component 2). In order to see underlying properties of the landscape for higher dimensions, we apply PCA with two principal components on the raw brute-force data points which results in the  landscapes shown in Fig.~\ref{fig:3paramPCAraw}. The landscapes in Fig.~\ref{fig:bruteforce} show both the high fidelity (red) and low fidelity (blue) regions. As the target is to achieve a high-fidelity state transfer, we can gain more insight into the space of effective control protocols by focusing on only regions with high target state fidelity, i.e. $F>0.95$, shown in Fig.~\ref{fig:afterapplyingPCAhighfid}.

 As it is evident from the PCA representation of the landscapes in Fig.~\ref{fig:afterapplyingPCAhighfid}, higher dimensional landscapes are filled with more high fidelity regions as compared to the low dimensional landscapes. From this observation, it is clear that by dividing the total evolution time $T$ in to more partitions (hence high dimensional landscape), we increase the chances of finding more control signals that will result in high fidelity (complete population transfer).  \par 
 
Clearly, we cannot simply try the brute-force combination of inputs for more than a small number of parameters and hope we can land in a combination which will result in high fidelity. Instead, in order to generate a control pulse with reasonably good fidelity, we use optimization algorithms such as SGD, GA, QL, DQN or PPO to find a control signal which will drive our system into its target state effectively.  Details of these algorithms and the experimental setups for each experiment are discussed next.

\begin{figure}[htbp]
    \centering
    \begin{subfigure}[b]{0.275\textwidth}
        \centering
        \includegraphics[width=\textwidth]{landscape_images/2_param_brute-force_95p_PCA_20241003.png} 
        \caption{}
        \label{fig:2paramPCAhighfid}
    \end{subfigure}
    \begin{subfigure}[b]{0.275\textwidth}
        \centering
        \includegraphics[width=\textwidth]{landscape_images/3_param_brute-force_95p_PCA_20241003.png} 
        \caption{}
        \label{fig:3paramPCAhighfid}
    \end{subfigure}
    \begin{subfigure}[b]{0.275\textwidth}
        \centering
        \includegraphics[width=\textwidth]{landscape_images/4_param_brute-force_95p_PCA_20241003.png} 
        \caption{}
        \label{fig:4paramPCAhighfid}
    \end{subfigure}
    \caption{Quantum control landscape after applying PCA and extracting regions of high fidelity (fidelity above 0.95): (a) Two parameter, (b) Three parameter, and (c) Four parameter, the dataset which is used to generate these plots is generated using a brute-force combination of inputs in the range [-1, 1]}
    \label{fig:afterapplyingPCAhighfid}
\end{figure}

\section{Experimental Setup and Computational Techniques}
\label{sec:experimentalsetup}
We can study the QCL generated when using SGD, GA, QL, DQN and PPO for $N$-parameter optimization using~\eqref{eq:LZ}. In order to generate the QCLs as a function of fidelity, we run 1000 experiments for each algorithm, hence generated 1000 data-points, we then pass the results through a 2 component PCA to visualize the landscape in 2D. Since the results of the PCA will depend on the number of data-points available, we first generated PCA loadings for 2, 3, and 4 parameters using data points created by the brute force combination of possible values between [-1, 1]. PCA loadings represent the coefficients assigned to the original variables when forming the principal components and they indicate the contribution of each original feature to the principal components. These PCA loadings are then used to transform the 2, 3 and 4 parameter results from the above five algorithms to a 2-component PCA transformation. In our study, Numpy and Qutip~\cite{JOHANSSON20131234} libraries are used to represent the quantum states and other related vectors and matrices resulting during the optimization process. A minimum infidelity (1-fidelity) value  of 0.001 is used, and when the infidelity value from a certain algorithm is below this value, we assume the target is achieved and the algorithm exits the optimization loop immediately. A brief summary of the algorithms used in this work is presented below.


\subsection{SGD with Momentum}
Stochastic Gradient Descent (SGD) is an optimization algorithm that is used to minimize the cost function (difference between target value and calculated value) in optimization problems. In the context of quantum control, SGD can be used to generate control pulses that would drive the initial state towards the target state as much as possible. Starting from a list of random values whose length is equal to the number of parameters, $N$, these values are iteratively updated by generating a set of infinitesimally small values which are added to or subtracted from the initial random values, until a pulse that can result in high fidelity (above the target fidelity threshold) is found or the total number of iterations are exhausted. 

To improve the convergence time of the traditional central difference based SGD algorithm used in \cite{Ferrie190404} and \cite{Zhang_2019}, we can add a momentum term. This addition results in improved performance when compared with the implementation without momentum. A learning rate of 0.01, and a momentum of 0.95 are used. The initial pulses consist of a set of random values between -1 and 1, with a length equal to the number of parameters $N$. Unlike the 500 iterations used in \cite{Zhang_2019}, a maximum iteration of 10,000 is used as the stopping criteria for the SGD so that the algorithm has enough time to find optimal results and avoid sub-optimal regions. If the algorithm is unable to find an optimal control pulse after the max iterations are exhausted, the control pulse from the final iteration is returned. 

\subsection{Genetic Algorithm (GA)}
Genetic algorithms (GA), a class of optimization techniques inspired by the principles of natural selection and genetics, operate by evolving a population of candidate solutions through iterative processes of selection, crossover, and mutation, aiming to improve the solutions over successive generations based on fitness criteria. 

In the context of quantum control, the genes will be the possible pulse amplitudes (we limit number of genes to be 100 in the range to [-1, 1] in our experiment), the chromosome will represent the complete control pulse of length $N$, the population will be the collection of the chromosomes, and we use a mutation rate of 0.3. The length of each chromosome will be equal to the number of parameters we are interested in. The fitness function is, naturally, the fidelity of the system after the selected chromosome is tested for its performance using the given Hamiltonian. In every iteration, chromosomes that are among the top 30 percent of the population (based on their fitness value) and those that are in the last 20 percent are selected to be part of the next generation of the population. The decision to include those in the bottom 20 percent is to enable the algorithm have enough genetic variations in the cross over and mutation stages of the algorithm. To keep the total population constant, the remaining 50 percent of the population is accounted for by generating new population through crossover and mutation. The initial population and chromosomes are generated by selecting genes randomly.


In our experiments, we limit the algorithm to have a maximum of 50 generations. If a chromosome that meets the fitness criteria is found early, it is returned; otherwise, the iteration continues until the final generation, after which the top chromosome based on fitness value is selected.

\subsection{Q-Learning (QL)}
Q-Learning is a model-free reinforcement learning algorithm that enables an agent to learn optimal actions within an environment by interacting with it. It operates by learning a Q-value function, which estimates the expected cumulative reward for taking a particular action in a given state and following an optimal policy thereafter. Through trial and error, Q-Learning updates these Q-values iteratively, using observed rewards to refine future decisions. This process allows the agent to eventually converge on an optimal policy, even without prior knowledge of the environment's dynamics.

QL in a given environment is characterized by two important variables: the state of the system/agent at any given time/episode and the action space of the system. While the state represents a snapshot of the environment at a given time, the action space of a QL algorithm is defined as the set of all possible actions an agent can take in a given environment, that change the agent's state and influence the rewards it receives, when executed. For our quantum control problem, the action space is represented by the set of possible amplitudes the pulse can have, where we limit this to be 100 values in the range [-1, 1]. 

The `state' of QL in our quantum control problem is represented by the state of the qubit at a given time in the evolution process. We use QL with epsilon-greedy strategy, where we explore the environment with a probability of $\epsilon$ and exploit the system by choosing the actions that have high Q-values  with a probability of 1-$\epsilon$. Because our quantum control problem has very short (i.e., few) time steps / episodes, the value of $\epsilon$ is set to 0.1 (10\% exploration and 90\% exploitation), to focus more on exploitation than exploration.  In this work, a learning rate of 0.001 and a reward decay factor of 0.9 are employed. The reward values are assigned as follows: -1 if the infidelity exceeds 0.5, 10 if the infidelity is below 0.5, 100 if the infidelity is below 0.1, and 500 if the infidelity is below 0.001. The initial actions are generated randomly. The maximum episodes the agent can take is limited to 500, and if the agent does not find an optimal path during this time, the state of the system at the end of the episode is returned.

\subsection{Deep Q-Network (DQN)}
Deep Q-Learning (DQN) is an extension of the Q-Learning algorithm that incorporates deep neural networks to handle environments with high-dimensional state spaces. While traditional Q-Learning relies on tabular representations (Q-table), DQN uses a neural network to approximate the Q-value function, enabling it to scale to more complex tasks. By integrating techniques like experience replay and target networks, DQN improves stability and convergence during training. 

The action space and state of the DQN algorithm are the same as that of the QL algorithm. The neural network we use is a simple 3 layer MLP (Multi Layer Perceptron) with hidden layers containing [64, 512, 256] units, respectively. Key hyper-parameters include a learning rate of 0.0001, an exploration fraction of 0.25, and a reward discount factor of 0.000001. Reward values are assigned as follows: 1 if the infidelity exceeds 0.5, 10 if the infidelity is below 0.5, 500 if the infidelity is below 0.1, and 5000 if the infidelity is below 0.001. The implementation details of our DQN algorithm are as follows:
\begin{itemize}
    \item We create a custom environment wrapped around gymnasium. Gymnasium~\cite{towers2024gymnasium} is a toolkit designed for developing single agent RL algorithms by providing environments for common RL experiments or the possibility to define custom environments, allowing researchers to easily design and run RL experiments tailored to specific needs.
    \item The neural net is implemented using Stable Baselines3. Stable Baseline3~\cite{stable-baselines3} is another popular library that implements state-of-the-art RL algorithms (like DQN and PPO) and simplifies the process of training, rapid experimentation and deploying RL models in Gymnasium environments.
    \item The input to the neural net is the state of the system at current time step and the output of the neural net is the optimal action for that state
\end{itemize}

\subsection{Proximal Policy Optimization (PPO)}
Proximal Policy Optimization (PPO) is a popular RL algorithm that strikes a balance between simplicity and efficiency. PPO simplifies the optimization process while maintaining stable updates by using a clipped objective function to prevent large, destabilizing policy updates, allowing for efficient learning across a variety of environments~\cite{schulman2017proximalpolicyoptimizationalgorithms}. PPO is widely used in RL due to its robustness, ease of implementation, and ability to perform well in both continuous and discrete action spaces.

Similar to DQN, we use Gymnasium to define a custom environment, where the action space, which consists of 100 discrete values within the range [-1, 1], is represented by the set of possible actions the qubit can take, and the `state' is represented by the state of the qubit at a given time. For the PPO algorithm, we use the implementation by Stable Baselines3, and used a 3 layer MLP with [64, 512, 256] units in each layer as the PPO's neural network. The following hyper-parameters are used in our experiments: a learning rate of 0.0001, entropy coefficient of 0.25, and a reward discount factor of 0.000001. Reward values are structured exactly the same to that of the DQN algorithm. For both DQN and PPO, initial actions are sampled from the action space randomly. Whereas DQN and PPO are highly effective algorithms for complex tasks, their use in simpler problems may be excessive. In such cases, the neural network could struggle to capture finer details, leading to sub-optimal learning.

\begin{figure}[htbp]
    \centering
    \begin{subfigure}[b]{0.29\textwidth}
        \centering
        \includegraphics[width=\textwidth]{landscape_images/2_param_SGD_from_PCA-Loadings_20241003.png} 
        \caption{}
        \label{fig:2paramPCASGD}
    \end{subfigure}
    \begin{subfigure}[b]{0.29\textwidth}
        \centering
        \includegraphics[width=\textwidth]{landscape_images/3_param_SGD_from_PCA-Loadings_20241003.png} 
        \caption{}
        \label{fig:3paramPCASGD}
    \end{subfigure}
    \begin{subfigure}[b]{0.29\textwidth}
        \centering
        \includegraphics[width=\textwidth]{landscape_images/4_param_SGD_from_PCA-Loadings_20241003.png} 
        \caption{}
        \label{fig:4paramPCASGD}
    \end{subfigure}
    
    \begin{subfigure}[b]{0.29\textwidth}
        \centering
        \includegraphics[width=\textwidth]{landscape_images/2_param_GA_from_PCA-Loadings_20241003.png} 
        \caption{}
        \label{fig:2paramPCAGA}
    \end{subfigure}
    \begin{subfigure}[b]{0.29\textwidth}
        \centering
        \includegraphics[width=\textwidth]{landscape_images/3_param_GA_from_PCA-Loadings_20241003.png} 
        \caption{}
        \label{fig:3paramPCAGA}
    \end{subfigure}
    \begin{subfigure}[b]{0.29\textwidth}
        \centering
        \includegraphics[width=\textwidth]{landscape_images/4_param_GA_from_PCA-Loadings_20241003.png} 
        \caption{}
        \label{fig:4paramPCAGA}
    \end{subfigure}
    \caption{The quantum landscape when using SGD and GA for 1000 tests and after applying PCA (a) SGD - 2 parameter, (b) SGD - 3 parameter, (c) SGD -  4 parameter, (d) GA - 2 parameter, (e) GA - 3 parameter, and (f) GA -  4 parameter}
    \label{fig:afterapplyingPCASGDGA}
\end{figure}


% \FloatBarrier

\section{Results Analysis and Discussion}
\label{sec:results}
In pursuit of discovering which algorithms among SGD, GA, QL, DQN and PPO are best at generating high fidelity control pulses and are able to explore the quantum landscape effectively, we conduct 1,000 tests for the 2, 3, and 4 parameter cases for each algorithm, and the resulting data points are passed through PCA to generate a 2D landscape. 


Fig.~\ref{fig:afterapplyingPCASGDGA}(a-c) shows the results after running SGD for a maximum of 10,000 iterations. From the results in Fig.~\ref{fig:afterapplyingPCASGDGA}(a-c), we observe that even though most points in the two dimensional quantum landscape correspond to high fidelity values, there are a considerable number of points which are in the sub optimal (low fidelity) region even after giving the algorithm enough time to converge (10,000 iterations). The presence of sub-optimal points suggests that SGD can struggle with convergence to globally optimal solutions.

In Fig.~\ref{fig:afterapplyingPCASGDGA}(d-f), we present the resulting landscape when the approach used is GA. As can be seen in Fig.~\ref{fig:afterapplyingPCASGDGA}(d-f), and unlike SGD, the GA results display high fidelity pulses across all parameter cases. This indicates that GA has a stronger capability to explore the parameter space effectively and converge to optimal solutions. However, it is important to note that despite the favorable results, GA may still encounter issues if the initial population is poorly sampled. Random initialization can trap GA in sub-optimal regions, although such cases were not evident in our experiments. Overall, the GA demonstrates greater robustness and reliability compared to SGD in generating high-fidelity pulses.


% \FloatBarrier

The results of our experiment using QL, after passing through PCA, are presented in Fig.~\ref{fig:afterapplyingPCAQL}. When compared to SGD, QL produces a landscape that is close to the ideal brute-force landscape, with the data points concentrated in high-fidelity regions. This suggests that QL, similar to GA, is a more capable algorithm for navigating quantum control landscapes, as it consistently outperforms SGD in terms of generating optimal control pulses. This landscape demonstrates QL's potential for effective optimization in quantum control tasks.

In the experiments we conduct using DQN and PPO, the results reveal important insights into reward structuring of these algorithms. It was discovered that for problems requiring fewer steps per episode, as is the case for 2, 3, and 4 parameter quantum control, delayed rewards often fail to guide the model towards optimal solutions due to the shorter episode length. On the contrary,  the model tends to achieve better results when immediate rewards are prioritized, as the system has less time to accumulate delayed feedback effectively.
 
\begin{figure}[htbp]
    \centering
    \begin{subfigure}[b]{0.29\textwidth}
        \centering
        \includegraphics[width=\textwidth]{landscape_images/2_param_QL_from_PCA-Loadings_20241003.png} 
        \caption{}
        \label{fig:2paramPCAQL}
    \end{subfigure}
    \begin{subfigure}[b]{0.29\textwidth}
        \centering
        \includegraphics[width=\textwidth]{landscape_images/3_param_QL_from_PCA-Loadings_20241003.png} 
        \caption{}
        \label{fig:3paramPCAQL}
    \end{subfigure}
    \begin{subfigure}[b]{0.29\textwidth}
        \centering
        \includegraphics[width=\textwidth]{landscape_images/4_param_QL_from_PCA-Loadings_20241003.png} 
        \caption{}
        \label{fig:4paramPCAQL}
    \end{subfigure}
    \caption{The quantum landscape when using QL for 1000 tests and after applying PCA (a) 2 parameter, (b) 3 parameter, and (c) 4 parameter}
    \label{fig:afterapplyingPCAQL}
\end{figure}

As is the case with the other algorithms we discussed above, the experiments were repeated 1000 times and the results of our experiment using DQN and PPO, after passing through PCA,  are presented in Fig.~\ref{fig:afterapplyingPCADQNPPO}. From the results in Fig.~\ref{fig:afterapplyingPCADQNPPO}, we see that the DQN and PPO results align closely (with the exception of the 4 parameter case) with those of QL and GA, suggesting that reinforcement learning methods, when structured with appropriate reward schemes, offer strong potential for high-fidelity quantum control optimization. The few results in the sub-optimal region for both DQN and PPO (especially for the 4 parameter case), may suggest these algorithms might be an overkill if the problem is relatively simple as in our single-qubit state transition problem. One possible reason for this could be that the neural network in DQN and PPO fails to learn the relationships between its inputs and outputs as the quantum system being explored is fairly simple for a neural network - suggesting we might want to use advanced RL algorithms only when the traditional algorithms fail to generate high fidelity pulses. Overall, for our single-qubit state transition quantum control problem, GA and QL standout as the best performing algorithms.


\begin{figure}[htbp]
    \centering
    \begin{subfigure}[b]{0.29\textwidth}
        \centering
        \includegraphics[width=\textwidth]{landscape_images/2_param_DQN_from_PCA-Loadings_20241003.png} 
        \caption{}
        \label{fig:2paramPCADQN}
    \end{subfigure}
    \begin{subfigure}[b]{0.29\textwidth}
        \centering
        \includegraphics[width=\textwidth]{landscape_images/3_param_DQN_from_PCA-Loadings_20241003.png} 
        \caption{}
        \label{fig:3paramPCADQN}
    \end{subfigure}
    \begin{subfigure}[b]{0.29\textwidth}
        \centering
        \includegraphics[width=\textwidth]{landscape_images/4_param_DQN_from_PCA-Loadings_20241003.png} 
        \caption{}
        \label{fig:4paramPCADQN}
    \end{subfigure}

    \begin{subfigure}[b]{0.29\textwidth}
        \centering
        \includegraphics[width=\textwidth]{landscape_images/2_param_PPO_from_PCA-Loadings_20241003.png} 
        \caption{}
        \label{fig:2paramPCAPPO}
    \end{subfigure}
    \begin{subfigure}[b]{0.29\textwidth}
        \centering
        \includegraphics[width=\textwidth]{landscape_images/3_param_PPO_from_PCA-Loadings_20241003.png} 
        \caption{}
        \label{fig:3paramPCAPPO}
    \end{subfigure}
    \begin{subfigure}[b]{0.29\textwidth}
        \centering
        \includegraphics[width=\textwidth]{landscape_images/4_param_PPO_from_PCA-Loadings_20241003.png} 
        \caption{}
        \label{fig:4paramPCAPPO}
    \end{subfigure}
    \caption{The quantum landscape when using DQN and PPO for 1000 tests and after applying PCA (a) 2 parameter-DQN, (b) 3 parameter-DQN, (c) 4 parameter-DQN (d) 2 parameter-PPO, (e) 3 parameter-PPO, and (f) 4 parameter-PPO}
    \label{fig:afterapplyingPCADQNPPO}
\end{figure}

Upon examining the landscape plots, it may initially seem that not all 1,000 test results are represented in each plot. This is due to overlapping points in the plots (i.e., where an approach finds the same solution multiple times), which creates the impression that fewer data points are present. For each algorithm and parameter count, we recorded the overlap of points to provide a comprehensive understanding. As an example, for the 3-parameter scenario, we present this overlap count in Fig.~\ref{fig:overlapSGDGAQLDQNPPO}. From this overlap plot, we see that while the points spread over the high fidelity regions for some algorithms, certain algorithms (DQN and PPO) tend to favor repeating specific combinations of inputs, resulting in clusters of points mainly in certain regions of the landscape. This clustering may suggest the tendency of certain algorithms to gravitate towards repeating certain pulses rather than exploring new regions of the parameter space.

\begin{figure}[h]
    \centering
    \begin{subfigure}[b]{0.3\textwidth}
        \centering
        \includegraphics[width=\textwidth]{overlap_images/3_param_SGD_Overlap_20241003.png} 
        \caption{}
        \label{fig:3paramSGDOVerlap}
    \end{subfigure}
    \begin{subfigure}[b]{0.3\textwidth}
        \centering
        \includegraphics[width=\textwidth]{overlap_images/3_param_GA_Overlap_20241003.png} 
        \caption{}
        \label{fig:3paramGAOVerlap}
    \end{subfigure}
    \begin{subfigure}[b]{0.3\textwidth}
        \centering
        \includegraphics[width=\textwidth]{overlap_images/3_param_QL_Overlap_20241003.png} 
        \caption{}
        \label{fig:3paramQLOVerlap}
    \end{subfigure}
    \begin{subfigure}[b]{0.3\textwidth}
        \centering
        \includegraphics[width=\textwidth]{overlap_images/3_param_DQN_Overlap_20241003.png} 
        \caption{}
        \label{fig:3paramDQNOVerlap}
    \end{subfigure}
    \begin{subfigure}[b]{0.3\textwidth}
        \centering
        \includegraphics[width=\textwidth]{overlap_images/3_param_PPO_Overlap_20241003.png} 
        \caption{}
        \label{fig:3paramPPOOVerlap}
    \end{subfigure}
    % \captionsetup{skip=5pt}
    % \setlength{\belowcaptionskip}{1pt}
    \caption{The overlap count for the 3 parameter case: the overlap count means the number of points in close proximity both in terms of coordinates and fidelity value. If certain number of points lie closer to each other in the 2D coordinate system and their fidelity values are close to each other, the overlap count in this region will be higher. Bigger circles represent higher count. (a) overlap count for 3 parameter-SGD, (b) overlap count for 3 parameter-GA, (c) overlap count for 3 parameter-QL (d) overlap count for 3 parameter-DQN, and (e) overlap count for 3 parameter-PPO}
    \label{fig:overlapSGDGAQLDQNPPO}
\end{figure}

Knowing the overlap count is relevant for a number of reasons: 
\begin{itemize}
    \item A high overlap count may indicate suboptimal results, suggesting that the algorithm may have missed high-fidelity regions, or
    \item it could imply that the quantum system is inherently complex, with limited solutions available to satisfy state transfer requirements, highlighting the need for more advanced control strategies.
\end{itemize}
On the other hand, if the overlap count remains low, it may imply that the algorithm has effectively explored and covered most high-fidelity regions in the solution space. This can indicate either that the algorithm is well-suited for the task or that the problem itself is relatively straightforward. In such cases, achieving successful state transfer may not require a complex or highly optimized quantum control strategy, as the solution can be reached with simpler techniques, saving computational resources and reducing algorithmic complexity.



To quantitatively evaluate how effectively each algorithm identifies pulses that result in high fidelity, we present the distribution of pulse counts as a function of fidelity in Fig.~\ref{fig:overlapPlotAll}. From the plots in Fig.~\ref{fig:2paramOverlapAll} we observe that 2-parameter SGD algorithm has many points concentrated in the low-fidelity region, indicating it struggles to find high-fidelity solutions. In contrast, GA has all its pulses within the high-fidelity region (close to 100\%), indicating it is highly effective in achieving high fidelity with two parameters. QL, DQN, and PPO display a mixed distribution, with most of their points also leaning toward higher fidelity, but they are less consistent than GA. For the 3-parameter case in Fig.~\ref{fig:3paramOVerlapAll}, SGD continues to struggle, GA maintains its dominance, achieving high fidelity with all pulses. QL, DQN, and PPO are distributed across high-fidelity regions as well, with DQN and PPO clustering closer to higher fidelity. In Fig.~\ref{fig:4paramOVerlapALL}, which explores the 4-parameter scenario, a more spread-out distribution is observed. SGD still has many points scattered across lower fidelity values, showing it is less efficient even with additional parameters. GA remains highly effective, with all points concentrated near 100\% fidelity. Interestingly, QL, DQN, and PPO also show a wider range in fidelity, but most points are in the high-fidelity region. PPO, in particular, has few pulses distributed across a wider fidelity range compared to GA. 
\begin{figure}[H]
    \centering
    \begin{subfigure}[b]{0.33\textwidth}
        \centering
        \includegraphics[width=\textwidth]{overlap_images/2_param_overlap_log_scaled_20241003.png} 
        \caption{}
        \label{fig:2paramOverlapAll}
    \end{subfigure}
    \begin{subfigure}[b]{0.33\textwidth}
        \centering
        \includegraphics[width=\textwidth]{overlap_images/3_param_overlap_log_scaled_20241003.png} 
        \caption{}
        \label{fig:3paramOVerlapAll}
    \end{subfigure}
    \begin{subfigure}[b]{0.33\textwidth}
        \centering
        \includegraphics[width=\textwidth]{overlap_images/4_param_overlap_log_scaled_20241003.png} 
        \caption{}
        \label{fig:4paramOVerlapALL}
    \end{subfigure}
    
    \caption{Pulse counts for SGD, GA, QL , DQN and PPO algorithm as a function of fidelity: (a) 2 parameter, (b) 3 parameter, and (c) 4 parameter. Note that the Y axis values are given in log scale to show the distribution of points more clearly }
    \label{fig:overlapPlotAll}
\end{figure}
In summary, an algorithm capable of generating a wide variety of high fidelity pulses is preferred over one that produces only a limited set, as having a broader selection of high fidelity pulses provides flexibility. This range is valuable since some pulses may not be easily realizable, making it essential to have alternative pulses that enhance the robustness and applicability of a control algorithm.

\section{Solution  Space Complexity}
\label{sec:solutionspacecomplexity}

In the previous section, we have seen that some approaches have the potential to yield more (i.e., increased variety) high-fidelity solutions, and others can fall into local optima traps. Yet, the question remains, how should we evaluate the complexity of a QCL in order to make informed choices regarding which approach(es) to use? In order to study the complexity of the solution space, we analyse how sparse/dense the landscapes are using clustering algorithms and distance measures. To find clusters in the solution space, we employ a cluster number agnostic algorithm-DBSCAN~\cite{ester1996density}, and to find the distances between points within a cluster;  we use the Euclidean distance. If two clusters are of comparable area and the distance between the points within one cluster is smaller than the distance between points within the other cluster, this tells us the first cluster is more dense. This, in turn, can reveal that an approach is more/less likely to find a high-fidelity solution in this region of the landscape. 

To measure how dense the landscape is, we define a term we called cluster density index (CDI), which is the inverse of how sparse a region is,  as:
\begin{equation}
\text{CDI} = \frac{\bar{A}}{\bar{D}}
\label{equ:sparsity}
\end{equation}
where, \textit{CDI} refers to the average Cluster Density Index, $\bar{A}$ is the average area of clusters calculated as mean of individual cluster areas which are calculated using Delaunay Triangulation~\cite{10.1145/235815.235821}, and $\bar{D}$ is the average, over all clusters, of the mean pairwise distances between points in a given cluster. In Table~\ref{tab:sparsityIndex}, $\bar{L}$ refers to the average inter-cluster distance.
\begin{table*} [htbp]
    \centering
    \caption{Cluster Density Index: describes how sparse / dense a landscape is}
    \begin{adjustbox}{max width=\textwidth}
    \begin{tabular}{|>{\centering\arraybackslash}m{0.041\linewidth}|>{\centering\arraybackslash}m{0.041\linewidth}|>{\centering\arraybackslash}m{0.041\linewidth}|>{\centering\arraybackslash}m{0.041\linewidth}|>{\centering\arraybackslash}m{0.041\linewidth}|>{\centering\arraybackslash}m{0.041\linewidth}|>{\centering\arraybackslash}m{0.041\linewidth}|>{\centering\arraybackslash}m{0.041\linewidth}|>{\centering\arraybackslash}m{0.041\linewidth}|>{\centering\arraybackslash}m{0.041\linewidth}|>{\centering\arraybackslash}m{0.041\linewidth}|>{\centering\arraybackslash}m{0.041\linewidth}|>{\centering\arraybackslash}m{0.041\linewidth}|>{\centering\arraybackslash}m{0.041\linewidth}|>{\centering\arraybackslash}m{0.041\linewidth}|>{\centering\arraybackslash}m{0.041\linewidth}|}  
    \hline
          \rule{0pt}{4.5ex} \multirow{2}{*}{} &  \multicolumn{3}{c|}{\textbf{SGD}} & \multicolumn{3}{c|}{\textbf{GA}} & \multicolumn{3}{c|}{\textbf{QL}} & \multicolumn{3}{c|}{\textbf{DQN}} & \multicolumn{3}{c|}{\textbf{PPO}} \\  \cline{2-16} 
          \rule{0pt}{5.5ex}& 2 param & 3 param & 4 param & 2 param & 3 param & 4 param & 2 param & 3 param & 4 param & 2 param & 3 param & 4 param & 2 param & 3 param & 4 param\\ \hline
         $\bar{L}$ & \gradient{0.9463}& \gradient{1.0501}& \gradient{1.2232}& \gradient{0.9391}& \gradient{0.9576}& \gradient{1.2380}& \gradient{0.8398}& \gradient{1.4588}& \gradient{1.3698}& \gradient{0.8919}& \gradient{1.1551}& \gradient{0.9895}& \gradient{0.8836}& \gradient{1.2656}& \gradient{1.1495}\\ \hline
         $\bar{D}$ & \gradient{0.0035}& \gradient{0.1280}& \gradient{0.1392}& \gradient{0.0234}& \gradient{0.1094}& \gradient{0.1418}& \gradient{0.0697}& \gradient{0.2178}& \gradient{0.1966}& \gradient{0.0423}& \gradient{0.0167}& \gradient{0.0253}& \gradient{0.1383}& \gradient{0.0273}& \gradient{0.1593}\\ \hline
         $\bar{A}$ & \gradient{0.0001}& \gradient{0.0251}& \gradient{0.1023}& \gradient{0.0010}& \gradient{0.0505}& \gradient{0.2282}& \gradient{0.0133}& \gradient{0.4129}& \gradient{0.4605}& \gradient{0.0036}& \gradient{0.0024}& \gradient{0.0027}& \gradient{0.0333}& \gradient{0.0185}& \gradient{0.0724}\\ \hline
         % Number of  clusters & \seqsplit{14}& \seqsplit{14}& \seqsplit{22}& \seqsplit{14}& \seqsplit{13}& \seqsplit{15}& \seqsplit{10}& \seqsplit{7}& \seqsplit{8}& \seqsplit{11}& \seqsplit{21}& \seqsplit{23}& \seqsplit{7}& \seqsplit{8}& \seqsplit{17}\\ \hline
         CDI & \gradient{0.0075}& \gradient{0.1962}& \gradient{0.7346}& \gradient{0.0449}& \gradient{0.4617}& \gradient{1.6096}& \gradient{0.1902}& \gradient{1.8957}& \gradient{2.3420}& \gradient{0.0844}& \gradient{0.1427}& \gradient{0.1075}& \gradient{0.2404}& \gradient{0.6766}& \gradient{0.4543}\\ \hline
    \end{tabular}
    \end{adjustbox}
    \label{tab:sparsityIndex}
\end{table*}

Using~\eqref{equ:sparsity}, we calculate and present the resulting Cluster Density Index in Table~\ref{tab:sparsityIndex}. A high CDI value is desirable, because a high CDI (according to~\eqref{equ:sparsity}), ideally, means there are high fidelity regions in the landscape which have large and dense clusters (it is worth to note that the landscapes are generated after using PCA on the output of the different algorithms). To increase the CDI value, $\bar{A}$ has to be high while $\bar{D}$ is kept low. Typically, higher CDI means there are many high fidelity points (higher cluster area) in the landscape with the distance between individual points being low (individual points are closer to each other in the cluster). The results in Table~\ref{tab:sparsityIndex} confirm that, in general, algorithms with larger clusters in the landscape, have higher cluster density index values. The best performing algorithms (GA and QL) have the highest CDI as compared to the remaining algorithms. When we compare the results in terms of parameter count, for the same algorithm, we see that for SGD, GA and QL increasing the number of parameters results in an increased CDI value. For DQN and PPO, 3 parameter pulses seem to be better than both 2 parameter and 4 parameter pulses - a detailed future study could be necessary to further investigate this outlier behavior.

However, it should be noted that a higher value of CDI does not necessarily mean that there are many high fidelity points in a given region - because we can get high CDI value even when $\bar{A}$ is very small (extremely few points in each cluster) provided that $\bar{D}$ is even smaller. For example, looking at the CDI value in Table~\ref{tab:sparsityIndex}, for 3 parameter PPO, it is relatively larger than the value for 2 parameter PPO. This is because the average cluster area is very small for 3 parameter PPO, but the average distance between the clusters is even smaller, resulting in a relatively higher CDI value for the 3 parameter PPO than that of 2 parameter. However, this does not mean 3 parameter PPO outperforms 2 parameter PPO in terms of exploring the QCL and generating control pulses with high fidelity (indeed a quick check of Fig.\ref{fig:afterapplyingPCADQNPPO} (d) and (e),  can reveal that 2 parameter PPO results are better at exploring the landscape when compared to 3 parameter case). While higher CDI values can sometimes result from small cluster areas paired with even smaller inter-point distances, these instances represent edge cases. In most scenarios, CDI serves as a reliable metric for assessing cluster quality, as it typically indicates regions of high fidelity with tightly packed points (we refer, for example, CDI results in Table~\ref{tab:sparsityIndex} for GA and QL and verify results by looking at Fig.~\ref{fig:afterapplyingPCASGDGA}(d-f) and Fig.~\ref{fig:afterapplyingPCAQL}). In general, CDI captures the relation between cluster area and point distribution, making it a valuable measure for solution space complexity analysis.

The CDI can be a very powerful tool for studying the QCL even at higher dimensions than we have considered. Because DBSCAN can work with higher dimensional data, and Euclidean distances can be calculated for any dimensions, CDI can potentially allow us to understand how probable it is to find high fidelity clusters in a landscape without even using dimensionality reduction techniques. 

% \FloatBarrier
\section{Conclusion}
In this work, we studied the dynamics of quantum control landscapes through the use of PCA as a dimensionality reduction technique, to visualize higher-dimensional quantum landscapes and algorithms best suited for solving quantum control problem. Our findings reveal that dimensionality reduction techniques are important to analyze the intricate nature of high dimensional quantum control. In addition, we noted that increasing the number of parameters (dividing the total evolution time into more discrete parts) yields landscapes with more high-fidelity regions. However, this does not guarantee a reduction in low-fidelity areas; rather, it enhances the chances of landing on high-fidelity regions when the number of parameters is increased. It should be noted that all dimensionality reduction techniques involve some degree of information loss; as such it would be highly relevant to compare and contrast characteristics of other dimensionality reduction techniques when applied to QCL.

Comparative analyses of traditional and machine learning algorithms in solving the quantum control problem for the given Hamiltonian demonstrates that while Stochastic Gradient Descent (SGD) can perform poorly, Genetic Algorithms (GA) excel at finding high fidelity pulses and exploring the landscape well. Among machine learning approaches, Q-learning (QL) showed promising results, while Deep Q-Network (DQN) and Proximal Policy Optimization (PPO) were less effective, suggesting that simpler algorithms may be advantageous for straightforward problems. Additionally, the design of the reward function in DQN and PPO significantly impacts performance, with immediate rewards being more beneficial for short episode tasks. 


The solution space complexity analysis through the use of cluster density index, indicates that both GA and QL achieved high cluster-density scores for 4-parameter quantum control - meaning those algorithms result in a larger QCL clusters with closely spaced individual points. When comparing the results in terms of parameter count, the cluster-density-score increase when the parameter count is increased from 2 parameter to 3 parameter and then to 4 parameter  with exceptions noted for DQN and PPO. 

Our results demonstrate that dimensionality reduction tools, in particular PCA, can be highly effective in capturing the relevant features of the control landscape. While we have restricted our analysis to a small number of parameters to ensure that we can easily access the full QCL via brute force for comparison, it is worth highlighting that many of the considered algorithms can accommodate optimizing a much larger number of parameters~\cite{CoopmansPRR} or be applied to more complex multi-qubit systems. Our framework for characterising the solution space can nevertheless be readily applied without incurring significant additional computational overhead. The obvious caveat then being that achieving control over these more complex systems will necessitate high dimensional control pulses and, therefore, it still remains to determine the tradeoff between how many principal components are required to maintain the important features of the QCL. Furthermore, our results help to get a deeper understanding of quantum control strategies, emphasizing the importance of parameter selection and reward design in optimizing algorithm performance.

Future research could explore why certain algorithms (example: DQN and PPO) under-perform in short RL episodes, and identify the types of problems that require RL versus those better suited for traditional algorithms. Further research could also establish clear metrics to distinguish between simple and complex problems, helping determine the most appropriate algorithms based on problem complexity and required computational resources.

\section*{Data availability}
The data generated during this study are available from the corresponding author upon reasonable request.

\section*{Code availability}
The code used in this work is accessible on GitHub through the following link: 
%if article is accepted for publication.
\href{https://github.com/Hafdream/Quantum-control-and-landscape}{click here for code}

\section*{Acknowledgements}

This publication has emanated from research conducted with the financial support of Science Foundation Ireland under Grant number 18/CRT/6183. For the purpose of Open Access, the author has applied a CC BY public copyright license to any Author Accepted Manuscript version arising from this submission.

% \bibliography{sample}
% \bibliographystyle{unsrt}
% \bibliography{references}
\documentclass{MITstyle}

%\usepackage[table]{xcolor}
\usepackage{chngcntr}
\usepackage{hyperref}
\usepackage{microtype}

\title{A Lightweight and Extensible Cell Segmentation and Classification Model for Whole Slide Images}

\author{Nikita Shvetsov~$^{1, }$\footnote{Correspondence e-mail: nikita.shvetsov@uit.no}, Thomas K. Kilvaer~$^{2, 3}$, Masoud Tafavvoghi~$^{4}$, Anders Sildnes~$^{1}$, \\ Kajsa Møllersen~$^{4}$, Lill-Tove Rasmussen Busund~$^{5, 6}$, Lars Ailo Bongo~$^{1}$ \\
%
\vspace{1em} % Space between authors and afilliations
%
\normalfont{\small $^{1}$Department of Computer Science, UiT The Arctic University of Norway}\\
\normalfont{\small $^{2}$Department of Oncology, University Hospital of North Norway}\\
\normalfont{\small $^{3}$Department of Clinical Medicine, UiT The Arctic University of Norway}\\
\normalfont{\small $^{4}$Department of Community Medicine, UiT The Arctic University of Norway}\\
\normalfont{\small $^{5}$Department of Medical Biology, UiT The Arctic University of Norway} \\
\normalfont{\small $^{6}$Department of Clinical Pathology, University Hospital of North Norway} %\vspace{2em}
}

\begin{document}
\maketitle

\section*{Abstract}

% \begin{abstract}
% Developing clinically useful cell-level analysis tools in digital pathology remains challenging due to limitations in dataset granularity, inconsistent annotations, computational demands of advanced models, and difficulties in integrating new technologies into clinical workflows. To address these challenges, we propose a multi-faceted solution that enhances data quality, model performance, and usability to create a lightweight and extensible cell segmentation and classification model.

% First, we update data labels by employing a cross-relabeling process that refines the labels of two existing datasets, PanNuke and MoNuSAC, to create a new unified dataset with enhanced granularity, encompassing seven distinct cell types. Second, we leverage the H-Optimus foundation model as a fixed encoder to improve feature representation for simultaneous cell segmentation and classification tasks. Third, to address the computational demands of foundation models, we employ knowledge distillation to reduce model size and complexity while maintaining comparable performance. Finally, to facilitate integration into clinical workflows, we integrate the distilled model into the QuPath software, a widely used open-source platform in digital pathology.

% Our results demonstrate improvements in cell segmentation and classification performance using the H‑Optimus-based model compared to a CNN-based model. Specifically, the average $R^2$ improved from 0.575 to 0.871, and the average $PQ$ score improved from 0.450 to 0.492, indicating better alignment with actual cell counts and enhanced segmentation and classification quality. Furthermore, the distilled student model maintains performance comparable to the larger foundation model while reducing the parameter count by a factor of 48.
% Overall, by reducing computational complexity and integrating it into existing workflows, the proposed approach may significantly impact diagnostic processes, reduce the workload of pathologists, and contribute to improved patient outcomes. Though our approach shows potential enhancements in efficiency and usability of cell segmentation and classification models in digital pathology, extensive validation is needed to deploy these models in clinical practice.
% \end{abstract}

%%% shortened abstract
\begin{abstract}
Developing clinically useful cell-level analysis tools in digital pathology remains challenging due to limitations in dataset granularity, inconsistent annotations, high computational demands, and difficulties integrating new technologies into workflows. To address these issues, we propose a solution that enhances data quality, model performance, and usability by creating a lightweight, extensible cell segmentation and classification model. 

First, we update data labels through cross-relabeling to refine annotations of PanNuke and MoNuSAC, producing a unified dataset with seven distinct cell types. Second, we leverage the H-Optimus foundation model as a fixed encoder to improve feature representation for simultaneous segmentation and classification tasks. Third, to address foundation models' computational demands, we distill knowledge to reduce model size and complexity while maintaining comparable performance. Finally, we integrate the distilled model into QuPath, a widely used open-source digital pathology platform. 

Results demonstrate improved segmentation and classification performance using the H-Optimus-based model compared to a CNN-based model. Specifically, average $R^2$ improved from 0.575 to 0.871, and average $PQ$ score improved from 0.450 to 0.492, indicating better alignment with actual cell counts and enhanced segmentation quality. The distilled model maintains comparable performance while reducing parameter count by a factor of 48. By reducing computational complexity and integrating into workflows, this approach may significantly impact diagnostics, reduce pathologist workload, and improve outcomes. Although the method shows promise, extensive validation is necessary prior to clinical deployment.
\end{abstract}
\clearpage

\section{Introduction}
In digital pathology, accurate segmentation and classification of cells are crucial for many diagnostic, prognostic, and predictive analyses \cite{Jaber_Beziaeva_etal._2019,Lin_Pan_etal._2022,Park_Ock_etal._2022,Shen_Choi_etal._2024}. Nowadays, developments in computational pathology offer multiple solutions \cite{H._Qu_P._Wu_etal._2020,Javed_Mahmood_etal._2020} to utilize cell-level datasets to train machine learning models that solve these problems. The quality and specificity of training datasets are critical for robust and accurate models. Adhering to the principle of "garbage in, garbage out", it is essential to ensure that these datasets are extensively and accurately labeled with distinct classes that reflect the diverse biological characteristics of different cell types. Unfortunately, the number of open-source datasets comprising such high-quality annotations is limited. Existing cell segmentation datasets \cite{Gamper_Koohbanani_etal._2019,Graham_Vu_etal._2019,Verma_Kumar_etal._2021} may offer extensive annotations for certain cell types while providing more general labels for others. For example, in PanNuke, which is one of the largest open-source datasets comprising labeled cells, various types of morphologically and functionally different inflammatory cells like macrophages and lymphocytes are clustered in a broad "inflammatory" class. Consequently, these classes are frequently omitted from analyses or aggregated into broader meta-classes \cite{Gamper_Koohbanani_etal._2020} and likely interfere with other cell classes included in the dataset. This and similar inconsistencies in annotation granularity limit the ability of machine learning models to learn the comprehensive and nuanced features necessary for accurate cell segmentation and classification. To address these challenges, methods for refining and standardizing dataset annotations are essential to enhance the quality of training data.

A complementary approach to mitigate the absence of high-quality training data is the use of foundation models. Foundation models as encoders are defined as large-scale, versatile networks pre-trained on vast, diverse datasets using self-supervised learning, contrasting with convolutional neural network (CNN) pre-trained encoders that rely on supervised learning with labeled data. In practice, foundation models leverage enormous amounts of weakly or unlabeled data from millions of whole slide images (WSIs) and employ self-attention mechanisms to capture long-range dependencies and global context \cite{Chen_Ding_etal._2024,Saillard_Jenatton_etal._2024,Vorontsov_Bozkurt_etal._2024,Xu_Usuyama_etal._2024}. As a consequence, foundation models are able to produce transferable feature representations across different cell types and tissue environments. The feature representations can be leveraged by decoder networks to produce segmentation masks and pixel-level classifications. Because foundation models have comprehensive feature representations, they can be effectively fine-tuned using much smaller amounts of cell-level data compared to the large datasets needed to train models from scratch. Furthermore, foundation models incorporate adversarial training elements or contrastive learning \cite{Chen_Ding_etal._2024,Xu_Usuyama_etal._2024}, enhancing their resilience and adaptability by exposing them to challenging and varied scenarios during training. This may result in more generalizable models, often making them well-suited for diverse and complex tasks in digital pathology.

Despite the inherent advantages of foundation models, their deployment for practical use faces its own obstacles. In particular, they require substantial computational power, financial investments and rigorous testing to ensure reliability and efficacy for a given task \cite{Akkus_Dangott_etal._2022,Dragomir_Cocuz_etal._2022,Go_2022,Jafri_Farooqui_etal._2024}. Moreover, while foundation models enhance feature representation and performance, they depend on the quality of available annotations for decoder fine-tuning and, like any other model, cannot resolve existing inconsistencies or ambiguities in data labels. Therefore, there remains a critical need for solutions that address both data quality and practical deployment considerations.
Further, integrating new technologies into existing clinical workflows often encounters resistance, as it necessitates adjustments to established diagnostic processes. So, there is a need to develop solutions that could be integrated into current practices, minimizing the burden on medical professionals to adopt new tools \cite{King_Williams_etal._2023}.

Existing solutions \cite{Goldsborough_Philps_etal._2024,Hörst_Rempe_etal._2024}, while addressing some aspects of these challenges, fall short in providing a comprehensive approach. To address the data quality and clinical deployment issues, we propose a multi-faceted solution that encompasses data refinement, model optimization, and integration with existing pathology tools (\hyperref[fig:fig1]{Figure 1}). The outcome is a lightweight cell segmentation and classification model that can be integrated into digital pathology workflows for practical clinical use.

\begin{figure}[h!]
    \centering
    \includegraphics[width=\textwidth, height=0.82\textheight, keepaspectratio]{images/Figure_1.pdf}
    \caption{Overview of the proposed solution, including 1) Data refinement using cross-relabeling, 2) Teacher model development and fine tuning, 3) Student model optimization with knowledge distillation and 4) Student model and QuPath integration}
    \label{fig:fig1}
\end{figure}
\clearpage

Our approach begins with preparing the data for the fine-tuning and training of the machine learning models. We create a refined dataset, acquired via cross-relabeling two cell-level datasets, enhancing annotation specificity and consistency of the labeled data. Subsequently, we create a cell segmentation and classification model based on the foundation model. We leverage the foundation model as a fixed encoder and fine-tune a decoder using the refined dataset to improve generalization across diverse tissue- and cell types.
To ensure that the model remains lightweight and deployable in a possibly resource-constrained environment, we employ knowledge distillation to approximate the functionality of the foundation model. Finally, to facilitate the practical application of our model in digital pathology workflows, we integrate it with the QuPath \cite{Bankhead_Loughrey_etal._2017} application. Each methodological component contributes to the overarching goal of enhancing model performance, generalizability, and usability in clinical settings.

The primary contributions of this paper are:
\begin{enumerate}
    \item \textit{Data labels refinement through cross-relabeling:}
    
    We propose a new method for refining labels of cell-level datasets through cross-relabeling. This method employs classification models to re-label broad and ambiguous instances, resulting in a more diverse dataset. Our evaluation demonstrates that these classification models achieve high accuracy on test subsets, indicating the reliability of the method for label refinement.

    \item \textit{Enhanced model performance via foundation models:}
    
    We employ a foundation model as a feature extractor for the cell segmentation and classification task. In comparison with training a CNN model from scratch, the foundation model backbone only needs fine-tuning, which significantly reduces training time, computational resources and data requirements. We show that using a foundation model encoder leads to better performance in cell segmentation and classification networks than using a CNN-based encoder. This improvement may enable the model to generalize more effectively across various tissue types and imaging methods.
    
    \item \textit{Model optimization through knowledge distillation:}
    
    We show that a smaller student model trained using knowledge distillation on the refined dataset obtained via our cross-relabeling approach from a foundation model achieves comparable performance in cell segmentation and quantification tasks. As a result, this model is more suitable for deployment in environments without high-performance computing resources.
    
    \item \textit{Integration with QuPath:}
    
    We integrate the distilled cell segmentation and classification model into QuPath, a widely used open-source digital pathology platform, to accelerate clinical adaptation by enabling pathologists to more easily incorporate advanced computational tools into their existing workflows.
\end{enumerate}

Through these methodological steps, we aim to bridge the gap between advanced machine learning techniques and practical clinical applications, making accurate and efficient digital pathology accessible in a broader range of healthcare settings.

\section{Refining Existing Datasets Using Cross-Relabeling}
To address the limitations of sparse and ambiguous labeling of cell-level datasets, we propose a generalizable cross-relabeling strategy that can be applied to any dataset containing broadly categorized or imprecisely labeled cell types. This approach involves training and subsequently leveraging classification models to refine broad categories into more specific or biologically relevant classes.
When applied to cell-level data, the methodology includes extracting individual cell images from the dataset patches, preprocessing these images to standardize the size and accommodate partial cells, and then training deep learning classifiers capable of distinguishing between the finer cell subtypes within the coarser categories. 
To illustrate our approach, we focus on the PanNuke \cite{Gamper_Koohbanani_etal._2020, Gamper_Koohbanani_etal._2019} and MoNuSAC \cite{Verma_Kumar_etal._2021} datasets that we have used to train models for cell quantification in our previous works \cite{Shvetsov_Grønnesby_etal._2022,Shvetsov_Sildnes_etal._2024}. We find that for better cell differentiation we have to introduce more granular labels. PanNuke includes a broad classification of "inflammatory" cells, encompassing lymphocytes, macrophages, and neutrophils. Each cell type differs significantly in structure, function, and clinical relevance. Conversely, MoNuSAC uses the label "epithelial" for a class that comprises both benign epithelial cells and malignant neoplastic cells. This practice makes it challenging to differentiate between benign and malignant epithelial cells in the dataset, which is a critical distinction when identifying tumor areas within tissue samples. To address these issues, we implement a cross-relabeling strategy as shown in \hyperref[fig:fig2]{Figure 2}. The key components are two classification models: one is trained on singular cell images from PanNuke data to classify the epithelial meta-class into epithelial and neoplastic classes. The other is trained on MoNuSAC to refine the inflammatory class into lymphocytes, neutrophils, and macrophages.

\begin{figure}[h!]
    \centering
    \includegraphics[width=\textwidth]{images/Figure_2.pdf}
    \caption{Refined dataset generation via cross relabeling}
    \label{fig:fig2}
\end{figure}

The refining approach consists of three consecutive steps. The first is the preprocessing step, in which we extract individual cells from both datasets (\hyperref[fig:fig3]{Figure 3}). The specifics of PanNuke and MoNuSAC patch preparation before cell preprocessing are provided in \hyperref[chap:S1]{Appendix S1}.

\begin{figure}[h!]
    \centering
    \includegraphics[width=\textwidth]{images/Figure_3.pdf}
    \caption{Cell instances preprocessing including (1) cell map extraction, (2) bounding box delineation, (3) adjusting cell boxes and (4) cropping and resizing of cell images}
    \label{fig:fig3}
\end{figure}

During preprocessing, we extract cell type maps from the ground truth label mask and calculate bounding boxes around each cell instance. To accommodate partial cells at patch borders, a common issue in cropped patch images, we employ mirror padding and extend the field of view of the cell label by 15 pixels to capture adjacent cells. We then crop and resize the identified regions to $64 \times 64$ pixels using bicubic interpolation.

The preprocessed PanNuke dataset comprises 68,031 neoplastic and 23,207 epithelial cell images, while MoNuSAC comprises  33,104 lymphocytes, 1,252 neutrophils, and 1,695 macrophages, which we subsequently use in training cell classification models and classifying the cell image data \hyperref[fig:S2]{Appendix Figure S2 (1)}. 

The next step is to train two distinct ResNet50-based classifiers tailored to address the specific labeling challenges inherent in each dataset. We use ResNet50 for classification models due to its proven effectiveness for image classification tasks in histopathology \cite{pan2022reviewmachinelearningapproaches}, and its compatibility with small images. For the PanNuke dataset, we design the classifier, trained on MoNuSAC data, to disaggregate the heterogeneous "inflammatory" cell category into distinct subtypes: lymphocytes, macrophages, and neutrophils. Similarly, for the MoNuSAC dataset, the classifier is trained on PanNuke data and distinguishes between benign and malignant epithelial cells within the overarching "epithelial" label. By applying these targeted classifiers to their respective datasets, we assign more specific labels to individual cell instances, thus enabling us to create a unified dataset.
To ensure a balanced representation of classes, we train both models on datasets that had been equalized to match the size of the least represented class. Thus, we obtain datasets comprising 23,207 samples per class for PanNuke and 1,252 samples per class for MoNuSAC data. Next, we partition both of them into training (70\%), validation (20\%), and testing (10\%) subsets. To mitigate the risk of overfitting, we use a single dropout layer with a rate of p=0.5 in both models and data augmentation using randomized color perturbations, rotation, and horizontal and vertical flipping. We employ AdamW optimizer and the cross-entropy loss function for the training criterion.

To evaluate the two trained models, we measure the classification accuracy on the respective test subsets. The accuracies on the test subset for both classifiers are presented in \hyperref[tab:1]{Table 1}. The PanNuke model achieves an average accuracy of 93.57\%, with higher accuracy for neoplastic cells (96.06\%) compared to epithelial cells (86.26\%). The confusion matrix in Figure A3.1 shows that the model predominantly distinguishes accurately between epithelial and neoplastic tissues, with a substantial number of correct classifications and relatively few misclassifications. The MoNuSAC model demonstrates an average accuracy of 98.92\%, excelling in classifying lymphocytes (99.67\%) and macrophages (94.12\%), with lower performance for neutrophils (85.71\%). The confusion matrix in Figure A3.2 shows that the model identifies lymphocytes and performs reasonably well with macrophages and neutrophils.

\begin{table}[h!]
\renewcommand{\arraystretch}{1.5}
  \centering
  \caption{Cell classification results for PanNuke and MoNuSAC trained models (CI 95\%).}
  \label{tab:1}
  \begin{tabular}{|l|c|c|}
   \hline
   %\rowcolor{gray!30}
    Accuracy               & PanNuke model              & MoNuSAC model              \\
    \hline
    Average      & 0.936 (0.931--0.941)         & 0.989 (0.986--0.993)        \\
    \hline
    Neoplastic   & 0.961 (0.956--0.965)         & -                          \\
    \hline
    Epithelial   & 0.863 (0.849--0.877)         & -                          \\
    \hline
    Lymphocytes  & -                          & 0.997 (0.995--0.999)        \\
    \hline
    Neutrophils  & -                          & 0.857 (0.796--0.918)        \\
    \hline
    Macrophages  & -                          & 0.941 (0.906--0.976)        \\
    \hline
  \end{tabular}
\end{table}

Finally, during the last step, we use the model trained on PanNuke data for epithelial cells in MoNuSAC and the model trained on MoNuSAC for the inflammatory cells class in PanNuke. Specifically, we use classifier models to relabel epithelial cells in MoNuSAC and inflammatory cells in PanNuke data. Then we combine cells with refined labels and the rest of the cells in both datasets to create a refined dataset (\hyperref[fig:S2]{Appendix Figure S2 (2)}). The process of relabeling cells and visualizing them on a patch is shown in \hyperref[fig:fig4]{Figure 4}. The cell counts in the refined dataset are provided in \hyperref[tab:S4]{Appendix Table S4}.

\begin{figure}[h!]
    \centering
    \includegraphics[width=\textwidth, height=0.42\textheight, keepaspectratio]{images/Figure_4.pdf}
    \caption{Cell relabeling procedure for epithelial and inflammatory cell classes}
    \label{fig:fig4}
\end{figure}

%\hfill

Relabeling and combining datasets have been explored in a prior study \cite{Parulekar_Kanwat_etal._2023}, where consecutive fine-tuning on multiple datasets was employed to account for hierarchical class label structures. While the method presented in \cite{Parulekar_Kanwat_etal._2023} is intuitive, it often lacks consistency and requires multiple fine-tuning runs, which can be cumbersome and time-consuming. 
In contrast, cross-relabeling simplifies this process by using specialized classification models tailored to each dataset's specific labeling challenges. This approach provides better transparency and produces a unified dataset encompassing seven distinct cell types across multiple tissue samples, enhancing data diversity for further model training or fine-tuning.

Despite these improvements, cross-relabeling does not entirely resolve issues related to poor labeling quality or the amount of labeled data. Specifically, our results show lower accuracies persist for underrepresented classes, such as macrophages, which may stem from a limited sample availability and intrinsic challenges in distinguishing these cells based solely on H\&E staining. Furthermore, while our method enhances label specificity, it relies on the initial quality of the broad labels; thus, any fundamental inaccuracies in the original annotations can propagate through the relabeling process. Addressing the overall problem of limited data labels may require integrating additional data sources or utilizing complementary immunohistochemical staining methods.
Although the reported performance metrics are obtained from evaluations on the native test sets of each dataset, it is important to note that the primary application of these classifiers is to perform cross-relabeling, where a model trained on one dataset (e.g., PanNuke) is applied to another (e.g., MoNuSAC) and vice versa. We acknowledge that a more systematic evaluation of cross-dataset generalization is needed and could be performed in future work.

Overall, the refined dataset produced by our approach can enhance the supervised training or fine-tuning of cell segmentation and classification models, especially those that utilize pre-trained foundation models to improve feature extraction robustness. In addition, these models can detect nuanced classes that enable researchers to conduct more detailed analyses of biological processes in computational pathology.

\section{Foundation models for robust cell segmentation and classification}

Accurate cell segmentation and classification in digital pathology are hindered by limited labeled data and the fact that conventional CNNs are unable to capture global contextual information due to their local receptive field constraints \cite{Gheflati_Rivaz_2022,Yang_Marcus_etal.}. Traditional approaches in cell quantification have predominantly relied on CNN encoders, such as ResNet50, given their proven effectiveness in semantic segmentation tasks \cite{Deshmane_2023,Graham_Vu_etal._2019,Mukasheva_Koishiyeva_etal._2024,Stringer_Wang_etal._2021}. However, approaches that include fine-tuning of pretrained CNNs, data augmentation, and stain normalization to partially increase data variability and address staining differences often fail to achieve the necessary generalization and robustness across diverse tissue types and staining conditions \cite{G._Wang_W._Li_etal._2018,Gao_Bagci_etal._2018,Karim_El_Khoury_Martin_Fockedey_etal._2021}.

To overcome these challenges, we leverage an encoder-decoder network that uses a foundation model as the encoder and a CNN upsampling decoder (\hyperref[fig:fig5]{Figure 5}) for simultaneous cell segmentation and classification in 2D patches extracted from WSIs. Foundation models with transformer-based architectures are viable alternatives to CNN-based encoders \cite{Shamshad_Khan_etal._2023,Sourget_2023}. They enable the creation of more advanced architectures that can decode or transform learned features more effectively \cite{Chen_Duan_etal._2023,Cheng_Misra_etal._2022,Xie_Wang_etal._2021}.

\begin{figure}[h!]
    \centering
    \includegraphics[width=\textwidth]{images/Figure_5.pdf}
    \caption{UNETR-like model with foundational model as backbone}
    \label{fig:fig5}
\end{figure}

By utilizing a transformer-based encoder, we incorporate global contextual information into the feature extraction process, which is a key advantage of such architectures \cite{Chen_Lu_etal._2021}. This foundation model integration facilitates accurate pixel-wise segmentation and classification without the need for extensive encoder training, thereby potentially improving generalization across varied cellular structures and tissue types.
In our implementation, we employ a modified UNETR \cite{Hatamizadeh_Tang_etal._2021} architecture that combines a vision transformer (ViT) \cite{Dosovitskiy_Beyer_etal._2021} encoder with a CNN-based decoder. The encoder utilizes the pretrained H-Optimus foundation model, which contains 1.1 billion parameters and is trained on over 500,000 H\&E stained WSIs \cite{Saillard_Jenatton_etal._2024}. We extract outputs from four evenly spaced transformer blocks $Z_i$, where $i \in [1, 14, 26, 38]$, to serve as residual connections for the CNN decoder. We select these blocks based on our observation that features from non-adjacent levels of the encoder lead to better overall performance on the test subset.

The CNN decoder upsamples the feature representations, acquired from the transformer blocks, to generate an intermediate vector that is handled by two task-specific layers that generate cell segmentation and classification masks. The first task-specific layer is the ‘Cellpose head’,  which is used to delineate cell instances. The layer generates horizontal and vertical gradient maps to form vector fields that are refined through gradient tracking in a post-processing step using the Cellpose algorithm \cite{Stringer_Wang_etal._2021}, known for its efficacy in cell segmentation tasks and generalizability across multiple domains \cite{Pachitariu_Stringer_2022,Stringer_Pachitariu_2024}. The second task-specific layer is the "Cell type head", which assigns labels to individual pixels. In the post-processing step, we determine the output classification label of each segmented cell instance by majority voting over the labeled pixels that comprise the cell in the segmentation map.

To evaluate model performance and measure the impact of adding a foundation model as backbone, we compare it to a ResNet50-based model. ResNet50 is a widely used solution for encoders in segmentation architectures in the medical domain \cite{Deshmane_2023,Graham_Vu_etal._2019,Mukasheva_Koishiyeva_etal._2024,Stringer_Wang_etal._2021}. For the H-Optimus-based model, we utilize frozen weights for the encoder and only fine-tune the decoder to take advantage of the extensive pre-training of the foundation model. For the ResNet50-based model we start with ImageNet \cite{Deng_Dong_etal.} weights and train both encoder and decoder parts. Hyperparameters for the training step are set to be identical, where possible, for comparable evaluation. 
For this evaluation, we deliberately use the PanNuke dataset to provide a standardized and controlled comparison between the H‑Optimus and ResNet50-based models (\hyperref[fig:S2]{Appendix Figure S2 (3)}). Specifically, we use two of the default PanNuke dataset splits (66\%) for training and validation, and reserve the third split (33\%) for testing.

To address the challenge of cell class imbalance in the PanNuke dataset, which is a common characteristic in most cell-level H\&E patch datasets, both models’ training processes employ a weighted loss function comprising cross-entropy and focal loss \cite{Lin_Goyal_etal._2018}. The focal loss component is adjusted with coefficients derived from each cell class' instance frequency, emphasizing learning from underrepresented classes and enhancing the model's sensitivity to rare but significant cellular patterns. The cross-entropy loss is augmented with spectral decoupling regularization \cite{Pezeshki_Kaba_etal._2021,Pohjonen_Stürenberg_etal._2022} and spatially varying label smoothing \cite{Islam_Glocker_2021}, which potentially stabilizes training and improves generalization in case of complex tissue morphologies. For optimization, we employ the \textit{AdamW} \cite{Loshchilov_Hutter_2019} to counter unbalanced class scenarios, with cosine annealing learning rate scheduler.

We utilize the scikit-learn library \cite{Van_der_Walt_Schönberger_etal._2014} and HoVer-Net \cite{Graham_Vu_etal._2019} implementations of $R^2$ (the coefficient of determination) and $PQ$ (panoptic quality) to evaluate our experiments. Complete mathematical formulations and detailed explanations of these metrics are provided in \hyperref[chap:S5]{Appendix S5}. To compute confidence intervals, we use nonparametric bootstrapping, where after calculating the metric on the full sample, we generated 1000 bootstrap replicates by resampling with replacement and then determined the 95\% confidence intervals as the 2.5th and 97.5th percentiles of the resulting empirical distribution.

%\hfill

The model comparisons are summarized in \hyperref[tab:2]{Table 2}. The H‑Optimus-based model achieves higher $R^2$ across all cell classes compared to the ResNet50-based model, which means that its predictions are more closely aligned with the PanNuke cell counts, indicating a stronger correlation with the observed data. Notably, the improvement of $R^2_{dead}$ may be an indicator of better global contextual representations provided by the foundation model backbone. In terms of segmentation and classification quality combined, measured by the PQ score, the H‑Optimus-based model demonstrates notable improvements across most cell classes. Overall, the average $R^2$ improved from 0.575 to 0.871, while the average $PQ$ score improved from 0.450 to 0.492, demonstrating better performance of the H-Optimus-based model.

\begin{table}[h!]
\renewcommand{\arraystretch}{1.5}
  \centering
  \caption{Cell quantification metrics for baseline and proposed models (CI 95\%).}
  \label{tab:2}
  \begin{tabular}{|l|c|c|}
    \hline
    %\rowcolor{gray!30}
    Metric             & Resnet50-based            & H-optimus-based              \\
    \hline
    $R^2_{neoplastic}$    & 0.681 (0.576--0.769)       & \textbf{0.941 (0.917--0.960)} \\
    \hline
    $R^2_{inflammatory}$  & 0.863 (0.778--0.903)       & \textbf{0.949 (0.918--0.966)} \\
    \hline
    $R^2_{connective}$    & 0.600 (0.488--0.698)       & 0.609 (0.436--0.772)          \\
    \hline
    $R^2_{dead}$          & 0.097 (-11.389--0.669)     & 0.925 (0.404--0.982)          \\
    \hline
    $R^2_{epithelial}$    & 0.635 (0.490--0.747)       & \textbf{0.930 (0.886--0.964)} \\
    \hline
    $PQ_{neoplastic}$       & 0.517 (0.499--0.535)       & \textbf{0.589 (0.575--0.604)} \\
    \hline
    $PQ_{inflammatory}$     & 0.455 (0.429--0.482)       & \textbf{0.528 (0.507--0.549)} \\
    \hline
    $PQ_{connective}$       & 0.416 (0.400--0.431)       & \textbf{0.451 (0.436--0.465)} \\
    \hline
    $PQ_{dead}$             & 0.374 (0.342--0.408)       & 0.292 (0.209--0.365)          \\
    \hline
    $PQ_{epithelial}$       & 0.488 (0.460--0.519)       & \textbf{0.599 (0.579--0.618)} \\
    \hline
  \end{tabular}
\end{table}

Our results  show that integrating the H‑Optimus foundation model within the UNETR architecture enhances the model's ability to segment and classify cells across diverse tissues from PanNuke data. The pretrained transformer encoder provides robust feature representations, resulting in higher average $R^2$ and $PQ$ scores compared to the CNN-based model. This leads to more reliable cell quantification and more accurate downstream analysis. Additionally, the streamlined fine-tuning process reduces computational overhead and training time, making the model more adaptable for new data.

Despite these advancements, the foundation model-based approach does not fully resolve all challenges related to cell segmentation and classification. We observe lower metric scores for underrepresented classes in the training data. Furthermore, foundation models typically encompass billions of parameters, resulting in substantial computational and memory requirements. It therefore poses challenges for deployment in resource-constrained environments, limiting their practical applicability in certain clinical settings.

\section{Model optimization via Knowledge Distillation}

To address the limitations posed by the extensive size of foundation models, we implement knowledge distillation — a model compression technique that leverages the teacher-student paradigm \cite{Hinton_Vinyals_etal._2015}. By training a smaller, more efficient student model to replicate the output of a larger, pre-trained teacher model, we retain performance while significantly reducing the model's complexity and resource requirements (\hyperref[fig:fig6]{Figure 6}).

\begin{figure}[h!]
    \centering
    \includegraphics[width=\textwidth, height=0.45\textheight, keepaspectratio]{images/Figure_6.pdf}
    \caption{Knowledge distillation framework for training a student model using a pre-trained teacher}
    \label{fig:fig6}
\end{figure}

We employ knowledge distillation to compress the H‑Optimus-based teacher model into a more efficient student model. The teacher model is the modified UNETR architecture with the H‑Optimus foundation model described in the previous chapter. The student model is based on a UNet architecture augmented with residual connections and incorporates a smaller ViT encoder with 9 million parameters \cite{Steiner_Kolesnikov_etal._2022,Wightman_2019}. 

First, we fine-tune the teacher model using the refined dataset from the cross-relabeling procedure (Section 2). Initially we train the decoder of the teacher model while keeping the encoder weights frozen. We split the refined dataset into train (70\%), validation (20\%) and test (10\%) subsets (\hyperref[fig:S2]{Appendix Figure S2 (4)}). During fine-tuning, we use the train and validation subsets, while leaving the test subset for model evaluation. We set the training procedure and model hyperparameters to be identical to those that were used to demonstrate the utility of foundation models for the simultaneous cell segmentation and classification task.

Next, we perform knowledge distillation from teacher to student using the refined dataset used to fine-tune the teacher model. The student model is trained to replicate the teacher model's outputs. We utilize a specialized loss function that aligns the student's predicted probability distribution with the teacher's, incorporating the teacher's class probability distribution derived from the output. Following the methodology of Hinton et al. \cite{Hinton_Vinyals_etal._2015}, we experiment with various hyperparameter settings for the temperature ($T$) and the balancing coefficients ($\alpha$ and $\beta$) in the loss function. We vary $T$ from 1 to 20 and adjust $\alpha$ and $\beta$ to balance the distillation and student losses. Through iterative tuning and evaluation, we identify that setting $T=14$, $\alpha=0.3$, and $\beta=0.7$ yields a configuration that converges and closely approximates the teacher model's performance during training.

Finally, we assess the performance of both models using the $R^2$ and $PQ$ (defined in \hyperref[chap:S5]{Appendix S5}) on the test set of the refined dataset (\hyperref[tab:3]{Table 3}). We observe that the 95\% confidence intervals overlap for most cell types, so we cannot claim statistically significant performance differences between the teacher and student models. One exception appears in the neoplastic class. The teacher model produces an $R^2$ of 0.919, while the student model shows an $R^2$ of 0.852. In addition, the student model achieves higher $PQ$ values for the neoplastic and connective classes, though the confidence intervals show overlap.

\begin{table}[h!]
\renewcommand{\arraystretch}{1.5}
  \centering
  \caption{Cell quantification metrics for teacher and distilled student models (CI 95\%).}
  \label{tab:3}
  \begin{tabular}{|l|c|c|}
    \hline
    %\rowcolor{gray!30}
    Metric & Teacher & Student \\
    \hline
    $R^2_{neoplastic}$    & \textbf{0.919} (0.898--0.939) & 0.852 (0.800--0.891) \\
    \hline
    $R^2_{lymphocyte}$    & 0.969 (0.956--0.977)         & 0.969 (0.956--0.978) \\
    \hline
    $R^2_{connective}$    & 0.694 (0.548--0.809)         & 0.618 (0.469--0.741) \\
    \hline
    $R^2_{dead}$          & 0.755 (0.400--0.908)         & 0.424 (0.100--0.731) \\
    \hline
    $R^2_{epithelial}$    & 0.922 (0.870--0.958)         & 0.843 (0.738--0.917) \\
    \hline
    $R^2_{macrophage}$    & 0.384 (-0.369--0.724)        & 0.704 (0.352--0.859) \\
    \hline
    $R^2_{neutrofil}$     & 0.854 (0.578--0.929)         & 0.833 (0.502--0.925) \\
    \hline
    $PQ_{neoplastic}$       & 0.581 (0.569--0.593)         & 0.601 (0.588--0.613) \\
    \hline
    $PQ_{lymphocyte}$       & 0.536 (0.520--0.553)         & 0.563 (0.544--0.579) \\
    \hline
    $PQ_{connective}$       & 0.436 (0.421--0.451)         & 0.457 (0.441--0.474) \\
    \hline
    $PQ_{dead}$             & 0.272 (0.235--0.315)         & 0.279 (0.201--0.369) \\
    \hline
    $PQ_{epithelial}$       & 0.522 (0.500--0.545)         & 0.530 (0.506--0.555) \\
    \hline
    $PQ_{macrophage}$       & 0.524 (0.459--0.588)         & 0.474 (0.405--0.543) \\
    \hline
    $PQ_{neutrofil}$        & 0.541 (0.490--0.592)         & 0.565 (0.522--0.607) \\
    \hline
  \end{tabular}
\end{table}


We further decompose the $PQ$ metric into its $SQ$ and $DQ$ components (\hyperref[tab:S6]{Appendix Table S6}). Both models produce nearly identical $SQ$ values, which indicates that they predict instance boundaries with similar precision. Although the student model shows some improvement in $DQ$ scores for certain classes, the confidence intervals overlap and do not confirm a statistically significant difference.

We observe that the student and teacher models yield comparable detection performance despite the student model using a much smaller and simpler architecture. A model with fewer parameters reduces the risk of overfitting when training data are scarce relative to the model’s complexity \cite{Farias_Ludermir_etal._2022}. The knowledge distillation process also encourages the student model to focus on the most generalizable detection features learned from the teacher. These factors enable the student model to achieve similar detection performance across different cell types.

Additionally, considering the model sizes reported in \hyperref[tab:4]{Table 4}, the distilled model achieves a significant reduction compared to the teacher model, with a 48-fold decrease in parameter count and a 5.5-fold reduction in on-disk size. In inference mode, the teacher model requires 16 GB of VRAM for a batch size of 32, while the distilled model only needs 3 GB of VRAM for the same batch size. These reductions make the distilled model significantly more practical for fine-tuning and deployment in resource-constrained environments.

\begin{table}[h!]
\renewcommand{\arraystretch}{1.5}
  \centering
  \caption{Parameter counts and size of teacher and distilled model}
  \label{tab:4}
  \adjustbox{max width=\textwidth}{%
  \begin{tabular}{|l|c|c|c|}
    \hline
    %\rowcolor{gray!30}
    Metric & H-optimus-based (Teacher) & mobileViT-based (Student) & Magnitude of difference \\
    \hline
    Parameters count       & 1,158,917,906   & \textbf{24,093,393}   & \textbf{48x}  \\
    \hline
    Estimated Total Size (MB) & 87,912       & \textbf{15,935}    & \textbf{5.5x} \\
    \hline
  \end{tabular}%
}
\end{table}

%\hfill

With recent advancements in complex network architectures and the use of pretrained encoders to achieve state-of-the-art performance \cite{Baumann_Dislich_etal._2024,Hörst_Rempe_etal._2024} in cell segmentation and classification tasks, model size, computational complexity, and processing times have increased. This limits the scalability and accessibility of these models. As we demonstrate, this may be mitigated using knowledge distillation. Studies in the field of natural language processing have demonstrated the efficacy of knowledge distillation in retaining the capabilities of the teacher model while achieving significant reductions in size and complexity \cite{Huangpu_Gao_2024,Sun_Yu_etal.}. 

We demonstrate the feasibility of knowledge distillation in digital pathology, specifically for cell segmentation and classification tasks. Moreover, we achieve this performance while also significantly reducing the parameter count. In addressing the challenge of knowledge transfer, we found that distillation from a transformer-based model to a smaller transformer is more straightforward than attempting to map transformer features to CNN blocks. In our experiments, using a CNN-based network as a student results in worse cell quantification performance due to the structural constraints of CNN feature space dimensions. 

Although our primary approach relies on a transformer-based student model that performs well, it can be further optimized to incorporate advantages from CNN architectures. For example, employing alternative techniques such as using ViT adapters \cite{Chen_Duan_etal._2023} or $1 \times 1$ convolutions to adjust feature map sizes may be beneficial for harnessing CNN advantages like enhanced local feature extraction. Moreover, if additional performance improvements are desired, the process can be further enhanced by applying supplementary knowledge distillation techniques, such as self-distillation \cite{Zhang_Song_etal._2019} or online distillation \cite{Houyon_Cioppa_etal._2023}.

Despite these promising results, further validation on independent datasets is necessary to fully understand the model's limitations. Underrepresented classes may pose challenges when addressing complex cases. Pathologists need to validate these models to adopt them in clinical settings. While the distilled models are smaller and more deployable, a technological gap persists because pathologists traditionally rely on established methods for inspecting WSIs and diagnosing diseases. Addressing the complexities involved in deploying models for inference and supporting pathologists in adopting new tools is essential for integrating these models into clinical workflows.

\section{Model integration with QuPath}
Digital pathology tools with graphical user interfaces are essential for visualizing and analyzing WSIs. To make our student model useful in clinical pathology workflows, it needs to be integrated into a tool that enables inspecting regions, creating annotations, and providing quantitative analyses of biomarkers. Therefore, we integrate the trained student model from the previous chapter into the QuPath open‑source platform \cite{Bankhead_Loughrey_etal._2017}. QuPath provides the required annotation, visualization, and analysis tools to interpret complex histological data, including workflows for cell segmentation, classification, and quantification (\hyperref[fig:fig7]{Figure 7}). 

\begin{figure}[h!]
    \centering
    \includegraphics[width=\textwidth]{images/Figure_7.pdf}
    \caption{Visualization of model-generated cell quantification annotations (left) and the corresponding unannotated slide (right) in QuPath}
    \label{fig:fig7}
\end{figure}

To identify the regions in a WSI critical for prognosticating tumor development, such as specific tumor areas or border regions without overlapping healthy tissue, the pathologist uses QuPath to outline these regions. Then, the pathologist initiates a cell segmentation and classification script through the QuPath interface for the selected regions. The resulting annotations and quantified cell information are then directly overlaid onto the WSI in the QuPath interface. Additional design and implementation details are in \hyperref[chap:S7]{Appendix S7}. 

Two common approaches for integrating deep learning models into QuPath are Java‑based native QuPath extensions \cite{Goldsborough_Philps_etal._2024} and the execution of RESTful API requests to a model server coupled with handling the response via an extension, as demonstrated in the application of cell segmentation models applied to immunofluorescence images \cite{Sugawara_2023}. While the community is actively working on these integration strategies, there is currently no universal solution that fully addresses all integration and performance requirements.

Extensions may offer better integration with QuPath, allowing slightly improved performance and more widespread usage of the built-in QuPath models, but they lack the flexibility to customize models and modify their behavior. For example, the newest version of QuPath includes models such as StarDist \cite{Weigert_Schmidt} and InstanSeg \cite{Goldsborough_Philps_etal._2024} that can perform cell segmentation. Both models pose limitations when applied to simultaneous cell segmentation and classification. StarDist performs well only on convex, round shapes by design, whereas some neoplastic, inflammatory, and connective cells exhibit complex and non-convex shapes. InstanSeg provides only semantic segmentation without assigning classes to the segmented cells.

%\hfill

In contrast, our approach offers an alternative integration strategy. It utilizes the paquo library to directly interact with QuPath’s internal application programming interface from within Python. This enables data exchange and processing without the need for intermediate conversion steps and provides greater control over model customization, retraining, and the incorporation of custom processing steps.

The integration of our custom model with QuPath underscores its potential to significantly enhance the diagnostic process by reducing the time burden on pathologists and enabling them to focus on more complex interpretative tasks using familiar software. Leveraging a tool that is already well-established among pathologists increases the likelihood of its adoption into daily clinical workflows. The quantitative data generated through the automated workflow is critical for both clinical decision-making and research, facilitating more accurate biomarker analysis, enabling robust statistical evaluations, and supporting hypothesis generation and testing. Additionally, by streamlining cell segmentation and classification, the tool enhances the scalability and reproducibility of pathological assessments, ultimately contributing to improved diagnostic accuracy and patient outcomes.

\section{Conclusion and future work}

In this study, we address critical challenges in digital pathology and tackle the usability and deployment issues of the developed models in standard computing environments without the need for high-performance computing systems. Our multi-faceted approach encompasses data refinement through cross-relabeling, leveraging foundation models for robust cell segmentation and classification, optimizing model performance via knowledge distillation, and integrating the optimized model into the QuPath software for practical application. This approach is used to construct a capable, versatile, and adjustable model for cell segmentation and classification, with enhanced performance and usability.

\begin{sloppypar}
While our approach shows potential in the field of computational pathology, certain limitations persist. 
For example, our implementation currently exhibits lower performance in detecting macrophages. 
This serves as an instance of the broader challenge of accurately identifying complex cell types. In order to address this issue, extending our approach to incorporate additional data sources, exploring alternative modeling approaches, and integrating other imaging modalities such as immunohistochemical staining may help improve detection accuracy. Moreover, although the distilled model reduces computational demands, integrating advanced deep learning models into clinical practice requires addressing technological gaps and potential resistance to adopting new tools within established diagnostic processes.
\end{sloppypar}

Future work could focus on several key areas to refine the proposed approach and facilitate its adoption in clinical environments. Enhancing the cell-relabeling process with additional datasets \cite{Graham_Jahanifar_etal._2021} could improve the representation of underrepresented cell types and enhance overall model performance. Also, incorporating additional data sources, such as multi-modal imaging or complementary staining methods, may address limitations related to cell type differentiation and class imbalance. Exploring other foundation models \cite{Vorontsov_Bozkurt_etal._2024,Zimmermann_Vorontsov_etal._2024} or introducing additional modalities \cite{Ding_Wagner_etal._2024,Vaidya_Zhang_etal._2025} may provide alternative architectures better suited to specific tasks or offer improved efficiency. Implementing more complex knowledge distillation techniques \cite{Houyon_Cioppa_etal._2023,Zhang_Song_etal._2019} could further optimize the model's performance and adaptability. Additionally, deeper integration with QuPath or other digital pathology software could provide pathologists more control over cell quantification analysis directly within the QuPath interface, thereby increasing accessibility and usability. Such enhancements would not only refine model performance but also ensure greater adaptability and scalability within various clinical environments. Finally, extensive validation of the model by pathologists and benchmarking against independent datasets are essential steps toward establishing the model's reliability and fostering confidence in its clinical utility.

\section*{Acknowledgments} 
This work was funded in part by the Research Council of Norway grant no. 309439 SFI Visual Intelligence, and the North Norwegian Health Authority grant no. HNF1521-20.

\bibliographystyle{IEEEtran}
\begin{sloppypar}
\begin{thebibliography}{99}

\bibitem{chaplot2020neural} Chaplot, Devendra Singh, et al. "Neural topological slam for visual navigation." Proceedings of the IEEE/CVF conference on computer vision and pattern recognition. 2020.

\bibitem{maksymets2021thda} Maksymets, Oleksandr, et al. "Thda: Treasure hunt data augmentation for semantic navigation." Proceedings of the IEEE/CVF International Conference on Computer Vision. 2021.

\bibitem{mezghan2022memory} Mezghan, Lina, et al. "Memory-augmented reinforcement learning for image-goal navigation." 2022 IEEE/RSJ International Conference on Intelligent Robots and Systems (IROS). IEEE, 2022.

\bibitem{al2022zero} Al-Halah, Ziad, Santhosh Kumar Ramakrishnan, and Kristen Grauman. "Zero experience required: Plug \& play modular transfer learning for semantic visual navigation." Proceedings of the IEEE/CVF Conference on Computer Vision and Pattern Recognition. 2022.

\bibitem{ye2021auxiliary} Ye, Joel, et al. "Auxiliary tasks and exploration enable objectgoal navigation." Proceedings of the IEEE/CVF international conference on computer vision. 2021.

\bibitem{chaplot2020object} Chaplot, Devendra Singh, et al. "Object goal navigation using goal-oriented semantic exploration." Advances in Neural Information Processing Systems 33 (2020)

\bibitem{ramakrishnan2022poni} Ramakrishnan, Santhosh Kumar, et al. "Poni: Potential functions for objectgoal navigation with interaction-free learning." Proceedings of the IEEE/CVF Conference on Computer Vision and Pattern Recognition. 2022.

\bibitem{ramrakhya2022habitat} Ramrakhya, Ram, et al. "Habitat-web: Learning embodied object-search strategies from human demonstrations at scale." Proceedings of the IEEE/CVF Conference on Computer Vision and Pattern Recognition. 2022.

\bibitem{mousavian2019visual} Mousavian, Arsalan, et al. "Visual representations for semantic target driven navigation." 2019 International Conference on Robotics and Automation (ICRA). IEEE, 2019.

\bibitem{dhariwal2021diffusion} Dhariwal, Prafulla, and Alexander Nichol. "Diffusion models beat gans on image synthesis." Advances in neural information processing systems 34 (2021)

\bibitem{ho2022classifier} Ho, Jonathan, and Tim Salimans. "Classifier-free diffusion guidance." arXiv preprint arXiv:2207.12598 (2022).

\bibitem{nichol2021glide} Nichol, Alex, et al. "Glide: Towards photorealistic image generation and editing with text-guided diffusion models." arXiv preprint arXiv:2112.10741 (2021)

\bibitem{brooks2023instructpix2pix} Brooks, Tim, Aleksander Holynski, and Alexei A. Efros. "Instructpix2pix: Learning to follow image editing instructions." Proceedings of the IEEE/CVF Conference on Computer Vision and Pattern Recognition. 2023.

\bibitem{fu2023guiding} Fu, Tsu-Jui, et al. "Guiding instruction-based image editing via multimodal large language models." arXiv preprint arXiv:2309.17102 (2023).

\bibitem{geng2024instructdiffusion} Geng, Zigang, et al. "Instructdiffusion: A generalist modeling interface for vision tasks." Proceedings of the IEEE/CVF Conference on Computer Vision and Pattern Recognition. 2024.

\bibitem{zhou2024minedreamer} Zhou, Enshen, et al. "Minedreamer: Learning to follow instructions via chain-of-imagination for simulated-world control." arXiv preprint arXiv:2403.12037 (2024).

\bibitem{zhou2023esc} Zhou, Kaiwen, et al. "Esc: Exploration with soft commonsense constraints for zero-shot object navigation." International Conference on Machine Learning. PMLR, 2023.

\bibitem{yu2023l3mvn} Yu, Bangguo, Hamidreza Kasaei, and Ming Cao. "L3mvn: Leveraging large language models for visual target navigation." 2023 IEEE/RSJ International Conference on Intelligent Robots and Systems (IROS). IEEE, 2023.

\bibitem{gadre2023cows} Gadre, Samir Yitzhak, et al. "Cows on pasture: Baselines and benchmarks for language-driven zero-shot object navigation." Proceedings of the IEEE/CVF Conference on Computer Vision and Pattern Recognition. 2023.

\bibitem{shah2023navigation} Shah, Dhruv, et al. "Navigation with large language models: Semantic guesswork as a heuristic for planning." Conference on Robot Learning. PMLR, 2023.

\bibitem{cai2024bridging} Cai, Wenzhe, et al. "Bridging zero-shot object navigation and foundation models through pixel-guided navigation skill." 2024 IEEE International Conference on Robotics and Automation (ICRA). IEEE, 2024.

\bibitem{yu2023co} Yu, Bangguo, Hamidreza Kasaei, and Ming Cao. "Co-NavGPT: Multi-robot cooperative visual semantic navigation using large language models." arXiv preprint arXiv:2310.07937 (2023).

\bibitem{wu2024voronav} Wu, Pengying, et al. "Voronav: Voronoi-based zero-shot object navigation with large language model." arXiv preprint arXiv:2401.02695 (2024).

\bibitem{qin2023mp5} Qin, Yiran, et al. "Mp5: A multi-modal open-ended embodied system in minecraft via active perception." arXiv preprint arXiv:2312.07472 (2023).

\bibitem{du2024learning} Du, Yilun, et al. "Learning universal policies via text-guided video generation." Advances in Neural Information Processing Systems 36 (2024).

\bibitem{ajay2024compositional} Ajay, Anurag, et al. "Compositional foundation models for hierarchical planning." Advances in Neural Information Processing Systems 36 (2024).

\bibitem{liang2024skilldiffuser} Liang, Zhixuan, et al. "Skilldiffuser: Interpretable hierarchical planning via skill abstractions in diffusion-based task execution." Proceedings of the IEEE/CVF Conference on Computer Vision and Pattern Recognition. 2024.

\bibitem{heusel2017gans} Heusel, Martin, et al. "Gans trained by a two time-scale update rule converge to a local nash equilibrium." Advances in neural information processing systems 30 (2017).

\bibitem{zhang2018unreasonable} Zhang, Richard, et al. "The unreasonable effectiveness of deep features as a perceptual metric." Proceedings of the IEEE conference on computer vision and pattern recognition. 2018.

\bibitem{brown2020language} Brown, Tom B. "Language models are few-shot learners." arXiv preprint arXiv:2005.14165 (2020).

\bibitem{podell2023sdxl} Podell, Dustin, et al. "Sdxl: Improving latent diffusion models for high-resolution image synthesis." arXiv preprint arXiv:2307.01952 (2023).

\bibitem{brohan2022rt} Brohan, Anthony, et al. "Rt-1: Robotics transformer for real-world control at scale." arXiv preprint arXiv:2212.06817 (2022).

\bibitem{brohan2023rt} Brohan, Anthony, et al. "Rt-2: Vision-language-action models transfer web knowledge to robotic control." arXiv preprint arXiv:2307.15818 (2023).

\bibitem{li2024manipllm} Li, Xiaoqi, et al. "Manipllm: Embodied multimodal large language model for object-centric robotic manipulation." Proceedings of the IEEE/CVF Conference on Computer Vision and Pattern Recognition. 2024.

\bibitem{shah2023vint} Shah, Dhruv, et al. "ViNT: A foundation model for visual navigation." arXiv preprint arXiv:2306.14846 (2023).

\bibitem{liu2024visual} Liu, Haotian, et al. "Visual instruction tuning." Advances in neural information processing systems 36 (2024).

\bibitem{hu2021lora} Hu, Edward J., et al. "Lora: Low-rank adaptation of large language models." arXiv preprint arXiv:2106.09685 (2021).

\bibitem{qin2023supfusion} Qin, Yiran, et al. "SupFusion: Supervised LiDAR-camera fusion for 3D object detection." Proceedings of the IEEE/CVF International Conference on Computer Vision. 2023.

\bibitem{qin2024worldsimbench} Qin, Yiran, et al. "Worldsimbench: Towards video generation models as world simulators." arXiv preprint arXiv:2410.18072 (2024).

\bibitem{yu2025gamefactory} Yu, Jiwen, et al. "GameFactory: Creating New Games with Generative Interactive Videos." arXiv preprint arXiv:2501.08325 (2025).

\bibitem{zhou2024code} Zhou, Enshen, et al. "Code-as-Monitor: Constraint-aware Visual Programming for Reactive and Proactive Robotic Failure Detection." arXiv preprint arXiv:2412.04455 (2024).

\bibitem{zhang2024ad} Zhang, Zaibin, et al. "AD-H: Autonomous Driving with Hierarchical Agents." arXiv preprint arXiv:2406.03474 (2024).

\bibitem{wang2024toward} Wang, Chaoqun, et al. "Toward Accurate Camera-based 3D Object Detection via Cascade Depth Estimation and Calibration." arXiv preprint arXiv:2402.04883 (2024).

\bibitem{huang2024story3d} Huang, Yuzhou, et al. "Story3d-agent: Exploring 3d storytelling visualization with large language models." arXiv preprint arXiv:2408.11801 (2024).

\bibitem{savinov2018semi} Savinov, Nikolay, Alexey Dosovitskiy, and Vladlen Koltun. "Semi-parametric topological memory for navigation." arXiv preprint arXiv:1803.00653 (2018).

\bibitem{majumdar2022zson} Majumdar, Arjun, et al. "Zson: Zero-shot object-goal navigation using multimodal goal embeddings." Advances in Neural Information Processing Systems 35 (2022): 32340-32352.

\bibitem{yadav2023offline} Yadav, Karmesh, et al. "Offline visual representation learning for embodied navigation." Workshop on Reincarnating Reinforcement Learning at ICLR 2023. 2023.

\bibitem{yadav2023ovrl} Yadav, Karmesh, et al. "Ovrl-v2: A simple state-of-art baseline for imagenav and objectnav." arXiv preprint arXiv:2303.07798 (2023).

\bibitem{sun2024fgprompt} Sun, Xinyu, et al. "FGPrompt: fine-grained goal prompting for image-goal navigation." Advances in Neural Information Processing Systems 36 (2024).

\bibitem{zhu2017target} Zhu, Yuke, et al. "Target-driven visual navigation in indoor scenes using deep reinforcement learning." 2017 IEEE international conference on robotics and automation (ICRA). IEEE, 2017.

\bibitem{koh2024generating} Koh, Jing Yu, Daniel Fried, and Russ R. Salakhutdinov. "Generating images with multimodal language models." Advances in Neural Information Processing Systems 36 (2024).

\bibitem{krantz2022instance} Krantz, Jacob, et al. "Instance-specific image goal navigation: Training embodied agents to find object instances." arXiv preprint arXiv:2211.15876 (2022).

\bibitem{schulman2017proximal} Schulman, John, et al. "Proximal policy optimization algorithms." arXiv preprint arXiv:1707.06347 (2017).

\bibitem{anderson2018evaluation} Anderson, Peter, et al. "On evaluation of embodied navigation agents." arXiv preprint arXiv:1807.06757 (2018).

\bibitem{lin2024navcot} Lin, Bingqian, et al. "NavCoT: Boosting LLM-Based Vision-and-Language Navigation via Learning Disentangled Reasoning." arXiv preprint arXiv:2403.07376 (2024).

\bibitem{NavGPT} Zhou, Gengze, Yicong Hong, and Qi Wu. "Navgpt: Explicit reasoning in vision-and-language navigation with large language models." Proceedings of the AAAI Conference on Artificial Intelligence.

\bibitem{hahn2021no} Hahn, Meera, et al. "No rl, no simulation: Learning to navigate without navigating." Advances in Neural Information Processing Systems 34 (2021): 26661-26673.

\bibitem{li2025t2isafety} Li, Lijun, et al. "T2ISafety: Benchmark for Assessing Fairness, Toxicity, and Privacy in Image Generation." arXiv preprint arXiv:2501.12612 (2025).

\bibitem{an2024agfsync} An, Jingkun, et al. "AGFSync: Leveraging AI-Generated Feedback for Preference Optimization in Text-to-Image Generation." arXiv preprint arXiv:2403.13352 (2024).


\end{thebibliography}
\end{sloppypar}

\clearpage
\beginsupplement
\section*{Appendix}
\renewcommand{\thesubsection}{S\arabic{subsection}}

\subsection{\label{chap:S1}PanNuke and MoNuSAC preprocessing}
The PanNuke dataset comprises a set of 7,901 RGB patches, each with dimensions of $256 \times 256$ pixels, which we set as the standard patch size for our analysis. In contrast, the MoNuSAC dataset encompasses 294 images of heterogeneous dimensions. To standardize the MoNuSAC images with our experiments, we implement a standardization protocol. Specifically, for images exceeding the dimensions of $256 \times 256$ pixels, we segment them into equal-sized patches and apply mirror padding to the remaining portions to avoid information loss at the peripherals. Patches with dimensions less than $128 \times 128$ pixels are excluded from the dataset due to the insufficient resolution to capture relevant cellular details. For patches where either dimension falls between 128 and 256 pixels, we employ upsampling to achieve the standard patch size. As a result, we obtain a total of 2,823 RGB patches derived from the MoNuSAC dataset for subsequent analysis. For additional details on the MoNuSAC data preparation process, refer to the source code \cite{Shvetsov_2025a}.
\clearpage

\subsection{\label{chap:S2}Data usage for the methodology}

\counterwithin{figure}{subsection}
\renewcommand{\thefigure}{S\arabic{subsection}}

\begin{figure}[h!]
    \centering
    \includegraphics[width=\textwidth, height=0.85\textheight, keepaspectratio]{images/A2.pdf}
    \caption{Overview of the methodology for cross-labeling, dataset refinement, and model comparison. (1) Cross-relabeling - training and testing cell classification models, (2) Cross-relabeling - using cell classification models to create refined dataset, (3) Fine-tuning and training models for comparison, (4) Student knowledge distillation with refined dataset}
    \label{fig:S2}
\end{figure}
\clearpage

\subsection{\label{chap:S3}Confusion matrices for classification models}
\counterwithin{figure}{subsection}
\renewcommand{\thefigure}{S\arabic{subsection}.\arabic{figure}}

\begin{figure}[h!]
    \centering
    \includegraphics[width=\textwidth, height=0.4\textheight, keepaspectratio]{images/A3_1.pdf}
    \caption{Confusion matrix for PanNuke trained model}
    \label{fig:S3.1}
\end{figure}

\begin{figure}[h!]
    \centering
    \includegraphics[width=\textwidth, height=0.4\textheight, keepaspectratio]{images/A3_2.pdf}
    \caption{Confusion matrix for MoNuSAC trained model}
    \label{fig:S3.2}
\end{figure}

\clearpage

\subsection{\label{chap:S4}Datasets cell counts}

\counterwithin{table}{subsection}
\renewcommand{\thetable}{S\arabic{subsection}}

\begin{table}[h!]
\renewcommand{\arraystretch}{2.0}
\centering
\caption{\label{tab:S4}Cell counts for PanNuke, MoNuSAC and refined datasets. Numbers in parentheses indicate preprocessed cell counts for cell classifier models training and testing.}
%\adjustbox{max width=\textwidth}{%
\begin{tabular}{|l|c|c|c|}
\hline
%\rowcolor{gray!30}
Cell type & PanNuke & MoNuSAC & Refined \\
\hline
Neoplastic & 77,403 (68,031) & - & 105,451 \\
\hline
Epithelial & 26,572 (23,207) & - & 29,926 \\
\hline
Epithelial (benign and malignant) & - & 31,402 & - \\
\hline
Inflammatory & 32,276 & - & - \\
\hline
Lymphocytes & - & 37,045 (33,104) & 65,275 \\
\hline
Neutrophils & - & 1,355 (1,252) & 3,833 \\
\hline
Macrophage & - & 1,842 (1,695) & 3,410 \\
\hline
Dead & 2,908 & - & 2,908 \\
\hline
Connective & 50,585 & - & 50,585 \\
\hline
\end{tabular}
%
%}
\end{table}



\clearpage

\subsection{\label{chap:S5}Definition of validation metrics}
\counterwithin{equation}{subsection}
\renewcommand{\theequation}{\arabic{equation}}

\subsubsection{\label{chap:S5.1}R\textsuperscript{2}}
The coefficient of determination, denoted as $R^2$, is a statistical measure that represents the proportion of variance in the dependent variable that is predictable from the independent variables. In the context of cell quantification in pathology, $R^2$ is used to assess how well the predicted quantities of different cell types in a patch align with the actual quantities observed in the ground truth data, with higher values representing more accurate quantification. $R^2$ is defined as
\begin{equation*}
R^2 = 1 - \frac{\sum_{i=1}^n (y_i - \hat{y}_i)^2}{\sum_{i=1}^n (y_i - \bar{y})^2},
\end{equation*}
where $y_i$ represents the actual number of cells of a specific type in the $i$-th image, $\hat{y}_i$ represents the predicted number of cells of that type in the $i$-th image, $\bar{y}$ is the mean of the actual numbers across all images, and $n$ is the total number of images in the dataset.

The $R^2$ metric has a range of $(-\infty, 1]$. An $R^2$ of 1 indicates perfect prediction, where all predicted values exactly match the actual values. An $R^2$ of 0 suggests that the model explains none of the variability of the response data around its mean. If $R^2$ is negative, it indicates that the model performs worse than a model that simply predicts the mean of the actual values for all observations.

\subsubsection{\label{chap:S5.2}PQ}
Panoptic Quality ($PQ$) is a comprehensive metric used to evaluate the performance of segmentation models in tasks that require both instance segmentation and classification. $PQ$ provides a single score that encapsulates both the detection accuracy (i.e., how many objects were correctly identified) and the segmentation quality (i.e., how accurately the objects' boundaries were delineated). This metric is particularly useful in multiclass scenarios where each pixel is classified into distinct categories, such as different cell types in pathology images.

$PQ$ is calculated as the product of two terms: Detection Quality ($DQ$) and Segmentation Quality ($SQ$). It can be expressed as
\begin{equation*}
PQ = DQ \cdot SQ,
\end{equation*}
where
\begin{equation*}
DQ = \frac{TP}{TP + 0.5\, FP + 0.5\, FN},
\end{equation*}
\begin{equation*}
SQ = \frac{\sum_{(p, g) \in \mathcal{M}} IoU(p, g)}{TP}.
\end{equation*}
In these formulas, $TP$ denotes the number of correctly matched instances between ground truth and prediction, $FP$ denotes the predicted instances that have no corresponding ground truth, $FN$ denotes the ground truth instances that were not detected, $IoU(p, g)$ is the Intersection over Union for a pair of matched instances $p$ (prediction) and $g$ (ground truth), and $\mathcal{M}$ is the set of matched pairs.

The $PQ$ metric is calculated for each class and is averaged across classes to provide a global performance measure.

The $PQ$ score has a range of $[0, 1.0]$, where a higher score indicates better performance in both detecting and segmenting the instances correctly. A $PQ$ of 1 signifies perfect identification and segmentation of all instances, whereas a $PQ$ of 0 indicates that no instances were correctly identified and segmented.

\clearpage

\subsection{\label{chap:S6}Segmentation and Detection quality metrics for teacher and student models}

\begin{table}[h!]
\renewcommand{\arraystretch}{2.0}
\centering
\caption{Segmentation and detection quality for student and teacher models (CI 95\%)}
\label{tab:S6}
%\adjustbox{max width=\textwidth}{%
\begin{tabular}{|l|c|c|}
\hline
%\rowcolor{gray!30}
Metric & Teacher & Student \\
\hline
$SQ_{neoplastic}$ & 0.819 (0.815--0.823) & 0.824 (0.819--0.828) \\
\hline
$SQ_{lymphocyte}$ & 0.795 (0.788--0.802) & 0.790 (0.783--0.796) \\
\hline
$SQ_{connective}$ & 0.770 (0.762--0.776) & 0.780 (0.772--0.786) \\
\hline
$SQ_{dead}$ & 0.659 (0.623--0.688) & 0.657 (0.624--0.695) \\
\hline
$SQ_{epithelial}$ & 0.780 (0.770--0.790) & 0.788 (0.779--0.797) \\
\hline
$SQ_{macrophage}$ & 0.788 (0.760--0.810) & 0.757 (0.730--0.783) \\
\hline
$SQ_{neutrofil}$ & 0.782 (0.761--0.801) & 0.775 (0.759--0.792) \\
\hline
$DQ_{neoplastic}$ & 0.706 (0.692--0.719) & 0.727 (0.712--0.741) \\
\hline
$DQ_{lymphocyte}$ & 0.675 (0.656--0.698) & 0.713 (0.691--0.734) \\
\hline
$DQ_{connective}$ & 0.566 (0.546--0.584) & 0.583 (0.565--0.602) \\
\hline
$DQ_{dead}$ & 0.410 (0.361--0.465) & 0.435 (0.306--0.561) \\
\hline
$DQ_{epithelial}$ & 0.668 (0.639--0.694) & 0.673 (0.644--0.702) \\
\hline
$DQ_{macrophage}$ & 0.657 (0.583--0.727) & 0.615 (0.531--0.703) \\
\hline
$DQ_{neutrofil}$ & 0.691 (0.625--0.753) & 0.729 (0.679--0.778) \\
\hline
\end{tabular}
%
%}
\end{table}

\clearpage

\subsection{\label{chap:S7}QuPath integration method}
We adopt an integration strategy leveraging the paquo \cite{Bayer_AG} library, a Python package that enables direct interaction with QuPath’s internal API, thereby facilitating seamless data exchange without intermediate conversion steps. The data processing pipeline (\hyperref[fig:S7]{Appendix Figure S7}) begins with the acquisition of WSIs and their associated annotations from QuPath, which are represented as Shapely \cite{Gillies_Wel_etal._2024} polygons. Utilizing paquo, we directly read, create, and modify these annotations and detections within a QuPath project in the Python environment. Images are then cropped using these polygons and processed by cell segmentation and classification models employing standard vision processing toolkits such as OpenCV, pyvips, and PyTorch. Additionally, QuPath employs Groovy scripts to initiate a Python process that starts the entire pipeline from QuPath graphical interface: fetching polygons, extracting images from them, and running deep learning model inference on the cropped images. 
The results are returned to QuPath, leveraging paquo's Python bindings to manipulate QuPath data while minimizing the computational overhead typically associated with cross-environment communication.

\counterwithin{figure}{subsection}
\renewcommand{\thefigure}{S\arabic{subsection}}

\begin{figure}[h!]
    \centering
    \includegraphics[width=\textwidth]{images/A7.pdf}
    \caption{QuPath integration workflow using Python environment}
    \label{fig:S7}
\end{figure}

Compared to traditional workflows that involve exporting annotations as GeoJSON, classifying them in Python, and reimporting them into QuPath, our approach offers several advantages. We eliminate the need to switch between programming languages, providing a cohesive and streamlined development process entirely within QuPath software and removing the necessity to use other tools. Meanwhile, we avoid storing annotations as intermediate JSON files unless required for external use or archiving. By conducting the entire inference and post-processing workflow within the Python environment, we leverage the power and flexibility of Python libraries for image processing and machine learning. This approach also enables adjustments to any set of labels and models, thereby improving its applicability.

%\hfill

The distilled model and QuPath integration code are packaged into a Docker container, enabling streamlined execution with the Docker engine. Detailed integration code and deployment instructions can be found in the GitHub repository \cite{Shvetsov_2025b}.

Despite these benefits, we acknowledge that the paquo library is a proof‑of‑concept project in its early development stage and has not been tested across all versions of QuPath.

\clearpage

\subsection{\label{chap:S8}Data and code availability statement}
All datasets, models, and code used in this study are publicly available and can be obtained from the repositories listed below. 
The PanNuke \cite{Gamper_Koohbanani_etal._2019} and MoNuSAC \cite{Verma_Kumar_etal._2021} datasets are publicly accessible, and download information along with detailed descriptions can be found in their respective articles. Preprocessing scripts for PanNuke and MoNuSAC data, as well as individual cell extraction scripts, are available on GitHub \cite{Shvetsov_2025a}. The H-Optimus foundation model used in our experiments can be downloaded from the HuggingFace repository \cite{hoptimus2024}, and model information is available on GitHub \cite{Saillard_Jenatton_etal._2024}. In addition, the integration code for QuPath and the distilled model packaged in a Docker container are provided in the repository \cite{Shvetsov_2025b}, and paquo Python library is available from the authors GitHub repository \cite{Bayer_AG}.
\clearpage

\end{document}
 



% \section*{Author contributions statement}
% H.W.F. conducted the experiment(s), H.W.F. and S.C. and S.C. conceived the ideas and analysed the results.  All authors reviewed the manuscript. 

% \section*{Additional information}
% Authors declare no conflict of interest.

\end{document}