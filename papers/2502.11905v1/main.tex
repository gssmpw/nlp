\documentclass[10pt]{wlscirep} %\documentclass[fleqn,10pt]{wlscirep}
\usepackage[utf8]{inputenc}
\usepackage[T1]{fontenc}
% ========================================================================
\usepackage{braket}
\usepackage{subcaption}
\usepackage{float}
\usepackage{dblfloatfix}
\usepackage{placeins}
\usepackage{adjustbox}
\usepackage{tabularx}
\usepackage{seqsplit}
\usepackage{multirow}
\usepackage{array}
\usepackage{soul}
\renewcommand{\arraystretch}{1.75}
\usepackage{colortbl}
\usepackage{pgfplots}
\usepackage{pgfplotstable}
\usepackage{pgf} % for calculating the values for gradient
% \usepackage[table]{xcolor}
\usepackage{etoolbox}
\definecolor{high}{HTML}{08ef15}  % the color for the highest number in your data set
\definecolor{low}{HTML}{2f5652}  % the color for the lowest number in your data set
\newcommand*{\opacity}{50}% here you can change the opacity of the background color!
%======================================
% Data set related!
\newcommand*{\minval}{0.0}% define the minimum value on your data set
\newcommand*{\maxval}{2.5}% define the maximum value in your data set!
%======================================
% gradient function!
\newcommand{\gradient}[1]{
    \small
    % The values are calculated linearly between \minval and \maxval
    \ifdimcomp{#1pt}{>}{\maxval pt}{#1}{
        \ifdimcomp{#1pt}{<}{\minval pt}{#1}{
            \pgfmathparse{int(round(100*(#1/(\maxval-\minval))-(\minval*(100/(\maxval-\minval)))))}
            \xdef\tempa{\pgfmathresult}
            \cellcolor{high!\tempa!low!\opacity} #1
    }}
}

%======================================


\usepackage{bm}
\makeatletter
\AtBeginDocument{\DeclareMathVersion{bold}
\SetSymbolFont{operators}{bold}{T1}{times}{b}{n}
% \SetSymbolFont{NewLetters}{bold}{T1}{times}{b}{it}
\SetMathAlphabet{\mathrm}{bold}{T1}{times}{b}{n}
\SetMathAlphabet{\mathit}{bold}{T1}{times}{b}{it}
\SetMathAlphabet{\mathbf}{bold}{T1}{times}{b}{n}
\SetMathAlphabet{\mathtt}{bold}{OT1}{pcr}{b}{n}
\SetSymbolFont{symbols}{bold}{OMS}{cmsy}{b}{n}
\renewcommand\boldmath{\@nomath\boldmath\mathversion{bold}}}
\makeatother

\def\BibTeX{{\rm B\kern-.05em{\sc i\kern-.025em b}\kern-.08em
    T\kern-.1667em\lower.7ex\hbox{E}\kern-.125emX}}
\pgfplotsset{compat=1.18}
% ========================================================================


\title{Exploring Quantum Control Landscape and Solution Space Complexity through Optimization Algorithms \& Dimensionality Reduction}

\author[1,3,*]{Haftu W. Fentaw}
\author[2,3]{Steve Campbell}
\author[1,3]{Simon Caton}
\affil[1]{School of Computer Science, University College Dublin, Dublin, Ireland}
\affil[2]{School of Physics, University College Dublin, Dublin, Ireland}
\affil[3]{Centre for Quantum Engineering, Science, and Technology, University College Dublin, Dublin, Ireland}
\affil[*]{haftu.fentaw@ucdconnect.ie}


\keywords{
Genetic Algorithms (GA), Principal Component Analysis (PCA), Quantum Control Landscape (QL), Reinforcement Learning (RL), Stochastic Gradient Descent (SGD)}
%\keywords{Keyword1, Keyword2, Keyword3}

\begin{abstract}
Understanding the quantum control landscape (QCL) is important for designing effective quantum control strategies. In this study, we analyze the QCL for a single two-level quantum system (qubit) using various control strategies. We employ Principal Component Analysis (PCA), to visualize and analyze the QCL for higher dimensional control parameters. Our results indicate that dimensionality reduction techniques such as PCA, can play an important role in understanding the complex nature of quantum control in higher dimensions. Evaluations of traditional control techniques and machine learning algorithms reveal that Genetic Algorithms (GA) outperform Stochastic Gradient Descent (SGD), while Q-learning (QL) shows great promise compared to Deep Q-Networks (DQN) and Proximal Policy Optimization (PPO). Additionally, our experiments highlight the importance of reward function design in DQN and PPO demonstrating that using immediate reward results in improved performance rather than delayed rewards for systems with short time steps. A study of solution space complexity was conducted by using Cluster Density Index (CDI) as a key metric for analyzing the density of optimal solutions in the landscape. The CDI reflects cluster quality and helps determine whether a given algorithm generates regions of high fidelity or not. Our results provide insights into effective quantum control strategies, emphasizing the significance of parameter selection and algorithm optimization.
\end{abstract}
\begin{document}

\flushbottom
\maketitle

\thispagestyle{empty}

\section{Introduction}

\label{sec:introduction}
As the demand for advanced computational capabilities continues to rise, there is a growing need for alternatives to solve complex problems that are either beyond the reach of classical computers or could use a boost with improved computational capability. Quantum computing has emerged as a viable alternative with the potential to tackle certain classes of problems currently not amenable with today's classical algorithms, such as factoring large numbers (Shor's algorithm) \cite{Shor_1997}, or speeding up certain algorithms as in quantum machine learning \cite{Biamonte_2017}. However, the inherent fragility of quantum states means that to harness this potential, effective quantum control strategies are essential for manipulating quantum states. In order to design an effective control strategy it is therefore important to understand the quantum control landscape (QCL). 

The QCL refers to the multidimensional space of control variables and represents the relationship between the control parameters of a quantum system and the associated performance measures, such as the distribution of candidate optimal solutions in any local neighborhood or the trajectories through control space to the optimal solution~\cite{C4CP03853C, 9502048}. The study of QCLs is critical for understanding the controllability, and corresponding optimality, that is achievable by manipulating the quantum systems using, e.g., external control fields. Exploration of this landscape helps in identifying the optimal path a quantum system takes during a transition to a target state \cite{Chakrabarti_2007}. By varying the properties of the control pulses (i.e. applying variable external fields), the state of the system can be driven to a desired state from its initial configuration~\cite{PhysRevA-84-012109}. 

For a closed system, the process during which this change in the state occurs is governed by the time-dependent Schr\"odinger equation
\begin{equation}
    \label{eq:SE}
    i\hbar\frac{\partial}{\partial t}\ket{\psi(t)} = H(t)\ket{\psi(t)}
\end{equation}
where $H(t)$ is the Hamiltonian of the system which describes the total energy and $\ket{\psi(t)}$ is the time evolved state of the system. In general, solving~\eqref{eq:SE} is a difficult problem for an arbitrary time-dependence. One approach is to determine $\ket{\psi(t)}$ using a discrete time analysis by assuming the overall evolution time, $T$, is divided into $N$ equal partitions of size $\Delta t=T/N$ such that the state at time $t+\Delta t$ is given by
\begin{equation}
\ket{\psi(t+\Delta t)} = e^{-\frac{i}{\hbar}H(t)\Delta t} \ket{\psi(t)}
\label{timeEvolution}
\end{equation}

In quantum control problems it is common to express the time-dependence in the Hamiltonian, $H(t)$ as
\begin{equation}
    H(t) = H_d + u(t)H_c  
    \label{Hamiltonian}
\end{equation}
where $H_d$ is the drift Hamiltonian and $u(t) H_c$ is the control Hamiltonian. The objective of quantum control is, therefore, to find a $u(t)$ that will maximize the probability of achieving some target objective, e.g. evolving to a given final state~\cite{Larocca_2018, Koch2022-bo}. The cost function for the control is then given by the quantum state fidelity
\begin{equation}
    F = |\braket{\psi(T)|\Psi}|^2
    \label{fidelity}
\end{equation}
where $\ket{\psi(T)}$ is the state at the end of the evolution and $\ket{\Psi}$ is the desired target state.

The QCL allows us to analyze the complexity of the solution space by plotting the combination of discrete field values that will result in high target state fidelity values. While for simple piece-wise constant pulses consisting of only two segments, i.e. $N=2$, %in equation \eqref{timeEvolution},
 the full QCL can be plotted, in order to visually inspect the complexity of the solution space for higher dimensions, i.e. $N>2$, we need to resort to the use of \textit{dimensionality reduction} techniques.

Finding the right combination of control parameters which will result in achieving a desired target is a complex task that requires the use of different optimization algorithms. There is a rich body of work exploring different approaches, including notable examples such as Gradient Ascent Pulse Engineering (GRAPE)~\cite{KHANEJA2005296}, stochastic gradient descent (SGD)-based works such as~\cite{Ferrie190404}, and reinforcement learning (RL)-based approaches as demonstrated in \cite{Zhang_2019}.  

In this work, the performance and suitability of traditional optimization algorithms, such as SGD, genetic algorithms (GA), and RL algorithms, for exploring the QCL are presented. The high level description of the state transfer problem and steps followed in our work is presented in Fig.~\ref{fig:highleveldiagram}. We focus on the control of a single two-level quantum system (qubit) initially in state $\ket{0}$ and aim to perform a bit flip operation to state $\ket{1}$ as seen in Fig.~\ref{fig:highleveldiagram}(a) using a control pulse. The qubit's evolution is governed by~\eqref{timeEvolution} and \eqref{Hamiltonian}, leading to a piece-wise constant control signal, as illustrated in Fig.~\ref{fig:highleveldiagram}(b). The QCL for different input combinations is generated using a brute-force approach (for later comparison), the results for each parameter are passed through a 2 component PCA to obtain a 2D landscape and the PCA loadings (which will be used in later stages of the analysis). Details of this are presented in section~\ref{sec:dimensionalityReduction}. Following this, Fig.~\ref{fig:highleveldiagram}(c), shows the experiments we perform using different optimization algorithms to generate the control pulses (we refer to section~\ref{sec:experimentalsetup}). After the control pulses are generated,  they are passed through PCA (using PCA loadings from previous step) to obtain the 2D landscape. Once the 2D landscapes are plotted, we perform a number of analyses to better understand the complexity of the landscape, and find the algorithms which can explore the landscape well. Details of this can be found in sections~\ref{sec:results} and~\ref{sec:solutionspacecomplexity}.

\begin{figure}[htbp]
    \centering
        \includegraphics[width=0.9\textwidth]{high_level_diagram/High_level_diagram.png} 
    \caption{High level diagram: The basic steps involved in this work: (a) the state transfer problem, (b) the brute-force solution (by converting the continuous control signal into discrete values ), (c) optimization algorithms based solution}
    \label{fig:highleveldiagram}
\end{figure}


The main contributions of this work are:
\begin{itemize}
    \item Investigation of how well different algorithms explore the solution space / landscape with the view to characterize the potential suitability of different techniques. 
    \item Introduced Dimensionality Reduction (DR) techniques to visualize and study higher dimensional quantum control landscapes
    \item Designed proper RL networks and reward signals specifically for short time step (small steps per episode) quantum control problems
    \item Analysis of the solution space complexity - to better understand how easy or difficult it is to find high fidelity solutions in certain regions of the landscape and which algorithms are best suited for this. 
\end{itemize}

\section{Prior Related work}
\label{sec:relatedwork}
The study of QCLs is well established at this point and one of the key insights we can extract from QCL analysis is whether the landscape is trap-free or it contains local minima that can hinder optimization~\cite{PhysRevA-84-012109}. It therefore follows that understanding the structure of the landscape can help avoid these traps and improve the efficiency of control strategies~\cite{PhysRevA-84-012109}. When no constraints are imposed on the available controls for the system, the QCL is devoid of local minima and is trap-free~\cite{Russell_2017}. However, in many real-world scenarios, some constraints are unavoidable. As shown, for example, in~\cite{Larocca_2020_PhysRevA.101.023410}, the QCL is not trap-free in most practical situations where constraints are imposed on the system, making it necessary to use more sophisticated optimization techniques to escape traps. 

In Ref.~\cite{Larocca_2018}, the authors conducted a detailed analysis of the QCL for a single qubit bit-flip using a two-parameter control system for various evolution times, both below and above the so-called quantum speed limit~\cite{Deffner2017}. The quantum speed limit, $T_{min}$ sets a fundamental lower bound on the time necessary to connect two quantum states and for a single qubit its value can be analytically determined as in~\cite{Hegerfeldt_2013, Poggi2013}. Their findings indicate that for $T/T_{min}<1$, the landscape is relatively simple, featuring only a single global maximum but importantly with a fidelity, $F<1$. As the total evolution time is increased well beyond the quantum speed limit, the landscape becomes more complex and there are a number of local maxima which are able to achieve perfect state transfer. We extend their work to explore details of the landscape in higher dimensions, and how the possible ridges and valleys in higher dimensions can influence the solution space.

It is worth highlighting that the QCL can also reveal important physical insights into the underlying quantum dynamics. For example, the presence of broad plateaus or ridges in the landscape may suggest that a wide range of control parameters can achieve near-optimal performance, highlighting the robustness of certain quantum systems to noise and perturbations\cite{PhysRevA.86.052117, PhysRevA-84-012109}. On the contrary, if a quantum system has steep landscapes, this suggests that the system is highly sensitive and it requires more precise control strategies to find the optimal control pulses. 

In order to study the details of the QCL at higher dimensions, a technique to reduce the number of dimensions or features of a dataset while retaining as much information as possible is necessary. Dimensionality reduction algorithms such as Principal Component Analysis (PCA)~\cite{MACKIEWICZ1993303}, t-distributed Stochastic Neighbor Embedding (t-SNE)~\cite{JMLR:v9:vandermaaten08a} and Uniform Manifold Approximation and Projection for Dimension Reduction~(UMAP)~\cite{mcinnes2020umapuniformmanifoldapproximation} are among the most commonly used techniques. The decision of choosing which algorithm to use depends on several factors including how sensitive the algorithm is to small changes in its parameters and how well the algorithm preserves the structure of the data after applying it. There are studies into these details (e.g. \cite{Huang2022_comprehensive_DR}) that can assist in this decision process.  For example, authors in Ref.~\cite{berger2024dimensionalityreductionclosedloopquantum} applied dimensionality reduction in quantum gate calibration, PCA is used in Ref.~\cite{li2018visualizinglosslandscapeneural} to visualize the loss landscape in neural networks, and in Ref.~\cite{fan2024manifoldconnectednessquantumcontrol} PCA is used to project quantum control trajectories into three dimensions. 
%In our study, we experimented with t-SNE, UMAP and PCA. 
Both t-SNE and UMAP require tuning parameters that significantly influence the results, often leading to inconsistent outcomes that can be difficult to interpret~\cite{10.5555/3546258.3546459}. For instance, t-SNE relies on parameters such as perplexity, learning rate, and the number of iterations, while UMAP depends on the number of neighbors and minimum distance between points. In contrast, PCA requires only the number of principal components to be specified, making it simpler to use and resulting in more stable and interpretable outcomes. For these reasons (and also based on initial experimentation with each of these methods), we will focus on PCA to visualise the QCL. 
%Nevertheless, as with any dimensionality reduction technique, PCA entails some degree of information loss, meaning the lower dimensional landscapes  may not capture all fine details. The QCLs in this work are based on the PCA-transformed topology.
%In this work, we chose PCA as our dimensionality reduction technique since PCA is less sensitive to parameter choices unlike  tSNE or UMAP~\cite{10.5555/3546258.3546459} as PCA does not require multiple input parameters, making the results easier to interpret and less sensitive to input variations. 

The QCL is typically studied using various optimization techniques to find the best control parameters that guide the system. Traditional control algorithms including Gradient-based methods, Genetic Algorithms (GA) and Bang-Bang Control are widely used for navigating the QCL and optimizing control parameters. While the authors in Ref.~\cite{Ferrie190404} introduced a central difference based gradient approach (self guided quantum tomography) and argued this iterative approach is more efficient than other methods, those in Ref.~\cite{Brown_2023} utilized genetic algorithms for optimal quantum state control and claimed they observed fast preparation times, and resilience to noise when using genetic algorithms. The drawbacks of gradient-based algorithms is that they can get trapped in local minima, especially in systems with complex landscapes, and their sensitivity to initial conditions~\cite{Netrapalli2019-gu}. While GAs can avoid local minima more effectively than gradient-based methods, they may still suffer from inefficiencies in high-dimensional search spaces~\cite{10.1371/journal.pone.0303088}. Authors in Ref.~\cite{Barnes2015-zj} introduced an analytical robust quantum control approach that not only yields explicit constraints on the control field but also ensure that the leading-order noise-induced errors in a qubit’s evolution cancel exactly.


In recent years, Reinforcement Learning (RL) based approaches have emerged as alternatives for quantum optimal control. The major advantage of RL is its ability to explore vast control landscapes autonomously, finding creative control solutions that may be difficult for traditional methods to uncover. For instance, authors in Ref.~\cite{s41534-019-0141-3} used trusted-region-policy-optimization (TRPO) RL algorithms to train agents for the control of two-qubit unitary gates by adding control noise into training environments and they found the agent demonstrates a two-order-of-magnitude reduction in average-gate-error compared to gradient based methods. This shows that RL algorithms are robust to noise and model uncertainties because they can adapt to noisy environments by learning policies that are more resilient to fluctuations in control parameters, making them practical for real-world quantum systems where noise and environmental factors affect control fidelity. RL algorithms can also handle high-dimensional control spaces and complex quantum dynamics by learning from interactions with the environment, thus reducing the need for precise system modeling. In Ref.~\cite{PhysRevX.12.011059}, the authors trained a model-free RL agent which learns through trial-and-error interaction with the quantum system and hence there is no need for the precise modeling of the physical system. They concluded that this is of immediate relevance allowing the adaptation of quantum control policies to the specific system in which they are deployed and complete elimination of model bias. 

Different RL based methods were compared with traditional optimization techniques in Ref.~\cite{Zhang_2019}, where the authors demonstrated that RL algorithms (TQL~\cite{sutton1999reinforcement}, DQN~\cite{mnih2015human} and PG~\cite{NIPS1999_464d828b}) can outperform traditional methods (SGD~\cite{Ferrie190404} and Krotov~\cite{krotov1996global}) when the problem size is scaled up. The authors showed that for a relatively small number of iterations (up to 500) SGD would fail to reach optimal solutions compared with ML techniques, however it is worth noting that allowing SGD more iterations can result in improved performance as detailed in section \ref{sec:experimentalsetup}.

Despite RL-based algorithms showing better performance, there are drawbacks to using RL in quantum control. RL methods often require a large number of training episodes, particularly for complex quantum systems. Additionally, RL systems are sensitive to hyper-parameters, such as the learning rate and exploration-exploitation trade-off, which must be carefully tuned for optimal performance~\cite{sutton1999reinforcement}. In this work, an effort is made to answer the question of when one should consider using RL techniques, which RL methods are beneficial, and which are excessive by visualizing the landscape and studying the solution space complexity.


Indeed, the fact that there are different options for quantum control notwithstanding, there has been limited research into visualizing and understanding the QCL in higher-dimensional parameter spaces. Therefore, this work contributes by providing: (a) a framework for the exploration and visualization of QCL in such complex spaces and (b) algorithms that we can utilize to get optimal control pulses, offering new insights and tools for optimizing quantum control.

\section{Visualizing Higher Dimensional Landscapes Using Principal Component Analysis (PCA)}
\label{sec:dimensionalityReduction}
PCA is a statistical technique for dimensionality reduction that transforms a higher dimensional data set into a lower dimensional representation. It seeks to maximise the amount of information retained while reducing the dimensionality of the data. By projecting data onto a new set of orthogonal axes, called principal components, PCA identifies the directions of maximum variance in the data. The first principal component accounts for the largest possible variance, with each subsequent component capturing the next highest variance under the constraint of being orthogonal to the preceding components. This process helps in extracting key features and making it easier to visualize and analyze high-dimensional datasets.

% Results of using t-SNE and UMAP is presented in the Appendix section. 

By applying PCA on 3-parameter or 4-parameter quantum landscapes, we can create 2-dimensional visualisations of the landscapes which can allow us to gain insight into the complexities and characteristics of the QCL for the Hamiltonian. To make things concrete, in this work we focus on the QCL generated for $N$-parameter pulses, i.e. the continuous control field $u(t)$ over total evolution time $T$ is divided into $N$ equal partitions (time steps), each partition with its corresponding amplitude, resulting in a piece-wise constant function for $u(t)$. 
\begin{figure*}[htbp]
    \centering
    \begin{subfigure}[b]{0.295\textwidth}
        \centering
        \includegraphics[width=\textwidth]{landscape_images/2_param_brute_force_raw_data_20241003_2.png} 
        \caption{}
        \label{fig:2paramraw}
    \end{subfigure}
    \begin{subfigure}[b]{0.295\textwidth}
        \centering
        \includegraphics[width=\textwidth]{landscape_images/3_param_brute_force_raw_data_20241003_2.png} 
        \caption{}
        \label{fig:3paramraw}
    \end{subfigure}
    \begin{subfigure}[b]{0.295\textwidth}
        \centering
        \includegraphics[width=\textwidth]{landscape_images/3_param_brute_force_PCA_20241003.png} 
        \caption{}
        \label{fig:3paramPCAraw}
    \end{subfigure}
    \caption{Quantum landscape using brute-force data: (a) 2 parameter~\cite{Larocca_2018},  (b) 3 parameter, and (c) 3 parameter after applying PCA. Data points were created by generating every possible combination of values in the range [-1, 1] for each axis, with each axis divided into 100 intervals. Axis labels a1, a2 and a3 represent parameters 1, 2 and 3 in each axis, pc1 and pc2 represent the first two principal components.}
    \label{fig:bruteforce}
\end{figure*}
We consider the evolution of a single qubit with $T/T_{min}=2$, initial state of $\ket{0}$ and target state of $\ket{1}$ where $\ket{0}$ and $\ket{1}$ are the eigenstates of $\sigma_z$ and assume the system is governed by the Hamiltonian      
\begin{equation}
\label{eq:LZ}
    H(t) = \frac{\sigma_{x}}{2} + 2u(t)\sigma_{z}
\end{equation} 
The Hamiltonian in Eq.~\eqref{eq:LZ} is closely related to the celebrated Landau-Zener model which captures a remarkably diverse range of physical settings, as discussed, for example in Ref.~\cite{IVAKHNENKO20231} and we refer to Refs.~\cite{Larocca_2018, Hegerfeldt_2013} where similar control techniques as will be considered below were employed.

The two parameter and three parameter quantum landscapes resulting from using brute-force combination of 100 values for each parameter in the range [-1, 1] are shown in Fig.~\ref{fig:2paramraw} and Fig.~\ref{fig:3paramraw}. As can be seen from Fig.~\ref{fig:3paramraw}, details of the landscape for the 3 parameter case are hard to see. (Note: throughout this paper, the axis labels a1, a2, and a3 represent parameter 1, parameter 2, and parameter 3, respectively, while pc1 and pc2 denote principal component 1 and principal component 2). In order to see underlying properties of the landscape for higher dimensions, we apply PCA with two principal components on the raw brute-force data points which results in the  landscapes shown in Fig.~\ref{fig:3paramPCAraw}. The landscapes in Fig.~\ref{fig:bruteforce} show both the high fidelity (red) and low fidelity (blue) regions. As the target is to achieve a high-fidelity state transfer, we can gain more insight into the space of effective control protocols by focusing on only regions with high target state fidelity, i.e. $F>0.95$, shown in Fig.~\ref{fig:afterapplyingPCAhighfid}.

 As it is evident from the PCA representation of the landscapes in Fig.~\ref{fig:afterapplyingPCAhighfid}, higher dimensional landscapes are filled with more high fidelity regions as compared to the low dimensional landscapes. From this observation, it is clear that by dividing the total evolution time $T$ in to more partitions (hence high dimensional landscape), we increase the chances of finding more control signals that will result in high fidelity (complete population transfer).  \par 
 
Clearly, we cannot simply try the brute-force combination of inputs for more than a small number of parameters and hope we can land in a combination which will result in high fidelity. Instead, in order to generate a control pulse with reasonably good fidelity, we use optimization algorithms such as SGD, GA, QL, DQN or PPO to find a control signal which will drive our system into its target state effectively.  Details of these algorithms and the experimental setups for each experiment are discussed next.

\begin{figure}[htbp]
    \centering
    \begin{subfigure}[b]{0.275\textwidth}
        \centering
        \includegraphics[width=\textwidth]{landscape_images/2_param_brute-force_95p_PCA_20241003.png} 
        \caption{}
        \label{fig:2paramPCAhighfid}
    \end{subfigure}
    \begin{subfigure}[b]{0.275\textwidth}
        \centering
        \includegraphics[width=\textwidth]{landscape_images/3_param_brute-force_95p_PCA_20241003.png} 
        \caption{}
        \label{fig:3paramPCAhighfid}
    \end{subfigure}
    \begin{subfigure}[b]{0.275\textwidth}
        \centering
        \includegraphics[width=\textwidth]{landscape_images/4_param_brute-force_95p_PCA_20241003.png} 
        \caption{}
        \label{fig:4paramPCAhighfid}
    \end{subfigure}
    \caption{Quantum control landscape after applying PCA and extracting regions of high fidelity (fidelity above 0.95): (a) Two parameter, (b) Three parameter, and (c) Four parameter, the dataset which is used to generate these plots is generated using a brute-force combination of inputs in the range [-1, 1]}
    \label{fig:afterapplyingPCAhighfid}
\end{figure}

\section{Experimental Setup and Computational Techniques}
\label{sec:experimentalsetup}
We can study the QCL generated when using SGD, GA, QL, DQN and PPO for $N$-parameter optimization using~\eqref{eq:LZ}. In order to generate the QCLs as a function of fidelity, we run 1000 experiments for each algorithm, hence generated 1000 data-points, we then pass the results through a 2 component PCA to visualize the landscape in 2D. Since the results of the PCA will depend on the number of data-points available, we first generated PCA loadings for 2, 3, and 4 parameters using data points created by the brute force combination of possible values between [-1, 1]. PCA loadings represent the coefficients assigned to the original variables when forming the principal components and they indicate the contribution of each original feature to the principal components. These PCA loadings are then used to transform the 2, 3 and 4 parameter results from the above five algorithms to a 2-component PCA transformation. In our study, Numpy and Qutip~\cite{JOHANSSON20131234} libraries are used to represent the quantum states and other related vectors and matrices resulting during the optimization process. A minimum infidelity (1-fidelity) value  of 0.001 is used, and when the infidelity value from a certain algorithm is below this value, we assume the target is achieved and the algorithm exits the optimization loop immediately. A brief summary of the algorithms used in this work is presented below.


\subsection{SGD with Momentum}
Stochastic Gradient Descent (SGD) is an optimization algorithm that is used to minimize the cost function (difference between target value and calculated value) in optimization problems. In the context of quantum control, SGD can be used to generate control pulses that would drive the initial state towards the target state as much as possible. Starting from a list of random values whose length is equal to the number of parameters, $N$, these values are iteratively updated by generating a set of infinitesimally small values which are added to or subtracted from the initial random values, until a pulse that can result in high fidelity (above the target fidelity threshold) is found or the total number of iterations are exhausted. 

To improve the convergence time of the traditional central difference based SGD algorithm used in \cite{Ferrie190404} and \cite{Zhang_2019}, we can add a momentum term. This addition results in improved performance when compared with the implementation without momentum. A learning rate of 0.01, and a momentum of 0.95 are used. The initial pulses consist of a set of random values between -1 and 1, with a length equal to the number of parameters $N$. Unlike the 500 iterations used in \cite{Zhang_2019}, a maximum iteration of 10,000 is used as the stopping criteria for the SGD so that the algorithm has enough time to find optimal results and avoid sub-optimal regions. If the algorithm is unable to find an optimal control pulse after the max iterations are exhausted, the control pulse from the final iteration is returned. 

\subsection{Genetic Algorithm (GA)}
Genetic algorithms (GA), a class of optimization techniques inspired by the principles of natural selection and genetics, operate by evolving a population of candidate solutions through iterative processes of selection, crossover, and mutation, aiming to improve the solutions over successive generations based on fitness criteria. 

In the context of quantum control, the genes will be the possible pulse amplitudes (we limit number of genes to be 100 in the range to [-1, 1] in our experiment), the chromosome will represent the complete control pulse of length $N$, the population will be the collection of the chromosomes, and we use a mutation rate of 0.3. The length of each chromosome will be equal to the number of parameters we are interested in. The fitness function is, naturally, the fidelity of the system after the selected chromosome is tested for its performance using the given Hamiltonian. In every iteration, chromosomes that are among the top 30 percent of the population (based on their fitness value) and those that are in the last 20 percent are selected to be part of the next generation of the population. The decision to include those in the bottom 20 percent is to enable the algorithm have enough genetic variations in the cross over and mutation stages of the algorithm. To keep the total population constant, the remaining 50 percent of the population is accounted for by generating new population through crossover and mutation. The initial population and chromosomes are generated by selecting genes randomly.


In our experiments, we limit the algorithm to have a maximum of 50 generations. If a chromosome that meets the fitness criteria is found early, it is returned; otherwise, the iteration continues until the final generation, after which the top chromosome based on fitness value is selected.

\subsection{Q-Learning (QL)}
Q-Learning is a model-free reinforcement learning algorithm that enables an agent to learn optimal actions within an environment by interacting with it. It operates by learning a Q-value function, which estimates the expected cumulative reward for taking a particular action in a given state and following an optimal policy thereafter. Through trial and error, Q-Learning updates these Q-values iteratively, using observed rewards to refine future decisions. This process allows the agent to eventually converge on an optimal policy, even without prior knowledge of the environment's dynamics.

QL in a given environment is characterized by two important variables: the state of the system/agent at any given time/episode and the action space of the system. While the state represents a snapshot of the environment at a given time, the action space of a QL algorithm is defined as the set of all possible actions an agent can take in a given environment, that change the agent's state and influence the rewards it receives, when executed. For our quantum control problem, the action space is represented by the set of possible amplitudes the pulse can have, where we limit this to be 100 values in the range [-1, 1]. 

The `state' of QL in our quantum control problem is represented by the state of the qubit at a given time in the evolution process. We use QL with epsilon-greedy strategy, where we explore the environment with a probability of $\epsilon$ and exploit the system by choosing the actions that have high Q-values  with a probability of 1-$\epsilon$. Because our quantum control problem has very short (i.e., few) time steps / episodes, the value of $\epsilon$ is set to 0.1 (10\% exploration and 90\% exploitation), to focus more on exploitation than exploration.  In this work, a learning rate of 0.001 and a reward decay factor of 0.9 are employed. The reward values are assigned as follows: -1 if the infidelity exceeds 0.5, 10 if the infidelity is below 0.5, 100 if the infidelity is below 0.1, and 500 if the infidelity is below 0.001. The initial actions are generated randomly. The maximum episodes the agent can take is limited to 500, and if the agent does not find an optimal path during this time, the state of the system at the end of the episode is returned.

\subsection{Deep Q-Network (DQN)}
Deep Q-Learning (DQN) is an extension of the Q-Learning algorithm that incorporates deep neural networks to handle environments with high-dimensional state spaces. While traditional Q-Learning relies on tabular representations (Q-table), DQN uses a neural network to approximate the Q-value function, enabling it to scale to more complex tasks. By integrating techniques like experience replay and target networks, DQN improves stability and convergence during training. 

The action space and state of the DQN algorithm are the same as that of the QL algorithm. The neural network we use is a simple 3 layer MLP (Multi Layer Perceptron) with hidden layers containing [64, 512, 256] units, respectively. Key hyper-parameters include a learning rate of 0.0001, an exploration fraction of 0.25, and a reward discount factor of 0.000001. Reward values are assigned as follows: 1 if the infidelity exceeds 0.5, 10 if the infidelity is below 0.5, 500 if the infidelity is below 0.1, and 5000 if the infidelity is below 0.001. The implementation details of our DQN algorithm are as follows:
\begin{itemize}
    \item We create a custom environment wrapped around gymnasium. Gymnasium~\cite{towers2024gymnasium} is a toolkit designed for developing single agent RL algorithms by providing environments for common RL experiments or the possibility to define custom environments, allowing researchers to easily design and run RL experiments tailored to specific needs.
    \item The neural net is implemented using Stable Baselines3. Stable Baseline3~\cite{stable-baselines3} is another popular library that implements state-of-the-art RL algorithms (like DQN and PPO) and simplifies the process of training, rapid experimentation and deploying RL models in Gymnasium environments.
    \item The input to the neural net is the state of the system at current time step and the output of the neural net is the optimal action for that state
\end{itemize}

\subsection{Proximal Policy Optimization (PPO)}
Proximal Policy Optimization (PPO) is a popular RL algorithm that strikes a balance between simplicity and efficiency. PPO simplifies the optimization process while maintaining stable updates by using a clipped objective function to prevent large, destabilizing policy updates, allowing for efficient learning across a variety of environments~\cite{schulman2017proximalpolicyoptimizationalgorithms}. PPO is widely used in RL due to its robustness, ease of implementation, and ability to perform well in both continuous and discrete action spaces.

Similar to DQN, we use Gymnasium to define a custom environment, where the action space, which consists of 100 discrete values within the range [-1, 1], is represented by the set of possible actions the qubit can take, and the `state' is represented by the state of the qubit at a given time. For the PPO algorithm, we use the implementation by Stable Baselines3, and used a 3 layer MLP with [64, 512, 256] units in each layer as the PPO's neural network. The following hyper-parameters are used in our experiments: a learning rate of 0.0001, entropy coefficient of 0.25, and a reward discount factor of 0.000001. Reward values are structured exactly the same to that of the DQN algorithm. For both DQN and PPO, initial actions are sampled from the action space randomly. Whereas DQN and PPO are highly effective algorithms for complex tasks, their use in simpler problems may be excessive. In such cases, the neural network could struggle to capture finer details, leading to sub-optimal learning.

\begin{figure}[htbp]
    \centering
    \begin{subfigure}[b]{0.29\textwidth}
        \centering
        \includegraphics[width=\textwidth]{landscape_images/2_param_SGD_from_PCA-Loadings_20241003.png} 
        \caption{}
        \label{fig:2paramPCASGD}
    \end{subfigure}
    \begin{subfigure}[b]{0.29\textwidth}
        \centering
        \includegraphics[width=\textwidth]{landscape_images/3_param_SGD_from_PCA-Loadings_20241003.png} 
        \caption{}
        \label{fig:3paramPCASGD}
    \end{subfigure}
    \begin{subfigure}[b]{0.29\textwidth}
        \centering
        \includegraphics[width=\textwidth]{landscape_images/4_param_SGD_from_PCA-Loadings_20241003.png} 
        \caption{}
        \label{fig:4paramPCASGD}
    \end{subfigure}
    
    \begin{subfigure}[b]{0.29\textwidth}
        \centering
        \includegraphics[width=\textwidth]{landscape_images/2_param_GA_from_PCA-Loadings_20241003.png} 
        \caption{}
        \label{fig:2paramPCAGA}
    \end{subfigure}
    \begin{subfigure}[b]{0.29\textwidth}
        \centering
        \includegraphics[width=\textwidth]{landscape_images/3_param_GA_from_PCA-Loadings_20241003.png} 
        \caption{}
        \label{fig:3paramPCAGA}
    \end{subfigure}
    \begin{subfigure}[b]{0.29\textwidth}
        \centering
        \includegraphics[width=\textwidth]{landscape_images/4_param_GA_from_PCA-Loadings_20241003.png} 
        \caption{}
        \label{fig:4paramPCAGA}
    \end{subfigure}
    \caption{The quantum landscape when using SGD and GA for 1000 tests and after applying PCA (a) SGD - 2 parameter, (b) SGD - 3 parameter, (c) SGD -  4 parameter, (d) GA - 2 parameter, (e) GA - 3 parameter, and (f) GA -  4 parameter}
    \label{fig:afterapplyingPCASGDGA}
\end{figure}


% \FloatBarrier

\section{Results Analysis and Discussion}
\label{sec:results}
In pursuit of discovering which algorithms among SGD, GA, QL, DQN and PPO are best at generating high fidelity control pulses and are able to explore the quantum landscape effectively, we conduct 1,000 tests for the 2, 3, and 4 parameter cases for each algorithm, and the resulting data points are passed through PCA to generate a 2D landscape. 


Fig.~\ref{fig:afterapplyingPCASGDGA}(a-c) shows the results after running SGD for a maximum of 10,000 iterations. From the results in Fig.~\ref{fig:afterapplyingPCASGDGA}(a-c), we observe that even though most points in the two dimensional quantum landscape correspond to high fidelity values, there are a considerable number of points which are in the sub optimal (low fidelity) region even after giving the algorithm enough time to converge (10,000 iterations). The presence of sub-optimal points suggests that SGD can struggle with convergence to globally optimal solutions.

In Fig.~\ref{fig:afterapplyingPCASGDGA}(d-f), we present the resulting landscape when the approach used is GA. As can be seen in Fig.~\ref{fig:afterapplyingPCASGDGA}(d-f), and unlike SGD, the GA results display high fidelity pulses across all parameter cases. This indicates that GA has a stronger capability to explore the parameter space effectively and converge to optimal solutions. However, it is important to note that despite the favorable results, GA may still encounter issues if the initial population is poorly sampled. Random initialization can trap GA in sub-optimal regions, although such cases were not evident in our experiments. Overall, the GA demonstrates greater robustness and reliability compared to SGD in generating high-fidelity pulses.


% \FloatBarrier

The results of our experiment using QL, after passing through PCA, are presented in Fig.~\ref{fig:afterapplyingPCAQL}. When compared to SGD, QL produces a landscape that is close to the ideal brute-force landscape, with the data points concentrated in high-fidelity regions. This suggests that QL, similar to GA, is a more capable algorithm for navigating quantum control landscapes, as it consistently outperforms SGD in terms of generating optimal control pulses. This landscape demonstrates QL's potential for effective optimization in quantum control tasks.

In the experiments we conduct using DQN and PPO, the results reveal important insights into reward structuring of these algorithms. It was discovered that for problems requiring fewer steps per episode, as is the case for 2, 3, and 4 parameter quantum control, delayed rewards often fail to guide the model towards optimal solutions due to the shorter episode length. On the contrary,  the model tends to achieve better results when immediate rewards are prioritized, as the system has less time to accumulate delayed feedback effectively.
 
\begin{figure}[htbp]
    \centering
    \begin{subfigure}[b]{0.29\textwidth}
        \centering
        \includegraphics[width=\textwidth]{landscape_images/2_param_QL_from_PCA-Loadings_20241003.png} 
        \caption{}
        \label{fig:2paramPCAQL}
    \end{subfigure}
    \begin{subfigure}[b]{0.29\textwidth}
        \centering
        \includegraphics[width=\textwidth]{landscape_images/3_param_QL_from_PCA-Loadings_20241003.png} 
        \caption{}
        \label{fig:3paramPCAQL}
    \end{subfigure}
    \begin{subfigure}[b]{0.29\textwidth}
        \centering
        \includegraphics[width=\textwidth]{landscape_images/4_param_QL_from_PCA-Loadings_20241003.png} 
        \caption{}
        \label{fig:4paramPCAQL}
    \end{subfigure}
    \caption{The quantum landscape when using QL for 1000 tests and after applying PCA (a) 2 parameter, (b) 3 parameter, and (c) 4 parameter}
    \label{fig:afterapplyingPCAQL}
\end{figure}

As is the case with the other algorithms we discussed above, the experiments were repeated 1000 times and the results of our experiment using DQN and PPO, after passing through PCA,  are presented in Fig.~\ref{fig:afterapplyingPCADQNPPO}. From the results in Fig.~\ref{fig:afterapplyingPCADQNPPO}, we see that the DQN and PPO results align closely (with the exception of the 4 parameter case) with those of QL and GA, suggesting that reinforcement learning methods, when structured with appropriate reward schemes, offer strong potential for high-fidelity quantum control optimization. The few results in the sub-optimal region for both DQN and PPO (especially for the 4 parameter case), may suggest these algorithms might be an overkill if the problem is relatively simple as in our single-qubit state transition problem. One possible reason for this could be that the neural network in DQN and PPO fails to learn the relationships between its inputs and outputs as the quantum system being explored is fairly simple for a neural network - suggesting we might want to use advanced RL algorithms only when the traditional algorithms fail to generate high fidelity pulses. Overall, for our single-qubit state transition quantum control problem, GA and QL standout as the best performing algorithms.


\begin{figure}[htbp]
    \centering
    \begin{subfigure}[b]{0.29\textwidth}
        \centering
        \includegraphics[width=\textwidth]{landscape_images/2_param_DQN_from_PCA-Loadings_20241003.png} 
        \caption{}
        \label{fig:2paramPCADQN}
    \end{subfigure}
    \begin{subfigure}[b]{0.29\textwidth}
        \centering
        \includegraphics[width=\textwidth]{landscape_images/3_param_DQN_from_PCA-Loadings_20241003.png} 
        \caption{}
        \label{fig:3paramPCADQN}
    \end{subfigure}
    \begin{subfigure}[b]{0.29\textwidth}
        \centering
        \includegraphics[width=\textwidth]{landscape_images/4_param_DQN_from_PCA-Loadings_20241003.png} 
        \caption{}
        \label{fig:4paramPCADQN}
    \end{subfigure}

    \begin{subfigure}[b]{0.29\textwidth}
        \centering
        \includegraphics[width=\textwidth]{landscape_images/2_param_PPO_from_PCA-Loadings_20241003.png} 
        \caption{}
        \label{fig:2paramPCAPPO}
    \end{subfigure}
    \begin{subfigure}[b]{0.29\textwidth}
        \centering
        \includegraphics[width=\textwidth]{landscape_images/3_param_PPO_from_PCA-Loadings_20241003.png} 
        \caption{}
        \label{fig:3paramPCAPPO}
    \end{subfigure}
    \begin{subfigure}[b]{0.29\textwidth}
        \centering
        \includegraphics[width=\textwidth]{landscape_images/4_param_PPO_from_PCA-Loadings_20241003.png} 
        \caption{}
        \label{fig:4paramPCAPPO}
    \end{subfigure}
    \caption{The quantum landscape when using DQN and PPO for 1000 tests and after applying PCA (a) 2 parameter-DQN, (b) 3 parameter-DQN, (c) 4 parameter-DQN (d) 2 parameter-PPO, (e) 3 parameter-PPO, and (f) 4 parameter-PPO}
    \label{fig:afterapplyingPCADQNPPO}
\end{figure}

Upon examining the landscape plots, it may initially seem that not all 1,000 test results are represented in each plot. This is due to overlapping points in the plots (i.e., where an approach finds the same solution multiple times), which creates the impression that fewer data points are present. For each algorithm and parameter count, we recorded the overlap of points to provide a comprehensive understanding. As an example, for the 3-parameter scenario, we present this overlap count in Fig.~\ref{fig:overlapSGDGAQLDQNPPO}. From this overlap plot, we see that while the points spread over the high fidelity regions for some algorithms, certain algorithms (DQN and PPO) tend to favor repeating specific combinations of inputs, resulting in clusters of points mainly in certain regions of the landscape. This clustering may suggest the tendency of certain algorithms to gravitate towards repeating certain pulses rather than exploring new regions of the parameter space.

\begin{figure}[h]
    \centering
    \begin{subfigure}[b]{0.3\textwidth}
        \centering
        \includegraphics[width=\textwidth]{overlap_images/3_param_SGD_Overlap_20241003.png} 
        \caption{}
        \label{fig:3paramSGDOVerlap}
    \end{subfigure}
    \begin{subfigure}[b]{0.3\textwidth}
        \centering
        \includegraphics[width=\textwidth]{overlap_images/3_param_GA_Overlap_20241003.png} 
        \caption{}
        \label{fig:3paramGAOVerlap}
    \end{subfigure}
    \begin{subfigure}[b]{0.3\textwidth}
        \centering
        \includegraphics[width=\textwidth]{overlap_images/3_param_QL_Overlap_20241003.png} 
        \caption{}
        \label{fig:3paramQLOVerlap}
    \end{subfigure}
    \begin{subfigure}[b]{0.3\textwidth}
        \centering
        \includegraphics[width=\textwidth]{overlap_images/3_param_DQN_Overlap_20241003.png} 
        \caption{}
        \label{fig:3paramDQNOVerlap}
    \end{subfigure}
    \begin{subfigure}[b]{0.3\textwidth}
        \centering
        \includegraphics[width=\textwidth]{overlap_images/3_param_PPO_Overlap_20241003.png} 
        \caption{}
        \label{fig:3paramPPOOVerlap}
    \end{subfigure}
    % \captionsetup{skip=5pt}
    % \setlength{\belowcaptionskip}{1pt}
    \caption{The overlap count for the 3 parameter case: the overlap count means the number of points in close proximity both in terms of coordinates and fidelity value. If certain number of points lie closer to each other in the 2D coordinate system and their fidelity values are close to each other, the overlap count in this region will be higher. Bigger circles represent higher count. (a) overlap count for 3 parameter-SGD, (b) overlap count for 3 parameter-GA, (c) overlap count for 3 parameter-QL (d) overlap count for 3 parameter-DQN, and (e) overlap count for 3 parameter-PPO}
    \label{fig:overlapSGDGAQLDQNPPO}
\end{figure}

Knowing the overlap count is relevant for a number of reasons: 
\begin{itemize}
    \item A high overlap count may indicate suboptimal results, suggesting that the algorithm may have missed high-fidelity regions, or
    \item it could imply that the quantum system is inherently complex, with limited solutions available to satisfy state transfer requirements, highlighting the need for more advanced control strategies.
\end{itemize}
On the other hand, if the overlap count remains low, it may imply that the algorithm has effectively explored and covered most high-fidelity regions in the solution space. This can indicate either that the algorithm is well-suited for the task or that the problem itself is relatively straightforward. In such cases, achieving successful state transfer may not require a complex or highly optimized quantum control strategy, as the solution can be reached with simpler techniques, saving computational resources and reducing algorithmic complexity.



To quantitatively evaluate how effectively each algorithm identifies pulses that result in high fidelity, we present the distribution of pulse counts as a function of fidelity in Fig.~\ref{fig:overlapPlotAll}. From the plots in Fig.~\ref{fig:2paramOverlapAll} we observe that 2-parameter SGD algorithm has many points concentrated in the low-fidelity region, indicating it struggles to find high-fidelity solutions. In contrast, GA has all its pulses within the high-fidelity region (close to 100\%), indicating it is highly effective in achieving high fidelity with two parameters. QL, DQN, and PPO display a mixed distribution, with most of their points also leaning toward higher fidelity, but they are less consistent than GA. For the 3-parameter case in Fig.~\ref{fig:3paramOVerlapAll}, SGD continues to struggle, GA maintains its dominance, achieving high fidelity with all pulses. QL, DQN, and PPO are distributed across high-fidelity regions as well, with DQN and PPO clustering closer to higher fidelity. In Fig.~\ref{fig:4paramOVerlapALL}, which explores the 4-parameter scenario, a more spread-out distribution is observed. SGD still has many points scattered across lower fidelity values, showing it is less efficient even with additional parameters. GA remains highly effective, with all points concentrated near 100\% fidelity. Interestingly, QL, DQN, and PPO also show a wider range in fidelity, but most points are in the high-fidelity region. PPO, in particular, has few pulses distributed across a wider fidelity range compared to GA. 
\begin{figure}[H]
    \centering
    \begin{subfigure}[b]{0.33\textwidth}
        \centering
        \includegraphics[width=\textwidth]{overlap_images/2_param_overlap_log_scaled_20241003.png} 
        \caption{}
        \label{fig:2paramOverlapAll}
    \end{subfigure}
    \begin{subfigure}[b]{0.33\textwidth}
        \centering
        \includegraphics[width=\textwidth]{overlap_images/3_param_overlap_log_scaled_20241003.png} 
        \caption{}
        \label{fig:3paramOVerlapAll}
    \end{subfigure}
    \begin{subfigure}[b]{0.33\textwidth}
        \centering
        \includegraphics[width=\textwidth]{overlap_images/4_param_overlap_log_scaled_20241003.png} 
        \caption{}
        \label{fig:4paramOVerlapALL}
    \end{subfigure}
    
    \caption{Pulse counts for SGD, GA, QL , DQN and PPO algorithm as a function of fidelity: (a) 2 parameter, (b) 3 parameter, and (c) 4 parameter. Note that the Y axis values are given in log scale to show the distribution of points more clearly }
    \label{fig:overlapPlotAll}
\end{figure}
In summary, an algorithm capable of generating a wide variety of high fidelity pulses is preferred over one that produces only a limited set, as having a broader selection of high fidelity pulses provides flexibility. This range is valuable since some pulses may not be easily realizable, making it essential to have alternative pulses that enhance the robustness and applicability of a control algorithm.

\section{Solution  Space Complexity}
\label{sec:solutionspacecomplexity}

In the previous section, we have seen that some approaches have the potential to yield more (i.e., increased variety) high-fidelity solutions, and others can fall into local optima traps. Yet, the question remains, how should we evaluate the complexity of a QCL in order to make informed choices regarding which approach(es) to use? In order to study the complexity of the solution space, we analyse how sparse/dense the landscapes are using clustering algorithms and distance measures. To find clusters in the solution space, we employ a cluster number agnostic algorithm-DBSCAN~\cite{ester1996density}, and to find the distances between points within a cluster;  we use the Euclidean distance. If two clusters are of comparable area and the distance between the points within one cluster is smaller than the distance between points within the other cluster, this tells us the first cluster is more dense. This, in turn, can reveal that an approach is more/less likely to find a high-fidelity solution in this region of the landscape. 

To measure how dense the landscape is, we define a term we called cluster density index (CDI), which is the inverse of how sparse a region is,  as:
\begin{equation}
\text{CDI} = \frac{\bar{A}}{\bar{D}}
\label{equ:sparsity}
\end{equation}
where, \textit{CDI} refers to the average Cluster Density Index, $\bar{A}$ is the average area of clusters calculated as mean of individual cluster areas which are calculated using Delaunay Triangulation~\cite{10.1145/235815.235821}, and $\bar{D}$ is the average, over all clusters, of the mean pairwise distances between points in a given cluster. In Table~\ref{tab:sparsityIndex}, $\bar{L}$ refers to the average inter-cluster distance.
\begin{table*} [htbp]
    \centering
    \caption{Cluster Density Index: describes how sparse / dense a landscape is}
    \begin{adjustbox}{max width=\textwidth}
    \begin{tabular}{|>{\centering\arraybackslash}m{0.041\linewidth}|>{\centering\arraybackslash}m{0.041\linewidth}|>{\centering\arraybackslash}m{0.041\linewidth}|>{\centering\arraybackslash}m{0.041\linewidth}|>{\centering\arraybackslash}m{0.041\linewidth}|>{\centering\arraybackslash}m{0.041\linewidth}|>{\centering\arraybackslash}m{0.041\linewidth}|>{\centering\arraybackslash}m{0.041\linewidth}|>{\centering\arraybackslash}m{0.041\linewidth}|>{\centering\arraybackslash}m{0.041\linewidth}|>{\centering\arraybackslash}m{0.041\linewidth}|>{\centering\arraybackslash}m{0.041\linewidth}|>{\centering\arraybackslash}m{0.041\linewidth}|>{\centering\arraybackslash}m{0.041\linewidth}|>{\centering\arraybackslash}m{0.041\linewidth}|>{\centering\arraybackslash}m{0.041\linewidth}|}  
    \hline
          \rule{0pt}{4.5ex} \multirow{2}{*}{} &  \multicolumn{3}{c|}{\textbf{SGD}} & \multicolumn{3}{c|}{\textbf{GA}} & \multicolumn{3}{c|}{\textbf{QL}} & \multicolumn{3}{c|}{\textbf{DQN}} & \multicolumn{3}{c|}{\textbf{PPO}} \\  \cline{2-16} 
          \rule{0pt}{5.5ex}& 2 param & 3 param & 4 param & 2 param & 3 param & 4 param & 2 param & 3 param & 4 param & 2 param & 3 param & 4 param & 2 param & 3 param & 4 param\\ \hline
         $\bar{L}$ & \gradient{0.9463}& \gradient{1.0501}& \gradient{1.2232}& \gradient{0.9391}& \gradient{0.9576}& \gradient{1.2380}& \gradient{0.8398}& \gradient{1.4588}& \gradient{1.3698}& \gradient{0.8919}& \gradient{1.1551}& \gradient{0.9895}& \gradient{0.8836}& \gradient{1.2656}& \gradient{1.1495}\\ \hline
         $\bar{D}$ & \gradient{0.0035}& \gradient{0.1280}& \gradient{0.1392}& \gradient{0.0234}& \gradient{0.1094}& \gradient{0.1418}& \gradient{0.0697}& \gradient{0.2178}& \gradient{0.1966}& \gradient{0.0423}& \gradient{0.0167}& \gradient{0.0253}& \gradient{0.1383}& \gradient{0.0273}& \gradient{0.1593}\\ \hline
         $\bar{A}$ & \gradient{0.0001}& \gradient{0.0251}& \gradient{0.1023}& \gradient{0.0010}& \gradient{0.0505}& \gradient{0.2282}& \gradient{0.0133}& \gradient{0.4129}& \gradient{0.4605}& \gradient{0.0036}& \gradient{0.0024}& \gradient{0.0027}& \gradient{0.0333}& \gradient{0.0185}& \gradient{0.0724}\\ \hline
         % Number of  clusters & \seqsplit{14}& \seqsplit{14}& \seqsplit{22}& \seqsplit{14}& \seqsplit{13}& \seqsplit{15}& \seqsplit{10}& \seqsplit{7}& \seqsplit{8}& \seqsplit{11}& \seqsplit{21}& \seqsplit{23}& \seqsplit{7}& \seqsplit{8}& \seqsplit{17}\\ \hline
         CDI & \gradient{0.0075}& \gradient{0.1962}& \gradient{0.7346}& \gradient{0.0449}& \gradient{0.4617}& \gradient{1.6096}& \gradient{0.1902}& \gradient{1.8957}& \gradient{2.3420}& \gradient{0.0844}& \gradient{0.1427}& \gradient{0.1075}& \gradient{0.2404}& \gradient{0.6766}& \gradient{0.4543}\\ \hline
    \end{tabular}
    \end{adjustbox}
    \label{tab:sparsityIndex}
\end{table*}

Using~\eqref{equ:sparsity}, we calculate and present the resulting Cluster Density Index in Table~\ref{tab:sparsityIndex}. A high CDI value is desirable, because a high CDI (according to~\eqref{equ:sparsity}), ideally, means there are high fidelity regions in the landscape which have large and dense clusters (it is worth to note that the landscapes are generated after using PCA on the output of the different algorithms). To increase the CDI value, $\bar{A}$ has to be high while $\bar{D}$ is kept low. Typically, higher CDI means there are many high fidelity points (higher cluster area) in the landscape with the distance between individual points being low (individual points are closer to each other in the cluster). The results in Table~\ref{tab:sparsityIndex} confirm that, in general, algorithms with larger clusters in the landscape, have higher cluster density index values. The best performing algorithms (GA and QL) have the highest CDI as compared to the remaining algorithms. When we compare the results in terms of parameter count, for the same algorithm, we see that for SGD, GA and QL increasing the number of parameters results in an increased CDI value. For DQN and PPO, 3 parameter pulses seem to be better than both 2 parameter and 4 parameter pulses - a detailed future study could be necessary to further investigate this outlier behavior.

However, it should be noted that a higher value of CDI does not necessarily mean that there are many high fidelity points in a given region - because we can get high CDI value even when $\bar{A}$ is very small (extremely few points in each cluster) provided that $\bar{D}$ is even smaller. For example, looking at the CDI value in Table~\ref{tab:sparsityIndex}, for 3 parameter PPO, it is relatively larger than the value for 2 parameter PPO. This is because the average cluster area is very small for 3 parameter PPO, but the average distance between the clusters is even smaller, resulting in a relatively higher CDI value for the 3 parameter PPO than that of 2 parameter. However, this does not mean 3 parameter PPO outperforms 2 parameter PPO in terms of exploring the QCL and generating control pulses with high fidelity (indeed a quick check of Fig.\ref{fig:afterapplyingPCADQNPPO} (d) and (e),  can reveal that 2 parameter PPO results are better at exploring the landscape when compared to 3 parameter case). While higher CDI values can sometimes result from small cluster areas paired with even smaller inter-point distances, these instances represent edge cases. In most scenarios, CDI serves as a reliable metric for assessing cluster quality, as it typically indicates regions of high fidelity with tightly packed points (we refer, for example, CDI results in Table~\ref{tab:sparsityIndex} for GA and QL and verify results by looking at Fig.~\ref{fig:afterapplyingPCASGDGA}(d-f) and Fig.~\ref{fig:afterapplyingPCAQL}). In general, CDI captures the relation between cluster area and point distribution, making it a valuable measure for solution space complexity analysis.

The CDI can be a very powerful tool for studying the QCL even at higher dimensions than we have considered. Because DBSCAN can work with higher dimensional data, and Euclidean distances can be calculated for any dimensions, CDI can potentially allow us to understand how probable it is to find high fidelity clusters in a landscape without even using dimensionality reduction techniques. 

% \FloatBarrier
\section{Conclusion}
In this work, we studied the dynamics of quantum control landscapes through the use of PCA as a dimensionality reduction technique, to visualize higher-dimensional quantum landscapes and algorithms best suited for solving quantum control problem. Our findings reveal that dimensionality reduction techniques are important to analyze the intricate nature of high dimensional quantum control. In addition, we noted that increasing the number of parameters (dividing the total evolution time into more discrete parts) yields landscapes with more high-fidelity regions. However, this does not guarantee a reduction in low-fidelity areas; rather, it enhances the chances of landing on high-fidelity regions when the number of parameters is increased. It should be noted that all dimensionality reduction techniques involve some degree of information loss; as such it would be highly relevant to compare and contrast characteristics of other dimensionality reduction techniques when applied to QCL.

Comparative analyses of traditional and machine learning algorithms in solving the quantum control problem for the given Hamiltonian demonstrates that while Stochastic Gradient Descent (SGD) can perform poorly, Genetic Algorithms (GA) excel at finding high fidelity pulses and exploring the landscape well. Among machine learning approaches, Q-learning (QL) showed promising results, while Deep Q-Network (DQN) and Proximal Policy Optimization (PPO) were less effective, suggesting that simpler algorithms may be advantageous for straightforward problems. Additionally, the design of the reward function in DQN and PPO significantly impacts performance, with immediate rewards being more beneficial for short episode tasks. 


The solution space complexity analysis through the use of cluster density index, indicates that both GA and QL achieved high cluster-density scores for 4-parameter quantum control - meaning those algorithms result in a larger QCL clusters with closely spaced individual points. When comparing the results in terms of parameter count, the cluster-density-score increase when the parameter count is increased from 2 parameter to 3 parameter and then to 4 parameter  with exceptions noted for DQN and PPO. 

Our results demonstrate that dimensionality reduction tools, in particular PCA, can be highly effective in capturing the relevant features of the control landscape. While we have restricted our analysis to a small number of parameters to ensure that we can easily access the full QCL via brute force for comparison, it is worth highlighting that many of the considered algorithms can accommodate optimizing a much larger number of parameters~\cite{CoopmansPRR} or be applied to more complex multi-qubit systems. Our framework for characterising the solution space can nevertheless be readily applied without incurring significant additional computational overhead. The obvious caveat then being that achieving control over these more complex systems will necessitate high dimensional control pulses and, therefore, it still remains to determine the tradeoff between how many principal components are required to maintain the important features of the QCL. Furthermore, our results help to get a deeper understanding of quantum control strategies, emphasizing the importance of parameter selection and reward design in optimizing algorithm performance.

Future research could explore why certain algorithms (example: DQN and PPO) under-perform in short RL episodes, and identify the types of problems that require RL versus those better suited for traditional algorithms. Further research could also establish clear metrics to distinguish between simple and complex problems, helping determine the most appropriate algorithms based on problem complexity and required computational resources.

\section*{Data availability}
The data generated during this study are available from the corresponding author upon reasonable request.

\section*{Code availability}
The code used in this work is accessible on GitHub through the following link: 
%if article is accepted for publication.
\href{https://github.com/Hafdream/Quantum-control-and-landscape}{click here for code}

\section*{Acknowledgements}

This publication has emanated from research conducted with the financial support of Science Foundation Ireland under Grant number 18/CRT/6183. For the purpose of Open Access, the author has applied a CC BY public copyright license to any Author Accepted Manuscript version arising from this submission.

% \bibliography{sample}
% \bibliographystyle{unsrt}
% \bibliography{references}
% This must be in the first 5 lines to tell arXiv to use pdfLaTeX, which is strongly recommended.
\pdfoutput=1
% In particular, the hyperref package requires pdfLaTeX in order to break URLs across lines.

\documentclass[11pt]{article}

% Change "review" to "final" to generate the final (sometimes called camera-ready) version.
% Change to "preprint" to generate a non-anonymous version with page numbers.
\usepackage{acl}

% Standard package includes
\usepackage{times}
\usepackage{latexsym}

% Draw tables
\usepackage{booktabs}
\usepackage{multirow}
\usepackage{xcolor}
\usepackage{colortbl}
\usepackage{array} 
\usepackage{amsmath}

\newcolumntype{C}{>{\centering\arraybackslash}p{0.07\textwidth}}
% For proper rendering and hyphenation of words containing Latin characters (including in bib files)
\usepackage[T1]{fontenc}
% For Vietnamese characters
% \usepackage[T5]{fontenc}
% See https://www.latex-project.org/help/documentation/encguide.pdf for other character sets
% This assumes your files are encoded as UTF8
\usepackage[utf8]{inputenc}

% This is not strictly necessary, and may be commented out,
% but it will improve the layout of the manuscript,
% and will typically save some space.
\usepackage{microtype}
\DeclareMathOperator*{\argmax}{arg\,max}
% This is also not strictly necessary, and may be commented out.
% However, it will improve the aesthetics of text in
% the typewriter font.
\usepackage{inconsolata}

%Including images in your LaTeX document requires adding
%additional package(s)
\usepackage{graphicx}
% If the title and author information does not fit in the area allocated, uncomment the following
%
%\setlength\titlebox{<dim>}
%
% and set <dim> to something 5cm or larger.

\title{Wi-Chat: Large Language Model Powered Wi-Fi Sensing}

% Author information can be set in various styles:
% For several authors from the same institution:
% \author{Author 1 \and ... \and Author n \\
%         Address line \\ ... \\ Address line}
% if the names do not fit well on one line use
%         Author 1 \\ {\bf Author 2} \\ ... \\ {\bf Author n} \\
% For authors from different institutions:
% \author{Author 1 \\ Address line \\  ... \\ Address line
%         \And  ... \And
%         Author n \\ Address line \\ ... \\ Address line}
% To start a separate ``row'' of authors use \AND, as in
% \author{Author 1 \\ Address line \\  ... \\ Address line
%         \AND
%         Author 2 \\ Address line \\ ... \\ Address line \And
%         Author 3 \\ Address line \\ ... \\ Address line}

% \author{First Author \\
%   Affiliation / Address line 1 \\
%   Affiliation / Address line 2 \\
%   Affiliation / Address line 3 \\
%   \texttt{email@domain} \\\And
%   Second Author \\
%   Affiliation / Address line 1 \\
%   Affiliation / Address line 2 \\
%   Affiliation / Address line 3 \\
%   \texttt{email@domain} \\}
% \author{Haohan Yuan \qquad Haopeng Zhang\thanks{corresponding author} \\ 
%   ALOHA Lab, University of Hawaii at Manoa \\
%   % Affiliation / Address line 2 \\
%   % Affiliation / Address line 3 \\
%   \texttt{\{haohany,haopengz\}@hawaii.edu}}
  
\author{
{Haopeng Zhang$\dag$\thanks{These authors contributed equally to this work.}, Yili Ren$\ddagger$\footnotemark[1], Haohan Yuan$\dag$, Jingzhe Zhang$\ddagger$, Yitong Shen$\ddagger$} \\
ALOHA Lab, University of Hawaii at Manoa$\dag$, University of South Florida$\ddagger$ \\
\{haopengz, haohany\}@hawaii.edu\\
\{yiliren, jingzhe, shen202\}@usf.edu\\}



  
%\author{
%  \textbf{First Author\textsuperscript{1}},
%  \textbf{Second Author\textsuperscript{1,2}},
%  \textbf{Third T. Author\textsuperscript{1}},
%  \textbf{Fourth Author\textsuperscript{1}},
%\\
%  \textbf{Fifth Author\textsuperscript{1,2}},
%  \textbf{Sixth Author\textsuperscript{1}},
%  \textbf{Seventh Author\textsuperscript{1}},
%  \textbf{Eighth Author \textsuperscript{1,2,3,4}},
%\\
%  \textbf{Ninth Author\textsuperscript{1}},
%  \textbf{Tenth Author\textsuperscript{1}},
%  \textbf{Eleventh E. Author\textsuperscript{1,2,3,4,5}},
%  \textbf{Twelfth Author\textsuperscript{1}},
%\\
%  \textbf{Thirteenth Author\textsuperscript{3}},
%  \textbf{Fourteenth F. Author\textsuperscript{2,4}},
%  \textbf{Fifteenth Author\textsuperscript{1}},
%  \textbf{Sixteenth Author\textsuperscript{1}},
%\\
%  \textbf{Seventeenth S. Author\textsuperscript{4,5}},
%  \textbf{Eighteenth Author\textsuperscript{3,4}},
%  \textbf{Nineteenth N. Author\textsuperscript{2,5}},
%  \textbf{Twentieth Author\textsuperscript{1}}
%\\
%\\
%  \textsuperscript{1}Affiliation 1,
%  \textsuperscript{2}Affiliation 2,
%  \textsuperscript{3}Affiliation 3,
%  \textsuperscript{4}Affiliation 4,
%  \textsuperscript{5}Affiliation 5
%\\
%  \small{
%    \textbf{Correspondence:} \href{mailto:email@domain}{email@domain}
%  }
%}

\begin{document}
\maketitle
\begin{abstract}
Recent advancements in Large Language Models (LLMs) have demonstrated remarkable capabilities across diverse tasks. However, their potential to integrate physical model knowledge for real-world signal interpretation remains largely unexplored. In this work, we introduce Wi-Chat, the first LLM-powered Wi-Fi-based human activity recognition system. We demonstrate that LLMs can process raw Wi-Fi signals and infer human activities by incorporating Wi-Fi sensing principles into prompts. Our approach leverages physical model insights to guide LLMs in interpreting Channel State Information (CSI) data without traditional signal processing techniques. Through experiments on real-world Wi-Fi datasets, we show that LLMs exhibit strong reasoning capabilities, achieving zero-shot activity recognition. These findings highlight a new paradigm for Wi-Fi sensing, expanding LLM applications beyond conventional language tasks and enhancing the accessibility of wireless sensing for real-world deployments.
\end{abstract}

\section{Introduction}

In today’s rapidly evolving digital landscape, the transformative power of web technologies has redefined not only how services are delivered but also how complex tasks are approached. Web-based systems have become increasingly prevalent in risk control across various domains. This widespread adoption is due their accessibility, scalability, and ability to remotely connect various types of users. For example, these systems are used for process safety management in industry~\cite{kannan2016web}, safety risk early warning in urban construction~\cite{ding2013development}, and safe monitoring of infrastructural systems~\cite{repetto2018web}. Within these web-based risk management systems, the source search problem presents a huge challenge. Source search refers to the task of identifying the origin of a risky event, such as a gas leak and the emission point of toxic substances. This source search capability is crucial for effective risk management and decision-making.

Traditional approaches to implementing source search capabilities into the web systems often rely on solely algorithmic solutions~\cite{ristic2016study}. These methods, while relatively straightforward to implement, often struggle to achieve acceptable performances due to algorithmic local optima and complex unknown environments~\cite{zhao2020searching}. More recently, web crowdsourcing has emerged as a promising alternative for tackling the source search problem by incorporating human efforts in these web systems on-the-fly~\cite{zhao2024user}. This approach outsources the task of addressing issues encountered during the source search process to human workers, leveraging their capabilities to enhance system performance.

These solutions often employ a human-AI collaborative way~\cite{zhao2023leveraging} where algorithms handle exploration-exploitation and report the encountered problems while human workers resolve complex decision-making bottlenecks to help the algorithms getting rid of local deadlocks~\cite{zhao2022crowd}. Although effective, this paradigm suffers from two inherent limitations: increased operational costs from continuous human intervention, and slow response times of human workers due to sequential decision-making. These challenges motivate our investigation into developing autonomous systems that preserve human-like reasoning capabilities while reducing dependency on massive crowdsourced labor.

Furthermore, recent advancements in large language models (LLMs)~\cite{chang2024survey} and multi-modal LLMs (MLLMs)~\cite{huang2023chatgpt} have unveiled promising avenues for addressing these challenges. One clear opportunity involves the seamless integration of visual understanding and linguistic reasoning for robust decision-making in search tasks. However, whether large models-assisted source search is really effective and efficient for improving the current source search algorithms~\cite{ji2022source} remains unknown. \textit{To address the research gap, we are particularly interested in answering the following two research questions in this work:}

\textbf{\textit{RQ1: }}How can source search capabilities be integrated into web-based systems to support decision-making in time-sensitive risk management scenarios? 
% \sq{I mention ``time-sensitive'' here because I feel like we shall say something about the response time -- LLM has to be faster than humans}

\textbf{\textit{RQ2: }}How can MLLMs and LLMs enhance the effectiveness and efficiency of existing source search algorithms? 

% \textit{\textbf{RQ2:}} To what extent does the performance of large models-assisted search align with or approach the effectiveness of human-AI collaborative search? 

To answer the research questions, we propose a novel framework called Auto-\
S$^2$earch (\textbf{Auto}nomous \textbf{S}ource \textbf{Search}) and implement a prototype system that leverages advanced web technologies to simulate real-world conditions for zero-shot source search. Unlike traditional methods that rely on pre-defined heuristics or extensive human intervention, AutoS$^2$earch employs a carefully designed prompt that encapsulates human rationales, thereby guiding the MLLM to generate coherent and accurate scene descriptions from visual inputs about four directional choices. Based on these language-based descriptions, the LLM is enabled to determine the optimal directional choice through chain-of-thought (CoT) reasoning. Comprehensive empirical validation demonstrates that AutoS$^2$-\ 
earch achieves a success rate of 95–98\%, closely approaching the performance of human-AI collaborative search across 20 benchmark scenarios~\cite{zhao2023leveraging}. 

Our work indicates that the role of humans in future web crowdsourcing tasks may evolve from executors to validators or supervisors. Furthermore, incorporating explanations of LLM decisions into web-based system interfaces has the potential to help humans enhance task performance in risk control.






\section{Related Work}
\label{sec:relatedworks}

% \begin{table*}[t]
% \centering 
% \renewcommand\arraystretch{0.98}
% \fontsize{8}{10}\selectfont \setlength{\tabcolsep}{0.4em}
% \begin{tabular}{@{}lc|cc|cc|cc@{}}
% \toprule
% \textbf{Methods}           & \begin{tabular}[c]{@{}c@{}}\textbf{Training}\\ \textbf{Paradigm}\end{tabular} & \begin{tabular}[c]{@{}c@{}}\textbf{$\#$ PT Data}\\ \textbf{(Tokens)}\end{tabular} & \begin{tabular}[c]{@{}c@{}}\textbf{$\#$ IFT Data}\\ \textbf{(Samples)}\end{tabular} & \textbf{Code}  & \begin{tabular}[c]{@{}c@{}}\textbf{Natural}\\ \textbf{Language}\end{tabular} & \begin{tabular}[c]{@{}c@{}}\textbf{Action}\\ \textbf{Trajectories}\end{tabular} & \begin{tabular}[c]{@{}c@{}}\textbf{API}\\ \textbf{Documentation}\end{tabular}\\ \midrule 
% NexusRaven~\citep{srinivasan2023nexusraven} & IFT & - & - & \textcolor{green}{\CheckmarkBold} & \textcolor{green}{\CheckmarkBold} &\textcolor{red}{\XSolidBrush}&\textcolor{red}{\XSolidBrush}\\
% AgentInstruct~\citep{zeng2023agenttuning} & IFT & - & 2k & \textcolor{green}{\CheckmarkBold} & \textcolor{green}{\CheckmarkBold} &\textcolor{red}{\XSolidBrush}&\textcolor{red}{\XSolidBrush} \\
% AgentEvol~\citep{xi2024agentgym} & IFT & - & 14.5k & \textcolor{green}{\CheckmarkBold} & \textcolor{green}{\CheckmarkBold} &\textcolor{green}{\CheckmarkBold}&\textcolor{red}{\XSolidBrush} \\
% Gorilla~\citep{patil2023gorilla}& IFT & - & 16k & \textcolor{green}{\CheckmarkBold} & \textcolor{green}{\CheckmarkBold} &\textcolor{red}{\XSolidBrush}&\textcolor{green}{\CheckmarkBold}\\
% OpenFunctions-v2~\citep{patil2023gorilla} & IFT & - & 65k & \textcolor{green}{\CheckmarkBold} & \textcolor{green}{\CheckmarkBold} &\textcolor{red}{\XSolidBrush}&\textcolor{green}{\CheckmarkBold}\\
% LAM~\citep{zhang2024agentohana} & IFT & - & 42.6k & \textcolor{green}{\CheckmarkBold} & \textcolor{green}{\CheckmarkBold} &\textcolor{green}{\CheckmarkBold}&\textcolor{red}{\XSolidBrush} \\
% xLAM~\citep{liu2024apigen} & IFT & - & 60k & \textcolor{green}{\CheckmarkBold} & \textcolor{green}{\CheckmarkBold} &\textcolor{green}{\CheckmarkBold}&\textcolor{red}{\XSolidBrush} \\\midrule
% LEMUR~\citep{xu2024lemur} & PT & 90B & 300k & \textcolor{green}{\CheckmarkBold} & \textcolor{green}{\CheckmarkBold} &\textcolor{green}{\CheckmarkBold}&\textcolor{red}{\XSolidBrush}\\
% \rowcolor{teal!12} \method & PT & 103B & 95k & \textcolor{green}{\CheckmarkBold} & \textcolor{green}{\CheckmarkBold} & \textcolor{green}{\CheckmarkBold} & \textcolor{green}{\CheckmarkBold} \\
% \bottomrule
% \end{tabular}
% \caption{Summary of existing tuning- and pretraining-based LLM agents with their training sample sizes. "PT" and "IFT" denote "Pre-Training" and "Instruction Fine-Tuning", respectively. }
% \label{tab:related}
% \end{table*}

\begin{table*}[ht]
\begin{threeparttable}
\centering 
\renewcommand\arraystretch{0.98}
\fontsize{7}{9}\selectfont \setlength{\tabcolsep}{0.2em}
\begin{tabular}{@{}l|c|c|ccc|cc|cc|cccc@{}}
\toprule
\textbf{Methods} & \textbf{Datasets}           & \begin{tabular}[c]{@{}c@{}}\textbf{Training}\\ \textbf{Paradigm}\end{tabular} & \begin{tabular}[c]{@{}c@{}}\textbf{\# PT Data}\\ \textbf{(Tokens)}\end{tabular} & \begin{tabular}[c]{@{}c@{}}\textbf{\# IFT Data}\\ \textbf{(Samples)}\end{tabular} & \textbf{\# APIs} & \textbf{Code}  & \begin{tabular}[c]{@{}c@{}}\textbf{Nat.}\\ \textbf{Lang.}\end{tabular} & \begin{tabular}[c]{@{}c@{}}\textbf{Action}\\ \textbf{Traj.}\end{tabular} & \begin{tabular}[c]{@{}c@{}}\textbf{API}\\ \textbf{Doc.}\end{tabular} & \begin{tabular}[c]{@{}c@{}}\textbf{Func.}\\ \textbf{Call}\end{tabular} & \begin{tabular}[c]{@{}c@{}}\textbf{Multi.}\\ \textbf{Step}\end{tabular}  & \begin{tabular}[c]{@{}c@{}}\textbf{Plan}\\ \textbf{Refine}\end{tabular}  & \begin{tabular}[c]{@{}c@{}}\textbf{Multi.}\\ \textbf{Turn}\end{tabular}\\ \midrule 
\multicolumn{13}{l}{\emph{Instruction Finetuning-based LLM Agents for Intrinsic Reasoning}}  \\ \midrule
FireAct~\cite{chen2023fireact} & FireAct & IFT & - & 2.1K & 10 & \textcolor{red}{\XSolidBrush} &\textcolor{green}{\CheckmarkBold} &\textcolor{green}{\CheckmarkBold}  & \textcolor{red}{\XSolidBrush} &\textcolor{green}{\CheckmarkBold} & \textcolor{red}{\XSolidBrush} &\textcolor{green}{\CheckmarkBold} & \textcolor{red}{\XSolidBrush} \\
ToolAlpaca~\cite{tang2023toolalpaca} & ToolAlpaca & IFT & - & 4.0K & 400 & \textcolor{red}{\XSolidBrush} &\textcolor{green}{\CheckmarkBold} &\textcolor{green}{\CheckmarkBold} & \textcolor{red}{\XSolidBrush} &\textcolor{green}{\CheckmarkBold} & \textcolor{red}{\XSolidBrush}  &\textcolor{green}{\CheckmarkBold} & \textcolor{red}{\XSolidBrush}  \\
ToolLLaMA~\cite{qin2023toolllm} & ToolBench & IFT & - & 12.7K & 16,464 & \textcolor{red}{\XSolidBrush} &\textcolor{green}{\CheckmarkBold} &\textcolor{green}{\CheckmarkBold} &\textcolor{red}{\XSolidBrush} &\textcolor{green}{\CheckmarkBold}&\textcolor{green}{\CheckmarkBold}&\textcolor{green}{\CheckmarkBold} &\textcolor{green}{\CheckmarkBold}\\
AgentEvol~\citep{xi2024agentgym} & AgentTraj-L & IFT & - & 14.5K & 24 &\textcolor{red}{\XSolidBrush} & \textcolor{green}{\CheckmarkBold} &\textcolor{green}{\CheckmarkBold}&\textcolor{red}{\XSolidBrush} &\textcolor{green}{\CheckmarkBold}&\textcolor{red}{\XSolidBrush} &\textcolor{red}{\XSolidBrush} &\textcolor{green}{\CheckmarkBold}\\
Lumos~\cite{yin2024agent} & Lumos & IFT  & - & 20.0K & 16 &\textcolor{red}{\XSolidBrush} & \textcolor{green}{\CheckmarkBold} & \textcolor{green}{\CheckmarkBold} &\textcolor{red}{\XSolidBrush} & \textcolor{green}{\CheckmarkBold} & \textcolor{green}{\CheckmarkBold} &\textcolor{red}{\XSolidBrush} & \textcolor{green}{\CheckmarkBold}\\
Agent-FLAN~\cite{chen2024agent} & Agent-FLAN & IFT & - & 24.7K & 20 &\textcolor{red}{\XSolidBrush} & \textcolor{green}{\CheckmarkBold} & \textcolor{green}{\CheckmarkBold} &\textcolor{red}{\XSolidBrush} & \textcolor{green}{\CheckmarkBold}& \textcolor{green}{\CheckmarkBold}&\textcolor{red}{\XSolidBrush} & \textcolor{green}{\CheckmarkBold}\\
AgentTuning~\citep{zeng2023agenttuning} & AgentInstruct & IFT & - & 35.0K & - &\textcolor{red}{\XSolidBrush} & \textcolor{green}{\CheckmarkBold} & \textcolor{green}{\CheckmarkBold} &\textcolor{red}{\XSolidBrush} & \textcolor{green}{\CheckmarkBold} &\textcolor{red}{\XSolidBrush} &\textcolor{red}{\XSolidBrush} & \textcolor{green}{\CheckmarkBold}\\\midrule
\multicolumn{13}{l}{\emph{Instruction Finetuning-based LLM Agents for Function Calling}} \\\midrule
NexusRaven~\citep{srinivasan2023nexusraven} & NexusRaven & IFT & - & - & 116 & \textcolor{green}{\CheckmarkBold} & \textcolor{green}{\CheckmarkBold}  & \textcolor{green}{\CheckmarkBold} &\textcolor{red}{\XSolidBrush} & \textcolor{green}{\CheckmarkBold} &\textcolor{red}{\XSolidBrush} &\textcolor{red}{\XSolidBrush}&\textcolor{red}{\XSolidBrush}\\
Gorilla~\citep{patil2023gorilla} & Gorilla & IFT & - & 16.0K & 1,645 & \textcolor{green}{\CheckmarkBold} &\textcolor{red}{\XSolidBrush} &\textcolor{red}{\XSolidBrush}&\textcolor{green}{\CheckmarkBold} &\textcolor{green}{\CheckmarkBold} &\textcolor{red}{\XSolidBrush} &\textcolor{red}{\XSolidBrush} &\textcolor{red}{\XSolidBrush}\\
OpenFunctions-v2~\citep{patil2023gorilla} & OpenFunctions-v2 & IFT & - & 65.0K & - & \textcolor{green}{\CheckmarkBold} & \textcolor{green}{\CheckmarkBold} &\textcolor{red}{\XSolidBrush} &\textcolor{green}{\CheckmarkBold} &\textcolor{green}{\CheckmarkBold} &\textcolor{red}{\XSolidBrush} &\textcolor{red}{\XSolidBrush} &\textcolor{red}{\XSolidBrush}\\
API Pack~\cite{guo2024api} & API Pack & IFT & - & 1.1M & 11,213 &\textcolor{green}{\CheckmarkBold} &\textcolor{red}{\XSolidBrush} &\textcolor{green}{\CheckmarkBold} &\textcolor{red}{\XSolidBrush} &\textcolor{green}{\CheckmarkBold} &\textcolor{red}{\XSolidBrush}&\textcolor{red}{\XSolidBrush}&\textcolor{red}{\XSolidBrush}\\ 
LAM~\citep{zhang2024agentohana} & AgentOhana & IFT & - & 42.6K & - & \textcolor{green}{\CheckmarkBold} & \textcolor{green}{\CheckmarkBold} &\textcolor{green}{\CheckmarkBold}&\textcolor{red}{\XSolidBrush} &\textcolor{green}{\CheckmarkBold}&\textcolor{red}{\XSolidBrush}&\textcolor{green}{\CheckmarkBold}&\textcolor{green}{\CheckmarkBold}\\
xLAM~\citep{liu2024apigen} & APIGen & IFT & - & 60.0K & 3,673 & \textcolor{green}{\CheckmarkBold} & \textcolor{green}{\CheckmarkBold} &\textcolor{green}{\CheckmarkBold}&\textcolor{red}{\XSolidBrush} &\textcolor{green}{\CheckmarkBold}&\textcolor{red}{\XSolidBrush}&\textcolor{green}{\CheckmarkBold}&\textcolor{green}{\CheckmarkBold}\\\midrule
\multicolumn{13}{l}{\emph{Pretraining-based LLM Agents}}  \\\midrule
% LEMUR~\citep{xu2024lemur} & PT & 90B & 300.0K & - & \textcolor{green}{\CheckmarkBold} & \textcolor{green}{\CheckmarkBold} &\textcolor{green}{\CheckmarkBold}&\textcolor{red}{\XSolidBrush} & \textcolor{red}{\XSolidBrush} &\textcolor{green}{\CheckmarkBold} &\textcolor{red}{\XSolidBrush}&\textcolor{red}{\XSolidBrush}\\
\rowcolor{teal!12} \method & \dataset & PT & 103B & 95.0K  & 76,537  & \textcolor{green}{\CheckmarkBold} & \textcolor{green}{\CheckmarkBold} & \textcolor{green}{\CheckmarkBold} & \textcolor{green}{\CheckmarkBold} & \textcolor{green}{\CheckmarkBold} & \textcolor{green}{\CheckmarkBold} & \textcolor{green}{\CheckmarkBold} & \textcolor{green}{\CheckmarkBold}\\
\bottomrule
\end{tabular}
% \begin{tablenotes}
%     \item $^*$ In addition, the StarCoder-API can offer 4.77M more APIs.
% \end{tablenotes}
\caption{Summary of existing instruction finetuning-based LLM agents for intrinsic reasoning and function calling, along with their training resources and sample sizes. "PT" and "IFT" denote "Pre-Training" and "Instruction Fine-Tuning", respectively.}
\vspace{-2ex}
\label{tab:related}
\end{threeparttable}
\end{table*}

\noindent \textbf{Prompting-based LLM Agents.} Due to the lack of agent-specific pre-training corpus, existing LLM agents rely on either prompt engineering~\cite{hsieh2023tool,lu2024chameleon,yao2022react,wang2023voyager} or instruction fine-tuning~\cite{chen2023fireact,zeng2023agenttuning} to understand human instructions, decompose high-level tasks, generate grounded plans, and execute multi-step actions. 
However, prompting-based methods mainly depend on the capabilities of backbone LLMs (usually commercial LLMs), failing to introduce new knowledge and struggling to generalize to unseen tasks~\cite{sun2024adaplanner,zhuang2023toolchain}. 

\noindent \textbf{Instruction Finetuning-based LLM Agents.} Considering the extensive diversity of APIs and the complexity of multi-tool instructions, tool learning inherently presents greater challenges than natural language tasks, such as text generation~\cite{qin2023toolllm}.
Post-training techniques focus more on instruction following and aligning output with specific formats~\cite{patil2023gorilla,hao2024toolkengpt,qin2023toolllm,schick2024toolformer}, rather than fundamentally improving model knowledge or capabilities. 
Moreover, heavy fine-tuning can hinder generalization or even degrade performance in non-agent use cases, potentially suppressing the original base model capabilities~\cite{ghosh2024a}.

\noindent \textbf{Pretraining-based LLM Agents.} While pre-training serves as an essential alternative, prior works~\cite{nijkamp2023codegen,roziere2023code,xu2024lemur,patil2023gorilla} have primarily focused on improving task-specific capabilities (\eg, code generation) instead of general-domain LLM agents, due to single-source, uni-type, small-scale, and poor-quality pre-training data. 
Existing tool documentation data for agent training either lacks diverse real-world APIs~\cite{patil2023gorilla, tang2023toolalpaca} or is constrained to single-tool or single-round tool execution. 
Furthermore, trajectory data mostly imitate expert behavior or follow function-calling rules with inferior planning and reasoning, failing to fully elicit LLMs' capabilities and handle complex instructions~\cite{qin2023toolllm}. 
Given a wide range of candidate API functions, each comprising various function names and parameters available at every planning step, identifying globally optimal solutions and generalizing across tasks remains highly challenging.



\section{Preliminaries}
\label{Preliminaries}
\begin{figure*}[t]
    \centering
    \includegraphics[width=0.95\linewidth]{fig/HealthGPT_Framework.png}
    \caption{The \ourmethod{} architecture integrates hierarchical visual perception and H-LoRA, employing a task-specific hard router to select visual features and H-LoRA plugins, ultimately generating outputs with an autoregressive manner.}
    \label{fig:architecture}
\end{figure*}
\noindent\textbf{Large Vision-Language Models.} 
The input to a LVLM typically consists of an image $x^{\text{img}}$ and a discrete text sequence $x^{\text{txt}}$. The visual encoder $\mathcal{E}^{\text{img}}$ converts the input image $x^{\text{img}}$ into a sequence of visual tokens $\mathcal{V} = [v_i]_{i=1}^{N_v}$, while the text sequence $x^{\text{txt}}$ is mapped into a sequence of text tokens $\mathcal{T} = [t_i]_{i=1}^{N_t}$ using an embedding function $\mathcal{E}^{\text{txt}}$. The LLM $\mathcal{M_\text{LLM}}(\cdot|\theta)$ models the joint probability of the token sequence $\mathcal{U} = \{\mathcal{V},\mathcal{T}\}$, which is expressed as:
\begin{equation}
    P_\theta(R | \mathcal{U}) = \prod_{i=1}^{N_r} P_\theta(r_i | \{\mathcal{U}, r_{<i}\}),
\end{equation}
where $R = [r_i]_{i=1}^{N_r}$ is the text response sequence. The LVLM iteratively generates the next token $r_i$ based on $r_{<i}$. The optimization objective is to minimize the cross-entropy loss of the response $\mathcal{R}$.
% \begin{equation}
%     \mathcal{L}_{\text{VLM}} = \mathbb{E}_{R|\mathcal{U}}\left[-\log P_\theta(R | \mathcal{U})\right]
% \end{equation}
It is worth noting that most LVLMs adopt a design paradigm based on ViT, alignment adapters, and pre-trained LLMs\cite{liu2023llava,liu2024improved}, enabling quick adaptation to downstream tasks.


\noindent\textbf{VQGAN.}
VQGAN~\cite{esser2021taming} employs latent space compression and indexing mechanisms to effectively learn a complete discrete representation of images. VQGAN first maps the input image $x^{\text{img}}$ to a latent representation $z = \mathcal{E}(x)$ through a encoder $\mathcal{E}$. Then, the latent representation is quantized using a codebook $\mathcal{Z} = \{z_k\}_{k=1}^K$, generating a discrete index sequence $\mathcal{I} = [i_m]_{m=1}^N$, where $i_m \in \mathcal{Z}$ represents the quantized code index:
\begin{equation}
    \mathcal{I} = \text{Quantize}(z|\mathcal{Z}) = \arg\min_{z_k \in \mathcal{Z}} \| z - z_k \|_2.
\end{equation}
In our approach, the discrete index sequence $\mathcal{I}$ serves as a supervisory signal for the generation task, enabling the model to predict the index sequence $\hat{\mathcal{I}}$ from input conditions such as text or other modality signals.  
Finally, the predicted index sequence $\hat{\mathcal{I}}$ is upsampled by the VQGAN decoder $G$, generating the high-quality image $\hat{x}^\text{img} = G(\hat{\mathcal{I}})$.



\noindent\textbf{Low Rank Adaptation.} 
LoRA\cite{hu2021lora} effectively captures the characteristics of downstream tasks by introducing low-rank adapters. The core idea is to decompose the bypass weight matrix $\Delta W\in\mathbb{R}^{d^{\text{in}} \times d^{\text{out}}}$ into two low-rank matrices $ \{A \in \mathbb{R}^{d^{\text{in}} \times r}, B \in \mathbb{R}^{r \times d^{\text{out}}} \}$, where $ r \ll \min\{d^{\text{in}}, d^{\text{out}}\} $, significantly reducing learnable parameters. The output with the LoRA adapter for the input $x$ is then given by:
\begin{equation}
    h = x W_0 + \alpha x \Delta W/r = x W_0 + \alpha xAB/r,
\end{equation}
where matrix $ A $ is initialized with a Gaussian distribution, while the matrix $ B $ is initialized as a zero matrix. The scaling factor $ \alpha/r $ controls the impact of $ \Delta W $ on the model.

\section{HealthGPT}
\label{Method}


\subsection{Unified Autoregressive Generation.}  
% As shown in Figure~\ref{fig:architecture}, 
\ourmethod{} (Figure~\ref{fig:architecture}) utilizes a discrete token representation that covers both text and visual outputs, unifying visual comprehension and generation as an autoregressive task. 
For comprehension, $\mathcal{M}_\text{llm}$ receives the input joint sequence $\mathcal{U}$ and outputs a series of text token $\mathcal{R} = [r_1, r_2, \dots, r_{N_r}]$, where $r_i \in \mathcal{V}_{\text{txt}}$, and $\mathcal{V}_{\text{txt}}$ represents the LLM's vocabulary:
\begin{equation}
    P_\theta(\mathcal{R} \mid \mathcal{U}) = \prod_{i=1}^{N_r} P_\theta(r_i \mid \mathcal{U}, r_{<i}).
\end{equation}
For generation, $\mathcal{M}_\text{llm}$ first receives a special start token $\langle \text{START\_IMG} \rangle$, then generates a series of tokens corresponding to the VQGAN indices $\mathcal{I} = [i_1, i_2, \dots, i_{N_i}]$, where $i_j \in \mathcal{V}_{\text{vq}}$, and $\mathcal{V}_{\text{vq}}$ represents the index range of VQGAN. Upon completion of generation, the LLM outputs an end token $\langle \text{END\_IMG} \rangle$:
\begin{equation}
    P_\theta(\mathcal{I} \mid \mathcal{U}) = \prod_{j=1}^{N_i} P_\theta(i_j \mid \mathcal{U}, i_{<j}).
\end{equation}
Finally, the generated index sequence $\mathcal{I}$ is fed into the decoder $G$, which reconstructs the target image $\hat{x}^{\text{img}} = G(\mathcal{I})$.

\subsection{Hierarchical Visual Perception}  
Given the differences in visual perception between comprehension and generation tasks—where the former focuses on abstract semantics and the latter emphasizes complete semantics—we employ ViT to compress the image into discrete visual tokens at multiple hierarchical levels.
Specifically, the image is converted into a series of features $\{f_1, f_2, \dots, f_L\}$ as it passes through $L$ ViT blocks.

To address the needs of various tasks, the hidden states are divided into two types: (i) \textit{Concrete-grained features} $\mathcal{F}^{\text{Con}} = \{f_1, f_2, \dots, f_k\}, k < L$, derived from the shallower layers of ViT, containing sufficient global features, suitable for generation tasks; 
(ii) \textit{Abstract-grained features} $\mathcal{F}^{\text{Abs}} = \{f_{k+1}, f_{k+2}, \dots, f_L\}$, derived from the deeper layers of ViT, which contain abstract semantic information closer to the text space, suitable for comprehension tasks.

The task type $T$ (comprehension or generation) determines which set of features is selected as the input for the downstream large language model:
\begin{equation}
    \mathcal{F}^{\text{img}}_T =
    \begin{cases}
        \mathcal{F}^{\text{Con}}, & \text{if } T = \text{generation task} \\
        \mathcal{F}^{\text{Abs}}, & \text{if } T = \text{comprehension task}
    \end{cases}
\end{equation}
We integrate the image features $\mathcal{F}^{\text{img}}_T$ and text features $\mathcal{T}$ into a joint sequence through simple concatenation, which is then fed into the LLM $\mathcal{M}_{\text{llm}}$ for autoregressive generation.
% :
% \begin{equation}
%     \mathcal{R} = \mathcal{M}_{\text{llm}}(\mathcal{U}|\theta), \quad \mathcal{U} = [\mathcal{F}^{\text{img}}_T; \mathcal{T}]
% \end{equation}
\subsection{Heterogeneous Knowledge Adaptation}
We devise H-LoRA, which stores heterogeneous knowledge from comprehension and generation tasks in separate modules and dynamically routes to extract task-relevant knowledge from these modules. 
At the task level, for each task type $ T $, we dynamically assign a dedicated H-LoRA submodule $ \theta^T $, which is expressed as:
\begin{equation}
    \mathcal{R} = \mathcal{M}_\text{LLM}(\mathcal{U}|\theta, \theta^T), \quad \theta^T = \{A^T, B^T, \mathcal{R}^T_\text{outer}\}.
\end{equation}
At the feature level for a single task, H-LoRA integrates the idea of Mixture of Experts (MoE)~\cite{masoudnia2014mixture} and designs an efficient matrix merging and routing weight allocation mechanism, thus avoiding the significant computational delay introduced by matrix splitting in existing MoELoRA~\cite{luo2024moelora}. Specifically, we first merge the low-rank matrices (rank = r) of $ k $ LoRA experts into a unified matrix:
\begin{equation}
    \mathbf{A}^{\text{merged}}, \mathbf{B}^{\text{merged}} = \text{Concat}(\{A_i\}_1^k), \text{Concat}(\{B_i\}_1^k),
\end{equation}
where $ \mathbf{A}^{\text{merged}} \in \mathbb{R}^{d^\text{in} \times rk} $ and $ \mathbf{B}^{\text{merged}} \in \mathbb{R}^{rk \times d^\text{out}} $. The $k$-dimension routing layer generates expert weights $ \mathcal{W} \in \mathbb{R}^{\text{token\_num} \times k} $ based on the input hidden state $ x $, and these are expanded to $ \mathbb{R}^{\text{token\_num} \times rk} $ as follows:
\begin{equation}
    \mathcal{W}^\text{expanded} = \alpha k \mathcal{W} / r \otimes \mathbf{1}_r,
\end{equation}
where $ \otimes $ denotes the replication operation.
The overall output of H-LoRA is computed as:
\begin{equation}
    \mathcal{O}^\text{H-LoRA} = (x \mathbf{A}^{\text{merged}} \odot \mathcal{W}^\text{expanded}) \mathbf{B}^{\text{merged}},
\end{equation}
where $ \odot $ represents element-wise multiplication. Finally, the output of H-LoRA is added to the frozen pre-trained weights to produce the final output:
\begin{equation}
    \mathcal{O} = x W_0 + \mathcal{O}^\text{H-LoRA}.
\end{equation}
% In summary, H-LoRA is a task-based dynamic PEFT method that achieves high efficiency in single-task fine-tuning.

\subsection{Training Pipeline}

\begin{figure}[t]
    \centering
    \hspace{-4mm}
    \includegraphics[width=0.94\linewidth]{fig/data.pdf}
    \caption{Data statistics of \texttt{VL-Health}. }
    \label{fig:data}
\end{figure}
\noindent \textbf{1st Stage: Multi-modal Alignment.} 
In the first stage, we design separate visual adapters and H-LoRA submodules for medical unified tasks. For the medical comprehension task, we train abstract-grained visual adapters using high-quality image-text pairs to align visual embeddings with textual embeddings, thereby enabling the model to accurately describe medical visual content. During this process, the pre-trained LLM and its corresponding H-LoRA submodules remain frozen. In contrast, the medical generation task requires training concrete-grained adapters and H-LoRA submodules while keeping the LLM frozen. Meanwhile, we extend the textual vocabulary to include multimodal tokens, enabling the support of additional VQGAN vector quantization indices. The model trains on image-VQ pairs, endowing the pre-trained LLM with the capability for image reconstruction. This design ensures pixel-level consistency of pre- and post-LVLM. The processes establish the initial alignment between the LLM’s outputs and the visual inputs.

\noindent \textbf{2nd Stage: Heterogeneous H-LoRA Plugin Adaptation.}  
The submodules of H-LoRA share the word embedding layer and output head but may encounter issues such as bias and scale inconsistencies during training across different tasks. To ensure that the multiple H-LoRA plugins seamlessly interface with the LLMs and form a unified base, we fine-tune the word embedding layer and output head using a small amount of mixed data to maintain consistency in the model weights. Specifically, during this stage, all H-LoRA submodules for different tasks are kept frozen, with only the word embedding layer and output head being optimized. Through this stage, the model accumulates foundational knowledge for unified tasks by adapting H-LoRA plugins.

\begin{table*}[!t]
\centering
\caption{Comparison of \ourmethod{} with other LVLMs and unified multi-modal models on medical visual comprehension tasks. \textbf{Bold} and \underline{underlined} text indicates the best performance and second-best performance, respectively.}
\resizebox{\textwidth}{!}{
\begin{tabular}{c|lcc|cccccccc|c}
\toprule
\rowcolor[HTML]{E9F3FE} &  &  &  & \multicolumn{2}{c}{\textbf{VQA-RAD \textuparrow}} & \multicolumn{2}{c}{\textbf{SLAKE \textuparrow}} & \multicolumn{2}{c}{\textbf{PathVQA \textuparrow}} &  &  &  \\ 
\cline{5-10}
\rowcolor[HTML]{E9F3FE}\multirow{-2}{*}{\textbf{Type}} & \multirow{-2}{*}{\textbf{Model}} & \multirow{-2}{*}{\textbf{\# Params}} & \multirow{-2}{*}{\makecell{\textbf{Medical} \\ \textbf{LVLM}}} & \textbf{close} & \textbf{all} & \textbf{close} & \textbf{all} & \textbf{close} & \textbf{all} & \multirow{-2}{*}{\makecell{\textbf{MMMU} \\ \textbf{-Med}}\textuparrow} & \multirow{-2}{*}{\textbf{OMVQA}\textuparrow} & \multirow{-2}{*}{\textbf{Avg. \textuparrow}} \\ 
\midrule \midrule
\multirow{9}{*}{\textbf{Comp. Only}} 
& Med-Flamingo & 8.3B & \Large \ding{51} & 58.6 & 43.0 & 47.0 & 25.5 & 61.9 & 31.3 & 28.7 & 34.9 & 41.4 \\
& LLaVA-Med & 7B & \Large \ding{51} & 60.2 & 48.1 & 58.4 & 44.8 & 62.3 & 35.7 & 30.0 & 41.3 & 47.6 \\
& HuatuoGPT-Vision & 7B & \Large \ding{51} & 66.9 & 53.0 & 59.8 & 49.1 & 52.9 & 32.0 & 42.0 & 50.0 & 50.7 \\
& BLIP-2 & 6.7B & \Large \ding{55} & 43.4 & 36.8 & 41.6 & 35.3 & 48.5 & 28.8 & 27.3 & 26.9 & 36.1 \\
& LLaVA-v1.5 & 7B & \Large \ding{55} & 51.8 & 42.8 & 37.1 & 37.7 & 53.5 & 31.4 & 32.7 & 44.7 & 41.5 \\
& InstructBLIP & 7B & \Large \ding{55} & 61.0 & 44.8 & 66.8 & 43.3 & 56.0 & 32.3 & 25.3 & 29.0 & 44.8 \\
& Yi-VL & 6B & \Large \ding{55} & 52.6 & 42.1 & 52.4 & 38.4 & 54.9 & 30.9 & 38.0 & 50.2 & 44.9 \\
& InternVL2 & 8B & \Large \ding{55} & 64.9 & 49.0 & 66.6 & 50.1 & 60.0 & 31.9 & \underline{43.3} & 54.5 & 52.5\\
& Llama-3.2 & 11B & \Large \ding{55} & 68.9 & 45.5 & 72.4 & 52.1 & 62.8 & 33.6 & 39.3 & 63.2 & 54.7 \\
\midrule
\multirow{5}{*}{\textbf{Comp. \& Gen.}} 
& Show-o & 1.3B & \Large \ding{55} & 50.6 & 33.9 & 31.5 & 17.9 & 52.9 & 28.2 & 22.7 & 45.7 & 42.6 \\
& Unified-IO 2 & 7B & \Large \ding{55} & 46.2 & 32.6 & 35.9 & 21.9 & 52.5 & 27.0 & 25.3 & 33.0 & 33.8 \\
& Janus & 1.3B & \Large \ding{55} & 70.9 & 52.8 & 34.7 & 26.9 & 51.9 & 27.9 & 30.0 & 26.8 & 33.5 \\
& \cellcolor[HTML]{DAE0FB}HealthGPT-M3 & \cellcolor[HTML]{DAE0FB}3.8B & \cellcolor[HTML]{DAE0FB}\Large \ding{51} & \cellcolor[HTML]{DAE0FB}\underline{73.7} & \cellcolor[HTML]{DAE0FB}\underline{55.9} & \cellcolor[HTML]{DAE0FB}\underline{74.6} & \cellcolor[HTML]{DAE0FB}\underline{56.4} & \cellcolor[HTML]{DAE0FB}\underline{78.7} & \cellcolor[HTML]{DAE0FB}\underline{39.7} & \cellcolor[HTML]{DAE0FB}\underline{43.3} & \cellcolor[HTML]{DAE0FB}\underline{68.5} & \cellcolor[HTML]{DAE0FB}\underline{61.3} \\
& \cellcolor[HTML]{DAE0FB}HealthGPT-L14 & \cellcolor[HTML]{DAE0FB}14B & \cellcolor[HTML]{DAE0FB}\Large \ding{51} & \cellcolor[HTML]{DAE0FB}\textbf{77.7} & \cellcolor[HTML]{DAE0FB}\textbf{58.3} & \cellcolor[HTML]{DAE0FB}\textbf{76.4} & \cellcolor[HTML]{DAE0FB}\textbf{64.5} & \cellcolor[HTML]{DAE0FB}\textbf{85.9} & \cellcolor[HTML]{DAE0FB}\textbf{44.4} & \cellcolor[HTML]{DAE0FB}\textbf{49.2} & \cellcolor[HTML]{DAE0FB}\textbf{74.4} & \cellcolor[HTML]{DAE0FB}\textbf{66.4} \\
\bottomrule
\end{tabular}
}
\label{tab:results}
\end{table*}
\begin{table*}[ht]
    \centering
    \caption{The experimental results for the four modality conversion tasks.}
    \resizebox{\textwidth}{!}{
    \begin{tabular}{l|ccc|ccc|ccc|ccc}
        \toprule
        \rowcolor[HTML]{E9F3FE} & \multicolumn{3}{c}{\textbf{CT to MRI (Brain)}} & \multicolumn{3}{c}{\textbf{CT to MRI (Pelvis)}} & \multicolumn{3}{c}{\textbf{MRI to CT (Brain)}} & \multicolumn{3}{c}{\textbf{MRI to CT (Pelvis)}} \\
        \cline{2-13}
        \rowcolor[HTML]{E9F3FE}\multirow{-2}{*}{\textbf{Model}}& \textbf{SSIM $\uparrow$} & \textbf{PSNR $\uparrow$} & \textbf{MSE $\downarrow$} & \textbf{SSIM $\uparrow$} & \textbf{PSNR $\uparrow$} & \textbf{MSE $\downarrow$} & \textbf{SSIM $\uparrow$} & \textbf{PSNR $\uparrow$} & \textbf{MSE $\downarrow$} & \textbf{SSIM $\uparrow$} & \textbf{PSNR $\uparrow$} & \textbf{MSE $\downarrow$} \\
        \midrule \midrule
        pix2pix & 71.09 & 32.65 & 36.85 & 59.17 & 31.02 & 51.91 & 78.79 & 33.85 & 28.33 & 72.31 & 32.98 & 36.19 \\
        CycleGAN & 54.76 & 32.23 & 40.56 & 54.54 & 30.77 & 55.00 & 63.75 & 31.02 & 52.78 & 50.54 & 29.89 & 67.78 \\
        BBDM & {71.69} & {32.91} & {34.44} & 57.37 & 31.37 & 48.06 & \textbf{86.40} & 34.12 & 26.61 & {79.26} & 33.15 & 33.60 \\
        Vmanba & 69.54 & 32.67 & 36.42 & {63.01} & {31.47} & {46.99} & 79.63 & 34.12 & 26.49 & 77.45 & 33.53 & 31.85 \\
        DiffMa & 71.47 & 32.74 & 35.77 & 62.56 & 31.43 & 47.38 & 79.00 & {34.13} & {26.45} & 78.53 & {33.68} & {30.51} \\
        \rowcolor[HTML]{DAE0FB}HealthGPT-M3 & \underline{79.38} & \underline{33.03} & \underline{33.48} & \underline{71.81} & \underline{31.83} & \underline{43.45} & {85.06} & \textbf{34.40} & \textbf{25.49} & \underline{84.23} & \textbf{34.29} & \textbf{27.99} \\
        \rowcolor[HTML]{DAE0FB}HealthGPT-L14 & \textbf{79.73} & \textbf{33.10} & \textbf{32.96} & \textbf{71.92} & \textbf{31.87} & \textbf{43.09} & \underline{85.31} & \underline{34.29} & \underline{26.20} & \textbf{84.96} & \underline{34.14} & \underline{28.13} \\
        \bottomrule
    \end{tabular}
    }
    \label{tab:conversion}
\end{table*}

\noindent \textbf{3rd Stage: Visual Instruction Fine-Tuning.}  
In the third stage, we introduce additional task-specific data to further optimize the model and enhance its adaptability to downstream tasks such as medical visual comprehension (e.g., medical QA, medical dialogues, and report generation) or generation tasks (e.g., super-resolution, denoising, and modality conversion). Notably, by this stage, the word embedding layer and output head have been fine-tuned, only the H-LoRA modules and adapter modules need to be trained. This strategy significantly improves the model's adaptability and flexibility across different tasks.


\section{Experiment}
\label{s:experiment}

\subsection{Data Description}
We evaluate our method on FI~\cite{you2016building}, Twitter\_LDL~\cite{yang2017learning} and Artphoto~\cite{machajdik2010affective}.
FI is a public dataset built from Flickr and Instagram, with 23,308 images and eight emotion categories, namely \textit{amusement}, \textit{anger}, \textit{awe},  \textit{contentment}, \textit{disgust}, \textit{excitement},  \textit{fear}, and \textit{sadness}. 
% Since images in FI are all copyrighted by law, some images are corrupted now, so we remove these samples and retain 21,828 images.
% T4SA contains images from Twitter, which are classified into three categories: \textit{positive}, \textit{neutral}, and \textit{negative}. In this paper, we adopt the base version of B-T4SA, which contains 470,586 images and provides text descriptions of the corresponding tweets.
Twitter\_LDL contains 10,045 images from Twitter, with the same eight categories as the FI dataset.
% 。
For these two datasets, they are randomly split into 80\%
training and 20\% testing set.
Artphoto contains 806 artistic photos from the DeviantArt website, which we use to further evaluate the zero-shot capability of our model.
% on the small-scale dataset.
% We construct and publicly release the first image sentiment analysis dataset containing metadata.
% 。

% Based on these datasets, we are the first to construct and publicly release metadata-enhanced image sentiment analysis datasets. These datasets include scenes, tags, descriptions, and corresponding confidence scores, and are available at this link for future research purposes.


% 
\begin{table}[t]
\centering
% \begin{center}
\caption{Overall performance of different models on FI and Twitter\_LDL datasets.}
\label{tab:cap1}
% \resizebox{\linewidth}{!}
{
\begin{tabular}{l|c|c|c|c}
\hline
\multirow{2}{*}{\textbf{Model}} & \multicolumn{2}{c|}{\textbf{FI}}  & \multicolumn{2}{c}{\textbf{Twitter\_LDL}} \\ \cline{2-5} 
  & \textbf{Accuracy} & \textbf{F1} & \textbf{Accuracy} & \textbf{F1}  \\ \hline
% (\rownumber)~AlexNet~\cite{krizhevsky2017imagenet}  & 58.13\% & 56.35\%  & 56.24\%& 55.02\%  \\ 
% (\rownumber)~VGG16~\cite{simonyan2014very}  & 63.75\%& 63.08\%  & 59.34\%& 59.02\%  \\ 
(\rownumber)~ResNet101~\cite{he2016deep} & 66.16\%& 65.56\%  & 62.02\% & 61.34\%  \\ 
(\rownumber)~CDA~\cite{han2023boosting} & 66.71\%& 65.37\%  & 64.14\% & 62.85\%  \\ 
(\rownumber)~CECCN~\cite{ruan2024color} & 67.96\%& 66.74\%  & 64.59\%& 64.72\% \\ 
(\rownumber)~EmoVIT~\cite{xie2024emovit} & 68.09\%& 67.45\%  & 63.12\% & 61.97\%  \\ 
(\rownumber)~ComLDL~\cite{zhang2022compound} & 68.83\%& 67.28\%  & 65.29\% & 63.12\%  \\ 
(\rownumber)~WSDEN~\cite{li2023weakly} & 69.78\%& 69.61\%  & 67.04\% & 65.49\% \\ 
(\rownumber)~ECWA~\cite{deng2021emotion} & 70.87\%& 69.08\%  & 67.81\% & 66.87\%  \\ 
(\rownumber)~EECon~\cite{yang2023exploiting} & 71.13\%& 68.34\%  & 64.27\%& 63.16\%  \\ 
(\rownumber)~MAM~\cite{zhang2024affective} & 71.44\%  & 70.83\% & 67.18\%  & 65.01\%\\ 
(\rownumber)~TGCA-PVT~\cite{chen2024tgca}   & 73.05\%  & 71.46\% & 69.87\%  & 68.32\% \\ 
(\rownumber)~OEAN~\cite{zhang2024object}   & 73.40\%  & 72.63\% & 70.52\%  & 69.47\% \\ \hline
(\rownumber)~\shortname  & \textbf{79.48\%} & \textbf{79.22\%} & \textbf{74.12\%} & \textbf{73.09\%} \\ \hline
\end{tabular}
}
\vspace{-6mm}
% \end{center}
\end{table}
% 

\subsection{Experiment Setting}
% \subsubsection{Model Setting.}
% 
\textbf{Model Setting:}
For feature representation, we set $k=10$ to select object tags, and adopt clip-vit-base-patch32 as the pre-trained model for unified feature representation.
Moreover, we empirically set $(d_e, d_h, d_k, d_s) = (512, 128, 16, 64)$, and set the classification class $L$ to 8.

% 

\textbf{Training Setting:}
To initialize the model, we set all weights such as $\boldsymbol{W}$ following the truncated normal distribution, and use AdamW optimizer with the learning rate of $1 \times 10^{-4}$.
% warmup scheduler of cosine, warmup steps of 2000.
Furthermore, we set the batch size to 32 and the epoch of the training process to 200.
During the implementation, we utilize \textit{PyTorch} to build our entire model.
% , and our project codes are publicly available at https://github.com/zzmyrep/MESN.
% Our project codes as well as data are all publicly available on GitHub\footnote{https://github.com/zzmyrep/KBCEN}.
% Code is available at \href{https://github.com/zzmyrep/KBCEN}{https://github.com/zzmyrep/KBCEN}.

\textbf{Evaluation Metrics:}
Following~\cite{zhang2024affective, chen2024tgca, zhang2024object}, we adopt \textit{accuracy} and \textit{F1} as our evaluation metrics to measure the performance of different methods for image sentiment analysis. 



\subsection{Experiment Result}
% We compare our model against the following baselines: AlexNet~\cite{krizhevsky2017imagenet}, VGG16~\cite{simonyan2014very}, ResNet101~\cite{he2016deep}, CECCN~\cite{ruan2024color}, EmoVIT~\cite{xie2024emovit}, WSCNet~\cite{yang2018weakly}, ECWA~\cite{deng2021emotion}, EECon~\cite{yang2023exploiting}, MAM~\cite{zhang2024affective} and TGCA-PVT~\cite{chen2024tgca}, and the overall results are summarized in Table~\ref{tab:cap1}.
We compare our model against several baselines, and the overall results are summarized in Table~\ref{tab:cap1}.
We observe that our model achieves the best performance in both accuracy and F1 metrics, significantly outperforming the previous models. 
This superior performance is mainly attributed to our effective utilization of metadata to enhance image sentiment analysis, as well as the exceptional capability of the unified sentiment transformer framework we developed. These results strongly demonstrate that our proposed method can bring encouraging performance for image sentiment analysis.

\setcounter{magicrownumbers}{0} 
\begin{table}[t]
\begin{center}
\caption{Ablation study of~\shortname~on FI dataset.} 
% \vspace{1mm}
\label{tab:cap2}
\resizebox{.9\linewidth}{!}
{
\begin{tabular}{lcc}
  \hline
  \textbf{Model} & \textbf{Accuracy} & \textbf{F1} \\
  \hline
  (\rownumber)~Ours (w/o vision) & 65.72\% & 64.54\% \\
  (\rownumber)~Ours (w/o text description) & 74.05\% & 72.58\% \\
  (\rownumber)~Ours (w/o object tag) & 77.45\% & 76.84\% \\
  (\rownumber)~Ours (w/o scene tag) & 78.47\% & 78.21\% \\
  \hline
  (\rownumber)~Ours (w/o unified embedding) & 76.41\% & 76.23\% \\
  (\rownumber)~Ours (w/o adaptive learning) & 76.83\% & 76.56\% \\
  (\rownumber)~Ours (w/o cross-modal fusion) & 76.85\% & 76.49\% \\
  \hline
  (\rownumber)~Ours  & \textbf{79.48\%} & \textbf{79.22\%} \\
  \hline
\end{tabular}
}
\end{center}
\vspace{-5mm}
\end{table}


\begin{figure}[t]
\centering
% \vspace{-2mm}
\includegraphics[width=0.42\textwidth]{fig/2dvisual-linux4-paper2.pdf}
\caption{Visualization of feature distribution on eight categories before (left) and after (right) model processing.}
% 
\label{fig:visualization}
\vspace{-5mm}
\end{figure}

\subsection{Ablation Performance}
In this subsection, we conduct an ablation study to examine which component is really important for performance improvement. The results are reported in Table~\ref{tab:cap2}.

For information utilization, we observe a significant decline in model performance when visual features are removed. Additionally, the performance of \shortname~decreases when different metadata are removed separately, which means that text description, object tag, and scene tag are all critical for image sentiment analysis.
Recalling the model architecture, we separately remove transformer layers of the unified representation module, the adaptive learning module, and the cross-modal fusion module, replacing them with MLPs of the same parameter scale.
In this way, we can observe varying degrees of decline in model performance, indicating that these modules are indispensable for our model to achieve better performance.

\subsection{Visualization}
% 


% % 开始使用minipage进行左右排列
% \begin{minipage}[t]{0.45\textwidth}  % 子图1宽度为45%
%     \centering
%     \includegraphics[width=\textwidth]{2dvisual.pdf}  % 插入图片
%     \captionof{figure}{Visualization of feature distribution.}  % 使用captionof添加图片标题
%     \label{fig:visualization}
% \end{minipage}


% \begin{figure}[t]
% \centering
% \vspace{-2mm}
% \includegraphics[width=0.45\textwidth]{fig/2dvisual.pdf}
% \caption{Visualization of feature distribution.}
% \label{fig:visualization}
% % \vspace{-4mm}
% \end{figure}

% \begin{figure}[t]
% \centering
% \vspace{-2mm}
% \includegraphics[width=0.45\textwidth]{fig/2dvisual-linux3-paper.pdf}
% \caption{Visualization of feature distribution.}
% \label{fig:visualization}
% % \vspace{-4mm}
% \end{figure}



\begin{figure}[tbp]   
\vspace{-4mm}
  \centering            
  \subfloat[Depth of adaptive learning layers]   
  {
    \label{fig:subfig1}\includegraphics[width=0.22\textwidth]{fig/fig_sensitivity-a5}
  }
  \subfloat[Depth of fusion layers]
  {
    % \label{fig:subfig2}\includegraphics[width=0.22\textwidth]{fig/fig_sensitivity-b2}
    \label{fig:subfig2}\includegraphics[width=0.22\textwidth]{fig/fig_sensitivity-b2-num.pdf}
  }
  \caption{Sensitivity study of \shortname~on different depth. }   
  \label{fig:fig_sensitivity}  
\vspace{-2mm}
\end{figure}

% \begin{figure}[htbp]
% \centerline{\includegraphics{2dvisual.pdf}}
% \caption{Visualization of feature distribution.}
% \label{fig:visualization}
% \end{figure}

% In Fig.~\ref{fig:visualization}, we use t-SNE~\cite{van2008visualizing} to reduce the dimension of data features for visualization, Figure in left represents the metadata features before model processing, the features are obtained by embedding through the CLIP model, and figure in right shows the features of the data after model processing, it can be observed that after the model processing, the data with different label categories fall in different regions in the space, therefore, we can conclude that the Therefore, we can conclude that the model can effectively utilize the information contained in the metadata and use it to guide the model for classification.

In Fig.~\ref{fig:visualization}, we use t-SNE~\cite{van2008visualizing} to reduce the dimension of data features for visualization.
The left figure shows metadata features before being processed by our model (\textit{i.e.}, embedded by CLIP), while the right shows the distribution of features after being processed by our model.
We can observe that after the model processing, data with the same label are closer to each other, while others are farther away.
Therefore, it shows that the model can effectively utilize the information contained in the metadata and use it to guide the classification process.

\subsection{Sensitivity Analysis}
% 
In this subsection, we conduct a sensitivity analysis to figure out the effect of different depth settings of adaptive learning layers and fusion layers. 
% In this subsection, we conduct a sensitivity analysis to figure out the effect of different depth settings on the model. 
% Fig.~\ref{fig:fig_sensitivity} presents the effect of different depth settings of adaptive learning layers and fusion layers. 
Taking Fig.~\ref{fig:fig_sensitivity} (a) as an example, the model performance improves with increasing depth, reaching the best performance at a depth of 4.
% Taking Fig.~\ref{fig:fig_sensitivity} (a) as an example, the performance of \shortname~improves with the increase of depth at first, reaching the best performance at a depth of 4.
When the depth continues to increase, the accuracy decreases to varying degrees.
Similar results can be observed in Fig.~\ref{fig:fig_sensitivity} (b).
Therefore, we set their depths to 4 and 6 respectively to achieve the best results.

% Through our experiments, we can observe that the effect of modifying these hyperparameters on the results of the experiments is very weak, and the surface model is not sensitive to the hyperparameters.


\subsection{Zero-shot Capability}
% 

% (1)~GCH~\cite{2010Analyzing} & 21.78\% & (5)~RA-DLNet~\cite{2020A} & 34.01\% \\ \hline
% (2)~WSCNet~\cite{2019WSCNet}  & 30.25\% & (6)~CECCN~\cite{ruan2024color} & 43.83\% \\ \hline
% (3)~PCNN~\cite{2015Robust} & 31.68\%  & (7)~EmoVIT~\cite{xie2024emovit} & 44.90\% \\ \hline
% (4)~AR~\cite{2018Visual} & 32.67\% & (8)~Ours (Zero-shot) & 47.83\% \\ \hline


\begin{table}[t]
\centering
\caption{Zero-shot capability of \shortname.}
\label{tab:cap3}
\resizebox{1\linewidth}{!}
{
\begin{tabular}{lc|lc}
\hline
\textbf{Model} & \textbf{Accuracy} & \textbf{Model} & \textbf{Accuracy} \\ \hline
(1)~WSCNet~\cite{2019WSCNet}  & 30.25\% & (5)~MAM~\cite{zhang2024affective} & 39.56\%  \\ \hline
(2)~AR~\cite{2018Visual} & 32.67\% & (6)~CECCN~\cite{ruan2024color} & 43.83\% \\ \hline
(3)~RA-DLNet~\cite{2020A} & 34.01\%  & (7)~EmoVIT~\cite{xie2024emovit} & 44.90\% \\ \hline
(4)~CDA~\cite{han2023boosting} & 38.64\% & (8)~Ours (Zero-shot) & 47.83\% \\ \hline
\end{tabular}
}
\vspace{-5mm}
\end{table}

% We use the model trained on the FI dataset to test on the artphoto dataset to verify the model's generalization ability as well as robustness to other distributed datasets.
% We can observe that the MESN model shows strong competitiveness in terms of accuracy when compared to other trained models, which suggests that the model has a good generalization ability in the OOD task.

To validate the model's generalization ability and robustness to other distributed datasets, we directly test the model trained on the FI dataset, without training on Artphoto. 
% As observed in Table 3, compared to other models trained on Artphoto, we achieve highly competitive zero-shot performance, indicating that the model has good generalization ability in out-of-distribution tasks.
From Table~\ref{tab:cap3}, we can observe that compared with other models trained on Artphoto, we achieve competitive zero-shot performance, which shows that the model has good generalization ability in out-of-distribution tasks.


%%%%%%%%%%%%
%  E2E     %
%%%%%%%%%%%%


\section{Conclusion}
In this paper, we introduced Wi-Chat, the first LLM-powered Wi-Fi-based human activity recognition system that integrates the reasoning capabilities of large language models with the sensing potential of wireless signals. Our experimental results on a self-collected Wi-Fi CSI dataset demonstrate the promising potential of LLMs in enabling zero-shot Wi-Fi sensing. These findings suggest a new paradigm for human activity recognition that does not rely on extensive labeled data. We hope future research will build upon this direction, further exploring the applications of LLMs in signal processing domains such as IoT, mobile sensing, and radar-based systems.

\section*{Limitations}
While our work represents the first attempt to leverage LLMs for processing Wi-Fi signals, it is a preliminary study focused on a relatively simple task: Wi-Fi-based human activity recognition. This choice allows us to explore the feasibility of LLMs in wireless sensing but also comes with certain limitations.

Our approach primarily evaluates zero-shot performance, which, while promising, may still lag behind traditional supervised learning methods in highly complex or fine-grained recognition tasks. Besides, our study is limited to a controlled environment with a self-collected dataset, and the generalizability of LLMs to diverse real-world scenarios with varying Wi-Fi conditions, environmental interference, and device heterogeneity remains an open question.

Additionally, we have yet to explore the full potential of LLMs in more advanced Wi-Fi sensing applications, such as fine-grained gesture recognition, occupancy detection, and passive health monitoring. Future work should investigate the scalability of LLM-based approaches, their robustness to domain shifts, and their integration with multimodal sensing techniques in broader IoT applications.


% Bibliography entries for the entire Anthology, followed by custom entries
%\bibliography{anthology,custom}
% Custom bibliography entries only
\bibliography{main}
\newpage
\appendix

\section{Experiment prompts}
\label{sec:prompt}
The prompts used in the LLM experiments are shown in the following Table~\ref{tab:prompts}.

\definecolor{titlecolor}{rgb}{0.9, 0.5, 0.1}
\definecolor{anscolor}{rgb}{0.2, 0.5, 0.8}
\definecolor{labelcolor}{HTML}{48a07e}
\begin{table*}[h]
	\centering
	
 % \vspace{-0.2cm}
	
	\begin{center}
		\begin{tikzpicture}[
				chatbox_inner/.style={rectangle, rounded corners, opacity=0, text opacity=1, font=\sffamily\scriptsize, text width=5in, text height=9pt, inner xsep=6pt, inner ysep=6pt},
				chatbox_prompt_inner/.style={chatbox_inner, align=flush left, xshift=0pt, text height=11pt},
				chatbox_user_inner/.style={chatbox_inner, align=flush left, xshift=0pt},
				chatbox_gpt_inner/.style={chatbox_inner, align=flush left, xshift=0pt},
				chatbox/.style={chatbox_inner, draw=black!25, fill=gray!7, opacity=1, text opacity=0},
				chatbox_prompt/.style={chatbox, align=flush left, fill=gray!1.5, draw=black!30, text height=10pt},
				chatbox_user/.style={chatbox, align=flush left},
				chatbox_gpt/.style={chatbox, align=flush left},
				chatbox2/.style={chatbox_gpt, fill=green!25},
				chatbox3/.style={chatbox_gpt, fill=red!20, draw=black!20},
				chatbox4/.style={chatbox_gpt, fill=yellow!30},
				labelbox/.style={rectangle, rounded corners, draw=black!50, font=\sffamily\scriptsize\bfseries, fill=gray!5, inner sep=3pt},
			]
											
			\node[chatbox_user] (q1) {
				\textbf{System prompt}
				\newline
				\newline
				You are a helpful and precise assistant for segmenting and labeling sentences. We would like to request your help on curating a dataset for entity-level hallucination detection.
				\newline \newline
                We will give you a machine generated biography and a list of checked facts about the biography. Each fact consists of a sentence and a label (True/False). Please do the following process. First, breaking down the biography into words. Second, by referring to the provided list of facts, merging some broken down words in the previous step to form meaningful entities. For example, ``strategic thinking'' should be one entity instead of two. Third, according to the labels in the list of facts, labeling each entity as True or False. Specifically, for facts that share a similar sentence structure (\eg, \textit{``He was born on Mach 9, 1941.''} (\texttt{True}) and \textit{``He was born in Ramos Mejia.''} (\texttt{False})), please first assign labels to entities that differ across atomic facts. For example, first labeling ``Mach 9, 1941'' (\texttt{True}) and ``Ramos Mejia'' (\texttt{False}) in the above case. For those entities that are the same across atomic facts (\eg, ``was born'') or are neutral (\eg, ``he,'' ``in,'' and ``on''), please label them as \texttt{True}. For the cases that there is no atomic fact that shares the same sentence structure, please identify the most informative entities in the sentence and label them with the same label as the atomic fact while treating the rest of the entities as \texttt{True}. In the end, output the entities and labels in the following format:
                \begin{itemize}[nosep]
                    \item Entity 1 (Label 1)
                    \item Entity 2 (Label 2)
                    \item ...
                    \item Entity N (Label N)
                \end{itemize}
                % \newline \newline
                Here are two examples:
                \newline\newline
                \textbf{[Example 1]}
                \newline
                [The start of the biography]
                \newline
                \textcolor{titlecolor}{Marianne McAndrew is an American actress and singer, born on November 21, 1942, in Cleveland, Ohio. She began her acting career in the late 1960s, appearing in various television shows and films.}
                \newline
                [The end of the biography]
                \newline \newline
                [The start of the list of checked facts]
                \newline
                \textcolor{anscolor}{[Marianne McAndrew is an American. (False); Marianne McAndrew is an actress. (True); Marianne McAndrew is a singer. (False); Marianne McAndrew was born on November 21, 1942. (False); Marianne McAndrew was born in Cleveland, Ohio. (False); She began her acting career in the late 1960s. (True); She has appeared in various television shows. (True); She has appeared in various films. (True)]}
                \newline
                [The end of the list of checked facts]
                \newline \newline
                [The start of the ideal output]
                \newline
                \textcolor{labelcolor}{[Marianne McAndrew (True); is (True); an (True); American (False); actress (True); and (True); singer (False); , (True); born (True); on (True); November 21, 1942 (False); , (True); in (True); Cleveland, Ohio (False); . (True); She (True); began (True); her (True); acting career (True); in (True); the late 1960s (True); , (True); appearing (True); in (True); various (True); television shows (True); and (True); films (True); . (True)]}
                \newline
                [The end of the ideal output]
				\newline \newline
                \textbf{[Example 2]}
                \newline
                [The start of the biography]
                \newline
                \textcolor{titlecolor}{Doug Sheehan is an American actor who was born on April 27, 1949, in Santa Monica, California. He is best known for his roles in soap operas, including his portrayal of Joe Kelly on ``General Hospital'' and Ben Gibson on ``Knots Landing.''}
                \newline
                [The end of the biography]
                \newline \newline
                [The start of the list of checked facts]
                \newline
                \textcolor{anscolor}{[Doug Sheehan is an American. (True); Doug Sheehan is an actor. (True); Doug Sheehan was born on April 27, 1949. (True); Doug Sheehan was born in Santa Monica, California. (False); He is best known for his roles in soap operas. (True); He portrayed Joe Kelly. (True); Joe Kelly was in General Hospital. (True); General Hospital is a soap opera. (True); He portrayed Ben Gibson. (True); Ben Gibson was in Knots Landing. (True); Knots Landing is a soap opera. (True)]}
                \newline
                [The end of the list of checked facts]
                \newline \newline
                [The start of the ideal output]
                \newline
                \textcolor{labelcolor}{[Doug Sheehan (True); is (True); an (True); American (True); actor (True); who (True); was born (True); on (True); April 27, 1949 (True); in (True); Santa Monica, California (False); . (True); He (True); is (True); best known (True); for (True); his roles in soap operas (True); , (True); including (True); in (True); his portrayal (True); of (True); Joe Kelly (True); on (True); ``General Hospital'' (True); and (True); Ben Gibson (True); on (True); ``Knots Landing.'' (True)]}
                \newline
                [The end of the ideal output]
				\newline \newline
				\textbf{User prompt}
				\newline
				\newline
				[The start of the biography]
				\newline
				\textcolor{magenta}{\texttt{\{BIOGRAPHY\}}}
				\newline
				[The ebd of the biography]
				\newline \newline
				[The start of the list of checked facts]
				\newline
				\textcolor{magenta}{\texttt{\{LIST OF CHECKED FACTS\}}}
				\newline
				[The end of the list of checked facts]
			};
			\node[chatbox_user_inner] (q1_text) at (q1) {
				\textbf{System prompt}
				\newline
				\newline
				You are a helpful and precise assistant for segmenting and labeling sentences. We would like to request your help on curating a dataset for entity-level hallucination detection.
				\newline \newline
                We will give you a machine generated biography and a list of checked facts about the biography. Each fact consists of a sentence and a label (True/False). Please do the following process. First, breaking down the biography into words. Second, by referring to the provided list of facts, merging some broken down words in the previous step to form meaningful entities. For example, ``strategic thinking'' should be one entity instead of two. Third, according to the labels in the list of facts, labeling each entity as True or False. Specifically, for facts that share a similar sentence structure (\eg, \textit{``He was born on Mach 9, 1941.''} (\texttt{True}) and \textit{``He was born in Ramos Mejia.''} (\texttt{False})), please first assign labels to entities that differ across atomic facts. For example, first labeling ``Mach 9, 1941'' (\texttt{True}) and ``Ramos Mejia'' (\texttt{False}) in the above case. For those entities that are the same across atomic facts (\eg, ``was born'') or are neutral (\eg, ``he,'' ``in,'' and ``on''), please label them as \texttt{True}. For the cases that there is no atomic fact that shares the same sentence structure, please identify the most informative entities in the sentence and label them with the same label as the atomic fact while treating the rest of the entities as \texttt{True}. In the end, output the entities and labels in the following format:
                \begin{itemize}[nosep]
                    \item Entity 1 (Label 1)
                    \item Entity 2 (Label 2)
                    \item ...
                    \item Entity N (Label N)
                \end{itemize}
                % \newline \newline
                Here are two examples:
                \newline\newline
                \textbf{[Example 1]}
                \newline
                [The start of the biography]
                \newline
                \textcolor{titlecolor}{Marianne McAndrew is an American actress and singer, born on November 21, 1942, in Cleveland, Ohio. She began her acting career in the late 1960s, appearing in various television shows and films.}
                \newline
                [The end of the biography]
                \newline \newline
                [The start of the list of checked facts]
                \newline
                \textcolor{anscolor}{[Marianne McAndrew is an American. (False); Marianne McAndrew is an actress. (True); Marianne McAndrew is a singer. (False); Marianne McAndrew was born on November 21, 1942. (False); Marianne McAndrew was born in Cleveland, Ohio. (False); She began her acting career in the late 1960s. (True); She has appeared in various television shows. (True); She has appeared in various films. (True)]}
                \newline
                [The end of the list of checked facts]
                \newline \newline
                [The start of the ideal output]
                \newline
                \textcolor{labelcolor}{[Marianne McAndrew (True); is (True); an (True); American (False); actress (True); and (True); singer (False); , (True); born (True); on (True); November 21, 1942 (False); , (True); in (True); Cleveland, Ohio (False); . (True); She (True); began (True); her (True); acting career (True); in (True); the late 1960s (True); , (True); appearing (True); in (True); various (True); television shows (True); and (True); films (True); . (True)]}
                \newline
                [The end of the ideal output]
				\newline \newline
                \textbf{[Example 2]}
                \newline
                [The start of the biography]
                \newline
                \textcolor{titlecolor}{Doug Sheehan is an American actor who was born on April 27, 1949, in Santa Monica, California. He is best known for his roles in soap operas, including his portrayal of Joe Kelly on ``General Hospital'' and Ben Gibson on ``Knots Landing.''}
                \newline
                [The end of the biography]
                \newline \newline
                [The start of the list of checked facts]
                \newline
                \textcolor{anscolor}{[Doug Sheehan is an American. (True); Doug Sheehan is an actor. (True); Doug Sheehan was born on April 27, 1949. (True); Doug Sheehan was born in Santa Monica, California. (False); He is best known for his roles in soap operas. (True); He portrayed Joe Kelly. (True); Joe Kelly was in General Hospital. (True); General Hospital is a soap opera. (True); He portrayed Ben Gibson. (True); Ben Gibson was in Knots Landing. (True); Knots Landing is a soap opera. (True)]}
                \newline
                [The end of the list of checked facts]
                \newline \newline
                [The start of the ideal output]
                \newline
                \textcolor{labelcolor}{[Doug Sheehan (True); is (True); an (True); American (True); actor (True); who (True); was born (True); on (True); April 27, 1949 (True); in (True); Santa Monica, California (False); . (True); He (True); is (True); best known (True); for (True); his roles in soap operas (True); , (True); including (True); in (True); his portrayal (True); of (True); Joe Kelly (True); on (True); ``General Hospital'' (True); and (True); Ben Gibson (True); on (True); ``Knots Landing.'' (True)]}
                \newline
                [The end of the ideal output]
				\newline \newline
				\textbf{User prompt}
				\newline
				\newline
				[The start of the biography]
				\newline
				\textcolor{magenta}{\texttt{\{BIOGRAPHY\}}}
				\newline
				[The ebd of the biography]
				\newline \newline
				[The start of the list of checked facts]
				\newline
				\textcolor{magenta}{\texttt{\{LIST OF CHECKED FACTS\}}}
				\newline
				[The end of the list of checked facts]
			};
		\end{tikzpicture}
        \caption{GPT-4o prompt for labeling hallucinated entities.}\label{tb:gpt-4-prompt}
	\end{center}
\vspace{-0cm}
\end{table*}
% \section{Full Experiment Results}
% \begin{table*}[th]
    \centering
    \small
    \caption{Classification Results}
    \begin{tabular}{lcccc}
        \toprule
        \textbf{Method} & \textbf{Accuracy} & \textbf{Precision} & \textbf{Recall} & \textbf{F1-score} \\
        \midrule
        \multicolumn{5}{c}{\textbf{Zero Shot}} \\
                Zero-shot E-eyes & 0.26 & 0.26 & 0.27 & 0.26 \\
        Zero-shot CARM & 0.24 & 0.24 & 0.24 & 0.24 \\
                Zero-shot SVM & 0.27 & 0.28 & 0.28 & 0.27 \\
        Zero-shot CNN & 0.23 & 0.24 & 0.23 & 0.23 \\
        Zero-shot RNN & 0.26 & 0.26 & 0.26 & 0.26 \\
DeepSeek-0shot & 0.54 & 0.61 & 0.54 & 0.52 \\
DeepSeek-0shot-COT & 0.33 & 0.24 & 0.33 & 0.23 \\
DeepSeek-0shot-Knowledge & 0.45 & 0.46 & 0.45 & 0.44 \\
Gemma2-0shot & 0.35 & 0.22 & 0.38 & 0.27 \\
Gemma2-0shot-COT & 0.36 & 0.22 & 0.36 & 0.27 \\
Gemma2-0shot-Knowledge & 0.32 & 0.18 & 0.34 & 0.20 \\
GPT-4o-mini-0shot & 0.48 & 0.53 & 0.48 & 0.41 \\
GPT-4o-mini-0shot-COT & 0.33 & 0.50 & 0.33 & 0.38 \\
GPT-4o-mini-0shot-Knowledge & 0.49 & 0.31 & 0.49 & 0.36 \\
GPT-4o-0shot & 0.62 & 0.62 & 0.47 & 0.42 \\
GPT-4o-0shot-COT & 0.29 & 0.45 & 0.29 & 0.21 \\
GPT-4o-0shot-Knowledge & 0.44 & 0.52 & 0.44 & 0.39 \\
LLaMA-0shot & 0.32 & 0.25 & 0.32 & 0.24 \\
LLaMA-0shot-COT & 0.12 & 0.25 & 0.12 & 0.09 \\
LLaMA-0shot-Knowledge & 0.32 & 0.25 & 0.32 & 0.28 \\
Mistral-0shot & 0.19 & 0.23 & 0.19 & 0.10 \\
Mistral-0shot-Knowledge & 0.21 & 0.40 & 0.21 & 0.11 \\
        \midrule
        \multicolumn{5}{c}{\textbf{4 Shot}} \\
GPT-4o-mini-4shot & 0.58 & 0.59 & 0.58 & 0.53 \\
GPT-4o-mini-4shot-COT & 0.57 & 0.53 & 0.57 & 0.50 \\
GPT-4o-mini-4shot-Knowledge & 0.56 & 0.51 & 0.56 & 0.47 \\
GPT-4o-4shot & 0.77 & 0.84 & 0.77 & 0.73 \\
GPT-4o-4shot-COT & 0.63 & 0.76 & 0.63 & 0.53 \\
GPT-4o-4shot-Knowledge & 0.72 & 0.82 & 0.71 & 0.66 \\
LLaMA-4shot & 0.29 & 0.24 & 0.29 & 0.21 \\
LLaMA-4shot-COT & 0.20 & 0.30 & 0.20 & 0.13 \\
LLaMA-4shot-Knowledge & 0.15 & 0.23 & 0.13 & 0.13 \\
Mistral-4shot & 0.02 & 0.02 & 0.02 & 0.02 \\
Mistral-4shot-Knowledge & 0.21 & 0.27 & 0.21 & 0.20 \\
        \midrule
        
        \multicolumn{5}{c}{\textbf{Suprevised}} \\
        SVM & 0.94 & 0.92 & 0.91 & 0.91 \\
        CNN & 0.98 & 0.98 & 0.97 & 0.97 \\
        RNN & 0.99 & 0.99 & 0.99 & 0.99 \\
        % \midrule
        % \multicolumn{5}{c}{\textbf{Conventional Wi-Fi-based Human Activity Recognition Systems}} \\
        E-eyes & 1.00 & 1.00 & 1.00 & 1.00 \\
        CARM & 0.98 & 0.98 & 0.98 & 0.98 \\
\midrule
 \multicolumn{5}{c}{\textbf{Vision Models}} \\
           Zero-shot SVM & 0.26 & 0.25 & 0.25 & 0.25 \\
        Zero-shot CNN & 0.26 & 0.25 & 0.26 & 0.26 \\
        Zero-shot RNN & 0.28 & 0.28 & 0.29 & 0.28 \\
        SVM & 0.99 & 0.99 & 0.99 & 0.99 \\
        CNN & 0.98 & 0.99 & 0.98 & 0.98 \\
        RNN & 0.98 & 0.99 & 0.98 & 0.98 \\
GPT-4o-mini-Vision & 0.84 & 0.85 & 0.84 & 0.84 \\
GPT-4o-mini-Vision-COT & 0.90 & 0.91 & 0.90 & 0.90 \\
GPT-4o-Vision & 0.74 & 0.82 & 0.74 & 0.73 \\
GPT-4o-Vision-COT & 0.70 & 0.83 & 0.70 & 0.68 \\
LLaMA-Vision & 0.20 & 0.23 & 0.20 & 0.09 \\
LLaMA-Vision-Knowledge & 0.22 & 0.05 & 0.22 & 0.08 \\

        \bottomrule
    \end{tabular}
    \label{full}
\end{table*}




\end{document}
 



% \section*{Author contributions statement}
% H.W.F. conducted the experiment(s), H.W.F. and S.C. and S.C. conceived the ideas and analysed the results.  All authors reviewed the manuscript. 

% \section*{Additional information}
% Authors declare no conflict of interest.

\end{document}