\section{Experiments}
\vspace{-0.5em}
\label{experiment}
\subsection{Setup}
\vspace{-0.5em}
\textbf{Datasets.}
We evaluate four real-world datasets: Amazon stock, Google stock \citep{cam2018stock}, electricity demand \citep{harries1999splice} and temperature in Delhi \citep{sumanth2017climate}. Besides, we evaluate the synthetic dataset under changepoint setting. In the subsequent sections, we will provide a detailed introduction to each of these datasets.


\textbf{Base predictors.}
We consider three diverse types of base predictors.

\begin{itemize}
    \item Prophet \citep{taylor2018forecasting}: As a Bayesian additive model, Prophet predicts the value $\hat{Y}_t$ as a function of time $t$, expressed as $\hat{Y}_t = g(t) + s(t) + h(t) + \epsilon_t$, where $g(t)$ models the overall trend, $s(t)$ accounts for periodic seasonal effects, $h(t)$ captures holiday effects, and $\epsilon_t$ represents the noise assumed to follow a normal distribution.
    \item AR (AutoRegressive Model): As a classic model, AR is defined as $\hat{Y}_t = \phi_1 Y_{t-1} +\phi_2 Y_{t-2} + ... + \phi_p Y_{t-p} + \epsilon_t$, where $\phi_1, \phi_2,..., \phi_p$ are the parameters, and $p$ is the order of AR model. Following \cite{pid_angelopoulos2024conformal}, we set $p=3$.
    \item Theta \citep{assimakopoulos2000theta}: As a decomposition-based forecasting approach, Theta model modify the curvature of a time series by applying a coefficient $\theta$ to its second differences. We typically use $\theta=0$ (the long-term trend) and $\theta=2$ (the short-term dynamics) to decompose the original series into two components.
\end{itemize}

\textbf{Baselines.}
We compare with four state-of-the-art methods: ACI \citep{aci_gibbs2021adaptive}, OGD, SF-OGD \citep{bhatnagar2023improved}, decay-OGD  \citep{angelopoulos2024online}, PID \citep{pid_angelopoulos2024conformal}. A detailed description of existing methods can be found in \Cref{More details on existing methods}.

\textbf{General implement.}
We choose target coverage $1-\alpha=90\%$ and construct asymmetric prediction sets using two-side quantile scores under $\alpha/2$ respectively. For each baseline, we select the most appropriate range of learning rates $\eta$ for the respective datasets and present the best results in the tables. For sets, all baselines will output asymmetric sets $[\hat{Y}_t-q^l_t, \hat{Y}_t+q^u_t]$ with upper score $q^u_t$ and lower score $q^l_t$ under half of the coverage level $\alpha/2$ respectively.

\textbf{Other implement.}
For EQ term, we set $f(x)=\frac{1}{1+\exp{(-cx))}}$ as Sigmoid function and $c=1$. For ECI-cutoff, we set $h=1$ and $h_t = h \cdot (\max\{s_{t-w+1}, \dots, s_t\} - \min\{s_{t-w+1}, \dots, s_t\})$. For ECI-integral, we set weights $w_i=\frac{0.95^{t-i}}{\sum_{j=1}^t0.95^{t-j}}$ for $1\leq i \leq t$. Specifically, PID, ECI, and its variants use adaptive learning rates $\eta_t = \eta \cdot (\max\{s_{t-w+1}, \dots, s_t\} - \min\{s_{t-w+1}, \dots, s_t\})$, where $w$ is window length. Unless necessary changes are made, the settings for all baselines adhere to the original papers and open-source codes.

\textbf{Overview of experimental results.}
We have conducted extensive experiments, including stock data in \Cref{results in finance domain}, electricity data in \Cref{results in energy domain}, climate data in \Cref{results in climate domain}, synthetic data in \Cref{Results in synthetic data}, and ablation study of hyperparameters in \Cref{Ablation study of hyperparameters}. Experimental results with Transformer as the base model can be found in \Cref{Experimental results with Transformer}. More comprehensive experimental results and discussion on the scorecaster and learning rates can be found in \Cref{More details on experiment}. Our code is available at \url{https://github.com/creator-xi/Error-quantified-Conformal-Inference}.


\subsection{Results in finance domain}
\vspace{-0.5em}
\label{results in finance domain}
We consider the uncertainty problem of forecasting stock prices, including Amazon (AMZN) and Google (GOOGL), collected over 9 years (from January 1, 2006 to December 31, 2014). Models will forecast the daily opening price of each of Amazon and Google stock on a log scale. 

The quantitative results are shown in \Cref{table amazon} and \Cref{table google}. For ACI, the occurrence of infinite sets is too frequent, due to updating $\alpha_t$ and adopting the $\alpha_t$-quantile of past scores as $q_t$. This implies that ACI may tend to conservatively expand the prediction sets in the face of more complex or volatile data to ensure high coverage rates. However, such a strategy is not always ideal in practical applications, as overly broad sets can reduce the precision and utility of the predictions. For OGD and SF-OGD, they achieve a good balance between coverage rate and set width to some extent. However, their performance overly relies on the selection of learning rate and may fail under many learning rate settings. For PID, its scorecasting can be seen as helping compensate for the base predictors' predictive accuracy. Thus, it can lead to improvements in the worse base predictor case (such as Prophet). However, in the case of better base predictors, it will instead widen the length of the prediction set. 

\begin{table}[ht]
\caption{The experimental results in the Amazon stock dataset with nominal level $\alpha = 10\%$. The best (shortest) width results are indicated with bold text.}
\label{table amazon}
\setlength{\tabcolsep}{1.25mm} % 调整列间距
\renewcommand{\arraystretch}{1.15} % 调整行间距
\begin{center}
\small
\begin{tabular}{c|ccc|ccc|ccc}
\hline
            & \multicolumn{3}{c|}{Prophet Model}            & \multicolumn{3}{c|}{AR Model}                  & \multicolumn{3}{c}{Theta Model}                \\
Method &
  \begin{tabular}[c]{@{}c@{}}Coverage\\ ( \%)\end{tabular} &
  \begin{tabular}[c]{@{}c@{}}Average\\ width\end{tabular} &
  \begin{tabular}[c]{@{}c@{}}Median\\ width\end{tabular} &
  \begin{tabular}[c]{@{}c@{}}Coverage\\ ( \%)\end{tabular} &
  \begin{tabular}[c]{@{}c@{}}Average\\ width\end{tabular} &
  \begin{tabular}[c]{@{}c@{}}Median\\ width\end{tabular} &
  \begin{tabular}[c]{@{}c@{}}Coverage\\ ( \%)\end{tabular} &
  \begin{tabular}[c]{@{}c@{}}Average\\ width\end{tabular} &
  \begin{tabular}[c]{@{}c@{}}Median\\ width\end{tabular} \\ \hline
ACI         & 90.2 & $\infty$   & 46.97  & 89.8 & $\infty$   & 13.77  & 89.7 & $\infty$   & 12.31  \\
OGD         & 89.6 & 55.15 & 30.00  & 89.9 & 19.10 & 15.00  & 89.8 & 18.07 & 14.50  \\
SF-OGD      & 89.5 & 61.47 & 31.75  & 89.9 & 24.44 & 21.05  & 90.0 & 23.88 & 21.14  \\
decay-OGD    & 89.9 & 97.22          & 36.2           & 89.7 & 20.23          & 14.01          & 89.2 & 17.49          & 13.46          \\
PID          & 89.8 & 52.56          & 39.09          & 89.6 & 59.22          & 37.93          & 89.5 & 61.19          & 40.20          \\
ECI          & 90.1 & 47.00    & 34.84 & 89.5 & 17.12 & 12.73          & 89.7 & 17.46 & 12.49 \\
ECI-cutoff   & 89.7 & 43.46          & \textbf{29.98} & 89.3 & \textbf{16.91} & 12.63 & 89.6 & \textbf{17.19} & 12.48          \\
ECI-integral & 89.8 & \textbf{42.01} & 30.02          & 89.5 & 16.99          & \textbf{12.62} & 89.6 & 17.20          & \textbf{12.46} \\ \hline
\end{tabular}
\end{center}
\end{table}

\begin{table}[ht]
\caption{The experimental results in the Google stock dataset with nominal level $\alpha = 10\%$.}
\label{table google}
\setlength{\tabcolsep}{1.25mm} % 调整列间距
\renewcommand{\arraystretch}{1.15} % 调整行间距
\begin{center}
\small
\begin{tabular}{c|ccc|ccc|ccc}
\hline
            & \multicolumn{3}{c|}{Prophet Model}            & \multicolumn{3}{c|}{AR Model}                  & \multicolumn{3}{c}{Theta Model}                \\
Method &
  \begin{tabular}[c]{@{}c@{}}Coverage\\ ( \%)\end{tabular} &
  \begin{tabular}[c]{@{}c@{}}Average\\ width\end{tabular} &
  \begin{tabular}[c]{@{}c@{}}Median\\ width\end{tabular} &
  \begin{tabular}[c]{@{}c@{}}Coverage\\ ( \%)\end{tabular} &
  \begin{tabular}[c]{@{}c@{}}Average\\ width\end{tabular} &
  \begin{tabular}[c]{@{}c@{}}Median\\ width\end{tabular} &
  \begin{tabular}[c]{@{}c@{}}Coverage\\ ( \%)\end{tabular} &
  \begin{tabular}[c]{@{}c@{}}Average\\ width\end{tabular} &
  \begin{tabular}[c]{@{}c@{}}Median\\ width\end{tabular} \\ \hline
ACI          & 90.0 & $\infty$   & 66.83  & 89.8 & $\infty$   & 18.64  & 90.5 & $\infty$   & 32.78  \\
OGD          & 89.7 & 57.60 & 46.00  & 90.7 & 33.76 & 23.00  & 89.9 & 31.49 & 29.50  \\
SF-OGD       & 89.6 & 58.92 & 47.78  & 89.9 & 28.31 & 24.42  & 90.0 & 34.04 & 31.48  \\
decay-OGD             & 89.9 & 77.23          & 50.18          & 90.2 & 46.53          & 26.77          & 90.2 & 55.32          & 33.71          \\
PID                   & 90.1 & 57.47          & 48.44          & 89.9 & 64.88          & 54.07          & 89.9 & 63.58          & 54.05          \\
ECI                   & 89.9 & 56.06          & 46.96          & 89.7 & 19.95          & \textbf{17.19} & 89.6 & 30.92          & 29.53          \\
ECI-cutoff            & 89.8 & 53.12          & 44.36          & 89.7 & \textbf{19.84} & 17.63          & 89.6 & 30.71          & 28.11          \\
ECI-integral          & 89.8 & \textbf{52.36} & \textbf{43.28} & 89.7 & 19.93          & 17.31          & 89.6 & \textbf{30.42} & \textbf{28.02}  \\ \hline
\end{tabular}
\end{center}
\vspace{-1em}
\end{table}


As for ECI, its performance is outstanding, especially in the control of set width. Compared to existing methods, ECI can provide shorter and more accurate prediction sets with little loss of coverage. ECI-cutoff extends ECI by introducing a truncation threshold to further reduce the redundancy of prediction sets. The experimental data also show that ECI-cutoff achieves the shortest set width of all methods in general. ECI-integral extends ECI by integrating the information from longer past time, leading to a better balance between coverage rate and set width.

\Cref{figure amazon prophet coverage} shows the coverage results on Amazon stock dataset with Prophet as the base predictor. Throughout the entire period, the more effectively a method maintains the nominal level (here is $90\%$), the more valid the method is. We can see ACI generally having larger oscillations, and OGD, and SF-OGD become increasingly oscillatory over time. \Cref{figure amazon prophet sets} compares the prediction sets of the variants of ECI with those of other methods. Consistent with the quantitative results, the variants of ECI have the shortest prediction sets.

\vspace{-0.9em}
\begin{figure*}[ht]
  \centering
  \includegraphics[width=1\textwidth]{figures/amzn_prophet_coverage.pdf}
  \vspace{-0.5em}
  \caption{Comparison results of coverage rate on Amazon stock dataset with Prophet model.}
  \vspace{-0.5em}
  \label{figure amazon prophet coverage}
\end{figure*}

\vspace{-0.5em}
\begin{figure*}[ht]
  \centering
  \includegraphics[width=1\textwidth]{figures/amzn_prophet_sets.pdf}
  \vspace{-0.5em}
  \caption{Comparison results of prediction sets on Amazon stock dataset with Prophet model.}
  \vspace{-0.5em}
  \label{figure amazon prophet sets}
\vspace{-1em}
\end{figure*}



\begin{table}[ht]
\caption{The experimental results in the electricity demand dataset with nominal level $\alpha = 10\%$.}
\label{table elec2}
\setlength{\tabcolsep}{1.25mm} % 调整列间距
\renewcommand{\arraystretch}{1.15} % 调整行间距
\begin{center}
\small
\begin{tabular}{c|ccc|ccc|ccc}
\hline
            & \multicolumn{3}{c|}{Prophet Model}            & \multicolumn{3}{c|}{AR Model}                  & \multicolumn{3}{c}{Theta Model}                \\
Method &
  \begin{tabular}[c]{@{}c@{}}Coverage\\ ( \%)\end{tabular} &
  \begin{tabular}[c]{@{}c@{}}Average\\ width\end{tabular} &
  \begin{tabular}[c]{@{}c@{}}Median\\ width\end{tabular} &
  \begin{tabular}[c]{@{}c@{}}Coverage\\ ( \%)\end{tabular} &
  \begin{tabular}[c]{@{}c@{}}Average\\ width\end{tabular} &
  \begin{tabular}[c]{@{}c@{}}Median\\ width\end{tabular} &
  \begin{tabular}[c]{@{}c@{}}Coverage\\ ( \%)\end{tabular} &
  \begin{tabular}[c]{@{}c@{}}Average\\ width\end{tabular} &
  \begin{tabular}[c]{@{}c@{}}Median\\ width\end{tabular} \\ \hline
ACI        & 90.1 & $\infty$   & 0.443  & 90.1 & $\infty$   & 0.105  & 90.2 & $\infty$   & \textbf{0.055} \\
OGD        & 89.8 & 0.433 & 0.435  & 90.0 & 0.133 & 0.115  & 90.1 & 0.081 & 0.075          \\
SF-OGD     & 89.9 & 0.419 & 0.426  & 90.0 & 0.129 & 0.116  & 90.3 & 0.106 & 0.095          \\
decay-OGD             & 90.1 & 0.531          & 0.521          & 90.1 & 0.122          & 0.099          & 90   & 0.100          & 0.059          \\
PID                   & 90.1 & \textbf{0.207} & \textbf{0.177} & 90   & 0.434          & 0.432          & 89.9 & 0.413          & 0.411          \\
ECI                   & 90   & 0.384          & 0.382          & 90   & \textbf{0.117} & 0.098          & 89.9 & \textbf{0.071} & \textbf{0.055} \\
ECI-cutoff            & 90   & 0.405          & 0.396          & 90.2 & 0.118          & \textbf{0.096} & 90.1 & 0.072          & \textbf{0.055} \\
ECI-integral          & 90.1 & 0.402          & 0.398          & 90   & \textbf{0.117} & 0.098          & 89.9 & 0.072          & \textbf{0.055} \\ \hline
\end{tabular}
\end{center}
\end{table}


\subsection{Results in energy domain}
\vspace{-0.5em}
\label{results in energy domain}
Then we consider the uncertainty problem of electricity demand. The dataset measures electricity demand in New South Wales, collected at half-hour increments from May 7th, 1996 to December 5th, 1998. All values are normalized in $[0, 1]$.

\Cref{table elec2} shows that ECI and ECI-integral have the shortest prediction sets with AR and Theta model, even maintaining the highest coverage in the AR model. We can note that PID stands out with Prophet model due to the scorecaster. In fact, This dataset has collected several other variables, such as the demand and price in Victoria, the amount of energy transfer between New South Wales and Victoria, and so on. These are given as covariates to the scorecaster and complement Prophet well. Other methods do not use this information.




\subsection{Results in climate domain}
\vspace{-0.5em}
\label{results in climate domain}
Finally, we consider the uncertainty problem of climate demand. The dataset measures the daily temperature in the city of Delhi over 15 years (from January 1, 2003 to April 24, 2017). \Cref{table daily-climate} shows that ECI-cutoff achieves best performance in general.

\begin{table}[ht]
\caption{The experimental results in the Delhi temperature dataset with nominal level $\alpha = 10\%$.}
\label{table daily-climate}
\setlength{\tabcolsep}{1.25mm} % 调整列间距
\renewcommand{\arraystretch}{1.15} % 调整行间距
\begin{center}
\small
\begin{tabular}{c|ccc|ccc|ccc}
\hline
            & \multicolumn{3}{c|}{Prophet Model}            & \multicolumn{3}{c|}{AR Model}                  & \multicolumn{3}{c}{Theta Model}                \\
Method &
  \begin{tabular}[c]{@{}c@{}}Coverage\\ ( \%)\end{tabular} &
  \begin{tabular}[c]{@{}c@{}}Average\\ width\end{tabular} &
  \begin{tabular}[c]{@{}c@{}}Median\\ width\end{tabular} &
  \begin{tabular}[c]{@{}c@{}}Coverage\\ ( \%)\end{tabular} &
  \begin{tabular}[c]{@{}c@{}}Average\\ width\end{tabular} &
  \begin{tabular}[c]{@{}c@{}}Median\\ width\end{tabular} &
  \begin{tabular}[c]{@{}c@{}}Coverage\\ ( \%)\end{tabular} &
  \begin{tabular}[c]{@{}c@{}}Average\\ width\end{tabular} &
  \begin{tabular}[c]{@{}c@{}}Median\\ width\end{tabular} \\ \hline
ACI          & 91.0 & $\infty$   & 8.49  & 90.0 & $\infty$   & 6.06  & 90.2 & $\infty$   & 6.48 \\
OGD          & 90.4 & 7.54  & 7.60  & 90.1 & 6.82  & 6.10  & 90.0 & 6.36  & 6.30          \\
SF-OGD       & 90.0 & 7.17  & 7.08  & 90.1 & 6.37  & 5.91  & 90.1 & 6.75  & 6.43          \\
decay-OGD             & 90.1 & 8.84          & 8.35          & 90   & 6.36          & \textbf{5.67} & 89.9 & 6.56          & 6.18          \\
PID                   & 90.1 & 7.65          & 7.65          & 89.7 & 8.92          & 8.86          & 89.7 & 8.77          & 8.79          \\
ECI                   & 90   & 7.2           & 7.22          & 90.1 & 6.39          & 6.1           & 90   & 6.41          & 6.27          \\
ECI-cutoff            & 90.1 & \textbf{7.01} & \textbf{6.96} & 90.1 & \textbf{6.28} & 5.97          & 90   & \textbf{6.27} & \textbf{6.17} \\
ECI-integral          & 90   & 7.21          & 7.3           & 90.2 & 6.39          & 6.11          & 90   & 6.38          & 6.26                \\ \hline
\end{tabular}
\end{center}
\end{table}

\vspace{-0.5em}
\subsection{Results in synthetic data}
\vspace{-0.5em}
\label{Results in synthetic data}
In this experiment, we compare the performance of our method with other baselines under a synthetic changepoint setting in \cite{barber2023conformal}. The data $\{X_i, Y_i\}_{i=1}^n$ are generated according to a linear model $Y_t = X_t^T \beta_t + \epsilon_t$, $X_t \sim \mathcal{N}(0, I_4)$, $\epsilon_t \sim \mathcal{N}(0,1)$. And we set: $\beta_t=\beta^{(0)}=(2,1,0,0)^\top$ for $t=1,\ldots,500$; $\beta_t=\beta^{(1)}=(0,-2,-1,0)^\top$ for $t=501,\ldots,1500$; and $\beta_t=\beta^{(2)}=(0,0,2,1)^\top$ for $t=1501,\ldots,2000$.
And two changes in the coefficients happen up to time $2000$.


We compare ECI and its variants with some competing methods about the coverage and prediction set width. \Cref{table synthetic data} show the result of coverage and set width, while the base predictor is Prophet model. In general, ECI-cutoff and ECI-integral achieve best performance.


\begin{table}[ht]
\caption{The experimental results in the synthetic data dataset with nominal level $\alpha = 10\%$.}
\label{table synthetic data}
\setlength{\tabcolsep}{1.25mm} % 调整列间距
\renewcommand{\arraystretch}{1.15} % 调整行间距
\begin{center}
\small
\begin{tabular}{c|ccc|ccc|ccc}
\hline
            & \multicolumn{3}{c|}{Prophet Model}            & \multicolumn{3}{c|}{AR Model}                  & \multicolumn{3}{c}{Theta Model}                \\
Method &
  \begin{tabular}[c]{@{}c@{}}Coverage\\ ( \%)\end{tabular} &
  \begin{tabular}[c]{@{}c@{}}Average\\ width\end{tabular} &
  \begin{tabular}[c]{@{}c@{}}Median\\ width\end{tabular} &
  \begin{tabular}[c]{@{}c@{}}Coverage\\ ( \%)\end{tabular} &
  \begin{tabular}[c]{@{}c@{}}Average\\ width\end{tabular} &
  \begin{tabular}[c]{@{}c@{}}Median\\ width\end{tabular} &
  \begin{tabular}[c]{@{}c@{}}Coverage\\ ( \%)\end{tabular} &
  \begin{tabular}[c]{@{}c@{}}Average\\ width\end{tabular} &
  \begin{tabular}[c]{@{}c@{}}Median\\ width\end{tabular} \\ \hline
ACI                   & 89.9 & $\infty$           & 8.2           & 89.9 & $\infty$           & 8.2           & 89.9 & $\infty$           & 8.43          \\
OGD                   & 90   & 8.49          & 8.5           & 89.9 & 8.39          & 8.4           & 89.9 & 8.73          & 8.7           \\
SF-OGD                & 90   & 12.48         & 11.56         & 90   & 12.58         & 11.69         & 89.9 & 12.7          & 11.88         \\
decay-OGD             & 90   & 8.3           & \textbf{8.22} & 90   & 8.26          & 8.21          & 90   & 8.57          & 8.6           \\
PID                   & 89.7 & 11.02         & 9.64          & 89.9 & 10.83         & 9.35          & 89.7 & 11.23         & 9.78          \\
ECI                   & 89.9 & \textbf{8.16} & 8.25          & 89.9 & 8.17          & 8.26          & 89.8 & 8.55          & 8.68          \\
ECI-cutoff            & 89.8 & 8.31          & 8.44          & 89.9 & \textbf{8.14} & \textbf{8.19} & 89.8 & 8.51          & 8.59          \\
ECI-integral          & 89.8 & 8.25          & 8.37          & 89.9 & 8.16          & 8.23          & 89.8 & \textbf{8.48} & \textbf{8.58}                \\ \hline
\end{tabular}
\end{center}
\end{table}


