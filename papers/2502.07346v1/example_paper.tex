%%%%%%%% ICML 2025 EXAMPLE LATEX SUBMISSION FILE %%%%%%%%%%%%%%%%%

\documentclass{article}

% Recommended, but optional, packages for figures and better typesetting:
\usepackage{microtype}
\usepackage{graphicx}
\usepackage{subfigure}
\usepackage{booktabs} % for professional tables

% hyperref makes hyperlinks in the resulting PDF.
% If your build breaks (sometimes temporarily if a hyperlink spans a page)
% please comment out the following usepackage line and replace
% \usepackage{icml2025} with \usepackage[nohyperref]{icml2025} above.
\usepackage{hyperref}


% Attempt to make hyperref and algorithmic work together better:
\newcommand{\theHalgorithm}{\arabic{algorithm}}

% Use the following line for the initial blind version submitted for review:
% \usepackage{icml2025}

% If accepted, instead use the following line for the camera-ready submission:
\usepackage[accepted]{icml2025}

% For theorems and such
\usepackage{amsmath}
\usepackage{amssymb}
\usepackage{mathtools}
\usepackage{amsthm}

% if you use cleveref..
\usepackage[capitalize,noabbrev]{cleveref}

%%%%%%%%%%%%%%%%%%%%%%%%%%%%%%%%
% THEOREMS
%%%%%%%%%%%%%%%%%%%%%%%%%%%%%%%%
\theoremstyle{plain}
\newtheorem{theorem}{Theorem}[section]
\newtheorem{proposition}[theorem]{Proposition}
\newtheorem{lemma}[theorem]{Lemma}
\newtheorem{corollary}[theorem]{Corollary}
\theoremstyle{definition}
\newtheorem{definition}[theorem]{Definition}
\newtheorem{assumption}[theorem]{Assumption}
\theoremstyle{remark}
\newtheorem{remark}[theorem]{Remark}

% Todonotes is useful during development; simply uncomment the next line
%    and comment out the line below the next line to turn off comments
%\usepackage[disable,textsize=tiny]{todonotes}
\usepackage[textsize=tiny]{todonotes}
\usepackage{booktabs}
\usepackage{amsmath,amsfonts}
\let\algorithmic\relax
\let\endalgorithmic\relax
\usepackage{array}
\usepackage{textcomp}
\usepackage{stfloats}
\usepackage{url}
\usepackage{verbatim}
\usepackage{graphicx}
\hyphenation{op-tical net-works semi-conduc-tor IEEE-Xplore}

%additional package
\usepackage[utf8]{inputenc}
\usepackage{multirow}
\usepackage{multicol}
\usepackage{color}
\usepackage{url}
\usepackage{enumitem}
\usepackage{diagbox}
\usepackage{flushend,cuted}
\usepackage{bbm}
\usepackage{xspace}
\usepackage{algpseudocode}
\usepackage{colortbl}
\usepackage{bbding}
\usepackage{amssymb}
\DeclareMathOperator*{\argmax}{arg\,max}
\DeclareMathOperator*{\argmin}{arg\,min}
\usepackage{listings}
\usepackage{ulem}
\usepackage{listings}
\usepackage{wrapfig}
\usepackage{soul}
\usepackage{makecell}
\usepackage{xcolor}
\newcommand\nsout[1]{\textcolor{red}{\sout{#1}}}
\newcommand{\kaiti}[1]{\begin{CJK*}{UTF8}{gkai} #1 \end{CJK*}}
\soulregister{\kaiti}7
\usepackage{pifont}
\newcommand{\mycircled}[1]{%
   \raisebox{2pt}{\textcircled{\raisebox{-0.9pt}{\kern-0.2pt #1}}}%
}
\usepackage{rotating}
\usepackage{stfloats}
\usepackage{CJKutf8}
\usepackage[encapsulated]{CJK}
\usepackage{tabularx}
% \usepackage{arydshln}

\newcommand{\ttinyfont}[1]{{\fontsize{5}{12}\selectfont #1}}
\newcommand{\name}{BenchMAX\xspace}
\newcommand{\ifeval}{xIFEval\xspace}
\newcommand{\lcb}{xLiveCodeBench\xspace}
\newcommand{\humaneval}{xHumanEval+\xspace}
\newcommand{\gpqa}{xGPQA\xspace}
\newcommand{\mgsm}{xMGSM\xspace}
\newcommand{\ruler}{xRULER\xspace}
\newcommand{\nexus}{xNexus\xspace}
\newcommand{\arenahard}{xArena-Hard\xspace}
\renewcommand{\arraystretch}{1.1}

% The \icmltitle you define below is probably too long as a header.
% Therefore, a short form for the running title is supplied here:
\icmltitlerunning{\name: A Comprehensive Multilingual Evaluation Suite for Large Language Models}

\begin{document}

\twocolumn[
\icmltitle{\includegraphics[width=0.7cm]{figure/languages.png}\ \name: A Comprehensive Multilingual Evaluation Suite \\ for Large Language Models}
          
% \icmltitle{Submission and Formatting Instructions for \\
%           International Conference on Machine Learning (ICML 2025)}

% It is OKAY to include author information, even for blind
% submissions: the style file will automatically remove it for you
% unless you've provided the [accepted] option to the icml2025
% package.

% List of affiliations: The first argument should be a (short)
% identifier you will use later to specify author affiliations
% Academic affiliations should list Department, University, City, Region, Country
% Industry affiliations should list Company, City, Region, Country

% You can specify symbols, otherwise they are numbered in order.
% Ideally, you should not use this facility. Affiliations will be numbered
% in order of appearance and this is the preferred way.
\icmlsetsymbol{equal}{*}

\begin{icmlauthorlist}
\icmlauthor{Xu Huang}{nju}
\icmlauthor{Wenhao Zhu}{nju}
\icmlauthor{Hanxu Hu}{uzh}
\icmlauthor{Conghui He}{shlab}
\icmlauthor{Lei Li}{cmu}
\icmlauthor{Shujian Huang}{nju}
\icmlauthor{Fei Yuan}{shlab}
\end{icmlauthorlist}

\icmlaffiliation{nju}{National Key Laboratory for Novel Software Technology, Nanjing University}
\icmlaffiliation{uzh}{University of Zurich}
\icmlaffiliation{shlab}{Shanghai Artificial Intelligence Laboratory}
\icmlaffiliation{cmu}{Carnegie Mellon University}
% \icmlaffiliation{comp}{Company Name, Location, Country}
% \icmlaffiliation{sch}{School of ZZZ, Institute of WWW, Location, Country}

\icmlcorrespondingauthor{Shujian Huang}{huangsj@nju.edu.cn}
\icmlcorrespondingauthor{Fei Yuan}{yuanfei@pjlab.org.cn}

% You may provide any keywords that you
% find helpful for describing your paper; these are used to populate
% the "keywords" metadata in the PDF but will not be shown in the document
\icmlkeywords{Machine Learning, ICML}

\vskip 0.3in
]

% this must go after the closing bracket ] following \twocolumn[ ...

% This command actually creates the footnote in the first column
% listing the affiliations and the copyright notice.
% The command takes one argument, which is text to display at the start of the footnote.
% The \icmlEqualContribution command is standard text for equal contribution.
% Remove it (just {}) if you do not need this facility.

%\printAffiliationsAndNotice{}  % leave blank if no need to mention equal contribution
\printAffiliationsAndNotice{\icmlEqualContribution} % otherwise use the standard text.

End-to-end imitation learning offers a promising approach for training robot policies. However, generalizing to new settings—such as unseen scenes, tasks, and object instances—remains a significant challenge. Although large-scale robot demonstration datasets have shown potential for inducing generalization, they are resource-intensive to scale. In contrast, human video data is abundant and diverse, presenting an attractive alternative. Yet, these human-video datasets lack action labels, complicating their use in imitation learning. Existing methods attempt to extract grounded action representations (e.g., hand poses), but resulting policies struggle to bridge the embodiment gap between human and robot actions.
% our approach
We propose an alternative approach: leveraging language-based reasoning from human videos - essential for guiding robot actions - to train generalizable robot policies. Building on recent advances in reasoning-based policy architectures, we introduce Reasoning through Action-free Data (RAD). RAD learns from both robot demonstration data (with reasoning and action labels) and action-free human video data (with only reasoning labels). The robot data teaches the model to map reasoning to low-level actions, while the action-free data enhances reasoning capabilities. Additionally, we will release a new dataset of 3,377 human-hand demonstrations compatible with the Bridge V2 benchmark. This dataset includes chain-of-thought reasoning annotations and hand-tracking data to help facilitate future work on reasoning-driven robot learning.
% experiments
Our experiments demonstrate that RAD enables effective transfer across the embodiment gap, allowing robots to perform tasks seen only in action-free data. Furthermore, scaling up action-free reasoning data significantly improves policy performance and generalization to novel tasks. These results highlight the promise of reasoning-driven learning from action-free datasets for advancing generalizable robot control. 
% releasing dataset
Website: \href{https://rad-generalization.github.io}{here}.


\begin{figure}[ht]
    \centering
    \includegraphics[width=0.8\linewidth]{graphs/greater_than_naive.pdf}
    \vspace{0.5cm}
    \includegraphics[width=0.8\linewidth]{graphs/p1_bottom.png}
    \vspace{-5pt}
    \caption{\textcolor{positional}{Positional} vs.\ \textcolor{nonpositional}{non-positional} circuits. In a \textcolor{nonpositional}{non-positional} circuit, the same edges must be included at all positions. A \textcolor{positional}{positional} circuit can distinguish between the same edge at different positions. This specificity yields better trade-offs between circuit size and faithfulness. It can also increase both precision and recall.}
    \label{fig:p1}
    \vspace{-5pt}
\end{figure}

\section{Introduction}

\looseness=-1
A primary goal of interpretability research is to characterize the internal mechanisms in language models (LMs) and other NLP models. 
A core approach in this area is \textbf{circuit discovery}---identifying the minimal subgraph within the model's computation graph that performs a specific task \citep{olah2021framework,olah-mech}.
Typically, the nodes of a circuit represent model components (e.g., attention heads, neurons, or layers).
While manual circuit discovery methods can yield position-specific insights \citep{wanginterpretability,goldowskydill2023localizingmodelbehaviorpath}, \emph{automatic methods often overlook positional information}, treating components as uniformly relevant across all input token positions \citep{conmytowards,syed2023attribution}. 
For instance, if an attention head is included in a circuit, it is assumed to contribute equally to the computation for every position in the input sequence.
The assumption that circuits are position-invariant ignores the fact that different positions often require distinct computations.
By ignoring positions, current methods limit their ability to capture mechanisms that operate across positions, such as interactions between attention heads across positions.

In this study, we start by demonstrating that positional agnosticism is a significant limitation (\S\ref{sec:motivating}). Then, to address these limitations, we introduce a new approach: position-aware edge attribution patching (PEAP; \S\ref{sec:full_circ_discovery}; Figure~\ref{fig:p1}). Current approaches  assume that if an edge is in a circuit, then the same edge will be in the circuit at all positions, thus leading to low precision. It is also assumed that an edge's importance should be aggregated across positions before deciding whether it should be included in the circuit; this can lead to cancellation effects, and thus low recall. PEAP instead allows us to compute the importance of cross-positional edges, and separately evaluates edge importance at each position. We show that this leads to smaller and more accurate circuits; see Figure~\ref{fig:p1}.

Incorporating positional information into circuit discovery is straightforward when inputs have the same length and structure across examples.

However, realistic datasets are not nearly this templatic.
How, then, can we incorporate positional information into automatic circuit discovery?
To address this challenge, we propose \textbf{schemas} (\S\ref{sec:schema}). 
Schemas assign semantic labels to spans of tokens, enabling information aggregation across examples even when the spans differ in length.

For example, in the input ``The \textcolor{positional}{war} lasted from 1453 to 14\underline{\hspace{1em}},'' the span ``\textcolor{positional}{war}'' could be labeled as ``\emph{Subject}''.
This enables handling spans with varying lengths: the phrase ``\textcolor{positional}{Black Plague}'' in another example can be treated as a single positional span with the same role as ``\textcolor{positional}{war}''.
In experiments with two LMs and three tasks, we find that circuits discovered using schemas achieve a better trade-off between circuit size and faithfulness to the model's behavior than position-agnostic circuits.
Importantly, position-aware circuits offer a more precise representation of the underlying mechanisms, providing a more concise foundation for mechanistic explanations.

We also present a fully automated pipeline for schema generation and application (\S\ref{sec:schema-generation}) using large language models (LLMs). 
We evaluate the quality of the generated schemas and their utility in discovering position-aware circuits (\S\ref{sec:schema-eval}).
Notably, circuits derived using automatically generated and applied schemas achieve comparable faithfulness scores to circuits discovered with human-designed and manually applied schemas.

We summarize our contributions as follows:
\begin{itemize}[noitemsep,leftmargin=*,topsep=1pt,parsep=1pt]
    \item Introduce a position-aware circuit discovery method, which obtains better faithfulness than position-agnostic discovery.  
    \item Introduce dataset schemas,  facilitating positional circuit discovery in more naturalistic settings. 
    \item Develop an automated schema generation and application pipeline with LLMs, yielding schemas that are comparable to manually-annotated ones.
\end{itemize}



\section{Related work}


Recent advances in single-image animatable head avatar generation can be categorized into mainly 2D-based and 3D-based approaches. 

\paragraph{\bf Image to 2D Animatable Avatar.}
2D-based methods, leveraging the power of convolutional neural networks (CNNs)~\cite{DBLP:conf/cvpr/KarrasLAHLA20,DBLP:conf/cvpr/IsolaZZE17,DBLP:conf/nips/GoodfellowPMXWOCB14}, often employ generative adversarial networks (GANs)~\cite{DBLP:conf/cvpr/StyleGAN} for direct image synthesis. Early approaches~\cite{DBLP:conf/cvpr/WangDYSW23,DBLP:conf/cvpr/BurkovPGL20,DBLP:conf/iccv/ZakharovSBL19} focus on injecting expression and pose features into the generator network, often utilizing architectures like U-Net or StyleGAN~\cite{DBLP:conf/cvpr/StyleGAN}.
Some other 2D methods~\cite{DBLP:journals/corr/abs-2407-03168,DBLP:conf/cvpr/ZhangQZZW0CW023,DBLP:conf/cvpr/HongZS022,DBLP:conf/mm/DrobyshevCKILZ22,DBLP:conf/cvpr/BurkovPGL20,DBLP:conf/nips/SiarohinLT0S19} represent expressions and poses as warping fields applied to the source image. 
Benefiting from advances in image and video diffusion networks, more recent 2D-based works~\cite{DBLP:journals/corr/abs-2410-07718,DBLP:journals/corr/abs-2406-08801,DBLP:conf/eccv/TianWZB24} get improved results with diffusion techniques. 
However, these methods still face challenges related to long generation times and significant computational resource demands. Audio-driven 2D control methods~\cite{DBLP:conf/cvpr/ZhangCWZSGSW23,DBLP:journals/corr/abs-2211-12368,DBLP:conf/iccv/GuoCLLBZ21} are easy to use but cannot explicitly control facial expressions and poses. 2D-based techniques often struggle with large pose or expression variations due to the lack of an explicit 3D structure, sometimes producing unrealistic distortions or identity changes. While some 2D methods~\cite{SadTalker,StyleHEAT,Pirenderer,DBLP:conf/cvpr/WangM021,MegaPortraits} incorporate 3D Morphable Models (3DMMs)~\cite{DBLP:conf/fgr/GerigMBELSV18,DBLP:journals/tog/LiBBL017,DBLP:conf/avss/PaysanKARV09,DBLP:conf/siggraph/BlanzV99} to mitigate these issues, they typically cannot achieve free-viewpoint rendering. 

\vspace{-0.1in}

\begin{figure*}[h]
    \centering
    \includegraphics[width=0.9\linewidth]{images/framework.pdf}
    \caption{\textbf{Overall Framework.} Our framework utilizes learnable query features attached to FLAME vertices to perform cross-attention with the extracted multi-level image features. The extracted features are then decoded to reconstruct the Gaussian avatar in the canonical space, which can be animated utilizing standard linear blend skinning (LBS) and corrective blendshapes as the FLAME model did and rendered in real-time on various platforms.}
    \label{fig:framework}
\end{figure*}

\paragraph{\bf Image to 3D Animatable Avatar.}
3D-aware methods offer improved geometric consistency and free-viewpoint rendering capabilities. Early 3D approaches~\cite{DBLP:conf/eccv/KhakhulinSLZ22,DBLP:conf/cvpr/XuYCWDJT20} utilize 3DMMs for head avatar reconstruction. With the advent of Neural Radiance Fields (NeRFs)~\cite{DBLP:conf/eccv/MildenhallSTBRN20}, many recent methods~\cite{DBLP:conf/siggraph/YuFZWYBCSWSW23,DBLP:conf/cvpr/MaZQLZ23,DBLP:conf/cvpr/LiZWZ0CZWB023,GPAvatar,ye2024real3d,deng2024portrait4d,deng2024portrait4d2,DBLP:conf/eccv/KiMC24,DBLP:conf/cvpr/BaiFWZSYS23,PointAvatar,Nerfies,INSTA} have adopted this representation for higher fidelity, particularly in modeling fine details like hair. However, NeRF-based~\cite{DBLP:conf/cvpr/ZhangZLHLWGCL024,HAvatar,DBLP:conf/cvpr/BaiTHSTQMDDOPTB23,AD-NeRF,DBLP:journals/tog/GaoZXHGZ22,DBLP:journals/tog/ParkSHBBGMS21,DBLP:conf/cvpr/AtharXSSS22,DBLP:journals/corr/abs-2112-05637,DBLP:conf/iccv/TretschkTGZLT21,DBLP:conf/cvpr/GafniTZN21,DBLP:conf/eccv/KiMC24,DBLP:conf/cvpr/BaiFWZSYS23,PointAvatar,Nerfies,DBLP:conf/siggraph/YuFZWYBCSWSW23,DBLP:conf/cvpr/MaZQLZ23,DBLP:conf/cvpr/LiZWZ0CZWB023} approaches often require extensive training data, including multi-view or single-view videos, raising privacy concerns and limiting generalization to unseen identities. Some methods~\cite{DBLP:conf/cvpr/SunWWLZZL23,DBLP:conf/3dim/ZhuangMKS22,DBLP:journals/pami/SunWZHWL24,DBLP:journals/tvcg/TangZYZCMW24,DBLP:conf/iclr/XuZLZBFS23} bypass this data requirement by training generators with random noise and then inverting them for identity-specific reconstruction, but inversion accuracy remains a challenge. Test-time optimization offers another alternative, but its computational cost limits practical applications. Several recent works~\cite{goha2023,hidenerf2023,gpavatar2024,ye2024real3d,ma2024cvthead,deng2024portrait4d,deng2024portrait4d2,GGHead} have explored one-shot 3D head reconstruction to address the limitations of data requirements and computational cost. These methods employ various techniques, such as tri-plane features, deformation fields, point-based expression fields, and vertex-feature transformers. Despite these advancements, NeRF-based methods often struggle with real-time rendering. 
Recently, 3D Gaussian Splatting~\cite{GaussianSplatting} has emerged as a promising alternative, offering both high-quality results and fast rendering speeds. However, existing Gaussian Splatting methods~\cite{GaussianAvatar,DBLP:conf/cvpr/XuCL00ZL24} typically rely on video data for training for each person, limiting their ability to generalize to new identities. Instead, the most recent work, GAGAvatar~\cite{GAGAvatar}, proposes a one-shot 3D Gaussian-based head avatar generation method. However, it still relies heavily on complex 2D neural post-processing to achieve optimal animation outcomes, thus it is not a pure 3D solution and the extra neural network hinders its application on various platforms. In contrast, our work generates Gaussian heads that are immediately animatable and renderable without additional networks or post-processing steps, enabling seamless integration into existing rendering pipelines for real-time animation and rendering across a wide range of platforms, including mobile phones. 

\section{Benchmark Construction}
In this section, we extend the evaluation of the core capabilities of LLMs into multilingual scenarios.
% As shown in Figure~\ref{fig:overview}, \name encompasses multiway parallel data spanning 17 languages~(\S~\ref{sec:lg_selection}), enabling fair cross-lingual assessments and comparisons. 
To ensure sufficient linguistic diversity, we select 16 non-English languages~(\S~\ref{sec:lg_selection}).
% Meanwhile, its comprehensive coverage of diverse tasks~(\S~\ref{sec:capability_selection}) facilitates the evaluation of multiple capabilities across a spectrum of linguistic contexts.
Meanwhile, a diverse set of tasks designed to evaluate 6 crucial LLM capabilities is chosen to facilitate comprehensive assessment~(\S~\ref{sec:capability_selection}).
% Notably, its precise human annotation~(\S~\ref{sec:construction_process}) guarantee the integrity and trustworthiness of the data.
Subsequently, we introduce a rigorous pipeline~(\S~\ref{sec:construction_process}) that incorporates human annotators and LLMs to obtain a high-quality benchmark.


\subsection{Language Selection}
\label{sec:lg_selection}
\name supports 17 selected languages to represent diverse language families and writing systems~(Table~\ref{tab:lg_selection}).
% 17 languages~(\textit{en, es, fr, de, ru, bn, ja, th, sw, zh, te, ar, ko, sr, cs, hu, vi}) supported by \name, cover diverse language families and script systems~(Table~\ref{tab:lg_selection}). 


\subsection{Capabilities Selection}
\label{sec:capability_selection}
% \renewcommand{\arraystretch}{1.4} % Default value: 1
% \begin{table*}[!ht]
%     \caption{Selection of core capabilities and details of task data.}
%     \label{tab:task_detail}
%     \vskip 0.15in
%     \centering
%     \resizebox{0.95\linewidth}{!}{
%     \begin{tabular}{c|c|c|c|c|c|c|c}
%     \toprule
%         \textbf{Capability} & \textbf{Dataset} & \textbf{\# Sample} & \textbf{Metric} & \textbf{Capability} & \textbf{Dataset} & \textbf{\# Sample} & \textbf{Metric} \\ 
%     \midrule
%         Instruction& IFeval & 429 & Accuracy & \multirow{2}{*}{Reasoning} & MGSM & 250 & \multirow{2}{*}{Exact Match} \\ 
%         \cline{2-4} \cline{6-7}
%         Following & Arena-hard & 500 & Win Rate & ~ & GPQA & 448 & ~ \\ 
%         \hline
%         Code & Humaneval+ & 164 & \multirow{2}{*}{Pass@1} & \multirow{2}{*}{Translation} & Flores+TED+WMT24 & 1012 & \multirow{2}{*}{spBLEU} \\ 
%         \cline{2-3} \cline{6-7}
%         Generation & LiveCodeBench & 713 & & & Domain Translation & 2781 & ~ \\ 
%         \hline
%         Long Context & RULER & 100 & Exact Match & Tool Use & Nexus & 318~\footnote{We only adopt the standardized\_queries subset which contains 318 samples.} & Accuracy \\ 
%     \bottomrule
%     \end{tabular}}
%     \vskip -0.1in
% \end{table*}


LLMs have demonstrated proficiency in understanding tasks such as text classification, sentiment analysis, and so on. 
However, their capabilities transcend text understanding, possessing the following intrinsic capabilities:

\begin{itemize} [nosep,itemsep=1pt,leftmargin=0.1cm]
    \item \textbf{Instruction Following:}  Following instructions capability is categorized into two distinct tasks based on evaluation paradigms: rule-based and model-based assessment.
    \item \textbf{Reasoning:} The capability to reason through intricate scenarios including both math reasoning and natural scientific~(physics, chemistry, and biology) reasoning tasks.
    \item \textbf{Code Generation:} We primarily consider Python executable code generation in two settings, function completion and programming problem solving.
    % \item \textbf{Long Context:} The ability to handle long contextual information. We mainly use synthetic long-context data to assess the ability. 
    \item \textbf{Long Context Modeling:} The ability to extract evidence from lengthy documents. We evaluate this capability through question-answering tasks with long documents ~(128k tokens).
    \item \textbf{Tool Use:} We assess the capability of utilizing tools effectively to correctly select and invoke a single function from multiple available functions based on given user queries.
    \item \textbf{Translation:} Translation involves accurately converting text between languages while preserving semantic meaning. Beyond traditional translation tasks, we introduce the Domain Translation task, a by-product of the \name construction process. This task challenges models to translate specialized terminology and determine whether specific segments should be translated.
\end{itemize}

% \begingroup
% \renewcommand{\arraystretch}{1.3} % Default value: 1
% \begin{table}[!t]
%     \caption{Selection of core capabilities and details of task data. For IFEval, we filter out all language specific instructions, thus remaining 429 samples. For Nexus dataset, we only adopt the standardized\_queries subset which contains 318 samples. For general translation datasets, the number of samples may vary in different translation directions, according to the number of parallel samples in TED and WMT24.}
%     \label{tab:task_detail}
%     \vskip 0.15in
%     \footnotesize
%     \centering
%     \resizebox{0.95\linewidth}{!}{
%     \begin{tabular}{c|c|c|c|c}
%     \toprule
%         \textbf{Capability} & \textbf{Category} & \textbf{Dataset} & \textbf{\# Samples} & \textbf{Metric} \\ 
%     \midrule
%         Instruction & Rule-based & IFeval & 429 & Accuracy  \\ 
%         \cline{2-5}
%         Following & Model-based & Arena-hard & 500 & Win Rate \\ 
%         \hline
%         \multirow{2}{*}{Reasoning} & Math & MGSM & 250 & \multirow{2}{*}{Exact Match} \\
%         \cline{2-4}
%         & Science & GPQA & 448 & \\
%         \hline
%         Code & \makecell{Function\\Completion} & Humaneval+ & 164 & \multirow{2}{*}{Pass@1}  \\ 
%         \cline{2-4}
%         Generation & \makecell{Problem\\Solving} & LiveCodeBench\_v4 & 713 & \\ 
%         \hline
%         Long Context & \makecell{Question\\Answering} & RULER & 800 & Exact Match \\
%         \hline
%         Tool Use & \makecell{Multiple\\Functions} & Nexus & 318 & Accuracy \\
%         \hline
%         \multirow{2}{*}{Translation} & General & Flores+TED+WMT24 & $\ge$1012 & \multirow{2}{*}{spBLEU} \\
%         \cline{2-4}
%         & Domain & Annotated data above& 2781 &  \\
%     \bottomrule
%     \end{tabular}}
%     \vskip -0.2in
% \end{table}
% \endgroup


Further details on the datasets, sample sizes, and evaluation metrics are provided in Table~\ref{tab:task_detail}.
More detailed information can be found in Appendix~\ref{sec:appendix-task}.


\begingroup
\renewcommand{\arraystretch}{1.3}
\begin{table}[t!]
    \caption{One example in rule-based instruction following task, which includes complex constraints. First, we enclose these constraints with special symbols and then use a machine translation system to translate from English to the target language. Finally, we reform the case structure by extracting the constraint from machine translation for human post-editing.}
    \label{tab:construction-case}
    \vskip 0.1in
    \centering 
    \tiny
    \begin{tabular}{|p{7.8cm}|}
        \hline
        \textbf{[Original Text]:} 
\{prompt: Create an ad copy by expanding "Get 40 miles per gallon on the highway" in the form of a QA with a weird style. Your response should contain less than 8 sentences. Do not include keywords 'mileage' or 'fuel' in your response. \\
instruction\_id\_list: ['length\_constraints: number\_sentences', 'keywords: forbidden\_words'] \\
kwargs: [\{'relation': 'less than', 'num\_sentences': 8\}, \{'forbidden\_words': ['mileage', 'fuel']\}]\} \\
        \hline
        
        \textbf{[Translation Input]:} 
Create an ad copy by expanding "Get 40 miles per gallon on the highway" in the form of a QA with a weird style. Your response should contain less than 8 sentences. Do not include keywords '\textcolor{red}{$<$b$>$}mileage\textcolor{red}{$<$/b$>$}' or '\textcolor{red}{$<$b$>$}fuel\textcolor{red}{$<$/b$>$}' in your response. \\
        \hline
        
        \textbf{[Google Translation Result]:} 
        \begin{CJK}{UTF8}{gbsn}以风格怪异的问答形式扩展“在高速公路上每加仑行驶 40 英里”来创建广告文案。您的回复应少于 8 个句子。请勿在回复中包含关键字“\textcolor{red}{$<$b$>$}里程\textcolor{red}{$<$/b$>$}”或“\textcolor{red}{$<$b$>$}燃料\textcolor{red}{$<$/b$>$}”。\end{CJK} \\
        
        % \begin{CJK}{UTF8}{gbsn}写一首关于名叫罗德尼的塞尔达粉丝的打油诗。请确保包含以下内容:\textcolor{red}{<b>}塞尔达\textcolor{red}{</b>}、\textcolor{red}{<b>}海拉鲁\textcolor{red}{</b>}、\textcolor{red}{<b>}林克\textcolor{red}{</b>}、\textcolor{red}{<b>}加农\textcolor{red}{</b>}。字数不得超过 100 个。\end{CJK}\\
        
        \hline
    
        \textbf{[Case Reform]} \{
        prompt: \begin{CJK}{UTF8}{gbsn}以风格怪异的问答形式扩展“在高速公路上每加仑行驶 40 英里”来创建广告文案。您的回复应少于 8 个句子。请勿在回复中包含关键字“里程”或“燃料”。\end{CJK}
        \\
instruction\_id\_list: ['length\_constraints:number\_sentences', 'keywords:forbidden\_words']
kwargs: [\{'relation': 'less than', 'num\_sentences': 8\}, {'forbidden\_words': [\begin{CJK}{UTF8}{gbsn}'里程', '燃料'\end{CJK}]}] \} \\ 
        \hline
        \textbf{[Human Post-Editing]} \{
        "prompt": \begin{CJK}{UTF8}{gbsn}以一种奇特风格的问答形式展开“在高速公路上每加仑行驶40英里”这句话,创建为一个广告文案。你的回答应该少于8句话。不要在你的回复中包含关键字“里程”或“燃料”。\end{CJK},\\ 
        instruction\_id\_list: ['length\_constraints:number\_sentences', 'keywords:forbidden\_words']
kwargs: [\{'relation': 'less than', 'num\_sentences': 8\}, {'forbidden\_words': [\begin{CJK}{UTF8}{gbsn}'里程', '燃料'\end{CJK}]}] \} \\ 
        \hline
    \end{tabular}
        % \vskip -0.55in
    % \vspace{-0.4cm}
\end{table}
\endgroup


\begin{figure*}[htbp]
    \centering
    \includegraphics[width=0.7\linewidth]{figure/process.pdf}
    \caption{The construction process of \name involves three steps: Step 1) translating data from English to non-English; Step 2) post-editing each sample by three human annotators; Step 3) selecting the final translation version.}
    \label{fig:overview}
    \vskip -0.2in
\end{figure*}


% follow complex instructions~(\textit{Instruction Following Capability}), reason through intricate scenarios~(\textit{Reasoning}), handle long contextual information~(\textit{Long Context Modeling Capability}), generate code autonomously~(\textit{Code Generation Capability}), utilize tools effectively~(\textit{Tool Usage Capability}), and navigate the complicated landscape of machine translation~(\textit{Translation Capability}). 
% Here, we extend the evaluation of these complex capabilities in multilingual scenarios, as shown in Table~\ref{tab:task_detail}.

% \begin{itemize} [nosep,itemsep=1pt,leftmargin=0.3cm]
%     \item Instruction Following Capability: In the light of varied evaluation methods - rule-based or model-based - we introduce two distinct tasks.
%     \item Reasoning: We include both mathematical reasoning and scientific (physics, chemistry, and biology) reasoning tasks 
%     \item Long Context Modeling Capability: We focus on the evaluation of long context in multilingual settings.
%     \item Code Generation Capability: We primarile consider Python code generation in two settings, function completion and programming problem solving.
%     \item Tool Usage Capability: We assess the ability to correctly select and invoke a single function from multiple available functions based on a given user query.
%     \item Translation Capability: Beyond traditional translation tasks, we introduce the Domain Translation task, a by-product of the \name construction process. This task challenges the model to determine whether a given segment should be translated.
% \end{itemize}



\subsection{Construction}
\label{sec:construction_process}
The way to obtain \name consists of three steps, as shown in Figure~\ref{fig:overview}: 1) translate data from English to non-English by machines; 2) post-edit each sample by three native annotators; 3) pick the final translation version by GPT-4o-mini.


% For Step 1, we first conduct a preliminary study by randomly selecting a small sample set and translating it with both GPT-4o and Google Translate. 
% Interestingly, GPT-4o's translations don't always outperform, even for high-resource languages. 
% Hence, our choice of translator is based on the performance of this preliminary study.
\paragraph{Step 1: Translating data from English to selected non-English languages by machine translation systems.}
We select between traditional translators such as Google Translate, and LLM-based ones like GPT-4o, depending on whether the task contains extractable constraints.
As illustrated in Figure~\ref{fig:step1}, if the task data contains constraints that are hard to extract, we prompt GPT-4o to translate the data and satisfy the constraints.
Otherwise, we use Google Translate along with extraction tools.
Extraction tools can include methods for extracting translated keywords by enclosing source keywords with special symbols, and for preserving source constraints by replacing constraints with placeholders before translation and restoring them afterwards.

\begin{figure}[!t]
    \centering
    \includegraphics[width=0.8\linewidth]{figure/step1.pdf}
    \caption{Flow chart illustrating the constraint extraction and preprocessing pipeline in the first step of our benchmark construction.}
    \label{fig:step1}
\end{figure}

% \begin{enumerate}
%     \item If the task data contains no keywords and constraints that are necessary for evaluation, use Google Translate.
%     \item If the task data contains extractable keywords or constraints, use Google Translate along with extraction tools.
%     \item If the task data contains keywords or constraints that are not easily extractable like program variable names, attempt to prompt GPT-4o for translation.
% \end{enumerate}


Taking an example of rule-based instruction following task as an example, as shown in Table~\ref{tab:construction-case}, it requires extra processing to extract constraints from the translated instruction, as they are needed for verification.
Inspired by~\citet{yuan-etal-2020-enhancing}, we enclose the keywords in the original instruction with special symbols, making them easy to extract from the translated result.
% In constraints-contained translation, we explore various groups of special symbols, and improve the recall of constraints~(the detailed recall results are displayed in Appendix~\ref{sec:append_if_keywords}.) through multiple rounds of extraction.
As shown in Table~\ref{tab:stat_symbol_translation}, we explore various groups of special symbols and different orders and calculate the recall rates of keywords.
Comparing to not using special symbols, apply any symbol group can greatly improve the recalls, while combining different symbol groups in multiple rounds can further improve the recalls.
We choose \textit{Order 1} as it can achieve better results with fewer groups than \textit{Order 2}.
The detailed recall results are in Appendix~\ref{sec:append_if_keywords}.


\begin{table}[t]
    \centering
    \caption{The recall rates using different groups of special symbols}
    \label{tab:stat_symbol_translation}
    \vskip 0.15in
    \scriptsize
    \begin{tabular}{cc|cccc}
        \toprule
         \multicolumn{2}{c|}{\multirow{2}{*}{\textbf{Setting}}} & \multicolumn{4}{c}{\textbf{Target Language}}   \\
         & & zh & es & fr & hu \\
         \midrule
         \multicolumn{2}{c|}{w/o special symbols} & 0.68 & 0.68 & 0.68 & 0.68 
         \\
         \multicolumn{2}{l|}{symbol 1: $<$b$>$ $<$/b$>$} & 0.91 & 0.89 & 0.88 & 0.93 \\
         \multicolumn{2}{l|}{symbol 2: (\texttt{ })} & 0.88 & 0.91 & 0.89 & 0.92 \\
         \multicolumn{2}{l|}{symbol 3: ([\texttt{ }])} & 0.82 & 0.89 & 0.87 & 0.92 \\
        \midrule
         \multirow{2}{*}{Order 1} & + symbol 1 & 0.91 & 0.89 & 0.88 & 0.93 \\
         & + symbol 2 & 0.93 & 0.93 & 0.90 & 0.95 \\
         \midrule
        \multirow{3}{*}{Order 2} & + symbol 2 & 0.88 & 0.91 & 0.89 & 0.92 \\
        & + symbol 1 & 0.90 & 0.93 & 0.90 & 0.95 \\
        & + symbol 3 & 0.92 & 0.93 & 0.90 & 0.95 \\
         \bottomrule
    \end{tabular}
    % \vskip -0.1in
\end{table}

% Furthermore, we adjust the verification rules to better adapt to multiple languages, and refine the the number-word constraint for non-English languages based on the English to their word count ratio in the Flores-200 corpus, ensuring fair comparison across languages.



% During Step~2, each native speaker annotator needs to create a new translation using the English input text and the result from Step~1. 
% Therefore, the basic requirement for annotators is proficiency in both English and their native language. Additionally, since GPQA is a knowledge-intensive task, we require annotators to have at least a Bachelor's degree and to consult relevant materials to ensure the accuracy of proper noun translations.
% \vskip 0.1in
\paragraph{Step 2: Post-editing each sample by three distinct native-speaking annotators in all tasks.}
To ensure high-quality dataset, we implement a multi-round annotation and verification process. 
1) Each sample is given to three native-speaking annotators who are proficient in English and their native language. Considering the specialized nature of datasets like Science reasoning, annotators are required to hold at least a Bachelor's degree. 
2) Two automatic verifiers - rule-based verifiers and model-based verifiers - are used to assess the quality of human annotation. Rule-based verifiers are used to ensure the satisfaction of constraints for certain tasks, such as the rule-based instruction following task.
For model-based verifiers, we utilize the GEMBA-SQM prompt~\cite{kocmi-federmann-2023-large} and employ Qwen2.5-72B-Instruct, a powerful multilingual model, estimating the quality of translations. 
Along with providing an overall score, the model offers detailed explanations of translation errors as feedback to annotators. 
Samples failing the rule-based verifier or scoring below a predefined threshold are identified as failed samples and refined in subsequent iterations. 
Each manually annotated dataset undergoes at least three iterations.



% \vskip 0.1in
\paragraph{Step 3: Selecting the final translation version by LLMs.}

Initially, we ask a fourth annotator, uninvolved in the annotation process, to choose the final version from the results revised by three individuals. Intriguingly, the selection by the fourth annotator exhibited a strong position bias, often favoring the initial annotation. This preference could be attributed to the uniformly high quality of the annotations, resulting in minimal discernible differences among them.

% In preliminary studies, we find that even human annotators have position bias, prefering the first translation when the three translations are similar.
Debiasing for human annotators is costly in terms of both time and finance, because three translations encompass all permutations of six.
Consequently, we employ GPT-4o-mini to select the final translation, as it is a powerful and balanced LLM across different languages.
In particular, following~\citet{li2024crowdsourced}, we adapt the LLM-Judge system instruction~(see Appendix~\ref{sec:prompts}) to suit pairwise translation evaluation.
% We conduct two battles to determine the winner among the three candidates.
We firstly shuffle the positions of the three translations and conduct two battles to select a final version.
In each battle between two translations, we perform two judgments by swapping their positions and determine one winner.
The winner of the first two translations then battles against the third translation, determining the final winner.



\begin{table}[t]
\centering
\caption{Results over the benchmark datasets. The mIoU is reported. %
}
\label{tab:sota_results}
\resizebox{\columnwidth}{!}{
\begin{tabular}{ccccccc}
\toprule
\textbf{Method} & \begin{tabular}[c]{@{}c@{}}Inference\\ Vocab. \end{tabular} & A-847 & PC-459 & A-150 & PC-59 & VOC-20 \\ \midrule
SAN \cite{xu2023side} & \checkmark & 12.4 & 15.7 & 27.5 & 53.8 & 94.0 \\
AttrSeg \cite{ma2024open} & \checkmark & -- & -- & -- & 56.3 & 91.6 \\
SCAN \cite{liu2024open} & \checkmark & 14.0 & 16.7 & 30.8 & \textbf{58.4} & \textbf{97.0} \\
EBSeg \cite{shan2024open} & \checkmark & 13.7 & 21.0 & 30.0 & 56.7 & 94.6 \\
SED \cite{xie2024sed} & \checkmark & 11.4 & 18.6 & 31.6 & 57.3 & 94.4 \\
CAT-Seg \cite{cho2024cat} & \checkmark & \textbf{16.0} & \textbf{23.8} & \textbf{31.8} & 57.5 & 94.6 \\ \midrule
CaSED + SAM \cite{conti2024vocabulary} & \xmark & -- & -- & 6.1 & 7.5 & 13.7 \\
CaSED + SAN \cite{conti2024vocabulary} & \xmark & -- & -- & 7.2 & 15.5 & 26.9 \\
DenseCaSED \cite{conti2024vocabulary} & \xmark & -- & -- & 8.6 & 13.4 & 20.5 \\
\textbf{Chick.-and-egg} (CaSED) & \xmark & 3.2 & 4.4 & 9.7 & \textbf{23.1} & \textbf{47.6} \\
\textbf{Chick.-and-egg} (RAM) & \xmark & \textbf{3.7} & \textbf{7.1} & \textbf{15.6} & 23.0 & 47.5  \\
\bottomrule
\end{tabular}
}
\end{table}

\section{Experiments}
\label{ch:results}

We conduct a comprehensive experimental analysis to investigate how different components affect VSS performance.
First, we evaluate the proposed two-stage approach on standard benchmarks to establish the baseline (\Cref{sec:benchmark}). We then present an in-depth analysis of the text encoder's behaviour and its impact on segmentation quality (\Cref{sec:text}). To better understand the relationship between the two-stages, we examine the image tagging accuracy and its influence on the segmentation task (\Cref{sec:tagging}). Finally, we study how different assignment thresholds in the evaluation protocol affect the reported performance (\Cref{sec:thresholds}).

\textbf{Implementation Details:}
The model is trained on the COCO-Stuff dataset \cite{caesar2018coco}, which contains 118k annotated images across 171 categories, following \cite{cho2024cat}. All results are based on CLIP \cite{radford2021learning} with a ViT-B/16 backbone. The image encoder and cost aggregation module are trained with per-pixel binary cross-entropy loss. 
The training parameters follow \cite{cho2024cat}. The batch size is 4, and models are trained for 80k iterations.
We performed image tagging and instance description using a frozen VLM model not trained on the testing dataset. More in detail, we examined the robustness of two models RAM \cite{zhang2024recognize} and Llava-1.6  \cite{liu2024llavanext}.

\textbf{Test Datasets:} The evaluation covers several datasets to ensure comprehensive testing. We used ADE20K \cite{zhou2019semantic} with both 150 and 847 class configurations, Pascal Context \cite{mottaghi2014role} with 59 and 459 class setups, and Pascal VOC \cite{everingham2010pascal} with its 20 classes. 

\begin{figure*}[t]
    \centering
    \resizebox{\textwidth}{!}{%
    \begin{tabular}{@{}ccccc@{}}
        
        
        \includegraphics[width=0.25\textwidth]{fig/qualitative/ADE_val_00000049_img.png} &
        \includegraphics[width=0.25\textwidth]{fig/qualitative_new/ADE_val_00000049_zeroseg.png} &
        \includegraphics[width=0.25\textwidth]{fig/qualitative/ADE_val_00000049_cased_labels_bigger.png} &
        \includegraphics[width=0.25\textwidth]{fig/qualitative/ADE_val_00000049_ours_labels_bigger.png} &
        \includegraphics[width=0.25\textwidth]{fig/qualitative/ADE_val_00000049_labels_bigger.png} \\[0.2cm]
        



        \includegraphics[width=0.25\textwidth] {fig/qualitative_new/ADE_val_00000683_img.png} &
        \includegraphics[width=0.25\textwidth] {fig/qualitative_new/ADE_val_00000683_zero_seg.png} &
        \includegraphics[width=0.25\textwidth]{fig/qualitative_new/ADE_val_00000683_real_image_cased.png} &
        \includegraphics[width=0.25\textwidth]{fig/qualitative_new/ADE_val_00000683_real_image.png} &
        \includegraphics[width=0.25\textwidth]{fig/qualitative_new/ADE_val_00000683_ground_truth.png} \\[0.2cm]


        \includegraphics[width=0.25\textwidth] {fig/qualitative_new/2007_008415_img.png} &
        \includegraphics[width=0.25\textwidth] {fig/qualitative_new/2007_008415_zeroseg.png} &
        \includegraphics[width=0.25\textwidth]{fig/qualitative_new/2007_008415_real_image_cased.png} &
        \includegraphics[width=0.25\textwidth]{fig/qualitative_new/2007_008415_real_image.png} &
        \includegraphics[width=0.25\textwidth]{fig/qualitative_new/2007_008415_ground_truth.png} \\[0.2cm]
        
        
        \textbf{Image} &
        \textbf{ZeroSeg} & 
        \textbf{Chicken-and-egg} (CaSED) & 
        \textbf{Chicken-and-egg} (RAM) & 
        \textbf{GT}
    \end{tabular}
    }
    \caption{Comparison of segmentation results across ZeroSeg \cite{rewatbowornwong2023zero} and \textbf{Chicken-and-Egg} (CaSED \cite{conti2024vocabulary} and RAM \cite{zhang2024recognize}), and ground-truth labels.}
    \label{fig:sota_qualitative_comparison}
\end{figure*}





\begin{table*}[t]
\centering
\caption{Results over the benchmark datasets by using soft assignment. † Results come from their original work. * mapped with Llama-2 \cite{ulger2024autovocabularysemanticsegmentation} rather than Sentence-BERT \cite{reimers2019sentence}. The soft assignment has threshold zero (i.e., all the words are assigned to a class in the evaluation vocabulary).}
\label{tab:mapping_results}
\resizebox{.99\textwidth}{!}{%
    \begin{tabular}{l|cc|cc|cccccccccc}
    \toprule
    \multirow{2}{*}{\textbf{Method}} & \multirow{2}{*}{\begin{tabular}[c]{@{}c@{}}\textbf{Vision}\\ \textbf{Backbone}\end{tabular}} & \multirow{2}{*}{\textbf{Stages}} &\multicolumn{2}{c}{\textbf{Components}} & A-847 & PC-459 & A-150 & PC-59 & VOC-20 \\
     & & & Tagging & Segmentation & &&&&& \\ \midrule
    Zero-Seg† \cite{rewatbowornwong2023zero} & ViT-B/16 & Mask2Tag & CLIP+GPT-2 & DINO & -- & -- & -- & 11.2 & 8.1 \\
    Auto-Seg† \cite{ulger2024autovocabularysemanticsegmentation} & ViT-L/16 & Tag2Mask & BLIP-2 & X-Decoder & 5.9* & -- & -- & 11.7* & \textbf{87.1}* \\
    TAG† \cite{kawano2024tag} & ViT-L/14 & Mask2Tag &CLIP+DB & DINO & -- & -- & 6.6 & 20.2 & 56.9 \\
    \textbf{Chicken-and-egg} (CaSED) & ViT-B/16 & Tag2Mask & CLIP+DB & CAT-Seg & 4.3 & 3.1 & 7.8 & \textbf{27.9} & 82.3 \\
    \textbf{Chicken-and-egg} (RAM) & ViT-B/16 & Tag2Mask & CLIP+Swin & CAT-Seg & \textbf{6.7} & \textbf{8.0} & \textbf{18.8} & 27.8 & 81.8 \\
        \bottomrule
    \end{tabular}
}
\end{table*}

\subsection{Benchmark Evaluation}\label{sec:benchmark}
We first conducted a comprehensive benchmark evaluation comparing existing approaches to establish a strong foundation for VSS and identify the most promising direction. This analysis served two key purposes: (1) to understand the current state-of-the-art performance in VSS and (2) to determine which baseline architecture would be the most suitable.

\textbf{Quantitatives:} \Cref{tab:sota_results} compares the mIoU across the Open-Vocabulary benchmarks. The proposed pipeline outperforms the previous VSS methods by a constant margin in all the datasets. To better accommodate VSS methods, they adopt a class remapping strategy that reduces penalization in cases where an exact class match is not found. This approach is reflected in \Cref{tab:mapping_results}, where the soft evaluation assignment takes place as described in \Cref{sec:assignment}.

\textbf{Qualitatives:} As shown in \Cref{fig:sota_qualitative_comparison}, the current approach fills the gap between the predictions and original dataset labels without a predefined vocabulary, offering finer-grained details across diverse scenarios (indoor and outdoor). %
The maps obtained suggest that current evaluation metrics might be overly pessimistic about the qualitative performance of the results. This issue arises from dataset limitations, where many instances struggle to find appropriate matches (e.g., in the third image, "husky" instead of "dog"). Mask2Tag methods like ZeroSeg \cite{rewatbowornwong2023zero} tend to over-segment the instances, getting improper text matches. On the other hand, Chicken-and-egg with CaSED tends to limit the number of predicted tags, %
while coupled with RAM it reaches the best compromise.




\subsection{Segmentatation Analysis} \label{sec:text}
\textbf{Perfect Tagger:} Our empirical results on the OVSS task - presented in \Cref{tab:gt_labels} - revealed that providing only image-specific text labels, %
rather than the entire vocabulary, during training led to improved segmentation performance when applying the same adjustment at inference. Although having access to inference labels is unrealistic, this setup represents the best achievable performance if tagger predictions were 100\% accurate. 
More in detail, in \Cref{tab:gt_labels}, the set of class names is defined for each batch as \(\mathcal{C}_b \subset \mathcal{C}\) during training, where \(\mathcal{C}_b\) represents the batch-specific subset of the entire class vocabulary \(\mathcal{C}\), dynamically selected based on the batch's unique context or requirements. This subset approach allows the model to focus on relevant classes without being overwhelmed by the entire vocabulary. However, we observed no gain when the text labels in inference are predicted from an image tagger. Nevertheless, this represents the upper bound currently obtainable with the state-of-the-art open-vocabulary method \cite{cho2024cat}. Moreover, we show in \Cref{tab:attvsadj} that when performing inference on perfect predictions (100\% accuracy from the tagger) we can boost performance by providing additional textual information.
\begin{table}[t]
\centering
\caption{\textbf{Comparison using CAT-Seg \cite{cho2024cat}, using ground truth classes as text embeddings at different stages}, where $T$ represents training and $I$ represents inference. The mIoU is reported on ADE-20K (A)\cite{zhou2019semantic}, Pascal Context (PC)\cite{mottaghi2014role}, and Pascal VOC (VOC) \cite{everingham2010pascal}.
}
\label{tab:gt_labels}
\resizebox{\columnwidth}{!}{%
\begin{tabular}{ccccccccc}
\toprule
\textbf{Method} & \multicolumn{2}{c}{\textbf{\begin{tabular}[c]{@{}c@{}}Only GT\\ Text Labels\end{tabular}}} & COCO & A-847 & PC-459 & A-150 & PC-59 & VOC-20 \\ \cmidrule{2-3}
 & T & I &  &  &  &  &  &  \\ \midrule
Base &  &  & 47.11 & 11.95 & 18.95 & 31.78 & 57.20 & 95.30 \\
L.Bound & \checkmark &  & 43.73 & 10.89 & 16.63 & 30.29 & 55.99 & 94.20 \\
U.Bound (I) &  & \checkmark & 56.15 & 12.38 & 18.38 & 45.53 & 69.77 & \textbf{95.87} \\
U.Bound & \checkmark & \checkmark & \textbf{64.03} & \textbf{13.98} & \textbf{24.04} & \textbf{51.21} & \textbf{72.79} & 94.38 \\
\bottomrule
\end{tabular}
}
\end{table}

\begin{table}[t]
\centering
\caption{All methods are based on \cite{cho2024cat}, changing textual descriptors, while performing inference on GT classes. (a)-(c) are trained using the predicted VLM information on COCO dataset.
}
\label{tab:attvsadj}
\resizebox{\columnwidth}{!}{%
\begin{tabular}{ccccccccc}
\toprule
\textbf{Method} & \multicolumn{2}{c}{\textbf{VLM input}} & COCO & A-847 & PC-459 & A-150 & PC-59 \\ \cmidrule{2-3}
& $Image$& $Text$ & & & \\ \midrule
Baseline \cite{cho2024cat} & & & 56.15 & 12.38 & 18.38 %
& 45.53 & 69.77 &\\ %
(a) Caption & \checkmark & & 58.17 & 12.71 & 17.07 %
& 47.09 & 71.04 &\\ %
(b) Class Adjectives & & \checkmark & 62.33 & 14.96 & 19.13 & 48.77 & 60.47 \\ %
(c) Instance Adjectives & \checkmark & \checkmark & \textbf{65.13} & \textbf{15.40} & \textbf{23.20} & \textbf{54.43} & \textbf{72.04} \\ %
\bottomrule
\end{tabular}
}
\end{table}

\begin{table*}[t]
\centering
\caption{Prompts for different algorithms for \cref{tab:attvsadj} results.}
\label{tab:prompt}
\resizebox{.9\textwidth}{!}{%
\begin{tabular}{cc}
\toprule
\textbf{\begin{tabular}[c]{@{}c@{}}Description\\ Level\end{tabular}} & \textbf{Prompts} \\ \midrule
Class & \begin{tabular}[c]{@{}c@{}}1. "Please group the classes in this list $<$dataset-class-list$>$ into groups of classes that are similar to each other \\ meaning they could be confused in an image. Every class should be in one group and only in one group. \\ Make sure there are no classes from the original list missing in your grouping. \\ This is an example of how the output should look: dog, cat, kitten, bird -- couch, desk, sofa, lamp -- knife, fork, plate"\\ 2. " The classes in the group are: $<$group$>$. Please generate a short list of adjectives for each class \\ that describe how the object looks in an image. The adjectives should be distinctive within each group meaning that \\ the same attribute should not appear for two classes in the same group. Generate at least one adjective for each class. \\ This is an example how the output should look. {giraffe: [tall, brown, spotted, yellow], tree: [tall, green], armchair: [comfortable]}\end{tabular}\\ \midrule
Instance & \begin{tabular}[c]{@{}c@{}}"The objects in the image are: $<$dataset-class-list$>$. Please generate a short list of adjectives\\ for each object that describes how the object looks in the image. \\ This is an example of how the output should look. \{giraffe: {[}tall, brown, spotted, interacting{]}, tree: {[}tall, green, leafy{]}\}"\end{tabular} \\ \bottomrule
\end{tabular}
}
\end{table*}

\textbf{Aiding Text Encoder with Descriptions:} Previous works \cite{ma2024open} used adjectives with the assumption to find the common class features that better describe each class. For example, a "dalmata" could be described as "a white dog with black spots". However, in typical recognition tasks, the categories are much broader, such as simply "dog", and a "dog" could be described very differently in terms of color and size. Hence, AttrSeg \cite{ma2024open} have focused on training strategies to find the optimal set of descriptions that could enhance class distinguishability while still being able to represent each class. While this approach has merit, it can result in the loss of fine-grained details. For instance, a "table" or "hat" could be of any size or color, and even a "wall" that is typically "white" could be "bricked" or some other texture.
Zhao et al. \cite{zhao2024gradient} experimented on CLIP's ability to identify different types of object attributes, including shape, material, color, size, and position. 
For shape and material attributes, CLIP showed a certain but limited knowledge, with the heat maps highlighting partial correct attention on obvious objects, but also exhibiting false positive and false negative errors. For color attributes, the results further verified that CLIP has a good ability to distinguish different colors.
For comparative attributes like size and position, CLIP produced some erroneous results, demonstrating that it relies more on the primary object (e.g., "cube", "red") rather than the comparative attribute (e.g., "small", "left"). Overall, their analysis suggests that CLIP has advantages with common perceptual attributes.
Therefore, we adopted a pre-trained VLM to find the corresponding descriptions given each image and its specific set of class names - the text labels of each image-, and we tried to enforce general language descriptions.
The prompts used are shown in \Cref{tab:prompt}. For generating captions, we employed the BLIP-2 model \cite{li2023blip} without any query input, whereas for the multimodal model, we utilized Llava-1.6 \cite{liu2024llavanext}. These models were selected because they both incorporate CLIP as their text encoder.
The text embedding of the captions is employed as a query within an additional cross-attention module, linking it to the embeddings of the classes. In the case of the adjectives, they are sampled and used within the template "A photo of a \{adjective\} \{class name\}".
We report the results in table \Cref{tab:attvsadj}, where adding image-specific content results beneficial, specially for large numbers of classes.
It is important to notice that, when using predicted labels from the image tagger or applying the complete set of image labels during inference, we did not observe the same benefit. %
In the VSS scenario, ambiguities with other classes are largely resolved during the CLIP segmentation stage by directly predicting the image's content using the image tagger. However, misclassifications may still occur at this stage, a behaviour explored in the next paragraph.
\begin{table}[t]
\centering
\caption{Class recognition accuracy of different VLMs with $T_\text{SBERT}$=$0.0$. \\ * using vocabulary.
\# FN = average number of missed classes, \# FP = average number of classes predicted but not in the ground truth.}
\label{tab:accuracy}
\resizebox{.5\textwidth}{!}{%
\begin{tabular}{ccccccccccc}
\toprule
\multirow{2}{*}{\textbf{\begin{tabular}[c]{@{}c@{}}Predicted\\ Classes\end{tabular}}} & \multirow{2}{*}{\textbf{\begin{tabular}[c]{@{}c@{}}Mapping\\ Model\end{tabular}}} & \multicolumn{3}{c}{A-150} & \multicolumn{3}{c}{PC-59} & \multicolumn{3}{c}{VOC-20} \\ \cmidrule{3-11} 
 &  & Acc & \#FP & \#FN & Acc & \#FP & \#FN & Acc & \#FP & \#FN \\ \midrule
CaSED & - & 10 & 10.7 & 7.8 & 22 & 9.3 & 4.0 & 50 & 9.5 & 0.9 \\
CaSED & SBERT & 23 & 7.4 & 6.8 & 42 & 5.4 & 3.1 & 84 & 4.2 & 0.3 \\
Llava-1.6 & - & 26 & 4.9 & 6.3 & 29 & 3.7 & 3.5 & 53 & \textbf{3.5} & 0.8 \\
Llava-1.6 & SBERT & 39 & \textbf{2.6} & 5.2 & 47 & \textbf{1.8} & 2.7 & 91 & 1.9 & 0.2 \\
RAM & - & 34 & 10.4 & 5.9 & 41 & 11.8 & 3.1 & 68 & 12.2 & 0.5 \\
RAM & SBERT & \textbf{46} & 5.7 & \textbf{4.8} & \textbf{61} & 5.4 & \textbf{2.2} & \textbf{96} & 4.8 & \textbf{0.1} \\
\midrule
RAM* & - & 79 & 16.7 & 1.95 & 80 & 6.2 & 1.1 & 97 & 1.5 & 0.1  \\
\bottomrule
\end{tabular}
}
\end{table}

\subsection{Image Tagging Analysis}\label{sec:tagging}
In \Cref{tab:accuracy}, we investigated various image tagging methods to understand how different types of errors affect the sensitivity of the segmentation module, particularly the text encoder since we use the tags as input to CLIP. We evaluated the impact of three architectures: a training-free method, CaSED \cite{conti2024vocabulary}, a multi-step trained method, RAM \cite{zhang2024recognize}, and a general-purpose multimodal model, Llava \cite{liu2024llavanext}.
CaSED %
uses a pre-trained vision-language model and an external database to extract candidate categories and assign the image to the best match. 
On the other hand, RAM %
generates large-scale image tags through automatic semantic parsing, followed by training a model to annotate images using both captioning and tagging tasks. A data engine then refines these annotations, and the model is retrained on this enhanced data, with final fine-tuning on a higher-quality subset.
\Cref{tab:accuracy} shows that using SBERT for evaluation avoids discarding words merely due to the absence of an exact match with the chosen word by the annotators. RAM achieves the best overall results across the evaluated datasets. In the table, the performance of Llava \cite{liu2024llavanext} %
demonstrates the versatility of powerful vision-language architectures. Note that the current baseline, RAM, does not reach a perfect accuracy even when the whole list of desired classes (i.e., non-vocabulary free), hence this represents the current limitation of such an approach. 
Furthermore, compared to CaSED, RAM demonstrates higher class recognition accuracy, but with more false positives on average. To investigate this further, we examined in \Cref{fig:miss_vs_false_sim} how the model is influenced by simulating a drop rate and false positives on top of the ground truth text classes in each image. In the table, the false positives are randomly selected from the vocabulary set. The influence of false negatives deeply influences the performance, while introducing false positives only leads to marginal degradation. These results confirm why RAM outperforms current alternatives: it has the fewest misclassifications, despite having a higher rate of false positives.

\begin{figure}
    \centering
    \includegraphics[width=.9\columnwidth, trim=0cm 0.55cm 0cm 0.75cm, clip]{fig/fpfn_full_stefano.png}
    \caption{Simulating missing classes or adding wrong ones over the OVSS baseline by assuming the labels are known at inference time.}
    \label{fig:miss_vs_false_sim}
\end{figure}


\subsection{Evaluation Assignment Thresholds} \label{sec:thresholds}
In \Cref{fig:thresh} we show the effect of providing different values for $T_\text{SBERT}$. Unlike Zero-Seg \cite{rewatbowornwong2023zero}, we did not observe a consistent trend in the optimal threshold across datasets. Respectively, $0.6-0.7$ for A-847, PC-459 and A-150, $0.5$ for PC-59 and $0.1$ for VOC-20. 
Our findings suggest that as the number of classes increases, we need to be more confident in the assignment, hence a higher threshold leads to a better score.

\begin{figure}
    \centering
    \includegraphics[width=.9\columnwidth, trim=0cm 0.55cm 0cm 0.75cm, clip]{fig/thresholds_stefano.png}
    \caption{Ablation over different thresholds for the evaluation mapping.%
    }
    \label{fig:thresh}
\end{figure}





\section{Analysis}

In this section, we analyze the effects of EvoStealer's components, the iteration number, and the experimental costs.

\subsection{Ablation Study} \label{ablation}
\begin{figure}[t]
    \centering
    \setlength{\abovecaptionskip}{6pt}
    \begin{subfigure}[t]{0.5\linewidth}
        \centering
        \includegraphics[width=\linewidth]{sources/beta_scan.png}
        \caption{}
        \label{fig:beta_scan}
    \end{subfigure}%
    \hfill
    \begin{subfigure}[t]{0.5\linewidth}
        \centering
        \includegraphics[width=\linewidth]{sources/iteration_scan.png}
        \caption{}
        \label{fig:iteration_scan}
    \end{subfigure}
    \caption{Ablation study on hyperparameters. The left figure demonstrates the effect of varying $\beta$ values during training, while the right figure highlights the impact of search iterations in MCTS.}
    \label{fig:beta_and_iteration}
\end{figure}
%如我们在section3_5所述,在子代的质量评估阶段,EvoStealer会随机生成一张in-domain的图片,并计算其与目标图片之间的相似度分数作为适应度分数的一部分来指导EvoStealer的进化方向。然而,这种代价也是昂贵的,例如,若设置population size为5,iterations为5时,需要生成25张图片,大概越需要2美刀。那么,能否不生成图片而仅计算prompt和目标图片之间的相似度来指导计划呢?我们对此进行研究,结果如表2所示。从表中可以看到。当去掉图片生成后,EvoStealer的平均性能略微下滑,在in-domain和out-of-domain上的性能均有一定的下降。这说明,在适应度分数中添加图片之间的相似度分数能够更好地指导EvoStealer的进化,是的收敛速度更快。但考虑到平衡价格因素,EvoStealer的成本大幅度下降,且EvoStealer的性能仍然优于其他baselines。因此,在可接受的范围内,去掉图片生成是一种性能和代价折中的可接受方案。
% As described in Section~\ref{section_fitness}, during the offspring quality evaluation phase, EvoStealer randomly generates a in-domain image and calculates its similarity score with the target image. This score is then used as part of the fitness score to guide EvoStealer’s evolutionary direction. However, this process is costly. For example, with a population size of 5 and 5 iterations, it requires generating 25 images, resulting in an estimated cost of around 2 dollars. This raises the question: is it possible to guide the evolution process by calculating the similarity between the prompt and the target image, without generating images? We investigated this possibility, and the results are shown in Table~\ref{tb:abalation}. As seen in the table, removing image generation leads to a decline in EvoStealer’s performance on both in-domain and out-of-domain data. This suggests that including the similarity score between images in the fitness score helps better guide EvoStealer's evolution, thereby accelerating convergence. However, considering cost balance, EvoStealer's overhead significantly decreases, while its performance still surpasses other baseline models. Therefore, removing image generation offers a trade-off between performance and cost, making it a viable solution within an acceptable range.

% 我们分别消去抽取模板中的supplements和适应性函数中的图片相似性评估,以探讨信息抽取和适应性函数对EvoStealer的影响,结果如表3所示。我们可以看到,消去任一模块都导致了整体性能的下降,特别是移除supplements,导致平均相似度下降2.79。这是因为,supplements提供了关于额外的细节描述,包括图片的细节和风格特征。如diffevo所述,supplements相对于单个modifire更长。因而移除supplements在视觉表现上更加明显。我们在附录D中提供了消去supplements的前后对比图。移除适应性函数的图片相似性评估,导致性能下降1.27。这说明,在适应性函数中添加生成图片和目标图片的相似度比较能够更好地指导进化过程,加速收敛。然而,这种下降幅度相对较小,我们认为适当扩充子代数量或者增加迭代轮数能够缓解此缺陷。
We remove the supplements from the extracted templates and the image similarity evaluation from the fitness function to examine their impact on EvoStealer. The results are shown in Table~\ref{tb:abalation}. As observed, removing either module results in decreased performance, with a more significant drop when supplements are removed—specifically, an average similarity reduction of 2.79\%. This is because supplements provide additional details, such as image features and style information. As noted in Section~\ref{diffevo}, supplements are longer than individual modifiers, so their removal has a more pronounced effect on visual performance. A comparison of the performance before and after removing supplements is provided in Appendix~\ref{app_ablation}. Removing the image similarity evaluation from the fitness function causes a performance decrease of 1.27\%, suggesting that including the comparison between the generated and target images in the fitness function helps guide the evolutionary process and accelerate convergence.
\begin{figure}[h!]
    \centering
    \includegraphics[width=\linewidth]{figures/test.pdf}
    \caption{The convergence curve of EvoStealer, with the left half showing changes in fitness score and the right half depicting performance changes of the optimal prompt template for in-domain and out-of-domain data.}
    \label{fig:iters}
\end{figure}

\begin{figure*}[h!]
    \centering
    \includegraphics[width=0.9\linewidth]{figures/cases.pdf}
    \caption{The attack results of EvoStealer compared to three baseline methods on both easy and hard examples. (a)-(d) represent EvoStealer, CLIP-Interrogator, PromptStealer, and BLIP2, respectively.}
    \label{fig:case}
\end{figure*}
\subsection{Effect of Number of Iterations}

% 我们随机选择20个cases去探讨EvoStealer的收敛,结果如图3所示。图左侧描述的是随着进化进行的适应性分数的变化,右侧描述的是最优模板在in-domain和out-of-domain上的分数变化。从中我们可以看到,随着进化的进行,fitness分数的最优分数和平均分数都在逐步增长,这表明随着进化的进行,EvoStealer能够产生适应度更高的子代。右图表明,随着进化的进行,EvoStealer所窃取的prompt template的质量也在逐步改善。不论是在in-domain还是out-of-domain数据上,提示模板的性能都在稳定的提升。
We select 10 groups of easy and 10 groups of hard cases to examine EvoStealer's convergence~(we use GPT-4o as the analysis model), with results shown in Figure~\ref{fig:iters}. The left section of the figure displays changes in the fitness score as evolution progresses, while the right section shows changes in the scores of the optimal templates for both in-domain and out-of-domain data. We observe that as evolution progresses, both the optimal and average fitness scores gradually increase, indicating that EvoStealer generates offspring with higher adaptability. The performance of the prompt templates steadily improves for both in-domain and out-of-domain data. Two examples are provided in Appendix~\ref{app_progress}


% \subsection{Impact of Different MLLMs}
% % 我们分别使用InterVL2-26B,GPT-4o-mini和GPT-4o进行对比,以探索不同的多模态模型对EvoStealer的影响。结果如表2所示。从中可知,GPT-4o性能最优,InterVL2-26B则相对于GPT-4o和GPT-4o-mini略微差一点,在in-domain和out-of-domain数据上相对于GPT-4o分别下降2.03%和2.28%。但对比表1中的其他baselines,使用InterVL2-26B的EvoStealer仍然显著优于其他baselines。由此我们可以得出两个结论。1. 使用视觉能力更强的模型,窃取的prompt template质量更优;2.现有的开源模型能力足够强,使用开源模型也能够窃取到质量高的prompt template。因此,在需要窃取大量的prompt template时,为了节省开支,使用开源模型是一种值得尝试的方式。
% We compare InterVL2-26B, GPT-4o-mini, and GPT-4o to investigate the impact of different multimodal models on EvoStealer. The results, shown in Table~\ref{tb:models}, indicate that GPT-4o performs the best, while InterVL2-26B slightly lags behind GPT-4o and GPT-4o-mini, with performance 2.03\% and 2.28\% lower than GPT-4o on in-domain and out-of-domain data, respectively. However, when compared to other baseline models in Table~\ref{tb:main_result}, EvoStealer based on InterVL2-26B still the other methods significantly. Two conclusions can be drawn from this: (1) Models with stronger visual capabilities generate higher-quality prompt templates; (2) Existing open-source models are sufficiently powerful, making them a feasible option for stealing high-quality prompt templates and reducing costs when large-scale prompt template stealing is required.

% % \vspace{-0.3cm}
\setlength{\tabcolsep}{0pt}
\renewcommand{\arraystretch}{0.95}
\setcounter{table}{1}
\begin{table*}[b]
    \small
    % \vspace{-5mm}
    \caption{Parametric models included in the experiments. Cond. = conditioning method, R.F. = receptive field in samples.
    PEQ = Parametric EQ, G = Gain, O = Offset, MLP = Multilayer Perceptron, RNL = Rational Non Linearity. Controllers: 
    .s = static, .d = dynamic, .sc = static conditional, .dc = dynamic conditional}
    \label{tab:models}
    % \vspace{-2mm}
    \centerline{
        \begin{tabular}{L{2.8cm}C{1.3cm}R{1.1cm}C{1.1cm}C{1.1cm}C{1.3cm}C{1.5cm}R{1.4cm}R{1.3cm}R{1.3cm}}
            \hline
            \hline
            Model
                & Cond.
                    & R.F.
                        & Blocks
                            & Kernel
                                & Dilation
                                    & Channels
                                        & \# Params 
                                            & FLOP/s 
                                                & MAC/s\\ 
            \hline
            TCN-F-45-S-16 & FiLM & 2047 & 5 & 7 & 4 & 16 & 15.0k & 736.5M & 364.3M\\
            TCN-TF-45-S-16 & TFiLM & 2047 & 5 & 7 & 4 & 16 & 42.0k & 762.8M & 364.2M\\
            TCN-TTF-45-S-16 & TTFiLM & 2047 & 5 & 7 & 4 & 16 & 17.3k & 744.0M & 367.4M\\
            TCN-TVF-45-S-16 & TVFiLM & 2047 & 5 & 7 & 4 & 16 & 17.7k & 740.4M & 366.2M\\
            \hline
            \hline
        \end{tabular}
    }
    \centerline{
        \begin{tabular}{L{2.8cm}C{1.3cm}R{1.1cm}C{1.2cm}C{2.3cm}C{1.5cm}R{1.4cm}R{1.3cm}R{1.3cm}}
            Model
                & Cond.
                    & R.F.
                        & Blocks
                            & State Dimension
                                & Channels
                                    & \# Params
                                        & FLOP/s 
                                            & MAC/s\\ 
            \hline
            S4-F-S-16 & FiLM & - & 4 & 4 & 16 & 8.9k & 135.2M & 53.8M\\
            S4-TF-S-16 & TFiLM & - & 4 & 4 & 16 & 30.0k & 155.6M & 53.8M\\
            S4-TTF-S-16 & TTFiLM & - & 4 & 4 & 16 & 10.2k & 141.0M & 56.3M\\
            S4-TVF-S-16 & TVFiLM & - & 4 & 4 & 16 & 11.6k & 138.9M & 55.3M\\
            \hline
            \hline
        \end{tabular}
    }
    \centerline{
        \begin{tabular}{L{3cm}C{7.2cm}R{1.4cm}R{1.3cm}R{1.3cm}}
            Model
                & Signal Chain
                    & \# Params
                        & FLOP/s 
                            & MAC/s\\
            \hline
            GB-C-DIST-MLP & PEQ.sc $\rightarrow$ G.sc $\rightarrow$ O.sc $\rightarrow$ MLP $\rightarrow$ G.sc $\rightarrow$ PEQ.sc & 4.5k & 202.8M & 101.4M\\
            GB-C-DIST-RNL & PEQ.sc $\rightarrow$ G.sc $\rightarrow$ O.sc $\rightarrow$ RNL $\rightarrow$ G.sc $\rightarrow$ PEQ.sc & 2.3k & 920.5k & 4.3k\\
            \hline
            GB-C-FUZZ-MLP & PEQ.sc $\rightarrow$ G.sc $\rightarrow$ O.dc $\rightarrow$ MLP $\rightarrow$ G.sc $\rightarrow$ PEQ.sc & 4.2k & 202.8M & 101.4M\\
            GB-C-FUZZ-RNL & PEQ.sc $\rightarrow$ G.sc $\rightarrow$ O.dc $\rightarrow$ RNL $\rightarrow$ G.sc $\rightarrow$ PEQ.sc & 2.0k & 988.9k & 3.6k\\
            \hline
            \hline
        \end{tabular}
    }
    % \vspace{-4mm}
\end{table*}

% \subsection{Effect of Population Size}



\subsection{Cost Analysis}
% \begin{table*}[]
\small
\centering
\setlength{\tabcolsep}{3.5pt}
\renewcommand{\arraystretch}{0.8}
\begin{tabular}{@{}cl|ccccc@{}}
\toprule
\textbf{\# Topics} & \textbf{Model} & \multicolumn{1}{l}{\textbf{\# Input Tokens}} & \multicolumn{1}{l}{\textbf{\# Output Tokens}} & \multicolumn{1}{l}{\textbf{\# LLM Calls}} & \multicolumn{1}{l}{\textbf{Cost (GPT-4)}} & \multicolumn{1}{l}{\textbf{Time (seconds)}} \\ \midrule
\multirow{3}{*}{2} & \modelTopic & 21383.08 & 3412.02 & 25.45 & 0.32 & 117.60 \\
 & Hierarchical & 31130.02 & 2536.66 & 13.15 & 0.39 & 83.13 \\
 & Incremental-\textit{Topic} & 59010.66 & 6115.04 & 15.15 & 0.77 & 214.39 \\ \midrule
\multirow{3}{*}{3} & \modelTopic & 30208.20 & 5040.38 & 37.38 & 0.45 & 149.54 \\
 & Hierarchical & 31144.83 & 2649.78 & 13.15 & 0.39 & 68.60 \\
 & Incremental-\textit{Topic} & 61344.07 & 8442.54 & 16.15 & 0.87 & 197.33 \\ \midrule
\multirow{3}{*}{4} & \modelTopic & 38286.40 & 6440.23 & 47.91 & 0.58 & 163.91 \\
 & Hierarchical & 31144.31 & 2740.31 & 13.15 & 0.39 & 88.75 \\
 & Incremental-\textit{Topic} & 62877.46 & 9966.45 & 17.15 & 0.93 & 312.55 \\ \midrule
\multirow{3}{*}{5} & \modelTopic & 47008.59 & 7918.92 & 58.94 & 0.71 & 186.32 \\
 & Hierarchical & 31160.88 & 2850.24 & 13.15 & 0.40 & 61.70 \\
 & Incremental-\textit{Topic} & 64893.95 & 11965.84 & 18.15 & 1.01 & 262.07 \\ \bottomrule
\end{tabular}
\caption{\label{appendix:table:cost_cqa} Number of LLM input/output tokens, LLM calls, GPT-4 Cost (USD), and Time (seconds) needed to run inference on a single DFQS example on ConflictingQA with the top-3 models. We report 5 runs and 20 examples.}
\end{table*}

\begin{table*}[]
\small
\centering
\setlength{\tabcolsep}{3.5pt}
\renewcommand{\arraystretch}{0.8}
\begin{tabular}{@{}cl|ccccc@{}}
\toprule
\multicolumn{1}{l}{\textbf{Dataset}} & \textbf{Model} & \multicolumn{1}{l}{\textbf{\# Input Tokens}} & \multicolumn{1}{l}{\textbf{\# Output Tokens}} & \multicolumn{1}{l}{\textbf{\# LLM Calls}} & \multicolumn{1}{l}{\textbf{Cost (GPT-4)}} & \multicolumn{1}{l}{\textbf{Time (seconds)}} \\ \midrule
\multirow{3}{*}{2} & \modelTopic & 17183.75 & 2722.40 & 20.30 & 0.25 & 94.81 \\
 & Hierarchical & 19181.59 & 2040.39 & 10.25 & 0.25 & 63.68 \\
 & Incremental-\textit{Topic} & 41656.87 & 5062.44 & 12.25 & 0.57 & 182.19 \\ 
 \midrule
\multirow{3}{*}{3} & \modelTopic & 24801.22 & 4136.12 & 30.40 & 0.37 & 126.83 \\
 & Hierarchical & 19182.58 & 2141.91 & 10.25 & 0.26 & 53.32 \\
 & Incremental-\textit{Topic} & 43119.51 & 6532.92 & 13.25 & 0.63 & 152.44 \\ \midrule
\multirow{3}{*}{4} & \modelTopic & 30677.67 & 5037.31 & 38.00 & 0.46 & 120.64 \\
 & Hierarchical & 19203.30 & 2253.17 & 10.25 & 0.26 & 73.35 \\
 & Incremental-\textit{Topic} & 43922.02 & 7327.88 & 14.25 & 0.66 & 241.54 \\ \midrule
\multirow{3}{*}{5} & \modelTopic & 36988.41 & 6049.93 & 46.09 & 0.55 & 139.71 \\
 & Hierarchical & 19211.74 & 2356.01 & 10.25 & 0.26 & 49.41 \\
 & Incremental-\textit{Topic} & 45113.12 & 8504.59 & 15.25 & 0.71 & 186.40 \\ \bottomrule
\end{tabular}
\caption{\label{appendix:table:cost_debate} Number of LLM input/output tokens, LLM calls, GPT-4 Cost (USD), and Time (seconds) needed to run inference on a single DFQS example on DebateQFS with the top-3 models. We report 5 runs and 20 examples.}
\end{table*}

\begin{table*}[]
\small
\centering
\setlength{\tabcolsep}{3.5pt}
\renewcommand{\arraystretch}{0.8}
\begin{tabular}{@{}cl|ccccc@{}}
\toprule
\multicolumn{1}{l}{\textbf{\# Topics}} & \textbf{Model} & \multicolumn{1}{l}{\textbf{\# Input Tokens}} & \multicolumn{1}{l}{\textbf{\# Output Tokens}} & \multicolumn{1}{l}{\textbf{\# LLM Calls}} & \multicolumn{1}{l}{\textbf{Cost (GPT-4)}} & \multicolumn{1}{l}{\textbf{Time (seconds)}} \\ 
\midrule
\multirow{3}{*}{ConflictingQA} & \modelTopic & 47008.59 & 7918.92 & 58.94 & 0.71 & 186.32 \\
 & \modelTopic Pick All & 53733.70 & 9596.75 & 71.75 & 0.83 & 303.13 \\
 & Hierarchical-\emph{Topic} & 168160.85 & 7485.50 & 66.75 & 1.91 & 210.80 \\ \midrule
\multirow{3}{*}{DebateQFS} & \modelTopic & 36988.41 & 6049.93 & 46.09 & 0.55 & 139.71 \\
& \modelTopic Pick All & 43098.85 & 7612.45 & 57.25 & 0.66 & 242.35 \\
& Hierarchical-\emph{Topic} & 105237.25 & 5278.35 & 52.25 & 1.21 & 139.96 \\ \bottomrule
\end{tabular}
\caption{\label{appendix:table:cost_weird} Number of LLM input/output tokens, LLM calls, GPT-4 Cost (USD), and Time (seconds) needed to run inference on a single DFQS example on ConflictingQA and DebateQFS with \modelTopic, the version of \modelTopic with no Moderator, and the version of Hierarchical merging that runs on each topic paragraph ($m=5$). We report 5 runs and 20 examples.}
\end{table*}


% 为了证明EvoStealer的实用性,我们对EvoStealer窃取一个prompt template进行代价分析。EvoStealer的开销主要包含3个部分:种群吹实话,差分进化(包含fitness函数)以及最终的图片生成。附录D提供了详细的估算过程介绍。由估算结果可知,EvoStealer窃取一个模板需要调用144次API,生成34张图像(包含9张最终的合成图像),共计约需消耗119.1k token,花费1.7美元。尽管所需费用少于平台的3-9美刀售价,但是并不具备显著的优势。然而,根据5.1的消融可知,EvoStealer可通过采用开源模型或者去掉fitness 函数中的图片相似度比较来降低开支,实现接近0成本的窃取。尽管它的性能不如完整的EvoStealer,但性能仍远超其他方法。

To assess the practicality of EvoStealer, we analyzed the cost of stealing a prompt template. The primary overhead of EvoStealer consists of three components: population initialization, differential evolution (including the fitness function), and image synthesis. A detailed cost estimation process is provided in Appendix~\ref{app_cost}. The results indicate that EvoStealer requires 144 API calls, generates 34 images (including 9 final synthesized images), and consumes approximately 119.1k tokens, amounting to a total cost of \$1.70. While this is lower than the platform’s pricing range of \$3–9, the cost advantage is not substantial. However, as demonstrated in the ablation study in Section~\ref{ablation}, costs can be further reduced by using open-source models or omitting image similarity calculations in the fitness function, enabling near-zero-cost stealing. Although this cost-reduced version performs slightly worse than the full EvoStealer model, it still significantly outperforms alternative approaches.


\section{Conclusion and future work}
In this study, we examined the ability of LLMs to produce self-generated counterfactual explanations (SCEs).
We design a prompt-based setup for evaluating the efficacy of \SCEs.
Our results show that LLMs consistently struggle with generating valid \SCEs. In many cases model prediction on a \SCE does not yield the same target prediction for which the model crafted the \SCE.
Surprisingly, we find that LLMs put significant emphasis on the context---the prediction on \SCE is significantly impacted by the presence of original prediction and instructions for generating the \SCE.
Based on this empirical evidence, we argue that LLMs are still far from being able to explain their own predictions counterfactually.
Our findings add to similar insights from recent studies on other forms of self-explanations~\cite{lanham2023measuring,tanneru2024quantifying}.



Our work opens several avenues for future work. Inspired by counterfactual data augmentation~\cite{sachdeva2023catfood}, one could include the counterfactual explanation capabilities a part of the LLM training process. This inclusion may enhance the counterfactual reasoning capabilities of the LLM. Follow ups should also explore the effect of prompt tuning, specifically, model-tailored prompts for generating \SCEs. These approaches might lead to better quality \SCEs.


We limited our investigation to open source models of upto 70B parameters. Extending our analysis to larger and more recent models, \eg, DeepSeek R1 671B, and closed source models like OpenAI o3 would be an interesting avenue for future work.

Finally, our experiments were limited to relatively simple tasks: classification and mathematics problems where the solution is an integer. This limitation was mainly due to the fact that it is difficult to automatically judge validity of answers for more open-ended language generation tasks like search and information retrieval. Scaling our analysis to such tasks would require significant human-annotation resources, and is an important direction for future investigations.


\normalem
\bibliography{example_paper}
\bibliographystyle{icml2025}

%%%%%%%%%%%%%%%%%%%%%%%%%%%%%%%%%%%%%%%%%%%%%%%%%%%%%%%%%%%%%%%%%%%%%%%%%%%%%%%
%%%%%%%%%%%%%%%%%%%%%%%%%%%%%%%%%%%%%%%%%%%%%%%%%%%%%%%%%%%%%%%%%%%%%%%%%%%%%%%
% APPENDIX
%%%%%%%%%%%%%%%%%%%%%%%%%%%%%%%%%%%%%%%%%%%%%%%%%%%%%%%%%%%%%%%%%%%%%%%%%%%%%%%
%%%%%%%%%%%%%%%%%%%%%%%%%%%%%%%%%%%%%%%%%%%%%%%%%%%%%%%%%%%%%%%%%%%%%%%%%%%%%%%
\newpage

\newpage
\appendix
\onecolumn

\renewcommand{\thetable}{A\arabic{table}} % Prefix table numbers with 'A'
\renewcommand{\thefigure}{A\arabic{figure}} % Prefix figure numbers with 'A'
\renewcommand{\theequation}{A\arabic{equation}} % Prefix equation numbers with 'A'

\setcounter{table}{0} % Reset table counter
\setcounter{figure}{0} % Reset figure counter
\setcounter{equation}{0} % Reset equation counter

\section*{Appendix}

\section{Optimal Brain Surgeon Derivation}
\label{OBS_ALGORITHM}

In the original setup in OBS, we have a local quadratic model for the loss $L$ given by:
$$
    \delta L = L(w + \delta w) \approx L(w) + \nabla_w L^T \delta w + \frac{1}{2} \delta w^T H \delta w
$$
Since OBS is a pruning-after-training approach, they discarded the 1-st order component. Reducing the expression for saliency as:
$$
    \delta L = \frac{1}{2} \delta w^T H \delta w
$$
To remove a single parameter, the authors of OBS introduced the constraint $e_q^T \delta w + w_q = 0$, with $e_q$ being the $q^{\text{th}}$ canonical basis vector. The pruning is defined as a constrained optimization problem of the form:
$$
    \min_{\delta w \in \mathbb{R^d}} \left( \frac{1}{2} \delta w^T H \delta w\right),
    ~~\text{s.t}~~
    e_q^T \delta w + w_q = 0.
$$
And the choice of which parameter to remove becomes:
$$
    \min_{q \in \mathcal{Q}} \left\{
        \min_{\delta w \in \mathbb{R^d}} \left( \frac{1}{2} \delta w^T H \delta w\right),
        ~~\text{s.t}~~
        e_q^T \delta w + w_q = 0
    \right\}.
$$
To solve the internal problem, we use a Lagrange multiplier $\lambda$ to write the problem as an unconstrained optimization case as follows:
$$
    \mathcal{L}(\delta w, \lambda) =
    \frac{1}{2} \delta w^T H \delta w +
    \lambda(e_q^T \delta w + w_q).
$$
Then, to find the stationary conditions, we compute the partial derivatives with respect to $\delta w$ and $\lambda$, and equate them to 0, obtaining:
$$
    \nabla_{\delta w} \mathcal{L} = 
    H \delta w + \lambda e_q = 0 
    \rightarrow
    \delta w = - \lambda H^{-1} e_q
$$
$$
    \nabla_{\lambda} \mathcal{L} =
    e_q^T \delta w + w_q = 0
    \rightarrow
    e_q^T \delta w = -w_q
$$
With some replacements, we get:
$$
    e_q^T \delta w = -w_q
    \rightarrow
    e_q^T \left( 
        - \lambda H^{-1} e_q
    \right) = -w_q
    \rightarrow
    - \lambda e_q^T H^{-1} e_q = -w_q
    \rightarrow
    \lambda = \frac{w_q}{e_q^T H^{-1} e_q} = \frac{w_q}{[H^{-1}]_{qq}}
$$
$$
    \delta w = - \frac{w_q H^{-1} e_q}{[H^{-1}]_{qq}}
$$
Replacing the expression for $\delta w$ in the saliency expression, we have:
\begin{align*}
    \delta L = \frac{1}{2} \delta w^T H \delta w
    &= \frac{1}{2}\left(
        - \frac{w_q H^{-1} e_q}{[H^{-1}]_{qq}}
    \right)^T
    H
    \left(
        - \frac{w_q H^{-1} e_q}{[H^{-1}]_{qq}}
    \right)
    \nonumber \\
    &= 
    \frac{w_q^2}{2[H^{-1}]_{qq}^2}
    \left(
        H^{-1} e_q
    \right)^T
    H
    \left(
        H^{-1} e_q
    \right)
    \nonumber \\
    &= 
    \frac{w_q^2}{2[H^{-1}]_{qq}^2}
    e_q ^T
    H^{-1}
    e_q
    = 
    \frac{w_q^2}{2[H^{-1}]_{qq}^2}
    [H^{-1}]_{qq}
    = 
    \frac{w_q^2}{2[H^{-1}]_{qq}}
    \nonumber \\
\end{align*}
%------------------------------------------------------------------------------------------------
\newpage
\section{Fisher Brain Surgeon Sensitivity Derivation}
\label{FBSS_ALGORITHM}
As we considered a PBT setting, it is not possible to ignore the first-order term in the local quadratic approximation of the error as it could still be informative. In this case, our model for sensitivity is given by: 
$$
    \delta L = \nabla_w L^T \delta w + \frac{1}{2} \delta w^T H \delta w
$$
The process to remove a single parameter remains similar; the constraint $e_q^T \delta w + w_q = 0$, with $e_q$ is still valid, redefining the optimization problem as:
$$
    \min_{\delta w \in \mathbb{R^d}} \left(
        \nabla_w L^T \delta w +  \frac{1}{2} \delta w^T H \delta w
    \right),
    ~~\text{s.t}~~
    e_q^T \delta w + w_q = 0.
$$
And the choice of which parameter to remove becomes:
$$
    \min_{q \in \mathcal{Q}} \left\{
        \min_{\delta w \in \mathbb{R^d}} \left(
            \nabla_w L^T \delta w + \frac{1}{2} \delta w^T H \delta w
        \right),
        ~~\text{s.t}~~
        e_q^T \delta w + w_q = 0
    \right\}.
$$
Using a Lagrange multiplier $\lambda$ as in the reference case, we solve the following unconstrained optimization problem:
$$
    \mathcal{L}(\delta w, \lambda) =
    \nabla_w L^T \delta w + 
    \frac{1}{2} \delta w^T H \delta w +
    \lambda(e_q^T \delta w + w_q).
$$
With the following stationary conditions:
$$
    \nabla_{\delta w} \mathcal{L} = 
    \nabla_w L + H \delta w + \lambda e_q = 0 
    \rightarrow
    \delta w = - (\lambda H^{-1}e_q + H^{-1} \nabla_w L)
$$
$$
    \nabla_{\lambda} \mathcal{L} =
    e_q^T \delta w + w_q = 0
    \rightarrow
    e_q^T \delta w = -w_q
$$
The expression for $\lambda$ is redefined as follows:
\begin{align*}
    e_q^T \left(
        - (\lambda H^{-1}e_q + H^{-1} \nabla_w L)
    \right) 
    &= -w_q
    \nonumber \\
    \lambda e_q^T H^{-1} e_q + e_q^T H^{-1} \nabla_w L
    &= w_q
    \nonumber \\
    \lambda [H^{-1}]_{qq} 
    &= w_q - e_q^T H^{-1} \nabla_w L
    \nonumber \\
    \lambda
    &= \frac{w_q - e_q^T H^{-1} \nabla_w L}{[H^{-1}]_{qq}}
\end{align*}
Replacing the expression for $\delta w$ in our sensitivity expression, we have:
\begin{align*}
    \delta L = \nabla_w L^T \delta w + \frac{1}{2} \delta w^T H \delta w
    &= 
    \nabla_w L^T \left[
        - (\lambda H^{-1}e_q + H^{-1} \nabla_w L)
    \right]
    \nonumber \\
    &+
    \frac{1}{2}\left[
        - (\lambda H^{-1}e_q + H^{-1} \nabla_w L)
    \right]^T
    H
    \left[
        - (\lambda H^{-1}e_q + H^{-1} \nabla_w L)
    \right]
    \nonumber \\
    &= 
    - \lambda \nabla_w L^T H^{-1}e_q - \nabla_w L^T H^{-1} \nabla_w L
    \nonumber \\
    &+
    \frac{1}{2}\left[
        (\lambda H^{-1}e_q)^T + (H^{-1} \nabla_w L)^T
    \right]
    \left[
        \lambda H H^{-1}e_q + H H^{-1} \nabla_w L)
    \right]
    \nonumber \\
    &= 
    - \lambda \nabla_w L^T H^{-1}e_q - \nabla_w L^T H^{-1} \nabla_w L
    \nonumber \\
    &+
    \frac{1}{2}\left[
        (\lambda H^{-1}e_q)^T + (H^{-1} \nabla_w L)^T
    \right]
    \left[
        \lambda e_q + \nabla_w L
    \right]
    \nonumber \\
    &= 
    - \lambda \nabla_w L^T H^{-1}e_q - \nabla_w L^T H^{-1} \nabla_w L
    \nonumber \\
    &+
    \frac{1}{2}\left[
        (\lambda H^{-1}e_q)^T \lambda e_q
        + (H^{-1} \nabla_w L)^T \lambda e_q
        + (\lambda H^{-1}e_q)^T \nabla_w L
        + (H^{-1} \nabla_w L)^T \nabla_w L
    \right]
    \nonumber \\
    &= 
    - \lambda \nabla_w L^T H^{-1}e_q - \nabla_w L^T H^{-1} \nabla_w L
    \nonumber \\
    &+
    \frac{1}{2}\left[
        \lambda^2 e_q^T H^{-1} e_q
        + \lambda \nabla_w L^T H^{-1} e_q
        + \lambda e_q^T H^{-1} \nabla_w L
        + \nabla_w L^T H^{-1} \nabla_w L
    \right]
    \nonumber \\
    &= 
    \frac{1}{2}\left[
        \lambda^2 [H^{-1}]_{qq}
        - \lambda \nabla_w L^T H^{-1} e_q
        + \lambda e_q^T H^{-1} \nabla_w L
        - \nabla_w L^T H^{-1} \nabla_w L
    \right]
    \nonumber \\
\end{align*}
Finally, replacing the $\lambda$:
\begin{align*}
    \delta L 
    &= 
    \frac{1}{2}\left[
        \lambda^2 [H^{-1}]_{qq}
        - \lambda \nabla_w L^T H^{-1} e_q
        + \lambda e_q^T H^{-1} \nabla_w L
        - \nabla_w L^T H^{-1} \nabla_w L
    \right]
    \nonumber \\
    &= 
    \frac{1}{2[H^{-1}]_{qq}}\left[
        (w_q - e_q^T H^{-1} \nabla_w L)^2 
        + (w_q - e_q^T H^{-1} \nabla_w L)(e_q^T H^{-1} \nabla_w L - \nabla_w L^T H^{-1} e_q)
        - \nabla_w L^T H^{-1} \nabla_w L
    \right]
    \nonumber \\
    &= 
    \frac{1}{2[H^{-1}]_{qq}}[
        w_q^2
        - 2 w_q (e_q^T H^{-1} \nabla_w L)
        + (e_q^T H^{-1} \nabla_w L)^2
        + w_q (e_q^T H^{-1} \nabla_w L)
    \nonumber \\
        &- w_q (\nabla_w L^T H^{-1} e_q)
        - (e_q^T H^{-1} \nabla_w L)(e_q^T H^{-1} \nabla_w L)
        + (e_q^T H^{-1} \nabla_w L)(\nabla_w L^T H^{-1} e_q)
        - \nabla_w L^T H^{-1} \nabla_w L
    ]
    \nonumber \\
    &= 
    \frac{1}{2[H^{-1}]_{qq}}[
        w_q^2
        - w_q (e_q^T H^{-1} \nabla_w L)
        + (e_q^T H^{-1} \nabla_w L)^2
    \nonumber \\
        &- w_q (\nabla_w L^T H^{-1} e_q)
        - (e_q^T H^{-1} \nabla_w L)^2
        + (e_q^T H^{-1} \nabla_w L)(\nabla_w L^T H^{-1} e_q)
        - \nabla_w L^T H^{-1} \nabla_w L
    ]
    \nonumber \\
    &= 
    \frac{1}{2[H^{-1}]_{qq}}\left[
        w_q^2
        - 2 w_q (e_q^T H^{-1} \nabla_w L)
        + (e_q^T H^{-1} \nabla_w L)^2
        - \nabla_w L^T H^{-1} \nabla_w L
    \right]
    \nonumber \\
    &= 
    \frac{1}{2[\hat{F}^{-1}]_{qq}}
    \left[
        w_q - (e_q^T \hat{F}^{-1} \nabla \mathcal{L}(w_0))
    \right]^2
\end{align*}

%------------------------------------------------------------------------------------------------

\newpage
\section{Training and Testing Details}
\label{appendix:training_parameters}

We perform an 80:20 stratified split, with a constant seed, on the CIFAR10/100 training dataset to obtain a validation set with the same class distribution. For both datasets, we have a training set with 40,000 samples, a validation set with 10,000 samples, and a testing set of 10,000 samples. Validation is performed after each training step, and the weights of the best-performing validation step (based on top-1 accuracy) are utilized for the final evaluation on the testing set. Table \ref{tab:table_training_parameters} summarizes the training parameters.

\begin{table}[h]
\caption{Training parameters used for ResNet18 and VGG19 on the CIFAR-10/100 datasets.}
\label{tab:table_training_parameters}
\vskip 0.15in
\begin{center}
\begin{small}
\begin{sc}
\begin{tabular}{lcc}
\toprule
Parameter & ResNet18 & VGG19 \\
\midrule
Number of steps       & 160 & 160 \\
Criterion             & CE & CE \\
Optimizer             & SGD & SGD \\
Learning rate         & 0.01 & 0.1 \\
Momentum              & 0.9 & 0.9 \\
Weight decay          & $5 \times 10^{-4}$ & $1 \times 10^{-4}$ \\
Learning rate drops   & [60, 120] & [60, 120] \\
Learning rate drop factor & 0.2 & 0.1 \\
\bottomrule
\end{tabular}
\end{sc}
\end{small}
\end{center}
\vskip -0.1in
\end{table}

%------------------------------------------------------------------------------------------------

\newpage
\section{Results CIFAR10}
\subsection{ResNet18}
\label{appendix:CIFAR10_ResNet18}

\begin{table}[h]
\caption{Performance of different sensitivity methods for pruning evaluated using ResNet18 on the CIFAR-10 testset. The right side of the table presents our proposed criteria. The mean accuracy and standard deviation are reported across three initialization seeds for various sparsity levels. Baseline, no pruning: $91.78 \pm 0.09$.}
\label{tab:resnet18_cifar10_compressors}
\vskip 0.15in
\begin{center}
\begin{small}
\begin{sc}
\resizebox{\textwidth}{!}{%
\begin{tabular}{lccccc|cccc}
\toprule
Sparsity  & Random & Magnitude & GN & SNIP & GraSP & FD & FP & FTS & FBSS \\
\midrule
0.10  & 91.71 ± 0.21 & 91.72 ± 0.07 & 91.57 ± 0.15 & 91.72 ± 0.07 & 89.16 ± 0.05 & 91.87 ± 0.13 & 91.63 ± 0.21 & 91.53 ± 0.12 & 91.76 ± 0.08 \\
0.20  & 91.63 ± 0.11 & 91.42 ± 0.12 & 91.51 ± 0.09 & 91.64 ± 0.16 & 88.69 ± 0.34 & 91.50 ± 0.12 & 91.65 ± 0.14 & 91.53 ± 0.15 & 91.54 ± 0.13 \\
0.30  & 91.45 ± 0.18 & 91.61 ± 0.13 & 91.68 ± 0.20 & 91.65 ± 0.08 & 88.67 ± 0.26 & 91.65 ± 0.18 & 91.44 ± 0.27 & 91.49 ± 0.05 & 91.62 ± 0.07 \\
0.40  & 91.59 ± 0.18 & 91.06 ± 0.16 & 91.61 ± 0.09 & 91.55 ± 0.08 & 88.24 ± 0.33 & 91.51 ± 0.05 & 91.38 ± 0.13 & 91.56 ± 0.28 & 91.39 ± 0.05 \\
0.50  & 91.60 ± 0.06 & 91.32 ± 0.13 & 91.44 ± 0.13 & 91.22 ± 0.07 & 87.69 ± 0.15 & 91.30 ± 0.18 & 91.58 ± 0.16 & 91.46 ± 0.19 & 91.41 ± 0.05 \\
0.60  & 91.10 ± 0.16 & 91.18 ± 0.16 & 91.59 ± 0.13 & 91.24 ± 0.04 & 87.48 ± 0.55 & 91.34 ± 0.07 & 91.35 ± 0.16 & 91.40 ± 0.11 & 91.38 ± 0.18 \\
0.70  & 91.17 ± 0.04 & 91.07 ± 0.07 & 91.19 ± 0.17 & 91.33 ± 0.18 & 87.26 ± 0.34 & 91.34 ± 0.23 & 91.42 ± 0.23 & 91.18 ± 0.18 & 91.27 ± 0.14 \\
0.80  & 90.78 ± 0.08 & 91.10 ± 0.12 & 90.95 ± 0.35 & 90.74 ± 0.10 & 87.18 ± 0.51 & 90.95 ± 0.11 & 91.08 ± 0.06 & 90.94 ± 0.22 & 90.73 ± 0.33 \\
0.90  & 89.35 ± 0.13 & 89.88 ± 0.28 & 90.39 ± 0.23 & 90.36 ± 0.34 & 86.60 ± 0.51 & 90.04 ± 0.21 & 90.20 ± 0.08 & 90.55 ± 0.23 & 89.22 ± 0.30 \\
0.95  & 87.59 ± 0.11 & 89.23 ± 0.19 & 89.00 ± 0.05 & 89.31 ± 0.17 & 86.50 ± 0.05 & 88.61 ± 0.28 & 89.50 ± 0.18 & 89.47 ± 0.32 & 87.58 ± 0.25 \\
0.98  & 83.47 ± 0.20 & 85.70 ± 0.33 & 86.43 ± 0.05 & 87.26 ± 0.28 & 85.99 ± 0.08 & 85.61 ± 0.20 & 86.97 ± 0.22 & 87.24 ± 0.32 & 83.40 ± 0.74 \\
0.99  & 78.28 ± 0.45 & 71.99 ± 0.28 & 83.47 ± 0.15 & 84.54 ± 0.04 & 84.56 ± 0.46 & 82.13 ± 0.28 & 83.74 ± 0.48 & 84.85 ± 0.18 & 77.60 ± 1.02 \\
\bottomrule
\end{tabular}}
\end{sc}
\end{small}
\end{center}
\vskip -0.1in
\end{table}

%------------------------------------------------------------------------------------------------
\clearpage
\subsection{VGG19}
\label{appendix:CIFAR10_VGG19}

As discussed earlier, introducing a warm-up phase effectively mitigates layer collapse in data-dependent pruning methods. Here, we evaluate the impact of different warm-up durations by comparing no warm-up, a single warm-up epoch, and five warm-up epochs. Table \ref{tab:VGG19_cifar10_compressors} demonstrates how performance drastically degrades with increasing sparsity, ultimately leading to layer collapse at 0.90 sparsity. However, as shown in the results, a single warm-up epoch is sufficient to prevent collapse and stabilize pruning performance. Moreover, as seen in Table \ref{tab:VGG19_cifar10_compressors_warmup5}, increasing the warm-up period to five epochs provides no substantial additional improvement. This indicates that prolonged warm-up training is not necessary; a single training step is enough to achieve gradient stabilization and overcome layer collapse.

\begin{table}[h]
\caption{Performance of different sensitivity methods for pruning evaluated using VGG19 on the CIFAR-10 test set. The right side of the table presents our proposed criteria. The mean accuracy and standard deviation are reported across three initialization seeds for various sparsity levels. Baseline, no pruning: $89.21 \pm 0.22$.}
\label{tab:VGG19_cifar10_compressors}
\vskip 0.15in
\begin{center}
\begin{small}
\begin{sc}
\resizebox{\textwidth}{!}{%
\begin{tabular}{lccccc|cccc}
\toprule
Sparsity  & Random & Magnitude & GN & SNIP & GraSP & FD & FP & FTS & FBSS \\
\midrule
0.10  & 88.40 ± 0.95 & 89.12 ± 0.55 & 90.14 ± 0.10 & 90.16 ± 0.18 & 87.81 ± 1.66 & 90.20 ± 0.29 & 90.21 ± 0.37 & 90.25 ± 0.38 & 89.06 ± 0.75 \\
0.20  & 89.19 ± 0.22 & 89.65 ± 0.60 & 89.59 ± 0.69 & 90.06 ± 0.04 & 89.57 ± 0.34 & 89.91 ± 0.28 & 90.28 ± 0.55 & 89.80 ± 0.28 & 88.89 ± 0.76 \\
0.30  & 88.93 ± 0.83 & 88.77 ± 1.07 & 90.23 ± 0.09 & 89.88 ± 0.59 & 89.14 ± 0.19 & 90.25 ± 0.09 & 89.97 ± 0.26 & 90.46 ± 0.41 & 89.06 ± 0.36 \\
0.40  & 88.28 ± 1.08 & 89.38 ± 0.53 & 90.50 ± 0.23 & 89.79 ± 0.67 & 88.20 ± 0.31 & 90.51 ± 0.12 & 90.37 ± 0.24 & 90.23 ± 0.14 & 10.00 ± 0.00 \\
0.50  & 88.96 ± 0.82 & 89.03 ± 0.59 & 90.46 ± 0.60 & 90.38 ± 0.25 & 88.67 ± 0.23 & 89.54 ± 0.86 & 90.47 ± 0.52 & 90.19 ± 0.31 & 10.00 ± 0.00 \\
0.60  & 88.15 ± 0.68 & 89.47 ± 0.18 & 89.95 ± 0.30 & 90.32 ± 0.25 & 88.82 ± 0.32 & 90.02 ± 0.40 & 90.18 ± 0.33 & 90.14 ± 0.36 & 10.00 ± 0.00 \\
0.70  & 88.02 ± 0.53 & 89.63 ± 0.44 & 89.69 ± 0.42 & 89.23 ± 0.19 & 89.62 ± 0.81 & 89.85 ± 0.08 & 90.01 ± 0.34 & 10.00 ± 0.00 & 10.00 ± 0.00 \\
0.80  & 88.28 ± 0.34 & 89.62 ± 0.91 & 85.72 ± 0.63 & 89.39 ± 0.43 & 88.82 ± 0.14 & 10.00 ± 0.00 & 88.29 ± 0.11 & 10.00 ± 0.00 & 10.00 ± 0.00 \\
0.90  & 85.82 ± 0.19 & 89.29 ± 0.79 & 10.00 ± 0.00 & 80.85 ± 0.62 & 24.28 ± 20.2 & 10.00 ± 0.00 & 10.00 ± 0.00 & 10.00 ± 0.00 & 10.00 ± 0.00 \\
0.95  & 84.41 ± 0.05 & 10.00 ± 0.00 & 10.00 ± 0.00 & 10.00 ± 0.00 & 10.00 ± 0.00 & 10.00 ± 0.00 & 10.00 ± 0.00 & 10.00 ± 0.00 & 10.00 ± 0.00 \\
0.98  & 80.04 ± 0.90 & 10.00 ± 0.00 & 10.00 ± 0.00 & 10.00 ± 0.00 & 10.00 ± 0.00 & 10.00 ± 0.00 & 10.00 ± 0.00 & 10.00 ± 0.00 & 10.00 ± 0.00 \\
0.99  & 76.89 ± 0.26 & 10.00 ± 0.00 & 10.00 ± 0.00 & 10.00 ± 0.00 & 10.00 ± 0.00 & 10.00 ± 0.00 & 10.00 ± 0.00 & 10.00 ± 0.00 & 10.00 ± 0.00 \\
\bottomrule
\end{tabular}}
\end{sc}
\end{small}
\end{center}
\vskip -0.1in
\end{table}
\newpage
%------------------------------------------------------------------------------------------------
\begin{table*}[h]
\caption{Performance of different compression methods evaluated after 1 warmup epoch using VGG19 on the CIFAR-10 dataset. We report the mean accuracy between three initialization seeds across various sparsity levels. Baseline, no pruning: $89.21 \pm 0.22$.}
\label{tab:VGG19_cifar10_compressors_warmup1}
\vskip 0.15in
\begin{center}
\begin{small}
\begin{sc}
\resizebox{\textwidth}{!}{%
\begin{tabular}{lccccc|cccc}
\toprule
Sparsity  & Random & Magnitude & GN & SNIP & GraSP & FD & FP & FTS & FBSS \\
\midrule
0.80  & 88.73 ± 0.38 & 88.35 ± 0.54 & 86.76 ± 0.27 & 87.39 ± 0.66 & 87.24 ± 0.25 & 87.14 ± 0.45 & 87.00 ± 0.87 & 87.68 ± 0.33 & 64.33 ± 15.91 \\
0.90  & 87.26 ± 0.42 & 88.62 ± 0.49 & 85.96 ± 0.75 & 86.75 ± 0.76 & 87.47 ± 0.33 & 86.69 ± 0.72 & 87.09 ± 0.31 & 87.42 ± 0.21 & 46.16 ± 7.62 \\
0.95  & 85.47 ± 0.64 & 87.68 ± 0.49 & 86.66 ± 0.27 & 86.00 ± 1.10 & 86.71 ± 1.24 & 85.71 ± 1.35 & 86.73 ± 0.36 & 87.56 ± 0.62 & 46.30 ± 5.32 \\
0.98  & 80.44 ± 0.30 & 86.61 ± 0.62 & 84.72 ± 1.69 & 87.22 ± 0.23 & 86.45 ± 0.64 & 80.34 ± 6.43 & 86.07 ± 0.39 & 86.36 ± 0.29 & 49.05 ± 4.31 \\
0.99  & 77.24 ± 0.73 & 83.69 ± 1.36 & 80.28 ± 2.04 & 83.49 ± 1.77 & 85.39 ± 0.43 & 75.11 ± 7.80 & 84.40 ± 1.27 & 85.35 ± 1.05 & 47.10 ± 4.41 \\
\bottomrule
\end{tabular}}
\end{sc}
\end{small}
\end{center}
\vskip -0.1in
\end{table*} 
%------------------------------------------------------------------------------------------------

\begin{table}[h]
\caption{Performance of different sensitivity methods for pruning evaluated after 5 warmup epochs using VGG19 on the CIFAR-10 testset. The right side of the table presents our proposed criteria. The mean accuracy and standard deviation are reported across three initialization seeds for various sparsity levels. Baseline, no pruning: $89.21 \pm 0.22$.}
\label{tab:VGG19_cifar10_compressors_warmup5}
\vskip 0.15in
\begin{center}
\begin{small}
\begin{sc}
\resizebox{\textwidth}{!}{%
\begin{tabular}{lccccc|cccc}
\toprule
Sparsity  & Random & Magnitude & GN & SNIP & GraSP & FD & FP & FTS & FBSS \\
\midrule
0.80  & 88.84 ± 0.43 & 88.41 ± 0.47 & 87.58 ± 0.52 & 88.15 ± 1.09 & 86.77 ± 1.14 & 87.28 ± 0.90 & 88.22 ± 0.82 & 86.68 ± 0.61 & 70.52 ± 9.25 \\
0.90  & 87.56 ± 0.62 & 88.60 ± 0.93 & 86.73 ± 0.37 & 87.89 ± 0.25 & 87.10 ± 0.47 & 87.50 ± 1.42 & 88.18 ± 0.47 & 86.98 ± 0.14 & 47.78 ± 1.26 \\
0.95 & 85.51 ± 0.69 & 87.66 ± 1.19 & 87.44 ± 0.46 & 87.71 ± 0.82 & 87.05 ± 0.16 & 86.83 ± 1.47 & 87.36 ± 0.52 & 87.00 ± 0.74 & 48.83 ± 2.52 \\
0.98 & 82.09 ± 0.17 & 86.24 ± 0.52 & 84.66 ± 1.33 & 86.55 ± 0.84 & 86.04 ± 0.66 & 85.44 ± 0.64 & 86.64 ± 0.13 & 84.89 ± 0.51 & 49.48 ± 0.85 \\
0.99 & 77.22 ± 1.03 & 83.93 ± 1.80 & 81.62 ± 2.17 & 84.53 ± 0.70 & 81.33 ± 5.77 & 81.71 ± 1.41 & 85.02 ± 0.69 & 83.78 ± 0.80 & 41.24 ± 1.55 \\
\bottomrule
\end{tabular}}
\end{sc}
\end{small}
\end{center}
\vskip -0.1in
\end{table}

%------------------------------------------------------------------------------------------------

\newpage
\section{Results CIFAR100}
\subsection{ResNet18}
\label{sec:resnet_cifar-100}

CIFAR-100 results exhibit a similar trend to those observed on CIFAR-10, further reinforcing the robustness of our proposed Fisher-Taylor Sensitivity (FTS) criterion. Across all evaluated sparsity levels, FTS consistently maintains strong performance, frequently ranking among the top-performing methods. This trend is particularly evident at extreme sparsities, where many pruning approaches suffer significant performance degradation. The stability of FTS across both datasets highlights its effectiveness in preserving network expressivity despite aggressive pruning.

\begin{table}[h]
\caption{Performance of different compression methods evaluated using ResNet18 on the CIFAR-100 dataset. We report the mean accuracy between three initialization seeds across various sparsity levels. Baseline, no pruning: $69.57 \pm 0.19$.}
\label{tab:resnet18_cifar100_compressors}
\vskip 0.15in
\begin{center}
\begin{small}
\begin{sc}
\resizebox{\textwidth}{!}{%
\begin{tabular}{lccccc|cccc}
\toprule
Sparsity  & Random & Magnitude & GN & SNIP & GraSP & FD & FP & FTS & FBSS \\
\midrule
0.10  & 69.16 ± 0.11 & 69.37 ± 0.14 & 69.63 ± 0.34 & 69.42 ± 0.07 & 64.26 ± 0.27 & 69.66 ± 0.30 & 69.08 ± 0.21 & 69.16 ± 0.11 & 69.07 ± 0.10 \\
0.20  & 69.16 ± 0.30 & 69.06 ± 0.24 & 69.19 ± 0.11 & 69.30 ± 0.08 & 63.28 ± 0.58 & 69.60 ± 0.30 & 69.35 ± 0.35 & 69.41 ± 0.43 & 69.07 ± 0.20 \\
0.30  & 69.36 ± 0.18 & 68.58 ± 0.36 & 69.37 ± 0.13 & 68.82 ± 0.17 & 62.02 ± 0.43 & 69.24 ± 0.40 & 68.84 ± 0.13 & 68.80 ± 0.55 & 68.96 ± 0.11 \\
0.40  & 69.41 ± 0.20 & 68.50 ± 0.29 & 69.16 ± 0.26 & 68.95 ± 0.19 & 61.18 ± 0.19 & 69.17 ± 0.16 & 68.88 ± 0.25 & 69.02 ± 0.21 & 68.92 ± 0.25 \\
0.50  & 69.12 ± 0.46 & 68.17 ± 0.20 & 68.94 ± 0.20 & 68.63 ± 0.11 & 61.11 ± 0.40 & 69.13 ± 0.13 & 68.68 ± 0.12 & 68.71 ± 0.12 & 68.71 ± 0.57 \\
0.60  & 68.66 ± 0.27 & 67.78 ± 0.35 & 68.77 ± 0.17 & 68.63 ± 0.42 & 61.40 ± 0.78 & 68.34 ± 0.43 & 67.98 ± 0.23 & 68.41 ± 0.14 & 68.60 ± 0.15 \\
0.70  & 67.95 ± 0.43 & 67.51 ± 0.24 & 68.29 ± 0.39 & 68.08 ± 0.18 & 59.43 ± 0.76 & 68.03 ± 0.46 & 67.96 ± 0.15 & 68.29 ± 0.06 & 68.16 ± 0.07 \\
0.80  & 67.26 ± 0.48 & 66.55 ± 0.19 & 67.20 ± 0.37 & 67.21 ± 0.38 & 59.08 ± 0.22 & 66.70 ± 0.05 & 67.05 ± 0.06 & 66.77 ± 0.65 & 66.62 ± 0.43 \\
0.90  & 64.75 ± 0.16 & 64.48 ± 0.18 & 64.87 ± 0.27 & 65.70 ± 0.08 & 59.16 ± 0.91 & 64.74 ± 0.44 & 65.46 ± 0.30 & 65.41 ± 0.13 & 63.90 ± 0.31 \\
0.95  & 61.01 ± 0.32 & 62.20 ± 0.06 & 62.20 ± 0.23 & 63.20 ± 0.20 & 57.91 ± 0.09 & 62.14 ± 0.42 & 63.22 ± 0.25 & 63.21 ± 0.47 & 61.25 ± 0.44 \\
0.98  & 54.72 ± 0.22 & 55.44 ± 0.18 & 57.34 ± 0.31 & 58.83 ± 0.35 & 54.85 ± 0.35 & 55.57 ± 0.17 & 58.05 ± 0.18 & 58.59 ± 0.12 & 55.02 ± 0.34 \\
0.99  & 45.62 ± 0.55 & 40.39 ± 0.36 & 50.46 ± 0.61 & 52.96 ± 0.10 & 49.13 ± 0.19 & 48.02 ± 0.32 & 49.98 ± 0.60 & 52.85 ± 0.24 & 44.91 ± 0.52 \\
\bottomrule
\end{tabular}}
\end{sc}
\end{small}
\end{center}
\vskip -0.1in
\end{table}

%------------------------------------------------------------------------------------------------
\clearpage
\subsection{VGG19}
The results on VGG19 with CIFAR-100 exhibit a similar trend to those observed on CIFAR-10, reinforcing the effectiveness of our proposed approach. Once again, we identify the occurrence of layer collapse at extreme sparsities when no warm-up is applied, leading to a significant drop in accuracy. Introducing a single warm-up epoch effectively resolves this issue, restoring pruning performance across all evaluated criteria. However, increasing the warm-up phase to five epochs does not yield any additional advantage, indicating that a brief warm-up period is sufficient to stabilize gradient-based importance scores and prevent collapse.

\label{sec:vgg_cifar-100}

\begin{table}[h]
\caption{Performance of different compression methods evaluated using VGG19 on the CIFAR-100 dataset. We report the mean accuracy between three initialization seeds across various sparsity levels. Baseline, no pruning: $58.96 \pm 2.30$.}
\label{tab:VGG19_cifar100_compressors}
\vskip 0.15in
\begin{center}
\begin{small}
\begin{sc}
\resizebox{\textwidth}{!}{%
\begin{tabular}{lccccc|cccc}
\toprule
Sparsity & Random & Magnitude & GN & SNIP & GraSP & FD & FP & FTS & FBSS \\
\midrule
0.10  & 60.31 ± 0.40 & 59.13 ± 1.29 & 61.93 ± 0.48 & 61.98 ± 0.29 & 59.32 ± 0.63 & 62.13 ± 0.61 & 60.45 ± 3.47 & 61.56 ± 1.04 & 58.79 ± 0.98 \\
0.20  & 60.43 ± 1.14 & 59.27 ± 0.34 & 62.64 ± 0.21 & 62.68 ± 0.24 & 61.21 ± 0.41 & 63.04 ± 0.43 & 62.71 ± 1.02 & 62.24 ± 0.44 & 60.48 ± 0.48 \\
0.30  & 58.32 ± 0.60 & 59.35 ± 1.43 & 62.61 ± 0.23 & 63.11 ± 0.35 & 59.30 ± 0.43 & 62.85 ± 0.42 & 61.43 ± 0.61 & 62.65 ± 0.54 & 58.77 ± 1.02 \\
0.40  & 56.50 ± 3.20 & 60.04 ± 1.02 & 62.36 ± 0.02 & 62.39 ± 0.55 & 56.34 ± 1.49 & 62.38 ± 0.75 & 61.56 ± 1.25 & 62.67 ± 0.06 & 1.00 ± 0.00 \\
0.50  & 58.47 ± 1.49 & 61.49 ± 1.22 & 62.02 ± 0.64 & 62.76 ± 0.50 & 54.43 ± 0.84 & 62.84 ± 0.33 & 62.25 ± 0.33 & 62.47 ± 0.42 & 1.00 ± 0.00 \\
0.60  & 57.54 ± 0.74 & 61.50 ± 0.30 & 62.55 ± 0.13 & 63.08 ± 0.55 & 56.76 ± 0.69 & 62.40 ± 0.57 & 62.70 ± 0.63 & 62.17 ± 0.23 & 1.00 ± 0.00 \\
0.70  & 57.63 ± 0.80 & 61.71 ± 0.25 & 60.85 ± 0.79 & 60.58 ± 0.39 & 57.76 ± 0.84 & 60.44 ± 0.34 & 60.92 ± 0.41 & 60.51 ± 1.67 & 1.00 ± 0.00 \\
0.80  & 57.84 ± 0.57 & 61.89 ± 1.02 & 55.09 ± 0.49 & 59.84 ± 0.29 & 58.39 ± 0.74 & 1.00 ± 0.00 & 43.16 ± 1.02 & 58.66 ± 2.28 & 1.00 ± 0.00 \\
0.90  & 58.41 ± 0.41 & 62.60 ± 0.91 & 1.00 ± 0.00 & 8.35 ± 10.39 & 42.88 ± 1.64 & 1.00 ± 0.00 & 1.00 ± 0.00 & 8.87 ± 11.13 & 1.00 ± 0.00 \\
0.95  & 54.84 ± 1.08 & 1.00 ± 0.00 & 1.00 ± 0.00 & 1.00 ± 0.00 & 1.00 ± 0.00 & 1.00 ± 0.00 & 1.00 ± 0.00 & 1.00 ± 0.00 & 1.00 ± 0.00 \\
0.98  & 50.21 ± 0.72 & 1.00 ± 0.00 & 1.00 ± 0.00 & 1.00 ± 0.00 & 1.00 ± 0.00 & 1.00 ± 0.00 & 1.00 ± 0.00 & 1.00 ± 0.00 & 1.00 ± 0.00 \\
0.99  & 46.69 ± 0.45 & 1.00 ± 0.00 & 1.00 ± 0.00 & 1.00 ± 0.00 & 1.00 ± 0.00 & 1.00 ± 0.00 & 1.00 ± 0.00 & 1.00 ± 0.00 & 1.00 ± 0.00 \\
\bottomrule
\end{tabular}}
\end{sc}
\end{small}
\end{center}
\vskip -0.1in
\end{table}

%------------------------------------------------------------------------------------------------

\begin{table}[h]
\caption{Performance of different compression methods evaluated after 1 warmup epoch using VGG19 on the CIFAR-100 dataset. We report the mean accuracy between three initialization seeds across various sparsity levels. Baseline, no pruning: $58.96 \pm 2.30$.}
\label{tab:VGG19_cifar100_compressors_warmup1}
\vskip 0.15in
\begin{center}
\begin{small}
\begin{sc}
\resizebox{\textwidth}{!}{%
\begin{tabular}{lccccc|cccc}
\toprule
Sparsity & Random & Magnitude & GN & SNIP & GraSP & FD & FP & FTS & FBSS \\
\midrule
0.80  & 60.39 ± 1.16 & 58.91 ± 0.41 & 52.81 ± 1.32 & 55.62 ± 2.27 & 55.15 ± 2.25 & 56.71 ± 0.31 & 58.03 ± 0.93 & 52.41 ± 3.07 & 52.74 ± 5.16 \\
0.90  & 58.90 ± 0.98 & 60.95 ± 0.81 & 50.56 ± 4.59 & 55.89 ± 2.05 & 56.01 ± 1.58 & 52.07 ± 3.24 & 53.65 ± 0.57 & 52.45 ± 3.75 & 19.65 ± 1.68 \\
0.95  & 56.10 ± 0.85 & 57.64 ± 2.63 & 50.34 ± 1.00 & 53.70 ± 3.60 & 56.16 ± 0.41 & 54.44 ± 1.38 & 53.24 ± 3.54 & 53.56 ± 1.26 & 17.24 ± 0.44 \\
0.98  & 50.97 ± 0.40 & 54.66 ± 2.56 & 43.43 ± 5.32 & 50.19 ± 1.59 & 54.64 ± 1.50 & 42.75 ± 1.91 & 50.59 ± 3.39 & 48.56 ± 5.25 & 16.42 ± 0.64 \\
0.99  & 46.52 ± 0.45 & 43.33 ± 5.83 & 33.90 ± 5.35 & 42.65 ± 5.32 & 45.98 ± 4.48 & 29.67 ± 8.49 & 49.11 ± 3.46 & 48.70 ± 2.59 & 13.25 ± 0.84 \\
\bottomrule
\end{tabular}}
\end{sc}
\end{small}
\end{center}
\vskip -0.1in
\end{table}


%------------------------------------------------------------------------------------------------

\begin{table}[h]
\caption{Performance of different compression methods evaluated after 5 warmup epochs using VGG19 on the CIFAR-100 dataset. We report the mean accuracy between three initialization seeds across various sparsity levels. Baseline, no pruning: $58.96 \pm 2.30$.}
\label{tab:VGG19_cifar100_compressors_warmup5}
\vskip 0.15in
\begin{center}
\begin{small}
\begin{sc}
\resizebox{\textwidth}{!}{%
\begin{tabular}{lccccc|cccc}
\toprule
Sparsity & Random & Magnitude & GN & SNIP & GraSP & FD & FP & FTS & FBSS \\
\midrule
0.80  & 60.41 ± 1.39 & 58.38 ± 0.85 & 60.86 ± 0.79 & 61.63 ± 0.45 & 56.25 ± 0.49 & 59.59 ± 0.76 & 59.37 ± 3.50 & 60.86 ± 0.53 & 46.93 ± 9.04 \\
0.90  & 60.32 ± 0.09 & 57.74 ± 1.64 & 57.77 ± 2.41 & 58.23 ± 4.07 & 56.27 ± 1.02 & 60.19 ± 0.63 & 61.23 ± 0.50 & 60.52 ± 0.37 & 21.66 ± 1.95 \\
0.95 & 57.86 ± 0.53 & 59.55 ± 1.15 & 56.09 ± 0.97 & 58.83 ± 0.65 & 55.26 ± 1.25 & 55.80 ± 2.77 & 59.83 ± 0.94 & 58.52 ± 1.32 & 19.98 ± 2.62 \\
0.98 & 51.75 ± 0.43 & 47.75 ± 7.63 & 52.26 ± 4.06 & 55.27 ± 1.69 & 54.59 ± 0.96 & 49.46 ± 4.98 & 57.40 ± 1.26 & 56.00 ± 1.08 & 17.59 ± 1.36 \\
0.99 & 47.59 ± 0.80 & 42.46 ± 7.95 & 46.58 ± 2.00 & 53.13 ± 0.84 & 53.91 ± 1.53 & 42.87 ± 4.63 & 53.17 ± 1.18 & 53.05 ± 2.14 & 13.92 ± 0.14 \\
\bottomrule
\end{tabular}}
\end{sc}
\end{small}
\end{center}
\vskip -0.1in
\end{table}


%------------------------------------------------------------------------------------------------
\clearpage

\section{Mask Batch Size for Other Sparsities}
The Effect of batch size on pruning performance across different sparsities. 
As sparsity increases, the effect of batch size on pruning performance becomes more pronounced. 
At lower sparsities (0.90, 0.95), the differences across batch sizes are less evident, suggesting that even smaller batches provide a reasonable estimation of parameter importance. However, at extreme sparsities (0.98, 0.99), we observe a clear trend where larger batch sizes consistently lead to better parameter selection, ultimately improving accuracy. This aligns with our hypothesis that larger batches help reduce variance in gradient estimation, leading to more stable and effective pruning decisions. 
\label{batch_size_heatmaps}

\begin{figure}[h]
    \centering
    \includegraphics[width=0.8\linewidth]{imgs/cifar10_resnet18_heatmap_warmup_0.png}
    \caption{Effect of batch size on pruning performance at increasing sparsities.}
    \label{fig:enter-label}
\end{figure}

%------------------------------------------------------------------------------------------------

\clearpage
\section{Comparison of our criteria with magnitude-based pruning}

Figure \ref{fig:our_criterion_vs_magnitude} illustrates the relationship between parameter magnitude and different sensitivity-based pruning metrics. Each point represents a model parameter, with red points indicating the top-ranked parameters selected for retention by each criterion. The green dashed line marks the 99th percentile of parameter magnitudes.

A key observation is that the most effective pruning criteria, such as Fisher-Taylor Sensitivity, tend to retain parameters with a broad range of magnitudes, including many that are relatively small (left of the green line). This shows that the estimated importance does not always prioritize parameters based on their magnitude. 


\begin{figure}[htp]
    \centering
    \includegraphics[width=0.9\linewidth]{imgs/cifar_10_mag_vs_criteria_s_99.png}
    \caption{Our criteria vs. Magnitude parameter selection for 99\% sparsity (ResNet18, CIFAR-10, Seed 0)} 
    \label{fig:our_criterion_vs_magnitude}
\end{figure}


%%%%%%%%%%%%%%%%%%%%%%%%%%%%%%%%%%%%%%%%%%%%%%%%%%%%%%%%%%%%%%%%%%%%%%%%%%%%%%%
%%%%%%%%%%%%%%%%%%%%%%%%%%%%%%%%%%%%%%%%%%%%%%%%%%%%%%%%%%%%%%%%%%%%%%%%%%%%%%%


\end{document}

% This document was modified from the file originally made available by
% Pat Langley and Andrea Danyluk for ICML-2K. This version was created
% by Iain Murray in 2018, and modified by Alexandre Bouchard in
% 2019 and 2021 and by Csaba Szepesvari, Gang Niu and Sivan Sabato in 2022.
% Modified again in 2023 and 2024 by Sivan Sabato and Jonathan Scarlett.
% Previous contributors include Dan Roy, Lise Getoor and Tobias
% Scheffer, which was slightly modified from the 2010 version by
% Thorsten Joachims & Johannes Fuernkranz, slightly modified from the
% 2009 version by Kiri Wagstaff and Sam Roweis's 2008 version, which is
% slightly modified from Prasad Tadepalli's 2007 version which is a
% lightly changed version of the previous year's version by Andrew
% Moore, which was in turn edited from those of Kristian Kersting and
% Codrina Lauth. Alex Smola contributed to the algorithmic style files.