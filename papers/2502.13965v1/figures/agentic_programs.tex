\begin{figure}[t]
\centering
\begin{subfigure}[b]{0.49\columnwidth}
    \includegraphics[width=\linewidth]{plots/chatbot-agent.pdf}
    \caption{Chatbot}
    \label{fig:chatbot_agent}
\end{subfigure}
\hfill
\begin{subfigure}[b]{0.49\columnwidth}
    \includegraphics[width=\linewidth]{plots/REACT-agent.pdf}
    \caption{ReAct Agent}
    \label{fig:react_agent}
\end{subfigure}
\hfill
\begin{subfigure}[b]{0.44\columnwidth}
    \includegraphics[width=\linewidth]{plots/mapreduce-agent.pdf}
    \caption{Map-Reduce}
    \label{fig:map_reduce_agent}
\end{subfigure}
\hfill
\begin{subfigure}[b]{0.53\columnwidth}
    \includegraphics[width=\linewidth]{plots/MCTS-agent.pdf}
    \caption{Monte Carlo Tree Search}
    \label{fig:mcts_agent}
\end{subfigure}
\caption{\small\textbf{Execution workflows for \textit{Agentic Programs}.} Agentic programs are highly dynamic execution workflows that follow a directed acyclic graph (DAG). It consists of \textcolor{blue}{LLM calls} from one or more LLM agents and \textcolor{orange}{external interrupts} (i.e. tool calls, humans).}
\label{fig:agentic_program}
\vspace{-5mm}
\end{figure}

% Nodes and edges represent computation and data flow respectively. 