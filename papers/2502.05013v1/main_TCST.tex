%%%%%%%%%%%%%%%%%%%%%%%%%%%%%%%%%%%%%%%%%%%%%%%%%%%%%%%%%%%%%%%%%%%%%%%%%%%%%%%%
%2345678901234567890123456789012345678901234567890123456789012345678901234567890
%        1         2         3         4         5         6         7         8

\documentclass[letterpaper, 10 pt, conference]{ieeeconf}  % Comment this line out
                                                          % if you need a4paper
%\documentclass[a4paper, 10pt, conference]{ieeeconf}      % Use this line for a4
                                                          % paper

\IEEEoverridecommandlockouts                              % This command is only
                                                          % needed if you want to
                                                          % use the \thanks command
\overrideIEEEmargins
% See the \addtolength command later in the file to balance the column lengths
% on the last page of the document

\usepackage[utf8]{inputenc}
\usepackage[T1]{fontenc}
% \usepackage{cite}

% The following packages can be found on http:\\www.ctan.org
% \usepackage{graphics} % for pdf, bitmapped graphics files
%\usepackage{epsfig} % for postscript graphics files
%\usepackage{mathptmx} % assumes new font selection scheme installed
%\usepackage{mathptmx} % assumes new font selection scheme installed
%\usepackage{amsmath} % assumes amsmath package installed
%\usepackage{amssymb}  % assumes amsmath package installed

\title{\LARGE \bf
% Economic Model Predictive Control for Station-Keeping on Near-Rectilinear Halo Orbit via Full-State Tracking
%Station-Keeping on Near-Rectilinear Halo Orbits via Full-State Targeting Model Predictive Control
Output-Feedback Full-State Targeting Model Predictive Control for Station-Keeping on Near-Rectilinear Halo Orbits
}

\author{Yuri Shimane$^{1}$, Stefano Di Cairano$^{2}$, Koki Ho$^{3}$, and Avishai Weiss$^{4}$% <-this % stops a space
%\thanks{*This work was not supported by any organization}% <-this % stops a space
\thanks{$^{1,3}$Y. Shimane and K. Ho are with the Daniel Guggenheim School of Aerospace Engineering, Georgia Institute of Technology, Atlanta, GA 30332, USA
        Emails: {\tt\small \{yuri.shimane,kokiho\} at gatech.edu}}%
\thanks{$^{2,4}$S. Di Cairano and A. Weiss are with Mitsubishi Electric Research Laboratories (MERL), Cambridge, MA 02139, USA
        Emails: {\tt\small \{dicairano,weiss\} at merl.com}}%
}

% ------------------------------------------------------ %
% \setlength{\marginparwidth}{2cm}
% \usepackage{todonotes}
\usepackage{graphicx}
\usepackage{siunitx}
\usepackage{algorithm}
\usepackage{algpseudocode}
\usepackage{amsfonts}
\usepackage{booktabs}
% \usepackage{caption}
% \captionsetup{compatibility=false}
\usepackage{amsmath}
\usepackage{amssymb}
\usepackage{balance}                        % for balancing last page
\usepackage[flushleft]{threeparttable}
% \usepackage{threeparttable}
\usepackage[font=footnotesize,labelformat=simple]{subcaption}

%\usepackage{subfigure}
% \usepackage{tikz}
% \usetikzlibrary{shapes.geometric, arrows, calc}
% Custom commands
\newtheorem{proposition}{Proposition}
\newcommand{\EMrot}{\mathrm{EM}}
\newcommand{\Frame}{\mathcal{F}}
\newcommand{\thetabold}{\boldsymbol{\theta}}
\newcommand{\varthetabold}{\boldsymbol{\vartheta}}
\newcommand{\Abold}{\boldsymbol{A}}
\newcommand{\Bbold}{\boldsymbol{B}}
\newcommand{\Cbold}{\boldsymbol{C}}
\newcommand{\Fbold}{\boldsymbol{F}}
\newcommand{\Gbold}{\boldsymbol{G}}
\newcommand{\Tbold}{\boldsymbol{T}}
\newcommand{\Pbold}{\boldsymbol{P}}
\newcommand{\Qbold}{\boldsymbol{Q}}
\newcommand{\Rbold}{\boldsymbol{R}}
\newcommand{\Sbold}{\boldsymbol{S}}
\newcommand{\Hbold}{\boldsymbol{H}}
\newcommand{\Lbold}{\boldsymbol{L}}
\newcommand{\Vbold}{\boldsymbol{V}}
\newcommand{\abold}{\boldsymbol{a}}
\newcommand{\dbold}{\boldsymbol{d}}
\newcommand{\rbold}{\boldsymbol{r}}
\newcommand{\vbold}{\boldsymbol{v}}
\newcommand{\ubold}{\boldsymbol{u}}
\newcommand{\Ubold}{\boldsymbol{U}}
\newcommand{\hbold}{\boldsymbol{h}}
\newcommand{\xbold}{\boldsymbol{x}}
\newcommand{\ybold}{\boldsymbol{y}}
\newcommand{\fbold}{\boldsymbol{f}}
\newcommand{\Ibold}{\boldsymbol{I}}
\newcommand{\Phibold}{\boldsymbol{\Phi}}
\newcommand{\Sun}{\mathrm{Sun}}
\newcommand{\Earth}{\mathrm{Earth}}
\newcommand{\Moon}{\mathrm{Moon}}
\newcommand{\sdc}[1]{{\bf  (SDC: #1)}}
\newcommand{\period}{T}

\newcommand{\red}[1]{\textcolor{black}{#1}}
\newcommand{\ACCrev}[1]{\textcolor{black}{#1}}

% ------------------------------------------------------ %

\begin{document}

\maketitle
\thispagestyle{empty}
\pagestyle{empty}


\begin{abstract}
We develop a model predictive control (MPC) policy for station-keeping (SK) on a Near-Rectilinear Halo Orbit (NRHO). 
The proposed policy achieves full-state tracking of a reference NRHO via a two-maneuver control horizon placed one revolution apart.
Our method abides by the typical mission requirement that at most one maneuver is used for SK during each NRHO revolution.
Simultaneously, the policy has sufficient controllability for full-state tracking, making it immune to phase deviation issues in the along-track direction of the reference NRHO, a common drawback of existing SK methods with a single maneuver per revolution. 
We report numerical simulations with a navigation filter to demonstrate the MPC's performance with output feedback. 
Our approach successfully maintains the spacecraft's motion in the vicinity of the reference in both space and phase, with tighter tracking than state-of-the-art SK methods and comparable delta-V performance.
\end{abstract}


% ========================================================================= %
\section{Introduction}
\red{With growing interest in lunar exploration, libration point orbits (LPO) offer unique locations to place both robotic and crewed spacecraft.
For example, the lunar Gateway is planned in the 9:2 resonant southern Near-Rectilinear Halo Orbit (NRHO) about the Earth-Moon L2 point~\cite{Lee2019}.
The instability of LPOs necessitates station-keeping (SK) maneuvers, also referred to as orbit maintenance maneuvers (OMMs), to be conducted by the spacecraft.
The purpose of SK is to maintain the spacecraft near a pre-computed reference LPO, or \textit{baseline}, under the presence of uncertainties such as state estimation error, modeling error, and control execution error.
Due to the stringent propellant budget, typically higher instability of LPOs compared to traditional orbits around planets and moons, and the low number of heritage missions flying on LPOs, SK techniques on LPOs are an active area of research.
}

%Libration point orbits (LPOs) are expected to play a central role in upcoming lunar exploration. Most notably, the Lunar Gateway will be placed into the 9:2 resonant southern Near-Rectilinear Halo Orbit (NRHO) about the Earth-Moon L2 point~\cite{Lee2019}. 
%Due to the instability of LPOs, station-keeping (SK) maneuvers, also referred to as orbit maintenance maneuvers (OMMs), are required.  
%SK involves maintaining the spacecraft in the vicinity of a pre-computed reference NRHO, or \textit{baseline}, despite estimation error, modeling error, and control execution error. 
%To date, few missions have flown on LPOs, and thus SK techniques for LPOs are still active areas of research.

\red{To accommodate mission operations, SK maneuvers are typically required to be as infrequent as possible~\cite{Davis2022}.
On the NRHO with an orbital period of about $6.55$~days, a typical requirement is for SK maneuvers to be conducted at most once every revolution about the Moon.
To adhere to this requirement, a commonly adopted approach is the \textit{$x$-axis crossing control}~\cite{Davis2022}, a shooting-based method for designing SK maneuvers. 
In the $x$-axis crossing control, a single 3-degrees-of-freedom (DOF) control maneuver is designed at each revolution to target a subset of the spacecraft state at a perilune along the baseline a few revolutions downstream.
Recently, the CAPSTONE mission~\cite{Cheetham2022} adopted this SK technique, and some variants are currently being studied for the upcoming Gateway mission~\cite{Davis2022}. 
}

% In LPO missions, SK maneuvers should be as infrequent as possible in order to allocate time for other activities, such as operating the mission's payloads, or communicating with ground stations on Earth.  
% In the case of the NRHO, a typical mission requirement is that, at most, a single SK maneuver per revolution about the Moon be conducted. 
% One popular approach that adheres to this requirement is \textit{$x$-axis crossing control} \cite{Davis2022}, which employs a shooting method to design the corrective maneuver. 
% In $x$-axis crossing control, a single 3-degrees-of-freedom (DOF) control maneuver is applied once every revolution such that a subset of the spacecraft state at perilune tracks the baseline. 
% The recent CAPSTONE mission~\cite{Cheetham2022} adopted this SK technique, and its variants are currently being studied for the upcoming Gateway mission~\cite{Davis2022}. 

\red{One drawback of the $x$-axis crossing control stems from the fact that at most three out of the six translational state components can be assigned.
To overcome this deficiency, $x$-axis crossing control leverages the LPO's plane of symmetry.
A subset of the predicted spacecraft state at the intersection with the plane of symmetry is matched with the corresponding state components along the baseline when it intersects the same plane.
Using the plane of symmetry results in a discrepancy between the epoch when the spacecraft crosses the plane and when the baseline crosses the plane.
As a result, the steered path may experience a phase angle disparity: the spacecraft's position along the orbit may drift ahead or behind the baseline.}
To date, the phase disparity has been treated by ad-hoc heuristics, e.g. augmenting the targeting scheme with the epoch at which the symmetry event occurs~\cite{Davis2022,Williams2023}, or \red{encapsulating} the targeting scheme \red{within} a constrained optimization problem formulation~\cite{Shimane2024PCSCOP}. 
For further details, see~\cite{Shirobokov2017} and references therein. 

% One challenge with $x$-axis crossing control is that at most three 
% state components can be assigned. 
% In spite of the lower DOF than the order of the system, $x$-axis crossing control maintains the spacecraft near the baseline by leveraging the NRHO's plane of symmetry: a subset of the predicted spacecraft state at the intersection with this plane of symmetry is matched to the corresponding subset of the baseline at its intersection with the same plane. 
% Due to the use of the plane of symmetry, there exists a discrepancy between the epoch in which the spacecraft state crosses the plane and the epoch in which the baseline crosses the plane. 
% The mismatch in epoch causes the steered path to experience a phase angle disparity, where the spacecraft's position along the orbit drifts ahead or behind the baseline. 
% Over a long mission duration, the phase disparity risks unexpected communication blackouts and/or eclipses. 
% To date, the phase disparity has been treated by ad-hoc heuristics, e.g. augmenting the targeting scheme with the epoch at which the symmetry event occurs~\cite{Davis2022,Williams2023}, or replacing the targeting scheme by a constrained optimization problem formulation~\cite{Shimane2024PCSCOP}. 
% For further details, see~\cite{Shirobokov2017} and references therein. 

Here, we propose a model predictive control (MPC) policy that overcomes the phase disparity via full-state targeting.
\red{The proposed MPC uses a control horizon with two maneuvers spaced one revolution apart, which provides sufficient controllability to track all six state components.
Simultaneously, the one-revolution control cadence ensures our approach is consistent with the operational requirement of conducting up to a single SK maneuver per revolution.
To minimize the propellant consumption explicitly, we employ an economic objective~\cite{Rawlings2012,Angeli2015} based solely on the control cost.
The proposed MPC, hereafter denoted as SKMPC, sequentially solves a second-order cone program (SOCP) that steers the state of the spacecraft to the vicinity of the baseline at the end of its targeting horizon.
At each iteration, the SOCP is re-instantiated by linearizing the dynamics about the steered state from the previous iteration; the SKMPC is terminated when the steered state propagated with the nonlinear dynamics lies sufficiently close to the baseline. 
We provide a brief discussion on the recursive feasibility of the SKMPC and numerically demonstrate its performance. 
%We find the SKMPC to result in comparable control cost to other approaches such as the $x$-axis crossing control scheme requiring ad-hoc modifications, with superior tracking performance. 
}
While other \ACCrev{MPC-based} approaches~\cite{Misra2018,Elango2022Eigenmotion,Padhi2024} also adopt a full-state tracking approach, they do not account for the requirement of a single maneuver per revolution. 
Our SKMPC meets this critical requirement, thus making it a promising approach for future missions. 
%, which however is a critical spacecraft mission requirement and is enforced by our SKMPC. 

% In this brief, we propose a model predictive control (MPC) policy that overcomes the phase disparity via full-state targeting. 
% By considering two maneuvers spaced one revolution apart within the MPC's control horizon, sufficient controllability is recovered to track all 6 state components. 
% Meanwhile, the maneuvers' one-revolution spacing ensures that our approach remains consistent with the requirement of using only one maneuver per revolution along the NRHO. 
% An economic objective~\cite{Rawlings2012,Angeli2015} based solely on the maneuver cost is adopted to ensure the SK cost is minimized. 
% The proposed MPC, hereafter denoted by SKMPC, consists of sequentially solving a second-order cone program (SOCP) that steers the state of the spacecraft to the vicinity of the baseline at the end of its targeting horizon; the SOCP is re-instantiated by linearizing the dynamics about the steered state from the previous iteration until the final state propagated with nonlinear dynamics lies sufficiently close to the baseline. 
% We briefly discuss the recursive feasibility of the proposed MPC policy and numerically demonstrate that its performance is comparable to other, ad-hoc approaches such as the $x$-axis crossing control scheme. 
% Other control theoretic approaches~\cite{Kalabic2015,Misra2018,Elango2022Eigenmotion} also adopt a full-state tracking approach, however they do not account for the realistic single maneuver per revolution requirement considered in this work. 

\red{
In this work, we extend \cite{ACC2025trackingMPC} by augmenting a navigation filter to estimate the full state of the spacecraft, validating the proposed approach in a realistic output-feedback scenario. 
Our simulation incorporates disturbances due to navigational uncertainty, dynamics modeling errors, control actuation errors, and random impulses imparted at scheduled times along the NRHO due to momentum wheel desaturation maneuvers. 
We provide comprehensive Monte Carlo results with varying disturbance levels, thereby quantifying the coupled performance of the filter and the SKMPC. 
}

% \red{
% The remainder of this brief is organized as follows. In Section~\ref{sec:background}, we introduce the spacecraft dynamics model, LPOs, NRHO stability, and the navigation filter.
% }
% %discuss NRHO stability, \red{and introduce the navigation filter}. 
% %background context on the SK problem on LPOs. 
% Section~\ref{sec:EMPCforSK} develops the MPC policy for SK on NRHOs. 
% Section~\ref{sec:experiment_setup} outlines the numerical experiment setup for demonstrating the proposed algorithm, and results are provided in Section~\ref{sec:numerical_results}. 
% Finally, we provide concluding remarks in Section~\ref{sec:conclusion}. 


% ========================================================================= %
\section{Background}
\label{sec:background}
First, we model the spacecraft dynamics, and then provide a brief introduction to LPOs and their stability. 
%, along with the associated stable directions. 
%Finally, the station-keeping problem on LPOs is presented. 

% ------------------------------------------------------- %
\subsection{Spacecraft Dynamics Model}
\red{We consider the spacecraft's motion} in the inertial frame $\mathcal{F}_{\rm Inr}$, centered at the Moon.
%The spacecraft's motion is modeled in the inertial frame $\mathcal{F}_{\rm Inr}$, centered at the Moon. 
The state of the spacecraft $\xbold \in \mathbb{R}^6$ consists of the Cartesian position $\rbold \in \mathbb{R}^3$ with respect to the Moon \red{and} the rate of change of $\rbold$ in $\mathcal{F}_{\rm Inr}$, denoted by $\vbold \in \mathbb{R}^3$. 
The equations of motion are given by~\cite{Vallado2001}
\begin{equation}
    \dot{\xbold} = \fbold\left[\xbold(t),t\right]
    % + \begin{bmatrix}
    %     \boldsymbol{0}_{3\times 3} \\ \Ibold_3
    % \end{bmatrix} \ubold(t),
    = 
    \begin{bmatrix}
        \vbold \\ 
        -\dfrac{\mu}{r^3}\rbold + \abold_{\rm J2} + \sum_{i} \abold_{N_i} + \abold_{\rm SRP}
    \end{bmatrix},
    \label{eq:nonlinear_dynamics}
\end{equation}
where $r = \|\rbold\|_2$, and $\mu$ is the gravitational parameter of the Moon.  
The derivative of $\vbold$ consists, in order, of the Keplerian acceleration due to the Moon, J2 perturbation of the Moon $\abold_{\rm J2}$, gravitational perturbations by other celestial bodies $\abold_{N_i}$, and the solar radiation pressure (SRP) $\abold_{\rm SRP}$. 
These terms are given by
\begin{subequations}
\begin{align}
    \abold_{\rm J2} &= 
    \Tbold^{\rm PA}_{\rm Inr} 
    \left( -\dfrac{3 \mu J_2 R_{\Moon}^2}{2r^5}
        \begin{bmatrix}
            \left( 1-5\frac{z_{\rm PA}^2}{r^2} \right)x_{\rm PA} \\
            \left( 1-5\frac{z_{\rm PA}^2}{r^2} \right)y_{\rm PA} \\
            \left( 3-5\frac{z_{\rm PA}^2}{r^2} \right)z_{\rm PA}
        \end{bmatrix}
    \right),
    %\nonumber
    \\
    \abold_{N_i} &= -\mu_i \left( \dfrac{\rbold_i}{r_i^3} + \dfrac{\dbold_i}{d_i^3} \right),
    %\nonumber
    \\
    \abold_{\rm SRP} &= P_{\Sun} \left( \dfrac{ \|\dbold_{\Earth} - \dbold_{\Sun}\|_2 }{r_{\Sun}} \right)^2 C_r \dfrac{A}{m} \dfrac{\rbold_{\Sun}}{r_{\Sun}},
    \label{eq:nonlinear_dynamics_perturbations_SRP}
    %\nonumber
\end{align}
\end{subequations}
respectively, where $J_2$ is the \ACCrev{coefficient due to the oblateness} of the Moon, $R_{\Moon}$ is the equatorial radius of the Moon, where $[x_{\rm PA}, y_{\rm PA}, z_{\rm PA}]$ is the position vector components of the spacecraft resolved in the Moon's principal axes frame $\mathcal{F}_{\rm PA}$, $\Tbold^{\rm PA}_{\rm Inr} \in \mathbb{R}^{3\times 3}$ is the transformation matrix from $\mathcal{F}_{\rm PA}$ to $\mathcal{F}_{\rm Inr}$, $\mu_i$ is the gravitational parameter of body $i$, $\dbold_i$ is the position of body $i$ with respect to the Moon, $d_i = \|\dbold_i\|_2$, $\rbold_i = \rbold - \dbold_i$ is the position of the spacecraft with respect to body $i$ in $\mathcal{F}_{\rm Inr}$, $r_i = \|\rbold_i\|_2$, $P_{\Sun}$ is the SRP magnitude at the 1 astronomical unit, $C_r$ is the radiation pressure coefficient, and $A/m$ is the pressure area-to-mass ratio of the spacecraft. 
%Expressions for these accelerations can be found in classical orbital mechanics textbooks \cite{prussing1993orbital}. 
\red{We include third-body perturbations due to the Earth and the Sun.}
%Third-body perturbations due to the Earth and Sun are included. 
%In this work, third-body perturbations of the Earth and the Sun are included. 
Note that  $\abold_{\rm J2}$, $\abold_{N_i}$ and $\abold_{\rm SRP}$ in equation~\eqref{eq:nonlinear_dynamics} are time-dependent, making $\fbold$ non-autonomous. 
Constants in the equations of motion and ephemerides of celestial bodies are taken from the SPICE toolkit~\cite{Acton2018}. 

An initial linear perturbation $\delta \xbold(t_0)$ can be linearly propagated to time $t$, denoted by $\delta \xbold(t)$, via the linear state-transition matrix (STM) $\Phibold(t,t_0) \in \mathbb{R}^6$, 
\begin{equation}
    \delta \xbold(t) = \Phibold(t,t_0) \delta \xbold(t_0)
    .
    \label{eq:delta_x_map_with_STM}
\end{equation}
The Jacobian of the dynamics may be used to construct the STM by solving the matrix initial value problem (IVP)
\begin{equation}
\begin{aligned}
    \dot{\Phibold}(t,t_0) &= \dfrac{\partial \fbold(\xbold,t)}{\partial \xbold}\Phibold(t,t_0),
    \\
    \Phibold(t_0,t_0) &= \Ibold_{n}. 
\end{aligned}
\label{eq:stm_matrix_ivp}
\end{equation}
We use the shorthand notations $\xbold_j = \xbold(t_j)$ and $\Phibold_{j,i} = \Phibold(t_j,t_i)$, \red{and} we express the block submatrices of $\Phibold_{j,i}$ as
%In the remainder of this work, the shorthand notations $\xbold_j = \xbold(t_j)$ and $\Phibold_{j,i} = \Phibold(t_j,t_i)$ are used. 
%For ease of notation, we express the four 3-by-3 block submatrices of $\Phibold_{j,i}$ as
\begin{equation}
    \Phibold_{j,i} =
    \begin{bmatrix}
        \Phibold_{j,i}^{\rbold \rbold} & 
        \Phibold_{j,i}^{\rbold \vbold} \\[0.1em]
        \Phibold_{j,i}^{\vbold \rbold} & 
        \Phibold_{j,i}^{\vbold \vbold} \\
    \end{bmatrix}.
\end{equation}

%The impact of a control action at some time $t_k$ to the state at time $t_{k+1}$ is obtained exactly by integrating the nonlinear dynamics \eqref{eq:nonlinear_dynamics}. 
Assuming impulsive thrust\footnote{Due to control executions lasting on the order of seconds to minutes along an orbit with a period on the order of days, all conventional thrusters are effectively impulsive in this application.} is available to control the spacecraft state, \ACCrev{the state at time $t_{k+1}$ with an impulse applied at time $t_k$ is given by
\begin{equation}    \label{eq:nonlinearControlDynamics}
    % \xbold_{k+1} = \xbold_k + 
    % \begin{bmatrix}
    %     \boldsymbol{0}_{3 \times 1} \\ \ubold_{k}
    % \end{bmatrix}
    % + \int_{t_k}^{t_{k+1}} \fbold[ \xbold(t), t] \mathrm{d}t
    % ,
    \begin{aligned}
        \xbold_{k+1} &= 
        \xbold_k + \int_{t_k}^{t_{k+1}} \fbold[ \xbold(t), t] 
        + \delta(t - t_k) \begin{bmatrix}
            \boldsymbol{0}_{3 \times 1} \\ \ubold_{k}
        \end{bmatrix}
        \mathrm{d}t
        ,
    \end{aligned}
\end{equation}
where the control $\ubold_k \in \mathbb{R}^3$ is an impulsive change in velocity and $\delta$ is the Dirac delta function.
Assuming $\|\ubold_{k}\|$ is much smaller compared to} the dominant forces \red{in~\eqref{eq:nonlinear_dynamics}}, we can \red{linearly} approximate~\eqref{eq:nonlinearControlDynamics} as
\ACCrev{
\begin{equation}\label{eq:linearizedDynamics}
\begin{aligned}
    \xbold_{k+1} &= 
    \xbold_{k} + 
    \int_{t_k}^{t_{k+1}} \fbold[ \xbold(t), t] \mathrm{d}t 
    + 
    \begin{bmatrix}
        \Phibold_{k+1,k}^{\rbold \vbold} \\[0.1em]
        \Phibold_{k+1,k}^{\vbold \vbold}
    \end{bmatrix}
    \ubold_k.
\end{aligned}
\end{equation}}
\red{
To facilitate the formulation of the SKMPC, we define the Earth-Moon rotating frame, $\mathcal{F}_{\rm EM}$, with its $x$-axis aligned with the Earth-Moon vector, $z$-axis aligned with the co-rotating angular velocity vector of the Earth and the Moon, and the $y$-axis completing the triad. 
Note that due to the co-orbital motion of the Earth and the Moon, $\mathcal{F}_{\rm EM}$ is dynamic. 
}

% ------------------------------------------------------- %
\subsection{Canonical Scales}
\red{The large discrepancy in magnitude between $\rbold$ components expressed in \SI{}{km} and $\vbold$ components expressed in \SI{}{km/s} causes the STM to have poor numerical conditioning.
To alleviate this effect,} the dynamics from equation~\eqref{eq:nonlinear_dynamics} can be resolved in terms of canonical scales, where $\rbold$ is in terms of some length unit $\mathrm{LU}$, and $\vbold$ is in terms of some velocity unit $\mathrm{VU}$. 
%There is a large discrepancy in orders of magnitude between $\rbold$ components expressed in \SI{}{km} and $\vbold$ components expressed in \SI{}{km/s}, which causes the STM to have poor numerical conditioning. 
%As a countermeasure, the dynamics from equation~\eqref{eq:nonlinear_dynamics} can be resolved in terms of canonical scales, where $\rbold$ is in terms of some length unit $\mathrm{LU}$, and $\vbold$ is in terms of some velocity unit $\mathrm{VU}$. 
We \red{choose} $\mathrm{LU} =$ \SI{100000}{km} \red{and} define $\mathrm{VU} \triangleq \sqrt{\mu / \mathrm{LU}}$. The canonical time unit $\mathrm{TU}$ simply follows as $\mathrm{TU} = \mathrm{LU}/\mathrm{VU}$. Once $\mathrm{LU}$, $\mathrm{TU}$, and $\mathrm{VU}$ are defined, all dynamical coefficients in equation~\eqref{eq:nonlinear_dynamics} \red{are} re-scaled accordingly. 
\red{Further detail on how to choose $\mathrm{LU}$ is provided in~\cite{ACC2025trackingMPC}.}

% The appropriate choice of $\mathrm{LU}$ is investigated by looking at the condition number $\kappa$ of the $N$ revolution STM $\Phibold_{t_0 + N \period, t_0}$ along the NRHO starting at an arbitrarily chosen initial state, 
% %where $\period$ is the approximate period of the NRHO, given by
% \begin{equation}
%     \kappa\left( \Phibold_{t_0 + \period, t_0} \right)
%     \triangleq \| \Phibold_{t_0 + \period, t_0} \|_2 \| \Phibold_{t_0 + \period, t_0}^{-1} \|_2
%     ,
% \end{equation}
% where $\period$ is the approximate period of the NRHO. 
% For each $\mathrm{LU}$ sampled from a range of values between \SI{1000}{km} and \SI{200000}{km}, $\Phibold_{t_0 + N \period, t_0}$ is constructed by integrating the canonically scaled matrix IVP~\eqref{eq:stm_matrix_ivp}. 
% Figure \ref{fig:STM_condition_number} shows the variation of $\kappa$ against $\mathrm{LU}$ for $N$ between 1 and 7 revolutions. In this work, to have an easily interpretable value that also results in reasonable $\kappa$, $\mathrm{LU} = $ \SI{100000}{km} is selected. 

% % figure on condition number
% \begin{figure}[ht]
%     \centering
%     \includegraphics[width=0.94\linewidth]{plots/STM_condition_number_compressed.pdf}
%     \caption{$N$ revolution STM condition number against canonical length unit}
%     \label{fig:STM_condition_number}
% \end{figure}


% ------------------------------------------------------- %
\subsection{Libration Point Orbits}
\red{Libration point orbits (LPOs) are bounded motions revolving around libration points of the three-body system such as the Earth-Moon-spacecraft system.}
%Libration point orbits (LPOs) refer to bounded motions that revolve around libration points of three-body systems, such as the Earth-Moon-spacecraft system. 
While periodic LPOs can only exist in simplified dynamics models such as the restricted three-body problems, quasi-periodic motion still exists in the full-ephemeris dynamics model~\eqref{eq:nonlinear_dynamics}. 
\red{LPOs occupy spatial regions and energy levels that may not be covered by ``traditional'' orbital motions revolving around planetary bodies, thus providing mission designers attractive alternative spacecraft destinations.}
%LPOs offer mission designers an attractive alternative to ``traditional'' orbital motions revolving around planetary bodies as they cover a different spatial region and at various energy levels. 
For example, the southern NRHO about the Earth-Moon L2 has been selected as the location for the Lunar Gateway, a planned crew station in cislunar space~\cite{ZimovanSpreen2022}. 
In this work, we use the 15-year-long baseline NRHO from NASA \cite{Lee2019}. 


% ------------------------------------------------------- %
\subsection{Stability on Near Rectilinear Halo Orbit}
Many LPOs, including the NRHO, possess both stable and unstable subspaces. 
The unstable subspace on LPOs necessitates station-keeping actions to prevent the spacecraft from diverging from the baseline, especially when considering uncertainties in navigation and thrust actuation. 

\red{To quantify the local instability along the NRHO, we evaluate} the 1-revolution finite-time Lyapunov exponent ($\mathrm{FTLE}$)
%The local stability along the NRHO may be quantified by evaluating the 1-revolution finite-time Lyapunov exponent ($\mathrm{FTLE}$) given by
\begin{equation}
    \mathrm{FTLE} =
    \dfrac{1}{| \period | } \ln{\sqrt{
        \lambda_{\max}\left(\Phibold_{t_0+\period, t_0}\right)
    }}
    ,
\end{equation}
where $\period \approx 6.55$ \SI{}{days} is the approximate orbital period of the NRHO, and $\lambda_{\max}\left(\Phibold_{t_0+\period, t_0}\right)$ is the largest eigenvalue of $\Phibold_{t_0+\period, t_0}$. 
Figure \ref{fig:nrho_ftle} shows the $\mathrm{FTLE}$ evaluated at various locations along the NRHO. 
\red{We introduce the \textit{osculating true anomaly} $\theta$ to} facilitate the discussion about the varying stability along the NRHO. 
\red{Following} the Keplerian definition for a spacecraft orbiting the Moon,
%we introduce the \textit{osculating true anomaly} $\theta$, which follows the traditional Keplerian definition for a spacecraft orbiting the Moon, given by
\begin{equation}
    \theta = \operatorname{atan2}\left(
        h v_r, h^2/r - \mu
    \right),
    %\label{eq:true_anomaly}
\end{equation}
where $h = \| \boldsymbol{h} \|_2 = \| \rbold \times \vbold \|_2$ is the angular momentum, and $v_r = \rbold \cdot \vbold / r$ is the radial velocity. 
%\red{We omit the time argument on} the right-hand side of~\eqref{eq:true_anomaly} for quantities defined with respect to $\rbold$ and $\vbold$ for the sake of conciseness. 
%Note that the time argument has been omitted from the right-hand side of~\eqref{eq:true_anomaly} for quantities defined with respect to $\rbold$ and $\vbold$ for the sake of conciseness. 
\red{It is apparent from Figure~\ref{fig:nrho_ftle} that} the dynamics are most sensitive at \textit{perilune} where $\theta = 0^{\circ}$ and the spacecraft is closest to the \red{Moon}, and least sensitive at \textit{apolune} where $\theta = 180^{\circ}$. 
For further details on the dynamics in NRHO, see~\cite{ZimovanSpreen2022} and references therein. 

\red{To make the SK activity as robust as possible against navigation and control actuation errors, SK maneuvers typically execute near apolune.
Let the \textit{maneuver true anomaly} $\theta_{\rm man}$ denote the osculating true anomaly where the SK maneuver is scheduled to occur.}
In accordance with operational plans for the Gateway~\cite{Davis2022}, \red{we use $\theta_{\rm man} = 200^{\circ}$.}
%SK maneuvers are typically placed around apolune where the dynamics are less sensitive, making the SK activity more robust to navigation and control execution errors. 
%In accordance with operational plans for the Gateway \cite{Davis2022}, controls are assumed to have to occur at an osculating true anomaly of $200^{\circ}$, denoted hereafter by the \textit{maneuver true anomaly} $\theta_{\rm man}$. 
We also choose to target the baseline at an apolune $N$ revolutions \red{downstream} to minimize the targeting sensitivity. 
In summary, the controller proposed in this work aims to design an SK maneuver at $\theta_{\rm man}$ \red{such that the steered state lies in the vicinity of the baseline at the $N^{\mathrm{th}}$ apolune into the future,} i.e., approximately $N$ revolutions later. 
%to steer the state near the baseline at the $N^{\mathrm{th}}$ apolune into the future, approximately $N$ revolutions later. 

% FTLE plot
\begin{figure}[ht]
    \centering
    \includegraphics[width=0.82\linewidth]{plots/nrho_ftle_compressed.pdf}
    \caption{NRHO state history in Moon-centered J2000 frame}
    \label{fig:nrho_ftle}
\end{figure}


% ------------------------------------------------------- %
\subsection{Navigation Filter}
We consider an extended Kalman filter (EKF) to estimate the spacecraft state. 
Let $\hat{\xbold} \in \mathbb{R}^6$ and $\Pbold \in \mathbb{R}^{6 \times 6}$ denote the state and covariance estimates of the filter, respectively. 
We briefly present the prediction and update steps of the EKF, along with the measurement model and the impulse events. %adopted in this work. 

\subsubsection{Prediction}
\red{
The prediction step from time $t_{k-1}$ to $t_k$ is given by
\ACCrev{
\begin{subequations}
\begin{align}
    % \hat{\xbold}_{k|k-1} &= \hat{\xbold}_{k-1|k-1} + \int_{t_{k-1}}^{t_k} \fbold [\xbold(t), t] \mathrm{d}t ,
    \hat{\xbold}_{k|k-1} &= 
    \hat{\xbold}_{k-1|k-1} +
    \int_{t_{k-1}}^{t_k} \fbold [\hat{\xbold}(t), t] \mathrm{d}t ,
    \nonumber \\
    \Pbold_{k|k-1} &= \Phibold_{k,k-1} \Pbold_{k-1|k-1} \Phibold_{k,k-1}^T + \Qbold_{k,k-1},
    \nonumber 
\end{align}
\end{subequations}
}
where $\Qbold_{k,k-1}$ is the process noise accounting for unmodelled disturbances. 
We adopt the unbiased random process noise model~\cite{Carpenter2018}
\begin{equation}
    \Qbold_{k,k-1} = \sigma_p^2 
    \begin{bmatrix}
        ({\Delta t^3}/{3}) \Ibold_{3}
        &
        ({\Delta t^2}/{2}) \Ibold_{3}
        \\%[0.3em]
        ({\Delta t^2}/{2}) \Ibold_{3}
        &
        \Delta t \Ibold_{3}
    \end{bmatrix}
    ,
    \nonumber %\label{eq:process_noise_random_walk}
\end{equation}
where $\sigma_p$ is a tuning parameter.
}

\subsubsection{Update}
\red{
At time $t_k$, a measurement is provided to the filter. 
The noisy measurement $\ybold_k \in \mathbb{R}^m$ is assumed to follow a multivariate normal distribution with zero mean and covariance $\Rbold_k$ such that 
\begin{equation}
    \ybold_k 
    = 
    \hbold[\xbold(t_k)] +
    \mathcal{N}(\boldsymbol{0}_{m \times 1}, \Rbold_k)
    .
    \nonumber
\end{equation}
Let $\Hbold_k = \partial \hbold[\xbold(t_k)] / \partial \xbold(t_k)$, the update step is
%denote the partials of the nominal measurement model $\hbold[\xbold(t_k)]$. 
%The update step is given by
\begin{subequations}
\begin{align}
    \Lbold_k &= \Pbold_{k|k-1} \Hbold_k^T 
    \left( \Hbold_k \Pbold_{k|k-1} \Hbold_k^T + \Rbold_k \right)^{-1},
    \label{eq:ekf_kalman_gain}
    \nonumber \\
    \hat{\xbold}_{k|k} &= \hat{\xbold}_{k|k-1} + \Lbold_k 
    \left( \ybold_k - \hbold_k[\xbold_{k|k-1}] \right),
    \nonumber \\
    \Pbold_{k|k} &= (\Ibold_6 - \Lbold_k \Hbold_k) \Pbold_{k|k-1}(\Ibold_6 - \Lbold_k \Hbold_k)^T 
    + \Lbold_k \Rbold_k \Lbold_k^T.
    \nonumber %\label{eq:ekf_joseph_update}
\end{align}
\end{subequations}
%The Joseph form of the covariance update has been adopted for superior numerical stability. 
}

\subsubsection{Measurements}
\red{
%In this work, 
We consider measurements based on range and range-rate. The corresponding measurement model and partials are
\begin{subequations}
\begin{align}
    \hbold(\xbold) 
    &= \begin{bmatrix}
        r
        \\
        \dot{r}
    \end{bmatrix} = \begin{bmatrix}
        \| \rbold \|_2 
        \\
        \rbold^T \vbold / \| \rbold \|_2
    \end{bmatrix}
    ,
    \nonumber
    \\[0.3em]
    \Hbold(\xbold) &= 
    \begin{bmatrix}
        \dfrac{x}{r} & \dfrac{y}{r} & \dfrac{z}{r} & 0 & 0 & 0\\
        \dfrac{v_x}{r} - \dfrac{x \dot{r}}{r^2} &
        \dfrac{v_y}{r} - \dfrac{y \dot{r}}{r^2} &
        \dfrac{v_z}{r} - \dfrac{z \dot{r}}{r^2} &
        \dfrac{x}{r} & \dfrac{y}{r} & \dfrac{z}{r}
    \end{bmatrix}
    .
    \nonumber
\end{align}
\end{subequations}
We assume a constant measurement covariance $\Rbold_k = \Rbold = \operatorname{diag}(\sigma_r^2, \sigma_{\dot{r}}^2)$
% given by
% \begin{equation}
%     \Rbold = \begin{bmatrix}
%         \sigma_r^2 & 0 \\ 0 & \sigma_{\dot{r}}^2
%     \end{bmatrix},
%     \nonumber
% \end{equation}
where $\sigma_r$ and $\sigma_{\dot{r}}$ are the standard deviations of the range and range-rate measurements. 
}

\subsubsection{Impulse Events}
\red{
We model SK maneuvers as resulting in a velocity impulse on the spacecraft $\Delta \vbold_k = \Delta \hat{\vbold}_k + \mathcal{N} (\boldsymbol{0}_{3\times 1}, \Vbold_k)$,
%$\Delta \vbold_k \in \mathbb{R}^3$ given by
% \begin{equation}
%     \Delta \vbold_k = \Delta \hat{\vbold}_k + \mathcal{N} (\boldsymbol{0}_{3\times 1}, \Vbold_k),
%     \nonumber
% \end{equation}
where $\Delta \hat{\vbold}$ is the expected impulse, and $\Vbold_k$ is the corresponding covariance. 
The maneuver estimate $\Delta \hat{\vbold}$ is computed by the SKMPC and $\Vbold_k = (\sigma_{\Delta \vbold, \mathrm{abs}} + \sigma_{\Delta \vbold, \mathrm{rel}} \|\Delta \hat{\vbold}\|_2)^2 \Ibold_3$,
%is given by
% \begin{equation}
%     \Vbold_k = (\sigma_{\Delta \vbold, \mathrm{abs}} + \sigma_{\Delta \vbold, \mathrm{rel}} \|\Delta \hat{\vbold}\|_2)^2 \Ibold_3
%     ,
%     \nonumber
% \end{equation}
where $\sigma_{\Delta \vbold, \mathrm{abs}}$ and $\sigma_{\Delta \vbold, \mathrm{rel}}$ are the absolute and relative standard deviation of the thruster. 
%For a desaturation event, we assume $\Delta \hat{\vbold} = \boldsymbol{0}_{3\times 1}$ and $\Vbold_k = \sigma_{\mathrm{desat}}^2 \Ibold_3$ where $\sigma_{\mathrm{desat}}$ is the standard deviation of the impulse imparted by the desaturation event.   
Then, the state and covariance estimates are updated via
\begin{subequations}
\begin{align}
    \hat{\xbold}_{k|k} &= \hat{\xbold}_{k|k-1} + 
    \begin{bmatrix}
        \boldsymbol{0}_{3\times 1} \\ \Delta \hat{\vbold}_k
    \end{bmatrix},
    % \begin{bmatrix}
    %     \boldsymbol{0}_{3\times 3} \\ \Ibold_3
    % \end{bmatrix} \Delta \hat{\vbold}_k,
    \nonumber
    \\
    \Pbold_{k|k} &= \Pbold_{k|k-1} +
    \begin{bmatrix}
        \boldsymbol{0}_{3\times 3} & \boldsymbol{0}_{3\times 3} \\
        \boldsymbol{0}_{3\times 3} &
        \Vbold_k
    \end{bmatrix}.
    \nonumber
\end{align}
\end{subequations}
}


% ========================================================================= %
\section{Full-State Targeting MPC for Station-Keeping on NRHO}
\label{sec:EMPCforSK}
\red{
The SKMPC computes an SK maneuver based on the state estimate~$\hat{\xbold}(t_0)$ at the current time $t_0$ from the EKF and a predicted future state $\hat{\xbold}(t_N)$ at some future target time $t_N > t_0$. 
In the remainder of this section, we omit the $\hat{(\cdot)}$ notation from estimated state quantities within the SKMPC.
%
% in the following description of the SKMPC, we assume the controller has access to the state estimate at the current time as well as a time prediction
% (what is the controller doing, what is the implication of the state estimate)
% in introducing the SKMPC policy, our controller does not have access to the true state but rather the output of the EKF, so XYZ
%
% In this section, we introduce the SKMPC. 
% We begin by formulating the discrete-time optimal control problem (OCP) with linearized dynamics, and provide a proposition for its recursive feasibility; we then present the sequential linearization scheme; finally, we provide a pseudo-code for the SKMPC scheme together with the sequential linearization. 
%
%We now introduce the SKMPC. 
% Note that the controller only has access to the state estimate $\hat{\xbold} = [\hat{\rbold}^T, \hat{\vbold}^T]^T$ of the filter rather than the true state $\xbold = [\rbold^T, \vbold^T]^T$. 
% In this section, we omit the $\hat{(\cdot)}$ notation from estimated state quantities to simplify the notation. 
}

% ------------------------------------------------------- %
\subsection{Problem Formulation}
Let $\mathcal{U}$ denote the admissible control set, $N$ denote the number of revolutions until the targeted apolune along the baseline, which occurs at some future time $t_N$, and $\mathcal{X}(t_N)$ denote the terminal constraint set at time $t_N$. 
\red{We consider a control horizon with \ACCrev{$2 \leq K \leq N$} impulsive maneuvers,} denoted by $\ubold_k \in \mathbb{R}^3$ for $k = 0,\ldots,K-1$.
%The control horizon consists of $K \leq N$ impulsive maneuvers, denoted by $\ubold_k \in \mathbb{R}^3$ for $k = 0,\ldots,K-1$.
\red{The} maneuvers are placed at the $K$ earliest instances in time \red{$t_k$} where $\theta(t_k) = \theta_{\rm man}$ between \red{the time when the controller is invoked, denoted by} $t_{\rm invoked}$, and $t_N$. 
Thus, for a maneuver time $t_k$, $k = 0,\ldots,K-1$,
\begin{equation}
    \label{eq:control_horizon_definition}
    \begin{cases}
        t_k \geq t_{\rm invoked} & k = 0 \\
        t_k > t_{k-1} & k > 0
    \end{cases}
    \text{  and  }
    \theta(t_k) = \theta_{\rm man}.
\end{equation}
Without loss of generality, we hereafter assume that the controller is invoked when $\theta(t_{\rm invoked}) = \theta_{\rm man}$, such that $t_0 = t_{\rm invoked}$. 
\red{The maneuvers are used to steer the propagated state at $t_N$ to reside in~$\mathcal{X}(t_N)$.}
%These maneuvers are designed to steer the current state to be in $\mathcal{X}(t_N)$.
\red{We formulate a minimization problem} with an economic sum-of-2-norm objective of the $K$ maneuvers, which corresponds directly to the propellant mass consumed via Tsiolkovsky's rocket equation \cite{Vallado2001}. 
The finite-horizon discrete-time OCP of the SKMPC is
\begin{subequations}
\label{eq:optim_socp_general}
\begin{align}
    \min_{\ubold_{0}, \ldots, \ubold_{K-1}}
        \quad& \sum_{k=0}^{K-1} \| \ubold_k \|_2
    \label{eq:objective}
    \\
    \text{s.t.} \quad
    &   
        % \int_{t_0}^{t_N} \fbold[\xbold(t), t] \mathrm{d}t + 
        %\xbold_N^0 + 
        \xbold_0^N +
        \sum_{k=0}^{K-1}
        \begin{bmatrix}
            \Phibold_{N,k}^{\rbold\vbold} \\
            \Phibold_{N,k}^{\vbold\vbold} 
        \end{bmatrix}
        \ubold_{k}
        \in \mathcal{X}(t_N),
    \label{eq:constraint_x_terminal_set}
    \\
    &   \ubold_k \in \mathcal{U}, \quad \forall k = 0,\ldots,K-1.
    %\ubold_k, \ubold_{k+1} \in \mathcal{U}
    \label{eq:constraint_u_admissible}
\end{align}
\end{subequations}
where $\xbold_0^N \triangleq \operatorname{vec}(\rbold_0^N,\vbold_0^N)$ denotes the initial state propagated until the end of the prediction horizon,
\begin{equation}
    \xbold_0^N = 
    \begin{bmatrix}
        \rbold_0^N \\ \vbold_0^N
    \end{bmatrix}
    = \xbold_0 + \int_{t_0}^{t_N} \fbold[\xbold(t), t] \mathrm{d}t 
    ,
    \label{eq:unsteered_state_integration}
\end{equation}
and $\xbold_0$ is the state at $t_0$. 
The STM submatrices $\Phibold_{N,k}^{\rbold\vbold}$ and $\Phibold_{N,k}^{\vbold\vbold}$ are constructed by linearizing about the free drift trajectory obtained by integrating~\eqref{eq:unsteered_state_integration}.
\red{Linearizing} about the free drift trajectory is akin to the EKF, as opposed to linearizing about the baseline path, which is akin to the linearized Kalman filter. 
The former results in a more accurate linearized model, since the free drift trajectory is closer to the desired controlled trajectory than the baseline.%, i.e. the nominal orbit.
%
% In Section~\ref{sec:SequentialLinearization}, 
% problem~\eqref{eq:optim_socp_general} is recast as an SOCP by introducing slack variables for the 2-norm of $\ubold_k$ for $k = 0,\ldots,K-1$, and the STMs in constraint~\eqref{eq:constraint_x_terminal_set} is recomputed about the spacecraft state along $\xbold_0^N$ in the first iteration and along $\xbold_0^N$ incorporating controls $\ubold_k$ from previous iteration on the second iteration onward.

\red{The linearized dynamics in~\eqref{eq:constraint_x_terminal_set} implies} that the control action $\ubold_k$ shifts the state within some trust-region $\boldsymbol{\delta} \in \mathbb{R}^6$,
%The use of the linearized dynamics in~\eqref{eq:constraint_x_terminal_set} implicitly assumes that the control actions $\ubold_k$ shifts the state within some trust-region $\boldsymbol{\delta} \in \mathbb{R}^6$, such that
\begin{equation}
    \left|
    % \int_{t_0}^{t_N} \fbold[\xbold(t), t] \mathrm{d}t
    \xbold_0^N
    -
    \left( \boldsymbol{F}_{\ubold}[\xbold_0, \ubold_0,\ldots,\ubold_{K-1}] \right)
    \right| \leq \boldsymbol{\delta}
    ,
    \label{eq:trust_region}
    % \nonumber
\end{equation}
\ACCrev{where $\boldsymbol{F}_{\ubold}$ integrates the nonlinear dynamics~\eqref{eq:nonlinearControlDynamics} from $t_0$ to $t_N$ with controls $\ubold_0,\ldots,\ubold_{K-1}$ provided in the argument,}
\begin{equation}    \label{eq:definitionFu}
    \begin{aligned}
        \Fbold_{\ubold} &= 
        \xbold_0 + \int_{t_0}^{t_N} \fbold[ \xbold(t), t] 
        + 
        \sum_{k=0}^{K-1} \delta(t - t_k) \begin{bmatrix}
            \boldsymbol{0}_{3 \times 1} \\ \ubold_{k}
        \end{bmatrix}
        \mathrm{d}t
        .
    \end{aligned}
\end{equation}
% \ACCrev{where $\boldsymbol{F}_{\ubold}$ is the dynamics $\fbold$ piece-wise integrated with impulsive controls applied at time instances $t_0, \ldots, t_{K-1}$ via~\eqref{eq:nonlinearControlDynamics}, and the inequality applies element-wise.} 
% where Fu is the nonlinear form of (14b), where ...

\red{In~\eqref{eq:constraint_u_admissible}, $\mathcal{U}$} is the set of controls upper-bounded by a maximum executable control magnitude $u_{\max}$,
%For the admissible control set $\mathcal{U}$, we consider the set of all controls with magnitudes upper-bounded by a maximum executable control magnitude $u_{\max}$,
\begin{equation}\label{eq:UmaxDefinition}
    \mathcal{U} =
    \left\{
        \ubold \in \mathbb{R}^3 : \|\ubold\|_2 \leq u_{\max}
    \right\}
    .
\end{equation}


% \red{ {\bf SDC: I am not sure that I would say this:}
% In the numerical implementation of the SKMPC, problem~\eqref{eq:optim_socp_general} is solved without explicitly enforcing constraint~\eqref{eq:constraint_u_admissible}, but the condition $\| \ubold_0 \|_2 \leq u_{\max}$ is checked a posteriori, and the SK is considered to be unsuccessful if the condition is not met. 
% The condition is checked with the first maneuver only, as it is the only one that gets implemented.
%; note that due to the fact that earlier maneuvers have larger impacts downstream for the same maneuver magnitude, the solution is likely to have monotonically descending maneuver magnitudes, $\ubold_0 > \ubold_1 > \ldots > \ubold_{K-1}$. 
% }

% ------------------------------------------------------- %
\subsection{Definition of Terminal Constraint Set}
\red{We construct $\mathcal{X}(t_N)$ as a 6D ellipsoid centered at the baseline state at $t_N$,} $\xbold_{N,\mathrm{ref}} \triangleq [\rbold_{N,\mathrm{ref}}^T, \vbold_{N,\mathrm{ref}}^T]^T$, 
% A straightforward choice to remain in the vicinity of the baseline is to consider a 6D ellipsoidal $\mathcal{X}(t_N)$. 
% As will be demonstrated in Section~\ref{sec:numerical_results}, this definition of $\mathcal{X}(t_N)$ yields $\Delta V$ cost performances that are comparable to other state-of-the-art schemes such as $x$-axis control. 
%\red{In this work, we employ a terminal constraint set }
% We define the ellipsoid centered at the baseline state at time $t_N$, denoted by $\xbold_{N,\mathrm{ref}} \triangleq [\rbold_{N,\mathrm{ref}}^T, \vbold_{N,\mathrm{ref}}^T]^T$, with radii $\epsilon_r$ in position components and $\epsilon_v$ in velocity components. 
%The corresponding set $\mathcal{X}(t_N)$ is given by
\begin{equation}
    \begin{aligned}
    &\mathcal{X}(t_N)
    =
    \\
    &\,\,
    \left\{
        \xbold \in \mathbb{R}^n :
        \| \rbold - {\rbold}_{N,\mathrm{ref}} \|_2 \leq \epsilon_r , 
        \| \vbold - {\vbold}_{N,\mathrm{ref}} \|_2 \leq \epsilon_v
    \right\},
    \end{aligned}
    \label{eq:ellipsoid_definition}
\end{equation}
% where ${\rbold}_{N,\mathrm{ref}}$ and ${\vbold}_{N,\mathrm{ref}}$ denote the position and velocity of the baseline trajectory at time $t_N$, respectively, while 
where $\epsilon_r$ and $\epsilon_v$ are the magnitude of the apses of the ellipsoid \red{along position and velocity components and are tuning parameters}. 
The terminal constraint~\eqref{eq:constraint_x_terminal_set} can be replaced by two second-order cone (SOC) constraints,
\begin{subequations}
    \label{eq:termnal_set_ellipsoid}
    \begin{align}
        % \left\| \begin{bmatrix}
        %     \Phibold_{N,k}^{\rbold\vbold} & \Phibold_{N,k+1}^{\rbold\vbold}
        % \end{bmatrix} \Ubold
        \left\| \sum_{k=0}^{K-1} \Phibold_{N,k}^{\rbold\vbold} \ubold_k
        + \rbold_0^N - {\rbold}_{N,\mathrm{ref}} \right\|_2 &\leq \epsilon_r ,
        \\
        % \left\| \begin{bmatrix}
        %     \Phibold_{N,k}^{\vbold\vbold} & \Phibold_{N,k+1}^{\vbold\vbold}
        % \end{bmatrix} \Ubold
        \left\| \sum_{k=0}^{K-1} \Phibold_{N,k}^{\vbold\vbold} \ubold_k
        + \vbold_0^N - {\vbold}_{N,\mathrm{ref}} \right\|_2 &\leq \epsilon_v .
    \end{align}
\end{subequations}
\red{
Note that $\epsilon_r$ and $\epsilon_v$ are easier to tune in the $\mathcal{F}_{\rm EM}$ frame due to the near-invariance of the apolune state of the NRHO in this frame. 
Thus we enforce~\eqref{eq:termnal_set_ellipsoid} with $\Phibold_{N,k}^{\rbold\vbold}$, $\Phibold_{N,k}^{\vbold\vbold}$, $\rbold^N_0$, and $\rbold_{N,\rm ref}$ realized in $\mathcal{F}_{\rm EM}$. 
}


% % - - - - - - - - - - - - - - - - - - - - - - - - - - - %
% \subsubsection{Stable Subspace}
% \label{sec:stable_subspace}
% An alternative terminal constraint set is considered by making use of the stable subspace of the baseline NRHO. 
% Specifically, the steered state is constrained to lie inside the conical combination of all stable basis vectors $\boldsymbol{y}^s_i$ for $i = 0,\ldots,S-1$. The corresponding set $\mathcal{X}(t_N)$ is given by
% \begin{equation}
%     \mathcal{X}(t_N)
%     =
%     \left\{
%         \xbold \in \mathbb{R}^n : 
%         \xbold \in {\xbold}_{N,\mathrm{ref}} \pm \sum_i \alpha_i \ybold^s_i
%     \right\}
%     .
%     \label{eq:stable_subspace_definition}
% \end{equation}
% This $\mathcal{X}(t_N)$ can be implemented by enforcing, in addition to conditions~\eqref{eq:termnal_set_ellipsoid}, the following linear constraints 
% \begin{subequations}
%     \label{eq:target_subspace}
%     \begin{align}
%         \xbold_0^N +
%         \sum_{k=0}^{K-1}
%         \begin{bmatrix}
%             \Phibold_{N,k}^{\rbold\vbold} \\
%             \Phibold_{N,k}^{\vbold\vbold} 
%         \end{bmatrix}
%         \ubold_{k}
%         &= {\xbold}_{N,\mathrm{ref}} \pm \sum_{i=0}^{S-1} \alpha_i \ybold^s_i
%         ,
%         \\
%         \alpha_i &\geq 0, \quad \forall i = 0,\ldots,S-1,
%     \end{align}
% \end{subequations}
% where $\ybold_i \in \mathbb{R}^6$ denote the $i^{\rm th}$ stable direction on the baseline at time $t_N$, and $\alpha_i$ for $i=0,\ldots,S-1$ are introduced as additional variables to scale along each stable direction. 


% ------------------------------------------------------- %
\subsection{Recursive Feasibility} %\red{(separate the case of ellipsoid vs subspace)}}
\label{sec:theoretical_results}
Next, we briefly discuss the recursive feasibility of problem~\eqref{eq:optim_socp_general} with input constraint~\eqref{eq:UmaxDefinition} and terminal set constraint~\eqref{eq:ellipsoid_definition}. 
\red{The non-autonomous dynamics~\eqref{eq:nonlinear_dynamics} results in the terminal set $\mathcal{X}(t_j)$ to also be time-dependent, complicating the application of the} classical approach for proving recursive feasibility of MPC. 
%The classical approach for proving recursive feasibility of MPC is not straightforward to apply to this application, as the dynamics are time-varying, and thus so too would the invariant terminal set $\mathcal{X}(t_j)$ need to be. 
Computing and storing such a time-varying set for the NRHO, which is not periodic but only quasi-periodic, may be prohibitive in practice. 
However, this specific application has some favorable conditions that help us recover guarantees of recursive feasibility.
First, for the considered family of orbits, the STM in~\eqref{eq:linearizedDynamics} ensures controllability of the linearized system around the nominal orbit, described by $\fbold[ \xbold_k, t]$. 
\red{Second, the available thrust upper-bounded by~$u_{\max}$ is} ``significantly larger'' than what is required in SK maneuvers, although the general desire is to minimize the requested thrust.

\begin{proposition}
    Let \ACCrev{$K \geq 2$} \red{correspond to the number of maneuvers} such that
    $${\rm rank}\left(\Bigg[        \left[  \begin{smallmatrix}
            \Phibold_{j+N,j}^{\rbold\vbold} \\
            \Phibold_{j+N,j}^{\vbold\vbold} 
        \end{smallmatrix} \right] \cdots       \left[  \begin{smallmatrix}
            \Phibold_{j+N,j+K-1}^{\rbold\vbold} \\
            \Phibold_{j+N,j+K-1}^{\vbold\vbold}
        \end{smallmatrix} \right]  \Bigg]  \right) = 6,\ \forall j=0,1,\ldots$$ For a large enough $u_{\rm max}$, if \eqref{eq:optim_socp_general} is feasible at time $t_{j-1}$ then it is feasible at $t_{j}$. Furthermore, the trajectory remains bounded in a set~$\mathcal{X}_{\rm bnd}$ at the apolune times $t_j$.
\end{proposition}


{\em Proof.}
Since~\eqref{eq:optim_socp_general} is feasible at $t_{j-1}$, there exists 
$U_K(t_{j-1}) = [\ubold(t_{j-1}) \ldots \ubold(t_{j+K-2})]$ such that
$\xbold(t_{j-1+N})\in\mathcal{X}(t_{j-1+N})$, and $\xbold(t_j)$ is obtained by applying $\ubold(t_{j-1})$ to~\eqref{eq:linearizedDynamics}.
Let $\bar \xbold(t_{j+N})= \int_{t_{j-1+N}}^{t_{j+N}} \fbold[\xbold(t_{j-1+N}),t]{\rm d}t$, where $\xbold(t_{j-1+N})$ is obtained by applying the entire sequence $U_K(t_{j-1})$ followed by open loop evolution.
We need to prove that it is possible to obtain a state perturbation $\delta \xbold$ such that 
$\bar \xbold(t_{j+N})+\delta \xbold\in \mathcal{X}(t_{j+N})$. 

Consider the candidate control sequence 
$U_K(t_{j}) = [\ubold(t_{j})\!+\!\delta\ubold(t_{j}), \ldots, \ubold(t_{j+K-1})\!+\!\delta\ubold(t_{j+K-1}), \ubold(t_{j+K})]$ and
\begin{equation*}
    \tilde \xbold(t_{j+N}) = \begin{bmatrix}
            \Phibold_{j+N,j+K}^{\rbold\vbold} \\
            \Phibold_{j+N,j+K}^{\vbold\vbold}
        \end{bmatrix} \ubold_{j+K}
        + \sum_{k=j+1}^{j+K-1}
        \begin{bmatrix}
            \Phibold_{j+N,k}^{\rbold\vbold} \\
            \Phibold_{j+N,k}^{\vbold\vbold} 
        \end{bmatrix}
        \delta \ubold_{k}.
\end{equation*}
% $\tilde \xbold(t_{j+N}) = \begin{bmatrix}
%         \Phibold_{j+N,j+K}^{\rbold\vbold} \\
%         \Phibold_{j+N,j+K}^{\vbold\vbold}
%     \end{bmatrix} \ubold_{j+K}
%     + \sum_{k=j+1}^{j+K-1}
%     \begin{bmatrix}
%         \Phibold_{j+N,k}^{\rbold\vbold} \\
%         \Phibold_{j+N,k}^{\vbold\vbold} 
%     \end{bmatrix}
% \delta \ubold_{k}.$
Then, $\xbold(t_{j+N}) = \bar \xbold(t_{j+N}) + \tilde \xbold(t_{j+N}) \in \mathcal{X}({t_{j+N}})$ is guaranteed by the controllability in $K$ steps for some $\Delta U_K(t_{j}) = [ \delta\ubold(t_{j}), \ldots, \delta\ubold(t_{j+K-1}), \ubold(t_{j+K}) ]$ which perturbs and extends the previous control sequence $U_K(t_{j-1})$. 
\red{For a large enough~$u_{\rm max}$,} the sequence $ U_K(t_{j}) $ is feasible. 
%For $u_{\rm max}$ large enough, the sequence $ U_K(t_{j}) $ is feasible. 
Due to the finite horizon and the bounded thrust, the trajectories remain bounded in a set~$\mathcal{X}_{\rm bnd}$ because the control strategy  enforces~\eqref{eq:constraint_x_terminal_set}, and using~\eqref{eq:ellipsoid_definition}, $\mathcal{X}(t)$ is bounded by its definition.
\QED

\red{With regards to the assumptions in Proposition 1, due to the quasi-periodic nature of the orbit,} the difference between $\mathcal{X}(t_{j-1+N})$ and $\mathcal{X}(t_{j+N})$ is usually small. 
The necessitated correction~$\delta \xbold$ is thus relatively small compared to the control authority~$u_{\max}$. 
% \red{resulting in relatively small correction~$\delta \xbold$ compared to the control authority~$u_{\max}$.}
Hence, the maximum thrust of the propulsion system will be sufficient to ensure the feasibility of the candidate control sequence. 
The rank condition is ensured by the controllability of the spacecraft in the NRHO orbit.

% Since the orbit is quasi-periodic, i.e., the difference between $\mathcal{X}(t_{j-1+N})$ and $\mathcal{X}(t_{j+N})$ is usually small, the required correction $\delta \xbold$ will also be relatively small, such that the maximum thrust of the propulsion system will be sufficient to ensure feasibility of the candidate control sequence. Furthermore, the rank condition is ensured by the controllability of the spacecraft in the NRHO orbit.


% ------------------------------------------------------- %
\subsection{Sequential Linearization Scheme}
\label{sec:SequentialLinearization}
\red{To improve on the prediction error introduced by the linearization in~\eqref{eq:constraint_x_terminal_set}, we employ a sequential linearization scheme.}
Sequential linearization has been previously found to improve the recursive convergence of optimization-based SK algorithms~\cite{Shimane2024PCSCOP,Elango2022}.
\red{In essence, sequential linearization involves re-solving problem~\eqref{eq:optim_socp_general}, each time re-linearizing the dynamics about the steered trajectory obtained from the previous solution.}
%While the STM yields a reasonably reliable prediction of small perturbations over time, the nonlinearity of the dynamics is sufficiently high that a sequential linearization scheme has been previously found to improve the recursive convergence of SK algorithms \cite{Shimane2024PCSCOP,Elango2022}, and is thus also adopted in this work. 

\red{We recast~\eqref{eq:optim_socp_general} as an SOCP} by introducing slack variables for the 2-norm of $\ubold_k$ for $k = 0,\ldots,K-1$ in the objective~\eqref{eq:objective}, replacing~\eqref{eq:constraint_x_terminal_set} by the SOC constraints~\eqref{eq:termnal_set_ellipsoid}, and using definition~\eqref{eq:UmaxDefinition} for $\mathcal{U}$ in constraint~\eqref{eq:constraint_u_admissible}. 

At each iteration \red{of the sequential linearization, we update} $\xbold_0^N$, $\Phibold^{\rbold \vbold}_{N,k}$ and $\Phibold^{\vbold \vbold}_{N,k}$ for $k=0,\ldots,K-1$ by considering the controlled trajectory with controls computed from the previous iteration.  
%The use of sequential linearization results in update schemes for both $\xbold_0^N$, $\Phibold^{\rbold \vbold}_{N,k}$ and $\Phibold^{\vbold \vbold}_{N,k}$ for $k=0,\ldots,K-1$.  
Let $\ubold_0^{(i)}, \ldots, \ubold_{K-1}^{(i)}$ denote the solution to problem~\eqref{eq:optim_socp_general} at the $i^{\rm th}$ iteration. 
On the next iteration, $\xbold_0^N$ is obtained by
\begin{equation}
    \xbold_0^N = 
    \begin{bmatrix}
        \rbold_0^N \\ \vbold_0^N
    \end{bmatrix}
    = \boldsymbol{F}_{\ubold}[\xbold_0, {\ubold}_{0,\rm prev}^{(i)}, \ldots, {\ubold}_{K-1,\rm prev}^{(i)}] ,
    \label{eq:ubar_steered_state_integration}
\end{equation}
% instead of equation~\eqref{eq:unsteered_state_integration}, where $\fbold_{\ubold}$ is the dynamics including the provided impulses applied at times $t_0, \ldots, t_{K-1}$; 
instead of equation~\eqref{eq:unsteered_state_integration}, \red{where $\boldsymbol{F}_{\ubold}$ is given by~\eqref{eq:definitionFu}};
in~\eqref{eq:ubar_steered_state_integration}, ${\ubold}_{k,\rm prev}^{(i)}$ is the cumulative $k^{\rm th}$ control
\begin{equation}
    {\ubold}_k^{(i)}
    =
    \begin{cases}
        \boldsymbol{0}_{3 \times 1}, & i = 0,
        \\
        \sum_{j=0}^{i-1} \ubold_{k,\rm prev}^{(j)}, & i > 0.
    \end{cases}
    \nonumber
\end{equation}
Furthermore, $\Phibold^{\rbold \vbold}_{N,k}$ and $\Phibold^{\vbold \vbold}_{N,k}$ are reconstructed by linearizing the nonlinear flow around~\eqref{eq:ubar_steered_state_integration}. 

Algorithm \ref{alg:mpc_tracking} summarizes the SKMPC algorithm with the sequential linearization scheme. 
At a given time instance $t_0$, the algorithm requires as input the current state estimate $\hat{\xbold}(t_0)$ treated as $\xbold_0$, targeted time $t_N$, terminal constraint set $\mathcal{X}(t_N)$, admissible control set $\mathcal{U}$, and the maximum number of iterations for linearization $M$. 
In algorithm~\ref{alg:mpc_tracking}, ${\Ubold} = \operatorname{vec}{[{\ubold}_{0,\rm prev},\ldots,{\ubold}_{K-1,\rm prev}]}$ is the vectorized cumulative controls computed from successive SOCP solves, and $\Ubold^{(i)} = \operatorname{vec}{[\ubold_0^{(i)},\ldots,\ubold_{K-1}^{(i)}]}$ is the solution to the SOCP at the $i^{\mathrm{th}}$ iteration. 
The algorithm makes use of the following functions: 
\begin{itemize}
    \item \verb|IVP| solves the initial value problem by integrating equation~\eqref{eq:ubar_steered_state_integration} along with the STM, applying the impulsive controls ${\ubold}_0,\ldots,{\ubold}_{K-1}$ at times $t_0,\ldots,t_{K-1}$. 
    \item \verb|SOCP| solves problem \eqref{eq:optim_socp_general} recast into an SOCP by convex solvers such as ECOS \cite{Domahidi2013ecos} or SCS \cite{ocpb2016}. 
\end{itemize}

The algorithm terminates once the nonlinear steered state $\xbold_0^N$ from~\eqref{eq:ubar_steered_state_integration} lies in~$\mathcal{X}(t_N)$ and returns the earliest control $\ubold(t_0) = \ubold_0$, which is executed. 
Then, the spacecraft remains in the corrected orbit until the next maneuver instance $t_1$, at which time Algorithm~\ref{alg:mpc_tracking} is called again with updated time indices, sliding the targeting horizon $t_N$ by one revolution, and a new sequence of controls is obtained.
%Then, the spacecraft state is propagated until $t_1$, at which time Algorithm~\ref{alg:mpc_tracking} is called again with updated time indices, sliding the targeting horizon $t_N$ by one revolution, and a new sequence of controls is obtained. 

% Once problem~\eqref{eq:optim_socp_general} is solved, a sequence of controls $\Ubold$, each spaced one revolution apart according to condition~\eqref{eq:control_horizon_definition}, is obtained. 
% The spacecraft executes $\ubold_0$ at time $t_0$, and the spacecraft state is propagated until time $t_1$; at this time, problem~\eqref{eq:optim_socp_general} is solved again with updated time indices, sliding the targeting horizon $t_N$ by one revolution, and a new sequence of controls is obtained. 
% Thus, since $\ubold_k$ for $k \geq 1$ are never implemented on the spacecraft, Algorithm~\ref{alg:mpc_tracking} only returns the earliest maneuver $\ubold_0$. 
%Note that the algorithm only returns the maneuver $\ubold_0$ to be implemented at $t_0$; upon executing this maneuver, the spacecraft state is propagated until $t_1$, and the control maneuvers are recomputed with updated time indices. 

% Pseudo-algorithm for controller
\begin{algorithm}
    \caption{Sequential SKMPC}
    \label{alg:mpc_tracking}
    \textbf{Inputs}: $t_0$, $t_N$, $\xbold_0$, $\mathcal{X}(t_N)$, $\mathcal{U}$, $M$
    \begin{algorithmic}[1]
        % \State $\xbold_0 \gets \hat{\xbold}(t_0)$
            %\Comment{Set initial controller state to current state estimate}
        \State ${\Ubold} \gets \boldsymbol{0}_{3K\times 1}$
        \For{$i = 0,\ldots,M-1$}
            \State $\xbold_0^N, \Phibold_{N,0}, \ldots, \Phibold_{N,K-1} \gets$
            \verb|IVP|$(t_0, t_N, \xbold_0, {\Ubold})$
            \If{$\xbold_0^N \in \mathcal{X}(t_N)$}
                \State break
            \EndIf
            \State $\Ubold^{(i)} \gets$
                \verb|SOCP|$(\mathcal{X}(t_N), \mathcal{U}, \xbold_0^N, \Phibold_{N,0},\ldots, \Phibold_{N,K-1})$
            \State ${\Ubold} \gets {\Ubold} + \Ubold^{(i)}$
        \EndFor
        \State $\ubold_0 \gets {\Ubold}_{[0:3]}$
    \end{algorithmic}
    \textbf{Outputs}: $\ubold_0$
\end{algorithm}

% ========================================================================= %
\section{Experiment Setup}
\label{sec:experiment_setup}
The SKMPC is tested on a realistic SK scenario for a spacecraft flying on the NRHO.
The simulation consists of recursively applying Algorithm~\ref{alg:mpc_tracking} for an extended number of revolutions spanning multiple years, subject to \red{navigation error from the EKF as well as dynamics model error, control execution error, and random impulses due to momentum wheel desaturation maneuvers imparted at predefined locations. The latter three errors are realized based on predefined Gaussian distributions.}
Each time the spacecraft arrives at $\theta(t) = 200^{\circ}$, we denote $t$ as $t_0$, and Algorithm~\ref{alg:mpc_tracking} is invoked using a control horizon defined by~\eqref{eq:control_horizon_definition} with $\theta_{\rm man} = 200^{\circ}$. 

% ------------------------------------------------------- %
\subsection{Error Models}
\red{
The simulation involves repeatedly applying the EKF and executing an SK control every time the spacecraft reaches $\theta(t) = \theta_{\rm man}$. 
The filter is initialized assuming an initial covariance $\Pbold_{0|0} = \operatorname{diag}([\sigma_{r_0}^2,\sigma_{r_0}^2,\sigma_{r_0}^2,\sigma_{v_0}^2,\sigma_{v_0}^2,\sigma_{v_0}^2])$ and an initial state estimate 
\begin{equation}
    \hat{\xbold}_{0|0} = \xbold(t_{\rm init}) + \mathcal{N}(\boldsymbol{0}_{6 \times 1}, \Pbold_{0|0})
    ,
    \nonumber
\end{equation}
where $\xbold(t_{\rm init})$ is the true state at the initial epoch $t_{\rm init}$. 
At each revolution, when the spacecraft arrives at $\theta_{\rm man}$, a maneuver is computed using the state estimate of the filter, $\hat{\xbold}$, as $\xbold_0$ in Algorithm~\ref{alg:mpc_tracking}. 
The true state of the spacecraft is imparted with a corrupted maneuver using the Gates model~\cite{Gates1963}. 
%
In addition, we incorporate dynamics error, which consists of variation in SRP magnitude, and random impulses imparted by momentum wheel desaturation maneuvers~\cite{Davis2022}. 
The former is modeled by relative perturbations $\delta (A/m)$ and $\delta C_r$ on $A/m$ and $C_r$ in~\eqref{eq:nonlinear_dynamics_perturbations_SRP}, and the latter is modeled by an additive velocity perturbation $\Delta \vbold$ with random direction and magnitude when the spacecraft arrives at $\theta(t) = \theta_{\rm desat}$, where $\theta_{\rm desat}$ are desaturation true anomalies dictated by mission requirements~\cite{Davis2022}. 
%
%Between each maneuver, we incorporate dynamics error according to the approach in~\cite{Davis2022}, realized by randomly varying $A/m$ and $C_r$ in the SRP term, and reaction wheel desaturation error, realized by applying a random impulse at prescribed location(s) along the orbit. 
}
% We assume realistic error models that exist in the SK operation of a spacecraft on an NRHO \cite{Davis2022}, consisting of:
% \begin{itemize}
%     \item dynamics error, realized through random relative variation of $A/m$ and $C_r$ in the SRP term,
%     \item reaction wheel desaturation error, realized by appending random impulses at prescribed locations along the orbit in each revolution, and
%     \item maneuver execution error, realized by corrupting the maneuver with the Gates model \cite{Gates1963}. 
% \end{itemize}
% The errors are implemented according to the recursive simulation setup in \cite{Shimane2024PCSCOP}, with error values summarized in Table~\ref{tab:error_parameters}; these values are taken to be corresponding to the assumed levels of uncertainties for the Gateway, provided in \cite{Bolliger2021}.

\red{
Table~\ref{tab:error_parameters} summarizes the error parameters, corresponding to the assumed levels of uncertainties for the Gateway \cite{Bolliger2021}, \ACCrev{along with the selected process noise parameter $\sigma_p$}. 
% The table also provides parameters of the range and range-rate measurements from \cite{Bolliger2021} and the process noise parameter $\sigma_p$.
\ACCrev{Note that the choice of $\sigma_p$ is dependent on the canonical scales in which the dynamics are expressed.}
}

% Error parameters
\begin{table}[ht]
    \caption{Simulation parameters}
    \label{tab:error_parameters}
    \centering
    \begin{tabular}{ll}
        \toprule
        Simulation parameter & Value  \\
        \midrule
        Average SRP $A/m$, \SI{}{m^2/kg} & $315/17900$ \\
        Average SRP $C_r$ & $2$ \\
        SRP rel. $\delta (A/m)$ 3-$\sigma_{A/m}$, \% & $30$ \\
        SRP rel. $\delta C_r$ 3-$\sigma_{C_r}$, \% & $15$ \\
        Desaturation velocity magnitude 3-$\sigma_{\rm desat}$, \SI{}{cm/s} & $1.0$ \\
        Desaturation true anomaly $\theta_{\rm desat}$, \SI{}{deg} & 
            \begin{tabular}[c]{@{}l@{}}$0^{\circ}$ or\\$330^{\circ}, 0^{\circ}$ or\\$330^{\circ}, 0^{\circ}, 30^{\circ}$\end{tabular}
            \\
        Maneuver rel. magnitude error 3-$\sigma_{\Delta \vbold,\rm rel}$, \% & $1.5$ \\
        Maneuver abs. magnitude error 3-$\sigma_{\Delta \vbold,\rm abs}$, \SI{}{mm/s} & $1.42$ \\
        Maneuver execution direction error 3-$\sigma_{\Delta \vbold,\rm dir}$, \SI{}{deg} & $1^{\circ}$ \\
        Initial position standard deviation 3-$\sigma_{r_0}$, \SI{}{km} & $10$ \\
        Initial velocity standard deviation 3-$\sigma_{v_0}$, \SI{}{mm/s} & $10$ \\
        Range measurement 3-$\sigma_r$, \SI{}{m} &   $1$ \\
        Range-rate measurement 3-$\sigma_{\dot{r}}$, \SI{}{mm/s} &   $0.1$ \\
        Process noise parameter $\sigma_p$ & $5 \times 10^{-5}$ \\
        \bottomrule
    \end{tabular}
\end{table}


% ------------------------------------------------------- %
\subsection{Navigation Update Model}
\red{
In accordance with the typical operation of ground-based tracking, we assume measurements are provided during \textit{tracking windows}, each lasting $\Delta t_{\rm track} = 1$ hour. 
Let $t_0$ and $t_1 \approx t_0 + \period$ denote two consecutive epochs where the maneuver is executed, such that $\theta(t_0) = \theta(t_1) = \theta_{\rm man}$. 
In each revolution, there is one post-maneuver tracking window starting $12$ hours after $t_0$, and three pre-maneuver tracking windows, starting $72$, $48$, and $7$ hours before $t_1$. 
During each tracking window, we provide $N_{\rm meas} = 10$ equally spaced measurements. 
% The starting times of each tracking window $t_{\mathrm{track},j}$ for $j = 1,\ldots,4$ are given by
% \begin{equation}
%     t_{\mathrm{track},j} = 
%     \begin{cases}
%         t_0 + \Delta t_{\mathrm{track},j} & j = 1
%         \\
%         t_1 - \Delta t_{\mathrm{track},j} & j > 1 %= 2,3,4
%     \end{cases}
%     . 
%     \nonumber 
% \end{equation}
% During each interval $t \in [t_{\mathrm{track},j}, t_{\mathrm{track},j} + \Delta t_{\mathrm{track},j}]$, we provide $N_{\rm meas} = 10$ equally spaced measurements. 
%Thus, the last measurement provided during the tracking window occurs $t_1 - 7 + \Delta t_{\rm track} = 6$ hours before the maneuver at $t_1$.
Thus, at $t_1$, Algorithm~\ref{alg:mpc_tracking} uses the EKF's predicted state estimate following the latest measurement update provided $t_1 - 7 + \Delta t_{\rm track} = 6$ hours earlier.
}

% \red{
% We define the \textit{data cutoff time} $\Delta t_{\rm cutoff} = \Delta t_4 - \Delta t_{\rm track}$, such that $t_1 - \Delta t_{\rm cutoff}$ corresponds to the final measurement update before the maneuver at $t_1$. 
% Note that $\Delta t_{\rm cutoff}$ has a direct implication on the cumulative SK cost since it impacts the navigation error at the control time. 
% % In the simulation, we fix $\Delta t_{\mathrm{track},1}$ through $\Delta t_{\mathrm{track},3}$, and run multiple cases with varying $\Delta t_{\mathrm{track},4}$. 
% The adopted values are provided in Table~\ref{tab:error_parameters}. 
% }


% ------------------------------------------------------- %
\subsection{Control Trigger Condition}
\ACCrev{To improve the delta-V performance of the SKMPC under navigation and execution errors, we consider a trigger condition to determine whether a maneuver is necessitated.}
% Proof provided is based 
% informal statement on why it's ok - epsilon definition swapping is ok
% "interpret 
% we might break recursive feasibility however, we can recover [..] because of the 
% because if hte prtedicted state is already feasible with 0 control, 
%
The condition checks if the unsteered state predicted until $t_N$ lies within an ellipsoid about the baseline with radii $\epsilon_{r,\mathrm{trig}}$ in position components and $\epsilon_{v,\mathrm{trig}}$ in velocity components
\begin{align}
\label{eq:control_trigger_conditions}
    \| \rbold_0^N - {\rbold}_{N,\mathrm{ref}} \|_2 \leq \epsilon_{r,\mathrm{trig}}, 
    \, %\nonumber \\
    \| \vbold_0^N - {\vbold}_{N,\mathrm{ref}} \|_2 \leq \epsilon_{v,\mathrm{trig}}.
    %\nonumber 
\end{align}
Tolerances $\epsilon_{r,\mathrm{trig}}$ and $\epsilon_{v,\mathrm{trig}}$ do not need to be the same as $\epsilon_r$ and $\epsilon_v$ in~\eqref{eq:termnal_set_ellipsoid}. 
In fact, choosing $\epsilon_{r/v} < \epsilon_{r/v,\mathrm{trig}}$ in general makes the closed loop more robust against uncertainties. 
\ACCrev{When using~\eqref{eq:control_trigger_conditions}, recursive feasibility may be recovered by considering the proof based on $\epsilon_{r,\mathrm{trig}}$ and $\epsilon_{v,\mathrm{trig}}$ instead of $\epsilon_{r}$ and $\epsilon_{v}$, assuming a sufficiently large $u_{\max}$.}

% ========================================================================= %
\section{Numerical Results}
\label{sec:numerical_results}
\red{
We conduct a Monte-Carlo experiment, where each sample consists of navigating and performing SK over $300$ revolutions along the NRHO, corresponding to over 5.3 years. 
% The controller is configured with a control horizon $K = 2$ and a prediction horizon $N = 6$. 
We use $K = 2$, $N = 6$, and $u_{\max} = 1\,\mathrm{m/s}$, with triggering thresholds $\epsilon_{r,\rm trig} = 100\,\mathrm{km}$ and $\epsilon_{v,\rm trig} = 20\,\mathrm{m/s}$, and terminal constraint radii $\epsilon_r = 25\,\mathrm{km}$ and $\epsilon_v = 5\,\mathrm{m/s}$. 
All thresholds are defined in $\mathcal{F}_{\rm EM}$. 
We conduct three separate experiments, using 1, 2, and 3 desaturation events per revolution, at $\theta_{\rm desat}$ provided in Table~\ref{tab:error_parameters}.}

\ACCrev{The dynamics is integrated using the explicit embedded Runge-Kutta Prince-Dormand (8,9) method from the GNU Scientific Library~\cite{gough2009gnu}.
The SKMPC takes an average of \SI{1.76}{sec} to solve on a single Intel i7-12700 CPU; the majority of the computational effort comes from propagating and constructing the STMs.}

% \red{
% In the remainder of this section, we first present the performance of the EKF, noting, in particular, the state estimate error at the maneuver location.
% Taking note of the navigation error, we present the controller's cost performance, followed by its tracking performance. 
% }

% Controller parameters are given in Table \ref{tab:controller_parameters}. 
% Specifically, two controllers are considered: configuration A uses the stable subspace-based terminal set, while configuration B uses the simpler ellipsoid-based terminal set. 
% We note that configuration B is expected to outperform A, as the smaller targeted set is expected to require larger controls, while the contracting behavior of the stable subspace is attenuated by the existence of noise. 

% For both controllers, the prediction horizon $N$ has been tuned after preliminary testing to result in no failed scenarios for 100 Monte Carlo runs. Each run consists of 600 revolutions, which corresponds to 10.8 years on the NRHO. 
% The choice of $N$ is impacted by $\kappa (\Phibold_{N,0})$ and by the various errors associated with the operation of the spacecraft. 
% Note also that configuration A necessitates a smaller $N$ as predicting the stable modes $\ybold^s_i$ is more susceptible to errors than predicting the final state $\xbold_0^N$. 


% ------------------------------------------------------- %
\subsection{Navigation Performance}
\red{
To assess the performance of the SKMPC, we first look at the navigation estimates provided to the controller. 
Figure~\ref{fig:filter_recurse} shows the estimation error of the EKF for the case involving 3 desaturation events; only the first 60 days are shown for the sake of clarity, as the filter performance is qualitatively similar across the remaining 240 days. 
The filter performance for 1 and 2 desaturation events are qualitatively similar. 
For assessing SK activities, we focus on navigation performance at the maneuver time. 
Table~\ref{tab:premaneuver_state_estim_error} shows the numerical $3$-$\sigma$ pre-maneuver state estimation error with 1, 2, and 3 desaturation events. 
As the number of desaturation events increases, the navigation error at the control epoch gets worse.}

\subsection{Cost Performance}
\red{
The SK cost increases as the navigation performance worsens, as shown in Table~\ref{tab:controller_performances}. 
%
Figure~\ref{fig:MC_cumulative_cost_history} shows the cumulative cost history with 3 desaturation events. Even in this case where the disturbance is largest, the cumulative cost follows a predominantly linear trend, indicating that the SKMPC is applying the appropriate level of control effort to keep the spacecraft motion near the baseline despite the uncertainties. 
%
Through preliminary experiments, we find the cumulative SK cost to be particularly sensitive to errors in velocity estimates. 
The costs reported in Table~\ref{tab:controller_performances} are in accordance to our previous study~\cite{ACC2025trackingMPC} which assumed a fixed navigation uncertainty $3$-$\sigma$'s of \SI{1.5}{km} and \SI{0.8}{cm/s}, and is comparable to those reported with the use of the $x$-axis crossing control~\cite{Davis2022}. 
}

% plot of filter
\begin{figure}[ht]
    \centering
    \includegraphics[width=0.950\linewidth]{plots_TCST/filter_residual_target6_horizon300_cutoff6_desat3.png}
    \caption{Estimation error in $\mathcal{F}_{\rm Inr}$ with 3 desaturation events}
    \label{fig:filter_recurse}
\end{figure}

% pre-maneuver state error
\begin{table}[ht]
    \centering
    \caption{Pre-maneuver state estimation error in $\mathcal{F}_{\rm EM}$}
    \label{tab:premaneuver_state_estim_error}
    \begin{tabular}{@{}lrrr@{}}
        \toprule
        Number of desaturation events & $1$ & $2$ & $3$ \\
        \midrule
        $(\hat{x} - x)$ 3-$\sigma$, \SI{}{km}       & 0.988 & 1.065 & 1.153 \\
        $(\hat{y} - y)$ 3-$\sigma$, \SI{}{km}       & 1.181 & 1.341 & 1.533 \\
        $(\hat{z} - z)$ 3-$\sigma$, \SI{}{km}       & 0.661 & 0.687 & 0.720 \\
        $(\hat{v}_x - v_x)$ 3-$\sigma$, \SI{}{cm/s} & 0.199 & 0.206 & 0.213 \\
        $(\hat{v}_y - v_y)$ 3-$\sigma$, \SI{}{cm/s} & 0.800 & 0.959 & 1.119 \\
        $(\hat{v}_z - v_z)$ 3-$\sigma$, \SI{}{cm/s} & 0.110 & 0.122 & 0.138 \\
        \bottomrule
    \end{tabular}
\end{table}

% table of controller performances
\begin{table}[ht!]
    \centering
    \caption{Cost statistics from Monte-Carlo experiments}
    \begin{tabular}{lrrr}
    \toprule
    Number of desaturation events & $1$ & $2$ & $3$ \\
    \midrule
    Per maneuver mean, \SI{}{cm/s}              & 2.42   & 2.95   & 3.69       \\
    Yearly mean, \SI{}{cm/s}                    & 120.63 & 162.58 & 196.12     \\
    Yearly standard deviation, \SI{}{cm/s}      & 7.87   & 8.78   & 10.20      \\
    Yearly $95^{\rm th}$ percentile, \SI{}{cm/s}& 133.32 & 177.52 & 213.11     \\
    \bottomrule
    \end{tabular}
    \label{tab:controller_performances}
\end{table}

% cumulative cost history
\begin{figure}[ht!]
    \centering
    \includegraphics[width=0.950\linewidth]{plots_TCST/cumulative_cost_vs_time_target6_horizon300_cutoff6_desat3.pdf}
    \caption{Cumulative cost history with 3 desaturation events}
    \label{fig:MC_cumulative_cost_history}
\end{figure}

% ------------------------------------------------------- %
\subsection{Tracking Performance}
\red{
We now analyze the tracking capability of the SKMPC. 
We first examine the deviation between the controlled spacecraft state and the baseline across the simulation horizon. 
We then look at the deviation of the epoch and state between the controlled spacecraft and the baseline. 
}

% - - - - - - - - - - - - - - - %
\subsubsection{Global Tracking Performance}
\red{
Figure~\ref{fig:global_tracking_error_target6_horizon300_cutoff6_desat3} shows the state deviation from the baseline over the first 60 days of the recursion. 
There is a clear periodic trend, where deviations are minimal except for the spikes at intervals of the NRHO period. 
These spikes correspond to perilune passes, where both the position and velocity vectors change rapidly. 
In such intervals, if there is a phase deviation, where the spacecraft leads ahead of or lags behind the baseline, the state deviation will be large even though the traced path itself may be close to the baseline. 
}
\begin{figure}[ht]
    \centering
    \includegraphics[width=0.950\linewidth]{plots_TCST/global_tracking_error_target6_horizon300_cutoff6_desat3.png}
    \caption{State deviation in $\mathcal{F}_{\rm EM}$}
    \label{fig:global_tracking_error_target6_horizon300_cutoff6_desat3}
\end{figure}

% - - - - - - - - - - - - - - - %
\subsubsection{Perilune Tracking Performance}
To isolate the effect of phase deviation, we compare the epochs and states at perilune passes to the corresponding perilune passes of the baseline. 
Figures~\ref{fig:perilune_deviation_epoch_target6_horizon300_cutoff6_desat3} and~\ref{fig:perilune_deviation_state_target6_horizon300_cutoff6_desat3} show the deviation of the epoch and state in $\mathcal{F}_{\rm EM}$ at each perilune passage. 
The SKMPC is found to keep the spacecraft to within $20$ minutes of perilune pass deviation, with the pass occurring within position deviations of about \SI{25}{km} and velocity deviations of about \SI{5}{m/s}. 
In $\mathcal{F}_{\rm EM}$, perilunes occur approximately along the $+z$ axis, with the spacecraft's motion approximately perpendicular to the position vector; thus, the error is found to be larger in $z$ compared to $x$ and $y$ in position components, and $v_x$, and $v_y$ compared to $v_z$ in velocity components. 
The tracking performance is improved with the SKMPC compared to the $x$-axis crossing control which reports perilune deviations of up to \SI{80}{km} in position and $48$ minutes in epoch~\cite{Davis2022}; with the SKMPC, we achieve a $3.2\times$ improvement in perilune position tracking, and a $2.4\times$ improvement in perilune epoch tracking. 
This improved tracking performance is due to the SKMPC avoiding phase drift through full-state tracking suffered by $x$-axis crossing control. 
In general, tighter tracking of the baseline is desirable since more stringent requirements can be met with regard to the spacecraft design or payload operations that require remaining closer to the intended path. 
For instance, the NRHO baseline for the Gateway is designed to be free of any Earth-shadowing eclipses~\cite{ZimovanSpreen2023}, and tight tracking can ensure no such eclipse occurs during the flight subject to uncertainties as well. 
% perilune epoch deviation
\begin{figure}
    \centering
    \includegraphics[width=0.95\linewidth]{plots_TCST/perilune_deviation_epoch_target6_horizon300_cutoff6_desat3.png}
    \caption{Epoch deviation at perilune passes}
    \label{fig:perilune_deviation_epoch_target6_horizon300_cutoff6_desat3}
\end{figure}
% perilune state deviation
\begin{figure}
    \centering
    \includegraphics[width=0.95\linewidth]{plots_TCST/perilune_deviation_state_target6_horizon300_cutoff6_desat3.png}
    \caption{State deviation in $\mathcal{F}_{\rm EM}$ at perilune passes}
    \label{fig:perilune_deviation_state_target6_horizon300_cutoff6_desat3}
\end{figure}


% ========================================================================= %
\section{Conclusion}
\label{sec:conclusion}
In this work, we proposed a targeting MPC for the SK problem on the NRHO. 
\red{This SKMPC achieves full-state tracking by taking into account two maneuvers within its control horizon. 
The maneuvers are placed one revolution apart, making our approach compatible with the single maneuver-per-revolution requirement typical in space missions on the NRHO. 
Through full-state tracking, the SKMPC overcomes the issue of uncontrolled drift in phase ahead or behind the tracked baseline typically encountered by other state-of-the-art SK schemes with single maneuver-per-revolution. 
}
%This SKMPC leverages two maneuvers within its control horizon to have sufficient degrees of freedom for full-state tracking. 
%Meanwhile, by placing each maneuver in the control horizon one revolution apart, the proposed SKMPC can be used as a single maneuver-per-revolution scheme, a common operational requirement in space missions on the NRHO to simplify the spacecraft's operation. 
\red{
We demonstrated the SKMPC results in more precise tracking with output-feedback using an EKF with range and range-rate measurements, in closed loop with high-fidelity dynamics, and subject to realistic error models. 
}
Our approach achieves cumulative maneuver costs comparable to SK approaches proposed in the astrodynamics literature, while resulting in tighter tracking of the reference orbit in both space and phase without requiring additional ad-hoc heuristics, as in $x$-axis crossing control. % necessitated by other approaches. 


% ========================================================================= %
% \section*{Acknowledgment}
% The authors thank Samet Uzun for helpful discussions. 
\balance


% ========================================================================= %
\bibliographystyle{IEEEtran}
\bibliography{IEEEabrv,references.bib}
% \bibliography{references.bib}

% \section*{References}

% Please number citations consecutively within brackets \cite{b1}. The 
% sentence punctuation follows the bracket \cite{b2}. Refer simply to the reference 
% number, as in \cite{b3}---do not use ``Ref. \cite{b3}'' or ``reference \cite{b3}'' except at 
% the beginning of a sentence: ``Reference \cite{b3} was the first $\ldots$''

% Number footnotes separately in superscripts. Place the actual footnote at 
% the bottom of the column in which it was cited. Do not put footnotes in the 
% abstract or reference list. Use letters for table footnotes.

% Unless there are six authors or more give all authors' names; do not use 
% ``et al.''. Papers that have not been published, even if they have been 
% submitted for publication, should be cited as ``unpublished'' \cite{b4}. Papers 
% that have been accepted for publication should be cited as ``in press'' \cite{b5}. 
% Capitalize only the first word in a paper title, except for proper nouns and 
% element symbols.

% For papers published in translation journals, please give the English 
% citation first, followed by the original foreign-language citation \cite{b6}.

% \begin{thebibliography}{00}
% \bibitem{b1} G. Eason, B. Noble, and I. N. Sneddon, ``On certain integrals of Lipschitz-Hankel type involving products of Bessel functions,'' Phil. Trans. Roy. Soc. London, vol. A247, pp. 529--551, April 1955.
% \bibitem{b2} J. Clerk Maxwell, A Treatise on Electricity and Magnetism, 3rd ed., vol. 2. Oxford: Clarendon, 1892, pp.68--73.
% \bibitem{b3} I. S. Jacobs and C. P. Bean, ``Fine particles, thin films and exchange anisotropy,'' in Magnetism, vol. III, G. T. Rado and H. Suhl, Eds. New York: Academic, 1963, pp. 271--350.
% \bibitem{b4} K. Elissa, ``Title of paper if known,'' unpublished.
% \bibitem{b5} R. Nicole, ``Title of paper with only first word capitalized,'' J. Name Stand. Abbrev., in press.
% \bibitem{b6} Y. Yorozu, M. Hirano, K. Oka, and Y. Tagawa, ``Electron spectroscopy studies on magneto-optical media and plastic substrate interface,'' IEEE Transl. J. Magn. Japan, vol. 2, pp. 740--741, August 1987 [Digests 9th Annual Conf. Magnetics Japan, p. 301, 1982].
% \bibitem{b7} M. Young, The Technical Writer's Handbook. Mill Valley, CA: University Science, 1989.
% \end{thebibliography}
% \vspace{12pt}
% \color{red}
% IEEE conference templates contain guidance text for composing and formatting conference papers. Please ensure that all template text is removed from your conference paper prior to submission to the conference. Failure to remove the template text from your paper may result in your paper not being published.

\end{document}
