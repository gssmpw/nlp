%%
%% This is file `sample-acmsmall.tex',
%% generated with the docstrip utility.
%%
%% The original source files were:
%%
%% samples.dtx  (with options: `acmsmall')
%%
%% IMPORTANT NOTICE:
%%
%% For the copyright see the source file.
%%
%% Any modified versions of this file must be renamed
%% with new filenames distinct from sample-acmsmall.tex.
%%
%% For distribution of the original source see the terms
%% for copying and modification in the file samples.dtx.
%%
%% This generated file may be distributed as long as the
%% original source files, as listed above, are part of the
%% same distribution. (The sources need not necessarily be
%% in the same archive or directory.)
%%
%% Commands for TeXCount
%TC:macro \cite [option:text,text]
%TC:macro \citep [option:text,text]
%TC:macro \citet [option:text,text]
%TC:envir table 0 1
%TC:envir table* 0 1
%TC:envir tabular [ignore] word
%TC:envir displaymath 0 word
%TC:envir math 0 word
%TC:envir comment 0 0
%%
%%
%% The first command in your LaTeX source must be the \documentclass command.
%\documentclass[acmsmall]{acmart}
%\documentclass[sigconf,screen,review,anonymous]{acmart}
%\documentclass[sigconf,screen,review]{acmart}
%camera-ready instruction
\documentclass[sigconf,screen,nonacm]{acmart}
%\acmBooktitle{Proceedings of the 32nd ACM Symposium on the Foundations of Software Engineering (FSE '24), November 15--19, 2024, Porto de Galinhas, Brazil}
%% NOTE that a single column version is required for
%% submission and peer review. This can be done by changing
%% the \doucmentclass[...]{acmart} in this template to
%% \documentclass[manuscript,screen]{acmart}
%%
%% To ensure 100% compatibility, please check the white list of
%% approved LaTeX packages to be used with the Master Article Template at
%% https://www.acm.org/publications/taps/whitelist-of-latex-packages
%% before creating your document. The white list page provides
%% information on how to submit additional LaTeX packages for
%% review and adoption.
%% Fonts used in the template cannot be substituted; margin
%% adjustments are not allowed.
%%

%% Rights management information.  This information is sent to you
%% when you complete the rights form.  These commands have SAMPLE
%% values in them; it is your responsibility as an author to replace
%% the commands and values with those provided to you when you
%% complete the rights form.
\setcopyright{rightsretained}
% \copyrightyear{2025}
% \acmYear{2025}
% \acmDOI{XXXXXXX.XXXXXXX}


%%
%% These commands are for a JOURNAL article.
%\acmJournal{JACM}
%\acmVolume{37}
%\acmNumber{4}
%\acmArticle{111}
%\acmMonth{8}

%%
%% Submission ID.
%% Use this when submitting an article to a sponsored event. You'll
%% receive a unique submission ID from the organizers
%% of the event, and this ID should be used as the parameter to this command.
%%\acmSubmissionID{123-A56-BU3}

%%
%% For managing citations, it is recommended to use bibliography
%% files in BibTeX format.
%%
%% You can then either use BibTeX with the ACM-Reference-Format style,
%% or BibLaTeX with the acmnumeric or acmauthoryear sytles, that include
%% support for advanced citation of software artefact from the
%% biblatex-software package, also separately available on CTAN.
%%
%% Look at the sample-*-biblatex.tex files for templates showcasing
%% the biblatex styles.
%%

%%
%% The majority of ACM publications use numbered citations and
%% references.  The command \citestyle{authoryear} switches to the
%% "author year" style.
%%
%% If you are preparing content for an event
%% sponsored by ACM SIGGRAPH, you must use the "author year" style of
%% citations and references.
%% Uncommenting
%% the next command will enable that style.
%%\citestyle{acmauthoryear}

%%
%% end of the preamble, start of the body of the document source.
\usepackage{xcolor}
\usepackage{wrapfig}

\usepackage{amsmath}
\usepackage{algorithm}
\usepackage{algpseudocode}

\usepackage{multirow}
\usepackage[normalem]{ulem}
\useunder{\uline}{\ul}{}
\usepackage{diagbox}

\usepackage{pifont}

\usepackage{bbding}

\usepackage{hhline}
\usepackage{colortbl}
\usepackage{framed}
\usepackage{mdframed}
\usepackage{amsmath}
\usepackage{array}
% \usepackage[table,xcdraw]{xcolor}
\usepackage[caption=false]{subfig}
\usepackage{listings}
\definecolor{darkred}{rgb}{0.6,0.0,0.0}


\definecolor{mycolor1}{RGB}{0, 0, 180}
\definecolor{mycolor2}{RGB}{31, 132, 31}
%\usepackage[true]{anonymous}
%\usepackage[true]{anonymous-acm}
% Reduce the vertical space before and after section titles

% Define custom spacing commands
\newcommand{\sectionspacing}{\vspace{-2.5ex}}
\newcommand{\subsectionspacing}{\vspace{-2ex}}
\newcommand{\subsubsectionspacing}{\vspace{-1.5ex}}
\newcommand{\aftersectionskip}{\vspace{1.5ex}}
\newcommand{\aftersubsectionskip}{\vspace{1ex}}
\newcommand{\aftersubsubsectionskip}{\vspace{0.5ex}}

\usepackage{microtype}
\usepackage{balance}

%camera-ready setting


\newcommand{\tool}{\textit{}}

\algnewcommand\algorithmicoutput{\textbf{Output:}}
\algnewcommand\OUTPUT{\item[\algorithmicoutput]}

\algnewcommand\algorithmicdefine{\textbf{Define:}}
\algnewcommand\DEFINE{\item[\algorithmicdefine]}

\newcommand{\tocheckJZ}[1]{{\color{red} #1}}

\newif\ifdoubleblind
% Uncomment the next line for double-blind submission
% \doubleblindtrue
% Uncomment the next line for camera-ready submission
\doubleblindfalse

\newenvironment{packed_enum}{
\vspace{-5pt}
\begin{enumerate}
  \setlength{\itemsep}{1pt}
  \setlength{\parskip}{0pt}
  \setlength{\parsep}{0pt}
}{\end{enumerate}}
\vspace{-10pt}

\setlength\parindent{0pt}
\setlength{\parskip}{0.1mm}


%\linespread{0.95}
%% \BibTeX command to typeset BibTeX logo in the docs
%\AtBeginDocument{%
%  \providecommand\BibTeX{{%
%    \normalfont B\kern-0.5em{\scshape i\kern-0.25em b}\kern-0.8em\TeX}}}


\let\oldemptyset\emptyset
\let\emptyset\varnothing
\def\BibTeX{{\rm B\kern-.05em{\sc i\kern-.025em b}\kern-.08em
    T\kern-.1667em\lower.7ex\hbox{E}\kern-.125emX}}

%\AtBeginDocument{%
%  \providecommand\BibTeX{{%
%  Bib\TeX}}}
%\usepackage[maxnames=2]{biblatex}
%\addbibresource{main.bib}


%%% The following is specific to FSE '24-IVR and the paper
%%% 'Testing Learning-Enabled Cyber-Physical Systems with Large-Language Models: A Formal Approach'
%%% by Xi Zheng, Aloysius K. Mok, Ruzica Piskac, Yong Jae Lee, Bhaskar Krishnamachari, Dakai Zhu, Oleg Sokolsky, and Insup Lee.
%%%
% \setcopyright{rightsretained}
%\acmDOI{10.1145/3663529.3663779}
% \acmYear{2025}
% \copyrightyear{2025}
%\acmSubmissionID{fsecomp24ivr-p36-p}
%\acmISBN{979-8-4007-0658-5/24/07}
%\acmConference[FSE Companion '25]{Companion Proceedings of the 32nd ACM International Conference on the Foundations of Software Engineering}{July 15--19, 2024}{Porto de Galinhas, Brazil}
% \acmBooktitle{Companion Proceedings of the 33rd ACM Symposium on the Foundations of Software Engineering (FSE '25), June 23--27, 2025, Trondheim, Norway}
%\acmBooktitle{Companion Proceedings of the 32nd ACM International Conference on the Foundations of Software Engineering (FSE Companion '24), July 15--19, 2024, Porto de Galinhas, Brazil}
%\received{18-JAN-2024}
%\received[accepted]{2024-04-09}

%\renewcommand{\shortauthors}{Zheng et al.}

\begin{document}

\title{NeuroStrata: Harnessing Neurosymbolic Paradigms for Improved Design, Testability, and Verifiability of Autonomous CPS}


\ifdoubleblind
  % Double-blind submission: Hide author names
  \author{Anonymous Authors}
\else
  % Camera-ready submission: Show author names
    \author{Xi Zheng}
    \orcid{0000-0002-2572-2355}
    \affiliation{%
      \institution{Macquarie University}
      \city{Sydney}
      \country{Australia}
    }
    \email{james.zheng@mq.edu.au}


    \author{Ziyang Li}
    \orcid{0000-0003-3925-9549}
    \affiliation{%
      \institution{University of Pennsylvania}
      \city{Philadelphia}
      \country{USA}
    }
    \email{liby99@seas.upenn.edu}

    \author{Ivan Ruchkin}
    \orcid{0000-0003-3546-414X}
    \affiliation{%
      \institution{University of Florida}
      \city{Gainesville}
      \country{USA}
    }
    \email{iruchkin@ece.ufl.edu}
    % \IEEEcompsocitemizethanks{
    % \IEEEcompsocthanksitem test
    % }


    \author{Ruzica Piskac}
    \orcid{0000-0002-3267-0776}
    \affiliation{%
      \institution{Yale University}
      \city{New Heaven}
      \country{USA}
    }
    \email{ruzica.piskac@yale.edu}

    \author{Miroslav Pajic}
    %\orcid{0000-0001-5282-0658}
    \affiliation{%
      \institution{Duke University}
      \city{Durham}
      \country{USA}
    }
    \email{miroslav.pajic@duke.edu}


    \thanks{Ivan Ruchkin is supported in part by the NSF Grant CCF-2403616. Any opinions, findings, conclusions, or recommendations expressed in this material are those of the authors and do not necessarily reflect the views of the National Science Foundation (NSF) or the United States Government. }
 %   \author{Armando Solar-Lezama}
    %\orcid{0000-0001-5282-0658}
 %   \affiliation{%
 %     \institution{Massachusetts Institute of Technology}
   %   \city{Cambridge}
   %   \country{USA}
   % }
   % \email{asolar@csail.mit.edu}

\fi




%%
%% The "title" command has an optional parameter,
%% allowing the author to define a "short title" to be used in page headers.
%\title{Testing Learning-Enabled Cyber-Physical Systems with Large-Language Models: A Formal Approach}

%%
%% The "author" command and its associated commands are used to define
%% the authors and their affiliations.
%% Of note is the shared affiliation of the first two authors, and the
%% "authornote" and "authornotemark" commands
%% used to denote shared contribution to the research.
%\authoranon{

%\author{Xi Zheng}
%\affiliation{
 % \institution{Macquarie University}
 % \country{Australia}
%}
%\email{james.zheng@mq.edu.au}


%\author{Aloysius K. Mok}
%\affiliation{
 %   \institution{University of Texas at Austin}
  %  \country{USA}
%}
%\email{mok@cs.utexas.edu}


%\author{Ruzica Piskac}
%\affiliation{
 %%   \institution{Yale University}
   %\country{USA}
%}
%\email{ruzica.piskac@yale.edu}

%\author{Yong Jae Lee}
%\affiliation{
 %   \institution{University of Wisconsin Madison}
  %  \country{USA}
%}
%\email{yongjaelee@cs.wisc.edu}

%\author{Yong Jae Lee}
%\email{yongjaelee@cs.wisc.edu}
%\affiliation{
 %   \institution{University of Wisconsin Madison}
  %  \country{USA}
%}

%\author{Bhaskar Krishnamachari}
%\affiliation{
 %   \institution{University of Southern California}
  % \country{USA}
%}
%\email{bkrishna@usc.edu}

%\author{Dakai Zhu}
%\affiliation{
    %\institution{University of Texas %at San Antonio}
   %\country{USA}
%}
%\email{Dakai.Zhu@utsa.edu}

%\author{Oleg Sokolsky}
%\affiliation{
 %   \institution{University of Pennsylvania}
  %  \country{USA}
%}
%\email{sokolsky@cis.upenn.edu}


%\author{Insup Lee}
%\affiliation{
%    \institution{University of Pennsylvania}
 %   \country{USA}
%}
%\email{lee@seas.upenn.edu}


%}


%\author{Xi Zheng, Aloysius K. Mok, Insup Lee, Oleg Sokolsky, Ruzica Piskac, Yong Jae Lee, Bhaskar Krishnamachari}
%\IEEEcompsocitemizethanks{\IEEEcompsocthanksitem X. Zheng is with the School
%of Computing, Macquarie University, Sydney,
%NSW.\protect
%E-mail: james.zheng@mq.edu.au,
%\IEEEcompsocthanksitem A. K. Mok is with the Department of Computer Science, University of Texas at Austin.\protect
%E-mail: mok@cs.utexas.edu
%\IEEEcompsocthanksitem I. Lee and O. Sokolsky are with the Department of Computer %E-mail: [lee, sokolsky]@cis.upenn.edu
%\IEEEcompsocthanksitem R. Piskac is with the Department of Computer Science, Yale University.
%\protect
%E-mail: ruzica.piskac@yale.edu
%\IEEEcompsocthanksitem Y.J. Lee is with the Department of Computer Science,
%University of Wisconsin Madison.
%\protect
%E-mail: yongjaelee@cs.wisc.edu
%\IEEEcompsocthanksitem B. Krishnamachari is with the Department of Computer Science, University of Southern California.
%\protect
%E-mail: bkrishna@usc.edu
%\IEEEcompsocthanksitem D. Kai is with the Department of Computer Science, University of Texas at Dallas.
%\protect
%E-mail: bkrishna@usc.edu
%}% <-this % stops an unwanted space



%%
%% By default, the full list of authors will be used in the page
%% headers. Often, this list is too long, and will overlap
%% other information printed in the page headers. This command allows
%% the author to define a more concise list
%% of authors' names for this purpose.
%\renewcommand{\shortauthors}{Zheng et al.}
%\renewcommand{\shortauthors}{Zheng, Li, Piskac, Ruchkin, Pajic, and Solar-Lezama}


%%
%% The abstract is a short summary of the work to be presented in the
%% article.
\begin{abstract}
Autonomous cyber-physical systems (CPSs) leverage AI for perception, planning, and control but face trust and safety certification challenges due to inherent uncertainties. The neurosymbolic paradigm replaces stochastic layers with interpretable symbolic AI, enabling determinism. While promising, challenges like multisensor fusion, adaptability, and verification remain. This paper introduces \textbf{NeuroStrata}, a neurosymbolic framework to enhance the testing and verification of autonomous CPS. We outline its key components, present early results, and detail future plans.
%Autonomous cyber-physical systems (CPSs) leverage AI for perception, planning, and control --- but face challenges in trust and safety certification due to their highly uncertain nature. The neurosymbolic paradigm replaces stochastic layers with interpretable symbolic AI, enabling deterministic systems. While promising, challenges such as multisensor fusion, adaptability, and verification persist. This paper introduces \textbf{NeuroStrata} --- a novel neurosymbolic framework designed to advance the neurosymbolic paradigm for enhancing the testing and verification of autonomous CPS. We outline the key components, present early results, and detail our future plans.%We describe the key components and early results.

%This paper explores advances and future directions for neurosymbolic paradigms to enhance the testing and verification of autonomous CPSs.



%have gained significant momentum in both industry and research due to their ability to leverage AI components for perception, planning, and control from complex sensor data. However, the stochastic nature of these AI components makes autonomous CPSs, such as autonomous vehicles, unmanned aerial systems, and robotics, untrustworthy and challenging to certify for safety or gain human trust. The neurosymbolic paradigm addresses this issue by replacing stochastic AI decision layers with symbolic AI, incorporating knowledge graphs or logic-based reasoning, to create interpretable and deterministic systems. Despite promising progress, challenges remain, including multisensor fusion, adaptability to deployment environments that require continuous learning, and verification of neurosymbolic components. This paper highlights current advances in the field and explores key directions for leveraging neurosymbolic paradigms to enhance the testing and verification of autonomous CPS.
\end{abstract}
%%
%% The code below is generated by the tool at http://dl.acm.org/ccs.cfm.
%% Please copy and paste the code instead of the example below.
%%
\begin{CCSXML}
<ccs2012>
   <concept>
       <concept_id>10011007.10011074.10011099</concept_id>
       <concept_desc>Software and its engineering~Software verification and validation</concept_desc>
       <concept_significance>500</concept_significance>
       </concept>
   <concept>
       <concept_id>10010520.10010553</concept_id>
       <concept_desc>Computer systems organization~Embedded and cyber-physical systems</concept_desc>
       <concept_significance>500</concept_significance>
       </concept>
   <concept>
       <concept_id>10010147.10010257</concept_id>
       <concept_desc>Computing methodologies~Machine learning</concept_desc>
       <concept_significance>300</concept_significance>
       </concept>
 </ccs2012>
\end{CCSXML}

\ccsdesc[500]{Software and its engineering~Software verification and validation}
\ccsdesc[500]{Computer systems organization~Embedded and cyber-physical systems}
\ccsdesc[300]{Computing methodologies~Machine learning}
\ccsdesc[300]{Theory of computation~Formal languages and automata theory}


%%
%% Keywords. The author(s) should pick words that accurately describe
%% the work being presented. Separate the keywords with commas.
\keywords{AI-based Systems, Cyber-Physical Systems, Neurosymbolic AI, Testing, Verification}

% \received{20 February 2007}
% \received[revised]{12 March 2009}
% \received[accepted]{5 June 2009}

%%
%% This command processes the author and affiliation and title
%% information and builds the first part of the formatted document.
\maketitle
%\pagestyle{plain}

\vspace{3mm}
\sectionspacing
\section{CONTEXT, MOTIVATION, AND AIMS}
\section{Introduction}
\label{sec:introduction}
The business processes of organizations are experiencing ever-increasing complexity due to the large amount of data, high number of users, and high-tech devices involved \cite{martin2021pmopportunitieschallenges, beerepoot2023biggestbpmproblems}. This complexity may cause business processes to deviate from normal control flow due to unforeseen and disruptive anomalies \cite{adams2023proceddsriftdetection}. These control-flow anomalies manifest as unknown, skipped, and wrongly-ordered activities in the traces of event logs monitored from the execution of business processes \cite{ko2023adsystematicreview}. For the sake of clarity, let us consider an illustrative example of such anomalies. Figure \ref{FP_ANOMALIES} shows a so-called event log footprint, which captures the control flow relations of four activities of a hypothetical event log. In particular, this footprint captures the control-flow relations between activities \texttt{a}, \texttt{b}, \texttt{c} and \texttt{d}. These are the causal ($\rightarrow$) relation, concurrent ($\parallel$) relation, and other ($\#$) relations such as exclusivity or non-local dependency \cite{aalst2022pmhandbook}. In addition, on the right are six traces, of which five exhibit skipped, wrongly-ordered and unknown control-flow anomalies. For example, $\langle$\texttt{a b d}$\rangle$ has a skipped activity, which is \texttt{c}. Because of this skipped activity, the control-flow relation \texttt{b}$\,\#\,$\texttt{d} is violated, since \texttt{d} directly follows \texttt{b} in the anomalous trace.
\begin{figure}[!t]
\centering
\includegraphics[width=0.9\columnwidth]{images/FP_ANOMALIES.png}
\caption{An example event log footprint with six traces, of which five exhibit control-flow anomalies.}
\label{FP_ANOMALIES}
\end{figure}

\subsection{Control-flow anomaly detection}
Control-flow anomaly detection techniques aim to characterize the normal control flow from event logs and verify whether these deviations occur in new event logs \cite{ko2023adsystematicreview}. To develop control-flow anomaly detection techniques, \revision{process mining} has seen widespread adoption owing to process discovery and \revision{conformance checking}. On the one hand, process discovery is a set of algorithms that encode control-flow relations as a set of model elements and constraints according to a given modeling formalism \cite{aalst2022pmhandbook}; hereafter, we refer to the Petri net, a widespread modeling formalism. On the other hand, \revision{conformance checking} is an explainable set of algorithms that allows linking any deviations with the reference Petri net and providing the fitness measure, namely a measure of how much the Petri net fits the new event log \cite{aalst2022pmhandbook}. Many control-flow anomaly detection techniques based on \revision{conformance checking} (hereafter, \revision{conformance checking}-based techniques) use the fitness measure to determine whether an event log is anomalous \cite{bezerra2009pmad, bezerra2013adlogspais, myers2018icsadpm, pecchia2020applicationfailuresanalysispm}. 

The scientific literature also includes many \revision{conformance checking}-independent techniques for control-flow anomaly detection that combine specific types of trace encodings with machine/deep learning \cite{ko2023adsystematicreview, tavares2023pmtraceencoding}. Whereas these techniques are very effective, their explainability is challenging due to both the type of trace encoding employed and the machine/deep learning model used \cite{rawal2022trustworthyaiadvances,li2023explainablead}. Hence, in the following, we focus on the shortcomings of \revision{conformance checking}-based techniques to investigate whether it is possible to support the development of competitive control-flow anomaly detection techniques while maintaining the explainable nature of \revision{conformance checking}.
\begin{figure}[!t]
\centering
\includegraphics[width=\columnwidth]{images/HIGH_LEVEL_VIEW.png}
\caption{A high-level view of the proposed framework for combining \revision{process mining}-based feature extraction with dimensionality reduction for control-flow anomaly detection.}
\label{HIGH_LEVEL_VIEW}
\end{figure}

\subsection{Shortcomings of \revision{conformance checking}-based techniques}
Unfortunately, the detection effectiveness of \revision{conformance checking}-based techniques is affected by noisy data and low-quality Petri nets, which may be due to human errors in the modeling process or representational bias of process discovery algorithms \cite{bezerra2013adlogspais, pecchia2020applicationfailuresanalysispm, aalst2016pm}. Specifically, on the one hand, noisy data may introduce infrequent and deceptive control-flow relations that may result in inconsistent fitness measures, whereas, on the other hand, checking event logs against a low-quality Petri net could lead to an unreliable distribution of fitness measures. Nonetheless, such Petri nets can still be used as references to obtain insightful information for \revision{process mining}-based feature extraction, supporting the development of competitive and explainable \revision{conformance checking}-based techniques for control-flow anomaly detection despite the problems above. For example, a few works outline that token-based \revision{conformance checking} can be used for \revision{process mining}-based feature extraction to build tabular data and develop effective \revision{conformance checking}-based techniques for control-flow anomaly detection \cite{singh2022lapmsh, debenedictis2023dtadiiot}. However, to the best of our knowledge, the scientific literature lacks a structured proposal for \revision{process mining}-based feature extraction using the state-of-the-art \revision{conformance checking} variant, namely alignment-based \revision{conformance checking}.

\subsection{Contributions}
We propose a novel \revision{process mining}-based feature extraction approach with alignment-based \revision{conformance checking}. This variant aligns the deviating control flow with a reference Petri net; the resulting alignment can be inspected to extract additional statistics such as the number of times a given activity caused mismatches \cite{aalst2022pmhandbook}. We integrate this approach into a flexible and explainable framework for developing techniques for control-flow anomaly detection. The framework combines \revision{process mining}-based feature extraction and dimensionality reduction to handle high-dimensional feature sets, achieve detection effectiveness, and support explainability. Notably, in addition to our proposed \revision{process mining}-based feature extraction approach, the framework allows employing other approaches, enabling a fair comparison of multiple \revision{conformance checking}-based and \revision{conformance checking}-independent techniques for control-flow anomaly detection. Figure \ref{HIGH_LEVEL_VIEW} shows a high-level view of the framework. Business processes are monitored, and event logs obtained from the database of information systems. Subsequently, \revision{process mining}-based feature extraction is applied to these event logs and tabular data input to dimensionality reduction to identify control-flow anomalies. We apply several \revision{conformance checking}-based and \revision{conformance checking}-independent framework techniques to publicly available datasets, simulated data of a case study from railways, and real-world data of a case study from healthcare. We show that the framework techniques implementing our approach outperform the baseline \revision{conformance checking}-based techniques while maintaining the explainable nature of \revision{conformance checking}.

In summary, the contributions of this paper are as follows.
\begin{itemize}
    \item{
        A novel \revision{process mining}-based feature extraction approach to support the development of competitive and explainable \revision{conformance checking}-based techniques for control-flow anomaly detection.
    }
    \item{
        A flexible and explainable framework for developing techniques for control-flow anomaly detection using \revision{process mining}-based feature extraction and dimensionality reduction.
    }
    \item{
        Application to synthetic and real-world datasets of several \revision{conformance checking}-based and \revision{conformance checking}-independent framework techniques, evaluating their detection effectiveness and explainability.
    }
\end{itemize}

The rest of the paper is organized as follows.
\begin{itemize}
    \item Section \ref{sec:related_work} reviews the existing techniques for control-flow anomaly detection, categorizing them into \revision{conformance checking}-based and \revision{conformance checking}-independent techniques.
    \item Section \ref{sec:abccfe} provides the preliminaries of \revision{process mining} to establish the notation used throughout the paper, and delves into the details of the proposed \revision{process mining}-based feature extraction approach with alignment-based \revision{conformance checking}.
    \item Section \ref{sec:framework} describes the framework for developing \revision{conformance checking}-based and \revision{conformance checking}-independent techniques for control-flow anomaly detection that combine \revision{process mining}-based feature extraction and dimensionality reduction.
    \item Section \ref{sec:evaluation} presents the experiments conducted with multiple framework and baseline techniques using data from publicly available datasets and case studies.
    \item Section \ref{sec:conclusions} draws the conclusions and presents future work.
\end{itemize}
\aftersectionskip

%\vspace{-1.5cm}

\sectionspacing
\section{Motivating System and the State-of-the-art}
\label{sec:relatedwork}
\section{Related Work}
The landscape of large language model vulnerabilities has been extensively studied in recent literature \cite{crothers2023machinegeneratedtextcomprehensive,shayegani2023surveyvulnerabilitieslargelanguage,Yao_2024,Huang2023ASO}, that propose detailed taxonomies of threats. These works categorize LLM attacks into distinct types, such as adversarial attacks, data poisoning, and specific vulnerabilities related to prompt engineering. Among these, prompt injection attacks have emerged as a significant and distinct category, underscoring their relevance to LLM security.

The following high-level overview of the collected taxonomy of LLM vulnerabilities is defined in \cite{Yao_2024}:
\begin{itemize}
    \item Adversarial Attacks: Data Poisoning, Backdoor Attacks
    \item Inference Attacks: Attribute Inference, Membership Inferences
    \item Extraction Attacks
    \item Bias and Unfairness
Exploitation
    \item Instruction Tuning Attacks: Jailbreaking, Prompt Injection.
\end{itemize}
Prompt injection attacks are further classified in \cite{shayegani2023surveyvulnerabilitieslargelanguage} into the following: Goal hijacking and \textbf{Prompt leakage}.

The reviewed taxonomies underscore the need for comprehensive frameworks to evaluate LLM security. The agentic approach introduced in this paper builds on these insights, automating adversarial testing to address a wide range of scenarios, including those involving prompt leakage and role-specific vulnerabilities.

\subsection{Prompt Injection and Prompt Leakage}

Prompt injection attacks exploit the blending of instructional and data inputs, manipulating LLMs into deviating from their intended behavior. Prompt injection attacks encompass techniques that override initial instructions, expose private prompts, or generate malicious outputs \cite{Huang2023ASO}. A subset of these attacks, known as prompt leakage, aims specifically at extracting sensitive system prompts embedded within LLM configurations. In \cite{shayegani2023surveyvulnerabilitieslargelanguage}, authors differentiate between prompt leakage and related methods such as goal hijacking, further refining the taxonomy of LLM-specific vulnerabilities.

\subsection{Defense Mechanisms}

Various defense mechanisms have been proposed to address LLM vulnerabilities, particularly prompt injection and leakage \cite{shayegani2023surveyvulnerabilitieslargelanguage,Yao_2024}. We focused on cost-effective methods like instruction postprocessing and prompt engineering, which are viable for proprietary models that cannot be retrained. Instruction preprocessing sanitizes inputs, while postprocessing removes harmful outputs, forming a dual-layer defense. Preprocessing methods include perplexity-based filtering \cite{Jain2023BaselineDF,Xu2022ExploringTU} and token-level analysis \cite{Kumar2023CertifyingLS}. Postprocessing employs another set of techniques, such as censorship by LLMs \cite{Helbling2023LLMSD,Inan2023LlamaGL}, and use of canary tokens and pattern matching \cite{vigil-llm,rebuff}, although their fundamental limitations are noted \cite{Glukhov2023LLMCA}. Prompt engineering employs carefully designed instructions \cite{Schulhoff2024ThePR} and advanced techniques like spotlighting \cite{Hines2024DefendingAI} to mitigate vulnerabilities, though no method is foolproof \cite{schulhoff-etal-2023-ignore}. Adversarial training, by incorporating adversarial examples into the training process, strengthens models against attacks \cite{Bespalov2024TowardsBA,Shaham2015UnderstandingAT}.

\subsection{Security Testing for Prompt Injection Attacks}

Manual testing, such as red teaming \cite{ganguli2022redteaminglanguagemodels} and handcrafted "Ignore Previous Prompt" attacks \cite{Perez2022IgnorePP}, highlights vulnerabilities but is limited in scale. Automated approaches like PAIR \cite{chao2024jailbreakingblackboxlarge} and GPTFUZZER \cite{Yu2023GPTFUZZERRT} achieve higher success rates by refining prompts iteratively or via automated fuzzing. Red teaming with LLMs \cite{Perez2022RedTL} and reinforcement learning \cite{anonymous2024diverse} uncovers diverse vulnerabilities, including data leakage and offensive outputs. Indirect Prompt Injection (IPI) manipulates external data to compromise applications \cite{Greshake2023NotWY}, adapting techniques like SQL injection to LLMs \cite{Liu2023PromptIA}. Prompt secrecy remains fragile, with studies showing reliable prompt extraction \cite{Zhang2023EffectivePE}. Advanced frameworks like Token Space Projection \cite{Maus2023AdversarialPF} and Weak-to-Strong Jailbreaking Attacks \cite{zhao2024weaktostrongjailbreakinglargelanguage} exploit token-space relationships, achieving high success rates for prompt extraction and jailbreaking.

\subsection{Agentic Frameworks for Evaluating LLM Security}

The development of multi-agent systems leveraging large language models (LLMs) has shown promising results in enhancing task-solving capabilities \cite{Hong2023MetaGPTMP, Wang2023UnleashingTE, Talebirad2023MultiAgentCH, Wu2023AutoGenEN, Du2023ImprovingFA}. A key aspect across various frameworks is the specialization of roles among agents \cite{Hong2023MetaGPTMP, Wu2023AutoGenEN}, which mimics human collaboration and improves task decomposition.

Agentic frameworks and the multi-agent debate approach benefit from agent interaction, where agents engage in conversations or debates to refine outputs and correct errors \cite{Wu2023AutoGenEN}. For example, debate systems improve factual accuracy and reasoning by iteratively refining responses through collaborative reasoning \cite{Du2023ImprovingFA}, while AG2 allows agents to autonomously interact and execute tasks with minimal human input.

These frameworks highlight the viability of agentic systems, showing how specialized roles and collaborative mechanisms lead to improved performance, whether in factuality, reasoning, or task execution. By leveraging the strengths of diverse agents, these systems demonstrate a scalable approach to problem-solving.

Recent research on testing LLMs using other LLMs has shown that this approach can be highly effective \cite{chao2024jailbreakingblackboxlarge, Yu2023GPTFUZZERRT, Perez2022RedTL}. Although the papers do not explicitly employ agentic frameworks they inherently reflect a pattern similar to that of an "attacker" and a "judge". \cite{chao2024jailbreakingblackboxlarge}  This pattern became a focal point for our work, where we put the judge into a more direct dialogue, enabling it to generate attacks based on the tested agent response in an active conversation.

A particularly influential paper in shaping our approach is Jailbreaking Black Box Large Language Models in Twenty Queries \cite{chao2024jailbreakingblackboxlarge}. This paper not only introduced the attacker/judge architecture but also provided the initial system prompts used for a judge.
\aftersectionskip


\sectionspacing
\section{NeuroStrata: Our Vision for Hierarchical Neurosymbolic Framework for Autonomous Systems}
\label{sec:roadmap}
%Version  5: 11th Jan
%change the first paragraph
\looseness=-1
To address the challenges of designing, testing, and verifying autonomous CPS, we propose a new \textbf{neurosymbolic framework}, \textbf{NeuroStrata}, tailored to the unique requirements of such systems. As shown in Figure \ref{fig:vision}, NeuroStrata combines neural adaptability with symbolic reasoning to enforce formal specifications across hierarchical DSLs that capture underlying safety and liveness properties. The framework structures \textit{Perception} and \textit{Planning \& Control} capabilities into high-level (symbolic-only) and middle- and low-level (neurosymbolic) modules. It ensures runtime reliability and adaptation via a two-phase process: \textit{top-down synthesis}, propagating symbolic specifications to neurosymbolic modules, and \textit{bottom-up adaptation}, where neurosymbolic outputs refine symbolic programs.


% \begin{wrapfigure}[16]{r}{0.75\textwidth}
\begin{figure*}
\centering
% \vspace{-4mm}
\includegraphics[width = 0.75\textwidth]
% \includegraphics[width = \columnwidth]
{figs/FSE25-NeuroStrata.png}
\caption{Proposed Vision for NeuroStrata: Hierarchical Neurosymbolic Programming for Autonomous CPS}
\label{fig:vision}
% \end{wrapfigure}
\end{figure*}

%Version 4: 9th Jan
%To address the challenges of testing and verifying autonomous CPS, we propose a new \textbf{neurosymbolic framework}, \textbf{NeuroStrata}, tailored to the unique requirements of such systems. As shown in Figure \ref{fig:vision}, NeuroStrata integrates scalable frameworks like Scallop~\cite{li2023scallop} with hierarchical DSLs to ensure formal specification enforcement and adaptability at all levels. This framework applies to both \textit{Perception} (context-awareness) and \textit{Planning \& Control}, employing a two-phase process: \textit{top-down synthesis (design-time)} and \textit{bottom-up adaptation (runtime)}.


\textbf{Modules.} 
%At design time, \textit{Spec Mining}, which is built on top of existing neurosymbolic distillation \cite{
%singireddy2023automaton,blazek2024automated,abir2024neuro},to extract formal safety and liveness specifications from training datasets. 
At design time, \textit{Specification Mining}, built on neurosymbolic distillation \cite{singireddy2023automaton,blazek2024automated,abir2024neuro}, extracts formal safety and liveness specifications from training datasets.
To cover more diverse safety and liveness violations and out-of-distribution scenarios beyond existing training data, we leverage recent work using large language models to analyze multi-modal sensor data \cite{zheng2024testing,deng2023target}, such as front-facing cameras in vehicles, to generate additional real-world crashes and unusual cases from various angles.
These specifications are propagated hierarchically across the system. In the perception stack, a high-level \textit{Scene Graph} encodes semantic relationships and interactions between objects (e.g., ``pedestrian crossing road''), represented as differentiable, adaptable programs that can be verified using formal tools like theorem provers. The middle-level \textit{Semantic Map} encodes spatial and semantic information such as road layouts and drivable areas, ensuring consistency with the scene graph via symbolic rules. The low-level \textit{Sensor Fusion and Signal Processing} integrates multi-modal sensor data (e.g., LiDAR, cameras, GPS) while enforcing constraints on accuracy and consistency, leveraging neurosymbolic reasoning for fusion and processing. Similarly, the planning and control stack follows a hierarchical structure. The high-level \textit{Global/Mission Planner} synthesizes deterministic programs to achieve overall system objectives, verified with formal methods such as theorem proving. The middle-level \textit{Local Planner} generates short-term trajectories that align with global plans while adapting to local changes, guided by symbolic reasoning. The low-level \textit{Actuation Control} converts trajectories into control commands (e.g., steering angle, throttle) and ensures compliance with constraints using runtime verification techniques.

\textbf{Specifications.} 
For \textit{perception}, high-level specifications govern system-wide context awareness, such as ensuring that pedestrians and vehicles do not spatially overlap in the scene graph or that all objects adhere to semantic relationships. Middle-level specifications enforce localized consistency, such as aligning lane boundaries with the semantic map and ensuring that detected objects are positioned correctly within the road layout. Low-level specifications address operational constraints, such as maintaining sensor fusion accuracy within a 0.1-meter error margin and ensuring consistent integration of multi-modal sensor data. For \textit{planning/control}, high-level specifications ensure system-wide safety and mission compliance, such as requiring the vehicle to remain within designated route bounds throughout its journey. Middle-level specifications enforce trajectory-level constraints, such as avoiding obstacles within a 2-meter radius or maintaining smooth transitions between trajectory points. Low-level specifications govern detailed actuation control, such as keeping the steering angle within physical limits and ensuring the stability of throttle and braking in response to control inputs. 
These hierarchical specifications for perception and planning/control ensure an integrated and reliable system design.
%Together, these hierarchical specifications for perception and planning/control enable a robust, consistent, and safe system design.

\textbf{Adaptation.} 
During runtime, NeuroStrata dynamically adapts its perception and planning modules to real-time data while maintaining formal specification compliance. For perception, sensor data flows upward through the hierarchy, where outputs from the low-level sensor fusion are validated against middle-level semantic map constraints, and updates propagate to the high-level scene graph. It evolves dynamically using differentiable program induction, compacting, and adapting specifications as needed. For planning and control, high-level mission planners adjust strategies based on changing conditions, while differentiable and adaptable control programs refine global plans and compact themselves in response to system data. Middle- and low-level components, such as local planners and actuation control, remain guided by symbolic reasoning to ensure safety and alignment with global objectives. This integration enables simultaneously adaptable and formally validated behavior throughout the system.

\textbf{Guarantees.} 
NeuroStrata ensures reliability through a hybrid validation framework. High-level deterministic programs, such as scene graphs and mission planners, are validated using formal verification tools like model checking and theorem proving. Middle- and low-level neurosymbolic components, such as semantic maps and sensor fusion, are guided by symbolic constraints and validated using white-box testing, runtime monitors, and error propagation analysis (e.g., approximate reachability verification~\cite{geng_bridging_2024} and conformance checking~\cite{habeeb_approximate_2024}). Together, this framework bridges the gap between deterministic high-level programs and adaptive, data-driven neuro-components, thus providing formal guarantees across all three levels of the hierarchy.



%Version 3: 8th Jan
%\begin{figure*}[tb!]
%\centering
%\includegraphics[width = 0.8\textwidth]
%{figs/FSE25vision.jpg}
%\caption{Proposed Vision for NeuroStrata: Hierarchical Neurosymbolic Programming for Autonomous CPS}
%\label{fig:vision}
%\end{figure*}
%To address the challenges of testing and verifying autonomous CPS, we propose the development of a new \textbf{neurosymbolic framework}, \textbf{NeuroStrata}, tailored to the unique requirements of such systems. As shown in Figure \ref{fig:vision},  NeuroStrata combines scalable frameworks like Scallop~\cite{li2023scallop} with hierarchical domain-specific languages (DSLs) for perception, prediction, and planning. The proposed language integrates the low-level symbolic reasoning capabilities of Scallop with the high-level hierarchical semantics of DSLs, ensuring that specifications propagate throughout all abstraction levels in the system.

%For the autonomous driving systems case study, high-level system safety and liveness properties, typically associated with the control module, are decomposed into corresponding properties for the ego-vehicle's planning, dynamic object trajectories (prediction), and perception outputs (e.g., lanes, traffic lights, obstacles). NeuroStrata enables this hierarchical transformation, ensuring safety-critical properties are verified from control (high-level) to prediction (middle-level) to perception (low-level).

%This hierarchical structure facilitates monitoring and enforcing specifications at every level. Safety and liveness properties described in the DSL guide program induction and enable the synthesis of concrete control programs. These programs are differentiable and adaptable to real-time environments, allowing dynamic adjustment of the control module based on evolving input and ensuring alignment with high-level DSL-related checkers.

%By leveraging NeuroStrata, limitations of traditional hard-coded, rule-based neurosymbolic systems are addressed through dynamic reconfiguration. For instance:
%- Path planning and prediction modules are trained and guided by DSLs to extract features and map concepts, akin to a concept learner~\cite{mao2019neuro}.
%- Trajectory prediction and planning are transformed into concrete programs that guide neural network training via backpropagation, with middle-level DSL checkers ensuring conformance to trajectory-level specifications.
%- The perception module operates similarly, with low-level DSL checkers validating object detection against symbolic specifications.

%By replacing traditional black-box AI components with modular, hierarchical neurosymbolic counterparts, NeuroStrata provides formal guarantees at each level of abstraction. The benefits include:

%\begin{itemize}
  %  \item \textbf{Perception}: Multi-modal sensor data is processed with symbolic reasoning integrated into neural networks. The low-level DSL validates fused data streams through logic-based consistency checks, ensuring reliable object detection that adheres to system specifications.
%    \item \textbf{Prediction and Planning}: Trajectory generation and motion planning use neurosymbolic paradigms validated by DSL checkers, guaranteeing safety and compliance with system requirements.
  %  \item \textbf{Control}: Deterministic yet adaptive, differentiable control programs are generated through program induction, leveraging the symbolic language to adhere to high-level safety and liveness properties.
%\end{itemize}

%NeuroStrata’s hierarchical design with compositional interfaces enables modular verification at each level while preserving system-wide guarantees. Furthermore, its integration of neurosymbolic reasoning allows the \textbf{automatic generation of high-level DSLs} as shown as the \textbf{Spec Mining} in Figure \ref{fig:vision}, overcoming scalability challenges faced by current methods like DreamCoder~\cite{ellis2021dreamcoder} or syntax-guided synthesis~\cite{bjorner2023formal}, which often rely on predefined DSLs and E-Graphs.

%NeuroStrata bridges the gap between stochastic AI components and deterministic safety-critical requirements in autonomous systems. For verification, we propose a hybrid framework combining traditional software verification techniques with symbolic reasoning as shown as \textbf{Neuro-Verifier} in Figure \ref{fig:vision}. This framework includes:
%- Formal methods (e.g., model checking, theorem proving and runtime verification) to validate symbolic layers like knowledge graphs and logical rules.
%- Error propagation analysis to assess the impact of neural network errors on symbolic reasoning, identifying failure points and ensuring robustness.

%Together, NeuroStrata enables the creation of safe, adaptable, and verifiable autonomous CPS that can handle real-world uncertainties while maintaining strict safety standards.




%Version 2: 26th Dec
%To address the challenges of testing and verifying autonomous CPS, we propose the development of a new \textbf{neurosymbolic language} tailored to the unique requirements of such systems. This language combines scalable frameworks like Scallop~\cite{li2023scallop} with domain-specific languages (DSLs) for tasks such as perception, prediction, and planning. The proposed neurosymbolic language integrates the low-logic reasoning capabilities of Scallop with the high-level hierarchical semantics of DSLs. For the autonomous driving systems case study, the high-level system safety and liveness properties, typically associated with the output of control modules, are decomposed into corresponding properties for the ego-vehicle's planning, dynamic object trajectories (prediction), and perception outputs (e.g., lane, traffic light, obstacles). This enables a hierarchical transformation of AI-enabled autonomous systems—from control (high-level) to prediction (middle-level) to perception (low-level).

%This hierarchical structure facilitates the monitoring of high-level specifications throughout the system and ensures safety and liveness properties are adhered to at each level. Safety and liveness descriptions in the DSL help monitor the input and output of the control module, guiding program induction and synthesizing concrete control programs. These control programs are differentiable and adaptable to real-time environments, making the control module incremental by continuously monitoring inputs and outputs, adapting to unknown environments, and aligning the evolving control program with high-level DSL-related checkers to ensure specification conformance.

%Such adaptable control modules solve the limitations of hard-coded, rule-based neurosymbolic systems by enabling dynamic reconfiguration. For instance, path planning and prediction modules are trained and guided by the DSL for feature extraction and concept mapping, similar to the concept learner~\cite{mao2019neuro}. In contrast, trajectory prediction and planning are transformed into concrete programs that guide neural network training (backpropagation), with the DSL-associated checker (middle-level) ensuring conformance to specifications at the trajectory level. Similarly, the perception module is constructed in a similar fashion, with a low-level DSL-associated checker operating at the object detection level. More specifcially, by leveraging the neurosymbolic paradigm, we can replace traditional black-box AI components in perception, prediction, planning, and control with modular, hierarchical alternatives that provide formal guarantees. These neurosymbolic components enable:

%\begin{itemize}
   % \item \textbf{Perception}: Symbolic reasoning is integrated with neural networks to produce interpretable decision-making for multi-modal sensor data, ensuring alignment with specifications defined in the DSL. The low-level DSL will define the relationships and dependencies between sensors, validate fused data streams through consistency checks and logic-based validation, and ensure the correct integration of sensor data for reliable object detection.
  %  \item \textbf{Prediction and Planning}: Neurosymbolic paradigms are employed for trajectory generation and motion planning, with formal guarantees validated by corresponding DSL checkers to ensure compliance with system specifications.
  %  \item \textbf{Control}: Outputs from perception, prediction, and planning modules are used to generate deterministic control programs via program induction, leveraging the proposed symbolic language to ensure adherence to system-level safety and liveness properties.
%\end{itemize}

%This hierarchical design with clear compositional interfaces facilitates modular verification at each level, ensuring system-wide guarantees are preserved. Moreover, the formal specifications extracted using neurosymbolic reasoning can \textbf{automatically generate high-level DSLs}, overcoming limitations in current methods. For example, existing techniques like DreamCoder~\cite{ellis2021dreamcoder} or syntax-guided synthesis~\cite{bjorner2023formal} rely heavily on predefined DSLs to constrain program induction. These methods are often overly simplistic or face scalability challenges due to their reliance on E-Graphs. By integrating neurosymbolic reasoning with Scallop, we can construct robust, scalable DSLs tailored to autonomous CPS, enabling more effective program induction and systematic verification. This integration bridges the gap between stochastic AI components and the deterministic requirements of safety-critical autonomous systems.

%As for the verification of neurosymbolic components challenges, we propose a hybrid verification framework that combines traditional software verification techniques with symbolic reasoning. This approach will leverage formal methods such as model checking and theorem proving to ensure the correctness of the symbolic layers, including knowledge graphs and logical rules. Additionally, we will implement error propagation analysis to trace how errors in neural networks impact the symbolic reasoning process, identifying failure points and ensuring that both the AI and symbolic components of the system function as expected. This framework will provide a more comprehensive and interpretable way to verify the safety and reliability of neurosymbolic autonomous systems. Together, these advancements pave the way for safe, adaptable, and verifiable autonomous CPS that can handle real-world uncertainties while maintaining strict safety standards.


%Version 1: 25th Dec
%To address the challenges of testing and verifying autonomous CPS, we propose the development of a \textbf{new neurosymbolic language} tailored to the unique requirements of such systems. This language combines scalable frameworks like Scallop~\cite{liang2021scallop} with domain-specific languages (DSLs) for tasks such as perception, prediction, and planning. A DSL can abstract domain-specific concepts like traffic rules, map constraints, and vehicle behavior into intuitive high-level constructs, while Scallop, as a declarative reasoning framework, provides probabilistic reasoning to handle complex, multi-modal data. For example, a DSL inspired by OpenScenario or Scenic could define traffic laws and road conditions, while Scallop reasons about these rules to adapt to unseen ``long-tail'' scenarios caused by covariate shifts (e.g., input distributions differing from training datasets). This integration enables reasoning over stochastic components, ensuring adaptability and interpretability by symbolically representing and reasoning about the uncertainties inherent in such systems.

%Building on this foundation, we envision a framework that integrates neurosymbolic paradigms across critical modules of autonomous CPS, enabling systematic and hierarchical improvements in \textbf{testability} and \textbf{verification}:

%First, we aim to \textbf{extract formal specifications} from data-driven components using neurosymbolic reasoning. Often, the difficulty lies in defining accurate specifications, which can have a more significant impact than implementation errors. Symbolic reasoning can capture system invariants and constraints from training data or domain knowledge, providing a formal basis for rigorous testing of AI components such as perception and planning.

%Second, we propose replacing traditional black-box AI components in perception, prediction, planning, and control with neurosymbolic alternatives that are both hierarchical and compositional. 
%\begin{itemize}
 %   \item \textbf{Perception}: Symbolic reasoning is integrated with neural networks to enable interpretable decision-making for multi-modal sensor data, ensuring alignment with underlying specifications defined in the DSL.
 %   \item \textbf{Prediction and Planning}: Neurosymbolic paradigms are employed for trajectory generation and motion planning, with built-in formal guarantees. These components are guided by the proposed DSL and validated by a corresponding checker to align decisions with specifications.
  %  \item \textbf{Control}: Outputs from the perception, prediction, and planning modules are leveraged to generate deterministic control programs through program induction using the proposed symbolic language. This ensures adherence to system-level safety and liveness properties.
%\end{itemize}
%By designing these components hierarchically, with clear compositional interfaces, we facilitate modular verification at each level while preserving system-wide guarantees.

%Moreover, the \textbf{formal specifications extracted} using neurosymbolic paradigms can \textbf{automatically generate high-level DSLs}, overcoming limitations in current approaches. For example, existing methods like DreamCoder~\cite{ellis2021dreamcoder} or syntax-guided synthesis~\cite{gao2022formal} rely heavily on predefined DSLs to constrain program induction. However, these approaches are often too simplistic or face scalability issues due to their underlying structures (e.g., E-Graphs). By integrating neurosymbolic reasoning with Scallop, we can construct robust and scalable DSLs tailored to autonomous CPS, enabling better program induction and systematic verification. This automated DSL generation bridges the gap between stochastic AI components and the deterministic requirements of safety-critical autonomous systems.Together, these advancements pave the way for safe, adaptable, and verifiable autonomous CPS, capable of handling real-world uncertainties while maintaining strict safety standards.


%To address the challenges posed by testing and verifying autonomous CPS, we envision leveraging neurosymbolic paradigms to fundamentally enhance the testability and verifiability of these systems. Neurosymbolic approaches combine the statistical power of neural networks with the expressiveness and interpretability of symbolic reasoning. In this vision, we propose a framework that integrates neurosymbolic paradigms across critical modules of autonomous CPS, enabling systematic and hierarchical improvements in testability and verification.

%First, we aim to \textbf{extract formal specifications} from data-driven components using neurosymbolic reasoning. Often in practice, the issue is that it is unclear what the specification for a component or task should be, and the errors in specifications can be more impactful than implementation issues. This approach enables the creation of precise, interpretable specifications that can facilitate model-based testing. For example, symbolic reasoning can be applied to capture system invariants and constraints derived from training data or domain knowledge, providing a formal basis for rigorous testing of AI components such as perception and planning.

%Second, we propose replacing key components of perception, prediction, planning, and control with \textbf{compositional and hierarchical neurosymbolic alternatives}. For instance, the perception module, traditionally dominated by black-box neural networks, can incorporate symbolic reasoning to handle multi-modal sensor data with interpretable decision-making. Similarly, the prediction and planning modules can utilize neurosymbolic paradigms to enable trajectory generation and motion planning with built-in formal guarantees. The control module can then leverage these neurosymbolic outputs to ensure actuation commands adhere to system-level safety and liveness properties. By designing these components hierarchically, with clear compositional interfaces, we enable modular verification at each level while preserving system-wide guarantees.


%To achieve the above vision, we advocate for the development of a \textbf{new neurosymbolic language} tailored to autonomous CPS. Building on the foundation of scalable frameworks like Scallop~\cite{liang2021scallop}. More specifically, we would like to combine Scallop with high-level Domain specific language (
%DSL) for autnomous CPS. In the context of autonomous driving systems, combining Scallop with a DSL tailored for tasks such as perception, prediction, and planning holds significant promise. A DSL can abstract domain-specific concepts like traffic rules, map constraints, and vehicle behavior into intuitive high-level constructs, while Scallop, as a declarative reasoning framework, provides probabilistic reasoning capabilities to handle complex, multi-modal data. For instance, a DSL based on OpenScenario or Scenic could define traffic laws or specific road conditions, while Scallop reasons about these rules to adapt to unseen ``long-tail'' scenarios caused by covariate shifts (e.g., input distributions differing from training datasets). This integration enables reasoning over stochastic components, such as object detection and prediction modules, to ensure adaptability and interpretability by symbolically representing and reasoning about the uncertainties inherent in such systems. 

%Once again, the formal specifications extracted using neurosymbolic paradigms can automatically generate the corresponding high-level DSL, addressing a key limitation in current approaches. For example, existing approaches such as DreamCoder~\cite{ellis2021dreamcoder}, which uses a wake-sleep library learning framework and E-Graph, and syntax-guided synthesis approaches for formal explainable AI (XAI)~\cite{gao2022formal}, rely on DSLs to constrain the program induction space. However, these approaches are either too simplistic, as in the XAI paper, or face scalability issues, as the underlying structures (e.g., E-Graphs) are not well-suited for handling complex, domain-specific requirements. To overcome these challenges, a neurosymbolic framework integrated with Scallop can enable the automatic construction of domain-specific languages that are robust, expressive, and scalable. This automatic generation of DSLs would allow for better constraints in program induction, enhancing the ability of autonomous driving systems to handle dynamic environments, ensure safety, and support systematic verification. Together, these advancements can bridge the gap between stochastic AI components and the deterministic requirements of safety-critical autonomous CPS. 



%In the context of autonomous driving systems, combining Scallop with a DSL tailored for tasks such as perception, prediction, and planning holds significant promise. A DSL can abstract domain-specific concepts like traffic rules, map constraints, and vehicle behavior into intuitive high-level constructs, while Scallop, as a declarative reasoning framework, provides probabilistic reasoning capabilities to handle complex, multi-modal data. For instance, a DSL based on OpenScenario or Scenic\cite{} \tocheckJZ{my google citations capabilities are constrained in China} could define traffic laws or specific road conditions, while Scallop reasons about these rules to adapt to unseen ``long-tail'' scenarios caused by covariate shifts (e.g., input distributions differing from training datasets). This integration enables reasoning over stochastic components, such as object detection and prediction modules, to ensure adaptability and interpretability by symbolically representing and reasoning about the uncertainties inherent in such systems. Moreover, Scallop’s explainable and modular reasoning engine can address scalability issues by processing these rules hierarchically and supporting efficient decision-making in dynamic real-world environments. Together, this combination can bridge the gap between current stochastic AI models and the deterministic requirements of safety-critical autonomous CPS, offering a pathway for robust, verifiable, and interpretable decision-making in autonomous driving.


%the proposed language would support domain-specific reasoning and learning, enabling efficient integration of neural and symbolic computations across the system. This language would allow for the co-design of neurosymbolic components, ensuring seamless interaction and consistency between modules while facilitating rigorous testing and verification.

%This vision establishes a roadmap for leveraging neurosymbolic paradigms to not only bridge the gap between data-driven learning and formal reasoning but also to create a systematic approach for the design, testing, and verification of autonomous CPS. By integrating these advancements, we aim to address the current limitations of testing and verification efforts, paving the way for safer and more reliable autonomous systems.\tocheckJZ{To Ruzica, Ivan and Ziyang, and the rest: I have outlined a high-level design here, which should be sufficient for drafting the initial results in the next section. Please feel free to review and suggest updates if you notice anything important missing. I plan to add more specific details next week.}
\aftersectionskip

\sectionspacing
\section{EARLY RESULTS AND FUTURE PLAN}
\label{sec:earlyresults}
% \tocheckJZ{ZiYang, the SE community would greatly appreciate seeing some earlier results demonstrating how the neurosymbolic paradigm can improve the testability and verifiability of autonomous CPS. Now that the design is clearer, could you provide a quick summary of your LASER work? Specifically, where you used LLMs to extract a customized spatial-temporal specification language from captions, and applied this to weakly supervise the generation of spatial-temporal scene graphs? This could effectively address the question of how neurosymbolic-extracted DSLs can support training scene graphs that align with formal specifications. RQ1 is designed for this purpose, while I will explore answers for RQ2.}

We conducted a preliminary case study to investigate a key \textbf{Research Question (RQ)}: ``can neurosymbolic reasoning complement neural-network training to align with underlying specifications?''. We also outline our future plans, along with the potential challenges and proposed solutions. %detail our future plans along with potential challenges and solutions. %Similarly, an initial exploration was performed for \textbf{Research Question 2 (RQ2)}: ``can program induction enable effective neurosymbolic reasoning?''
%Our preliminary findings are promising, showing how a neurosymbolic approach not only enhances interpretability but also ensures alignment with formal specifications, particularly for tasks such as scene graph generation and spatial-temporal reasoning. These results align well with our roadmap and provide a strong foundation for future exploration.


\subsection{Assessing neurosymbolic reasoning to align neural network training with specifications}
\label{sec:symbolicReasoning}

In this study, we investigate the capability of differentiable neurosymbolic reasoning to align perceptual neural networks with specifications.
We explore the application of a high-level visual perception system trained by aligning its output with specifications.
In this application, the goal is to infer spatio-temporal scene graphs (STSG) from videos (e.g., ones taken by ego-centric cameras), where the scene graphs must align with a given spatio-temporal specification.
Figure~\ref{fig:laser-illus} depicts a specification for a traffic scene which is described in natural language but then formalized into a temporal logic formula.
Notice that the specification consists of logical symbols like \textit{exists} ($\exists$), \textit{and} ($\wedge$), \textit{not} ($\neg$), and \textit{finally} ($\Diamond$).
A neurosymbolic approach to solving this task comprises of a neural model for STSG extraction and a differentiable symbolic component for aligning the predicted STSG with the given specification.
Being differentiable, the loss computed from the alignment process can be used to supervise the neural model.
We evaluated our work on three datasets: OpenPVSG \cite{yang2023panoptic}, 20BN-Something-Something \cite{goyal2017something}, and MUGEN \cite{hayes2022mugen}, each with diverse temporal properties. Our approach outperforms current baselines on downstream tasks while offering explainability\footnote{Reference to the full report was anonymized for the review process.}.
This specification alignment provides high confidence in our top-down synthesis approach and in guiding neural training with our specifications.
%Preliminary results have shown that the approach achieve superior data-efficiency and preciseness, as the approach can leverage widely-available weak labels for videos \cite{huang2024laser}.
%Utilizing the neurosymbolic paradigm, we may eliminate the reliance on low-level annotations which might be noisy and biased.
%Instead, high-level, verifiable, and transparent specifications may be used to train the underlying perception models.

\begin{figure}
    \centering
    \includegraphics[width=\linewidth]{figs/laser-illus.pdf}
    \caption{Aligning STSG with natural language description via temporal logic specifications.}
    \label{fig:laser-illus}
\end{figure}

%Leveraging ChatGpt4.0, we interpreted a Texas traffic rule handbook, creating a DSL (Fig.~\ref{fig:schema}) for transforming these rules into traffic scenarios. This DSL stands out by emphasizing semantic descriptions over exact coordinates, setting it apart from other DSLs like OpenScenario and Scenic~\cite{openScenario, geoscenario, Scenic}.
%Using ChatGpt4.0, we interpreted the Texas traffic rule handbook to create a DSL (Fig.~\ref{fig:schema}) that transforms these rules into traffic scenarios. This DSL, focusing on semantic descriptions rather than precise coordinates, differentiates from others such as OpenScenario and Scenic~\cite{openScenario, geoscenario, Scenic}. %R3 comment about hallucination
%{\color{blue}
%We implemented multi-level validation to ensure the correctness of the DSL specifications, mitigating hallucination issues.
 %}
% We then translated these DSL specifications into test scripts for the CARLA simulation platform~\cite{dosovitskiy2017carla}, uncovering significant bugs in autonomous driving systems. The DSL comprises elements like \textit{Environment}, \textit{Road network}, \textit{Actor}, and \textit{Oracle}, each capturing intricate scenario semantics. This approach enabled us to generate diverse, semantically rich test scenarios that revealed rule violations in real-world autonomous systems, aiding developers in pinpointing specific issues. In our experiments, we observed that the MMFN model~\cite{zhang2022mmfn} failed to stop at the stop sign, Autoware~\cite{kato2018autoware} collided with a front vehicle, and LAV~\cite{chen2022lav} did not respond to a pedestrian crossing the road. We reported these issues, along with detailed log data, to the developers of these widely-used autonomous driving systems. They utilised our logs to pinpoint the root causes of these rule violations.
%{\color{blue}
%Further details about the DSL, including insights, examples, and the methodology for translating DSLs into test scripts, along with replicable artifacts, are provided in~\cite{deng2023target}.
%} %R2 comment about artifacts and more details of test scripts generation
%Further insights and examples are detailed in Figure~\ref{fig:schema} and supplementary material~\footnote{Supplementary material: \url{https://shorturl.at/gFPX3}}.


%Utilizing ChatGpt4.0, we parsed a Texas traffic rule handbook to extract human knowledge through specialized prompts, crafting a DSL for translating rules into traffic scenarios. This DSL, unique in its abstraction from concrete parameters, focuses on semantic descriptions based on traffic rule handbooks, distinguishing it from DSLs like OpenScenario and Scenic~\cite{openScenario, geoscenario, Scenic}. The synthesized test scripts from the rule DSL representation through template-based synthesis for the CARLA simulation platform~\cite{dosovitskiy2017carla} revealed significant bugs in autonomous driving systems. The DSL's components: \textit{Environment}, \textit{Road network}, \textit{Actor}, and \textit{Oracle}, encapsulate complex semantics of traffic scenarios, supported by a rich set of elements drawn from various sources including the traffic rule handbook~\cite{texasDriverHandbook} and OpenXOntology~\cite{openxontology}. This hierarchical structure allows the creation of diverse test scenarios with rich semantic details, avoiding reliance on overly concrete parameters. The
%generated scenarios are able to reveal rule violations for a few real-world autonomous driving systems and produced detailed execution logs are also able to help developers to locate the exact issues leading to rule violations.
%Further details and examples are showcased in Figure~\ref{fig:schema} and supplementary material~\footnote{Supplementary material: \url{https://shorturl.at/gFPX3}}.

%We leveraged ChatGpt4.0 to parse a Texas traffic rule handbook, crafting specialized prompts to guide the Language Model in extracting human knowledge embedded within these rules. We devised a domain-specific language (DSL) that melds syntactic simplicity with semantic richness, facilitating the Language Model in translating the original rules into DSL-specified traffic scenarios. We designed specific prompt template with ChatGpt4.0, propelling the Language Model to embody a test expert role for autonomous driving systems. We elucidated the DSL rules, illustrated examples, and engaged the Language Model to generate traffic scenarios in DSL format. Through a template-based program synthesis approach, we synthesized test scripts for the CARLA simulation platform~\cite{dosovitskiy2017carla}, unveiling intriguing violations acknowledged as genuine bugs by autonomous system developers. %Delve into Figure~\ref{fig:all scenarios} and \cite{deng2023target} for an in-depth exploration.

%This DSL stands apart from existing DSLs such as OpenScenario and Scenic~\cite{openScenario, geoscenario, Scenic} due to its abstraction from concrete parameters like exact coordinates of entities, instead focusing on a higher-level semantic description derived from traffic rule handbooks. The DSL, depicted in a context-free grammar format in Figure~\ref{fig:schema}, encompasses four principal components: \textbf{\textit{Environment}}, illustrating the temporal and weather conditions; \textbf{\textit{Road network}}, portraying the geographical and infrastructural context; \textbf{\textit{Actor}}, describing the dynamic entities involved and their respective states; and \textbf{\textit{Oracle}}, governing the ego vehicle's appropriate behaviors.

%Each of these components is further subdivided to encapsulate specific semantics, with a rich set of optional elements defined for each subcomponent, drawing from sources like the traffic rule handbook~\cite{texasDriverHandbook} and OpenXOntology~\cite{openxontology}. For instance, optional elements under \textit{road type} encapsulate typical road configurations like intersections and roundabouts. This hierarchical structure, demonstrated in Figure~\ref{fig:schema}, and the extensive list of optional elements available in the supplementary material~\footnote{Supplementary material: \url{https://shorturl.at/gFPX3}}, furnish a robust framework capable of capturing a wide spectrum of semantic details in test scenarios derived from traffic rules, thereby enriching the capability to generate meaningful test scenarios without delving into the nitty-gritty details of exact coordinates or overly concrete parameters.


%In our recent study~\cite{deng2023target}, we employed ChatGpt to parse a Texas traffic rule handbook, using specially crafted prompts to guide the LLM in extracting human knowledge from these rules. We created a domain-specific language (DSL) that's syntactically simple yet semantically rich, aiding the LLM in generating DSL-specified traffic scenarios from the original rules. Table~\ref{tab:prompt_knowledge_extraction} illustrates our initial prompt, urging the LLM to act as a test expert for autonomous driving systems. We introduced the DSL rules, demonstrated examples, and had the LLM generate traffic scenarios in DSL format. Through template-based program synthesis, we created test scripts for the CARLA simulation platform~\cite{dosovitskiy2017carla}, uncovering intriguing violations confirmed as real bugs by autonomous system developers. See Figure~\ref{fig:all scenarios} and \cite{deng2023target} for more details.


%In our recent work~\cite{deng2023target}, we utilized ChatGpt to parse an entire traffic rule handbook from Texas. We used specifcialized designed prompt to guide the LLM to extract human knowledge from those traffic rules. We also design a domain-specific-language which is syntactically simple and semantically rich enough for wide range of traffic rules. The LLM is guided to generate DSL-specifcied traffic scenario from the original rule description. In Table~\ref{tab:prompt_knowledge_extraction}, we showed the first-stage prompt where we ask the LLM to take a role as a test expert for autonomous driving systems. We first explain the DSL rules and shows some demostrations.  Then we ask the LLM to directly generate the traffic scenario in our DSL format. We then use some template-based program synthesis approach to generate test scripts for the target simulation platform (CARLA~\cite{dosovitskiy2017carla}}. After running the test scripts, we are able to generate the corresponding driving scenarios for each input rule and find some interesting violations which later confirmed by the corresponding autonomous driving systems developers as real bugs. Some of the examples are shown in
%Figure~\ref{fig:all scenarios} and we refer readers to find more details in~\cite{deng2023target}.


%\begin{table}[h!]
%\begin{tabular}{p{\columnwidth}}
%\hline
%\rowcolor[HTML]{C0C0C0}
%\textbf{Role Setting}                                                                                                                               \\
%\begin{tabular}[c]{@{}l@{}}You are a test expert for autonomous driving systems. Your task is \\to generate specific test  scenario representations from given traffic \\rules.\end{tabular}                                                                                  \\
%\rowcolor[HTML]{C0C0C0}
%\textbf{Prompt}                                                                                                                            \\
%\begin{tabular}[c]{@{}l@{}}Below is the definition of a domain-specific language to represent \\test scenarios for autonomous driving systems:\\ \{\textit{Details of DSL}\}\\ \\ Below are the lists of commonly used elements for each \\subcomponent. When creating the scenario representation, consider \\the following elements first for each subcomponent. If no element can \\ describe the close meaning, create a new element by yourself.\\ \{\textit{Details of element lists}\}\\ \\ Below is an example of an input traffic rule and the corresponding \\scenario representation:\\ Traffic rule: \{\textit{content of the traffic rule}\}\\ Scenario representation: \{\textit{content of the scenario representation}\} \\ \\ Based on the above descriptions and examples, convert the following \\traffic rule to corresponding \\scenario representation: \\ \{\textit{input traffic rule}\} \end{tabular} \\ \hline
%\end{tabular}
%\caption{The prompt template for knowledge extraction}
%\label{tab:prompt_knowledge_extraction}
%\end{table}

%\begin{figure*}


%centering
%\subfloat[Traffic rule violation (Auto model~\cite{carla_auto}): Not Give way to the other vehicle in roundabout]{\includegraphics[height=0.16\textwidth,width=.3\textwidth]{figs/roundabout.jpg}\label{fig:violation_roundabout}}\hfil%
%\subfloat[Traffic rule violation (Auto model~\cite{carla_auto}): Not give way to the other vehicle at an intersection]
%{\includegraphics[height=0.16\textwidth,width=.3\textwidth]{figs/collision.jpg}\label{fig:violation_intersection}}\hfil
%\subfloat[Traffic rule violation (MMFN model~\cite{zhang2022mmfn}): Not stop before the stop sign]
%{\includegraphics[height=0.16\textwidth,width=.3\textwidth]{figs/no_stop_before_sign.jpg}\label{fig:violation_stop}}\hfil
%\subfloat[MMFN model~\cite{zhang2022mmfn} collides on the roadside]{\includegraphics[height=0.16\textwidth,width=.3\textwidth]{figs/mmfn_problem.jpg}\label{fig:violation_mmfn}}\hfil
%\subfloat[Autoware~\cite{kato2018autoware} collides with the front vehicle]{\includegraphics[height=0.16\textwidth,width=.3\textwidth]{figs/autoware_collison.jpg}\label{fig:violation_autoware}}\hfil
%\subfloat[LAV~\cite{chen2022lav} does not move even if the pedestrian has crossed the road]{\includegraphics[height=0.16\textwidth,width=.3\textwidth]{figs/1087.jpg}\label{fig:violation_person}}\hfil

% \vspace{-1em}
%\caption{Examples of detected problems on ADSs and Carla simulator}\label{figure}
% \vspace{-1em}

%\label{fig:all scenarios}
%\end{figure*}

%In one of our current work, we take cues from the prevalent use of front-facing cameras that capture real-world
%traffic accidents for autonomous vehicles. We propose the use of existing video data, chronicling
%various failures in autonomous driving, to train a video-language multi-modal large language model
%(LLM). The purpose is twofold: first, to generate a vast repository of realistic, high-quality data; and second, to extract formal specifications underlying these crash scenarios. Leveraging this enriched data, we aim to apply data-driven learning methodologies to distill these formal specifications. The language model itself can also be programmed to generate these formal conditions directly. This provides the foundation for implementing model-based testing, an approach far more formalized than existing strategies like search-based testing.

\subsection{Future Research Plan}
To advance \textsc{NeuroStrata}, we propose a \textit{six-step future plan} with concrete steps to address challenges at each stage.

\looseness=-1
First, \textit{generating diverse training datasets} will leverage recent advancements in model-based testing that utilize LLMs to analyze multi-modal sensor data~\cite{zheng2024testing,deng2023target}. The multi-faceted challenge lies in ensuring the generated datasets are diverse, representative of real-world scenarios, and capable of addressing edge cases. Solutions can be tailored around prior work by accessing diverse sensor datasets and logs, leveraging advanced multi-modal LLMs, and integrating domain-specific constraints with iterative refinement based on industrial partner feedback.

Second, \textit{designing suitable DSLs} is essential for capturing hierarchical and semantically rich specifications. These DSLs enable experts to encode operational constraints for sensor fusion, signal processing, and physical actuation control. Challenges include ensuring the DSLs are intuitive for domain experts while expressive enough to handle complex requirements. Solutions involve co-designing DSLs with autonomy and robotics specialists, developing language automation, and designing usable visual interfaces. By providing a bridge between formal methods and practical application, these DSLs empower experts to play an active role in system design.

Third, \textit{developing a specification mining module} based on neurosymbolic distillation will extract formal safety and liveness specifications from training datasets and LLM interactions. Aligning mined specifications with real-world requirements and handling noisy/hallucinated data are key challenges. Hybrid approaches that combine symbolic reasoning with neural embeddings, as well as active learning techniques, can iteratively refine the mined specifications to ensure accuracy and physical grounding.

%precision.

%Fourth, at design time, \textsc{NeuroStrata} will \textit{synthesize multi-level specifications} for perception and planning/control modules, perform top-down synthesis of symbolic and neurosymbolic components, and validate these components using formal verification techniques. The challenge of scalability for systems with diverse scenarios and high-dimensional inputs can be addressed through modular architectures that compartmentalize functionalities and parallelized verification methods to reduce computational costs. Adaptive abstraction techniques can further focus verification efforts on critical system components, ensuring scalability without sacrificing precision.

%Fourth, at design time, \textsc{NeuroStrata} will \textit{synthesize multi-level specifications} for perception and planning/control modules, perform top-down synthesis of symbolic and neurosymbolic components, and validate them using formal verification. Scalability challenges with diverse scenarios and high-dimensional inputs can be addressed through modular architectures, parallelized verification, and adaptive abstraction techniques to focus on critical components without compromising

%Fifth, during runtime adaptation and validation, \textit{symbolic programs will be dynamically refined} to adapt to real-world changes while maintaining compliance with specifications. Prior work, such as DreamCoder~\cite{ellis2021dreamcoder}, demonstrates the feasibility of program induction through iterative refinement. Challenges include maintaining computational efficiency for real-time adaptation and ensuring formal guarantees. These challenges can be addressed by optimizing runtime validators, incorporating lightweight symbolic reasoning techniques, and employing just-in-time verification to minimize overhead while ensuring reliability.
%Fourth, for design-time synthesis and verification, we will focus on enforcing \textit{multi-level specifications} and synthesis for perception and planning/control modules. This involves defining a hierarchical structure of specifications, starting from sensor data processing constraints to high-level mission planning objectives. We plan to use modular architectures to separate functionalities, enabling scalable synthesis of symbolic and neurosymbolic components through a top-down approach. Formal verification will validate these components against specifications, with parallelized processes and adaptive abstraction techniques addressing scalability issues. These steps ensure the framework is robust across diverse scenarios and high-dimensional inputs.
%Fourth, for design-time synthesis and verification, we will enforce \textit{multi-level specifications} for perception and planning/control modules. This entails enforcing the hierarchical structure of specifications defined in our DSL, ranging from low-level sensor data processing constraints to high-level mission planning objectives. Modular architectures will be employed to separate functionalities, enabling scalable top-down synthesis of symbolic and neurosymbolic components. Formal verification will ensure these components adhere to specifications, while parallelized processes and adaptive abstraction techniques will address scalability challenges, ensuring robustness across diverse scenarios and high-dimensional inputs.
%Fourth, for design-time synthesis and verification, we will enforce \textit{multi-level specifications} for perception and planning/control modules. This involves utilizing the hierarchical structure of specifications defined in our DSL, spanning from low-level sensor data processing constraints to high-level mission planning objectives. Modular architectures will be used to separate functionalities, enabling scalable top-down synthesis of symbolic and neurosymbolic components. Formal verification will ensure compliance with specifications, while parallelized processes and adaptive abstraction techniques will address scalability challenges, ensuring robustness across diverse scenarios and high-dimensional inputs.
Fourth, for design-time synthesis and verification, we will enforce \textit{multi-level specifications} for perception and planning/control modules, leveraging the hierarchical structure  defined in our DSL. Modular architectures will enable scalable top-down synthesis of symbolic and neurosymbolic components, while formal verification ensures compliance with specifications. Parallelized processes and adaptive abstraction techniques will address scalability challenges, ensuring robustness across diverse scenarios and high-dimensional inputs.

Fifth, for runtime adaptation and validation, we will develop mechanisms to \textit{dynamically refine symbolic programs} for real-world changes while ensuring specification compliance. Inspired by program induction approaches like DreamCoder~\cite{ellis2021dreamcoder}, \textsc{NeuroStrata} will iteratively refine symbolic representations using real-time data. Challenges include maintaining computational efficiency and real-time guarantees. To address these, we will optimize runtime validators, integrate lightweight symbolic reasoning for faster adaptation, and implement efficient runtime verification to ensure reliability and compliance with minimal overhead. These advancements will enable \textsc{NeuroStrata} to adapt to dynamic environments and evolving operational conditions.

%Fifth, for runtime adaptation and validation, we aim to develop mechanisms for \textit{dynamically refining symbolic programs} to adapt to real-world changes while ensuring specification compliance. Inspired by program induction approaches such as DreamCoder~\cite{ellis2021dreamcoder}, \textsc{NeuroStrata} will iteratively refine symbolic representations based on real-time data. Challenges include maintaining computational efficiency and ensuring real-time guarantees. To address this, we will optimize runtime validators to efficiently process changes, integrate lightweight symbolic reasoning for faster adaptation, and implement efficient runtime verification to maintain reliability and compliance without introducing significant overhead. These advancements will enable \textsc{NeuroStrata} to handle dynamic environments and evolving operational conditions effectively.


Finally, for \textit{industrial deployment}, \textsc{NeuroStrata} will be applied to autonomous driving systems, delivery drones, cargo drones, and passenger aircraft --- as facilitated by our partners. Key challenges include seamless integration into existing systems, adherence to stringent safety standards, and building trust among stakeholders. Solutions include close collaborative projects, iterative deployment in increasingly open environments, and the creation of comprehensive documentation and training programs to facilitate adoption.

%By addressing these challenges with targeted solutions, \textsc{NeuroStrata} aims to bridge theoretical advancements and practical applications, transforming the testing and verification of autonomous CPS in critical real-world scenarios.

%To advance \textsc{NeuroStrata}, we propose a comprehensive future plan with concrete steps and address the challenges associated with each. First, generating diverse training datasets will leverage LLMs to synthesize edge cases and out-of-distribution scenarios. The main challenge is ensuring dataset relevance and diversity while avoiding overfitting; this can be mitigated by incorporating domain-specific constraints and iterative feedback from industrial partners. Second, developing a specification mining module based on neurosymbolic distillation will enable the extraction of formal safety and liveness specifications. The challenge lies in aligning these specifications with real-world requirements and managing noisy data, which can be addressed through hybrid symbolic-neural approaches and active learning techniques. Third, at design time, we will synthesize multi-level specifications for perception and planning/control, perform top-down synthesis, and validate components using formal methods. Achieving scalability across diverse scenarios remains a challenge but can be tackled by leveraging modular architectures and parallelized verification techniques. Fourth, during runtime adaptation and validation, symbolic programs will be dynamically refined to ensure compliance with observed real-world conditions. Earlier results from DreamCoder \cite{ellis2021dreamcoder}, though not relevant to autonomous systems it shows capability that by looking at some few examples of buildings, the synthesized program can learn the underlying funcitons for building such as arch, bridge and pyramaid, and then in the end through iterative process of combination and compression, create realistic buildings similar to the examples but with more diversiteis,  demonstrate the feasibility of aligning neural outputs with symbolic specifications and adapting programs in real-time to maintain performance under unseen scenarios. The key challenges here are ensuring computational efficiency and maintaining formal guarantees during adaptation, which can be addressed by optimizing runtime validators and using lightweight symbolic reasoning methods. Finally, for industry deployment, \textsc{NeuroStrata} will be applied to autonomous driving systems, delivery drones, cargo drones, and passenger aircraft. Challenges include seamless integration into existing systems, adherence to industry-specific safety standards, and user trust, which can be solved by close collaboration with industry stakeholders and iterative testing in controlled environments. These steps will ensure that \textsc{NeuroStrata} bridges theoretical advancements and practical adoption in critical autonomous CPS applications.

%\begin{figure}
 %   \centering
  %  \includegraphics[width=\linewidth]{figs/DreamCoder-Big.png}
   % \caption{Learning tower building tasks.}
    %\label{fig:dreamcoderexample}
%\end{figure}

%In this study, we eliminated the reliance on specialized DSLs and adopted a learning-to-synthesize paradigm \cite{
%balog2016deepcoder,devlin2017robustfill}.
%This approach trains a neural search policy to efficiently learn a library of reusable components for solving similar problems.
%The learned library is encapsulated in a generative model, which scores candidate programs and samples random tasks to further train the neural search policy.
%The search policy approximates the constraints traditionally defined by DSLs, enabling scalability.
%The developed system can take a collection of synthesized programs and extract a set of components that compactly represent these programs using e-graph matching.
%As shown in the example tasks in the Figure~\ref{fig:dreamcoderexample}, our learned program showcases representative results, demonstrating the ability of our synthesized program to generate an underlying building blocks library from a few provided shape examples.
%Through an iterative incremental learning process involving compaction and filtering, the program can subsequently create buildings that are not only as realistic as the samples but also exhibit greater diversity.
%\tocheckJZ{To Ziyang:As this involves a double-anonymous review process, we are not allowed to reference our existing papers. Instead, we can provide a high-level summary of our research aligned with the research question and include one most convincing figure with an explanation. I am considering from the Dreamcoder paper Figure 9A as the input and Figure 9D as the output; however, we need to redraw the figure to avoid plagiarism. We can also explore a suitable figure from the DreamCoder paper as inspiration such as the one in figure 11A to learn phsyical laws, redraw it, and replace the corresponding description accordingly. Thank you.}
%This system shows significant potential by adaptively learning synthesized programs that are compact, scalable, and closely approximate the oracle solution. This adaptability instills strong confidence in the robustness and adaptability of our high-level components for perception and planning, which we hope to synthesize in a similar learning-based way.% ensuring their robustness and adaptability.

%\begin{figure*}[tb!]
%\centering
%\includegraphics[width = .6\textwidth]
%{figs/Collision-overturn.jpeg}
%\caption{A traffic collision example}
%\label{fig:accident}
%\end{figure*}
%\begin{wrapfigure}{R}{0.25\textwidth}
%\vspace{-1.0cm}
%\begin{center}
%\includegraphics[width=0.25\textwidth]{figs/Collision-overturn.png}

\aftersectionskip
%\section{A case study}
%\input{Casestudy}
%\label{sec:casestudy}

\sectionspacing
\section{CONCLUSION}
\label{sec:conclusion}
We present RiskHarvester, a risk-based tool to compute a security risk score based on the value of the asset and ease of attack on a database. We calculated the value of asset by identifying the sensitive data categories present in a database from the database keywords. We utilized data flow analysis, SQL, and Object Relational Mapper (ORM) parsing to identify the database keywords. To calculate the ease of attack, we utilized passive network analysis to retrieve the database host information. To evaluate RiskHarvester, we curated RiskBench, a benchmark of 1,791 database secret-asset pairs with sensitive data categories and host information manually retrieved from 188 GitHub repositories. RiskHarvester demonstrates precision of (95\%) and recall (90\%) in detecting database keywords for the value of asset and precision of (96\%) and recall (94\%) in detecting valid hosts for ease of attack. Finally, we conducted an online survey to understand whether developers prioritize secret removal based on security risk score. We found that 86\% of the developers prioritized the secrets for removal with descending security risk scores.
\aftersectionskip

\clearpage

\balance
\bibliographystyle{ACM-Reference-Format}
\bibliography{main}
%\printbibliography

\end{document}
\endinput
%%
%% End of file `sample-acmsmall.tex'.
