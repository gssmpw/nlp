% This must be in the first 5 lines to tell arXiv to use pdfLaTeX, which is strongly recommended.
\pdfoutput=1
% In particular, the hyperref package requires pdfLaTeX in order to break URLs across lines.

\documentclass[11pt]{article}

% Change "review" to "final" to generate the final (sometimes called camera-ready) version.
% Change to "preprint" to generate a non-anonymous version with page numbers.
% \usepackage[review]{acl}
\usepackage[preprint]{acl}
 % \usepackage[final]{acl}

% Standard package includes
\usepackage{times}
\usepackage{latexsym}
\usepackage{booktabs} 
\usepackage{graphicx} 
\usepackage{lipsum}   
\usepackage{caption}  
\usepackage{geometry} 
\usepackage{algorithm}
\usepackage{algorithmicx}
\usepackage{algpseudocode}
\usepackage{subcaption} 
\usepackage{multirow}

\geometry{a4paper, margin=1in} 
%\newcommand{\uq-eval}{MCQA-Eval}




% For proper rendering and hyphenation of words containing Latin characters (including in bib files)
\usepackage[T1]{fontenc}
% For Vietnamese characters
% \usepackage[T5]{fontenc}
% See https://www.latex-project.org/help/documentation/encguide.pdf for other character sets

% This assumes your files are encoded as UTF8
\usepackage[utf8]{inputenc}

% This is not strictly necessary, and may be commented out,
% but it will improve the layout of the manuscript,
% and will typically save some space.
\usepackage{microtype}

% This is also not strictly necessary, and may be commented out.
% However, it will improve the aesthetics of text in
% the typewriter font.
\usepackage{inconsolata}

%Including images in your LaTeX document requires adding
%additional package(s)
\usepackage{graphicx}


\newcommand{\dlc}[1]{\textcolor{blue}{dlc: #1}}

\newcommand{\hua}[1]{\textcolor{green}{hua: #1}}

\newcommand{\cc}[1]{\textcolor{magenta}{ccc: #1}}

% If the title and author information does not fit in the area allocated, uncomment the following
%
%\setlength\titlebox{<dim>}
%
% and set <dim> to something 5cm or larger.

\title{\uqeval: Efficient Confidence Evaluation in NLG with Gold-Standard Correctness Labels}
%: Reliable Evaluation for NLG Confidence
% MCQA-Eval: A Multiple-Choice Approach to Evaluating Confidence in Natural Language Generation
% Beyond Correctness Functions: A Multiple-Choice Framework for Evaluating Confidence in NLG
%MCQA-Eval: A Multiple-Choice Reformulation of Confidence Evaluation in NLG
% Confidence Evaluation in NLG with Economical Gold-Standard Correctness Labels
% Gold-Labeled Confidence Evaluation for NLG via a Multiple-Choice Framework

% Author information can be set in various styles:
% For several authors from the same institution:
% \author{Author 1 \and ... \and Author n \\
%         Address line \\ ... \\ Address line}
% if the names do not fit well on one line use
%         Author 1 \\ {\bf Author 2} \\ ... \\ {\bf Author n} \\
% For authors from different institutions:
% \author{Author 1 \\ Address line \\  ... \\ Address line
%         \And  ... \And
%         Author n \\ Address line \\ ... \\ Address line}
% To start a separate ``row'' of authors use \AND, as in
% \author{Author 1 \\ Address line \\  ... \\ Address line
%         \AND
%         Author 2 \\ Address line \\ ... \\ Address line \And
%         Author 3 \\ Address line \\ ... \\ Address line}

\author{Xiaoou Liu*$^{1}$, Zhen Lin*, Longchao Da$^{1}$, Chacha Chen$^{2}$, Shubhendu Trivedi, Hua Wei$^{1}$\\
  % \texttt{shubhendu@csail.mit.edu}\\
  %\texttt{xiaoouli@asu.edu}
  %\texttt{longchao@asu.edu}
  ${^1}$ Arizona State University\\
  ${^2}$ University of Chicago\\
  Correspondence: \texttt{xiaoouli@asu.edu} \\
% ${^2}$ Carle’s Illinois College of Medicine, University of Illinois at Urbana-Champaign\\
% Chacha Chen University of Chicago chacha@uchicago.edu
}

% \author{First Author \\
%   Affiliation / Address line 1 \\
%   Affiliation / Address line 2 \\
%   Affiliation / Address line 3 \\
%   \texttt{email@domain} \\\And
%   Second Author \\
%   Affiliation / Address line 1 \\
%   Affiliation / Address line 2 \\
%   Affiliation / Address line 3 \\
%   \texttt{email@domain} \\}

%\author{
%  \textbf{First Author\textsuperscript{1}},
%  \textbf{Second Author\textsuperscript{1,2}},
%  \textbf{Third T. Author\textsuperscript{1}},
%  \textbf{Fourth Author\textsuperscript{1}},
%\\
%  \textbf{Fifth Author\textsuperscript{1,2}},
%  \textbf{Sixth Author\textsuperscript{1}},
%  \textbf{Seventh Author\textsuperscript{1}},
%  \textbf{Eighth Author \textsuperscript{1,2,3,4}},
%\\
%  \textbf{Ninth Author\textsuperscript{1}},
%  \textbf{Tenth Author\textsuperscript{1}},
%  \textbf{Eleventh E. Author\textsuperscript{1,2,3,4,5}},
%  \textbf{Twelfth Author\textsuperscript{1}},
%\\
%  \textbf{Thirteenth Author\textsuperscript{3}},
%  \textbf{Fourteenth F. Author\textsuperscript{2,4}},
%  \textbf{Fifteenth Author\textsuperscript{1}},
%  \textbf{Sixteenth Author\textsuperscript{1}},
%\\
%  \textbf{Seventeenth S. Author\textsuperscript{4,5}},
%  \textbf{Eighteenth Author\textsuperscript{3,4}},
%  \textbf{Nineteenth N. Author\textsuperscript{2,5}},
%  \textbf{Twentieth Author\textsuperscript{1}}
%\\
%\\
%  \textsuperscript{1}Affiliation 1,
%  \textsuperscript{2}Affiliation 2,
%  \textsuperscript{3}Affiliation 3,
%  \textsuperscript{4}Affiliation 4,
%  \textsuperscript{5}Affiliation 5
%\\
%  \small{
%    \textbf{Correspondence:} \href{mailto:email@domain}{email@domain}
%  }
%}

\usepackage{bbm}
\usepackage{graphicx}
\usepackage{amsmath,amssymb,amsthm,amsfonts}

\usepackage{paralist}
\usepackage{bm}
\usepackage{xspace}
\usepackage{url}
\usepackage{prettyref}
\usepackage{boxedminipage}
\usepackage{wrapfig}
\usepackage{ifthen}
\usepackage{color}
\usepackage{xspace}

\newcommand{\ii}{{\sc Indicator-Instance}\xspace}
\newcommand{\midd}{{\sf mid}}


\usepackage{amsmath,amsthm,amsfonts,amssymb}
\usepackage{mathtools}
\usepackage{graphicx}


% \usepackage{fullpage}

\usepackage{nicefrac}

\newtheorem{inftheorem}{Informal Theorem}
\newtheorem{claim}{Claim}
\newtheorem*{definition*}{Definition}
\newtheorem{example}{Example}

\DeclareMathOperator*{\argmax}{arg\,max}
\DeclareMathOperator*{\argmin}{arg\,min}
\usepackage{subcaption}

\newtheorem{problem}{Problem}
\usepackage[utf8]{inputenc}
\newcommand{\rank}{\mathsf{rank}}
\newcommand{\tr}{\mathsf{Tr}}
\newcommand{\tv}{\mathsf{TV}}
\newcommand{\opt}{\mathsf{OPT}}
\newcommand{\rr}{\textsc{R}\space}
\newcommand{\alg}{\textsf{Alg}\space}
\newcommand{\sd}{\textsf{sd}_\lambda}
\newcommand{\lblq}{\mathfrak{lq} (X_1)}
\newcommand{\diag}{\textsf{diag}}
\newcommand{\sign}{\textsf{sgn}}
\newcommand{\BC}{\texttt{BC} }
\newcommand{\MM}{\texttt{MM} }
\newcommand{\Nexp}{N_{\mathrm{exp}}}
\newcommand{\Nrep}{N_{\mathrm{replay}}}
\newcommand{\Drep}{D_{\mathrm{replay}}}
\newcommand{\Nsim}{N_{\mathrm{sim}}}
\newcommand{\piBC}{\pi^{\texttt{BC}}}
\newcommand{\piRE}{\pi^{\texttt{RE}}}
\newcommand{\piEMM}{\pi^{\texttt{MM}}}
\newcommand{\mmd}{\texttt{Mimic-MD} }
\newcommand{\RE}{\texttt{RE} }
\newcommand{\dem}{\pi^E}
\newcommand{\Rlint}{\mathcal{R}_{\mathrm{lin,t}}}
\newcommand{\Rlipt}{\mathcal{R}_{\mathrm{lip,t}}}
\newcommand{\Rlin}{\mathcal{R}_{\mathrm{lin}}}
\newcommand{\Rlip}{\mathcal{R}_{\mathrm{lip}}}
\newcommand{\Rmax}{R_{\mathrm{max}}}
\newcommand{\Rall}{\mathcal{R}_{\mathrm{all}}}
\newcommand{\Rdet}{\mathcal{R}_{\mathrm{det}}}
\newcommand{\Fmax}{F_{\mathrm{max}}}
\newcommand{\Nmax}{\mathcal{N}_{\mathrm{max}}}
\newcommand{\piref}{\pi^{\mathrm{ref}}}
\newcommand{\green}{\text{\color{green!75!black} green}\;}
\newcommand{\thetaBC}{\widehat{\theta}^{\textsf{BC}}}
\newcommand{\ent}{\mathcal{E}_{\Theta,n,\delta}}
\newcommand{\eNt}{\mathcal{E}_{\Theta_t,\Nexp,\delta}}
\newcommand{\eNtH}{\mathcal{E}_{\Theta_t,\Nexp,\delta/H}}

\newcommand{\eref}[1]{(\ref{#1})}
\newcommand{\sref}[1]{Sec. \ref{#1}}
\newcommand{\dr}{\widehat{d}_{\mathrm{replay}}}
\newcommand{\figref}[1]{Fig. \ref{#1}}

\usepackage{xcolor}
\definecolor{expert}{HTML}{008000}
\definecolor{error}{HTML}{f96565}
\newcommand{\GKS}[1]{{\textcolor{violet}{\textbf{GKS: #1}}}}
\newcommand{\Q}[1]{{\textcolor{red}{\textbf{Question #1}}}}
\newcommand{\ZSW}[1]{{\textcolor{orange}{\textbf{ZSW: #1}}}}
\newcommand{\JAB}[1]{{\textcolor{teal}{\textbf{JAB: #1}}}}
\newcommand{\jab}[1]{{\textcolor{teal}{\textbf{JAB: #1}}}}
\newcommand{\SAN}[1]{{\textcolor{blue}{\textbf{SC: #1}}}}
\newcommand{\scnote}[1]{\SAN{#1}}
\newcommand{\norm}[1]{\left\lVert #1 \right\rVert}

\usepackage{color-edits}
\addauthor{sw}{blue}

\usepackage{thmtools}
\usepackage{thm-restate}

\usepackage{tikz}
\usetikzlibrary{arrows,calc} 
\newcommand{\tikzAngleOfLine}{\tikz@AngleOfLine}
\def\tikz@AngleOfLine(#1)(#2)#3{%
\pgfmathanglebetweenpoints{%
\pgfpointanchor{#1}{center}}{%
\pgfpointanchor{#2}{center}}
\pgfmathsetmacro{#3}{\pgfmathresult}%
}

\declaretheoremstyle[
    headfont=\normalfont\bfseries, 
    bodyfont = \normalfont\itshape]{mystyle} 
\declaretheorem[name=Theorem,style=mystyle,numberwithin=section]{thm}

% \usepackage{algorithm}
% \usepackage{algorithmic}
\usepackage[linesnumbered,algoruled,boxed,lined,noend]{algorithm2e}

\usepackage{listings}
\usepackage{amsmath}
\usepackage{amsthm}
\usepackage{tikz}
\usepackage{caption}
\usepackage{mdwmath}
\usepackage{multirow}
\usepackage{mdwtab}
\usepackage{eqparbox}
\usepackage{multicol}
\usepackage{amsfonts}
\usepackage{tikz}
\usepackage{multirow,bigstrut,threeparttable}
\usepackage{amsthm}
\usepackage{bbm}
\usepackage{epstopdf}
\usepackage{mdwmath}
\usepackage{mdwtab}
\usepackage{eqparbox}
\usetikzlibrary{topaths,calc}
\usepackage{latexsym}
\usepackage{cite}
\usepackage{amssymb}
\usepackage{bm}
\usepackage{amssymb}
\usepackage{graphicx}
\usepackage{mathrsfs}
\usepackage{epsfig}
\usepackage{psfrag}
\usepackage{setspace}
\usepackage[%dvips,
            CJKbookmarks=true,
            bookmarksnumbered=true,
            bookmarksopen=true,
%						bookmarks=false,
            colorlinks=true,
            citecolor=red,
            linkcolor=blue,
            anchorcolor=red,
            urlcolor=blue
            ]{hyperref}
%\usepackage{algorithm}
\usepackage[linesnumbered,algoruled,boxed,lined]{algorithm2e}
\usepackage{algpseudocode}
\usepackage{stfloats}
\RequirePackage[numbers]{natbib}

\usepackage{comment}
\usepackage{mathtools}
\usepackage{blkarray}
\usepackage{multirow,bigdelim,dcolumn,booktabs}

\usepackage{xparse}
\usepackage{tikz}
\usetikzlibrary{calc}
\usetikzlibrary{decorations.pathreplacing,matrix,positioning}

\usepackage[T1]{fontenc}
\usepackage[utf8]{inputenc}
\usepackage{mathtools}
\usepackage{blkarray, bigstrut}
\usepackage{gauss}

\newenvironment{mygmatrix}{\def\mathstrut{\vphantom{\big(}}\gmatrix}{\endgmatrix}

\newcommand{\tikzmark}[1]{\tikz[overlay,remember picture] \node (#1) {};}

%% Adapted form https://tex.stackexchange.com/questions/206898/braces-for-cases-in-tabular-environment/207704#207704
\newcommand*{\BraceAmplitude}{0.4em}%
\newcommand*{\VerticalOffset}{0.5ex}%  
\newcommand*{\HorizontalOffset}{0.0em}% 
\newcommand*{\blocktextwid}{3.0cm}%
\NewDocumentCommand{\InsertLeftBrace}{%
	O{} % #1 = draw options
	O{\HorizontalOffset,\VerticalOffset} % #2 = optional brace shift options
	O{\blocktextwid} % #3 = optional text width
	m   % #4 = top tikzmark
	m   % #5 = bottom tikzmark
	m   % #6 = node text
}{%
	\begin{tikzpicture}[overlay,remember picture]
	\coordinate (Brace Top)    at ($(#4.north) + (#2)$);
	\coordinate (Brace Bottom) at ($(#5.south) + (#2)$);
	\draw [decoration={brace, amplitude=\BraceAmplitude}, decorate, thick, draw=black, #1]
	(Brace Bottom) -- (Brace Top) 
	node [pos=0.5, anchor=east, align=left, text width=#3, color=black, xshift=\BraceAmplitude] {#6};
	\end{tikzpicture}%
}%
\NewDocumentCommand{\InsertRightBrace}{%
	O{} % #1 = draw options
	O{\HorizontalOffset,\VerticalOffset} % #2 = optional brace shift options
	O{\blocktextwid} % #3 = optional text width
	m   % #4 = top tikzmark
	m   % #5 = bottom tikzmark
	m   % #6 = node text
}{%
	\begin{tikzpicture}[overlay,remember picture]
	\coordinate (Brace Top)    at ($(#4.north) + (#2)$);
	\coordinate (Brace Bottom) at ($(#5.south) + (#2)$);
	\draw [decoration={brace, amplitude=\BraceAmplitude}, decorate, thick, draw=black, #1]
	(Brace Top) -- (Brace Bottom) 
	node [pos=0.5, anchor=west, align=left, text width=#3, color=black, xshift=\BraceAmplitude] {#6};
	\end{tikzpicture}%
}%
\NewDocumentCommand{\InsertTopBrace}{%
	O{} % #1 = draw options
	O{\HorizontalOffset,\VerticalOffset} % #2 = optional brace shift options
	O{\blocktextwid} % #3 = optional text width
	m   % #4 = top tikzmark
	m   % #5 = bottom tikzmark
	m   % #6 = node text
}{%
	\begin{tikzpicture}[overlay,remember picture]
	\coordinate (Brace Top)    at ($(#4.west) + (#2)$);
	\coordinate (Brace Bottom) at ($(#5.east) + (#2)$);
	\draw [decoration={brace, amplitude=\BraceAmplitude}, decorate, thick, draw=black, #1]
	(Brace Top) -- (Brace Bottom) 
	node [pos=0.5, anchor=south, align=left, text width=#3, color=black, xshift=\BraceAmplitude] {#6};
	\end{tikzpicture}%
}%

\usetikzlibrary{patterns}

\definecolor{cof}{RGB}{219,144,71}
\definecolor{pur}{RGB}{186,146,162}
\definecolor{greeo}{RGB}{91,173,69}
\definecolor{greet}{RGB}{52,111,72}

% provide arXiv number if available:
% \arxiv{cs.IT/1502.00326}

% put your definitions there:

%\newtheorem{remark}{Remark} \def\remref#1{Remark~\ref{#1}}
%\newtheorem{conjecture}{Conjecture} \def\remref#1{Remark~\ref{#1}}
%\newtheorem{example}{Example}

%\theorembodyfont{\itshape}
%\newtheorem{theorem}{Theorem}
%\newtheorem{proposition}{Proposition}
%\newtheorem{lemma}{Lemma} \def\lemref#1{Lemma~\ref{#1}}
%\newtheorem{corollary}{Corollary}


%\theorembodyfont{\rmfamily}
%\newtheorem{definition}{Definition}
%\numberwithin{equation}{section}
% \theoremstyle{plain}
% \newtheorem{theorem}{Theorem}
% \newtheorem{Example}{Example}
% \newtheorem{lemma}{Lemma}
% \newtheorem{remark}{Remark}
% \newtheorem{corollary}{Corollary}
% \newtheorem{definition}{Definition}
% \newtheorem{conjecture}{Conjecture}
% \newtheorem{question}{Question}
% \newtheorem*{induction}{Induction Hypothesis}
% \newtheorem*{folklore}{Folklore}
% \newtheorem{assumption}{Assumption}

\def \by {\bar{y}}
\def \bx {\bar{x}}
\def \bh {\bar{h}}
\def \bz {\bar{z}}
\def \cF {\mathcal{F}}
\def \bP {\mathbb{P}}
\def \bE {\mathbb{E}}
\def \bR {\mathbb{R}}
\def \bF {\mathbb{F}}
\def \cG {\mathcal{G}}
\def \cM {\mathcal{M}}
\def \cB {\mathcal{B}}
\def \cN {\mathcal{N}}
\def \var {\mathsf{Var}}
\def\1{\mathbbm{1}}
\def \FF {\mathbb{F}}


\newenvironment{keywords}
{\bgroup\leftskip 20pt\rightskip 20pt \small\noindent{\bfseries
Keywords:} \ignorespaces}%
{\par\egroup\vskip 0.25ex}
\newlength\aftertitskip     \newlength\beforetitskip
\newlength\interauthorskip  \newlength\aftermaketitskip















%%%%%%%%%%%%%%%%%%%%%%%%%%%% by Wu %%%%%%%%%%%%%%%%%%%%%%%%%%%%
\usepackage{xspace}

\newcommand{\Lip}{\mathrm{Lip}}
\newcommand{\stepa}[1]{\overset{\rm (a)}{#1}}
\newcommand{\stepb}[1]{\overset{\rm (b)}{#1}}
\newcommand{\stepc}[1]{\overset{\rm (c)}{#1}}
\newcommand{\stepd}[1]{\overset{\rm (d)}{#1}}
\newcommand{\stepe}[1]{\overset{\rm (e)}{#1}}
\newcommand{\stepf}[1]{\overset{\rm (f)}{#1}}


\newcommand{\floor}[1]{{\left\lfloor {#1} \right \rfloor}}
\newcommand{\ceil}[1]{{\left\lceil {#1} \right \rceil}}

\newcommand{\blambda}{\bar{\lambda}}
\newcommand{\reals}{\mathbb{R}}
\newcommand{\naturals}{\mathbb{N}}
\newcommand{\integers}{\mathbb{Z}}
\newcommand{\Expect}{\mathbb{E}}
\newcommand{\expect}[1]{\mathbb{E}\left[#1\right]}
\newcommand{\Prob}{\mathbb{P}}
\newcommand{\prob}[1]{\mathbb{P}\left[#1\right]}
\newcommand{\pprob}[1]{\mathbb{P}[#1]}
\newcommand{\intd}{{\rm d}}
\newcommand{\TV}{{\sf TV}}
\newcommand{\LC}{{\sf LC}}
\newcommand{\PW}{{\sf PW}}
\newcommand{\htheta}{\hat{\theta}}
\newcommand{\eexp}{{\rm e}}
\newcommand{\expects}[2]{\mathbb{E}_{#2}\left[ #1 \right]}
\newcommand{\diff}{{\rm d}}
\newcommand{\eg}{e.g.\xspace}
\newcommand{\ie}{i.e.\xspace}
\newcommand{\iid}{i.i.d.\xspace}
\newcommand{\fracp}[2]{\frac{\partial #1}{\partial #2}}
\newcommand{\fracpk}[3]{\frac{\partial^{#3} #1}{\partial #2^{#3}}}
\newcommand{\fracd}[2]{\frac{\diff #1}{\diff #2}}
\newcommand{\fracdk}[3]{\frac{\diff^{#3} #1}{\diff #2^{#3}}}
\newcommand{\renyi}{R\'enyi\xspace}
\newcommand{\lpnorm}[1]{\left\|{#1} \right\|_{p}}
\newcommand{\linf}[1]{\left\|{#1} \right\|_{\infty}}
\newcommand{\lnorm}[2]{\left\|{#1} \right\|_{{#2}}}
\newcommand{\Lploc}[1]{L^{#1}_{\rm loc}}
\newcommand{\hellinger}{d_{\rm H}}
\newcommand{\Fnorm}[1]{\lnorm{#1}{\rm F}}
%% parenthesis
\newcommand{\pth}[1]{\left( #1 \right)}
\newcommand{\qth}[1]{\left[ #1 \right]}
\newcommand{\sth}[1]{\left\{ #1 \right\}}
\newcommand{\bpth}[1]{\Bigg( #1 \Bigg)}
\newcommand{\bqth}[1]{\Bigg[ #1 \Bigg]}
\newcommand{\bsth}[1]{\Bigg\{ #1 \Bigg\}}
\newcommand{\xxx}{\textbf{xxx}\xspace}
\newcommand{\toprob}{{\xrightarrow{\Prob}}}
\newcommand{\tolp}[1]{{\xrightarrow{L^{#1}}}}
\newcommand{\toas}{{\xrightarrow{{\rm a.s.}}}}
\newcommand{\toae}{{\xrightarrow{{\rm a.e.}}}}
\newcommand{\todistr}{{\xrightarrow{{\rm D}}}}
\newcommand{\eqdistr}{{\stackrel{\rm D}{=}}}
\newcommand{\iiddistr}{{\stackrel{\text{\iid}}{\sim}}}
%\newcommand{\var}{\mathsf{var}}
\newcommand\indep{\protect\mathpalette{\protect\independenT}{\perp}}
\def\independenT#1#2{\mathrel{\rlap{$#1#2$}\mkern2mu{#1#2}}}
\newcommand{\Bern}{\text{Bern}}
\newcommand{\Poi}{\mathsf{Poi}}
\newcommand{\iprod}[2]{\left \langle #1, #2 \right\rangle}
\newcommand{\Iprod}[2]{\langle #1, #2 \rangle}
\newcommand{\indc}[1]{{\mathbf{1}_{\left\{{#1}\right\}}}}
\newcommand{\Indc}{\mathbf{1}}
\newcommand{\regoff}[1]{\textsf{Reg}_{\mathcal{F}}^{\text{off}} (#1)}
\newcommand{\regon}[1]{\textsf{Reg}_{\mathcal{F}}^{\text{on}} (#1)}

\definecolor{myblue}{rgb}{.8, .8, 1}
\definecolor{mathblue}{rgb}{0.2472, 0.24, 0.6} % mathematica's Color[1, 1--3]
\definecolor{mathred}{rgb}{0.6, 0.24, 0.442893}
\definecolor{mathyellow}{rgb}{0.6, 0.547014, 0.24}


\newcommand{\red}{\color{red}}
\newcommand{\blue}{\color{blue}}
\newcommand{\nb}[1]{{\sf\blue[#1]}}
\newcommand{\nbr}[1]{{\sf\red[#1]}}

\newcommand{\tmu}{{\tilde{\mu}}}
\newcommand{\tf}{{\tilde{f}}}
\newcommand{\tp}{\tilde{p}}
\newcommand{\tilh}{{\tilde{h}}}
\newcommand{\tu}{{\tilde{u}}}
\newcommand{\tx}{{\tilde{x}}}
\newcommand{\ty}{{\tilde{y}}}
\newcommand{\tz}{{\tilde{z}}}
\newcommand{\tA}{{\tilde{A}}}
\newcommand{\tB}{{\tilde{B}}}
\newcommand{\tC}{{\tilde{C}}}
\newcommand{\tD}{{\tilde{D}}}
\newcommand{\tE}{{\tilde{E}}}
\newcommand{\tF}{{\tilde{F}}}
\newcommand{\tG}{{\tilde{G}}}
\newcommand{\tH}{{\tilde{H}}}
\newcommand{\tI}{{\tilde{I}}}
\newcommand{\tJ}{{\tilde{J}}}
\newcommand{\tK}{{\tilde{K}}}
\newcommand{\tL}{{\tilde{L}}}
\newcommand{\tM}{{\tilde{M}}}
\newcommand{\tN}{{\tilde{N}}}
\newcommand{\tO}{{\tilde{O}}}
\newcommand{\tP}{{\tilde{P}}}
\newcommand{\tQ}{{\tilde{Q}}}
\newcommand{\tR}{{\tilde{R}}}
\newcommand{\tS}{{\tilde{S}}}
\newcommand{\tT}{{\tilde{T}}}
\newcommand{\tU}{{\tilde{U}}}
\newcommand{\tV}{{\tilde{V}}}
\newcommand{\tW}{{\tilde{W}}}
\newcommand{\tX}{{\tilde{X}}}
\newcommand{\tY}{{\tilde{Y}}}
\newcommand{\tZ}{{\tilde{Z}}}

\newcommand{\sfa}{{\mathsf{a}}}
\newcommand{\sfb}{{\mathsf{b}}}
\newcommand{\sfc}{{\mathsf{c}}}
\newcommand{\sfd}{{\mathsf{d}}}
\newcommand{\sfe}{{\mathsf{e}}}
\newcommand{\sff}{{\mathsf{f}}}
\newcommand{\sfg}{{\mathsf{g}}}
\newcommand{\sfh}{{\mathsf{h}}}
\newcommand{\sfi}{{\mathsf{i}}}
\newcommand{\sfj}{{\mathsf{j}}}
\newcommand{\sfk}{{\mathsf{k}}}
\newcommand{\sfl}{{\mathsf{l}}}
\newcommand{\sfm}{{\mathsf{m}}}
\newcommand{\sfn}{{\mathsf{n}}}
\newcommand{\sfo}{{\mathsf{o}}}
\newcommand{\sfp}{{\mathsf{p}}}
\newcommand{\sfq}{{\mathsf{q}}}
\newcommand{\sfr}{{\mathsf{r}}}
\newcommand{\sfs}{{\mathsf{s}}}
\newcommand{\sft}{{\mathsf{t}}}
\newcommand{\sfu}{{\mathsf{u}}}
\newcommand{\sfv}{{\mathsf{v}}}
\newcommand{\sfw}{{\mathsf{w}}}
\newcommand{\sfx}{{\mathsf{x}}}
\newcommand{\sfy}{{\mathsf{y}}}
\newcommand{\sfz}{{\mathsf{z}}}
\newcommand{\sfA}{{\mathsf{A}}}
\newcommand{\sfB}{{\mathsf{B}}}
\newcommand{\sfC}{{\mathsf{C}}}
\newcommand{\sfD}{{\mathsf{D}}}
\newcommand{\sfE}{{\mathsf{E}}}
\newcommand{\sfF}{{\mathsf{F}}}
\newcommand{\sfG}{{\mathsf{G}}}
\newcommand{\sfH}{{\mathsf{H}}}
\newcommand{\sfI}{{\mathsf{I}}}
\newcommand{\sfJ}{{\mathsf{J}}}
\newcommand{\sfK}{{\mathsf{K}}}
\newcommand{\sfL}{{\mathsf{L}}}
\newcommand{\sfM}{{\mathsf{M}}}
\newcommand{\sfN}{{\mathsf{N}}}
\newcommand{\sfO}{{\mathsf{O}}}
\newcommand{\sfP}{{\mathsf{P}}}
\newcommand{\sfQ}{{\mathsf{Q}}}
\newcommand{\sfR}{{\mathsf{R}}}
\newcommand{\sfS}{{\mathsf{S}}}
\newcommand{\sfT}{{\mathsf{T}}}
\newcommand{\sfU}{{\mathsf{U}}}
\newcommand{\sfV}{{\mathsf{V}}}
\newcommand{\sfW}{{\mathsf{W}}}
\newcommand{\sfX}{{\mathsf{X}}}
\newcommand{\sfY}{{\mathsf{Y}}}
\newcommand{\sfZ}{{\mathsf{Z}}}


\newcommand{\calA}{{\mathcal{A}}}
\newcommand{\calB}{{\mathcal{B}}}
\newcommand{\calC}{{\mathcal{C}}}
\newcommand{\calD}{{\mathcal{D}}}
\newcommand{\calE}{{\mathcal{E}}}
\newcommand{\calF}{{\mathcal{F}}}
\newcommand{\calG}{{\mathcal{G}}}
\newcommand{\calH}{{\mathcal{H}}}
\newcommand{\calI}{{\mathcal{I}}}
\newcommand{\calJ}{{\mathcal{J}}}
\newcommand{\calK}{{\mathcal{K}}}
\newcommand{\calL}{{\mathcal{L}}}
\newcommand{\calM}{{\mathcal{M}}}
\newcommand{\calN}{{\mathcal{N}}}
\newcommand{\calO}{{\mathcal{O}}}
\newcommand{\calP}{{\mathcal{P}}}
\newcommand{\calQ}{{\mathcal{Q}}}
\newcommand{\calR}{{\mathcal{R}}}
\newcommand{\calS}{{\mathcal{S}}}
\newcommand{\calT}{{\mathcal{T}}}
\newcommand{\calU}{{\mathcal{U}}}
\newcommand{\calV}{{\mathcal{V}}}
\newcommand{\calW}{{\mathcal{W}}}
\newcommand{\calX}{{\mathcal{X}}}
\newcommand{\calY}{{\mathcal{Y}}}
\newcommand{\calZ}{{\mathcal{Z}}}

\newcommand{\bara}{{\bar{a}}}
\newcommand{\barb}{{\bar{b}}}
\newcommand{\barc}{{\bar{c}}}
\newcommand{\bard}{{\bar{d}}}
\newcommand{\bare}{{\bar{e}}}
\newcommand{\barf}{{\bar{f}}}
\newcommand{\barg}{{\bar{g}}}
\newcommand{\barh}{{\bar{h}}}
\newcommand{\bari}{{\bar{i}}}
\newcommand{\barj}{{\bar{j}}}
\newcommand{\bark}{{\bar{k}}}
\newcommand{\barl}{{\bar{l}}}
\newcommand{\barm}{{\bar{m}}}
\newcommand{\barn}{{\bar{n}}}
\newcommand{\baro}{{\bar{o}}}
\newcommand{\barp}{{\bar{p}}}
\newcommand{\barq}{{\bar{q}}}
\newcommand{\barr}{{\bar{r}}}
\newcommand{\bars}{{\bar{s}}}
\newcommand{\bart}{{\bar{t}}}
\newcommand{\baru}{{\bar{u}}}
\newcommand{\barv}{{\bar{v}}}
\newcommand{\barw}{{\bar{w}}}
\newcommand{\barx}{{\bar{x}}}
\newcommand{\bary}{{\bar{y}}}
\newcommand{\barz}{{\bar{z}}}
\newcommand{\barA}{{\bar{A}}}
\newcommand{\barB}{{\bar{B}}}
\newcommand{\barC}{{\bar{C}}}
\newcommand{\barD}{{\bar{D}}}
\newcommand{\barE}{{\bar{E}}}
\newcommand{\barF}{{\bar{F}}}
\newcommand{\barG}{{\bar{G}}}
\newcommand{\barH}{{\bar{H}}}
\newcommand{\barI}{{\bar{I}}}
\newcommand{\barJ}{{\bar{J}}}
\newcommand{\barK}{{\bar{K}}}
\newcommand{\barL}{{\bar{L}}}
\newcommand{\barM}{{\bar{M}}}
\newcommand{\barN}{{\bar{N}}}
\newcommand{\barO}{{\bar{O}}}
\newcommand{\barP}{{\bar{P}}}
\newcommand{\barQ}{{\bar{Q}}}
\newcommand{\barR}{{\bar{R}}}
\newcommand{\barS}{{\bar{S}}}
\newcommand{\barT}{{\bar{T}}}
\newcommand{\barU}{{\bar{U}}}
\newcommand{\barV}{{\bar{V}}}
\newcommand{\barW}{{\bar{W}}}
\newcommand{\barX}{{\bar{X}}}
\newcommand{\barY}{{\bar{Y}}}
\newcommand{\barZ}{{\bar{Z}}}

\newcommand{\hX}{\hat{X}}
\newcommand{\Ent}{\mathsf{Ent}}
\newcommand{\awarm}{{A_{\text{warm}}}}
\newcommand{\thetaLS}{{\widehat{\theta}^{\text{\rm LS}}}}

\newcommand{\jiao}[1]{\langle{#1}\rangle}
\newcommand{\gaht}{\textsc{GoodActionHypTest}\;}
\newcommand{\iaht}{\textsc{InitialActionHypTest}\;}
\newcommand{\true}{\textsf{True}\;}
\newcommand{\false}{\textsf{False}\;}

% \usepackage[capitalize,noabbrev]{cleveref}
% \crefname{lemma}{Lemma}{Lemmas}
% \Crefname{lemma}{Lemma}{Lemmas}
% \crefname{thm}{Theorem}{Theorems}
% \Crefname{thm}{Theorem}{Theorems}
% \Crefname{assumption}{Assumption}{Assumptions}
% \Crefname{inftheorem}{Informal Theorem}{Informal Theorems}
% \crefformat{equation}{(#2#1#3)}

% % if you use cleveref..
% \usepackage[capitalize,noabbrev]{cleveref}
% \crefname{lemma}{Lemma}{Lemmas}
% \crefname{proposition}{Proposition}{Propositions}
% \crefname{remark}{Remark}{Remarks}
% \crefname{corollary}{Corollary}{Corollaries}
% \crefname{definition}{Definition}{Definitions}
% \crefname{conjecture}{Conjecture}{Conjectures}
% \crefname{figure}{Fig.}{Figures}


\begin{document}
\maketitle
\begin{abstract}
Large Language Models (LLMs) require robust confidence estimation, particularly in critical domains like healthcare and law where unreliable outputs can lead to significant consequences. 
Despite much recent work in confidence estimation, current evaluation frameworks rely on \textit{correctness functions}---various heuristics that are often noisy, expensive, and possibly introduce systematic biases. % through human evaluation limitations and reference matching constraints.
These methodological weaknesses tend to distort evaluation metrics and thus the comparative ranking of confidence measures.
We introduce \uqeval, an evaluation framework for assessing confidence measures in Natural Language Generation (NLG) that eliminates dependence on an explicit correctness function by leveraging gold-standard correctness labels from multiple-choice datasets.
%on correctness labels through structured multiple-choice QA datasets.
\uqeval enables systematic comparison of both internal state-based white-box (e.g. logit-based) and consistency-based black-box confidence measures, providing a unified evaluation methodology across different approaches.
Through extensive experiments on multiple LLMs and widely used QA datasets, we report that \uqeval provides efficient and more reliable assessments of confidence estimation methods than existing approaches.%\footnote{Code and data will be released upon publication.}


%With the rapid advancement of Large Language Models(LLMs), confidence estimation has become increasing important in assessing the reliability of LLM outputs, especially in high stake applications such as healthcare and law, where incorrect or uncertain responses have significant consequences. 
%Despite progress in confidence estimation, existing evaluation methods heavily rely on correctness labels, which introduce systematic biases and inconsistencies due to the limitations of human evaluation, reference matching techniques. 

%These issues impact the reliability of evaluation metrics and the ranking of different confidence estimation methods. \cc{the previous sentence need to be improved its not clear}
%In this paper, we introduce \uqeval, a novel evaluation framework for assessing confidence measures in Natural Language Generation (NLG). Our method leverages multiple-choice QA datasets, providing a structured approach that eliminates the reliance on correctness labels. 
%Our framwork is applicable to both logit-based white-box \hua{inconsistent term - logit-based, internal-states-based} and consistency-based black-box confidence estimation techniques, ensuring a robust comparison across different confidence estimation methods. We conduct extensive experiments on multiple LLMs and widely used QA datasets to demonstrate the effectiveness and efficiency of our proposed framework\footnote{We will release our code and data upon publication.}. 
\end{abstract}

\section{Introduction}\label{sec:intro}

Large Language Models (LLMs) demonstrate strong performance across natural language processing tasks, yet their architectural complexity and limited interpretability can produce unreliable outputs. 
This presents significant challenges in critical domains such as healthcare, where output errors carry serious consequences. 
Confidence estimation methods have emerged to quantify output reliability. 
The field connects closely with uncertainty quantification in natural language generation, as both address output trustworthiness. 
Current approaches divide into consistency-based methods, which analyze agreement across multiple outputs, and internal-states methods that leverage model-specific features like output probabilities.
Despite advances in these approaches, developing robust evaluation frameworks remains a central challenge.

%Large Language Models (LLMs) have achieved notable success in various natural language processing (NLP) tasks, such as text generation, question answering, and reasoning. However, their complex architectures and lack of interpretability often lead to uncertain, incorrect, or ambiguous outputs. This raises critical concerns about their reliability, especially in high-stakes domains like healthcare and decision-making, where errors can have severe consequences. Confidence estimation has emerged as a key tool for addressing these concerns by quantifying the trustworthiness of LLM outputs.


%In natural language generation (NLG), confidence estimation is closely tied to uncertainty quantification (UQ), as both aim to assess the reliability of model outputs. 
%% Confidence scores are often derived from uncertainty measures, with higher uncertainty corresponding to lower confidence. 
%Existing methods for confidence estimation in LLMs can be broadly categorized into two types: consistency-based black-box methods, which rely on agreement among multiple generated responses, and internal-states-based methods, which use model-specific information such as output probabilities or self-evaluation mechanisms. 
%While these approaches have advanced the field, the evaluation of confidence estimation methods remains a significant challenge in their evaluation process.\cc{consider say we mainly do UQ evaluation earlier, cut the context. at least merge the first two paragraphs}

%Currently, the evaluation frameworks for confidence measures in NLG primarily rely on correctness labels to compute metrics such as area under the receiver operating characteristic curve (AUROC) and accuracy-rejection curve (AUARC). 
%These frameworks follow a general pipeline: 
%(1) generating predictions from the model,
%(2) assigning correctness labels to the predictions using a \textit{correctness function} $f(\cdot)$, and 
%(3) calculating evaluation metrics based on these labels. 
%However, the reliance on correct labels introduces several limitations. 
%Human evaluation, though reliable, is time-consuming and impractical for large-scale datasets. 
%Automated reference-based metrics like BLEU and ROUGE fail to account for semantically correct but differently phrased responses. 
%LLM-based evaluators offer flexibility but still suffer from systematic biases~\cite{lin2022truthfulqa}, such as favoring outputs from similar models\cc{add citation here?} or being sensitive to prompt variations—issues that lack concrete evidence in prior work\cc{add citation here?}. 

Current evaluation frameworks for NLG confidence measures rely on correctness labels to compute metrics such as AUROC and AUARC. 
These frameworks follow a three-step process: generating model predictions, labeling correctness via a function $f(\cdot)$, and calculating metrics. 
This label-dependent approach faces several constraints. While human evaluation provides reliable correctness ground truth, it cannot scale to large datasets. 
Metrics based on reference matching, such as BLEU and ROUGE, fail to recognize semantically equivalent responses phrased differently.
% Reference matching-based metrics like BLEU and ROUGE miss semantically equivalent responses with different phrasing. 
% LLM-based evaluators offer greater capability and scalability, but are still noisy, and could introduce systematic biases, such as favor over responses generated by similar LMs~\cite{panickssery2024llm} or over length~\cite{lin2022truthfulqa}.
LLM-based evaluators offer greater capability but remain noisy and may introduce systematic biases, such as favoring responses generated by themselves or similar LMs~\cite{panickssery2024llm}, or preferring longer responses~\cite{lin2022truthfulqa}.
Moreover, running such evaluators could be expensive.
% LLM-based evaluators improve capability and scalability but remain noisy and may introduce systematic biases, such as favoring responses generated by similar LMs~\cite{panickssery2024llm} or preferring longer responses~\cite{lin2022truthfulqa}.
% (also observed in our own experiments) or over length~\cite{lin2022truthfulqa}.
% The prompt used to eli
%including outputs from similar models\cc{add citation here?} and prompt sensitivity—phenomena that remain empirically understudied\cc{add citation here?}.


%More critically, flaws in the correctness function $f(\cdot)$ can propagate through the evaluation pipeline, affecting evaluation metrics like AUROC. 
%This fragility can lead to misleading conclusions about which confidence estimation method performs better, particularly when different methods yield similar performance. 
%These issues highlight the need for alternative evaluation frameworks with reliable correctness labels.

% More critically, flaws in the correctness function $f(\cdot)$ propagate through the evaluation pipeline, affecting metrics like AUROC. 

Flaws in the correctness function $f(\cdot)$ propagate through the evaluation pipeline, affecting metrics like AUROC. This sensitivity becomes particularly problematic when comparing confidence estimation methods with similar performance. Such limitations underscore the need for evaluation frameworks that establish correctness more reliably.


%In this paper, we propose a simple evaluation framework that addresses these limitations by leveraging multiple-choice question-answering (QA) datasets. 
%Our approach removes the dependence on a correctness function by using the structure of multiple-choice QA tasks to evaluate confidence measures directly. 
%Importantly, our framework is complementary to existing pipelines rather than a replacement; it provides an additional perspective to validate the discriminative power of confidence measures across different evaluation settings.

In this paper, we propose \uqeval, a simple, efficient yet effective evaluation framework that eliminates the dependence on unreliable correctness functions.
% leverages multiple-choice question-answering (QA) datasets to address these limitations.
\textit{The key insight is to leverage multiple-choice question-answering (QA) datasets, which inherently provide gold-standard answer choices at no cost.} 
With these definitive labels, our framework bypasses the ambiguity of determining correctness via correctness function $f(\cdot)$ and ensures an objective assessment of confidence estimation methods. 
% By leveraging multiple-choice QA datasets, our proposed evaluation framework ensures an objective assessment of confidence estimation methods.
% Unlike conventional evaluation pipelines that rely on an often noisy correctness function $f(\cdot)$ to determine whether a model’s prediction is right or wrong, our approach directly utilizes gold-standard answer choices in multiple-choice QA tasks. 
% \textit{Our approach eliminates the dependence on unreliable correctness functions} and ensures an objective assessment of confidence estimation methods.
% By exploiting the inherent structure of multiple-choice QA datasets, our approach eliminates dependence on unreliable correctness. 
Rather than replacing existing evaluation pipelines, our framework complements them, offering an additional lens to assess the discriminative power of confidence estimation methods.
% across different evaluation settings.
\cref{fig:pipeline} shows how our proposal (green) and the existing evaluation pipeline (blue) differ, yet complement each other.
% In this paper, we propose a simple yet effective evaluation framework that eliminates reliance on noisy correctness functions. Our key insight is to leverage multiple-choice question-answering (QA) datasets, which inherently provide gold-standard answer choices. By using these definitive labels, our framework bypasses the ambiguity of determining correctness via 
% f(⋅) and ensures an objective assessment of confidence estimation methods. 
% Rather than replacing existing evaluation pipelines, our approach complements them, offering an additional lens to assess the discriminative power of confidence measures. \cref{fig:pipeline} illustrates how our proposal (green) differs from, yet aligns with, the existing pipeline (blue).
Our contributions are summarized as follows:
\begin{itemize}[leftmargin=*,nosep]
    \item We demonstrate that commonly used evaluation methods for NLG confidence measures are sensitive to noise in correctness labels, which can lead to misleading conclusions about evaluation metrics and rankings of different confidence estimation approaches.
    %We show that the most popular evaluation methods for NLG confidence measures are prone to noise in the ``correctness label`, which could further lead to problematic conclusions. %(about which measure is better)
    \item We propose a simple yet effective method that utilizes multiple-choice QA datasets to evaluate confidence measures, supporting both internal-states-based white-box and consistency-based black-box methods.
    \item Extensive experiments across recent LLMs and QA datasets verify that \uqeval produces stable evaluations broadly consistent with existing methods, while eliminating the need for expensive correctness functions.
    %We conduct extensive experiments on recent LLMs and popular QA datasets to verify that \uqeval provides stable evaluation results that are often consistent with existing evaluation methods, without the costly correctness function.
\end{itemize}



% Reward hacking is a well-known issue in reinforcement learning, affecting both traditional RL and RLHF in LLMs~\cite{weng2024rewardhack}.
\subsection{Reward Hacking in Traditional RL}  
Reward hacking arises when an RL agent exploits flaws or ambiguities in the reward function to achieve high rewards without performing the intended task~\cite{weng2024rewardhack}. This aligns with Goodhart’s Law: \emph{When a measure becomes a target, it ceases to be a good measure.} For example: 
A bicycle agent rewarded for not falling and moving toward a goal (but not penalized for moving away) learns to circle the goal indefinitely~\cite{Randlv1998LearningTD}.  
A walking agent in the DMControl suite, rewarded for matching a target speed, learns to walk unnaturally using only one leg~\cite{lee2021pebblefeedbackefficientinteractivereinforcement}.  
An RL agent allowed to modify its body grows excessively long legs to fall forward and reach the goal~\cite{Ha2018designrl}.  
In the Elevator Action ALE game, the agent repeatedly kills the first enemy on the first floor to accumulate small rewards~\cite{toromanoff2019deepreinforcementlearningreally}.  
% A robot trained to stay on track learns to reverse along straight paths by alternating left and right turns instead of following curves~\cite{Vamplew2004}.

\citet{amodei2016concrete} propose several potential mitigation strategies to address reward hacking, including
\emph{(1) Adversarial Reward Functions}: Treating the reward function as an adaptive agent capable of responding to new strategies where the model achieves high rewards but receives low human ratings.
\emph{(2) Model Lookahead}: Assigning rewards based on anticipated future states; for example, penalizing the agent with negative rewards if it attempts to modify the reward function~\cite{everitt2016selfmodificationpolicyutilityfunction}.
\emph{(3) Adversarial Blinding}: Restricting the model’s access to specific variables to prevent it from learning information that could facilitate reward hacking~\cite{ajakan2015domainadversarialneuralnetworks}.
\emph{(4) Careful Engineering}: Designing systems to avoid certain types of reward hacking by isolating the agent’s actions from its reward signals, such as through sandboxing techniques~\cite{The_AGI_Containment_Problem}.
\emph{(5) Trip Wires}: Deliberately introducing vulnerabilities into the system and setting up monitoring mechanisms to detect and alert when reward hacking occurs.

\subsection{Reward Hacking in RLHF of LLMs}  
Reward hacking in RLHF for large language models has been extensively studied. \citet{gao2023scaling} systematically investigate the scaling laws of reward hacking in small models, while \citet{wen2024languagemodelslearnmislead} demonstrate that language models can learn to mislead humans through RLHF. Beyond exploiting the training process, reward hacking can also target evaluators. Although using LLMs as judges is a natural choice given their increasing capabilities, this approach is imperfect and can introduce biases. For instance, LLMs may favor their own responses when evaluating outputs from different model families~\cite{liu2024llmsnarcissisticevaluatorsego} or exhibit positional bias when assessing responses in sequence~\cite{wang2023largelanguagemodelsfair}.  

To mitigate reward hacking, several methods have been proposed. Reward ensemble techniques have shown promise in addressing this issue~\cite{Eisenstein2023HelpingOH, Rame2024WARMOT, ahmed2024scalableensemblingmitigatingreward, coste2023reward, zhang2024improvingreinforcementlearninghuman}, and shaping methods have also proven straightforward and effective~\cite{yang2024regularizinghiddenstatesenables, jinnai2024regularizedbestofnsamplingmitigate}. \citet{miao2024informmitigatingrewardhacking} introduce an information bottleneck to filter irrelevant noise, while \citet{moskovitz2023confrontingrewardmodeloveroptimization} employ constrained RLHF to prevent reward over-optimization. \citet{Chen2024ODINDR} propose the ODIN method, which uses a linear layer to separately output quality and length rewards, reducing their correlation through an orthogonal loss function. Similarly,
~\citet{sun2023salmon} train instructable reward models to give a more comprehensive reward signal from multiple objectives. \citet{Dai2023SafeRS} constrain reward magnitudes using regularization terms. ~\citet{liu2024rrmrobustrewardmodel} curate diverse pairwise training data. Additionally, post-processing techniques have been explored, such as the log-sigmoid centering transformation introduced by \citet{Wang2024TransformingAC}.  



\section{Preliminaries}
\label{sec:preliminary}

\begin{figure*} [t]
	\centering
   \vspace{-2ex}
	\subfigure[\dataset]{
		\includegraphics[width=0.225\linewidth]{figures/crop_data_ratio.pdf}
		\label{fig:data-overall}
	} 
     \subfigure[Tool Data]{
		\includegraphics[width=0.225\linewidth]{figures/crop_tool_ratio.pdf}
		\label{fig:data-tool}
	}
    \subfigure[Retrieved Data]{
		\includegraphics[width=0.225\linewidth]{figures/crop_retr_ratio.pdf}
		\label{fig:data-retr}
	}
    \subfigure[t-SNE: Retrieved Data]{
		\includegraphics[width=0.225\linewidth]{figures/tsne-new-v3.pdf}
		\label{fig:tsne-retr}
	}
    % \vspace{-1ex}
	\caption{Data composition of (a) the entire \dataset, (b) seed data collection (\cref{sec:data-phase1}), and (c) retrieved agent data from the open web (\cref{sec:data-phase2}). A t-SNE visualization (d) depicts seed data (\textbf{colorful} points, with each color representing different data sources), retrieved data (\textbf{black}), and general text (\textcolor{gray}{\textbf{gray}}) within the semantic space, where retrieved data is closer to the selected seed data than to the general text. Detailed data sources are in \cref{app:data-pretrain}.
 }
\vspace{-2ex}
\label{fig:data}
\end{figure*}


\noindent \textbf{Problem Formulation.} We conceptualize leveraging LLMs as autonomous agents for problem-solving as a planning process.
Initially, we augment the LLM agent with access to a pool of candidate API functions, denoted as $\mathcal{A}=\{\text{API}_0,\text{API}_1,\cdots,\text{API}_m$\}, along with a natural language task description $g\in\mathcal{G}$ from the task space $\mathcal{G}$. 
The objective of the LLM agent is to translate the task description $g$ into an ordered sequence of $T_g$ API function calls $p_g=\{a_0,\cdots,a_{T_g}\}$.
Specifically, considering the task description $g$ as the initial state $s_0$, we then sample the plan $p_g$ by prompting the LLM agent with the API definitions $\mathcal{I}$ and demonstration samples $\mathcal{D}$ as follows: $p_g\sim\rho(a_0,a_1,\cdots,a_{T_g}|s_0;\mathcal{I},\mathcal{D}):\mathcal{G}\times\mathcal{I}\times\mathcal{D}\to\Delta(\mathcal{A}^{T_g})$, where $\Delta(\cdot)$ denotes a probability simplex function. 
The final output is derived after executing the entire plan $y\sim\pi(y|s_0,a_1,a_2,\cdots,a_{T_g})$, where $\pi(\cdot)$ denotes a plan executor.

During this procedure, we focus on three fundamental capabilities of LLM agents:

\noindent \textbf{Accurate Function Calling.} It involves accurately understanding the API definitions and demonstration samples to generate correct API function calls with corresponding parameters in a given scenario.
Specifically, the model should accurately understand the API definitions $\mathcal{I}$ and demonstration samples $\mathcal{D}$, as well as generate an accurate API function call in the given scenario $p(a_t|s_0,a_1,\cdots,a_{t-1},\mathcal{I},\mathcal{D})$, where $a_t$ is the ground-truth API function call with corresponding parameters at $t$-th step.

\noindent \textbf{Intrinsic Reasoning and Planning.} It refers to the intrinsic reasoning and planning ability to devise a sequence of multiple tool functions as a solution when addressing complex (multi-step) real-world problems. In such cases, LLMs are often required to generate a sequence of API function calls, $p(a_1,a_2,\cdots,a_{T_g}|s_0;\mathcal{I},\mathcal{D})$, where $\{a_1,a_2,\cdots,a_{T_g}\}$ constitutes the ground-truth solution plan of length $T_g$.  
This process relies on intrinsic reasoning embedded within the model parameters; enhanced reasoning capabilities lead to a solution plan with a higher chance of success.


\noindent \textbf{Adaptation with Environment Feedback.} It focuses on adapting the current plan or action based on environmental feedback when the environments support interaction with the LLM agent. When such feedback is available, it is crucial for the agent to adjust its actions accordingly: $p(a_t|s_0,a_1,o_1,a_2,\cdots,o_{t-1};\mathcal{I},\mathcal{D})$,
where $o_{k}$ represents the feedback from the environment after the $k$-th action. 
Incorporating environmental feedback allows the agent to take reflections to refine its plan and improve task performance iteratively.



\section{ \uqeval: A framework for Assessing Confidence Estimation}\label{sec:method}
At a high level, existing evaluation frameworks
% the evaluation pipeline described in \cref{sec:prelim:old_eval} 
for $C_{\mathcal{M}}$ includes three main steps (blue path in \cref{fig:pipeline}):
\begin{enumerate}[nosep]
    \item Generate $\predSeq$ from $\mathcal{M}$ given the input $\xInput_i$.
    \item Determine the correctness label of $\predSeq$ using the function $\acc(\cdot,\xInput)$.
    \item Compute evaluation metrics such as AUROC. A higher metric value indicates that $C_{\mathcal{M}}$ is a ``better'' confidence estimation.
\end{enumerate}
The main limitation of this general pipeline lies in $\acc$ in step 2. 
Existing evaluation frameworks all implicitly assume step 1---that the confidence measure $C_{\mathcal{M}}$ must apply to generated sequences $\predSeq$. 
While this might hold for consistency-based uncertainty measures, where response divergence indicates uncertainty, it does not extend to confidence measures. 
In other words, we could relax step 1 in order to improve step 2.
% Consider, for instance, Eccentricity~\cite{lin2024generating}: The uncertainty measure $U_{ecc}$ computes the average distance between sampled generation embeddings and their centroid, while the confidence measure evaluates a specific generation. 
% Consequently, we could simply use the sampled generations to construct the embedding space, yet we can still measure the distance of \textit{any} sequence to the center of embeddings.


%The previous discussion suggests that existing evaluation frameworks all bear an implicit assumption: The confidence measure $C_{\mathcal{M}}$ must be applied to the generations $\predSeq$.
%While this might be true for the case of most existing consistency-based uncertainty measures, where a high degree of divergence of the sampled responses is a strong indicator of high uncertainty, this is \textit{not} the case for confidence measures.
%Take Eccentricity~\cite{lin2024generating} as an example: The uncertainty measure $U_{ecc}$ is based on the average distance of the embeddings of the sampled generations to the center of all embeddings and the confidence measure is that of a particular generation. 
%Consequently, we could simply use the sampled generations to construct the embedding space, yet we can still measure the distance of \textit{any} sequence to the center of embeddings. 

\textit{Our main proposal in this paper is to adapt multiple-choice datasets to evaluate confidence measures designed for free-form NLG.}
Unlike free-form NLG datasets, multiple-choice datasets provide inherent correctness values for options, eliminating the need for an explicit correctness function. 
If we simply ``pretend'' that these options are free-form generations from the base LM, we can directly evaluate the confidence measure quality. 
As \cref{fig:pipeline} shows, the approach differs from existing evaluation pipelines only in applying confidence estimation methods to multiple-choice options.




Consider the QASC~\cite{khot2020qasc} dataset as an example,
each problem comes with a question $\xInput$ and a few choices, $o_1,\ldots,o_K$. 
Unlike what such datasets were designed for, we re-format the input prompt as a free-form NLG question, as illustrated in \cref{fig:qasc_example}, as if the base LLM generated each option itself, in different runs.
In what follows, we first explain explain slight nuances in applying internal state-based white-box confidence measures as well as consistency-based black-box ones. 
%shows a reformatted question from the QASC dataset.

\begin{figure}[t]
  \includegraphics[width=\columnwidth]{figures/qasc_example.pdf}
  % \caption{A reformatted question example from the QASC dataset. The Question and Choices are directly from the original dataset, while our prompt is specifically designed for LLM input to generate open-form responses.}
  \caption{
  We reformat each option from the multiple-choice question (left), by injecting the \smash{\colorbox{yellow!40}{{{\color{blue}option}}}} to a free-form QA \smash{\colorbox{green!40}{prompt}}.
  One could typically apply any confidence estimation method by treating this \smash{\colorbox{yellow!40}{{{\color{blue}option}}}} as if it was generated by the base LM.
  For black-box confidence measures that require additional responses, we only feed the \smash{\colorbox{green!40}{prompt}} to the base LM.
  }
  \label{fig:qasc_example}
\vspace{-3mm}
\end{figure}



\textbf{Logit or Internal State-Based Measures} typically examine the internals of a LM when it generates a particular response.
The nature of the free-form generation task allows us to simply plug-in the option $o_i$ into the corresponding location of the prompt, and extract similar information that allows us to evaluate the confidence\footnote{In fact, this was the practice to compute \baselineSL for actual generations. For example, \url{https://github.com/lorenzkuhn/semantic_uncertainty/blob/main/code/get_likelihoods.py} and \url{https://huggingface.co/docs/transformers/perplexity}.}.
% One concern is whether these options are too ``different'' from what the LM would otherwise generate itself.
% As exemplified in \cref{fig:true_distribution}, $C(o_i)$ in general shares a similar distribution to $C(\predSeq_i)$. 

% Taking CommonsenseQA~\footnote{A multiple-choice dataset for our experiment is described in Section~\ref{sec:experiments}} as an example, we compare the logit distribution of the correct answer choices with the distributions of other LLM-generated responses. 
% As shown in \cref{fig:true_distribution}, the distributions exhibit notable similarities, indicating that logit-based confidence estimation can capture underlying patterns shared between correct answer choices and free-form generations.

% \begin{figure}[t]
%   \includegraphics[width=\columnwidth]{figures/true_distribution_new.png}
%   \caption{Confidence score distribution for the \baselinePTrue method on the CQA dataset. The blue distribution represents 20 open-form responses, while the red distribution corresponds to the correct option. }
%   \label{fig:true_distribution}
% \end{figure}


\textbf{Consistency-based Confidence Measures}
Unlike logit-based or internal-state-based measures, consistency-based confidence measures typically rely on an estimate of the predictive distribution, denoted as $\PredDist$, and any response that is closer to the center of the distribution (in the ``semantic space'') is considered to be of higher confidence. 
Consider methods from~\citet{lin2024generating} as an example. To preserve the integrity of the predictive distribution, we first sample $n$ responses from $\PredDist$ as usual, and then iteratively include one option $o_i$ at a time to compute its associated confidence score~\cite{rivera-etal-2024-combining,manakul-etal-2023-selfcheckgpt}. 
\cref{alg:confidence_score} outlines this process. 


\renewcommand{\algorithmicrequire}{\textbf{Input:}}
\renewcommand{\algorithmicensure}{\textbf{Output:}}

\begin{algorithm}[t]
\small
% \caption{Confidence Score Computation in the Black-Box Method}
\caption{Consistency-based Confidence Estimation for Any Sequences}
\label{alg:confidence_score}
\begin{algorithmic}[1]
    \Require $\xInput$, $\mathcal{M}$, candidate sequences $A = \{a_1, \dots, a_K\}$
    \Ensure $\{C_{\mathcal{M}}(\xInput,a_1), \dots,C_{\mathcal{M}}(\xInput,a_K)\}$ 
    
    \State Generate $S = \{\predSeq_1, \dots, \predSeq_{n}\}$ using $\mathcal{M}$ for question $\xInput$
    \State Compute pairwise similarity matrix $M$ of $S$.% \in \mathbb{R}^{|S'| \times |S'|}$
    % \State Construct full response set $S' = S \cup A$, where $|S'| = n+K$
    % \State Compute pairwise similarity matrix $M_{sim} \in \mathbb{R}^{|S'| \times |S'|}$
    
    \For{each $a_i \in A$}
        \State Compute a new similarity matrix $M_i$ of $S\cup\{a_i\}$, reusing $M$. % \in 
        % \State Form subset $S_i = S \cup \{a_i\}$, where $|S_i| = n+1$
        % \State Extract pairwise similarity matrix $M_{sim}^{(i)} \in \mathbb{R}^{|S_i| \times |S_i|}$
        \State Compute confidence score $C_{\mathcal{M}}(\xInput,a_i)$ using $M_i$. %degree matrix or eccentricity of $M_{sim}^{(i)}$
    \EndFor

    \State \Return $\{C_{\mathcal{M}}(\xInput,a_1), \dots,C_{\mathcal{M}}(\xInput,a_K)\}$ 
\end{algorithmic}
\end{algorithm}

% Since computing the similarity matrix is the most computationally expensive step, the subsequent 5 confidence score calculations reuse precomputed similarity values. As a result, the additional computations take less than 1 minute in total, ensuring efficiency.

% Consistency-based confidence measures are a little different from .


\paragraph{Remarks}
Our proposal relaxes step 1 at the beginning of this section, allowing for $\predSeq^*=o_i$ not sampled from $\PredDist$.
This is not to be misunderstood as a proposal to \textit{replace} the current pipeline (\cref{sec:prelim:old_eval})---rather, it is \textit{complementary}.
The rationale is that if a good confidence measure predicts the correctness well, it should perform well in \textit{both} evaluation frameworks.
In fact, any $o_i\in\Sigma^*$ that does not violate the generation configuration, has a non-zero probability to be sampled from $\PredDist$, and a robust confidence measure should be expected to model it well.
% \fontred{
% In fact, any $o_i$, as long as it does not violate the generation config, could be sampled from $\PredDist$ given enough time.
% % As we will see in \cref{sec:exp}, even though $o_i$ are not sampled from $\PredDist$, we do not observe a big distribution shift in terms of the confidence values as well.
% }
% Note that we do not advise \textit{replacing} the existing valuation 

% It is important to note that our method only obtains correctness labels for $o_i$. Consequently, when computing AUROC, AUARC, and other evaluation metrics, we only consider confidence values associated with these options.\textcolor{red}{already mention it in \cref{sec:metrics} }


\section{Experiments}

We conduct comprehensive experiments across multiple datasets and model architectures to validate our method's ability to decouple explanation robustness from classification robustness. Our evaluation addresses three key research questions:
\begin{itemize}
    \item \textbf{RQ1:} Does \ours have better quantify uncertainties?
    \item \textbf{RQ2:} How do different ensemble methods and information from both dimensions help?
    \item \textbf{RQ3:} Is\ours robust to different settings? 
\end{itemize}


\begin{table}[H]
\centering
\resizebox{!}{0.11\textwidth}{
\begin{tabular}{@{}lc@{}}
\toprule
\textbf{Measure} & \textbf{Details} \\ 
\midrule
$U_{\textit{Eigv}}(Dis)$ & \multicolumn{1}{c}{Spectral eigenvalue on the disagreement.} \\ 
$U_{\textit{Ecc}}(Dis)$ & \multicolumn{1}{c}{Average distance in responses' disagreement.} \\ 
$U_{\textit{Degree}}(Dis)$ & \multicolumn{1}{c}{Degree of disagreement similarity Matrix.} \\ 
$U_{\textit{Eigv}}(Agre)$ & \multicolumn{1}{c}{Spectral eigenvalue on the agreement.} \\ 
$U_{\textit{Ecc}}(Agre)$ & \multicolumn{1}{c}{Average distance in responses' agreement.} \\ 
$U_{\textit{Degree}}(Agre)$ & \multicolumn{1}{c}{Degree Matrix of agreement Matrix.} \\ 
$p(true)$ & \multicolumn{1}{c}{Entropy of knowledge dimension responses} \\ 
$D-UE$ & \multicolumn{1}{c}{eigenvalue from Laplacian of a directional graph} \\ 

\bottomrule
\end{tabular}}
\vspace{-1mm}
\caption{The baseline methods and explanations.}
\vspace{-5mm}
\label{tab:baslines}
\end{table}

\subsection{Experimental Setup}
\label{sec:setup}
\subsubsection{Datasets} As mentioned in \cref{sec:background}, following prior works~\cite{lin2022teaching}, we focus on open-form question-answering 
(QA) tasks in this paper. We adopt 4 different classic QA datasets. Coqa~\cite{reddy2019coqa} is a conversational question-answering dataset that contains dialogues with free-form answers grounded in diverse passages, which is the easiest dataset among all datasets. HotpotQA~\cite{yang2018hotpotqa} is a multi-hop QA dataset that demands reasoning over multiple Wikipedia paragraphs to derive correct answers. NQ-Open~\cite{kwiatkowski2019natural} consists of real-world queries from Google Search, requiring models to retrieve and answer questions without explicit context, which is the hardest dataset. 
\subsubsection{Models to Evaluate} We evaluate \ours on Llama family~\cite{touvron2023llama}, which is the one of the most popular LLMs. In detail, we use Llama-2-13b and Llama-2-7B to demonstrate the effectiveness of \ours with different model sizes and use Llama-3.1-8B~\cite{dubey2024llama} to that \ours could also work on the different version of Llama. To further demonstrate the generalization ability for other architectures,  we also use Phi4~\cite{abdin2024phi} and Deepseek-R1-distill-7B~\cite{guo2025deepseek} in our paper.


%\textcolor{red}{Not sure whether there is a section labeled as "sec eva metric" refered by Sec. }

\subsubsection{Evaluation Metrics} Effective uncertainty measures should accurately represent the reliability of LLM responses, with higher uncertainty more likely leading to incorrect generations and vice versa~\cite{lin2023generating,kuhn2023semantic}. Following prior works~\cite{lin2023generating,da2024llm}, we mainly use UQ values to predict whether an answer is correct or not. Following prior works~\cite{lin2023generating,da2024llm}, we will use Area Under Receiver Operating Characteristic (AUROC) and Area Under Accuracy Rejection Curve (AUARC) as evaluation metrics, where a higher AUROC or AUARC demonstrates better uncertainty measures. To compute AUROC and AUARC, the accuracy of each original response is required. Following previous works~\cite{da2024llm,lin2023generating}, we use another LLM to provide correctness from 0-100 to each response. If the correctness is greater than 70, we label the response as correct. In this paper, we use Qwen-34B~\cite{bai2023qwen} to evaluate the correctness.

\subsubsection{Knowledge Extracted Models} In this paper, we mainly use llama-2-13b~\cite{touvron2023llama} as the auxiliary models to extract the knowledge dimension of responses. To demonstrate the robustness of \ours with different knowledge-extracted models, we also contain the results for different LLMs as knowledge-extracted models.

\subsubsection{Baselines} In this paper, we compared \ours with baselines that use semantic dimension response and knowledge dimension response. For semantic dimension, we mainly compared with methods that come from \citet{lin2023generating}. In detail, we incorporate six distinct methods from \citet{lin2023generating}, which differ based on the operations applied after computing similarity and whether they utilize agreement (entailment) probabilities or disagreement (contradiction) logits to construct the similarity matrix. For knowledge dimension, we use D-UE~\cite{da2024llm} and $p(true)$~\cite{kadavath2022language} as the baselines. Note that we use $p(true)$ on the knowledge dimension of response. We show the detailed explanations of all baselines in \cref{tab:baslines}

\begin{figure*}[t]
\centering
\begin{minipage}[t]{0.32\linewidth}
  \centering
  \includegraphics[width=\linewidth,trim=0 0 0 1cm, clip]{images/ablation.pdf}
  \captionof{figure}{Ablation studies that show that \ours fully utilizes all the information from both dimensions.}
  \label{fig:ablation}
\end{minipage}\hfill
\begin{minipage}[t]{0.32\linewidth}
  \centering
  \includegraphics[width=\linewidth]{images/knowledge_extract.pdf}
  \captionof{figure}{Performance for different knowledge extract models on CoQA and NQ\_Open with llama3.1.}
  \label{fig:knowledge_extract}
\end{minipage}\hfill
\begin{minipage}[t]{0.32\linewidth}
  \centering
  \includegraphics[width=\linewidth]{images/Jacc.pdf}
  \captionof{figure}{Performance that uses Jaccard similarity on CoQA and NQ\_Open with llama3.1.}
  \label{fig:jacc}
\end{minipage}
\end{figure*}

\subsection{Does \ours have better quantify uncertainties? (RQ1)}
\label{sec:main_result}
In this section, we explore whether \ours has better uncertainties compared with state-of-the-art uncertainty quantification methods. In \cref{tab:main_results}, we compare \ours with 8 baselines across three different datasets and five different models as introduced in \cref{sec:setup} In detail, we have the following observations:

\noindent $\bullet$ Compared with all baseline methods, \ours achieves the best performance overall. Especially when we consider AUROC. For AUARC, \ours achieves the best performance for NQ\_Open while \ours also achieves the comparable performance for CoQA in most scenarios. These results demonstrate that \ours has better quantify uncertainties overall. \\
\noindent $\bullet$ Among all datasets, \ours achieves the highest performance improvement on NQ\_Open, which is the most difficult dataset among all datasets and may lose to baselines for an easier dataset like CoQA. This indicates \ours could perform even better when the task is harder, where uncertainty quantification is more important. \\
\noindent $\bullet$ Two different ensemble methods show very similar results. Min strategy performs better than the sum strategy under $61.51\%$ situations, indicating that difficult datasets might also have more complex structures that single one tensor decomposition might oversight some information while using min structure could reduce such oversight by considering the best cases. However, both ensemble methods show a better performance than all baselines, which proves the effectiveness of tensor decomposition. \\

From these results, we get a conclusion that overall, \ours have better uncertainties.


\subsection{How do different ensemble methods and information from both dimensions help? (RQ2)}
\label{sec:ablation}
In this section, we use more experiments to prove the necessity of using information from both semantic and knowledge dimensions as well as using tensor decomposition. In detail, we consider the following methods: 1) \ours with only semantic responses, 2) \ours with only knowledge responses and 3) Concatenating similarity matrices from semantic and knowledge dimensions into a 2D matrix and applying SVD, 4) only using one tensor decomposition. In \cref{fig:ablation}, we show the comparison between \ours and other methods.  The results show that \ours consistently outperforms its variants and SVD method that repeated information will dominate the features, showing the effectiveness of our framework.




\subsection{Is \ours robust to different settings? (RQ3)}
\subsubsection{Different Knowledge Extracted Models} Knowledge extracted models influence the claim extraction in \ours as stated in \cref{subsubsec:knowledge}. Therefore, in this section, We test the robustness of \ours on various knowledge extracted models. unlike using llama2-13b in \cref{sec:main_result} and \cref{sec:ablation}, we conduct experiments on CoQA and NQ\_open using llama2-7b and llama3.1 as the knowledge extracted models, We show the results in \cref{fig:knowledge_extract}. From the figure, we can see that using Phi4 could even achieve a better result, indicating \ours has more potential with the development of LLMs. 

\subsubsection{Different Accuracy Thresholds} Different accuracy thresholds lead to different accuracy and influence the evaluation of uncertainties. In the previous experiments, we all set the accuracy threshold to 70 as mentioned in \cref{sec:setup}.  To test the robustness of \ours under different accuracy thresholds, we choose an extra dataset TriviaQA~\cite{joshi2017triviaqa}, which is considered the easiest dataset, and NQ\_Open, which is the most challenging dataset in our paper to conduct experiments. We show the results with accuracy thresholds of 70 and 90 in \cref{tab:Accuracy_threshold}. From the results, we can see that increasing the accuracy threshold decreases the performance of all baselines while the performance of \ours could even increase for datasets with different difficulties, showing the robustness of \ours in different settings. 

\subsubsection{Different Similarity Metrics} Finally, different similarity metrics lead to different similarity matrices. Therefore, to test whether \ours also has a good performance for different similarities, we use Jaccard similarity instead of using an NLI model in this section and the results are presented in \cref{fig:jacc}. The results show that using Jaccard similarity will boost the performance for a simple dataset like CoQA but hurt the performance for a difficult dataset like NQ\_Open. This is because the answer to a simple question might not have a deeper semantic meaning that requires NLI models. However, \ours can still outperform baseline methods that also use Jaccard similarity, showing the robustness of \ours.





\begin{table}[h]
    \centering
    \resizebox{0.5\textwidth}{!}{
    \begin{tabular}{lcccc}
        \toprule
        \multirow{2}{*}{Methods} & \multicolumn{2}{c}{Accuracy Threshold: 0.7} & \multicolumn{2}{c}{Accuracy Threshold: 0.9} \\
        \cmidrule(lr){2-3} \cmidrule(lr){4-5}
        & AUROC & AUARC & AUROC & AUARC \\
        \midrule
        \multicolumn{5}{c}{\textbf{Dataset: TriviaQA} [Easy]} \\
        \midrule
        Eigv(Dis) & 0.8261 & 0.8094 & 0.8100& 0.7604\\
        Ecc(Dis) & 0.8063& 0.7940&0.7892 & 0.7415\\
        Degree(Dis) &0.8399 & 0.8163&0.8259 & 0.7694\\
        Eigv(Agre) &0.8436 &0.8116 &0.8351 & 0.7721 \\
        Ecc(Agre) & \textbf{0.8510}&0.8189 & 0.8374&0.7721 \\
        Degree(Agre) &0.8396 &\textbf{0.8397} &0.8384 & 0.7739\\
        \ours-Sum &0.8428 &0.8144 & 0.8438&0.7749 \\
        \ours-Min &0.8431 &0.8149 & \textbf{0.8440} & \textbf{0.7754}\\
        \midrule
        \multicolumn{5}{c}{\textbf{Dataset: NQ\_Open} [Hard]} \\
        \midrule
        Eigv(Dis) & 0.6162 & 0.7300 &0.5636 &0.6017 \\
        Ecc(Dis) & 0.6210& 0.7330& 0.5658&0.5941 \\
        Degree(Dis) &0.6130 & 0.7168&0.5662 &0.6033 \\
        Eigv(Agre) &0.6258 &0.7276 & 0.6146& 0.6290 \\
        Ecc(Agre) & 0.6273&0.7311 &0.6239 &0.6344\\
        Degree(Agre) &0.6286 &0.7355 & 0.6221&0.6299 \\
        \ours-Sum &\textbf{0.6334} &\textbf{0.7410} &\textbf{0.6351} &\textbf{0.6430} \\
        \ours-Min &0.6332 &0.7409 & 0.6350 &0.6429 \\
        \bottomrule
    \end{tabular}
    }
    \caption{Comparison of different methods across different accuracy thresholds on TrivialQA and NQ\_Open with llama2-13B. The results show that our methods outperform baselines after increasing the accuracy threshold, indicating that our methods have an advantage on more difficult datasets.}
    \vspace{-7mm}
    \label{tab:Accuracy_threshold}
\end{table}



\section{Conclusion}
In this paper, we propose ChineseEcomQA, a scalable question-answering benchmark designed to rigorously assess LLMs on fundamental e-commerce concepts. ChineseEcomQA is characterized by three core features: Focus on Fundamental Concept, E-Commerce Generalizability, and Domain-Specific Expertise, which collectively enable systematic evaluation of LLMs' e-commerce knowledge. Leveraging ChineseEcomQA, we conduct extensive evaluations on mainstream LLMs, yielding critical insights into their capabilities and limitations. Our findings not only highlight performance disparities across models but also delineate actionable directions for advancing LLM applications in the e-commerce domain.

\bibliography{main}

\cleardoublepage

\appendix
% \section{Notation Table}

\def \TabNotation{
\begin{table}[]
\centering
\resizebox{!}{0.18\textwidth}{
\begin{tabular}{@{}lc@{}}
\toprule
\textbf{Notation} & \textbf{Explanation} \\ 
\midrule
$\mathcal{M}$ & Language model \\ 
$\Sigma$ & Vocabulary \\
$\xInput$ & Input prompt \\ 
$\predSeq$ & Output responds \\ 
$A= \{ a_i,...a_m\}$ & Set of reference answers \\ 
$C_{\mathcal{M}}(\xInput,\predSeq)$ & Confidence score of $\predSeq$ \\ 
$f(\predSeq,\xInput)$ & Correctness function \\
$sim(\predSeq,A)$ & Similarity score \\
$\tau$ & LLM predefined threshold \\
$n$ & Number of open-form responses \\
$o_i$ & Options in QA dataset \\
$K$ & Number of options \\
\bottomrule
\end{tabular}}
\vspace{-1mm}
\caption{The notation used in this paper}
\vspace{-5mm}
\label{tab:notations}
\end{table}
}
%\TabNotation

% \section{Example Appendix}
% \label{sec:appendix}

\section{Experiments Details}\label{appendix:sec:exp_imp}

\subsection{Dataset Description}\label{sec:datasetDes}


\begin{itemize}
    \item \textbf{C-QA} A multiple-choice dataset designed for commonsense question answering. Each question requires world knowledge and reasoning to determine the correct answer from 5 given choices. The dataset consists of 1,221 test questions.
    
    \item \textbf{QASC} A multiple-choice commonsense reasoning dataset with 8 answer choices per question. Compared to C-QA, QASC presents a higher level of difficulty. While the dataset was originally designed for multi-hop reasoning, our focus is not on evaluating the reasoning capabilities of LLMs. Therefore, we do not provide the supporting facts to the model and instead present only the question. For our experiments, we use the original validation set, which includes 926 questions.
    
    \item \textbf{MedQA} A multiple-choice dataset with 5 options for answers, specifically designed for medical QA. 
    It covers three languages: English, simplified Chinese, and traditional Chinese, and contains 12,723, 34,251, and 14,123 questions for the three languages, respectively.
    The questions are sourced from professional medical board exams, making this dataset particularly challenging due to its reliance on specialized medical knowledge. 
    For our experiments, we randomly selected the first 1,000 questions from the English dataset.
    
    \item \textbf{RACE-m and RACE-h} used in this paper are derived from the RACE (\textbf{R}e\textbf{A}ding \textbf{C}omprehension dataset from \textbf{E}xaminations) dataset, a large-scale machine reading comprehension dataset introduced by Lai et al~\cite{lai2017race}. 
    RACE comprises 27,933 passages and 97,867 questions collected from English examinations for Chinese students aged 12–18. 
    These datasets evaluate a model’s ability to comprehend complex passages and answer questions based on contextual reasoning. 
    Each question is accompanied by four answer choices, with only one correct option. 
    For our experiments, we randomly sampled 1,000 questions from the entire dataset using a fixed random seed of 42 to ensure reproducibility.
\end{itemize}

% \textbf{RACE datasets:} The RACE-h and RACE-m datasets used in this paper are derived from the RACE (\textbf{R}e\textbf{A}ding \textbf{C}omprehension dataset from \textbf{E}xaminations) dataset, a large-scale machine reading comprehension dataset introduced by Lai et al~\cite{lai2017race}. 
% RACE comprises 27,933 passages and 97,867 questions collected from English examinations for Chinese students aged 12–18. 
% The dataset is split into two subsets: RACE-M, which includes 28,293 questions from middle school exams, and RACE-H, containing 69,574 questions from high school exams. Each question in the dataset is paired with four candidate answers, only one of which is correct. Unlike other machine reading comprehension datasets generated through heuristics or crowdsourcing, RACE's questions are designed by domain experts to test human reading and comprehension skills, making it a unique resource for evaluating large language understanding of models. For our specific study, since collecting responses and conducting evaluations is relatively time-consuming, so we conducted a random sample of 1,000 questions extracted from the entire dataset using a random seed of 42 to ensure reproducibility.


% \begin{table}[t!]
%     \centering
%     \begin{tabular}{lcc}
%         \toprule
%         Method & Description & \\
%         \midrule
%         \multicolumn{3}{c}{\textbf{Black-Box Methods}} \\
%         \midrule
%         Ecc(C) & \multicolumn{1}{c}{..} \\ 
%         Deg(C) & \multicolumn{1}{c}{..} \\ 
%         Ecc(E) & \multicolumn{1}{c}{..} \\ 
%         Deg(E) & \multicolumn{1}{c}{..} \\ 
%         Ecc(J) & \multicolumn{1}{c}{..} \\ 
%         Deg(J) & \multicolumn{1}{c}{..} \\ 
%         \midrule
%         \multicolumn{3}{c}{\textbf{White-Box Methods}} \\
%         \midrule
%         P(true) & \multicolumn{1}{c}{..} \\ 
%         CSL & \multicolumn{1}{c}{..} \\ 
%         CSL-next & \multicolumn{1}{c}{..} \\ 
%         SL & \multicolumn{1}{c}{..} \\ 
%         SL(norm) & \multicolumn{1}{c}{..} \\ 
%         TokenSAR & \multicolumn{1}{c}{..} \\ 
%         \bottomrule
%     \end{tabular}
%     \caption{All the baseline methods}
%     \vspace{-5mm}
%     \label{tab:similarity_matrix_stat}
% \end{table}
%\section{Implement Confidence Estimation Methods}


\subsection{Prompt Details}
\label{sec:appendix_prompt}
\begin{itemize}
    \item We use the following prompt to collect open-form responses for each of the 5 datasets separately.
    
\includegraphics[width=.9\columnwidth]{figures/generate_prompt.png}

    \item We use the following prompt to elicit P(True) confidence score.
    The ``Possible Answer'' is an option from the multiple-choice dataset.
    
\includegraphics[width=.9\columnwidth]{figures/ptrue_prompt.png}
\end{itemize}






\subsection{Computation Resources}
To efficiently process multiple queries, we used vLLM~\cite{kwon2023efficientvllm} for parallel inference.
All experiments were conducted on a Linux server running Ubuntu, equipped with an A100 80GB GPU.


\subsection{Response Generation }
For black-box methods, we mostly adopt the experimental configurations from~\citet{lin2024generating}. 
Sampling-based black-box confidence measures use $n=20$ open-form responses per question. 
The temperature settings for different LLMs are kept at their default values.


\section{Additional Experiments Results}\label{sec:full_results}

\subsection{Full Results of Different Evaluation Metrics}
In the main text, due to space constraints, we only show a subset of the AUROC results.
Here, \cref{appendix:tab:bb:auc,appendix:tab:wb:auc} show the AUROC and AUARC for black-box and white-box confidence measures, respectively. 
Similarly, \cref{appendix:tab:bb:calib,appendix:tab:wb:calib} present RCE and ECE results.
Note that all ECE are based on \textit{calibrated} confidence measures for fair comparisons, as some original confidence measures are not even constrained to $[0,1]$.
For the calibration step, we applied histogram binning method~\cite{KDD_HistogramBinning} on all methods.%, and compute the adaptive calibration error (ACE)~\cite{Nixon_2019_CVPR_Workshops}.


% full table
%%%%%%%%%%%%%%%%%%%%%%%%%%%%%%%%%%%%%%%%%%%%%%%%%%%%%%%%%%%%%%%%%%%%%%%%%%%%%%%%%%%%%%%%%%%%%%%%%%%%%%%%%%%%
%white
%%%%%%%%%%%%%%%%%%%%%%%%%%%%%%%%%%%%%%%%%%%%%%%%%%%%%%%%%%%%%%%%%%%%%%%%%%%%%%%%%%%%%%%%%%%%%%%%%%%%%%%%%%%%%%%%
\begin{table*}[h!]
\centering
\resizebox{\textwidth}{!}{%
\begin{tabular}{llcccccccccccc}
\toprule
\multirow{2}{*}{\textbf{Dataset}} & \multirow{2}{*}{\textbf{Model}} & \multicolumn{6}{c}{\textbf{AUROC $\Uparrow$}} & \multicolumn{6}{c}{\textbf{AUARC} $\Uparrow$} \\ 
\cmidrule(lr){3-8} \cmidrule(lr){9-14}
 &  & Ecc(C) & Deg(C) & Ecc(E) & Deg(E) & Ecc(J) & Deg(J) & Ecc(C) & Deg(C) & Ecc(E) & Deg(E) & Ecc(J) & Deg(J) \\ 
\midrule
\multirow{4}{*}{C-QA}
 & Llama2-7b   & 60.981 & 66.651 & 78.629 & 72.771 & 67.081 & 71.668 & 29.386 & 33.266 & 38.221 & 35.858 & 34.681 & 36.915 \\
 & Llama3-8b   & 57.590 & 62.592 & 80.004 & 73.734 & 65.886 & 76.583 &32.062 & 33.232 & 38.414 & 32.648 & 37.150 & 38.596\\
 & Phi4        & 67.879 & 68.413 & 80.712 & 69.976 & 71.447 & 75.278 & 32.123 & 31.596 & 19.294 & 30.032 & 28.570 & 24.739 \\
 & Qwen2.5-32b & 71.409 & 73.931 & 81.885 & 77.087 & 69.473 & 74.645 & 34.775 & 37.399 & 39.926 & 37.964 & 36.776 & 38.808 \\
\midrule
\multirow{4}{*}{QASC}
 & Llama2-7b   & 58.949 & 61.978 & 73.221 & 69.200 & 61.659 & 66.877 & 17.509 & 19.628 & 25.724 & 23.556 & 21.469 & 23.251 \\
 & Llama3-8b   & 55.121 & 55.446 & 74.912 & 72.033 & 64.124 & 72.657 & 15.785 & 15.952 & 25.199 & 24.163 & 23.198 & 25.786 \\
 & Phi4        & 65.100 & 65.553 & 76.980 & 67.692 & 68.496 & 71.209 & 20.297 & 21.063 & 26.740 & 21.422 & 24.067 & 24.308 \\
 & Qwen2.5-32b & 62.218 & 61.611 & 74.546 & 71.702 & 64.658 & 69.131 & 19.522 & 19.830 & 25.695 & 24.306 & 23.182 & 24.510  \\
\midrule
\multirow{4}{*}{MedQA}
 & Llama2-7b   & 53.683 & 54.129 & 52.076 & 52.963 & 53.137 & 53.778 & 21.956 & 23.105 & 21.160 & 22.863 & 23.454 & 23.371 \\
 & Llama3-8b   & 52.824 & 53.971 & 51.641 & 53.523 & 55.257 & 59.552 & 21.125 & 22.103 & 20.390 & 22.164 & 25.598 & 26.617 \\
 & Phi4        & 60.055 & 59.512 & 54.945 & 55.261 & 57.815 & 65.067 & 25.081 & 25.410 & 22.077 & 22.940 & 27.573 &29.201 \\
 & Qwen2.5-32b & 60.071 & 61.737 & 54.727 & 58.454 & 61.564 & 63.783 & 24.998 & 28.045 & 22.246 & 26.331 & 29.848 & 30.054 \\
\midrule
\multirow{4}{*}{RACE-m}
 & Llama2-7b  & 65.473 & 64.304 & 61.022 & 59.245 & 67.480 & 67.760 & 34.147 & 36.637 & 32.570 & 33.994 & 38.844 & 38.904 \\
 & Llama3-8b & 62.385 & 63.351 & 61.872 & 58.711 & 68.391 & 73.267 & 30.774 & 35.054 & 31.639 & 32.491 & 41.231 & 43.055 \\
 & Phi4     & 66.461 & 64.344 & 64.492 & 58.981 & 68.124 & 72.304 & 34.312 & 35.355 & 32.903 & 32.232 & 41.311 & 41.895  \\
 & Qwen2.5-32b & 65.425 & 67.627 & 60.268 & 61.309 & 75.420 & 75.746 & 34.393 & 37.409 & 32.092 & 34.850 & 44.281 & 44.585 \\
\midrule
\multirow{4}{*}{RACE-h}
 & Llama2-7b    & 58.991 & 53.597 & 57.178 & 54.037 & 59.300 & 59.856 & 34.147 & 36.637 & 32.570 & 33.994 & 38.844 & 38.904 \\
 & Llama3-8b  & 56.372 & 53.560 & 58.456 & 54.004 & 57.488 & 63.788 & 27.959 & 28.483 & 29.120 & 27.823 & 33.912 & 36.139 \\
 & Phi4    & 60.550 & 53.867 & 61.263 & 54.442 & 59.639 & 64.385 & 30.733 & 28.641 & 31.411 & 28.157 & 34.519 & 35.710    \\
 & Qwen2.5-32b   & 60.012 & 54.781 & 55.984 & 55.657 & 64.985 & 66.130 & 31.049 & 29.180 & 30.459 & 28.921 & 37.620 & 37.734 \\
\bottomrule
\end{tabular}%
}
\caption{AUROC and AUARC for black-box methods, across different models and datasets}
\label{appendix:tab:bb:auc}
\end{table*}


% only roc
% \begin{table*}[h!]
% \centering
% \resizebox{0.5\textwidth}{!}{%
% \begin{tabular}{llcccccc}
% \toprule
% \multirow{2}{*}{\textbf{Dataset}} & \multirow{2}{*}{\textbf{Model}} & \multicolumn{6}{c}{\textbf{AUROC $\Uparrow$}} \\
% \cmidrule(lr){3-8}
%  &  & Ecc(C) & Deg(C) & Ecc(E) & Deg(E) & Ecc(J) & Deg(J) \\
% \midrule
% \multirow{4}{*}{C-QA}
%  & Llama2-7b   & 60.981 & 66.651 & 78.629 & 72.771 & 67.081 & 71.668 \\
%  & Llama3-8b   & 57.590 & 62.592 & 80.004 & 73.734 & 65.886 & 76.583 \\
%  & Phi4        & 67.879 & 68.413 & 80.712 & 69.976 & 71.447 & 75.278 \\
%  & Qwen2.5-32b & 71.409 & 73.931 & 81.885 & 77.087 & 69.473 & 74.645 \\
% \midrule
% \multirow{4}{*}{QASC}
%  & Llama2-7b   & 58.949 & 61.978 & 73.221 & 69.200 & 61.659 & 66.877 \\
%  & Llama3-8b   & 55.121 & 55.446 & 74.912 & 72.033 & 64.124 & 72.657 \\
%  & Phi4        & 65.100 & 65.553 & 76.980 & 67.692 & 68.496 & 71.209 \\
%  & Qwen2.5-32b & 62.218 & 61.611 & 74.546 & 71.702 & 64.658 & 69.131 \\
% \midrule
% \multirow{4}{*}{MedQA}
%  & Llama2-7b   & 53.683 & 54.129 & 52.076 & 52.963 & 53.137 & 53.778 \\
%  & Llama3-8b   & 52.824 & 53.971 & 51.641 & 53.523 & 55.257 & 59.552 \\
%  & Phi4        & 60.055 & 59.512 & 54.945 & 55.261 & 57.815 & 65.067 \\
%  & Qwen2.5-32b & 60.071 & 61.737 & 54.727 & 58.454 & 61.564 & 63.783 \\
% \midrule
% \multirow{4}{*}{RACE-m}
%  & Llama2-7b   & 65.473 & 64.304 & 61.022 & 59.245 & 67.480 & 67.760 \\
%  & Llama3-8b   & 62.385 & 63.351 & 61.872 & 58.711 & 68.391 & 73.267 \\
%  & Phi4        & 66.461 & 64.344 & 64.492 & 58.981 & 68.124 & 72.304 \\
%  & Qwen2.5-32b & 65.425 & 67.627 & 60.268 & 61.309 & 75.420 & 75.746 \\
% \midrule
% \multirow{4}{*}{RACE-h}
%  & Llama2-7b   & 58.991 & 53.597 & 57.178 & 54.037 & 59.300 & 59.856 \\
%  & Llama3-8b   & 56.372 & 53.560 & 58.456 & 54.004 & 57.488 & 63.788 \\
%  & Phi4        & 60.550 & 53.867 & 61.263 & 54.442 & 59.639 & 64.385 \\
%  & Qwen2.5-32b & 60.012 & 54.781 & 55.984 & 55.657 & 64.985 & 66.130 \\
% \bottomrule
% \end{tabular}%
% }
% \caption{Black Box Methods Performance Metrics Across Different Models and Datasets (AUROC)}
% \label{tab:metrics_table}
% \end{table*}



% \begin{table*}[h!]
% \centering

%%%%%%%%%%%%%%%%%%%%%%%%%%%%%%%%%%%%%%%%%%%%%%%%%%%%%%%%%%%%%%%%%%%%%%%%%%%%%%%%%%%%%%%%%%%%%%%%%%%%%%%%%%%%
%black
%%%%%%%%%%%%%%%%%%%%%%%%%%%%%%%%%%%%%%%%%%%%%%%%%%%%%%%%%%%%%%%%%%%%%%%%%%%%%%%%%%%%%%%%%%%%%%%%%%%%%%%%%%%%%%%%


\begin{table*}[h!]
\centering
\resizebox{\textwidth}{!}{%
\begin{tabular}{llcccccccccccc}
\toprule
\multirow{2}{*}{\textbf{Dataset}} & \multirow{2}{*}{\textbf{Model}} & \multicolumn{6}{c}{\textbf{AUROC} $\Uparrow$} & \multicolumn{6}{c}{\textbf{AUARC} $\Uparrow$} \\ \cmidrule(lr){3-8} \cmidrule(lr){9-14}
 &  & P(true) & CSL & CSL-next & SL & Perplexity & TokenSAR & P(true) & CSL & CSL-next & SL & Perlexity & TokenSAR \\ \midrule
\multirow{4}{*}{C-QA}
 & Llama2-7b   & 62.278 & 78.253 & 74.799 & 81.390 & 76.958 & 77.888 &  28.401 & 38.231 & 36.213 & 40.178 & 37.579 & 37.450 \\
 & Llama3-8b   & 82.760 & 78.423 & 73.068 & 81.731 & 75.503 & 75.385 & 40.235 & 38.191 & 35.096 & 40.152 & 36.368 & 35.453  \\
 & Phi4        & 86.184 & 77.382 & 73.477 & 78.903 & 75.471 & 75.722 & 42.447 & 37.984 & 35.749 & 38.452 & 36.928 & 36.630  \\
 & Qwen2.5-32b & 89.892 & 82.486 & 77.087 & 82.143 & 78.964 & 79.064 & 45.449 & 40.802 & 38.003 & 40.596 & 38.674 & 38.215\\%[1ex]
 \midrule
\multirow{4}{*}{QASC}
 & Llama2-7b   & 66.198 & 77.535 & 76.053 & 79.589 & 77.637 & 77.696 &  19.815 & 25.986 & 25.494 & 27.632 & 26.324 & 25.921\\
 & Llama3-8b   & 86.069 & 77.970 & 73.090 & 80.718 & 74.531 & 75.006 &  30.127 & 26.215 & 24.251 & 28.253 & 24.442 & 24.308 \\
 & Phi4        & 84.478 & 77.556 & 74.596 & 78.661 & 75.678 & 76.222 & 29.977 & 26.068 & 25.246 & 27.064 & 25.463 &25.307 \\
 & Qwen2.5-32b & 88.998 & 79.324 & 73.895 & 78.598 & 74.485 & 75.175 & 32.992 & 26.810 & 24.608 & 27.387 & 24.069 & 23.992  \\%[1ex]
  \midrule
\multirow{4}{*}{MedQA}
 & Llama2-7b   & 54.660 & 55.144 & 55.852 & 54.766 & 55.766 & 55.703 & 22.414 & 24.437 & 24.888 & 24.246 & 24.848 & 24.795  \\
 & Llama3-8b   & 77.493 & 57.384 & 57.894 & 57.919 & 57.592 & 57.530 & 36.884 & 24.072 & 25.225 & 25.879 & 24.973 & 24.803 \\
 & Phi4 & 86.888 & 65.550 & 64.284 & 63.287 & 65.588 & 65.696 &42.615 & 31.671 & 31.050 & 30.888 &31.752 & 31.775  \\
 & Qwen2.5-32b & 80.131 & 63.264 & 63.712 & 63.109 & 62.564 & 62.164 &40.197 & 27.495 & 27.754 & 29.382 & 27.440 & 27.221  \\%[1ex]
  \midrule
\multirow{4}{*}{RACE-m}
 & Llama2-7b  & 63.965 & 69.194 & 70.819 & 67.568 & 71.823 & 71.984 & 35.543 & 38.429 & 39.404 & 38.870 & 40.030 & 40.133 \\
 & Llama3-8b   & 82.118 & 67.317 & 70.875 & 69.321 & 69.851 & 70.029 & 47.145 & 36.953 & 40.206 & 40.508 & 39.144 & 39.232 \\
 & Phi4        & 90.543 & 68.334 & 69.5354 & 68.8049 & 69.025 & 69.188 & 52.457 & 36.638 & 38.717 & 40.314 & 37.972 & 38.057 \\
 & Qwen2.5-32b  & 56.049 &67.294 & 69.102 & 73.267 & 69.147 & 69.279 & 29.283 & 34.913 & 36.873 & 42.373 & 36.220 & 36.318 \\%[1ex]
  \midrule
\multirow{4}{*}{RACE-h}
 & Llama2-7b   & 61.265 & 61.905 & 62.481 & 59.889 & 63.486 & 63.465 & 35.543 & 38.429 & 39.404 & 38.870 & 40.030 & 40.133 \\
 & Llama3-8b    & 79.466 & 60.775 & 63.868 & 61.253 & 64.134 & 64.146 & 44.910 & 31.300 & 34.086 & 33.463 & 33.973 & 33.974 \\
 & Phi4       & 87.172 & 62.253 & 62.680 & 60.178 & 63.391 & 63.383 &  50.250 & 32.395 & 33.484 & 33.243 & 33.547 & 33.537 \\
 & Qwen2.5-32b   & 52.811 & 61.837 & 64.047 & 63.555 & 64.050 & 64.024 & 27.605 & 31.279 & 32.714 & 34.462 & 32.462 & 32.458 \\
\bottomrule
\end{tabular}%
}
\caption{AUROC and AUARC for white-box methods, across different models and datasets}
\label{appendix:tab:wb:auc}
\end{table*}

% \begin{table*}[h!]
% \centering
% \resizebox{0.5\textwidth}{!}{%
% \begin{tabular}{llcccccc}
% \toprule
% \multirow{2}{*}{\textbf{Dataset}} & \multirow{2}{*}{\textbf{Model}} & \multicolumn{6}{c}{\textbf{AUROC} $\Uparrow$} \\
% \cmidrule(lr){3-8}
%  &  & P(true) & CSL & CSL-next & SL & Perplexity & TokenSAR \\ 
% \midrule
% \multirow{4}{*}{C-QA}
%  & Llama2-7b   & 62.278 & 78.253 & 74.799 & 81.390 & 76.958 & 77.888 \\
%  & Llama3-8b   & 82.760 & 78.423 & 73.068 & 81.731 & 75.503 & 75.385 \\
%  & Phi4        & 86.184 & 77.382 & 73.477 & 78.903 & 75.471 & 75.722 \\
%  & Qwen2.5-32b & 89.892 & 82.486 & 77.087 & 82.143 & 78.964 & 79.064 \\
% \midrule
% \multirow{4}{*}{QASC}
%  & Llama2-7b   & 66.198 & 77.535 & 76.053 & 79.589 & 77.637 & 77.696 \\
%  & Llama3-8b   & 86.069 & 77.970 & 73.090 & 80.718 & 74.531 & 75.006 \\
%  & Phi4        & 84.478 & 77.556 & 74.596 & 78.661 & 75.678 & 76.222 \\
%  & Qwen2.5-32b & 88.998 & 79.324 & 73.895 & 78.598 & 74.485 & 75.175 \\
% \midrule
% \multirow{4}{*}{MedQA}
%  & Llama2-7b   & 54.660 & 55.144 & 55.852 & 54.766 & 55.766 & 55.703 \\
%  & Llama3-8b   & 77.493 & 57.384 & 57.894 & 57.919 & 57.592 & 57.530 \\
%  & Phi4        & 86.888 & 65.550 & 64.284 & 63.287 & 65.588 & 65.696 \\
%  & Qwen2.5-32b & 80.131 & 63.264 & 63.712 & 63.109 & 62.564 & 62.164 \\
% \midrule
% \multirow{4}{*}{RACE-m}
%  & Llama2-7b   & 63.965 & 69.194 & 70.819 & 67.568 & 71.823 & 71.984 \\
%  & Llama3-8b   & 82.118 & 67.317 & 70.875 & 69.321 & 69.851 & 70.029 \\
%  & Phi4        & 90.543 & 68.334 & 69.5354 & 68.8049 & 69.025 & 69.188 \\
%  & Qwen2.5-32b & 56.049 & 67.294 & 69.102 & 73.267 & 69.147 & 69.279 \\
% \midrule
% \multirow{4}{*}{RACE-h}
%  & Llama2-7b   & 61.265 & 61.905 & 62.481 & 59.889 & 63.486 & 63.465 \\
%  & Llama3-8b   & 79.466 & 60.775 & 63.868 & 61.253 & 64.134 & 64.146 \\
%  & Phi4        & 87.172 & 62.253 & 62.680 & 60.178 & 63.391 & 63.383 \\
%  & Qwen2.5-32b & 52.811 & 61.837 & 64.047 & 63.555 & 64.050 & 64.024 \\
% \bottomrule
% \end{tabular}%
% }
% \caption{White Box Methods Performance Metrics Across Different Models and Datasets (AUROC)}
% \label{tab:metrics_table}
% \end{table*}
%%%%%%%%%%%%%%%%%%%%%%%%%%%%%%%%%%%%%%%%%%%%%%%%%%%%%%%%%%%%%%%%%%%%%%%%%%%%%%%%%%%%%%%%
%next
%%%%%%%%%%%%%%%%%%%%%%%%%%%%%%%%%%%%%%%%%%%%%%%%%%%%%%%%%%%%%%%%%%%%%%%%%%%%%%%%%%%%%%%%
%%%%%%%%%%%%%%%%%%%%%%%%%%%%%%%%%%%%%%%%%%%%%%%%%%%%%%%%%%%%%%%%%%%%%%%%%%%%%%%%%%%%%%%%

\begin{table*}[t]
\centering
\resizebox{\textwidth}{!}{%
\begin{tabular}{llcccccccccccc}
\toprule
\multirow{2}{*}{\textbf{Dataset}} & \multirow{2}{*}{\textbf{Model}} & \multicolumn{6}{c}{\textbf{RCE}} & \multicolumn{6}{c}{\textbf{Calibration ECE}} \\ \cmidrule(lr){3-8} \cmidrule(lr){9-14}
 &  & Ecc(C) & Deg(C) & Ecc(E) & Deg(E) & Ecc(J) & Deg(J) & Ecc(C) & Deg(C) & Ecc(E) & Deg(E) & Ecc(J) & Deg(J) \\ \midrule
\multirow{4}{*}{C-QA} 
 & Llama2-7b    & 0.2857  & 0.143722 & 0.117486 & 0.084357 & 0.271789 & 0.198744 & 0.014457 & 0.064792 & 0.025161 & 0.009014 & 0.009546 & 0.031801 \\
 & Llama3-8b    & 0.28071 & 0.15255  & 0.06311  & 0.041246 & 0.362527 & 0.153761 & 0.013865 & 0.044074 & 0.031566 & 0.016865 & 0.008845 & 0.060919 \\
 & Phi4         & 0.18881 & 0.115068 & 0.067507 & 0.038771 & 0.225698 & 0.218135 & 0.017734 & 0.059135 & 0.040364 & 0.024237 & 0.019987 & 0.056875 \\
 & Qwen2.5-32b  & 0.16192 & 0.114378 & 0.080021 & 0.055613 & 0.278165 & 0.198222 & 0.0111   & 0.087857 & 0.043406 & 0.016647 & 0.014439 & 0.051092 \\%[1ex]
  \midrule
\multirow{4}{*}{QASC} 
 & Llama2-7b    & 0.25132 & 0.162559 & 0.193186 & 0.121908 & 0.331258 & 0.252667 & 0.013984 & 0.020481 & 0.019263 & 0.012321 & 0.003108 & 0.022164 \\
 & Llama3-8b    & 0.28697 & 0.231308 & 0.083146 & 0.057512 & 0.401264 & 0.230094 & 0.003117 & 0.005336 & 0.004844 & 0.009398 & 0.010951 & 0.022145 \\
 & Phi4         & 0.19064 & 0.104986 & 0.066258 & 0.063753 & 0.23061  & 0.225091 & 0.004181 & 0.015734 & 0.012447 & 0.01108  & 0.003271 & 0.026654 \\
 & Qwen2.5-32b  & 0.25004 & 0.142512 & 0.091264 & 0.084393 & 0.31938  & 0.272657 & 0.010503 & 0.020774 & 0.012144 & 0.009716 & 0.004127 & 0.023387 \\%[1ex]
  \midrule
\multirow{4}{*}{MedQA} 
 & Llama2-7b    & 0.19817 & 0.188788 & 0.231296 & 0.243174 & 0.263178 & 0.213793 & 0.005909 & 0.006271 & 0.006057 & 0.01008  & 0.007157 & 0.008915 \\
 & Llama3-8b    & 0.21067 & 0.190038 & 0.286932 & 0.194414 & 0.290058 & 0.146904 & 0.006035 & 0.006757 & 0.006424 & 0.006872 & 0.01166  & 0.007277 \\
 & phi4         & 0.09127 & 0.09877  & 0.208792 & 0.132527 & 0.308812 & 0.087518 & 0.008327 & 0.018021 & 0.0156   & 0.008231 & 0.020912 & 0.016443 \\
 & Qwen2.5-32b  & 0.09064 & 0.089393 & 0.194414 & 0.087518 & 0.234422 & 0.118149 & 0.006312 & 0.01598  & 0.011337 & 0.021417 & 0.014092 & 0.021119 \\%[1ex]
  \midrule
\multirow{4}{*}{RACE-m} 
 & Llama2-7b  & 0.09876 & 0.31881 & 0.17315 & 0.27630 & 0.14502 & 0.16065 & 0.04523 & 0.07009 & 0.01980 & 0.01965  & 0.00778 & 0.01433 \\
 & Llama3-8b  & 0.10252 & 0.32068 & 0.12877 & 0.27005 & 0.21254 & 0.04500 & 0.00939 & 0.08513 & 0.00962 & 0.04675 & 0.025705  & 0.03261 \\
 & phi4          & 0.06001 & 0.31756  & 0.11252 & 0.26817 & 0.150655 & 0.07501 & 0.01699 & 0.07599 & 0.03366   & 0.01936 & 0.016385 &  0.01542 \\
 & Qwen2.5-32b  & 0.19378 &  0.32756 &  0.18253 & 0.27505 & 0.09689 & 0.1187 & 0.024623 & 0.10445  & 0.02922 & 0.05540 & 0.01300 & 0.02171 \\%[1ex]
  \midrule
\multirow{4}{*}{RACE-h} 
 & Llama2-7b  & 0.12565 & 0.36069 & 0.22441 & 0.40383 & 0.29568 & 0.30881 & 0.01702 & 0.06116 & 0.01635 & 0.01577  & 0.020679 & 0.01569 \\
 & Llama3-8b   & 0.20316 & 0.37007 & 0.18816 & 0.42070 & 0.26192 & 0.05938 & 0.01754 & 0.06838 & 0.01672 & 0.02324 & 0.02597  & 0.02622\\
 & phi4        & 0.09751 & 0.36757  & 0.14627 & 0.38820 &  0.26880 & 0.15878 & 0.01928 & 0.06393 & 0.021709   & 0.02191 & 0.02294 & 0.02502 \\
 & Qwen2.5-32b  & 0.11564 & 0.35069 & 0.21441 & 0.35569 & 0.28505 & 0.205666 & 0.01679 & 0.06562  & 0.01794 &0.02833 & 0.015137 & 0.01438 \\[1ex]
\bottomrule
\end{tabular}%
}
\caption{RCE and (calibrated) ECE for black-box methods, across different models and datasets}
\label{appendix:tab:bb:calib}
\end{table*}




\begin{table*}[t]
\centering
\resizebox{\textwidth}{!}{%
\begin{tabular}{llcccccccccccc}
\toprule
\multirow{2}{*}{\textbf{Dataset}} & \multirow{2}{*}{\textbf{Model}} & \multicolumn{6}{c}{\textbf{RCE}} & \multicolumn{6}{c}{\textbf{Calibration ECE}} \\ \cmidrule(lr){3-8} \cmidrule(lr){9-14}
 &  & P(true) & CSL & CSL-next & SL & SL(norm) & TokenSAR & P(true) & CSL & CSL-next & SL & SL(norm) & TokenSAR \\ \midrule
\multirow{4}{*}{C-QA} 
 & Llama2-7b    & 0.084386 & 0.0506   & 0.041895  & 0.041267     & 0.038126 & 0.034997 & 0.0102   & 0.035637  & 0.041958  & 0.023881        & 0.04454 & 0.027278 \\
 & Llama3-8b    & 0.040614 & 0.03563  & 0.068102  & 0.031902     & 0.057489 & 0.038742 & 0.01871  & 0.034739  & 0.050008  & 0.022294        & 0.04291 & 0.026352 \\
 & Phi4         & 0.043731 & 0.04626  & 0.046892  & 0.043771     & 0.041858 & 0.03501  & 0.0583   & 0.034232  & 0.055943  & 0.019535        & 0.04302 & 0.030655 \\
 & Qwen2.5-32b  & 0.058105 & 0.02999  & 0.044359  & 0.032513     & 0.044363 & 0.059406 & 0.0369   & 0.022175  & 0.046935  & 0.021905        & 0.03671 & 0.021438 \\%[1ex]
  \midrule
\multirow{4}{*}{QASC} 
 & Llama2-7b    & 0.077448 & 0.04685  & 0.078796  & 0.051258     & 0.043136 & 0.045007 & 0.01119  & 0.024505  & 0.037871  & 0.023245        & 0.0326  & 0.023127 \\
 & Llama3-8b    & 0.030627 & 0.04811  & 0.117522  & 0.050664     & 0.08503  & 0.043753 & 0.00894  & 0.020665  & 0.038958  & 0.025687        & 0.03274 & 0.020785 \\
 & Phi4         & 0.082518 & 0.04437  & 0.116905  & 0.066942     & 0.088115 & 0.049376 & 0.02122  & 0.021401  & 0.0415    & 0.028242        & 0.02548 & 0.033083 \\
 & Qwen2.5-32b  & 0.11997  & 0.04878  & 0.062505  & 0.081237     & 0.073773 & 0.041861 & 0.03096  & 0.014358  & 0.040047  & 0.025111        & 0.02665 & 0.023483 \\%[1ex]
   \midrule
\multirow{4}{*}{MedQA} 
 & Llama2-7b    & 0.181911 & 0.19254  & 0.19879   & 0.191288     & 0.228796 & 0.238798 & 0.00606  & 0.015623  & 0.007533  & 0.00791         & 0.00669 & 0.007449 \\
 & Llama3-8b    & 0.028131 & 0.08939  & 0.121274  & 0.207542     & 0.163158 & 0.178161 & 0.0166   & 0.012721  & 0.008949  & 0.03            & 0.00861 & 0.010613 \\
 & phi4         & 0.05126  & 0.09127  & 0.115648  & 0.176285     & 0.119399 & 0.116273 & 0.02853  & 0.046391  & 0.05184   & 0.058272        & 0.05787 & 0.05535  \\
 & Qwen2.5-32b  & 0.078141 & 0.06126  & 0.07314   & 0.128151     & 0.088143 & 0.075015 & 0.03067  & 0.020881  & 0.033491  & 0.047763        & 0.03295 & 0.032673 \\%[1ex]
   \midrule
\multirow{4}{*}{RACE-m} 
 & Llama2-7b  & 0.16253 & 0.26130 & 0.22254 & 0.13502     & 0.24317 & 0.24567 & 0.00741 & 0.01935 & 0.03113 & 0.01820  & 0.061396 & 0.062452 \\
 & Llama3-8b    & 0.05938 & 0.18003 & 0.09814 & 0.12752     & 0.10252 & 0.12189 & 0.05006 & 0.05534 & 0.04303 & 0.04812        & 0.01986  & 0.02156 \\
 & phi4        & 0.09689 & 0.15753  & 0.09314 & 0.09689     & 0.13127 & 0.13565 & 0.02585 & 0.04808 &0.02775  & 0.032335        & 0.01727 & 0.01938 \\
 & Qwen2.5-32b  & 0.16940 & 0.17566 & 0.17691 & 0.17691     & 0.24567 & 0.25255 &0.00695& 0.04720  & 0.05091 &  0.07564        & 0.07986 &  0.08066 \\%[1ex]
   \midrule
\multirow{4}{*}{RACE-h} 
 & Llama2-7b & 0.17566 & 0.31818 & 0.32318 & 0.33256     & 0.31818 & 0.32256 & 0.01748 & 0.02600 & 0.01719 & 0.01613         & 0.021382 & 0.021339 \\
 & Llama3-8b   & 0.05563 & 0.22316 & 0.12189 & 0.15565     & 0.163782 & 0.149404 & 0.045399 & 0.01684 & 0.031577 & 0.034098        & 0.030134  & 0.030341 \\
 & phi4       & 0.08939 & 0.19566  & 0.150030 & 0.13315     & 0.19316 & 0.19566 & 0.019294 & 0.035576 &  0.02874   & 0.03037        & 0.02238 & 0.040637 \\
 & Qwen2.5-32b    & 0.24754 & 0.22254 & 0.21754 & 0.21316     & 0.29505 & 0.30006 & 0.016826 & 0.02004  &0.02105 & 0.022801        & 0.03156 & 0.04110 \\
\bottomrule
\end{tabular}%
}
\caption{RCE and (calibrated) ECE for white-box methods, across different models and datasets}
\label{appendix:tab:wb:calib}
\end{table*}







\subsection{Additional Visualizations for ROC Curves}
\cref{appendix:fig:ROC} presents the ROC curves for \phiName.
\baselinePTrue achieves much better performance than other confidence measures on the more challenging datasets, likely because \phiName is a relatively advanced model.
On the easier datasets, where we could observe a bigger performance gap between different confidence measures, it is also interesting to see that the general shapes (and rankings at different FPR) are relatively consistent across C-QA and QASC, suggesting stability of \uqeval.



\def \FigAUROCVisHorizontalBar{
\begin{figure}[t]
  \centering
  \begin{subfigure}[b]{\columnwidth}
    \centering
    \includegraphics[width=\columnwidth]{figures/qasc_blackbox.png}
    \caption{AUROC of different black-box methods.}
    \label{fig:cqa_blackbox}
  \end{subfigure}
  \begin{subfigure}[b]{\columnwidth}
    \centering
    \includegraphics[width=\columnwidth]{figures/qasc_whitebox.png}
    \caption{AUROC of different white-box methods.}
    \label{fig:another_dataset}
  \end{subfigure}

  \caption{(a) and (b) show the performance of 4 different LLMs and 12 different confidence estimation methods on the QASC dataset. A higher AUROC indicates better performance.}
  \label{fig:llm_perspective}
\end{figure}

\begin{figure}[t]
  \centering
  \begin{subfigure}[b]{\columnwidth}
    \centering
    \includegraphics[width=\columnwidth]{figures/medqa_blackbox.png}
    \caption{AUROC of different black-box methods.}
    \label{fig:cqa_blackbox}
  \end{subfigure}
  \begin{subfigure}[b]{\columnwidth}
    \centering
    \includegraphics[width=\columnwidth]{figures/medqa_whitebox.png}
    \caption{AUROC of different white-box methods.}
    \label{fig:another_dataset}
  \end{subfigure}

  \caption{(a) and (b) show the performance of 4 different LLMs and 12 different confidence estimation methods on the MedQA dataset. A higher AUROC indicates a better performance.}
  \label{fig:llm_perspective}
\end{figure}
}
%/FigAUROCVisHorizontalBar


% \begin{figure*}
%     \centering
%     \includegraphics[width=0.99\linewidth]{figures/comparison_qwen.pdf}
%     \caption{The comparison of different evaluation metrics using our method to quantify Qwen-2.5-32b model's uncertainty on datasets: RACE-h (harder) and RACE-m (easier).}
%     \label{fig:compareQwen}
% \end{figure*}

% \begin{figure*}
%     \centering
%     \includegraphics[width=0.99\linewidth]{figures/comparison_llama7b.pdf}
%     \caption{The comparison of different evaluation metrics using our method to quantify Llama2-7b model's uncertainty on datasets: RACE-h (harder) and RACE-m (easier).}
%     \label{fig:compareLlama2}
% \end{figure*}

% \begin{figure*}
%     \centering
%     \includegraphics[width=0.99\linewidth]{figures/comparison_llama8b.pdf}
%     \caption{The comparison of different evaluation metrics using our method to quantify Llama3-8b model's uncertainty on datasets: RACE-h (harder) and RACE-m (easier).}
%     \label{fig:compareLlama3}
% \end{figure*}

\begin{figure*}[htbp]
    \centering
    % Row 1: Two subfigures side by side
    \begin{subfigure}[b]{0.45\textwidth}
        \centering
        \includegraphics[width=\textwidth]{figures/c-qa.png}
        \caption{C-QA Dataset}
        \label{fig:subfig1}
    \end{subfigure}
    \hfill
    \begin{subfigure}[b]{0.45\textwidth}
        \centering
        \includegraphics[width=\textwidth]{figures/qasc.png}
        \caption{QASC Dataset}
        \label{fig:subfig2}
    \end{subfigure}

    % Row 2: Two subfigures side by side
    \begin{subfigure}[b]{0.45\textwidth}
        \centering
        \includegraphics[width=\textwidth]{figures/extra/phi4_cqa_race_m_10_update1.png}
        \caption{RACE-m Dataset}
        \label{fig:subfig3}
    \end{subfigure}
    \hfill
    \begin{subfigure}[b]{0.45\textwidth}
        \centering
        \includegraphics[width=\textwidth]{figures/extra/phi4_cqa_race_h_10_update1.png}
        \caption{RACE-h Dataset}
        \label{fig:subfig4}
    \end{subfigure}

    % Row 3: One subfigure occupying most of the width
    \begin{subfigure}[b]{0.45\textwidth}
        \centering
        \includegraphics[width=\textwidth]{figures/medqa.png}
        \caption{MedQA Dataset}
        \label{fig:subfig5}
    \end{subfigure}
    
    \caption{Comparison of different evaluation metrics using our method to quantify the \phiName model's confidence scores across five datasets (C-QA, QASC, RACE-m, RACE-h, MedQA), with increasing difficulty.}
    \label{appendix:fig:ROC}
\end{figure*}

% \section{AI Assistant Usage}

% We used GPT for grammar checking and Copilot as an assistive tool.


% \section{Engines}

% To produce a PDF file, pdf\LaTeX{} is strongly recommended (over original \LaTeX{} plus dvips+ps2pdf or dvipdf). Xe\LaTeX{} also produces PDF files, and is especially suitable for text in non-Latin scripts.

% \section{Document Body}

% \subsection{Footnotes}

% Footnotes are inserted with the \verb|\footnote| command.\footnote{This is a footnote.}

% \subsection{Tables and figures}

% See Table~\ref{tab:accents} for an example of a table and its caption.
% \textbf{Do not override the default caption sizes.}

% \begin{table}
%   \centering
%   \begin{tabular}{lc}
%     \hline
%     \textbf{Command} & \textbf{Output} \\
%     \hline
%     \verb|{\"a}|     & {\"a}           \\
%     \verb|{\^e}|     & {\^e}           \\
%     \verb|{\`i}|     & {\`i}           \\
%     \verb|{\.I}|     & {\.I}           \\
%     \verb|{\o}|      & {\o}            \\
%     \verb|{\'u}|     & {\'u}           \\
%     \verb|{\aa}|     & {\aa}           \\\hline
%   \end{tabular}
%   \begin{tabular}{lc}
%     \hline
%     \textbf{Command} & \textbf{Output} \\
%     \hline
%     \verb|{\c c}|    & {\c c}          \\
%     \verb|{\u g}|    & {\u g}          \\
%     \verb|{\l}|      & {\l}            \\
%     \verb|{\~n}|     & {\~n}           \\
%     \verb|{\H o}|    & {\H o}          \\
%     \verb|{\v r}|    & {\v r}          \\
%     \verb|{\ss}|     & {\ss}           \\
%     \hline
%   \end{tabular}
%   \caption{Example commands for accented characters, to be used in, \emph{e.g.}, Bib\TeX{} entries.}
%   \label{tab:accents}
% \end{table}

% As much as possible, fonts in figures should conform
% to the document fonts. See Figure~\ref{fig:experiments} for an example of a figure and its caption.

% Using the \verb|graphicx| package graphics files can be included within figure
% environment at an appropriate point within the text.
% The \verb|graphicx| package supports various optional arguments to control the
% appearance of the figure.
% You must include it explicitly in the \LaTeX{} preamble (after the
% \verb|\documentclass| declaration and before \verb|\begin{document}|) using
% \verb|\usepackage{graphicx}|.

% \begin{figure}[t]
%   \includegraphics[width=\columnwidth]{example-image-golden}
%   \caption{A figure with a caption that runs for more than one line.
%     Example image is usually available through the \texttt{mwe} package
%     without even mentioning it in the preamble.}
%   \label{fig:experiments}
% \end{figure}

% \begin{figure*}[t]
%   \includegraphics[width=0.48\linewidth]{example-image-a} \hfill
%   \includegraphics[width=0.48\linewidth]{example-image-b}
%   \caption {A minimal working example to demonstrate how to place
%     two images side-by-side.}
% \end{figure*}

% \subsection{Hyperlinks}

% Users of older versions of \LaTeX{} may encounter the following error during compilation:
% \begin{quote}
% \verb|\pdfendlink| ended up in different nesting level than \verb|\pdfstartlink|.
% \end{quote}
% This happens when pdf\LaTeX{} is used and a citation splits across a page boundary. The best way to fix this is to upgrade \LaTeX{} to 2018-12-01 or later.

% \subsection{Citations}

% \begin{table*}
%   \centering
%   \begin{tabular}{lll}
%     \hline
%     \textbf{Output}           & \textbf{natbib command} & \textbf{ACL only command} \\
%     \hline
%     \citep{Gusfield:97}       & \verb|\citep|           &                           \\
%     \citealp{Gusfield:97}     & \verb|\citealp|         &                           \\
%     \citet{Gusfield:97}       & \verb|\citet|           &                           \\
%     \citeyearpar{Gusfield:97} & \verb|\citeyearpar|     &                           \\
%     \citeposs{Gusfield:97}    &                         & \verb|\citeposs|          \\
%     \hline
%   \end{tabular}
%   \caption{\label{citation-guide}
%     Citation commands supported by the style file.
%     The style is based on the natbib package and supports all natbib citation commands.
%     It also supports commands defined in previous ACL style files for compatibility.
%   }
% \end{table*}

% Table~\ref{citation-guide} shows the syntax supported by the style files.
% We encourage you to use the natbib styles.
% You can use the command \verb|\citet| (cite in text) to get ``author (year)'' citations, like this citation to a paper by \citet{Gusfield:97}.
% You can use the command \verb|\citep| (cite in parentheses) to get ``(author, year)'' citations \citep{Gusfield:97}.
% You can use the command \verb|\citealp| (alternative cite without parentheses) to get ``author, year'' citations, which is useful for using citations within parentheses (e.g. \citealp{Gusfield:97}).

% A possessive citation can be made with the command \verb|\citeposs|.
% This is not a standard natbib command, so it is generally not compatible
% with other style files.

% \subsection{References}

% \nocite{Ando2005,andrew2007scalable,rasooli-tetrault-2015}

% The \LaTeX{} and Bib\TeX{} style files provided roughly follow the American Psychological Association format.
% If your own bib file is named \texttt{custom.bib}, then placing the following before any appendices in your \LaTeX{} file will generate the references section for you:
% \begin{quote}
% \begin{verbatim}
% \bibliography{custom}
% \end{verbatim}
% \end{quote}

% You can obtain the complete ACL Anthology as a Bib\TeX{} file from \url{https://aclweb.org/anthology/anthology.bib.gz}.
% To include both the Anthology and your own .bib file, use the following instead of the above.
% \begin{quote}
% \begin{verbatim}
% \bibliography{anthology,custom}
% \end{verbatim}
% \end{quote}

% Please see Section~\ref{sec:bibtex} for information on preparing Bib\TeX{} files.

% \subsection{Equations}

% An example equation is shown below:
% \begin{equation}
%   \label{eq:example}
%   A = \pi r^2
% \end{equation}

% Labels for equation numbers, sections, subsections, figures and tables
% are all defined with the \verb|\label{label}| command and cross references
% to them are made with the \verb|\ref{label}| command.

% This an example cross-reference to Equation~\ref{eq:example}.

% \subsection{Appendices}

% Use \verb|\appendix| before any appendix section to switch the section numbering over to letters. See Appendix~\ref{sec:appendix} for an example.

% \section{Bib\TeX{} Files}
% \label{sec:bibtex}

% Unicode cannot be used in Bib\TeX{} entries, and some ways of typing special characters can disrupt Bib\TeX's alphabetization. The recommended way of typing special characters is shown in Table~\ref{tab:accents}.

% Please ensure that Bib\TeX{} records contain DOIs or URLs when possible, and for all the ACL materials that you reference.
% Use the \verb|doi| field for DOIs and the \verb|url| field for URLs.
% If a Bib\TeX{} entry has a URL or DOI field, the paper title in the references section will appear as a hyperlink to the paper, using the hyperref \LaTeX{} package.

% \section*{Acknowledgments}

% This document has been adapted
% by Steven Bethard, Ryan Cotterell and Rui Yan
% from the instructions for earlier ACL and NAACL proceedings, including those for
% ACL 2019 by Douwe Kiela and Ivan Vuli\'{c},
% NAACL 2019 by Stephanie Lukin and Alla Roskovskaya,
% ACL 2018 by Shay Cohen, Kevin Gimpel, and Wei Lu,
% NAACL 2018 by Margaret Mitchell and Stephanie Lukin,
% Bib\TeX{} suggestions for (NA)ACL 2017/2018 from Jason Eisner,
% ACL 2017 by Dan Gildea and Min-Yen Kan,
% NAACL 2017 by Margaret Mitchell,
% ACL 2012 by Maggie Li and Michael White,
% ACL 2010 by Jing-Shin Chang and Philipp Koehn,
% ACL 2008 by Johanna D. Moore, Simone Teufel, James Allan, and Sadaoki Furui,
% ACL 2005 by Hwee Tou Ng and Kemal Oflazer,
% ACL 2002 by Eugene Charniak and Dekang Lin,
% and earlier ACL and EACL formats written by several people, including
% John Chen, Henry S. Thompson and Donald Walker.
% Additional elements were taken from the formatting instructions of the \emph{International Joint Conference on Artificial Intelligence} and the \emph{Conference on Computer Vision and Pattern Recognition}.

% Bibliography entries for the entire Anthology, followed by custom entries
%\bibliography{anthology,custom}
% Custom bibliography entries only




\end{document}
